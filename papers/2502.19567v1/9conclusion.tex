% \section{Conclusion}\label{sec:conclusion}

% condensed maybe too short?
Our work demonstrates that robust security measures can be integrated into ML lifecycles without significant performance impact.
The combination of hardware-backed security guarantees with provenance tracking provides a foundation for securing ML pipelines.
Our case studies show \atlas' ability to maintain strong security properties while preserving established development practices.

We identify three directions for further enhancing ML pipeline security:
\begin{enumerate*}
    \item extending hardware security coverage to specialized ML accelerators will enable more comprehensive protection of ML computations;
    \item standardizing provenance tracking across the industry will significantly improve interoperability and adoption;
    \item developing robust algorithmic verification methods and model guardrails will strengthen protection against sophisticated attacks targeting model behavior, allowing for flexible integration of emerging defenses.
\end{enumerate*}

% org
%\section{Conclusion and Future Work}\label{sec:conclusion}
%
%This paper presented a comprehensive framework for ensuring ML model integrity and provenance throughout the model lifecycle.
%Our solution combines hardware-backed security guarantees with cryptographic mechanisms to create an end-to-end verifiable pipeline for ML operations.
%The implementation demonstrates that robust security measures can be integrated into existing ML workflows without significant performance impact.
%
%Our evaluation with BERT model pipelines suggests the practical feasibility of continuous integrity verification in larger-scale environments.
%The framework's modular design and flexible integration capabilities could enable organizations to maintain strong security guarantees while preserving their
%established development practices.
%
%Key contributions include:
%\begin{itemize}
%    \item Novel integration of hardware security features with ML pipeline security\cite{tdx2022}
%    \item Staged verification system with comprehensive integrity checks
%    \item Scalable implementation with support for incremental validation
%    \item Practical deployment methodology for enterprise environments
%\end{itemize}
%
%Several directions for future research and development emerge from this work:
%
%\begin{itemize}
%\item \textbf{Scalability Enhancements:} While our current implementation demonstrates effective verification
%through component classification and staged validation, further optimization opportunities exist for very large-scale
%deployments. This includes enhanced parallel verification mechanisms and more sophisticated caching strategies for
%distributed deployments.
%
%\item \textbf{RAG Integration:} As Retrieval-Augmented Generation becomes increasingly prevalent,
%extending our framework to handle dynamic content retrieval and real-time verification presents new challenges and opportunities for securing ML systems.
%
%\item \textbf{Algorithmic Security:} Integration of watermarking and fingerprinting techniques would provide additional layers of security through
%algorithmic verification methods. These techniques could enable ownership verification and unauthorized modification detection at the model behavior level.
%
%\item \textbf{Threat Detection:} The evolving landscape of ML security requires continuous advancement of our detection capabilities.
%Future work will focus on identifying and mitigating emerging attack vectors, particularly those targeting model provenance and integrity verification systems.
%
%\item \textbf{Distributed Training:} Supporting secure distributed training environments while maintaining strong provenance guarantees represents
%a significant challenge that requires further research and development.
%\end{itemize}
%
%These advancements will further strengthen the framework's ability to ensure trustworthy ML
%operations across diverse deployment scenarios while maintaining verifiable provenance throughout the model lifecycle.
