\section{Experiments}
\begin{table*}[h]
    \centering
    \small
    \renewcommand\tabcolsep{3.5pt}
    \begin{tabular}{l l c c c c c c c c c c c c }
    \toprule
    \multirow{3}{*}{Models} & \multirow{3}{*}{Strategy} & \multicolumn{6}{c}{GSM8K} & \multicolumn{6}{c}{MATH} \\
    & & \multicolumn{2}{c}{N=10} & \multicolumn{2}{c}{N=20} & \multicolumn{2}{c}{N=40} & \multicolumn{2}{c}{N=10} & \multicolumn{2}{c}{N=20} & \multicolumn{2}{c}{N=40}\\ 
    \cmidrule(lr){3-4} \cmidrule(lr){5-6} \cmidrule(lr){7-8} \cmidrule(lr){9-10} \cmidrule(lr){11-12} \cmidrule(lr){13-14}
     & & Mean & Max & Mean & Max & Mean & Max & Mean & Max & Mean & Max & Mean & Max \\
     \toprule

\multirow{2}{*}{Qwen2.5-1.5B} & Fix & 65.4 & 67.6& 67.2 & 69.6& 68.2 & \textbf{70.9}& 31.4 & \textbf{36.1}& 34.5 & 38.6& 36.5 & 40.8 \\
& Dynamic & \textbf{65.7} & \textbf{67.7}& \textbf{67.8} & \textbf{69.8}& \textbf{68.9} & \textbf{70.9}& \textbf{32.4} & 36.0 & \textbf{36.5} & \textbf{38.9}& \textbf{38.7} & \textbf{41.0} \\
\midrule
\multirow{2}{*}{Qwen2.5-1.5B-Instruct} & Fix & 79.0 & \textbf{80.7}& 80.3 & 82.2& 81.1 & 83.2& 51.9 & \textbf{53.8}& 53.3 & 55.0& 54.1 & 55.8 \\
& Dynamic & \textbf{79.2} & \textbf{80.7}& \textbf{80.8} & \textbf{82.4}& \textbf{81.6} & \textbf{83.5}& \textbf{52.3} & 53.6 & \textbf{53.9} & \textbf{55.2}& \textbf{54.6} & \textbf{55.9} \\
\midrule
\multirow{2}{*}{Qwen2.5-7B} & Fix & 84.6 & 86.1& 85.7 & 87.7& 86.3 & 88.9& 48.7 & 52.0& 50.7 & 53.8& 51.8 & 54.9 \\
& Dynamic & \textbf{84.7} & \textbf{86.3}& \textbf{86.1} & \textbf{88.1}& \textbf{86.8} & \textbf{89.0}& \textbf{49.6} & \textbf{52.3}& \textbf{51.9} & \textbf{53.9}& \textbf{53.2} & \textbf{55.1} \\
\midrule
\multirow{2}{*}{Qwen2.5-7B-Instruct} & Fix & \textbf{90.8} & 91.9& 91.2 & 92.2& 91.4 & \textbf{92.4}& 65.9 & 66.6& 66.6 & \textbf{67.3}& 66.9 & \textbf{67.7} \\
& Dynamic & \textbf{90.8} & \textbf{92.0}& \textbf{91.4} & \textbf{92.3}& \textbf{91.7} & \textbf{92.4}& \textbf{66.1} & \textbf{66.7}& \textbf{66.8} & \textbf{67.3}& \textbf{67.2} & 67.6 \\
\midrule
\multirow{2}{*}{Qwen2.5-Math-1.5B} & Fix & 80.1 & \textbf{83.3}& 82.0 & 84.6& 82.9 & \textbf{85.6}& 41.5 & 44.1& 43.1 & \textbf{46.0}& 44.2 & 47.1 \\
& Dynamic & \textbf{80.7} & 83.2 & \textbf{82.9} & \textbf{84.7}& \textbf{83.9} & \textbf{85.6}& \textbf{41.9} & \textbf{44.2}& \textbf{44.1} & \textbf{46.0}& \textbf{45.2} & \textbf{47.2} \\
\midrule
\multirow{2}{*}{Qwen2.5-Math-1.5B-Instruct} & Fix & 87.1 & 88.2& 87.7 & \textbf{88.8}& 87.9 & 89.0& 64.2 & 65.1& 64.7 & \textbf{65.8}& 64.9 & \textbf{66.1} \\
& Dynamic & \textbf{87.2} & \textbf{88.4}& \textbf{87.9} & \textbf{88.8}& \textbf{88.2} & \textbf{89.2}& \textbf{64.3} & \textbf{65.2}& \textbf{64.8} & 65.6 & \textbf{65.1} & 65.9 \\
\midrule
\multirow{2}{*}{Qwen2.5-Math-7B} & Fix & 82.2 & 85.0& 84.6 & 87.1& 85.8 & 88.2& 52.7 & 56.1& 54.9 & 57.9& 56.3 & 59.4 \\
& Dynamic & \textbf{83.0} & \textbf{85.4}& \textbf{85.5} & \textbf{87.4}& \textbf{86.8} & \textbf{88.5}& \textbf{53.4} & \textbf{56.2}& \textbf{56.2} & \textbf{58.4}& \textbf{57.7} & \textbf{59.7} \\
\midrule
\multirow{2}{*}{Qwen2.5-Math-7B-Instruct} & Fix & 94.9 & 95.8& 95.2 & \textbf{96.0}& 95.4 & \textbf{96.2}& 68.8 & 70.1& 69.5 & \textbf{70.9}& 69.7 & \textbf{70.9} \\
& Dynamic & \textbf{95.1} & \textbf{95.9}& \textbf{95.4} & \textbf{96.0}& \textbf{95.6} & \textbf{96.2}& \textbf{69.3} & \textbf{70.4}& \textbf{70.0} & 70.7 & \textbf{70.2} & \textbf{70.9} \\
\midrule
\multirow{2}{*}{Llama-3-8B} & Fix & 58.2 & 63.0& 60.9 & 65.8& 62.5 & 67.4& 18.6 & 21.5& 20.3 & 23.5& 21.7 & 25.1 \\
& Dynamic & \textbf{59.3} & \textbf{63.4}& \textbf{62.6} & \textbf{66.1}& \textbf{64.3} & \textbf{67.6}& \textbf{19.3} & \textbf{21.9}& \textbf{22.1} & \textbf{24.2}& \textbf{23.6} & \textbf{25.5} \\
\midrule
\multirow{2}{*}{Llama-3-8B-Instruct} & Fix & 66.6 & 72.0& 70.2 & 76.1& 72.2 & 78.6& \textbf{20.1} & 24.4& 21.3 & 26.8& 22.1 & 28.7 \\
& Dynamic & \textbf{67.1} & \textbf{72.7}& \textbf{71.6} & \textbf{76.9}& \textbf{74.1} & \textbf{79.5}& \textbf{20.1} & \textbf{25.0}& \textbf{21.4} & \textbf{26.9}& \textbf{22.3} & \textbf{28.8} \\
\midrule
\multirow{2}{*}{Gemma-2-2B} & Fix & 29.1 & 32.2& 31.0 & 33.9& 32.3 & \textbf{34.9}& 14.6 & 16.5& 16.1 & \textbf{18.2}& 16.8 & 18.2 \\
& Dynamic & \textbf{29.7} & \textbf{32.3}& \textbf{32.3} & \textbf{34.2}& \textbf{33.5} & 34.7 & \textbf{15.1} & \textbf{16.7}& \textbf{16.8} & 17.9 & \textbf{17.6} & \textbf{18.4} \\
\midrule

\multirow{2}{*}{Phi-1.5} & Fix & 35.1 & 37.6& 37.0 & 39.5& 38.1 & 40.7& 4.0 & 4.7& 4.6 & 5.0& 5.0 & 5.5 \\
& Dynamic & \textbf{35.6} & \textbf{37.7}& \textbf{37.8} & \textbf{39.6}& \textbf{39.0} & \textbf{40.8}& \textbf{4.2} & \textbf{5.0}& \textbf{4.8} & \textbf{5.2}& \textbf{5.3} & \textbf{5.7} \\
\midrule
\multirow{2}{*}{DeepSeek-Math-7B-Instruct} & Fix & \textbf{87.4} & \textbf{88.6}& 88.1 & 89.5& 88.5 & \textbf{90.1}& 44.4 & 45.8& 46.2 & \textbf{48.2}& 47.1 & \textbf{49.5} \\
& Dynamic & \textbf{87.4} & \textbf{88.6}& \textbf{88.3} & \textbf{89.8}& \textbf{88.7} & \textbf{90.1}& \textbf{44.8} & \textbf{46.1}& \textbf{46.6} & \textbf{48.2}& \textbf{47.8} & \textbf{49.5} \\
\midrule
\multirow{2}{*}{Llama-3.2-3B-Instruct} & Fix & \textbf{86.2} & \textbf{87.4}& 87.2 & 88.4& 87.7 & 88.9& 48.7 & 49.8& 50.2 & 51.4& 51.2 & 52.4 \\
& Dynamic & \textbf{86.2} & 87.3& \textbf{87.5} & \textbf{88.6}& \textbf{88.1} & \textbf{89.2}& \textbf{49.0} & \textbf{50.1}& \textbf{50.6} & \textbf{51.6}& \textbf{51.7} & \textbf{52.7} \\
\midrule
\multirow{2}{*}{Mistral-7B-Instruct-v0.3} & Fix & 46.1 & 48.9& 49.3 & 53.3& 51.5 & 57.2& 17.1 & 18.3& 19.2 & 20.8& 20.8 & 22.4 \\
& Dynamic & \textbf{46.6} & \textbf{49.7}& \textbf{50.4} & \textbf{55.0}& \textbf{52.8} & \textbf{58.9}& \textbf{17.6} & \textbf{19.0}& \textbf{20.2} & \textbf{21.0}& \textbf{22.2} & \textbf{23.5} \\


    \bottomrule
    \end{tabular}
        
    \caption{Evaluation results by using 15 models from different base architectures on GSM8K\citep{GSM8K} and MATH\citep{MATH}. Dynamic temperature sampling achieves superior average and maximum performance across a wide range of settings.}
    \label{tb:method_results}
\end{table*}

\subsection{Experiment Setup}
\paragraph{Datasets and Models}
We evaluate our method on two widely-used mathematical reasoning benchmarks: GSM8K \citep{GSM8K} and MATH \citep{MATH}. 
Experiments span multiple model families to assess generalizability, including Qwen \citep{qwen}, Llama \citep{llama}, Mistral\citep{mistral}, DeepSeek\citep{deepseek}, Gemma\citep{gemma} and Phi\citep{phi}.
\paragraph{Implementation Details}
To systematically compare dynamic versus static temperature strategies, we test initial temperatures $T_{0}\in\{0.1,0.2,...,1.0\}$ with sampling budgets $N\in\{10, 20, 40\}$.
\paragraph{Metric}
To provide an intuitive and efficient evaluation of the differences between methods, we calculate both the average and maximum accuracy for fixed-temperature sampling and dynamic-temperature sampling across all temperatures. The evaluation is conducted from the perspectives of robustness and effectiveness. Formally:
\begin{align}
    Mean &= \frac{1}{N_T}\sum_{t \in T_0}Acc_t \\
    Max &= \underset{t \in T_0}{max} \;Acc_t 
\end{align}

\subsection{Results}
From the results presented in Table \ref{tb:method_results} through 15 models, we can find:
\paragraph{Dynamic temperature sampling mitigates the performance degradation associated with fixed-temperature sampling.}
We find that the average accuracy across different temperatures for dynamic temperature sampling outperforms fixed-temperature sampling in the majority of models. This suggests that our method is not constrained by the temperature range and can identify samples that are more effective for self-consistency performance across various temperatures. To some extent, this approach mitigates the performance loss caused by ineffective sampling at a single fixed temperature.
\paragraph{For different samples, dynamic temperature sampling searches for a more suitable temperature for each sample.}
Similarly, we observe that dynamic temperature sampling also provides a certain improvement in terms of the maximum accuracy. This can be attributed to the fact that different samples require different temperature ranges. Fixed-temperature sampling can only achieve the desired accuracy for the dataset as a whole, whereas dynamic temperature sampling automatically searches for a more optimal temperature for each individual sample, maximizing the performance of self-consistency optimization across various temperatures.
\subsection{Analysis}
\paragraph{Visualization}
We provide a detailed analysis of the model's accuracy at different temperatures. Figure \ref{fig:analysis_acc} presents the accuracy and temperature curve for the Qwen2.5-Math-7B model. We observe that, with sampling sizes of 20 and 40, both low temperature ranges (0.1-0.4) and high temperature ranges (0.7-1.0) exhibit notable improvements. This suggests that dynamic temperature sampling yields more robust results. However, with a sampling size of 10, the performance in the low temperature range is almost identical to that of fixed-temperature sampling, primarily due to the constraints of the sample size. In the more optimal temperature range (0.4-0.7), the performance of dynamic and fixed-temperature sampling is similar, which aligns with our expectations and indicates that 
the intermediate temperature has already achieved a balanced trade-off.
\begin{figure*}[t]
\centering
\includegraphics[width=1.0\linewidth]{figs/method_gsm8k_Qwen2.5-Math-7B.pdf}
\caption{A detailed results of the model's accuracy across different temperatures. Our method achieves better performance under both lower and higher initial temperatures.}
\label{fig:analysis_acc}
\end{figure*} 
\begin{figure*}[t]
\centering
\includegraphics[width=1.0\linewidth]{figs/analysis_tem_ratio_gsm8k_Qwen2.5-Math-7B.pdf}
\caption{Proportions of samples with temperature increases, decreases, or stability during dynamic temperature sampling.}
\label{fig:analysis_tem_ratio}
\end{figure*} 
\begin{figure*}[!ht]
\centering
\includegraphics[width=1.0\linewidth]{figs/analysis_infsd_ratio_gsm8k_Qwen2.5-Math-7B_3.pdf}
\caption{Proportion of FSD instances reaching the dead zone, where dynamic temperature sampling results in a higher proportion.}
\label{fig:analysis_fsd_ratio}
\end{figure*} 
\paragraph{Direction Analysis of Temperature Variation}
Taking the sample level into account, we first analyzed the proportions of samples that experienced temperature increases, decreases, or remained constant throughout the dynamic temperature sampling process, as illustrated in Figure \ref{fig:analysis_tem_ratio}. We observed that in the low temperature range, at least 80\% of the samples experienced an increase in temperature. This observation is consistent with our hypothesis derived from dataset-level considerations, which suggests that increasing the temperature tends to result in higher expected accuracies. As the temperature rises, the proportion of samples experiencing temperature increases gradually declines, indicating that for some samples at the current sampling size, excessive temperatures are insufficient to confidently select the correct answer. Consequently, lowering the temperature becomes necessary to enhance FSD. Additionally, we noticed that with higher sampling sizes, the proportion of samples undergoing temperature increases is higher compared to low sampling sizes, which aligns with our analysis presented in Section \ref{sec:diversity}.

\paragraph{Proportion of Optimal Temperature Range}
We analyze the proportion of FSD instances that ultimately reach the dead zone. We consider reaching the dead zone as an indication that the sample operates within an optimal temperature range. As shown in Figure~\ref{fig:analysis_fsd_ratio}, dynamic temperature sampling results in a higher proportion of FSD instances entering the dead zone compared to fixed-temperature sampling, suggesting that our method enables better alignment for a larger number of samples.