\begin{abstract}
Self-consistency improves reasoning by aggregating diverse stochastic samples, yet the dynamics behind its efficacy remain underexplored. We reframe self-consistency as a dynamic distributional alignment problem, revealing that decoding temperature not only governs sampling randomness but also actively shapes the latent answer distribution.  Given that high temperatures require prohibitively large sample sizes to stabilize, while low temperatures risk amplifying biases, we propose a confidence-driven mechanism that dynamically calibrates temperature: sharpening the sampling distribution under uncertainty to align with high-probability modes, and promoting exploration when confidence is high. Experiments on mathematical reasoning tasks show this approach outperforms fixed-diversity baselines under limited samples, improving both average and best-case performance across varying initial temperatures without additional data or modules. This establishes self-consistency as a synchronization challenge between sampling dynamics and evolving answer distributions.
\end{abstract}