\documentclass[journal]{IEEEtran}
\usepackage{graphicx}
\usepackage{amsmath}
\usepackage{algorithm}
%\usepackage{algorithmic}
\usepackage{hyperref}
\usepackage{booktabs} % For formal tables
\usepackage{tabularx} % For adjustable-width columns

\usepackage{booktabs} % 使用三线表
\usepackage{tabularx} % 允许列宽自动调整
\usepackage{caption}  % 更好的标题控制
\usepackage{mathtools}
\usepackage{amssymb}%manthbb
% \usepackage{ulem}
\usepackage{amsmath}
\usepackage{multirow}

\usepackage{array}
\usepackage{makecell}
\usepackage{textgreek}
\usepackage{comment}
\usepackage{bm}
\usepackage{booktabs}
\usepackage{color}
\usepackage{tabularx}
\usepackage{algorithmicx}
\usepackage{algpseudocode}
\usepackage{subcaption}
\usepackage[normalem]{ulem}
\newtheorem{theorem}{Theorem}%[Chapter]
\newtheorem{lemma}{Lemma}
\newtheorem{claim}{Claim}
\newtheorem{proposition}
{Proposition}
\newtheorem{corollary}{Corollary}
\algnewcommand\Input{\item[\textbf{Input:}]}  % Define Input
\algnewcommand\Output{\item[\textbf{Output:}]}  % Define Output
\usepackage{graphicx} % Required for inserting images
\hyphenation{op-tical net-works semi-conduc-tor}

\begin{document}
%\title{SuRLLC: \underline{S}ecure \underline{U}ltra-\underline{r}eliable and \underline{L}ow \underline{L}atency \underline{C}ommunication in NOMA-UAV Intelligent Transportation Systems}
%\author{Kan Yu,~\IEEEmembership{Member,~IEEE}, Xiao Zhao, Xiaowu Liu,~\IEEEmembership{Member,~IEEE}, Zhiyong Feng,~\IEEEmembership{Senior Member,~IEEE}, Dong Li~\IEEEmembership{Senior Member,~IEEE}, and Jiguo Yu,~\IEEEmembership{Fellow,~IEEE}

%\title{A Novel Physical Layer Security in Internet of Vehicles: from Perspective of Sensing Interference}

\title{First Glimpse on Physical Layer Security in Internet of Vehicles: Transformed from Communication Interference to Sensing Interference}
 
\author{Kaixuan Li, Kan Yu,~\IEEEmembership{Member,~IEEE}, Xiaowu Liu, Qixun Zhang,~\IEEEmembership{Member,~IEEE}, Zhiyong Feng,~\IEEEmembership{Senior Member,~IEEE}, and Dong Li,~\IEEEmembership{Senior Member,~IEEE}
 %<-this % stops a space
\thanks{This work is supported by the National Natural Science Foundation of China with Grant 62301076, the Macao Young Scholars Program with Grant AM2023015, Fundamental Research Funds for the Central Universities with Grant  24820232023YQTD01, National Natural Science Foundation of China with Grants 62341101 and 62321001, Beijing Municipal Natural Science Foundation with Grant L232003, and National Key Research and Development Program of China with Grant 2022YFB4300403, and the Science and Technology Development Fund, Macau SAR, under Grant 0188/2023/RIA3.
 %the Natural Science Foundation of Shandong Province with Grants ZR2021QF050 and ZR2021MF075, Major Program of Shandong Provincial Natural Science Foundation for the Fundamental Research under Grant ZR2022ZD03, the Pilot Project for Integrated Innovation of Science, Education and Industry of Qilu University of Technology (Shandong Academy of Sciences) under Grant 2022XD001, the Talent Cultivation Promotion Program of Computer Science and Technology in Qilu University of Technology (Shandong Academy of Sciences) under Grant 2023PY059, the Colleges and Universities 20 Terms Foundation of Jinan City with Grant 202228093, and the Science and Technology Development Fund, Macau SAR, under Grant 0029/2021/AGJ.
}

\thanks{K. Li is with the School of Computer Science, Qufu Normal University, Rizhao, P.R. China. E-mail: lkx0311@126.com;}
\thanks{K. Yu (\emph{the corresponding author}) is with the School of Computer Science and Engineering, Macau University of Science and Technology, Taipa, Macau, 999078, P. R. China;
the Key Laboratory of Universal Wireless Communications, Ministry of Education, Beijing University of Posts and Telecommunications, Beijing, 100876, P.R. China. E-mail: kanyu1108@126.com;}
\thanks{X. Liu is with the School of Computer Science, Qufu Normal University, Rizhao, P.R. China. E-mail: liuxw@qfnu.edu.cn;}
\thanks{Q. Zhang is with the Key Laboratory of Universal
Wireless Communications, Ministry of Education, Beijing University of Posts and Telecommunications, Beijing, 100876, P.R. China. E-mail: zhangqixun@bupt.edu.cn;}
\thanks{Z. Feng is with the Key Laboratory of Universal Wireless Communications, Ministry of Education, Beijing University of Posts and Telecommunications, Beijing, 100876, P.R. China. E-mail: fengzy@bupt.edu.cn;}
\thanks{D. Li is with the School of Computer Science and Engineering, Macau University of Science and Technology, Taipa, Macau, China. E-mail: dli@must.edu.mo.}

%S. Wang is with the School of Artificial Intelligence, Beijing Normal University, Beijing 100875, P.R. China. E-mail: wangshengling@bnu.edu.cn\protect\\
%X. Cheng is with the Department of Computer Science, The George Washington University, Washington DC 20052. E-mail: cheng@gwu.edu\protect\\
%D. Yu is with School of Computer Science and Technology, Shandong University, Qingdao, 266510, PR China. E-mail: dxyu@cs.hku.hk\protect\\


%\thanks{X. Liu is with School of Computer Science, Qufu Normal University, Rizhao, 276826, P.R. China. E-mail: liuxw@qfnu.edu.cn;}
%\thanks{G. Li is with School of Computer Science, Qufu Normal University, Rizhao, 276826, P.R. China. E-mail: guangshunli@qfnu.edu.cn.}
}




% The paper headers
\markboth{IEEE Transactions on Communications,~Vol.~, No.~, 2024}%
{Shell \Baogui Huang{\textit{et al.}}: Shortest Link Scheduling Under SINR}
\maketitle
\begin{abstract}
Integrated sensing and communication (ISAC) plays a crucial role in the Internet of Vehicles (IoV), serving as a key factor in enhancing driving safety and traffic efficiency. To address the security challenges of the confidential information transmission caused by the inherent openness nature of wireless medium, different from current physical layer security (PLS) methods, which depends on the \emph{additional communication interference} costing extra power resources, in this paper, we investigate a novel PLS solution, under which the \emph{inherent radar sensing interference} of the vehicles is utilized to secure wireless communications. To measure the performance of PLS methods in ISAC-based IoV systems, we first define an improved security performance metric called by transmission reliability and sensing accuracy based secrecy rate (TRSA\_SR), and derive closed-form expressions of connection outage probability (COP), secrecy outage probability (SOP), success ranging probability (SRP) for evaluating transmission reliability, security and sensing accuracy, respectively. Furthermore, we formulate an optimization problem to maximize the TRSA\_SR by utilizing radar sensing interference and joint design of the communication duration, transmission power and straight trajectory of the legitimate transmitter. Finally, the non-convex feature of formulated problem is solved through the problem decomposition and alternating optimization.  
Simulations indicate that compared with traditional PLS methods obtaining a non-positive STC, the proposed method achieves a secrecy rate of 3.92bps/Hz for different settings of noise power.

\iffalse

Physical layer security (PLS) offers an attractive solution to enhance wireless network security.


Inter-vehicle communication plays a crucial role in vehicle networks, serving as a key factor in enhancing traffic safety and optimizing traffic efficiency. Nevertheless, communication signal transmission faces two challenges, including information eavesdropped and complex interference, due to the broadcast nature of wireless communication signals and the propagation characteristics of line-of-sight (LoS) links. Physical layer security (PLS) schemes utilize the transmission characteristics of wireless channels to ensure the reliable and secure transmission of secret information. Among these, artificial noise (AN) schemes effectively enhance communication performance by interfering with eavesdroppers. However, existing PLS schemes are not suitable for Internet of vehicles (IoV) as they fail to consider the impact of sensing interference arising from vehicle radars. When sensing signals couple with communication frequency bands, this can have a detrimental effect on system communication performance. Sensing signals will interfere with communication signals once their frequency bands couple. This paper addresses the reliable and secure transmission of communication signals in IoV, where sensing signal coupling and eavesdroppers are present.  We propose the \textcolor{red}{intrinsic} physical layer security (IPLS) scheme tailored for IoV and establish the analytical framework. This scheme fully exploits the existing sensing interference to suppress eavesdroppers' communication channels and assess communication confidentiality under a range of metrics. Furthermore, we optimize signal transmission time, sender power, and trajectory. Simulations indicate that the IPLS scheme substantially improves the security while sacrifices reliability of the communication compared with PLS. In addition, joint optimisation of communication time, trajectory, and communication power schemes can significantly improve secrecy rates and enable secure and reliable transmission.
\fi
\end{abstract}
\begin{IEEEkeywords}
Physical layer security; Secrecy rate maximization; Integrated 
sensing and communication; Internet of Vehicles; Inherent sensing interference 
\end{IEEEkeywords}

\IEEEpeerreviewmaketitle

\section{Introduction}\label{sec:introduction}

The IoV, equipped with ISAC capabilities, serves as a significant enabler of intelligent transportation systems (ITS). Driven by the demands for high-precision sensing, ultra-low communication latency, and ultra-high data rate, 
the transition of communication and sensing technologies towards millimeter-wave (mmWave) frequency bands is an irresistible general trend \cite{Que2023Joint,Qi2024Multiuse}. Due to the inherent openness nature of the wireless medium, strong directional beamforming (BF) of mmWave, and vehicles' mobility, the confidentiality and security of information transmission in IoV systems become extremely challenging. For example, if the BF of the confidential information is aimed at the eavesdroppers (Eves), there will be no security at all. Consequently, the IoV faces pressing security challenges that require immediate attention. 

Aiming at maximizing difference effects between legitimate channel and eavesdropping channel, PLS emerges as an effective supplement of traditional encryption methods to address security challenges by utilizing the inherent unpredictability of wireless channels with a lower complexity satisfying the features of IoV \cite{shannon1949communication,wyner1975wire}. In fact, with the help of communication interference, the core idea behind PLS implementation is to suppress the quality of the eavesdropping channel to be inferior to that of the legitimate channel.
However, these PLS methods cannot be directly extended to secure wireless communications in
ISAC-assisted IoV systems,
\iffalse
\footnote{\textcolor{red}{The current communication frequency bands are evolving towards the mmWave frequency range (30-300GHz), and in the future may even extend to the terahertz frequency range \cite{Zhuo2023Performance}. This will potentially couple with the frequency range of mmWave radar, which indicates the development of ISAC means that communication and sensing tasks can be performed on the same frequency band.}}, 
\fi
since communication-sensing coupled interference exists rather than only communication interference when the communication and sensing functions work in the same or similar frequency band, along with the strong directional BF and vehicles' mobility in the context of ISAC-based IoV scenarios. Against this background, new effective PLS methods urgently need to be proposed for satisfying requirements of the communication, sensing, and security in ISAC-based IoV systems. 

\iffalse
Aiming at maximizing difference effects between legitimate channel and \textcolor{blue}{eavesdropping channel, physical layer security (PLS)} emerges as an effective supplement of traditional encryption methods to address \textcolor{blue}{security challenges}, by utilizing the inherent unpredictability of wireless channels with a lower complexity satisfying the features of IoV \cite{shannon1949communication,wyner1975wire}, \sout{namely limited computing capabilities of the vehicles and lower delay constraint. }
\textcolor{blue}{In fact, with the help of wireless interference, channel fading and noise, the core idea behind the implementation of PLS technologies is to suppress the quality of the eavesdropping channel to be inferior to that of the legitimate channel.}
However, these PLS methods cannot be directly extended to secure wireless communications in
ISAC-assisted IoV systems
\footnote{\textcolor{red}{The current communication frequency bands are evolving towards the millimeter wave frequency range (30-300GHz), and in the future may even extend to the terahertz frequency range \cite{Zhuo2023Performance}. This will potentially couple with the frequency range of mmWave radar, which indicates the development of ISAC means that communication and sensing tasks can be performed on the same frequency band.}}, 
since communication-sensing coupled interference exists rather than only communication interference, along with the strong directional BF and vehicles' mobility in the context of ISAC-based IoV scenarios. \textcolor{blue}{Therefore, new effective PLS methods need be proposed to satisfy requirements of the communication, sensing, and security.} 
\fi


The sensing interference is generally considered harmful to communication systems, and transceiver design \cite{Shlezinger2021DeepSIC}, BF design \cite{Liu2019Multi} and collaboration \cite{Li2022JointT} are popular methods to deal with the sensing interference for mitigating the effects on transmission reliability and sensing accuracy. From the perspective of PLS design, similar to communication interference, it can be also used to protect the confidential information from being eavesdropped. In addition, strong directional sensing BF potentially provides an effective means to only suppress the quality of wiretap channel rather than legitimate channel.
Then a natural question arises: \emph{\textbf{``is it possible and how to make a transformation from `communication interference based security' to `sensing interference based security'?''}}. This problem remains unclear and is not solved in existing works regarding the traditional PLS.

\iffalse
\textcolor{red}{In fact, the core idea behind the implementation of PLS technologies is to suppress the quality of the eavesdropping channel to be inferior to that of the legitimate channel through the communication interference. While coupled communication-sensing interference presents the following challenges in PLS design: 1) In ISAC systems, communication interference and sensing interference are coupled. Many current works mainly focused on how to suppress the sensing interference with the performance of communication systems. In nature, coupled interference leads to the poor universality of traditional PLS schemes based on communication interference, making it difficult to directly apply them to secure information transmission; 2) Generally, the sensing interference is generally considered harmful to communication systems. From the perspective of PLS, it can be used to protect confidential information from being eavesdropped. In particular, based on the perfect channel state information (CSI) assumption of the Eve, the strong directionality of mmWave sensing BF can effectively further suppress the wiretap channel.}

\textcolor{red}{However, the directional sensing BF poses severe security challenges for information transmission in ISAC-based IoV systems. For instance, when the directional sensing BF is aimed at the legitimate vehicles, the quality of the legitimate channel is likely to be worse than that of the eavesdropping channel, making it difficult to achieve PLS. Nevertheless, \textbf{the sensing capabilities of the vehicles and coupled sensing-communication interference also provide new opportunities for leveraging PLS technologies to enhance the system security}. In other words, sensing capabilities of vehicles can obtain the perfect CSI of Eves, and the sensing interference can be utilized to suppress the eavesdropping channel, instead of traditional communication interference.}



Therefore, in ISAC-assisted IoV, the coupled interference between communication and sensing restricts the mutual performance, while highly-directional sensing interference provides potentially an effective means to only suppress the quality of wiretap channel. Further more, the sensing interference is generally considered harmful to communication systems in general. From the perspective of PLS, it can be used to protect confidential information from being eavesdropped.
Then a natural question arises: \emph{\textbf{``Is it possible and how to make a transformation from `communication-security' to `sensing-security'?'' }}.
\fi


To demonstrate the rationality and validity of above transformation, in the context of an ISAC based IoV scenario, where a transmitting vehicle (Alice), a receiving vehicle (Bob), multiple moving vehicles (Carols) monitoring the target by using sensing capabilities, and an Eve are considered, in this paper, we study the joint design of the transmission power and straight trajectory of the Alice, and utilize the strong sensing BF of Carols to suppress the quality of eavesdropping channel, while satisfying the expected sensing accuracy of Carols, transmission reliability and security of Bob.
To the best of our knowledge, this is the first paper to only utilize sensing interference for secure wireless communications. The main contributions of this paper can be summarized as follows.
\iffalse
To solve above challenges, the  interference exploitation provides new opportunities for saving system energy and enhancing system safety. In other words, sensing capabilities of vehicles can obtain the perfect CSI of Eves, and the sensing interference can be utilized to suppress the eavesdropping channel, instead of traditional communication interference.
To the best of our knowledge, this is the first paper to only utilize sensing interference for secure wireless communications. The main contributions of this paper can be summarized as follows:
\fi
\begin{itemize}
    \item To address communication-sensing coupled interference, we derive closed-form expressions for connection outage probability (COP), secrecy outage probability (SOP), and success range probability (SRP), which offer significant insights into how key system parameters influence sensing accuracy, transmission reliability, and security; 
    \item To achieve the desired communication and sensing performance under constraints from COP, SOP, and SRP, we introduce a transmission reliability and sensing accuracy-based secrecy rate (TRSA\_SR) metric. We formulate an optimization problem to maximize TRSA\_SR through joint design of transmission time (for sensing interference utilization), power allocation, and the trajectory of Alice;
    \item Given the non-convex nature of the optimization problem, an alternating algorithm is proposed to iteratively optimize Alice’s transmission power allocation and trajectory. Additionally, we provide detailed analyses of the algorithm’s time complexity and convergence;
    \item Simulation results validate the effectiveness of sensing interference based PLS method. Notably, when traditional PLS approaches fail to achieve a positive TRSA\_SR, our proposed scheme achieves a TRSA\_SR of 3.92 bps/Hz for different settings of noise power.
\end{itemize}

\iffalse
\begin{itemize}
  \item 
  Using tools from stochastic geometry, we derived closed-form expressions for the success COP, SOP and SRP. These metrics are essential for evaluating the performance of sensing, communication, and security.  Furthermore, we construct an optimization problem aiming at maximizing the secrecy rate by jointly designing the vehicle's transmit power and moving trajectory, subject to constraints on SRP, COP, and SOP. This proposed optimization framework offers significant insights and practical guidance for optimizing the performance of the IoV system.
  %\item \textcolor{blue}{To track this NP-hard problem, the probabilistic constraint of outage probability is transformed and solved by Markov inequality and Marcum Q-function. After which, the problem is decomposed to three subproblems, which consist of the power allocation, communication UAV trajectory design, and artificial noise optimization. In each subproblem, the Dinkelbach’s method is adopted to convert the objective from fractional to the subtractive form, which has a low complexity and can reach convergence through efficient iteration optimization.}
  \item Before dealing with the the non-convexity of the optimization problem, due to limited maximum horizontal angle of sensing BF and vehicles' mobility, we first find the communication duration between legitimate vehicles, under which the sensing interference can cover the Eve. Then, for the non-convex problem, it is decomposed into two subproblems, which consist of the vehicle's transmit power allocation optimization and the straight trajectory optimization. After transforming the sub-problems into convex ones, an alternating algorithm (AO) is proposed to alternately optimize the power allocation and straight trajectory, which has a low complexity and converge quickly.
 % \item Before dealing with the the non-convexity of the optimization problem, due to limited maximum horizontal angle of sensing BF and vehicles' mobility, we first find the communication duration between legitimate vehicles, under which the sensing interference can cover the Eve. Then, for the non-convex problem via BCD method, it is decomposed to two subproblems, which consist of the vehicle's transmit power allocation optimization and the straight trajectory optimization. The sub-problems can be simplified by introducing Lagrangian Multiplier for the former, and auxiliary variables, Taylor expansion and SCA method for the latter, respectively. After transforming the sub-problems into convex ones, an alternating algorithm (AO) is proposed to alternately optimize the power allocation and straight trajectory, which has a low complexity and converge quickly.
  %\item This paper consider the interference of sensing signals on communication signals and construct the IPLS model using stochastic geometry methods. We propose a new performance metric, joint sensing and communication based Secrecy Rate Maximization (JSAC-SRM), to measure the secrecy rate under the constraints of communication reliability, communication security, and sensing accuracy. Moreover, we derive closed-form expressions for connection outage probability (COP), secrecy outage probability (SOP), successful ranging probability (SRP) and JSAC-SRM.
  %\item As communication signals are not continuously transmitting, it is necessary to optimize the timing of signal transmission in order to maximize the interference to eavesdroppers. Furthermore, the communication transmission power and driving trajectory of the transmitter are optimised in order to maximise the secrecy capacity.
  \item Simulations corroborate the effectiveness of our proposed scheme in terms of ``sensing interference based security'' and performance superiority in the achieved sensing accuracy, transmission reliability and security, as well as the secrecy rate.
\end{itemize}
\fi

\iffalse
\begin{enumerate}
    \item The coupled interference between communication and sensing restricts the mutual performance, while highly-directional sensing interference provides potentially an effective means to only suppress the quality of wiretap channel. But the interplay between coupled interference and PLS, as well as their performances, is still not clear. As a result, the question that whether it is feasible to make a transformation from "communication-security" to "sensing-security" need to be answered.
    \item The characteristics of the potential LoS channel follow a Gamma distribution, which makes it challenging to establish a closed-form expression of secrecy rate. Consequently, some significant insights into how key system parameters affect reliability, security, and latency cannot be gained.
    \item Traditional methods of UAV trajectory design primarily rely on experience and simplified models. These approaches often overlook the inherent randomness and complex channel environment associated with UAV communication. This limitation can potentially expose the transmitted information to Eves.
\end{enumerate}
\fi


\iffalse

joint Sensing and Communication (JSAC) integration. With the demand of higher precision sensing (mm), lower communication latency (ms), and higher data rate (Tbps), the frequency bands for 

a transformative approach


play a pivotal role in realizing the full potential of 

the IoV faces pressing security challenges that need to be addressed urgently.

joint Sensing and Communication (JSAC) integration is a critical advancement for vehicular networks, which are central to the development of intelligent transportation systems.

by merging the capabilities of sensing and communication

for accurately detecting and understanding the surroundings 


ensure timely information exchange for critical applications

ensure seamless operation and data processing.

To achieve effectively and safely vehicle-to-infrastructure (V2I), autonomous driving, collision avoidance, and traffic management
\fi

\iffalse
It aims to facilitate information sharing among entities through vehicle-to-everything (V2X) communication and connectivity to the Internet, thus enhancing traffic safety and optimizing efficiency \cite{chen2019gas}.
These communications include V2V communication, vehicle-to-roadside (V2R) communication, vehicle-to-person (V2P) communication, etc \cite{sharma2019survey}, enabling information exchange and integration among multiple agents. To obtain road condition information and assist drivers in safe driving, vehicles in IoV are typically equipped with sensing devices such as cameras, millimeter-wave radar and LiDAR \cite{hasch2012millimeter}. The commonly used 77GHz automotive millimeter-wave radar utilizes emitted probing signals to measure target distance, velocity, and other motion information, enabling functions such as automatic parking, adaptive cruise control, lane departure warning, etc\cite{fang2020analysis}.

\textcolor{red}{the concept of the secrecy rate}



Data transmission processes are susceptible to tampering or forgery by unauthorized users, due to the broadcast nature of wireless communication. Additionally, attackers could exploit this vulnerability to impersonate authorized entities and disseminate false information to nearby vehicles \cite{gyawali2020challenges}. Communication signals in the IoV typically contain information related to traffic and safety, such as vehicle location, user information, and traffic conditions \cite{shi2023qkbaka}. The potential compromise of such information by eavesdroppers poses a significant threat to both traffic safety and user privacy.  Consequently, effective measures are imperative to mitigate the risk of data breaches and bolster security within the IoV ecosystem. While traditional communication protocols have relied on encryption to safeguard communication integrity, the extensive key management requirements and intricate encryption algorithms substantially elevate bandwidth consumption and latency in the IoV \cite{zheng2022physical}. 
\fi

\iffalse
In this context, the PLS scheme, which is designed to reduce the risk of data leakage by exploiting the characteristics of the physical layer. It has become a promising security solution. Unlike key-based security approaches, PLS technology, rooted in information theory \cite{shannon1949communication}, fundamentally prevents unauthorized users from eavesdropping on information, ensuring secure information transmission. Wyner introduced the concept of the wiretap channel model \cite{wyner1975wire}, leveraging channel differences between sender-legitimate receiver and sender-eavesdropper pairs, rendering eavesdroppers incapable of obtaining confidential information. PLS schemes provide additional protection for communication networks, serving as complementary technologies to traditional key-based security approaches, fundamentally mitigating the risk of unauthorized access to confidential information. Traditional PLS techniques employ strategies such as AN \cite{zhou2010secure}, relay-assisted security \cite{dong2009improving}, beamforming \cite{he2010cooperation}, and others. 
\fi

\iffalse
However, traditional AN scheme face three challenges. 1) Traditional PLS schemes overlook the impact of sensing signals on communication performance, which may lead to decreased communication performance and compromised security. Communication will utilize higher-frequency millimeter-wave bands in the future due to their low latency and large bandwidth characteristics \cite{song2021frequency}, especially for IoV scenarios that demand real-time responsiveness \cite{rappaport2017overview}. However, this may lead to overlapping between the millimeter-wave communication bands and millimeter-wave radar bands \cite{griffiths2014radar}. When electromagnetic wave bands are close or identical, interference will occur between the two signals. This means that not only will there be interference between communication signals, but also sensing signals will affect communication performance \cite{li2016optimum}. These interferences will result in increased bit error rate of communication signals, thereby reducing communication and sensing performance. 2) AN scheme also affects legitimate receivers while interfering with eavesdroppers. AN technology transmits artificially generated noise to interfere with eavesdroppers. Although the interference signal greatly reduces the risk of information leakage, it also causes some interference to legitimate receiving nodes because wireless signals are broadcasted in all directions. Especially in highly dynamic vehicular network scenarios, it is difficult to achieve beam alignment. 3) AN generated increases system energy consumption and also increases interference to vehicles in the vicinity.

Traditional PLS schemes must tackle three sub-problems to address the aforementioned challenges. Firstly, it is crucial to thoroughly consider the mutual influence between communication signals and perceptual signals, thereby constructing a PLS analysis framework that accounts for perceptual interference. Secondly, adopting effective solutions is essential to ensure that interference signals primarily impact eavesdropping nodes while minimizing disruption to legitimate receivers. Lastly, optimizing the utilization of internal interference signals is key to enhancing communication performance without introducing additional interference.
In addressing the interference exploitation issue, \emph{KabirIn et al.} regard multi-user interference as a valuable resource that can be used to provide energy and efficient power, thereby reducing system power consumption \cite{kabir2018robust}. Furthermore, they employ the use of downlink interference with the objective of saving energy in both the uplink and downlink in \cite{kabir2018interference}. In \cite{hong2021interference}, \emph{Hong et al.} propose a method that reverses the direction of destructive interference signals while preserving beneficial interference signals, utilizing known interference in radar and communication systems to increase signal power and consequently improve signal-to-noise ratio. The utilization of interference signals in these literatures serves to enhance the integrated communication-sensing performance within the system, thereby leading to energy savings. Interference exploitation schemes offer valuable insights into the PLS problem, although they have not been widely implemented thus far.

In the context of IoV networks considering the coupling between communication and sensing, we propose the IPLS scheme with eavesdroppers present, and investigate the impact of sensing signals on communication performance. To the best of our knowledge, this is the first paper to utilize sensing signals for achieving secure communication. Specifically, the primary contributions of this paper are as follows:
\begin{itemize}
  \item In response to the challenges faced by traditional PLS approaches, we propose a novel IPLS scheme. By utilizing signals generated from sensing signals to suppress eavesdroppers' channels, secure transmission of communication signals is achieved.
  \item This paper consider the interference of sensing signals on communication signals and construct the IPLS model using stochastic geometry methods. 
  We propose a new performance metric, joint sensing and communication based Secrecy Rate Maximization (JSAC-SRM), to measure the secrecy rate under the constraints of communication reliability, communication security, and sensing accuracy. Moreover, we derive closed-form expressions for connection outage probability (COP), secrecy outage probability (SOP), successful ranging probability (SRP) and JSAC-SRM.
  \item As communication signals are not continuously transmitting, it is necessary to optimize the timing of signal transmission in order to maximize the interference to eavesdroppers. Furthermore, the communication transmission power and driving trajectory of the transmitter are optimised in order to maximise the secrecy capacity.
  \item Simulation results validate the theoretical analysis of the IPLS scheme and demonstrate its superior security and reliability performance compared to PLS scheme.
\end{itemize}
\fi

The remainder of the paper is organized as follows: The related works are demonstrated in Section \ref{sec:related work}. In Section \ref{sec:network model}, the network model and performance metrics are presented, then the optimization for secrecy rate maximization is formulated. In Section \ref{sec:opt}, closed-form expressions of COP, SOP and SRP are derived at first, then communication duration of confidential information is optimized, and transmit power optimization and trajectory optimization are discussed, respectively. Simulations demonstrate the effectiveness of the proposed scheme in Section \ref{sec:simulation}. Finally, conclusions and future works are discussed in Section \ref{sec:conclusion}.


\iffalse
The rest of the paper is organized as follows. The related work is included in Section \ref{sec:related work}. In Section \ref{sec:network model}, we present the network model based on the IPLS. In Section \ref{sec:performance metrics}, we derive closed-form expressions for performance metrics. In Section \ref{sec:optimize}, we optimize the performance to maximize secrecy capacity and simulated results are shown in Section \ref{sec:simulation}. Finally, conclusion is given in Section \ref{sec:conclusion}.
\fi

\begin{table*}[t]  
    \caption{A Brief summary of the current PLS methods considering reliability, sensing and security}
    \centering
    \label{tab:methods}
    \begin{tabular}{|>{\centering\arraybackslash}m{1.5cm}|>{\centering\arraybackslash}m{3.5cm}|>{\centering\arraybackslash}m{2cm}|>{\centering\arraybackslash}m{1.0cm}|>{\centering\arraybackslash}m{1.0cm}|>{\centering\arraybackslash}m{5.2cm}|}  % 使用 m{} 实现上下左右居中
        \hline  % 顶部的边框
        \textbf{Reference}   & \textbf{Core ideas of PLS design}     & \textbf{Communication index}   & \textbf{Sensing index}  & \textbf{Security index}  & \textbf{Summaries} \\ \hline % 表头文字加粗
        \cite{wang2020physical}    & Joint design AN power and legitimate signal's BF    & $\checkmark$    & $\times$ & $\checkmark$ & \multirow{4}{5.2cm}{\centering 1) Secrecy enhancement empowered by communication interference; \\2) Extra energy consumption due to the generation of additional AN signals; \\3) Without considering the effects of communication-sensing coupled interference on the performance of communication, sensing, and security} \\ \cline{1-5}
        \cite{Zhang2019Transmit,yin2021uav} & Joint design of AN's power, legitimate signal's BF & $\checkmark$ & $\times$ & $\times$  & \\ \cline{1-5}
        \cite{Li2019UAV,Xu2021Low} & Joint design of AN's power and UAV's trajectory  & $\checkmark$ & $\times$ & $\times$  & \\ \cline{1-5}
        \cite{Zhou2018Improving} & Joint design of AN's power and UAV's trajectory  & $\checkmark$ & $\times$ & $\checkmark$  & \\ \cline{1-5}
         \cite{Chen2024Physical} & RIS/BF-based scenario & $\checkmark$ & $\times$ & $\checkmark$  & \\ \hline
         
        \cite{Liu2022Dynamic,Li2022Optimization} & $\times$ & $\times$ & $\times$ & $\times$  &  Optimized algorithms and driving safety \\ \hline
         \cite{Fang2020Stochastic,Zhang2024Coexistence} & $\times$& $\times$ & $\checkmark$   & $\times$ & without considering PLS \\ \hline

          \cite{chu2023joint} & Sensing signal-assisted BF, or AN-assisted BF & $\checkmark$ & $\checkmark$   & $\times$ & 1) PLS design based on sensing/AN signals; 2) Without considering the mobility of vehicles \\ \hline
          
         Method in this paper & \textbf{\emph{Sensing interference empowered joint design of transmission power and trajectory of vehicles}} & $\checkmark$ & $\checkmark$  & $\checkmark$ &
        1) PLS design based on sensing interference;
        2) Considering the vehicles' mobility, trajectory and moving time\\[1.2em] \hline
        %1) Using sensing signals to implement PLS; 2) Using existing signals within the system as interference signals; \textcolor{blue}{3) Considering dynamic scenarios} \\[2em] \hline
    \end{tabular}
\end{table*}
\section{Related Works}\label{sec:related work}

%\textcolor{red}{studies on AN, moving RIS, UAV moving trajectory optimization}

%\textcolor{red}{Traditional idea: Communication interference centered PLS design}
Wireless PLS technologies become particularly critical in the IoV, in terms of secure autonomous driving, vehicle-to-infrastructure. Concurrently, many efforts on the PLS design focused on the communication interference utilization by considering different legitimate and eavesdropping channels, such as the artificial noise (AN) scheme, joint design of the phase shift matrix and transmission power of reconfigurable intelligent surface (RIS), and the joint design of the transmission power and trajectory of the UAV.

\iffalse
\sout{Wireless PLS becomes particularly critical in the IoV, in terms of secure autonomous driving, vehicle-to-infrastructure communication, and emergency communication. Concurrently, many efforts on the PLS design focused on the communication interference utilization by considering different legitimate and eavesdropping channels. }From the perspective of communication interference utilization and trajectory optimization, the most representative methods includes AN-based BF design, joint design of the phase shift matrix and moving trajectory of IRS, joint design of the transmit power and trajectory of the UAV, and joint design of the BF and moving trajectory of the movable antenna (MA). 
\fi

\subsection{The communication interference based PLS in IoV systems}
AN scheme is one of most classic PLS techniques to effectively reduce the quality of eavesdropping channel, thereby enhancing the security.
In particular, \emph{Wang et al.} explored PLS in cellular vehicular networks, under which the legitimate user transmits confidential information while generating AN signals simultaneously for enhancing communication security \cite{wang2020physical}. Similarly, by design optimal AN-aided BF, \emph{Zhang et al.} proposed a Layered PLS model that minimizes transmission power while satisfying secrecy rate requirements \cite{Zhang2019Transmit}. The importance of attracting AN to the PLS were also emphasized in \cite{yin2021uav,jin2024enhanced}.
Featured by establishing line-of-sight (LoS) channels, the UAV and RIS provide new potential advantages in suppressing the wiretap channel. On the basis of this idea, in \cite{Li2019UAV}, \emph{Li et al.} regarded moving UAVs as jammers to generate AN signals for interfering with Eves, and jointly optimized the UAVs' trajectory and transmission power to enhance the security. \emph{Chen et al.} utilized RIS-assisted V2V communications and derived the upper bound of the secrecy capacity and the approximate expression of the secrecy outage probability under Rayleigh fading channel \cite{Chen2024Physical}.
It was demonstrated that optimizing UAV trajectory can effectively improve communication performance in \cite{Xu2021Low,Zhou2018Improving}.

\iffalse
\sout{By using the AN-assisted BF and optimizing power allocation between legitimate signals and AN signals, \emph{Jin et al.} emphasized the importance of attracting AN to the PLS \cite{jin2024enhanced}.Furthermore, considering the role of AN-BF in characterizing the trade-off between the reliability and security, they introduced the concept of effective secrecy throughput to quantify the average data rate of secure transmission of confidential information.} %Similar works have been also done in  \cite{Saqib2024reconfigurable,Liu2022Throughput,Li2023Multi}.
In addition, featured by line-of-sight (LoS), the UAV and RIS provide new potential advantages in suppressing the Eves' channel. On the basis of this idea, in \cite{Li2019UAV}, \emph{Li et al.} regarded moving UAVs as jammers interfering with Eves, and jointly optimized the UAVs' trajectory and transmit power for generating jamming signals to enhance the security. \textcolor{blue}{\emph{Chen et al.} utilized RIS-assisted vehicle-to-vehicle (V2V) communication and derived the upper bound of the secrecy capacity and the approximate expression for the secrecy outage probability (SOP) under double Rayleigh fading channels \cite{Chen2024Physical}.
Furthermore, in \cite{Li2019UAV,Xu2021Low}, it was demonstrated that optimizing UAV trajectories can effectively improve communication performance. \sout{Although various vehicles trajectory optimization schemes, such as \cite{Liu2022Dynamic, Li2022Optimization}, have been proposed for V2V communication, studies specifically leveraging vehicle trajectory optimization to enhance communication performance remain relatively scarce.} }
\fi

\iffalse
\sout{While \emph{Xu et al.} divided the UAV's transmit power into two parts: one for transmitting confidential signals and another one for generating AN signals. Next, they optimized the power allocation ratio and the UAV's trajectory to maximize the average secrecy rate \cite{Xu2021Low}. Furthermore, the impact of UAV' jamming power and positions on the reliability and security was studied in \cite{Zhou2018Improving}. In addition, the UAVs can be utilized as relays to generate the AN signals to confuse the Eves. Therefore, \emph{Yin et al.} applied this idea to  
achieve secure vehicle communications by optimizing the transmit power of UAVs and satellite transmission beams together \cite{yin2021uav}.}
\fi


However, in the context of ISAC-enabled IoV systems, communication and sensing frequency bands are often close, overlapping, or even identical, which results in communication-sensing coupled interference. %\cite{Zhang2024Coexistence,su2020secure}. 
Therefore, above PLS methods centered on communication interference cannot be directly extended to the secure information transmission in IoV systems featured by coupled interference. Moreover, although various vehicles trajectory optimization schemes, such as \cite{Liu2022Dynamic, Li2022Optimization}, have been proposed for V2V communication, studies specifically leveraging vehicle trajectory optimization to secure wireless communications remain relatively scarce.

In \cite{chu2023joint}, \emph{ Chu et al.} considered the design of 
PLS methods in an ISAC-based system. With the assumption of Eves' perfect CSI known beforehand, strong radar sensing signals were used to suppress the eavesdropping channel, then joint optimization of secure transmission BF and radar receiving filters was done for enhancing the system security. Otherwise, 
all available power resources of the base station and radar were utilized to design BF matrix and radar receiving filters for generating AN signals as much as possible to Eves. However, the aforementioned scenario is based on the assumption that all user locations are fixed, and legitimate users are highly susceptible to radar interference. This makes it difficult to apply to highly dynamic IoV scenarios.
In addition, many studies have revealed that optimizing the phase-shift matrix of RIS in IRS-enabled ISAC systems can extend their coverage \cite{qin2023joint,salem2022active}.
%\cite{zhang2022active,abeywickrama2020intelligent,qin2023joint}. 
Accordingly, \emph{Salem et al.} proposed a PLS solution to maximize achievable secrecy rate,
by jointly designing the receiving BF for radar, reflection coefficient matrix of RIS, and BS' transmitting BF \cite{salem2022active}.
Although these results point out that the sensing signals can assist transmission reliability, they do not address the communication security from the perspective of PLS constrained by communication-sensing coupled interference.

\subsection{Radar sensing interference modeling in ISAC-based IoV}

\iffalse
To describe radar sensing capability, \emph{Martin et al.} introduced the concept of \emph{interruption} into radar networks, using radar interruption as a performance metric defined as the situation where a radar network cannot detect a specific object due to interference from others \cite{braun2013co}. \emph{Brooker} showed a high probability of interference in overlapping frequency bands by studying interference under different conditions and sensor types. Furthermore, they investigated mutual interference between mmWave radar systems operating in the 77 GHz and 94 GHz frequency bands \cite{brooker2007mutual}. Researchers have developed mathematical models to gain a deeper understanding of the mutual interference between automotive radars and to predict the degree of sensing interference in different scenarios. In \cite{al2018stochastic}, \emph{Al-Hourani et al.} pioneered the modelling of automotive radar interference based on stochastic geometry tools. They analyzed the interference between opposing lane radars using Poisson and lattice models, and proposed to estimate the probability of success of radar ranging based on closed-form interference statistics. In \cite{Fang2020Stochastic}, \emph{Fang et al.} studied radar interference in bidirectional multilane scenarios and modelled target radar cross section (RCS) fluctuations using Swerling I and Chi-Square models, obtaining closed-form expressions for radar SRP. These studies provide the foundation for the integration of communication and sensing.
\fi

To measure the performance of the sensing capability, in \cite{braun2013co}, \emph{Martin et al.} introduced the concept of \emph{interruption} into radar networks, which refers to the situation where a radar network cannot detect a specific object due to interference from others. A high probability of interference exists in overlapping frequency bands was proved by \emph{Brooker}, then mutual interference between mmWave radar systems operating in 77 GHz and 94 GHz frequency bands was investigated \cite{brooker2007mutual}. Subsequently, many researches have developed mathematical models to make a deeper understanding of the mutual interference among radars and predict the degree of sensing interference in different scenarios. 
For instance, 
in \cite{al2018stochastic}, \emph{Al-Hourani et al.} pioneered the modelling of automotive radar sensing interference based on stochastic geometry tools. They analyzed the interference between opposing lane radars using Poisson and Lattice models and estimated the SRP based on closed-form interference statistics. 
In \cite{Fang2020Stochastic}, \emph{Fang et al.} studied radar sensing interference in bidirectional multilane scenarios and modelled target radar cross section (RCS) fluctuations using Swerling I model and Chi-Square model, obtaining closed-form expressions for radar sensing SRP. These studies provide the foundation for the integration of communication and sensing in IoV systems.

To better understand communication-sensing coexisting networks, in \cite{Zhang2024Coexistence}, considering radar and communication systems, \emph{Zhang et al.} focused on joint design of communication precoders, radar transmit waveforms, and receiving filters to suppress mutual interference between communication and sensing. 
To deal with the communication-sensing coupled interference, in \cite{su2020secure}, \emph{Su et al.} introduced AN scheme at the source to minimize the SINR at sensing radar, thereby ensuring the expected signal strength of legitimate users. However, the transmit power is increased by introducing external AN into the system. To mitigate power cost, \emph{Lynggaard} employed the assumption of perfect CSI to predict the minimum transmit power required to overcome interference, at the cost of an increased computational complexity \cite{lynggaard2018using}. From above works, to some extent, we can find that although the proposed methods have improved the performance of communication and sensing, additional power consumption and computational burden were also introduced.

Table \ref{tab:methods} summaries the existing PLS methods and highlights the differences between our proposed and other works.
When applying the radar sensing interference in PLS design in IoV, the sensing accuracy, transmission reliability and security should be considered simultaneously. Therefore, on the one hand, the performance analytical framework of ISAC systems considering communication-sensing coupled interference should be established, which describes the relationship between communication and sensing capabilities. On other hand, from the perspective of radar sensing interference utilization, no studies investigate the confidential communication by jointly designing transmission power and trajectory of the source.
This motivates our research in this work. 
%\iffalse
In addition, the difference between our work and previous studies is described as follows.
\begin{itemize}
  \item Different from PLS methods based on AN schemes \cite{wang2020physical,Zhang2019Transmit,Xu2021Low}, which did not consider the communication-sensing coupled interference, we make an effective transformation from ``communication interference based security'' to ``sensing interference based security'';
  \item In the context of ISAC-IoV, different from using sensing/AN interference in studies of PLS design \cite{salem2022active,su2020secure}, only the sensing interference is utilized to secure wireless communications, without extra energy resources;
  %we fully utilize the perceived interference signals within the system to suppress the eavesdropping channel capacity, thereby ensuring the secure transmission of communication signals. Different from the same frequency interference suppression scheme in \cite{goppelt2011analytical}\cite{su2020secure}\cite{lynggaard2018using}, this approach aims to make full use of existing internal signals to enhance system performance.
  %\item \textcolor{green}{Due to significant interference caused by sensing signals, the current popular communication interference metrics cannot be used to evaluate sensing-enabled ISAC networks. We provide a new method for measuring communication reliability and sensing accuracy, taking into account the impact of sensing performance on communication.}
  \item Instead of transmitter trajectory optimization \cite{Li2019UAV,Xu2021Low,Sun2021Unmanned}, the duration of confidential information transmission constrained by limited horizontal angle of Carols' sensing beamwidth, the trajectory and transmission power of the Alice are considered simultaneously.
\end{itemize}
%\fi


\section{Network Model and Preliminaries}\label{sec:network model}
To ensure safety driving, each vehicle posses capabilities of sensing and communication working at the same mmWave frequency band. 
In this way, vehicles can sense, monitor and gather the information of a given object or surroundings, and further share the information by communicating with each other. Moreover, there exists coupling-interference between sensing and communications. As shown in Fig. \ref{fig:network model}, given a two-way four-lane scenario, a legitimate source vehicle (called as Alice)
%with the position of $[x_{\rm Alice}, y_{\rm Alice}]^\textbf{T}$ 
transmits the confidential information to an intended receiving vehicle (called as Bob) within the communication duration of $T$, in the presence of multiple Eves, which try to intercept the transmitted information from the Alice. In particular, the Alice, Bob and Eves drive in the same direction. In addition, other legitimate vehicles, moving in opposite direction with the Alice, Bob and Eves and acting as interferers (called as Carols) since they generate sensing signals by using the forward-sensing radars can decrease the quality of legitimate channel and eavesdropping channel, sense the target ahead by using their forward-sensing radars and have no communication requirement. Without loss of generality, it is assumed that the positions of Eves and Carols follow the independent Poisson Point Process (PPP) with the densities of $\lambda_i$ and $\lambda_e$, and the sets of corresponding positions are denoted by $\Phi_i$ and $\Phi_e$, respectively. 
%For the $i$-th interferer and the $j$-th Eve, the positions can be presented as $[x_{i,\rm Inter}, y_{i, \rm Inter}]^\textbf{T}$ and $[x_{j,\rm Eve}, y_{j, \rm Eve}]^\textbf{T}$, respectively.
 

\begin{figure}[!ht]
\centering
\includegraphics[width=2.6in]{results/system_model_11_11.jpg}
\caption{\small System model in an ISAC-IoV}
\label{fig:network model}
\end{figure}

A simplified version of Fig. \ref{fig:network model} is given in Fig.\ref{fig:scenario}, a Cartesian coordinate system can be established with the Alice, Bob, and Eves moving along the positive $x$-axis, while Carols moving along the negative $x$-axis. 
All vehicles are assumed to travel in the center of the road. To ensure the road safety, the width of each lane is denoted by $D_{\rm lane}$ and affordable minimum following distance between any two vehicles is $D^{\min}_{\rm foll}$. The horizontal distance and vertical distance between the Bob and the $i$-th Carol are denoted as $X^{\rm hor}_{c_i\rightarrow b}$ and $Y^{\rm ver}_{c_i\rightarrow b}$, respectively. Similarly, the horizontal distance and vertical distance between the $i$-th Carol and the $j$-th Eve are denoted as $X^{\rm hor}_{c_i\rightarrow e_j}$ and $Y^{\rm ver}_{c_i\rightarrow e_j}$, respectively. 
%In addition, $P_t$ and $P_s$ denote the transmit power and sensing power of the vehicles. 
Due to the limited sensing power, the maximum sensing distance and maximum horizontal angle of sensing beamwidth of the vehicles are limited, which are denoted as $R_{\max}$, and $\theta$, respectively. 

In addition, for better describing physical layer security, we discretize the time duration $T$ into $N$ time slots, indexed by $k$, with the length of $T /N$, under which the transmitting/sensing signal is generated at the begin of the time slot, and received by the vehicles at the end of the current time slot or another one. It is assumed that the Alice transmits the confidential information at each time slot $1 \leq k \leq T /N$. The speed and acceleration at the $k$-th time slot of the $i$-th vehicle are denoted by $v_i[k]$ and $a_i[k]$, and allowable maximization ones are represented as $v_{\max}$ and $a_{\max}$, respectively.
\iffalse
We consider a self-organizing network composed of legitimate vehicles, eavesdroppers, and interfering vehicles situated in a two-dimensional space, as illustrated in Fig. \ref{fig:network model}. The system model is constructed using random geometric methods. It is assumed that all vehicles possess sensing capability. Specifically: 
1) Communication Functionality: Legitimate transmitting vehicle intend to send communication signals to the only legitimate receiver. Nevertheless, there are illegal users who attempt to eavesdrop on confidential information. Assuming that the transmitter vehicle is equipped with a single antenna for communication with power $q$, and communicates using the 77 GHz band. Assuming the positions of eavesdroppers and interferers are simulated using independent Poisson Point Processes (PPP). Without loss of generality, let the sets of positions for the transmitter and eavesdropper be $\Phi_a $ and ${\Phi}_e $ respectively, with densities $\lambda_a $ and ${\lambda }_e$. Due to the presence of various obstacles during the communication and sensing process, Rayleigh fading is used to simulate the channel fading. 
2) Sensing Functionality: Suppose all vehicles are equipped with millimeter-wave radars. The forward-facing radar utilises a 77GHz medium-range radar with a horizontal beamwidth of $\theta =60{}^\circ$ and a maximum detection range of $R_{max}$. Due to the fact that all forward-facing mmwave radars operate on the same frequency band, the overlapping frequency band ranges between two vehicles' forward-facing radars result in sensing interference. The communication frequency band is the same as the forward radar sensing frequency band, leading to potential interference from the radar sensing signals on the communication. Vehicles share an antenna for transmission and reception during sensing. Let ${\Phi }_i$ denote the set of vehicles traveling in the same and opposite direction as Eve, with The density of interfering vehicles ${\lambda }_i$. All forward-facing radars transmit signals with power $P$. In addition, due to the numerous impediments encountered in the communication sensing process, it is necessary to simulate the channel condition using Rayleigh fading. Table \ref{tab:terms and meanings} presents several significant notations and their corresponding meanings.
\fi
\begin{figure}[!ht]
\centering
\includegraphics[width=2.2in]{results/Scenario}
\caption{\small A simplified version of an ISAC-IoV}
\label{fig:scenario}
\end{figure}

\iffalse
In a specific communication scenario, illustrated in Fig.\ref{fig:scenario}, the legitimate user (Alice) transmits communication signals to the legitimate receiver (Bob), while there are eavesdroppers intending to intercept confidential information. In the opposite lane, there are many interfering vehicles, and their sensing signals will affect the reception of communication signals. Assuming perfect knowledge of the channel state information (CSI) between the legitimate user and eavesdroppers, a Cartesian coordinate system can be established with the Alice as the origin $\left( 0,0 \right)$ and the direction of the its movement as the positive $x$-axis. The coordinates of the vehicles in the system are designated as $\left( x,y \right)$. The width of the lane is represented by the variable $D$, and all vehicles are assumed to travel in the centre of the road, with their own widths neglected. The horizontal distances between Bob and  a specific interfering vehicle, Carol, on the $x$-axis and $y$-axis are designated as $X_{bc}$ and $Y_{bc}$, respectively. The horizontal distances between the eavesdropper Eve and Carol on the $x$-axis and $y$-axis are denoted as $X_{ec}$ and $Y_{ec}$, respectively.
\fi

%对向行驶车辆的雷达系统会直接对本车的雷达系统产生的干扰信号, 强度高且频繁出现,因为它们的雷达信号是正面对准的。首先考虑这种干扰源,有助于确保模型能够处理最严重和最常见的干扰情况。非目标雷达回波包括来自道路旁建筑物、路标以及其他环境物体的反射信号。这些回波信号通常较弱,且多为短暂的间歇性干扰,可以通过滤波和信号处理技术进行有效抑制。因此,可以合理地忽略这种干扰,以简化模型并专注于主要干扰源的处理。


\subsection{Radar sensing model}
As shown in Fig. \ref{fig:scenario}, when the Alice senses and monitors the target Bob ahead by using radar echo, the precision of radar performance is affected by two kinds of interfering signals: 1) the radar echo from non-intended target; 2) radar echos from Carols in the opposite lances. For the sake of simplified analysis, only the latter interference is focused \cite{Fang2020Stochastic,Ghatak2022Radar}\footnote{In radar systems of the vehicles, the radar echoes generated by vehicles in the opposite lanes directly affect the radar system of the current vehicle.
On the one hand, these non-intended echoes are strong and occur frequently because their radar signals are directly aligned with each other. Prioritizing them helps ensure that the model can address the most severe and common interference scenarios.
On the other hand, non-target radar echoes include reflected signals from buildings, road signs, and other environmental objects along the road. These echo signals are usually weak and often present as short, intermittent interference. They can be effectively mitigated through filtering and signal processing techniques. Therefore, it is reasonable to ignore this type of interference to simplify the model and focus on addressing the primary sources of interference.}.
Generally, the RCS of the target is employed to ascertain the extent being detected. Due to the mobility of the vehicles, the radar echo from a target is fluctuating from scan to scan, which is vital to the precision of radar performance. From the conclusion in \cite{lewinski1983nonstationary}, it can be noticed that Chi-square model is a reasonable model to evaluate the fluctuation effectively.
The probability density function (PDF) of degree 2 $\delta$ under Chi-square model can be expressed as
\begin{equation} \label{eq:p_sigma }
p \left( \sigma  \right) =\frac{\delta }{\Gamma \left( \delta \right){\bar{\sigma }_{av}}^{\\delta}}{{\left( \sigma  \right)}^{\delta-1}}\exp \left( -\frac{\delta}{{\bar{\sigma }_{av}}}\sigma  \right),\sigma \ge 0,
\end{equation}
where $\delta$ is the degree of freedom (DoF), and $\bar{\sigma}_{av}$ denotes and average value of the RCS. To simplify the notation, the distance between the vehicle and its sensing target is denoted as $R_{\rm{tar}}$.
Moreover, based on the conclusion in \cite{series2014systems}, for a given RCS of the target, denoted by $\sigma$, the radar echo of a vehicle to the desired target can be written as 
%\begin{equation}
%{P}_{\text{each}}={\frac{P_{\rm sen}{{G}_{t}}{{G}_{r}}{{\lambda }_{w}}^{2}\sigma }{{{\left( 4\pi  \right)}^{3}}{{\textcolor{red}{R_\rm{tar}}}}^{2\alpha }}} ,
%\end{equation}
\begin{displaymath}
{P}_{\text{echo}}=\frac{P_{\rm sen}G_t G_r \lambda^{2}_{w}\sigma}{(4\pi)^3{R_{\rm tar}}^{2\alpha}},
\end{displaymath}
where $G_t$ and $G_r$ represent the gains of the radar transmitting antenna and receiving antenna, respectively, $\lambda _{w}$ denotes the wavelength of the sensing signal, $\alpha$ represents the path loss exponent, $P_{\rm sen}$ is the sensing power of the vehicles.

\iffalse
The sensing of vehicles equipped with radar devices generates interference to signals on the same frequency. When Alice communicates using millimeter-wave signals on the same frequency band as the forward-facing radar, it is susceptible to interference from vehicles in the opposite lane. These interferences may originate from direct interference of sensing signals from vehicles in the opposite lane or from reflections or scattering signals from other lanes. For the sake of calculation, this paper only considers the former. 
\fi

%\textcolor{red}{What is the relationship between Eq. (1) and Eq. (2)?}



Due to vehicles' mobility, as well as limited sensing distance and horizontal sensing beamwidth of the radars in vehicles, the channel qualities of the Bob and Eves are time-varying. From the perspective of the physical layer security, the Bob does not suffer from the sensing interference generated by the Carols, while the Eves are covered by radar sensing interference as much as possible. Accordingly, on the one hand, the horizontal distance the Bob and the $i$-th Carol should satisfy the condition 
\begin{equation}\label{eq:distance bob without interference}
X^{\rm hor}_{c_i\rightarrow b}> \sqrt{R_{\max}^2 - {Y^{\rm hor}_{c_i\rightarrow b}}{^2}} ~~\text{or}~~ X^{\rm hor}_{c_i\rightarrow b}< \frac{Y^{\rm hor}_{c_i\rightarrow b}}{\tan \theta/2}.
\end{equation}
Otherwise, the Bob will be affected by Carol's sensing signals. On the other hand, to position the Eves within the Carols' sensing range, the horizontal distance between the $i$-th Carol and the $j$-th Eve should fulfil the following condition 
\begin{equation}\label{eq:distance eve being interfered}
\frac{Y^{\rm hor}_{c_i\rightarrow e_j}}{\tan\theta/2}\le X^{\rm hor}_{c_i\rightarrow e_j} \le \sqrt{{{R}_{\max }}^{2}-{Y^{\rm hor}_{c_i\rightarrow e_j}}^{2}}.    
\end{equation}
\iffalse
%To improve the SINR at the legitimate receiver, Bob endeavors to minimize interference from Carol. Neglecting signal propagation delay, the horizontal distance between Bob and Carol should satisfy condition ${{X}_{bc}}\ge \sqrt{{{R}_{\max }}^{2}-{{Y}_{bc}}^{2}}$ or ${{X}_{bc}}\le \frac{{{Y}_{bc}}}{\tan \frac{\theta }{2}}$. Otherwise, Bob may be affected by Carol's sensing signals. Conversely, it is preferable to position Eve within Carol's sensing range in order to disrupt its activities, thus maintaining secure transmission of communication signals. Consequently, the horizontal distance between the two should ideally fulfil the conditions of condition $\frac{{{Y}_{ec}}}{\tan \frac{\theta }{2}}\le {{X}_{ec}}\le \sqrt{{{R}_{\max }}^{2}-{{Y}_{ec}}^{2}}$. 
\fi
%%%%%%%%%%%%%%%%%%%%%%%%%%%%
In addition, considering the sensing accuracy of the Alice, When the Alice is sensing and monitoring the Bob, the condition that it does not suffer from potential sensing interference generated by the Carols should satisfy
\begin{equation}\label{eq:distance alice being interfered}
{{X}^{\rm hor}_{c_i\rightarrow a}}> \sqrt{{{R}_{\max }}^{2}-{{Y}^{\rm ver}_{c_i\rightarrow a}}^{2}} ~~ \text{or} ~~{{X}^{\rm hor}_{c_i\rightarrow a}}< \frac{{{Y}^{\rm ver}_{c_i\rightarrow a}}}{\tan \theta/2}.
\end{equation}

Let ${{\Phi }_{c_i\to b}}=\left\{c_i:c_i\in {{\Phi }_{i}},\mathcal{M}\left( i \right)=1 \right\}$, ${{\Phi }_{c_i\to e_j}}=\left\{c_i:\right.$ $\left.c_i\in {{\Phi }_{i}},\right.$$\left. e_j\in\Phi_e,\mathcal{H}\left( i, j \right)=1 \right\}$, and ${{\Phi }_{c_i\to a}}=\left\{c_i:c_i\in {{\Phi }_{i}},\mathcal{F}\left( i \right)=1 \right\}$ denote the sets of Carols covering the Bob, the $j$-th Eve, and the Alice by using sensing signals, respectively, where $\mathcal{M}\left( i \right)$, $\mathcal{H}\left( i, j \right)$, and $\mathcal{F}\left( i \right)$ are indicator functions and given by 
\begin{displaymath}
\mathcal{M}\left( i \right)= \begin{cases}
  & 1,~~\text{otherwise;} \\ 
 & 0,~~\text{satisfying the Eq. \eqref{eq:distance bob without interference};}\\ 
\end{cases} 
\end{displaymath}
\begin{displaymath}
\mathcal{H}\left( i,j \right)= \begin{cases}
  & 0,~~\text{otherswise;}\\ 
  & 1,~~\text{satisfying Eq. \eqref{eq:distance eve being interfered};}
\end{cases}
\end{displaymath}
and 
\begin{displaymath}
\mathcal{F}\left( i \right)= \begin{cases}
  & 1, ~~\text{otherwise;} \\ 
 & 0, ~~\text{satisfying the Eq. \eqref{eq:distance alice being interfered};}\\ 
\end{cases}
\end{displaymath}
respectively. Based on the results in \cite{Fang2020Stochastic}, the cumulative interference caused by Carols in the opposing lane to the Bob and the $j$-th Eve can be represented as
\begin{displaymath}
{{I}_{\rm Bob}}=\sum\limits_{c_i\in {{\Phi }_{c_i\to b}}}{{{A}_{\rm ea}}\cdot S\cdot h_{c_i\rightarrow b}}\cdot{\Vert(X^{\rm hor}_{c_i\rightarrow b},Y^{\rm ver}_{c_i\rightarrow b})\Vert}^{-\alpha },
\end{displaymath}
\begin{displaymath}
{{I}_{e_j}}=\sum\limits_{c_i\in {{\Phi }_{c_i\to e_j}}}{{{A}_{\rm ea}}\cdot S\cdot h_{c_i\rightarrow e_j}}\cdot{\Vert(X^{\rm hor}_{c_i\rightarrow e_j},Y^{\rm ver}_{c_i\rightarrow e_j})\Vert}^{-\alpha },
\end{displaymath}
and
\begin{displaymath}
{{I}_{\rm Alice}}=\sum\limits_{c_i\in {{\Phi }_{c_i\to a}}}{{{A}_{\rm ea}}\cdot S\cdot h_{c_i\rightarrow a}}\cdot{\Vert(X^{\rm hor}_{c_i\rightarrow a},Y^{\rm ver}_{c_i\rightarrow a})\Vert}^{-\alpha },
\end{displaymath}
where ${{A}_{\rm ea}}=\frac{{{G}_{r}}{{\lambda }_{\omega }}^{2}}{4\pi }$ denotes the effective aperture of the radar receiver, ${S}=\frac{{P}{G_t}}{4\pi }$ is the power density at the unit distance from the interfering source, $h$ is the small scale fading caused by multi-path propagation \cite{Fang2020Stochastic}. The path loss is proportional to
${\Vert(X^{\rm hor}_{c_i\rightarrow b},Y^{\rm ver}_{c_i\rightarrow b})\Vert}^{-\alpha}$, where $\alpha \in [2, 6]$ denotes the path-loss exponent, and $\Vert\cdot\Vert$ refers to the Euclidean distance between the Carol and the the typical vehicle (i.e., Bob, Eve, and Alice).

\iffalse
When the Alice is sensing and monitoring the Bob, it suffers from potential sensing interference generated by the Carols, affecting the sensing accuracy. Due to the limited maximum sensing distance and horizontal sensing beamwidth of the Carols, 
let ${{\Phi }_{i\to a}}=\left\{c_i:c_i\in {{\Phi }_{i}},\mathcal{F}\left( i \right)=1 \right\}$ be the set of Carols that the Alice is covered by their sensing signals, where
\begin{equation}
\mathcal{F}\left( i \right)= \begin{cases}
  & 1, \text{otherswise;} \\ 
 & 0,{{X}_{ac}}\ge \sqrt{{{R}_{\max }}^{2}-{{Y}_{ac}}^{2}} ~ \text{or} {{X}_{c_i\rightarrow a}}\le \frac{{{Y}_{c_i\rightarrow a}}}{\tan \frac{\theta }{2}}.\\ 
\end{cases}
\end{equation}
And the interference experienced by Alice in sensing is represented as
\begin{equation}
{{I}_{a}}=\sum\nolimits_{i\in {{\Phi }_{i\to a}}}{{{A}_{e}}Sh}{{L}_{ai}}^{-\alpha }.
\end{equation}

The SINR is defined as the ratio of the received power of the desired signal to the combined interference power (noise and interference) at the receiver. Therefore, the psensing SINR at Alice can be represented as 
\begin{equation}
{ {\gamma}^{\rm sen}_{\rm Alice}  }=\frac{P^{echo}_a}{\sigma^2_a+\sum\nolimits_{i\in {{\Phi }_{i\to a}}}{{{A}_{e}}Sh}{{L}_{ai}}^{-\alpha }}
\end{equation}
\fi


\subsection{Communication model}
When the Alice transmits the confidential information to the Bob, the latter suffers from the potential sensing interference generated by the Carols. In addition, the Bob also receives its radar echo signal using sensing and monitoring the target. As a result, the received signal-interference-noise ratio (SINR) at the Bob can be expressed as
\begin{equation}\label{eq:sinr Bob}
{ {\gamma}^{\rm com}_{\rm Bob}  }=\frac{P_{\rm com}h_{a\rightarrow b} {\Vert(X^{\rm hor}_{a\rightarrow b},Y^{\rm ver}_{a\rightarrow b})\Vert}^{-\alpha}}{\sigma^{2}_{b}+ P^{\rm Bob}_{\rm echo} + I_{\rm Bob}},
\end{equation}
where $P_{\rm com}$ denotes the transmit power of the Alice, $h_{a\rightarrow b}$ denotes the small scale fading between the Alice and the Bob, $\sigma^{2}_{b}$ is the additive white gaussian noise (AWGN) at the Bob, and $P^{\rm Bob}_{\rm {echo}}$ refers to the radar echo power from the itself.

Similarly, when the $j$-th Eve tries to intercept the transmitted message, the received SINR at the $j$-th Eve can be represented as
\begin{equation}
{ {\gamma}^{\rm com}_{e_j}}=\frac{P_{\rm com}h_{a\rightarrow e_j} {\Vert(X^{\rm hor}_{a\rightarrow e_j},Y^{\rm ver}_{a\rightarrow e_j})\Vert}^{-\alpha}}{\sigma^{2}_{e_j}+ P^{e_j}_{\rm {echo}} + I_{e_j}},
\end{equation}
where $h_{a\rightarrow e_j}$ denotes the small scale fading between the Alice and the $j$-th Eve, $\sigma^{2}_{e_j}$ is the AWGN at the $j$-th Eve, and $P^{e_j}_{\rm {echo}}$ refers to the radar echo power from the itself.

\subsection{Assumption-Effectiveness of the Perfect CSI}
The sensing capability helps acquire the CSI and positions of the vehicles accurately, thereby enhancing the communication performance \cite{yu2023integrated}. Similar channel estimation techniques recently have extended to the studies of IoV \cite{chu2023joint,lynggaard2018using}. Hence, we assume the availability of perfect CSI for the Alice-Bob, during the whole transmission period. On the other hand, for the Eve, to better intercept
the confidentiality information transmitted by Alice, it is more likely that the Carols can sense the Eves by using their forward-sensing radars. Accordingly, information sharing among legitimate vehicles can facilitate channel estimation of the Alice-Eve. Therefore, the acquired CSI is expected to be also perfect. Moreover, apart from forward-sensing radars, the vehicles typically integrate various sensors (e.g., smart cameras and light detection and ranging) \cite{Sengupta2019A}, the perfect CSI of the Bob and Eve can be also obtained.

\iffalse
Although Eve can also obtain Alice's CSI to intercept information, Alice can further optimize trajectories through sideway radar. Additionally, Carol can obtain the positions of Bob and Eve through detection. When Bob moves out of Carol's sensing range, Carol can strengthen interference against Eve. Apart from forward mmWave radar, vehicles may also be equipped with cameras and other devices. Through the integration of various sensors, perfect CSI of Bob and Eve can be obtained.
\fi


\iffalse
Only one transmitter is considered, and it is not affected by AN interference in this scenario. Therefore, neither Bob nor Eve experiences interference from other communication signals. While Bob and Eves receive communication signals, they also experience sensing echo signals generated by their own detection of road conditions. Consequently, it can be expressed as follows
\begin{equation}\label{eq:sinr Bob}
{ {\gamma}^{sen}_b  }=\frac{qh_{ab}{{\left| {{L}_{ab}} \right|}^{-\alpha }}}{W_b+ P^{\rm{echo}}_b + I_b}
\end{equation}
\begin{equation}\label{eq: sinr eve}
{ {\gamma}^{sen}_e  }=\frac{qh_{ae}{{\left| {{L}_{ae}} \right|}^{-\alpha }}}{W_e+ P^{\rm{echo}}_e + I_e}
\end{equation}
where $L_{ab}$ represents the distance between the signals of Alice and Bob, $L_{ae}$ represents the distance between the signals of Alice and Eve, $W_b$ and $W_e$ denote the noise power received by Bob and Eve, respectively. $P^{\rm{echo}}_b$ and $P^{\rm{echo}}_e$ denote their own sensing echo power since all vehicles are continuously sensing in real-time.
\fi

\subsection{Performance metrics: transmission reliability and security, sensing accuracy, and secrecy rate}
\textbf{The connection outage probability (COP)}: Based on the SINR expression in Eq. \eqref{eq:sinr Bob}, the transmission reliability can be measured by determining the COP that the transmitted information from Alice fails to be received successfully at the Bob. Mathematically, the COP can be described as \cite{Yu2024SuRLLC}
\begin{equation} \label{eq:P_co}
{p_{co}}=1 - \mathrm{Pr}\left[ {\gamma^{\rm com}_{\rm Bob} \ge {\beta _b}} \right],
\end{equation}
where $\beta_b$ denotes a given threshold for decoding confidential information determined by the hardware.

\iffalse
and Eq. \eqref{eq: sinr eve},  

\subsubsection{COP} 
The COP represents the probability of message failure to be received at the Bob. It is defined as the probability that the legitimate receiver is unable to decode the message due to their channel capacity falling below a predefined threshold ${\beta}_b$. The aforementioned metric is employed to ascertain the reliability of the communication. 
The COP at Bob can be represented as
\begin{equation} \label{eq:P_co}
\begin{aligned}
{p_{co}}& =  \mathrm{Pr}\left[ {\gamma_b^{com} < {\beta _b}} \right]\\
&=1 - \mathrm{Pr}\left[ {\gamma_b^{com} \ge {\beta _b}} \right]
\end{aligned}
\end{equation}
\iffalse
\begin{equation} \label{eq:COP0}
{P_{co}} = \mathrm{Pr}\left[ {\gamma_b^{com} \le {\beta _b}} \right],
\end{equation}
where $\gamma_b^{com}$ represents the communication SINR at Bob. Representing the threshold $\beta _b$ as
\begin{equation} \label{eq:beta_b}
{\beta _b} = 2^{R_b} -1,
\end{equation}
\fi

\fi

\textbf{The secrecy outage probability (SOP)}: To describe the security level that the Eves intercept the confidential information successfully, the SOP is introduced and refers to that at least an Eve's channel capacity exceeds a given decoding threshold $\beta _e$. Mathematically, the SOP can be represented as
\begin{equation}\label{eq:concept of sop}
{p_{so}} = 1 - \mathrm{Pr}\left[ {\gamma^{\rm com}_{e_j} < {\beta _e}|\text{for all $e_j\in \Phi_e$}} \right].
\end{equation}



\iffalse
Representing the threshold $\beta _e$ as
\begin{equation} \label{eq:beta_e}
{\beta _e} = 2^{R_e} -1,
\end{equation}
\fi

\textbf{The success range probability (SRP)}: Similarly, to evaluate the sensing accuracy level of the Alice, the SRP refers to  the probability of successfully detecting an intended target. That is, the SRP at Alice can be expressed as
\begin{equation} \label{eq:SRP0}
{{p}_{sr}}=\mathrm{Pr} \left[ \gamma^{\rm sen}_{\rm Alice}>\beta_s \right],
\end{equation}
where $\beta_s$ is a given sensing threshold of the Alice determined by the hardware, and $\gamma^{\rm sen}_{\rm Alice}$ is represented as
\begin{displaymath} 
{\gamma}^{\rm sen}_{\rm Alice} =\frac{P^{\rm Alice}_{\rm{echo}}}{\sigma^2_a+\sum\limits_{c_i\in {{\Phi }_{c_i\rightarrow a}}}{{{A}_{\rm ea}}Sh_{c_i\rightarrow a}}{\Vert (X^{\rm hor}_{c_i\rightarrow a}, Y^{\rm ver}_{c_i\rightarrow a})\Vert}^{-\alpha }}.
\end{displaymath}


%\subsubsection{The transmission Reliability and Sensing Accuracy based Secrecy Rate ($TRSA\_SR$) in the Worst Case} 
\textbf{The transmission reliability and sensing accuracy based secrecy rate ($\rm TRSA\_SR$) in worst case}: The secrecy rate is an intuitive performance metric that describes channel dominance \cite{Vuppala2018On}. When the secrecy rate is positive, it means that the perfect secrecy can be achieved. In this paper, to simultaneously ensure both the expected performances in terms of communication and sensing, we define the concept of the $\rm TRSA\_SR$ in the worst case, which is represented as
\begin{equation}
\tau[k]  =( 1-p_{co}){{p}_{sr}}\frac{1}{N}\sum\limits_{k=1}^{N} \left[ {R_{\rm Bob}[k]}-\max_{e_j\in\Phi_e}{{R_{e_j}[k]}}\right]^+,
\end{equation}
where $R_{\rm Bob}[k]=\log_2(1 + \gamma^{\rm com}_{\rm Bob})$ represents the channel capacity from the Alice to the Bob, and $R_{e_j}[k]=\log_2(1+\gamma^{\rm com}_{e_j})$ represents the channel capacity from the Alice to the $j$-th Eve at $k$-th time slot, $a=[b-c]^+$ indicates that $a=b-c$ if $a\geq b$, and $a=0$ otherwise. When the secrecy rate is negative, the transmit power is set to zero. 

\iffalse
The original STC only ensures the success probability of communication performance. It is described as the achievable rate of successful transmission of confidential messages per unit network area. However, in IoV, there are numerous sensing signals, and it is essential to consider both the success probability of communication success and the probability of sensing detection success comprehensively. 

Success Range Probability (SRP)
Consequently, we propose JSAC-SRM as a performance index to assess the effective data transmission rate achievable by the communication system, while simultaneously ensuring both communication security and sensing accuracy. It can be expressed as
\begin{equation}
\tau  =( 1-P_{co}){{P}_{sr}}\frac{1}{N}\sum\limits_{k=1}^{N} \left[ {R_{\rm Bob}[k]}-\textcolor{red}{\max_{e_j\in\Phi_e}{{R_{e_j}[k]}}} \right]^+,
\end{equation}
where $R_{\rm Bob}=\log_2(1 + \gamma^{\rm com}_{\rm Bob})$ represents the channel capacity from the Alice to the Bob, and $R_{e_j}=\log_2(1+\gamma^{\rm com}_{e_j})$ represents the channel capacity from the Alice to the $j$-th Eve at $k$-th time slot, $a=[b-c]^+$ indicates that $a=b-c$ if $a\geq b$, and $a=0$ otherwise.

$R_s = R_b -R_e$ represents the difference in transmission rates between legitimate receivers and eavesdroppers, i.e., the secrecy rate. ${{\left[ {{R}_{t}}-{{R}_{e}} \right]}^{+}}=\mathrm{\max} \left\{ {{R}_{t}}-{{R}_{e}},0 \right\}$, and when the secrecy rate is positive, secure transmission is achievable. When the secrecy rate is negative, the transmission power is set to zero. Therefore, it can be rephrased as $R_t-R_e$.
\fi

\subsection{Optimization problem formulation }
Due to limited horizontal angle of Carols' radar sensing beamwidth and vehicles' mobility, the Eve only suffers from radar sensing interference during a series of time slots. Therefore, we should first determine the duration of confidential information transmission. Let $k_{\rm start}$ and $k_{\rm{end}}$ denote the indexes of the earliest time slot and the latest time slot for confidential signal transmission at the Alice, respectively. Then, to achieve the communication reliability and security, as well as sensing accuracy in the IoV, we investigate the joint optimization design of the transmit power $\textbf{P}_{\rm com} \triangleq {\left[ P_{\rm com}[k_{\rm start}],\ldots ,P_{\rm com}[k_{\rm end}] \right]}^T $ and straight trajectory along the $x$-axis $\textbf{X}_{\rm Alice} \triangleq {\left[ X_{\rm Alice}[k_{\rm start}],\ldots ,X_{\rm Alice}[k_{\rm end}] \right]}^T$ to maximize the $\rm TRSA\_SR$ of the system over all $N$ time slots. Note that, due to the fact that the Bob and the Eves suffer from the radar sensing interference generated by the Carols. Therefore, before jointly optimizing the transmit power and straight trajectory of the Alice, we should find the optimal potential time duration of information transmission, under which the Eves are covered by the radar sensing interference of the Carols, denoted by $[t_{\rm start}, t_{\rm end}]$, where $1\leq k_{\rm start}\leq k_{\rm end}\leq T/N$. To sum up, we formulated the $\rm TRSA\_SR$ maximization problem as follows.
\begin{subequations} \label{eq:opt0}
\begin{align}
\text{Objective:}~&\max\limits_{\textbf{P}_{\rm com}, \textbf{X}_{\rm Alice}} \tau[k] \\
%( 1-p_{co}){{p}_{sr}}\frac{1}{N}\sum\limits_{k=1}^{N} \left[ {R_{\rm Bob}[k]}-\textcolor{red}{\max_{e_j\in\Phi_e}{{R_{e_j}[k]}}} \right]^+\\
\mbox{s.t.}\quad
& 0\le P_{\rm com}\left[ k \right]\le {{P}^{\rm com}_{\max }},  
\forall k_{\rm start}\le k\le k_{\rm end}  
  \label{eq:opt0-P},  \\
&|{{v}_{i}}[k] |\le {v_{\max }} \label{eq:opt0-v}, \\
& |{{a}_{i}}[k]|\le a_{\max } \label{eq:opt0-a},\\
& X^{\rm hor}_{a\rightarrow b}[ k] \le D^{\min}_{\rm foll} \label{eq:opt0-x},\\
&{1-{p}_{co}}\ge \zeta _{\min }^{\rm rel} \label{eq:opt0-Pco},\\
&{p}_{so}\le \zeta _{\max }^{\rm sec}\label{eq:opt0-Pso}  ,\\
&{p}_{sr}\ge \zeta _{\min }^{\rm sen}\label{eq:opt0-Psr},
\end{align}
\end{subequations}
where $P^{\rm com}_{\max}$ is the maximum transmission power for confidential information in Eq. \eqref{eq:opt0-P}, constraint Eq. \eqref{eq:opt0-v} restricts vehicle speeds, and $|\cdot|$ is the absolute value since the velocity is a vector and possesses directionality. Next, constraint \eqref{eq:opt0-a} limits vehicle acceleration and constraint \eqref{eq:opt0-x} imposes a safety distance between two vehicles in the same lane for safety. Constraints \eqref{eq:opt0-Pco} and \eqref{eq:opt0-Pso} respectively enforce reliability and security thresholds for communication, denoted by $\zeta _{\min }^{\rm rel}$ and $\zeta _{\max }^{\rm sec}$. Constraint \eqref{eq:opt0-Psr} represents performance requirements for sensing accuracy, where $\zeta _{\min }^{\rm sen}$ is the minimum SRP.
However, problem \eqref{eq:opt0} is non-convex
optimization, which is difficult to be solved directly, due to the following reasons: 1) the operator ${\left[ \cdot \right]}^+$ is non-smoothness for the objective function; 2) the non-convexity of objective function with respect to ($\textbf{X}_{\rm Alice}$, $\textbf{P}_{\rm{com}}$) without the operation of ${\left[ \cdot \right]}^+$; 3) the transmission power $\textbf{P}_{\rm{com}}$ is coupled with $\textbf{X}_{\rm Alice}$, which makes the problem more challenging to be solved. 
Although there is no general approach to solving problem \eqref{eq:opt0} optimally, in the following section \ref{sec:opt}, we propose an algorithm to address problem \eqref{eq:opt0} effectively. 

\iffalse
Eq. \eqref{eq:opt0-2} concerns communication power constraints, where $q_{max}$ represents the maximum power of communication. Eq. \eqref{eq:opt0-3} pertains to vehicle speed constraints, where $v$ and $v_{\max}$ represent respectively the average and maximum velocity of Alice. Eq. \eqref{eq:opt0-4} deals with vehicle acceleration constraints, where $a_0^{max}$ represents the maximum acceleration of the transmitter. Eq. \eqref{eq:opt0-5} and \eqref{eq:opt0-6} represent reliability and sensing performance constraints, $\zeta _{\min }^{co}$ and $\zeta _{\min }^{each}$ are the minimum levels that satisfy reliability and successful sensing, respectively. \textcolor{red}{$|\cdot|$ is xxx, since xxxx.}
\fi



\iffalse
To facilitate computation, we optimize under the assumption of one Eve and one Carol, but the method can be extended to scenarios with multiple Eves and Carols. We divide Alice's time spent traveling on this road segment evenly into $n$ time slots. $t\left[ 0 \right]$ represents the moment it enters the road segment, while $t\left[ n  \right]$ represents the moment it exits the road segment. The objectives are as follows: 1) To determine the optimal transmission time window $[t_{\rm start}, t_{\rm end}]$ for the communication signal, where $1\leq t_{\rm start}\leq t_{\rm end}\leq\frac{T}{N}$. 2) Building upon the first objective, optimizing the joint transmission power $\textbf{P}_{\rm com} \triangleq {\left[ P_{\rm com}[t_{\rm start}],\ldots ,P_{\rm com}[t_{\rm end}] \right]}^T $ and trajectory along the $x$-axis $\textbf{X}_{\rm Alice} \triangleq {\left[ X_{\rm Alice}[t_{\rm start}],\ldots ,X_{\rm Alice}[t_{\rm end}] \right]}^T$ of the sender to maximize the secrecy rate. The secrecy rate at $k$-th time slot as Eq. \eqref{eq:Rs[k]}. The maximization of secrecy rate optimization problem can be formulated as
%优化问题 
\begin{subequations} \label{eq:opt0}
\begin{align}
\max\limits_{q,x} ~&( 1-P_{co}){{P}_{sr}}\frac{1}{N}\sum\limits_{k=1}^{N} \left[ {R_{\rm Bob}[k]}-\textcolor{red}{\max_{e_j\in\Phi_e}{{R_{e_j}[k]}}} \right]^+\\
\mbox{s.t.}\quad
&0\le q\left[ k \right]\le {{q}_{\max }},\forall t_{\rm start}\le k\le t_{\rm end} \label{eq:opt0-2},\\ 
&|{{v}_{i}}(k) |\le {v_{\max }} \label{eq:opt0-3}, \\
& |{{a}_{i}}(k)|\le a_{\max } \label{eq:opt0-4},\\
&{1-{P}_{co}}\ge \zeta _{\min }^{\rm rel} \label{eq:opt0-5},\\
&{P}_{sr}\ge \zeta _{\min }^{\rm sen},\\
&{P}_{so}\le \zeta _{\max }^{\rm sec}.\label{eq:opt0-6}
\end{align}
\end{subequations}
Eq. \eqref{eq:opt0-2} concerns communication power constraints, where $q_{max}$ represents the maximum power of communication. Eq. \eqref{eq:opt0-3} pertains to vehicle speed constraints, where $v$ and $v_{\max}$ represent respectively the average and maximum velocity of Alice. Eq. \eqref{eq:opt0-4} deals with vehicle acceleration constraints, where $a_0^{max}$ represents the maximum acceleration of the transmitter. Eq. \eqref{eq:opt0-5} and \eqref{eq:opt0-6} represent reliability and sensing performance constraints, $\zeta _{\min }^{co}$ and $\zeta _{\min }^{each}$ are the minimum levels that satisfy reliability and successful sensing, respectively. \textcolor{red}{$|\cdot|$ is xxx, since xxxx.}
\fi


\iffalse
\newcounter{TempEqCnt} % 创建临时变量TempEqCnt
\setcounter{TempEqCnt}{\value{equation}} 
%\setcounter{equation}{x} 
\begin{figure*}
\begin{equation} \label{eq:Rs[k]}
R_s \left[ k \right] = {{{\log }_{2}}\left[ 1+\frac{{q\left[k\right] }{{h}_{ab}}}{\left( W_b+P_b^{\rm{echo}}+{I_b} \right)\left( {{\left( x[k]-{{x}_{b}}[k] \right)}^{2}}+{{\left( {{y}_{a}}-{{y}_{b}} \right)}^{2}} \right)} \right] }
{-{{\log }_{2}}\left( 1+\frac{{q\left[k\right] }{{h}_{ae}}}{\left( W{}_{e}+P_e^{\rm{echo}}+{{I}_{e}} \right)\left( {{\left( x[k]-{{x}_{e}}[k] \right)}^{2}}+{{\left( {{y}_{a}}-{{y}_{e}} \right)}^{2}} \right)} \right) }
\end{equation}
\hrulefill
\end{figure*}
\setcounter{equation}{\value{TempEqCnt}} 
\fi




%在有建筑物、车辆和其他障碍物的城市环境中,信号会经历显著的衰减,
%alpha=4是一个合理的近似。并且,经验研究和测量通常支持在某些环境中使用alpha=4,与观察到的信号衰减模式非常匹配。
%因此,在推导无线物理层安全分析中的中断概率闭式表达式时,假设路径损耗指数为4是合理的。此假设不仅简化了分析过程,准确表示了常见的城市环境,有效地建模了严重的衰减场景,并且得到了经验研究的支持。


\section{Performance Analysis and Confidential Rate Maximisation}\label{sec:opt}
In this section, first of all, with the help of the tools of stochastic geometry, we establish the closed-form expressions of the COP, SOP, and SRP. Then, the non-convex optimization problem is effectively solved by utilizing methods of Lagrangian multipliers, auxiliary variables, first Taylor expansion and successive convex approximation (SCA). Finally, we conduct a theoretical analysis of the convergence and computational complexity of the proposed algorithm.

\subsection{Analytical framework of closed-form expressions of the COP, SOP, SRP and secrecy rate maximization} 
Based on the PDF of the RCS in Eq. \eqref{eq:p_sigma } and the tools of stochastic geometry, we consider the scenario of $\alpha=4$ across time slots, and derive the closed-form results. This case allows the closed-form results to be derived, because under this case the complex integration involved in the calculation can be solved, and provides a guideline to analyze the transmission reliability and security, as well as the sensing accuracy. The corresponding reasonability for deriving closed-form expression has been discussed in the studies of \cite{ Yu2024SuRLLC,Yu2024A}.\footnote{In urban environments with buildings, vehicles, and other obstacles, the settings of $\alpha=4$ is a reasonable approximation that the signal experiences significant attenuation. In addition, empirical studies often support the use of $\alpha=4$ in certain environments, as it closely matches the observed signal attenuation patterns. Therefore, the assumption of $\alpha=4$ not only simplifies the derivation of closed-form expressions for outage probabilities in PLS analysis, but also accurately represents common urban environments.
}

\begin{theorem}
Given a distance between the Alice and the Bob and RCS of the target sensed by the latter, denoted by $\bar{X}^{\rm hor}_{a\rightarrow b}$ and $\bar{\sigma}_{\rm Bob}$, the closed-form expression of the corresponding COP is given by 
\begin{equation} \label{eq:P_co final}
\begin{aligned}
{{P}_{\text{co}}} &=1-\exp \left[ -\pi {{\lambda }_{i}}{{\left( \frac{{{\beta }_{b}}{{A}_{\rm ea}}S}{P_{\rm {com}}} \right)}^{\frac{1}{2}}}
{{\bar{X}}_{a\rightarrow b} ^{\rm {hor}}} {^{2}}
{\cdot C \left(4 \right)} \right.\\
&\left. ~~~ -\frac{{{\beta }_{b}}}{P_{\rm com}} \left({\sigma^{2}_{b}+ {\frac{P{{G}_{t}}{{G}_{r}}{{\lambda }_{\omega}}^{2}\bar{\sigma}_{\rm Bob} }{{{\left( 4\pi  \right)}^{3}}
{R_{\rm{tar}}}^{8 }}}} \right){\bar{X}^{\rm hor}_{a\rightarrow b}}{^{4}}\right],
\end{aligned}
\end{equation}
where $C \left(\alpha \right)= \Gamma \left( 1+\frac{2}{\alpha } \right) \Gamma \left( 1-\frac{2}{\alpha } \right)$, and $\Gamma(\cdot )$ is the Gamma function. %The detailed proof processes refers to  \cite{zhu2014secrecy}.
\end{theorem}

\iffalse
\subsection{COP} 
For the communication signal transmitted by Alice, with a unique legitimate receiver Bob. The expression representing the COP in closed-form is given by
\begin{equation} \label{eq:P_co final}
\begin{aligned}
{{P}_{\text{co}}} &=1-\exp \left[ -\pi {{\lambda }_{i\to b}}{{\left( \frac{{{\beta }_{b}}{{A}_{e}}S}{q} \right)}^{\frac{1}{2}}}{{L}_{ab}}^{2} C \left(4 \right) \right.\\
&\left. ~~~ -\frac{{{\beta }_{b}}}{q} \left({{W}_{b}+ {\frac{P{{G}_{t}}{{G}_{r}}{{\lambda }_{\omega}}^{2}\sigma }{{{\left( 4\pi  \right)}^{3}}{{R}_{\max }}^{8 }}}} \right){{L}_{ab}}^{4} \right],
\end{aligned}
\end{equation}
where $C \left(\alpha \right)= \Gamma \left( 1+\frac{2}{\alpha } \right) \Gamma \left( 1-\frac{2}{\alpha } \right)$. $\Gamma(\cdot )$ is the gamma function. The specific proof process is as follows \cite{zhu2014secrecy}\cite{Yu2023The}.
\fi

\begin{IEEEproof}
Due to the limited space of the manuscript, the detail derivation process of Eq. \eqref{eq:P_co final} can refer to \cite{Lkx2024}.
\end{IEEEproof}

\begin{theorem}
For the case of $\alpha=4$ and a given RCS of the target, denoted by $\bar{\sigma}_{e_j}$, sensed by the $j$-th Eve, the upper bound and lower bound of the SOP happening at this Eve are given by Eq. \eqref{eq:sop_u} and Eq. \eqref{eq:SOP_L}, respectively, where $ \mathrm{Erfc} \left( z \right)= { \frac{2}{\sqrt{\pi}}} {\int_{z}^{\infty} { e^{-t^2}}  } dt$ represents the complementary error function. %\cite{abramowitz1948handbook}.
\begin{figure*}[t] % t选项是尽量放置在页面顶部
\begin{equation} \label{eq:sop_u}
\begin{aligned}
p_{so}^{\rm upper}= & 1-\exp \left[ -\frac{{{\lambda }_{e}}{{\pi }^{\frac{3}{2}}}}{2\sqrt{\frac{{{\beta }_{e}}}{P_{\rm com}}\left( {\sigma^{2}_{e}+ P^{e_j}_{\rm{echo} }} \right)}}\cdot \exp \left[ \frac{{{\left( \pi {{\lambda }_{i}}C\left( 4 \right) \right)}^{2}}{{A}_{\rm ea}}S}{4 \left( {\sigma^{2}_{e}+ P^{e_j}_{\rm{echo}} } \right) } \right] \cdot \mathrm{Erfc}\left[ \sqrt{\frac{{{A}_{\rm ea}}S}{4 \left( {\sigma^{2}_{e}+ P^{e_j}_{\rm {echo}}} \right)}} \right] \right]
\end{aligned}
\end{equation}
and
\begin{equation} \label{eq:SOP_L}
\begin{aligned}
p_{so}^{\rm lower}= &\exp \left[ -\frac{{{\lambda }_{e}}{{\pi }^{\frac{3}{2}}}}{2\sqrt{\frac{{{\beta }_{e}}}{P_{\rm com}}\left( {\sigma^{2}_{e}+ P^{e_j}_{\rm{echo}}} \right)}} \cdot \exp \left[ \frac{{{\left( \sqrt{\frac{{{\beta }_{e}}{{A}_{\rm ea}}S}{P_{\rm com}}}{{\lambda }_{i}}\pi C\left( 4 \right)+{{\lambda }_{e}} \right)}^{2}}}{4\frac{{{\beta }_{e}}}{P_{\rm com}}\left( {\sigma^{2}_{e}+ P^{e_j}_{\rm{echo}}} \right)} \right] \cdot \mathrm{Erfc}\left[ \frac{\sqrt{\frac{{{\beta }_{e}}{{A}_{\rm ea}}S}{P_{\rm com}}}{{\lambda }_{i}}\pi C\left( 4 \right)+{{\lambda }_{e}}}{2\sqrt{\frac{{{\beta }_{e}}}{P_{\rm com}}\left( {\sigma^{2}_{e}+ P^{e_j}_{\rm{echo}}} \right)}} \right] \right]
\end{aligned}
\end{equation}
where $P^{e_j}_{\rm{echo}} =\frac{P_{\rm sen}G_tG_r{\lambda_{w}}^2 \bar{\sigma}_{e_j}}
{( 4\pi )^3{R_{\rm{tar} }}^{2\alpha }}$.
\hrule % 分割线
\end{figure*}
\end{theorem}

\begin{IEEEproof}
Due to the limited space of the manuscript, the detail derivation process of Eq. \eqref{eq:sop_u} and Eq. \eqref{eq:SOP_L} can refer to \cite{Lkx2024}.
\end{IEEEproof}

\begin{theorem}
For the case of $\alpha=4$, the SRP of the Alice can be represented as
\begin{equation}\label{eq: closed expression srp}
\begin{aligned}
  & {{p}_{sr}} =\frac{\Gamma \left( k,\frac{4\pi \delta \beta_s { R_{\rm{tar}}^{8}}}{{\bar{\sigma }_{av}}}\left( \frac{{{\left( 4\pi  \right)}^{2}}}{{\bar{\sigma }_{av}}P_{\rm sen}{{G}_{t}}{{G}_{r}}{{\lambda }_{w}}^{2}}{\sigma^{2}_{a}}+\frac{{{\lambda }_{i}}}{3}{{\left( 2D_{\rm lane} \right)}^{-3}} \right) \right)}{\Gamma \left(\rm{\delta} \right)} \\ 
\end{aligned}
\end{equation}
\end{theorem}

\begin{IEEEproof}
Due to the limited space of the manuscript, the detail derivation process of Eq. \eqref{eq: closed expression srp} can refer to \cite{Lkx2024}.
\end{IEEEproof}

\iffalse
\subsection{JSAC-SRM}
In the case of $P_{co}=\varrho_{co}$, the ommunication threshold ${\beta_b}$ at Bob can be solved as Eq. \eqref{eq:beta_b^*}.

When $P_{so}=\varrho_{so}$, the ommunication threshold ${\beta_e}$ at Eve is expressed as 
\begin{equation}\label{eq:beta_e^*}
\begin{aligned}
\beta_e^* = &{{\left( \frac{\ln \left( \frac{1}{1-{{\varrho }_{so}}} \right){{\lambda }_{i\to e}}{{\sqrt{q}}^{{}}}{{\pi }^{\frac{3}{2}}}}{2\sqrt{{W}_{e} + P_e^{\rm{echo}} }}\exp \left( \frac{A_eS{{\left( \pi {{\lambda }_{i\to e}}C\left( 4 \right) \right)}^{2}}}{4\left( {W}_{e} + P_e^{\rm{echo}} \right)} \right) \right. }}\\
&{{\left.  \mathrm{Erfc}\left( \sqrt{\frac{AeS}{4\left( {W}_{e} + P_e^{\rm{echo}} \right)}} \right) \right)}^{2}}.
\end{aligned}
\end{equation}
Substituting formulas \eqref{eq:beta_b^*} and \eqref{eq:beta_e^*} into ${\beta_b} = 2^{R_b} -1$ and ${\beta_e} = 2^{R_e} -1$ respectively, we can obtain the values of Rt for ensuring reliable communication and Re for secure communication, represented as Eq. \eqref{eq:rb ss-stc} and Eq. \eqref{eq:re ss-stc}.
\newcounter{TempEqCnt2} % 创建临时变量TempEqCnt
\setcounter{TempEqCnt2}{\value{equation}} 
%\setcounter{equation}{x} 
\begin{figure*}
\begin{equation}\label{eq:beta_b^*}
\beta_b^* = q{{\left( \frac{A_eS{{\left( \pi {{\lambda }_{i\to b}}C\left( 4 \right) \right)}^{2}}+\sqrt{4{\left( {W}_{b} + P_b^{\rm{echo}} \right)}{{L}_{ab}}^{4}\ln \left( \frac{1}{1-{{\varrho }_{co}}} \right)}}{2{{W}_{b}}{{L}_{ab}}^{2}} \right)}^{2}}.
\end{equation}
\begin{equation} \label{eq:rb ss-stc}
 {{R}_{b}^{*}}={{\log }_{2}}\left[ 1+q{{\left( \frac{A_eS{{\left( \pi {{\lambda }_{i\to b}}C\left( 4 \right) \right)}^{2}}+\sqrt{4\left( {W}_{b} + P_b^{\rm{echo}} \right){{L}_{ab}}^{4}\ln \left( \frac{1}{1-{{\gamma }_{co}}} \right)}}{2\left( {W}_{b} + P_b^{\rm{echo}} \right){{L}_{ab}}^{2}} \right)}^{2}} \right]  
\end{equation}
\begin{equation} \label{eq:re ss-stc}
{{R}_{e}^{*}}={{\log }_{2}}\left[ 1+{{\left( \frac{\ln \left( \frac{1}{1-{{\gamma }_{so}}} \right){{\lambda }_{i\to e}}{{\sqrt{q}}^{{}}}{{\pi }^{\frac{3}{2}}}}{2\sqrt{\left( {W}_{e} + P_e^{\rm{echo}} \right)}}\exp \left( \frac{A_eS{{\left( \pi {{\lambda }_{i\to e}}C\left( 4 \right) \right)}^{2}}}{4\left( {W}_{e} + P_e^{\rm{echo}} \right)} \right) \mathrm{Erfc}\left( \sqrt{\frac{AeS}{4\left( {W}_{e} + P_e^{\rm{echo}} \right)}} \right) \right)}^{2}} \right]
\end{equation}
\hrulefill
\end{figure*}
\setcounter{equation2}{\value{TempEqCnt} 
%\setcounter{equation}{y} 

Finally, under the given COP, SOP and $P_a =\psi$, the achievable rate of successful transmission of confidential messages per unit network area is obtained.
\begin{equation}
\tau =\left( 1-{\varrho_{co}} \right)\psi {{\lambda }_{a}}\left( {{R}_{b}^{*}}-{{\operatorname{R}}_{e}^{*}} \right)
\end{equation}

\fi

\subsection{Optimizing transmission time and confidentiality rate} 
In this subsection, since the Eves are more closer to the Alice than the Bob, to achieve the highest $\rm TRSA\_SR$, the Alice tries to transmit the confidential information to the Bob only when the Eves are covered by radar sensing signals of the Carols. In this way, we need to determine the earliest transmit time and the latest end time for information transmission between the Alice and the Bob.
Furthermore, we optimize jointly the design of the transmit power and straight trajectory of the Alice to maximize the $\rm TRSA\_SR$. 
To facilitate mathematical calculations, we make an assumption that a single Eve and a single Carol are considered, but the method can be extended to scenarios with multiple Eves and Carols.

\iffalse
For communications with low real-time requirements, optimizing the transmission time may provide the greatest security of information transfer. In this section, we optimized the transmission time, communication power, and trajectory of the transmitter. The transmission time slots of communication signals were optimised initially, with the objective of selecting moments when eavesdroppers experience higher interference levels to transmit communication signals. In order to achieve the highest possible secrecy rate, the power and trajectory of the transmitter were optimised alternately. In the following analysis, we consider a scenario with one of each of transmitter (Alice), receiver (Bob), eavesdropper (Eve) and interferer (Carol), as shown in Fig. \ref{fig:earliest_time}. It should be noted that this method can be extended to scenarios involving multiple eavesdroppers and interferers.
\fi

\iffalse
\begin{figure}[!ht]
\centering
\includegraphics[width=2.3in]{results/Earliest_time.pdf}
\caption{\small Earliest transmit time of the Alice}
\label{fig:earliest_time}
\end{figure}

\begin{figure}[!ht]
\centering
\includegraphics[width=2.3in]{results/Latest_time.pdf}
\caption{\small Latest transmit time of the Alice}
\label{fig:latest_time}
\end{figure}
\fi

\subsubsection{\textbf{Communication duration for information transmission }}
Let $c=\frac{1}{\sqrt{\mu \varepsilon }}$ denote the propagation speeds of communication and sensing signals, where $\mu$ and $\varepsilon$ respectively represent the permittivity and permeability of the medium \cite{Sun2021Unmanned}. Initially, all vehicles move at a constant speed $v$. In the $k$-th time slot, the positions of Alice, Bob, Eve, and Carol are respectively $\left(x_{a}[k],0 \right)$, $\left( x_{b}[k],0\right)$, $\left(  x_{e}[k], D_{\rm{ lane}}\right)$, and $\left(  x_{c}[k], 2D_{\rm{ lane}}\right)$.


%\textcolor{red}{Let $t[k_a]$, $t[k^{\rm ear}_c]$, denote the earliest time starting to transmit the confidential information and sensing signal by the Alice and Carol, $t[k_e]$ denote the time receiving both above signals by the Eve.}

%\textcolor{red}{Let $k^{\rm ear}_c$ and $k^{\rm ear}_a$ denote the time slot starting to transmit the sensing signal and confidential signal by the Carol and Alice, respectively.}

\textit{\textbf{Earliest transmission time of the Alice:}}
Let $ k_{c}^{\rm {ear}}$ denote the index of the earliest time slot that the Carol generates radar sensing signal, and $k_{e}^{\rm{ear}}$ denote another index of the earliest time slot that the Eve receives the signal.
%the sensing signal emitted by Carol in $ k_{c}^{\rm {ear}}$-th time slot should precisely reach Eve at $k_{e}^{\rm{ear}}$-th time slot, while the signal emitted at $ \left(k_{c}^{\rm ear}-1 \right)$-th time slot does not interfere with Eve. 
Next, we first determine the relationship between $ k_{c}^{\rm {ear}}$ and $ k_{e}^{\rm {ear}}$.
%(while the Eve receives the confidential signal at $k^{\rm ear}_a$-th time slot ($k^{\rm sta}_a\geq k^{\rm ear}_c)$)
For this case, the maximum horizontal distance between the Carol and the Eve is $X^{\max, \rm{hor}}_{c\rightarrow e}=\sqrt{{{R}_{\max }}^{2}-{{D_{\rm lane}}^{2}}}$, and
the maximum propagation time of the radar sensing signal reaching to the Eve is $ t^{\max,\rm{sen}}_{{c\rightarrow e}} = R_{\max}/c$. Therefore, the sum of the moving distance of the Eve and the $X^{\max, \rm{hor}}_{c\rightarrow e}$ equals the horizontal distance between the Carol and the Eve at the $k_{c}^{\rm ear}$-th time slot.

\iffalse
In order to secure information transmission, it is essential to ensure that the Eves are covered by sensing range of the Carols when eavesdropping. Let $t_{\rm start}$ be the earliest tranmit time of the Alice for information transmission. Considering the propagation delay of the signals, if the communication signal sent by the Alice at time $t[a_1]$ and the sensing signal generated by the Carol at time $t\left[ c_1 \right]$ both arrive at Eve simultaneously at time $t \left[ m_1 \right]$, then the maximum detection range of the radar corresponds to the maximum detection horizontal distance. 

The horizontal length of the maximum sensing distance of the Carol between it and the Eve along the x-axis is $X_{\max }^{c\rightarrow e}=\sqrt{{{R}_{\max }}^{2}-{{D_{\rm lane}}^{2}}}$. This implies that if the distance between the Carol and the Eve on the x-axis is less than this value, the sensing signal emitted by Carol will suppress Eve's channel. At this moment, the needed propagation time for the interference signal to reach Eve is $ t_1^{c\rightarrow e} = \frac {R_{\max}}{c}$. Therefore, the one-way horizontal distance from the radar signal detection to Eve should satisfy 
\fi

\begin{displaymath}
X^{\max,\rm{hor}}_{c\rightarrow e}+ v t^{\max ,\rm{sen} }_{c\rightarrow e} = {x_c}\left[ k_c^{\rm ear} \right]-{x_e}\left[ k_c^{\rm {ear}} \right],
\end{displaymath}
where ${x_c}\left[ k_c^{\rm ear} \right]$ and 
${x_e   }\left[ k_c^{\rm ear} \right]$ represent the positions of Carol and Eve at the $k_c^{\rm ear}$-th time slot, respectively. The needed time that the Eve receives radar sensing signal equals to the sum of the time that Carol sends the radar sensing signal and the radar sensing signal propagation time, $ t \left[ {k_{e}^{\rm ear}} \right]= t^{\max ,\rm{sen}}_{c\rightarrow e} + t \left[{k_c^{\rm ear}} \right]$. The values of $k_c^{\rm ear}$ and $k_e^{\rm ear}$ are
\begin{displaymath}
{k_c^{\rm{ear}}}=\frac{{{x}_{e}\left[ 0 \right] }-{{x}_{c}\left[ 0 \right] }-X^{\max ,\rm{hor}}_{c\rightarrow e } - vt^{\max,\rm{sen}}_{c\rightarrow e}}{2v},
\end{displaymath}
and
\begin{displaymath}
{k_e^{\rm{ear}}}=\frac{{{x}_{e}\left[ 0 \right] }-{x}_{c}\left[ 0 \right] -X^{\max ,\rm{hor}}_{c\rightarrow e }+vt^{\max,\rm{sen}}_{c\rightarrow e}}{2v},
\end{displaymath}
respectively.
Let $k_a^{\rm {ear}}$ be the index of the earliest time slot that the Alice starts to transmit confidential message, and $t^{\rm {com}}_{{a\rightarrow e}}$ be the needed time that the Eve receives the message. Since the Alice and Eve are moving at a constant speed initially, thus $t^{\rm {com}}_{{a\rightarrow e}}$ can be given by
\begin{displaymath}
{{\left( vt^{\rm {com}}_{{a\rightarrow e}}+{{x}_{e} \left[ 0 \right]} \right)}^{2}}+{{D_{\rm lane}}^{2}}={{\left( ct^{\rm {com}}_{{a\rightarrow e}}\right)}^{2}}.
\end{displaymath}
Then, we can get 
\iffalse
Based on the time $ t\left [k_e^{\rm{ear}} \right]$ when Eve receives the radar sensing and communication signal at the same time, the earliest time $t\left[k_a^{\rm {ear}}\right]$ when Alice could have sent the signal can be determined. Since Alice and Eve are moving at a constant speed and the distance between them remains unchanged, the communication signal propagation time is fixed and denoted as $t^{\rm {com}}_{{a\rightarrow e}}$, 
\begin{equation}
{{\left( vt^{\rm {com}}_{{a\rightarrow e}}+{{x}_{e} \left[ 0 \right]} \right)}^{2}}+{{D_{\rm lane}}^{2}}={{\left( ct^{\rm {com}}_{{a\rightarrow e}}\right)}^{2}}.
\end{equation}
where $ct^{\rm {com}}_{{a\rightarrow e}}$ is the distance of communication signal propagation, $vt^{\rm {com}}_{{a\rightarrow e}}+{{x}_{e} \left[0\right]}$ is the horizontal distance from the moment of Alice transmitting the signal to Eve receiving it, and $D_{\rm lane}$ is the vertical distance between them. Upon completion of the calculations, the result is 
\fi
\begin{displaymath}
t^{\rm{com}}_{{a\rightarrow e}}=\frac{vx_e \left[ 0 \right] +\sqrt{{{c}^{2}}{{v}^{2}}+{{c}^{2}}{{D_{\rm lane}}^{2}}-{{v}^{2}}{{D_{\rm lane}}^{2}}}}{ \left(  {{c}^{2}}-{{v}^{2}} \right)}.
\end{displaymath}
Moreover, the begin of time slot that the Alice starts to transmit the confidential information equals to the difference between the time that the Eve receives the sensing signal and that of the confidential signal, i.e., $k_a^{\rm {ear}} =  k_e^{\rm {ear}}- t_{a\rightarrow e}^{\rm {com}}  $. Therefore, the index of the earliest time slot that the Alice starts to transmit can be calculated as
\begin{displaymath}
\begin{aligned}
    k_a^{\rm {ear}} = &\frac{{{x}_{e}\left[ 0 \right]}-{{x}_{c}\left[ 0 \right]}-\sqrt{{{R}_{\max }}^{2}-{{D_{\rm lane}}^{2}}}+v\frac{{{R}_{\max }}}{c}}{2v} \\
&-\frac{vx_e \left[ 0 \right]+\sqrt{{{c}^{2}}{{v}^{2}}+{{c}^{2}}{{D_{\rm lane}}^{2}}-{{v}^{2}}{{D_{\rm lane}}^{2}}}}{\left( {{c}^{2}}-{{v}^{2}} \right)}.
\end{aligned}
\end{displaymath}
Recall that $k_{\rm start}$ denote the index of the earliest time slot for the Alice transmitting message.
To ensure the physical layer security potentially, it is expected that the Eve should be covered by sensing signal from the Carol before receiving the confidential message from the Alice, namely $k_{\rm start}\geq k_a^{\rm {ear}}$. 

\textit{\textbf{Latest transmission time of the Alice:}}
Let $k_{c}^{\rm {last}} $ be the index of the index of the latest time slot that the Carol generates the radar sensing signal, and $k_{e}^{\rm {last}}$ denote another index of the lasted time slot when the Eve receives the signal. Next, we first determine the relationship between $k_{c}^{\rm {last}} $ and $k_{e}^{\rm {last}}$. For this case, the minimum horizontal distance between the Carol and the Eve is $X^{\min,\rm{hor} }_{c\to e}=\frac{D_{\rm{ lane}}}{\tan \frac{\theta }{2}}$, and the distance between the Bob and the Carol is then given by $D_{\min }^{c\to e}= \sqrt{X{{^{\min, \rm{hor}}_{c\to e}}^{2}}+{{D_{\rm{ lane}}}^{2}}}$. The minimum propagation time of the radar sensing signal reaching to the Eve is $ t_{c\rightarrow e}^{\min ,\rm{sen}} = D_{\min }^{c\to e}/c$. Therefore, the sum of the moving distance of the Eve and the $X^{\min,\rm{hor} }_{c\to e}=\frac{D_{\rm{ lane}}}{\tan \frac{\theta }{2}}$ equals the horizontal distance between the Carol and the Eve at the $k_{c}^{\rm {ear}}$-th time slot.
\begin{displaymath}
X^{\min ,\rm{hor}}_{c\rightarrow e}+ v t_{c\rightarrow e}^{\min ,\rm{sen}} = {x_c}\left[ k_c^{\rm {last}} \right]-{x_e}\left[ k_c^{\rm {last}} \right],
\end{displaymath}
where ${x_c}\left[ k_c^{\rm {last}} \right]$ and ${x_c}\left[ k_e^{\rm {last}} \right]$ represent the positions of Carol and Eve at the $k_e^{\rm {last}}$-th time slot, respectively. The needed time that
the Eve receives the radar sensing signal equals to the sum of the time that Carol sends the radar sensing signal and the radar sensing signal
propagation time, $ { k_{e}^{\rm {last}} } = {k_c^{\rm {last}}}  +t_{c\rightarrow e}^{\min ,\rm{sen}}$. The values of $k_c^{\rm {last}}$ and $k_e^{\rm {last}}$ are
\begin{displaymath}
{k_c^{\rm {last}}}=\frac{{{x}_{e}\left[ 0 \right] }-{{x}_{c}\left[ 0 \right] }-X^{\min ,\rm{hor}}_{c\rightarrow e} - vt^{\min ,\rm{sen} }_{c\rightarrow e}}{2v},
\end{displaymath}
and
\begin{displaymath}
{k_e^{\rm {last}}}=\frac{{{x}_{e}\left[ 0 \right] }-{{x}_{c}\left[ 0 \right] }-X^{\min ,\rm{hor}}_{c\rightarrow e} + vt^{\min ,\rm{sen} }_{c\rightarrow e}}{2v},
\end{displaymath}
respectively. Moreover, since the time required for the communication signal to be transmitted and received remains constant, $ k_a^{\rm {last}}=  k_e^{\rm {last}} - t_{a\rightarrow e}^{\rm {com}}  $, then the last time when Alice sends the signal can be calculated
\begin{displaymath}
\begin{aligned}
    k_a^{\rm {last}} = &\frac{{{x}_{e}\left[ 0 \right]}-{{x}_{c}\left[ 0 \right]}+\frac{D_{\rm{ lane}}}{\tan \frac{\theta }{2}}\left( \frac{\sec \frac{\theta }{2}}{c}-1 \right)}{2v} \\
&-\frac{vx_e \left[ 0 \right]+\sqrt{{{c}^{2}}{{v}^{2}}+{{c}^{2}}{{D_{\rm lane}}^{2}}-{{v}^{2}}{{D_{\rm lane}}^{2}}}}{\left( {{c}^{2}}-{{v}^{2}} \right)}.
\end{aligned}
\end{displaymath}
Recall that $k_{\rm{end}}$ denote the index of the latest time slot of the Alice transmitting message.  To ensure the PLS potentially, it is expected that the Eve should be covered by radar sensing signal from the Carol before receiving the confidential message from the Alice, namely $k_{\rm{end}} \le  k_a^{\rm {last}} $.


%在发射机和接收机之间视距(LOS)路径无障碍的情况下,η=2 是一个合理的近似。
%当源节点与目的节点之间存在LoS链路时,alpha=2是一种合理的近似,
%这种配置提供了比较的基线,使得能够在考虑其他因素之前,了解理想条件下的基本性能极限,然后研究额外的障碍物和环境因素(例如建筑物、植被)如何增加路径损耗指数并影响系统性能。
%此假设简化了分析过程,准确表示了自由空间条件,提供了比较的基线,并且在特定场景中得到了经验研究的支持。因此,这一假设在理论简化和实际应用之间达到了平衡。

\subsubsection{ \textbf{Joint optimization of transmission power and straight trajectory of the Alice}} 
By jointly optimizing transmission power of the Alice $\textbf{P}_{\rm {com}} \triangleq {\left[ P_{\rm {com}}[k_{\rm start}],\ldots ,P_{\rm {com}}[k_{\rm end}] \right]}^T $ and its straight trajectory along the $x$-axis $\textbf{X}_{\rm Alice} \triangleq {\left[ X_{\rm Alice}[k_{\rm start}],\ldots ,X_{\rm Alice}[k_{\rm end}] \right]}^T$, we aim to maximize the $\rm TRSA\_SR$ over all time slots
%for the $k$-th time slot satisfying $k_{\rm start}\le k \le k_{\rm end}$, 
under effective communication duration from the $k_{\rm start}$-th time slot to the $k_{\rm end}$-th time slot.
%of $[{t[k_{\rm start}] }, {t[k_{\rm end}]}]$. 
%It is assumed that Alice travels along the current road in a straight line, and its average speed during this time period remains $v$, 
With the settings of $\alpha=2$ \footnote{At the beginning of Section V, we have made an assumption of the number of Alice, Bob, Carol and Eve is one. Thus, LoS channel between Alice and Bob/Eve exists most likely. In addition, all vehicles are going in a straight line. It's a reasonable approximation of signal attenuation mode under LoS channel by setting $\alpha=2$. Furthermore, this configuration provides a baseline for comparisons, enabling an understanding of the fundamental performance limits under ideal conditions before considering other factors. Subsequently, it allows for the study of how additional environmental factors (such as noise power) affects the system performance.}, the problem \eqref{eq:opt0} is formulated as Eq. \eqref{eq:opt1}.
\begin{figure*}[t] % t选项是尽量放置在页面顶部
\begin{equation} \label{eq:opt1}
\begin{aligned}
\text{Objective:}~~&\max\limits_{\textbf{P}_{\rm {com}}, ~\textbf{X}_{\rm Alice}} 
\sum\limits_{k={{t}_{\rm{start}}}}^{{{t}_{\rm{end}}}}\left[ {{\log}_{2}}\left( 1+\frac{{P_{\rm {com}}[k] }{{h}_{a\rightarrow b}}}{( \sigma^{2}_{b}+{{I}_{\rm Bob}} ){{\Vert X^{\rm{hor}}_{a\rightarrow b}\Vert}^{2}}} \right)  -{{\log }_{2}}\left( 1+\frac{{P_{\rm {com}}\left[k\right] }{{h}_{a\rightarrow e}}}{\left(\sigma^{2}_{e}+{{I}_{e}} \right){\Vert(X^{\rm{hor}}_{a\rightarrow e},Y^{\rm ver}_{a\rightarrow e})\Vert}^{2} } \right) \right]\\
\mbox{s.t.}\quad
& \eqref{eq:opt0-P},\eqref{eq:opt0-v},\eqref{eq:opt0-a},\eqref{eq:opt0-Pco},\eqref{eq:opt0-Pso},\eqref{eq:opt0-Psr},\eqref{eq:opt-s.t.x_a}.
\end{aligned}
\end{equation}
\hrule % 分割线
\end{figure*}
which is also non-convex and difficult to be solved directly.

In the following, we decompose the problem \eqref{eq:opt1} into two sub-problems for obtaining feasible solutions via the BCD method, namely consisting of transmission power optimization and straight trajectory optimization. In particular, on the one hand, sub-problem (P1) is to optimize the transmit power for a given positions of the Alice. On the other hand, sub-problem (P2) is to optimize the straight trajectory of the Alice for a given transmit power. Finally, by alternately optimizing these two sub-problems to obtain a locally optimal solution, the $\rm TRSA\_SR$ maximization can be obtained.

\textit{\textbf{Transmission power optimization of the Alice:}} Let ${s_1 \left[ k \right]}=\frac{{{h}_{a\rightarrow b}}}{\left( \sigma^{2}_{b}+{{I}_{\rm Bob}} \right){{\Vert X^{\rm{hor}}_{a\rightarrow b} \Vert}^{2}}}$ and ${s_2 \left[ k \right]}=\frac{{{h}_{a\rightarrow e}}}{\left( \sigma^{2}_{e}+{{I}_{e}} \right){\Vert(X^{\rm{hor}}_{a\rightarrow e},Y^{\rm ver}_{a\rightarrow e})\Vert}^{2}}$.
Given a straight trajectory $\bar{\textbf{X}}_{\rm Alice}$ of the Alice, the problem \eqref{eq:opt1} can be simplified into
\begin{subequations} \label{eq:P1}
\begin{align}
\mathrm{P1:}~~& \max\limits_{\textbf{P}_{\rm {com}}} 
\sum\limits_{k=k_{\rm start}}^{k_{\rm end}}\left[ \log_{2}\left( 1+{s_1[k]}P_{\rm {com}}[k]\right)\right.\nonumber\\
&~~~~~~\left.-{{\log }_{2}}\left( 1+{s_2 \left[ k \right]}{P_{\rm {com}}\left[k\right] } \right)\right]\\
\mbox{s.t.}~
& \eqref{eq:opt0-P}.
\end{align}
\end{subequations}
%where ${s_1 \left[ k \right]}=\frac{{{h}_{a\rightarrow b}}}{\left( \sigma^{2}_{b}+{{I}_{\rm Bob}} \right){{\Vert X^{\rm{hor}}_{a\rightarrow b} \Vert}^{2}}}$ and ${s_2 \left[ k \right]}=\frac{{{h}_{a\rightarrow e}}}{\left( \sigma^{2}_{e}+{{I}_{e}} \right){\Vert(X^{\rm{hor}}_{a\rightarrow e},Y^{\rm ver}_{a\rightarrow e})\Vert}^{2}}$. 
The non-negative weighted sum still retains concavity because the $\rm TRSA\_SR$ is positive and the  objective is a concave function. Therefore, the Lagrangian Multiplier method can be used for the solution \cite{Gopala2008On}, 
namely
\begin{displaymath}
\frac{{{s}_{1}[k]}}{1+{{s}_{1}[k]}P_{\rm {com}}\left[ k \right]}-\frac{{{s}_{2}[k]}}{1+{{s}_{2}[k]}P_{\rm {com}}\left[ k \right]}-\lambda_{\rm lag}=0,
\end{displaymath}
where $\lambda_{\rm lag}$ denotes the Lagrange multiplier, which is obtained from the result of the last iteration. Accordingly, the solution for $P_{\rm {com}}\left[ k \right]$ is
\begin{equation} \label{eq:P-opt}
\begin{aligned} 
P_{\rm {com}}\left[ k \right] &=\frac{1}{2}\left[ \sqrt{{{\left( \frac{1}{{{s}_{1}}\left[ k \right]}-\frac{1}{{{s}_{2}}\left[ k \right]} \right)}^{2}}+\frac{4}{\lambda _{lag}}\left( \frac{1}{{{s}_{2}}\left[ k \right]}-\frac{1}{{{s}_{1}}\left[ k \right]} \right)} \right. \\
& ~~~\left. -\left( \frac{1}{{{s}_{1}}\left[ k \right]}+\frac{1}{{{s}_{2}}\left[ k \right]} \right) \right]. 
\end{aligned}
\end{equation}
Therefore, the optimal solution of the transmit power for problem \eqref{eq:P1} in the $k$-th time slot is represented as
\begin{equation}
{{P}^{*}_{\rm com}}\left[ k \right]=\min \left\{ P_{\rm com}\left[ k \right],{{P}^{\rm com}_{\max }} \right\}.
\end{equation}
Finally, $\textbf{P}^{*}_{\rm com}=\{{{P}^{*}_{\rm com}}[k_{\rm start}], ..., {{P}^{*}_{\rm com}}[k], ..., {{P}^{*}_{\rm com}}[k_{\rm end}]\}$. It is worth noting that when the channel quality for the Eve is better than that of the Bob, the transmit power is set to 0.

\textit{\textbf{Straight trajectory optimization of the Alice:}} For a given transmit power allocation $\bar{\textbf{P}}_{\rm com}$, the objective function is also non-concave by optimizing the driving trajectory of the Alice, rendering it impossible to directly be solved. Therefore, it is necessary to transform the function into a concave one. To facilitate the solution, relaxation variables $u\triangleq {\left[ u[k_{\rm start}],\ldots ,u[k_{\rm end}] \right]}^T$ and $w\triangleq {\left[ w[k_{\rm start}],\ldots ,w[k_{\rm end}] \right]}^T$  are introduced%\cite{Zhang2019Securing}
, and the problem can be rewritten as \cite{Yu2024SuRLLC}
\begin{subequations} \label{eq:P2}
\begin{align}
\mathrm{P2:} \max\limits_{\textbf{X}_{\rm Alice}} & 
\sum\limits_{k={k_{\rm{start}}}}^{{k_{\rm{end}}}}{\left[ {{\log }_{2}}\left( 1+ \frac{s_3\left[k\right]}{u\left[k\right]} \right) -{{\log }_{2}}\left( 1+ \frac{s_4\left[k\right]}{w\left[k\right]} \right)\right]}\\
\mbox{s.t.}\quad
&w\left[ k \right]-x_a{{\left[ k \right]}^{2}}+2{{x}_{e}}\left[ k \right]x_a\left[ k \right]\nonumber \\
&~~~~-{{x}_{e}}{{\left[ k \right]}^{2}} - D_{\rm{lane}}^2 \le 0,~\forall k, \label{eq:s.t.w}\\ 
 & x_a{{\left[ k \right]}^{2}}-2{{x}_{b}}\left[ k \right]x_a\left[ k \right]+{{x}_{b}}{{\left[ k \right]}^{2}}\nonumber\\
 &~~~~-u\left[ k \right]\le 0,~\forall k,  \label{eq:s.t.u}\\ 
& \eqref{eq:opt0-v},\eqref{eq:opt0-a},\eqref{eq:opt0-x},\eqref{eq:opt0-Pco},\eqref{eq:opt0-Pso},\eqref{eq:opt0-Psr}.\nonumber 
\end{align}
\end{subequations}
where ${{s}_{3}}\left[ k \right]=\frac{P_{\rm com}\left[ k \right]{{h}_{a\rightarrow b}}}{\sigma^{2}_{b}+{{I}_{\rm Bob}}}$ and ${{s}_{4}}\left[ k \right]=\frac{P_{\rm com}\left[ k \right]{{h}_{a\rightarrow e}}}{\sigma^{2}_{e}+{{I}_{e}}}$.
It can be shown that the function ${{\log }_{2}}\left( 1+\frac{{{s}_{3}}\left[ k \right]}{u\left[ k \right]} \right)$ is convex with respect to $u\left[ k \right]$, and $-{x_a\left[ k \right]}^2$ is concave with respect to $x_a\left[ k \right]$. Consequently, the Taylor expansion of a convex function provides a lower bound for the original function, while the Taylor expansion of a concave function provides an upper bound. We propose an iterative algorithm to solve problem approximately by applying the SCA method. The algorithm obtains an approximate solution to problem \eqref{eq:P2} by maximizing a concave lower bound of its objective function within a convex feasible region, which is detailed as follows. Given the initial points $\left( x_0,u_0 \right)$, where $x_0\triangleq {\left[ x_0[k_{\rm start}],\ldots ,x_0[k_{\rm end}] \right]}^T$ the original function is approximated as
\begin{equation}
\begin{aligned}
{{\log }_{2}}\left( 1+\frac{{{s}_{3}}\left[ k \right]}{u\left[ k \right]} \right) 
&\ge {{\log }_{2}}\left( 1+\frac{{{s}_{3}}\left[ k \right]}{{{u}_{0}}\left[ k \right]} \right) \\
&-\frac{{{s}_{3}}\left[ k \right]\left( u\left[ k \right]-{{u}_{0}}\left[ k \right] \right)}{\ln 2\left( u_{0}^{2}\left[ k \right]+{{s}_{3}}\left[ k \right]{{u}_{0}}\left[ k \right] \right)} ,\label{eq:opt-log}
\end{aligned}
\end{equation}
and $-x_a^2[k]$ can be represented as
\begin{equation}
-{{x_a}^{2}}\left[ k \right]\le x_{0}^{2}\left[ k \right]-2{{x}_{0}}\left[ k \right]x_a\left[ k \right] \label{eq:opt-s.t.x_a}.
\end{equation}
Neglecting the constant term, the optimization problem can be reduced to
\begin{subequations} \label{eq:P2*}
\begin{align}
\mathrm{P2':} \max\limits_{\textbf{X}_{\rm Alice}} & 
\sum\limits_{k={k_{\rm{start}}}}^{{k_{\rm{end}}}}{\left[  
-\frac{{{s}_{3}}\left[ k \right] u\left[ k \right]}{\ln 2\left( u_{0}^{2}\left[ k \right]+{{s}_{4}}\left[ k \right]{{u}_{0}}\left[ k \right] \right)} \right.} \nonumber \\
&{\left.~~~~~ -{{\log }_{2}}\left( 1+ \frac{s_4\left[k\right]}{v\left[k\right]} \right)\right]} \\
\mbox{s.t.}\quad
&w\left[ k \right]+ x_0{{\left[ k \right]}^{2}} -2x_0[k]x_a[k] +2{{x}_{e}}\left[ k \right]x_a\left[ k \right] \nonumber \\
&-{{x}_{e}}{{\left[ k \right]}^{2}} -D_{\rm{lane}}^2 \label{eq:s.t.u2} \le 0,\\ 
& \eqref{eq:opt0-v},\eqref{eq:opt0-a},\eqref{eq:opt0-x},\eqref{eq:opt0-Pco},\eqref{eq:opt0-Pso},\eqref{eq:opt0-Psr}, \eqref{eq:s.t.u}.\nonumber 
\end{align}
\end{subequations}
The final objective function obtained is concave,
which can be effectively solved using various existing algorithms and toolboxes, such as CVX and interior-point methods. Due to the introduction of slack variables and the use of first-order Taylor method in \eqref{eq:opt-log} to approximate the upper bound of $\rm TRSA\_SR$, the optimal solution of problem \eqref{eq:P2*} can only serve as the lower bound of original question.  



\subsection{AO algorithm for $\rm TRSA\_SR$ maximization}
By using AO algorithm to jointly optimize the transmit power and straight trajectory of the Alice, the $\rm TRSA\_SR$ maximization can ultimately be achieved. By alternately optimizing subproblems, the complexity of the problem is effectively reduced, and an approximate solution is obtained through a gradual convergence approach. A tight approximate solution can be obtained by solving the problem \eqref{eq:opt0}.

\begin{algorithm}[H] 
	\caption{Proposed Algorithm for Problem \eqref{eq:opt0}}
	\begin{algorithmic}[1]  
		\State  \textbf{Initialization:} the  maximum iteration number $I_{\max}$, initial solutions of transmit power $P^{0}_{\rm com}$, Alice's position $X^{0}_{\rm Alice}$, $\tau_0 = f_{\eqref{eq:opt0}}(P^{0}_{\rm com}, X^{0}_{\rm Alice})$, index of initial iteration $i=0$, and convergence threshold $\tau_{\varepsilon}$\\
		%calculate the earlist and last communication time $k_{start}$ and $k_{end}$\\
		\textbf{Repeat}\\
		~~~~$i = i + 1$\\
		~~~~Update $P^{i}_{\rm com}$ with initial point $\left({P^{i-1}_{\rm com}}, {X^{ i-1 }_{\rm Alice}} \right)$ with Eq. \eqref{eq:P-opt}\\
		~~~~Update $X^{i}_{\rm Alice}$ with initial point $\left({P^{i}_{\rm com}}, X^{i-1}_{\rm Alice} \right)$ with Eq. $\eqref{eq:P2*}$\\
		~~~~$\tau^i=f_{\eqref{eq:opt0}}({P^{i}_{\rm com},X^{i}_{\rm Alice}})$.\\
		\textbf{Until} $\tau^{i} -\tau^{i-1}< \tau_{\varepsilon}$ or the maximum iteration number $I_{\max}$ is reached
		%\State  \textbf{Output:} $P^{i}_{\rm com}$, $X^{i}_{\rm Alice}$, and $\tau^{i}$
	\end{algorithmic}
	\label{al:HSVM}
\end{algorithm}

\iffalse
\begin{algorithm}[H] 
\caption{Proposed Algorithm for Problem \eqref{eq:opt0}}
\begin{algorithmic}[1]   \label{algorithm1}
\State  \textbf{Initialization:} the maximum iteration number $I_{\rm ter}$, $k=1$, and convergence threshold $\tau_T$\\
\textbf{repeat}  \\
    ~~~~$P_{\rm com}[k] \leftarrow$ outputted by Eq. \eqref{eq:P-opt} with $P_{\rm com}[k-1]$ \\
    ~~~~$X_{\rm Alice}[k] \leftarrow$ outputted by Eq. \eqref{eq:P2*} with $X_{\rm Alice}[k-1]$ \\
   ~~~~calculate the optimal secrecy rate $\tau[k]$\\ 
    ~~~~\textbf{if} $\tau[k] >\tau[k-1]$ \textbf{do}\\
        ~~~~~~~~Transmit power updating: $P_{\rm com}[k-1]\leftarrow P_{\rm com}[k]$\\
        ~~~~~~~~Position updating: $X_{\rm Alice}[k-1] \leftarrow X_{\rm Alice}[k]$\\
        ~~~~~~~~$\rm TRSA\_SR$ updating: $\tau[k-1] \leftarrow \tau[k]$\\
    ~~~~\textbf{end if} \\
    \textbf{Until} $|R_{\rm{op}} -R_{\rm{previous}}|<\tau_T$ or the maximum iteration number $I_{\rm ter}$ is reached 
\State  \textbf{Output:} $P_{\rm com}[k-1]$, $X_{\rm Alice}[k-1]$, and $\tau[k-1]$
\end{algorithmic}
\label{al:HSVM}
\end{algorithm}
\fi


\iffalse
\begin{algorithm}[H] 
\caption{AO Algorithm for $\rm TRSA\_SR$ Maximization}
\begin{algorithmic}[1]   \label{algorithm1}
\State  \textbf{Initialization:} the  maximum iteration number $I_{\rm ter}$, previous communication power $\rm{p}_{\rm{previous}}$, previous position $\rm{x}_{\rm{previous}}$, previous secrecy rate $R_{\rm{previous}}$, and convergence threshold $\tau_T$\\
\textbf{repeat}  \\
    ~~~~$\rm{p}_{\rm{op}} \leftarrow$ outputted by Eq. \eqref{eq:P-opt} with $\rm{p}^{\rm{previous}}_{\rm com}$ \\
    ~~~~$\rm{x}_{\rm{op}} \leftarrow$ outputted by Eq. \eqref{eq:P2*} with $\rm{x}^{\rm{previous}}_{\rm Alice}$ \\
   ~~~~calculate the optimal secrecy rate $R_{\rm{op}}$\\ 
    ~~~~\textbf{if} $R_{\rm{op}} > R_{\rm{previous}}$ \\
    ~~~~\textbf{then} \\
        ~~~~~~~~$\rm{p}^{\rm{previous}}_{\rm com} \leftarrow \rm{p}_{\rm{op}}$, $\rm{x}^{\rm{previous}}_{\rm Alice} \leftarrow \rm{x}_{\rm{op}}$, $R_{\rm{previous}} \leftarrow R_{\rm{op}}$\\
    ~~~~\textbf{end if} \\
    \textbf{Until} $|R_{\rm{op}} -R_{\rm{previous}}|<\tau_T$ or the maximum iteration number $I_{\rm ter}$ is reached 
\State  \textbf{Output:} $\rm{p}_{\rm{previous}}, \rm{x}_{\rm{previous}}, and R_{\rm{previous}}$
\end{algorithmic}
\label{al:HSVM}
\end{algorithm}
\fi

\subsubsection{Convergence analysis}
 The optimization algorithm, based on alternating iterations in Algorithm \ref{al:HSVM}, defines the solution of the $\left( i-1 \right)$-th iteration $\left( P^{i-1}_{\rm com}, X^{i-1}_{\rm Alice}\right)$, and the objective function 
$\max\limits_{P^{i-1}_{\rm {com}}, X^{i-1}_{\rm{Alice}
}}\tau^{i-1}$.
In the $i$-th iteration, given $X^{i-1}_{\rm{Alice}}$, the objective can be obtained by solving sub-problem $\mathrm{P1}$, which yields $X^{i}_{\rm{Alice}}$ and we can get $\max\limits_{P^{i}_{\rm com},X^{i-1}_{\rm{Alice}}}\tau^{i-1} \ge \max\limits_{P^{i-1}_{\rm com}, X^{i-1}_{\rm{Alice}}}\tau^{i-1}$.
Similarly, sub-problem $P2'$ can output $X^{i}_{\rm{Alice}}$. Furthermore, $\max\limits_{ P^{i}_{\rm com},X^{i}_{\rm{Alice}}}\tau^{i} \ge \max\limits_{P^{i}_{\rm com}, X^{i-1}_{\rm{Alice}}}\tau^{i-1}$ holds.
It can be demonstrated that the objective function value is non-decreasing following each iteration. Due to the constraints on the power values, the objective function has an upper bound, which proves the convergence of the algorithm.

\subsubsection{complexity analysis} The problem \eqref{eq:opt0} is divided into two subproblems using the BCD method, and the trajectory optimization problem can be solved by using the CVX toolbox, which automatically transforms optimization problems into standard form and solves them using interior-point methods. 
Based on the conclusion that when one of block problems decomposed by BCD algorithm is solved by SCA algorithm, the corresponding computational complexity is the same as that of SCA algorithm for solving a single problem \cite{He2023full}, and results in \cite{Sun2021Unmanned}, the computational complexity of interior point method in each iteration is $O(N^3\log(\epsilon^{-1})$, where $N$ is the number of variables, and $\epsilon$ is the iteration accuracy. To sum up, the complexity of Algorithm \ref{al:HSVM} is $O\left( I_{\rm{iter}} N^3\log(\epsilon^{-1}) \right)$.

\iffalse
\textcolor{red}{\textbf{Complexity:} 
The problem \eqref{eq:opt0} is divided into two subproblems using the BCD method, and the trajectory optimization problem is solved using the CVX toolbox, which automatically transforms optimization problems into standard form and solves them using interior-point methods. According to XX, the complexity of the interior point method can be expressed as $O\left(  n^{3.5} \right)$. And the complexity of the power optimization is $O\left(  n \right)$. Since the computational complexity of each iteration is primarily determined by the complexity of the most computationally intensive sub-block, the time complexity of Algorithm 1 is primarily determined by the trajectory optimization part. Therefore, the overall complexity of the algorithm is $O\left( I_{\rm{iter}} n^{3.5} \right)$, where $I_{\rm{iter}}$ is the number of items. }

where $N$ is the number of variables. the interior point method with the complexity of $O(N^3\log(\epsilon^{-1})$ can be adopted to solve them, where $\epsilon$ is the iteration accuracy.

\textcolor{red}{The problem is divided into two subproblems using the BCD method, and the trajectory optimization problem is solved using the successive convex approximation (SCA) algorithm. Since the computational complexity of each iteration mainly depends on the complexity of solving a single block problem, the time complexity of Algorithm 1 is primarily determined by the trajectory optimization part. Therefore, the overall complexity of the algorithm is $O\left( I_{\rm{iter}} N^{3.5} \right)$, where $I_{\rm{iter}}$ denotes the number
of iterations required for convergence, and $N$ is the number of variables. }
\fi


\iffalse
After determining the transmission period $\left( a1, a2\right)$, the number of iterations for the OA algorithm is set to $iter$. Initially, the optimization of Alice's trajectory is performed using the CVX toolbox. The complexity of this step depends on the efficiency of CVX itself, typically ranging from $O \left( T^2 \right)$ to $O \left( T^3 \right)$. Alice's acceleration, speed, and position are optimized through a for loop, during the period from $a_1$ to $a_2$, resulting in a time complexity of $O \left( \left( a_2 - a_1\right) ^2 \right)$ to $O \left( \left( a_2 - a_1\right) ^3 \right)$. Subsequently, the transmission power is optimized, with a time complexity of $O\left( a_2 - a_1 \right)$. After each iteration, the results are saved, adding an additional complexity of $O\left( a_2 - a_1 \right)$ per iteration. Therefore, the overall time complexity of the algorithm is $O \left( \textcolor{red}{N_{\rm iter}} * \left(\left( a_2 - a_1\right) ^2 +2*\left( a_2 - a_1 \right)    \right) \right)$.
\fi

\iffalse
\begin{algorithm}[H] 
\caption{Joint optimization of the transmitter's transmission power and its trajectory.}
\begin{algorithmic}[1]  
\Statex  \textbf{Initialization:} the  maximum iteration count $I_{\max}$, previous communication transmission power $p_0$, Alice's previous position $\rm{x}_0$, the previous secret rate $R_0 = f(p_0,x_0) $.\\
%calculate the earlist and last communication time $k_{start}$ and $k_{end}$\\
Repeat\\
    ~~~~$I_{ter} = I_{ter} + 1$\\
    ~~~~Update $p_{iter}$ with initial point $\left({p_{iter-1}}, {x_{ iter-1 }} \right)$ with Eq. \eqref{eq:P-opt}\\
    ~~~~Update $x_{iter}$ with initial point $\left({p_{iter}}, x_{iter-1} \right)$ with Eq. $\eqref{eq:P2*}$\\
    ~~~~$R_{iter}=f({p_{iter},x_{iter}})$.\\
    ~~~~Until $\frac{R_{iter} -R_{iter-1}}{R_{iter}} \le \varepsilon$ or $iter=I_{\max}$\\
end Repeat
\Statex  \textbf{Output:} $p_{iter},x_{iter},R_{iter}$.
\end{algorithmic}
\label{al:HSVM}
\end{algorithm}
\fi


\section{Simulation Results}\label{sec:simulation}
In the section, considering the sensing interference empowered PLS, we provide numerical simulations to evaluate the performance of our proposed joint design of straight trajectory and transmit power of the Alice (denoted as \textbf{SI-PLS with ST-TP}). The vehicles moves in the center of the road with the maximum speed $v_{\max}=20 $m/s and maximum acceleration $a_{\max}=3 $m/$\rm s^2$, and affordable minimum following distance between two vehicles $D^{\min}_{\rm foll}=5$m. The maximum transmission power and sensing power are 50dBm and 50dBm, respectively. The maximum iteration number is 30. For comparison, we also consider traditional PLS scheme with no extra power consumption (denoted as \textbf{traditional PLS with no PC}, such as \cite{yin2021uav,jin2024enhanced,Li2024Joint}). Some key parameters and their values are shown in Table \ref{tab:sim para validation}.

\iffalse
\textcolor{red}{This section presents an evaluation of the effectiveness of the proposed IPLS scheme through simulations, with a comparison to the AN scheme. We investigate the impact of key system parameters on COP, SOP, and $R_s$. The simulations are conducted in a scenario with one transmitting node, one eavesdropping node, and multiple interfering nodes. Only direct perception interference from oncoming vehicles is considered. Additionally, in the PLS scheme, the AN is emitted by a node with a power of 30 dBm at a distance of 200 meters. Some key parameters are provided in Tab. \ref{tab:sim para validation} \cite{huang2019v2x}.}
\fi


\begin{table}[!htb]  \caption{Simulated parameters and values}
\centering
\label{tab:sim para validation}
\begin{tabular}{lll}
\toprule
  Symbol & Meanings &Values\\
\midrule
  $\alpha$            & path-loss exponent                            & [2, 4]         \\
  $R_{\max}$           & maximum sensing distance                    & 200 m \textsuperscript{\cite{series2014systems}}\\
  $F$                 &  frequency band of sensing radar                & 77 GHz         \\
  $\lambda_w$    & wavelength of radar detection signal                                  & 0.0039 m     \\
  $G_t (G_r)$                 & transmitting/receiving antenna gain                                  & 45 dBi \textsuperscript{\cite{al2018stochastic}} \\
  
  $P_{\rm{com}}^{\max}$           & maximum communication power       & 50 dBm    \\
  $P_{\rm{sen}}^{\max}$           & maximum sensing power       & 50 dBm \textsuperscript{\cite{series2014systems} }   \\
  $P_{\rm{com}}$                  & communication power               & 10 dBm    \\
  $P_{\rm sen}$                & sensing power of vehicle's radar      & 10 dBm \textsuperscript{\cite{al2018stochastic}}   \\
  $\theta$             &radar horizontal angle                   & ${60}^\circ $       \\
  $a_0^{\rm{max}}$         & maximum acceleration of the vehicles               & 3 $ \mathrm{m/s^2}$ \\
  $v_{\rm{max}}$           & maximum speed of the vehicles                  & 20 m/s       \\%\\
  $v_{av}$            & average speed of the vehicles                         & 16 m/s       \\
  $D_{\rm lane}$                 & road width                                    & 3.6 m \textsuperscript{\cite{al2018stochastic}}      \\
  $\lambda_I$    & the density of Carols & 0.001 \textsuperscript{\cite{Yu2023The}} \\
  \bottomrule
\end{tabular}
\end{table}

\subsection{Theoretical evaluations of the COP, SOP and SRP}
First, we consider the performance comparison of \textbf{SI-PLS with ST-TP}, \textbf{traditional PLS with no PC} \cite{yin2021uav,jin2024enhanced,Li2024Joint} and with multiple Carols and Eves. We analyze the effects of the noise power, communication/sensing power, and radar horizontal beamwidth on communication reliability and security, as well as the sensing accuracy, as shown in Fig. \ref{fig:COP-Noise} to Fig. \ref{fig:Rs-P0}.

\begin{figure*}
    \centering
\begin{minipage}{0.24\textwidth}
  \centering
  \includegraphics[width=1.55in]{results/COP_NoisePower_alpha.pdf}
  \caption{\small COP vs. the noise power}
  \label{fig:COP-Noise}
\end{minipage}
    \hfill
\begin{minipage}{0.24\textwidth}
   \centering
  \includegraphics[height=1.3in,width=1.55in]{results/SOP_NoisePower_alpha.pdf}
  \caption{\small SOP vs. noise power}
  \label{fig:SOP-Noise}
\end{minipage}
    \hfill
\begin{minipage}{0.24\textwidth}
        \centering
  \includegraphics[width=1.55in]{results/COP_P_alpha.pdf}
  \caption{\small COP vs. commun. power}
  \label{fig:COP-P}
\end{minipage}
  \hfill
\begin{minipage}{0.24\textwidth}
        \centering
  \includegraphics[width=1.55in]{results/SOP_P_alpha.pdf}
  \caption{\small SOP vs. commun. power}
  \label{fig:SOP-P}
\end{minipage}
\end{figure*}

Fig. \ref{fig:COP-Noise} and Fig. \ref{fig:SOP-Noise} illustrate the variations in COP and SOP with the noise power under different settings of the path loss conditions. The results indicate that as the noise power increases, COP gradually increases, while SOP decreases. This is because that the noise deteriorates the reliability performance in the sense that the Bob cannot recover messages successfully, and simultaneously helps the security performance in the
sense that Eves cannot recover messages successfully.
In addition, higher path loss results in the weaker signal strength received by Bob and Eves, decreasing reliability and improving security. The corresponding reasons are similar to these of the noise power. Compared with the \textbf{traditional PLS with no PC} \cite{yin2021uav,jin2024enhanced,Li2024Joint}, on the average, the COP and SOP of \textbf{SI-PLS with ST-TP} can be decreased by $82\%$ and $41\%$ for the settings of $\alpha=2$, respectively. 
This phenomenon indicates that the proposed scheme performs better in terms of secure communications, particularly under conditions of a lower noise power and higher path loss. %Although the PLS scheme compromises some security, secure communication can still be achieved through the optimizations in the next subsection.

The impact of the transmit power on transmission reliability and security is shown in Fig. \ref{fig:COP-P} and Fig. \ref{fig:SOP-P}, respectively. The results indicate that increasing transmit power can achieve better reliability but gives a poorer security, which further corroborates the findings in Fig. \ref{fig:COP-Noise} and Fig. \ref{fig:SOP-Noise}. That is, for \textbf{traditional PLS with no PC} \cite{yin2021uav,jin2024enhanced,Li2024Joint}, a higher path loss exponent contributes to an improved security but a compromised reliability, as this scheme ignores the radar sensing interference. While for \textbf{SI-PLS with ST-TP}, when the transmit power is lower than $10^{-2}\rm{W}$, a greater path loss exponent improves reliability by reducing radar sensing interference from the Carols to the Bob. However, the communication signal is more susceptible to path loss exponent at higher levels of transmit power, necessitating the mitigation of smaller path losses. For the settings of a high path loss, on the average, the COP and SOP of the \textbf{SI-PLS with ST-TP} are reduced by $50\%$ and $35\%$, respectively, when compared to that of the \textbf{traditional PLS with no PC} \cite{yin2021uav,jin2024enhanced,Li2024Joint}. It is evident that traditional paradigms ignore the sensing interference signals, which consequently results in a decline in reliability when such interference is taken into consideration. However, this deficit can be effectively mitigated through the optimization of vehicle trajectory and communication power.

To further validate the impact of the sensing power, Carols density, and radar horizontal beamwidth on the performance of transmission reliability and security achieved by \textbf{SI-PLS with ST-TP}, Fig. \ref{fig:COP-P0} and Fig. \ref{fig:SOP-P0} consider the impact of the radar sensing power transmission reliability and security, respectively. It can be observed that increasing sensing power benefits the security enhancement, as the Carols provide more interference to the Eves, but it compromises reliability performance. The reasons are similar to these in Fig. \ref{fig:COP-Noise} and Fig. \ref{fig:SOP-Noise}. 

%In addition, as shown in Fig. \ref{fig:COP-Density} and Fig. \ref{fig:SOP-Density}, on the one hand, over the density of Carols increasing, the COP increases while the SOP decreases, since there is more sensing interference generated by Carols. On the other hand, a larger horizontal angle results in more interference for Bob and Eves, since a greater covering range is provided, resulting in a higher COP and a lower SOP. Therefore, it is necessary to set parameters reasonably to make a better trade-off between the reliability and security.
\begin{figure*}[htbp]
    \centering
\begin{minipage}{0.24\textwidth}
        \centering
\includegraphics[height=1.3in,width=1.5in]{results/COP_P0_noisepower.pdf}
  \caption{\small COP vs. sensing power}
  \label{fig:COP-P0}
\end{minipage}
  \hfill
\begin{minipage}{0.24\textwidth}
  \centering
  \includegraphics[width=1.55in]{results/SOP_P0_noisepower.pdf}
  \caption{\small SOP vs. sensing power}
  \label{fig:SOP-P0}
\end{minipage}
  \hfill
\begin{minipage}{0.24\textwidth}
  \centering
  \includegraphics[width=1.55in]{results/SRP_range_detectionpower.pdf}
  \caption{\small SRP vs. sensing range}
  \label{fig:SRP_Range(detectionpower)}
\end{minipage}
  \hfill
\begin{minipage}{0.24\textwidth}
	\centering
	\includegraphics[width=1.55in]{results/SRP_Range_density.pdf}
	\caption{\small SRP vs. sensing range}
	\label{fig:SRP-Density}
\end{minipage}
\end{figure*}

Finally, the influence of sensing range with different settings of sensing powers and Carols' densities on the sensing accuracy (i.e., SRP) are shown in Fig. \ref{fig:SRP_Range(detectionpower)} and Fig. \ref{fig:SRP-Density}, respectively. The results reveal that as the sensing range increases, the SRP experiences a significant decline. This is primarily caused by the severe degradation of the sensing signal over the distance increasing. For the cases of sensing power with 30dBm and 40dBm, the performance of sensing accuracy drops sharply when the detection range is below 10m. The reason behind this phenomenon is that the decline becomes slower once the detection distance increases, since the increased interference between the sensing signals that affects the overall sensing performance. In addition, there will be an increased level of interference to the sensing signal due to the presence of a greater number of interferers.


\iffalse
\textcolor{blue}{Finally, we observe the impact of Carol's density, sensing power, sensing distance, and horizontal beamwidth of vehicular radar on the sensing accuracy. As shown in Fig. \ref{fig:SRP_Range(detectionpower)}, compared with the radar sensing power of 30dBm, for the settings of the radar sensing power with 40dBm and 50dBm, over the maximum sensing distance increasing, the SRP decreases quickly first when $R_{\max}$ is below 10m, and then reaches equilibrium. The reason behind this phenomenon is that, for a smaller radar sensing distance, more interference is generated by Carols with a greater radar sensing power; While for settings of sensing power with 30dBm, the SRP shows a gradual decline, since although the expected radar echo and radar sensing interference are decreased over $R_{\max}$ increasing, the difference between them does not be significant due to a smaller sensing power. Fig. \ref{fig:SRP-Density} further illustrates that as the sensing range increases, the SRP decreases significantly, primarily due to the gradual attenuation of the sensed signal over distance. In addition, when the density of interferers increases, the sensing performance is further compromised. This degradation occurs due to the interference between signals, where the presence of multiple nearby sources disrupts the accuracy of the sensed signal.}
\fi

\iffalse
\textcolor{red}{
shows the variation trend of SRP with distance In the case of noise power equaling 30$\rm{dBm}$. It can be observed that as the detection distance decreases, the probability of successful detection increases. SRP changes slowly with distance when the radar detection power is 50$\rm{dBm}$. At detection powers of 10$\rm{dBm}$ and 30$\rm{dBm}$, the change is more pronounced for distances less than 10$\rm{m}$, becoming more gradual between 10$\rm{m}$ and 100$\rm{m}$. }
\fi





\iffalse
Considering the influence of different Carol's densities and maximum sensing distance, as shown in Fig. \ref{fig:SRP-DetectionPower(range)}, it can be noticed that the SRP also presents a downward trend, and the reason is similar to that of Fig. \ref{fig:SRP-Density}. But the decreasing trend in Fig. \ref{fig:SRP-DetectionPower(range)} is relatively slow, which can greatly
give significant insights that, in a crowded connected car environment, using a smaller sensing power will help lower power consumption and enhance sensing accuracy.

\begin{figure}[!ht]
\centering
\includegraphics[width=2.2in]{SRP_DetectionPower_range.pdf}
\caption{\small SRP VS. Detection power in different sensing distance}
\label{fig:SRP-DetectionPower(range)}
\end{figure}
\fi

\subsection{Optimization of $TRSA\_SR$ maximization}
To better observe the performance of \textbf{SI-PLS with ST-TP} and simplify representations of the earliest transmit time and latest transmit time of the Alice, we consider a scenario with an Eve and a Carol, and all vehicles move in a straight line along the current road. The maximum speed of the Alice is 20 m/s, while other vehicles move at a constant speed of  16 m/s. As shown in Fig. \ref{fig:Position}, the initial position of the vehicles is at $t[0]$, the secrecy rate by jointly optimizing the trajectory and transmit power of the Alice between  $t[k_{\rm start}]$ and $t[k_{\rm end}]$.


\begin{figure}[h]
\centering
\includegraphics[width=3in]{results/Position.pdf}
\caption{\small vehicle positions in different time slots}
\label{fig:Position}
\end{figure}

\begin{figure}[h]
\centering
\includegraphics[width=2.2in]{results/OPT_P_T_P0.pdf}
\caption{\small communication power vs. different time slots}
\label{fig:OPT-P_T}
\end{figure}

\begin{figure}[h]
\centering
\includegraphics[width=2.2in]{results/OPT_V_Rs.pdf}
\caption{\small moving speed and secrecy rate vs. different time slots}
\label{fig:OPT-V-Rs}
\end{figure}



At the beginning of 10-th time slot, the Eve suffers from the sensing interference generated by the Carol until 58-th time slot. During this time, the Alice optimizes the straight trajectory by adjusting its speed while keeping a safe distance with the Bob. Fig. \ref{fig:Position} and Fig. \ref{fig:OPT-P_T} describe the straight trajectory and transmit power of the Alice across different time slots. The variation of corresponding secrecy rate is shown at the bottom of Fig. \ref{fig:OPT-V-Rs}. It can be noticed from the top of Fig. \ref{fig:OPT-V-Rs} that  
although the Alice begins to optimize its trajectory and transmit power at the $10$-th time slot, the wiretap channel quality remains superior to that of the Bob, resulting in poor signal security, and then the transmit power is set to 0. From the $49$-th time slot to the 52-th time slot, by optimizing the straight trajectory and transmit power, the Bob obtains a better channel quality than that of the Eve. But the achieved secrecy rate is relatively lower, since the sensing interference affects the Bob and Eve simultaneously. Furthermore, the achieved secrecy rate rises rapidly at 53-th time slot, since the Bob moves out the radar sensing range of the Carol, only the Eve suffers from its sensing interference, and the straight trajectory and transmit power are also optimized. At the 58-th time slot, the Eve moves out the radar sensing range of the Carol, under which the Alice stops transmitting the confidential information and then the secrecy rate gets to zero.

\iffalse
\begin{figure}[!ht]
\centering
\includegraphics[width=2in]{results/OPT_P_T_P0.pdf}
\caption{\small transmit power vs. different time slots}
\label{fig:OPT-P_T}
\end{figure}

\begin{figure}[!ht]
\centering
\includegraphics[width=2in]{results/OPT_V_Rs.pdf}
\caption{\small driving speed and secrecy rate vs. different time slots}
\label{fig:OPT-V-Rs}
\end{figure}
\fi

\iffalse
resulting in poor signal security and the transmission power being set to zero. Alice sends communication signals when Bob's channel quality is better than Eve's. Alice optimizes power to achieve maximum secrecy rate if Bob's channel is superior. During $49-52$ time slots in this figure, the channel difference between the two is minimal, resulting in lower secrecy rates. After the $52$-th time slot, Bob moves out of Carol's interference perception range, so Alice maintains high levels of transmission power and secrecy rate. Furthermore, Fig. \ref{fig:OPT-P_T} illustrates the variation of communication transmission power with time slots under different sensing powers. In time slots 48-52, there is a significant difference in communication power. This is because higher sensing power leads to increased interference for vehicles. To achieve reliable transmission, Alice must use higher communication power.
\fi

%Alice's speed changes as in Fig. \ref{fig:OPT-V-t}. 



\begin{figure}[h]
\centering
\includegraphics[width=2.2in]{results/OPT_Iterm.pdf}
\caption{\small $\rm TRSA\_SR$ vs. iterations with different noise powers}
\label{fig:OPT-iter}
\end{figure}


Fig. \ref{fig:OPT-iter} shows the convergence curve of the $\rm TRSA\_SR$ over the number of iterations. The secrecy rate achieved by using \textbf{traditional PLS with no PC} \cite{yin2021uav,jin2024enhanced,Li2024Joint} is zero because the Eve is closer to the Alice and has a better channel quality than that of the Bob. For \textbf{SI-PLS with ST-TP}, it can be seen that before the initial stage when the number of iterations is less than 4, the average secrecy rate gradually increases. After executing 5 iterations, $\rm TRSA\_SR$ reaches a relative stable stage, which means that the algorithm has converged to an optimal solution. To sum up, it can be concluded that the proposed algorithm effectively solves problem \eqref{eq:opt0}. 


\iffalse
\textcolor{red}{Fig. \ref{fig:OPT-iter} illustrates the changes in average secrecy rate across different iterations. It can be observed that the optimization results fluctuate significantly in the initial stages but gradually increase over time. Starting from the $5$-th iteration, the results begin to stabilize, ultimately converging to the optimal solution. The higher average secrecy rate obtained in the first and second iterations is due to the initial transmission power being 0 in time slots $48-52$, resulting in a higher average for time slots $53-58$. With Alice's trajectory optimization, secure transmission is now feasible in time slots 48-52, while the secrecy rate is much lower than the average secrecy rate. Therefore, the average over time slots $48-58$ has decreased.}
\fi


\begin{figure}[h]
\centering
\includegraphics[width=2.2in]{results/Rs_Noisepower_P0.pdf}
\caption{\small $\rm TRSA\_SR$ vs. noise power with different sensing powers}
\label{fig:Rs-Noise}
\end{figure}

\begin{figure}[h]
\centering
\includegraphics[width=2.2in]{results/Rs_P0_Rmax.pdf}
\caption{\small $\rm TRSA\_SR$ vs. sensing power with different sensing ranges}
\label{fig:Rs-P0}
\end{figure}
Fig. \ref{fig:Rs-Noise} and Fig. \ref{fig:Rs-P0} present how average $\rm TRSA\_SR$ varies with different values of the noise power, radar sensing power and radar sensing range, respectively. It can be observed from Fig.\ref{fig:Rs-Noise} that, on the one hand, $\rm TRSA\_SR$ decreases over noise power increasing, since it affects the channel quality of the Bob more severely when compared with that of the Eve. On the other hand, a greater radar sensing power can obtain a higher $\rm TRSA\_SR$. The reason is that, via joint design of straight trajectory and transmit power, the Carol can suppress the Eve more heavily compared with the Bob, which also validates the effectiveness of our proposed \textbf{SI-PLS with ST-TP}. Furthermore, Fig. \ref{fig:Rs-P0} demonstrates this conclusion. In addition, we can observe that the radar sensing power should be selected properly for a greater sensing range. The reason is that although a greater radar sensing power can achieve a higher $\rm TRSA\_SR$, a greater radar sensing range with a greater sensing power can determinate the channel quality of the Bob more seriously, then $\rm TRSA\_SR$ decreases. Combining with the results in Fig. \ref{fig:Position}, the sensing accuracy and transmission security can be ensured by selecting a proper setting of the radar sensing power and radar sensing range.

\iffalse
The secrecy rate rises as the sensing power increases, since higher sensing power effectively inhibits Eve from acquiring information, especially when Bob is out of Carol's interference range. Additionally, the extended radar range will lead to secrecy rate decreased 
because with greater range, Bob and Eve experience prolonged sensing interference. It is noteworthy that excessive sensing power may cause a reduction in secrecy rate, as it increases sensing interference to Bob when $R_{\max}=220m$.
\fi

\iffalse
\textcolor{green}{Overall, the IPLS scheme effectively enhances communication security performance, albeit with some sacrifice in reliability. By optimizing transmission timing, trajectory, and communication power control for Alice, we can substantially improve the reliability of the scheme while enhancing the secrecy rate, thereby achieving both secure and dependable transmission.}
\fi

\section{Conclusion}\label{sec:conclusion}
In this paper, we leverage the radar sensing interference to enable PLS of the IoV, rather than the communication interference, which is the core idea of traditional PLS techniques. The metrics for sensing accuracy, transmission reliability and security have been thoroughly evaluated. By jointly designing the transmit power and straight trajectory of the Alice, we formulate an optimization problem to maximize the secrecy rate of sensing interference enabled PLS. The optimization problem is non-convex and difficult to be solved directly, and then is decomposed two sub-problems and effectively addressed by using BCD and SCA methods. Furthermore, an AO algorithm is introduced to solve the original optimization problem. The feasibility of the proposed algorithm is verified by time complexity and convergence analysis. Simulations demonstrated the effectiveness of the proposed PLS method, and the impact of key system parameters on the sensing accuracy, transmission reliability and security. 

\iffalse
In this paper, we propose a new framework, the sensing enabled IPLS scheme, to address the issues present in the traditional PLS scheme within IoV environments. This scheme uses sensing interference to replace AN for disrupting eavesdroppers. Firstly, by constructing a communication-sensing coexisting network model, we derive closed-form expressions for the performance metrics of communication and sensing, include COP, SOP, SRP and JSAC-SRM Then, we maximize the interference experienced by the eavesdropper by optimizing the transmission timing. Additionally, we jointly optimize the transmitter's power and trajectory to maximize the secrecy rate. Finally, simulations validate the effectiveness of the IPLS scheme. Finally, we validated the impacts of various network parameters on communication ability and employed simulation study to validate the efficiency of IPLS scheme.
\fi


\iffalse
{\appendix[APPENDIX A]
Based on the concept of the COP in Eq. \eqref{eq:P_co}, we get Eq. \eqref{eq:COP1}.
\begin{figure*}[t] % t选项是尽量放置在页面顶部
\begin{equation} \label{eq:COP1}
\begin{aligned}
{{p}_{co}}  \quad 
%& =1-\mathrm{Pr}\left[ \frac{qh_{ab}{{L}_{ab}}^{-\alpha }}{W_b+ P^{each}_b+{{I}_{b}}}\ge {{\beta }_{b}} \right] \\ 
& =1-\mathrm{Pr}\left[ {\frac{P_{\rm com}{{h}_{a\rightarrow b}}{\bar{X}^{\rm hor}_{a\rightarrow b}}{^{-\alpha }}}{{\sigma^{2}_{b}+ P^{\rm Bob}_{\rm{echo}}}+\sum\limits_{c_i\in {{\Phi }_{c_i\rightarrow b}}}{{{A}_{\rm ea}}S{{h}_{c_i\rightarrow a}}{\Vert(X^{\rm hor}_{c_i\rightarrow b}, Y^{\rm ver}_{c_i\rightarrow b}) \Vert}{^{-\alpha} }}} }\geq \beta_b \right] \\ 
%& =1-\mathrm{Pr}\left[ {{h}_{a\rightarrow b}}\ge \frac{{{\beta }_{b}}{{\bar{X}^{\rm hor}_{a\rightarrow b}}{^\alpha}}}{P_{\rm com}}\left( {\sigma^{2}_{b}+ P^{\rm Bob}_{\rm each}}  +\sum\limits_{c_i\in {{\Phi }_{c_i\rightarrow b}}}{{{A}_{\rm ea}}S{{h}_{c_i\rightarrow b}}{\Vert(X^{\rm hor}_{c_i\rightarrow b}, Y^{\rm ver}_{c_i\rightarrow b}) \Vert}^{-\alpha }} \right) \right] \\ 
& \overset{(a)}=1-\exp \left[ -\frac{{{\beta }_{b}}{{\bar{X}^{\rm hor}_{a\rightarrow b}}{^\alpha}}}{P_{\rm com}}\left( {\sigma^{2}_{b}+ P^{\rm Bob}_{\rm{echo}}}+\mathbb{E}_h \left[ { \sum\limits_{c_i\in {{\phi }_{c_i\rightarrow b}}}{{{A}_{\rm ea}}Sh_{c_i\rightarrow b}{\Vert(X^{\rm hor}_{c_i\rightarrow b}, Y^{\rm ver}_{c_i\rightarrow b}) \Vert}^{-\alpha }} }\right] \right) \right] 
\end{aligned}
\end{equation}
%where $F=\frac{{{\beta }_{b}}{{\bar{X}^{\rm hor}_{a\rightarrow b}}^\alpha}}{P_{\rm com}}$.

\end{figure*}
In particular, considering the Rayleigh fading model, term (a) holds due to $\mathrm{Pr}\left[ h \ge z \right] =  \exp \left( -z \right)$ \cite{zhu2014secrecy}, and detail derived process is given in Eq. \eqref{eq: h expection}.
\begin{figure*}[t] % t选项是尽量放置在页面顶部
\begin{equation}\label{eq: h expection}
\begin{aligned}
{{\mathbb{E}}_{h}}\left[ \exp \left( -\sum\limits_{c_i\in {{\Phi }_{c_i\to b}}}{{{A}_{\rm ea}}Sh_{c_\rightarrow b}{\Vert(X^{\rm hor}_{c_i\rightarrow b}, Y^{\rm ver}_{c_i\rightarrow b}) \Vert}^{-\alpha }} \right) \right] 
 & \overset{(b)}={{\mathbb{E}}_{h}}\left[ \exp \left[ -2\pi {{\lambda }_{i}}\int_{0}^{+\infty }{r\left( 1-\exp \left( {{A}_{\rm ea}}S{h_{c_i\rightarrow b}}{{r}^{-\alpha }} \right) \right)}dr \right] \right] \\ 
 & \overset{(c)}=\exp \left[ -\pi {{\lambda }_{i}}{{\left( {{A}_{\rm ea}}S \right)}^{\frac{2}{\alpha }}}\Gamma \left( 1+\frac{2}{\alpha } \right)\Gamma \left( 1-\frac{2}{\alpha } \right) \right] \\ 
\end{aligned}
\end{equation}
\hrule % 分割线
\end{figure*}
where term $(b)$ holds based on the fact of %\cite{courant1965introduction}.
\begin{displaymath}
\int_{{{R}^{2}}}{f\left( r \right)dr}=\int_{0}^{2\pi }{d\theta }\int_{0}^{+\infty }{rf\left( r \right)}dr=2\pi \int_{0}^{+\infty }{rf\left( r \right)}dr
\end{displaymath}


To sum up, the theorem yields the claim.

\vspace{-1em}
\renewcommand{\appendixname}{}
\appendix[APPENDIX B]
Based on the concept of the SOP in Eq. \eqref{eq:concept of sop}, we know that the SOP happens once at least an Eve recovers the transmitted information successfully. Let $\mathcal{Z}_E \left( e\right)$ be a function that the value is one if there exists an Eve $e_j\in \Phi_e$ intercepting the transmitted information; otherwise, the value is 0 \cite{zhu2014secrecy,Yu2023The}. If all transmissions are secure, the value of the $\prod\limits_{e\in {{\Phi }_{e}}}{\left( 1-{{\mathcal{Z}}_{E}}\left( e \right) \right)}$ is 0. Mathematically, we get
\begin{displaymath} \label{eq:sop1}
\begin{aligned}
{{p}_{so}} 
& =1-{{\mathbb{E}}_{{{\Phi }_{i\to e}}}}\left\{ {{\mathbb{E}}_{{{\Phi }_{e}}}}\left\{ {{\mathbb{E}}_{{h}}}\left\{ \prod\limits_{e\in {{\Phi }_{e}}}{\left( 1-{{\mathcal{Z}}_{E}}\left( e \right) \right)} \right\} \right\} \right\} \\ 
& \overset{(d)}=1-{{\mathbb{E}}_{{{\Phi }_{c_i\to e_j}}}}\left\{ {{\mathbb{E}}_{{{\Phi }_{e}}}}\left\{ \prod\limits_{e\in {{\Phi }_{e}}} \left[ 1-\Pr \left[ \gamma^{\rm com}_{e_j}\ge {{\beta }_{e}}\left| \right.\right.\right. \right.\right.\\
&~~~\left.\left.\left.\left.{{\Phi }_{e}},{{\Phi }_{c_i\rightarrow e_j}} \right] \right] \right\} \right\} \\ 
& =1-{{\mathbb{E}}_{{{\Phi }_{c_i\rightarrow e_j}}}}\left\{ \exp \left[ -{{\lambda }_{e}}\int_{{{R}^{2}}}{\Pr \left[  \gamma^{\rm com}_{e_j}\ge {{\beta }_{e}}\left| {{\Phi }_{c_i\rightarrow e_j}} \right. \right]} \right] \right\}\\ 
\end{aligned}
\end{displaymath}
term $(d)$ holds because the fading coefficients of each link are independent. Next, for ease of representation, assuming that the density of the Carols generating radar sensing interference to any Eves is the same. According to the Jensen's inequality, % \cite{Garling2007}, 
we get
\begin{displaymath}
\begin{aligned}
  {{p}_{so}}
  &\le 1-\exp \left[ -{{\lambda }_{e}}\int_{{{R}^{2}}}{\Pr[\gamma_{e}^{\rm com}\ge {{\beta }_{e}} ]}dr \right] \\ 
 & \overset{(e)}= 1-\exp \left[ -2\pi {{\lambda }_{e}}\int_{0}^{\infty }\exp \left[ -{{\left( \frac{{{\beta }_{e}}{{A}_{\rm ea}}S}{P_{\rm com}} \right)}^{\frac{2}{\alpha }}}\pi {{\lambda }_{i}}{{r}^{2}}C\left( \alpha  \right) \right. \right. \\
 & \left. \left. ~~~~ -\frac{{{\beta }_{e}}}{P_{\rm com}}{ \left( \sigma^{2}_{e} + P^{e_j}_{\rm{echo}} \right)}{{r}^{\alpha }} \right]rdr \right] ,\\ 
\end{aligned}
\end{displaymath}
where step $(e)$ is obtained through polar coordinate transformation. Also, $\Pr[\gamma^{\rm com}_{e_j}\geq \beta_e]$ can be derived by using the same method in Eq. \eqref{eq:COP1}.

The lower bound of SOP is obtained by considering the success probability at the Eve nearest to the Alice, denoted by $e_1$, and the distance between them is denoted as $d_{a\rightarrow e_1}$. According to the characteristics of transmission distance under nearest receiver transmission (NRT) model \cite{zhu2014secrecy,Yu2023The}, the PDF of $d_{a\rightarrow e_1}$ is given by $f\left( {{d}_{a\rightarrow {{e}_{1}}}} \right)=2{{\lambda }_{e}}\pi {d_{a\rightarrow e_1}}\exp \left( -{{\lambda }_{e}}\pi {d^{2}_{a\rightarrow e_1}} \right)$. Then, we have
\begin{displaymath}
\begin{aligned}
 {{p}_{so}}  &  \ge \Pr \left[ \gamma_{{{e}_{1}}}^{\rm com}\ge {{\beta }_{e}} \right] \\ 
% & =\int_{0}^{\infty }{\Pr \left[ \gamma_{{{e}_{1}}}^{\rm com}\ge {{\beta }_{e}} \right]f\left( {{d}_{a\rightarrow{{e}_{1}}}} \right)d}{{d}_{a\rightarrow{{e}_{1}}}} \\ 
 & =2{{\lambda }_{e}}\pi \int_{0}^{\infty }\exp \left[ -{{\left( \frac{{{\beta }_{e}}{{A}_{\rm ea}}S}{P_{\rm com}} \right)}^{\frac{2}{\alpha }}}\pi {{\lambda }_{i}}C\left( \alpha  \right){r}^{2} \right. \\
 & \left. ~~~ -\frac{{{\beta }_{e}}}{P_{\rm com}}{ \left( \sigma^{2}_{e} + P^{e_1}_{\rm{echo}} \right)}{r}^{\alpha }-\pi {{\lambda }_{i}}{r}^{2} \right]{r}d{r}.  
\end{aligned}
\end{displaymath}
Substituting $\alpha = 4$ to calculate the upper bound and lower bound of SOP as Eq. \eqref{eq:sop_u} and Eq. \eqref{eq:SOP_L}, respectively.


\vspace{-1em}
\renewcommand{\appendixname}{}
\appendix[APPENDIX C]
Combining with the concept of the SRP in Eq. \eqref{eq:SRP0}, the sensing interference at the Alice, and using similar expression of Eq. \eqref{eq:sinr Bob}, we get
\begin{displaymath}
\begin{aligned}
 {{p}_{sr}}  & =\Pr\left[ \sigma >\eta {R_{\rm{tar}}^{2\alpha }}\left( {{\sigma}^{2}_{a}}+{{I}_{\rm Alice}} \right) \right],\\ 
\end{aligned}
\end{displaymath}
where $\eta =\frac{{{\left( 4\pi  \right)}^{3}}\beta_s }{P{{G}_{t}}{{G}_{r}}{{\lambda }_{w}}^{2}}$, and ${\sigma}^{2}_{a}$ is the AWGN at the Alice.

By using the PDF of $\sigma$ in Eq. \eqref{eq:p_sigma }, above formula is further obtained as
\begin{displaymath} \label{eq:Pa}
\begin{aligned}
  {{p}_{sr}} & =\int_{\eta {R_{\rm{tar}}^{2\alpha }}\left( {{\sigma}^{2}_{a}}+{{I}_{\rm Alice}} \right)}^{+\infty }{p\left( \sigma  \right)d}\sigma  \\ 
% & =\frac{{{\delta}^{\delta}}}{\Gamma \left( \delta \right){\bar{\sigma }_{av}}^{\delta}}\int_{\eta { {R_{\rm{tar}} }^{2\alpha }}\left( {{\sigma}^{2}_{a}}+{{I}_{\rm Alice}}  \right)}^{+\infty }{{{\sigma }^{\delta-1}}\exp \left( -\frac{ \rm{\delta} }{{\bar{\sigma }_{av}}}\sigma  \right)d}\sigma  \\ 
 & \overset{(f)} = \Gamma \left( \delta,\frac{\delta}{{\bar{\sigma }_{av}}}\eta {{ R_{\rm{tar}} }^{2\alpha }}\left( {{\sigma}^{2}_{a}}+\mathbb{E}_h\left[ {{I}_{\rm Alice}} \right] \right) \right)/\Gamma \left(\delta \right). \\ 
\end{aligned}
\end{displaymath}
Term $(f)$ holds because of $\Gamma \left( m,x \right)=\int_{x}^{\infty }{{{e}^{-t}}{{t}^{m-1}}dt}$. Moreover, based on the characteristics of Rayleigh fading gain, $\mathbb{E}_h\left[ {{I}_{\rm Alice}} \right]$ can be derived as follows:
\begin{displaymath}
\begin{aligned}
  \mathbb{E}_h\left[ {{I}_{\rm Alice}} \right] & ={\mathbb{E}_{{{\Phi }_{c_i\to a}},h_{c_i\rightarrow a}}}\left[ \sum\nolimits_{c_i\in {{\Phi }_{c_i\rightarrow a}}}{{{A}_{\rm ea}}S{h_{c_i\rightarrow a}}{{d}_{c_i\rightarrow a}}^{-\alpha }} \right] \\ 
 & \overset{(g)}={{\lambda }_{i}}{{A}_{\rm ea}}S\cdot \int_{2D_{\rm lane}}^{+\infty }{{{r}^{-\alpha }}dr} \\ 
 & =\frac{{{\lambda }_{i}}{{A}_{\rm ea}}S}{\alpha -1}{{\left( 2D_{\rm lane} \right)}^{1-\alpha }}. 
\end{aligned}
\end{displaymath}
Term $(g)$ is valid because  ${{\Phi }_{c_i\to a}}$ and $h_{c_i\rightarrow a}$ are independently identically distribution. Finally, the SRP in Eq. \eqref{eq: closed expression srp} can be obtained by substituting $\alpha=4$ into above formula.
}
\fi

\bibliographystyle{IEEEtran}
\bibliography{ref}
\end{document}

