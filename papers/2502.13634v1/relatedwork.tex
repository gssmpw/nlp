\section{Related Works}
\label{sec:related work}

%\textcolor{red}{studies on AN, moving RIS, UAV moving trajectory optimization}

%\textcolor{red}{Traditional idea: Communication interference centered PLS design}
Wireless PLS technologies become particularly critical in the IoV, in terms of secure autonomous driving, vehicle-to-infrastructure. Concurrently, many efforts on the PLS design focused on the communication interference utilization by considering different legitimate and eavesdropping channels, such as the artificial noise (AN) scheme, joint design of the phase shift matrix and transmission power of reconfigurable intelligent surface (RIS), and the joint design of the transmission power and trajectory of the UAV.

\iffalse
\sout{Wireless PLS becomes particularly critical in the IoV, in terms of secure autonomous driving, vehicle-to-infrastructure communication, and emergency communication. Concurrently, many efforts on the PLS design focused on the communication interference utilization by considering different legitimate and eavesdropping channels. }From the perspective of communication interference utilization and trajectory optimization, the most representative methods includes AN-based BF design, joint design of the phase shift matrix and moving trajectory of IRS, joint design of the transmit power and trajectory of the UAV, and joint design of the BF and moving trajectory of the movable antenna (MA). 
\fi

\subsection{The communication interference based PLS in IoV systems}
AN scheme is one of most classic PLS techniques to effectively reduce the quality of eavesdropping channel, thereby enhancing the security.
In particular, \emph{Wang et al.} explored PLS in cellular vehicular networks, under which the legitimate user transmits confidential information while generating AN signals simultaneously for enhancing communication security \cite{wang2020physical}. Similarly, by design optimal AN-aided BF, \emph{Zhang et al.} proposed a Layered PLS model that minimizes transmission power while satisfying secrecy rate requirements \cite{Zhang2019Transmit}. The importance of attracting AN to the PLS were also emphasized in \cite{yin2021uav,jin2024enhanced}.
Featured by establishing line-of-sight (LoS) channels, the UAV and RIS provide new potential advantages in suppressing the wiretap channel. On the basis of this idea, in \cite{Li2019UAV}, \emph{Li et al.} regarded moving UAVs as jammers to generate AN signals for interfering with Eves, and jointly optimized the UAVs' trajectory and transmission power to enhance the security. \emph{Chen et al.} utilized RIS-assisted V2V communications and derived the upper bound of the secrecy capacity and the approximate expression of the secrecy outage probability under Rayleigh fading channel \cite{Chen2024Physical}.
It was demonstrated that optimizing UAV trajectory can effectively improve communication performance in \cite{Xu2021Low,Zhou2018Improving}.

\iffalse
\sout{By using the AN-assisted BF and optimizing power allocation between legitimate signals and AN signals, \emph{Jin et al.} emphasized the importance of attracting AN to the PLS \cite{jin2024enhanced}.Furthermore, considering the role of AN-BF in characterizing the trade-off between the reliability and security, they introduced the concept of effective secrecy throughput to quantify the average data rate of secure transmission of confidential information.} %Similar works have been also done in  \cite{Saqib2024reconfigurable,Liu2022Throughput,Li2023Multi}.
In addition, featured by line-of-sight (LoS), the UAV and RIS provide new potential advantages in suppressing the Eves' channel. On the basis of this idea, in \cite{Li2019UAV}, \emph{Li et al.} regarded moving UAVs as jammers interfering with Eves, and jointly optimized the UAVs' trajectory and transmit power for generating jamming signals to enhance the security. \textcolor{blue}{\emph{Chen et al.} utilized RIS-assisted vehicle-to-vehicle (V2V) communication and derived the upper bound of the secrecy capacity and the approximate expression for the secrecy outage probability (SOP) under double Rayleigh fading channels \cite{Chen2024Physical}.
Furthermore, in \cite{Li2019UAV,Xu2021Low}, it was demonstrated that optimizing UAV trajectories can effectively improve communication performance. \sout{Although various vehicles trajectory optimization schemes, such as \cite{Liu2022Dynamic, Li2022Optimization}, have been proposed for V2V communication, studies specifically leveraging vehicle trajectory optimization to enhance communication performance remain relatively scarce.} }
\fi

\iffalse
\sout{While \emph{Xu et al.} divided the UAV's transmit power into two parts: one for transmitting confidential signals and another one for generating AN signals. Next, they optimized the power allocation ratio and the UAV's trajectory to maximize the average secrecy rate \cite{Xu2021Low}. Furthermore, the impact of UAV' jamming power and positions on the reliability and security was studied in \cite{Zhou2018Improving}. In addition, the UAVs can be utilized as relays to generate the AN signals to confuse the Eves. Therefore, \emph{Yin et al.} applied this idea to  
achieve secure vehicle communications by optimizing the transmit power of UAVs and satellite transmission beams together \cite{yin2021uav}.}
\fi


However, in the context of ISAC-enabled IoV systems, communication and sensing frequency bands are often close, overlapping, or even identical, which results in communication-sensing coupled interference. %\cite{Zhang2024Coexistence,su2020secure}. 
Therefore, above PLS methods centered on communication interference cannot be directly extended to the secure information transmission in IoV systems featured by coupled interference. Moreover, although various vehicles trajectory optimization schemes, such as \cite{Liu2022Dynamic, Li2022Optimization}, have been proposed for V2V communication, studies specifically leveraging vehicle trajectory optimization to secure wireless communications remain relatively scarce.

In \cite{chu2023joint}, \emph{ Chu et al.} considered the design of 
PLS methods in an ISAC-based system. With the assumption of Eves' perfect CSI known beforehand, strong radar sensing signals were used to suppress the eavesdropping channel, then joint optimization of secure transmission BF and radar receiving filters was done for enhancing the system security. Otherwise, 
all available power resources of the base station and radar were utilized to design BF matrix and radar receiving filters for generating AN signals as much as possible to Eves. However, the aforementioned scenario is based on the assumption that all user locations are fixed, and legitimate users are highly susceptible to radar interference. This makes it difficult to apply to highly dynamic IoV scenarios.
In addition, many studies have revealed that optimizing the phase-shift matrix of RIS in IRS-enabled ISAC systems can extend their coverage \cite{qin2023joint,salem2022active}.
%\cite{zhang2022active,abeywickrama2020intelligent,qin2023joint}. 
Accordingly, \emph{Salem et al.} proposed a PLS solution to maximize achievable secrecy rate,
by jointly designing the receiving BF for radar, reflection coefficient matrix of RIS, and BS' transmitting BF \cite{salem2022active}.
Although these results point out that the sensing signals can assist transmission reliability, they do not address the communication security from the perspective of PLS constrained by communication-sensing coupled interference.

\subsection{Radar sensing interference modeling in ISAC-based IoV}

\iffalse
To describe radar sensing capability, \emph{Martin et al.} introduced the concept of \emph{interruption} into radar networks, using radar interruption as a performance metric defined as the situation where a radar network cannot detect a specific object due to interference from others \cite{braun2013co}. \emph{Brooker} showed a high probability of interference in overlapping frequency bands by studying interference under different conditions and sensor types. Furthermore, they investigated mutual interference between mmWave radar systems operating in the 77 GHz and 94 GHz frequency bands \cite{brooker2007mutual}. Researchers have developed mathematical models to gain a deeper understanding of the mutual interference between automotive radars and to predict the degree of sensing interference in different scenarios. In \cite{al2018stochastic}, \emph{Al-Hourani et al.} pioneered the modelling of automotive radar interference based on stochastic geometry tools. They analyzed the interference between opposing lane radars using Poisson and lattice models, and proposed to estimate the probability of success of radar ranging based on closed-form interference statistics. In \cite{Fang2020Stochastic}, \emph{Fang et al.} studied radar interference in bidirectional multilane scenarios and modelled target radar cross section (RCS) fluctuations using Swerling I and Chi-Square models, obtaining closed-form expressions for radar SRP. These studies provide the foundation for the integration of communication and sensing.
\fi

To measure the performance of the sensing capability, in \cite{braun2013co}, \emph{Martin et al.} introduced the concept of \emph{interruption} into radar networks, which refers to the situation where a radar network cannot detect a specific object due to interference from others. A high probability of interference exists in overlapping frequency bands was proved by \emph{Brooker}, then mutual interference between mmWave radar systems operating in 77 GHz and 94 GHz frequency bands was investigated \cite{brooker2007mutual}. Subsequently, many researches have developed mathematical models to make a deeper understanding of the mutual interference among radars and predict the degree of sensing interference in different scenarios. 
For instance, 
in \cite{al2018stochastic}, \emph{Al-Hourani et al.} pioneered the modelling of automotive radar sensing interference based on stochastic geometry tools. They analyzed the interference between opposing lane radars using Poisson and Lattice models and estimated the SRP based on closed-form interference statistics. 
In \cite{Fang2020Stochastic}, \emph{Fang et al.} studied radar sensing interference in bidirectional multilane scenarios and modelled target radar cross section (RCS) fluctuations using Swerling I model and Chi-Square model, obtaining closed-form expressions for radar sensing SRP. These studies provide the foundation for the integration of communication and sensing in IoV systems.

To better understand communication-sensing coexisting networks, in \cite{Zhang2024Coexistence}, considering radar and communication systems, \emph{Zhang et al.} focused on joint design of communication precoders, radar transmit waveforms, and receiving filters to suppress mutual interference between communication and sensing. 
To deal with the communication-sensing coupled interference, in \cite{su2020secure}, \emph{Su et al.} introduced AN scheme at the source to minimize the SINR at sensing radar, thereby ensuring the expected signal strength of legitimate users. However, the transmit power is increased by introducing external AN into the system. To mitigate power cost, \emph{Lynggaard} employed the assumption of perfect CSI to predict the minimum transmit power required to overcome interference, at the cost of an increased computational complexity \cite{lynggaard2018using}. From above works, to some extent, we can find that although the proposed methods have improved the performance of communication and sensing, additional power consumption and computational burden were also introduced.

Table \ref{tab:methods} summaries the existing PLS methods and highlights the differences between our proposed and other works.
When applying the radar sensing interference in PLS design in IoV, the sensing accuracy, transmission reliability and security should be considered simultaneously. Therefore, on the one hand, the performance analytical framework of ISAC systems considering communication-sensing coupled interference should be established, which describes the relationship between communication and sensing capabilities. On other hand, from the perspective of radar sensing interference utilization, no studies investigate the confidential communication by jointly designing transmission power and trajectory of the source.
This motivates our research in this work. 
%\iffalse
In addition, the difference between our work and previous studies is described as follows.
\begin{itemize}
  \item Different from PLS methods based on AN schemes \cite{wang2020physical,Zhang2019Transmit,Xu2021Low}, which did not consider the communication-sensing coupled interference, we make an effective transformation from ``communication interference based security'' to ``sensing interference based security'';
  \item In the context of ISAC-IoV, different from using sensing/AN interference in studies of PLS design \cite{salem2022active,su2020secure}, only the sensing interference is utilized to secure wireless communications, without extra energy resources;
  %we fully utilize the perceived interference signals within the system to suppress the eavesdropping channel capacity, thereby ensuring the secure transmission of communication signals. Different from the same frequency interference suppression scheme in \cite{goppelt2011analytical}\cite{su2020secure}\cite{lynggaard2018using}, this approach aims to make full use of existing internal signals to enhance system performance.
  %\item \textcolor{green}{Due to significant interference caused by sensing signals, the current popular communication interference metrics cannot be used to evaluate sensing-enabled ISAC networks. We provide a new method for measuring communication reliability and sensing accuracy, taking into account the impact of sensing performance on communication.}
  \item Instead of transmitter trajectory optimization \cite{Li2019UAV,Xu2021Low,Sun2021Unmanned}, the duration of confidential information transmission constrained by limited horizontal angle of Carols' sensing beamwidth, the trajectory and transmission power of the Alice are considered simultaneously.
\end{itemize}
%\fi