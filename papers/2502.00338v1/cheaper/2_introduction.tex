\section{Introduction}
Accurate weather forecasting is crucial for disaster prevention, optimizing agriculture and energy planning, and ensuring water resource management~\cite{chen2023fengwu,pathak2022fourcastnet,ahmed2021improved}. Traditional Numerical Weather Prediction (NWP) methods~\cite{bauer2015quiet} rely on the numerical solution of atmospheric dynamic equations~\cite{achatz2023multiscale,buzzicotti2023spatio}, ensuring consistency across spatiotemporal scales from a physical perspective. However, with the growing volume of observational and historical data, and the increasing demand for high-resolution and long-term forecasts, NWP methods often struggle with computational costs and fail to fully leverage the potential value of vast data.

\begin{figure}[t]
  \centering
\includegraphics[width=0.95\linewidth]{figures/typhoon.png}
\vspace{-10pt}
\caption{Forecast results of extreme cyclones.  (a) OneForecast's predicted wind speed for Typhoon Molva (2020) at 850 hPa pressure level with a 60-hour lead time. (b)–(c) the predicted cyclone tracks of Typhoon Yagi (2018) and Typhoon Molva (2020) using different models}
  \label{fig:icml_intro}
\vspace{-15pt}
\end{figure}

Recent developments in deep learning (DL) methods offer new perspectives for weather forecasting. Early spatio-temporal prediction algorithms~\cite{wu2024earthfarsser}, such as ConvLSTM~\cite{10.5555/2969239.2969329} and PredRNN~\cite{wang2022predrnn}, focus on regional precipitation. Recent large-scale scientific computing models, like Pangu-weather~\cite{bi2023accurate}, GraphCast~\cite{lam2023learning}, and NowcastNet~\cite{zhang2023skilful}, achieve significant results in medium- and short-term forecasts and show high potential in extreme event prediction (e.g., precipitation and Typhoon track prediction)~\cite{chen2024machine, espeholt2022deep,gong2024cascast, wu2024neural}. However, pure AI methods still face several core challenges:

\ding{182}~\textbf{\textit{Global and regional high-resolution forecasts are hard to balance.}} Regional predictions often lack boundary information, making it difficult to effectively \underline{nest} global data. \ding{183}~\textbf{\textit{Extreme events and long-term forecasts suffer from over-smoothing.}} They fail to capture high-frequency disturbances, leading to reduced forecast accuracy. \ding{184}~\textbf{\textit{Lack of dynamic system modeling capability.}} This is especially true for capturing complex interactions between nodes at multiple scales and learning high-frequency node-edge features.

To address these challenges, we propose~\method{}, a global-regional nested weather forecasting framework based on Graph Neural Networks (GNNs). Inspired by heuristic learning from numerical methods, we construct a multi-scale graph structure based on dynamical systems and multi-grid theory~\cite{he2019mgnet,hemgno}, refining it for the target region to capture local high-frequency features with greater detail. Additionally, to solve issues like over-smoothing in extreme events and long-term forecasts, which hinder the capture of high-frequency disturbances, we introduce an adaptive information propagation mechanism. This mechanism deepens the integration of node and edge features through dynamic gating units. Finally, for regional high-resolution forecasting, we adopt a nested grid strategy~\cite{phillips1973strategy} that inherits large-scale background information from the global scale, significantly alleviating the boundary information loss. \textit{Through this integrated framework, we aim to effectively capture high-frequency features and extreme events across global to regional scales, as well as from short-term to long-term forecasts.}

The method most similar to ours is Graph-EFM~\cite{oskarsson2024probabilistic}. It also uses a hierarchical graph neural network for global and regional weather modeling. However, for high-resolution regional forecasts, it treats global low-resolution data as non-trainable forcing conditions, making it unable to adaptively couple multi-scale information based on actual needs. And it doesn't treats the forecasts of the global model in the region as forcing, which unable to fully utilize the information of the global model. In contrast, our neural nested grid method applies trainable local refinement to the target region in the network structure. It also updates boundary and background information with global model future forcing dynamically through end-to-end training during global-regional coupling. This design better captures the interaction between large-scale global backgrounds and high-frequency regional details. Our experiments (Sec~\ref{Regional}) show that our method achieves greater stability and accuracy in long-term rolling inference.


The contribution of this paper can be summarized as follows:~(1)~\textit{Global-Regional Unified Forecasting Framework.} We propose a Graph Neural Network method that supports both global scale and regional high-resolution forecasting, achieving high-accuracy results for multi-scale and multi-time frame weather forecasts within the same framework. (2)~\textit{Adaptive Information Propagation Mechanism.} Through Dynamic Gating Units and graph attention modules, we deeply integrate node and edge features, more accurately capturing extreme events and other high-frequency disturbance signals within the multi-scale graph structure. As shown in Figure~\ref{fig:icml_intro}, OneForecast delivers better performance in tracking extreme events like typhoons. (3)~\textit{Nested Grid and Long-Term Forecasting.} By using a nested grid to merge global and regional information, we overcome the boundary loss issue in regional forecasting. This method effectively mitigates the loss of details caused by over-smoothing in long-term forecasts.
