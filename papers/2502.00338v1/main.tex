%%%%%%%% ICML 2025 EXAMPLE LATEX SUBMISSION FILE %%%%%%%%%%%%%%%%%

\documentclass{article}

% Recommended, but optional, packages for figures and better typesetting:
\usepackage{microtype}
\usepackage{graphicx}
\usepackage{subfigure}
\usepackage{booktabs} % for professional tables

% hyperref makes hyperlinks in the resulting PDF.
% If your build breaks (sometimes temporarily if a hyperlink spans a page)
% please comment out the following usepackage line and replace
% \usepackage{icml2025} with \usepackage[nohyperref]{icml2025} above.
\usepackage{hyperref}
\usepackage{microtype}
\usepackage{graphicx}
\usepackage{subfigure}
\usepackage{booktabs} 
\usepackage{multirow,bm}
\usepackage{tikz}
\usepackage{caption}
\usepackage{threeparttable}
\usepackage{tikz-cd,mathtools}
\usepackage{mathtools}
\usetikzlibrary{arrows, matrix} 
\usepackage{xcolor}
\definecolor{darkblue}{rgb}{0.0, 0.0, 0.55}
\definecolor{darkred}{rgb}{0.55, 0.0, 0.0}
\usepackage{hyperref}
\newcommand{\theHalgorithm}{\arabic{algorithm}}


\usepackage{amssymb}
\usepackage{mathtools}
\usepackage{amsthm}
\usepackage{multirow}
\usepackage[capitalize,noabbrev]{cleveref}

\usepackage{pifont}

%%%%%%%%%%%%%%%%%%%%%%%%%%%%%%%%
% THEOREMS
%%%%%%%%%%%%%%%%%%%%%%%%%%%%%%%%
\theoremstyle{plain}
\newtheorem{theorem}{Theorem}[section]
\newtheorem{proposition}[theorem]{Proposition}
\newtheorem{lemma}[theorem]{Lemma}
\newtheorem{corollary}[theorem]{Corollary}
\theoremstyle{definition}
\newtheorem{definition}[theorem]{Definition}
\newtheorem{assumption}[theorem]{Assumption}
\theoremstyle{remark}
\newtheorem{remark}[theorem]{Remark}




% Use the following line for the initial blind version submitted for review:
% \usepackage{icml2025}

% If accepted, instead use the following line for the camera-ready submission:
\usepackage[accepted]{icml2025}
% \usepackage{icml2025}
\makeatletter
\def\toptitlebar{\hrule height1pt \vskip .15in}
\def\bottomtitlebar{\vskip .15in \hrule height1pt \vskip .3in}
\makeatother
% For theorems and such
\usepackage{amsmath}
\usepackage{amssymb}
\usepackage{mathtools}
\usepackage{amsthm}
\usepackage{algorithm}
\usepackage{algorithmic}
\usepackage{amsmath}
% if you use cleveref..
\usepackage[capitalize,noabbrev]{cleveref}

%%%%%%%%%%%%%%%%%%%%%%%%%%%%%%%%
% THEOREMS
%%%%%%%%%%%%%%%%%%%%%%%%%%%%%%%%
% \theoremstyle{plain}
% \newtheorem{theorem}{Theorem}[section]
% \newtheorem{proposition}[theorem]{Proposition}
% \newtheorem{lemma}[theorem]{Lemma}
% \newtheorem{corollary}[theorem]{Corollary}
% \theoremstyle{definition}
% \newtheorem{definition}[theorem]{Definition}
% \newtheorem{assumption}[theorem]{Assumption}
% \theoremstyle{remark}
% \newtheorem{remark}[theorem]{Remark}

% Todonotes is useful during development; simply uncomment the next line
%    and comment out the line below the next line to turn off comments
%\usepackage[disable,textsize=tiny]{todonotes}
\usepackage[textsize=tiny]{todonotes}
\def\method{OneForecast}
% The \icmltitle you define below is probably too long as a header.
% Therefore, a short form for the running title is supplied here:
\icmltitlerunning{OneForecast: A Universal Framework for Global and Regional Weather Forecasting}

\begin{document}

\twocolumn[

\icmltitle{OneForecast: A Universal Framework for Global and Regional\\Weather Forecasting}

\icmlsetsymbol{equal}{*}

\begin{icmlauthorlist}
\vskip -0.15in
\icmlauthor{Yuan Gao}{equal,thu1}
\icmlauthor{Hao Wu}{equal,thu1,tx,ustc}
\icmlauthor{Ruiqi Shu}{equal,thu1}
\icmlauthor{Huanshuo Dong}{ustc}
\icmlauthor{Fan Xu}{ustc}
\icmlauthor{Rui Chen}{thu2}
\icmlauthor{Yibo Yan}{hkust,hkustgz}
\icmlauthor{Qingsong Wen}{comp}
\icmlauthor{Xuming Hu}{hkust,hkustgz}
\icmlauthor{Kun Wang}{ntu}
\icmlauthor{Jiahao Wu}{polyu}
\icmlauthor{Qing Li}{polyu}
\icmlauthor{Hui Xiong}{hkust,hkustgz}
\icmlauthor{Xiaomeng Huang}{thu1}

%\icmlauthor{}{sch}
%\icmlauthor{}{sch}
\end{icmlauthorlist}
\vskip -0.03in
\icmlaffiliation{thu1}{Department of Earth System Science, Ministry of Education Key Laboratory for Earth System Modeling, Institute for Global Change Studies, Tsinghua University}
\icmlaffiliation{tx}{TEG, Tencent}
\icmlaffiliation{ustc}{Department and Computer and Science, University of Science and Technology of China}
\icmlaffiliation{thu2}{Institute for Interdisciplinary Information Sciences, Tsinghua University}
\icmlaffiliation{hkust}{Department of Computer Science and Engineering, The HongKong University of Science and Technology}
\icmlaffiliation{hkustgz}{AI Thrust, The Hong Kong University of Science and Technology (Guangzhou)}
\icmlaffiliation{ntu}{School of Computer Science and Engineering, Nanyang Technological University}
\icmlaffiliation{comp}{Squirrel Ai Learning}
\icmlaffiliation{polyu}{Department of Computing, The Hong Kong Polytechnic University}

\icmlcorrespondingauthor{Xiaomeng Huang}{hxm@tsinghua.edu.cn}

% You may provide any keywords that you
% find helpful for describing your paper; these are used to populate
% the "keywords" metadata in the PDF but will not be shown in the document
\icmlkeywords{Machine Learning, ICML}

\vskip 0.1in
]

% this must go after the closing bracket ] following \twocolumn[ ...

% This command actually creates the footnote in the first column
% listing the affiliations and the copyright notice.
% The command takes one argument, which is text to display at the start of the footnote.
% The \icmlEqualContribution command is standard text for equal contribution.
% Remove it (just {}) if you do not need this facility.

%\printAffiliationsAndNotice{}  % leave blank if no need to mention equal contribution
 % otherwise use the standard text.
\printAffiliationsAndNotice{\icmlEqualContribution}
\begin{abstract}
\begin{abstract}

We introduce \ours, a novel framework for scene-level appearance transfer from a single style image to a real-world scene represented by multiple views. The method combines explicit semantic correspondences with multi-view consistency to achieve precise and coherent stylization.
Unlike conventional stylization methods that apply a reference style globally, \ours uses open-vocabulary segmentation to establish dense, instance-level correspondences between the style and real-world images. This ensures that each object is stylized with semantically matched textures.
\ours first transfers the style to a single view using a training-free semantic-attention mechanism in a diffusion model.
It then lifts the stylization to additional views via a learned warp-and-refine network guided by monocular depth and pixel-wise correspondences.
Experiments show that \ours consistently outperforms prior methods in structure preservation, perceptual style similarity, and multi-view coherence.
User studies further validate its ability to produce photo-realistic, semantically faithful results.
Our code, pretrained models, and dataset will be publicly released, to support new applications in interior design, virtual staging, and 3D-consistent stylization.

\end{abstract}

\end{abstract}

 %


\begin{figure*}[t!]
    \centering
    \includegraphics[width=0.7\textwidth]{./Comparison.pdf}
    \caption{Comparison between conventional wireless system (left) and PASS (right).}
    \label{comparison}
    \vspace{-0.5cm}
\end{figure*} 

\section{Introduction} \label{sec:intro}

\IEEEPARstart{S}INCE Marconi demonstrated the feasibility of wireless communication in the late 19th century, the technology has undergone significant evolution and remarkable transformations. To address the unpredictable and dynamic nature of wireless channels, numerous advancements have been made in the air interface design, channel coding, source compression, and communication protocols for improving data rates and enhancing reliability. Among these advancements, multiple-input multiple-output (MIMO) has been one of the most important evolutionary techniques for wireless communication over the past few decades. By exploiting antenna arrays, MIMO brings about multiple benefits, such as enhanced signal strength through beamforming, mitigation of multi-path fading, and efficient spatial-domain multiplexing of users~\cite{bjornson2023twenty}. Since the advent of the third generation (3G) system, MIMO has been a fundamental component of wireless communication standards. However, during that era, the size of antenna arrays in MIMO systems was generally limited. The breakthrough came when Marzetta demonstrated the significant benefits of deploying an infinite number of antennas in 2010~\cite{marzetta2010noncooperative}, revealing the potential of MIMO to enhance communication performance while reducing system complexity. This revelation paved the way for the concept of massive MIMO, i.e., employing large-scale antenna arrays at base stations. Over time, massive MIMO has evolved into a key research focus and has become a reality with the deployment of 5G networks. 


However, massive MIMO has faced numerous challenges, as it is expected to transition from “Massive” in 5G (typically with 32-64 antennas) to “Gigantic” in 6G~\cite{Xtext, bjornson2024enabling}, where the number of antennas is expected to scale to hundreds or even thousands. One of the key obstacles is the complexity and cost of implementing massive MIMO since each antenna typically needs to be fed by a dedicated radio-frequency (RF) chain. Exploiting low-resolution analog-to-digital converters in RF chains or hybrid analog-digital antenna arrays with a limited number of RF chains were common methods to address this challenge, especially in the millimeter-wave band~\cite{heath2016overview}. More recently, advancements in metamaterials have paved the way for new antenna technologies, exemplified by waveguide-fed metasurface antennas~\cite{smith2017analysis, shlezinger2021dynamic, di2024reconfigurable}, which facilitate the ultra-dense deployment of antenna elements at a significantly lower cost and making massive MIMO implementation more feasible.

Flexible-antenna technique is a new evolution of MIMO. Unlike massive MIMO focusing on enlarging the wireless channel dimension, the flexible-antenna technique focuses on enabling the reconfiguration of the wireless channel. One of the most well-known approaches in this domain is the reconfigurable intelligent surface (RIS) technique~\cite{huang2019reconfigurable, wu2019intelligent, mu2021simultaneously}. By deploying RIS between transceivers, the wireless channel can be intelligently reconfigured by adjusting the phase shifts of the signals reflected/refracted by the RIS. More recently, fluid antennas~\cite{new2024tutorial} and movable antennas~\cite{zhu2023movable} have emerged as promising flexible-antenna technologies. The fundamental concept behind these approaches is to implement antenna arrays where individual antenna elements can dynamically adjust their positions within a spatial region, thus creating favorable channel conditions to enhance communication performance. 

Nevertheless, as shown on the left of Fig. \ref{comparison}, both massive MIMO and flexible-antenna techniques have limited capability in fundamentally addressing free-space pathloss and line-of-sight (LoS) blockage, two major causes of signal attenuation in wireless communications. While massive MIMO can achieve high beamforming gains to strengthen signals, it cannot combat LoS blockage and to effectively mitigate free-space pathloss, particularly for cell-edge users. RISs have been considered as a promising solution to overcome LoS blockage by creating virtual LoS paths. However, the double fading effect caused by signal reflection results in much higher pathloss compared to a direct LoS channel~\cite{ozdogan2019intelligent}. Additionally, fluid and movable antennas are typically capable of adjusting their positions only within a few wavelengths, making them more effective for mitigating small-scale fading rather than addressing large-scale pathloss. It is worthy to point out that all the aforementioned MIMO systems are lack of antenna array reconfigurability, i.e., once an antenna array is built, adding or removing antennas is no longer possible.

Pinching-Antenna SyStem (PASS) is a revolutionary technique for addressing the challenges of free-space pathloss and LoS blockage encountered by conventional multi-antenna technologies. This technique was originally proposed and prototyped by NTT DOCOMO in 2022~\cite{suzuki2022pinching}. As illustrated on the right of Fig. \ref{comparison}, PASS employs a dielectric waveguide as its primary transmission medium, which is known for its exceptionally low propagation loss (e.g., 0.01 dB/m \cite{pozar2021microwave}). By pinching a small separated dielectric element, referred to as a \emph{pinching antenna}, onto the waveguide, the system enables signal emission from the waveguide into the pinching antenna, which then radiates the signal into free space. Building on this principle, waveguides can be pre-deployed to extend service coverage, allowing pinching antennas to be placed at positions close to users. This strategic placement transforms the wireless system into a \emph{near-wired} system and hence establishes strong LoS links with users, effectively minimizing free-space path loss and mitigating blockage issues. Additionally, unlike existing MIMO systems, PASS allows both the number and positions of pinching antennas to be easily adjusted by simply pinching them to or releasing them from the waveguide~\cite{suzuki2022pinching}. This feature provides a low-cost and scalable approach to implementing MIMO while also facilitating the so-called \emph{pinching beamforming}, which enhances communication performance by dynamically optimizing antenna positions \cite{liu2025pinching}.

Given the successful prototyping of PASS by NTT DOCOMO, theoretical research on this topic has been steadily growing, though it remains in its early stages. The first theoretical study on PASS for the communication system design was presented in \cite{ding2024flexible}, where the authors provided a comprehensive analysis and developed low-complexity pinching beamforming designs for fundamental single-user and two-user scenarios. The array gain achieved by multiple pinching antennas on a waveguide was analyzed in \cite{ouyang2025array}, unveiling the optimal number of antennas and their spacing for maximizing the beamforming gain. 
% The authors of \cite{tegos2024minimum} studied an uplink PASS system and proposed an iterative antenna position optimization algorithm to maximize the sum rate under perfect phase alignment conditions. In \cite{wang2024antenna}, the authors investigated a downlink PASS system and introduced a matching theory-based optimization method for activating pinching antennas at preconfigured discrete positions. Their findings also highlighted the advantages of using non-orthogonal multiple access (NOMA) in PASS. Expanding on this,
The authors of \cite{bereyhi2025downlink} explored a downlink PASS architecture utilizing multiple waveguides, each equipped with a single pinching antenna, and proposed a greedy approach for jointly optimizing the transmit and pinching beamforming. Meanwhile, \cite{guo2025deep} examined a more generalized scenario, where multiple pinching antennas were deployed on each waveguide, and introduced a graph neural network (GNN)-based deep learning method to address the corresponding joint beamforming optimization problem.

Although PASS has attracted growing attention, several key challenges remain unsolved. On the one hand, the physics modeling of PASS is still underdeveloped, which is crucial for establishing an accurate signal model. In existing studies \cite{ding2024flexible, ouyang2025array, bereyhi2025downlink, guo2025deep}, it is commonly assumed that all signal power within the waveguide is fully radiated into free space and that each pinching antenna on a waveguide emits identical radiation power—an assumption analogous to conventional MIMO systems. However, pinching antennas operate fundamentally differently from traditional electronic antennas, and such assumptions may lack a solid physical foundation and fail to accurately reflect real-world behaviors. On the other hand, most existing works design PASS under simplified assumptions \cite{ding2024flexible, ouyang2025array, bereyhi2025downlink}, such as a single user, a single waveguide, a single pinching antenna per waveguide, or perfectly aligned signal phases. Although the GNN-based deep learning model proposed in \cite{guo2025deep} is capable of handling more complex scenarios with arbitrary numbers of users, waveguides, and pinching antennas, it suffers from a key limitation: the model parameters need to be retrained once the system configuration changes, limiting its generalization ability. Motivated by these challenges, this paper aims to develop a fundamental physics-based signal model for PASS and explore joint beamforming designs for more general scenarios. The key contributions of this work are summarized as follows:
\begin{itemize}
    \item We propose a physics-based hardware model for PASS, in which a pinching antenna is modeled as an open-ended directional waveguide coupler to facilitate the adjustment of radiation characteristics and simplify signal modeling. Based on this model, we characterize the relationship between the electromagnetic (EM) fields within the waveguide and those radiated by the pinching antennas using coupled-mode theory.
    \item We derive a novel signal model for PASS based on the proposed physics framework, revealing the inherent coupling effect between the radiation power of pinching antennas deployed on the same waveguide. Leveraging this coupling relationship, we introduce two simplified power models and their respective implementation methods: equal power and proportional power models.
    \item We formulate a joint transmit and pinching beamforming optimization problem to minimize the transmit power in a general PASS system with arbitrary numbers of users, waveguides, and pinching antennas, considering both continuous and discrete activation of pinching antennas. To solve this highly nonconvex, coupled, and multimodal optimization problem, we propose two algorithms: the penalty-based alternating optimization algorithm and the zero-forcing (ZF)-based low-complexity algorithm.
    \item We provide comprehensive numerical results to validate the advantages of PASS and the effectiveness of the proposed algorithm. The results demonstrate that 1) the ZF-based algorithm delivers performance comparable to the penalty-based algorithm but has a low complexity, 2) PASS significantly reduces transmit power, achieving a reduction of over 95\% compared to conventional and massive MIMO, 3) a dense set of available antenna positions is required for discrete activation to achieve similar performance to continuous activation, and 4) the proportional power model exhibits performance comparable to the equal power model.
\end{itemize}

The rest of this paper is structured as follows. Section \ref{sec:model} introduces the proposed physics-based hardware model and signal model for PASS. Section \ref{sec:beamforing} presents the general system model for downlink PASS and introduces a penalty-based alternating optimization method and a ZF-based algorithm for solving the joint beamforming optimization problem. Numerical evaluations and performance comparisons under various system configurations are presented in Section \ref{sec:results}. Finally, Section \ref{sec:conclusion} summarizes the findings and concludes the paper.


\emph{Notations:} Scalars are denoted using regular typeface, vectors and matrices are represented in boldface, and Euclidean subspaces are indicated with calligraphic letters. The set of complex and real numbers are denoted by $\mathbb{C}$ and $\mathbb{R}$, respectively. The inverse, conjugate, transpose, conjugate transpose, and trace operators are denoted by $(\cdot)^{-1}$, $(\cdot)^*$, $(\cdot)^T$, $(\cdot)^H$, and $\mathrm{tr}(\cdot)$, respectively. The absolute value, Euclidean norm, Frobenius norm, and maximum norm are denoted by $|\cdot|$, $\|\cdot\|$, $\|\cdot\|_F$, and $\|\cdot\|_\infty$ respectively. The real part of a complex number of demoted by $\Re \{\cdot\}$. The entry in the $n$-th row and $m$-th column of a matrix $\mathbf{X}$ is denoted by $[\mathbf{X}]_{n,m}$. An identity matrix of dimension $N \times N$ is denoted by $\mathbf{I}_N$. The big-O notation is given by $O(\cdot)$. A diagonal matrix with diagonal entries $x_1,\dots,x_N$ is denoted as $\mathrm{diag}(x_1,\dots,x_N)$.    




% \begin{figure*}[t!]
% \centering
% \begin{subfigure}[t]{0.48\textwidth}
%     \centering
%     \includegraphics[height=0.5\textwidth]{./Comparison_conventional.pdf}
% \end{subfigure}
% \hspace{-1.5cm}
% \begin{subfigure}[t]{0.48\textwidth}
%     \centering
%     \includegraphics[height=0.5\textwidth]{./Comparison_PASSpdf.pdf}
% \end{subfigure}
% \caption{Comparison between conventional wireless system (left) and PASS (right).}
% \end{figure*} 


% \section{Related Work}
\subsection{Deep Learning based Weather Forecasting}
\textbf{Global Weather Forecasting.} Global weather forecasting has seen significant progress with deep learning models. FourCastNet, based on Fourier neural operators, provides global forecasts comparable to traditional numerical methods like IFS, but at much higher speeds~\cite{pathak2022fourcastnet}. Pangu, utilizing the Swin Transformer, exceeds NWP methods, incorporating earth-specific location embeddings for better performance~\cite{bi2023accurate}. The Spherical Fourier Neural Operator (SFNO) extends Fourier methods using spherical harmonics, offering more stable long-term predictions~\cite{bonev2023spherical}. FuXi focuses on long-term forecasting, achieving a 15-day forecasts comparable to ECMWF~\cite{chen2023fuxi}. GraphCast leverages message-passing networks to improve efficiency and forecasting accuracy~\cite{lam2023learning}, and GenCast builds on this to enhance ensemble forecasting~\cite{price2023gencast}. Further, diffusion models like those in~\cite{li2024generative} generate probabilistic ensembles by sampling, while NeuralGCM~\cite{kochkov2024neural} focuses on atmospheric circulation with a dynamic core, offering climate simulation capabilities but at higher training and inference costs. 

\textbf{Regional Weather Forecasting.} The goal of regional weather forecasting is to enhance local prediction accuracy with high-resolution models. CorrDiff~\cite{mardani2023generative} combines U-Net and diffusion models to improve local forecasts. MetaWeather~\cite{kim2024metaweather} adapts global forecasts to regional contexts using meta-learning. GNNs are also widely applied in regional forecasting, with Graphcast~\cite{lam2023learning} enhancing accuracy by modeling complex spatial dependencies. MetNet-3~\cite{espeholt2022deep} offers high-accuracy forecasts for weather variables, such as precipitation, temperature, and wind speed, at 2-minute intervals and 1–4 km resolution, outperforming traditional models like HRRR. NowcastNet~\cite{zhang2023skilful} and DGMR~\cite{ravuri2021skilful} excel in short-term extreme precipitation forecasts using deep generative models and radar data. In spatiotemporal prediction, NMO~\cite{wu2024neural} models the evolution of physical dynamics, providing new insights for local weather forecasting. Similarly, SimVP~\cite{gao2022simvp} and PastNet~\cite{wu2024pastnet} achieve good results in forecasting local precipitation evolution using spatiotemporal convolution methods.
    
% Despite these advances, none of these methods effectively address the challenge of balancing global and regional high-resolution forecasts or capturing the fine-grained, dynamic interactions important for extreme event prediction.
    
\subsection{Numerical analysis methods}
Multigrid methods~\cite{mccormick1987multigrid,wesseling1995introduction,hackbusch2013multi,bramble2019multigrid,hiptmair1998multigrid,brandt1983multigrid,borzi2009multigrid} and nested grid strategies~\cite{miyakoda1977one,zhang2012nested,sullivan1996grid} are widely used to solve PDEs and handle multi-scale problems~\cite{debreu2008two,xue2000advanced}. Multigrid methods use grids of different resolutions to transfer information and accelerate iterations. They efficiently solve large-scale problems and improve computational accuracy. By eliminating low-frequency errors on coarse grids and high-frequency errors on fine grids, multigrid methods effectively handle error convergence at different scales~\cite{he2019mgnet,he2023mgno,shao2022fast}. Nested grid strategies embed higher-resolution fine grids into regions of interest based on a global coarse grid to capture local complex physical phenomena in detail. In weather forecasting, this method provides large-scale background fields on a global scale while refining the grid for target regions to accurately simulate the evolution of local weather systems and the occurrence of extreme events~\cite{bacon2000dynamically}. 

% Our proposed neural nested grid method helps address challenges like boundary information loss in regional forecasting and multi-scale feature capture.

\section{Additional Results}
%
We present more additional results in Figure \ref{fig_0.25-day}, \ref{fig_0.5-day}, \ref{fig_1.0-day} \ref{fig_1.5-day}, \ref{fig_2.0-day}, \ref{fig_2.5-day}, \ref{fig_3.0-day}, \ref{fig_3.5-day}, \ref{fig_4.0-day}, \ref{fig_4.5-day}, \ref{fig_5.0-day}, \ref{fig_5.5-day}, \ref{fig_6.0-day}, \ref{fig_6.5-day}, \ref{fig_7.0-day}, \ref{fig_7.5-day},
\ref{fig_8.0-day}, \ref{fig_8.5-day}, \ref{fig_9.0-day}, \ref{fig_9.5-day},
\ref{fig_10.0-day}, including 18 variables that are importmant to weather forecasting, each with results ranging from 6 hours to 10 days. These additional results further demonstrate the effectiveness of OneForecast. Same as the Figure \ref{fig:visual_results}
, the initial conditions is 00:00 UTC, 1 January 2020.


\begin{figure*}[h]
\centering
\includegraphics[width=1\linewidth]{figures/fig_0.25-day.jpg}
\vspace{-20pt}
\caption{6-hour forecast results of different models.}
\label{fig_0.25-day}
\end{figure*}

\begin{figure*}[h]
\centering
\includegraphics[width=1\linewidth]{figures/fig_0.5-day.jpg}
\vspace{-20pt}
\caption{0.5-day forecast results of different models.}
\label{fig_0.5-day}
\end{figure*}

\begin{figure*}[h]
\centering
\includegraphics[width=1\linewidth]{figures/fig_1.0-day.jpg}
\vspace{-20pt}
\caption{1-day forecast results of different models.}
\label{fig_1.0-day}
\end{figure*}

\begin{figure*}[h]
\centering
\includegraphics[width=1\linewidth]{figures/fig_1.5-day.jpg}
\vspace{-20pt}
\caption{1.5-day forecast results of different models.}
\label{fig_1.5-day}
\end{figure*}

\begin{figure*}[h]
\centering
\includegraphics[width=1\linewidth]{figures/fig_2.0-day.jpg}
\vspace{-20pt}
\caption{2-day forecast results of different models.}
\label{fig_2.0-day}
\end{figure*}


\begin{figure*}[h]
\centering
\includegraphics[width=1\linewidth]{figures/fig_2.5-day.jpg}
\vspace{-20pt}
\caption{2.5-day forecast results of different models.}
\label{fig_2.5-day}
\end{figure*}

\begin{figure*}[h]
\centering
\includegraphics[width=1\linewidth]{figures/fig_3.0-day.jpg}
\vspace{-20pt}
\caption{3-day forecast results of different models.}
\label{fig_3.0-day}
\end{figure*}

\begin{figure*}[h]
\centering
\includegraphics[width=1\linewidth]{figures/fig_3.5-day.jpg}
\vspace{-20pt}
\caption{3.5-day forecast results of different models.}
\label{fig_3.5-day}
\end{figure*}

\begin{figure*}[h]
\centering
\includegraphics[width=1\linewidth]{figures/fig_4.0-day.jpg}
\vspace{-20pt}
\caption{4-day forecast results of different models.}
\label{fig_4.0-day}
\end{figure*}

\begin{figure*}[h]
\centering
\includegraphics[width=1\linewidth]{figures/fig_4.5-day.jpg}
\vspace{-20pt}
\caption{4.5-day forecast results of different models.}
\label{fig_4.5-day}
\end{figure*}


\begin{figure*}[h]
\centering
\includegraphics[width=1\linewidth]{figures/fig_5.0-day.jpg}
\vspace{-20pt}
\caption{5.0-day forecast results of different models.}
\label{fig_5.0-day}
\end{figure*}

\begin{figure*}[h]
\centering
\includegraphics[width=1\linewidth]{figures/fig_5.5-day.jpg}
\vspace{-20pt}
\caption{5.5-day forecast results of different models.}
\label{fig_5.5-day}
\end{figure*}

\begin{figure*}[h]
\centering
\includegraphics[width=1\linewidth]{figures/fig_6.0-day.jpg}
\vspace{-20pt}
\caption{6.0-day forecast results of different models.}
\label{fig_6.0-day}
\end{figure*}

\begin{figure*}[h]
\centering
\includegraphics[width=1\linewidth]{figures/fig_6.5-day.jpg}
\vspace{-20pt}
\caption{6.5-day forecast results of different models.}
\label{fig_6.5-day}
\end{figure*}

\begin{figure*}[h]
\centering
\includegraphics[width=1\linewidth]{figures/fig_7.0-day.jpg}
\vspace{-20pt}
\caption{7.0-day forecast results of different models.}
\label{fig_7.0-day}
\end{figure*}

\begin{figure*}[h]
\centering
\includegraphics[width=1\linewidth]{figures/fig_7.5-day.jpg}
\vspace{-20pt}
\caption{7.5-day forecast results of different models.}
\label{fig_7.5-day}
\end{figure*}

\begin{figure*}[h]
\centering
\includegraphics[width=1\linewidth]{figures/fig_8.0-day.jpg}
\vspace{-20pt}
\caption{8.0-day forecast results of different models.}
\label{fig_8.0-day}
\end{figure*}

\begin{figure*}[h]
\centering
\includegraphics[width=1\linewidth]{figures/fig_8.5-day.jpg}
\vspace{-20pt}
\caption{8.5-day forecast results of different models.}
\label{fig_8.5-day}
\end{figure*}

\begin{figure*}[h]
\centering
\includegraphics[width=1\linewidth]{figures/fig_9.0-day.jpg}
\vspace{-20pt}
\caption{9.0-day forecast results of different models.}
\label{fig_9.0-day}
\end{figure*}

\begin{figure*}[h]
\centering
\includegraphics[width=1\linewidth]{figures/fig_9.5-day.jpg}
\vspace{-20pt}
\caption{9.5-day forecast results of different models.}
\label{fig_9.5-day}
\end{figure*}

\begin{figure*}[h]
\centering
\includegraphics[width=1\linewidth]{figures/fig_10.0-day.jpg}
\vspace{-20pt}
\caption{10.0-day forecast results of different models.}
\label{fig_10.0-day}
\end{figure*}
\section{Method}
\begin{figure*}
  \centering
  \includegraphics[width=1\linewidth]{figures/icml_main.png}
     \vspace{-16pt}
  \caption{Overview of Our~\method{}. \textbf{(a)} The overall architecture includes input variables, an encoder, a message passing module, a decoder, and visualization of forecast variables; \textbf{(b)} The global forecasts module uses rollout technology to generate future forecasts; \textbf{(c)} The neural nested grid method specializes in regional high-resolution weather forecasts tasks; and \textbf{(d)} The ensemble forecasting module generates long-term forecast results.}
  \label{fig:ICML_yuan}
\vspace{-10pt}
\end{figure*}

\textbf{Problem Definition}\label{sec:problem}
In this study, we model weather forecasting as an autoregressive problem~\cite{lam2023learning}. At each time step $t$, we use the meteorological state comprising surface variables $\mathbf{X}_t$ and pressure level variables $\mathbf{P}_t$ to forecast the state at the next time step. We concatenate the surface and pressure level variables along the channel dimension to form the combined input: $\mathbf{Z}_t = [\mathbf{X}_t, \mathbf{P}_t] \in \mathbb{R}^{N \times d}$, where $N = H \times W$ represents the number of grid locations (nodes), and $d = d_x + d_p$ is the total number of variables. Here, $d_x$ and $d_p$ are the numbers of surface level and pressure level variables, respectively. In our setup, the initial input contains 69 variables: 4 surface level variables and 65 pressure level variables. Our model aims to forecast the combined variables at the next time step $\hat{\mathbf{Z}}_{t+1}$ using the current input $\mathbf{Z}_t$, capturing the spatiotemporal evolution of the atmosphere: $\hat{\mathbf{Z}}_{t+1} = \text{Model}(\mathbf{Z}_t; \Theta),$ where $\Theta$ denotes the model parameters. The training objective is to minimize the relative mean squared error (MSE) between the forecasts and the true values across all time steps: $    \min_{\Theta} \frac{1}{T} \sum_{t=0}^{T-1} \frac{ \left\| \hat{\mathbf{Z}}_{t+1} - \mathbf{Z}_{t+1} \right\|_2^2}{\left\| \mathbf{Z}_{t+1} \right\|_2^2}$. During inference, we adopt a rollout strategy to forecast longer sequences. Starting from the initial state $\mathbf{Z}_0$, the model recursively uses its previous forecasts as the next input: $ \hat{\mathbf{Z}}_{t+1} = \text{Model}(\hat{\mathbf{Z}}_t; \Theta), \quad t = 0, 1, 2, \dots, T-1.$ This strategy allows the model to generate extended weather forecasts using its own forecasts.

\subsection{Earth-specific Region Refined Graph Encoder}
In the encoder of~\method{}, inspired by~\cite{fortunato2022multiscale} ~\cite{lam2023learning}, we introduce an Earth-specific Region Refined Multi-scale Graph to improve the interaction of node features in complex dynamical systems. Inspired by the idea of multigrid methods~\cite{he2019mgnet}, we construct a multi-level Graph Neural Network architecture that includes grids of multiple granularities. Each grid has the same number of nodes but different grid densities, thereby capturing spatial features at different scales. Specifically, we define the multi-scale graph structure as:
\begin{equation}
\mathcal{G}=\left(\mathcal{V}^G, \mathcal{V}, \mathcal{E}^{(1)}, \mathcal{E}^{(2)}, \ldots, \mathcal{E}^{(L)}, \mathcal{E}^{(R)}, \mathcal{E}^{\mathrm{G} 2 \mathrm{M}}, \mathcal{E}^{\mathrm{M} 2 \mathrm{G}}\right),
\end{equation}
where ${\mathcal{V}}^{G}$ represents the set of lat-lon grid nodes, with a total of $N = H \times W$ nodes; $\mathcal{V}$ represents the mesh nodes. $\mathcal{E}^{(l)}$ denotes the edge set at the $l$-th scale, corresponding to grids of different granularities, where $l = 1, 2, \dots, L$, and $\mathcal{E}^{(R)}$ represents regional refined edges. $\mathcal{E}^{\mathrm{G} 2 \mathrm{M}}$ and $\mathcal{E}^{\mathrm{M} 2 \mathrm{G}}$ are the unidirectional edges that connect lat-lon grid nodes and mesh nodes. All scales share the same set of nodes $\mathcal{V}$. More details can be found in Appendix \ref{Appendix:model_details}.

In the encoder, we first map the input meteorological state $\mathbf{Z}_t \in \mathbb{R}^{N \times d}$ to the initial node feature representation:
\begin{equation}
\mathbf{h}_i^{(0)} = \phi (\mathbf{Z}_{t, i}), \quad i = 1, 2, \ldots, N,
\end{equation}
where $\phi(\cdot)$ is the feature mapping function, and $\mathbf{Z}_{t,i}$ is the input feature at node $i$. Next, we iteratively update the node features on the multi-scale graph structure. At iteration $k$, the feature update formula for node $i$ is:
\begin{equation}
\mathbf{h}_i^{(k)} = \sigma \left( \sum_{l=1}^L \sum_{j \in \mathcal{N}_i^{(l)}} \mathbf{W}^{(l)} \mathbf{h}_j^{(k-1)} + \mathbf{b}^{(l)} \right),
\end{equation}
where $\mathcal{N}_i^{(l)}$ is the set of nodes adjacent to node $i$ at the $l$-th scale, $\mathbf{W}^{(l)}$ is the weight matrix at the $l$-th scale, $\mathbf{b}$ is the bias term, and $\sigma(\cdot)$ is the activation function. To enhance the forecasting accuracy in specific regions, we introduce a region-refined grid on the finest global grid. For nodes within the target region, we add denser edge connections to capture local high-frequency features. In this way, the update of node features not only considers global multi-scale information but also incorporates region-specific fine-grained information.

\subsection{Multi-stream Messaging}
To address the issue of information transmission between nodes in complex dynamic systems, we propose a module called \textit{Multi-stream Messaging} (MSM). This module consists of an adaptive messaging mechanism, including a dynamic multi-head gated edge update module and a multi-head node attention mechanism. And OneForecast includes 16 MSMs for messaging.

\textbf{Dynamic Multi-head Gated Edge Update Module.} Unlike traditional message passing methods based on MLPs, we introduce dynamic gating and multi-head mechanisms to control the information flow more precisely. For each edge, we concatenate its own features with those of the source node and the target node:
\begin{equation}
\mathbf{c}_i = \operatorname{Concat}\left( \mathbf{e}_i, \mathbf{h}_{s(i)}, \mathbf{h}_{d(i)} \right) \in \mathbb{R}^{D_e + 2 D_h},
\end{equation}
where $\mathbf{e}_i$ is the feature of edge $i$, $\mathbf{h}_{s(i)}$ and $\mathbf{h}_{d(i)}$ are the features of the source and target nodes of edge $i$, $D_e$ is the edge feature dimension, and $D_h$ is the node feature dimension. Next, we generate gating vectors through a two-layer MLP to regulate the information flow. Specifically, the gating vector is divided into three parts: edge feature update gate $g_e$, source node feature gate $g_s$, and destination node feature gate $g_d$. First, we perform the first layer linear transformation and activation:
\begin{equation}
\mathbf{z}_i = \operatorname{SiLU}\left( \mathbf{W}_1 \mathbf{c}_i + \mathbf{b}_1 \right),
\end{equation}
where $\mathbf{W}_1 \in \mathbb{R}^{h \times (D_e + 2 D_h)}$ is the weight matrix of the first layer, and $\mathbf{b}_1 \in \mathbb{R}^h$ is the bias term. Then, we perform the second layer linear transformation and Sigmoid activation:
\begin{equation}
\mathbf{g}_i = \sigma\left( \mathbf{W}_2 \mathbf{z}_i + \mathbf{b}_2 \right) \in \mathbb{R}^{3 H D},
\end{equation}
where, $\mathbf{W}_2 \in \mathbb{R}^{3 H D\times h}$ is the weight matrix of the linear transformation of the second layer, and $D$ is the feature dimension for each gate. For each head $h$, the gating values $\mathbf{g}_i^{(h,e)}$, $\mathbf{g}_i^{(h,s)}$, and $\mathbf{g}_i^{(h,d)}$ are vectors of dimension $D$, corresponding to the edge feature gate, source node feature gate, and destination node feature gate, respectively.

Subsequently, we use Edge Sum MLP (ESMLP) to perform linear transformation and nonlinear activation on the edge features to generate the updated edge features:
\begin{equation}
\mathbf{e}_i' = \operatorname{ESMLP}_e\left( \mathbf{e}_i, \mathbf{h}_{s(i)}, \mathbf{h}_{d(i)} \right) \in \mathbb{R}^{D_e'},
\end{equation}
where $D_e'$ is the dimension of the updated edge features. Finally, we combine the gating vectors and the updated edge features to generate the final updated edge features through weighted averaging and residual connections:
\begin{equation}
\begin{aligned}
\mathbf{e}_i^{\text{new}} = &\frac{1}{3} \sum_{h=1}^H \bigg(
\mathbf{g}_i^{(h,e)} \odot \mathbf{e}_i^{\prime}
+ \mathbf{g}_i^{(h,s)} \odot \mathbf{h}_{s(i)} \\
&\quad + \mathbf{g}_i^{(h,d)} \odot \mathbf{h}_{d(i)}
\bigg) + \mathbf{e}_i,
\end{aligned}
\end{equation}
where $\odot$ denotes element-wise multiplication.

\textbf{Multi-head Node Attention Mechanism.} Compared to traditional message passing mechanisms, multi-head attention mechanisms can more precisely capture complex dependencies between nodes and dynamically adjust the way information is aggregated through attention weights. For each edge $e_i = (j \rightarrow k)$, we use an MLP to calculate the attention score:
\begin{equation}
\mathbf{a}_i = \operatorname{MLP}_a\left( \mathbf{e}_i^{\text{new}} \right) \in \mathbb{R}^H,
\end{equation}
then, we normalize the attention scores:
\begin{equation}
\alpha_i^{(h)} = \frac{\exp\left( \mathbf{a}_i^{(h)} \right)}{ \sum_{e_j \in \mathcal{E}(k)} \exp\left( \mathbf{a}_j^{(h)} \right) }, \quad \forall h = 1, 2, \dots, H,
\end{equation}
where $\mathcal{E}(k)$ denotes the set of all incoming edges to node $k$, and $\alpha_i^{(h)}$ is the attention weight of the $h$-th attention head for edge $e_i$. Next, we perform weighted aggregation of the edge features. For each node $k$, based on the attention weights, we compute the weighted sum of the features of all edges incoming to node $k$, generating the aggregated feature for each head:
\begin{equation}
\mathbf{m}_k^{(h)} = \sum_{e_i \in \mathcal{E}(k)} \alpha_i^{(h)} \cdot \mathbf{e}_i^{\text{new}} \in \mathbb{R}^{D_e'},
\end{equation}
then, we flatten and concatenate the aggregated features from all heads:
\begin{equation}
\mathbf{M}_k = \operatorname{Flatten}\left[ \mathbf{m}_k^{(1)}, \mathbf{m}_k^{(2)}, \dots, \mathbf{m}_k^{(H)} \right] \in \mathbb{R}^{D_e' \cdot H}.
\end{equation}
Finally, we concatenate the aggregated edge features with the original node features and, through an MLP, perform a nonlinear transformation to generate the updated node features:
\begin{equation}
\mathbf{h}_k^{\text{new}} = \operatorname{MLP}_n\left( \operatorname{Concat}\left( \mathbf{M}_k, \mathbf{h}_k \right) \right) + \mathbf{h}_k.
\end{equation}
In summary, in each iteration, we use the multi-stream messaging module to update the node features. Specifically, the node feature update formula is:
\begin{equation}
\mathbf{h}_i^{(k)} = \sigma \left( \sum_{l=1}^{L} \operatorname{MSM}\left( \mathbf{h}_i^{(k-1)}, \mathcal{E}^{(l)} \right) + \mathbf{b} \right),
\end{equation}
where $\operatorname{MSM}$ represents the aforementioned multi-stream messaging operation, $\mathcal{E}^{(l)}$ is the set of edges at the $l$-th scale, and $\sigma(\cdot)$ is the activation function. In the region-refined graph structure, for nodes within the target region, we additionally consider the set of edges within the region $\mathcal{E}^{\text{region}}$ to capture finer local information.

\textbf{Theoretical Analysis.} From a theoretical perspective, we explain why our method helps capture high-frequency information. This enhances long-term prediction ability and improves the ability to detect extreme events.
\begin{theorem}\label{thm:bias}
\textbf{High-pass Filtering Property of Multi-stream Messaging.} Considering the improved multi-stream message passing mechanism, suppose the graph signal $\bm{f} \in \mathbb{R}^N$ has a spectrum $\hat{\bm{f}} = \bm{U}^\top \bm{f}$ under the graph Fourier basis $\bm{U} = [\bm{u}_1, ..., \bm{u}_N]$, where $\bm{L} = \bm{U} \bm{\Lambda} \bm{U}^\top$ is the normalized graph Laplacian matrix and $\bm{\Lambda} = \operatorname{diag}(\lambda_1, ..., \lambda_N)$ is its eigenvalue diagonal matrix ($0 \leq \lambda_1 \leq ... \leq \lambda_N \leq 2$). Define the frequency response function of the message passing operator as $\rho: \lambda \mapsto \mathbb{R}$. If the dynamic gating weights satisfy:

\begin{equation}
    g^{(h,e)}_i, g^{(h,s)}_i, g^{(h,d)}_i \propto |\lambda_i - 1| + \epsilon \quad (\epsilon > 0)
\end{equation}

then there exist constants $\alpha > 0$ and $\kappa > 0$ such that the frequency response of the operator satisfies:

\begin{equation}
    \rho(\lambda_i) \geq \alpha |\lambda_i - 1| \quad \text{and} \quad \rho(\lambda_i) \geq \kappa \lambda_i
\end{equation}

that is, the operator is a strictly high-pass filter.
\end{theorem}
The proof of Theorem~\ref{thm:bias} can be found in Appendix~\ref{appendix_theorems}.
\subsection{Decoding and Optimization}
The decoder's goal is to decode the latent information back to meteorological variables on the latitude-longitude grid. We obtain the updated feature representation $\mathbf{h}_i$ for each node. For each node $i$, the decoder applies the mapping function:
\begin{equation}
\hat{\mathbf{Z}}_{t+1, i} = \psi(\mathbf{h}_i),
\end{equation}
where $\psi(\cdot)$ is an MLP that converts the latent node features into the predicted variables $\hat{\mathbf{Z}}_{t+1, i}$ for the next time step.

We use relative $L_2$ loss function for model training. The loss function is defined as:
\begin{equation}
\mathcal{L} = \frac{1}{K H W} \sum_{k=1}^K  \sum_{i=1}^H \sum_{j=1}^W \frac{\left( \hat{x}_{i, j, k}^{t + l \delta t} - x_{i, j, k}^{t + l \delta t} \right)^2}{\left( x_{i, j, k}^{t + l \delta t} \right)^2},
\end{equation}
where $\hat{x}_{i, j, k}^{t + l \delta t}$ and $x_{i, j, k}^{t + l \delta t}$ are the predicted and true values for variable (channel) $k$ at spatial location $(i, j)$ and time $t + l \delta t$; $K$ is the number of variables (channels); $H$ and $W$ are the height and width of the spatial dimensions, respectively; $\delta t$ is the time interval of single-step prediction (we use $\delta t = 6$ hours).

\subsection{Downstream Tasks}

We consider three principal downstream tasks:

\textbf{Global Weather Forecasting.} As detailed in Section~\ref{sec:problem} and illustrated in Figure~\ref{fig:ICML_yuan}(b), we employ a rollout approach during inference, using the trained model for multi-step extrapolation. Specifically, starting from the initial state~$\mathbf{Z}_0$, the model recursively uses its previous predictions as inputs for subsequent time steps, generating a sequence of future global weather forecasts.

\textbf{Regional High-Resolution Forecasting.} To enhance the accuracy of high-resolution forecasts in specific regions, we propose a neural nested grid method, illustrated in Figure~\ref{fig:ICML_yuan}(c). This method combines global low-resolution future forecasts with regional high-resolution data to produce detailed forecasts for the target region. We first input the global low-resolution data at time~$t$ into the pre-trained global model to obtain the global forecasts $\hat{\mathbf{Z}}_{t+1}^{\text{global}}$ at time~$t+1$. We extract $\hat{\mathbf{Z}}_{t+1}^{\text{global1}}$ from $\hat{\mathbf{Z}}_{t+1}^{\text{global}}$, which shares the same spatial range as the region, and $\hat{\mathbf{Z}}_{t+1}^{\text{global2}}$, which includes the boundary of the region (the boundary are defined as two grid points around the region). Both $\hat{\mathbf{Z}}_{t+1}^{\text{global1}}$ and $\hat{\mathbf{Z}}_{t+1}^{\text{global2}}$ are then interpolated to match the resolution of the high-resolution regional data, which are concatenate as $\hat{\mathbf{Z}}_{t+1}^{\text{global}}$ to acted as global force. We then combine the regional high-resolution data at time~$t$ with the $\hat{\mathbf{Z}}_{t+1}^{\text{global}}$ to form the input to the regional model. The global forecasts provide the necessary boundary conditions for the regional forecasts. The regional model then produces the high-resolution forecasts for the regional state at time~$t+1$:
\begin{equation}\small
       \hat{\mathbf{Z}}_{t+1}^{\text{region}} = \text{Model}_{\text{region}}\left( \operatorname{Concat}\left( \hat{\mathbf{Z}}_{t+1}^{\text{global}},\, \mathbf{Z}_t^{\text{region}} \right);\, \Theta_{\text{region}} \right).
\end{equation}

\textbf{Long-Term and Esemble Weather Forecasting.} The initial condition of the atmospheric state is uncertain, so reasonable quantification of this uncertainty is conducive to improve to forecast performance. In this work, we set N=50. To account for the uncertainty in the atmospheric initial state for long-term ensemble forecasting, we generate~$N$ perturbed initial conditions~$\mathbf{Z}_0^{(n)}$ by adding Perlin noise~$\varepsilon^{(n)}$ to~$\mathbf{Z}_0$~\cite{chen2023fuxi}. Each perturbed initial condition is input into the model, and through recursive rollout over~$T$ time steps, we obtain individual forecasts~$\hat{\mathbf{Z}}_{t+1}^{(n)}$. Finally, at each time step~$t+1$, we compute the ensemble mean prediction~$ \hat{\mathbf{Z}}_{t+1}^{\text{ensemble}} = \frac{1}{N} \sum_{n=1}^N \hat{\mathbf{Z}}_{t+1}^{(n)}$ by averaging the forecasts from all~$N$ ensemble members. 

% \subsection{Theoretical Analysis}
% To illustrate the ability of graph neural networks with different grid resolutions to express data information, we denote the continuous form of the input data as \( f(x) \). For simplicity, assume the data is one-dimensional, with \( x \in [-\pi, \pi] \). For the input data, we use the trigonometric functions \( \{1, \cos t, \sin t, \cos 2t, \sin 2t, \cdots \} \) as an orthogonal basis, and expand the input data in this basis to obtain signals of different frequencies:
%  \[
%  x(t) = \frac{a_0}{2} + \sum_{n=1}^{\infty} \left( a_n \cos nt + b_n \sin nt \right)
%  \]

%  Assume the grid on the data is \( \{x_1, x_2, \cdots, x_N\} \in [-\pi, \pi] \). For any sampling point \( x_i \), we have:
%  \[
%  \cos(k_1 x_i) \approx \cos(k_2 x_i)
%  \]
%  where
%  \[
%  k_1 \approx k_2 - \frac{2\pi n}{x_i}, \quad n \in \mathbb{Z}
%  \]
%  This indicates that at the point \( x_i \), signals with frequencies \( k_1 \) and \( k_2 \) will experience aliasing.

%  Furthermore, for
%  \[
%  k_1' \approx k_2' - \frac{2\pi n}{\prod_{i=1}^{N} x_i}, \quad n \in \mathbb{Z}
%  \]
%  the signals with frequencies \( k_1' \) and \( k_2' \) will experience aliasing at all sampling points, meaning that the discrete data will cause the model to misinterpret the high-frequency signal of \( k_2' \) as the low-frequency signal of \( k_1' \).

%  To precisely describe this phenomenon, we introduce the famous \textbf{Nyquist-Shannon sampling theorem}:
%  \begin{theorem}
%  When the sampling rate of a continuous-time signal is less than twice the highest frequency component in the signal's Fourier series representation, aliasing will occur.
%  \end{theorem}
%  This theorem indicates that to capture higher-frequency signals through discrete sampling points, the sampling frequency must be increased. In the process of converting data into grids, this means that to allow a neural network to learn higher-frequency information, a finer grid is required. That is, graph neural networks learn high-frequency information more accurately on finer grids than on coarser grids, reducing distortion.
\section{Experiment}
In this section, we conduct extensive experiments to evaluate the performance of various LLMs on our Hellaswag-Pro benchmark. Our study is guided by three key research questions:
\textbf{RQ1}: How do different LLMs perform across all variants?
\textbf{RQ2}: What is the relative difficulty of different variants?
\textbf{RQ3}: How robust are LLMs to diverse prompts during evaluation?

\subsection{Experiment Setup} 
\subsubsection{Model Selection and Implementation Details}
We select 41 representative commercial and open-source models, including English LLMs, such as GPT-4o, Claude-3.5-Sonnet, Gemini-1.5-Pro,Mistral series, Llama3 series and Chinese LLMs, like Qwen-Max,  Qwen2.5 series, InternLM-2.5 series, Yi-1.5 series, Baichuan-2 series and DeepSeek series.

We integrate both Chinese HellaSwag and HellaSwagPro into the lm-evaluation-harness platform. For the open-source models, we use the default settings of lm-evaluation-harness: do\_sample is set to false and the temperature is set to the default value of the hugging-face library. For the closed-source models, we set the temperature to 0.7. In addition, we set the maximum output length to 1024.

\subsubsection{Prompt Strategy}
Taking into account the influence of language and shot, we design 9 prompting strategies, including Direct, CN-CoT, EN-CoT, CN-XLT and EN-XLT. The last four setups include both zero-shot and few-shot variants.\footnote{
For open-source models, Direct adopts an approach similar to the official implementation of HellaSwag, computing the log-likelihood for each option and selecting the one with the highest log-likelihood. And we report normalized accuracy that accounts for the impact of option length. Other prompting strategies use a generation setup and report accuracy based on exact match.}
\textbf {(1)Direct}: LLMs makes the selection directly without any CoT process.
\textbf{(2)CN-CoT}: LLMs performs CoT in Chinese, regardless of dataset language.
\textbf{(3)EN-CoT}: Similar to CN-CoT, but CoT is conducted in English. 
\textbf{(4)CN-XLT}: LLMs are instructed to first translate English questions and options to Chinese, and then reason in Chinese.
\textbf{(5)EN-XLT}: Similar to CN-XLT, but translates from Chinese dataset to English and reasons in English. 

%\textbf {CN-CoT}: LLMs perform Chinese reasoning and then output the answer and 3 shots are provided.
%\textbf {CN-CoT}: Similar as CNCoTFewShot without any shots.
%\textbf {EN-CoT}: The reasoning process in English is executed and then the answer is output and 3 shots are provided.
%\textbf {CN-XLT}: Inspired by this, we instruct LLMs to translate questions in Chinese and then output the answer after performing reasoning in Chinese too. And 3 shots are provided.
%\textbf {EN-XLT}: Inspired by this, we instruct LLMs to translate questions in Englsih and then output the answer after performing reasoning in Englsih too. Three shots are provided.

\subsubsection{Evaluation metric}

To comprehensively evaluate the robustness of each LLM, we consider four metrics: 
% Original Accuracy (\textbf{OA}), Average Robust Accuracy (\textbf{ARA}), Robust Loss Accuracy (\textbf{RLA}), and  Consistent Robust Accuracy (\textbf{CRA}).
\noindent %
\textbf{- Original Accuracy (OA)} measures accuracy on original problems.
\begin{equation}\label{eq1}
OA=\frac{\sum_{(x, y) \in D} \mathds{1}[L M(x), y]}{|D|}.
\end{equation}
\noindent %
\textbf{- Average Robust Accuracy  (ARA)} represents average accuracy across all variants, gauging overall performance on the robustness tasks.
\begin{equation}\label{eq2}
ARA=\frac{\sum_{\left(x^{\prime}, y^{\prime}\right) \in D_{R}} \mathds{1}\left(L M\left(x^{\prime}, y^{\prime}\right)\right.}{\left|D_{R}\right|}.
\end{equation}

\noindent %
\textbf{- Robust Loss Accuracy (RLA)} is the difference between ARA and OA, indicating performance degradation on robustness data versus original data.
%\begin{tiny}
%\begin{equation}\label{eq3}
%RLA=\frac{\sum_{\left(x^{\prime}, y^{\prime}\right) \in D_{R}} %\mathds{1}\left(L M\left(x^{\prime}, y^{\prime}\right)\right.}{\left|D_{R}\right|}-\frac{\sum_{(x, y) \in D}\mathds{1}[L M(x), y]}{|D|}
%\end{equation}
%\end{tiny}
\begin{equation}\label{eq3}
RLA= OA - ARA.
\end{equation}
\noindent %
\textbf{- Consistent Robust Accuracy (CRA)} shows accuracy when the model correctly answers both original and variant data, reflecting the model do understand the problem.
% consistency in problem-solving.
\begin{equation}\label{eq4}
CRA=\frac{\sum_{x, y, x^{\prime}, y^{\prime}}\mathds{1}[L M(x), y] \cdot \mathds{1}[L M(x^{\prime}), y^{\prime}]}{\left|D_{R}\right|}.
\end{equation}
For all equation above, $D$ denotes the original dataset, where $x$ represents the input question and options, and $y$ represents the correct label, while $D_{R}$ is the robust dataset with $x^{\prime}$ and $y^{\prime}$ representing similar to $x$ and $y$.


\begin{table*}[ht]
\centering
\setlength{\tabcolsep}{5pt}
% \footnotesize
\scalebox{0.6}{
% Please add the following required packages to your document preamble:
% \usepackage{multirow}
% \usepackage[table,xcdraw]{xcolor}
% Beamer presentation requires \usepackage{colortbl} instead of \usepackage[table,xcdraw]{xcolor}
% Please add the following required packages to your document preamble:
% \usepackage{multirow}
% \usepackage[table,xcdraw]{xcolor}
% Beamer presentation requires \usepackage{colortbl} instead of \usepackage[table,xcdraw]{xcolor}
\begin{tabular}{ccccccccccccc}
\hline
\multicolumn{1}{c|}{{ }}& \multicolumn{4}{c|}{Chinese}& \multicolumn{4}{c|}{English}& \multicolumn{4}{c}{AVG}\\ \cline{2-13} 
\multicolumn{1}{c|}{\multirow{-2}{*}{{ Model}}} & { OA(\%)$\uparrow$}& { ARA(\%)$\uparrow$} & {RLA(\%)$\downarrow$}& \multicolumn{1}{l|}{{CRA(\%)$\uparrow$}} & { OA(\%)$\uparrow$}& { ARA(\%)$\uparrow$} & { RLA(\%)$\downarrow$}& \multicolumn{1}{l|}{{CRA(\%)$\uparrow$}} & {OA(\%)$\uparrow$}& { ARA(\%)$\uparrow$} & {RLA(\%)$\downarrow$}& { CRA(\%)$\uparrow$} \\ \hline
\multicolumn{1}{c|}{{ Human}} & 96.41& 97.79& -1.38 & \multicolumn{1}{l|}{92.03}& 95.56& 96.04& -0.48 & \multicolumn{1}{l|}{90.02}& 95.99 & 96.92 & -0.93& 91.03 \\ \hline
\multicolumn{13}{c}{\textit{Close-source LLMs}}\\ 
\multicolumn{1}{c|}{{ GPT-4o}}& { 91.37} & { 81.97} & { 9.40}& \multicolumn{1}{l|}{{ 75.55}} & { \textbf{88.63}} & { \textbf{70.17}} & { \textbf{18.46}} & \multicolumn{1}{l|}{{ \textbf{63.06}}} & { 90.00} & { \textbf{76.07}} & { \textbf{13.93}} & { \textbf{69.31}} \\
\multicolumn{1}{c|}{{ Claude3.5}}& { \textbf{95.37}} & { 80.15} & { 15.22} & \multicolumn{1}{l|}{{ 75.04}} & { 85.11} & { 66.02} & { 19.08} & \multicolumn{1}{l|}{{ 57.20}} & { 90.24} & { 73.09} & { 17.15} & { 66.12} \\
\multicolumn{1}{c|}{{ Gemini-1.5-Pro}}& { 90.62} & { 78.36} & { 12.26} & \multicolumn{1}{l|}{{ 70.48}} & { 87.75} & { 60.74} & { 27.01} & \multicolumn{1}{l|}{{ 58.27}} & { 89.19} & { 69.55} & { 19.63} & { 64.38} \\
\multicolumn{1}{c|}{{ Qwen-Max}}& { 93.50} & { \textbf{84.82}} & { \textbf{8.68}}& \multicolumn{1}{l|}{{ \textbf{78.91}}} & { 87.60} & { 62.61} & { 24.99} & \multicolumn{1}{l|}{{ 59.65}} & { \textbf{90.55}} & { 73.72} & { 16.83} & { 69.28} \\ \hline
\multicolumn{13}{c}{\textit{Chinese open-source LLMs}} \\ 
\multicolumn{1}{c|}{{ Qwen2.5-0.5B}}& { 60.75} & { 45.18} & { \textbf{15.57}} & \multicolumn{1}{l|}{{ 28.70}} & { 49.50} & { 38.21} & { \textbf{11.29}} & \multicolumn{1}{l|}{{ 20.57}} & { 55.13} & { 41.70} & { \textbf{13.43}} & { 24.64} \\
\multicolumn{1}{c|}{{ Qwen2.5-1.5B}}& { 63.25} & { 46.16} & { 17.09} & \multicolumn{1}{l|}{{ 29.89}} & { 56.88} & { 39.57} & { 17.30} & \multicolumn{1}{l|}{{ 23.48}} & { 60.06} & { 42.87} & { 17.20} & { 26.69} \\
\multicolumn{1}{c|}{{ Qwen2.5-3B}}& { 67.50} & { 48.75} & { 18.75} & \multicolumn{1}{l|}{{ 33.79}} & { 61.75} & { 39.98} & { 21.77} & \multicolumn{1}{l|}{{ 25.75}} & { 64.63} & { 44.37} & { 20.26} & { 29.77} \\
\multicolumn{1}{c|}{{ Qwen2.5-7B}}& { 67.63} & { 50.59} & { 17.04} & \multicolumn{1}{l|}{{ 35.62}} & { 65.63} & { 43.93} & { 21.70} & \multicolumn{1}{l|}{{ 30.77}} & { 66.63} & { 47.26} & { 19.37} & { 33.20} \\
\multicolumn{1}{c|}{{ Qwen2.5-14B}} & { 69.00} & { 51.41} & { 17.59} & \multicolumn{1}{l|}{{ 35.84}} & { 68.50} & { 45.20} & { 23.30} & \multicolumn{1}{l|}{{ 32.12}} & { 68.75} & { 48.30} & { 20.45} & { 33.98} \\
\multicolumn{1}{c|}{{ Qwen2.5-32B}} & { 69.75} & { 53.11} & { 16.64} & \multicolumn{1}{l|}{{ 37.54}} & { 70.00} & { 46.10} & { 23.90} & \multicolumn{1}{l|}{{ 32.68}} & { 69.88} & { 49.61} & { 20.27} & { 35.11} \\
\multicolumn{1}{c|}{{ Qwen2.5-72B}} & { \textbf{70.87}} & { \textbf{54.75}} & { 16.12} & \multicolumn{1}{l|}{{ \textbf{39.64}}} & { \textbf{72.00}} & { \textbf{47.75}} & { 24.25} & \multicolumn{1}{l|}{{\textbf{ 35.12}}} & { \textbf{71.44}} & { \textbf{51.25}} & {20.19} & { \textbf{37.38}} \\ \hdashline[0.5pt/5pt]
\multicolumn{1}{c|}{{ Baichuan2-7B}}& { 67.00} & { 46.16} & { 20.84} & \multicolumn{1}{l|}{{ 31.50}} & { 60.62} & { 39.04} & { 21.58} & \multicolumn{1}{l|}{{ 25.21}} & { 63.81} & { 42.60} & { 21.21} & { 28.36} \\
\multicolumn{1}{c|}{{ Baichua2-13B}}& { 69.13} & { 46.98} & { 22.15} & \multicolumn{1}{l|}{{ 33.45}} & { 64.62} & { 38.82} & { 25.80} & \multicolumn{1}{l|}{{ 26.07}} & { 66.88} & { 42.90} & { 23.97} & { 29.76} \\ \hdashline[0.5pt/5pt]
\multicolumn{1}{c|}{{ DeepSeek-7B}} & { 68.13} & { 47.96} & { 20.17} & \multicolumn{1}{l|}{{ 33.30}} & { 63.38} & { 40.39} & { 22.99} & \multicolumn{1}{l|}{{ 26.70}} & { 65.76} & { 44.18} & { 21.58} & { 30.00} \\
\multicolumn{1}{c|}{{ DeepSeek-67B}}& { 71.50} & { 49.21} & { 22.29} & \multicolumn{1}{l|}{{ 35.89}} & { 71.37} & { 40.63} & { 30.75} & \multicolumn{1}{l|}{{ 29.71}} & { 71.44} & { 44.92} & { 26.52} & { 32.80} \\ \hdashline[0.5pt/5pt]
\multicolumn{1}{c|}{{ InternLM2.5-1.8B}}& { 61.62} & { 42.07} & { 19.55} & \multicolumn{1}{l|}{{ 26.99}} & { 55.37} & { 38.46} & { 16.91} & \multicolumn{1}{l|}{{ 22.61}} & { 58.50} & { 40.27} & { 18.23} & { 24.80} \\
\multicolumn{1}{c|}{{ InternLM2.5-7B}}& { 67.25} & { 49.77} & { 17.48} & \multicolumn{1}{l|}{{ 34.57}} & { 69.50} & { 40.89} & { 28.61} & \multicolumn{1}{l|}{{ 29.75}} & { 68.38} & { 45.33} & { 23.04} & { 32.16} \\
\multicolumn{1}{c|}{{ InternLM2.5-20B}} & { 67.37} & { 48.08} & { 19.29} & \multicolumn{1}{l|}{{ 33.21}} & { 73.62} & { 41.11} & { 32.51} & \multicolumn{1}{l|}{{ 31.23}} & { 70.50} & { 44.60} & { 25.90} & { 32.22} \\ \hdashline[0.5pt/5pt]
\multicolumn{1}{c|}{{ Yi-1.5-6B}} & { 67.00} & { 49.59} & { 17.41} & \multicolumn{1}{l|}{{ 34.27}} & { 64.38} & { 39.37} & { 25.01} & \multicolumn{1}{l|}{{ 26.62}} & { 65.69} & { 44.48} & { 21.21} & { 30.45} \\
\multicolumn{1}{c|}{{ Yi-1.5-9B}} & { 68.50} & { 50.18} & { 18.32} & \multicolumn{1}{l|}{{ 35.55}} & { 66.37} & { 39.58} & { 26.79} & \multicolumn{1}{l|}{{ 27.48}} & { 67.44} & { 44.88} & { 22.56} & { 31.52} \\
\multicolumn{1}{c|}{{ Yi-1.5-34B}}& { 71.00} & { 52.23} & { 18.77} & \multicolumn{1}{l|}{{ 38.09}} & { 71.00} & { 40.75} & { 30.25} & \multicolumn{1}{l|}{{ 29.91}} & { 71.00} & { 46.49} & { 24.51} & { 34.00} \\ \hline
\multicolumn{13}{c}{\textit{English open-source LLMs}} \\ 
\multicolumn{1}{c|}{{ Llama3-8B}} & { 59.13} & { 46.62} & { 12.51} & \multicolumn{1}{l|}{{ 28.23}} & { 66.25} & { 40.21} & { 26.04} & \multicolumn{1}{l|}{{ 27.34}} & { 62.69} & { 43.42} & { 19.27} & { 27.79} \\
\multicolumn{1}{c|}{{ Llama3-70B}}& { 65.75} & { 48.63} & { 17.12} & \multicolumn{1}{l|}{{ 32.70}} & { \textbf{72.50}} & { 41.27} & { 31.23} & \multicolumn{1}{l|}{{\textbf{ 30.63}}} & {\textbf{ 69.13}} & { 44.95} & { 24.18} & { 31.67} \\ \hdashline[0.5pt/5pt]
\multicolumn{1}{c|}{{ Mistral-7B-v0.2}} & { 57.75} & { 46.25} & { \textbf{11.50}} & \multicolumn{1}{l|}{{ 27.57}} & { 67.50} & { \textbf{41.52}} & { 25.98} & \multicolumn{1}{l|}{{ 28.93}} & { 62.63} & { 43.88} & { 18.74} & { 28.25} \\
\multicolumn{1}{c|}{{ Mixtral-8x7B-v0.1}} & { 63.62} & { 46.80} & { 16.82} & \multicolumn{1}{l|}{{ 30.82}} & { 69.75} & { 41.21} & { 28.54} & \multicolumn{1}{l|}{{ 29.39}} & { 66.69} & { 44.01} & { 22.68} & { 30.11} \\
\multicolumn{1}{c|}{{ Mixtral-8x22B-v0.1}}& { 66.00} & {\textbf{ 50.73}} & { 15.27} & \multicolumn{1}{l|}{{ \textbf{34.32}}} & { 72.12} & { 41.25} & { 30.87} & \multicolumn{1}{l|}{{ 30.61}} & { 69.06} & { \textbf{45.99}} & { 23.07} & { \textbf{32.47}} \\ \hdashline[0.5pt/5pt]
\multicolumn{1}{c|}{{ Gemma-2-2B}}& { 61.88} & { 45.38} & { 16.51} & \multicolumn{1}{l|}{{ 29.02}} & { 59.62} & { 39.13} & { \textbf{20.50}} & \multicolumn{1}{l|}{{ 24.88}} & { 60.75} & { 42.25} & {\textbf{ 18.50}} & { 26.95} \\
\multicolumn{1}{c|}{{ Gemma-2-9B}}& { \textbf{69.13}} & { 46.75} & { 22.38} & \multicolumn{1}{l|}{{ 33.29}} & { 64.88} & { 39.80} & { 25.08} & \multicolumn{1}{l|}{{ 26.91}} & { 67.01} & { 43.28} & { 23.73} & { 30.10} \\
\multicolumn{1}{c|}{{ Gemma-2-27B}} & { 63.38} & { 48.52} & { 14.86} & \multicolumn{1}{l|}{{ 31.96}} & { 71.88} & { 40.91} & { 30.97} & \multicolumn{1}{l|}{{ 30.25}} & { 67.63} & { 44.71} & { 22.92} & { 31.11} \\ \hline
\end{tabular}
}
\caption{TODO: bolded is not result. Results of existing LLMs on our HellaSwag-Pro dataset using \textbf{Direct} prompt. ``AVG'' indicates the average performance of each model on Chinese and English parts of the dataset.
The best results for each metric in each model category are \textbf{bolded}. }
\label{tab:main experiment.}
\end{table*}

\subsection{Model Performance (RQ1)}
\paragraph{Overall Performance}
Table \ref{tab:main experiment.} provides a comprehensive evaluation of various LLMs across four performance metrics\footnote{The results of instruct/chat models of Qwen2.5, Llama3 and Mixtral latest series are shown in Appendix.}. The main observations are as follow:
\begin{itemize}[leftmargin=*,topsep=0pt]
% \setlength{}{0}
    \item Upon evaluating all available models, we found that all performed well in overall accuracy (e.g., GPT-4 scored 90.00 in AVG OA, Claude 3.5 scored 90.24 in AVG OA). However, all models struggled with variations of the questions, as evidenced by a positive RLA value for each model. In contrast, humans received a negative RLA value, suggesting that the question variants were not more challenging than the originals. This disparity further illustrates that current LLMs lack a true understanding of the reasoning process and can easily be misled by question variants.
    \item When comparing open-source and close-source models, the close-source models demonstrate stronger capabilities in both OA and ARA scores, similar to most existing benchmarks. Overall, the RLA values for close-source models are also smaller, indicating that they are more robust in commonsense reasoning tasks compared to open-source models.
    \item When we compare models within the same series (e.g., Qwen, Llama), we observe that larger models often achieve higher scores on OA, ARA, and CRA. However, they are also more susceptible to variations, i.e., they have higher RLA values, a phenomenon particularly evident in English datasets. We attribute this phenomenon to the fact that larger models, compared to smaller ones, may have memorized more data, allowing them to rely on memorization to solve some problems more easily and making them more prone to the influence of variations~\cite{}.
\end{itemize}
% 1. When evaluating all available models, We find although 
% 2. When comparing the opensource LLMs and close source LLMs, 
% 3. When looking into each serious details
% \noindent
% \textbf{Overall Model Performance.}
% 1. close-source > open-source 2. the large the better 3. all have a performance decline when meeting varients.

% To evaluate the performance of various models, we observed patterns consistent with current mainstream trends: closed-source models generally outperform open-source models across metrics. 
% For instance, the closed-source model GPT-4o achieved scores of 90.00 in OA, 76.07 in ARA, and 69.31 in CRA, whereas the open-source model Qwen2.5-72B scored 71.44, 51.25, and 37.38, respectively. 
% Furthermore, within each model series, performance tends to improve with larger model sizes. 
% Nevertheless, even the strongest closed-source models struggle with variations in questions, as indicated by positive values in RLA for all models. In contrast, human performance yields a negative RLA value, highlighting that current LLMs do not genuinely grasp the reasoning process and are prone to falling into traps set by question variants. 
% This suggests that there is still significant room for improvement in developing models that can robustly understand and reason through complex linguistic challenges.
% It reveals a consistent pattern across Chinese, English, and average scores, with close-sourced LLMs generally outperforming open-sourced models. 
% However, all models exhibit a significant drop in performance when faced with robust variants, as indicated by RLA and CRA. Among closed-source models, GPT-4o demonstrates the highest ARA of 76.07\% in average scores, demonstrating its overwhelming superiority. Among open-sourced models, larger models tend to perform better, with Qwen2.5-72B achieving the highest OA (71.44\%) and ARA (51.25\%) in the average scores. However, even these top performers still struggle with robustness, as evidenced by the substantial RLA of 13.93\% for GPT-4o and 20.19\% for Qwen2.5-72B. Interestingly, some English open-sourced models, such as Llama3-70B and Mixtral-8x22B-v0.1, show competitive performance in English tasks but lag in Chinese tasks, highlighting the importance of language-specific training.

% \noindent
% \textbf{Chinese Models vs English Models.}
% Chinese models generally demonstrate higher OA in Chinese tasks compared to English tasks, with Qwen-Max achieving 93.50\% OA in Chinese versus 87.60\% in English. Conversely, English models tend to perform better in English tasks, exemplified by Llama3-70B's 72.50\% OA in English compared to 65.75\% in Chinese. 
% However, both Chinese and English models exhibit important drops in ARA across languages, indicating challenges in maintaining performance when faced with variations. This trend suggests that while models may excel in their primary language, they struggle with robustness across linguistic boundaries. 
% Notably, larger models tend to achieve higher ARA scores but also experience more substantial RLA, as seen with Qwen2.5-0.5B (41.70\% ARA, 13.43\% RLA in total) and Qwen2.5-72B (51.25\% ARA, 20.19\% RLA in total). 
% This pattern indicates that while increased model size enhances overall performance, it doesn't necessarily improve robustness proportionally. 
% The discrepancy between OA and ARA across languages underscores the need for improved cross-lingual robustness in language models, particularly as they scale in size and capability.


% \noindent
% \textbf{Comparison between Chinese and English datasets.}
% Generally, models demonstrate higher accuracy on the Chinese dataset compared to the English one, as evidenced by the consistently higher OA, ARA and CRA scores. For instance, GPT-4o achieves an OA of 91.37\%, an ARA of 81.97\% , an CRA of 75.55\% on the Chinese dataset, compared to 88.63\% and 70.17\% respectively on the English dataset. This trend is observed across most models, suggesting that the Chinese dataset is easier than English one. Moreover, the RLA values are typically lower for Chinese, indicating smaller performance drops when dealing with robust variants of Chinese questions. For example, Qwen-Max shows an RLA of 8.68\% for Chinese versus 24.99\% for English, highlighting a more consistent performance in Chinese. The CRA scores further reinforce this observation, with models generally maintaining higher consistency in correct answers for both original and variant Chinese questions.
% We attribute this phenomenon to the fact that blablabla

\noindent
\textbf{Reasoning Transferable Capability.}
% 为了进一步
To further analyze whether the model can transfer reasoning ability from the original question to its variant, Figure \ref{consis} presents the distribution of model performance on the original question and variant pairs. For all models, the pairs of (HellaSwag \ding{51} HellaSwag-Pro \ding{55}) occupy a significant proportion, indicating a challenge in transferring reasoning capabilities for current LLMs to more complex scenarios. Looking deeply, closed-source models like GPT-4 and Qwen-Max achieve around a 69\% portion of (HellaSwag \ding{51} HellaSwag-Pro \ding{51}) and a 3\% portion of (HellaSwag \ding{55} HellaSwag-Pro \ding{55}), while in contrast, open-source models struggle with around a 30\% portion of (HellaSwag \ding{51} HellaSwag-Pro \ding{51}) and a 20\% portion of (HellaSwag \ding{55} HellaSwag-Pro \ding{55}), further indicating the robustness of reasoning abilities in closed-source models.
% If a model can get both the original question and the variant right, we consider it to have transferable reasoning ability. Table \ref{consis} presents the distribution of model performance on the original question and variant pairs. Among all models, the pairs of (HellaSwag \ding{51}HellaSwag-Pro \ding{55}) account for a considerable proportion, i 
% The closed-source models like GPT-4o and Qwen-Max achieve around 69\% portion of (HellaSwag \ding{51}HellaSwag-Pro \ding{51}) and 3\% portion of (HellaSwag \ding{55} HellaSwag-Pro \ding{55}), indicating stronger reasoning transfer ability than other models. In contrast, open-source models struggle more, with around 30\% portion of (HellaSwag \ding{51}HellaSwag-Pro \ding{51}) and 20\% portion of (HellaSwag \ding{55} HellaSwag-Pro \ding{55}). 
% A notable trend is observed among the Qwen2.5 series, where increasing model size from 7B to 72B parameters correlates with improved performance on correct answers for both datasets (33.20\% to 37.38\%) and decreased failure rates (17.69\% to 14.7\%). It underscores the importance of model size in commonsense reasoning tasks.

\begin{figure}[t]
\centering
\setlength{\abovecaptionskip}{0.1cm}
\setlength{\belowcaptionskip}{0cm}
\includegraphics[width=\linewidth,scale=1.00]{images/consis.pdf}
\caption{Analysis of the transferable ability of model reasoning based on question pair performance. The green part, where both the original and the variant data are right, represents the transferable performance of model reasoning.}
\label{consis}
\vspace{-15pt}
\end{figure}

\begin{figure*}[ht]
\centering
\setlength{\abovecaptionskip}{0.1cm}
\setlength{\belowcaptionskip}{0cm}
\includegraphics[width=\linewidth,scale=1.00]{images/xing.pdf}
\caption{The impact of different few-shot prompts on model performance. With - as the separator, the first two parts of the legend represent the prompt name, and the third part represents the language of the dataset.}
\label{xing}
\vspace{-15pt}
\end{figure*}

\begin{figure}[ht]
\centering
\setlength{\abovecaptionskip}{0.1cm}
\setlength{\belowcaptionskip}{0cm}
\includegraphics[width=1.05\linewidth,scale=1.05]{images/zhu.pdf}
\caption{The RLA Distribution for 7 variants of commonsense reasoning. Parts below the 0 axis indicate that the model’s performance on the variant is improved compared to the original problem.}
\label{fig:zhu}
\vspace{-15pt}
\end{figure}


\subsection{Variant Analysis (RQ2)}
To further analyze the impact of different variants, we assessed the contribution of each variant to the RLA score. A higher contribution indicates that the model is more likely to make errors in that type. Figure~\ref{fig:zhu} presents the overall results, and the key observations are as follows:
\begin{itemize}[leftmargin=*]
    \item For problem restatement, causal inference, and sentence ordering, these three categories are the least challenging. Almost all models, particularly the close-source and Qwen series models, perform well on these variants, indicating that current LLMs can effectively handle these forms and we do not pay more attention on this kind of varients.
    \item For reverse conversion and critical testing, these two varients each contribute about 10\% to the RLA score. This indicates that current LLMs struggle to fully generalize to these simple scenarios, possibly because these types of questions are not commonly encountered, and reaserchers should pay some attention to this type of varients.
    \item For negative transformation and scenario refinement, this are the two most difficult tasks, with negative transformation being particularly challenging. For almost all models, these two varients accounts for more than 50\% of the RLA score. This may be due to intuitively counterintuitive questions—such as the use of "will not"  or counterfactual scenarios in scenario refinement. These setups are less common in LLM training data and cannot be easily tackled through memory alone. Only those LLMs which truely understand the question could answer the varient correctly, wihch better reflect the true performance of the model.. In the future, researchers should focus more on enhancing LLM's capability to address such types of questions.
\end{itemize}

% 1. Problem restCausal Inference 
% To further analysis the impact of different varients, we further 
% Figure \ref{fig: zhu} presents a comprehensive analysis of various LLMs' performance across different variant types. Negative transformation emerges as the most challenging task for all models, with scores consistently above 50.00\% and peaking at 78.38\% for Gemini-1.5-Pro. Conversely, problem restatement appears to be the least challenging, with most models scoring in the negative range. Intriguingly, smaller models like Qwen2.5-0.5B demonstrate unexpected strengths in certain areas, such as sentence sorting (7.75\%), outperforming some larger counterparts. A detailed analysis of each variant type follows.

% \noindent
% \textbf{Causal inference.} In this category, scores vary widely from -4.73\% for Qwen-Max to 12.25\% for Baichuan2-13B, illustrating differing degrees of sensitivity to causal reasoning among the models. Smaller models, such as Qwen2.5-0.5B and Qwen2.5-1.5B, achieve better scores, indicating relatively stronger robustness in causal reasoning. Conversely, larger models, like Baichuan2-13B, have higher scores, suggesting greater sensitivity to the challenges of inferring causality.

% \noindent
% \textbf{Critical testing.} Larger models, including Qwen2.5-72B and DeepSeek-67B, exhibit higher RLA scores of 30.50\% and 31.37\%, respectively, suggesting increased sensitivity when dealing with incomplete key information. In contrast, GPT-4o achieves the lowest score, highlighting its superior robustness in critical reasoning. This trend indicates that more complex models might struggle to handle incomplete contexts, underscoring potential areas for improvement in sophisticated architectures.

% \noindent
% \textbf{Negative transformation.} This aspect remains consistently challenging for all models, with scores ranging from 48.88\% to 78.38\%. Advanced commercial models like Gemini-1.5-Pro and Claude-3.5 also score higher (78.38\% and 76.43\%, respectively), indicating a prevalent sensitivity issue in reasoning processes when handling negations, irrespective of model size or architecture.

% \noindent
% \textbf{Problem restatement.} The negative values in this category for nearly all models suggest it is not particularly challenging. This is surprising, given that previous models were quite sensitive to sentence representation.

% \noindent
% \textbf{Reverse conversion.} This variation, which involves swapping the roles of the question and answer, seems to specifically impact larger models. For example, Qwen2.5-72B and DeepSeek-67B exhibit higher RLA scores of 24.38\% and 27.43\%, respectively, indicating heightened sensitivity to reverse reasoning compared to their performance on original questions.

% \noindent
% \textbf{Scenario refinement.} The scores range from 16.06\% for Gemma-2-2B to 32.56\% for Qwen2.5-72B, with larger models displaying more sensitivity in adapting to counterfactual predictions. This suggests that larger models may rely more heavily on general commonsense rather than flexibly adapting to specific contexts. Consequently, increased model complexity might adversely affect adaptability to scenario changes, underscoring the need for enhanced flexibility in advanced models.

% \noindent
% \textbf{Sentence sorting.} This category exhibits the most varied results across models. Some larger models like DeepSeek-67B and InternLM2.5-20B display higher scores (26.69\% and 26.68\%), indicating sensitivity, while others like Qwen2.5-72B and Gemini-1.5-Pro excel with lower scores (-9.88\% and -1.07\%, respectively). This suggests that sentence sorting ability may depend more on specific training approaches rather than being solely contingent on model size.


\subsection{Prompt Robustness (RQ3)}
% To investigate how prompt  influence our benchmark, we apply sereral prompt strategy on our datasets and showcase the average performance of all models on different kind of prompt strategies.
% Table~\ref{prompt} illustrates the final results. For both Chinese and English datasets, CN LLMs achieve the highest performance using CN-CoT-Few-Shot, followed closely by EN-CoT-Few-Shot, with overall performance scores of 67.36\% and 67.03\%, respectively. In contrast, English LLMs perform best with the EN-CoT-Few-Shot, reaching 67.55\% on the Chinese dataset and 60.36\% on the English dataset.
% Contrary to previous findings, translating the dataset to the model's advantage language before performing reasoning does not enhance performance. Moreover, Figure~\ref{xing} also shows the similar phenomenon. Conducting CoT reasoning in the model’s advantage language generally leads to better outcomes compared to Direct. Additionally, increasing the number of shots consistently improves performance across most configurations, highlighting the benefits of exposing models to multiple examples. 
To explore the impact of various prompt strategies on our benchmarks, we evaluated several approaches across our datasets and present the average performance of all models using different prompting techniques. Table~\ref{prompt} summarizes the results. For both Chinese and English datasets, Chinese LLMs performed best with the CN-CoT-Few-Shot strategy, followed closely by EN-CoT-Few-Shot, achieving overall scores of 67.36\% and 67.03\%, respectively. Conversely, English LLMs showed optimal performance with the EN-CoT-Few-Shot approach, attaining 67.55\% on the Chinese dataset and 60.36\% on the English dataset.
Besides, translating datasets into the model's native language before reasoning did not enhance performance. This phenomenon is further illustrated in Figure~\ref{xing}. Conducting CoT reasoning in the model's native language generally yields better results compared to direct reasoning. Furthermore, increasing the number of examples (shots) consistently boosts performance across most configurations, emphasizing the advantages of exposing models to multiple examples.
% Overall, the interaction between question language, prompt language, and the number of shots underscores the importance of aligning these factors to optimize task performance and robustness in LLMs.



% Please add the following required packages to your document preamble:
% \usepackage{multirow}
% Please add the following required packages to your document preamble:
% \usepackage{multirow}
\begin{table}[t]
\setlength{\tabcolsep}{8pt}
% \footnotesize
\scalebox{0.65}{
\begin{tabular}{c|l|lll}
\hline
\multicolumn{1}{l|}{Dataset}  & Prompt  & CN LLMs & EN LLMs &  LLMs \\ \hline
\multirow{7}{*}{\begin{tabular}[c]{@{}c@{}}Chinese\\ HellaSwag-Pro\end{tabular}} & Direct  & 48.95& 41.16& 45.06  \\
& CN-CoT-Few  & \textbf{71.04}& 51.90& 61.47  \\
& EN-CoT-Few  & 70.95& \textbf{67.55}& \textbf{69.25}  \\
& EN-XLT-Few  & 41.48& 28.69& 35.09  \\
& CN-CoT-Zero & 44.82& 23.89& 34.36  \\
& EN-CoT-Zero & 45.38& 31.39& 38.39  \\
& EN-XLT-Zero & 28.57& 12.93& 20.75  \\ \hline
\multirow{7}{*}{\begin{tabular}[c]{@{}c@{}}English\\ HellaSwag-Pro\end{tabular}} & Direct  & 47.46& 40.66& 44.06  \\
& CN-CoT-Few  & \textbf{63.67}& 47.24& 55.46  \\
& EN-CoT-Few  & 63.12& \textbf{60.36}& \textbf{61.74}  \\
& CN-XLT-Few  & 48.77& 16.61& 32.69  \\
& CN-CoT-Zero & 34.89& 18.25& 26.57  \\
& EN-CoT-Zero & 42.41& 31.03& 36.72  \\
& CN-XLT-Zero & 16.36& 11.22& 13.79  \\ \hline
\multirow{9}{*}{HellaSwag-Pro}& Direct  & 48.21& 40.91& 44.83  \\
& CN-CoT-Few  & \textbf{67.36}& 49.57& 58.46  \\
& EN-CoT-Few  & 67.03& \textbf{63.95}& \textbf{65.49}  \\
& CN-XLT-Few  & 59.91& 34.26& 47.08  \\
& EN-XLT-Few  & 52.30& 44.52& 48.41  \\
& CN-CoT-Zero & 39.86& 21.07& 30.46  \\
& EN-CoT-Zero & 43.90& 31.21& 37.55  \\
& CN-XLT-Zero & 30.59& 17.55& 24.07  \\
& EN-XLT-Zero & 35.49& 21.98& 28.74  \\ \hline
\end{tabular}
}
\caption{Average ARA of all open-source models on different prompts. CN-LLMs contains 17 LLMs, and EN-LLMs contains 7 LLMs. The bast results for each dataset are \textbf{bolded}.}
\label{prompt}
\end{table}




\section{Concluding Remarks}
In this paper, we proposed a novel approach utilizing multimodal LLMs to generate gesture-aware speech recognition transcripts for patients with language disorders. Our framework integrates verbal speech and iconic gestures, enabling the generation of enriched transcripts that capture the latent meaning conveyed through both modalities. Through extensive experimentation, we demonstrated that the proposed method effectively contextualizes incomplete or disfluent speech by incorporating gesture information, leading to more accurate and meaningful representations of the speaker's intent. These findings highlight the potential of our approach to significantly contribute to the field of speech and language therapy, offering innovative tools that can enhance the quality of life for individuals with language disorders by facilitating better communication and assessment methods.

\subsection{Ethical Statement} 
Our dataset was obtained from AphasiaBank with the approval of the Institutional Review Board (IRB) and adheres to the data sharing guidelines set by TalkBank\footnote{https://talkbank.org/share/ethics.html}. This includes complying with the Ground Rules for all TalkBank databases, which are based on the American Psychological Association Code of Ethics~\cite{american2002ethical}.

\subsection{Limitation \& Future Work} 
%This study represents a preliminary investigation into using multimodal LLMs to generate gesture-aware speech recognition transcripts. 
While the results are promising, we recognize several limitations and outline our plans to extend this work further.

One primary limitation is the absence of a definitive ground truth for quantitative evaluation. Since our model generates transcripts by synthesizing speech and gesture data from scratch, traditional benchmarks, such as comparisons with standard speech recognition outputs, are insufficient. Moreover, existing original transcripts lack gesture annotations, making direct comparisons challenging. In future work, we aim to address this gap by collaborating with certified pathologists to conduct qualitative assessments, such as A-B preference tests, to evaluate the effectiveness of gesture-enriched transcripts in accurately conveying the speaker's intentions.

To support quantitative evaluations, we plan to develop novel metrics that assess transcript quality, including grammar accuracy, semantic consistency, and the integration of multimodal information. Such metrics will provide a more objective basis for assessing our model's performance and facilitate comparisons with other multimodal and unimodal approaches.

Another limitation of this study is its focus on structured gestures from a specific task, the Peanut Butter Sandwich Task. While this task offers a controlled context for testing our approach, it does not encompass the diversity of gestures and communication patterns seen in everyday scenarios. As part of our future work, we plan to expand the scope of our model to include tasks such as the Cinderella Story Recall Task~\cite{bird1996cinderella}, which involves unstructured and complex narrative gestures. This expansion will allow us to evaluate the adaptability and robustness of our model in handling varied linguistic and gestural contexts.

In summary, while this study establishes a strong foundation for gesture-aware speech recognition, we aim to refine and extend our methods through collaborative qualitative evaluations, the development of robust quantitative metrics, and broader task applications. These efforts will ensure that our approach continues to evolve, ultimately contributing to more effective communication tools and interventions for individuals with language disorders.




\clearpage


% references without citing it in the main text, use \nocite
\nocite{langley00}
\section*{Acknowledgements}
This work was supported by the National Natural Science Foundation of China (42125503, 42430602).
\bibliography{example_paper}
\bibliographystyle{icml2025}


%%%%%%%%%%%%%%%%%%%%%%%%%%%%%%%%%%%%%%%%%%%%%%%%%%%%%%%%%%%%%%%%%%%%%%%%%%%%%%%
%%%%%%%%%%%%%%%%%%%%%%%%%%%%%%%%%%%%%%%%%%%%%%%%%%%%%%%%%%%%%%%%%%%%%%%%%%%%%%%
% APPENDIX
%%%%%%%%%%%%%%%%%%%%%%%%%%%%%%%%%%%%%%%%%%%%%%%%%%%%%%%%%%%%%%%%%%%%%%%%%%%%%%%
%%%%%%%%%%%%%%%%%%%%%%%%%%%%%%%%%%%%%%%%%%%%%%%%%%%%%%%%%%%%%%%%%%%%%%%%%%%%%%%
\newpage
\appendix
\onecolumn


\section{Metric}
\label{sec:metric}

\textbf{Mean Squared Error (MSE)} Mean Squared Error (MSE) is a common statistical metric used to assess the difference between predicted and actual values. The formula is:
\begin{equation}
    MSE = \frac{1}{n} \sum_{i=1}^{n} (y_i - \hat{y}_i)^2
\end{equation}
where $ n $ is the number of samples, $ y_i $ is the actual value, and $ \hat{y}_i $ is the predicted value.

\textbf{Relative L2 Error} Relative L2 error measures the relative difference between predicted and actual values, commonly used in time series prediction. The formula is:
\begin{equation}
    \text{Relative L2 Error} = \frac{\| Y_{\text{pred}} - Y_{\text{true}} \|_2}{\| Y_{\text{true}} \|_2}
\end{equation}
where $ Y_{\text{pred}} $ is the predicted value and $ Y_{\text{true}} $ is the actual value.

\textbf{Structural Similarity Index Measure (SSIM)} The Structural Similarity Index (SSIM) measures the similarity between two images in terms of luminance, contrast, and structure. The formula is:
\begin{equation}
    SSIM(x, y) = \frac{(2\mu_x \mu_y + C_1)(2\sigma_{xy} + C_2)}{(\mu_x^2 + \mu_y^2 + C_1)(\sigma_x^2 + \sigma_y^2 + C_2)}
\end{equation}
where $ \mu_x $ and $ \mu_y $ are the mean values, $ \sigma_x $ and $ \sigma_y $ are the standard deviations, $ \sigma_{xy} $ is the covariance.

\section{Related Work}
\subsection{Deep Learning based Weather Forecasting}
\textbf{Global Weather Forecasting.} Global weather forecasting has seen significant progress with deep learning models. FourCastNet, based on Fourier neural operators, provides global forecasts comparable to traditional numerical methods like IFS, but at much higher speeds~\cite{pathak2022fourcastnet}. Pangu, utilizing the Swin Transformer, exceeds NWP methods, incorporating earth-specific location embeddings for better performance~\cite{bi2023accurate}. The Spherical Fourier Neural Operator (SFNO) extends Fourier methods using spherical harmonics, offering more stable long-term predictions~\cite{bonev2023spherical}. FuXi focuses on long-term forecasting, achieving a 15-day forecasts comparable to ECMWF~\cite{chen2023fuxi}. GraphCast leverages message-passing networks to improve efficiency and forecasting accuracy~\cite{lam2023learning}, and GenCast builds on this to enhance ensemble forecasting~\cite{price2023gencast}. Further, diffusion models like those in~\cite{li2024generative} generate probabilistic ensembles by sampling, while NeuralGCM~\cite{kochkov2024neural} focuses on atmospheric circulation with a dynamic core, offering climate simulation capabilities but at higher training and inference costs. 

\textbf{Regional Weather Forecasting.} The goal of regional weather forecasting is to enhance local prediction accuracy with high-resolution models. CorrDiff~\cite{mardani2023generative} combines U-Net and diffusion models to improve local forecasts. MetaWeather~\cite{kim2024metaweather} adapts global forecasts to regional contexts using meta-learning. GNNs are also widely applied in regional forecasting, with Graphcast~\cite{lam2023learning} enhancing accuracy by modeling complex spatial dependencies. MetNet-3~\cite{espeholt2022deep} offers high-accuracy forecasts for weather variables, such as precipitation, temperature, and wind speed, at 2-minute intervals and 1–4 km resolution, outperforming traditional models like HRRR. NowcastNet~\cite{zhang2023skilful} and DGMR~\cite{ravuri2021skilful} excel in short-term extreme precipitation forecasts using deep generative models and radar data. In spatiotemporal prediction, NMO~\cite{wu2024neural} models the evolution of physical dynamics, providing new insights for local weather forecasting. Similarly, SimVP~\cite{gao2022simvp} and PastNet~\cite{wu2024pastnet} achieve good results in forecasting local precipitation evolution using spatiotemporal convolution methods.
    
% Despite these advances, none of these methods effectively address the challenge of balancing global and regional high-resolution forecasts or capturing the fine-grained, dynamic interactions important for extreme event prediction.
    
\subsection{Numerical analysis methods}
Multigrid methods~\cite{mccormick1987multigrid,wesseling1995introduction,hackbusch2013multi,bramble2019multigrid,hiptmair1998multigrid,brandt1983multigrid,borzi2009multigrid} and nested grid strategies~\cite{miyakoda1977one,zhang2012nested,sullivan1996grid} are widely used to solve PDEs and handle multi-scale problems~\cite{debreu2008two,xue2000advanced}. Multigrid methods use grids of different resolutions to transfer information and accelerate iterations. They efficiently solve large-scale problems and improve computational accuracy. By eliminating low-frequency errors on coarse grids and high-frequency errors on fine grids, multigrid methods effectively handle error convergence at different scales~\cite{he2019mgnet,he2023mgno,shao2022fast}. Nested grid strategies embed higher-resolution fine grids into regions of interest based on a global coarse grid to capture local complex physical phenomena in detail. In weather forecasting, this method provides large-scale background fields on a global scale while refining the grid for target regions to accurately simulate the evolution of local weather systems and the occurrence of extreme events~\cite{bacon2000dynamically}. 

% Our proposed neural nested grid method helps address challenges like boundary information loss in regional forecasting and multi-scale feature capture.

\section{Additional Results}
%
We present more additional results in Figure \ref{fig_0.25-day}, \ref{fig_0.5-day}, \ref{fig_1.0-day} \ref{fig_1.5-day}, \ref{fig_2.0-day}, \ref{fig_2.5-day}, \ref{fig_3.0-day}, \ref{fig_3.5-day}, \ref{fig_4.0-day}, \ref{fig_4.5-day}, \ref{fig_5.0-day}, \ref{fig_5.5-day}, \ref{fig_6.0-day}, \ref{fig_6.5-day}, \ref{fig_7.0-day}, \ref{fig_7.5-day},
\ref{fig_8.0-day}, \ref{fig_8.5-day}, \ref{fig_9.0-day}, \ref{fig_9.5-day},
\ref{fig_10.0-day}, including 18 variables that are importmant to weather forecasting, each with results ranging from 6 hours to 10 days. These additional results further demonstrate the effectiveness of OneForecast. Same as the Figure \ref{fig:visual_results}
, the initial conditions is 00:00 UTC, 1 January 2020.


\begin{figure*}[h]
\centering
\includegraphics[width=1\linewidth]{figures/fig_0.25-day.jpg}
\vspace{-20pt}
\caption{6-hour forecast results of different models.}
\label{fig_0.25-day}
\end{figure*}

\begin{figure*}[h]
\centering
\includegraphics[width=1\linewidth]{figures/fig_0.5-day.jpg}
\vspace{-20pt}
\caption{0.5-day forecast results of different models.}
\label{fig_0.5-day}
\end{figure*}

\begin{figure*}[h]
\centering
\includegraphics[width=1\linewidth]{figures/fig_1.0-day.jpg}
\vspace{-20pt}
\caption{1-day forecast results of different models.}
\label{fig_1.0-day}
\end{figure*}

\begin{figure*}[h]
\centering
\includegraphics[width=1\linewidth]{figures/fig_1.5-day.jpg}
\vspace{-20pt}
\caption{1.5-day forecast results of different models.}
\label{fig_1.5-day}
\end{figure*}

\begin{figure*}[h]
\centering
\includegraphics[width=1\linewidth]{figures/fig_2.0-day.jpg}
\vspace{-20pt}
\caption{2-day forecast results of different models.}
\label{fig_2.0-day}
\end{figure*}


\begin{figure*}[h]
\centering
\includegraphics[width=1\linewidth]{figures/fig_2.5-day.jpg}
\vspace{-20pt}
\caption{2.5-day forecast results of different models.}
\label{fig_2.5-day}
\end{figure*}

\begin{figure*}[h]
\centering
\includegraphics[width=1\linewidth]{figures/fig_3.0-day.jpg}
\vspace{-20pt}
\caption{3-day forecast results of different models.}
\label{fig_3.0-day}
\end{figure*}

\begin{figure*}[h]
\centering
\includegraphics[width=1\linewidth]{figures/fig_3.5-day.jpg}
\vspace{-20pt}
\caption{3.5-day forecast results of different models.}
\label{fig_3.5-day}
\end{figure*}

\begin{figure*}[h]
\centering
\includegraphics[width=1\linewidth]{figures/fig_4.0-day.jpg}
\vspace{-20pt}
\caption{4-day forecast results of different models.}
\label{fig_4.0-day}
\end{figure*}

\begin{figure*}[h]
\centering
\includegraphics[width=1\linewidth]{figures/fig_4.5-day.jpg}
\vspace{-20pt}
\caption{4.5-day forecast results of different models.}
\label{fig_4.5-day}
\end{figure*}


\begin{figure*}[h]
\centering
\includegraphics[width=1\linewidth]{figures/fig_5.0-day.jpg}
\vspace{-20pt}
\caption{5.0-day forecast results of different models.}
\label{fig_5.0-day}
\end{figure*}

\begin{figure*}[h]
\centering
\includegraphics[width=1\linewidth]{figures/fig_5.5-day.jpg}
\vspace{-20pt}
\caption{5.5-day forecast results of different models.}
\label{fig_5.5-day}
\end{figure*}

\begin{figure*}[h]
\centering
\includegraphics[width=1\linewidth]{figures/fig_6.0-day.jpg}
\vspace{-20pt}
\caption{6.0-day forecast results of different models.}
\label{fig_6.0-day}
\end{figure*}

\begin{figure*}[h]
\centering
\includegraphics[width=1\linewidth]{figures/fig_6.5-day.jpg}
\vspace{-20pt}
\caption{6.5-day forecast results of different models.}
\label{fig_6.5-day}
\end{figure*}

\begin{figure*}[h]
\centering
\includegraphics[width=1\linewidth]{figures/fig_7.0-day.jpg}
\vspace{-20pt}
\caption{7.0-day forecast results of different models.}
\label{fig_7.0-day}
\end{figure*}

\begin{figure*}[h]
\centering
\includegraphics[width=1\linewidth]{figures/fig_7.5-day.jpg}
\vspace{-20pt}
\caption{7.5-day forecast results of different models.}
\label{fig_7.5-day}
\end{figure*}

\begin{figure*}[h]
\centering
\includegraphics[width=1\linewidth]{figures/fig_8.0-day.jpg}
\vspace{-20pt}
\caption{8.0-day forecast results of different models.}
\label{fig_8.0-day}
\end{figure*}

\begin{figure*}[h]
\centering
\includegraphics[width=1\linewidth]{figures/fig_8.5-day.jpg}
\vspace{-20pt}
\caption{8.5-day forecast results of different models.}
\label{fig_8.5-day}
\end{figure*}

\begin{figure*}[h]
\centering
\includegraphics[width=1\linewidth]{figures/fig_9.0-day.jpg}
\vspace{-20pt}
\caption{9.0-day forecast results of different models.}
\label{fig_9.0-day}
\end{figure*}

\begin{figure*}[h]
\centering
\includegraphics[width=1\linewidth]{figures/fig_9.5-day.jpg}
\vspace{-20pt}
\caption{9.5-day forecast results of different models.}
\label{fig_9.5-day}
\end{figure*}

\begin{figure*}[h]
\centering
\includegraphics[width=1\linewidth]{figures/fig_10.0-day.jpg}
\vspace{-20pt}
\caption{10.0-day forecast results of different models.}
\label{fig_10.0-day}
\end{figure*}


\section{Detailed Mathematical Proof}
\label{sec:proof}
\textbf{Proof of Theorem 1}

Now we have N augmented data and we need to select the best from them. We consider both the quality and the diversity of these data and get the sampling strategy from an optimization problem.

We model the sampling strategy as a multinomial distribution supported on all the augmented data $S = \{\mathbf{X}_j\}_{j=1}^N$, which means that the sampling strategy $\pi=(\pi_1,...,\pi_N)^\top$ is the corresponding probabilities of selecting $\mathbf{X}_1,...,\mathbf{X}_N$, then we can model the expectation of the similarity as:
$$\begin{aligned}
 & \mathbb{E}_{Y_x,Y_{x^{\prime}}\in\mathcal{C}}\{g(x,x^{\prime})\mid S\} \\
 & =\quad\int g(\mathbf{x},\mathbf{x}^{\prime})\boldsymbol{\pi}(\mathbf{x})\mathrm{Pr}_{S}(Y_{x}\in\mathcal{C}\mid\boldsymbol{x}=\mathbf{x})\boldsymbol{\pi}(\mathbf{x}^{\prime})\mathrm{Pr}_{S}(Y_{x}\in\mathcal{C}\mid\boldsymbol{x}=\mathbf{x}^{\prime})d\mathbf{x}d\mathbf{x}^{\prime} \\
 & =\quad\sum_{i,j=1}^Ng(\mathbf{X}_i,\mathbf{X}_j)\pi_i\pi_j\mathrm{Pr}_{S}(Y_x\in\mathcal{C}\mid\boldsymbol{x}=\mathbf{X}_i)\mathrm{Pr}_{S}(Y_x\in\mathcal{C}\mid\boldsymbol{x}=\mathbf{X}_j),
\end{aligned}$$
where the set $\mathcal{C}$ denotes the criterion of selection we are using, the function $g$ can be chosen as any similarity metric function and $x$ means a random variable.

The core to solving the above optimization problem is to use predictive inference to approximate the conditional probability of $\{Y_x\in\mathcal{C}\}$ given $x = \mathbf{X}$
Let $\mu ( \mathbf{x} ) : = \mathbb{E} ( Y\mid \mathbf{X} = \mathbf{x} )$ be the oracle associated with $( \mathbf{X} , Y) .$ Denote $\theta_j=\mathbb{I}\{Y_j\in\mathcal{C}\}$. As the augmented data
$\mathbf{X}_1,...,\mathbf{X}_N$ are independently identically distributed, $\theta_1,...,\theta_N$ can be regarded as independent Bernoulli($q)$ variables with $q=\Pr(Y_j\in\mathcal{C}).$ The probability distribution of the predicted result $W_j$ for $j=1,...,N$ is
$$\Pr(W_j\mid\theta_j)=(1-\theta_j)f_0+\theta_jf_1,\quad$$
where $f_0$ and $f_1$ are the conditional distributions of $W_j$ on $Y_j \in \mathcal{C}$ or not.

Denote $T(w) = \frac{(1-q)f_0(W_j)}{f(W_j)}$, we can rewrite the expectation of the similarity as
$$\mathbb{E}_{Y_x,Y_{x^{\prime}}\in\mathcal{C}}\{g(x,x^{\prime})|S\}=\sum_{i,j=1}^Ng(\mathbf{X}_i,\mathbf{X}_j)\pi_i\pi_j(1-T_i)(1-T_j)=\boldsymbol{\pi}^\top A_\mathbb{T}\boldsymbol{\pi},$$

Next, we use the expectation to control the quality of the data.
$$\mathbb{E}\{\mathbb{I}(Y_x\not\in\mathcal{C})\mid S\}=\sum_{i=1}^N\Pr(Y_i\not\in\mathcal{C}\mid\mathbf{X}_i)\pi_i=\sum_{i=1}^N\pi_iT_i\leq\alpha,$$

In all, the optimization problem can be modeled as 
\begin{align}
    & \arg\min_{\boldsymbol{\pi}}\quad h(\boldsymbol{\pi},\mathbb{T}):=\boldsymbol{\pi}^\top A_\mathbb{T}\boldsymbol{\pi}, \\
    & \text{subject to} \quad
        \begin{cases}
            \sum_{i = 1}^N\pi_iT_i\leq\alpha, \\
            \sum_{i = 1}^N\pi_i = 1, \\
            0\leq\pi_i\leq m^{-1}, \quad 1\leq i\leq N.
        \end{cases}
\end{align}

where $m$ is used to control the maximum selection.

The best selection of K is determined by the strategy $\pi$ which serves as the solution to the above optimization problem.

\section{Additional Experiments}
\label{sec:more_experiments}
\subsection{Long-term forecasting experiment expansion}

In the long-term forecasting experiments, we compare the performance of different backbone models on the SWE benchmark, evaluating the relative L2 error for three variables (U, V, and H). Our setup inputs 5 frames and predicts 50 frames. For the SimVP-v2 model, using \method{} reduces the relative L2 error for SWE (u) from 0.0187 to 0.0154, SWE (v) from 0.0387 to 0.0342, and SWE (h) from 0.0443 to 0.0397. We visualize SWE (h) in 3D as shown in Figure~\ref{fig:case} [\textcolor{red}{I}]. For the ConvLSTM model, applying \method{} reduces the relative L2 error for SWE (u) from 0.0487 to 0.0321, SWE (v) from 0.0673 to 0.0351, and SWE (h) from 0.0762 to 0.0432. For the FNO model, using \method{} reduces the relative L2 error for SWE (u) from 0.0571 to 0.0502, SWE (v) from 0.0832 to 0.0653, and SWE (h) from 0.0981 to 0.0911. Overall, \method{} significantly improves the long-term forecasting accuracy of different backbone models.

\begin{figure*}[h]
    \centering
    \includegraphics[width=\textwidth]{image/casestudy.pdf}
    \caption{
    \textcolor{red}{I.} 3D visualization of the SWE(h), showing Ground-truth, SimVP-V2+BeamVQ predictions, and Error at T=1, 10, 20, 30, 40, 50. The first row shows Ground-truth, the second SimVP-V2+BeamVQ predictions, and the third Error. \textcolor{red}{II.} A case study. Building fire simulation with ventilation settings added to Wu's Prometheus~\cite{wu2024prometheus}. (a) Layout and HRR growth. (b) Comparison of physical metrics for different methods. (c) Ground-truth, ResNet+BeamVQ, and ResNet predictions.
    }
    \label{fig:case} 
\end{figure*}


\subsection{Experiment Statistical Significance}
\label{sec:significance}
To measure the statistical significance of our main experiment results, we choose three backbones to train on two datasets to run 5 times. 
Table~\ref{tab:significance} records the average and standard deviation of the test MSE loss.
The results prove that our method is statistically significant to outperform the baselines
because our confidence interval is always upper than the confidence interval of the baselines. 
Due to limited computation resources, we do not cover all ten backbones and five datasets, 
but we believe these results have shown that our method has consistent advantages.


\begin{table}[h]
\label{tab:significance}
\centering
\begin{scriptsize}
    \begin{sc}
    \caption{ The average and standard deviation of MSE in 5 runs}
    \label{tab:significance}
    \centering
        \renewcommand{\multirowsetup}{\centering}
        \setlength{\tabcolsep}{10pt}
        \begin{tabular}{l|cc|cc}
            \toprule
            
            \multirow{4}{*}{Model} & \multicolumn{4}{c}{Benchmarks}  \\
            \cmidrule(lr){2-5}
            & \multicolumn{2}{c}{NSE} &   \multicolumn{2}{c}{SEVIR}   \\
            \cmidrule(lr){2-5}
           & Ori & + BeamVQ & Ori & + BeamVQ  \\
            \midrule
            ConvLSTM &0.4092$\pm$0.0002 &\textbf{0.1277$\pm$0.0001}  & 0.1762 0.0007  & \textbf{0.1279$\pm$0.0009}  \\
            FNO &  0.2227$\pm$0.0003 &\textbf{0.1007 $\pm$0.0002}& 0.0787$\pm$0.0012 & \textbf{ 0.0437$\pm$0.0013} \\
            CNO & 0.2192 $\pm$0.0008 &\textbf{ 0.1492$\pm$0.0011}& 0.0057$\pm$0.0005 & \textbf{ 0.0053$\pm$0.0006} \\
            \bottomrule
        \end{tabular}
    \end{sc}

\end{scriptsize}
\end{table}

\section{Related Work}
\subsection{Deep Learning based Weather Forecasting}
\textbf{Global Weather Forecasting.} Global weather forecasting has seen significant progress with deep learning models. FourCastNet, based on Fourier neural operators, provides global forecasts comparable to traditional numerical methods like IFS, but at much higher speeds~\cite{pathak2022fourcastnet}. Pangu, utilizing the Swin Transformer, exceeds NWP methods, incorporating earth-specific location embeddings for better performance~\cite{bi2023accurate}. The Spherical Fourier Neural Operator (SFNO) extends Fourier methods using spherical harmonics, offering more stable long-term predictions~\cite{bonev2023spherical}. FuXi focuses on long-term forecasting, achieving a 15-day forecasts comparable to ECMWF~\cite{chen2023fuxi}. GraphCast leverages message-passing networks to improve efficiency and forecasting accuracy~\cite{lam2023learning}, and GenCast builds on this to enhance ensemble forecasting~\cite{price2023gencast}. Further, diffusion models like those in~\cite{li2024generative} generate probabilistic ensembles by sampling, while NeuralGCM~\cite{kochkov2024neural} focuses on atmospheric circulation with a dynamic core, offering climate simulation capabilities but at higher training and inference costs. 

\textbf{Regional Weather Forecasting.} The goal of regional weather forecasting is to enhance local prediction accuracy with high-resolution models. CorrDiff~\cite{mardani2023generative} combines U-Net and diffusion models to improve local forecasts. MetaWeather~\cite{kim2024metaweather} adapts global forecasts to regional contexts using meta-learning. GNNs are also widely applied in regional forecasting, with Graphcast~\cite{lam2023learning} enhancing accuracy by modeling complex spatial dependencies. MetNet-3~\cite{espeholt2022deep} offers high-accuracy forecasts for weather variables, such as precipitation, temperature, and wind speed, at 2-minute intervals and 1–4 km resolution, outperforming traditional models like HRRR. NowcastNet~\cite{zhang2023skilful} and DGMR~\cite{ravuri2021skilful} excel in short-term extreme precipitation forecasts using deep generative models and radar data. In spatiotemporal prediction, NMO~\cite{wu2024neural} models the evolution of physical dynamics, providing new insights for local weather forecasting. Similarly, SimVP~\cite{gao2022simvp} and PastNet~\cite{wu2024pastnet} achieve good results in forecasting local precipitation evolution using spatiotemporal convolution methods.
    
% Despite these advances, none of these methods effectively address the challenge of balancing global and regional high-resolution forecasts or capturing the fine-grained, dynamic interactions important for extreme event prediction.
    
\subsection{Numerical analysis methods}
Multigrid methods~\cite{mccormick1987multigrid,wesseling1995introduction,hackbusch2013multi,bramble2019multigrid,hiptmair1998multigrid,brandt1983multigrid,borzi2009multigrid} and nested grid strategies~\cite{miyakoda1977one,zhang2012nested,sullivan1996grid} are widely used to solve PDEs and handle multi-scale problems~\cite{debreu2008two,xue2000advanced}. Multigrid methods use grids of different resolutions to transfer information and accelerate iterations. They efficiently solve large-scale problems and improve computational accuracy. By eliminating low-frequency errors on coarse grids and high-frequency errors on fine grids, multigrid methods effectively handle error convergence at different scales~\cite{he2019mgnet,he2023mgno,shao2022fast}. Nested grid strategies embed higher-resolution fine grids into regions of interest based on a global coarse grid to capture local complex physical phenomena in detail. In weather forecasting, this method provides large-scale background fields on a global scale while refining the grid for target regions to accurately simulate the evolution of local weather systems and the occurrence of extreme events~\cite{bacon2000dynamically}. 

% Our proposed neural nested grid method helps address challenges like boundary information loss in regional forecasting and multi-scale feature capture.

\section{Additional Results}
%
We present more additional results in Figure \ref{fig_0.25-day}, \ref{fig_0.5-day}, \ref{fig_1.0-day} \ref{fig_1.5-day}, \ref{fig_2.0-day}, \ref{fig_2.5-day}, \ref{fig_3.0-day}, \ref{fig_3.5-day}, \ref{fig_4.0-day}, \ref{fig_4.5-day}, \ref{fig_5.0-day}, \ref{fig_5.5-day}, \ref{fig_6.0-day}, \ref{fig_6.5-day}, \ref{fig_7.0-day}, \ref{fig_7.5-day},
\ref{fig_8.0-day}, \ref{fig_8.5-day}, \ref{fig_9.0-day}, \ref{fig_9.5-day},
\ref{fig_10.0-day}, including 18 variables that are importmant to weather forecasting, each with results ranging from 6 hours to 10 days. These additional results further demonstrate the effectiveness of OneForecast. Same as the Figure \ref{fig:visual_results}
, the initial conditions is 00:00 UTC, 1 January 2020.


\begin{figure*}[h]
\centering
\includegraphics[width=1\linewidth]{figures/fig_0.25-day.jpg}
\vspace{-20pt}
\caption{6-hour forecast results of different models.}
\label{fig_0.25-day}
\end{figure*}

\begin{figure*}[h]
\centering
\includegraphics[width=1\linewidth]{figures/fig_0.5-day.jpg}
\vspace{-20pt}
\caption{0.5-day forecast results of different models.}
\label{fig_0.5-day}
\end{figure*}

\begin{figure*}[h]
\centering
\includegraphics[width=1\linewidth]{figures/fig_1.0-day.jpg}
\vspace{-20pt}
\caption{1-day forecast results of different models.}
\label{fig_1.0-day}
\end{figure*}

\begin{figure*}[h]
\centering
\includegraphics[width=1\linewidth]{figures/fig_1.5-day.jpg}
\vspace{-20pt}
\caption{1.5-day forecast results of different models.}
\label{fig_1.5-day}
\end{figure*}

\begin{figure*}[h]
\centering
\includegraphics[width=1\linewidth]{figures/fig_2.0-day.jpg}
\vspace{-20pt}
\caption{2-day forecast results of different models.}
\label{fig_2.0-day}
\end{figure*}


\begin{figure*}[h]
\centering
\includegraphics[width=1\linewidth]{figures/fig_2.5-day.jpg}
\vspace{-20pt}
\caption{2.5-day forecast results of different models.}
\label{fig_2.5-day}
\end{figure*}

\begin{figure*}[h]
\centering
\includegraphics[width=1\linewidth]{figures/fig_3.0-day.jpg}
\vspace{-20pt}
\caption{3-day forecast results of different models.}
\label{fig_3.0-day}
\end{figure*}

\begin{figure*}[h]
\centering
\includegraphics[width=1\linewidth]{figures/fig_3.5-day.jpg}
\vspace{-20pt}
\caption{3.5-day forecast results of different models.}
\label{fig_3.5-day}
\end{figure*}

\begin{figure*}[h]
\centering
\includegraphics[width=1\linewidth]{figures/fig_4.0-day.jpg}
\vspace{-20pt}
\caption{4-day forecast results of different models.}
\label{fig_4.0-day}
\end{figure*}

\begin{figure*}[h]
\centering
\includegraphics[width=1\linewidth]{figures/fig_4.5-day.jpg}
\vspace{-20pt}
\caption{4.5-day forecast results of different models.}
\label{fig_4.5-day}
\end{figure*}


\begin{figure*}[h]
\centering
\includegraphics[width=1\linewidth]{figures/fig_5.0-day.jpg}
\vspace{-20pt}
\caption{5.0-day forecast results of different models.}
\label{fig_5.0-day}
\end{figure*}

\begin{figure*}[h]
\centering
\includegraphics[width=1\linewidth]{figures/fig_5.5-day.jpg}
\vspace{-20pt}
\caption{5.5-day forecast results of different models.}
\label{fig_5.5-day}
\end{figure*}

\begin{figure*}[h]
\centering
\includegraphics[width=1\linewidth]{figures/fig_6.0-day.jpg}
\vspace{-20pt}
\caption{6.0-day forecast results of different models.}
\label{fig_6.0-day}
\end{figure*}

\begin{figure*}[h]
\centering
\includegraphics[width=1\linewidth]{figures/fig_6.5-day.jpg}
\vspace{-20pt}
\caption{6.5-day forecast results of different models.}
\label{fig_6.5-day}
\end{figure*}

\begin{figure*}[h]
\centering
\includegraphics[width=1\linewidth]{figures/fig_7.0-day.jpg}
\vspace{-20pt}
\caption{7.0-day forecast results of different models.}
\label{fig_7.0-day}
\end{figure*}

\begin{figure*}[h]
\centering
\includegraphics[width=1\linewidth]{figures/fig_7.5-day.jpg}
\vspace{-20pt}
\caption{7.5-day forecast results of different models.}
\label{fig_7.5-day}
\end{figure*}

\begin{figure*}[h]
\centering
\includegraphics[width=1\linewidth]{figures/fig_8.0-day.jpg}
\vspace{-20pt}
\caption{8.0-day forecast results of different models.}
\label{fig_8.0-day}
\end{figure*}

\begin{figure*}[h]
\centering
\includegraphics[width=1\linewidth]{figures/fig_8.5-day.jpg}
\vspace{-20pt}
\caption{8.5-day forecast results of different models.}
\label{fig_8.5-day}
\end{figure*}

\begin{figure*}[h]
\centering
\includegraphics[width=1\linewidth]{figures/fig_9.0-day.jpg}
\vspace{-20pt}
\caption{9.0-day forecast results of different models.}
\label{fig_9.0-day}
\end{figure*}

\begin{figure*}[h]
\centering
\includegraphics[width=1\linewidth]{figures/fig_9.5-day.jpg}
\vspace{-20pt}
\caption{9.5-day forecast results of different models.}
\label{fig_9.5-day}
\end{figure*}

\begin{figure*}[h]
\centering
\includegraphics[width=1\linewidth]{figures/fig_10.0-day.jpg}
\vspace{-20pt}
\caption{10.0-day forecast results of different models.}
\label{fig_10.0-day}
\end{figure*}

%%%%%%%%%%%%%%%%%%%%%%%%%%%%%%%%%%%%%%%%%%%%%%%%%%%%%%%%%%%%%%%%%%%%%%%%%%%%%%%
%%%%%%%%%%%%%%%%%%%%%%%%%%%%%%%%%%%%%%%%%%%%%%%%%%%%%%%%%%%%%%%%%%%%%%%%%%%%%%%


\end{document}


% This document was modified from the file originally made available by
% Pat Langley and Andrea Danyluk for ICML-2K. This version was created
% by Iain Murray in 2018, and modified by Alexandre Bouchard in
% 2019 and 2021 and by Csaba Szepesvari, Gang Niu and Sivan Sabato in 2022.
% Modified again in 2023 and 2024 by Sivan Sabato and Jonathan Scarlett.
% Previous contributors include Dan Roy, Lise Getoor and Tobias
% Scheffer, which was slightly modified from the 2010 version by
% Thorsten Joachims & Johannes Fuernkranz, slightly modified from the
% 2009 version by Kiri Wagstaff and Sam Roweis's 2008 version, which is
% slightly modified from Prasad Tadepalli's 2007 version which is a
% lightly changed version of the previous year's version by Andrew
% Moore, which was in turn edited from those of Kristian Kersting and
% Codrina Lauth. Alex Smola contributed to the algorithmic style files.
