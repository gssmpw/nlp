\newpage
\appendix

\renewcommand{\figurename}{Supplementary Figure}
\renewcommand{\tablename}{Supplementary Table}
\setcounter{figure}{0}
\setcounter{table}{0}

    



\section{Details of datasets}
This section provides additional details about the dataset used to evaluate the downstream tasks. \Cref{tab:disease_definition} lists the ICD-10 codes and medications used to identify the diagnoses for each disease. \Cref{tab:characteristic} presents the distribution of patient characteristics for each disease. \Cref{fig:nyu_langone_prevalence,fig:nyu_longisland_prevalence} illustrates the prevalence of each disease in the downstream tasks for the NYU Langone and NYU Long Island datasets, highlighting the imbalances present in these tasks.

\begin{table}[!htpb]
    \centering
    \caption{The definition of diseases in EHR by diagnosis codes and medications.}
    \begin{tabular}{lr}
    \toprule
         Disease &  Definition in EHR \\
    \midrule
       IPH  &  I61.0, I61.1, I61.2, I61.3, I61.4, I61.8, I61.9 \\
       IVH  &  I61.5, P52.1, P52.2, P52.3  \\
       ICH  &  IPH + IVH + I61.6, I62.9, P10.9, P52.4, P52.9 \\
       SDH  &  S06.5, I62.0 \\
       EDH  &  S06.4, I62.1 \\
       SAH  &  I60.*, S06.6, P52.5, P10.3  \\
       Tumor  &  C71.*, C79.3, D33.0, D33.1, D33.2, D33.3, D33.7, D33.9  \\
       Hydrocephalus  &  G91.* \\
       Edema  &  G93.1, G93.5, G93.6, G93.82, S06.1 \\
       \multirow{2}{*}{ADRD}  &  G23.1, G30.*, G31.01, G31.09, G31.83, G31.85, G31.9, F01.*, F02.*, F03.*, G31.84, G31.1, \\ 
       & \textbf{Medication:} DONEPEZIL, RIVASTIGMINE, GALANTAMINE, MEMANTINE, TACRINE \\ 
    \bottomrule
    \end{tabular}
    \label{tab:disease_definition}
\end{table}

\begin{table}[!htbp]
\centering
\caption{Demographic characteristics of patients associated with scans from the NYU Langone dataset, matched with electronic health records (EHR) and utilized in downstream tasks.}
\label{tab:characteristic}

 The characteristic table on NYU Langone dataset matched with EHR.
\begin{tabular}{ll|rr|r}
\toprule
                       \textbf{Cohort} &  &           \textbf{Male (\%)} &          \textbf{Female (\%)} &     \textbf{Age (std)} \\
\midrule
 --- & All (n=270,205) & 128,113 (47.41\%) & 142,092 (52.59\%) & 63.64 (19.68) \\
\midrule
       Tumor & Neg (n=260,704) & 123,338 (47.31\%) & 137,366 (52.69\%) & 63.85 (19.72) \\
             & Pos (n=9,501) &   4,775 (50.26\%) &   4,726 (49.74\%) & 57.80 (17.67) \\
\midrule
HCP & Neg (n=253,000) & 118,881 (46.99\%) & 134,119 (53.01\%) & 63.67 (19.72) \\
              & Pos (n=17,205) &   9,232 (53.66\%) &   7,973 (46.34\%) & 63.18 (19.11) \\
\midrule
Edema & Neg (n=242,576) & 112,987 (46.58\%) & 129,589 (53.42\%) & 63.96 (19.84) \\
      & Pos (n=27,629) &  15,126 (54.75\%) &  12,503 (45.25\%) & 60.81 (17.97) \\
\midrule
ADRD  & Neg (n=232,667) & 111,159 (47.78\%) & 121,508 (52.22\%) & 61.31 (19.55) \\
      & Pos (n=37,538) &  16,954 (45.16\%) &  20,584 (54.84\%) & 78.09 (13.30) \\
\midrule
          IPH & Neg (n=251,308) & 117,692 (46.83\%) & 133,616 (53.17\%) & 63.58 (19.82) \\
              & Pos (n=18,897) &  10,421 (55.15\%) &   8,476 (44.85\%) & 64.39 (17.69) \\
\midrule
          IVH & Neg (n=258,232) & 121,686 (47.12\%) & 136,546 (52.88\%) & 63.65 (19.79) \\
              & Pos (n=11,973) &   6,427 (53.68\%) &   5,546 (46.32\%) & 63.45 (17.19) \\
\midrule
          SDH & Neg (n=248,468) & 114,869 (46.23\%) & 133,599 (53.77\%) & 63.44 (19.78) \\
              & Pos (n=21,737) &  13,244 (60.93\%) &   8,493 (39.07\%) & 65.95 (18.33) \\
\midrule
          EDH & Neg (n=265,431) & 125,113 (47.14\%) & 140,318 (52.86\%) & 63.77 (19.64) \\
              & Pos (n=4,774) &   3,000 (62.84\%) &   1,774 (37.16\%) & 56.53 (20.75) \\
\midrule
          SAH & Neg (n=251,594) & 118,424 (47.07\%) & 133,170 (52.93\%) & 63.79 (19.76) \\
              & Pos (n=18,611) &   9,689 (52.06\%) &   8,922 (47.94\%) & 61.59 (18.49) \\
\midrule
          ICH & Neg (n=229,851) & 105,498 (45.90\%) & 124,353 (54.10\%) & 63.41 (19.93) \\
              & Pos (n=40,354) &  22,615 (56.04\%) &  17,739 (43.96\%) & 64.93 (18.14) \\
\bottomrule
\end{tabular}
\end{table}


\begin{table}[!h]
    \centering
    \caption*{\textbf{Supplementary \Cref{tab:characteristic} Continued.} Demographic characteristics of patients associated with scans from the NYU Long Island dataset, matched with electronic health records (EHR) and utilized in downstream tasks.}
\begin{tabular}{ll|rr|r}
\toprule
                       \textbf{Cohort} &  &           \textbf{Male (\%)} &          \textbf{Female (\%)} &     \textbf{Age (std)} \\
\midrule
--- & All (n=22,158) & 9,580 (43.23\%) & 12,578 (56.77\%) & 68.33 (18.14) \\
\midrule
Tumor & Neg (n=21,578) & 9,275 (42.98\%) & 12,303 (57.02\%) & 68.59 (18.08) \\
      & Pos (n=580) &   305 (52.59\%) &    275 (47.41\%) & 58.78 (17.79) \\
\midrule
HCP   & Neg (n=20,653) & 8,718 (42.21\%) & 11,935 (57.79\%) & 69.05 (17.90) \\
      & Pos (n=1,505) &   862 (57.28\%) &    643 (42.72\%) & 58.52 (18.48) \\
\midrule
Edema & Neg (n=19,402) & 8,068 (41.58\%) & 11,334 (58.42\%) & 68.89 (18.27) \\
      & Pos (n=2,756) & 1,512 (54.86\%) &  1,244 (45.14\%) & 64.36 (16.66) \\
\midrule
ADRD  & Neg (n=19,537) & 8,391 (42.95\%) & 11,146 (57.05\%) & 66.78 (18.28) \\
      & Pos (n=2,621) & 1,189 (45.36\%) &  1,432 (54.64\%) & 79.90 (11.77) \\
\midrule
IPH   & Neg (n=19,357) & 7,974 (41.19\%) & 11,383 (58.81\%) & 68.97 (18.27) \\
      & Pos (n=2,801) & 1,606 (57.34\%) &  1,195 (42.66\%) & 63.89 (16.48) \\
\midrule
IVH   & Neg (n=19,636) & 8,164 (41.58\%) & 11,472 (58.42\%) & 68.96 (18.22) \\
      & Pos (n=2,522) & 1,416 (56.15\%) &  1,106 (43.85\%) & 63.43 (16.66) \\
\midrule
SDH   & Neg (n=20,885) & 8,870 (42.47\%) & 12,015 (57.53\%) & 68.33 (18.21) \\
      & Pos (n=1,273) &   710 (55.77\%) &    563 (44.23\%) & 68.37 (16.83) \\
\midrule
EDH   & Neg (n=21,912) & 9,443 (43.10\%) & 12,469 (56.90\%) & 68.33 (18.16) \\
      & Pos (n=246) &   137 (55.69\%) &    109 (44.31\%) & 68.19 (15.59) \\
\midrule
SAH   & Neg (n=20,652) & 8,824 (42.73\%) & 11,828 (57.27\%) & 68.68 (18.12) \\
      & Pos (n=1,506) &   756 (50.20\%) &    750 (49.80\%) & 63.58 (17.65) \\
\midrule
ICH   & Neg (n=18,388) & 7,456 (40.55\%) & 10,932 (59.45\%) & 68.92 (18.35) \\
      & Pos (n=3,770) & 2,124 (56.34\%) &  1,646 (43.66\%) & 65.48 (16.77) \\
\bottomrule
\end{tabular}
\end{table}

\begin{figure}[!ht]
    \centering
    \includegraphics[width=0.8\textwidth]{images/NYU_Langone_prevalence.pdf}
    \caption{Disease prevalence of NYU Langone }
    \label{fig:nyu_langone_prevalence}
\end{figure}

\begin{figure}[!h]
    \centering
    \includegraphics[width=0.8\textwidth]{images/NYU_Longisland_prevalence.pdf}
    \caption{Disease prevalence of NYU Longisland dataset}
    \label{fig:nyu_longisland_prevalence}
\end{figure}



\section{Data augmentation details}
\label{sec:dataaug_details}
We applied Random Flipping across all three dimensions, Random Shift Intensity with offset $0.1$ for both pre-training and fine-tuning. For DINO training. random Gaussian Smoothing with sigma=$(0.5-1.0)$ is applied across all dimensions, Random Gamma Adjust is applied with gamma=$(0.2-1.0)$.


\section{Additional experiment results}
This section provides additional experimental results with more details.
Supplementary \Cref{fig:channels-ablation,fig:patches-ablation} compares the performance of the foundation model using different numbers of channels and patch sizes, demonstrating that the architecture design of our foundation model is optimal. 

Supplementary \Cref{fig:radar-comparison-merlin} compares our foundation model with a foundation CT model from previous studies, Merlin\cite{blankemeier2024merlinvisionlanguagefoundation}, which was trained on abdomen CT scans with corresponding radiology report pairs. Our model demonstrates superior performance on head CT scans.

Supplementary \Cref{fig:probing-comparison-gemini} compares our foundation model with Google CT Foundation model~\cite{yang2024advancingmultimodalmedicalcapabilities}, which was trained on large scale and diverse CT scans from different anatomy with corresponding radiology report pairs. Our model consistently shows improved performance across the board even though our model was pre-trained with less samples.

Supplementary \Cref{fig:probing_comparison} compares the performance on downstream tasks with various supervised tuning methods applied to foundation models pretrained with the MAE and DINO frameworks. Per-pathology comparisons are shown in Supplementary \Cref{fig:probing-comparison-perpath,fig:probing-comparison-perpath-dino}. Meanwhile, supplementary \Cref{fig:boxplot_scaling} complements \Cref{fig:scaling_law}, illustrating the per-pathology performances of foundation models pretrained with different scales of training data.

Supplementary \Cref{fig:batch_effect,fig:thickness-ablation} studies the impact of batch effect caused by different CT scan protocols of slice thickness and machine manufacturer. Detailed per-pathology performances are shown in Supplementary \Cref{fig:slice_thickness_per_pathology,fig:manufacturer_per_pathology}.

\begin{figure}[!htpb]
    \centering
    \makebox[\textwidth][l]{%
        \hspace{0.3\textwidth}\textbf{NYU Langone}
    } \\[0.2cm]
    \includegraphics[trim={0 0 0 0},clip,height=0.3\textwidth, width=0.3\textwidth]{figures/abla_chans/AUC_chans_NYU.pdf}
    \includegraphics[trim={0 0 0 0},clip,height=0.3\textwidth, width=0.55\textwidth]{figures/abla_chans/AP_chans_NYU.pdf}\\
    \makebox[\textwidth][l]{
        \hspace{0.34\textwidth}\textbf{RSNA}
    } \\[0.2cm]
    \includegraphics[trim={0 0 0 0},clip,height=0.3\textwidth, width=0.3\textwidth]{figures/abla_chans/AUC_chans_RSNA.pdf}
    \includegraphics[height=0.3\textwidth, width=0.55\textwidth]{figures/abla_chans/AP_chans_RSNA.pdf} 
    \caption{\textbf{Comparison of Different Channels Performance.} This plot compares the performance of models trained using different numbers of channels (channels from multiple HU intervals vs. a single HU interval). Across two datasets, models using three channels from different HU intervals consistently outperform those using a single channel with a fixed HU interval. All models were pre-trained on $100\%$ of the pretraining data with MAE.}
    \label{fig:channels-ablation}
\end{figure}


\begin{figure}[!htpb]
    \centering
    \makebox[\textwidth][l]{%
        \hspace{0.3\textwidth}\textbf{NYU Langone}
    } \\[0.2cm]
    \includegraphics[trim={0 0 0 0},clip,height=0.3\textwidth, width=0.3\textwidth]{figures/abla_patches/AUC_patches_NYU.pdf}
    \includegraphics[trim={0 0 0 0},clip,height=0.3\textwidth, width=0.55\textwidth]{figures/abla_patches/AP_patches_NYU.pdf}\\
    \makebox[\textwidth][l]{
        \hspace{0.34\textwidth}\textbf{RSNA}
    } \\[0.2cm]
    \includegraphics[trim={0 0 0 0},clip,height=0.3\textwidth, width=0.3\textwidth]{figures/abla_patches/AUC_patches_RSNA.pdf}
    \includegraphics[height=0.3\textwidth, width=0.55\textwidth]{figures/abla_patches/AP_patches_RSNA.pdf} 
    \caption{\textbf{Comparison of Different Patches Performance.} This plot compares the performance of models trained with different patch sizes (12 vs. 16). The results demonstrate that smaller patch sizes consistently achieve better performance. All models were pre-trained on $100\%$ of the pretraining data with MAE.}
    \label{fig:patches-ablation}
\end{figure}


\begin{figure*}
    \centering
    \makebox[\textwidth][l]{%
        \hspace{0.06\textwidth}
        \textbf{NYU Langone} \hspace{0.06\textwidth} \textbf{NYU Long Island} \hspace{0.11\textwidth} \textbf{RSNA} \hspace{0.18\textwidth} \textbf{CQ500}
    } \\[0.2cm]
    \includegraphics[trim={0 0 0 0},clip,height=0.21\textwidth, width=0.21\textwidth]{figures/abla_radarplot_merlin/AUC_NYU.pdf}
    \includegraphics[trim={0 0 0 0},clip,height=0.21\textwidth, width=0.21\textwidth]{figures/abla_radarplot_merlin/AUC_Longisland.pdf}
    \includegraphics[trim={0 0 0 0},clip,height=0.21\textwidth, width=0.21\textwidth]{figures/abla_radarplot_merlin/AUC_RSNA.pdf}
    \includegraphics[trim={0 0 0 0},clip,height=0.21\textwidth, width=0.35\textwidth]{figures/abla_radarplot_merlin/AUC_CQ500.pdf}\\[0.2cm]
    \includegraphics[height=0.21\textwidth, width=0.21\textwidth]{figures/abla_radarplot_merlin/AP_NYU.pdf} 
    \includegraphics[height=0.21\textwidth, width=0.21\textwidth]{figures/abla_radarplot_merlin/AP_Longisland.pdf} 
    \includegraphics[height=0.21\textwidth, width=0.21\textwidth]{figures/abla_radarplot_merlin/AP_RSNA.pdf}
    \includegraphics[height=0.21\textwidth, width=0.35\textwidth]{figures/abla_radarplot_merlin/AP_CQ500.pdf}
    \caption{\textbf{Comparison to previous 3D Foundation Model.} This plot compares the performance of our model with Merlin~\cite{blankemeier2024merlinvisionlanguagefoundation} and models trained from scratch across four datasets for our model and ResNet50-3D. Our DINO trained model is used in this comparison. Our model demonstrates consistently superior performance across majority of diseases, with the exception of epidural hemorrhage (EDH) in the CQ500 dataset.}
    \label{fig:radar-comparison-merlin}
\end{figure*}



\begin{figure*}
    \centering
    \makebox[\textwidth][l]{%
        \hspace{0.10\textwidth}
        \textbf{NYU Langone} \hspace{0.08\textwidth} \textbf{NYU Long Island} \hspace{0.1\textwidth} \textbf{RSNA} \hspace{0.15\textwidth} \textbf{CQ500}
    } \\[0.2cm]
    \includegraphics[trim={0 0 0 0},clip, width=0.22\textwidth]{figures/abla_probing/AUC_NYU.pdf}
    \includegraphics[trim={0 0 0 0},clip, width=0.22\textwidth]{figures/abla_probing/AUC_Longisland.pdf}
    \includegraphics[trim={0 0 0 0},clip, width=0.22\textwidth]{figures/abla_probing/AUC_RSNA.pdf}
    \includegraphics[trim={0 0 0 0},clip, width=0.28\textwidth]{figures/abla_probing/AUC_CQ500.pdf}
    \\[0.2cm]
    \includegraphics[width=0.22\textwidth]{figures/abla_probing/AP_NYU.pdf} 
    \includegraphics[width=0.22\textwidth]{figures/abla_probing/AP_Longisland.pdf} 
    \includegraphics[width=0.22\textwidth]{figures/abla_probing/AP_RSNA.pdf}
    \includegraphics[width=0.28\textwidth]{figures/abla_probing/AP_CQ500.pdf}
    \caption{\textbf{Comparison of Different Downstream Training Methods.} This plot illustrates the downstream performance of models evaluated using fine-tuning and various probing methods across four datasets. In most cases, the DINO pre-trained model outperforms the MAE pre-trained model. All models were pre-trained on $100\%$ of the available pretraining data.}
    \label{fig:probing_comparison}
\end{figure*}


\begin{figure}
\centering
\makebox[\textwidth][l]{%
    \hspace{0.39\textwidth}\textbf{RSNA}
} \\[0.2cm]
\includegraphics[trim={0 0 0mm 0},clip,height=0.27\textwidth]{figures/abla_gemini/AUC_RSNA_Gemini.pdf}
\includegraphics[trim={0 0 5mm 0},clip,height=0.27\textwidth]{figures/abla_gemini/AP_RSNA_Gemini.pdf}

\makebox[\textwidth][l]{%
    \hspace{0.38\textwidth}\textbf{CQ500}
} \\[0.2cm]
\includegraphics[trim={0 0 10mm 0},clip,height=0.345\textwidth]{figures/abla_gemini/AUC_CQ500_Gemini.pdf}
\includegraphics[trim={0 0 5mm 0},clip,height=0.345\textwidth]{figures/abla_gemini/AP_CQ500_Gemini.pdf}

\caption{\textbf{Performance comparison of linear probing for Our Model vs. Google CT Foundation model} This plot compares our model performance vs. Google CT Foundation model\cite{yang2024advancing} and Merlin \cite{blankemeier2024merlinvisionlanguagefoundation} across all diseases on RSNA and CQ500. Since Google CT Foundation moudel requires uploading data to Google Cloud (not allowed on our private data) for requesting model embeddings with model weights inaccessible, only public dataset comparison is provided in this study. Similar to other evaluations, we observed that our model outperforms Google CT Foundation model across the board with the only exception on Midline Shift for Google CT Foundation model and EDH for Merlin.}
\label{fig:probing-comparison-gemini}
\end{figure}



\begin{figure}
    \centering
    \makebox[\textwidth][l]{%
        \hspace{0.35\textwidth}\textbf{NYU Langone}
    } \\[0.2cm]
    \includegraphics[trim={0 0 120mm 0},clip,height=0.255\textwidth]{figures/abla_probing_perpath/DINO_AUC_NYU_Langone.pdf}
    \includegraphics[trim={0 0 0 0},clip,height=0.255\textwidth]{figures/abla_probing_perpath/DINO_AP_NYU_Langone.pdf} \\
    \makebox[\textwidth][l]{
        \hspace{0.35\textwidth}\textbf{NYU Long Island}
    } \\[0.2cm]
    \includegraphics[trim={0 0 120mm 0},clip,height=0.255\textwidth]{figures/abla_probing_perpath/DINO_AUC_NYU_Long_Island.pdf}
    \includegraphics[trim={0 0 0 0},clip,height=0.255\textwidth]{figures/abla_probing_perpath/DINO_AP_NYU_Long_Island.pdf} 
    \makebox[\textwidth][l]{
        \hspace{0.4\textwidth}\textbf{RSNA}
    } \\[0.2cm]
    \includegraphics[trim={0 0 120mm 0},clip,height=0.24\textwidth]{figures/abla_probing_perpath/DINO_AUC_RSNA.pdf}
    \hspace{5mm}
    \includegraphics[trim={0 0 0 0},clip,height=0.24\textwidth]{figures/abla_probing_perpath/DINO_AP_RSNA.pdf} 
    \makebox[\textwidth][l]{
        \hspace{0.4\textwidth}\textbf{CQ500}
    } \\[0.2cm]
    \includegraphics[trim={0 0 120mm 0},clip,height=0.30\textwidth]{figures/abla_probing_perpath/DINO_AUC_CQ500.pdf} \hspace{5mm}
    \includegraphics[trim={0 0 0 0},clip,height=0.30\textwidth]{figures/abla_probing_perpath/DINO_AP_CQ500.pdf} 
    \caption{\textbf{Performance comparison of supervised finetuning methods per pathology on the foundation model trained with DINO.} This plot breaks down the average performance across all diseases shown in Supplementary \Cref{fig:probing_comparison}. The results show that fine-tuning the entire network achieves the best performance in most scenarios. However, linear probing closely approaches finetuning performance for many diseases especially on small or imbalanced dataset, underscoring the capability of our pre-trained models to generate representations that adapt effectively to diverse disease detection tasks.}
    \label{fig:probing-comparison-perpath-dino}
\end{figure}

\begin{figure}
    \centering
    \makebox[\textwidth][l]{%
        \hspace{0.35\textwidth}\textbf{NYU Langone}
    } \\[0.2cm]
    \includegraphics[trim={0 0 0 0},clip,height=0.24\textwidth, width=0.3\textwidth]{figures/abla_probing_perpath/AUC_NYU.pdf}
    \includegraphics[trim={0 0 0 0},clip,height=0.24\textwidth, width=0.45\textwidth]{figures/abla_probing_perpath/AP_NYU.pdf}\\
    \makebox[\textwidth][l]{
        \hspace{0.35\textwidth}\textbf{NYU Long Island}
    } \\[0.2cm]
    \includegraphics[trim={0 0 0 0},clip,height=0.24\textwidth, width=0.3\textwidth]{figures/abla_probing_perpath/AUC_Longisland.pdf}
    \includegraphics[trim={0 0 0 0},clip,height=0.24\textwidth, width=0.45\textwidth]{figures/abla_probing_perpath/AP_Longisland.pdf} 
    \makebox[\textwidth][l]{
        \hspace{0.4\textwidth}\textbf{RSNA}
    } \\[0.2cm]
    \includegraphics[trim={0 0 0 0},clip,height=0.24\textwidth, width=0.3\textwidth]{figures/abla_probing_perpath/AUC_RSNA.pdf}
    \includegraphics[height=0.24\textwidth, width=0.45\textwidth]{figures/abla_probing_perpath/AP_RSNA.pdf} 
    \makebox[\textwidth][l]{
        \hspace{0.4\textwidth}\textbf{CQ500}
    } \\[0.2cm]
    \includegraphics[trim={0 0 120mm 0},clip,height=0.24\textwidth]{figures/abla_probing_perpath/AUC_CQ500.pdf}
    \includegraphics[trim={0 0 0 0},clip,height=0.24\textwidth]{figures/abla_probing_perpath/AP_CQ500.pdf} 
    \caption{\textbf{Performance comparison of supervised finetuning methods per pathology on the foundation model trained with MAE.} The results reveal that attentive probing is significantly more effective than linear probing, consistent with findings from~\cite{Chen2024}. Furthermore, for many diseases, the performance of probing models approaches that of fine-tuning, demonstrating that our pre-trained models produce adaptable representations capable of detecting diverse diseases.}
    \label{fig:probing-comparison-perpath}
\end{figure}









\begin{figure}
    \centering
    \textbf{NYU Langone} \\
    \includegraphics[trim={0 0 0 0},clip,height=0.24\textwidth, width=0.38\textwidth]{figures/abla_perpath_perf/AUC_NYU.pdf}
    \includegraphics[height=0.24\textwidth, width=0.45\textwidth]{figures/abla_perpath_perf/AP_NYU.pdf} \\
    \textbf{NYU Long Island} \\
    \includegraphics[trim={0 0 0 0},clip,height=0.24\textwidth, width=0.38\textwidth]{figures/abla_perpath_perf/AUC_Longisland.pdf}
    \includegraphics[height=0.24\textwidth, width=0.45\textwidth]{figures/abla_perpath_perf/AP_Longisland.pdf} \\
    \textbf{RSNA} \\
    \includegraphics[trim={0 0 0 0},clip,height=0.24\textwidth, width=0.38\textwidth]{figures/abla_perpath_perf/AUC_RSNA.pdf}
    \includegraphics[height=0.24\textwidth, width=0.45\textwidth]{figures/abla_perpath_perf/AP_RSNA.pdf}\\
    \textbf{CQ500} \\
    \includegraphics[trim={0 0 0 0},clip,height=0.24\textwidth, width=0.38\textwidth]{figures/abla_perpath_perf/AUC_CQ500.pdf}
    \includegraphics[height=0.24\textwidth, width=0.45\textwidth]{figures/abla_perpath_perf/AP_CQ500.pdf}
    \caption{\textbf{Performance for Different Percentage of Pre-training Samples (Per-Pathology).} This plot illustrates label efficiency for individual pathologies using Tukey plots, alongside the average performance across all diseases shown in \Cref{fig:scaling_law}. The results indicate that the majority of pathologies show improved downstream performance as the amount of pretraining data increases.}
    \label{fig:boxplot_scaling}
\end{figure}


\newpage

\section{Time complexity increase with reduced patch size}
\label{apd:self_attention_rate}
Assume we have 3D image input of shape $H\times W\times D$, patch size $P$ and reducing factor $s$. By time complexity of self-attention $O(n^2 d)$ for sequence length $n$ and embedding dimension $d$, the new time complexity after reducing patch size can be derived as
\begin{align*}
    O(n^2d)&=O((\frac{H\times W\times D}{(\frac{P}{s})^3})^2d) \\
           &=O((\frac{H\times W\times D}{P^3})^2 s^6d)  \\
           &=O(s^6)O(n_{ori}^2d)
\end{align*}
where $n_{ori}=\frac{H\times W\times D}{P^3}$ is the original sequence length before reducing patch size.



















\newpage
\begin{figure}[ht]
    \centering
    \includegraphics[width=\textwidth]{images/tsne_embedding_visualization_per_pathology.png}
    \caption{The 2D projection with t-SNE of CT volume representation extracted from the foundation model. Interestingly, certain subgroups still exhibited slightly better AUCs. For instance, scans with slice thicknesses between 1–4 mm (represented by light green points in the upper panel of \Cref{fig:batch_effect}) align with a specialized protocol for CT angiography (CTA), which uses contrast dye to improve diagnosis on particular diseases.}
    \label{fig:batch_effect}
\end{figure}


\begin{figure*}[ht]
    \centering
    \begin{subfigure}[b]{0.33\textwidth}
        \centering
        \includegraphics[width=\textwidth]{images/AUROC_vs_Slice_thickness_binned.png}
        \caption{AUROC Performance}
    \end{subfigure}
    \hfill
    \begin{subfigure}[b]{0.33\textwidth}
        \centering
        \includegraphics[width=\textwidth]{images/AUPRC_vs_Slice_thickness_binned.png}
        \caption{AUPRC Performance}
    \end{subfigure}
    \hfill
    \begin{subfigure}[b]{0.33\textwidth}
        \centering
        \includegraphics[width=\textwidth]{images/Histogram_of_slice_thickness_distribution_across_scans.png}
        \caption{Histogram of slice thickness distribution}
    \end{subfigure}
    \caption{The downstream task performances on various ranges of slice thickness.}
    \label{fig:thickness-ablation}
\end{figure*}


\begin{figure*}[ht]
    \centering
    \begin{subfigure}[b]{\textwidth}
        \centering
        \includegraphics[width=\textwidth]{images/AUROC_vs_slice_thickness_for_each_disease_category.png}
        \caption{AUROC Performance}
    \end{subfigure}
    \hfill
    \begin{subfigure}[b]{\textwidth}
        \centering
        \includegraphics[width=\textwidth]{images/AUPRC_vs_slice_thickness_for_eachdisease_category.png}
        \caption{AUPRC Performance}
    \end{subfigure}
    \hfill
    \begin{subfigure}[b]{\textwidth}
        \centering
        \includegraphics[width=\textwidth]{images/Ratio_of_positive_labels_vs_slice_thickness_for_each_disease_category.png}
        \caption{Ratio of Positive Labels}
    \end{subfigure}
    \caption{Performance for Each Slice Thickness Bin (Per Pathology).}
    \label{fig:slice_thickness_per_pathology}
\end{figure*}


\begin{figure*}[ht]
    \centering
    \begin{subfigure}[b]{0.3\textwidth}
        \centering
        \includegraphics[width=\textwidth]{images/AUROC_by_Disease_and_Manufacturer.png}
        \caption{AUROC Performance}
    \end{subfigure}
    \hfill
    \begin{subfigure}[b]{0.3\textwidth}
        \centering
        \includegraphics[width=\textwidth]{images/AUPRC_by_Disease_and_Manufacturer.png}
        \caption{AUPRC Performance}
    \end{subfigure}
    \hfill
    \begin{subfigure}[b]{0.39\textwidth}
        \centering
        \includegraphics[width=\textwidth]{images/Positive_Label_Ratio_by_Disease_and_Manufacturer.png}
        \caption{Distribution of Scans from Each Manufacturer}
    \end{subfigure}
    \caption{Performance for Each Manufacturer (Per Pathology).}
    \label{fig:manufacturer_per_pathology}
\end{figure*}





