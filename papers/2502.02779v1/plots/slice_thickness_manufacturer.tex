\begin{figure}[ht]
    \centering
    \includegraphics[width=\textwidth]{images/tsne_embedding_visualization_per_pathology.png}
    \caption{The 2D projection with t-SNE of CT volume representation extracted from the foundation model. Interestingly, certain subgroups still exhibited slightly better AUCs. For instance, scans with slice thicknesses between 1–4 mm (represented by light green points in the upper panel of \Cref{fig:batch_effect}) align with a specialized protocol for CT angiography (CTA), which uses contrast dye to improve diagnosis on particular diseases.}
    \label{fig:batch_effect}
\end{figure}


\begin{figure*}[ht]
    \centering
    \begin{subfigure}[b]{0.33\textwidth}
        \centering
        \includegraphics[width=\textwidth]{images/AUROC_vs_Slice_thickness_binned.png}
        \caption{AUROC Performance}
    \end{subfigure}
    \hfill
    \begin{subfigure}[b]{0.33\textwidth}
        \centering
        \includegraphics[width=\textwidth]{images/AUPRC_vs_Slice_thickness_binned.png}
        \caption{AUPRC Performance}
    \end{subfigure}
    \hfill
    \begin{subfigure}[b]{0.33\textwidth}
        \centering
        \includegraphics[width=\textwidth]{images/Histogram_of_slice_thickness_distribution_across_scans.png}
        \caption{Histogram of slice thickness distribution}
    \end{subfigure}
    \caption{The downstream task performances on various ranges of slice thickness.}
    \label{fig:thickness-ablation}
\end{figure*}


\begin{figure*}[ht]
    \centering
    \begin{subfigure}[b]{\textwidth}
        \centering
        \includegraphics[width=\textwidth]{images/AUROC_vs_slice_thickness_for_each_disease_category.png}
        \caption{AUROC Performance}
    \end{subfigure}
    \hfill
    \begin{subfigure}[b]{\textwidth}
        \centering
        \includegraphics[width=\textwidth]{images/AUPRC_vs_slice_thickness_for_eachdisease_category.png}
        \caption{AUPRC Performance}
    \end{subfigure}
    \hfill
    \begin{subfigure}[b]{\textwidth}
        \centering
        \includegraphics[width=\textwidth]{images/Ratio_of_positive_labels_vs_slice_thickness_for_each_disease_category.png}
        \caption{Ratio of Positive Labels}
    \end{subfigure}
    \caption{Performance for Each Slice Thickness Bin (Per Pathology).}
    \label{fig:slice_thickness_per_pathology}
\end{figure*}


\begin{figure*}[ht]
    \centering
    \begin{subfigure}[b]{0.3\textwidth}
        \centering
        \includegraphics[width=\textwidth]{images/AUROC_by_Disease_and_Manufacturer.png}
        \caption{AUROC Performance}
    \end{subfigure}
    \hfill
    \begin{subfigure}[b]{0.3\textwidth}
        \centering
        \includegraphics[width=\textwidth]{images/AUPRC_by_Disease_and_Manufacturer.png}
        \caption{AUPRC Performance}
    \end{subfigure}
    \hfill
    \begin{subfigure}[b]{0.39\textwidth}
        \centering
        \includegraphics[width=\textwidth]{images/Positive_Label_Ratio_by_Disease_and_Manufacturer.png}
        \caption{Distribution of Scans from Each Manufacturer}
    \end{subfigure}
    \caption{Performance for Each Manufacturer (Per Pathology).}
    \label{fig:manufacturer_per_pathology}
\end{figure*}




