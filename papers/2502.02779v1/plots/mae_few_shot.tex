\begin{figure}[htbp]
    \centering
    NYU Langone \\
    \includegraphics[trim={0 0 140pt 0},clip,height=0.3\textwidth]{figures/fig2/fewshot_NYU_AUC.pdf}
    \includegraphics[height=0.3\textwidth]{figures/fig2/fewshot_NYU_AP.pdf} \\
    NYU Long Island \\
    \includegraphics[trim={0 0 140pt 0},clip,height=0.3\textwidth]{figures/fig2/fewshot_NYU_Long_Island_AUC.pdf}
    \includegraphics[height=0.3\textwidth]{figures/fig2/fewshot_NYU_Long_Island_AP.pdf} \\
    RSNA \\
    \includegraphics[trim={0 0 125pt 0},clip,height=0.295\textwidth]{figures/fig2/fewshot_RSNA_AUC.pdf}
    \includegraphics[height=0.295\textwidth]{figures/fig2/fewshot_RSNA_AP.pdf}
    \caption{\textbf{Few-shot performance of the foundation model.}  The plots display the per-pathology AUC and average precision (AP) of the disease detection model under a few-shot learning setting, evaluated with varying numbers of training samples from the NYU Langone, NYU Long Island, and RSNA datasets. CQ500 is excluded since its small dataset size gives no enough positive samples for many diseases. Few-shot learning performance is compared to supervised finetuning with all training data (denoted by stars), demonstrating the strong generalization ability of the foundation model with limited training data. The confidence intervals are computed by 5 repeated experiments on resampling the training data and retraining the model.}
    \label{fig:fewshot}
\end{figure}
