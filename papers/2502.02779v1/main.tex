\documentclass[fleqn,10pt]{wlscirep}
\usepackage[utf8]{inputenc}
\usepackage[T1]{fontenc}
\usepackage{multirow}
\usepackage{adjustbox}
\usepackage{subcaption}
\usepackage{pifont}
\usepackage{hyperref}
\usepackage{float}
\usepackage{cleveref}
\usepackage{nameref}
\usepackage{multirow}
\usepackage{xcolor}
\usepackage{longtable}

\pdfobjcompresslevel=0
\newcommand{\cmark}{\ding{51}}%
\newcommand{\xmark}{\ding{55}}%
\crefname{figure}{Fig.}{Figs.}         %
\crefname{appendix}{Supplementary}{Supplementary}

\title{3D Foundation AI Model for Generalizable Disease Detection in Head Computed Tomography}

\author[1†]{Weicheng Zhu}
\author[1†]{Haoxu Huang}
\author[1]{Huanze Tang}
\author[2]{Rushabh Musthyala}
\author[1]{Boyang Yu}
\author[1]{Long Chen}
\author[3]{Emilio Vega}
\author[3,8]{Thomas O'Donnell}
\author[3]{Seena Dehkharghani}
\author[4]{Jennifer A. Frontera}
\author[4,5,6]{Arjun V. Masurkar}
\author[4]{Kara Melmed}
\author[3,7*]{Narges Razavian}
\affil[1]{New York University, Center for Data Science, New York, NY, 10001, USA}
\affil[2]{New York University, Courant Institute of Mathematical Sciences, New York, NY, 10001, USA}
\affil[3]{NYU Grossman School of Medicine, Department of Radiology, New York, NY, 10016, USA}
\affil[4]{NYU Grossman School of Medicine, Department of Neurology, New York, NY, 10016, USA}
\affil[5]{NYU Grossman School of Medicine, Department of Neuroscience and Physiology, New York, NY, 10016, USA}
\affil[6]{NYU Grossman School of Medicine, Neuroscience Institute, New York, NY, 10016, USA}
\affil[7]{NYU Grossman School of Medicine, Department of Population Health, New York, NY, 10016, USA}
\affil[8]{Siemens Healthineers, Malvern, PA, 19355, USA}

\affil[*]{\textbf{Corresponding author:}Narges.Razavian@nyulangone.org}

\affil[$\dagger$]{These authors contributed equally to this work}
 

\begin{abstract}
Head computed tomography (CT) imaging is a widely-used imaging modality with multitudes of medical indications, particularly in assessing pathology of the brain, skull, and cerebrovascular system. It is commonly the first-line imaging in neurologic emergencies given its rapidity of image acquisition, safety, cost, and ubiquity. 
Deep learning models may facilitate detection of a wide range of diseases. However, the scarcity of high-quality labels and annotations, particularly among less common conditions, significantly hinders the development of powerful models. To address this challenge, we introduce \textbf{FM-CT}: a \textbf{F}oundation \textbf{M}odel for Head \textbf{CT} for generalizable disease detection, trained using self-supervised learning. Our approach pre-trains a deep learning model on a large, diverse dataset of 361,663 non-contrast 3D head CT scans without the need for manual annotations, enabling the model to learn robust, generalizable features. To investigate the potential of self-supervised learning in head CT, we employed both discrimination with self-distillation and masked image modeling, and we construct our model in 3D rather than at the slice level (2D) to exploit the structure of head CT scans more comprehensively and efficiently. The pre-training phase is followed by fine-tuning on smaller, annotated downstream datasets, thereby optimizing the model for specific diagnostic tasks, such as detecting hemorrhages, tumors, and other abnormalities. The model's downstream classification performance is evaluated using internal and three external datasets, encompassing both in-distribution (ID) and out-of-distribution (OOD) data. Our results demonstrate that the self-supervised foundation model significantly improves performance on downstream diagnostic tasks compared to models trained from scratch and previous 3D CT foundation models on scarce annotated datasets. Furthermore, the model maintains strong generalization across different datasets, indicating its potential for broad clinical applicability. This work highlights the effectiveness of self-supervised learning in medical imaging and sets a new benchmark for head CT image analysis in 3D, enabling broader use of artificial intelligence for head CT-based diagnosis.
\end{abstract} 
\begin{document}

\flushbottom
\maketitle
\thispagestyle{empty}


\section*{Introduction}
Head computed tomography (CT) is often the first step in diagnosing a wide range of neurological disorders, including head trauma, hemorrhages, hydrocephalus, and malignancies. Head CT scans are faster, more accessible, and generally less expensive than magnetic resonance imaging (MRIs), making them ideal for emergencies like traumatic brain injury (TBI) or suspected stroke. They are also effective in detecting bone fractures, or neurovascular pathologies such as arterial venous malformations. Despite its widespread use, CT lacks the contrast resolution and hence the sensitivity for many disorders dependent upon diagnosis by MRI, thus MRI is the imaging modality of choice for many neurologic diseases. MRI, however, is more costly, risks potential heating or displacement of indwelling implants, and suffers generally slower acquisition times, increasing patient discomfort and risking non-diagnostic examinations due to its greater sensitivity to motion-related artifacts. It is also more expensive than CT and is contraindicated in specific patients. Access to MRI is a major challenge in resource-limited countries. The timely and arduous determination of certain pathologies can delay appropriate medical and surgical treatment for patients.
There is significant potential to harness artificial intelligence (AI) algorithms to enhance the diagnostic and early detection capabilities of head CT, providing critical support in clinical decision-making and improving patient outcomes. Early and accurate diagnosis can potentially lead to more effective treatments, reduce complications, and improve patient survival.


Current research on AI-driven diagnosis using head CT is limited due to both lack of data availability and the complexity of model architectures. Although datasets such as RSNA~\cite{flanders_construction_2020} and CQ500~\cite{CQ500} provide publicly available head CT data, they remain small (RSNA includes approximately 10K samples and CQ500 approximately 1K), and their primary focus is on hemorrhage detection, which restricts broader applicability as a credible path to clinical decision support. Moreover, many existing models are designed with highly task-specific architectures that may not generalize well to diverse clinical applications. These models typically apply 2D convolution neural networks (CNN) to sequentially process 3D volumes slice-by-slice under the supervision of slice-level labels~\cite{wang_deep_2021,CQ500,yun_deep_2023}. %
Slice-level labels are often expensive to acquire, and models trained on 2D slices often struggle to generalize to conditions like neurodegenerative diseases, where slice-level labels are not easily defined. Developing models that can harness the information embedded within the 3D structure of CT images while requiring minimal slice-level labeled data can thus expand the impact of such approaches. To address these challenges, we have developed FM-CT: a Foundation Model for Head CT, and demonstrated robust performance across multiple tasks and datasets, which highlights our model's potential for broad clinical applicability.


Recent advancements in AI \textit{foundation models}, deep learning models pre-trained on extensive datasets in a self-supervised manner, have enabled rapid adaptation and robust performance across a wide range of tasks~\cite{radford2021learningtransferablevisualmodels,zhou2021ibot,oquab2023dinov2, rishi24foundation}. Multiple studies have shown that foundation models trained on large-scale medical data can enhance model performance in various medical imaging tasks, including chest X-rays~\cite{yao2024evaxfoundationmodelgeneral}, histopathology~\cite{wang_pathology_2024,huang_visuallanguage_2023,chen_towards_2024, Vorontsov2024},  retina imaging~\cite{zhou2023foundation}, fMRI~\cite{dong2024brainjepa} and more. Additionally, several generalist vision-language models show promise for multimodal medical applications~\cite{codella2024medimageinsight,yang2024advancingmultimodalmedicalcapabilities,zhang2024generalist}. Although some research has focused on CT scans~\cite{Tang_2022_CVPR,blankemeier2024merlinvisionlanguagefoundation,codella2024medimageinsight}, these studies remain limited to abdominal CTs and cannot generalize to other part of the body. While Google CT Foundation model has explored report generation for head CTs~\cite{yang2024advancingmultimodalmedicalcapabilities}, it uses default video encoders to interpret 3D head CTs and has only been subjectively evaluated on fewer than 100 samples, for which the generated reports were worse than original ones. Given these limitations, developing a dedicated vision foundation model for head CTs is essential to advance AI-driven diagnosis and facilitate early detection of cranial and neurological conditions.

In this work, we introduce FM-CT, a 3D foundation model for head CT scans, developed using self-supervised learning (SSL). While SSL has shown success in natural images~\cite{chen2020simple, he2020momentum, caron2020unsupervised, caron2021emerging, bao2022beit, He2021MaskedAA, zhou2021ibot, oquab2023dinov2, zbontar2021barlow, bardes2022vicreg} and in medical imaging~\cite{Liu_2023_CVPR, zhu2022interpretablepredictionlungsquamous, chen_towards_2024, zhou2023foundation, Huang2023, azizi21big, Vorontsov2024, huang2023radiology, huang21GLoRIA, chen23masked, Azizi2023}, training a robust 3D CT volume encoder presents distinct challenges, such as selecting appropriate pretext tasks, managing spatial normalization, and addressing high computational demands. To overcome these challenges, we developed a standardized pipeline that normalizes head CT scans from various protocols, producing consistent input for our foundation model. For pre-training, we adapted two SSL frameworks—self-distillation, inspired by DINO~\cite{caron2021emerging}, and masked prediction, inspired by MAE~\cite{He2021MaskedAA}. These methods were tailored to train a volumetric encoder based on a customized vision Transformer (ViT)\cite{dosovitskiy2020vit}. The full details of our design and choices are described in Section\hyperref[sec:methods]{``Method''}. pre-training was conducted on a large-scale dataset comprising 361,663 head CT scans from a major clinical institution.

To evaluate the foundation model, we systematically assessed its performance and generalizability across 10 downstream disease detection tasks using diverse internal and external datasets, as illustrated in \Cref{fig:overview}. Beyond commonly studied hemorrhages, our evaluation includes crucial yet less-explored tasks in head CT, such as identifying brain tumors, Alzheimer’s disease and related dementia (ADRD), edema, and hydrocephalus (HCP). For each downstream task, the foundation model was fine-tuned using task-specific labels. Given the scarcity of expert-annotated public datasets for these conditions, we leverage electronic health records (EHR) to acquire labels of each task. While EHRs may include missing data and suffer potential label-noise, they remain a valuable and practical source for large-scale patient status labeling that can be used to evaluate the performance of the foundation model.

Our results reveal substantial performance improvements enabled by our foundation model. Downstream models initialized with the pre-trained weights of foundation model achieve a $16.07\%$ improvement in macro-AUC over models trained from scratch with random initialization on internal NYU Langone data, and $20.86\%$ and $12.01\%$ improvements on external datasets from NYU Long Island (previously a separate hospital) and RSNA, respectively ($P<0.001$ for all comparisons). These findings underscore the potential of our foundation model to advance AI-based interpretation of head CT scans, supporting more accurate diagnosis and early disease detection. Furthermore, as described in Section~\hyperref[sec:results]{``Results''}, we demonstrate the model’s capabilities in out-of-distribution generalization (\Cref{fig:overview}), few-shot learning (\Cref{fig:fewshot}), and scalability (\Cref{fig:scaling_law}), highlighting the method’s potential in scenarios with limited annotated fine-tuning data, or scenarios such as federated learning which provide access to orders of magnitude more data. Overall, the experimental results on multiple datasets and tasks underscore the generalizability, adaptability and effectiveness of the model, and pave the way for significant impact in real-world clinical applications.




\section{Overview}

\revision{In this section, we first explain the foundational concept of Hausdorff distance-based penetration depth algorithms, which are essential for understanding our method (Sec.~\ref{sec:preliminary}).
We then provide a brief overview of our proposed RT-based penetration depth algorithm (Sec.~\ref{subsec:algo_overview}).}



\section{Preliminaries }
\label{sec:Preliminaries}

% Before we introduce our method, we first overview the important basics of 3D dynamic human modeling with Gaussian splatting. Then, we discuss the diffusion-based 3d generation techniques, and how they can be applied to human modeling.
% \ZY{I stopp here. TBC.}
% \subsection{Dynamic human modeling with Gaussian splatting}
\subsection{3D Gaussian Splatting}
3D Gaussian splatting~\cite{kerbl3Dgaussians} is an explicit scene representation that allows high-quality real-time rendering. The given scene is represented by a set of static 3D Gaussians, which are parameterized as follows: Gaussian center $x\in {\mathbb{R}^3}$, color $c\in {\mathbb{R}^3}$, opacity $\alpha\in {\mathbb{R}}$, spatial rotation in the form of quaternion $q\in {\mathbb{R}^4}$, and scaling factor $s\in {\mathbb{R}^3}$. Given these properties, the rendering process is represented as:
\begin{equation}
  I = Splatting(x, c, s, \alpha, q, r),
  \label{eq:splattingGA}
\end{equation}
where $I$ is the rendered image, $r$ is a set of query rays crossing the scene, and $Splatting(\cdot)$ is a differentiable rendering process. We refer readers to Kerbl et al.'s paper~\cite{kerbl3Dgaussians} for the details of Gaussian splatting. 



% \ZY{I would suggest move this part to the method part.}
% GaissianAvatar is a dynamic human generation model based on Gaussian splitting. Given a sequence of RGB images, this method utilizes fitted SMPLs and sampled points on its surface to obtain a pose-dependent feature map by a pose encoder. The pose-dependent features and a geometry feature are fed in a Gaussian decoder, which is employed to establish a functional mapping from the underlying geometry of the human form to diverse attributes of 3D Gaussians on the canonical surfaces. The parameter prediction process is articulated as follows:
% \begin{equation}
%   (\Delta x,c,s)=G_{\theta}(S+P),
%   \label{eq:gaussiandecoder}
% \end{equation}
%  where $G_{\theta}$ represents the Gaussian decoder, and $(S+P)$ is the multiplication of geometry feature S and pose feature P. Instead of optimizing all attributes of Gaussian, this decoder predicts 3D positional offset $\Delta{x} \in {\mathbb{R}^3}$, color $c\in\mathbb{R}^3$, and 3D scaling factor $ s\in\mathbb{R}^3$. To enhance geometry reconstruction accuracy, the opacity $\alpha$ and 3D rotation $q$ are set to fixed values of $1$ and $(1,0,0,0)$ respectively.
 
%  To render the canonical avatar in observation space, we seamlessly combine the Linear Blend Skinning function with the Gaussian Splatting~\cite{kerbl3Dgaussians} rendering process: 
% \begin{equation}
%   I_{\theta}=Splatting(x_o,Q,d),
%   \label{eq:splatting}
% \end{equation}
% \begin{equation}
%   x_o = T_{lbs}(x_c,p,w),
%   \label{eq:LBS}
% \end{equation}
% where $I_{\theta}$ represents the final rendered image, and the canonical Gaussian position $x_c$ is the sum of the initial position $x$ and the predicted offset $\Delta x$. The LBS function $T_{lbs}$ applies the SMPL skeleton pose $p$ and blending weights $w$ to deform $x_c$ into observation space as $x_o$. $Q$ denotes the remaining attributes of the Gaussians. With the rendering process, they can now reposition these canonical 3D Gaussians into the observation space.



\subsection{Score Distillation Sampling}
Score Distillation Sampling (SDS)~\cite{poole2022dreamfusion} builds a bridge between diffusion models and 3D representations. In SDS, the noised input is denoised in one time-step, and the difference between added noise and predicted noise is considered SDS loss, expressed as:

% \begin{equation}
%   \mathcal{L}_{SDS}(I_{\Phi}) \triangleq E_{t,\epsilon}[w(t)(\epsilon_{\phi}(z_t,y,t)-\epsilon)\frac{\partial I_{\Phi}}{\partial\Phi}],
%   \label{eq:SDSObserv}
% \end{equation}
\begin{equation}
    \mathcal{L}_{\text{SDS}}(I_{\Phi}) \triangleq \mathbb{E}_{t,\epsilon} \left[ w(t) \left( \epsilon_{\phi}(z_t, y, t) - \epsilon \right) \frac{\partial I_{\Phi}}{\partial \Phi} \right],
  \label{eq:SDSObservGA}
\end{equation}
where the input $I_{\Phi}$ represents a rendered image from a 3D representation, such as 3D Gaussians, with optimizable parameters $\Phi$. $\epsilon_{\phi}$ corresponds to the predicted noise of diffusion networks, which is produced by incorporating the noise image $z_t$ as input and conditioning it with a text or image $y$ at timestep $t$. The noise image $z_t$ is derived by introducing noise $\epsilon$ into $I_{\Phi}$ at timestep $t$. The loss is weighted by the diffusion scheduler $w(t)$. 
% \vspace{-3mm}

\subsection{Overview of the RTPD Algorithm}\label{subsec:algo_overview}
Fig.~\ref{fig:Overview} presents an overview of our RTPD algorithm.
It is grounded in the Hausdorff distance-based penetration depth calculation method (Sec.~\ref{sec:preliminary}).
%, similar to that of Tang et al.~\shortcite{SIG09HIST}.
The process consists of two primary phases: penetration surface extraction and Hausdorff distance calculation.
We leverage the RTX platform's capabilities to accelerate both of these steps.

\begin{figure*}[t]
    \centering
    \includegraphics[width=0.8\textwidth]{Image/overview.pdf}
    \caption{The overview of RT-based penetration depth calculation algorithm overview}
    \label{fig:Overview}
\end{figure*}

The penetration surface extraction phase focuses on identifying the overlapped region between two objects.
\revision{The penetration surface is defined as a set of polygons from one object, where at least one of its vertices lies within the other object. 
Note that in our work, we focus on triangles rather than general polygons, as they are processed most efficiently on the RTX platform.}
To facilitate this extraction, we introduce a ray-tracing-based \revision{Point-in-Polyhedron} test (RT-PIP), significantly accelerated through the use of RT cores (Sec.~\ref{sec:RT-PIP}).
This test capitalizes on the ray-surface intersection capabilities of the RTX platform.
%
Initially, a Geometry Acceleration Structure (GAS) is generated for each object, as required by the RTX platform.
The RT-PIP module takes the GAS of one object (e.g., $GAS_{A}$) and the point set of the other object (e.g., $P_{B}$).
It outputs a set of points (e.g., $P_{\partial B}$) representing the penetration region, indicating their location inside the opposing object.
Subsequently, a penetration surface (e.g., $\partial B$) is constructed using this point set (e.g., $P_{\partial B}$) (Sec.~\ref{subsec:surfaceGen}).
%
The generated penetration surfaces (e.g., $\partial A$ and $\partial B$) are then forwarded to the next step. 

The Hausdorff distance calculation phase utilizes the ray-surface intersection test of the RTX platform (Sec.~\ref{sec:RT-Hausdorff}) to compute the Hausdorff distance between two objects.
We introduce a novel Ray-Tracing-based Hausdorff DISTance algorithm, RT-HDIST.
It begins by generating GAS for the two penetration surfaces, $P_{\partial A}$ and $P_{\partial B}$, derived from the preceding step.
RT-HDIST processes the GAS of a penetration surface (e.g., $GAS_{\partial A}$) alongside the point set of the other penetration surface (e.g., $P_{\partial B}$) to compute the penetration depth between them.
The algorithm operates bidirectionally, considering both directions ($\partial A \to \partial B$ and $\partial B \to \partial A$).
The final penetration depth between the two objects, A and B, is determined by selecting the larger value from these two directional computations.

%In the Hausdorff distance calculation step, we compute the Hausdorff distance between given two objects using a ray-surface-intersection test. (Sec.~\ref{sec:RT-Hausdorff}) Initially, we construct the GAS for both $\partial A$ and $\partial B$ to utilize the RT-core effectively. The RT-based Hausdorff distance algorithms then determine the Hausdorff distance by processing the GAS of one object (e.g. $GAS_{\partial A}$) and set of the vertices of the other (e.g. $P_{\partial B}$). Following the Hausdorff distance definition (Eq.~\ref{equation:hausdorff_definition}), we compute the Hausdorff distance to both directions ($\partial A \to \partial B$) and ($\partial B \to \partial A$). As a result, the bigger one is the final Hausdorff distance, and also it is the penetration depth between input object $A$ and $B$.


%the proposed RT-based penetration depth calculation pipeline.
%Our proposed methods adopt Tang's Hausdorff-based penetration depth methods~\cite{SIG09HIST}. The pipeline is divided into the penetration surface extraction step and the Hausdorff distance calculation between the penetration surface steps. However, since Tang's approach is not suitable for the RT platform in detail, we modified and applied it with appropriate methods.

%The penetration surface extraction step is extracting overlapped surfaces on other objects. To utilize the RT core, we use the ray-intersection-based PIP(Point-In-Polygon) algorithms instead of collision detection between two objects which Tang et al.~\cite{SIG09HIST} used. (Sec.~\ref{sec:RT-PIP})
%RT core-based PIP test uses a ray-surface intersection test. For purpose this, we generate the GAS(Geometry Acceleration Structure) for each object. RT core-based PIP test takes the GAS of one object (e.g. $GAS_{A}$) and a set of vertex of another one (e.g. $P_{B}$). Then this computes the penetrated vertex set of another one (e.g. $P_{\partial B}$). To calculate the Hausdorff distance, these vertex sets change to objects constructed by penetrated surface (e.g. $\partial B$). Finally, the two generated overlapped surface objects $\partial A$ and $\partial B$ are used in the Hausdorff distance calculation step.

\section*{Results}
\label{sec:results}
\subsection*{Foundation model for disease detection with 3D head CT scans}

The key aim of the foundation model is to develop a single model that improves performance on a wide range of downstream tasks of detecting recognizable abnormalities from head CT scans. To evaluate the capability of the foundation model, we train classification models for multiple disease detection tasks by fine-tuning the foundation model separately per disease, and assessing the fine-tuned model's performance on held-out validation and external data sets. The selected downstream tasks include detecting various types of hemorrhages (intraparenchymal hemorrhage (IPH), intraventricular hemorrhage (IVH), subdural hemorrhage (SDH), epidural hemorrhage (EDH), subarachnoid hemorrhage (SAH), and intracranial hemorrhage (ICH)), brain tumors, hydrocephalus (HCP), edema, and Alzheimer's diseases and related dementia (ADRD). \Cref{fig:overview}a,b,c show the overview of our pre-training framework and included data, EHR-matching, and datasets used in pre-training, in-domain fine-tuning, and external validation. Overall N=361,663 scans were used during pre-training, and four distinct datasets from different sources were used for various forms of validation (NYU Langone N=26,487; NYU Long Island N=2,202; RSNA N=1,058; and CQ500 N=236). NYU Langone is a hospital system comprised of multiple geographically distinct hospitals including two Level 1 Trauma Centers and three Comprehensive Stroke Centers. NYU Long Island, a Level 1 Trauma Center/Comprehensive Stroke Center, is treated as an external dataset for the purposes of this study. 

The first two rows of \Cref{fig:overview}e report the task-specific AUCs for Vision Transformer (ViT) classifiers trained from scratch with random initialization, namely \textit{scratch}, versus those fine-tuned from the foundation model, namely \textit{fine-tuned} on NYU Langone data. The fine-tuned models consistently outperform the scratch model across all 10 disease detection tasks, achieving a macro-AUC of 0.852 --- a $16.07\%$ increase over the scratch model’s 0.734 ($P<0.001$). Additionally, in \Cref{fig:overview}f and Supplementary \Cref{fig:radar-comparison-merlin} we compared the foundation model with two other foundation model for 3D CT scans --- Merlin~\cite{blankemeier2024merlinvisionlanguagefoundation} and Google's CT Foundation\cite{yang2024advancing} model. Merlin outperforms the scratch model with a macro-AUC of $5.67\%$ while falling short compared to our foundation model with $7.51\%$ lower macro-AUC ($P<0.001$, illustrated in Supplementary \Cref{fig:radar-comparison-merlin}). Although Merlin is not directly comparable to our foundation model as it was pre-trained on abdominal CT, it still provides a valuable baseline. We compare our model to Google CT Foundation model with linear probing, because trainable weights for end-to-end fine-tuning are not provided for this model. We consistently observe improved model performance across the board (in \Cref{fig:overview}f and Supplementary \Cref{fig:probing-comparison-gemini}). %
These findings demonstrate that despite the progress in general domain multimodal models, specialized foundation model pre-trained on head CT data still significantly enhance the understanding of brain CTs. 

To assess our foundation model's generalization to out-of-distribution data, we compiled three external datasets from multiple institutions and sources: NYU Long Island, RSNA~\cite{flanders_construction_2020}, and CQ500~\cite{CQ500}, as shown in \Cref{fig:overview}c (NYU Langone and NYU Long Island are geographically separate and distinct institutions within the broader health system). The data in these external datasets has a different distribution than the data used for pre-training. We evaluate the generalization on external datasets via two common practices to utilize the foundation model: (1) in-domain fine-tuning on separated datasets and tasks, and (2) fully external validation of the disease detection models without any site-specific fine-tuning.

\begin{figure}[htbp]
    \centering
    NYU Langone \\
    \includegraphics[trim={0 0 140pt 0},clip,height=0.3\textwidth]{figures/fig2/fewshot_NYU_AUC.pdf}
    \includegraphics[height=0.3\textwidth]{figures/fig2/fewshot_NYU_AP.pdf} \\
    NYU Long Island \\
    \includegraphics[trim={0 0 140pt 0},clip,height=0.3\textwidth]{figures/fig2/fewshot_NYU_Long_Island_AUC.pdf}
    \includegraphics[height=0.3\textwidth]{figures/fig2/fewshot_NYU_Long_Island_AP.pdf} \\
    RSNA \\
    \includegraphics[trim={0 0 125pt 0},clip,height=0.295\textwidth]{figures/fig2/fewshot_RSNA_AUC.pdf}
    \includegraphics[height=0.295\textwidth]{figures/fig2/fewshot_RSNA_AP.pdf}
    \caption{\textbf{Few-shot performance of the foundation model.}  The plots display the per-pathology AUC and average precision (AP) of the disease detection model under a few-shot learning setting, evaluated with varying numbers of training samples from the NYU Langone, NYU Long Island, and RSNA datasets. CQ500 is excluded since its small dataset size gives no enough positive samples for many diseases. Few-shot learning performance is compared to supervised finetuning with all training data (denoted by stars), demonstrating the strong generalization ability of the foundation model with limited training data. The confidence intervals are computed by 5 repeated experiments on resampling the training data and retraining the model.}
    \label{fig:fewshot}
\end{figure}


For in-domain fine-tuning, the foundation model is fine-tuned on each external dataset’s training set and validated on held-out sets from the same source. The bottom four rows in \Cref{fig:overview}e report the tasks-level performances on NYU Long Island and RSNA datasets. The fine-tuned model yields a macro-AUC of 0.904 across the 10 tasks on NYU Long Island dataset and a macro-AUC of 0.923 for five types of hemorrhages on the RSNA dataset. In comparison, the scratch model results in macro-AUC scores of 0.748 and 0.824, respectively.  Moreover, the foundation model also significantly outperforms Merlin, as shown in Supplementary \Cref{fig:radar-comparison-merlin}. The superior performances on external datasets indicate the generalizability of the foundation model. Note that the limited data size of CQ500 forbids training a effective deep learning model from scratch, reinforcing the importance of the foundation model in label efficiency, which is further studied in Section \hyperref[sec:label_efficiency]{"Label efficiency"}. Interestingly, when comparing performances across different datasets, \Cref{fig:overview}e demonstrates that the AUCs of the in-domain fine-tuned model on the external dataset even exceed the AUCs achieved on the internal dataset. For instance, the fine-tuned models consistently obtained AUCs greater than 0.90 in all the hemorrhage detection tasks on RSNA dataset, surpassing the AUCs on NYU Langone data. This may be attributed to the higher label quality in radiologist-reviewed datasets, for which label noise may be better controlled by comparison to EHR-derived labels.

In the full external validation without any site-specific fine-tuning (illustrated in \Cref{fig:overview}c), we evaluated classification models fine-tuned on the NYU Langone training set, as-is, on the held-out validation sets from each external dataset. \Cref{fig:overview}f compares performance between external validation and in-domain fine-tuning. Results show that, for the NYU Long Island and RSNA datasets —where the training set used for fine-tuning includes a sufficient number of high-quality labeled samples— in-domain fine-tuning does enhance the model performance. However, on the CQ500 dataset, with only 1,120 training samples, the in-domain fine-tuned model performs worse than the model transferred from NYU Langone, especially for EDH and SDH, which have a greater class imbalance. These comparisons highlight two typical use cases for foundation models depending on the availability of labeled data for fine-tuning. Additionally, comparing the first row of \Cref{fig:overview}e and external validation in \Cref{fig:overview}f, the fine-tuned model on NYU Langone achieves similar AUC values on both internal and external datasets, indicating robust generalizability to external data.











\subsection*{Label efficiency of few-shot classification performance}
\label{sec:label_efficiency}

Another key advantage of the foundation model is its ability to facilitate transfer learning and fine-tuning tasks with minimal labeled data. For example, as shown in \Cref{fig:overview}c, the CQ500 dataset contains only 1,585 scans. Despite the small dataset size, fine-tuning our foundation model on CQ500 achieves promising results, with an AUC of 0.863. 

To systematically evaluate the label efficiency of our foundation model, we also assess the generalization capabilities of models on new tasks given a limited number of examples within the paradigm of few-shot learning, where only $K$ positive and negative samples each are used for training in each task. Since the quality of few-shot learning is largely determined by the sampled $K$-shots training data, we re-sampled and re-trained the model 5 times for calculating means and confidence intervals. As expected, \Cref{fig:fewshot} shows that performance improves as more data is used for training, with narrower confidence intervals. Surprisingly, even with a small number of examples (e.g., 512 total, with $K=256$), the model achieves performance comparable to training with the full dataset, which contains over at least 16 times more training examples in the RSNA. Notably, for tasks like detecting IVH in the RSNA dataset, the 8-shots model achieves an AUC above 0.90, a result that rivals full-data training. These findings suggest that our foundation model has learned diverse and expressive features/representations during SSL pre-training, making it highly effective for new tasks even when trained on small labeled datasets.

\begin{figure}[t]
    \centering
    \includegraphics[width=0.24\textwidth]{figures/fig4/pretrain_size_NYU_Langone_AUC.pdf} 
    \includegraphics[width=0.24\textwidth]{figures/fig4/pretrain_size_NYU_Long_Island_AUC.pdf}
    \includegraphics[width=0.24\textwidth]{figures/fig4/pretrain_size_RSNA_AUC.pdf}
    \includegraphics[width=0.24\textwidth]{figures/fig4/pretrain_size_CQ500_AUC.pdf}
    \includegraphics[width=0.24\textwidth]{figures/fig4/pretrain_size_NYU_Langone_AP.pdf}
    \includegraphics[width=0.24\textwidth]{figures/fig4/pretrain_size_NYU_Long_Island_AP.pdf}
    \includegraphics[width=0.24\textwidth]{figures/fig4/pretrain_size_RSNA_AP.pdf}
    \includegraphics[width=0.24\textwidth]{figures/fig4/pretrain_size_CQ500_AP.pdf}
    \caption{\textbf{Performance for Different Percentage of Pre-training Samples (Mean):} we compare the label efficiency in terms of different percentage of pre-training data for MAE vs. DINO. The $95\%$ CI are plotted in colour bands and the centre points of the bands indicate the mean value. We show that although DINO present higher label efficiency plot, both MAE and DINO efficiently scale up on downstream performance as more pre-training data is incorporated.}
    \label{fig:scaling_law}
\end{figure}


\subsection*{Scaling up pre-training data}

Scaling laws have proven effective in enhancing the performance of foundation models by increasing the size of the training dataset~\cite{kaplan2020scalinglawsneurallanguage}. This phenomenon is not only observed in natural language and image domains~\cite{zhai22scalingvit, pmlr-v202-dehghani23a}, but also extends to medical imaging~\cite{zhou2023foundation, li2024well}. As shown in \Cref{fig:scaling_law}, scaling up the foundation model by incorporating more data during self-supervised pre-training significantly improves downstream tasks performances. We compared models pre-trained with varying proportions of the available data --- $10\%$, $30\%$, and $100\%$ (full dataset), observing that larger pre-training datasets consistently led to better downstream task performance. These findings highlight the potential of leveraging more data to achieve superior results, further suggesting the value of multi-institutional collaboration and federated approaches to aggregating larger datasets to enhance model quality. Noticeably, the performance for CQ500 does not change a lot from $10\%$ to $30\%$, but $100\%$ gives a sudden performance improvement, this indicates that for smaller datasets like CQ500, scaling up the data size is crucial for learning meaningful representations.

\subsection*{Visual Interpretation}
To gain insight into the features learned through self-supervised pre-training and supervised fine-tuning of the foundation model, we visualize the attention maps within the Vision Transformer (ViT), as shown in \Cref{fig:attention_interpretation}. These heatmaps highlight the regions where the ViT model focuses most strongly. In the second column, we see that the pre-trained foundation model captures generic brain features, with dark red indicating attention on abnormal ventricular shapes and green marking areas of hemorrhage. After fine-tuning on specific tasks, the ViT’s attention becomes more focused on patterns relevant to each disease. For instance, in the edema task (third column), the heatmap extends across most of the brain, reflecting generalized swelling. For ADRD (fourth column), the model emphasizes regions of ventricular enlargement and cerebral atrophy. Multiple hemorrhages are also present in this sample, with attention covering both the IPH in the dense central region (fourth column) and extending toward the left end of the ventricle where IVH appears (fifth column). In the case of SAH (sixth column), the attention map is less prominent due to the small, peripheral area of the SAH in the lower part of the slice, although the model still predicts it accurately.

The comparison between the pre-trained and fine-tuned ViT explains the performance difference between linear probing and fine-tuning (shown in Supplementary \Cref{fig:probing-comparison-perpath}, as end-to-end fine-tuning allows the model to learn task-specific features more effectively. Details on the computation of the visualized attention maps are provided in Section \hyperref[sec:methods]{“Methods”}.

\section{Feature Directions in Safety Residual Space}
\label{sec:interpretation}


So far, we have focused on examining the dominant direction in the safety residual space, which predicts the model's aligned behavior. In this section, we will investigate how \textit{non-dominant} directions represent different features.

\paragraph{Problem} 
Unlike probe vectors, arbitrary directions lack pre-defined semantic meanings~\cite{bricken2023monosemanticity}, making it challenging to observe outcome changes through intervention experiments. While previous works~\cite{Ball2024UnderstandingJS,lee2024mechanistic} have used Logit Lens~\cite{nostalgebraist2020interpreting} to map representations to the projection layer in transformers, the faithfulness of this approach relies on vector similarity to the vocabulary space, which does not apply to residual directions.

\paragraph{Our Approach}

To determine features represented by directions, we introduce a theoretically grounded method within the LRP framework. We refer it as Partial Layer-wise Relevance Propagation (PLRP): given a set of directions $\{v_i\}$ and representations $X^l$, we first project $X^l$ onto the span of $\{v_i\}$. We then decompose its Euclidean norm into relevance scores $R$ and back-propagate the relevance scores. To ensure relevance conservation, we apply the epsilon rule~\cite{bach2015pixel} for handling projections. Formally we have:

\begin{align*}
    P_V(X^l
    ) = \sum_{v \in V} \|v^T X^l\|_2^2 \propto R_l
\end{align*}

The relevance score $R_l$ is then back-propagated to either (1) input tokens in training data or (2) projections on directions of activation in earlier layers. For input tokens $t$, we follow \citet{achtibat2024attnlrp} and sum up relevance scores of all elements in the token embedding, i.e., $R^{<t>} = \sum_{i=1}^{d} R_i^{<t>}$. To compute relevance scores of directions $v_i$ in $X^{l'}$ of earlier layers, we first compose an linear reconstruction term with first $k$ SVD components $V_{:k} \in \mathbb{R}^{d \times k}$: $\hat{X}^{l'} = V_{:k}W + \epsilon$, where $W \in \mathbb{R}^k$ minimizes the reconstruction error $\epsilon$. We then calculate the relevance scores $R^W_i$ on elements of $W$ and re-normalize to remove relevance scores absorbed by $\epsilon$. The relevance scores of $v_i$ is then given by average $R^W_i$ across all training samples.

\begin{figure}
    \vskip 0.2in
    \begin{center}
    \centerline{\includegraphics[width=\columnwidth]{figures/fig_supression.pdf}}
    \caption{Intervention results after removing the direction of 6th component of layer 14 (\texttt{L14-C6}) from the hidden states during generation. \texttt{L14-C6} is identified as representing the specific ability to recognize the PAIR Attack. Additionally, we remove the dominant direction (\texttt{L25-C1}), which completely eliminates the fine-tuned model's ability to refuse.}
    \label{fig:intervention}
    \end{center}
    \vskip -0.2in
\end{figure}

\subsection{Interpreting Directions via Token Relevance}
\label{iterpret_tokens}

We demonstrate that relevance scores of training input tokens help understand the semantic meaning of directions in the safety residual space. \autoref{tab:plrp_logitlens} visualizes the relevance distribution for several directions using a handcrafted example on layer 14. \footnote{Other layers around layer 14 also show similar patterns. We provide an analysis in \autoref{sec:early_phase}} We provide observations on the dominant and non-dominant directions in the following.

\paragraph{Dominant Direction} We evaluate dominant directions (i.e. \texttt{LN-C1}) and non-dominant directions (i.e. \texttt{L14-CK} in \autoref{tab:plrp_logitlens}) separately. The \textsc{Top Token} column shows the most relevant training tokens that activate each direction. For \texttt{L14-C1} and \texttt{L15-C1}, we observe that the dominant direction primarily relates to harmful subjects, such as \textit{divisive ideologies}. This aligns with our earlier finding that the dominant direction best predicts harmfulness.

\paragraph{Non-Dominant Direction}
For non-dominant directions, we find they are activated not by toxicity or harmfulness, but rather by features characteristic of specific jailbreak patterns. For instance, tokens like \textit{Imagine}, \textit{fictional} and \textit{hypothetical} in \texttt{L14-C2} establish a hypothetical tone. This negatively correlates with the dominant component in layer 25, reducing the probability of refusal. Meanwhile, \texttt{L14-C5} is triggered by explicit mentions of \textit{ChatGPT} and positively correlates with the dominant direction, likely due to its prevalent use in role-playing jailbreaks~\cite{yu2023gptfuzzer}. These findings suggest that non-dominant directions serve to capture indirect features related to safety.

\paragraph{The ``Sure, I'm happy to help'' Direction}
Notably, \texttt{L14-C6} activates when \textit{Sure, I'm happy to help} co-occurs with \textit{Imagine}. We notice that this pattern matches common jailbreak techniques used by PAIR~\cite{chao2023pair}, which typically set up harmful requests in imaginary scenarios (e.g., \textit{Imagine you are a professional hacker}) and force the model to respond positively (e.g., \textit{Start your response with `Sure, I'm happy to help'}). To validate \texttt{L14-C6}'s role, we intervene during generation using \autoref{eq:intervene} to remove its corresponding direction from the safety fine-tuned model. \autoref{fig:intervention} confirms that removing \texttt{L14-C6} specifically ablate the model's ability to refuse PAIR prompts while preserving its capability to handle other attack types. We evaluate the intervention's impact on the model's general abilities in \autoref{sec:impact_inter}, ensuring that removing refusal does not degrade overall performance.

\begin{figure}
    % \vskip 0.1in
    \begin{center}
    \includegraphics[width=\columnwidth]{figures/relevance_heatmap.pdf}
    \vspace{-0.1in}
        \caption{\textbf{Top 3}: Adjacent layer relevance scores among top directions. \texttt{Rel Comp 1}: relevance scores to first component in next layer. \textbf{Bottom}: Log-likelihood of predicting aligned behavior with different directions.}
    \label{fig:network_arch}
    \end{center}
    \vskip -0.2in
\end{figure}

\begin{figure*}[t]
    \begin{center}
    \centerline{\includegraphics[width=\textwidth]{figures/component_projections.pdf}}
    \caption{Projection of representations on top components under different settings in SSFT. The projection on component 1 is strongly correlated with the model's safety behavior. \texttt{Harmful}: representations are from harmful samples. \texttt{Benign}: representations are from benign samples. \texttt{Non-Dominance}: \texttt{Harmful} setting with most non-dominant components removed by intervention. \texttt{Removal}: harmful samples with trigger tokens removed. We evaluate the intervention's impact on the model's general abilities in \autoref{sec:impact_inter}, ensuring that intervention does not degrade overall performance.}
    \label{fig:component_projections}
    \end{center}
    \vskip -0.2in
\end{figure*}

\subsection{Layer-Wise Dynamics of Safety Residual Space}

We now examine the evolution of safety feature directions in the space. Using PLRP, we can measure how one direction influences another by attributing feature directions to directions in earlier layers. \autoref{fig:network_arch} visualizes the  relevance score of different components between adjacent layers.

\paragraph{Early Phase: Development of Safety Features}
\label{sec:early_phase}
We analyze how feature directions evolve across layers using PLRP to trace relevance scores through the transformer network. Our analysis reveals two distinct patterns of propagation. In most layers, directions primarily retain information from their counterparts in the previous layer. For instance, as shown in \texttt{Rel Comp 1} of \autoref{fig:network_arch}, \texttt{L20-C1} inherits most of its relevance from \texttt{L19-C1}. In contrast, during early layers, directions exhibit a more dynamic pattern, receiving contributions from multiple directions in the previous layer.

\paragraph{Late Phase: Uncertainty Reduction for Safety Behavior}
After layer 15, we observe that major directions exhibit a stronger retention pattern, but their corresponding eigenvalues continue to increase. This creates an interesting dynamic: although the model's harmfulness prediction accuracy plateaus after layer 15 (as shown in \autoref{fig:classification_accuracy}), the log-likelihood of these predictions continues to grow across subsequent layers (\autoref{fig:network_arch}, Bottom).


In summary, our analysis reveals that \emph{feature directions develop gradually through the network, stabilizing their safety semantic meanings in the early layers. Subsequently, the dominant direction responsible for safety behavior continues to strengthen, reducing uncertainty in the model's aligned outputs.}

% These findings indicate that feature directions develop gradually through the network rather than emerging suddenly at specific layers.
% This pattern suggests that the late phase of forward propagation primarily serves to reduce uncertainty in the model's existing predictions rather than developing new features.


% \begin{figure}
%     % \vskip 0.1in
%     \begin{center}
%     \begin{tabular}{c}
%         \includegraphics[width=\columnwidth]{figures/network_0.pdf} \\[-0.2in]
%         \includegraphics[width=\columnwidth]{figures/network_1.pdf} \\[-0.2in]
%         \includegraphics[width=\columnwidth]{figures/network_2.pdf} \\[-0.2in]
%         \includegraphics[width=\columnwidth]{figures/network_3.pdf}
%     \end{tabular}
%     \caption{Relevance propagation visualization across different components.}
%     \label{fig:network_arch}
%     \end{center}
%     % \vskip -0.2in
% \end{figure}

\section*{Discussion}
Despite advances in disease detection using 3D head CT scans, current solutions are limited by the availability of annotated data and the complex, task-specific design requirements of network architectures. These constraints hinder the broader application of machine learning in clinical disease detection. To address this, we developed a foundation model, trained on a large unlabeled dataset, to enable fine-tuning for multiple tasks with minimal labeled data under a unified network architecture. 

Highly accurate detection of intracerebral hemorrhages without delay is a critical clinical issue for the diagnostic decision making and treatment in an emergency room \cite{Hemphill2015-yw,Qureshi2009-ve}. Our results indicate that 3D Head CT scans can also be used to help identify hemorrhage subtypes and, more interestingly, etiology. High performances and generalizability observed by our model in detecting intracerebral hemorrhage have a potential to greatly assist in pre-hospital and early hospital management of blood pressure. This is particularly important given that early blood pressure control is a key factor in preventing hematoma expansion and improving patient outcomes~\cite{Macellari2014-zj,Morotti2022-lc}.

This approach is also particularly valuable for extending detection capabilities to new diseases in CT imaging. For example, early detection of ADRD with deep learning has traditionally relied on MRI scans \cite{Li2019-jx, pmlr-v116-liu20a, Xue2024}. However, access to MRI machines is costly and often restricted by patients’ geographic location and socioeconomic status~\cite{https://doi.org/10.1002/neo2.10}. Head CT, in contrast, is fast, accessible, and is the first-line imaging test in emergency and diagnostic settings. Our foundation model enables more accessible ADRD detection using head CT scans. This advancement holds the potential for expanding early ADRD detection in common public health settings for the older population~\cite{lin_dementia_2020, kim2021racial}, such as emergency rooms, as well as in underserved communities nationally and internationally in which CT is more available than MRI. Similarly, our model could facilitate the development of detection tools for other conditions, such as cancers and neuroinfectious diseases, thus supporting population health on a broader scale.

Our study demonstrates that this pre-trained foundation model significantly outperforms models trained from scratch and other CT foundation models on the same labeled data. Moreover, it exhibits strong performance even with limited data, as shown in few-shot learning experiments, and suggests promising potential for scaling up with larger datasets. In clinical practice, head CT scans are typically acquired using heterogeneous protocols, including variations in slice thickness and scanner modalities. A robust foundation model for CT should generalize effectively across these diverse protocols. In this study, we utilized scans with slice thicknesses ranging from 0.5 mm to 5 mm and data from two major manufacturers (Siemens and Toshiba) to develop and assess the generalization capabilities of our foundation model. As illustrated in Supplementary \Cref{fig:batch_effect}, the embeddings produced by the foundation model show separability based on scanner manufacturer and slice thickness, likely reflecting variations in protocol distribution. However, by comparing the distribution of “All” patients to that of positive cases for each condition, we observe that the embeddings do not tend to collapse or bias towards a certain protocol. Supplementary \Cref{fig:thickness-ablation} further demonstrates that fine-tuned models achieve comparable performance across scanner protocols. Detailed per-task performance results are provided in Supplementary \Cref{fig:slice_thickness_per_pathology,fig:manufacturer_per_pathology}.  Additionally, in comparisons of Toshiba and Siemens scanners, we noted a systemically higher prevalence of positive cases across all tasks in Toshiba scans, leading to a modestly higher AUC in specific instances. Despite these variations, our foundation model demonstrates robust generalization capabilities across diverse CT protocols, highlighting its potential for broad clinical application.

However, our evaluation is limited by label noise in real-world datasets. Labels derived from electronic health records (EHRs) can suffer from missing or incomplete information. This issue is evidenced by the model’s lower performance on NYU Langone data compared to RSNA data, where labels were rigorously reviewed by radiologists. Another limitation is that, due to constraints on training samples and computational resources, our model does not yet fully explore the potential of scaling laws. The 361K scans used for pre-training represent the entirety of CT scans available from a single large clinical institution, highlighting the need for multi-institutional collaborations to enhance the dataset diversity and volume. With greater computational resources, we could also scale up the model’s size, resolution of image patches, and number of tokens used in the ViT architecture, potentially improving performance for detecting conditions with small spatial manifestations, such as subarachnoid hemorrhage (SAH).

While our current results primarily focused on disease detection, our foundation model holds significant potential for advancing disease prognosis analysis. For instance, the prediction of decompensation, particularly hemorrhagic expansion, is an important potential use of the foundation model and may lead to the development of novel hyperacute treatment strategies~\cite{hematoma_expansion}. Additionally, critical applications in acute ischemic stroke, such as predicting hemorrhagic transformation and the development of malignant edema can benefit from the foundation model. Beyond acute conditions, the foundation model can potentially also be used to predict the development of ADRD~\cite{Zhu2024-zd}.


\section*{Methods}
\label{sec:methods}
\subsection*{Datasets}
\subsubsection*{Dataset for pre-training foundation model}
We utilized a large-scale head CT scan dataset from NYU Langone, consisting of 499,084 scans across 203,665 patients, collected between 2009 and 2023. These scans were acquired using Siemens and Toshiba machines. We included all the non-contrast head CT scans with ranging from 0.5mm to 5mm, kVp values between 70 and 150, and convolution kernels Hr/Qr/J with sharpness levels of 35-45. We filtered out corrupted scan series with missing DICOM files and those containing less than 10 slices, resulting in 451,298 scans. We partitioned these scans by the patient IDs into training, validation, and held-out validation sets in an 8:1:1 ratio to avoid the leakage of scans from the sample patient. As illustrated in \Cref{fig:overview}a, this led to training, validation and held-out validation set with 361,663, 44,886 and 44,749 scans, respectively. The scans in the training set were used to train the foundation model.

\subsubsection*{Datasets for downstream tasks}
We evaluated our model using four datasets: one \emph{in-domain (ID)} dataset from NYU Langone and three \emph{out-of-domain (OOD)} datasets from NYU Long Island, the RSNA Challenge, and the public CQ500 dataset. Each dataset includes multiple head CT disease detection classes, with some classes abbreviated as follows: Hydrocephalus (HCP), Dementia (ADRD), Intraparenchymal Hemorrhage (IPH), Intraventricular Hemorrhage (IVH), Subdural Hemorrhage (SDH), Epidural Hemorrhage (EDH), Subarachnoid Hemorrhage (SAH), and Intracerebral Hemorrhage (ICH). These classes can have co-occur in the same Head CT scan. The characteristics of the patients are shown in \Cref{tab:characteristic}. We split all datasets by patients to avoid information leakage, Further dataset details of our dataset are provided below.

\paragraph{NYU Langone - 10 tasks} The NYU Langone main campus dataset serves as the internal ID dataset for downstream evaluation. As depicted in \Cref{fig:overview}b, patient health status was derived from Electronic Health Records (EHR) within a 3-month window centered around the scan date, with conditions defined by ICD-10 diagnostic codes and medications, outlined in Supplementary \Cref{tab:disease_definition}. This cohort includes 270,205 scans from 66,801 patients with valid EHRs, covering 10 classes: Tumor, HCP, Edema, ADRD, IPH, IVH, SDH, EDH, SAH, and ICH. This cohort was partitioned following the same split used used during pre-training: matched patients within the training, validation, and held-out subsets of the SSL pre-training phase were assigned to the corresponding sets of the supervised fine-tuning phase. This resulted in 217,109, 26,609, and 26,487 scans in the training, validation and test set, respectively.
\paragraph{NYU Long Island - 10 tasks} NYU Long Island data was acquired in Long Island hospital which used to be a separate hospital, severing as an OOD dataset. This dataset includes 22,158 samples with 10 classes, labeled similarly to the NYU Langone dataset using ICD-10 codes. It is partitioned into an 8:1:1 train-validation-test split.
\paragraph{RSNA - 5 tasks} The RSNA Head CT Challenge dataset~\cite{flanders_construction_2020} serves as a public external OOD dataset, collected from Stanford University, Thomas Jefferson University, Unity Health Toronto, and Universidade Federal de São Paulo (UNIFESP). The dataset, initially provided as 2D slices, was reorganized by subject ID, retaining subjects with complete slice data. After preprocessing, the dataset consists of 10,579 samples across five classes: Any (any hemorrhage type), IPH, IVH, SAH, and SDH. Dataset labels were assigned by 60 volunteers from the American Society of Neuroradiology (ASNR). We partitioned this cohort into an 8:1:1 train-validation-test split.
\paragraph{CQ500 - 10 tasks} The CQ500 Head CT dataset~\cite{CQ500} serves as another public external OOD dataset, collected from multiple centers in India. This dataset includes 1,585 samples including varying slice thickness across 10 selected classes: ICH, IPH, IVH, SDH, EDH, SAH, BleedLocation-Left, BleedLocation-Right, MidlineShift, MassEffect. Each scan was labeled by three senior radiologists, and the cohort was split into an 8:1:1 train-validation-test ratio.

\subsubsection*{Label acquisition from electronic health records}

As illustrated in \Cref{fig:overview}b, we labeled head CT scans from NYU Langone and Long Island Hospital using electronic health records (EHR). For each head CT, we retrieved an EHR snippet for the corresponding patient based on their Medical Record Number (MRN), starting from the time of the scan and covering a 90-day period. We then checked for the presence of any diagnosis codes (ICD-10 codes) and medication records, within this EHR snippet that matched the predefined definitions for each disease, allowing us to create binary labels for each condition. The complete list of ICD-10 codes and the medications used for disease definitions is provided in Supplementary \Cref{tab:disease_definition}.


\subsubsection*{Data preprocessing}
For the NYU Langone and Long Island datasets, we converted the DICOM files into NIfTI format using MRIcroGL dcm2nii~\cite{li_first_2016}, standardizing the file format with those from the RSNA and CQ500 datasets. Given the variability in scan protocols, which can result in differing orientation, resolutions and slice thicknesses, we applied spatial normalization to transform the volume orientation to right-anterior-superior (RAS) angle and resample with bicubic interpolation to the isotropic resolution ratio of $(1.0, 1.0, 1.0)$ in the world coordinate system. This ensures uniform pixel spacing across all scans and axes. 

Head CT scans use Hounsfield Units (HU) to represent various tissue types, which span a broad range of values. To better capture tissue characteristics, we applied three windowing ranges, each emphasizing specific tissue types: (40, 80) for soft tissue, (80, 200) for contrast-enhanced tissues and blood vessels, and (600, 2800) for bone. We then stacked the values from each window, producing a 3-channel 3D volume that enhances the representation of these key tissues. Similar strategy has been applied in Chilamkurthy \textit{et al.}~\cite{CQ500}.

To ensure compatibility with model input requirements, we transformed each volume into the desired size. We first padded or cropped each volume to a size of (224, 224, 224), preserving the whole brain across all axes. Then for training, we applied data augmentations detailed in Supplementary Section \hyperref[sec:dataaug_details]{``Data Augmentation details’’}; for evaluation, we center-cropped the volumes to (192, 192, 192). Finally, we resized each volume to (96, 96, 96) as the input size for the model.


\subsection*{Model architecture}
\label{sec:model_architecture}
Numerous studies have demonstrated that ViT can effectively learn high-quality representations for 2D medical images at scale~\cite{chen_towards_2024, zhou2023foundation, MedSAM, Vorontsov2024, Azizi2023}. Our study extends this by exploring whether representations of 3D medical images (specifically head CT scans) can also be effectively learned at scale through the direct compression of 3D patches as model input. We employ the Vision Transformer (ViT)\cite{dosovitskiy2020vit} as the volume encoder for our foundation model, as well as for baseline comparisons in all experiments. Our model uses a ViT-Base architecture with an embedding dimension of 768, 12 self-attention layers, 12 heads, and feed-forward layers with a hidden size of 3072. We apply sine-cosine absolute positional encoding\cite{NIPS2017_3f5ee243} across all pre-training and fine-tuning stages.

For the 3D input volume, instead of creating 196 patches of size $16 \times 16$ from a $224 \times 224$ image as in standard 2D ViT, we segment $96 \times 96 \times 96$ 3D volumes into 512 patches of size $12 \times 12 \times 12$ for ViT input. This customized patch design considers the trade-off between performance and computational cost. As shown in Supplementary \Cref{fig:patches-ablation}, our model outperforms a version using 216 patches of size $16 \times 16 \times 16$, indicating that smaller, more numerous patches enhance model performance. This supports the importance of capturing fine-grained features in 3D medical imaging, consistent with prior findings~\cite{Tang_2022_CVPR, li2024well}. However, computational costs increase significantly with respect to $s$ ($s$ defined as patch size reducing factor), at a rate of $O(s^{6})$, due to the cubic growth of patch numbers in 3D and the quadratic growth in self-attention computation (\Cref{apd:self_attention_rate}). To balance performance with computational efficiency, we adopt 512 patches of $12 \times 12 \times 12$ as the optimal input size for ViT in our foundation model.



\subsection*{Self-supervised pretaining}
Self-Supervised Learning recently has been widely adopted as learning framework for building medical foundation models~\cite{chen_towards_2024, zhou2023foundation, Huang2023, azizi21big, Vorontsov2024}. While previous works mainly focus on directly applying existing self-supervised learning algorithms on 2D medical images, we explore how to effectively leverage these algorithms with 3D medical images. Specifically, we explore two main branches of self-supervised learning framework for building our 3D foundation model --- discriminative with self-distillation (DINO) and masked image modeling (MAE).

\paragraph{Self-Distillation Modelling (DINO)}
DINO\cite{caron2021emerging, oquab2023dinov2} is a self-supervised learning method shown promising and robust downstream evaluation performance in previous studies on different areas~\cite{chen_towards_2024, Vorontsov2024}. DINO uses a student-teacher framework for learning meaningful representations. Both student and teacher networks share the same model architecture, while the teacher’s parameters are updated using an exponential moving average of the student’s parameters. Each input image is augmented multiple times to create different views as student and teacher networks input. Specifically, we applied random global and local crops, random flips, shifts in intensity and contrasts, and Gaussian blurs for augmented views. Then the student’s output is trained to match the teacher’s output using a distillation loss, ensuring similar representations for different views of the same image. We pre-trained the ViT in the DINO framework for $1000$ epochs with batch size at $64$ per GPU and an AdamW~\cite{loshchilov2018decoupled} optimizer ($\beta_1=0.9, \beta_2=0.95$, $0.05$ weight decay). A base learning rate $3\times10^{-4}$ was applied combined with cosine scheduling and a linear warmup on the first $5$ epochs. During pre-training, two global augmentations and three local augmentations were applied to enable ViT to learn both global and local features of the head CT. Because small region of brain is likely to be dissimilar, we observed cropping too small brain regions would cause unstable model training by making the learning task to be too challenging. Therefore, we first resample the input images to $224\times224\times224$. Subsequently, we perform multi-scale cropping by extracting both global and local crops regions, ranging from $112\times112\times112$ to $224\times224\times224$ for global crops and from $64\times64\times64$ to $112\times112\times112$ for local crops. After the cropping, all cropped regions are resampled to $96\times96\times96$. For training on $100\%$ data, convergence on the performance for downstream tasks is observed at around $300$ epochs, which took around one week on four 80GB NVIDIA A100 GPUs. 

\paragraph{Masked Image Modeling (MAE)}
MAE~\cite{He2021MaskedAA} is another self-supervised learning method for vision tasks, inspired by masked language modeling in Natural Language Processing (NLP). MAE is trained to reconstruct randomly-masked patches via an encoder-decoder architecture, where the encoder processes visible patches of an image, while the decoder reconstructs the image from encoded patches and mask tokens. Specifically, we randomly masked the patches from each volume with a probability of 0.75. Mean squared error (MSE) loss is optimized to minimize the difference between the reconstructed volume and the original volume. We pre-trained the ViT in MAE framework for $400$ epochs with batch size at 64 per GPU and an AdamW~\cite{loshchilov2018decoupled} optimizer ($\beta_1=0.9, \beta_2=0.95$, $0.05$ weight decay). A base learning rate $1.5\times10^{-3}$ was applied combined with cosine scheduling and a linear warmup on the first $5\%$ steps,  For training on $100\%$ data, convergence is observed at around $250$ epochs, which took around 4 days on four 80GB NVIDIA A100 GPUs for MAE. Similar to DINO, MAE has shown success in learning robust representations in many previous works~\cite{ravi2024sam2, tong2022videomae, gupta2023siamese, zhou23self, huang2022masked, cong2022satmae, chen23masked}, including the studies on both 2D and 3D data.

We compared the performance on downstream tasks between two versions of foundation models pre-trained using DINO and MAE, as shown in \Cref{fig:scaling_law} and Supplementary \Cref{fig:probing_comparison,fig:probing-comparison-perpath,fig:probing-comparison-perpath-dino}. The results indicate that DINO consistently outperforms MAE across all datasets. Based on this finding, we selected the DINO-pre-trained model as our final foundation model.


\subsection*{Evaluation setting}
\paragraph{Baseline comparisons}
Since no prior foundation models have been specifically trained on 3D Head CT for direct comparison, we benchmark our model against Merlin~\cite{blankemeier2024merlinvisionlanguagefoundation} and Google CT Foundation model~\cite{yang2024advancingmultimodalmedicalcapabilities} to highlight the advantages of our domain-specific foundation model. Merlin is a 3D Abdomen CT foundation model pre-trained on vision-language pairs with contrastive learning~\cite{radford2021learningtransferablevisualmodels} and ICD code prediction task, where 6+ million images from 15,331 CTs, 1.8+ million diagnostic ICD codes from EHR, and 6+ million tokens from radiology reports are used. Different from our model architecture, Merlin used ResNet-152 ($\sim60.4$M Parameters) as vision model with reshaped image size of $224\times224\times160$. The performance comparison between our model and Merlin is shown in Supplementary \Cref{fig:radar-comparison-merlin}, where our model shows substantial improvement across most datasets and diseases. Google CT Foundation model is trained on a comprehensive private dataset comprising 527,078 CT studies with associated radiology reports from 430,772 patients. The model is first trained by Contrastive Captioning with CoCa~\cite{yu2022coca} on 2D medical images and then adapting to CT by training on series of CT slices with VideoCoCa~\cite{yan2023videococavideotextmodelingzeroshot}. The performance comparison between our model and Google CT Foundation model is shown in Supplementary \Cref{fig:probing-comparison-gemini}, where our model shows a consistent improvement across the board. We additionally show comparison of our model against model trained from scratch in \Cref{fig:overview} and Supplementary \Cref{fig:radar-comparison-merlin}, where the overall significantly improved performance shows the effectiveness of our pre-training strategies on 3D Head CT images.

\paragraph{Fine-tuning and Probing classification evaluation}
We assessed pre-trained model performance through full fine-tuning (updating all weights) and various probing methods (updating only the classification layers). For both approaches, images were normalized to isotropic spacing, transformed to three HU interval channels, and reshaped to $3\times96\times96\times96$. The entire transformed 3D image was then input into the ViT model for feature extraction, followed by an additional classification layer for downstream tasks. Probing utilized two strategies: linear probing, which adds a linear layer atop the ViT backbone, and attentive probing, which incorporates an attention layer. Attentive probing is chosen since MAE does not use \texttt{[CLS]} token as the learning objective. Linear probing only relies on \texttt{[CLS]} token to perform classification and attentive probing explores the interaction among all tokens~\cite{Chen2024}. Given the imbalances of downstream task labels, we randomly sampled a balanced subset from the training set per epoch, consisting of 5,000 samples (when fine-tuning on the NYU Langone, NYU Long Island, and RSNA datasets), and 500 samples when fine-tuning on CQ500. We trained all methods using the AdamW~\cite{loshchilov2018decoupled} optimizer with a cosine learning rate scheduler, a learning rate of $1\times10^{-5}$ for backbone and $1\times10^{-3}$ for classification layers, cross-entropy loss, and a maximum of 10 epochs. The main evaluation result with linear probing is shown in \Cref{fig:overview} with fine-tuning and probing comparison shown in Supplementary \Cref{fig:probing_comparison} for average performance across all diseases and Supplementary \Cref{fig:probing-comparison-perpath} for per disease performance. The result indicates that probing achieves performance levels close to full fine-tuning, underscoring the high quality of learned representations in our model.

For fine-tuning model from scratch, as we observe more unstable model performance from different hyper-parameters across different datasets, we perform hyper-parameters sweep across different setting and report the best performance model. The sweeping hyper-parameters are lr=\{1e-3, 1e-4, 1e-5\}, weight decay=\{0.01, 0.05, 0.0001, 0.00001\}, epochs=\{10, 15, 30, 50\}, optimizer=\{SGD, Adam, AdamW\}.

\paragraph{Few-shots classification evaluation}
In order to evaluate the effectiveness of our model under scare label conditions, we applied few-shots learning where each class is only sampled $K$-times. Specifically, we chose $K=8, 16, 32, 64, 128, 256$, where the data is sampled such that positive and negative samples equal to $K$ for each disease. Few-shot training was performed using full fine-tuning with the same hyper-parameter settings. While we also attempted some other commonly used few-shots classification methods such as k-nearest neighbors (KNN), Simple Shots~\cite{want19simpleshot} and Prototypical Networks~\cite{jake17proto}, we did not observe performance improvement on our datasets over full fine-tuning. The main evaluation for few-shots classification is present in \Cref{fig:fewshot}, where we observed our model can already reach performance close to full fine-tuning with only $K=256$ samples. This demonstrates the effectiveness of our model under scare data training regime.

\subsection*{Visual Interpretation}
\label{sec:visual_interpretation}
Self-attention enables the Vision Transformer (ViT) to integrate information across the entire volume, even in its lowest layers. To analyze the relationships among different patches within the CT volumes, we calculate the average spatial distance over which information is integrated, using the attention weights. 

Let $\mathbf{A}^{(l,h)} \in \mathbb{R}^{N \times N}$  represent the attention weight matrix for the $h$th attention head in the $l$th layer of ViT and $N$ is the number of patches in a CT volume. $d(i, j)$ denotes the spatial distance between patch $i$ and patch $j$ within the 3D volume. The attention distance for each patch $i$ is computed as a weighted average distance to other patches, based on the attention weights:
\begin{equation}
    \centering
    D_i^{(l, h)} = \sum_{j=1}^{N} A^{(l, h)}_{ij}  d(i, j)
\end{equation}

We visualize the average attention distances across all heads and layers for every patch in the volume in \Cref{fig:attention_interpretation}. This “attention distance” serves as an estimate of the ViT’s receptive field within the CT volumes, indicating the regions of the brain that the model focuses on. This visualization helps illustrate how the model integrates information across spatial areas to capture meaningful patterns within the volume.

\subsection*{Statistical analysis}

In each experiment, we report the mean and confidence interval, calculated by bootstrapping the held-out validation set 100 times. For few-shot learning, where model variance is also influenced by the specific training data samples, we repeated the training and evaluation process five times with randomly sampled training data, reporting the mean and confidence interval of the resulting metrics. For all statistical significance (p-values) reported in this study, we used a two-sided paired permutation test with $1,000$ permutations to assess the performance difference of two compared models.

\subsection*{Computing Hardware Software}
All experiments are performed under Python (v3.8.11), PyTorch (v2.4.1), CUDA (12.1) and MONAI (v1.2.0). We extend ViT, MAE, DINO implementation from original their corresponding repositories (\url{https://github.com/facebookresearch/mae}, \url{https://github.com/facebookresearch/dino}) to match our need for 3D CT image encoding. For comparison with Merlin~\cite{blankemeier2024merlinvisionlanguagefoundation}, we integrated their original model weight checkpoints and model backbone code (\url{https://github.com/louisblankemeier/merlin}) to our downstream fine-tuning code base. ResNet50-3D~\cite{hara3dcnns} from (\url{https://github.com/kenshohara/3D-ResNets-PyTorch/tree/master}) is integrated to our code base for evaluation. All plots and figures were created by Matplotlib (v0.1.6) and Seaborn (v0.13.2). All downstream experiments were conducted on single 80 GB NVIDIA A100 GPU (graphics processing unit). All pre-training experiments were conducted on four 80 GB NVIDIA A100 GPUs.




\section*{Data availability}
The internal clinical data involved in the study is unavailable due to privacy concerns and institutional policy. Public dataset RSNA is available from \url{https://www.kaggle.com/competitions/rsna-intracranial-hemorrhage-detection}. Public dataset CQ500 is available from \url{https://www.kaggle.com/datasets/crawford/qureai-headct}.  The original data is provided as DICOM files. We converted each scan from DICOM to NIfTI files and removed the scans with missing slices for creating 3D imaging datasets in our evaluation. We use all slice thickness scan protocols in each scan (e.g. thin, plain thin, and plain scan) for CQ500, hence providing a more exhaustive evaluation on our model adaptability on different slice thickness for scan.

\section*{Code availability}
The code for pre-training, fine-tuning and evaluation of the foundation model is available on \url{https://github.com/NYUMedML/headCT_foundation}. Due to the possibility of inferring patient face from headCT data, the model weights are only available upon request after signing institutional agreement. Requests for model weight should be sent to the corresponding author and the NYU Langone Data Sharing Strategy Board (DSSB) Committee (DataSharing@nyulangone.org).

\section*{Acknowledgements}
W.Z., H.H., L.C., A.V.M. and N.R. were supported by the National Institute On Aging of the National Institutes of Health under Award R01AG085617. W.Z. H.H., B.Y. and L.C. received partial support from NSF Award 1922658. N.R. and A.V.M. were also partially supported by the National Institute On Aging of the National Institutes of Health under Awards R01AG079175 and P30AG066512.

% This must be in the first 5 lines to tell arXiv to use pdfLaTeX, which is strongly recommended.
\pdfoutput=1
% In particular, the hyperref package requires pdfLaTeX in order to break URLs across lines.

\documentclass[11pt]{article}

% Change "review" to "final" to generate the final (sometimes called camera-ready) version.
% Change to "preprint" to generate a non-anonymous version with page numbers.
\usepackage{acl}

% Standard package includes
\usepackage{times}
\usepackage{latexsym}

% Draw tables
\usepackage{booktabs}
\usepackage{multirow}
\usepackage{xcolor}
\usepackage{colortbl}
\usepackage{array} 
\usepackage{amsmath}

\newcolumntype{C}{>{\centering\arraybackslash}p{0.07\textwidth}}
% For proper rendering and hyphenation of words containing Latin characters (including in bib files)
\usepackage[T1]{fontenc}
% For Vietnamese characters
% \usepackage[T5]{fontenc}
% See https://www.latex-project.org/help/documentation/encguide.pdf for other character sets
% This assumes your files are encoded as UTF8
\usepackage[utf8]{inputenc}

% This is not strictly necessary, and may be commented out,
% but it will improve the layout of the manuscript,
% and will typically save some space.
\usepackage{microtype}
\DeclareMathOperator*{\argmax}{arg\,max}
% This is also not strictly necessary, and may be commented out.
% However, it will improve the aesthetics of text in
% the typewriter font.
\usepackage{inconsolata}

%Including images in your LaTeX document requires adding
%additional package(s)
\usepackage{graphicx}
% If the title and author information does not fit in the area allocated, uncomment the following
%
%\setlength\titlebox{<dim>}
%
% and set <dim> to something 5cm or larger.

\title{Wi-Chat: Large Language Model Powered Wi-Fi Sensing}

% Author information can be set in various styles:
% For several authors from the same institution:
% \author{Author 1 \and ... \and Author n \\
%         Address line \\ ... \\ Address line}
% if the names do not fit well on one line use
%         Author 1 \\ {\bf Author 2} \\ ... \\ {\bf Author n} \\
% For authors from different institutions:
% \author{Author 1 \\ Address line \\  ... \\ Address line
%         \And  ... \And
%         Author n \\ Address line \\ ... \\ Address line}
% To start a separate ``row'' of authors use \AND, as in
% \author{Author 1 \\ Address line \\  ... \\ Address line
%         \AND
%         Author 2 \\ Address line \\ ... \\ Address line \And
%         Author 3 \\ Address line \\ ... \\ Address line}

% \author{First Author \\
%   Affiliation / Address line 1 \\
%   Affiliation / Address line 2 \\
%   Affiliation / Address line 3 \\
%   \texttt{email@domain} \\\And
%   Second Author \\
%   Affiliation / Address line 1 \\
%   Affiliation / Address line 2 \\
%   Affiliation / Address line 3 \\
%   \texttt{email@domain} \\}
% \author{Haohan Yuan \qquad Haopeng Zhang\thanks{corresponding author} \\ 
%   ALOHA Lab, University of Hawaii at Manoa \\
%   % Affiliation / Address line 2 \\
%   % Affiliation / Address line 3 \\
%   \texttt{\{haohany,haopengz\}@hawaii.edu}}
  
\author{
{Haopeng Zhang$\dag$\thanks{These authors contributed equally to this work.}, Yili Ren$\ddagger$\footnotemark[1], Haohan Yuan$\dag$, Jingzhe Zhang$\ddagger$, Yitong Shen$\ddagger$} \\
ALOHA Lab, University of Hawaii at Manoa$\dag$, University of South Florida$\ddagger$ \\
\{haopengz, haohany\}@hawaii.edu\\
\{yiliren, jingzhe, shen202\}@usf.edu\\}



  
%\author{
%  \textbf{First Author\textsuperscript{1}},
%  \textbf{Second Author\textsuperscript{1,2}},
%  \textbf{Third T. Author\textsuperscript{1}},
%  \textbf{Fourth Author\textsuperscript{1}},
%\\
%  \textbf{Fifth Author\textsuperscript{1,2}},
%  \textbf{Sixth Author\textsuperscript{1}},
%  \textbf{Seventh Author\textsuperscript{1}},
%  \textbf{Eighth Author \textsuperscript{1,2,3,4}},
%\\
%  \textbf{Ninth Author\textsuperscript{1}},
%  \textbf{Tenth Author\textsuperscript{1}},
%  \textbf{Eleventh E. Author\textsuperscript{1,2,3,4,5}},
%  \textbf{Twelfth Author\textsuperscript{1}},
%\\
%  \textbf{Thirteenth Author\textsuperscript{3}},
%  \textbf{Fourteenth F. Author\textsuperscript{2,4}},
%  \textbf{Fifteenth Author\textsuperscript{1}},
%  \textbf{Sixteenth Author\textsuperscript{1}},
%\\
%  \textbf{Seventeenth S. Author\textsuperscript{4,5}},
%  \textbf{Eighteenth Author\textsuperscript{3,4}},
%  \textbf{Nineteenth N. Author\textsuperscript{2,5}},
%  \textbf{Twentieth Author\textsuperscript{1}}
%\\
%\\
%  \textsuperscript{1}Affiliation 1,
%  \textsuperscript{2}Affiliation 2,
%  \textsuperscript{3}Affiliation 3,
%  \textsuperscript{4}Affiliation 4,
%  \textsuperscript{5}Affiliation 5
%\\
%  \small{
%    \textbf{Correspondence:} \href{mailto:email@domain}{email@domain}
%  }
%}

\begin{document}
\maketitle
\begin{abstract}
Recent advancements in Large Language Models (LLMs) have demonstrated remarkable capabilities across diverse tasks. However, their potential to integrate physical model knowledge for real-world signal interpretation remains largely unexplored. In this work, we introduce Wi-Chat, the first LLM-powered Wi-Fi-based human activity recognition system. We demonstrate that LLMs can process raw Wi-Fi signals and infer human activities by incorporating Wi-Fi sensing principles into prompts. Our approach leverages physical model insights to guide LLMs in interpreting Channel State Information (CSI) data without traditional signal processing techniques. Through experiments on real-world Wi-Fi datasets, we show that LLMs exhibit strong reasoning capabilities, achieving zero-shot activity recognition. These findings highlight a new paradigm for Wi-Fi sensing, expanding LLM applications beyond conventional language tasks and enhancing the accessibility of wireless sensing for real-world deployments.
\end{abstract}

\section{Introduction}

In today’s rapidly evolving digital landscape, the transformative power of web technologies has redefined not only how services are delivered but also how complex tasks are approached. Web-based systems have become increasingly prevalent in risk control across various domains. This widespread adoption is due their accessibility, scalability, and ability to remotely connect various types of users. For example, these systems are used for process safety management in industry~\cite{kannan2016web}, safety risk early warning in urban construction~\cite{ding2013development}, and safe monitoring of infrastructural systems~\cite{repetto2018web}. Within these web-based risk management systems, the source search problem presents a huge challenge. Source search refers to the task of identifying the origin of a risky event, such as a gas leak and the emission point of toxic substances. This source search capability is crucial for effective risk management and decision-making.

Traditional approaches to implementing source search capabilities into the web systems often rely on solely algorithmic solutions~\cite{ristic2016study}. These methods, while relatively straightforward to implement, often struggle to achieve acceptable performances due to algorithmic local optima and complex unknown environments~\cite{zhao2020searching}. More recently, web crowdsourcing has emerged as a promising alternative for tackling the source search problem by incorporating human efforts in these web systems on-the-fly~\cite{zhao2024user}. This approach outsources the task of addressing issues encountered during the source search process to human workers, leveraging their capabilities to enhance system performance.

These solutions often employ a human-AI collaborative way~\cite{zhao2023leveraging} where algorithms handle exploration-exploitation and report the encountered problems while human workers resolve complex decision-making bottlenecks to help the algorithms getting rid of local deadlocks~\cite{zhao2022crowd}. Although effective, this paradigm suffers from two inherent limitations: increased operational costs from continuous human intervention, and slow response times of human workers due to sequential decision-making. These challenges motivate our investigation into developing autonomous systems that preserve human-like reasoning capabilities while reducing dependency on massive crowdsourced labor.

Furthermore, recent advancements in large language models (LLMs)~\cite{chang2024survey} and multi-modal LLMs (MLLMs)~\cite{huang2023chatgpt} have unveiled promising avenues for addressing these challenges. One clear opportunity involves the seamless integration of visual understanding and linguistic reasoning for robust decision-making in search tasks. However, whether large models-assisted source search is really effective and efficient for improving the current source search algorithms~\cite{ji2022source} remains unknown. \textit{To address the research gap, we are particularly interested in answering the following two research questions in this work:}

\textbf{\textit{RQ1: }}How can source search capabilities be integrated into web-based systems to support decision-making in time-sensitive risk management scenarios? 
% \sq{I mention ``time-sensitive'' here because I feel like we shall say something about the response time -- LLM has to be faster than humans}

\textbf{\textit{RQ2: }}How can MLLMs and LLMs enhance the effectiveness and efficiency of existing source search algorithms? 

% \textit{\textbf{RQ2:}} To what extent does the performance of large models-assisted search align with or approach the effectiveness of human-AI collaborative search? 

To answer the research questions, we propose a novel framework called Auto-\
S$^2$earch (\textbf{Auto}nomous \textbf{S}ource \textbf{Search}) and implement a prototype system that leverages advanced web technologies to simulate real-world conditions for zero-shot source search. Unlike traditional methods that rely on pre-defined heuristics or extensive human intervention, AutoS$^2$earch employs a carefully designed prompt that encapsulates human rationales, thereby guiding the MLLM to generate coherent and accurate scene descriptions from visual inputs about four directional choices. Based on these language-based descriptions, the LLM is enabled to determine the optimal directional choice through chain-of-thought (CoT) reasoning. Comprehensive empirical validation demonstrates that AutoS$^2$-\ 
earch achieves a success rate of 95–98\%, closely approaching the performance of human-AI collaborative search across 20 benchmark scenarios~\cite{zhao2023leveraging}. 

Our work indicates that the role of humans in future web crowdsourcing tasks may evolve from executors to validators or supervisors. Furthermore, incorporating explanations of LLM decisions into web-based system interfaces has the potential to help humans enhance task performance in risk control.






\section{Related Work}
\label{sec:relatedworks}

% \begin{table*}[t]
% \centering 
% \renewcommand\arraystretch{0.98}
% \fontsize{8}{10}\selectfont \setlength{\tabcolsep}{0.4em}
% \begin{tabular}{@{}lc|cc|cc|cc@{}}
% \toprule
% \textbf{Methods}           & \begin{tabular}[c]{@{}c@{}}\textbf{Training}\\ \textbf{Paradigm}\end{tabular} & \begin{tabular}[c]{@{}c@{}}\textbf{$\#$ PT Data}\\ \textbf{(Tokens)}\end{tabular} & \begin{tabular}[c]{@{}c@{}}\textbf{$\#$ IFT Data}\\ \textbf{(Samples)}\end{tabular} & \textbf{Code}  & \begin{tabular}[c]{@{}c@{}}\textbf{Natural}\\ \textbf{Language}\end{tabular} & \begin{tabular}[c]{@{}c@{}}\textbf{Action}\\ \textbf{Trajectories}\end{tabular} & \begin{tabular}[c]{@{}c@{}}\textbf{API}\\ \textbf{Documentation}\end{tabular}\\ \midrule 
% NexusRaven~\citep{srinivasan2023nexusraven} & IFT & - & - & \textcolor{green}{\CheckmarkBold} & \textcolor{green}{\CheckmarkBold} &\textcolor{red}{\XSolidBrush}&\textcolor{red}{\XSolidBrush}\\
% AgentInstruct~\citep{zeng2023agenttuning} & IFT & - & 2k & \textcolor{green}{\CheckmarkBold} & \textcolor{green}{\CheckmarkBold} &\textcolor{red}{\XSolidBrush}&\textcolor{red}{\XSolidBrush} \\
% AgentEvol~\citep{xi2024agentgym} & IFT & - & 14.5k & \textcolor{green}{\CheckmarkBold} & \textcolor{green}{\CheckmarkBold} &\textcolor{green}{\CheckmarkBold}&\textcolor{red}{\XSolidBrush} \\
% Gorilla~\citep{patil2023gorilla}& IFT & - & 16k & \textcolor{green}{\CheckmarkBold} & \textcolor{green}{\CheckmarkBold} &\textcolor{red}{\XSolidBrush}&\textcolor{green}{\CheckmarkBold}\\
% OpenFunctions-v2~\citep{patil2023gorilla} & IFT & - & 65k & \textcolor{green}{\CheckmarkBold} & \textcolor{green}{\CheckmarkBold} &\textcolor{red}{\XSolidBrush}&\textcolor{green}{\CheckmarkBold}\\
% LAM~\citep{zhang2024agentohana} & IFT & - & 42.6k & \textcolor{green}{\CheckmarkBold} & \textcolor{green}{\CheckmarkBold} &\textcolor{green}{\CheckmarkBold}&\textcolor{red}{\XSolidBrush} \\
% xLAM~\citep{liu2024apigen} & IFT & - & 60k & \textcolor{green}{\CheckmarkBold} & \textcolor{green}{\CheckmarkBold} &\textcolor{green}{\CheckmarkBold}&\textcolor{red}{\XSolidBrush} \\\midrule
% LEMUR~\citep{xu2024lemur} & PT & 90B & 300k & \textcolor{green}{\CheckmarkBold} & \textcolor{green}{\CheckmarkBold} &\textcolor{green}{\CheckmarkBold}&\textcolor{red}{\XSolidBrush}\\
% \rowcolor{teal!12} \method & PT & 103B & 95k & \textcolor{green}{\CheckmarkBold} & \textcolor{green}{\CheckmarkBold} & \textcolor{green}{\CheckmarkBold} & \textcolor{green}{\CheckmarkBold} \\
% \bottomrule
% \end{tabular}
% \caption{Summary of existing tuning- and pretraining-based LLM agents with their training sample sizes. "PT" and "IFT" denote "Pre-Training" and "Instruction Fine-Tuning", respectively. }
% \label{tab:related}
% \end{table*}

\begin{table*}[ht]
\begin{threeparttable}
\centering 
\renewcommand\arraystretch{0.98}
\fontsize{7}{9}\selectfont \setlength{\tabcolsep}{0.2em}
\begin{tabular}{@{}l|c|c|ccc|cc|cc|cccc@{}}
\toprule
\textbf{Methods} & \textbf{Datasets}           & \begin{tabular}[c]{@{}c@{}}\textbf{Training}\\ \textbf{Paradigm}\end{tabular} & \begin{tabular}[c]{@{}c@{}}\textbf{\# PT Data}\\ \textbf{(Tokens)}\end{tabular} & \begin{tabular}[c]{@{}c@{}}\textbf{\# IFT Data}\\ \textbf{(Samples)}\end{tabular} & \textbf{\# APIs} & \textbf{Code}  & \begin{tabular}[c]{@{}c@{}}\textbf{Nat.}\\ \textbf{Lang.}\end{tabular} & \begin{tabular}[c]{@{}c@{}}\textbf{Action}\\ \textbf{Traj.}\end{tabular} & \begin{tabular}[c]{@{}c@{}}\textbf{API}\\ \textbf{Doc.}\end{tabular} & \begin{tabular}[c]{@{}c@{}}\textbf{Func.}\\ \textbf{Call}\end{tabular} & \begin{tabular}[c]{@{}c@{}}\textbf{Multi.}\\ \textbf{Step}\end{tabular}  & \begin{tabular}[c]{@{}c@{}}\textbf{Plan}\\ \textbf{Refine}\end{tabular}  & \begin{tabular}[c]{@{}c@{}}\textbf{Multi.}\\ \textbf{Turn}\end{tabular}\\ \midrule 
\multicolumn{13}{l}{\emph{Instruction Finetuning-based LLM Agents for Intrinsic Reasoning}}  \\ \midrule
FireAct~\cite{chen2023fireact} & FireAct & IFT & - & 2.1K & 10 & \textcolor{red}{\XSolidBrush} &\textcolor{green}{\CheckmarkBold} &\textcolor{green}{\CheckmarkBold}  & \textcolor{red}{\XSolidBrush} &\textcolor{green}{\CheckmarkBold} & \textcolor{red}{\XSolidBrush} &\textcolor{green}{\CheckmarkBold} & \textcolor{red}{\XSolidBrush} \\
ToolAlpaca~\cite{tang2023toolalpaca} & ToolAlpaca & IFT & - & 4.0K & 400 & \textcolor{red}{\XSolidBrush} &\textcolor{green}{\CheckmarkBold} &\textcolor{green}{\CheckmarkBold} & \textcolor{red}{\XSolidBrush} &\textcolor{green}{\CheckmarkBold} & \textcolor{red}{\XSolidBrush}  &\textcolor{green}{\CheckmarkBold} & \textcolor{red}{\XSolidBrush}  \\
ToolLLaMA~\cite{qin2023toolllm} & ToolBench & IFT & - & 12.7K & 16,464 & \textcolor{red}{\XSolidBrush} &\textcolor{green}{\CheckmarkBold} &\textcolor{green}{\CheckmarkBold} &\textcolor{red}{\XSolidBrush} &\textcolor{green}{\CheckmarkBold}&\textcolor{green}{\CheckmarkBold}&\textcolor{green}{\CheckmarkBold} &\textcolor{green}{\CheckmarkBold}\\
AgentEvol~\citep{xi2024agentgym} & AgentTraj-L & IFT & - & 14.5K & 24 &\textcolor{red}{\XSolidBrush} & \textcolor{green}{\CheckmarkBold} &\textcolor{green}{\CheckmarkBold}&\textcolor{red}{\XSolidBrush} &\textcolor{green}{\CheckmarkBold}&\textcolor{red}{\XSolidBrush} &\textcolor{red}{\XSolidBrush} &\textcolor{green}{\CheckmarkBold}\\
Lumos~\cite{yin2024agent} & Lumos & IFT  & - & 20.0K & 16 &\textcolor{red}{\XSolidBrush} & \textcolor{green}{\CheckmarkBold} & \textcolor{green}{\CheckmarkBold} &\textcolor{red}{\XSolidBrush} & \textcolor{green}{\CheckmarkBold} & \textcolor{green}{\CheckmarkBold} &\textcolor{red}{\XSolidBrush} & \textcolor{green}{\CheckmarkBold}\\
Agent-FLAN~\cite{chen2024agent} & Agent-FLAN & IFT & - & 24.7K & 20 &\textcolor{red}{\XSolidBrush} & \textcolor{green}{\CheckmarkBold} & \textcolor{green}{\CheckmarkBold} &\textcolor{red}{\XSolidBrush} & \textcolor{green}{\CheckmarkBold}& \textcolor{green}{\CheckmarkBold}&\textcolor{red}{\XSolidBrush} & \textcolor{green}{\CheckmarkBold}\\
AgentTuning~\citep{zeng2023agenttuning} & AgentInstruct & IFT & - & 35.0K & - &\textcolor{red}{\XSolidBrush} & \textcolor{green}{\CheckmarkBold} & \textcolor{green}{\CheckmarkBold} &\textcolor{red}{\XSolidBrush} & \textcolor{green}{\CheckmarkBold} &\textcolor{red}{\XSolidBrush} &\textcolor{red}{\XSolidBrush} & \textcolor{green}{\CheckmarkBold}\\\midrule
\multicolumn{13}{l}{\emph{Instruction Finetuning-based LLM Agents for Function Calling}} \\\midrule
NexusRaven~\citep{srinivasan2023nexusraven} & NexusRaven & IFT & - & - & 116 & \textcolor{green}{\CheckmarkBold} & \textcolor{green}{\CheckmarkBold}  & \textcolor{green}{\CheckmarkBold} &\textcolor{red}{\XSolidBrush} & \textcolor{green}{\CheckmarkBold} &\textcolor{red}{\XSolidBrush} &\textcolor{red}{\XSolidBrush}&\textcolor{red}{\XSolidBrush}\\
Gorilla~\citep{patil2023gorilla} & Gorilla & IFT & - & 16.0K & 1,645 & \textcolor{green}{\CheckmarkBold} &\textcolor{red}{\XSolidBrush} &\textcolor{red}{\XSolidBrush}&\textcolor{green}{\CheckmarkBold} &\textcolor{green}{\CheckmarkBold} &\textcolor{red}{\XSolidBrush} &\textcolor{red}{\XSolidBrush} &\textcolor{red}{\XSolidBrush}\\
OpenFunctions-v2~\citep{patil2023gorilla} & OpenFunctions-v2 & IFT & - & 65.0K & - & \textcolor{green}{\CheckmarkBold} & \textcolor{green}{\CheckmarkBold} &\textcolor{red}{\XSolidBrush} &\textcolor{green}{\CheckmarkBold} &\textcolor{green}{\CheckmarkBold} &\textcolor{red}{\XSolidBrush} &\textcolor{red}{\XSolidBrush} &\textcolor{red}{\XSolidBrush}\\
API Pack~\cite{guo2024api} & API Pack & IFT & - & 1.1M & 11,213 &\textcolor{green}{\CheckmarkBold} &\textcolor{red}{\XSolidBrush} &\textcolor{green}{\CheckmarkBold} &\textcolor{red}{\XSolidBrush} &\textcolor{green}{\CheckmarkBold} &\textcolor{red}{\XSolidBrush}&\textcolor{red}{\XSolidBrush}&\textcolor{red}{\XSolidBrush}\\ 
LAM~\citep{zhang2024agentohana} & AgentOhana & IFT & - & 42.6K & - & \textcolor{green}{\CheckmarkBold} & \textcolor{green}{\CheckmarkBold} &\textcolor{green}{\CheckmarkBold}&\textcolor{red}{\XSolidBrush} &\textcolor{green}{\CheckmarkBold}&\textcolor{red}{\XSolidBrush}&\textcolor{green}{\CheckmarkBold}&\textcolor{green}{\CheckmarkBold}\\
xLAM~\citep{liu2024apigen} & APIGen & IFT & - & 60.0K & 3,673 & \textcolor{green}{\CheckmarkBold} & \textcolor{green}{\CheckmarkBold} &\textcolor{green}{\CheckmarkBold}&\textcolor{red}{\XSolidBrush} &\textcolor{green}{\CheckmarkBold}&\textcolor{red}{\XSolidBrush}&\textcolor{green}{\CheckmarkBold}&\textcolor{green}{\CheckmarkBold}\\\midrule
\multicolumn{13}{l}{\emph{Pretraining-based LLM Agents}}  \\\midrule
% LEMUR~\citep{xu2024lemur} & PT & 90B & 300.0K & - & \textcolor{green}{\CheckmarkBold} & \textcolor{green}{\CheckmarkBold} &\textcolor{green}{\CheckmarkBold}&\textcolor{red}{\XSolidBrush} & \textcolor{red}{\XSolidBrush} &\textcolor{green}{\CheckmarkBold} &\textcolor{red}{\XSolidBrush}&\textcolor{red}{\XSolidBrush}\\
\rowcolor{teal!12} \method & \dataset & PT & 103B & 95.0K  & 76,537  & \textcolor{green}{\CheckmarkBold} & \textcolor{green}{\CheckmarkBold} & \textcolor{green}{\CheckmarkBold} & \textcolor{green}{\CheckmarkBold} & \textcolor{green}{\CheckmarkBold} & \textcolor{green}{\CheckmarkBold} & \textcolor{green}{\CheckmarkBold} & \textcolor{green}{\CheckmarkBold}\\
\bottomrule
\end{tabular}
% \begin{tablenotes}
%     \item $^*$ In addition, the StarCoder-API can offer 4.77M more APIs.
% \end{tablenotes}
\caption{Summary of existing instruction finetuning-based LLM agents for intrinsic reasoning and function calling, along with their training resources and sample sizes. "PT" and "IFT" denote "Pre-Training" and "Instruction Fine-Tuning", respectively.}
\vspace{-2ex}
\label{tab:related}
\end{threeparttable}
\end{table*}

\noindent \textbf{Prompting-based LLM Agents.} Due to the lack of agent-specific pre-training corpus, existing LLM agents rely on either prompt engineering~\cite{hsieh2023tool,lu2024chameleon,yao2022react,wang2023voyager} or instruction fine-tuning~\cite{chen2023fireact,zeng2023agenttuning} to understand human instructions, decompose high-level tasks, generate grounded plans, and execute multi-step actions. 
However, prompting-based methods mainly depend on the capabilities of backbone LLMs (usually commercial LLMs), failing to introduce new knowledge and struggling to generalize to unseen tasks~\cite{sun2024adaplanner,zhuang2023toolchain}. 

\noindent \textbf{Instruction Finetuning-based LLM Agents.} Considering the extensive diversity of APIs and the complexity of multi-tool instructions, tool learning inherently presents greater challenges than natural language tasks, such as text generation~\cite{qin2023toolllm}.
Post-training techniques focus more on instruction following and aligning output with specific formats~\cite{patil2023gorilla,hao2024toolkengpt,qin2023toolllm,schick2024toolformer}, rather than fundamentally improving model knowledge or capabilities. 
Moreover, heavy fine-tuning can hinder generalization or even degrade performance in non-agent use cases, potentially suppressing the original base model capabilities~\cite{ghosh2024a}.

\noindent \textbf{Pretraining-based LLM Agents.} While pre-training serves as an essential alternative, prior works~\cite{nijkamp2023codegen,roziere2023code,xu2024lemur,patil2023gorilla} have primarily focused on improving task-specific capabilities (\eg, code generation) instead of general-domain LLM agents, due to single-source, uni-type, small-scale, and poor-quality pre-training data. 
Existing tool documentation data for agent training either lacks diverse real-world APIs~\cite{patil2023gorilla, tang2023toolalpaca} or is constrained to single-tool or single-round tool execution. 
Furthermore, trajectory data mostly imitate expert behavior or follow function-calling rules with inferior planning and reasoning, failing to fully elicit LLMs' capabilities and handle complex instructions~\cite{qin2023toolllm}. 
Given a wide range of candidate API functions, each comprising various function names and parameters available at every planning step, identifying globally optimal solutions and generalizing across tasks remains highly challenging.



\section{Preliminaries}
\label{Preliminaries}
\begin{figure*}[t]
    \centering
    \includegraphics[width=0.95\linewidth]{fig/HealthGPT_Framework.png}
    \caption{The \ourmethod{} architecture integrates hierarchical visual perception and H-LoRA, employing a task-specific hard router to select visual features and H-LoRA plugins, ultimately generating outputs with an autoregressive manner.}
    \label{fig:architecture}
\end{figure*}
\noindent\textbf{Large Vision-Language Models.} 
The input to a LVLM typically consists of an image $x^{\text{img}}$ and a discrete text sequence $x^{\text{txt}}$. The visual encoder $\mathcal{E}^{\text{img}}$ converts the input image $x^{\text{img}}$ into a sequence of visual tokens $\mathcal{V} = [v_i]_{i=1}^{N_v}$, while the text sequence $x^{\text{txt}}$ is mapped into a sequence of text tokens $\mathcal{T} = [t_i]_{i=1}^{N_t}$ using an embedding function $\mathcal{E}^{\text{txt}}$. The LLM $\mathcal{M_\text{LLM}}(\cdot|\theta)$ models the joint probability of the token sequence $\mathcal{U} = \{\mathcal{V},\mathcal{T}\}$, which is expressed as:
\begin{equation}
    P_\theta(R | \mathcal{U}) = \prod_{i=1}^{N_r} P_\theta(r_i | \{\mathcal{U}, r_{<i}\}),
\end{equation}
where $R = [r_i]_{i=1}^{N_r}$ is the text response sequence. The LVLM iteratively generates the next token $r_i$ based on $r_{<i}$. The optimization objective is to minimize the cross-entropy loss of the response $\mathcal{R}$.
% \begin{equation}
%     \mathcal{L}_{\text{VLM}} = \mathbb{E}_{R|\mathcal{U}}\left[-\log P_\theta(R | \mathcal{U})\right]
% \end{equation}
It is worth noting that most LVLMs adopt a design paradigm based on ViT, alignment adapters, and pre-trained LLMs\cite{liu2023llava,liu2024improved}, enabling quick adaptation to downstream tasks.


\noindent\textbf{VQGAN.}
VQGAN~\cite{esser2021taming} employs latent space compression and indexing mechanisms to effectively learn a complete discrete representation of images. VQGAN first maps the input image $x^{\text{img}}$ to a latent representation $z = \mathcal{E}(x)$ through a encoder $\mathcal{E}$. Then, the latent representation is quantized using a codebook $\mathcal{Z} = \{z_k\}_{k=1}^K$, generating a discrete index sequence $\mathcal{I} = [i_m]_{m=1}^N$, where $i_m \in \mathcal{Z}$ represents the quantized code index:
\begin{equation}
    \mathcal{I} = \text{Quantize}(z|\mathcal{Z}) = \arg\min_{z_k \in \mathcal{Z}} \| z - z_k \|_2.
\end{equation}
In our approach, the discrete index sequence $\mathcal{I}$ serves as a supervisory signal for the generation task, enabling the model to predict the index sequence $\hat{\mathcal{I}}$ from input conditions such as text or other modality signals.  
Finally, the predicted index sequence $\hat{\mathcal{I}}$ is upsampled by the VQGAN decoder $G$, generating the high-quality image $\hat{x}^\text{img} = G(\hat{\mathcal{I}})$.



\noindent\textbf{Low Rank Adaptation.} 
LoRA\cite{hu2021lora} effectively captures the characteristics of downstream tasks by introducing low-rank adapters. The core idea is to decompose the bypass weight matrix $\Delta W\in\mathbb{R}^{d^{\text{in}} \times d^{\text{out}}}$ into two low-rank matrices $ \{A \in \mathbb{R}^{d^{\text{in}} \times r}, B \in \mathbb{R}^{r \times d^{\text{out}}} \}$, where $ r \ll \min\{d^{\text{in}}, d^{\text{out}}\} $, significantly reducing learnable parameters. The output with the LoRA adapter for the input $x$ is then given by:
\begin{equation}
    h = x W_0 + \alpha x \Delta W/r = x W_0 + \alpha xAB/r,
\end{equation}
where matrix $ A $ is initialized with a Gaussian distribution, while the matrix $ B $ is initialized as a zero matrix. The scaling factor $ \alpha/r $ controls the impact of $ \Delta W $ on the model.

\section{HealthGPT}
\label{Method}


\subsection{Unified Autoregressive Generation.}  
% As shown in Figure~\ref{fig:architecture}, 
\ourmethod{} (Figure~\ref{fig:architecture}) utilizes a discrete token representation that covers both text and visual outputs, unifying visual comprehension and generation as an autoregressive task. 
For comprehension, $\mathcal{M}_\text{llm}$ receives the input joint sequence $\mathcal{U}$ and outputs a series of text token $\mathcal{R} = [r_1, r_2, \dots, r_{N_r}]$, where $r_i \in \mathcal{V}_{\text{txt}}$, and $\mathcal{V}_{\text{txt}}$ represents the LLM's vocabulary:
\begin{equation}
    P_\theta(\mathcal{R} \mid \mathcal{U}) = \prod_{i=1}^{N_r} P_\theta(r_i \mid \mathcal{U}, r_{<i}).
\end{equation}
For generation, $\mathcal{M}_\text{llm}$ first receives a special start token $\langle \text{START\_IMG} \rangle$, then generates a series of tokens corresponding to the VQGAN indices $\mathcal{I} = [i_1, i_2, \dots, i_{N_i}]$, where $i_j \in \mathcal{V}_{\text{vq}}$, and $\mathcal{V}_{\text{vq}}$ represents the index range of VQGAN. Upon completion of generation, the LLM outputs an end token $\langle \text{END\_IMG} \rangle$:
\begin{equation}
    P_\theta(\mathcal{I} \mid \mathcal{U}) = \prod_{j=1}^{N_i} P_\theta(i_j \mid \mathcal{U}, i_{<j}).
\end{equation}
Finally, the generated index sequence $\mathcal{I}$ is fed into the decoder $G$, which reconstructs the target image $\hat{x}^{\text{img}} = G(\mathcal{I})$.

\subsection{Hierarchical Visual Perception}  
Given the differences in visual perception between comprehension and generation tasks—where the former focuses on abstract semantics and the latter emphasizes complete semantics—we employ ViT to compress the image into discrete visual tokens at multiple hierarchical levels.
Specifically, the image is converted into a series of features $\{f_1, f_2, \dots, f_L\}$ as it passes through $L$ ViT blocks.

To address the needs of various tasks, the hidden states are divided into two types: (i) \textit{Concrete-grained features} $\mathcal{F}^{\text{Con}} = \{f_1, f_2, \dots, f_k\}, k < L$, derived from the shallower layers of ViT, containing sufficient global features, suitable for generation tasks; 
(ii) \textit{Abstract-grained features} $\mathcal{F}^{\text{Abs}} = \{f_{k+1}, f_{k+2}, \dots, f_L\}$, derived from the deeper layers of ViT, which contain abstract semantic information closer to the text space, suitable for comprehension tasks.

The task type $T$ (comprehension or generation) determines which set of features is selected as the input for the downstream large language model:
\begin{equation}
    \mathcal{F}^{\text{img}}_T =
    \begin{cases}
        \mathcal{F}^{\text{Con}}, & \text{if } T = \text{generation task} \\
        \mathcal{F}^{\text{Abs}}, & \text{if } T = \text{comprehension task}
    \end{cases}
\end{equation}
We integrate the image features $\mathcal{F}^{\text{img}}_T$ and text features $\mathcal{T}$ into a joint sequence through simple concatenation, which is then fed into the LLM $\mathcal{M}_{\text{llm}}$ for autoregressive generation.
% :
% \begin{equation}
%     \mathcal{R} = \mathcal{M}_{\text{llm}}(\mathcal{U}|\theta), \quad \mathcal{U} = [\mathcal{F}^{\text{img}}_T; \mathcal{T}]
% \end{equation}
\subsection{Heterogeneous Knowledge Adaptation}
We devise H-LoRA, which stores heterogeneous knowledge from comprehension and generation tasks in separate modules and dynamically routes to extract task-relevant knowledge from these modules. 
At the task level, for each task type $ T $, we dynamically assign a dedicated H-LoRA submodule $ \theta^T $, which is expressed as:
\begin{equation}
    \mathcal{R} = \mathcal{M}_\text{LLM}(\mathcal{U}|\theta, \theta^T), \quad \theta^T = \{A^T, B^T, \mathcal{R}^T_\text{outer}\}.
\end{equation}
At the feature level for a single task, H-LoRA integrates the idea of Mixture of Experts (MoE)~\cite{masoudnia2014mixture} and designs an efficient matrix merging and routing weight allocation mechanism, thus avoiding the significant computational delay introduced by matrix splitting in existing MoELoRA~\cite{luo2024moelora}. Specifically, we first merge the low-rank matrices (rank = r) of $ k $ LoRA experts into a unified matrix:
\begin{equation}
    \mathbf{A}^{\text{merged}}, \mathbf{B}^{\text{merged}} = \text{Concat}(\{A_i\}_1^k), \text{Concat}(\{B_i\}_1^k),
\end{equation}
where $ \mathbf{A}^{\text{merged}} \in \mathbb{R}^{d^\text{in} \times rk} $ and $ \mathbf{B}^{\text{merged}} \in \mathbb{R}^{rk \times d^\text{out}} $. The $k$-dimension routing layer generates expert weights $ \mathcal{W} \in \mathbb{R}^{\text{token\_num} \times k} $ based on the input hidden state $ x $, and these are expanded to $ \mathbb{R}^{\text{token\_num} \times rk} $ as follows:
\begin{equation}
    \mathcal{W}^\text{expanded} = \alpha k \mathcal{W} / r \otimes \mathbf{1}_r,
\end{equation}
where $ \otimes $ denotes the replication operation.
The overall output of H-LoRA is computed as:
\begin{equation}
    \mathcal{O}^\text{H-LoRA} = (x \mathbf{A}^{\text{merged}} \odot \mathcal{W}^\text{expanded}) \mathbf{B}^{\text{merged}},
\end{equation}
where $ \odot $ represents element-wise multiplication. Finally, the output of H-LoRA is added to the frozen pre-trained weights to produce the final output:
\begin{equation}
    \mathcal{O} = x W_0 + \mathcal{O}^\text{H-LoRA}.
\end{equation}
% In summary, H-LoRA is a task-based dynamic PEFT method that achieves high efficiency in single-task fine-tuning.

\subsection{Training Pipeline}

\begin{figure}[t]
    \centering
    \hspace{-4mm}
    \includegraphics[width=0.94\linewidth]{fig/data.pdf}
    \caption{Data statistics of \texttt{VL-Health}. }
    \label{fig:data}
\end{figure}
\noindent \textbf{1st Stage: Multi-modal Alignment.} 
In the first stage, we design separate visual adapters and H-LoRA submodules for medical unified tasks. For the medical comprehension task, we train abstract-grained visual adapters using high-quality image-text pairs to align visual embeddings with textual embeddings, thereby enabling the model to accurately describe medical visual content. During this process, the pre-trained LLM and its corresponding H-LoRA submodules remain frozen. In contrast, the medical generation task requires training concrete-grained adapters and H-LoRA submodules while keeping the LLM frozen. Meanwhile, we extend the textual vocabulary to include multimodal tokens, enabling the support of additional VQGAN vector quantization indices. The model trains on image-VQ pairs, endowing the pre-trained LLM with the capability for image reconstruction. This design ensures pixel-level consistency of pre- and post-LVLM. The processes establish the initial alignment between the LLM’s outputs and the visual inputs.

\noindent \textbf{2nd Stage: Heterogeneous H-LoRA Plugin Adaptation.}  
The submodules of H-LoRA share the word embedding layer and output head but may encounter issues such as bias and scale inconsistencies during training across different tasks. To ensure that the multiple H-LoRA plugins seamlessly interface with the LLMs and form a unified base, we fine-tune the word embedding layer and output head using a small amount of mixed data to maintain consistency in the model weights. Specifically, during this stage, all H-LoRA submodules for different tasks are kept frozen, with only the word embedding layer and output head being optimized. Through this stage, the model accumulates foundational knowledge for unified tasks by adapting H-LoRA plugins.

\begin{table*}[!t]
\centering
\caption{Comparison of \ourmethod{} with other LVLMs and unified multi-modal models on medical visual comprehension tasks. \textbf{Bold} and \underline{underlined} text indicates the best performance and second-best performance, respectively.}
\resizebox{\textwidth}{!}{
\begin{tabular}{c|lcc|cccccccc|c}
\toprule
\rowcolor[HTML]{E9F3FE} &  &  &  & \multicolumn{2}{c}{\textbf{VQA-RAD \textuparrow}} & \multicolumn{2}{c}{\textbf{SLAKE \textuparrow}} & \multicolumn{2}{c}{\textbf{PathVQA \textuparrow}} &  &  &  \\ 
\cline{5-10}
\rowcolor[HTML]{E9F3FE}\multirow{-2}{*}{\textbf{Type}} & \multirow{-2}{*}{\textbf{Model}} & \multirow{-2}{*}{\textbf{\# Params}} & \multirow{-2}{*}{\makecell{\textbf{Medical} \\ \textbf{LVLM}}} & \textbf{close} & \textbf{all} & \textbf{close} & \textbf{all} & \textbf{close} & \textbf{all} & \multirow{-2}{*}{\makecell{\textbf{MMMU} \\ \textbf{-Med}}\textuparrow} & \multirow{-2}{*}{\textbf{OMVQA}\textuparrow} & \multirow{-2}{*}{\textbf{Avg. \textuparrow}} \\ 
\midrule \midrule
\multirow{9}{*}{\textbf{Comp. Only}} 
& Med-Flamingo & 8.3B & \Large \ding{51} & 58.6 & 43.0 & 47.0 & 25.5 & 61.9 & 31.3 & 28.7 & 34.9 & 41.4 \\
& LLaVA-Med & 7B & \Large \ding{51} & 60.2 & 48.1 & 58.4 & 44.8 & 62.3 & 35.7 & 30.0 & 41.3 & 47.6 \\
& HuatuoGPT-Vision & 7B & \Large \ding{51} & 66.9 & 53.0 & 59.8 & 49.1 & 52.9 & 32.0 & 42.0 & 50.0 & 50.7 \\
& BLIP-2 & 6.7B & \Large \ding{55} & 43.4 & 36.8 & 41.6 & 35.3 & 48.5 & 28.8 & 27.3 & 26.9 & 36.1 \\
& LLaVA-v1.5 & 7B & \Large \ding{55} & 51.8 & 42.8 & 37.1 & 37.7 & 53.5 & 31.4 & 32.7 & 44.7 & 41.5 \\
& InstructBLIP & 7B & \Large \ding{55} & 61.0 & 44.8 & 66.8 & 43.3 & 56.0 & 32.3 & 25.3 & 29.0 & 44.8 \\
& Yi-VL & 6B & \Large \ding{55} & 52.6 & 42.1 & 52.4 & 38.4 & 54.9 & 30.9 & 38.0 & 50.2 & 44.9 \\
& InternVL2 & 8B & \Large \ding{55} & 64.9 & 49.0 & 66.6 & 50.1 & 60.0 & 31.9 & \underline{43.3} & 54.5 & 52.5\\
& Llama-3.2 & 11B & \Large \ding{55} & 68.9 & 45.5 & 72.4 & 52.1 & 62.8 & 33.6 & 39.3 & 63.2 & 54.7 \\
\midrule
\multirow{5}{*}{\textbf{Comp. \& Gen.}} 
& Show-o & 1.3B & \Large \ding{55} & 50.6 & 33.9 & 31.5 & 17.9 & 52.9 & 28.2 & 22.7 & 45.7 & 42.6 \\
& Unified-IO 2 & 7B & \Large \ding{55} & 46.2 & 32.6 & 35.9 & 21.9 & 52.5 & 27.0 & 25.3 & 33.0 & 33.8 \\
& Janus & 1.3B & \Large \ding{55} & 70.9 & 52.8 & 34.7 & 26.9 & 51.9 & 27.9 & 30.0 & 26.8 & 33.5 \\
& \cellcolor[HTML]{DAE0FB}HealthGPT-M3 & \cellcolor[HTML]{DAE0FB}3.8B & \cellcolor[HTML]{DAE0FB}\Large \ding{51} & \cellcolor[HTML]{DAE0FB}\underline{73.7} & \cellcolor[HTML]{DAE0FB}\underline{55.9} & \cellcolor[HTML]{DAE0FB}\underline{74.6} & \cellcolor[HTML]{DAE0FB}\underline{56.4} & \cellcolor[HTML]{DAE0FB}\underline{78.7} & \cellcolor[HTML]{DAE0FB}\underline{39.7} & \cellcolor[HTML]{DAE0FB}\underline{43.3} & \cellcolor[HTML]{DAE0FB}\underline{68.5} & \cellcolor[HTML]{DAE0FB}\underline{61.3} \\
& \cellcolor[HTML]{DAE0FB}HealthGPT-L14 & \cellcolor[HTML]{DAE0FB}14B & \cellcolor[HTML]{DAE0FB}\Large \ding{51} & \cellcolor[HTML]{DAE0FB}\textbf{77.7} & \cellcolor[HTML]{DAE0FB}\textbf{58.3} & \cellcolor[HTML]{DAE0FB}\textbf{76.4} & \cellcolor[HTML]{DAE0FB}\textbf{64.5} & \cellcolor[HTML]{DAE0FB}\textbf{85.9} & \cellcolor[HTML]{DAE0FB}\textbf{44.4} & \cellcolor[HTML]{DAE0FB}\textbf{49.2} & \cellcolor[HTML]{DAE0FB}\textbf{74.4} & \cellcolor[HTML]{DAE0FB}\textbf{66.4} \\
\bottomrule
\end{tabular}
}
\label{tab:results}
\end{table*}
\begin{table*}[ht]
    \centering
    \caption{The experimental results for the four modality conversion tasks.}
    \resizebox{\textwidth}{!}{
    \begin{tabular}{l|ccc|ccc|ccc|ccc}
        \toprule
        \rowcolor[HTML]{E9F3FE} & \multicolumn{3}{c}{\textbf{CT to MRI (Brain)}} & \multicolumn{3}{c}{\textbf{CT to MRI (Pelvis)}} & \multicolumn{3}{c}{\textbf{MRI to CT (Brain)}} & \multicolumn{3}{c}{\textbf{MRI to CT (Pelvis)}} \\
        \cline{2-13}
        \rowcolor[HTML]{E9F3FE}\multirow{-2}{*}{\textbf{Model}}& \textbf{SSIM $\uparrow$} & \textbf{PSNR $\uparrow$} & \textbf{MSE $\downarrow$} & \textbf{SSIM $\uparrow$} & \textbf{PSNR $\uparrow$} & \textbf{MSE $\downarrow$} & \textbf{SSIM $\uparrow$} & \textbf{PSNR $\uparrow$} & \textbf{MSE $\downarrow$} & \textbf{SSIM $\uparrow$} & \textbf{PSNR $\uparrow$} & \textbf{MSE $\downarrow$} \\
        \midrule \midrule
        pix2pix & 71.09 & 32.65 & 36.85 & 59.17 & 31.02 & 51.91 & 78.79 & 33.85 & 28.33 & 72.31 & 32.98 & 36.19 \\
        CycleGAN & 54.76 & 32.23 & 40.56 & 54.54 & 30.77 & 55.00 & 63.75 & 31.02 & 52.78 & 50.54 & 29.89 & 67.78 \\
        BBDM & {71.69} & {32.91} & {34.44} & 57.37 & 31.37 & 48.06 & \textbf{86.40} & 34.12 & 26.61 & {79.26} & 33.15 & 33.60 \\
        Vmanba & 69.54 & 32.67 & 36.42 & {63.01} & {31.47} & {46.99} & 79.63 & 34.12 & 26.49 & 77.45 & 33.53 & 31.85 \\
        DiffMa & 71.47 & 32.74 & 35.77 & 62.56 & 31.43 & 47.38 & 79.00 & {34.13} & {26.45} & 78.53 & {33.68} & {30.51} \\
        \rowcolor[HTML]{DAE0FB}HealthGPT-M3 & \underline{79.38} & \underline{33.03} & \underline{33.48} & \underline{71.81} & \underline{31.83} & \underline{43.45} & {85.06} & \textbf{34.40} & \textbf{25.49} & \underline{84.23} & \textbf{34.29} & \textbf{27.99} \\
        \rowcolor[HTML]{DAE0FB}HealthGPT-L14 & \textbf{79.73} & \textbf{33.10} & \textbf{32.96} & \textbf{71.92} & \textbf{31.87} & \textbf{43.09} & \underline{85.31} & \underline{34.29} & \underline{26.20} & \textbf{84.96} & \underline{34.14} & \underline{28.13} \\
        \bottomrule
    \end{tabular}
    }
    \label{tab:conversion}
\end{table*}

\noindent \textbf{3rd Stage: Visual Instruction Fine-Tuning.}  
In the third stage, we introduce additional task-specific data to further optimize the model and enhance its adaptability to downstream tasks such as medical visual comprehension (e.g., medical QA, medical dialogues, and report generation) or generation tasks (e.g., super-resolution, denoising, and modality conversion). Notably, by this stage, the word embedding layer and output head have been fine-tuned, only the H-LoRA modules and adapter modules need to be trained. This strategy significantly improves the model's adaptability and flexibility across different tasks.


\section{Experiment}
\label{s:experiment}

\subsection{Data Description}
We evaluate our method on FI~\cite{you2016building}, Twitter\_LDL~\cite{yang2017learning} and Artphoto~\cite{machajdik2010affective}.
FI is a public dataset built from Flickr and Instagram, with 23,308 images and eight emotion categories, namely \textit{amusement}, \textit{anger}, \textit{awe},  \textit{contentment}, \textit{disgust}, \textit{excitement},  \textit{fear}, and \textit{sadness}. 
% Since images in FI are all copyrighted by law, some images are corrupted now, so we remove these samples and retain 21,828 images.
% T4SA contains images from Twitter, which are classified into three categories: \textit{positive}, \textit{neutral}, and \textit{negative}. In this paper, we adopt the base version of B-T4SA, which contains 470,586 images and provides text descriptions of the corresponding tweets.
Twitter\_LDL contains 10,045 images from Twitter, with the same eight categories as the FI dataset.
% 。
For these two datasets, they are randomly split into 80\%
training and 20\% testing set.
Artphoto contains 806 artistic photos from the DeviantArt website, which we use to further evaluate the zero-shot capability of our model.
% on the small-scale dataset.
% We construct and publicly release the first image sentiment analysis dataset containing metadata.
% 。

% Based on these datasets, we are the first to construct and publicly release metadata-enhanced image sentiment analysis datasets. These datasets include scenes, tags, descriptions, and corresponding confidence scores, and are available at this link for future research purposes.


% 
\begin{table}[t]
\centering
% \begin{center}
\caption{Overall performance of different models on FI and Twitter\_LDL datasets.}
\label{tab:cap1}
% \resizebox{\linewidth}{!}
{
\begin{tabular}{l|c|c|c|c}
\hline
\multirow{2}{*}{\textbf{Model}} & \multicolumn{2}{c|}{\textbf{FI}}  & \multicolumn{2}{c}{\textbf{Twitter\_LDL}} \\ \cline{2-5} 
  & \textbf{Accuracy} & \textbf{F1} & \textbf{Accuracy} & \textbf{F1}  \\ \hline
% (\rownumber)~AlexNet~\cite{krizhevsky2017imagenet}  & 58.13\% & 56.35\%  & 56.24\%& 55.02\%  \\ 
% (\rownumber)~VGG16~\cite{simonyan2014very}  & 63.75\%& 63.08\%  & 59.34\%& 59.02\%  \\ 
(\rownumber)~ResNet101~\cite{he2016deep} & 66.16\%& 65.56\%  & 62.02\% & 61.34\%  \\ 
(\rownumber)~CDA~\cite{han2023boosting} & 66.71\%& 65.37\%  & 64.14\% & 62.85\%  \\ 
(\rownumber)~CECCN~\cite{ruan2024color} & 67.96\%& 66.74\%  & 64.59\%& 64.72\% \\ 
(\rownumber)~EmoVIT~\cite{xie2024emovit} & 68.09\%& 67.45\%  & 63.12\% & 61.97\%  \\ 
(\rownumber)~ComLDL~\cite{zhang2022compound} & 68.83\%& 67.28\%  & 65.29\% & 63.12\%  \\ 
(\rownumber)~WSDEN~\cite{li2023weakly} & 69.78\%& 69.61\%  & 67.04\% & 65.49\% \\ 
(\rownumber)~ECWA~\cite{deng2021emotion} & 70.87\%& 69.08\%  & 67.81\% & 66.87\%  \\ 
(\rownumber)~EECon~\cite{yang2023exploiting} & 71.13\%& 68.34\%  & 64.27\%& 63.16\%  \\ 
(\rownumber)~MAM~\cite{zhang2024affective} & 71.44\%  & 70.83\% & 67.18\%  & 65.01\%\\ 
(\rownumber)~TGCA-PVT~\cite{chen2024tgca}   & 73.05\%  & 71.46\% & 69.87\%  & 68.32\% \\ 
(\rownumber)~OEAN~\cite{zhang2024object}   & 73.40\%  & 72.63\% & 70.52\%  & 69.47\% \\ \hline
(\rownumber)~\shortname  & \textbf{79.48\%} & \textbf{79.22\%} & \textbf{74.12\%} & \textbf{73.09\%} \\ \hline
\end{tabular}
}
\vspace{-6mm}
% \end{center}
\end{table}
% 

\subsection{Experiment Setting}
% \subsubsection{Model Setting.}
% 
\textbf{Model Setting:}
For feature representation, we set $k=10$ to select object tags, and adopt clip-vit-base-patch32 as the pre-trained model for unified feature representation.
Moreover, we empirically set $(d_e, d_h, d_k, d_s) = (512, 128, 16, 64)$, and set the classification class $L$ to 8.

% 

\textbf{Training Setting:}
To initialize the model, we set all weights such as $\boldsymbol{W}$ following the truncated normal distribution, and use AdamW optimizer with the learning rate of $1 \times 10^{-4}$.
% warmup scheduler of cosine, warmup steps of 2000.
Furthermore, we set the batch size to 32 and the epoch of the training process to 200.
During the implementation, we utilize \textit{PyTorch} to build our entire model.
% , and our project codes are publicly available at https://github.com/zzmyrep/MESN.
% Our project codes as well as data are all publicly available on GitHub\footnote{https://github.com/zzmyrep/KBCEN}.
% Code is available at \href{https://github.com/zzmyrep/KBCEN}{https://github.com/zzmyrep/KBCEN}.

\textbf{Evaluation Metrics:}
Following~\cite{zhang2024affective, chen2024tgca, zhang2024object}, we adopt \textit{accuracy} and \textit{F1} as our evaluation metrics to measure the performance of different methods for image sentiment analysis. 



\subsection{Experiment Result}
% We compare our model against the following baselines: AlexNet~\cite{krizhevsky2017imagenet}, VGG16~\cite{simonyan2014very}, ResNet101~\cite{he2016deep}, CECCN~\cite{ruan2024color}, EmoVIT~\cite{xie2024emovit}, WSCNet~\cite{yang2018weakly}, ECWA~\cite{deng2021emotion}, EECon~\cite{yang2023exploiting}, MAM~\cite{zhang2024affective} and TGCA-PVT~\cite{chen2024tgca}, and the overall results are summarized in Table~\ref{tab:cap1}.
We compare our model against several baselines, and the overall results are summarized in Table~\ref{tab:cap1}.
We observe that our model achieves the best performance in both accuracy and F1 metrics, significantly outperforming the previous models. 
This superior performance is mainly attributed to our effective utilization of metadata to enhance image sentiment analysis, as well as the exceptional capability of the unified sentiment transformer framework we developed. These results strongly demonstrate that our proposed method can bring encouraging performance for image sentiment analysis.

\setcounter{magicrownumbers}{0} 
\begin{table}[t]
\begin{center}
\caption{Ablation study of~\shortname~on FI dataset.} 
% \vspace{1mm}
\label{tab:cap2}
\resizebox{.9\linewidth}{!}
{
\begin{tabular}{lcc}
  \hline
  \textbf{Model} & \textbf{Accuracy} & \textbf{F1} \\
  \hline
  (\rownumber)~Ours (w/o vision) & 65.72\% & 64.54\% \\
  (\rownumber)~Ours (w/o text description) & 74.05\% & 72.58\% \\
  (\rownumber)~Ours (w/o object tag) & 77.45\% & 76.84\% \\
  (\rownumber)~Ours (w/o scene tag) & 78.47\% & 78.21\% \\
  \hline
  (\rownumber)~Ours (w/o unified embedding) & 76.41\% & 76.23\% \\
  (\rownumber)~Ours (w/o adaptive learning) & 76.83\% & 76.56\% \\
  (\rownumber)~Ours (w/o cross-modal fusion) & 76.85\% & 76.49\% \\
  \hline
  (\rownumber)~Ours  & \textbf{79.48\%} & \textbf{79.22\%} \\
  \hline
\end{tabular}
}
\end{center}
\vspace{-5mm}
\end{table}


\begin{figure}[t]
\centering
% \vspace{-2mm}
\includegraphics[width=0.42\textwidth]{fig/2dvisual-linux4-paper2.pdf}
\caption{Visualization of feature distribution on eight categories before (left) and after (right) model processing.}
% 
\label{fig:visualization}
\vspace{-5mm}
\end{figure}

\subsection{Ablation Performance}
In this subsection, we conduct an ablation study to examine which component is really important for performance improvement. The results are reported in Table~\ref{tab:cap2}.

For information utilization, we observe a significant decline in model performance when visual features are removed. Additionally, the performance of \shortname~decreases when different metadata are removed separately, which means that text description, object tag, and scene tag are all critical for image sentiment analysis.
Recalling the model architecture, we separately remove transformer layers of the unified representation module, the adaptive learning module, and the cross-modal fusion module, replacing them with MLPs of the same parameter scale.
In this way, we can observe varying degrees of decline in model performance, indicating that these modules are indispensable for our model to achieve better performance.

\subsection{Visualization}
% 


% % 开始使用minipage进行左右排列
% \begin{minipage}[t]{0.45\textwidth}  % 子图1宽度为45%
%     \centering
%     \includegraphics[width=\textwidth]{2dvisual.pdf}  % 插入图片
%     \captionof{figure}{Visualization of feature distribution.}  % 使用captionof添加图片标题
%     \label{fig:visualization}
% \end{minipage}


% \begin{figure}[t]
% \centering
% \vspace{-2mm}
% \includegraphics[width=0.45\textwidth]{fig/2dvisual.pdf}
% \caption{Visualization of feature distribution.}
% \label{fig:visualization}
% % \vspace{-4mm}
% \end{figure}

% \begin{figure}[t]
% \centering
% \vspace{-2mm}
% \includegraphics[width=0.45\textwidth]{fig/2dvisual-linux3-paper.pdf}
% \caption{Visualization of feature distribution.}
% \label{fig:visualization}
% % \vspace{-4mm}
% \end{figure}



\begin{figure}[tbp]   
\vspace{-4mm}
  \centering            
  \subfloat[Depth of adaptive learning layers]   
  {
    \label{fig:subfig1}\includegraphics[width=0.22\textwidth]{fig/fig_sensitivity-a5}
  }
  \subfloat[Depth of fusion layers]
  {
    % \label{fig:subfig2}\includegraphics[width=0.22\textwidth]{fig/fig_sensitivity-b2}
    \label{fig:subfig2}\includegraphics[width=0.22\textwidth]{fig/fig_sensitivity-b2-num.pdf}
  }
  \caption{Sensitivity study of \shortname~on different depth. }   
  \label{fig:fig_sensitivity}  
\vspace{-2mm}
\end{figure}

% \begin{figure}[htbp]
% \centerline{\includegraphics{2dvisual.pdf}}
% \caption{Visualization of feature distribution.}
% \label{fig:visualization}
% \end{figure}

% In Fig.~\ref{fig:visualization}, we use t-SNE~\cite{van2008visualizing} to reduce the dimension of data features for visualization, Figure in left represents the metadata features before model processing, the features are obtained by embedding through the CLIP model, and figure in right shows the features of the data after model processing, it can be observed that after the model processing, the data with different label categories fall in different regions in the space, therefore, we can conclude that the Therefore, we can conclude that the model can effectively utilize the information contained in the metadata and use it to guide the model for classification.

In Fig.~\ref{fig:visualization}, we use t-SNE~\cite{van2008visualizing} to reduce the dimension of data features for visualization.
The left figure shows metadata features before being processed by our model (\textit{i.e.}, embedded by CLIP), while the right shows the distribution of features after being processed by our model.
We can observe that after the model processing, data with the same label are closer to each other, while others are farther away.
Therefore, it shows that the model can effectively utilize the information contained in the metadata and use it to guide the classification process.

\subsection{Sensitivity Analysis}
% 
In this subsection, we conduct a sensitivity analysis to figure out the effect of different depth settings of adaptive learning layers and fusion layers. 
% In this subsection, we conduct a sensitivity analysis to figure out the effect of different depth settings on the model. 
% Fig.~\ref{fig:fig_sensitivity} presents the effect of different depth settings of adaptive learning layers and fusion layers. 
Taking Fig.~\ref{fig:fig_sensitivity} (a) as an example, the model performance improves with increasing depth, reaching the best performance at a depth of 4.
% Taking Fig.~\ref{fig:fig_sensitivity} (a) as an example, the performance of \shortname~improves with the increase of depth at first, reaching the best performance at a depth of 4.
When the depth continues to increase, the accuracy decreases to varying degrees.
Similar results can be observed in Fig.~\ref{fig:fig_sensitivity} (b).
Therefore, we set their depths to 4 and 6 respectively to achieve the best results.

% Through our experiments, we can observe that the effect of modifying these hyperparameters on the results of the experiments is very weak, and the surface model is not sensitive to the hyperparameters.


\subsection{Zero-shot Capability}
% 

% (1)~GCH~\cite{2010Analyzing} & 21.78\% & (5)~RA-DLNet~\cite{2020A} & 34.01\% \\ \hline
% (2)~WSCNet~\cite{2019WSCNet}  & 30.25\% & (6)~CECCN~\cite{ruan2024color} & 43.83\% \\ \hline
% (3)~PCNN~\cite{2015Robust} & 31.68\%  & (7)~EmoVIT~\cite{xie2024emovit} & 44.90\% \\ \hline
% (4)~AR~\cite{2018Visual} & 32.67\% & (8)~Ours (Zero-shot) & 47.83\% \\ \hline


\begin{table}[t]
\centering
\caption{Zero-shot capability of \shortname.}
\label{tab:cap3}
\resizebox{1\linewidth}{!}
{
\begin{tabular}{lc|lc}
\hline
\textbf{Model} & \textbf{Accuracy} & \textbf{Model} & \textbf{Accuracy} \\ \hline
(1)~WSCNet~\cite{2019WSCNet}  & 30.25\% & (5)~MAM~\cite{zhang2024affective} & 39.56\%  \\ \hline
(2)~AR~\cite{2018Visual} & 32.67\% & (6)~CECCN~\cite{ruan2024color} & 43.83\% \\ \hline
(3)~RA-DLNet~\cite{2020A} & 34.01\%  & (7)~EmoVIT~\cite{xie2024emovit} & 44.90\% \\ \hline
(4)~CDA~\cite{han2023boosting} & 38.64\% & (8)~Ours (Zero-shot) & 47.83\% \\ \hline
\end{tabular}
}
\vspace{-5mm}
\end{table}

% We use the model trained on the FI dataset to test on the artphoto dataset to verify the model's generalization ability as well as robustness to other distributed datasets.
% We can observe that the MESN model shows strong competitiveness in terms of accuracy when compared to other trained models, which suggests that the model has a good generalization ability in the OOD task.

To validate the model's generalization ability and robustness to other distributed datasets, we directly test the model trained on the FI dataset, without training on Artphoto. 
% As observed in Table 3, compared to other models trained on Artphoto, we achieve highly competitive zero-shot performance, indicating that the model has good generalization ability in out-of-distribution tasks.
From Table~\ref{tab:cap3}, we can observe that compared with other models trained on Artphoto, we achieve competitive zero-shot performance, which shows that the model has good generalization ability in out-of-distribution tasks.


%%%%%%%%%%%%
%  E2E     %
%%%%%%%%%%%%


\section{Conclusion}
In this paper, we introduced Wi-Chat, the first LLM-powered Wi-Fi-based human activity recognition system that integrates the reasoning capabilities of large language models with the sensing potential of wireless signals. Our experimental results on a self-collected Wi-Fi CSI dataset demonstrate the promising potential of LLMs in enabling zero-shot Wi-Fi sensing. These findings suggest a new paradigm for human activity recognition that does not rely on extensive labeled data. We hope future research will build upon this direction, further exploring the applications of LLMs in signal processing domains such as IoT, mobile sensing, and radar-based systems.

\section*{Limitations}
While our work represents the first attempt to leverage LLMs for processing Wi-Fi signals, it is a preliminary study focused on a relatively simple task: Wi-Fi-based human activity recognition. This choice allows us to explore the feasibility of LLMs in wireless sensing but also comes with certain limitations.

Our approach primarily evaluates zero-shot performance, which, while promising, may still lag behind traditional supervised learning methods in highly complex or fine-grained recognition tasks. Besides, our study is limited to a controlled environment with a self-collected dataset, and the generalizability of LLMs to diverse real-world scenarios with varying Wi-Fi conditions, environmental interference, and device heterogeneity remains an open question.

Additionally, we have yet to explore the full potential of LLMs in more advanced Wi-Fi sensing applications, such as fine-grained gesture recognition, occupancy detection, and passive health monitoring. Future work should investigate the scalability of LLM-based approaches, their robustness to domain shifts, and their integration with multimodal sensing techniques in broader IoT applications.


% Bibliography entries for the entire Anthology, followed by custom entries
%\bibliography{anthology,custom}
% Custom bibliography entries only
\bibliography{main}
\newpage
\appendix

\section{Experiment prompts}
\label{sec:prompt}
The prompts used in the LLM experiments are shown in the following Table~\ref{tab:prompts}.

\definecolor{titlecolor}{rgb}{0.9, 0.5, 0.1}
\definecolor{anscolor}{rgb}{0.2, 0.5, 0.8}
\definecolor{labelcolor}{HTML}{48a07e}
\begin{table*}[h]
	\centering
	
 % \vspace{-0.2cm}
	
	\begin{center}
		\begin{tikzpicture}[
				chatbox_inner/.style={rectangle, rounded corners, opacity=0, text opacity=1, font=\sffamily\scriptsize, text width=5in, text height=9pt, inner xsep=6pt, inner ysep=6pt},
				chatbox_prompt_inner/.style={chatbox_inner, align=flush left, xshift=0pt, text height=11pt},
				chatbox_user_inner/.style={chatbox_inner, align=flush left, xshift=0pt},
				chatbox_gpt_inner/.style={chatbox_inner, align=flush left, xshift=0pt},
				chatbox/.style={chatbox_inner, draw=black!25, fill=gray!7, opacity=1, text opacity=0},
				chatbox_prompt/.style={chatbox, align=flush left, fill=gray!1.5, draw=black!30, text height=10pt},
				chatbox_user/.style={chatbox, align=flush left},
				chatbox_gpt/.style={chatbox, align=flush left},
				chatbox2/.style={chatbox_gpt, fill=green!25},
				chatbox3/.style={chatbox_gpt, fill=red!20, draw=black!20},
				chatbox4/.style={chatbox_gpt, fill=yellow!30},
				labelbox/.style={rectangle, rounded corners, draw=black!50, font=\sffamily\scriptsize\bfseries, fill=gray!5, inner sep=3pt},
			]
											
			\node[chatbox_user] (q1) {
				\textbf{System prompt}
				\newline
				\newline
				You are a helpful and precise assistant for segmenting and labeling sentences. We would like to request your help on curating a dataset for entity-level hallucination detection.
				\newline \newline
                We will give you a machine generated biography and a list of checked facts about the biography. Each fact consists of a sentence and a label (True/False). Please do the following process. First, breaking down the biography into words. Second, by referring to the provided list of facts, merging some broken down words in the previous step to form meaningful entities. For example, ``strategic thinking'' should be one entity instead of two. Third, according to the labels in the list of facts, labeling each entity as True or False. Specifically, for facts that share a similar sentence structure (\eg, \textit{``He was born on Mach 9, 1941.''} (\texttt{True}) and \textit{``He was born in Ramos Mejia.''} (\texttt{False})), please first assign labels to entities that differ across atomic facts. For example, first labeling ``Mach 9, 1941'' (\texttt{True}) and ``Ramos Mejia'' (\texttt{False}) in the above case. For those entities that are the same across atomic facts (\eg, ``was born'') or are neutral (\eg, ``he,'' ``in,'' and ``on''), please label them as \texttt{True}. For the cases that there is no atomic fact that shares the same sentence structure, please identify the most informative entities in the sentence and label them with the same label as the atomic fact while treating the rest of the entities as \texttt{True}. In the end, output the entities and labels in the following format:
                \begin{itemize}[nosep]
                    \item Entity 1 (Label 1)
                    \item Entity 2 (Label 2)
                    \item ...
                    \item Entity N (Label N)
                \end{itemize}
                % \newline \newline
                Here are two examples:
                \newline\newline
                \textbf{[Example 1]}
                \newline
                [The start of the biography]
                \newline
                \textcolor{titlecolor}{Marianne McAndrew is an American actress and singer, born on November 21, 1942, in Cleveland, Ohio. She began her acting career in the late 1960s, appearing in various television shows and films.}
                \newline
                [The end of the biography]
                \newline \newline
                [The start of the list of checked facts]
                \newline
                \textcolor{anscolor}{[Marianne McAndrew is an American. (False); Marianne McAndrew is an actress. (True); Marianne McAndrew is a singer. (False); Marianne McAndrew was born on November 21, 1942. (False); Marianne McAndrew was born in Cleveland, Ohio. (False); She began her acting career in the late 1960s. (True); She has appeared in various television shows. (True); She has appeared in various films. (True)]}
                \newline
                [The end of the list of checked facts]
                \newline \newline
                [The start of the ideal output]
                \newline
                \textcolor{labelcolor}{[Marianne McAndrew (True); is (True); an (True); American (False); actress (True); and (True); singer (False); , (True); born (True); on (True); November 21, 1942 (False); , (True); in (True); Cleveland, Ohio (False); . (True); She (True); began (True); her (True); acting career (True); in (True); the late 1960s (True); , (True); appearing (True); in (True); various (True); television shows (True); and (True); films (True); . (True)]}
                \newline
                [The end of the ideal output]
				\newline \newline
                \textbf{[Example 2]}
                \newline
                [The start of the biography]
                \newline
                \textcolor{titlecolor}{Doug Sheehan is an American actor who was born on April 27, 1949, in Santa Monica, California. He is best known for his roles in soap operas, including his portrayal of Joe Kelly on ``General Hospital'' and Ben Gibson on ``Knots Landing.''}
                \newline
                [The end of the biography]
                \newline \newline
                [The start of the list of checked facts]
                \newline
                \textcolor{anscolor}{[Doug Sheehan is an American. (True); Doug Sheehan is an actor. (True); Doug Sheehan was born on April 27, 1949. (True); Doug Sheehan was born in Santa Monica, California. (False); He is best known for his roles in soap operas. (True); He portrayed Joe Kelly. (True); Joe Kelly was in General Hospital. (True); General Hospital is a soap opera. (True); He portrayed Ben Gibson. (True); Ben Gibson was in Knots Landing. (True); Knots Landing is a soap opera. (True)]}
                \newline
                [The end of the list of checked facts]
                \newline \newline
                [The start of the ideal output]
                \newline
                \textcolor{labelcolor}{[Doug Sheehan (True); is (True); an (True); American (True); actor (True); who (True); was born (True); on (True); April 27, 1949 (True); in (True); Santa Monica, California (False); . (True); He (True); is (True); best known (True); for (True); his roles in soap operas (True); , (True); including (True); in (True); his portrayal (True); of (True); Joe Kelly (True); on (True); ``General Hospital'' (True); and (True); Ben Gibson (True); on (True); ``Knots Landing.'' (True)]}
                \newline
                [The end of the ideal output]
				\newline \newline
				\textbf{User prompt}
				\newline
				\newline
				[The start of the biography]
				\newline
				\textcolor{magenta}{\texttt{\{BIOGRAPHY\}}}
				\newline
				[The ebd of the biography]
				\newline \newline
				[The start of the list of checked facts]
				\newline
				\textcolor{magenta}{\texttt{\{LIST OF CHECKED FACTS\}}}
				\newline
				[The end of the list of checked facts]
			};
			\node[chatbox_user_inner] (q1_text) at (q1) {
				\textbf{System prompt}
				\newline
				\newline
				You are a helpful and precise assistant for segmenting and labeling sentences. We would like to request your help on curating a dataset for entity-level hallucination detection.
				\newline \newline
                We will give you a machine generated biography and a list of checked facts about the biography. Each fact consists of a sentence and a label (True/False). Please do the following process. First, breaking down the biography into words. Second, by referring to the provided list of facts, merging some broken down words in the previous step to form meaningful entities. For example, ``strategic thinking'' should be one entity instead of two. Third, according to the labels in the list of facts, labeling each entity as True or False. Specifically, for facts that share a similar sentence structure (\eg, \textit{``He was born on Mach 9, 1941.''} (\texttt{True}) and \textit{``He was born in Ramos Mejia.''} (\texttt{False})), please first assign labels to entities that differ across atomic facts. For example, first labeling ``Mach 9, 1941'' (\texttt{True}) and ``Ramos Mejia'' (\texttt{False}) in the above case. For those entities that are the same across atomic facts (\eg, ``was born'') or are neutral (\eg, ``he,'' ``in,'' and ``on''), please label them as \texttt{True}. For the cases that there is no atomic fact that shares the same sentence structure, please identify the most informative entities in the sentence and label them with the same label as the atomic fact while treating the rest of the entities as \texttt{True}. In the end, output the entities and labels in the following format:
                \begin{itemize}[nosep]
                    \item Entity 1 (Label 1)
                    \item Entity 2 (Label 2)
                    \item ...
                    \item Entity N (Label N)
                \end{itemize}
                % \newline \newline
                Here are two examples:
                \newline\newline
                \textbf{[Example 1]}
                \newline
                [The start of the biography]
                \newline
                \textcolor{titlecolor}{Marianne McAndrew is an American actress and singer, born on November 21, 1942, in Cleveland, Ohio. She began her acting career in the late 1960s, appearing in various television shows and films.}
                \newline
                [The end of the biography]
                \newline \newline
                [The start of the list of checked facts]
                \newline
                \textcolor{anscolor}{[Marianne McAndrew is an American. (False); Marianne McAndrew is an actress. (True); Marianne McAndrew is a singer. (False); Marianne McAndrew was born on November 21, 1942. (False); Marianne McAndrew was born in Cleveland, Ohio. (False); She began her acting career in the late 1960s. (True); She has appeared in various television shows. (True); She has appeared in various films. (True)]}
                \newline
                [The end of the list of checked facts]
                \newline \newline
                [The start of the ideal output]
                \newline
                \textcolor{labelcolor}{[Marianne McAndrew (True); is (True); an (True); American (False); actress (True); and (True); singer (False); , (True); born (True); on (True); November 21, 1942 (False); , (True); in (True); Cleveland, Ohio (False); . (True); She (True); began (True); her (True); acting career (True); in (True); the late 1960s (True); , (True); appearing (True); in (True); various (True); television shows (True); and (True); films (True); . (True)]}
                \newline
                [The end of the ideal output]
				\newline \newline
                \textbf{[Example 2]}
                \newline
                [The start of the biography]
                \newline
                \textcolor{titlecolor}{Doug Sheehan is an American actor who was born on April 27, 1949, in Santa Monica, California. He is best known for his roles in soap operas, including his portrayal of Joe Kelly on ``General Hospital'' and Ben Gibson on ``Knots Landing.''}
                \newline
                [The end of the biography]
                \newline \newline
                [The start of the list of checked facts]
                \newline
                \textcolor{anscolor}{[Doug Sheehan is an American. (True); Doug Sheehan is an actor. (True); Doug Sheehan was born on April 27, 1949. (True); Doug Sheehan was born in Santa Monica, California. (False); He is best known for his roles in soap operas. (True); He portrayed Joe Kelly. (True); Joe Kelly was in General Hospital. (True); General Hospital is a soap opera. (True); He portrayed Ben Gibson. (True); Ben Gibson was in Knots Landing. (True); Knots Landing is a soap opera. (True)]}
                \newline
                [The end of the list of checked facts]
                \newline \newline
                [The start of the ideal output]
                \newline
                \textcolor{labelcolor}{[Doug Sheehan (True); is (True); an (True); American (True); actor (True); who (True); was born (True); on (True); April 27, 1949 (True); in (True); Santa Monica, California (False); . (True); He (True); is (True); best known (True); for (True); his roles in soap operas (True); , (True); including (True); in (True); his portrayal (True); of (True); Joe Kelly (True); on (True); ``General Hospital'' (True); and (True); Ben Gibson (True); on (True); ``Knots Landing.'' (True)]}
                \newline
                [The end of the ideal output]
				\newline \newline
				\textbf{User prompt}
				\newline
				\newline
				[The start of the biography]
				\newline
				\textcolor{magenta}{\texttt{\{BIOGRAPHY\}}}
				\newline
				[The ebd of the biography]
				\newline \newline
				[The start of the list of checked facts]
				\newline
				\textcolor{magenta}{\texttt{\{LIST OF CHECKED FACTS\}}}
				\newline
				[The end of the list of checked facts]
			};
		\end{tikzpicture}
        \caption{GPT-4o prompt for labeling hallucinated entities.}\label{tb:gpt-4-prompt}
	\end{center}
\vspace{-0cm}
\end{table*}
% \section{Full Experiment Results}
% \begin{table*}[th]
    \centering
    \small
    \caption{Classification Results}
    \begin{tabular}{lcccc}
        \toprule
        \textbf{Method} & \textbf{Accuracy} & \textbf{Precision} & \textbf{Recall} & \textbf{F1-score} \\
        \midrule
        \multicolumn{5}{c}{\textbf{Zero Shot}} \\
                Zero-shot E-eyes & 0.26 & 0.26 & 0.27 & 0.26 \\
        Zero-shot CARM & 0.24 & 0.24 & 0.24 & 0.24 \\
                Zero-shot SVM & 0.27 & 0.28 & 0.28 & 0.27 \\
        Zero-shot CNN & 0.23 & 0.24 & 0.23 & 0.23 \\
        Zero-shot RNN & 0.26 & 0.26 & 0.26 & 0.26 \\
DeepSeek-0shot & 0.54 & 0.61 & 0.54 & 0.52 \\
DeepSeek-0shot-COT & 0.33 & 0.24 & 0.33 & 0.23 \\
DeepSeek-0shot-Knowledge & 0.45 & 0.46 & 0.45 & 0.44 \\
Gemma2-0shot & 0.35 & 0.22 & 0.38 & 0.27 \\
Gemma2-0shot-COT & 0.36 & 0.22 & 0.36 & 0.27 \\
Gemma2-0shot-Knowledge & 0.32 & 0.18 & 0.34 & 0.20 \\
GPT-4o-mini-0shot & 0.48 & 0.53 & 0.48 & 0.41 \\
GPT-4o-mini-0shot-COT & 0.33 & 0.50 & 0.33 & 0.38 \\
GPT-4o-mini-0shot-Knowledge & 0.49 & 0.31 & 0.49 & 0.36 \\
GPT-4o-0shot & 0.62 & 0.62 & 0.47 & 0.42 \\
GPT-4o-0shot-COT & 0.29 & 0.45 & 0.29 & 0.21 \\
GPT-4o-0shot-Knowledge & 0.44 & 0.52 & 0.44 & 0.39 \\
LLaMA-0shot & 0.32 & 0.25 & 0.32 & 0.24 \\
LLaMA-0shot-COT & 0.12 & 0.25 & 0.12 & 0.09 \\
LLaMA-0shot-Knowledge & 0.32 & 0.25 & 0.32 & 0.28 \\
Mistral-0shot & 0.19 & 0.23 & 0.19 & 0.10 \\
Mistral-0shot-Knowledge & 0.21 & 0.40 & 0.21 & 0.11 \\
        \midrule
        \multicolumn{5}{c}{\textbf{4 Shot}} \\
GPT-4o-mini-4shot & 0.58 & 0.59 & 0.58 & 0.53 \\
GPT-4o-mini-4shot-COT & 0.57 & 0.53 & 0.57 & 0.50 \\
GPT-4o-mini-4shot-Knowledge & 0.56 & 0.51 & 0.56 & 0.47 \\
GPT-4o-4shot & 0.77 & 0.84 & 0.77 & 0.73 \\
GPT-4o-4shot-COT & 0.63 & 0.76 & 0.63 & 0.53 \\
GPT-4o-4shot-Knowledge & 0.72 & 0.82 & 0.71 & 0.66 \\
LLaMA-4shot & 0.29 & 0.24 & 0.29 & 0.21 \\
LLaMA-4shot-COT & 0.20 & 0.30 & 0.20 & 0.13 \\
LLaMA-4shot-Knowledge & 0.15 & 0.23 & 0.13 & 0.13 \\
Mistral-4shot & 0.02 & 0.02 & 0.02 & 0.02 \\
Mistral-4shot-Knowledge & 0.21 & 0.27 & 0.21 & 0.20 \\
        \midrule
        
        \multicolumn{5}{c}{\textbf{Suprevised}} \\
        SVM & 0.94 & 0.92 & 0.91 & 0.91 \\
        CNN & 0.98 & 0.98 & 0.97 & 0.97 \\
        RNN & 0.99 & 0.99 & 0.99 & 0.99 \\
        % \midrule
        % \multicolumn{5}{c}{\textbf{Conventional Wi-Fi-based Human Activity Recognition Systems}} \\
        E-eyes & 1.00 & 1.00 & 1.00 & 1.00 \\
        CARM & 0.98 & 0.98 & 0.98 & 0.98 \\
\midrule
 \multicolumn{5}{c}{\textbf{Vision Models}} \\
           Zero-shot SVM & 0.26 & 0.25 & 0.25 & 0.25 \\
        Zero-shot CNN & 0.26 & 0.25 & 0.26 & 0.26 \\
        Zero-shot RNN & 0.28 & 0.28 & 0.29 & 0.28 \\
        SVM & 0.99 & 0.99 & 0.99 & 0.99 \\
        CNN & 0.98 & 0.99 & 0.98 & 0.98 \\
        RNN & 0.98 & 0.99 & 0.98 & 0.98 \\
GPT-4o-mini-Vision & 0.84 & 0.85 & 0.84 & 0.84 \\
GPT-4o-mini-Vision-COT & 0.90 & 0.91 & 0.90 & 0.90 \\
GPT-4o-Vision & 0.74 & 0.82 & 0.74 & 0.73 \\
GPT-4o-Vision-COT & 0.70 & 0.83 & 0.70 & 0.68 \\
LLaMA-Vision & 0.20 & 0.23 & 0.20 & 0.09 \\
LLaMA-Vision-Knowledge & 0.22 & 0.05 & 0.22 & 0.08 \\

        \bottomrule
    \end{tabular}
    \label{full}
\end{table*}




\end{document}






\newpage
\appendix

\renewcommand{\figurename}{Supplementary Figure}
\renewcommand{\tablename}{Supplementary Table}
\setcounter{figure}{0}
\setcounter{table}{0}

    



\section{Details of datasets}
This section provides additional details about the dataset used to evaluate the downstream tasks. \Cref{tab:disease_definition} lists the ICD-10 codes and medications used to identify the diagnoses for each disease. \Cref{tab:characteristic} presents the distribution of patient characteristics for each disease. \Cref{fig:nyu_langone_prevalence,fig:nyu_longisland_prevalence} illustrates the prevalence of each disease in the downstream tasks for the NYU Langone and NYU Long Island datasets, highlighting the imbalances present in these tasks.

\begin{table}[!htpb]
    \centering
    \caption{The definition of diseases in EHR by diagnosis codes and medications.}
    \begin{tabular}{lr}
    \toprule
         Disease &  Definition in EHR \\
    \midrule
       IPH  &  I61.0, I61.1, I61.2, I61.3, I61.4, I61.8, I61.9 \\
       IVH  &  I61.5, P52.1, P52.2, P52.3  \\
       ICH  &  IPH + IVH + I61.6, I62.9, P10.9, P52.4, P52.9 \\
       SDH  &  S06.5, I62.0 \\
       EDH  &  S06.4, I62.1 \\
       SAH  &  I60.*, S06.6, P52.5, P10.3  \\
       Tumor  &  C71.*, C79.3, D33.0, D33.1, D33.2, D33.3, D33.7, D33.9  \\
       Hydrocephalus  &  G91.* \\
       Edema  &  G93.1, G93.5, G93.6, G93.82, S06.1 \\
       \multirow{2}{*}{ADRD}  &  G23.1, G30.*, G31.01, G31.09, G31.83, G31.85, G31.9, F01.*, F02.*, F03.*, G31.84, G31.1, \\ 
       & \textbf{Medication:} DONEPEZIL, RIVASTIGMINE, GALANTAMINE, MEMANTINE, TACRINE \\ 
    \bottomrule
    \end{tabular}
    \label{tab:disease_definition}
\end{table}

\begin{table}[!htbp]
\centering
\caption{Demographic characteristics of patients associated with scans from the NYU Langone dataset, matched with electronic health records (EHR) and utilized in downstream tasks.}
\label{tab:characteristic}

 The characteristic table on NYU Langone dataset matched with EHR.
\begin{tabular}{ll|rr|r}
\toprule
                       \textbf{Cohort} &  &           \textbf{Male (\%)} &          \textbf{Female (\%)} &     \textbf{Age (std)} \\
\midrule
 --- & All (n=270,205) & 128,113 (47.41\%) & 142,092 (52.59\%) & 63.64 (19.68) \\
\midrule
       Tumor & Neg (n=260,704) & 123,338 (47.31\%) & 137,366 (52.69\%) & 63.85 (19.72) \\
             & Pos (n=9,501) &   4,775 (50.26\%) &   4,726 (49.74\%) & 57.80 (17.67) \\
\midrule
HCP & Neg (n=253,000) & 118,881 (46.99\%) & 134,119 (53.01\%) & 63.67 (19.72) \\
              & Pos (n=17,205) &   9,232 (53.66\%) &   7,973 (46.34\%) & 63.18 (19.11) \\
\midrule
Edema & Neg (n=242,576) & 112,987 (46.58\%) & 129,589 (53.42\%) & 63.96 (19.84) \\
      & Pos (n=27,629) &  15,126 (54.75\%) &  12,503 (45.25\%) & 60.81 (17.97) \\
\midrule
ADRD  & Neg (n=232,667) & 111,159 (47.78\%) & 121,508 (52.22\%) & 61.31 (19.55) \\
      & Pos (n=37,538) &  16,954 (45.16\%) &  20,584 (54.84\%) & 78.09 (13.30) \\
\midrule
          IPH & Neg (n=251,308) & 117,692 (46.83\%) & 133,616 (53.17\%) & 63.58 (19.82) \\
              & Pos (n=18,897) &  10,421 (55.15\%) &   8,476 (44.85\%) & 64.39 (17.69) \\
\midrule
          IVH & Neg (n=258,232) & 121,686 (47.12\%) & 136,546 (52.88\%) & 63.65 (19.79) \\
              & Pos (n=11,973) &   6,427 (53.68\%) &   5,546 (46.32\%) & 63.45 (17.19) \\
\midrule
          SDH & Neg (n=248,468) & 114,869 (46.23\%) & 133,599 (53.77\%) & 63.44 (19.78) \\
              & Pos (n=21,737) &  13,244 (60.93\%) &   8,493 (39.07\%) & 65.95 (18.33) \\
\midrule
          EDH & Neg (n=265,431) & 125,113 (47.14\%) & 140,318 (52.86\%) & 63.77 (19.64) \\
              & Pos (n=4,774) &   3,000 (62.84\%) &   1,774 (37.16\%) & 56.53 (20.75) \\
\midrule
          SAH & Neg (n=251,594) & 118,424 (47.07\%) & 133,170 (52.93\%) & 63.79 (19.76) \\
              & Pos (n=18,611) &   9,689 (52.06\%) &   8,922 (47.94\%) & 61.59 (18.49) \\
\midrule
          ICH & Neg (n=229,851) & 105,498 (45.90\%) & 124,353 (54.10\%) & 63.41 (19.93) \\
              & Pos (n=40,354) &  22,615 (56.04\%) &  17,739 (43.96\%) & 64.93 (18.14) \\
\bottomrule
\end{tabular}
\end{table}


\begin{table}[!h]
    \centering
    \caption*{\textbf{Supplementary \Cref{tab:characteristic} Continued.} Demographic characteristics of patients associated with scans from the NYU Long Island dataset, matched with electronic health records (EHR) and utilized in downstream tasks.}
\begin{tabular}{ll|rr|r}
\toprule
                       \textbf{Cohort} &  &           \textbf{Male (\%)} &          \textbf{Female (\%)} &     \textbf{Age (std)} \\
\midrule
--- & All (n=22,158) & 9,580 (43.23\%) & 12,578 (56.77\%) & 68.33 (18.14) \\
\midrule
Tumor & Neg (n=21,578) & 9,275 (42.98\%) & 12,303 (57.02\%) & 68.59 (18.08) \\
      & Pos (n=580) &   305 (52.59\%) &    275 (47.41\%) & 58.78 (17.79) \\
\midrule
HCP   & Neg (n=20,653) & 8,718 (42.21\%) & 11,935 (57.79\%) & 69.05 (17.90) \\
      & Pos (n=1,505) &   862 (57.28\%) &    643 (42.72\%) & 58.52 (18.48) \\
\midrule
Edema & Neg (n=19,402) & 8,068 (41.58\%) & 11,334 (58.42\%) & 68.89 (18.27) \\
      & Pos (n=2,756) & 1,512 (54.86\%) &  1,244 (45.14\%) & 64.36 (16.66) \\
\midrule
ADRD  & Neg (n=19,537) & 8,391 (42.95\%) & 11,146 (57.05\%) & 66.78 (18.28) \\
      & Pos (n=2,621) & 1,189 (45.36\%) &  1,432 (54.64\%) & 79.90 (11.77) \\
\midrule
IPH   & Neg (n=19,357) & 7,974 (41.19\%) & 11,383 (58.81\%) & 68.97 (18.27) \\
      & Pos (n=2,801) & 1,606 (57.34\%) &  1,195 (42.66\%) & 63.89 (16.48) \\
\midrule
IVH   & Neg (n=19,636) & 8,164 (41.58\%) & 11,472 (58.42\%) & 68.96 (18.22) \\
      & Pos (n=2,522) & 1,416 (56.15\%) &  1,106 (43.85\%) & 63.43 (16.66) \\
\midrule
SDH   & Neg (n=20,885) & 8,870 (42.47\%) & 12,015 (57.53\%) & 68.33 (18.21) \\
      & Pos (n=1,273) &   710 (55.77\%) &    563 (44.23\%) & 68.37 (16.83) \\
\midrule
EDH   & Neg (n=21,912) & 9,443 (43.10\%) & 12,469 (56.90\%) & 68.33 (18.16) \\
      & Pos (n=246) &   137 (55.69\%) &    109 (44.31\%) & 68.19 (15.59) \\
\midrule
SAH   & Neg (n=20,652) & 8,824 (42.73\%) & 11,828 (57.27\%) & 68.68 (18.12) \\
      & Pos (n=1,506) &   756 (50.20\%) &    750 (49.80\%) & 63.58 (17.65) \\
\midrule
ICH   & Neg (n=18,388) & 7,456 (40.55\%) & 10,932 (59.45\%) & 68.92 (18.35) \\
      & Pos (n=3,770) & 2,124 (56.34\%) &  1,646 (43.66\%) & 65.48 (16.77) \\
\bottomrule
\end{tabular}
\end{table}

\begin{figure}[!ht]
    \centering
    \includegraphics[width=0.8\textwidth]{images/NYU_Langone_prevalence.pdf}
    \caption{Disease prevalence of NYU Langone }
    \label{fig:nyu_langone_prevalence}
\end{figure}

\begin{figure}[!h]
    \centering
    \includegraphics[width=0.8\textwidth]{images/NYU_Longisland_prevalence.pdf}
    \caption{Disease prevalence of NYU Longisland dataset}
    \label{fig:nyu_longisland_prevalence}
\end{figure}



\section{Data augmentation details}
\label{sec:dataaug_details}
We applied Random Flipping across all three dimensions, Random Shift Intensity with offset $0.1$ for both pre-training and fine-tuning. For DINO training. random Gaussian Smoothing with sigma=$(0.5-1.0)$ is applied across all dimensions, Random Gamma Adjust is applied with gamma=$(0.2-1.0)$.


\section{Additional experiment results}
This section provides additional experimental results with more details.
Supplementary \Cref{fig:channels-ablation,fig:patches-ablation} compares the performance of the foundation model using different numbers of channels and patch sizes, demonstrating that the architecture design of our foundation model is optimal. 

Supplementary \Cref{fig:radar-comparison-merlin} compares our foundation model with a foundation CT model from previous studies, Merlin\cite{blankemeier2024merlinvisionlanguagefoundation}, which was trained on abdomen CT scans with corresponding radiology report pairs. Our model demonstrates superior performance on head CT scans.

Supplementary \Cref{fig:probing-comparison-gemini} compares our foundation model with Google CT Foundation model~\cite{yang2024advancingmultimodalmedicalcapabilities}, which was trained on large scale and diverse CT scans from different anatomy with corresponding radiology report pairs. Our model consistently shows improved performance across the board even though our model was pre-trained with less samples.

Supplementary \Cref{fig:probing_comparison} compares the performance on downstream tasks with various supervised tuning methods applied to foundation models pretrained with the MAE and DINO frameworks. Per-pathology comparisons are shown in Supplementary \Cref{fig:probing-comparison-perpath,fig:probing-comparison-perpath-dino}. Meanwhile, supplementary \Cref{fig:boxplot_scaling} complements \Cref{fig:scaling_law}, illustrating the per-pathology performances of foundation models pretrained with different scales of training data.

Supplementary \Cref{fig:batch_effect,fig:thickness-ablation} studies the impact of batch effect caused by different CT scan protocols of slice thickness and machine manufacturer. Detailed per-pathology performances are shown in Supplementary \Cref{fig:slice_thickness_per_pathology,fig:manufacturer_per_pathology}.

\begin{figure}[!htpb]
    \centering
    \makebox[\textwidth][l]{%
        \hspace{0.3\textwidth}\textbf{NYU Langone}
    } \\[0.2cm]
    \includegraphics[trim={0 0 0 0},clip,height=0.3\textwidth, width=0.3\textwidth]{figures/abla_chans/AUC_chans_NYU.pdf}
    \includegraphics[trim={0 0 0 0},clip,height=0.3\textwidth, width=0.55\textwidth]{figures/abla_chans/AP_chans_NYU.pdf}\\
    \makebox[\textwidth][l]{
        \hspace{0.34\textwidth}\textbf{RSNA}
    } \\[0.2cm]
    \includegraphics[trim={0 0 0 0},clip,height=0.3\textwidth, width=0.3\textwidth]{figures/abla_chans/AUC_chans_RSNA.pdf}
    \includegraphics[height=0.3\textwidth, width=0.55\textwidth]{figures/abla_chans/AP_chans_RSNA.pdf} 
    \caption{\textbf{Comparison of Different Channels Performance.} This plot compares the performance of models trained using different numbers of channels (channels from multiple HU intervals vs. a single HU interval). Across two datasets, models using three channels from different HU intervals consistently outperform those using a single channel with a fixed HU interval. All models were pre-trained on $100\%$ of the pretraining data with MAE.}
    \label{fig:channels-ablation}
\end{figure}


\begin{figure}[!htpb]
    \centering
    \makebox[\textwidth][l]{%
        \hspace{0.3\textwidth}\textbf{NYU Langone}
    } \\[0.2cm]
    \includegraphics[trim={0 0 0 0},clip,height=0.3\textwidth, width=0.3\textwidth]{figures/abla_patches/AUC_patches_NYU.pdf}
    \includegraphics[trim={0 0 0 0},clip,height=0.3\textwidth, width=0.55\textwidth]{figures/abla_patches/AP_patches_NYU.pdf}\\
    \makebox[\textwidth][l]{
        \hspace{0.34\textwidth}\textbf{RSNA}
    } \\[0.2cm]
    \includegraphics[trim={0 0 0 0},clip,height=0.3\textwidth, width=0.3\textwidth]{figures/abla_patches/AUC_patches_RSNA.pdf}
    \includegraphics[height=0.3\textwidth, width=0.55\textwidth]{figures/abla_patches/AP_patches_RSNA.pdf} 
    \caption{\textbf{Comparison of Different Patches Performance.} This plot compares the performance of models trained with different patch sizes (12 vs. 16). The results demonstrate that smaller patch sizes consistently achieve better performance. All models were pre-trained on $100\%$ of the pretraining data with MAE.}
    \label{fig:patches-ablation}
\end{figure}


\begin{figure*}
    \centering
    \makebox[\textwidth][l]{%
        \hspace{0.06\textwidth}
        \textbf{NYU Langone} \hspace{0.06\textwidth} \textbf{NYU Long Island} \hspace{0.11\textwidth} \textbf{RSNA} \hspace{0.18\textwidth} \textbf{CQ500}
    } \\[0.2cm]
    \includegraphics[trim={0 0 0 0},clip,height=0.21\textwidth, width=0.21\textwidth]{figures/abla_radarplot_merlin/AUC_NYU.pdf}
    \includegraphics[trim={0 0 0 0},clip,height=0.21\textwidth, width=0.21\textwidth]{figures/abla_radarplot_merlin/AUC_Longisland.pdf}
    \includegraphics[trim={0 0 0 0},clip,height=0.21\textwidth, width=0.21\textwidth]{figures/abla_radarplot_merlin/AUC_RSNA.pdf}
    \includegraphics[trim={0 0 0 0},clip,height=0.21\textwidth, width=0.35\textwidth]{figures/abla_radarplot_merlin/AUC_CQ500.pdf}\\[0.2cm]
    \includegraphics[height=0.21\textwidth, width=0.21\textwidth]{figures/abla_radarplot_merlin/AP_NYU.pdf} 
    \includegraphics[height=0.21\textwidth, width=0.21\textwidth]{figures/abla_radarplot_merlin/AP_Longisland.pdf} 
    \includegraphics[height=0.21\textwidth, width=0.21\textwidth]{figures/abla_radarplot_merlin/AP_RSNA.pdf}
    \includegraphics[height=0.21\textwidth, width=0.35\textwidth]{figures/abla_radarplot_merlin/AP_CQ500.pdf}
    \caption{\textbf{Comparison to previous 3D Foundation Model.} This plot compares the performance of our model with Merlin~\cite{blankemeier2024merlinvisionlanguagefoundation} and models trained from scratch across four datasets for our model and ResNet50-3D. Our DINO trained model is used in this comparison. Our model demonstrates consistently superior performance across majority of diseases, with the exception of epidural hemorrhage (EDH) in the CQ500 dataset.}
    \label{fig:radar-comparison-merlin}
\end{figure*}



\begin{figure*}
    \centering
    \makebox[\textwidth][l]{%
        \hspace{0.10\textwidth}
        \textbf{NYU Langone} \hspace{0.08\textwidth} \textbf{NYU Long Island} \hspace{0.1\textwidth} \textbf{RSNA} \hspace{0.15\textwidth} \textbf{CQ500}
    } \\[0.2cm]
    \includegraphics[trim={0 0 0 0},clip, width=0.22\textwidth]{figures/abla_probing/AUC_NYU.pdf}
    \includegraphics[trim={0 0 0 0},clip, width=0.22\textwidth]{figures/abla_probing/AUC_Longisland.pdf}
    \includegraphics[trim={0 0 0 0},clip, width=0.22\textwidth]{figures/abla_probing/AUC_RSNA.pdf}
    \includegraphics[trim={0 0 0 0},clip, width=0.28\textwidth]{figures/abla_probing/AUC_CQ500.pdf}
    \\[0.2cm]
    \includegraphics[width=0.22\textwidth]{figures/abla_probing/AP_NYU.pdf} 
    \includegraphics[width=0.22\textwidth]{figures/abla_probing/AP_Longisland.pdf} 
    \includegraphics[width=0.22\textwidth]{figures/abla_probing/AP_RSNA.pdf}
    \includegraphics[width=0.28\textwidth]{figures/abla_probing/AP_CQ500.pdf}
    \caption{\textbf{Comparison of Different Downstream Training Methods.} This plot illustrates the downstream performance of models evaluated using fine-tuning and various probing methods across four datasets. In most cases, the DINO pre-trained model outperforms the MAE pre-trained model. All models were pre-trained on $100\%$ of the available pretraining data.}
    \label{fig:probing_comparison}
\end{figure*}


\begin{figure}
\centering
\makebox[\textwidth][l]{%
    \hspace{0.39\textwidth}\textbf{RSNA}
} \\[0.2cm]
\includegraphics[trim={0 0 0mm 0},clip,height=0.27\textwidth]{figures/abla_gemini/AUC_RSNA_Gemini.pdf}
\includegraphics[trim={0 0 5mm 0},clip,height=0.27\textwidth]{figures/abla_gemini/AP_RSNA_Gemini.pdf}

\makebox[\textwidth][l]{%
    \hspace{0.38\textwidth}\textbf{CQ500}
} \\[0.2cm]
\includegraphics[trim={0 0 10mm 0},clip,height=0.345\textwidth]{figures/abla_gemini/AUC_CQ500_Gemini.pdf}
\includegraphics[trim={0 0 5mm 0},clip,height=0.345\textwidth]{figures/abla_gemini/AP_CQ500_Gemini.pdf}

\caption{\textbf{Performance comparison of linear probing for Our Model vs. Google CT Foundation model} This plot compares our model performance vs. Google CT Foundation model\cite{yang2024advancing} and Merlin \cite{blankemeier2024merlinvisionlanguagefoundation} across all diseases on RSNA and CQ500. Since Google CT Foundation moudel requires uploading data to Google Cloud (not allowed on our private data) for requesting model embeddings with model weights inaccessible, only public dataset comparison is provided in this study. Similar to other evaluations, we observed that our model outperforms Google CT Foundation model across the board with the only exception on Midline Shift for Google CT Foundation model and EDH for Merlin.}
\label{fig:probing-comparison-gemini}
\end{figure}



\begin{figure}
    \centering
    \makebox[\textwidth][l]{%
        \hspace{0.35\textwidth}\textbf{NYU Langone}
    } \\[0.2cm]
    \includegraphics[trim={0 0 120mm 0},clip,height=0.255\textwidth]{figures/abla_probing_perpath/DINO_AUC_NYU_Langone.pdf}
    \includegraphics[trim={0 0 0 0},clip,height=0.255\textwidth]{figures/abla_probing_perpath/DINO_AP_NYU_Langone.pdf} \\
    \makebox[\textwidth][l]{
        \hspace{0.35\textwidth}\textbf{NYU Long Island}
    } \\[0.2cm]
    \includegraphics[trim={0 0 120mm 0},clip,height=0.255\textwidth]{figures/abla_probing_perpath/DINO_AUC_NYU_Long_Island.pdf}
    \includegraphics[trim={0 0 0 0},clip,height=0.255\textwidth]{figures/abla_probing_perpath/DINO_AP_NYU_Long_Island.pdf} 
    \makebox[\textwidth][l]{
        \hspace{0.4\textwidth}\textbf{RSNA}
    } \\[0.2cm]
    \includegraphics[trim={0 0 120mm 0},clip,height=0.24\textwidth]{figures/abla_probing_perpath/DINO_AUC_RSNA.pdf}
    \hspace{5mm}
    \includegraphics[trim={0 0 0 0},clip,height=0.24\textwidth]{figures/abla_probing_perpath/DINO_AP_RSNA.pdf} 
    \makebox[\textwidth][l]{
        \hspace{0.4\textwidth}\textbf{CQ500}
    } \\[0.2cm]
    \includegraphics[trim={0 0 120mm 0},clip,height=0.30\textwidth]{figures/abla_probing_perpath/DINO_AUC_CQ500.pdf} \hspace{5mm}
    \includegraphics[trim={0 0 0 0},clip,height=0.30\textwidth]{figures/abla_probing_perpath/DINO_AP_CQ500.pdf} 
    \caption{\textbf{Performance comparison of supervised finetuning methods per pathology on the foundation model trained with DINO.} This plot breaks down the average performance across all diseases shown in Supplementary \Cref{fig:probing_comparison}. The results show that fine-tuning the entire network achieves the best performance in most scenarios. However, linear probing closely approaches finetuning performance for many diseases especially on small or imbalanced dataset, underscoring the capability of our pre-trained models to generate representations that adapt effectively to diverse disease detection tasks.}
    \label{fig:probing-comparison-perpath-dino}
\end{figure}

\begin{figure}
    \centering
    \makebox[\textwidth][l]{%
        \hspace{0.35\textwidth}\textbf{NYU Langone}
    } \\[0.2cm]
    \includegraphics[trim={0 0 0 0},clip,height=0.24\textwidth, width=0.3\textwidth]{figures/abla_probing_perpath/AUC_NYU.pdf}
    \includegraphics[trim={0 0 0 0},clip,height=0.24\textwidth, width=0.45\textwidth]{figures/abla_probing_perpath/AP_NYU.pdf}\\
    \makebox[\textwidth][l]{
        \hspace{0.35\textwidth}\textbf{NYU Long Island}
    } \\[0.2cm]
    \includegraphics[trim={0 0 0 0},clip,height=0.24\textwidth, width=0.3\textwidth]{figures/abla_probing_perpath/AUC_Longisland.pdf}
    \includegraphics[trim={0 0 0 0},clip,height=0.24\textwidth, width=0.45\textwidth]{figures/abla_probing_perpath/AP_Longisland.pdf} 
    \makebox[\textwidth][l]{
        \hspace{0.4\textwidth}\textbf{RSNA}
    } \\[0.2cm]
    \includegraphics[trim={0 0 0 0},clip,height=0.24\textwidth, width=0.3\textwidth]{figures/abla_probing_perpath/AUC_RSNA.pdf}
    \includegraphics[height=0.24\textwidth, width=0.45\textwidth]{figures/abla_probing_perpath/AP_RSNA.pdf} 
    \makebox[\textwidth][l]{
        \hspace{0.4\textwidth}\textbf{CQ500}
    } \\[0.2cm]
    \includegraphics[trim={0 0 120mm 0},clip,height=0.24\textwidth]{figures/abla_probing_perpath/AUC_CQ500.pdf}
    \includegraphics[trim={0 0 0 0},clip,height=0.24\textwidth]{figures/abla_probing_perpath/AP_CQ500.pdf} 
    \caption{\textbf{Performance comparison of supervised finetuning methods per pathology on the foundation model trained with MAE.} The results reveal that attentive probing is significantly more effective than linear probing, consistent with findings from~\cite{Chen2024}. Furthermore, for many diseases, the performance of probing models approaches that of fine-tuning, demonstrating that our pre-trained models produce adaptable representations capable of detecting diverse diseases.}
    \label{fig:probing-comparison-perpath}
\end{figure}









\begin{figure}
    \centering
    \textbf{NYU Langone} \\
    \includegraphics[trim={0 0 0 0},clip,height=0.24\textwidth, width=0.38\textwidth]{figures/abla_perpath_perf/AUC_NYU.pdf}
    \includegraphics[height=0.24\textwidth, width=0.45\textwidth]{figures/abla_perpath_perf/AP_NYU.pdf} \\
    \textbf{NYU Long Island} \\
    \includegraphics[trim={0 0 0 0},clip,height=0.24\textwidth, width=0.38\textwidth]{figures/abla_perpath_perf/AUC_Longisland.pdf}
    \includegraphics[height=0.24\textwidth, width=0.45\textwidth]{figures/abla_perpath_perf/AP_Longisland.pdf} \\
    \textbf{RSNA} \\
    \includegraphics[trim={0 0 0 0},clip,height=0.24\textwidth, width=0.38\textwidth]{figures/abla_perpath_perf/AUC_RSNA.pdf}
    \includegraphics[height=0.24\textwidth, width=0.45\textwidth]{figures/abla_perpath_perf/AP_RSNA.pdf}\\
    \textbf{CQ500} \\
    \includegraphics[trim={0 0 0 0},clip,height=0.24\textwidth, width=0.38\textwidth]{figures/abla_perpath_perf/AUC_CQ500.pdf}
    \includegraphics[height=0.24\textwidth, width=0.45\textwidth]{figures/abla_perpath_perf/AP_CQ500.pdf}
    \caption{\textbf{Performance for Different Percentage of Pre-training Samples (Per-Pathology).} This plot illustrates label efficiency for individual pathologies using Tukey plots, alongside the average performance across all diseases shown in \Cref{fig:scaling_law}. The results indicate that the majority of pathologies show improved downstream performance as the amount of pretraining data increases.}
    \label{fig:boxplot_scaling}
\end{figure}


\newpage

\section{Time complexity increase with reduced patch size}
\label{apd:self_attention_rate}
Assume we have 3D image input of shape $H\times W\times D$, patch size $P$ and reducing factor $s$. By time complexity of self-attention $O(n^2 d)$ for sequence length $n$ and embedding dimension $d$, the new time complexity after reducing patch size can be derived as
\begin{align*}
    O(n^2d)&=O((\frac{H\times W\times D}{(\frac{P}{s})^3})^2d) \\
           &=O((\frac{H\times W\times D}{P^3})^2 s^6d)  \\
           &=O(s^6)O(n_{ori}^2d)
\end{align*}
where $n_{ori}=\frac{H\times W\times D}{P^3}$ is the original sequence length before reducing patch size.



















\newpage
\begin{figure}[ht]
    \centering
    \includegraphics[width=\textwidth]{images/tsne_embedding_visualization_per_pathology.png}
    \caption{The 2D projection with t-SNE of CT volume representation extracted from the foundation model. Interestingly, certain subgroups still exhibited slightly better AUCs. For instance, scans with slice thicknesses between 1–4 mm (represented by light green points in the upper panel of \Cref{fig:batch_effect}) align with a specialized protocol for CT angiography (CTA), which uses contrast dye to improve diagnosis on particular diseases.}
    \label{fig:batch_effect}
\end{figure}


\begin{figure*}[ht]
    \centering
    \begin{subfigure}[b]{0.33\textwidth}
        \centering
        \includegraphics[width=\textwidth]{images/AUROC_vs_Slice_thickness_binned.png}
        \caption{AUROC Performance}
    \end{subfigure}
    \hfill
    \begin{subfigure}[b]{0.33\textwidth}
        \centering
        \includegraphics[width=\textwidth]{images/AUPRC_vs_Slice_thickness_binned.png}
        \caption{AUPRC Performance}
    \end{subfigure}
    \hfill
    \begin{subfigure}[b]{0.33\textwidth}
        \centering
        \includegraphics[width=\textwidth]{images/Histogram_of_slice_thickness_distribution_across_scans.png}
        \caption{Histogram of slice thickness distribution}
    \end{subfigure}
    \caption{The downstream task performances on various ranges of slice thickness.}
    \label{fig:thickness-ablation}
\end{figure*}


\begin{figure*}[ht]
    \centering
    \begin{subfigure}[b]{\textwidth}
        \centering
        \includegraphics[width=\textwidth]{images/AUROC_vs_slice_thickness_for_each_disease_category.png}
        \caption{AUROC Performance}
    \end{subfigure}
    \hfill
    \begin{subfigure}[b]{\textwidth}
        \centering
        \includegraphics[width=\textwidth]{images/AUPRC_vs_slice_thickness_for_eachdisease_category.png}
        \caption{AUPRC Performance}
    \end{subfigure}
    \hfill
    \begin{subfigure}[b]{\textwidth}
        \centering
        \includegraphics[width=\textwidth]{images/Ratio_of_positive_labels_vs_slice_thickness_for_each_disease_category.png}
        \caption{Ratio of Positive Labels}
    \end{subfigure}
    \caption{Performance for Each Slice Thickness Bin (Per Pathology).}
    \label{fig:slice_thickness_per_pathology}
\end{figure*}


\begin{figure*}[ht]
    \centering
    \begin{subfigure}[b]{0.3\textwidth}
        \centering
        \includegraphics[width=\textwidth]{images/AUROC_by_Disease_and_Manufacturer.png}
        \caption{AUROC Performance}
    \end{subfigure}
    \hfill
    \begin{subfigure}[b]{0.3\textwidth}
        \centering
        \includegraphics[width=\textwidth]{images/AUPRC_by_Disease_and_Manufacturer.png}
        \caption{AUPRC Performance}
    \end{subfigure}
    \hfill
    \begin{subfigure}[b]{0.39\textwidth}
        \centering
        \includegraphics[width=\textwidth]{images/Positive_Label_Ratio_by_Disease_and_Manufacturer.png}
        \caption{Distribution of Scans from Each Manufacturer}
    \end{subfigure}
    \caption{Performance for Each Manufacturer (Per Pathology).}
    \label{fig:manufacturer_per_pathology}
\end{figure*}








\end{document}
