\documentclass[fleqn,10pt]{wlscirep}
\usepackage[utf8]{inputenc}
\usepackage[T1]{fontenc}
\usepackage{multirow}
\usepackage{adjustbox}
\usepackage{subcaption}
\usepackage{pifont}
\usepackage{hyperref}
\usepackage{float}
\usepackage{cleveref}
\usepackage{nameref}
\usepackage{multirow}
\usepackage{xcolor}
\usepackage{longtable}

\pdfobjcompresslevel=0
\newcommand{\cmark}{\ding{51}}%
\newcommand{\xmark}{\ding{55}}%
\crefname{figure}{Fig.}{Figs.}         %
\crefname{appendix}{Supplementary}{Supplementary}

\title{3D Foundation AI Model for Generalizable Disease Detection in Head Computed Tomography}

\author[1†]{Weicheng Zhu}
\author[1†]{Haoxu Huang}
\author[1]{Huanze Tang}
\author[2]{Rushabh Musthyala}
\author[1]{Boyang Yu}
\author[1]{Long Chen}
\author[3]{Emilio Vega}
\author[3,8]{Thomas O'Donnell}
\author[3]{Seena Dehkharghani}
\author[4]{Jennifer A. Frontera}
\author[4,5,6]{Arjun V. Masurkar}
\author[4]{Kara Melmed}
\author[3,7*]{Narges Razavian}
\affil[1]{New York University, Center for Data Science, New York, NY, 10001, USA}
\affil[2]{New York University, Courant Institute of Mathematical Sciences, New York, NY, 10001, USA}
\affil[3]{NYU Grossman School of Medicine, Department of Radiology, New York, NY, 10016, USA}
\affil[4]{NYU Grossman School of Medicine, Department of Neurology, New York, NY, 10016, USA}
\affil[5]{NYU Grossman School of Medicine, Department of Neuroscience and Physiology, New York, NY, 10016, USA}
\affil[6]{NYU Grossman School of Medicine, Neuroscience Institute, New York, NY, 10016, USA}
\affil[7]{NYU Grossman School of Medicine, Department of Population Health, New York, NY, 10016, USA}
\affil[8]{Siemens Healthineers, Malvern, PA, 19355, USA}

\affil[*]{\textbf{Corresponding author:}Narges.Razavian@nyulangone.org}

\affil[$\dagger$]{These authors contributed equally to this work}
 

\begin{abstract}
Head computed tomography (CT) imaging is a widely-used imaging modality with multitudes of medical indications, particularly in assessing pathology of the brain, skull, and cerebrovascular system. It is commonly the first-line imaging in neurologic emergencies given its rapidity of image acquisition, safety, cost, and ubiquity. 
Deep learning models may facilitate detection of a wide range of diseases. However, the scarcity of high-quality labels and annotations, particularly among less common conditions, significantly hinders the development of powerful models. To address this challenge, we introduce \textbf{FM-CT}: a \textbf{F}oundation \textbf{M}odel for Head \textbf{CT} for generalizable disease detection, trained using self-supervised learning. Our approach pre-trains a deep learning model on a large, diverse dataset of 361,663 non-contrast 3D head CT scans without the need for manual annotations, enabling the model to learn robust, generalizable features. To investigate the potential of self-supervised learning in head CT, we employed both discrimination with self-distillation and masked image modeling, and we construct our model in 3D rather than at the slice level (2D) to exploit the structure of head CT scans more comprehensively and efficiently. The pre-training phase is followed by fine-tuning on smaller, annotated downstream datasets, thereby optimizing the model for specific diagnostic tasks, such as detecting hemorrhages, tumors, and other abnormalities. The model's downstream classification performance is evaluated using internal and three external datasets, encompassing both in-distribution (ID) and out-of-distribution (OOD) data. Our results demonstrate that the self-supervised foundation model significantly improves performance on downstream diagnostic tasks compared to models trained from scratch and previous 3D CT foundation models on scarce annotated datasets. Furthermore, the model maintains strong generalization across different datasets, indicating its potential for broad clinical applicability. This work highlights the effectiveness of self-supervised learning in medical imaging and sets a new benchmark for head CT image analysis in 3D, enabling broader use of artificial intelligence for head CT-based diagnosis.
\end{abstract} 
\begin{document}

\flushbottom
\maketitle
\thispagestyle{empty}


\section*{Introduction}
Head computed tomography (CT) is often the first step in diagnosing a wide range of neurological disorders, including head trauma, hemorrhages, hydrocephalus, and malignancies. Head CT scans are faster, more accessible, and generally less expensive than magnetic resonance imaging (MRIs), making them ideal for emergencies like traumatic brain injury (TBI) or suspected stroke. They are also effective in detecting bone fractures, or neurovascular pathologies such as arterial venous malformations. Despite its widespread use, CT lacks the contrast resolution and hence the sensitivity for many disorders dependent upon diagnosis by MRI, thus MRI is the imaging modality of choice for many neurologic diseases. MRI, however, is more costly, risks potential heating or displacement of indwelling implants, and suffers generally slower acquisition times, increasing patient discomfort and risking non-diagnostic examinations due to its greater sensitivity to motion-related artifacts. It is also more expensive than CT and is contraindicated in specific patients. Access to MRI is a major challenge in resource-limited countries. The timely and arduous determination of certain pathologies can delay appropriate medical and surgical treatment for patients.
There is significant potential to harness artificial intelligence (AI) algorithms to enhance the diagnostic and early detection capabilities of head CT, providing critical support in clinical decision-making and improving patient outcomes. Early and accurate diagnosis can potentially lead to more effective treatments, reduce complications, and improve patient survival.


Current research on AI-driven diagnosis using head CT is limited due to both lack of data availability and the complexity of model architectures. Although datasets such as RSNA~\cite{flanders_construction_2020} and CQ500~\cite{CQ500} provide publicly available head CT data, they remain small (RSNA includes approximately 10K samples and CQ500 approximately 1K), and their primary focus is on hemorrhage detection, which restricts broader applicability as a credible path to clinical decision support. Moreover, many existing models are designed with highly task-specific architectures that may not generalize well to diverse clinical applications. These models typically apply 2D convolution neural networks (CNN) to sequentially process 3D volumes slice-by-slice under the supervision of slice-level labels~\cite{wang_deep_2021,CQ500,yun_deep_2023}. %
Slice-level labels are often expensive to acquire, and models trained on 2D slices often struggle to generalize to conditions like neurodegenerative diseases, where slice-level labels are not easily defined. Developing models that can harness the information embedded within the 3D structure of CT images while requiring minimal slice-level labeled data can thus expand the impact of such approaches. To address these challenges, we have developed FM-CT: a Foundation Model for Head CT, and demonstrated robust performance across multiple tasks and datasets, which highlights our model's potential for broad clinical applicability.


Recent advancements in AI \textit{foundation models}, deep learning models pre-trained on extensive datasets in a self-supervised manner, have enabled rapid adaptation and robust performance across a wide range of tasks~\cite{radford2021learningtransferablevisualmodels,zhou2021ibot,oquab2023dinov2, rishi24foundation}. Multiple studies have shown that foundation models trained on large-scale medical data can enhance model performance in various medical imaging tasks, including chest X-rays~\cite{yao2024evaxfoundationmodelgeneral}, histopathology~\cite{wang_pathology_2024,huang_visuallanguage_2023,chen_towards_2024, Vorontsov2024},  retina imaging~\cite{zhou2023foundation}, fMRI~\cite{dong2024brainjepa} and more. Additionally, several generalist vision-language models show promise for multimodal medical applications~\cite{codella2024medimageinsight,yang2024advancingmultimodalmedicalcapabilities,zhang2024generalist}. Although some research has focused on CT scans~\cite{Tang_2022_CVPR,blankemeier2024merlinvisionlanguagefoundation,codella2024medimageinsight}, these studies remain limited to abdominal CTs and cannot generalize to other part of the body. While Google CT Foundation model has explored report generation for head CTs~\cite{yang2024advancingmultimodalmedicalcapabilities}, it uses default video encoders to interpret 3D head CTs and has only been subjectively evaluated on fewer than 100 samples, for which the generated reports were worse than original ones. Given these limitations, developing a dedicated vision foundation model for head CTs is essential to advance AI-driven diagnosis and facilitate early detection of cranial and neurological conditions.

In this work, we introduce FM-CT, a 3D foundation model for head CT scans, developed using self-supervised learning (SSL). While SSL has shown success in natural images~\cite{chen2020simple, he2020momentum, caron2020unsupervised, caron2021emerging, bao2022beit, He2021MaskedAA, zhou2021ibot, oquab2023dinov2, zbontar2021barlow, bardes2022vicreg} and in medical imaging~\cite{Liu_2023_CVPR, zhu2022interpretablepredictionlungsquamous, chen_towards_2024, zhou2023foundation, Huang2023, azizi21big, Vorontsov2024, huang2023radiology, huang21GLoRIA, chen23masked, Azizi2023}, training a robust 3D CT volume encoder presents distinct challenges, such as selecting appropriate pretext tasks, managing spatial normalization, and addressing high computational demands. To overcome these challenges, we developed a standardized pipeline that normalizes head CT scans from various protocols, producing consistent input for our foundation model. For pre-training, we adapted two SSL frameworks—self-distillation, inspired by DINO~\cite{caron2021emerging}, and masked prediction, inspired by MAE~\cite{He2021MaskedAA}. These methods were tailored to train a volumetric encoder based on a customized vision Transformer (ViT)\cite{dosovitskiy2020vit}. The full details of our design and choices are described in Section\hyperref[sec:methods]{``Method''}. pre-training was conducted on a large-scale dataset comprising 361,663 head CT scans from a major clinical institution.

To evaluate the foundation model, we systematically assessed its performance and generalizability across 10 downstream disease detection tasks using diverse internal and external datasets, as illustrated in \Cref{fig:overview}. Beyond commonly studied hemorrhages, our evaluation includes crucial yet less-explored tasks in head CT, such as identifying brain tumors, Alzheimer’s disease and related dementia (ADRD), edema, and hydrocephalus (HCP). For each downstream task, the foundation model was fine-tuned using task-specific labels. Given the scarcity of expert-annotated public datasets for these conditions, we leverage electronic health records (EHR) to acquire labels of each task. While EHRs may include missing data and suffer potential label-noise, they remain a valuable and practical source for large-scale patient status labeling that can be used to evaluate the performance of the foundation model.

Our results reveal substantial performance improvements enabled by our foundation model. Downstream models initialized with the pre-trained weights of foundation model achieve a $16.07\%$ improvement in macro-AUC over models trained from scratch with random initialization on internal NYU Langone data, and $20.86\%$ and $12.01\%$ improvements on external datasets from NYU Long Island (previously a separate hospital) and RSNA, respectively ($P<0.001$ for all comparisons). These findings underscore the potential of our foundation model to advance AI-based interpretation of head CT scans, supporting more accurate diagnosis and early disease detection. Furthermore, as described in Section~\hyperref[sec:results]{``Results''}, we demonstrate the model’s capabilities in out-of-distribution generalization (\Cref{fig:overview}), few-shot learning (\Cref{fig:fewshot}), and scalability (\Cref{fig:scaling_law}), highlighting the method’s potential in scenarios with limited annotated fine-tuning data, or scenarios such as federated learning which provide access to orders of magnitude more data. Overall, the experimental results on multiple datasets and tasks underscore the generalizability, adaptability and effectiveness of the model, and pave the way for significant impact in real-world clinical applications.




\begin{figure*}[t]
\begin{center}
\includegraphics[width=.85\linewidth]{fig_overview_v3.pdf}
\end{center}
\caption{
FastAtlas Overview: In each frame, we compute charts spanning fully or partially visible triangles (a), determine texture space bounding boxes for the visible portions of the view-space projections of each chart, and tightly pack these boxes into atlases (b, here $2K \times 2K$). We simultaneously bijectively parameterize and shade the charts into their atlas boxes, obtaining high quality texture space shading (c), and use this shading to render the shaded frames (d).}
\label{fig:overview}
\label{fig:alg_overview}
\end{figure*}

\section{Overview}
\label{sec:overview}
Our work has two core contributions: a real-time, GPU-based algorithm for tight packing of general parameterized charts into compact atlases; and a real-time TSS method that
utilizes this packing.  

\paragraph*{FastAtlas Packing.}
FastAtlas runs entirely on the GPU as a series of compute shaders. It takes the bounding boxes of parameterized charts as input, and packs them into an atlas (Fig~\ref{fig:overview}b, Sec.~\ref{sec:pack}). As such, the only input it requires are the dimensions of the bounding boxes.
Its outputs are deterministic; identical input charts are packed into identical atlases. This is critical for TSS and similar applications, as it ensures that consecutive frames taken from the same camera view have the same shading. Even minute shading differences across such frames can cause sampling jitter, leading to undesirable flicker \cite{baker2012rock}. 
While prior methods such as \cite{mueller2018shading,hladky2019tessellated,hladky2021snakebinning,Neff2022MSA} cap the dimensions of the charts that can be packed as-is for a given atlas size, and scale down all charts that exceed these dimensions, we scale all charts by the same factor, and do so only when strictly necessary to achieve packing success (Figs~\ref{fig:atlas},~\ref{fig:sas_issues}). 

\paragraph*{TSS using FastAtlas.}
Our end-to-end TSS atlas generation method combines the packing method above with a novel approach for computing seamless per-frame charts. 
We define our charts as the connected components of the visible surfaces in each frame (Fig.~\ref{fig:overview}a), and efficiently compute them using a parallel union-find algorithm (Sec.~\ref{sec:visible}). Since the boundaries of these charts coincide with the contours of the rendered surface, they are {\em invisible} to the viewer. This approach 
eliminates the artifacts caused by shading discontinuities along visible seams (Fig.~\ref{fig:seams}). 

\begin{parWithWrapFigure}
\begin{wrapfigure}{l}{.27\columnwidth}%
\includegraphics[width=\linewidth]{fig_inset_view_plane.pdf}%
\end{wrapfigure}
We bijectively parametrize the {\em visible portions} of our charts by projecting them to view space (inset). This maps a constant number of texels to each pixel in the final rendered output, evenly distributing residual undersampling error across all image pixels. While conceptually straightforward, efficiently parameterizing charts containing partially visible triangles using viewspace projection is non-trivial, as the visible portions may no longer be triangular (e.g. green triangle in the inset); applying naive projection to triangles with vertices behind the camera may produce ill-posed results. Clipping triangles before projection is both computationally expensive and significantly complicates downstream operations. We avoid explicit clipping by observing that all that is required for atlas packing is the dimensions of, potentially conservative, bounding boxes of these projected visible portions. We compute such bounding boxes without explicit chart clipping by adapting a conservative screen coverage estimator \shortcite{Blinn:CalculatingScreenCoverage} (Sec.~\ref{sec:box}). We then pack the computed boxes using FastAtlas. 
\end{parWithWrapFigure}

Finally, we shade the visible portion of each chart into its corresponding atlas bounding box (Fig~\ref{fig:overview}c). 
The resulting texture is then used during rasterization as a standard texture map (Fig. ~\ref{fig:overview}d). 
Our framework is compatible with all existing approaches for texture space shading, including forward shading, raytraced illumination, or deferred shading in texture space \cite{baker:2016}. In the examples shown, we use the standard forward shading based rendering pipeline included in the G3D Innovation Engine \cite{G3D17}, a commercial grade renderer.


\section*{Results}
\label{sec:results}
\subsection*{Foundation model for disease detection with 3D head CT scans}

The key aim of the foundation model is to develop a single model that improves performance on a wide range of downstream tasks of detecting recognizable abnormalities from head CT scans. To evaluate the capability of the foundation model, we train classification models for multiple disease detection tasks by fine-tuning the foundation model separately per disease, and assessing the fine-tuned model's performance on held-out validation and external data sets. The selected downstream tasks include detecting various types of hemorrhages (intraparenchymal hemorrhage (IPH), intraventricular hemorrhage (IVH), subdural hemorrhage (SDH), epidural hemorrhage (EDH), subarachnoid hemorrhage (SAH), and intracranial hemorrhage (ICH)), brain tumors, hydrocephalus (HCP), edema, and Alzheimer's diseases and related dementia (ADRD). \Cref{fig:overview}a,b,c show the overview of our pre-training framework and included data, EHR-matching, and datasets used in pre-training, in-domain fine-tuning, and external validation. Overall N=361,663 scans were used during pre-training, and four distinct datasets from different sources were used for various forms of validation (NYU Langone N=26,487; NYU Long Island N=2,202; RSNA N=1,058; and CQ500 N=236). NYU Langone is a hospital system comprised of multiple geographically distinct hospitals including two Level 1 Trauma Centers and three Comprehensive Stroke Centers. NYU Long Island, a Level 1 Trauma Center/Comprehensive Stroke Center, is treated as an external dataset for the purposes of this study. 

The first two rows of \Cref{fig:overview}e report the task-specific AUCs for Vision Transformer (ViT) classifiers trained from scratch with random initialization, namely \textit{scratch}, versus those fine-tuned from the foundation model, namely \textit{fine-tuned} on NYU Langone data. The fine-tuned models consistently outperform the scratch model across all 10 disease detection tasks, achieving a macro-AUC of 0.852 --- a $16.07\%$ increase over the scratch model’s 0.734 ($P<0.001$). Additionally, in \Cref{fig:overview}f and Supplementary \Cref{fig:radar-comparison-merlin} we compared the foundation model with two other foundation model for 3D CT scans --- Merlin~\cite{blankemeier2024merlinvisionlanguagefoundation} and Google's CT Foundation\cite{yang2024advancing} model. Merlin outperforms the scratch model with a macro-AUC of $5.67\%$ while falling short compared to our foundation model with $7.51\%$ lower macro-AUC ($P<0.001$, illustrated in Supplementary \Cref{fig:radar-comparison-merlin}). Although Merlin is not directly comparable to our foundation model as it was pre-trained on abdominal CT, it still provides a valuable baseline. We compare our model to Google CT Foundation model with linear probing, because trainable weights for end-to-end fine-tuning are not provided for this model. We consistently observe improved model performance across the board (in \Cref{fig:overview}f and Supplementary \Cref{fig:probing-comparison-gemini}). %
These findings demonstrate that despite the progress in general domain multimodal models, specialized foundation model pre-trained on head CT data still significantly enhance the understanding of brain CTs. 

To assess our foundation model's generalization to out-of-distribution data, we compiled three external datasets from multiple institutions and sources: NYU Long Island, RSNA~\cite{flanders_construction_2020}, and CQ500~\cite{CQ500}, as shown in \Cref{fig:overview}c (NYU Langone and NYU Long Island are geographically separate and distinct institutions within the broader health system). The data in these external datasets has a different distribution than the data used for pre-training. We evaluate the generalization on external datasets via two common practices to utilize the foundation model: (1) in-domain fine-tuning on separated datasets and tasks, and (2) fully external validation of the disease detection models without any site-specific fine-tuning.

\begin{figure}[htbp]
    \centering
    NYU Langone \\
    \includegraphics[trim={0 0 140pt 0},clip,height=0.3\textwidth]{figures/fig2/fewshot_NYU_AUC.pdf}
    \includegraphics[height=0.3\textwidth]{figures/fig2/fewshot_NYU_AP.pdf} \\
    NYU Long Island \\
    \includegraphics[trim={0 0 140pt 0},clip,height=0.3\textwidth]{figures/fig2/fewshot_NYU_Long_Island_AUC.pdf}
    \includegraphics[height=0.3\textwidth]{figures/fig2/fewshot_NYU_Long_Island_AP.pdf} \\
    RSNA \\
    \includegraphics[trim={0 0 125pt 0},clip,height=0.295\textwidth]{figures/fig2/fewshot_RSNA_AUC.pdf}
    \includegraphics[height=0.295\textwidth]{figures/fig2/fewshot_RSNA_AP.pdf}
    \caption{\textbf{Few-shot performance of the foundation model.}  The plots display the per-pathology AUC and average precision (AP) of the disease detection model under a few-shot learning setting, evaluated with varying numbers of training samples from the NYU Langone, NYU Long Island, and RSNA datasets. CQ500 is excluded since its small dataset size gives no enough positive samples for many diseases. Few-shot learning performance is compared to supervised finetuning with all training data (denoted by stars), demonstrating the strong generalization ability of the foundation model with limited training data. The confidence intervals are computed by 5 repeated experiments on resampling the training data and retraining the model.}
    \label{fig:fewshot}
\end{figure}


For in-domain fine-tuning, the foundation model is fine-tuned on each external dataset’s training set and validated on held-out sets from the same source. The bottom four rows in \Cref{fig:overview}e report the tasks-level performances on NYU Long Island and RSNA datasets. The fine-tuned model yields a macro-AUC of 0.904 across the 10 tasks on NYU Long Island dataset and a macro-AUC of 0.923 for five types of hemorrhages on the RSNA dataset. In comparison, the scratch model results in macro-AUC scores of 0.748 and 0.824, respectively.  Moreover, the foundation model also significantly outperforms Merlin, as shown in Supplementary \Cref{fig:radar-comparison-merlin}. The superior performances on external datasets indicate the generalizability of the foundation model. Note that the limited data size of CQ500 forbids training a effective deep learning model from scratch, reinforcing the importance of the foundation model in label efficiency, which is further studied in Section \hyperref[sec:label_efficiency]{"Label efficiency"}. Interestingly, when comparing performances across different datasets, \Cref{fig:overview}e demonstrates that the AUCs of the in-domain fine-tuned model on the external dataset even exceed the AUCs achieved on the internal dataset. For instance, the fine-tuned models consistently obtained AUCs greater than 0.90 in all the hemorrhage detection tasks on RSNA dataset, surpassing the AUCs on NYU Langone data. This may be attributed to the higher label quality in radiologist-reviewed datasets, for which label noise may be better controlled by comparison to EHR-derived labels.

In the full external validation without any site-specific fine-tuning (illustrated in \Cref{fig:overview}c), we evaluated classification models fine-tuned on the NYU Langone training set, as-is, on the held-out validation sets from each external dataset. \Cref{fig:overview}f compares performance between external validation and in-domain fine-tuning. Results show that, for the NYU Long Island and RSNA datasets —where the training set used for fine-tuning includes a sufficient number of high-quality labeled samples— in-domain fine-tuning does enhance the model performance. However, on the CQ500 dataset, with only 1,120 training samples, the in-domain fine-tuned model performs worse than the model transferred from NYU Langone, especially for EDH and SDH, which have a greater class imbalance. These comparisons highlight two typical use cases for foundation models depending on the availability of labeled data for fine-tuning. Additionally, comparing the first row of \Cref{fig:overview}e and external validation in \Cref{fig:overview}f, the fine-tuned model on NYU Langone achieves similar AUC values on both internal and external datasets, indicating robust generalizability to external data.











\subsection*{Label efficiency of few-shot classification performance}
\label{sec:label_efficiency}

Another key advantage of the foundation model is its ability to facilitate transfer learning and fine-tuning tasks with minimal labeled data. For example, as shown in \Cref{fig:overview}c, the CQ500 dataset contains only 1,585 scans. Despite the small dataset size, fine-tuning our foundation model on CQ500 achieves promising results, with an AUC of 0.863. 

To systematically evaluate the label efficiency of our foundation model, we also assess the generalization capabilities of models on new tasks given a limited number of examples within the paradigm of few-shot learning, where only $K$ positive and negative samples each are used for training in each task. Since the quality of few-shot learning is largely determined by the sampled $K$-shots training data, we re-sampled and re-trained the model 5 times for calculating means and confidence intervals. As expected, \Cref{fig:fewshot} shows that performance improves as more data is used for training, with narrower confidence intervals. Surprisingly, even with a small number of examples (e.g., 512 total, with $K=256$), the model achieves performance comparable to training with the full dataset, which contains over at least 16 times more training examples in the RSNA. Notably, for tasks like detecting IVH in the RSNA dataset, the 8-shots model achieves an AUC above 0.90, a result that rivals full-data training. These findings suggest that our foundation model has learned diverse and expressive features/representations during SSL pre-training, making it highly effective for new tasks even when trained on small labeled datasets.

\begin{figure}[t]
    \centering
    \includegraphics[width=0.24\textwidth]{figures/fig4/pretrain_size_NYU_Langone_AUC.pdf} 
    \includegraphics[width=0.24\textwidth]{figures/fig4/pretrain_size_NYU_Long_Island_AUC.pdf}
    \includegraphics[width=0.24\textwidth]{figures/fig4/pretrain_size_RSNA_AUC.pdf}
    \includegraphics[width=0.24\textwidth]{figures/fig4/pretrain_size_CQ500_AUC.pdf}
    \includegraphics[width=0.24\textwidth]{figures/fig4/pretrain_size_NYU_Langone_AP.pdf}
    \includegraphics[width=0.24\textwidth]{figures/fig4/pretrain_size_NYU_Long_Island_AP.pdf}
    \includegraphics[width=0.24\textwidth]{figures/fig4/pretrain_size_RSNA_AP.pdf}
    \includegraphics[width=0.24\textwidth]{figures/fig4/pretrain_size_CQ500_AP.pdf}
    \caption{\textbf{Performance for Different Percentage of Pre-training Samples (Mean):} we compare the label efficiency in terms of different percentage of pre-training data for MAE vs. DINO. The $95\%$ CI are plotted in colour bands and the centre points of the bands indicate the mean value. We show that although DINO present higher label efficiency plot, both MAE and DINO efficiently scale up on downstream performance as more pre-training data is incorporated.}
    \label{fig:scaling_law}
\end{figure}


\subsection*{Scaling up pre-training data}

Scaling laws have proven effective in enhancing the performance of foundation models by increasing the size of the training dataset~\cite{kaplan2020scalinglawsneurallanguage}. This phenomenon is not only observed in natural language and image domains~\cite{zhai22scalingvit, pmlr-v202-dehghani23a}, but also extends to medical imaging~\cite{zhou2023foundation, li2024well}. As shown in \Cref{fig:scaling_law}, scaling up the foundation model by incorporating more data during self-supervised pre-training significantly improves downstream tasks performances. We compared models pre-trained with varying proportions of the available data --- $10\%$, $30\%$, and $100\%$ (full dataset), observing that larger pre-training datasets consistently led to better downstream task performance. These findings highlight the potential of leveraging more data to achieve superior results, further suggesting the value of multi-institutional collaboration and federated approaches to aggregating larger datasets to enhance model quality. Noticeably, the performance for CQ500 does not change a lot from $10\%$ to $30\%$, but $100\%$ gives a sudden performance improvement, this indicates that for smaller datasets like CQ500, scaling up the data size is crucial for learning meaningful representations.

\subsection*{Visual Interpretation}
To gain insight into the features learned through self-supervised pre-training and supervised fine-tuning of the foundation model, we visualize the attention maps within the Vision Transformer (ViT), as shown in \Cref{fig:attention_interpretation}. These heatmaps highlight the regions where the ViT model focuses most strongly. In the second column, we see that the pre-trained foundation model captures generic brain features, with dark red indicating attention on abnormal ventricular shapes and green marking areas of hemorrhage. After fine-tuning on specific tasks, the ViT’s attention becomes more focused on patterns relevant to each disease. For instance, in the edema task (third column), the heatmap extends across most of the brain, reflecting generalized swelling. For ADRD (fourth column), the model emphasizes regions of ventricular enlargement and cerebral atrophy. Multiple hemorrhages are also present in this sample, with attention covering both the IPH in the dense central region (fourth column) and extending toward the left end of the ventricle where IVH appears (fifth column). In the case of SAH (sixth column), the attention map is less prominent due to the small, peripheral area of the SAH in the lower part of the slice, although the model still predicts it accurately.

The comparison between the pre-trained and fine-tuned ViT explains the performance difference between linear probing and fine-tuning (shown in Supplementary \Cref{fig:probing-comparison-perpath}, as end-to-end fine-tuning allows the model to learn task-specific features more effectively. Details on the computation of the visualized attention maps are provided in Section \hyperref[sec:methods]{“Methods”}.

\begin{figure}[t]
    \centering
    \includegraphics[height=0.6\linewidth]{figures/fig5/fig5_2.pdf}
    \caption{Visualization of ViT attentions on the scan.}
    \label{fig:attention_interpretation}
\end{figure}


\section*{Discussion}
Despite advances in disease detection using 3D head CT scans, current solutions are limited by the availability of annotated data and the complex, task-specific design requirements of network architectures. These constraints hinder the broader application of machine learning in clinical disease detection. To address this, we developed a foundation model, trained on a large unlabeled dataset, to enable fine-tuning for multiple tasks with minimal labeled data under a unified network architecture. 

Highly accurate detection of intracerebral hemorrhages without delay is a critical clinical issue for the diagnostic decision making and treatment in an emergency room \cite{Hemphill2015-yw,Qureshi2009-ve}. Our results indicate that 3D Head CT scans can also be used to help identify hemorrhage subtypes and, more interestingly, etiology. High performances and generalizability observed by our model in detecting intracerebral hemorrhage have a potential to greatly assist in pre-hospital and early hospital management of blood pressure. This is particularly important given that early blood pressure control is a key factor in preventing hematoma expansion and improving patient outcomes~\cite{Macellari2014-zj,Morotti2022-lc}.

This approach is also particularly valuable for extending detection capabilities to new diseases in CT imaging. For example, early detection of ADRD with deep learning has traditionally relied on MRI scans \cite{Li2019-jx, pmlr-v116-liu20a, Xue2024}. However, access to MRI machines is costly and often restricted by patients’ geographic location and socioeconomic status~\cite{https://doi.org/10.1002/neo2.10}. Head CT, in contrast, is fast, accessible, and is the first-line imaging test in emergency and diagnostic settings. Our foundation model enables more accessible ADRD detection using head CT scans. This advancement holds the potential for expanding early ADRD detection in common public health settings for the older population~\cite{lin_dementia_2020, kim2021racial}, such as emergency rooms, as well as in underserved communities nationally and internationally in which CT is more available than MRI. Similarly, our model could facilitate the development of detection tools for other conditions, such as cancers and neuroinfectious diseases, thus supporting population health on a broader scale.

Our study demonstrates that this pre-trained foundation model significantly outperforms models trained from scratch and other CT foundation models on the same labeled data. Moreover, it exhibits strong performance even with limited data, as shown in few-shot learning experiments, and suggests promising potential for scaling up with larger datasets. In clinical practice, head CT scans are typically acquired using heterogeneous protocols, including variations in slice thickness and scanner modalities. A robust foundation model for CT should generalize effectively across these diverse protocols. In this study, we utilized scans with slice thicknesses ranging from 0.5 mm to 5 mm and data from two major manufacturers (Siemens and Toshiba) to develop and assess the generalization capabilities of our foundation model. As illustrated in Supplementary \Cref{fig:batch_effect}, the embeddings produced by the foundation model show separability based on scanner manufacturer and slice thickness, likely reflecting variations in protocol distribution. However, by comparing the distribution of “All” patients to that of positive cases for each condition, we observe that the embeddings do not tend to collapse or bias towards a certain protocol. Supplementary \Cref{fig:thickness-ablation} further demonstrates that fine-tuned models achieve comparable performance across scanner protocols. Detailed per-task performance results are provided in Supplementary \Cref{fig:slice_thickness_per_pathology,fig:manufacturer_per_pathology}.  Additionally, in comparisons of Toshiba and Siemens scanners, we noted a systemically higher prevalence of positive cases across all tasks in Toshiba scans, leading to a modestly higher AUC in specific instances. Despite these variations, our foundation model demonstrates robust generalization capabilities across diverse CT protocols, highlighting its potential for broad clinical application.

However, our evaluation is limited by label noise in real-world datasets. Labels derived from electronic health records (EHRs) can suffer from missing or incomplete information. This issue is evidenced by the model’s lower performance on NYU Langone data compared to RSNA data, where labels were rigorously reviewed by radiologists. Another limitation is that, due to constraints on training samples and computational resources, our model does not yet fully explore the potential of scaling laws. The 361K scans used for pre-training represent the entirety of CT scans available from a single large clinical institution, highlighting the need for multi-institutional collaborations to enhance the dataset diversity and volume. With greater computational resources, we could also scale up the model’s size, resolution of image patches, and number of tokens used in the ViT architecture, potentially improving performance for detecting conditions with small spatial manifestations, such as subarachnoid hemorrhage (SAH).

While our current results primarily focused on disease detection, our foundation model holds significant potential for advancing disease prognosis analysis. For instance, the prediction of decompensation, particularly hemorrhagic expansion, is an important potential use of the foundation model and may lead to the development of novel hyperacute treatment strategies~\cite{hematoma_expansion}. Additionally, critical applications in acute ischemic stroke, such as predicting hemorrhagic transformation and the development of malignant edema can benefit from the foundation model. Beyond acute conditions, the foundation model can potentially also be used to predict the development of ADRD~\cite{Zhu2024-zd}.


\section*{Methods}
\label{sec:methods}
\subsection*{Datasets}
\subsubsection*{Dataset for pre-training foundation model}
We utilized a large-scale head CT scan dataset from NYU Langone, consisting of 499,084 scans across 203,665 patients, collected between 2009 and 2023. These scans were acquired using Siemens and Toshiba machines. We included all the non-contrast head CT scans with ranging from 0.5mm to 5mm, kVp values between 70 and 150, and convolution kernels Hr/Qr/J with sharpness levels of 35-45. We filtered out corrupted scan series with missing DICOM files and those containing less than 10 slices, resulting in 451,298 scans. We partitioned these scans by the patient IDs into training, validation, and held-out validation sets in an 8:1:1 ratio to avoid the leakage of scans from the sample patient. As illustrated in \Cref{fig:overview}a, this led to training, validation and held-out validation set with 361,663, 44,886 and 44,749 scans, respectively. The scans in the training set were used to train the foundation model.

\subsubsection*{Datasets for downstream tasks}
We evaluated our model using four datasets: one \emph{in-domain (ID)} dataset from NYU Langone and three \emph{out-of-domain (OOD)} datasets from NYU Long Island, the RSNA Challenge, and the public CQ500 dataset. Each dataset includes multiple head CT disease detection classes, with some classes abbreviated as follows: Hydrocephalus (HCP), Dementia (ADRD), Intraparenchymal Hemorrhage (IPH), Intraventricular Hemorrhage (IVH), Subdural Hemorrhage (SDH), Epidural Hemorrhage (EDH), Subarachnoid Hemorrhage (SAH), and Intracerebral Hemorrhage (ICH). These classes can have co-occur in the same Head CT scan. The characteristics of the patients are shown in \Cref{tab:characteristic}. We split all datasets by patients to avoid information leakage, Further dataset details of our dataset are provided below.

\paragraph{NYU Langone - 10 tasks} The NYU Langone main campus dataset serves as the internal ID dataset for downstream evaluation. As depicted in \Cref{fig:overview}b, patient health status was derived from Electronic Health Records (EHR) within a 3-month window centered around the scan date, with conditions defined by ICD-10 diagnostic codes and medications, outlined in Supplementary \Cref{tab:disease_definition}. This cohort includes 270,205 scans from 66,801 patients with valid EHRs, covering 10 classes: Tumor, HCP, Edema, ADRD, IPH, IVH, SDH, EDH, SAH, and ICH. This cohort was partitioned following the same split used used during pre-training: matched patients within the training, validation, and held-out subsets of the SSL pre-training phase were assigned to the corresponding sets of the supervised fine-tuning phase. This resulted in 217,109, 26,609, and 26,487 scans in the training, validation and test set, respectively.
\paragraph{NYU Long Island - 10 tasks} NYU Long Island data was acquired in Long Island hospital which used to be a separate hospital, severing as an OOD dataset. This dataset includes 22,158 samples with 10 classes, labeled similarly to the NYU Langone dataset using ICD-10 codes. It is partitioned into an 8:1:1 train-validation-test split.
\paragraph{RSNA - 5 tasks} The RSNA Head CT Challenge dataset~\cite{flanders_construction_2020} serves as a public external OOD dataset, collected from Stanford University, Thomas Jefferson University, Unity Health Toronto, and Universidade Federal de São Paulo (UNIFESP). The dataset, initially provided as 2D slices, was reorganized by subject ID, retaining subjects with complete slice data. After preprocessing, the dataset consists of 10,579 samples across five classes: Any (any hemorrhage type), IPH, IVH, SAH, and SDH. Dataset labels were assigned by 60 volunteers from the American Society of Neuroradiology (ASNR). We partitioned this cohort into an 8:1:1 train-validation-test split.
\paragraph{CQ500 - 10 tasks} The CQ500 Head CT dataset~\cite{CQ500} serves as another public external OOD dataset, collected from multiple centers in India. This dataset includes 1,585 samples including varying slice thickness across 10 selected classes: ICH, IPH, IVH, SDH, EDH, SAH, BleedLocation-Left, BleedLocation-Right, MidlineShift, MassEffect. Each scan was labeled by three senior radiologists, and the cohort was split into an 8:1:1 train-validation-test ratio.

\subsubsection*{Label acquisition from electronic health records}

As illustrated in \Cref{fig:overview}b, we labeled head CT scans from NYU Langone and Long Island Hospital using electronic health records (EHR). For each head CT, we retrieved an EHR snippet for the corresponding patient based on their Medical Record Number (MRN), starting from the time of the scan and covering a 90-day period. We then checked for the presence of any diagnosis codes (ICD-10 codes) and medication records, within this EHR snippet that matched the predefined definitions for each disease, allowing us to create binary labels for each condition. The complete list of ICD-10 codes and the medications used for disease definitions is provided in Supplementary \Cref{tab:disease_definition}.


\subsubsection*{Data preprocessing}
For the NYU Langone and Long Island datasets, we converted the DICOM files into NIfTI format using MRIcroGL dcm2nii~\cite{li_first_2016}, standardizing the file format with those from the RSNA and CQ500 datasets. Given the variability in scan protocols, which can result in differing orientation, resolutions and slice thicknesses, we applied spatial normalization to transform the volume orientation to right-anterior-superior (RAS) angle and resample with bicubic interpolation to the isotropic resolution ratio of $(1.0, 1.0, 1.0)$ in the world coordinate system. This ensures uniform pixel spacing across all scans and axes. 

Head CT scans use Hounsfield Units (HU) to represent various tissue types, which span a broad range of values. To better capture tissue characteristics, we applied three windowing ranges, each emphasizing specific tissue types: (40, 80) for soft tissue, (80, 200) for contrast-enhanced tissues and blood vessels, and (600, 2800) for bone. We then stacked the values from each window, producing a 3-channel 3D volume that enhances the representation of these key tissues. Similar strategy has been applied in Chilamkurthy \textit{et al.}~\cite{CQ500}.

To ensure compatibility with model input requirements, we transformed each volume into the desired size. We first padded or cropped each volume to a size of (224, 224, 224), preserving the whole brain across all axes. Then for training, we applied data augmentations detailed in Supplementary Section \hyperref[sec:dataaug_details]{``Data Augmentation details’’}; for evaluation, we center-cropped the volumes to (192, 192, 192). Finally, we resized each volume to (96, 96, 96) as the input size for the model.


\subsection*{Model architecture}
\label{sec:model_architecture}
Numerous studies have demonstrated that ViT can effectively learn high-quality representations for 2D medical images at scale~\cite{chen_towards_2024, zhou2023foundation, MedSAM, Vorontsov2024, Azizi2023}. Our study extends this by exploring whether representations of 3D medical images (specifically head CT scans) can also be effectively learned at scale through the direct compression of 3D patches as model input. We employ the Vision Transformer (ViT)\cite{dosovitskiy2020vit} as the volume encoder for our foundation model, as well as for baseline comparisons in all experiments. Our model uses a ViT-Base architecture with an embedding dimension of 768, 12 self-attention layers, 12 heads, and feed-forward layers with a hidden size of 3072. We apply sine-cosine absolute positional encoding\cite{NIPS2017_3f5ee243} across all pre-training and fine-tuning stages.

For the 3D input volume, instead of creating 196 patches of size $16 \times 16$ from a $224 \times 224$ image as in standard 2D ViT, we segment $96 \times 96 \times 96$ 3D volumes into 512 patches of size $12 \times 12 \times 12$ for ViT input. This customized patch design considers the trade-off between performance and computational cost. As shown in Supplementary \Cref{fig:patches-ablation}, our model outperforms a version using 216 patches of size $16 \times 16 \times 16$, indicating that smaller, more numerous patches enhance model performance. This supports the importance of capturing fine-grained features in 3D medical imaging, consistent with prior findings~\cite{Tang_2022_CVPR, li2024well}. However, computational costs increase significantly with respect to $s$ ($s$ defined as patch size reducing factor), at a rate of $O(s^{6})$, due to the cubic growth of patch numbers in 3D and the quadratic growth in self-attention computation (\Cref{apd:self_attention_rate}). To balance performance with computational efficiency, we adopt 512 patches of $12 \times 12 \times 12$ as the optimal input size for ViT in our foundation model.



\subsection*{Self-supervised pretaining}
Self-Supervised Learning recently has been widely adopted as learning framework for building medical foundation models~\cite{chen_towards_2024, zhou2023foundation, Huang2023, azizi21big, Vorontsov2024}. While previous works mainly focus on directly applying existing self-supervised learning algorithms on 2D medical images, we explore how to effectively leverage these algorithms with 3D medical images. Specifically, we explore two main branches of self-supervised learning framework for building our 3D foundation model --- discriminative with self-distillation (DINO) and masked image modeling (MAE).

\paragraph{Self-Distillation Modelling (DINO)}
DINO\cite{caron2021emerging, oquab2023dinov2} is a self-supervised learning method shown promising and robust downstream evaluation performance in previous studies on different areas~\cite{chen_towards_2024, Vorontsov2024}. DINO uses a student-teacher framework for learning meaningful representations. Both student and teacher networks share the same model architecture, while the teacher’s parameters are updated using an exponential moving average of the student’s parameters. Each input image is augmented multiple times to create different views as student and teacher networks input. Specifically, we applied random global and local crops, random flips, shifts in intensity and contrasts, and Gaussian blurs for augmented views. Then the student’s output is trained to match the teacher’s output using a distillation loss, ensuring similar representations for different views of the same image. We pre-trained the ViT in the DINO framework for $1000$ epochs with batch size at $64$ per GPU and an AdamW~\cite{loshchilov2018decoupled} optimizer ($\beta_1=0.9, \beta_2=0.95$, $0.05$ weight decay). A base learning rate $3\times10^{-4}$ was applied combined with cosine scheduling and a linear warmup on the first $5$ epochs. During pre-training, two global augmentations and three local augmentations were applied to enable ViT to learn both global and local features of the head CT. Because small region of brain is likely to be dissimilar, we observed cropping too small brain regions would cause unstable model training by making the learning task to be too challenging. Therefore, we first resample the input images to $224\times224\times224$. Subsequently, we perform multi-scale cropping by extracting both global and local crops regions, ranging from $112\times112\times112$ to $224\times224\times224$ for global crops and from $64\times64\times64$ to $112\times112\times112$ for local crops. After the cropping, all cropped regions are resampled to $96\times96\times96$. For training on $100\%$ data, convergence on the performance for downstream tasks is observed at around $300$ epochs, which took around one week on four 80GB NVIDIA A100 GPUs. 

\paragraph{Masked Image Modeling (MAE)}
MAE~\cite{He2021MaskedAA} is another self-supervised learning method for vision tasks, inspired by masked language modeling in Natural Language Processing (NLP). MAE is trained to reconstruct randomly-masked patches via an encoder-decoder architecture, where the encoder processes visible patches of an image, while the decoder reconstructs the image from encoded patches and mask tokens. Specifically, we randomly masked the patches from each volume with a probability of 0.75. Mean squared error (MSE) loss is optimized to minimize the difference between the reconstructed volume and the original volume. We pre-trained the ViT in MAE framework for $400$ epochs with batch size at 64 per GPU and an AdamW~\cite{loshchilov2018decoupled} optimizer ($\beta_1=0.9, \beta_2=0.95$, $0.05$ weight decay). A base learning rate $1.5\times10^{-3}$ was applied combined with cosine scheduling and a linear warmup on the first $5\%$ steps,  For training on $100\%$ data, convergence is observed at around $250$ epochs, which took around 4 days on four 80GB NVIDIA A100 GPUs for MAE. Similar to DINO, MAE has shown success in learning robust representations in many previous works~\cite{ravi2024sam2, tong2022videomae, gupta2023siamese, zhou23self, huang2022masked, cong2022satmae, chen23masked}, including the studies on both 2D and 3D data.

We compared the performance on downstream tasks between two versions of foundation models pre-trained using DINO and MAE, as shown in \Cref{fig:scaling_law} and Supplementary \Cref{fig:probing_comparison,fig:probing-comparison-perpath,fig:probing-comparison-perpath-dino}. The results indicate that DINO consistently outperforms MAE across all datasets. Based on this finding, we selected the DINO-pre-trained model as our final foundation model.


\subsection*{Evaluation setting}
\paragraph{Baseline comparisons}
Since no prior foundation models have been specifically trained on 3D Head CT for direct comparison, we benchmark our model against Merlin~\cite{blankemeier2024merlinvisionlanguagefoundation} and Google CT Foundation model~\cite{yang2024advancingmultimodalmedicalcapabilities} to highlight the advantages of our domain-specific foundation model. Merlin is a 3D Abdomen CT foundation model pre-trained on vision-language pairs with contrastive learning~\cite{radford2021learningtransferablevisualmodels} and ICD code prediction task, where 6+ million images from 15,331 CTs, 1.8+ million diagnostic ICD codes from EHR, and 6+ million tokens from radiology reports are used. Different from our model architecture, Merlin used ResNet-152 ($\sim60.4$M Parameters) as vision model with reshaped image size of $224\times224\times160$. The performance comparison between our model and Merlin is shown in Supplementary \Cref{fig:radar-comparison-merlin}, where our model shows substantial improvement across most datasets and diseases. Google CT Foundation model is trained on a comprehensive private dataset comprising 527,078 CT studies with associated radiology reports from 430,772 patients. The model is first trained by Contrastive Captioning with CoCa~\cite{yu2022coca} on 2D medical images and then adapting to CT by training on series of CT slices with VideoCoCa~\cite{yan2023videococavideotextmodelingzeroshot}. The performance comparison between our model and Google CT Foundation model is shown in Supplementary \Cref{fig:probing-comparison-gemini}, where our model shows a consistent improvement across the board. We additionally show comparison of our model against model trained from scratch in \Cref{fig:overview} and Supplementary \Cref{fig:radar-comparison-merlin}, where the overall significantly improved performance shows the effectiveness of our pre-training strategies on 3D Head CT images.

\paragraph{Fine-tuning and Probing classification evaluation}
We assessed pre-trained model performance through full fine-tuning (updating all weights) and various probing methods (updating only the classification layers). For both approaches, images were normalized to isotropic spacing, transformed to three HU interval channels, and reshaped to $3\times96\times96\times96$. The entire transformed 3D image was then input into the ViT model for feature extraction, followed by an additional classification layer for downstream tasks. Probing utilized two strategies: linear probing, which adds a linear layer atop the ViT backbone, and attentive probing, which incorporates an attention layer. Attentive probing is chosen since MAE does not use \texttt{[CLS]} token as the learning objective. Linear probing only relies on \texttt{[CLS]} token to perform classification and attentive probing explores the interaction among all tokens~\cite{Chen2024}. Given the imbalances of downstream task labels, we randomly sampled a balanced subset from the training set per epoch, consisting of 5,000 samples (when fine-tuning on the NYU Langone, NYU Long Island, and RSNA datasets), and 500 samples when fine-tuning on CQ500. We trained all methods using the AdamW~\cite{loshchilov2018decoupled} optimizer with a cosine learning rate scheduler, a learning rate of $1\times10^{-5}$ for backbone and $1\times10^{-3}$ for classification layers, cross-entropy loss, and a maximum of 10 epochs. The main evaluation result with linear probing is shown in \Cref{fig:overview} with fine-tuning and probing comparison shown in Supplementary \Cref{fig:probing_comparison} for average performance across all diseases and Supplementary \Cref{fig:probing-comparison-perpath} for per disease performance. The result indicates that probing achieves performance levels close to full fine-tuning, underscoring the high quality of learned representations in our model.

For fine-tuning model from scratch, as we observe more unstable model performance from different hyper-parameters across different datasets, we perform hyper-parameters sweep across different setting and report the best performance model. The sweeping hyper-parameters are lr=\{1e-3, 1e-4, 1e-5\}, weight decay=\{0.01, 0.05, 0.0001, 0.00001\}, epochs=\{10, 15, 30, 50\}, optimizer=\{SGD, Adam, AdamW\}.

\paragraph{Few-shots classification evaluation}
In order to evaluate the effectiveness of our model under scare label conditions, we applied few-shots learning where each class is only sampled $K$-times. Specifically, we chose $K=8, 16, 32, 64, 128, 256$, where the data is sampled such that positive and negative samples equal to $K$ for each disease. Few-shot training was performed using full fine-tuning with the same hyper-parameter settings. While we also attempted some other commonly used few-shots classification methods such as k-nearest neighbors (KNN), Simple Shots~\cite{want19simpleshot} and Prototypical Networks~\cite{jake17proto}, we did not observe performance improvement on our datasets over full fine-tuning. The main evaluation for few-shots classification is present in \Cref{fig:fewshot}, where we observed our model can already reach performance close to full fine-tuning with only $K=256$ samples. This demonstrates the effectiveness of our model under scare data training regime.

\subsection*{Visual Interpretation}
\label{sec:visual_interpretation}
Self-attention enables the Vision Transformer (ViT) to integrate information across the entire volume, even in its lowest layers. To analyze the relationships among different patches within the CT volumes, we calculate the average spatial distance over which information is integrated, using the attention weights. 

Let $\mathbf{A}^{(l,h)} \in \mathbb{R}^{N \times N}$  represent the attention weight matrix for the $h$th attention head in the $l$th layer of ViT and $N$ is the number of patches in a CT volume. $d(i, j)$ denotes the spatial distance between patch $i$ and patch $j$ within the 3D volume. The attention distance for each patch $i$ is computed as a weighted average distance to other patches, based on the attention weights:
\begin{equation}
    \centering
    D_i^{(l, h)} = \sum_{j=1}^{N} A^{(l, h)}_{ij}  d(i, j)
\end{equation}

We visualize the average attention distances across all heads and layers for every patch in the volume in \Cref{fig:attention_interpretation}. This “attention distance” serves as an estimate of the ViT’s receptive field within the CT volumes, indicating the regions of the brain that the model focuses on. This visualization helps illustrate how the model integrates information across spatial areas to capture meaningful patterns within the volume.

\subsection*{Statistical analysis}

In each experiment, we report the mean and confidence interval, calculated by bootstrapping the held-out validation set 100 times. For few-shot learning, where model variance is also influenced by the specific training data samples, we repeated the training and evaluation process five times with randomly sampled training data, reporting the mean and confidence interval of the resulting metrics. For all statistical significance (p-values) reported in this study, we used a two-sided paired permutation test with $1,000$ permutations to assess the performance difference of two compared models.

\subsection*{Computing Hardware Software}
All experiments are performed under Python (v3.8.11), PyTorch (v2.4.1), CUDA (12.1) and MONAI (v1.2.0). We extend ViT, MAE, DINO implementation from original their corresponding repositories (\url{https://github.com/facebookresearch/mae}, \url{https://github.com/facebookresearch/dino}) to match our need for 3D CT image encoding. For comparison with Merlin~\cite{blankemeier2024merlinvisionlanguagefoundation}, we integrated their original model weight checkpoints and model backbone code (\url{https://github.com/louisblankemeier/merlin}) to our downstream fine-tuning code base. ResNet50-3D~\cite{hara3dcnns} from (\url{https://github.com/kenshohara/3D-ResNets-PyTorch/tree/master}) is integrated to our code base for evaluation. All plots and figures were created by Matplotlib (v0.1.6) and Seaborn (v0.13.2). All downstream experiments were conducted on single 80 GB NVIDIA A100 GPU (graphics processing unit). All pre-training experiments were conducted on four 80 GB NVIDIA A100 GPUs.




\section*{Data availability}
The internal clinical data involved in the study is unavailable due to privacy concerns and institutional policy. Public dataset RSNA is available from \url{https://www.kaggle.com/competitions/rsna-intracranial-hemorrhage-detection}. Public dataset CQ500 is available from \url{https://www.kaggle.com/datasets/crawford/qureai-headct}.  The original data is provided as DICOM files. We converted each scan from DICOM to NIfTI files and removed the scans with missing slices for creating 3D imaging datasets in our evaluation. We use all slice thickness scan protocols in each scan (e.g. thin, plain thin, and plain scan) for CQ500, hence providing a more exhaustive evaluation on our model adaptability on different slice thickness for scan.

\section*{Code availability}
The code for pre-training, fine-tuning and evaluation of the foundation model is available on \url{https://github.com/NYUMedML/headCT_foundation}. Due to the possibility of inferring patient face from headCT data, the model weights are only available upon request after signing institutional agreement. Requests for model weight should be sent to the corresponding author and the NYU Langone Data Sharing Strategy Board (DSSB) Committee (DataSharing@nyulangone.org).

\section*{Acknowledgements}
W.Z., H.H., L.C., A.V.M. and N.R. were supported by the National Institute On Aging of the National Institutes of Health under Award R01AG085617. W.Z. H.H., B.Y. and L.C. received partial support from NSF Award 1922658. N.R. and A.V.M. were also partially supported by the National Institute On Aging of the National Institutes of Health under Awards R01AG079175 and P30AG066512.

\documentclass{MITstyle}

%\usepackage[table]{xcolor}
\usepackage{chngcntr}
\usepackage{hyperref}
\usepackage{microtype}

\title{A Lightweight and Extensible Cell Segmentation and Classification Model for Whole Slide Images}

\author{Nikita Shvetsov~$^{1, }$\footnote{Correspondence e-mail: nikita.shvetsov@uit.no}, Thomas K. Kilvaer~$^{2, 3}$, Masoud Tafavvoghi~$^{4}$, Anders Sildnes~$^{1}$, \\ Kajsa Møllersen~$^{4}$, Lill-Tove Rasmussen Busund~$^{5, 6}$, Lars Ailo Bongo~$^{1}$ \\
%
\vspace{1em} % Space between authors and afilliations
%
\normalfont{\small $^{1}$Department of Computer Science, UiT The Arctic University of Norway}\\
\normalfont{\small $^{2}$Department of Oncology, University Hospital of North Norway}\\
\normalfont{\small $^{3}$Department of Clinical Medicine, UiT The Arctic University of Norway}\\
\normalfont{\small $^{4}$Department of Community Medicine, UiT The Arctic University of Norway}\\
\normalfont{\small $^{5}$Department of Medical Biology, UiT The Arctic University of Norway} \\
\normalfont{\small $^{6}$Department of Clinical Pathology, University Hospital of North Norway} %\vspace{2em}
}

\begin{document}
\maketitle

\section*{Abstract}

% \begin{abstract}
% Developing clinically useful cell-level analysis tools in digital pathology remains challenging due to limitations in dataset granularity, inconsistent annotations, computational demands of advanced models, and difficulties in integrating new technologies into clinical workflows. To address these challenges, we propose a multi-faceted solution that enhances data quality, model performance, and usability to create a lightweight and extensible cell segmentation and classification model.

% First, we update data labels by employing a cross-relabeling process that refines the labels of two existing datasets, PanNuke and MoNuSAC, to create a new unified dataset with enhanced granularity, encompassing seven distinct cell types. Second, we leverage the H-Optimus foundation model as a fixed encoder to improve feature representation for simultaneous cell segmentation and classification tasks. Third, to address the computational demands of foundation models, we employ knowledge distillation to reduce model size and complexity while maintaining comparable performance. Finally, to facilitate integration into clinical workflows, we integrate the distilled model into the QuPath software, a widely used open-source platform in digital pathology.

% Our results demonstrate improvements in cell segmentation and classification performance using the H‑Optimus-based model compared to a CNN-based model. Specifically, the average $R^2$ improved from 0.575 to 0.871, and the average $PQ$ score improved from 0.450 to 0.492, indicating better alignment with actual cell counts and enhanced segmentation and classification quality. Furthermore, the distilled student model maintains performance comparable to the larger foundation model while reducing the parameter count by a factor of 48.
% Overall, by reducing computational complexity and integrating it into existing workflows, the proposed approach may significantly impact diagnostic processes, reduce the workload of pathologists, and contribute to improved patient outcomes. Though our approach shows potential enhancements in efficiency and usability of cell segmentation and classification models in digital pathology, extensive validation is needed to deploy these models in clinical practice.
% \end{abstract}

%%% shortened abstract
\begin{abstract}
Developing clinically useful cell-level analysis tools in digital pathology remains challenging due to limitations in dataset granularity, inconsistent annotations, high computational demands, and difficulties integrating new technologies into workflows. To address these issues, we propose a solution that enhances data quality, model performance, and usability by creating a lightweight, extensible cell segmentation and classification model. 

First, we update data labels through cross-relabeling to refine annotations of PanNuke and MoNuSAC, producing a unified dataset with seven distinct cell types. Second, we leverage the H-Optimus foundation model as a fixed encoder to improve feature representation for simultaneous segmentation and classification tasks. Third, to address foundation models' computational demands, we distill knowledge to reduce model size and complexity while maintaining comparable performance. Finally, we integrate the distilled model into QuPath, a widely used open-source digital pathology platform. 

Results demonstrate improved segmentation and classification performance using the H-Optimus-based model compared to a CNN-based model. Specifically, average $R^2$ improved from 0.575 to 0.871, and average $PQ$ score improved from 0.450 to 0.492, indicating better alignment with actual cell counts and enhanced segmentation quality. The distilled model maintains comparable performance while reducing parameter count by a factor of 48. By reducing computational complexity and integrating into workflows, this approach may significantly impact diagnostics, reduce pathologist workload, and improve outcomes. Although the method shows promise, extensive validation is necessary prior to clinical deployment.
\end{abstract}
\clearpage

\section{Introduction}
In digital pathology, accurate segmentation and classification of cells are crucial for many diagnostic, prognostic, and predictive analyses \cite{Jaber_Beziaeva_etal._2019,Lin_Pan_etal._2022,Park_Ock_etal._2022,Shen_Choi_etal._2024}. Nowadays, developments in computational pathology offer multiple solutions \cite{H._Qu_P._Wu_etal._2020,Javed_Mahmood_etal._2020} to utilize cell-level datasets to train machine learning models that solve these problems. The quality and specificity of training datasets are critical for robust and accurate models. Adhering to the principle of "garbage in, garbage out", it is essential to ensure that these datasets are extensively and accurately labeled with distinct classes that reflect the diverse biological characteristics of different cell types. Unfortunately, the number of open-source datasets comprising such high-quality annotations is limited. Existing cell segmentation datasets \cite{Gamper_Koohbanani_etal._2019,Graham_Vu_etal._2019,Verma_Kumar_etal._2021} may offer extensive annotations for certain cell types while providing more general labels for others. For example, in PanNuke, which is one of the largest open-source datasets comprising labeled cells, various types of morphologically and functionally different inflammatory cells like macrophages and lymphocytes are clustered in a broad "inflammatory" class. Consequently, these classes are frequently omitted from analyses or aggregated into broader meta-classes \cite{Gamper_Koohbanani_etal._2020} and likely interfere with other cell classes included in the dataset. This and similar inconsistencies in annotation granularity limit the ability of machine learning models to learn the comprehensive and nuanced features necessary for accurate cell segmentation and classification. To address these challenges, methods for refining and standardizing dataset annotations are essential to enhance the quality of training data.

A complementary approach to mitigate the absence of high-quality training data is the use of foundation models. Foundation models as encoders are defined as large-scale, versatile networks pre-trained on vast, diverse datasets using self-supervised learning, contrasting with convolutional neural network (CNN) pre-trained encoders that rely on supervised learning with labeled data. In practice, foundation models leverage enormous amounts of weakly or unlabeled data from millions of whole slide images (WSIs) and employ self-attention mechanisms to capture long-range dependencies and global context \cite{Chen_Ding_etal._2024,Saillard_Jenatton_etal._2024,Vorontsov_Bozkurt_etal._2024,Xu_Usuyama_etal._2024}. As a consequence, foundation models are able to produce transferable feature representations across different cell types and tissue environments. The feature representations can be leveraged by decoder networks to produce segmentation masks and pixel-level classifications. Because foundation models have comprehensive feature representations, they can be effectively fine-tuned using much smaller amounts of cell-level data compared to the large datasets needed to train models from scratch. Furthermore, foundation models incorporate adversarial training elements or contrastive learning \cite{Chen_Ding_etal._2024,Xu_Usuyama_etal._2024}, enhancing their resilience and adaptability by exposing them to challenging and varied scenarios during training. This may result in more generalizable models, often making them well-suited for diverse and complex tasks in digital pathology.

Despite the inherent advantages of foundation models, their deployment for practical use faces its own obstacles. In particular, they require substantial computational power, financial investments and rigorous testing to ensure reliability and efficacy for a given task \cite{Akkus_Dangott_etal._2022,Dragomir_Cocuz_etal._2022,Go_2022,Jafri_Farooqui_etal._2024}. Moreover, while foundation models enhance feature representation and performance, they depend on the quality of available annotations for decoder fine-tuning and, like any other model, cannot resolve existing inconsistencies or ambiguities in data labels. Therefore, there remains a critical need for solutions that address both data quality and practical deployment considerations.
Further, integrating new technologies into existing clinical workflows often encounters resistance, as it necessitates adjustments to established diagnostic processes. So, there is a need to develop solutions that could be integrated into current practices, minimizing the burden on medical professionals to adopt new tools \cite{King_Williams_etal._2023}.

Existing solutions \cite{Goldsborough_Philps_etal._2024,Hörst_Rempe_etal._2024}, while addressing some aspects of these challenges, fall short in providing a comprehensive approach. To address the data quality and clinical deployment issues, we propose a multi-faceted solution that encompasses data refinement, model optimization, and integration with existing pathology tools (\hyperref[fig:fig1]{Figure 1}). The outcome is a lightweight cell segmentation and classification model that can be integrated into digital pathology workflows for practical clinical use.

\begin{figure}[h!]
    \centering
    \includegraphics[width=\textwidth, height=0.82\textheight, keepaspectratio]{images/Figure_1.pdf}
    \caption{Overview of the proposed solution, including 1) Data refinement using cross-relabeling, 2) Teacher model development and fine tuning, 3) Student model optimization with knowledge distillation and 4) Student model and QuPath integration}
    \label{fig:fig1}
\end{figure}
\clearpage

Our approach begins with preparing the data for the fine-tuning and training of the machine learning models. We create a refined dataset, acquired via cross-relabeling two cell-level datasets, enhancing annotation specificity and consistency of the labeled data. Subsequently, we create a cell segmentation and classification model based on the foundation model. We leverage the foundation model as a fixed encoder and fine-tune a decoder using the refined dataset to improve generalization across diverse tissue- and cell types.
To ensure that the model remains lightweight and deployable in a possibly resource-constrained environment, we employ knowledge distillation to approximate the functionality of the foundation model. Finally, to facilitate the practical application of our model in digital pathology workflows, we integrate it with the QuPath \cite{Bankhead_Loughrey_etal._2017} application. Each methodological component contributes to the overarching goal of enhancing model performance, generalizability, and usability in clinical settings.

The primary contributions of this paper are:
\begin{enumerate}
    \item \textit{Data labels refinement through cross-relabeling:}
    
    We propose a new method for refining labels of cell-level datasets through cross-relabeling. This method employs classification models to re-label broad and ambiguous instances, resulting in a more diverse dataset. Our evaluation demonstrates that these classification models achieve high accuracy on test subsets, indicating the reliability of the method for label refinement.

    \item \textit{Enhanced model performance via foundation models:}
    
    We employ a foundation model as a feature extractor for the cell segmentation and classification task. In comparison with training a CNN model from scratch, the foundation model backbone only needs fine-tuning, which significantly reduces training time, computational resources and data requirements. We show that using a foundation model encoder leads to better performance in cell segmentation and classification networks than using a CNN-based encoder. This improvement may enable the model to generalize more effectively across various tissue types and imaging methods.
    
    \item \textit{Model optimization through knowledge distillation:}
    
    We show that a smaller student model trained using knowledge distillation on the refined dataset obtained via our cross-relabeling approach from a foundation model achieves comparable performance in cell segmentation and quantification tasks. As a result, this model is more suitable for deployment in environments without high-performance computing resources.
    
    \item \textit{Integration with QuPath:}
    
    We integrate the distilled cell segmentation and classification model into QuPath, a widely used open-source digital pathology platform, to accelerate clinical adaptation by enabling pathologists to more easily incorporate advanced computational tools into their existing workflows.
\end{enumerate}

Through these methodological steps, we aim to bridge the gap between advanced machine learning techniques and practical clinical applications, making accurate and efficient digital pathology accessible in a broader range of healthcare settings.

\section{Refining Existing Datasets Using Cross-Relabeling}
To address the limitations of sparse and ambiguous labeling of cell-level datasets, we propose a generalizable cross-relabeling strategy that can be applied to any dataset containing broadly categorized or imprecisely labeled cell types. This approach involves training and subsequently leveraging classification models to refine broad categories into more specific or biologically relevant classes.
When applied to cell-level data, the methodology includes extracting individual cell images from the dataset patches, preprocessing these images to standardize the size and accommodate partial cells, and then training deep learning classifiers capable of distinguishing between the finer cell subtypes within the coarser categories. 
To illustrate our approach, we focus on the PanNuke \cite{Gamper_Koohbanani_etal._2020, Gamper_Koohbanani_etal._2019} and MoNuSAC \cite{Verma_Kumar_etal._2021} datasets that we have used to train models for cell quantification in our previous works \cite{Shvetsov_Grønnesby_etal._2022,Shvetsov_Sildnes_etal._2024}. We find that for better cell differentiation we have to introduce more granular labels. PanNuke includes a broad classification of "inflammatory" cells, encompassing lymphocytes, macrophages, and neutrophils. Each cell type differs significantly in structure, function, and clinical relevance. Conversely, MoNuSAC uses the label "epithelial" for a class that comprises both benign epithelial cells and malignant neoplastic cells. This practice makes it challenging to differentiate between benign and malignant epithelial cells in the dataset, which is a critical distinction when identifying tumor areas within tissue samples. To address these issues, we implement a cross-relabeling strategy as shown in \hyperref[fig:fig2]{Figure 2}. The key components are two classification models: one is trained on singular cell images from PanNuke data to classify the epithelial meta-class into epithelial and neoplastic classes. The other is trained on MoNuSAC to refine the inflammatory class into lymphocytes, neutrophils, and macrophages.

\begin{figure}[h!]
    \centering
    \includegraphics[width=\textwidth]{images/Figure_2.pdf}
    \caption{Refined dataset generation via cross relabeling}
    \label{fig:fig2}
\end{figure}

The refining approach consists of three consecutive steps. The first is the preprocessing step, in which we extract individual cells from both datasets (\hyperref[fig:fig3]{Figure 3}). The specifics of PanNuke and MoNuSAC patch preparation before cell preprocessing are provided in \hyperref[chap:S1]{Appendix S1}.

\begin{figure}[h!]
    \centering
    \includegraphics[width=\textwidth]{images/Figure_3.pdf}
    \caption{Cell instances preprocessing including (1) cell map extraction, (2) bounding box delineation, (3) adjusting cell boxes and (4) cropping and resizing of cell images}
    \label{fig:fig3}
\end{figure}

During preprocessing, we extract cell type maps from the ground truth label mask and calculate bounding boxes around each cell instance. To accommodate partial cells at patch borders, a common issue in cropped patch images, we employ mirror padding and extend the field of view of the cell label by 15 pixels to capture adjacent cells. We then crop and resize the identified regions to $64 \times 64$ pixels using bicubic interpolation.

The preprocessed PanNuke dataset comprises 68,031 neoplastic and 23,207 epithelial cell images, while MoNuSAC comprises  33,104 lymphocytes, 1,252 neutrophils, and 1,695 macrophages, which we subsequently use in training cell classification models and classifying the cell image data \hyperref[fig:S2]{Appendix Figure S2 (1)}. 

The next step is to train two distinct ResNet50-based classifiers tailored to address the specific labeling challenges inherent in each dataset. We use ResNet50 for classification models due to its proven effectiveness for image classification tasks in histopathology \cite{pan2022reviewmachinelearningapproaches}, and its compatibility with small images. For the PanNuke dataset, we design the classifier, trained on MoNuSAC data, to disaggregate the heterogeneous "inflammatory" cell category into distinct subtypes: lymphocytes, macrophages, and neutrophils. Similarly, for the MoNuSAC dataset, the classifier is trained on PanNuke data and distinguishes between benign and malignant epithelial cells within the overarching "epithelial" label. By applying these targeted classifiers to their respective datasets, we assign more specific labels to individual cell instances, thus enabling us to create a unified dataset.
To ensure a balanced representation of classes, we train both models on datasets that had been equalized to match the size of the least represented class. Thus, we obtain datasets comprising 23,207 samples per class for PanNuke and 1,252 samples per class for MoNuSAC data. Next, we partition both of them into training (70\%), validation (20\%), and testing (10\%) subsets. To mitigate the risk of overfitting, we use a single dropout layer with a rate of p=0.5 in both models and data augmentation using randomized color perturbations, rotation, and horizontal and vertical flipping. We employ AdamW optimizer and the cross-entropy loss function for the training criterion.

To evaluate the two trained models, we measure the classification accuracy on the respective test subsets. The accuracies on the test subset for both classifiers are presented in \hyperref[tab:1]{Table 1}. The PanNuke model achieves an average accuracy of 93.57\%, with higher accuracy for neoplastic cells (96.06\%) compared to epithelial cells (86.26\%). The confusion matrix in Figure A3.1 shows that the model predominantly distinguishes accurately between epithelial and neoplastic tissues, with a substantial number of correct classifications and relatively few misclassifications. The MoNuSAC model demonstrates an average accuracy of 98.92\%, excelling in classifying lymphocytes (99.67\%) and macrophages (94.12\%), with lower performance for neutrophils (85.71\%). The confusion matrix in Figure A3.2 shows that the model identifies lymphocytes and performs reasonably well with macrophages and neutrophils.

\begin{table}[h!]
\renewcommand{\arraystretch}{1.5}
  \centering
  \caption{Cell classification results for PanNuke and MoNuSAC trained models (CI 95\%).}
  \label{tab:1}
  \begin{tabular}{|l|c|c|}
   \hline
   %\rowcolor{gray!30}
    Accuracy               & PanNuke model              & MoNuSAC model              \\
    \hline
    Average      & 0.936 (0.931--0.941)         & 0.989 (0.986--0.993)        \\
    \hline
    Neoplastic   & 0.961 (0.956--0.965)         & -                          \\
    \hline
    Epithelial   & 0.863 (0.849--0.877)         & -                          \\
    \hline
    Lymphocytes  & -                          & 0.997 (0.995--0.999)        \\
    \hline
    Neutrophils  & -                          & 0.857 (0.796--0.918)        \\
    \hline
    Macrophages  & -                          & 0.941 (0.906--0.976)        \\
    \hline
  \end{tabular}
\end{table}

Finally, during the last step, we use the model trained on PanNuke data for epithelial cells in MoNuSAC and the model trained on MoNuSAC for the inflammatory cells class in PanNuke. Specifically, we use classifier models to relabel epithelial cells in MoNuSAC and inflammatory cells in PanNuke data. Then we combine cells with refined labels and the rest of the cells in both datasets to create a refined dataset (\hyperref[fig:S2]{Appendix Figure S2 (2)}). The process of relabeling cells and visualizing them on a patch is shown in \hyperref[fig:fig4]{Figure 4}. The cell counts in the refined dataset are provided in \hyperref[tab:S4]{Appendix Table S4}.

\begin{figure}[h!]
    \centering
    \includegraphics[width=\textwidth, height=0.42\textheight, keepaspectratio]{images/Figure_4.pdf}
    \caption{Cell relabeling procedure for epithelial and inflammatory cell classes}
    \label{fig:fig4}
\end{figure}

%\hfill

Relabeling and combining datasets have been explored in a prior study \cite{Parulekar_Kanwat_etal._2023}, where consecutive fine-tuning on multiple datasets was employed to account for hierarchical class label structures. While the method presented in \cite{Parulekar_Kanwat_etal._2023} is intuitive, it often lacks consistency and requires multiple fine-tuning runs, which can be cumbersome and time-consuming. 
In contrast, cross-relabeling simplifies this process by using specialized classification models tailored to each dataset's specific labeling challenges. This approach provides better transparency and produces a unified dataset encompassing seven distinct cell types across multiple tissue samples, enhancing data diversity for further model training or fine-tuning.

Despite these improvements, cross-relabeling does not entirely resolve issues related to poor labeling quality or the amount of labeled data. Specifically, our results show lower accuracies persist for underrepresented classes, such as macrophages, which may stem from a limited sample availability and intrinsic challenges in distinguishing these cells based solely on H\&E staining. Furthermore, while our method enhances label specificity, it relies on the initial quality of the broad labels; thus, any fundamental inaccuracies in the original annotations can propagate through the relabeling process. Addressing the overall problem of limited data labels may require integrating additional data sources or utilizing complementary immunohistochemical staining methods.
Although the reported performance metrics are obtained from evaluations on the native test sets of each dataset, it is important to note that the primary application of these classifiers is to perform cross-relabeling, where a model trained on one dataset (e.g., PanNuke) is applied to another (e.g., MoNuSAC) and vice versa. We acknowledge that a more systematic evaluation of cross-dataset generalization is needed and could be performed in future work.

Overall, the refined dataset produced by our approach can enhance the supervised training or fine-tuning of cell segmentation and classification models, especially those that utilize pre-trained foundation models to improve feature extraction robustness. In addition, these models can detect nuanced classes that enable researchers to conduct more detailed analyses of biological processes in computational pathology.

\section{Foundation models for robust cell segmentation and classification}

Accurate cell segmentation and classification in digital pathology are hindered by limited labeled data and the fact that conventional CNNs are unable to capture global contextual information due to their local receptive field constraints \cite{Gheflati_Rivaz_2022,Yang_Marcus_etal.}. Traditional approaches in cell quantification have predominantly relied on CNN encoders, such as ResNet50, given their proven effectiveness in semantic segmentation tasks \cite{Deshmane_2023,Graham_Vu_etal._2019,Mukasheva_Koishiyeva_etal._2024,Stringer_Wang_etal._2021}. However, approaches that include fine-tuning of pretrained CNNs, data augmentation, and stain normalization to partially increase data variability and address staining differences often fail to achieve the necessary generalization and robustness across diverse tissue types and staining conditions \cite{G._Wang_W._Li_etal._2018,Gao_Bagci_etal._2018,Karim_El_Khoury_Martin_Fockedey_etal._2021}.

To overcome these challenges, we leverage an encoder-decoder network that uses a foundation model as the encoder and a CNN upsampling decoder (\hyperref[fig:fig5]{Figure 5}) for simultaneous cell segmentation and classification in 2D patches extracted from WSIs. Foundation models with transformer-based architectures are viable alternatives to CNN-based encoders \cite{Shamshad_Khan_etal._2023,Sourget_2023}. They enable the creation of more advanced architectures that can decode or transform learned features more effectively \cite{Chen_Duan_etal._2023,Cheng_Misra_etal._2022,Xie_Wang_etal._2021}.

\begin{figure}[h!]
    \centering
    \includegraphics[width=\textwidth]{images/Figure_5.pdf}
    \caption{UNETR-like model with foundational model as backbone}
    \label{fig:fig5}
\end{figure}

By utilizing a transformer-based encoder, we incorporate global contextual information into the feature extraction process, which is a key advantage of such architectures \cite{Chen_Lu_etal._2021}. This foundation model integration facilitates accurate pixel-wise segmentation and classification without the need for extensive encoder training, thereby potentially improving generalization across varied cellular structures and tissue types.
In our implementation, we employ a modified UNETR \cite{Hatamizadeh_Tang_etal._2021} architecture that combines a vision transformer (ViT) \cite{Dosovitskiy_Beyer_etal._2021} encoder with a CNN-based decoder. The encoder utilizes the pretrained H-Optimus foundation model, which contains 1.1 billion parameters and is trained on over 500,000 H\&E stained WSIs \cite{Saillard_Jenatton_etal._2024}. We extract outputs from four evenly spaced transformer blocks $Z_i$, where $i \in [1, 14, 26, 38]$, to serve as residual connections for the CNN decoder. We select these blocks based on our observation that features from non-adjacent levels of the encoder lead to better overall performance on the test subset.

The CNN decoder upsamples the feature representations, acquired from the transformer blocks, to generate an intermediate vector that is handled by two task-specific layers that generate cell segmentation and classification masks. The first task-specific layer is the ‘Cellpose head’,  which is used to delineate cell instances. The layer generates horizontal and vertical gradient maps to form vector fields that are refined through gradient tracking in a post-processing step using the Cellpose algorithm \cite{Stringer_Wang_etal._2021}, known for its efficacy in cell segmentation tasks and generalizability across multiple domains \cite{Pachitariu_Stringer_2022,Stringer_Pachitariu_2024}. The second task-specific layer is the "Cell type head", which assigns labels to individual pixels. In the post-processing step, we determine the output classification label of each segmented cell instance by majority voting over the labeled pixels that comprise the cell in the segmentation map.

To evaluate model performance and measure the impact of adding a foundation model as backbone, we compare it to a ResNet50-based model. ResNet50 is a widely used solution for encoders in segmentation architectures in the medical domain \cite{Deshmane_2023,Graham_Vu_etal._2019,Mukasheva_Koishiyeva_etal._2024,Stringer_Wang_etal._2021}. For the H-Optimus-based model, we utilize frozen weights for the encoder and only fine-tune the decoder to take advantage of the extensive pre-training of the foundation model. For the ResNet50-based model we start with ImageNet \cite{Deng_Dong_etal.} weights and train both encoder and decoder parts. Hyperparameters for the training step are set to be identical, where possible, for comparable evaluation. 
For this evaluation, we deliberately use the PanNuke dataset to provide a standardized and controlled comparison between the H‑Optimus and ResNet50-based models (\hyperref[fig:S2]{Appendix Figure S2 (3)}). Specifically, we use two of the default PanNuke dataset splits (66\%) for training and validation, and reserve the third split (33\%) for testing.

To address the challenge of cell class imbalance in the PanNuke dataset, which is a common characteristic in most cell-level H\&E patch datasets, both models’ training processes employ a weighted loss function comprising cross-entropy and focal loss \cite{Lin_Goyal_etal._2018}. The focal loss component is adjusted with coefficients derived from each cell class' instance frequency, emphasizing learning from underrepresented classes and enhancing the model's sensitivity to rare but significant cellular patterns. The cross-entropy loss is augmented with spectral decoupling regularization \cite{Pezeshki_Kaba_etal._2021,Pohjonen_Stürenberg_etal._2022} and spatially varying label smoothing \cite{Islam_Glocker_2021}, which potentially stabilizes training and improves generalization in case of complex tissue morphologies. For optimization, we employ the \textit{AdamW} \cite{Loshchilov_Hutter_2019} to counter unbalanced class scenarios, with cosine annealing learning rate scheduler.

We utilize the scikit-learn library \cite{Van_der_Walt_Schönberger_etal._2014} and HoVer-Net \cite{Graham_Vu_etal._2019} implementations of $R^2$ (the coefficient of determination) and $PQ$ (panoptic quality) to evaluate our experiments. Complete mathematical formulations and detailed explanations of these metrics are provided in \hyperref[chap:S5]{Appendix S5}. To compute confidence intervals, we use nonparametric bootstrapping, where after calculating the metric on the full sample, we generated 1000 bootstrap replicates by resampling with replacement and then determined the 95\% confidence intervals as the 2.5th and 97.5th percentiles of the resulting empirical distribution.

%\hfill

The model comparisons are summarized in \hyperref[tab:2]{Table 2}. The H‑Optimus-based model achieves higher $R^2$ across all cell classes compared to the ResNet50-based model, which means that its predictions are more closely aligned with the PanNuke cell counts, indicating a stronger correlation with the observed data. Notably, the improvement of $R^2_{dead}$ may be an indicator of better global contextual representations provided by the foundation model backbone. In terms of segmentation and classification quality combined, measured by the PQ score, the H‑Optimus-based model demonstrates notable improvements across most cell classes. Overall, the average $R^2$ improved from 0.575 to 0.871, while the average $PQ$ score improved from 0.450 to 0.492, demonstrating better performance of the H-Optimus-based model.

\begin{table}[h!]
\renewcommand{\arraystretch}{1.5}
  \centering
  \caption{Cell quantification metrics for baseline and proposed models (CI 95\%).}
  \label{tab:2}
  \begin{tabular}{|l|c|c|}
    \hline
    %\rowcolor{gray!30}
    Metric             & Resnet50-based            & H-optimus-based              \\
    \hline
    $R^2_{neoplastic}$    & 0.681 (0.576--0.769)       & \textbf{0.941 (0.917--0.960)} \\
    \hline
    $R^2_{inflammatory}$  & 0.863 (0.778--0.903)       & \textbf{0.949 (0.918--0.966)} \\
    \hline
    $R^2_{connective}$    & 0.600 (0.488--0.698)       & 0.609 (0.436--0.772)          \\
    \hline
    $R^2_{dead}$          & 0.097 (-11.389--0.669)     & 0.925 (0.404--0.982)          \\
    \hline
    $R^2_{epithelial}$    & 0.635 (0.490--0.747)       & \textbf{0.930 (0.886--0.964)} \\
    \hline
    $PQ_{neoplastic}$       & 0.517 (0.499--0.535)       & \textbf{0.589 (0.575--0.604)} \\
    \hline
    $PQ_{inflammatory}$     & 0.455 (0.429--0.482)       & \textbf{0.528 (0.507--0.549)} \\
    \hline
    $PQ_{connective}$       & 0.416 (0.400--0.431)       & \textbf{0.451 (0.436--0.465)} \\
    \hline
    $PQ_{dead}$             & 0.374 (0.342--0.408)       & 0.292 (0.209--0.365)          \\
    \hline
    $PQ_{epithelial}$       & 0.488 (0.460--0.519)       & \textbf{0.599 (0.579--0.618)} \\
    \hline
  \end{tabular}
\end{table}

Our results  show that integrating the H‑Optimus foundation model within the UNETR architecture enhances the model's ability to segment and classify cells across diverse tissues from PanNuke data. The pretrained transformer encoder provides robust feature representations, resulting in higher average $R^2$ and $PQ$ scores compared to the CNN-based model. This leads to more reliable cell quantification and more accurate downstream analysis. Additionally, the streamlined fine-tuning process reduces computational overhead and training time, making the model more adaptable for new data.

Despite these advancements, the foundation model-based approach does not fully resolve all challenges related to cell segmentation and classification. We observe lower metric scores for underrepresented classes in the training data. Furthermore, foundation models typically encompass billions of parameters, resulting in substantial computational and memory requirements. It therefore poses challenges for deployment in resource-constrained environments, limiting their practical applicability in certain clinical settings.

\section{Model optimization via Knowledge Distillation}

To address the limitations posed by the extensive size of foundation models, we implement knowledge distillation — a model compression technique that leverages the teacher-student paradigm \cite{Hinton_Vinyals_etal._2015}. By training a smaller, more efficient student model to replicate the output of a larger, pre-trained teacher model, we retain performance while significantly reducing the model's complexity and resource requirements (\hyperref[fig:fig6]{Figure 6}).

\begin{figure}[h!]
    \centering
    \includegraphics[width=\textwidth, height=0.45\textheight, keepaspectratio]{images/Figure_6.pdf}
    \caption{Knowledge distillation framework for training a student model using a pre-trained teacher}
    \label{fig:fig6}
\end{figure}

We employ knowledge distillation to compress the H‑Optimus-based teacher model into a more efficient student model. The teacher model is the modified UNETR architecture with the H‑Optimus foundation model described in the previous chapter. The student model is based on a UNet architecture augmented with residual connections and incorporates a smaller ViT encoder with 9 million parameters \cite{Steiner_Kolesnikov_etal._2022,Wightman_2019}. 

First, we fine-tune the teacher model using the refined dataset from the cross-relabeling procedure (Section 2). Initially we train the decoder of the teacher model while keeping the encoder weights frozen. We split the refined dataset into train (70\%), validation (20\%) and test (10\%) subsets (\hyperref[fig:S2]{Appendix Figure S2 (4)}). During fine-tuning, we use the train and validation subsets, while leaving the test subset for model evaluation. We set the training procedure and model hyperparameters to be identical to those that were used to demonstrate the utility of foundation models for the simultaneous cell segmentation and classification task.

Next, we perform knowledge distillation from teacher to student using the refined dataset used to fine-tune the teacher model. The student model is trained to replicate the teacher model's outputs. We utilize a specialized loss function that aligns the student's predicted probability distribution with the teacher's, incorporating the teacher's class probability distribution derived from the output. Following the methodology of Hinton et al. \cite{Hinton_Vinyals_etal._2015}, we experiment with various hyperparameter settings for the temperature ($T$) and the balancing coefficients ($\alpha$ and $\beta$) in the loss function. We vary $T$ from 1 to 20 and adjust $\alpha$ and $\beta$ to balance the distillation and student losses. Through iterative tuning and evaluation, we identify that setting $T=14$, $\alpha=0.3$, and $\beta=0.7$ yields a configuration that converges and closely approximates the teacher model's performance during training.

Finally, we assess the performance of both models using the $R^2$ and $PQ$ (defined in \hyperref[chap:S5]{Appendix S5}) on the test set of the refined dataset (\hyperref[tab:3]{Table 3}). We observe that the 95\% confidence intervals overlap for most cell types, so we cannot claim statistically significant performance differences between the teacher and student models. One exception appears in the neoplastic class. The teacher model produces an $R^2$ of 0.919, while the student model shows an $R^2$ of 0.852. In addition, the student model achieves higher $PQ$ values for the neoplastic and connective classes, though the confidence intervals show overlap.

\begin{table}[h!]
\renewcommand{\arraystretch}{1.5}
  \centering
  \caption{Cell quantification metrics for teacher and distilled student models (CI 95\%).}
  \label{tab:3}
  \begin{tabular}{|l|c|c|}
    \hline
    %\rowcolor{gray!30}
    Metric & Teacher & Student \\
    \hline
    $R^2_{neoplastic}$    & \textbf{0.919} (0.898--0.939) & 0.852 (0.800--0.891) \\
    \hline
    $R^2_{lymphocyte}$    & 0.969 (0.956--0.977)         & 0.969 (0.956--0.978) \\
    \hline
    $R^2_{connective}$    & 0.694 (0.548--0.809)         & 0.618 (0.469--0.741) \\
    \hline
    $R^2_{dead}$          & 0.755 (0.400--0.908)         & 0.424 (0.100--0.731) \\
    \hline
    $R^2_{epithelial}$    & 0.922 (0.870--0.958)         & 0.843 (0.738--0.917) \\
    \hline
    $R^2_{macrophage}$    & 0.384 (-0.369--0.724)        & 0.704 (0.352--0.859) \\
    \hline
    $R^2_{neutrofil}$     & 0.854 (0.578--0.929)         & 0.833 (0.502--0.925) \\
    \hline
    $PQ_{neoplastic}$       & 0.581 (0.569--0.593)         & 0.601 (0.588--0.613) \\
    \hline
    $PQ_{lymphocyte}$       & 0.536 (0.520--0.553)         & 0.563 (0.544--0.579) \\
    \hline
    $PQ_{connective}$       & 0.436 (0.421--0.451)         & 0.457 (0.441--0.474) \\
    \hline
    $PQ_{dead}$             & 0.272 (0.235--0.315)         & 0.279 (0.201--0.369) \\
    \hline
    $PQ_{epithelial}$       & 0.522 (0.500--0.545)         & 0.530 (0.506--0.555) \\
    \hline
    $PQ_{macrophage}$       & 0.524 (0.459--0.588)         & 0.474 (0.405--0.543) \\
    \hline
    $PQ_{neutrofil}$        & 0.541 (0.490--0.592)         & 0.565 (0.522--0.607) \\
    \hline
  \end{tabular}
\end{table}


We further decompose the $PQ$ metric into its $SQ$ and $DQ$ components (\hyperref[tab:S6]{Appendix Table S6}). Both models produce nearly identical $SQ$ values, which indicates that they predict instance boundaries with similar precision. Although the student model shows some improvement in $DQ$ scores for certain classes, the confidence intervals overlap and do not confirm a statistically significant difference.

We observe that the student and teacher models yield comparable detection performance despite the student model using a much smaller and simpler architecture. A model with fewer parameters reduces the risk of overfitting when training data are scarce relative to the model’s complexity \cite{Farias_Ludermir_etal._2022}. The knowledge distillation process also encourages the student model to focus on the most generalizable detection features learned from the teacher. These factors enable the student model to achieve similar detection performance across different cell types.

Additionally, considering the model sizes reported in \hyperref[tab:4]{Table 4}, the distilled model achieves a significant reduction compared to the teacher model, with a 48-fold decrease in parameter count and a 5.5-fold reduction in on-disk size. In inference mode, the teacher model requires 16 GB of VRAM for a batch size of 32, while the distilled model only needs 3 GB of VRAM for the same batch size. These reductions make the distilled model significantly more practical for fine-tuning and deployment in resource-constrained environments.

\begin{table}[h!]
\renewcommand{\arraystretch}{1.5}
  \centering
  \caption{Parameter counts and size of teacher and distilled model}
  \label{tab:4}
  \adjustbox{max width=\textwidth}{%
  \begin{tabular}{|l|c|c|c|}
    \hline
    %\rowcolor{gray!30}
    Metric & H-optimus-based (Teacher) & mobileViT-based (Student) & Magnitude of difference \\
    \hline
    Parameters count       & 1,158,917,906   & \textbf{24,093,393}   & \textbf{48x}  \\
    \hline
    Estimated Total Size (MB) & 87,912       & \textbf{15,935}    & \textbf{5.5x} \\
    \hline
  \end{tabular}%
}
\end{table}

%\hfill

With recent advancements in complex network architectures and the use of pretrained encoders to achieve state-of-the-art performance \cite{Baumann_Dislich_etal._2024,Hörst_Rempe_etal._2024} in cell segmentation and classification tasks, model size, computational complexity, and processing times have increased. This limits the scalability and accessibility of these models. As we demonstrate, this may be mitigated using knowledge distillation. Studies in the field of natural language processing have demonstrated the efficacy of knowledge distillation in retaining the capabilities of the teacher model while achieving significant reductions in size and complexity \cite{Huangpu_Gao_2024,Sun_Yu_etal.}. 

We demonstrate the feasibility of knowledge distillation in digital pathology, specifically for cell segmentation and classification tasks. Moreover, we achieve this performance while also significantly reducing the parameter count. In addressing the challenge of knowledge transfer, we found that distillation from a transformer-based model to a smaller transformer is more straightforward than attempting to map transformer features to CNN blocks. In our experiments, using a CNN-based network as a student results in worse cell quantification performance due to the structural constraints of CNN feature space dimensions. 

Although our primary approach relies on a transformer-based student model that performs well, it can be further optimized to incorporate advantages from CNN architectures. For example, employing alternative techniques such as using ViT adapters \cite{Chen_Duan_etal._2023} or $1 \times 1$ convolutions to adjust feature map sizes may be beneficial for harnessing CNN advantages like enhanced local feature extraction. Moreover, if additional performance improvements are desired, the process can be further enhanced by applying supplementary knowledge distillation techniques, such as self-distillation \cite{Zhang_Song_etal._2019} or online distillation \cite{Houyon_Cioppa_etal._2023}.

Despite these promising results, further validation on independent datasets is necessary to fully understand the model's limitations. Underrepresented classes may pose challenges when addressing complex cases. Pathologists need to validate these models to adopt them in clinical settings. While the distilled models are smaller and more deployable, a technological gap persists because pathologists traditionally rely on established methods for inspecting WSIs and diagnosing diseases. Addressing the complexities involved in deploying models for inference and supporting pathologists in adopting new tools is essential for integrating these models into clinical workflows.

\section{Model integration with QuPath}
Digital pathology tools with graphical user interfaces are essential for visualizing and analyzing WSIs. To make our student model useful in clinical pathology workflows, it needs to be integrated into a tool that enables inspecting regions, creating annotations, and providing quantitative analyses of biomarkers. Therefore, we integrate the trained student model from the previous chapter into the QuPath open‑source platform \cite{Bankhead_Loughrey_etal._2017}. QuPath provides the required annotation, visualization, and analysis tools to interpret complex histological data, including workflows for cell segmentation, classification, and quantification (\hyperref[fig:fig7]{Figure 7}). 

\begin{figure}[h!]
    \centering
    \includegraphics[width=\textwidth]{images/Figure_7.pdf}
    \caption{Visualization of model-generated cell quantification annotations (left) and the corresponding unannotated slide (right) in QuPath}
    \label{fig:fig7}
\end{figure}

To identify the regions in a WSI critical for prognosticating tumor development, such as specific tumor areas or border regions without overlapping healthy tissue, the pathologist uses QuPath to outline these regions. Then, the pathologist initiates a cell segmentation and classification script through the QuPath interface for the selected regions. The resulting annotations and quantified cell information are then directly overlaid onto the WSI in the QuPath interface. Additional design and implementation details are in \hyperref[chap:S7]{Appendix S7}. 

Two common approaches for integrating deep learning models into QuPath are Java‑based native QuPath extensions \cite{Goldsborough_Philps_etal._2024} and the execution of RESTful API requests to a model server coupled with handling the response via an extension, as demonstrated in the application of cell segmentation models applied to immunofluorescence images \cite{Sugawara_2023}. While the community is actively working on these integration strategies, there is currently no universal solution that fully addresses all integration and performance requirements.

Extensions may offer better integration with QuPath, allowing slightly improved performance and more widespread usage of the built-in QuPath models, but they lack the flexibility to customize models and modify their behavior. For example, the newest version of QuPath includes models such as StarDist \cite{Weigert_Schmidt} and InstanSeg \cite{Goldsborough_Philps_etal._2024} that can perform cell segmentation. Both models pose limitations when applied to simultaneous cell segmentation and classification. StarDist performs well only on convex, round shapes by design, whereas some neoplastic, inflammatory, and connective cells exhibit complex and non-convex shapes. InstanSeg provides only semantic segmentation without assigning classes to the segmented cells.

%\hfill

In contrast, our approach offers an alternative integration strategy. It utilizes the paquo library to directly interact with QuPath’s internal application programming interface from within Python. This enables data exchange and processing without the need for intermediate conversion steps and provides greater control over model customization, retraining, and the incorporation of custom processing steps.

The integration of our custom model with QuPath underscores its potential to significantly enhance the diagnostic process by reducing the time burden on pathologists and enabling them to focus on more complex interpretative tasks using familiar software. Leveraging a tool that is already well-established among pathologists increases the likelihood of its adoption into daily clinical workflows. The quantitative data generated through the automated workflow is critical for both clinical decision-making and research, facilitating more accurate biomarker analysis, enabling robust statistical evaluations, and supporting hypothesis generation and testing. Additionally, by streamlining cell segmentation and classification, the tool enhances the scalability and reproducibility of pathological assessments, ultimately contributing to improved diagnostic accuracy and patient outcomes.

\section{Conclusion and future work}

In this study, we address critical challenges in digital pathology and tackle the usability and deployment issues of the developed models in standard computing environments without the need for high-performance computing systems. Our multi-faceted approach encompasses data refinement through cross-relabeling, leveraging foundation models for robust cell segmentation and classification, optimizing model performance via knowledge distillation, and integrating the optimized model into the QuPath software for practical application. This approach is used to construct a capable, versatile, and adjustable model for cell segmentation and classification, with enhanced performance and usability.

\begin{sloppypar}
While our approach shows potential in the field of computational pathology, certain limitations persist. 
For example, our implementation currently exhibits lower performance in detecting macrophages. 
This serves as an instance of the broader challenge of accurately identifying complex cell types. In order to address this issue, extending our approach to incorporate additional data sources, exploring alternative modeling approaches, and integrating other imaging modalities such as immunohistochemical staining may help improve detection accuracy. Moreover, although the distilled model reduces computational demands, integrating advanced deep learning models into clinical practice requires addressing technological gaps and potential resistance to adopting new tools within established diagnostic processes.
\end{sloppypar}

Future work could focus on several key areas to refine the proposed approach and facilitate its adoption in clinical environments. Enhancing the cell-relabeling process with additional datasets \cite{Graham_Jahanifar_etal._2021} could improve the representation of underrepresented cell types and enhance overall model performance. Also, incorporating additional data sources, such as multi-modal imaging or complementary staining methods, may address limitations related to cell type differentiation and class imbalance. Exploring other foundation models \cite{Vorontsov_Bozkurt_etal._2024,Zimmermann_Vorontsov_etal._2024} or introducing additional modalities \cite{Ding_Wagner_etal._2024,Vaidya_Zhang_etal._2025} may provide alternative architectures better suited to specific tasks or offer improved efficiency. Implementing more complex knowledge distillation techniques \cite{Houyon_Cioppa_etal._2023,Zhang_Song_etal._2019} could further optimize the model's performance and adaptability. Additionally, deeper integration with QuPath or other digital pathology software could provide pathologists more control over cell quantification analysis directly within the QuPath interface, thereby increasing accessibility and usability. Such enhancements would not only refine model performance but also ensure greater adaptability and scalability within various clinical environments. Finally, extensive validation of the model by pathologists and benchmarking against independent datasets are essential steps toward establishing the model's reliability and fostering confidence in its clinical utility.

\section*{Acknowledgments} 
This work was funded in part by the Research Council of Norway grant no. 309439 SFI Visual Intelligence, and the North Norwegian Health Authority grant no. HNF1521-20.

\bibliographystyle{IEEEtran}
\begin{sloppypar}
\begin{thebibliography}{99}

\bibitem{chaplot2020neural} Chaplot, Devendra Singh, et al. "Neural topological slam for visual navigation." Proceedings of the IEEE/CVF conference on computer vision and pattern recognition. 2020.

\bibitem{maksymets2021thda} Maksymets, Oleksandr, et al. "Thda: Treasure hunt data augmentation for semantic navigation." Proceedings of the IEEE/CVF International Conference on Computer Vision. 2021.

\bibitem{mezghan2022memory} Mezghan, Lina, et al. "Memory-augmented reinforcement learning for image-goal navigation." 2022 IEEE/RSJ International Conference on Intelligent Robots and Systems (IROS). IEEE, 2022.

\bibitem{al2022zero} Al-Halah, Ziad, Santhosh Kumar Ramakrishnan, and Kristen Grauman. "Zero experience required: Plug \& play modular transfer learning for semantic visual navigation." Proceedings of the IEEE/CVF Conference on Computer Vision and Pattern Recognition. 2022.

\bibitem{ye2021auxiliary} Ye, Joel, et al. "Auxiliary tasks and exploration enable objectgoal navigation." Proceedings of the IEEE/CVF international conference on computer vision. 2021.

\bibitem{chaplot2020object} Chaplot, Devendra Singh, et al. "Object goal navigation using goal-oriented semantic exploration." Advances in Neural Information Processing Systems 33 (2020)

\bibitem{ramakrishnan2022poni} Ramakrishnan, Santhosh Kumar, et al. "Poni: Potential functions for objectgoal navigation with interaction-free learning." Proceedings of the IEEE/CVF Conference on Computer Vision and Pattern Recognition. 2022.

\bibitem{ramrakhya2022habitat} Ramrakhya, Ram, et al. "Habitat-web: Learning embodied object-search strategies from human demonstrations at scale." Proceedings of the IEEE/CVF Conference on Computer Vision and Pattern Recognition. 2022.

\bibitem{mousavian2019visual} Mousavian, Arsalan, et al. "Visual representations for semantic target driven navigation." 2019 International Conference on Robotics and Automation (ICRA). IEEE, 2019.

\bibitem{dhariwal2021diffusion} Dhariwal, Prafulla, and Alexander Nichol. "Diffusion models beat gans on image synthesis." Advances in neural information processing systems 34 (2021)

\bibitem{ho2022classifier} Ho, Jonathan, and Tim Salimans. "Classifier-free diffusion guidance." arXiv preprint arXiv:2207.12598 (2022).

\bibitem{nichol2021glide} Nichol, Alex, et al. "Glide: Towards photorealistic image generation and editing with text-guided diffusion models." arXiv preprint arXiv:2112.10741 (2021)

\bibitem{brooks2023instructpix2pix} Brooks, Tim, Aleksander Holynski, and Alexei A. Efros. "Instructpix2pix: Learning to follow image editing instructions." Proceedings of the IEEE/CVF Conference on Computer Vision and Pattern Recognition. 2023.

\bibitem{fu2023guiding} Fu, Tsu-Jui, et al. "Guiding instruction-based image editing via multimodal large language models." arXiv preprint arXiv:2309.17102 (2023).

\bibitem{geng2024instructdiffusion} Geng, Zigang, et al. "Instructdiffusion: A generalist modeling interface for vision tasks." Proceedings of the IEEE/CVF Conference on Computer Vision and Pattern Recognition. 2024.

\bibitem{zhou2024minedreamer} Zhou, Enshen, et al. "Minedreamer: Learning to follow instructions via chain-of-imagination for simulated-world control." arXiv preprint arXiv:2403.12037 (2024).

\bibitem{zhou2023esc} Zhou, Kaiwen, et al. "Esc: Exploration with soft commonsense constraints for zero-shot object navigation." International Conference on Machine Learning. PMLR, 2023.

\bibitem{yu2023l3mvn} Yu, Bangguo, Hamidreza Kasaei, and Ming Cao. "L3mvn: Leveraging large language models for visual target navigation." 2023 IEEE/RSJ International Conference on Intelligent Robots and Systems (IROS). IEEE, 2023.

\bibitem{gadre2023cows} Gadre, Samir Yitzhak, et al. "Cows on pasture: Baselines and benchmarks for language-driven zero-shot object navigation." Proceedings of the IEEE/CVF Conference on Computer Vision and Pattern Recognition. 2023.

\bibitem{shah2023navigation} Shah, Dhruv, et al. "Navigation with large language models: Semantic guesswork as a heuristic for planning." Conference on Robot Learning. PMLR, 2023.

\bibitem{cai2024bridging} Cai, Wenzhe, et al. "Bridging zero-shot object navigation and foundation models through pixel-guided navigation skill." 2024 IEEE International Conference on Robotics and Automation (ICRA). IEEE, 2024.

\bibitem{yu2023co} Yu, Bangguo, Hamidreza Kasaei, and Ming Cao. "Co-NavGPT: Multi-robot cooperative visual semantic navigation using large language models." arXiv preprint arXiv:2310.07937 (2023).

\bibitem{wu2024voronav} Wu, Pengying, et al. "Voronav: Voronoi-based zero-shot object navigation with large language model." arXiv preprint arXiv:2401.02695 (2024).

\bibitem{qin2023mp5} Qin, Yiran, et al. "Mp5: A multi-modal open-ended embodied system in minecraft via active perception." arXiv preprint arXiv:2312.07472 (2023).

\bibitem{du2024learning} Du, Yilun, et al. "Learning universal policies via text-guided video generation." Advances in Neural Information Processing Systems 36 (2024).

\bibitem{ajay2024compositional} Ajay, Anurag, et al. "Compositional foundation models for hierarchical planning." Advances in Neural Information Processing Systems 36 (2024).

\bibitem{liang2024skilldiffuser} Liang, Zhixuan, et al. "Skilldiffuser: Interpretable hierarchical planning via skill abstractions in diffusion-based task execution." Proceedings of the IEEE/CVF Conference on Computer Vision and Pattern Recognition. 2024.

\bibitem{heusel2017gans} Heusel, Martin, et al. "Gans trained by a two time-scale update rule converge to a local nash equilibrium." Advances in neural information processing systems 30 (2017).

\bibitem{zhang2018unreasonable} Zhang, Richard, et al. "The unreasonable effectiveness of deep features as a perceptual metric." Proceedings of the IEEE conference on computer vision and pattern recognition. 2018.

\bibitem{brown2020language} Brown, Tom B. "Language models are few-shot learners." arXiv preprint arXiv:2005.14165 (2020).

\bibitem{podell2023sdxl} Podell, Dustin, et al. "Sdxl: Improving latent diffusion models for high-resolution image synthesis." arXiv preprint arXiv:2307.01952 (2023).

\bibitem{brohan2022rt} Brohan, Anthony, et al. "Rt-1: Robotics transformer for real-world control at scale." arXiv preprint arXiv:2212.06817 (2022).

\bibitem{brohan2023rt} Brohan, Anthony, et al. "Rt-2: Vision-language-action models transfer web knowledge to robotic control." arXiv preprint arXiv:2307.15818 (2023).

\bibitem{li2024manipllm} Li, Xiaoqi, et al. "Manipllm: Embodied multimodal large language model for object-centric robotic manipulation." Proceedings of the IEEE/CVF Conference on Computer Vision and Pattern Recognition. 2024.

\bibitem{shah2023vint} Shah, Dhruv, et al. "ViNT: A foundation model for visual navigation." arXiv preprint arXiv:2306.14846 (2023).

\bibitem{liu2024visual} Liu, Haotian, et al. "Visual instruction tuning." Advances in neural information processing systems 36 (2024).

\bibitem{hu2021lora} Hu, Edward J., et al. "Lora: Low-rank adaptation of large language models." arXiv preprint arXiv:2106.09685 (2021).

\bibitem{qin2023supfusion} Qin, Yiran, et al. "SupFusion: Supervised LiDAR-camera fusion for 3D object detection." Proceedings of the IEEE/CVF International Conference on Computer Vision. 2023.

\bibitem{qin2024worldsimbench} Qin, Yiran, et al. "Worldsimbench: Towards video generation models as world simulators." arXiv preprint arXiv:2410.18072 (2024).

\bibitem{yu2025gamefactory} Yu, Jiwen, et al. "GameFactory: Creating New Games with Generative Interactive Videos." arXiv preprint arXiv:2501.08325 (2025).

\bibitem{zhou2024code} Zhou, Enshen, et al. "Code-as-Monitor: Constraint-aware Visual Programming for Reactive and Proactive Robotic Failure Detection." arXiv preprint arXiv:2412.04455 (2024).

\bibitem{zhang2024ad} Zhang, Zaibin, et al. "AD-H: Autonomous Driving with Hierarchical Agents." arXiv preprint arXiv:2406.03474 (2024).

\bibitem{wang2024toward} Wang, Chaoqun, et al. "Toward Accurate Camera-based 3D Object Detection via Cascade Depth Estimation and Calibration." arXiv preprint arXiv:2402.04883 (2024).

\bibitem{huang2024story3d} Huang, Yuzhou, et al. "Story3d-agent: Exploring 3d storytelling visualization with large language models." arXiv preprint arXiv:2408.11801 (2024).

\bibitem{savinov2018semi} Savinov, Nikolay, Alexey Dosovitskiy, and Vladlen Koltun. "Semi-parametric topological memory for navigation." arXiv preprint arXiv:1803.00653 (2018).

\bibitem{majumdar2022zson} Majumdar, Arjun, et al. "Zson: Zero-shot object-goal navigation using multimodal goal embeddings." Advances in Neural Information Processing Systems 35 (2022): 32340-32352.

\bibitem{yadav2023offline} Yadav, Karmesh, et al. "Offline visual representation learning for embodied navigation." Workshop on Reincarnating Reinforcement Learning at ICLR 2023. 2023.

\bibitem{yadav2023ovrl} Yadav, Karmesh, et al. "Ovrl-v2: A simple state-of-art baseline for imagenav and objectnav." arXiv preprint arXiv:2303.07798 (2023).

\bibitem{sun2024fgprompt} Sun, Xinyu, et al. "FGPrompt: fine-grained goal prompting for image-goal navigation." Advances in Neural Information Processing Systems 36 (2024).

\bibitem{zhu2017target} Zhu, Yuke, et al. "Target-driven visual navigation in indoor scenes using deep reinforcement learning." 2017 IEEE international conference on robotics and automation (ICRA). IEEE, 2017.

\bibitem{koh2024generating} Koh, Jing Yu, Daniel Fried, and Russ R. Salakhutdinov. "Generating images with multimodal language models." Advances in Neural Information Processing Systems 36 (2024).

\bibitem{krantz2022instance} Krantz, Jacob, et al. "Instance-specific image goal navigation: Training embodied agents to find object instances." arXiv preprint arXiv:2211.15876 (2022).

\bibitem{schulman2017proximal} Schulman, John, et al. "Proximal policy optimization algorithms." arXiv preprint arXiv:1707.06347 (2017).

\bibitem{anderson2018evaluation} Anderson, Peter, et al. "On evaluation of embodied navigation agents." arXiv preprint arXiv:1807.06757 (2018).

\bibitem{lin2024navcot} Lin, Bingqian, et al. "NavCoT: Boosting LLM-Based Vision-and-Language Navigation via Learning Disentangled Reasoning." arXiv preprint arXiv:2403.07376 (2024).

\bibitem{NavGPT} Zhou, Gengze, Yicong Hong, and Qi Wu. "Navgpt: Explicit reasoning in vision-and-language navigation with large language models." Proceedings of the AAAI Conference on Artificial Intelligence.

\bibitem{hahn2021no} Hahn, Meera, et al. "No rl, no simulation: Learning to navigate without navigating." Advances in Neural Information Processing Systems 34 (2021): 26661-26673.

\bibitem{li2025t2isafety} Li, Lijun, et al. "T2ISafety: Benchmark for Assessing Fairness, Toxicity, and Privacy in Image Generation." arXiv preprint arXiv:2501.12612 (2025).

\bibitem{an2024agfsync} An, Jingkun, et al. "AGFSync: Leveraging AI-Generated Feedback for Preference Optimization in Text-to-Image Generation." arXiv preprint arXiv:2403.13352 (2024).


\end{thebibliography}
\end{sloppypar}

\clearpage
\beginsupplement
\section*{Appendix}
\renewcommand{\thesubsection}{S\arabic{subsection}}

\subsection{\label{chap:S1}PanNuke and MoNuSAC preprocessing}
The PanNuke dataset comprises a set of 7,901 RGB patches, each with dimensions of $256 \times 256$ pixels, which we set as the standard patch size for our analysis. In contrast, the MoNuSAC dataset encompasses 294 images of heterogeneous dimensions. To standardize the MoNuSAC images with our experiments, we implement a standardization protocol. Specifically, for images exceeding the dimensions of $256 \times 256$ pixels, we segment them into equal-sized patches and apply mirror padding to the remaining portions to avoid information loss at the peripherals. Patches with dimensions less than $128 \times 128$ pixels are excluded from the dataset due to the insufficient resolution to capture relevant cellular details. For patches where either dimension falls between 128 and 256 pixels, we employ upsampling to achieve the standard patch size. As a result, we obtain a total of 2,823 RGB patches derived from the MoNuSAC dataset for subsequent analysis. For additional details on the MoNuSAC data preparation process, refer to the source code \cite{Shvetsov_2025a}.
\clearpage

\subsection{\label{chap:S2}Data usage for the methodology}

\counterwithin{figure}{subsection}
\renewcommand{\thefigure}{S\arabic{subsection}}

\begin{figure}[h!]
    \centering
    \includegraphics[width=\textwidth, height=0.85\textheight, keepaspectratio]{images/A2.pdf}
    \caption{Overview of the methodology for cross-labeling, dataset refinement, and model comparison. (1) Cross-relabeling - training and testing cell classification models, (2) Cross-relabeling - using cell classification models to create refined dataset, (3) Fine-tuning and training models for comparison, (4) Student knowledge distillation with refined dataset}
    \label{fig:S2}
\end{figure}
\clearpage

\subsection{\label{chap:S3}Confusion matrices for classification models}
\counterwithin{figure}{subsection}
\renewcommand{\thefigure}{S\arabic{subsection}.\arabic{figure}}

\begin{figure}[h!]
    \centering
    \includegraphics[width=\textwidth, height=0.4\textheight, keepaspectratio]{images/A3_1.pdf}
    \caption{Confusion matrix for PanNuke trained model}
    \label{fig:S3.1}
\end{figure}

\begin{figure}[h!]
    \centering
    \includegraphics[width=\textwidth, height=0.4\textheight, keepaspectratio]{images/A3_2.pdf}
    \caption{Confusion matrix for MoNuSAC trained model}
    \label{fig:S3.2}
\end{figure}

\clearpage

\subsection{\label{chap:S4}Datasets cell counts}

\counterwithin{table}{subsection}
\renewcommand{\thetable}{S\arabic{subsection}}

\begin{table}[h!]
\renewcommand{\arraystretch}{2.0}
\centering
\caption{\label{tab:S4}Cell counts for PanNuke, MoNuSAC and refined datasets. Numbers in parentheses indicate preprocessed cell counts for cell classifier models training and testing.}
%\adjustbox{max width=\textwidth}{%
\begin{tabular}{|l|c|c|c|}
\hline
%\rowcolor{gray!30}
Cell type & PanNuke & MoNuSAC & Refined \\
\hline
Neoplastic & 77,403 (68,031) & - & 105,451 \\
\hline
Epithelial & 26,572 (23,207) & - & 29,926 \\
\hline
Epithelial (benign and malignant) & - & 31,402 & - \\
\hline
Inflammatory & 32,276 & - & - \\
\hline
Lymphocytes & - & 37,045 (33,104) & 65,275 \\
\hline
Neutrophils & - & 1,355 (1,252) & 3,833 \\
\hline
Macrophage & - & 1,842 (1,695) & 3,410 \\
\hline
Dead & 2,908 & - & 2,908 \\
\hline
Connective & 50,585 & - & 50,585 \\
\hline
\end{tabular}
%
%}
\end{table}



\clearpage

\subsection{\label{chap:S5}Definition of validation metrics}
\counterwithin{equation}{subsection}
\renewcommand{\theequation}{\arabic{equation}}

\subsubsection{\label{chap:S5.1}R\textsuperscript{2}}
The coefficient of determination, denoted as $R^2$, is a statistical measure that represents the proportion of variance in the dependent variable that is predictable from the independent variables. In the context of cell quantification in pathology, $R^2$ is used to assess how well the predicted quantities of different cell types in a patch align with the actual quantities observed in the ground truth data, with higher values representing more accurate quantification. $R^2$ is defined as
\begin{equation*}
R^2 = 1 - \frac{\sum_{i=1}^n (y_i - \hat{y}_i)^2}{\sum_{i=1}^n (y_i - \bar{y})^2},
\end{equation*}
where $y_i$ represents the actual number of cells of a specific type in the $i$-th image, $\hat{y}_i$ represents the predicted number of cells of that type in the $i$-th image, $\bar{y}$ is the mean of the actual numbers across all images, and $n$ is the total number of images in the dataset.

The $R^2$ metric has a range of $(-\infty, 1]$. An $R^2$ of 1 indicates perfect prediction, where all predicted values exactly match the actual values. An $R^2$ of 0 suggests that the model explains none of the variability of the response data around its mean. If $R^2$ is negative, it indicates that the model performs worse than a model that simply predicts the mean of the actual values for all observations.

\subsubsection{\label{chap:S5.2}PQ}
Panoptic Quality ($PQ$) is a comprehensive metric used to evaluate the performance of segmentation models in tasks that require both instance segmentation and classification. $PQ$ provides a single score that encapsulates both the detection accuracy (i.e., how many objects were correctly identified) and the segmentation quality (i.e., how accurately the objects' boundaries were delineated). This metric is particularly useful in multiclass scenarios where each pixel is classified into distinct categories, such as different cell types in pathology images.

$PQ$ is calculated as the product of two terms: Detection Quality ($DQ$) and Segmentation Quality ($SQ$). It can be expressed as
\begin{equation*}
PQ = DQ \cdot SQ,
\end{equation*}
where
\begin{equation*}
DQ = \frac{TP}{TP + 0.5\, FP + 0.5\, FN},
\end{equation*}
\begin{equation*}
SQ = \frac{\sum_{(p, g) \in \mathcal{M}} IoU(p, g)}{TP}.
\end{equation*}
In these formulas, $TP$ denotes the number of correctly matched instances between ground truth and prediction, $FP$ denotes the predicted instances that have no corresponding ground truth, $FN$ denotes the ground truth instances that were not detected, $IoU(p, g)$ is the Intersection over Union for a pair of matched instances $p$ (prediction) and $g$ (ground truth), and $\mathcal{M}$ is the set of matched pairs.

The $PQ$ metric is calculated for each class and is averaged across classes to provide a global performance measure.

The $PQ$ score has a range of $[0, 1.0]$, where a higher score indicates better performance in both detecting and segmenting the instances correctly. A $PQ$ of 1 signifies perfect identification and segmentation of all instances, whereas a $PQ$ of 0 indicates that no instances were correctly identified and segmented.

\clearpage

\subsection{\label{chap:S6}Segmentation and Detection quality metrics for teacher and student models}

\begin{table}[h!]
\renewcommand{\arraystretch}{2.0}
\centering
\caption{Segmentation and detection quality for student and teacher models (CI 95\%)}
\label{tab:S6}
%\adjustbox{max width=\textwidth}{%
\begin{tabular}{|l|c|c|}
\hline
%\rowcolor{gray!30}
Metric & Teacher & Student \\
\hline
$SQ_{neoplastic}$ & 0.819 (0.815--0.823) & 0.824 (0.819--0.828) \\
\hline
$SQ_{lymphocyte}$ & 0.795 (0.788--0.802) & 0.790 (0.783--0.796) \\
\hline
$SQ_{connective}$ & 0.770 (0.762--0.776) & 0.780 (0.772--0.786) \\
\hline
$SQ_{dead}$ & 0.659 (0.623--0.688) & 0.657 (0.624--0.695) \\
\hline
$SQ_{epithelial}$ & 0.780 (0.770--0.790) & 0.788 (0.779--0.797) \\
\hline
$SQ_{macrophage}$ & 0.788 (0.760--0.810) & 0.757 (0.730--0.783) \\
\hline
$SQ_{neutrofil}$ & 0.782 (0.761--0.801) & 0.775 (0.759--0.792) \\
\hline
$DQ_{neoplastic}$ & 0.706 (0.692--0.719) & 0.727 (0.712--0.741) \\
\hline
$DQ_{lymphocyte}$ & 0.675 (0.656--0.698) & 0.713 (0.691--0.734) \\
\hline
$DQ_{connective}$ & 0.566 (0.546--0.584) & 0.583 (0.565--0.602) \\
\hline
$DQ_{dead}$ & 0.410 (0.361--0.465) & 0.435 (0.306--0.561) \\
\hline
$DQ_{epithelial}$ & 0.668 (0.639--0.694) & 0.673 (0.644--0.702) \\
\hline
$DQ_{macrophage}$ & 0.657 (0.583--0.727) & 0.615 (0.531--0.703) \\
\hline
$DQ_{neutrofil}$ & 0.691 (0.625--0.753) & 0.729 (0.679--0.778) \\
\hline
\end{tabular}
%
%}
\end{table}

\clearpage

\subsection{\label{chap:S7}QuPath integration method}
We adopt an integration strategy leveraging the paquo \cite{Bayer_AG} library, a Python package that enables direct interaction with QuPath’s internal API, thereby facilitating seamless data exchange without intermediate conversion steps. The data processing pipeline (\hyperref[fig:S7]{Appendix Figure S7}) begins with the acquisition of WSIs and their associated annotations from QuPath, which are represented as Shapely \cite{Gillies_Wel_etal._2024} polygons. Utilizing paquo, we directly read, create, and modify these annotations and detections within a QuPath project in the Python environment. Images are then cropped using these polygons and processed by cell segmentation and classification models employing standard vision processing toolkits such as OpenCV, pyvips, and PyTorch. Additionally, QuPath employs Groovy scripts to initiate a Python process that starts the entire pipeline from QuPath graphical interface: fetching polygons, extracting images from them, and running deep learning model inference on the cropped images. 
The results are returned to QuPath, leveraging paquo's Python bindings to manipulate QuPath data while minimizing the computational overhead typically associated with cross-environment communication.

\counterwithin{figure}{subsection}
\renewcommand{\thefigure}{S\arabic{subsection}}

\begin{figure}[h!]
    \centering
    \includegraphics[width=\textwidth]{images/A7.pdf}
    \caption{QuPath integration workflow using Python environment}
    \label{fig:S7}
\end{figure}

Compared to traditional workflows that involve exporting annotations as GeoJSON, classifying them in Python, and reimporting them into QuPath, our approach offers several advantages. We eliminate the need to switch between programming languages, providing a cohesive and streamlined development process entirely within QuPath software and removing the necessity to use other tools. Meanwhile, we avoid storing annotations as intermediate JSON files unless required for external use or archiving. By conducting the entire inference and post-processing workflow within the Python environment, we leverage the power and flexibility of Python libraries for image processing and machine learning. This approach also enables adjustments to any set of labels and models, thereby improving its applicability.

%\hfill

The distilled model and QuPath integration code are packaged into a Docker container, enabling streamlined execution with the Docker engine. Detailed integration code and deployment instructions can be found in the GitHub repository \cite{Shvetsov_2025b}.

Despite these benefits, we acknowledge that the paquo library is a proof‑of‑concept project in its early development stage and has not been tested across all versions of QuPath.

\clearpage

\subsection{\label{chap:S8}Data and code availability statement}
All datasets, models, and code used in this study are publicly available and can be obtained from the repositories listed below. 
The PanNuke \cite{Gamper_Koohbanani_etal._2019} and MoNuSAC \cite{Verma_Kumar_etal._2021} datasets are publicly accessible, and download information along with detailed descriptions can be found in their respective articles. Preprocessing scripts for PanNuke and MoNuSAC data, as well as individual cell extraction scripts, are available on GitHub \cite{Shvetsov_2025a}. The H-Optimus foundation model used in our experiments can be downloaded from the HuggingFace repository \cite{hoptimus2024}, and model information is available on GitHub \cite{Saillard_Jenatton_etal._2024}. In addition, the integration code for QuPath and the distilled model packaged in a Docker container are provided in the repository \cite{Shvetsov_2025b}, and paquo Python library is available from the authors GitHub repository \cite{Bayer_AG}.
\clearpage

\end{document}






\documentclass[journal,compsoc]{IEEEtran}
\usepackage{epsfig}
\usepackage{graphicx}
\usepackage{amsmath}
\usepackage{amssymb}
\usepackage{algorithm, algorithmic}

\usepackage{diagbox}
\usepackage{float}
\usepackage{afterpage}
\usepackage{bm}
\usepackage{subfig}

%\usepackage{tabu}
\usepackage{multirow}
\usepackage{color}
\usepackage{tablefootnote}
\usepackage{adjustbox}
\usepackage{wrapfig}

\usepackage{hyperref}       % hyperlinks
\usepackage{url}            % simple URL typesetting
\usepackage{booktabs}       % professional-quality tables
\usepackage{amsfonts}       % blackboard math symbols
\usepackage{nicefrac}       % compact symbols for 1/2, etc.
\usepackage{microtype}      % microtypography
\usepackage{times}
\usepackage{epsfig}
%\usepackage{tabu}
%\usepackage{overpic}
\usepackage{bbding}
\usepackage{etoolbox}
\usepackage{paralist}
\usepackage{ulem}
\usepackage{tikz}

\usepackage{makecell}

\usepackage{xcolor,colortbl}

% \usepackage[pagebackref=true,breaklinks=true,colorlinks,bookmarks=false]{hyperref}


\newcolumntype{Y}{p{0.5cm}<{\centering}}
\newcommand{\mc}[2]{\multicolumn{#1}{c}{#2}}
\definecolor{Gray}{gray}{0.5}
\definecolor{LightCyan}{rgb}{0.88,1,1}

\newcolumntype{a}{>{\columncolor{Gray}}c}
\newcolumntype{b}{>{\columncolor{white}}c}



\DeclareMathOperator*{\cat}{Cat}


\def\H{\operatorname{H}}
\def\I{\operatorname{I}}
\def\KL{\operatorname{KL}}


\def\etal{\textit{et al}.}
\def\ie{\textit{i.e.}}
\def\eg{\textit{e.g.}}
\def\etc{\textit{etc}}
\def\wrt{\textit{w.r.t. }}

\def\bz{\textcolor{blue}}
\def\xc{\textcolor{red}}
\newcommand{\tb}[1]{\textbf{#1}}
\newcommand{\bc}[1]{\textcolor[RGB]{192,0,0}{#1}}
\newcommand{\rc}[1]{\textcolor{blue}{#1}}
% \newcommand{\rb}[1]{\textcolor{teal}{#1}}
% \newcommand{\bb}[1]{\textcolor{blue}{#1}}
\newcommand{\bb}[1]{\textcolor[RGB]{192,0,0}{\textbf{#1}}}
\newcommand{\rb}[1]{\textcolor{blue}{\textbf{#1}}}
\newcommand{\todo}[1]{{\color{blue}{[TODO: #1]}}}

% \newcommand{\rev}[1]{\textcolor{red}{#1}}
\newcommand{\rev}[1]{{#1}}
\renewcommand{\thefootnote}{\fnsymbol{footnote}}


\normalem
\begin{document}

\title{Rotation-Adaptive Point Cloud Domain Generalization via Intricate Orientation Learning \\
—— Supplementary Material —— }

\author{{Bangzhen~Liu,~Chenxi~Zheng,~Xuemiao~Xu,~Cheng Xu,~Huaidong~Zhang, \\ and~Shengfeng~He,~\IEEEmembership{Senior Member,~IEEE}}

\thanks{ Bangzhen Liu,~Chenxi~Zheng, and~Xuemiao~Xu are with the School of Computer Science and Engineering, South China University of Technology, Guangzhou, China. E-mail: liubz.scut@gmail.com,~cszcx@mail.scut.edu.cn, and~xuemx@scut.edu.cn.}
\thanks{ Cheng Xu is with the Centre for Smart Health, The Hong Kong Polytechnic University, Hong Kong. E-mail: cschengxu@gmail.com}
\thanks{ Huaidong Zhang is with the School of Future Technology, South China University of Technology, Guangzhou, China. E-mail: huaidongz@scut.edu.cn.}
\thanks{ Shengfeng He is with the School of Computing and Information Systems, Singapore Management University, Singapore. E-mail: shengfenghe@smu.edu.sg.}
}

\markboth{IEEE Transactions on Pattern Analysis and Machine Intelligence}%
{Shell \MakeLowercase{\textit{Liu et al.}}: Rotation-Adaptive Point Cloud Domain Generalization via Intricate Orientation Learning}


\maketitle

\IEEEdisplaynontitleabstractindextext

\IEEEpeerreviewmaketitle


\section{More Experimental Results} \label{sec1}

It is worth noting that the three sub-datasets used in PointDA are all category-wise imbalanced, as shown in Table~\ref{table:dataset}, which indicates that the micro-average precision score (\textit{Acc.}) reported by previous studies is inappropriate to assess the generalizability of cross-domain classification. In the main paper, we instead report the results of PointDA in the form of the macro-average precision score (\textit{Avg.}) for a more convincing evaluation. We also report the extra evaluations in the form of \textit{Acc.} in Table~\ref{tab:pointda10_acc} for reference. Our method still outperforms all the competitors in the average metric over the six cross-domain tasks.

\begin{table}[h]
    \caption{{Number of samples for each category in PointDA~\cite{qin2019pointdan}.}}
    \vspace{-2ex}
    \label{table:dataset}
    \scriptsize
    % \begin{center}
    \setlength{\tabcolsep}{0.05cm}{
      \resizebox{0.48\textwidth}{!}{
        \begin{tabular}{c|c|c|c|c|c|c|c|c|c|c|c}
          \hline  & Tub & Bed & Shelf & Case & Chair & Lamp & Monit. & Plant & Sofa & Table & Total\\
          \hline 
          ModelNet & 106 & 515 & 572 & 200 & 889 & 124 & 465 & 240 & 680 & 392 & 4183 \\
          ShapeNet & 599 & 167 & 310 & 1076 & 4612 & 1620 & 762 & 158 & 2198 & 5876 & 17378 \\
          ScanNet & 98 & 329 & 464 & 650 & 2578 & 161 & 210 & 88 & 495 & 1037 & 6110 \\
          \hline
          \end{tabular}
      }
     }
    % \end{center}
    \vspace{-2mm}
  \end{table}
  

\noindent\textbf{Evaluation on Aligned Dataset.} {We additionally implement our method under the traditional aligned data scenario, where the rotation only happens on the z-axis. In this case, we adapt our intricate orientation mining approach to specifically identify the most intricate orientations along the z-axis. 
The comparisons with state-of-the-art 3DDG methods are shown in Table~\ref{tab:align}, where the results of competitors are directly borrowed from their papers. Our method surpasses the baselines on all six tasks, demonstrating its effectiveness. The proposed orientation-aware contrastive training enables the model to gain a more comprehensive understanding of point clouds from various challenging perspectives, thereby enhancing the generalizability of the learned features. We notice that our method is slightly inferior on M$\to$S* and S$\to$S*. Since the orientational shift is our major concern, we do not have a special design for capturing geometric information under self-occlusions. However, in this case, our method still outperforms the two 3DDG methods on three out of the six tasks, while achieving the best average accuracy. Furthermore, the experimental results also reveal the presence of rotational shifts in the aligned data scenes, demonstrating the potential of our method for solving this problem.}

\begin{table}[h] % table for OSDA setting on Office31
    \caption{Comparison of the \textit{Acc.} ($\%$) under the 3D domain generalization setting. The best records are marked in \textbf{bold}.}
    \label{tab:align} 
    \vspace{-3ex}
    \small
    \begin{center}
    \setlength{\tabcolsep}{0.1cm}{
    \resizebox{0.48\textwidth}{!}{
    \begin{tabular}{c|c|c|c|c|c|c|c}
    \hline

    Methods
    &{M$\to$S} & {M$\to$S*} & {S$\to$M} & {S$\to$S*} & {S*$\to$M} & {S*$\to$S} & {Avg}\\

    \hline
    Supervised                      &{93.9} &{78.4} &{96.2} &{78.4} &{96.2} &{93.9} &{89.5}\\
    w/o Adapt                       &{83.3} &{43.8} &{75.5} &{42.5} &{63.8} &{64.2} &{62.2}\\
    \hline
    {Metasets~\cite{huang2021metasets}} &\tb{86.0} &{52.3} &{67.3} &{42.1} &{69.8} &{69.5} &{64.5}\\
    {PDG~\cite{wei2022learning}}        &{85.6} &\tb{57.9} &{73.1} &{50.0} &{70.3} &{66.3} &{67.2}\\
    % \hline
    {Ours}                              &{83.8} &{46.0} &\tb{83.2} &{45.5} &\tb{76.4} &\tb{70.3} &\tb{67.5}\\
    \hline

    \end{tabular}

    }
    }
    \end{center}
    \vspace{-3mm}
\end{table}


\begin{table*}[h] % table for OSDA setting on Office31
    \caption{Comparison of the micro-average precision score \textit{Acc.}~($\%$) under the orientation-aware 3D domain generalization setting. The top 2 records are marked in \bc{red} and \rc{blue}.}
    \label{tab:pointda10_acc} 
    \vspace{-3ex}
    \small
    \begin{center}
    \setlength{\tabcolsep}{0.35cm}{
    \resizebox{1\textwidth}{!}{
    \begin{tabular}{c|c|c|c|c|c|c|c|c}
    \hline

    {Methods}
    &Type &{M$\to$S} & {M$\to$S*} & {S$\to$M} & {S$\to$S*} & {S*$\to$M} & {S*$\to$S} & {AVG} \\

    
    % \cline{2-19}
    % \cline{14-19}  
    % \cmidrule $\pm$ r{2-13}
    % \cmidrule $\pm$ r{14-19}
    \hline
    Supervised                            &\multirow{2}{*}{-}            &{86.6 $\pm$ 6.3}    &{69.6 $\pm$ 3.2}    &{88.4 $\pm$ 14.7}   &{69.6 $\pm$ 3.2}    &{88.4 $\pm$ 14.7}  &{86.6 $\pm$ 6.3}    &{81.5}\\
    w/o Adapt                             &          &{57.9 $\pm$ 15.1}   &{28.7 $\pm$ 5.1}    &{54.1 $\pm$ 8.6}   &{28.8 $\pm$ 4.7}    &{43.0 $\pm$ 4.8}  &{42.3 $\pm$ 6.0}    &{42.5}\\
    \hline   
    VN~\cite{Deng_2021_ICCV}              &\multirow{2}{*}{RE}             &{70.5 $\pm$ 0.0} &{30.6 $\pm$ 0.0} &{66.7 $\pm$ 0.0} &{32.0 $\pm$ 0.0} &{39.4 $\pm$ 0.0} &{44.8 $\pm$ 0.0}   &{47.3} \\  
    SVN~\cite{su2022svnet}               &            &{66.8 $\pm$ 0.6} &{32.3 $\pm$ 0.4}   &{62.0 $\pm$ 0.5} &{30.0 $\pm$ 0.6}    &{38.0 $\pm$ 0.9} &{42.2 $\pm$ 1.1}   &{45.7} \\ 
    EOMP~\cite{luo2022equivariant}        &              &{61.4 $\pm$ 0.8} &{28.1 $\pm$ 0.3}    &{60.5 $\pm$ 0.6} &{37.0 $\pm$ 0.7}  &{27.9 $\pm$ 0.8} &{37.2 $\pm$ 0.9}    &{42.0} \\
    
    \hline
    SPRIN~\cite{you2021prin}              &\multirow{5}{*}{RI}             &{68.2 $\pm$ 0.4} &{30.1 $\pm$ 0.6}   &{71.8 $\pm$ 0.6} &{30.4 $\pm$ 0.6}    &{46.8 $\pm$ 0.6} &{49.3 $\pm$ 0.5}  &{49.4}\\
    RIPCA~\cite{li2021closer}              &            &{70.3 $\pm$ 1.2} &{33.0 $\pm$ 0.7}   &{70.4 $\pm$ 0.9} &{39.1 $\pm$ 1.3}    &\rc{49.9 $\pm$ 1.6} &{50.6 $\pm$ 2.2}  &\rc{52.2}\\
    RIConv++~\cite{zhang2022riconv}        &              &{28.8 $\pm$ 0.6} &{14.2 $\pm$ 0.5}   &{55.1 $\pm$ 0.7} &{38.9 $\pm$ 0.5}    &{34.8 $\pm$ 0.7} &{47.3 $\pm$ 0.5}  &{36.5}\\
    PaRI~\cite{chen2022devil}              &             &{36.1 $\pm$ 0.0} &{29.3 $\pm$ 0.3}   &{51.8 $\pm$ 0.8} &\bc{44.8 $\pm$ 0.4}    &{43.3 $\pm$ 0.9} &{49.4 $\pm$ 0.1} &{42.5} \\
    LocoTrans~\cite{chen2024local}              &                  &\bc{76.7 $\pm$ 0.0}   &{34.5 $\pm$ 0.3}    &\rc{74.3 $\pm$ 0.4}  &\rc{43.6 $\pm$ 0.2}  &{41.6 $\pm$ 0.6} &{41.5 $\pm$ 0.0}    &{52.0}\\
    \hline
    PointDAN~\cite{qin2019pointdan}       &\multirow{5}{*}{DA}              &{59.8 $\pm$ 15.1}   &{29.5 $\pm$ 4.0}    &{55.2 $\pm$ 6.9}   &{24.0 $\pm$ 4.8}    &{38.0 $\pm$ 4.8}  &{47.4 $\pm$ 6.1}    &{42.3}\\
    DefRec~\cite{achituve2021self}        &              &{57.2 $\pm$ 13.3}   &{33.1 $\pm$ 4.6}    &{54.4 $\pm$ 8.0}    &{33.1 $\pm$ 4.4}   &{38.8 $\pm$ 6.5}  &{48.2 $\pm$ 5.7}    &{44.1}\\
    GAST~\cite{zou2021geometry}           &              &{27.7 $\pm$ 4.2}   &{7.0 $\pm$ 0.6}     &{40.8 $\pm$ 2.6}   &{5.8 $\pm$ 0.8}     &{30.7 $\pm$ 1.5}  &{50.7 $\pm$ 3.6}     &{27.1}\\
    MLSP~\cite{liang2022point}            &              &{66.5 $\pm$ 15.5}   &{32.8 $\pm$ 4.3}    &{59.7 $\pm$ 5.1}   &{30.0 $\pm$ 6.4}    &{46.3 $\pm$ 5.0}  &{52.2 $\pm$ 5.8}     &{47.9}\\
    SDDA~\cite{cardace2023self}           &              &{65.0 $\pm$ 14.5}  &\rc{37.8 $\pm$ 3.4}     &{61.4 $\pm$ 5.4}    &{40.1 $\pm$ 4.1}   &{40.7 $\pm$ 6.3}  &\rc{53.3 $\pm$ 6.4}     &{49.7}\\
    PCFEA~\cite{wang2024progressive}     &                   &{62.0 $\pm$ 13.6}   &{9.3 $\pm$ 0.2}   &{42.7 $\pm$ 8.7}   &{43.1 $\pm$ 4.0}   &{47.1 $\pm$ 4.0}   &\bc{54.0 $\pm$ 4.6}    &{43.0}    \\
    \hline
    {Metasets~\cite{huang2021metasets}}   &\multirow{3}{*}{DG}               &{53.9 $\pm$ 1.4} &\bc{40.3 $\pm$ 0.9}   &{32.2 $\pm$ 12.3}  &{33.5 $\pm$ 1.7} &{24.5 $\pm$ 4.6}    &{39.8 $\pm$ 10.0} &{37.4}\\
    {PDG~\cite{wei2022learning}}          &             &{25.4 $\pm$ 29.5}   &{21.2 $\pm$ 18.0}   &{38.4 $\pm$ 18.5}   &{8.1 $\pm$ 3.2}   &{30.3 $\pm$ 4.8}   &{29.7 $\pm$ 12.0}    &{25.5}\\
    % \hline
    {Ours}                                &              &\rb{70.8 $\pm$ 2.0}   &{37.2 $\pm$ 1.2}    &\bb{80.7 $\pm$ 0.6}   &{34.0 $\pm$ 1.0}     &\bb{50.0 $\pm$ 2.5}   &{47.1 $\pm$ 3.2}     &\bb{53.3}  \\
    \hline

    \end{tabular}

    }
    }
    \end{center}
    \vspace{-3mm}
\end{table*}


\noindent\textbf{Analysis of Hyper-parameter Sensitivity.} 
We evaluate the effects of varying $\lambda_{oc}$ and $\lambda_{ms}$, by changing the value while keeping the other frozen as 0.1. As Fig.~\ref{fig:ablation}(a) and Fig.~\ref{fig:ablation}(b) show, $\lambda_{oc}$ is insensitive across a large range, while larger $\lambda_{ms}$ may slightly decrease the performance of our model. According to the variation of performance curves, we choose $\lambda_{oc}=0.01$ and $\lambda_{ms}=0.01$ as the model setting in our main paper.
\begin{figure}[h]
    % \flushleft
    \centering
    \subfloat[Ablation of $\lambda_{oc}$]{%[b]{0.45\textwidth}
        \label{fig:plot_lambda_oc}
        \includegraphics[width=0.22\textwidth]{./resources/supp/ablation_cons_weights.pdf}
    }
    % \hspace{2mm}
    \subfloat[Ablation of $\lambda_{ms}$]{%[b]{0.45\textwidth}
        \label{fig:plot_lambda_ms}
        \includegraphics[width=0.22\textwidth]{./resources/supp/ablation_reg_weights.pdf}
    }
    % \vspace{-2mm}
    \caption{The curves of performance \wrt varying $\lambda_{oc}$ and $\lambda_{ms}$.}
    \label{fig:ablation}
    % \vspace{-5.5mm}
\end{figure} 


\noindent\textbf{Analysis of Training Stability.} 
We plot the curves of the proposed orientation consistency loss and the marginal separation loss over the training stage to demonstrate the convergence of our intricate orientational learning. As Fig.~\ref{fig:plot}(a) and Fig.~\ref{fig:plot}(b) show, all the losses gradually decrease and converge to a convincing degree. The blue curves are the orientation consistency loss, which periodically bursts every 20 epochs. This is due to the update of the intricate orientation set, which gradually adapts the model to all the intricate orientations. At the end of the training stage, the amplification tends to be stable, indicating the consistency of the object towards various rotations.


\begin{figure}[h]
    % \flushleft
    \centering
    \subfloat[M$\to$S]{%[b]{0.45\textwidth}
        \label{fig:m2s_loss}
        \includegraphics[width=0.23\textwidth]{./resources/supp/m2s_loss.pdf}
    }
    % \hspace{2mm}
    \subfloat[M$\to$S*]{%[b]{0.45\textwidth}
        \label{fig:m2ss_loss}
        \includegraphics[width=0.23\textwidth]{./resources/supp/m2ss_loss.pdf}
    }
    \vspace{-2ex}
    \caption{The training curves (\ie, $L_{cls}$, $L_{oc}$, and $L_{ms}$) on M$\to$S (a) and M$\to$S* (b).}
    \label{fig:plot}
    \vspace{-3mm}
\end{figure} 



\noindent\textbf{Analysis of Time Complexity.} {We report the computational costs of training/testing one batch of data in milliseconds for different compared methods in Table~\ref{tab:complexity}. The results are obtained by accumulating the running times within a single training/testing epoch and calculating the mean value w.r.t. one batch.} Due to the process of diversifying the intricate orientation set, our method introduces extra computational costs in the training phase. Nonetheless, our method yields the best performance among these methods while achieving the second-best inferencing speed, which is more efficient than the other RE and RI methods that require extra time-consuming modules for practical applications. 


\begin{table}[h]
    \caption{Time statistics (ms) of training/testing on one batch of data.}
    \label{tab:complexity}
    \vspace{-3ex}
    \small
    \begin{center}
    \setlength{\tabcolsep}{0.3cm}{
    \resizebox{0.4\textwidth}{!}{
      \begin{tabular}{c|c|c|c|c} 
        \hline 
        Methods & Type & Avg. & $T_{train}$& $T_{test}$ \\
        \hline
        VN~\cite{Deng_2021_ICCV} & RE & 41.3 & 808 & 27.9 \\
        SPRIN~\cite{you2021prin} & RI & 43.9 & 1551 & 370.2 \\
        RIPCA~\cite{li2021closer} & RI & 46.6 & 717 & 23.3 \\
        MLSP~\cite{liang2022point} & DA & 43.2 & 825 & 36.5 \\
        SDDA~\cite{cardace2023self} & DA & 43.1 & \bf{567} & \bf{12.0} \\
        \hline
        Ours & DG & \bf{49.6} & 2114 & 14.5 \\
        \hline
        \end{tabular}
    }
    }
    \end{center}
    \vspace{-3mm}
  \end{table}
  


\section{Extra Visualizations and Analysis} \label{sec2}

\noindent\textbf{The Learned Intricate Augmented Samples.} {In Fig.~\ref{fig:intricat_angle}, we select several point clouds and provide visualizations of how their intricate orientations evolve during training. We trained the model on ModelNet and optimized the intricate set on the testing set every 20 epochs. Each row of the point cloud sequence shows the current pose of the given point cloud augmented by its corresponding intricate orientation at that specific epoch. 
Beneath each sequence, we also visualize the distribution of predicted probabilities and the consistency of prediction over different testing orientations. 
Specifically, for each point cloud, we obtain the predicted probabilities of its 64 testing variants $P = {\{P_a|P_a = \left[p^1_a, ..., p^C_a\right]\}}^A_{a=1}$, where $A=64$ is the number of testing orientation series and $C=10$ is the number of categories. 
The visualized probabilities' distribution $P_m$ is calculated by averaging the predictions over the 64 testing rotation series, such that $P_m = \left[\frac{1}{A}\sum_{j=1}^{A}p^1_j, ..., \frac{1}{A}\sum_{j=1}^{A}p^C_j\right]$. 
To evaluate the predicted consistency, we adopt the entropy as the metric and calculate the consistency $Ent_m$ over the 64 predicted probabilities by 
\begin{equation*}
  Ent_m = \left[\frac{1}{A}\sum_{j=1}^{A}p^1_j log p^1_j, ..., \frac{1}{A}\sum_{j=1}^{A}p^C_j log p^C_j\right].
\end{equation*}
As the number of training epochs increases, both the confidence and output consistency of the model are enhanced. For samples located near the decision boundaries, such as row 6 and row 9, learning with intricate orientation mining could significantly alleviate the ambiguity of learned features, thereby producing a more robust and generalizable classifier for downstream tasks.
}




\begin{figure*}[h]
    % \flushleft
    \centering
    \subfloat[Metasets]{%[b]{0.45\textwidth}
        \label{fig:cm1_m2s}
        % \centering
        \includegraphics[width=0.24\textwidth]{./resources/supp/conf_mat_modelnet2shapenet_Metaset.pdf}
        % \vspace{10mm}
    }
    \subfloat[PDG]{%[b]{0.45\textwidth}
        \label{fig:cm2_m2s}
        % \centering
        \includegraphics[width=0.24\textwidth]{./resources/supp/conf_mat_modelnet2shapenet_PDG.pdf}
        % \vspace{10mm}
    }
    \subfloat[Ours]{%[b]{0.45\textwidth}
        \label{fig:cm3_m2s}
        % \centering
        \includegraphics[width=0.24\textwidth]{./resources/supp/conf_mat_modelnet2shapenet.pdf}
        % \vspace{10mm}
    }
    \vspace{-2ex}
    \caption{The confusion matrices of Metaset, PDG, and our method on M$\to$S. Zoom in for details.}
    \vspace{-3ex}
    \label{fig:visualization_m2s}
\end{figure*} 
% \vspace{-4mm}
\begin{figure*}[h]
    % \flushleft
    \centering
    \subfloat[Metasets]{%[b]{0.45\textwidth}
        \label{fig:cm1_s2m}
        % \centering
        \includegraphics[width=0.24\textwidth]{./resources/supp/conf_mat_shapenet2modelnet_Metaset.pdf}
        % \vspace{10mm}
    }
    \subfloat[PDG]{%[b]{0.45\textwidth}
        \label{fig:cm2_s2m}
        % \centering
        \includegraphics[width=0.24\textwidth]{./resources/supp/conf_mat_shapenet2modelnet_PDG.pdf}
        % \vspace{10mm}
    }
    \subfloat[Ours]{%[b]{0.45\textwidth}
        \label{fig:cm3_s2m}
        % \centering
        \includegraphics[width=0.24\textwidth]{./resources/supp/conf_mat_shapenet2modelnet.pdf}
        % \vspace{10mm}
    }
    \vspace{-2ex}
    \caption{The confusion matrices of Metaset, PDG, and our method on S$\to$M. Zoom in for details.}
    \label{fig:visualization_s2m}
\end{figure*} 


\begin{figure*}[h]
    % \flushleft
    \centering
    \subfloat{\label{fig:0}
        \begin{minipage}[b]{1.0\textwidth}\centering
            \includegraphics[width=0.95\textwidth]{./resources/supp/underline.pdf} 
            \\
            \includegraphics[width=0.9\textwidth]{./resources/supp/all_title.pdf} 
            \\
            \includegraphics[width=0.9\textwidth]{./resources/supp/all_cat_0_cropped.pdf} 
            \\
            \vspace{-3mm}
            \includegraphics[width=1.0\textwidth]{./resources/supp/all_predictions_cat_0_cropped.pdf}
        \end{minipage}
    }\vspace{-3mm}

    \subfloat{\label{fig:1}
        \begin{minipage}[b]{1.0\textwidth}\centering
            \includegraphics[width=0.9\textwidth]{./resources/supp/all_cat_1_cropped.pdf} 
            \\
            \vspace{-3mm}
            \includegraphics[width=1.0\textwidth]{./resources/supp/all_predictions_cat_1_cropped.pdf}
        \end{minipage}
    }\vspace{-3mm}

    \subfloat{\label{fig:2}
        \begin{minipage}[b]{1.0\textwidth}\centering
            \includegraphics[width=0.9\textwidth]{./resources/supp/all_cat_2_cropped.pdf} 
            \\
            \vspace{-3mm}
            \includegraphics[width=1.0\textwidth]{./resources/supp/all_predictions_cat_2_cropped.pdf}
        \end{minipage}
    }\vspace{-3mm}

    \subfloat{\label{fig:4}
        \begin{minipage}[b]{1.0\textwidth}\centering
            \includegraphics[width=0.9\textwidth]{./resources/supp/all_cat_3_cropped.pdf} 
            \\
            \vspace{-3mm}
            \includegraphics[width=1.0\textwidth]{./resources/supp/all_predictions_cat_3_cropped.pdf}
        \end{minipage}
    }\vspace{-3mm}

    \subfloat{\label{fig:6}
        \begin{minipage}[b]{1.0\textwidth}\centering
            \includegraphics[width=0.9\textwidth]{./resources/supp/all_cat_4_cropped.pdf} 
            \\
            \vspace{-3mm}
            \includegraphics[width=1.0\textwidth]{./resources/supp/all_predictions_cat_4_cropped.pdf}
        \end{minipage}
    }\vspace{-3mm}

    \subfloat{\label{fig:7}
        \begin{minipage}[b]{1.0\textwidth}\centering
            \includegraphics[width=0.9\textwidth]{./resources/supp/all_cat_5_cropped.pdf} 
            \\
            \vspace{-3mm}
            \includegraphics[width=1.0\textwidth]{./resources/supp/all_predictions_cat_5_cropped.pdf}
        \end{minipage}
    }\vspace{-3mm}

    \subfloat{\label{fig:8}
        \begin{minipage}[b]{1.0\textwidth}\centering
            \includegraphics[width=0.9\textwidth]{./resources/supp/all_cat_6_cropped.pdf} 
            \\
            \vspace{-3mm}
            \includegraphics[width=1.0\textwidth]{./resources/supp/all_predictions_cat_6_cropped.pdf}
        \end{minipage}
    }

    \caption{Visualization of the learned intricate orientation series on ModelNet (M). Each row of point cloud sequence records the transformation of the point cloud's poses after augmention by its corresponding intricate rotational angle during the training procedure. The interval of recording is 20 epoch. The statistic measurements at every records, including the predicted probability and rotational consistency, are presented beneath each point cloud sequence. The category which the current point cloud belongs to is marked in \bc{red}.}
    \label{fig:intricat_angle}
    % \vspace{-6mm}
\end{figure*}



\noindent\textbf{Confusion Matrices.} {we provide the evaluation results for Metasets~\cite{huang2021metasets}, PDG~\cite{wei2022learning}, and our method in the form of confusion matrix on the target domain. ShapeNet is a dataset whose samples are highly imbalanced across different categories, while ModelNet is much more balanced. The confusion matrices of the three approaches are shown in  Fig.~\ref{fig:visualization_m2s} and Fig.~\ref{fig:visualization_s2m}. Compared with the other two 3D domain generalization methods, our method has more compact confusion matrices under the orientation shift. For M$\to$S, both our method and Metaset are separated relatively well while PDG has much inaccurate classification on class "Plant". The inner reason is that the part-based feature utilized by PDG may encounter confusing local expressions, such as the plane of the table and the bottom of a potted plant. For S$\to$M, our method achieves more balanced and concise results. We observe that the sample of class "monitor" is much easier to misclassify into "bed" due to the similar plane structure of their surface. Similar trends happen for the categories "table" and "cabinet", which have less discriminative features in the view of shape. }

{In summary, the single shape cannot serve as a discriminative representation in some cases. This is the limitation of shape representation under the orientation shift since there are a lot of objects whose shapes are similar but belong to different categories. In this case, extra visual (\eg, texture or color), linguistic information, or spatial cues are important to provide complement representation, which may benefit the problem of cross-domain generalization under orientation shift. We will plan to investigate the function of these features in our future work.  }

\section{Gradient of the rotation parameters} \label{sec3}
In this section, we provide detailed calculations about the optimizable parameters $\Theta$ concerning a given model $F$. Considering the objective of optimizing $\Theta$ within a standard classification task, we have the following objective:
\begin{equation}
  \hat{\Theta} = \mathop{\arg\max_{\Theta}}L(w_{opt}, \hat{P}, y), 
\end{equation}
where $w_{opt}$ is the freeze parameter of $F$, $(\hat{P}, y)$ are the augmented point cloud and label:
\begin{equation}
  \begin{split}
    ~\hat{P}&=f(\hat{\Theta}, P)  \\
    &=R_{\theta_{x}}\cdot R_{\theta_{y}}\cdot R_{\theta_{z}}\cdot P.
  \end{split}
\end{equation}
According to the chain rules, the gradient of $\hat{\Theta}$ is calculated by:
\begin{equation}
  \begin{split}
  \frac{\partial L}{\partial \hat{\Theta}} &= \frac{\partial L}{\partial \hat{P}} \frac{\partial \hat{P}}{\partial \hat{\Theta}} \\
  &=\frac{\partial L}{\partial \hat{P}}
  \left(
  \frac{\partial R_{\theta_{x}}}{\partial \theta_{x}}
  R_{\theta_{y}}
  R_{\theta_{z}} \quad
  R_{\theta_{x}}
  \frac{\partial R_{\theta_{y}}}{\partial \theta_{y}}
  R_{\theta_{z}} \quad
  R_{\theta_{x}}
  R_{\theta_{y}}
  \frac{\partial R_{\theta_{z}}}{\partial \theta_{z}}
  \right)P,
  \end{split}
\end{equation}
where 
\begin{equation}
  \begin{split}
R_{\theta_{x}} = 
\begin{pmatrix}
  1 & 0 & 0 \\
  0 & \cos\theta_{x} & -\sin\theta_{x} \\
  0 & \sin\theta_{x} & \cos\theta_{x} 
\end{pmatrix},
\\
R_{\theta_{y}} = 
\begin{pmatrix}
  \cos\theta_{y} & 0 & \sin\theta_{y} \\
  0 & 1 & 0 \\
  -\sin\theta_{y} & 0 & \cos\theta_{y}
\end{pmatrix},
\\
R_{\theta_{z}} = 
\begin{pmatrix}
  \cos\theta_{z} & -\sin\theta_{z} & 0  \\
  \sin\theta_{z} & \cos\theta_{z} & 0 \\
  0 & 0 & 0
\end{pmatrix},
\end{split}
\end{equation}
and 
\begin{equation}
    \begin{split}
\frac{\partial R_{\theta_{x}}}{\partial \theta_{x}} = 
\begin{pmatrix}
  0 & 0 & 0 \\
  0 & -\sin\theta_{x} & -\cos\theta_{x} \\
  0 & \cos\theta_{x} & -\sin\theta_{x} 
\end{pmatrix}, \\
\frac{\partial R_{\theta_{y}}}{\partial \theta_{y}} = 
\begin{pmatrix}
  -\sin\theta_{y} & 0 & \cos\theta_{y} \\
  0 & 0 & 0 \\
  -\cos\theta_{y} & 0 & -\sin\theta_{y}
\end{pmatrix}, \\
\frac{\partial R_{\theta_{z}}}{\partial \theta_{z}} = 
\begin{pmatrix}
  -\sin\theta_{z} & -\cos\theta_{z} & 0  \\
  \cos\theta_{z} & -\sin\theta_{z} & 0 \\
  0 & 0 & 1
\end{pmatrix}.
\end{split}
\end{equation}



\section{Theoretical Analysis for Rotation-Adaptive Point Cloud Domain Generalization} \label{sec4}

In this section, we provide theoretical proof demonstrating how orientational consistency functions to bridge the domain gap, analyzed from the perspective of mutual information reduction.

Let $X\!=\!(U, V)$ represent a 3D point cloud, where $U$ corresponds to orientation-dependent variables and $V$ to orientation-independent variables. In our case, we assume that the ranges of $U$ and $V$ remain consistent across domains.
For $X_s\!\sim\!p_\mathrm{src}(x)$, where $p_\mathrm{src}(x)$ denotes the source domain data distribution, the marginal distributions \wrt $U_s$ and $V_s$ are expressed by:
\begin{equation}
 p_\mathrm{src}(u)=\int p_\mathrm{src}(x) \mathrm{d}v, \quad p_\mathrm{src}(v)=\int p_\mathrm{src}(x) \mathrm{d}u.
\end{equation}
Considering the data distribution $X_a\!\sim\!p_\mathrm{aug}(x)$ after augmentation, where each sample is assumed to be uniformly sampled \wrt orientations, the marginal distributions \wrt $U_a$ and $V_a$ are given by:
\begin{equation}
 p_\mathrm{aug}(u)=\mathcal{U}(\mathcal{D}_{U_a}), \quad p_\mathrm{aug}(v)=p_\mathrm{src}(v),
\end{equation}
where $\mathcal{U}(\cdot)$ denotes a uniform distribution over the measurable domain $\mathcal{D}_{U_a}$ of ${U_a}$. For simplicity, the subscript of $U_a$ in $\mathcal{D}_{U_a}$ is omitted without causing ambiguity in the subsequent analysis. In this work, we adopt the proposed orientation-aware contrastive learning framework to approximately achieve this, where ${U_a}$ is represented by Euler angles and $\mathcal{D}_{U}:=[-\pi, \pi)^3$. 



Based on the definition of joint entropy, the entropy of $p_\mathrm{src}(x)$ can be expressed in terms of its marginal entropies \wrt $U_s$ and $V_s$, along with an additional term presenting the mutual information between these two components:
\begin{equation}
\begin{aligned}
    \H(X_s) 
 =& \H(U_s)+\H(V_s)-\I(U_s;V_s) \\
 =& \mathbb{E}_{u\sim p_\mathrm{src}(u)}[-\log p_\mathrm{src}(u)] + \mathbb{E}_{v\sim p_\mathrm{src}(v)}[-\log p_\mathrm{src}(v)] \\
    &- \mathbb{E}_{x\sim p_\mathrm{src}(x)}\log \frac{p_\mathrm{src}(x)}{p_\mathrm{src}(u)p_\mathrm{src}(v)},
\end{aligned}
\end{equation}
where $\I(U_s;V_s)$ represents the mutual information between $U_s$ and $V_s$ in $p_\mathrm{src}(x)$. 
Since $p_\mathrm{aug}(u)$ follows a uniform distribution and $U_a$ and $V_a$ of $p_\mathrm{aug}(x)$ are independent under this setting, the entropy of $p_\mathrm{aug}(x)$ is given by $\I_\mathrm{aug}(U_a;V_a)\!=\!0$, and the entropy of $p_\mathrm{aug}(u)$ corresponds to the measure of $\mathcal{D}_{U}$, denoted as $m(\mathcal{D}_{U})$. 
Thus, the entropy of $p_\mathrm{aug}(x)$ can be simplified as follows:
\begin{equation}
\begin{aligned}
    \H(X_a) &= \H(U_a)+\H(V_a) \\
    &= \log m(\mathcal{D}_{U}) + \mathbb{E}_{v\sim p_\mathrm{aug}(v)}[-\log p_\mathrm{aug}(v)],
\end{aligned}
\end{equation}
where $m(\mathcal{D}_{U})\!=\!(2\pi)^3$ in our case.

We use the KL divergence to quantify the distributional shift between the source and the target distribution.
For any $X_t\!\sim\!p_\mathrm{tgt}(x)$, where $p_\mathrm{tgt}(x)$ represents the target domain distribution, the KL divergence between $p_\mathrm{tgt}(x)$ and $p_\mathrm{src}(x)$ (or $p_\mathrm{aug}(x)$) can be computed once the cross-entropy between them is known. 
However, directly calculating the cross-entropy between $p_\mathrm{tgt}(x)$ and $p_\mathrm{src}(x)$ (or $p_\mathrm{aug}(x)$) is intractable, and it is often treated as an optimization objective to minimize. Notably, the cross-entropy between $p_\mathrm{tgt}(x)$ and $p_\mathrm{src}(x)$ (or $p_\mathrm{aug}(x)$) shares the same upper bound, as the samples $X_s$, $X_a$, and $X_t$ all share the same dimensionality:
\begin{equation}
    \sup_{p_\mathrm{src}}{\H(p_\mathrm{tgt}, p_\mathrm{src})} = \sup_{p_\mathrm{aug}}{\H(p_\mathrm{tgt}, p_\mathrm{aug})} = \log ({m(\mathcal{D}_U) \times m(\mathcal{D}_V)}).
\end{equation}
Here, $\mathcal{D}_V$ is the measurable domain of $V_s$, $V_a$, and $V_t$.
It is straightforward to prove that $\H_\mathrm{aug}(X_a) > \H_\mathrm{src}(X_s)$, as the mutual information is non-negative and entropy reaches its upper bound when the distribution is uniform.
Therefore, the relation between the upper bound of the KL divergence from $p_\mathrm{tgt}(x)$ to $p_\mathrm{src}(x)$ and from $p_\mathrm{tgt}(x)$ to $p_\mathrm{aug}(x)$ can be expressed as:
\begin{equation}
\begin{aligned}
    \sup_{p_\mathrm{src}}{\KL(p_\mathrm{tgt}||p_\mathrm{src})} &= \sup_{p_\mathrm{src}}{\H(p_\mathrm{tgt};p_\mathrm{src})} - \sup_{p_\mathrm{src}}{\H(X_\mathrm{s})} \\
    &> \sup_{p_\mathrm{aug}}{\H(p_\mathrm{tgt};p_\mathrm{aug})} - \sup_{p_\mathrm{aug}}{\H(X_\mathrm{a})} \\
    &= \sup_{p_\mathrm{aug}}{\KL(p_\mathrm{tgt}||p_\mathrm{aug})}. \label{eq:ieq}
\end{aligned}
\end{equation}
As revealed in Eq.~\ref{eq:ieq}, the upper bound of $\KL(p_\mathrm{tgt}||p_\mathrm{aug})$ is consistently lower than $\KL(p_\mathrm{tgt}||p_\mathrm{src})$, demonstrating the effectiveness of orientation invariance in reducing the domain shift under the disturbance of varying rotations. Consequently, the final upper bound of $\KL(p_\mathrm{tgt}||p_\mathrm{aug})$ is formally given as follows:
\begin{equation}
    \sup_{p_\mathrm{aug}}{\KL(p_\mathrm{tgt}||p_\mathrm{aug})} = \log m(\mathcal{D}_V) - \mathbb{E}_{v\sim p_\mathrm{aug}(v)}[-\log p(v)].
\end{equation}


\section{Limitation and Future Work} 
Although our method shows commendable advantages in handling cross-domain orientational shifts, it faces challenges with other complex types of domain shifts, such as heavy occlusions. This is because our framework does not offer an explicit design for tackling these domain shifts. Addressing this limitation, possibly through constructing a more powerful and versatile feature space resilient to multiple domain shifts via self-supervised pre-training, is a goal for future work.

{\small
\bibliographystyle{ieee_fullname}
\bibliography{egbib}
}

\end{document} 



\end{document}
