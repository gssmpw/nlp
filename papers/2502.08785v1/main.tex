% This is samplepaper.tex, a sample chapter demonstrating the
% LLNCS macro package for Springer Computer Science proceedings;
% Version 2.21 of 2022/01/12
%
\documentclass[runningheads]{llncs}
%

\usepackage[T1]{fontenc}
% T1 fonts will be used to generate the final print and online PDFs,
% so please use T1 fonts in your manuscript whenever possible.
% Other font encondings may result in incorrect characters.
%
\usepackage{graphicx}
% Used for displaying a sample figure. If possible, figure files should
% be included in EPS format.
%
% If you use the hyperref package, please uncomment the following two lines
% to display URLs in blue roman font according to Springer's eBook style:
\usepackage{hyperref}
\hypersetup{
    colorlinks=true,
    linkcolor=black,
    citecolor=black,
    urlcolor=blue
}
%\usepackage{color}
%\renewcommand\UrlFont{\color{blue}\rmfamily}
\urlstyle{rm}
%

\usepackage{glossaries}
\usepackage{subcaption}
% Redefine caption format for figures
\DeclareCaptionFormat{bold-figure}{\textbf{#1#2}#3\par}
\captionsetup[figure]{format=bold-figure}

% Redefine caption format for tables
\DeclareCaptionFormat{bold-table}{\textbf{#1#2}#3\par}
\captionsetup[table]{format=bold-table}
% \usepackage{algorithm}
% \usepackage{algpseudocode}
% \usepackage{amsmath}
% \usepackage{amsfonts}
% \usepackage{amssymb}
% \usepackage{mathtools}
\usepackage{nicematrix}
\usepackage[numbers,sort&compress]{natbib}

\usepackage{xcolor,colortbl}
% \usepackage[table]{xcolor}% http://ctan.org/pkg/xcolor

\begin{document}

\newacronym{rl}{RL}{Reinforcement Learning}
\newacronym{drl}{DRL}{Deep Reinforcement Learning}
\newacronym{mdp}{MDP}{Markov Decision Process}
\newacronym{ppo}{PPO}{Proximal Policy Optimization}
\newacronym{sac}{SAC}{Soft Actor-Critic}
\newacronym{epvf}{EPVF}{Explicit Policy-conditioned Value Function}
\newacronym{unf}{UNF}{Universal Neural Functional}
\usepackage{enumitem}
\usepackage{multirow}
\usepackage{tcolorbox}
\usepackage{moreverb}
\usepackage{cprotect}
\usepackage{fancyvrb}
\usepackage{framed}

\newcommand{\myname}[0]{RLTHF}
\newcommand{\yifei}[1]{\textcolor{purple}{Yifei: #1}}
\newcommand{\bbb}[1]{\noindent\textbf{#1}}


%
\title{Decision Tree Based Wrappers for Hearing Loss}
%
%\titlerunning{Abbreviated paper title}
% If the paper title is too long for the running head, you can set
% an abbreviated paper title here
%
\author{Miguel Rabuge\orcidID{0009-0008-0914-0495} \and
Nuno Lourenço\orcidID{0000-0002-2154-0642}}
%
\authorrunning{M. Rabuge and N. Lourenço}
% First names are abbreviated in the running head.
% If there are more than two authors, 'et al.' is used.
%
\institute{
CISUC/LASI – Centre for Informatics and Systems of the University of Coimbra, Department of Informatics Engineering, University of Coimbra
    \email{\{rabuge,naml\}@dei.uc.pt}\\
}
%
\maketitle              % typeset the header of the contribution
%
\begin{abstract}
% 154 palavras
Audiology entities are using \gls{ML} models to guide their screening towards people at risk. \gls{FE} focuses on optimizing data for \gls{ML} models, with evolutionary methods being effective in feature selection and construction tasks. This work aims to benchmark an evolutionary \gls{FE} wrapper, using models based on decision trees as proxies. The FEDORA framework is applied to a \gls{HL} dataset, being able to reduce data dimensionality and statistically maintain baseline performance. Compared to traditional methods, FEDORA demonstrates superior performance, with a maximum balanced accuracy of 76.2\%, using 57 features. The framework also generated an individual that achieved 72.8\% balanced accuracy using a single feature.

\keywords{Feature Engineering \and Grammatical Evolution \and Audiology}
\end{abstract}
%
%
%
\section{Introduction}

Deep Reinforcement Learning (DRL) has emerged as a transformative paradigm for solving complex sequential decision-making problems. By enabling autonomous agents to interact with an environment, receive feedback in the form of rewards, and iteratively refine their policies, DRL has demonstrated remarkable success across a diverse range of domains including games (\eg Atari~\citep{mnih2013playing,kaiser2020model}, Go~\citep{silver2018general,silver2017mastering}, and StarCraft II~\citep{vinyals2019grandmaster,vinyals2017starcraft}), robotics~\citep{kalashnikov2018scalable}, communication networks~\citep{feriani2021single}, and finance~\citep{liu2024dynamic}. These successes underscore DRL's capability to surpass traditional rule-based systems, particularly in high-dimensional and dynamically evolving environments.

Despite these advances, a fundamental challenge remains: DRL agents typically rely on deep neural networks, which operate as black-box models, obscuring the rationale behind their decision-making processes. This opacity poses significant barriers to adoption in safety-critical and high-stakes applications, where interpretability is crucial for trust, compliance, and debugging. The lack of transparency in DRL can lead to unreliable decision-making, rendering it unsuitable for domains where explainability is a prerequisite, such as healthcare, autonomous driving, and financial risk assessment.

To address these concerns, the field of Explainable Deep Reinforcement Learning (XRL) has emerged, aiming to develop techniques that enhance the interpretability of DRL policies. XRL seeks to provide insights into an agent’s decision-making process, enabling researchers, practitioners, and end-users to understand, validate, and refine learned policies. By facilitating greater transparency, XRL contributes to the development of safer, more robust, and ethically aligned AI systems.

Furthermore, the increasing integration of Reinforcement Learning (RL) with Large Language Models (LLMs) has placed RL at the forefront of natural language processing (NLP) advancements. Methods such as Reinforcement Learning from Human Feedback (RLHF)~\citep{bai2022training,ouyang2022training} have become essential for aligning LLM outputs with human preferences and ethical guidelines. By treating language generation as a sequential decision-making process, RL-based fine-tuning enables LLMs to optimize for attributes such as factual accuracy, coherence, and user satisfaction, surpassing conventional supervised learning techniques. However, the application of RL in LLM alignment further amplifies the explainability challenge, as the complex interactions between RL updates and neural representations remain poorly understood.

This survey provides a systematic review of explainability methods in DRL, with a particular focus on their integration with LLMs and human-in-the-loop systems. We first introduce fundamental RL concepts and highlight key advances in DRL. We then categorize and analyze existing explanation techniques, encompassing feature-level, state-level, dataset-level, and model-level approaches. Additionally, we discuss methods for evaluating XRL techniques, considering both qualitative and quantitative assessment criteria. Finally, we explore real-world applications of XRL, including policy refinement, adversarial attack mitigation, and emerging challenges in ensuring interpretability in modern AI systems. Through this survey, we aim to provide a comprehensive perspective on the current state of XRL and outline future research directions to advance the development of interpretable and trustworthy DRL models.
\section{Background}
\label{sec:background}

\noindent
In this section, we first overview the principles governing transformer architecture. Next, we present a concise overview of DP-SFGs, which we employ to map OTA circuits into transformer-friendly sequential data. Finally, we describe a precomputed LUT-based width estimator to translate DP-SFG parameters to transistor widths.
\vspace{-1mm}
\subsection{The transformer architecture}

\noindent
The transformer~\cite{vaswani_17} is viewed as one of the most promising deep learning architectures for sequential data prediction in NLP.  It relies on an attention mechanism that reveals interdependencies among sequence elements, even in long sequences. The architecture takes a series of inputs \((x_1, x_2, x_3, \ldots, x_n\)) and generates corresponding outputs \((y_1, y_2, y_3, \ldots, y_n\)).

\begin{figure}[b]
\vspace{-5mm}
\centering
\includegraphics[width=0.5\textwidth, bb=0 0 370 190]{fig/TransformermODEL.pdf}
\vspace{-5mm}
\caption{Architecture of a transformer.}
\label{fig:simpleTrans}
% \vspace{-2mm}
\end{figure}

The simplified architecture shown in Fig.~\ref{fig:simpleTrans} consists of $N$ identical stacked encoder blocks, followed by $N$ identical stacked decoder blocks. The encoder and decoder is fed by an input embedding block, which converts a discrete input sequence to a continuous representation for neural processing. Additionally, a positional encoding block encodes the relative or absolute positional details of each element in the sequence using sine-cosine encoding functions at different frequencies. This allows the model to comprehend the position of each element in the sequence, thus understanding its context. Each encoder block comprises a multi-head self-attention block and a position-wise feed-forward network (FFN); each decoder block, which has a similar structure to the encoder, consists of an additional multi-head cross-attention block, stacked between the multi-head self-attention and feed-forward blocks. The attention block tracks the correlation between elements in the sequence and builds a contextual representation of interdependencies using a scaled dot-product between the query ($Q$), key ($K$), and value ($V$) vectors:
\begin{equation}
\text{{Attention}}(Q, K, V) = \text{softmax}\left(\frac{QK^T}{\sqrt{d_k}}\right)V,
\end{equation}
where $d_k$ is the dimension of the query and key vectors. The FFN consists of two fully connected networks with an activation function and dropout after each network to avoid overfitting. The model features residual connections across the attention blocks and FFN to mitigate vanishing gradients and facilitate information flow.

\subsection{Driving-point signal flow graphs}

\noindent
The input data sequence to the transformer must encode information that relates the parameters of a circuit to its performance metrics.  Our method for representing circuit performance is based on the signal flow graph (SFG).  The classical SFG proposed by Mason~\cite{Mason53} provides a graph representation of linear time-invariant (LTI) systems, and maps on well to the analysis of linear analog circuits such as amplifiers. In our work, we employ the driving-point signal flow graph (DP-SFG)~\cite{ochoa_98,schmid_18}. The vertices of this graph are the set of excitations (voltage and current sources) in the circuit and internal states (e.g., voltages) in the circuit.  
% An edge is drawn between vertices that have an electrical relationship, and the weight on each edge is the gain of the edge;
An edge connects vertices with an electrical relationship, and the edge weight is the gain; 
for example, if a vertex $z$ has two incoming edges from vertices $x$ and $y$, with gains $a$ and $b$, respectively, then $z = ax + by$, using the principle of superposition in LTI systems.  To effectively use superposition to assess the impact of each node on every other node, the DP-SFG introduces auxiliary voltages at internal nodes of the circuit that are not connected to excitations. These auxiliary sources are structured to not to alter any currents or voltages in the original circuit, and simplifies the SFG formulation for circuit analysis.
% enable easy formulation of the SFG to analyze circuit behavior. 

\begin{figure}[t]
% \vspace{-6mm}
\centering
\includegraphics[width=0.9\linewidth, bb=0 0 320 140]{fig/DPSFG.pdf}
\vspace{-0.25cm}
\caption{~(a) Schematic and (b) DP-SFG for an active inductor.}
\label{fig:DP-SFG_ex}
\vspace{-5mm}
\end{figure}

Fig.~\ref{fig:DP-SFG_ex}(a) shows a circuit of an active inductor, which is an inductor-less circuit that replicates the behavior of an inductor over a certain range of frequencies. Fig.~\ref{fig:DP-SFG_ex}(b) shows the equivalent DP-SFG. In Section~\ref{sec:dp-sfg}, we provide a detailed explanation that shows how a circuit may be mapped to its equivalent DP-SFG. 


\ignore{
\subsection{Lookup table for MOSFET sizing}
\label{sec:LUT}

\noindent
As seen in Fig.~\ref{fig:DP-SFG_ex}, the edge weights in a DP-SFG include circuit parameters such as the transistor transconductance, $g_m$, and various capacitances in the circuit.  The circuit may be optimized to find values of these parameters that meet specifications, but ultimately these must be translated into physical transistor parameters such as the transistor width.   In older technologies, the square-law model for MOS transistors could be used to perform a translation between DP-SFG parameters and transistor widths, but square-law behavior is inadequate for capturing the complexities of modern MOS transistor models.
In this work, we use a precomputed lookup table (LUT) that rapidly performs the mapping to device sizes while incorporating the complexities of advanced MOS models.

\begin{figure}[htbp]
\vspace{-0.4cm}
\centering
\includegraphics[height=4cm]{fig/lut_fig_1.pdf}
\vspace{-0.55cm}
\caption{LUT generation using three DOFs, $V_{gs}$, $V_{ds}$ and $L$.}
\label{fig:lutgen}
\vspace{-0.1cm}
\end{figure}

The LUT is indexed by the $V_{gs}$, $V_{ds}$, and length $L$ of the transistor, and provides four outputs: the drain current ($I_d$), transconductance ($g_m$), source-drain conductance ($g_{ds}$), and drain-source capacitance ($C_{ds}$).
The entries of the LUT are computed by performing a nested DC sweep simulation across the three input indices for the MOSFET with a specific reference width, $W_{ref}$, as shown in Fig.~\ref{fig:lutgen}, and for each input combination, the four outputs are recorded.
\blueHL{Empirically, we see that the impact of $V_{sb}$ is small enough that it can be neglected, and therefore we set $V_{sb} = 0$ in the sweeps used to create the LUT.}

Our methodology uses this LUT, together with the $g_m/I_d$ methodology~\cite{silviera_96}, to translate circuit parameters predicted by the transformer to transistor widths. The cornerstone of this methodology relies on the inherent width independence of the ratio $g_m/I_d$ to estimate the unknown device width: this makes it feasible to use an LUT characterized for a reference width $W_{ref}$. 
We will elaborate on this procedure further in Section~\ref{sec:precomputedLUTs}, and show how the LUT, together with the $g_m/I_d$ method, can effectively estimate the device widths corresponding to the transformer outputs.
% \redHL{\sout{required to achieve equivalent DC operating characteristics within the circuit. Section III D \redHL{Do not hardcode section numbers!!} provides an in-depth explanation of the implementation details of this methodology.}}
}
\section{Approach}

In this work, FEDORA \cite{rabuge2024comparison} will be applied to a \gls{HL} detection problem using three distinct classifiers based on decision trees, namely basic \gls{DT} models and their bagging and boosting counterparts, \gls{RF} and \gls{XGB}, as proxy models in the evolutionary framework. Figure \ref{fig:fedora} illustrates the inner workings of the framework.

\begin{figure}
    \centering
    \includegraphics[scale=0.5]{figures/fedora-general.pdf}
    \caption{FEDORA: Feature Engineering through Discovery of Reliable Attributes}
    \label{fig:fedora}
\end{figure}

The framework starts by splitting the original dataset into training (40\%), validation (40\%) and test subsets (20\%). The training and validation subsets are given to the evolutionary process, where \gls{SGE} will generate individuals that select and construct a new dataset from the original one, through a context-free grammar. These transformations will be applied to the training and validation subsets, which will then be used to train the proxy model and validate the transformation, respectively. The fitness is given by the validation error, namely (1 - Balanced Accuracy). After the specified generations of the evolutionary process, the individual with the lowest validation error is returned. This individual is then applied to the three subsets and its ability to generalize to unseen data is evaluated. This assessment involves training a range of Machine Learning (ML) models using both the training and validation subsets and subsequently evaluating their performance on the test set.
\section{Experimental Setup}

% Problem and Dataset
This study addresses detecting \gls{HL} with contextual attributes through binary classification. The dataset generation process is fully defined in \cite{miranda2022hytea}. The dataset has 60 features and cannot be publicly published due to sensitive patient screening information.

% Experimental Setup
Regarding the experimental settings, Table \ref{table:setup} summarizes the parameters of the framework for each one of the three experiments. Most settings are alike, only diverging in the proxy model. All the models used the default package parameters, except for the \gls{RF} where the n\_estimators and max\_depth parameters were defined to 5. The grammar used in the experiments enables the selection and construction of algebraic-type features and is available here \footnote{\href{https://github.com/miguelrabuge/fedora/blob/main/examples/audiology/audiology.pybnf}{ github.com/miguelrabuge/fedora/blob/main/examples/audiology/audiology.pybnf}}.


\begin{table}
\centering
\caption{Experimental Settings}
\label{table:setup}
\begin{NiceTabular}{|l|wc{2cm}|wc{2cm}|wc{2cm}|}
\hline
\multicolumn{1}{|c}{\textbf{Parameters}} & \multicolumn{3}{c|}{\textbf{Experiments}} \\
\hline
Proxy Model & DT & RF & XGB \\
\hline
Population & \multicolumn{3}{c|}{200} \\
\hline
Generations & \multicolumn{3}{c|}{100} \\
\hline
Runs & \multicolumn{3}{c|}{30} \\
\hline
Elitism & \multicolumn{3}{c|}{10\%} \\
\hline
Crossover Rate & \multicolumn{3}{c|}{0.9} \\
\hline
Mutation Rate & \multicolumn{3}{c|}{0.1} \\
\hline
Minimum Tree Depth & \multicolumn{3}{c|}{3} \\
\hline
Maximum Tree Depth & \multicolumn{3}{c|}{10} \\
\hline
Selection & \multicolumn{3}{c|}{Tournament (size 3)} \\
\hline
Fitness & \multicolumn{3}{c|}{1 - Balanced Accuracy} \\
\hline
\end{NiceTabular}
\end{table}

Four types of models were selected as testing models: \gls{DT}, \gls{RF}, \gls{XGB} and \gls{MLP}. These models will assess the generalization performance of the FEDORA individuals, comparing its balanced accuracy scores with the baseline and other \gls{FE} methods, such as \gls{PCA}, \gls{UMAP}, \glspl{SOM} and \glspl{AE}.

Each \gls{FE} technique will use the same number of features as the FEDORA individual. For instance, if the FEDORA individual has 15 features, both the number of \gls{PCA} and \gls{UMAP} components would be equal to 15, the 2D SOM grid would have dimensions of 15x1, and the code size of the AE would be set to 15. The \gls{AE} parameters consist of 50 neurons for the single hidden layers, with linear activation functions, and using mean squared error as the error metric. Its training involves using a batch size of 32, running for 50 epochs, using Stochastic Gradient Descent.
\section{Results and Discussion}
\label{ref:results}
We present the main results of our approach compared to state-of-the-art methods across the six time series tasks on nine datasets.
Overall, our method has demonstrated a substantial performance improvement, reaching up to 10--15\% in some tasks, while increasing the shift consistency up to 50--60\% compared to previous techniques.
\begin{table*}[h]
\caption{Performance comparison of our method and other techniques for HR estimation}
\begin{adjustbox}{width=1\columnwidth,center}
\label{tab:performance_ppg}
\renewcommand{\arraystretch}{0.8}
\begin{tabular}{@{}lllllllllll@{}}
\toprule
\multirow{2}{*}{Method} & \multicolumn{4}{l}{IEEE SPC22} & \multicolumn{4}{l}{DaLiA}  \\ 
\cmidrule(r{15pt}){2-5}  \cmidrule(r{15pt}){6-10}  
& S-Cons (\%) $\uparrow$ & RMSE $\downarrow$ & MAE $\downarrow$ & $\rho$ (\%) $\uparrow$ & S-Cons (\%) $\uparrow$ & RMSE $\downarrow$ & MAE $\downarrow$ & $\rho$ (\%) $\uparrow$ \\
\midrule
Baseline & 61.99\small$\pm$1.19 & 18.39\small$\pm$2.96 & 10.28\small$\pm$1.41 & 62.64\small$\pm$5.74 & 32.08\small$\pm$0.22 & 9.86\small$\pm$0.23 & 4.40\small$\pm$0.03 & 86.01\small$\pm$0.51 \\
Aug. & 76.48\small$\pm$1.77 & 18.73\small$\pm$1.15 & 10.42\small$\pm$0.40 & 64.06\small$\pm$3.70 & 52.77\small$\pm$0.39 & 9.85\small$\pm$0.21 & 4.47\small$\pm$0.06 & 85.99\small$\pm$0.49 & \\
LPF & 76.88\small$\pm$0.73 & 20.20\small$\pm$1.54  & 13.44\small$\pm$0.82 & 65.40\small$\pm$1.92 & 38.67\small$\pm$0.30 & 10.01\small$\pm$0.30 & 4.67\small$\pm$0.12 & 85.68\small$\pm$0.51 & \\
APS & 73.99\small$\pm$1.06 & 19.42\small$\pm$0.60 & 12.98\small$\pm$0.29 & 65.27\small$\pm$1.32 & 44.33\small$\pm$0.16 & 10.45\small$\pm$0.40 & 5.01\small$\pm$0.17 & 84.69\small$\pm$0.85 & \\

WaveletNet & 51.71\small$\pm$1.95 & 21.56\small$\pm$1.01 & 14.61\small$\pm$0.34 & 60.74\small$\pm$4.37 & 36.71\small$\pm$3.04 & 15.46\small$\pm$0.64 & 7.67\small$\pm$0.23 & 76.13\small$\pm$1.86 & \\

Canonicalize & 63.52\small$\pm$1.20 & 19.02\small$\pm$0.62 & 10.40\small$\pm$0.69 & 61.27\small$\pm$1.07 & 32.01\small$\pm$0.33 & 9.77\small$\pm$0.12 & 4.39\small$\pm$0.05 & 86.02\small$\pm$0.30 & \\
\midrule

Ours & \textbf{100\small$\pm$0.00} & \textbf{16.25\small$\pm$0.72} & \textbf{9.45\small$\pm$0.03} & \textbf{70.12\small$\pm$2.10}& \textbf{100\small$\pm$0.00} & \textbf{9.75\small$\pm$0.15} & \textbf{4.39\small$\pm$0.03} & \textbf{86.06\small$\pm$0.19} \\
Ours\small+\scriptsize LPF & 100\small$\pm$0.00 & 20.34\small$\pm$1.62 & 13.77\small$\pm$0.84 & 65.60\small$\pm$2.31 & 100\small$\pm$0.00 & 10.72\small$\pm$0.11 & 5.30\small$\pm$0.03  & 84.12\small$\pm$0.23 \\
Ours\small+\scriptsize APS & 100\small$\pm$0.00 & 18.81\small$\pm$1.59 & 12.32\small$\pm$0.84 & 67.01\small$\pm$3.79 & 100\small$\pm$0.00 & 10.47\small$\pm$0.09 & 5.10\small$\pm$0.03 & 84.62\small$\pm$0.31 & \\
\bottomrule
\end{tabular}
\end{adjustbox}
\end{table*}

\begin{table*}[t]
\caption{Performance comparison of ours and other techniques in \textit{ECG} datasets for CVD classification}
\begin{adjustbox}{width=1\columnwidth,center}
\label{tab:performance_ecg}
\renewcommand{\arraystretch}{0.8}
\begin{tabular}{@{}lllllllllll@{}}
\toprule
\multirow{2}{*}{Method} & \multicolumn{4}{l}{Chapman} & \multicolumn{4}{l}{PhysioNet 2017}  \\ 
\cmidrule(r{15pt}){2-5}  \cmidrule(r{15pt}){6-10}  
& S-Cons (\%) $\uparrow$ & Acc $\uparrow$ & F1 $\uparrow$ & AUC (\%)$\uparrow$ & S-Cons (\%) $\uparrow$ & Acc $\uparrow$ & F1 $\uparrow$ & AUC $\uparrow$ \\
\midrule
Baseline & 98.53\small$\pm$0.17 & 91.32\small$\pm$0.23 & 91.22\small$\pm$0.24 & 98.34\small$\pm$0.16 & 98.37\small$\pm$0.15 & 83.22\small$\pm$0.72 & 73.50\small$\pm$1.99 & 93.21\small$\pm$0.30 \\
Aug. & 99.00\small$\pm$0.16 & 91.96\small$\pm$0.19 & 91.89\small$\pm$0.22 & 98.45\small$\pm$0.18 & 98.96\small$\pm$0.17 & 82.28\small$\pm$1.18 & 72.32\small$\pm$2.20 & 93.20\small$\pm$0.42  \\
LPF & 98.69\small$\pm$0.14 & 92.01\small$\pm$0.23  & 91.94\small$\pm$0.58 & 98.50\small$\pm$0.24 & 98.94\small$\pm$0.39 & 84.40\small$\pm$0.16 & 75.68\small$\pm$0.76 & 93.80\small$\pm$0.32 & \\
APS & 98.60\small$\pm$0.17 & 90.69\small$\pm$0.89 & 89.44\small$\pm$1.00 & 98.31\small$\pm$0.24 & --- & --- & --- & --- \\

WaveletNet & 91.02\small$\pm$1.14 & 90.87\small$\pm$1.02 & 90.02\small$\pm$1.00 & 97.94\small$\pm$0.21 & 65.03\small$\pm$0.71 & 76.06\small$\pm$0.64 & 63.35\small$\pm$3.40 & 87.02\small$\pm$0.29\\

Canonicalize & 98.80\small$\pm$0.24 & 91.93\small$\pm$0.13 & 90.87\small$\pm$0.18 & 98.42\small$\pm$0.15 & 98.26\small$\pm$0.31 & 83.34\small$\pm$0.46 & 73.97\small$\pm$0.67 & 93.68\small$\pm$0.31\\

\midrule
Ours & \textbf{100\small$\pm$0.00} & \textbf{92.10\small$\pm$0.25} & 91.93\small$\pm$0.85 & 98.47\small$\pm$0.15 & \textbf{100\small$\pm$0.00} & 83.15\small$\pm$0.65 & 74.12\small$\pm$1.80 & 93.28\small$\pm$0.31 \\
Ours\small+\small LPF & 100\small$\pm$0.00 & 92.05\small$\pm$0.52 & \textbf{91.96\small$\pm$0.54} & \textbf{98.51\small$\pm$0.10} & 100\small$\pm$0.00 & \textbf{85.20\small$\pm$0.40} & \textbf{77.50\small$\pm$1.21} & \textbf{94.20\small$\pm$0.19} \\
Ours\small+\small APS & 100\small$\pm$0.00 & 91.61\small$\pm$1.11 & 91.10\small$\pm$0.56 & 98.36\small$\pm$0.20 & --- & --- & --- & --- \\
\bottomrule
\end{tabular}
\end{adjustbox}
\end{table*}

The experimental results from all the time series tasks are given in Tables~\ref{tab:performance_ppg},~\ref{tab:performance_ecg},~\ref{tab:performance_eeg} and~\ref{tab:performance_imu}.
These tables demonstrate that the previous techniques fail to provide shift-invariant models when applied to time series without limiting shifts.
Additionally, the models exhibit extremely low consistency (as low as 32\%) in HR prediction. 
More importantly, applying state-of-the-art methods to enhance shift consistency in deep learning models for predicting the heart rate results in performance degradation.


We believe the main reason for the small improvements in the consistency of previous techniques is that the research to date has tended to focus on limited shifts rather than considering the whole shift space as literature is mostly concerned about images.
While restricting shifts can be a valid assumption in computer vision, where the main reasoning is that the object being classified should not be near the boundary.
This assumption does not apply to time series, where the whole signal carries the information~\citep{demirel2023chaos} additional to local waveform features, and as such, there is no explicit boundary condition or input area to consider for limiting the range of shifts.
% \begin{wraptable}[10]{r}{0.55\textwidth}
% \vspace{-4mm}
% \caption{\label{tab:eeg_main}Performance comparison of ours with other techniques in \textit{EEG} for sleep stage classification}
% \begin{adjustbox}{width=0.55\textwidth}
% \label{tab:appendix_sleep}
% \renewcommand{\arraystretch}{0.9}
% \begin{tabular}{@{}lllllll@{}}
% \toprule
% \multirow{2}{*}{Method} & \multicolumn{4}{l}{Sleep-EDF}   \\ 
% \cmidrule(r{15pt}){2-6}  
% & S-Cons $\uparrow$ & Acc $\uparrow$ &  W-F1 $\uparrow$ & $\kappa$ $\uparrow$ \\
% \midrule
% Baseline & 95.06\small$\pm$0.61 & 75.41\small$\pm$2.01 &  74.87\small$\pm$1.92 & 67.12\small$\pm$2.96  \\
% % Baseline (2$\times$) & 91.09\small$\pm$1.26 & 73.88\small$\pm$2.10 &  74.32\small$\pm$2.86 & 65.14\small$\pm$2.94  \\
% Aug. & 99.00\small$\pm$0.17 & 74.89\small$\pm$1.11 & 74.03\small$\pm$1.46 & 65.89\small$\pm$1.81  \\
% LPF & 92.43\small$\pm$1.24 & 73.56\small$\pm$2.93  & 76.01\small$\pm$1.98 & 65.68\small$\pm$3.46  \\
% APS & --- & --- & --- & ---  \\

% WaveletNet & 84.40\small$\pm$5.90 & 73.54\small$\pm$4.78 & 72.74\small$\pm$3.45 & 64.66\small$\pm$4.12  \\

% Canonicalize & 93.95\small$\pm$0.51 & 77.12\small$\pm$2.21 & 70.14\small$\pm$2.25 & 69.81\small$\pm$2.76  \\

% \midrule
% Ours & \textbf{100\small$\pm$0.00} & \textbf{77.90\small$\pm$1.92}  & \textbf{76.77\small$\pm$2.58} & \textbf{70.01\small$\pm$1.10} \\
% Ours\small+\small LPF & 100\small$\pm$0.00 & 73.12\small$\pm$1.89 & 75.34\small$\pm$1.61 & 64.98\small$\pm$2.27 & \\
% \bottomrule
% \end{tabular}
% \end{adjustbox}
% \end{wraptable}
\begin{table*}[b]
\caption{Performance comparison of our method with other techniques on an \textit{EEG} dataset for sleep stage classification and an \textit{audio} dataset for lung sound classification in respiratory health assessment}
\begin{adjustbox}{width=1\columnwidth,center}
\label{tab:performance_eeg}
\renewcommand{\arraystretch}{0.8}
\begin{tabular}{@{}lllllllllll@{}}
\toprule
\multirow{2}{*}{Method} & \multicolumn{4}{l}{Sleep-EDF} & \multicolumn{4}{l}{Respiratory}  \\ 
\cmidrule(r{15pt}){2-5}  \cmidrule(r{15pt}){6-9}  
& S-Cons (\%) $\uparrow$ & Acc $\uparrow$ & W-F1 $\uparrow$ & $\kappa$ $\uparrow$ & S-Cons (\%) $\uparrow$ & Acc $\uparrow$ & F1 $\uparrow$ & W-F1 $\uparrow$  \\
\midrule
Baseline & 95.06\small$\pm$0.61 & 75.41\small$\pm$2.01 & 74.87\small$\pm$1.92 & 67.12\small$\pm$2.96 &  99.10\small$\pm$0.43 & 25.21\small$\pm$5.60 & 57.01\small$\pm$3.62 & 21.21\small$\pm$5.98 \\
Aug. & 99.00\small$\pm$0.17 & 74.89\small$\pm$1.11 & 74.03\small$\pm$1.46 & 65.89\small$\pm$1.81 & 99.68\small$\pm$0.42 & 20.32\small$\pm$5.18 & 45.81\small$\pm$3.51 & 15.31\small$\pm$6.07  \\
LPF & 92.43\small$\pm$1.24 & 73.56\small$\pm$2.93  & 76.01\small$\pm$1.98 & 65.68\small$\pm$3.46 & 99.50\small$\pm$0.42 & 19.47\small$\pm$9.78  & 46.53\small$\pm$3.04 & 11.89\small$\pm$4.98  \\

WaveletNet & 84.40\small$\pm$5.90 & 73.54\small$\pm$4.78 & 72.74\small$\pm$3.45 & 64.66\small$\pm$4.12 & 91.38\small$\pm$2.40 & 28.57\small$\pm$10.81 & 44.23\small$\pm$7.12 & 17.10\small$\pm$7.81 \\

Canonicalize & 93.95\small$\pm$0.51 & 77.12\small$\pm$2.21 & 70.14\small$\pm$2.25 & 69.81\small$\pm$2.76 & 98.28\small$\pm$0.64 & 22.68\small$\pm$10.52 & 45.33\small$\pm$5.75 & 15.30\small$\pm$5.33 \\

\midrule
Ours & \textbf{100\small$\pm$0.00} & \textbf{77.90\small$\pm$1.92}  & \textbf{76.77\small$\pm$2.58} & \textbf{70.01\small$\pm$1.10} & \textbf{100\small$\pm$0.00} & \textbf{33.10\small$\pm$5.12} & \textbf{60.13\small$\pm$4.67} & \textbf{28.33\small$\pm$6.55} \\
Ours\small+\small LPF & 100\small$\pm$0.00 & 73.12\small$\pm$1.89 & 75.34\small$\pm$1.61 & 64.98\small$\pm$2.27 & 100\small$\pm$0.00 & 25.77\small$\pm$2.12 & 51.82\small$\pm$2.10 & 17.99\small$\pm$4.15 \\
\bottomrule
\end{tabular}
\end{adjustbox}
\end{table*}

The empirical results support our motivation for proposing a differentiable bijective function that maps samples with different shifts to the same point on the data manifold, avoiding the limited shift assumption.
Additionally, applying low-pass filtering to prevent aliasing can degrade performance for certain tasks, where the interaction between frequencies plays a critical role~\citep{Science_EEG}.
\begin{table*}[t]
\centering
\caption{Performance comparison of our method with others in \textit{IMU} datasets for Activity and Step}
\begin{adjustbox}{width=1\columnwidth,center}
\label{tab:performance_imu}
\renewcommand{\arraystretch}{0.7}
\begin{tabular}{@{}lllllllllll@{}}
\toprule
\multirow{2}{*}{Method} & \multicolumn{3}{l}{UCIHAR} & \multicolumn{3}{l}{HHAR} & \multicolumn{3}{l}{Clemson} \\ 
\cmidrule(r{15pt}){2-4}  \cmidrule(r{15pt}){5-7}  \cmidrule(r{15pt}){8-10} \\ 
& S-Cons (\%) $\uparrow$ & Acc $\uparrow$ & F1 $\uparrow$ & S-Cons (\%) $\uparrow$ & Acc $\uparrow$ & F1 $\uparrow$ 
& S-Cons (\%) $\uparrow$ & MAPE $\downarrow$ & MAE $\downarrow$ \\
\midrule
Baseline & 94.07\small$\pm$1.38 & 85.39\small$\pm$2.30 & 83.20\small$\pm$2.94 & 98.27\small$\pm$0.33 &  91.87\small$\pm$1.36 & 91.16\small$\pm$1.38 & 54.31\small$\pm$4.40 & 4.76\small$\pm$0.11 & 2.74\small$\pm$0.08 \\
Aug. & 96.55\small$\pm$0.80 & 85.42\small$\pm$4.50 & 83.69\small$\pm$6.74 & 98.38\small$\pm$0.28 & 91.97\small$\pm$0.44 &91.31\small$\pm$0.49& 61.01\small$\pm$4.88 & 4.08\small$\pm$0.14 & 2.29\small$\pm$0.07  \\
LPF & 95.05\small$\pm$0.21 & 83.96\small$\pm$3.44 & 81.08\small$\pm$4.21 & 98.10\small$\pm$0.10 & 92.10\small$\pm$0.80 &91.43\small$\pm$0.94& 59.77\small$\pm$4.40 & 4.16\small$\pm$0.16 & 2.35\small$\pm$0.11  \\
APS & 96.40\small$\pm$0.03 & 81.75\small$\pm$4.11 & 79.01\small$\pm$5.33 & 98.30\small$\pm$0.24 & 91.83\small$\pm$1.35 &91.01\small$\pm$1.47 & 45.50\small$\pm$2.69 & 4.74\small$\pm$0.16 & 2.69\small$\pm$0.07 \\

WaveletNet & 94.56\small$\pm$1.31 & 82.78\small$\pm$4.62 & 80.73\small$\pm$5.59 & 96.76\small$\pm$0.15 & 90.72\small$\pm$0.38 & 90.71\small$\pm$0.39 & 59.14\small$\pm$3.10 & 5.20\small$\pm$0.66 & 2.95\small$\pm$0.41 \\


Canonicalize & 97.72\small$\pm$0.37 & 84.10\small$\pm$2.10 & 81.89\small$\pm$2.89 & 98.27\small$\pm$0.07 & 91.56\small$\pm$1.18 & 90.73\small$\pm$1.10 & 55.47\small$\pm$4.87 & 4.54\small$\pm$0.46 & 2.59\small$\pm$0.29 \\

\midrule
Ours & \textbf{100\small$\pm$0.00} & \textbf{87.71\small$\pm$1.98} & \textbf{85.67\small$\pm$2.47} & \textbf{100\small$\pm$0.00} & 91.93\small$\pm$1.14 & 91.12\small$\pm$1.03 & \textbf{100\small$\pm$0.00} & 4.28\small$\pm$0.34 & 2.43\small$\pm$0.21 \\
Ours\small+\small LPF & 100\small$\pm$0.00 & 84.78\small$\pm$2.46 & 82.58\small$\pm$2.62 & 100\small$\pm$0.00 & \textbf{92.51\small$\pm$0.55} &\textbf{91.80\small$\pm$0.62} & 100\small$\pm$0.00 & \textbf{3.75\small$\pm$0.33} & \textbf{2.12\small$\pm$0.18}  \\
Ours\small+\small APS & 100\small$\pm$0.00 & 82.96\small$\pm$1.79 & 81.10\small$\pm$1.73 & 100\small$\pm$0.00 & 91.38\small$\pm$0.32 & 90.64\small$\pm$0.32 & 100\small$\pm$0.00 & 3.87\small$\pm$0.19 & 2.19\small$\pm$0.11  \\
\bottomrule
\end{tabular}
\end{adjustbox}
\end{table*}
\paragraph{Time delay as adversary?}
% Another interesting outcome from the results is the significant consistency decrease of models as the number of output classes increases.
An interesting finding from our experiments is the notable decline in model consistency as the number of output classes increases.
This behavior in the models is similar to previous findings on adversarial examples, indicating that the robustness decreases with a higher number of classes~\citep{adversary_output}.
During our experiments, we observed the same phenomenon where the small shifts of the input change the output to another class, particularly when the task complexity increased with a higher number of classes.
For example, in the case of HR estimation (Table~\ref{tab:performance_ppg}), even short shifts (as low as 10--100\,ms) can lead to a change in the prediction by over 80\,bpm, despite no alteration in the periodicity of the signal, which is the main feature for this task.


Normally, it is expected that models learn the periodicities in these signals and infer the heart rate. 
However, our results indicate that the models learn something else or in a different way, because as the signal undergoes a slight shift, the model prediction jumps more than 100\%, even though the periodicity of the waveform remains unchanged with the shift operation.


We believe these drastic output changes arise from the model's sensitivity to (shortcut) features~\citep{geirhos_shortcut_2020, zhang_21}, resulting in a performance decrease when evaluated on samples different from those encountered during training.
Since our proposed transformation function works as an adaptive linear constraint in the data space, it reduces the potential points where samples can exist, thereby enhancing overall performance.


One distinct result from our experiments is that when previous shift-invariancy techniques are applied to the heart rate prediction task, the average error rate of the models increases by 7--10\%.
This performance decrease can be easily observed in the DaLiA (Table~\ref{tab:performance_ppg}) for the adaptive sampling technique.
The performance discrepancy between tasks can be attributed to the dataset and signal characteristics.
Since DaLiA contains impulse random noise with multiple periodicities, the norm-based subsampling can inadvertently emphasize the noisy waveforms instead of the desired pattern during the subsampling of feature maps, leading to a decrease in prediction performance. 
% Overall, results from our experiments imply that the effectiveness of the methods should be evaluated across diverse time series tasks with several datasets to understand and evaluate their true generalization, consistency, and performance.


We conduct detailed ablation experiments to further investigate the impact of various components, with a particular focus on the effect of the proposed mapping function under different modifications, i.e., modified loss for optimization, on the overall model's performance across time series tasks.



% -------------------------------------------------- %

\subsection{Ablation Study}
\label{sec:ablation}
We present a comprehensive investigation of our method and the effect of its components on the performance.
Mainly, we investigate the effect of guiding the proposed transformation with different loss functions and without any guidance.
First, we map all samples to a single manifold $\mathcal{M}^{\phi_0}$, i.e., $\mathcal{T}(\mathbf{x}, \phi)$ is applied with a constant $\phi = 0$ instead of learning the angle for each sample.
We experimented with different values of $\phi \sim (-\pi, \pi]$, but observed no significant change in the performance when the mapped manifold is constant for samples.
Second, we modify the loss for training the guidance network to increase the variance of angles---increasing the possible manifolds where data can be found---without changing the cross-entropy loss from the classification network as in Equation~\ref{eq:ablation_loss}, $(\hat{\mathcal{L}}_G)$.
Finally, we train both networks only with the cross-entropy loss $(\mathcal{L}^{\prime}_G = \mathcal{L}_C)$.
\begin{table}[b]
\centering
\caption{\label{tab:performance_hr_ablation} Ablation experiments for \textit{HR} (left) and \textit{IMU} (right) tasks}
\vspace{-2mm}
 \begin{subtable}[t]{0.47\linewidth}
    \centering
  \begin{adjustbox}{width=\columnwidth,center}
\begin{tabular}{@{}lllllll@{}}
\toprule
\multirow{2}{*}{Method} & \multicolumn{3}{l}{IEEE SPC22} & \multicolumn{3}{l}{DaLiA$_{PPG}$} \\ 
\cmidrule(r{15pt}){2-4}  \cmidrule(r{15pt}){5-7} \\ 
&  MAE $\downarrow$ & RMSE $\downarrow$ & $\rho$ $\uparrow$ & MAE $\downarrow$ & RMSE $\downarrow$ & $\rho$ $\uparrow$ \\
\midrule
$\mathcal{T}(\mathbf{x}, \phi)$  & 11.15  & 19.18  & 62.07 & 4.77 & 10.13 & 85.35 \\
$\mathcal{L}^{\prime}_G$ & 9.80 & 17.16 & 66.80 & 4.60 & 10.10 & 85.52  \\
$\hat{\mathcal{L}}_G$ & 9.45 & 17.00 & 69.10 & 4.41 & \textbf{9.63} & \textbf{86.35}  \\
Ours  &\textbf{9.45} & \textbf{16.25} & \textbf{70.12} & \textbf{4.39} & 9.75 & 86.06 \\
\midrule
Change  & \textcolor{Green}{+1.70} & \textcolor{Green}{+2.97} & \textcolor{Green}{+8.05} & \textcolor{Green}{+0.38} & \textcolor{Green}{+0.38} & \textcolor{Green}{+0.71} \\
\bottomrule
\end{tabular}
\end{adjustbox}
    \end{subtable}%
    \quad \quad
 \begin{subtable}[b]{0.47\linewidth}
        \begin{adjustbox}{width=\columnwidth,center}
\begin{tabular}{@{}lllllll@{}}
\toprule
\multirow{2}{*}{Method} & \multicolumn{2}{l}{UCIHAR} & \multicolumn{2}{l}{HHAR} & \multicolumn{2}{l}{Clemson} \\ 
\cmidrule(r{15pt}){2-3} \cmidrule(r{15pt}){4-5}  \cmidrule(r{15pt}){6-7} \\ 
&  Acc $\uparrow$ & F1 $\uparrow$ & Acc $\uparrow$ & F1 $\uparrow$ & MAPE $\downarrow$ & MAE $\downarrow$ \\
\midrule
$\mathcal{T}(\mathbf{x}, \phi)$ & 84.67 & 82.65 & \textbf{92.33} & \textbf{91.56} & 4.64  & 2.67 \\
$\mathcal{L}^{\prime}_G$ & 84.30 & 82.49 & 91.98 & 91.18  & 4.42 & 2.52 \\
$\hat{\mathcal{L}}_G$ & 84.82 & 81.99 & 91.51 & 90.83  & 4.31 & 2.45 \\
Ours  &\textbf{85.81} & \textbf{83.81} & 91.83 & 91.12 & \textbf{4.28} & \textbf{2.43} \\
\midrule
Change (\%)  & \textcolor{Green}{+1.14} & \textcolor{Green}{+1.16} & \textcolor{WildStrawberry}{-0.50} & \textcolor{WildStrawberry}{-0.44} & \textcolor{Green}{+0.36} & \textcolor{Green}{+0.24}  \\
\bottomrule
\end{tabular}
        \end{adjustbox}
    \end{subtable}
\end{table}
We compared these three variants of the learning techniques with the original proposed implementation as each represents distinct approaches for manipulating the data space.
For example, when all samples are mapped to a single manifold, the variations in samples decrease significantly since there is only one possible phase angle for the chosen harmonic with period $\mathrm{T}_0$.
Additionally, the relationships among all sinusoidal components remain invariant, given that the proposed transformation is a linear function of the frequency.
Conversely, optimizing the guidance network to increase the variance of angles, thereby favoring a greater sample diversity, expands the possible variations for samples.
\begin{equation}\label{eq:ablation_loss}
    \hat{\mathcal{L}}_G = \mathcal{L}_C - \sqrt{\text{Var}_{\mathbf{x} \sim \mathcal{B}} \left( f_{\theta_G}\left(|\mathcal{F}(\mathbf{x})|\right) \right)}
\end{equation}
Tables~\ref{tab:performance_hr_ablation} and~\ref{tab:performance_sleep_ablation} summarize the results where we exclude the consistency metric from the tables as the models that include the proposed transformation are always completely shift-invariant.
The first row ($\mathcal{T}(\mathbf{x}, \phi)$) in the tables shows the performance when all the samples are mapped to a single manifold i.e., without a guidance network for learning the mapping.
The second row ($\mathcal{L}^{\prime}_G$) represents the performance when the guidance network is only optimized using the categorical cross-entropy loss.
The third row ($\hat{\mathcal{L}}_G $) presents the performance when the variance of angles is optimized to increase during training.
And, the last row (Ours) is the original implementation of the proposed method.
% \begin{wraptable}[9]{R}{7cm}
% \vspace{-4mm}
% \centering
% \caption{Ablation experiments for \textit{CVD} task}
% \begin{adjustbox}{width=0.5\columnwidth,center}
% \label{tab:performance_ecg_ablation}
% \renewcommand{\arraystretch}{0.7}
% \begin{tabular}{@{}lllllll@{}}
% \toprule
% \multirow{2}{*}{Method} & \multicolumn{3}{l}{Chapman} & \multicolumn{3}{l}{PhysioNet} \\ 
% \cmidrule(r{15pt}){2-4}  \cmidrule(r{15pt}){5-7} \\ 
% &  Acc $\uparrow$ & F1 $\uparrow$ & AUC $\uparrow$ & Acc $\uparrow$ & F1 $\uparrow$ & AUC $\uparrow$ \\
% \midrule
% $\mathcal{T}(\mathbf{x}, \phi)$  & 91.82  & 90.76 & 98.36 & 83.12 & 73.67  & 93.24 \\
% $\mathcal{L}^{\prime}_G$ & 91.27 & 90.10 & 98.38 & 82.81 & 73.75 & 93.45 \\
% $\hat{\mathcal{L}}_G$ & 91.88 & 90.84 & 98.44 & \textbf{83.30}  & 73.90 & \textbf{93.51} \\
% Ours  &\textbf{92.10} & \textbf{91.93} & \textbf{98.40} & 83.15 & \textbf{74.12} & 93.30 \\
% \midrule
% Change (\%)  & \textcolor{Green}{+0.28} & \textcolor{Green}{+1.17} & \textcolor{Green}{+0.04} & \textcolor{Green}{+0.03} & \textcolor{Green}{+0.45} & \textcolor{Green}{+0.06} \\
% \bottomrule
% \end{tabular}
% \end{adjustbox}
% \end{wraptable}
We also report the change when the mapping function is guided using the network $f_{G_{\theta}}$ and optimized using the loss defined in Equation~\ref{eq:lossess}, as opposed to being a fixed, non-learnable function.
\begin{table}[t]
\vspace{-3mm}
\centering
\caption{\label{tab:performance_sleep_ablation} Ablation experiments for \textit{EEG} (left) and \textit{ECG} (right) tasks}
\renewcommand{\arraystretch}{0.9}
 \begin{subtable}[b]{0.47\linewidth}
    \centering
  \begin{adjustbox}{width=\columnwidth,center}
\begin{tabular}{@{}lllllll@{}}
\toprule
\multirow{2}{*}{Method} & \multicolumn{3}{l}{Sleep-EDF}   \\ 
\cmidrule(r{15pt}){2-5}  
& Acc $\uparrow$ & F1 $\uparrow$ & W-F1 $\uparrow$ & $\kappa$ $\uparrow$ \\
\midrule
$\mathcal{T}(\mathbf{x}, \phi)$ & 75.54\small$\pm$2.39 & 66.96\small$\pm$1.78 & 75.53\small$\pm$2.29 & 67.08\small$\pm$0.03  \\
$\mathcal{L}^{\prime}_G$ & 77.21\small$\pm$1.51 & 67.67\small$\pm$1.67 & 76.89\small$\pm$1.71 & 69.39\small$\pm$0.02  \\
$\hat{\mathcal{L}}_G$ & 77.75\small$\pm$1.23 & \textbf{68.04}\small$\pm$1.16 & \textbf{77.01}\small$\pm$1.07 & 69.94\small$\pm$0.01  \\
Ours & \textbf{77.80}\small$\pm$1.95 & 67.01\small$\pm$2.65 & 76.77\small$\pm$2.58 & \textbf{70.01\small$\pm$1.10} \\
\midrule
Change &  \textcolor{Green}{+2.26}  &  \textcolor{Green}{+0.05}  & \textcolor{Green}{+1.24}  & \textcolor{Green}{+2.93}  \\
\bottomrule
\end{tabular}
\end{adjustbox}
    \end{subtable}%
    \quad \quad
 \begin{subtable}[b]{0.47\linewidth}
        \begin{adjustbox}{width=\columnwidth,center}
\label{tab:performance_ecg_ablation}
\renewcommand{\arraystretch}{0.7}
\begin{tabular}{@{}lllllll@{}}
\toprule
\multirow{2}{*}{Method} & \multicolumn{3}{l}{Chapman} & \multicolumn{3}{l}{PhysioNet} \\ 
\cmidrule(r{15pt}){2-4}  \cmidrule(r{15pt}){5-7} \\ 
&  Acc $\uparrow$ & F1 $\uparrow$ & AUC $\uparrow$ & Acc $\uparrow$ & F1 $\uparrow$ & AUC $\uparrow$ \\
\midrule
$\mathcal{T}(\mathbf{x}, \phi)$  & 91.82  & 90.76 & 98.36 & 83.12 & 73.67  & 93.24 \\
$\mathcal{L}^{\prime}_G$ & 91.27 & 90.10 & 98.38 & 82.81 & 73.75 & 93.45 \\
$\hat{\mathcal{L}}_G$ & 91.88 & 90.84 & 98.44 & \textbf{83.30}  & 73.90 & \textbf{93.51} \\
Ours  &\textbf{92.10} & \textbf{91.93} & \textbf{98.40} & 83.15 & \textbf{74.12} & 93.30 \\
\midrule
Change (\%)  & \textcolor{Green}{+0.28} & \textcolor{Green}{+1.17} & \textcolor{Green}{+0.04} & \textcolor{Green}{+0.03} & \textcolor{Green}{+0.45} & \textcolor{Green}{+0.06} \\
\bottomrule
\end{tabular}
        \end{adjustbox}
    \end{subtable}
\end{table}



As can be seen from the tables, when the models are trained by guiding the transformation function (with $f_{G_{\theta}}$), the performance of the models increases significantly up to 8\%, except for the HHAR dataset with a marginal performance decrease of 0.5\%.
Importantly, adding the guidance network does not bring any additional parameters that help the learning, meaning that the model achieves improved generalization with the same capacity.
Furthermore, the additional model parameters introduced to the overall framework approximately amount to one percent of those in the classifier.
\begin{wraptable}{r}{0.45\textwidth}
\caption{Ablation experiments for \textit{Audio}}
\begin{adjustbox}{width=0.45\columnwidth,center}
\centering
\begin{tabular}{@{}llllll@{}}
\toprule
\multirow{2}{*}{Method} & \multicolumn{3}{l}{Respiratory}   \\ 
\cmidrule(r{15pt}){2-4}  
& Acc $\uparrow$ & F1 $\uparrow$ & W-F1 $\uparrow$  \\
\midrule
$\mathcal{T}(\mathbf{x}, \phi)$ & 21.28\small$\pm$7.43 & 55.03\small$\pm$2.89 & 18.14\small$\pm$6.39   \\
$\mathcal{L}^{\prime}_G$ & 27.17\small$\pm$6.71 & 55.58\small$\pm$9.18 & 21.46\small$\pm$4.07  \\
$\hat{\mathcal{L}}_G$ & 28.57\small$\pm$8.31 & 54.28\small$\pm$6.56 & 23.73\small$\pm$4.65   \\
Ours & \textbf{33.10\small$\pm$5.12} & \textbf{60.13\small$\pm$4.67} & \textbf{28.33\small$\pm$6.55}  \\
\midrule
Change &  \textcolor{Green}{+11.82}  &  \textcolor{Green}{+5.10}  & \textcolor{Green}{+10.19}  \\
\bottomrule
\end{tabular}
\end{adjustbox}
\end{wraptable}
While the performance increase can be associated with the decreased possible variations in the signals, our ablation experiments show that decreasing the variations blindly using the transformation with the same angle, decreases performance. 
Therefore, it is important to guide the transformation function for reducing the dimensionality, i.e., the space and time variations of a signal, of the whole data space. 
Overall, the results obtained from the ablation study and main experiments support the previous propositions and our motivation for introducing a novel diffeomorphism for preventing the inconsistency of deep learning models to the time shifts while increasing the generalization capability.


Additional results (i.e., the extended experiments and ablations) regarding the performance of the proposed method can be found in Appendix~\ref{appendix:Additional_Results}.
Investigations regarding the performance improvements of the proposed diffeomorphism with different model networks are given in Appendix~\ref{appendix:other_networks}.
Detailed analysis of the guidance network with its effect is given in Appendix~\ref{appendix:visual_examples}.
We provide an extended discussion of related work in Appendix~\ref{appendix:extensive_related_work} and outline limitations and future directions in Appendix~\ref{appendix:limitations}.

% We extended the related work section in Appendix~\ref{appendix:extensive_related_work}.
% We discussed the limitations and future work in Appendix~\ref{appendix:limitations}.




\section{Conclusion}
In this study, we introduce \ours, a novel framework designed to achieve lossless acceleration in generating ultra-long sequences with \acp{llm}. By analyzing and addressing three challenges, \ours significantly enhances the efficiency of the generation process. Our experimental results demonstrate that \ours achieves over $3\times$ acceleration across various model scales and architectures. Furthermore, \ours effectively mitigates issues related to repetitive content, ensuring the quality and coherence of the generated sequences. These advancements position \ours as a scalable and effective solution for ultra-long sequence generation tasks.












\begin{credits}
\subsubsection{\ackname} This work was partially funded by project A4A: Audiology for All (CENTRO-01-0247- FEDER-047083) financed by the Operational Program for Competitiveness and Internationalisation of PORTUGAL 2020 through the European Regional Development Fund, by project No. 7059 - Neuraspace - AI fights Space Debris, reference C644877546-00000020, supported by the RRP - Recovery and Resilience Plan and the European Next Generation EU Funds, following Notice No. 02/C05-i01/2022, Component 5 - Capitalization and Business Innovation - Mobilizing Agendas for Business Innovation, based upon work from COST Action Randomised Optimisation Algorithms Research Network (ROAR-NET), CA22137, supported by COST (European Cooperation in Science and Technology). This work is financed through national funds by FCT - Fundação para a Ciência e a Tecnologia, I.P., in the framework of the Project UIDB/00326/2020 and UIDP/00326/2020.

\subsubsection{\discintname}
The authors have no relevant competing interests to declare that are relevant to the content of this article.
\end{credits}
%
% ---- Bibliography ----
%
% BibTeX users should specify bibliography style 'splncs04'.
% References will then be sorted and formatted in the correct style.

\bibliographystyle{splncs04}
\bibliography{references}


\end{document}
