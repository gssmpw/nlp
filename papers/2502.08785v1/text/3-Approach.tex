\section{Approach}

In this work, FEDORA \cite{rabuge2024comparison} will be applied to a \gls{HL} detection problem using three distinct classifiers based on decision trees, namely basic \gls{DT} models and their bagging and boosting counterparts, \gls{RF} and \gls{XGB}, as proxy models in the evolutionary framework. Figure \ref{fig:fedora} illustrates the inner workings of the framework.

\begin{figure}
    \centering
    \includegraphics[scale=0.5]{figures/fedora-general.pdf}
    \caption{FEDORA: Feature Engineering through Discovery of Reliable Attributes}
    \label{fig:fedora}
\end{figure}

The framework starts by splitting the original dataset into training (40\%), validation (40\%) and test subsets (20\%). The training and validation subsets are given to the evolutionary process, where \gls{SGE} will generate individuals that select and construct a new dataset from the original one, through a context-free grammar. These transformations will be applied to the training and validation subsets, which will then be used to train the proxy model and validate the transformation, respectively. The fitness is given by the validation error, namely (1 - Balanced Accuracy). After the specified generations of the evolutionary process, the individual with the lowest validation error is returned. This individual is then applied to the three subsets and its ability to generalize to unseen data is evaluated. This assessment involves training a range of Machine Learning (ML) models using both the training and validation subsets and subsequently evaluating their performance on the test set.