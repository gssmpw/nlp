\section{Related Works}
Zongxing et al. provided a comprehensive review of modalities for simultaneous gesture recognition and force assessment, including vision-based systems, surface electromyography (sEMG), ultrasound imaging, and hybrid approaches \cite{zongxing2023human}. Most of the modalities focus on discrete force levels and gesture recognition for human-machine interaction systems. In contrast, our work focuses on combining discrete manipulation skill classification with continuous force estimation, addressing challenges specific to robotic control in industrial environments. Using sEMG, Hu et al. proposed a myoelectric control framework for synchronous gesture recognition and force estimation \cite{hu2022novel}. Their work improves robustness under variable force conditions, targeting applications in prosthetics and exoskeletons, estimating discrete force-levels for wearable assistive technologies. Our approach diverges by using ultrasound imaging to classify discrete manipulation skills and estimate continuous force values aimed at robot control. 

Zengyu et al. introduced a two-stage cascade model utilizing A-mode ultrasound for gesture classification and force estimation \cite{zengyu2022simultaneous}. Their method focuses on discrete force levels and gestures for human-machine interfaces. In contrast, we employ B-mode ultrasound imaging to integrate manipulation skill recognition with continuous force prediction, specifically designed for precision in industrial robotic manipulation tasks. Peng et al. developed a flexible, wearable B-mode ultrasound transducer for simultaneous recognition of hand movements and force levels, achieving high accuracy in both amputees and non-disabled individuals \cite{peng2023novel}. Although their work focuses on discrete classifications of movements and force levels for prosthetic applications, our approach advances these capabilities by utilizing B-mode ultrasound imaging for discrete manipulation skill recognition and continuous force estimation. This enables adaptive robotic control while simultaneously capturing force and skill data to facilitate more efficient LfD.