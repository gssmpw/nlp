\section{Related Work}
\label{sec:related-work}
There is no doubt that the emergence of advanced tools such as ChatGPT has brought new opportunities to enhance law enforcement ____, risk management ____, improve efficiency ____, and expand operational capacity ____. The embedded AI-driven capabilities can be leveraged across multiple tasks that have been explored in different scenarios and tailored settings.

Perlman ____ demonstrated the use cases based on prompting ChatGPT with questions to test its capabilities in document generation, legal analysis, information and research. The author stated that the obtained responses ``were imperfect and at times problematic", while the generated documents were incomplete. Nevertheless, attorneys may be willing to prepare initial drafts of complex legal instruments, easily adopting the firm's style and incorporating its substantive knowledge. Interestingly, one of the author's other closing observations was that ``AI will not eliminate the need for lawyers, but it does portend the end of lawyering as we know it". We also agree that the AI-driven revolution in legal practice is undeniable, and resistance seems futile.

Armstrong ____ found that ChatGPT-4 was able, although sometimes imperfectly, to engage in legal reasoning about law and factual data, answer questions about legal opinions or documents, and even analyze, draft and summarize legal cases, write motions, draft patents, and write reports. However, in some cases ChatGPT performed particularly poorly on this set of prompts, occasionally producing unsatisfactory results with some serious errors. To conclude, since ChatGPT presented fiction as fact (and with confidence), legal professionals should not rely on ChatGPT responses, and thus it is not a substitute for careful reading and case analysis. 

At the University of Minnesota Law School, Choi et al. ____ tested ChatGPT to generate answers to four real exams, including Constitutional Law: Federalism and Separation of Powers, Employee Benefits, Taxation, and Torts. In all four courses, over 95 multiple-choice questions and 12 essay questions, its performance was on average at the level of a C+ student, achieving a low but passing grade. In general, ChatGPT performed better on the essay sections of the exams than on the multiple choice questions. Note that if this performance were consistent throughout law school, its grades would be sufficient for a student to graduate.

Jan and others ____ asked a stirring question: ChatGPT as an artificial lawyer? In general, the results obtained were not accurate enough to deliver legal information directly to inexperts. ChatGPT sometimes generates false information with high confidence, especially about laws and cases. In this regard, it was less willing to revise answers, even when challenged multiple times. However, ChatGPT was able to provide a seamless interactive experience with a minimal learning curve, allowing users to describe their legal issues in fragmented language and clarify or refine details as the conversation progressed. 

Apparently, there is no consensus among researchers about the effectiveness of ChatGPT in analyzing legal documents. Some studies highlight its potential for drafting legal documents and assisting with legal reasoning, while acknowledging its frequent inaccuracies and incomplete results. Others emphasize its ability to pass legal assessments at a basic level, suggesting limited but still useful capabilities. 

However, concerns persist about its tendency to generate incorrect legal information with high confidence, making it unreliable for professionals who require precise legal analysis. While some see ChatGPT as a game-changing tool in legal practice, others caution against overreliance due to its apparent limitations. These differing perspectives suggest that while such a tool holds promise, its role in legal practice remains a subject of ongoing debate.

In light of the above, we are pleased to join this lively discussion, and in our study we aim to investigate the effectiveness of ChatGPT by testing its capabilities on a self-developed Polish language dataset in the classification of evidence under the Polish Penal Code.