\documentclass{article}
\section{Summary of Mathematical Notations}
\label{sec:notations}

We summarize the main mathematical notations used in the main paper in Table \ref{table:notations}.

\begin{table*}[h]
	\centering
	\caption{Summary of main mathematical notations.}
	%\resizebox{1\columnwidth}{!}{
		\begin{tabular}{c|l}
			\toprule
			Notation  & Description  \\
			\hline
            $ \mathcal{G} $ & a document graph \\
			$ \mathcal{D} $ & a corpus of documents, $ \mathcal{D}=\{d_i\}_{i=1}^{N} $ \\
			$ N $ & number of documents in the corpus, $ N=|\mathcal{D}| $ \\
			$ d_i $ & document $ i $ containing a sequence of words, $ d_i=\{w_{i,v}\}_{v=1}^{|d_i|}\subset\mathcal{V} $ \\
			$ \mathcal{V} $ & vocabulary \\
			$ |d_i| $ & number of words in document $ i $ \\
			$ \mathcal{E} $ & a set of graph edges connecting documents, $ \mathcal{E}=\{e_{ij}\} $ \\
			$ \mathcal{N}(i) $ & the neighbor set of document $ i $ \\
            $ \Bbb H^{n,K} $ & Hyperboloid model with dimension $ n $ and curvature $ -1/K $ \\
			$ \mathcal{T}_{\textbf{x}}\Bbb H^{n,K} $ & tangent (Euclidean) space around hyperbolic vector $ x\in\Bbb H^{n,K} $ \\
			$ \exp_{\textbf{x}}^K(\textbf{v}) $ & exponential map, projecting tangent vector $ \textbf{v} $ to hyperbolic space \\
			$ \log_{\textbf{x}}^K(\textbf{y}) $ & logarithmic map, projecting hyperbolic vector $ \textbf{y} $ to $ \textbf{x} $'s tangent space \\
			$ d_{\mathcal{L}}^K(\textbf{x},\textbf{y}) $ & hyperbolic distance between hyperbolic vectors $ \textbf{x} $ and $ \textbf{y} $ \\
			$ \text{PT}_{\textbf{x}\rightarrow\textbf{y}}^K(\textbf{v}) $ & parallel transport, transporting $ \textbf{v} $ from $ \textbf{x} $'s tangent space to $ \textbf{y} $'s \\
			$ H $ & length of a path on topic tree \\
            $ \sigma(t,i) $ & similarity between topic $ t $ and document $ i $ \\
            $ \bm{\pi}_i $ & path distribution of document $ i $ over topic tree \\
            $ \textbf{z}_{t,p} $ & hyperbolic ancestral hidden state of topic $ t $ \\
            $ \textbf{z}_{t,s} $ & hyperbolic fraternal hidden state of topic $ t $ \\
			$ \textbf{z}_t $ & hyperbolic hidden state of topic $ t $ \\
            $ \sigma(h,i) $ & similarity between topic $ t $ and document $ i $ \\
            $ \textbf{z}_h $ & hyperbolic hidden state of level $ h $ \\
			$ \bm{\delta}_i $ & level distribution of document $ i $ over topic tree \\
            $ \bm{\theta}_i $ & topic distribution of document $ i $ over topic tree \\
            $ \textbf{e}_i $ & hierarchical tree embedding of document $ i $ \\
            $ T $ & number of topics on topic tree \\
            $ \textbf{g}_i $ & hierarchical graph embedding of document $ i $ \\
			$ \textbf{U} $ & a matrix of word embeddings, $ \textbf{U}\in\Bbb R^{|\mathcal{V}|\times(n+1)} $ \\
			$ \bm{\beta} $ & topic-word distribution $ \bm{\beta}\in\Bbb R^{T\times |\mathcal{V}|} $ \\
			\bottomrule
		\end{tabular}
	%}
	%\vspace{-0.2cm}
	\label{table:notations}
\end{table*}
\usepackage[accepted]{icml2025}
% Recommended, but optional, packages for figures and better typesetting:
\usepackage{microtype}
\usepackage{graphicx}
\usepackage{subcaption}
\usepackage{booktabs} % for professional tables
\usepackage[most]{tcolorbox}  % Load tcolorbox package
\newtcblisting{templatebox}{colback=white,colframe=black, listing only, listing options={
    breakautoindent=false, 
    breaklines=true,
    columns=fullflexible,
    breakindent=0pt,
    breakatwhitespace=true,
    basicstyle=\small\rmfamily,
    language=
}}
\newcommand{\theHalgorithm}{\arabic{algorithm}}


% For theorems and such
\usepackage{amsmath,amssymb,amsfonts,amsthm,mathtools,mathrsfs}  % Consolidated math packages
% \usepackage[capitalize,noabbrev]{cleveref}

%%%%%%%%%%%%%%%%%%%%%%%%%%%%%%%%
% THEOREMS
%%%%%%%%%%%%%%%%%%%%%%%%%%%%%%%%
\theoremstyle{plain}
\newtheorem{theorem}{Theorem}[section]
\newtheorem{proposition}[theorem]{Proposition}
\newtheorem{lemma}[theorem]{Lemma}
\newtheorem{corollary}[theorem]{Corollary}
\theoremstyle{definition}
\newtheorem{definition}[theorem]{Definition}
\newtheorem{assumption}[theorem]{Assumption}
\theoremstyle{remark}
\newtheorem{remark}[theorem]{Remark}
\newtheorem{hypothesis}{Hypothesis}[section]

\def\shownotes{0}  %set 1 to show author notes
\ifnum\shownotes=1
\newcommand{\authnote}[2]{{\scriptsize $\ll$\textsf{#1 notes: #2}$\gg$}}
\else
\newcommand{\authnote}[2]{}
\fi
\newcommand\RZ[1]{\textcolor{red}{\authnote{RZ}{#1}}}
\newcommand\hanlin[1]{\textcolor{blue}{\authnote{Hanlin}{#1}}}
\newcommand\sm[1]{\textcolor{red}{\authnote{Song}{#1}}}
\newcommand\tianyu[1]{\textcolor{brown}{\authnote{Tianyu}{#1}}}
% \usepackage[textsize=tiny]{todonotes}




\icmltitlerunning{How Do LLMs Perform Two-Hop Reasoning in Context?}

\newcommand{\Wup}{\bW_{\sf up}}
\newcommand{\Wgate}{\bW_{\sf gate}}
\newcommand{\Wdown}{\bW_{\sf down}}
\newcommand{\bup}{\mathbf{b}_{\sf up}}
\newcommand{\bgate}{\mathbf{b}_{\sf gate}}
\newcommand{\bdown}{\mathbf{b}_{\sf down}}
\newcommand{\silu}{\sigma_{\sf silu}}
% \newcommand{\bos}{\texttt{$\langle \texttt{s}\rangle$}}


\icmltitlerunning{How Do LLMs Perform Two-Hop Reasoning in Context?}



\newcommand{\fix}{\marginpar{FIX}}
\newcommand{\new}{\marginpar{NEW}}


\begin{document}

\twocolumn[
\icmltitle{How Do LLMs Perform Two-Hop Reasoning in Context?}

\begin{icmlauthorlist}
\icmlauthor{Tianyu Guo$^*$}{berkeley}
\icmlauthor{Hanlin Zhu$^*$}{berkeley}
\icmlauthor{Ruiqi Zhang}{berkeley}
\icmlauthor{Jiantao Jiao}{berkeley}
\icmlauthor{Song Mei}{berkeley}
\icmlauthor{Michael I. Jordan}{berkeley}
\icmlauthor{Stuart Russell}{berkeley}
\end{icmlauthorlist}

\icmlaffiliation{berkeley}{UC Berkeley, Berkeley, CA, USA}

\icmlcorrespondingauthor{Tianyu Guo}{tianyu_guo@berkeley.edu}
\icmlcorrespondingauthor{Hanlin Zhu}{hanlinzhu@berkeley.edu}

\icmlkeywords{Machine Learning, Large Language Models, Reasoning, Training Dynamics}

\vskip 0.3in
]
\printAffiliationsAndNotice{\icmlEqualContribution} % Use this to indicate equal contribution

\begin{abstract}
\tianyu{I have slighted revised the abstract}
``Socrates is human. All humans are mortal. Therefore, Socrates is mortal.'' This classical example demonstrates two-hop reasoning, where a conclusion logically follows from two connected premises. \RZ{'This illustrates XXXX' looks strange. I would say 'The classical example "Socrates is human. All humans are mortal. Therefore, Socrates is mortal" demonstrates a fundamental two-hop reasoning process, where a conclusion follows from two premises.'} While transformer-based Large Language Models (LLMs) can make two-hop reasoning \RZ{'at such reasoning' seems weird. I am not sure whether using 'in context' here is good since people may confuse it with 'in-context' in ICL. Maybe 'within a textual context'?}, they tend to collapse to random guessing when faced with distracting premises \RZ{I would not say 'they may'. The tone is weak.}. To understand the underlying mechanism, we train a three-layer transformer on synthetic two-hop reasoning tasks. \RZ{I would say 'We train a three-layer XXXXX on XXX and analyze the dynamics of XXXXX'}The training dynamics show two stages: a slow learning phase, where the 3-layer transformer performs random guessing like LLMs, followed by an abrupt phase transitions, where the 3-layer transformer suddenly reaches $100\%$ accuracy. \RZ{I think we should remove the 'Initially' in this sentence. It makes the sentence less fluent.} Through reverse engineering, we explain the inner mechanisms for how models learn to randomly guess between distractions initially, and how they learn to ignore distractions eventually. We further propose a three-parameter model that supports the causal claims for the mechanisms to the training dynamics of the transformer. Finally, experiments on LLMs \RZ{Llama?} suggest that the discovered mechanisms generalize across scales. Our methodologies provide new perspectives for scientific understandings of LLMs and our findings provide new insights into how reasoning emerges during training. \RZ{The last sentence is weird. The paper is not on 'optimizing complex models' since we do not propose any techniques on optimization side.}
\end{abstract}

\section{Introduction}


\begin{figure}[t]
\centering
\includegraphics[width=0.6\columnwidth]{figures/evaluation_desiderata_V5.pdf}
\vspace{-0.5cm}
\caption{\systemName is a platform for conducting realistic evaluations of code LLMs, collecting human preferences of coding models with real users, real tasks, and in realistic environments, aimed at addressing the limitations of existing evaluations.
}
\label{fig:motivation}
\end{figure}

\begin{figure*}[t]
\centering
\includegraphics[width=\textwidth]{figures/system_design_v2.png}
\caption{We introduce \systemName, a VSCode extension to collect human preferences of code directly in a developer's IDE. \systemName enables developers to use code completions from various models. The system comprises a) the interface in the user's IDE which presents paired completions to users (left), b) a sampling strategy that picks model pairs to reduce latency (right, top), and c) a prompting scheme that allows diverse LLMs to perform code completions with high fidelity.
Users can select between the top completion (green box) using \texttt{tab} or the bottom completion (blue box) using \texttt{shift+tab}.}
\label{fig:overview}
\end{figure*}

As model capabilities improve, large language models (LLMs) are increasingly integrated into user environments and workflows.
For example, software developers code with AI in integrated developer environments (IDEs)~\citep{peng2023impact}, doctors rely on notes generated through ambient listening~\citep{oberst2024science}, and lawyers consider case evidence identified by electronic discovery systems~\citep{yang2024beyond}.
Increasing deployment of models in productivity tools demands evaluation that more closely reflects real-world circumstances~\citep{hutchinson2022evaluation, saxon2024benchmarks, kapoor2024ai}.
While newer benchmarks and live platforms incorporate human feedback to capture real-world usage, they almost exclusively focus on evaluating LLMs in chat conversations~\citep{zheng2023judging,dubois2023alpacafarm,chiang2024chatbot, kirk2024the}.
Model evaluation must move beyond chat-based interactions and into specialized user environments.



 

In this work, we focus on evaluating LLM-based coding assistants. 
Despite the popularity of these tools---millions of developers use Github Copilot~\citep{Copilot}---existing
evaluations of the coding capabilities of new models exhibit multiple limitations (Figure~\ref{fig:motivation}, bottom).
Traditional ML benchmarks evaluate LLM capabilities by measuring how well a model can complete static, interview-style coding tasks~\citep{chen2021evaluating,austin2021program,jain2024livecodebench, white2024livebench} and lack \emph{real users}. 
User studies recruit real users to evaluate the effectiveness of LLMs as coding assistants, but are often limited to simple programming tasks as opposed to \emph{real tasks}~\citep{vaithilingam2022expectation,ross2023programmer, mozannar2024realhumaneval}.
Recent efforts to collect human feedback such as Chatbot Arena~\citep{chiang2024chatbot} are still removed from a \emph{realistic environment}, resulting in users and data that deviate from typical software development processes.
We introduce \systemName to address these limitations (Figure~\ref{fig:motivation}, top), and we describe our three main contributions below.


\textbf{We deploy \systemName in-the-wild to collect human preferences on code.} 
\systemName is a Visual Studio Code extension, collecting preferences directly in a developer's IDE within their actual workflow (Figure~\ref{fig:overview}).
\systemName provides developers with code completions, akin to the type of support provided by Github Copilot~\citep{Copilot}. 
Over the past 3 months, \systemName has served over~\completions suggestions from 10 state-of-the-art LLMs, 
gathering \sampleCount~votes from \userCount~users.
To collect user preferences,
\systemName presents a novel interface that shows users paired code completions from two different LLMs, which are determined based on a sampling strategy that aims to 
mitigate latency while preserving coverage across model comparisons.
Additionally, we devise a prompting scheme that allows a diverse set of models to perform code completions with high fidelity.
See Section~\ref{sec:system} and Section~\ref{sec:deployment} for details about system design and deployment respectively.



\textbf{We construct a leaderboard of user preferences and find notable differences from existing static benchmarks and human preference leaderboards.}
In general, we observe that smaller models seem to overperform in static benchmarks compared to our leaderboard, while performance among larger models is mixed (Section~\ref{sec:leaderboard_calculation}).
We attribute these differences to the fact that \systemName is exposed to users and tasks that differ drastically from code evaluations in the past. 
Our data spans 103 programming languages and 24 natural languages as well as a variety of real-world applications and code structures, while static benchmarks tend to focus on a specific programming and natural language and task (e.g. coding competition problems).
Additionally, while all of \systemName interactions contain code contexts and the majority involve infilling tasks, a much smaller fraction of Chatbot Arena's coding tasks contain code context, with infilling tasks appearing even more rarely. 
We analyze our data in depth in Section~\ref{subsec:comparison}.



\textbf{We derive new insights into user preferences of code by analyzing \systemName's diverse and distinct data distribution.}
We compare user preferences across different stratifications of input data (e.g., common versus rare languages) and observe which affect observed preferences most (Section~\ref{sec:analysis}).
For example, while user preferences stay relatively consistent across various programming languages, they differ drastically between different task categories (e.g. frontend/backend versus algorithm design).
We also observe variations in user preference due to different features related to code structure 
(e.g., context length and completion patterns).
We open-source \systemName and release a curated subset of code contexts.
Altogether, our results highlight the necessity of model evaluation in realistic and domain-specific settings.





\section{Preliminaries}
\label{sec:prelim}
\label{sec:term}
We define the key terminologies used, primarily focusing on the hidden states (or activations) during the forward pass. 

\paragraph{Components in an attention layer.} We denote $\Res$ as the residual stream. We denote $\Val$ as Value (states), $\Qry$ as Query (states), and $\Key$ as Key (states) in one attention head. The \attlogit~represents the value before the softmax operation and can be understood as the inner product between  $\Qry$  and  $\Key$. We use \Attn~to denote the attention weights of applying the SoftMax function to \attlogit, and ``attention map'' to describe the visualization of the heat map of the attention weights. When referring to the \attlogit~from ``$\tokenB$'' to  ``$\tokenA$'', we indicate the inner product  $\langle\Qry(\tokenB), \Key(\tokenA)\rangle$, specifically the entry in the ``$\tokenB$'' row and ``$\tokenA$'' column of the attention map.

\paragraph{Logit lens.} We use the method of ``Logit Lens'' to interpret the hidden states and value states \citep{belrose2023eliciting}. We use \logit~to denote pre-SoftMax values of the next-token prediction for LLMs. Denote \readout~as the linear operator after the last layer of transformers that maps the hidden states to the \logit. 
The logit lens is defined as applying the readout matrix to residual or value states in middle layers. Through the logit lens, the transformed hidden states can be interpreted as their direct effect on the logits for next-token prediction. 

\paragraph{Terminologies in two-hop reasoning.} We refer to an input like “\Src$\to$\brga, \brgb$\to$\Ed” as a two-hop reasoning chain, or simply a chain. The source entity $\Src$ serves as the starting point or origin of the reasoning. The end entity $\Ed$ represents the endpoint or destination of the reasoning chain. The bridge entity $\Brg$ connects the source and end entities within the reasoning chain. We distinguish between two occurrences of $\Brg$: the bridge in the first premise is called $\brga$, while the bridge in the second premise that connects to $\Ed$ is called $\brgc$. Additionally, for any premise ``$\tokenA \to \tokenB$'', we define $\tokenA$ as the parent node and $\tokenB$ as the child node. Furthermore, if at the end of the sequence, the query token is ``$\tokenA$'', we define the chain ``$\tokenA \to \tokenB$, $\tokenB \to \tokenC$'' as the Target Chain, while all other chains present in the context are referred to as distraction chains. Figure~\ref{fig:data_illustration} provides an illustration of the terminologies.

\paragraph{Input format.}
Motivated by two-hop reasoning in real contexts, we consider input in the format $\bos, \text{context information}, \query, \answer$. A transformer model is trained to predict the correct $\answer$ given the query $\query$ and the context information. The context compromises of $K=5$ disjoint two-hop chains, each appearing once and containing two premises. Within the same chain, the relative order of two premises is fixed so that \Src$\to$\brga~always precedes \brgb$\to$\Ed. The orders of chains are randomly generated, and chains may interleave with each other. The labels for the entities are re-shuffled for every sequence, choosing from a vocabulary size $V=30$. Given the $\bos$ token, $K=5$ two-hop chains, \query, and the \answer~tokens, the total context length is $N=23$. Figure~\ref{fig:data_illustration} also illustrates the data format. 

\paragraph{Model structure and training.} We pre-train a three-layer transformer with a single head per layer. Unless otherwise specified, the model is trained using Adam for $10,000$ steps, achieving near-optimal prediction accuracy. Details are relegated to Appendix~\ref{app:sec_add_training_detail}.


% \RZ{Do we use source entity, target entity, and mediator entity? Or do we use original token, bridge token, end token?}





% \paragraph{Basic notations.} We use ... We use $\ve_i$ to denote one-hot vectors of which only the $i$-th entry equals one, and all other entries are zero. The dimension of $\ve_i$ are usually omitted and can be inferred from contexts. We use $\indicator\{\cdot\}$ to denote the indicator function.

% Let $V > 0$ be a fixed positive integer, and let $\vocab = [V] \defeq \{1, 2, \ldots, V\}$ be the vocabulary. A token $v \in \vocab$ is an integer in $[V]$ and the input studied in this paper is a sequence of tokens $s_{1:T} \defeq (s_1, s_2, \ldots, s_T) \in \vocab^T$ of length $T$. For any set $\mathcal{S}$, we use $\Delta(\mathcal{S})$ to denote the set of distributions over $\mathcal{S}$.

% % to a sequence of vectors $z_1, z_2, \ldots, z_T \in \real^{\dout}$ of dimension $\dout$ and length $T$.

% Let $\mU = [\vu_1, \vu_2, \ldots, \vu_V]^\transpose \in \real^{V\times d}$ denote the token embedding matrix, where the $i$-th row $\vu_i \in \real^d$ represents the $d$-dimensional embedding of token $i \in [V]$. Similarly, let $\mP = [\vp_1, \vp_2, \ldots, \vp_T]^\transpose \in \real^{T\times d}$ denote the positional embedding matrix, where the $i$-th row $\vp_i \in \real^d$ represents the $d$-dimensional embedding of position $i \in [T]$. Both $\mU$ and $\mP$ can be fixed or learnable.

% After receiving an input sequence of tokens $s_{1:T}$, a transformer will first process it using embedding matrices $\mU$ and $\mP$ to obtain a sequence of vectors $\mH = [\vh_1, \vh_2, \ldots, \vh_T] \in \real^{d\times T}$, where 
% \[
% \vh_i = \mU^\transpose\ve_{s_i} + \mP^\transpose\ve_{i} = \vu_{s_i} + \vp_i.
% \]

% We make the following definitions of basic operations in a transformer.

% \begin{definition}[Basic operations in transformers] 
% \label{defn:operators}
% Define the softmax function $\softmax(\cdot): \real^d \to \real^d$ over a vector $\vv \in \real^d$ as
% \[\softmax(\vv)_i = \frac{\exp(\vv_i)}{\sum_{j=1}^d \exp(\vv_j)} \]
% and define the softmax function $\softmax(\cdot): \real^{m\times n} \to \real^{m \times n}$ over a matrix $\mV \in \real^{m\times n}$ as a column-wise softmax operator. For a squared matrix $\mM \in \real^{m\times m}$, the causal mask operator $\mask(\cdot): \real^{m\times m} \to \real^{m\times m}$  is defined as $\mask(\mM)_{ij} = \mM_{ij}$ if $i \leq j$ and  $\mask(\mM)_{ij} = -\infty$ otherwise. For a vector $\vv \in \real^n$ where $n$ is the number of hidden neurons in a layer, we use $\layernorm(\cdot): \real^n \to \real^n$ to denote the layer normalization operator where
% \[
% \layernorm(\vv)_i = \frac{\vv_i-\mu}{\sigma}, \mu = \frac{1}{n}\sum_{j=1}^n \vv_j, \sigma = \sqrt{\frac{1}{n}\sum_{j=1}^n (\vv_j-\mu)^2}
% \]
% and use $\layernorm(\cdot): \real^{n\times m} \to \real^{n\times m}$ to denote the column-wise layer normalization on a matrix.
% We also use $\nonlin(\cdot)$ to denote element-wise nonlinearity such as $\relu(\cdot)$.
% \end{definition}

% The main components of a transformer are causal self-attention heads and MLP layers, which are defined as follows.

% \begin{definition}[Attentions and MLPs]
% \label{defn:attn_mlp} 
% A single-head causal self-attention $\attn(\mH;\mQ,\mK,\mV,\mO)$ parameterized by $\mQ,\mK,\mV \in \real^{{\dqkv\times \din}}$ and $\mO \in \real^{\dout\times\dqkv}$ maps an input matrix $\mH \in \real^{\din\times T}$ to
% \begin{align*}
% &\attn(\mH;\mQ,\mK,\mV,\mO) \\
% =&\mO\mV\layernorm(\mH)\softmax(\mask(\layernorm(\mH)^\transpose\mK^\transpose\mQ\layernorm(\mH))).
% \end{align*}
% Furthermore, a multi-head attention with $M$ heads parameterized by $\{(\mQ_m,\mK_m,\mV_m,\mO_m) \}_{m=1}^M$ is defined as 
% \begin{align*}
%     &\Attn(\mH; \{(\mQ_m,\mK_m,\mV_m,\mO_m) \}_{m\in[M]}) \\ =& \sum_{m=1}^M \attn(\mH;\mQ_m,\mK_m,\mV_m,\mO_m) \in \real^{\dout \times T}.
% \end{align*}
% An MLP layer $\mlp(\mH;\mW_1,\mW_2)$ parameterized by $\mW_1 \in \real^{\dhidden\times \din}$ and $\mW_2 \in \real^{\dout \times \dhidden}$ maps an input matrix $\mH = [\vh_1, \ldots, \vh_T] \in \real^{\din \times T}$ to
% \begin{align*}
%     &\mlp(\mH;\mW_1,\mW_2) = [\vy_1, \ldots, \vy_T], \\ \text{where } &\vy_i = \mW_2\nonlin(\mW_1\layernorm(\vh_i)), \forall i \in [T].
% \end{align*}

% \end{definition}

% In this paper, we assume $\din=\dout=d$ for all attention heads and MLPs to facilitate residual stream unless otherwise specified. Given \Cref{defn:operators,defn:attn_mlp}, we are now able to define a multi-layer transformer.

% \begin{definition}[Multi-layer transformers]
% \label{defn:transformer}
%     An $L$-layer transformer $\transformer(\cdot): \vocab^T \to \Delta(\vocab)$ parameterized by $\mP$, $\mU$, $\{(\mQ_m^{(l)},\mK_m^{(l)},\mV_m^{(l)},\mO_m^{(l)})\}_{m\in[M],l\in[L]}$,  $\{(\mW_1^{(l)},\mW_2^{(l)})\}_{l\in[L]}$ and $\Wreadout \in \real^{V \times d}$ receives a sequence of tokens $s_{1:T}$ as input and predict the next token by outputting a distribution over the vocabulary. The input is first mapped to embeddings $\mH = [\vh_1, \vh_2, \ldots, \vh_T] \in \real^{d\times T}$ by embedding matrices $\mP, \mU$ where 
%     \[
%     \vh_i = \mU^\transpose\ve_{s_i} + \mP^\transpose\ve_{i}, \forall i \in [T].
%     \]
%     For each layer $l \in [L]$, the output of layer $l$, $\mH^{(l)} \in \real^{d\times T}$, is obtained by 
%     \begin{align*}
%         &\mH^{(l)} =  \mH^{(l-1/2)} + \mlp(\mH^{(l-1/2)};\mW_1^{(l)},\mW_2^{(l)}), \\
%         & \mH^{(l-1/2)} = \mH^{(l-1)} + \\ & \quad \Attn(\mH^{(l-1)}; \{(\mQ_m^{(l)},\mK_m^{(l)},\mV_m^{(l)},\mO_m^{(l)}) \}_{m\in[M]}), 
%     \end{align*}
%     where the input $\mH^{(l-1)}$ is the output of the previous layer $l-1$ for $l > 1$ and the input of the first layer $\mH^{(0)} = \mH$. Finally, the output of the transformer is obtained by 
%     \begin{align*}
%         \transformer(s_{1:T}) = \softmax(\Wreadout\vh_T^{(L)})
%     \end{align*}
%     which is a $V$-dimensional vector after softmax representing a distribution over $\vocab$, and $\vh_T^{(L)}$ is the $T$-th column of the output of the last layer, $\mH^{(L)}$.
% \end{definition}



% For each token $v \in \vocab$, there is a corresponding $d_t$-dimensional token embedding vector $\embed(v) \in \mathbb{R}^{d_t}$. Assume the maximum length of the sequence studied in this paper does not exceed $T$. For each position $t \in [T]$, there is a corresponding positional embedding  







% \section{Experiments}
\label{sec:Experiments} 

We conduct several experiments across different problem settings to assess the efficiency of our proposed method. Detailed descriptions of the experimental settings are provided in \cref{sec:apendix_experiments}.
%We conduct experiments on optimizing PINNs for convection, wave PDEs, and a reaction ODE. 
%These equations have been studied in previous works investigating difficulties in training PINNs; we use the formulations in \citet{krishnapriyan2021characterizing, wang2022when} for our experiments. 
%The coefficient settings we use for these equations are considered challenging in the literature \cite{krishnapriyan2021characterizing, wang2022when}.
%\cref{sec:problem_setup_additional} contains additional details.

%We compare the performance of Adam, \lbfgs{}, and \al{} on training PINNs for all three classes of PDEs. 
%For Adam, we tune the learning rate by a grid search on $\{10^{-5}, 10^{-4}, 10^{-3}, 10^{-2}, 10^{-1}\}$.
%For \lbfgs, we use the default learning rate $1.0$, memory size $100$, and strong Wolfe line search.
%For \al, we tune the learning rate for Adam as before, and also vary the switch from Adam to \lbfgs{} (after 1000, 11000, 31000 iterations).
%These correspond to \al{} (1k), \al{} (11k), and \al{} (31k) in our figures.
%All three methods are run for a total of 41000 iterations.

%We use multilayer perceptrons (MLPs) with tanh activations and three hidden layers. These MLPs have widths 50, 100, 200, or 400.
%We initialize these networks with the Xavier normal initialization \cite{glorot2010understanding} and all biases equal to zero.
%Each combination of PDE, optimizer, and MLP architecture is run with 5 random seeds.

%We use 10000 residual points randomly sampled from a $255 \times 100$ grid on the interior of the problem domain. 
%We use 257 equally spaced points for the initial conditions and 101 equally spaced points for each boundary condition.

%We assess the discrepancy between the PINN solution and the ground truth using $\ell_2$ relative error (L2RE), a standard metric in the PINN literature. Let $y = (y_i)_{i = 1}^n$ be the PINN prediction and $y' = (y'_i)_{i = 1}^n$ the ground truth. Define
%\begin{align*}
%    \mathrm{L2RE} = \sqrt{\frac{\sum_{i = 1}^n (y_i - y'_i)^2}{\sum_{i = 1}^n y'^2_i}} = \sqrt{\frac{\|y - y'\|_2^2}{\|y'\|_2^2}}.
%\end{align*}
%We compute the L2RE using all points in the $255 \times 100$ grid on the interior of the problem domain, along with the 257 and 101 points used for the initial and boundary conditions.

%We develop our experiments in PyTorch 2.0.0 \cite{paszke2019pytorch} with Python 3.10.12.
%Each experiment is run on a single NVIDIA Titan V GPU using CUDA 11.8.
%The code for our experiments is available at \href{https://github.com/pratikrathore8/opt_for_pinns}{https://github.com/pratikrathore8/opt\_for\_pinns}.


\subsection{2D Allen Cahn Equation}
\begin{figure*}[t]
    \centering
    \includegraphics[scale=0.38]{figs/Burgers_operator.pdf}
    \caption{1D Burgers' Equation (Operator Learning): Steady-state solutions for different initializations $u_0$ under varying viscosity $\varepsilon$: (a) $\varepsilon = 0.5$, (b) $\varepsilon = 0.1$, (c) $\varepsilon = 0.05$. The results demonstrate that all final test solutions converge to the correct steady-state solution. (d) Illustration of the evolution of a test initialization $u_0$ following homotopy dynamics. The number of residual points is $\nres = 128$.}
    \label{fig:Burgers_result}
\end{figure*}
First, we consider the following time-dependent problem:
\begin{align}
& u_t = \varepsilon^2 \Delta u - u(u^2 - 1), \quad (x, y) \in [-1, 1] \times [-1, 1] \nonumber \\
& u(x, y, 0) = - \sin(\pi x) \sin(\pi y) \label{eq.hom_2D_AC}\\
& u(-1, y, t) = u(1, y, t) = u(x, -1, t) = u(x, 1, t) = 0. \nonumber
\end{align}
We aim to find the steady-state solution for this equation with $\varepsilon = 0.05$ and define the homotopy as:
\begin{equation}
    H(u, s, \varepsilon) = (1 - s)\left(\varepsilon(s)^2 \Delta u - u(u^2 - 1)\right) + s(u - u_0),\nonumber
\end{equation}
where $s \in [0, 1]$. Specifically, when $s = 1$, the initial condition $u_0$ is automatically satisfied, and when $s = 0$, it recovers the steady-state problem. The function $\varepsilon(s)$ is given by
\begin{equation}
\varepsilon(s) = 
\left\{\begin{array}{l}
s, \quad s \in [0.05, 1], \\
0.05, \quad s \in [0, 0.05].
\end{array}\right.\label{eq:epsilon_t}
\end{equation}

Here, $\varepsilon(s)$ varies with $s$ during the first half of the evolution. Once $\varepsilon(s)$ reaches $0.05$, it remains fixed, and only $s$ continues to evolve toward $0$. As shown in \cref{fig:2D_Allen_Cahn_Equation}, the relative $L_2$ error by homotopy dynamics is $8.78 \times 10^{-3}$, compared with the result obtained by PINN, which has a $L_2$ error of $9.56 \times 10^{-1}$. This clearly demonstrates that the homotopy dynamics-based approach significantly improves accuracy.

\subsection{High Frequency Function Approximation }
We aim to approximate the following function:
$u=    \sin(50\pi x), \quad x \in [0,1].$
The homotopy is defined as $H(u,\varepsilon) = u - \sin(\frac{1}{\varepsilon}\pi x), $
where $\varepsilon \in [\frac{1}{50},\frac{1}{15}]$.

\begin{table}[htbp!]
    \caption{Comparison of the lowest loss achieved by the classical training and homotopy dynamics for different values of $\varepsilon$ in approximating $\sin\left(\frac{1}{\varepsilon} \pi x\right)$
    }
    \vskip 0.15in
    \centering
    \tiny
    \begin{tabular}{|c|c|c|c|c|} 
    \hline 
    $ $ & $\varepsilon = 1/15$ & $\varepsilon = 1/35$ & $\varepsilon = 1/50$ \\ \hline 
    Classical Loss                & 4.91e-6     & 7.21e-2     & 3.29e-1       \\ \hline 
    Homotopy Loss $L_H$                      & 1.73e-6     & 1.91e-6     & \textbf{2.82e-5}       \\ \hline
    \end{tabular}
    % On convection, \al{} provides 14.2$\times$ and 1.97$\times$ improvement over Adam or \lbfgs{} on L2RE. 
    % On reaction, \al{} provides 1.10$\times$ and 1.99$\times$ improvement over Adam or \lbfgs{} on L2RE.
    % On wave, \al{} provides 6.32$\times$ and 6.07$\times$ improvement over Adam or \lbfgs{} on L2RE.}
    \label{tab:loss_approximate}
\end{table}

As shown in \cref{fig:high_frequency_result}, due to the F-principle \cite{xu2024overview}, training is particularly challenging when approximating high-frequency functions like $\sin(50\pi x)$. The loss decreases slowly, resulting in poor approximation performance. However, training based on homotopy dynamics significantly reduces the loss, leading to a better approximation of high-frequency functions. This demonstrates that homotopy dynamics-based training can effectively facilitate convergence when approximating high-frequency data. Additionally, we compare the loss for approximating functions with different frequencies $1/\varepsilon$ using both methods. The results, presented in \cref{tab:loss_approximate}, show that the homotopy dynamics training method consistently performs well for high-frequency functions.





\subsection{Burgers Equation}
In this example, we adopt the operator learning framework to solve for the steady-state solution of the Burgers equation, given by:
\begin{align}
& u_t+\left(\frac{u^2}{2}\right)_x - \varepsilon u_{xx}=\pi \sin (\pi x) \cos (\pi x), \quad x \in[0, 1]\nonumber\\
& u(x, 0)=u_0(x),\label{eq:1D_Burgers} \\
& u(0, t)=u(1, t)=0, \nonumber 
\end{align}
with Dirichlet boundary conditions, where $u_0 \in L_{0}^2((0, 1); \mathbb{R})$ is the initial condition and $\varepsilon \in \mathbb{R}$ is the viscosity coefficient. We aim to learn the operator mapping the initial condition to the steady-state solution, $G^{\dagger}: L_{0}^2((0, 1); \mathbb{R}) \rightarrow H_{0}^r((0, 1); \mathbb{R})$, defined by $u_0 \mapsto u_{\infty}$ for any $r > 0$. As shown in Theorem 2.2 of \cite{KREISS1986161} and Theorems 2.5 and 2.7 of \cite{hao2019convergence}, for any $\varepsilon > 0$, the steady-state solution is independent of the initial condition, with a single shock occurring at $x_s = 0.5$. Here, we use DeepONet~\cite{lu2021deeponet} as the network architecture. 
The homotopy definition, similar to ~\cref{eq.hom_2D_AC}, can be found in \cref{Ap:operator}. The results can be found in \cref{fig:Burgers_result} and \cref{tab:loss_burgers}. Experimental results show that the homotopy dynamics strategy performs well in the operator learning setting as well.


\begin{table}[htbp!]
    \caption{Comparison of loss between classical training and homotopy dynamics for different values of $\varepsilon$ in the Burgers equation, along with the MSE distance to the ground truth shock location, $x_s$.}
    \vskip 0.15in
    \centering
    \tiny
    \begin{tabular}{|c|c|c|c|c|} 
    \hline  
    $ $ & $\varepsilon = 0.5$ & $\varepsilon = 0.1$ & $\varepsilon = 0.05$ \\ \hline 
    Homotopy Loss $L_H$                &  7.55e-7     & 3.40e-7     & 7.77e-7       \\ \hline 
    L2RE                      & 1.50e-3     & 7.00e-4     & 2.52e-2       \\ \hline
        MSE Distance $x_s$                      & 1.75e-8     & 9.14e-8      & 1.2e-3      \\ \hline
    \end{tabular}
    % On convection, \al{} provides 14.2$\times$ and 1.97$\times$ improvement over Adam or \lbfgs{} on L2RE. 
    % On reaction, \al{} provides 1.10$\times$ and 1.99$\times$ improvement over Adam or \lbfgs{} on L2RE.
    % On wave, \al{} provides 6.32$\times$ and 6.07$\times$ improvement over Adam or \lbfgs{} on L2RE.}
    \label{tab:loss_burgers}
\end{table}



% \begin{itemize}
%     \item Relate the curvature in the problem to the differential operator. Use this to demonstrate why the problem is ill-conditioned
%     \item Give an argument for why using Adam + L-BFGS is better than just using L-BFGS outright. The idea is that Adam lowers the errors to the point where the rest of the optimization becomes convex \ldots
%     \item Show why we need second-order methods. We would like to prove that once we are close to the optimum, second-order methods will give condition-number free linear convergence. Specialize this to the Gauss-Newton setting, with the randomized low-rank approximation.
%     % \item Show that it is not possible to get superlinear convergence under the interpolation assumption with an overparameterized neural network. This should be true b/c the Hessian at the optimum will have rank $\min(n, d)$, and when $d > n$, this guarantees that we cannot have strong convexity.
% \end{itemize}
\section{Restricted Attention Models}
\label{sec:restricted_attention}


% The input of each layer $\ell$ is a sequence of embedding vectors $\mH^{(\ell-1)} = [\vh^{(\ell-1)}_1, \vh^{(\ell-1)}_2, \ldots, \vh^{(\ell-1)}_T ] \in \real^{d\times T}$, and the output is also a sequence of embeddings $\mH^{(\ell)} = [\vh^{(\ell)}_1, \vh^{(\ell)}_2, \ldots, \vh^{(\ell)}_T ] \in \real^{d\times T}$. The final output is $\softmax(\Wreadout\vh_T^{(L)}) \in \Delta(\vocab)$ where $\Wreadout \in \real^{V \times d}$ is the readout matrix.

% For each layer $\ell$, we have
% \begin{align*}
% \vh_i^{(\ell)} &= \sum_{j \leq i} q_{ij}^{(\ell)} \mV^{(\ell)} \vh_j^{(\ell-1)}, \\
% \text{ where } q_{ij}^{(\ell)} &= \frac{\exp(S_{ij}^{(\ell)})}{\sum_{k\leq i} \exp(S_{ik}^{(\ell)}) + \xi^{(\ell)}}. 
% \end{align*}

% Note that $\xi^{(\ell)} > 0$ is the attention logit for attention sink in layer $\ell$, where a query token can dump attention logits when there are no key tokens requiring attention~\citep{guo2024active}.

% $S_{ij}^{(\ell)}$ denotes the attention logit from token $i$ to token $j$ in layer $\ell$. Note that since we use causal attention,  $S_{ij}^{(\ell)}$ is non-zero only if $i \geq j$. Below, we provide the value of all non-zero terms among $\{S^{(\ell)}_{ij}\}$.

% \begin{align*}
% S_{ij}^{(1)} = \alpha, \quad i = \idx(\child), \, j = i-1
% \end{align*}

% \begin{align*}
% S_{ij}^{(2)} &=
% \begin{cases}
%     \beta_1 \langle \vh_i^{(1)}, \vh_j^{(1)} \rangle & i = \idx(\query),  j = \idx(\target  \brg) \\
%     \beta_2 \langle \vh_i^{(1)}, \vh_j^{(1)} \rangle & i \in \idx(\target \ed),  j = \idx(\brg \, \of \, i) \\
%     \lambda & i \in \idx(\ed),  j \in \idx(\parent)
% \end{cases}
% \end{align*}

% \begin{align*}
% S_{ij}^{(3)} &=
% \begin{cases}
%     \gamma \langle \vh_i^{(2)}, \vh_j^{(2)} \rangle + \eta & i = \idx(\query), \, j = \idx(\target \, \ed) \\
%     \eta & i = \idx(\query), \, j \in \idx(\child)
% \end{cases}
% \end{align*}


\subsection{Details of the mechanism of in-context two-hop reasoning}
\label{sec:illustration_mechanism}

% \Cref{fig:mechanism_illustration} shows the internal mechanism of a three-layer transformer to perform in-context two-hop reasoning.

In this section, we discuss the internal mechanism of a three-layer transformer to perform in-context two-hop reasoning in detail. For better visualization, we use illustrative attention maps and relegate the real attention map to Appendix~\ref{app:sec_DI_mechanism}.

\begin{figure}[h]
    \centering
    \includegraphics[width=0.7\linewidth]{figs/mlp_zero_out.pdf}
    \caption{Accuracies of the full model and the ablated models with skipped MLPs in different training stages.}
    \label{fig:mlp_zero_out}
\end{figure}

\paragraph{Attention-only model suffices to solve our in-context two-hop reasoning tasks.} We conducted ablation studies on the effect of MLP layers. \Cref{fig:mlp_zero_out} shows that even if we skip all MLP layers during inference, the model's performance on the validation set during different training stages remains nearly the same. Note that we kept all the MLP layers during the training, which provides strong evidence that MLP layers play nearly no role in the in-context two-hop reasoning tasks. Therefore, we mainly focus on attention-only transformers in this section. 

\begin{figure}[h]
    \centering
    \includegraphics[width=0.7\linewidth]{figs/acc_dynamics.pdf}
    \caption{Training Dynamics of the three-layer transformer.}
    \label{fig:acc_dynamics}
\end{figure}

\paragraph{Phase transition during training.} According to \Cref{fig:acc_dynamics}, the model mainly experienced two different phases during training. In the early stage (e.g., around training step 400), the model learned to predict an end token randomly (in \Cref{fig:acc_dynamics}, our data contains one target reasoning chain and one distracting reasoning chain, so the curve in \Cref{fig:acc_dynamics} shows that in the early stage, the model will predict the target end or non-target end with roughly equal probability). After a sharp phase transition after more than 800 training steps, the model learns to predict the correct answer perfectly. We call the interim mechanism that the model learned during the early stage \emph{uniform guessing mechanism}, and observed that the final mechanism the model learned to make the perfect prediction is a \emph{sequential query mechanism}. Besides, previous literature~\citep{sanford2024transformers} posited that the transformer performs in-context two-hop reasoning by a mechanism they theoretically constructed, which we called \emph{double induction head mechanism}. Below, we explain the uniform guessing mechanism and the sequential query mechanism layer by layer in detail. The details of the double induction head mechanism are deferred to \Cref{app:sec_DI_mechanism}.

\paragraph{The first layer.} For all the above three mechanisms, in the first layer, each child token pays all attention to its parent token by positional encoding and then copies its parent token to its buffer space to be used in subsequent layers. This is pictorially illustrated in \Cref{fig:mech:L1_copy}.

\begin{figure}[h]
    \centering
    \includegraphics[width=0.8\linewidth]{figs/mechanism/L1_copy.pdf}
    \caption{An illustration of the mechanism of the first layer. Each child token pays attention to its parent token and copies it to its buffer space.}
    \label{fig:mech:L1_copy}
\end{figure}

\begin{table}[h]
    \centering
    \renewcommand{\arraystretch}{1} % Adjust row height for better spacing
    \begin{tabular}{lccc}
        \toprule
        \textbf{Step} & \textit{Pre}-$\Brg$ & \textit{Self}-$\Ed$ & \textit{Post}-$\Brg$ \\
        \midrule
        $\mathbf{0}$ & 0.00 & 0.01 & 0.02 \\
        $\mathbf{400}$ & -1.55 & 7.14 & -0.02 \\
        $\mathbf{2000}$ & -4.80 & 11.50 & -0.05 \\
        \bottomrule
    \end{tabular}
    \caption{\textbf{The logit lens shows effect of \Ed~tokens' value states in layer 3 on the final output.} Through the logit lens, the value states become a logit score for the next token prediction. We track three groups of logits in the logit lens: The ``\textit{Pre}-\brg'' denotes the logits correspond to \brg~tokens precede the \Ed~token. The increasing logit score indicates the increasing suppression effect of the value states for \Ed~tokens to the prediction of \brg~tokens; The``\textit{Self}-\Ed'' denotes the logit corresponds to the \Ed~token itself, The increasing score indicates that \Ed's value states have increasing propensity to predict itself; The ``\textit{Post}-\brg'' denotes the \brg~tokens succeed the \Ed~token. The zero scores indicate that the suppression effect is formed in-context, not from memorization.}
    \label{tab:logit_lens_value}
\end{table}


\paragraph{The interim mechanism in early stage: uniform guessing.} During the early stage (e.g., around 400 steps), the model learns to randomly pick an end token by distinguishing between the end and bridge tokens among all child tokens. The underlying mechanism we observed is as follows. In the second layer, as shown in \Cref{fig:mech:L2_guess}, each child token will attend equally to all previous parent tokens and copy (the superposition of) them to its buffer.  The buffer space will later be used in the last layer, where the query entity (i.e., the last token in the input sequence) will attend equally to all child tokens as shown in \Cref{fig:mech:L3_guess}.  Thanks to the information on the buffer space collected in the second layer, the value state of each copied child token through the logit lens contains not only itself but also all parent tokens before that child token with a negative sign in the corresponding coordinate (\Cref{fig:mech:logit_lens_guess}). By aggregating the value states of all child tokens in the last layer, the query entity can distinguish the end token from the bridge token since the positive value of the bridge token due to $\brga$, a child token, is canceled out by the negative value caused by $\brgb$, a parent token, in the same corresponding coordinate. As a result, this interim mechanism learned in the early stage can uniformly guess an end entity as its prediction.

\begin{figure}[h]
    \centering
    \includegraphics[width=0.7\linewidth]{figs/mechanism/L2_guess.pdf}
    \caption{The uniform guessing mechanism in the second layer. Each child token pays attention to all the previous parent tokens.}
    \label{fig:mech:L2_guess}
\end{figure}

\begin{figure}[h]
    \centering
    \includegraphics[width=0.8\linewidth]{figs/mechanism/L3_guess.pdf}
    \caption{The uniform guessing mechanism in the third layer. The query entity pays attention to all the child tokens.}
    \label{fig:mech:L3_guess}
\end{figure}

\begin{figure}[h]
    \centering
    \includegraphics[width=\linewidth]{figs/mechanism/logit_lens_guess.pdf}
    \caption{Value states of the last layer through logit lens for the uniform guessing mechanism. The value of the coordinate of bridge tokens will cancel out after aggregating, which enables the query token to distinguish between the bridge entity and the end entity.}
    \label{fig:mech:logit_lens_guess}
\end{figure}

\paragraph{The (observed) sequential query mechanism after phase transition.} After the phase transition during training, the model achieves nearly perfect accuracy, and we observed a sequential query mechanism during that stage. After copying parent entities in the first layer (\Cref{fig:mech:L1_copy}), in the second layer, the query entity will pay attention to the bridge entity whose corresponding source entity in the buffer space matches the query entity. Then, it copies this bridge entity to the query entity's buffer space (\Cref{fig:mech:L2_sequential}). In the last layer, the query entity uses the collected bridge entity from the last layer to do the query again and obtains the corresponding end entity, which is exactly the expected answer (\Cref{fig:mech:L3_sequential}). Note that in a $k$-hop reasoning setting, the above sequential query mechanism performs one hop per layer and thus requires $(k+1)$ layers.

\begin{figure}[h]
    \centering
    \includegraphics[width=0.5\linewidth]{figs/mechanism/L2_sequential.pdf}
    \caption{Sequential query mechanism in the second layer. The query entity pays attention to the target \brga  entity.}
    \label{fig:mech:L2_sequential}
\end{figure}

\begin{figure}[h]
    \centering
    \includegraphics[width=0.5\linewidth]{figs/mechanism/L3_sequential.pdf}
    \caption{Sequential query mechanism in the third layer.}
    \label{fig:mech:L3_sequential}
\end{figure}

\paragraph{The model is prone to learn the sequential query mechanism.} Although the double induction head mechanism is theoretically optimal in terms of the number of layers required to perform multi-hop reasoning tasks (see \Cref{app:sec_DI_mechanism} for details), even for the simplest two-hop reasoning, the model tends to learn the sequential query mechanism according to our observations. To further understand why the model prefers the seemingly less efficient sequential query mechanism, as well as why the model abruptly learned how to solve the task during the rapid phase transition, we study a three-parameter model that fully captures the training dynamics of the three-layer transformers on our two-hop reasoning settings in \Cref{sec:three_param_model}.

\subsection{Training dynamics of the three-parameter model}
\label{sec:three_param_model}
Although the dynamics of the 3-layer transformer can be fully interpreted through uniform guessing and sequential querying, our claims are primarily based on empirical observations. To provide further evidence and gain a deeper understanding of the phase transition in the form of sequential querying, we propose studying a \textit{three-parameter dynamical system}.

Following the approach of \citet{reddy2023mechanistic}, we use the three-parameter model to simulate the dynamics of two potential mechanisms: the sequential query mechanism and the double induction head mechanism. If the dynamics of the three-parameter model match with 3-layer transformers, we gain strong evidence that the dynamics of the 3-layer transformer are truly driven by the proposed mechanism. For convenience, we define the approximate softmax operator as:
\[
\pseudosoftmax(u, M) = \frac{\exp(u)}{\exp(u)+M}.
\]
Intuitively, the approximate softmax gives the probability of an item with logit $u$ where the remaining $M$ logits are all zero. It is useful especially when we consider the ``average attention weights'' from all child to parent tokens. 

Given a residual state $u$, we use $\content(u)$, $\buffer_1(u)$, $\buffer_2(u)$ to denote the original content (i.e., the token embedding and the positional embedding) of token $u$, the buffer space of $u$ in the first layer, and the buffer space of $u$ in the second layer, respectively.

We first consider the next-token prediction \logit~on the query token. In the logit lens, as illustrated in \Cref{tab:logit_lens_value}, the value states of \Ed~tokens have large logits on itself. Therefore, we assume that $\Val(\target-\Ed) = \xi \cdot \bm{e}_{\Ed} \in \R^V$, with $\xi>0$ and $\bm{e}_{\Ed}$ being a one-hot vector in $\R^V$ that is non-zero on the index of $\Ed$. In our simulation, we fix $\xi=30$. Additionally, the attention from the query token increasingly concentrates on the target-\Ed~token along the training dynamics.  We approximate the attention weight from the query to the \target-\Ed~with $\pseudosoftmax(\attlogit(\query\to\target-\Ed), 2N)$.
Ignoring other irrelevant terms, the loss can therefore be approximated through
\begin{equation}\small
\text{Loss} = -\log(\pseudosoftmax(\attlogit(\query\to\target-\Ed)) \xi). \label{eqn:loss}
\end{equation}

To model the \attlogit, recall that the attention logit between tokens $h$ and $u$ is given by $h^\top K^{(\ell)}Q^{(\ell)} u$, with $K^{(\ell)}$, $Q^{(\ell)}$ being the weights of the key, query matrices in layer $\ell$. We rescale the $K^{(\ell)}Q^{(\ell)}$ matrices so that 
$$\alpha=\|K^{(1)}Q^{(1)}\|_2, \beta=\|K^{(2)}Q^{(2)}\|_2, \gamma=\|K^{(3)}Q^{(3)}\|_2.$$ 
We only set $\alpha$, $\beta$, and $\gamma$ as the three trainable models, which reflects the evolution of the weights in transformers.
Following the mechanism illustrated in \Cref{fig:mech:L3_sequential}, $\attlogit(\query\to\target-\Ed)$ is given by the inner product between $\buffer_2(\query)$ and $\buffer_1(\target-\Ed)$, scaled by $\gamma$. We therefore set that
\begin{align}
~& \attlogit(\query\to\target-\Ed) \nonumber \nonumber \\
= ~& \gamma \cdot \langle \buffer_2(\query), \buffer_1(\target-\Ed)\rangle.\label{eqn:gamma}
\end{align}
Similarly, as illustrated in \Cref{fig:mech:L2_sequential}, the $\buffer_2(\query)$ is proportional to the attention from query token to $\brga$ in the second layer. The query token uses its $\content(\query)$ to fit the $\buffer_1(\brga)$, copying $\content(\brg)$ to the residual stream. Therefore, we set $\buffer_2(\query)=\pseudosoftmax(\attlogit(\query\to\brga), 2N)\cdot \content(\brg)$, with 
\begin{align} 
 ~& \attlogit(\query\to\brga) \nonumber \\
= ~& \beta \langle \buffer_1(\brga), \content(\brg) \rangle. \label{eqn:beta}
\end{align}
The $\buffer_1(\cdot)$ is merely given by the copying head in the first layer, which purely relies on the positional information. We assume that the \attlogit~is only affected by the scale of the $Q^{(1)}K^{(1)}$. Therefore, for any $\child$ and $\parent$, we assume
\begin{equation}\label{eqn:alpha}
\attlogit(\child\to\parent)=\alpha. 
\end{equation}
Since the child tokens uniformly have $1$ to $2N-1$ tokens ($N$ being the number of premises) in front of them, but each of them has the same attention logit $\alpha$ with their parent, we can use $\pseudosoftmax(\alpha, N)$ to approximate the average attention weights for copying. We therefore set $\buffer_1(\brga)=\pseudosoftmax(\alpha, N)\content(\Src)$ and $\buffer_1(\Ed)=\pseudosoftmax(\alpha, N)\content(\brg)$. At last, since the task is entirely in-context, there is no fixed relationship between any tokens. We assume that $\langle\content(a),\content(b)\rangle = \bm{1}(a=b)$, which means that $\{\content(\cdot)\}$ is an orthonormal basis. By combining Equations~\eqref{eqn:gamma}, \eqref{eqn:beta}, and \eqref{eqn:alpha}, and \eqref{eqn:loss}, we get a model with three parameters $\alpha$, $\beta$, and $\gamma$ that simulates the sequential query mechanism. We optimize the loss function in Eq.~\eqref{eqn:loss} by updating $\alpha$, $\beta$, and $\gamma$ through gradient descent. 

\begin{figure}
    \centering
    \begin{subfigure}[t]{0.23\textwidth}
        \centering 
        \caption{\small Loss}
\includegraphics[width=\textwidth]{figs/loss_3.png}
    \end{subfigure}
    \begin{subfigure}[t]{0.23\textwidth}
    \centering
        \subcaption{Parameters}
\includegraphics[width=\textwidth]{figs/param_3.png}
    \end{subfigure}
    \caption{\textbf{The simulations results of the 3-parameter model} \textit{Left (a)}:  The loss function remains unchanged until a sudden phase transition occurs, after which it stabilizes at zero. \textit{Right (b)}: The parameters also went through a sudden phase transition around the step 1000.}
    \label{fig:3_param_dnamics}
\end{figure}

\Cref{fig:3_param_dnamics} presents the simulation results of the 3-parameter model. Since the model does not incorporate the random guessing mechanism, the loss remains unchanged during the first 1000 steps. This supports the hypothesis that random guessing contributes to the slow learning phase observed in the dynamics of the three-layer transformer.
Both the parameters and the loss function go through a sudden phase transition around step 1000, suggesting that the emergence of the sequential query mechanism is the driving force behind the abrupt drop in loss.

% \subsection{Experimental results}

% \hanlin{The following part might be deleted, and we will simply use the above formula.}

% The input of each layer $l$ is a sequence of embedding vectors $\mH^{(l-1)} = [\vh^{(l-1)}_1, \vh^{(l-1)}_2, \ldots, \vh^{(l-1)}_T ] \in \real^{d\times T}$, and the output is also a sequence of embeddings $\mH^{(l)} = [\vh^{(l)}_1, \vh^{(l)}_2, \ldots, \vh^{(l)}_T ] \in \real^{d\times T}$. The final output is $\softmax(\Wreadout\vh_T^{(L)}) \in \Delta(\vocab)$ where $\Wreadout \in \real^{V \times d}$ is the readout matrix. \todo{might want to use a restricted readout function to reduce the number of parameters}

% \paragraph{The first layer.} In the first layer, each child node pays attention to its parent node and copies its value state. As a result, we assume that
% \begin{equation}
% \label{eq:restricted_attn_layer_1}
% \begin{aligned}
%     \vh^{(1)}_i = \vh^{(0)}_i + \indicator\{ i > 1, i \text{ is odd} \} \cdot \Val^{(1)}(\vh^{(0)}_{i-1}). 
% \end{aligned}
% \end{equation}

% \paragraph{The second layer.} The second layer is the most interesting part and is the focus of our analysis. There is an interim mechanism appeared in the early training stage ($\beta_0$), and two candidate mechanisms after phase transition: one is the sequential query mechanism observed in our experiments ($\beta_1$), and the other was posited by previous literature ($\beta_2$). Mathematically, we have
% \begin{equation}
% \label{eq:restricted_attn_layer_2}
% \begin{aligned}
% &\Key^{(2)}(\vh^{(1)}_i) \cdot \Qry^{(2)}(\vh^{(1)}_j)  \\ 
%      = & \beta_0 \cdot \indicator\{i \text{ is even, } j \text{ is odd, } i < j \} \\
%     & + \beta_1 \cdot \indicator\{s_{i-1}=s_j, i+1 <j=T \text{ and } i \text{ is odd } \}
%     \\  & + \beta_2 \cdot \indicator\{s_{i-1}=s_j, i<j \text{ and } i, j \text{ are odd }\}
% \end{aligned}
% \end{equation}

% \paragraph{The last layer.} In the last layer, we only need to keep track of $\vh^{(3)}_T$. In either the sequential query mechanism or binary lifting mechanism, 
% \begin{equation}
% \label{eq:restricted_attn_layer_3}
% \begin{aligned}
% &\Key^{(3)}(\Val^{(2)}(\Val^{(1)}(\vh^{(0)}_i)) \cdot \Qry^{(3)}(\vh^{(0)}_T)   =  \indicator\{s_i = s_T\} \\ 
%      & \beta_0 \cdot \indicator\{i \text{ is even, } j \text{ is odd, } i < j \} \\
%     & + \beta_1 \cdot \indicator\{s_{i-1}=s_j, i+1 <j=T \text{ and } i \text{ is odd } \}
%     \\  & + \beta_2 \cdot \indicator\{s_{i-1}=s_j, i<j \text{ and } i, j \text{ are odd }\}
% \end{aligned}
% \end{equation}
\section{Implications for Large Language Models}
\label{sec:impl_llm}
In this section, we discuss the implications of the mechanisms discovered from three-layer transformers in real-world large language models. We use Llama2-7B-base, a widely used pre-trained language model, for all the experiments in this section.

\subsection{Experimental setting}
\label{sec:llama-initial}
We evaluate the in-context two-hop reasoning performance of LLaMA2-7B. To this end, we design fixed templates for in-context reasoning chains that cover various topics. For example, a template may take the form: ”[A] is a species in the genus [B]. The genus [B] belongs to the family [C].” Given multiple two-hop chains, we split each template into two premises and randomly order them according to the data generation procedure illustrated in \Cref{fig:data_illustration}.
We then append a query sentence, such as “Therefore, [A] is classified under the family,” after the context. Finally, we randomly sample from a set of artificial names to replace placeholders [A], [B], and so on. We perform a forward pass on the entire sequence and compute the next-token probability for the last token—in this example, “family.” Further details on the dataset are provided in Appendix\ref{app:dataset_detail}.

We set the number of two-hop chains to  K=2  and compute the next-token probability. We track the probability corresponding to the target \Ed, which is the correct answer. When \Ed~consists of multiple tokens, we consider only the first token. \Cref{fig:llm-onehop} compares the probability of LLaMA2-7B predicting the correct answer with and without distracting information. The results indicate a consistent reduction in probability by half across different template categories.

\begin{figure}[h]
    \centering
    \includegraphics[width=0.9\linewidth]{figs/llama_onehop.pdf}
    \caption{\textbf{Two-hop reasoning with distractions} When exposed to a single distraction, the probability of LLaMA2-7B predicting the correct target \Ed~drops by half. This decrease is consistent across different categories of the two-hop reasoning format.}
    \label{fig:llm-onehop}
\end{figure}

\subsection{Length generalization}
\label{subsec:length_gen}

According to our analysis of the training dynamics of three-layer transformers in the previous section, we observed two mechanisms: the random guessing mechanism during the early stage and the sequential query mechanism after the sharp phase transition. Below, we study which mechanism Llama2-7B uses to solve the two-hop reasoning task with distracting reasoning chains. 

\Cref{fig:llm-length-gen} shows the model's performance on in-context two-hop reasoning with different numbers of distracting chains in the prompt, both before and after fine-tuning. 

For the original pre-trained Llama2-7B model without fine-tuning, although it can make correct predictions when there is no distracting chain, its performance drops drastically as long as there exist distracting chains. Moreover, the probability of the model predicting the target end is close to the average probability of predicting any specific distracting end, which shows that the model adopts nearly uniform random guessing.  

In contrast, after fine-tuning, the model robustly makes the correct prediction. The most remarkable phenomenon is that although the model is fine-tuned on data with only two two-hop reasoning chains (one target chain and one distracting chain), the model still makes correct predictions with high probability, even if the number of chains in the prompt is as large as five. Compared to the original model, it is salient that the original model performs random guessing while the fine-tuned model learns the correct mechanism to solve the in-context two-hop reasoning problem according to its ability on length generalization.

\begin{figure}[h]
    \centering
    \includegraphics[width=\linewidth]{figs/llama_length_gen.pdf}
    \caption{The plot shows Llama's performance on in-context two-hop reasoning with various numbers of distracting reasoning chains in the prompt both before and after fine-tuning. Each point in the plot is averaged over 1000 samples. The $x$-axis represents the number of two-hop reasoning chains in the prompt. The $y$-axis represents either the probability of the model predicting the correct answer, i.e., the target end entity, or the average probability of predicting any specific distracting end entity. The fine-tuned model is trained on data with only two reasoning chains in the prompt (one target chain and one distracting chain).}
    \label{fig:llm-length-gen}
\end{figure}
\section{Conclusion}
In this work, we propose a simple yet effective approach, called SMILE, for graph few-shot learning with fewer tasks. Specifically, we introduce a novel dual-level mixup strategy, including within-task and across-task mixup, for enriching the diversity of nodes within each task and the diversity of tasks. Also, we incorporate the degree-based prior information to learn expressive node embeddings. Theoretically, we prove that SMILE effectively enhances the model's generalization performance. Empirically, we conduct extensive experiments on multiple benchmarks and the results suggest that SMILE significantly outperforms other baselines, including both in-domain and cross-domain few-shot settings.


\bibliographystyle{plainnat}
\bibliography{main_arxiv}

\newpage
\appendix
\onecolumn
\subsection{Lloyd-Max Algorithm}
\label{subsec:Lloyd-Max}
For a given quantization bitwidth $B$ and an operand $\bm{X}$, the Lloyd-Max algorithm finds $2^B$ quantization levels $\{\hat{x}_i\}_{i=1}^{2^B}$ such that quantizing $\bm{X}$ by rounding each scalar in $\bm{X}$ to the nearest quantization level minimizes the quantization MSE. 

The algorithm starts with an initial guess of quantization levels and then iteratively computes quantization thresholds $\{\tau_i\}_{i=1}^{2^B-1}$ and updates quantization levels $\{\hat{x}_i\}_{i=1}^{2^B}$. Specifically, at iteration $n$, thresholds are set to the midpoints of the previous iteration's levels:
\begin{align*}
    \tau_i^{(n)}=\frac{\hat{x}_i^{(n-1)}+\hat{x}_{i+1}^{(n-1)}}2 \text{ for } i=1\ldots 2^B-1
\end{align*}
Subsequently, the quantization levels are re-computed as conditional means of the data regions defined by the new thresholds:
\begin{align*}
    \hat{x}_i^{(n)}=\mathbb{E}\left[ \bm{X} \big| \bm{X}\in [\tau_{i-1}^{(n)},\tau_i^{(n)}] \right] \text{ for } i=1\ldots 2^B
\end{align*}
where to satisfy boundary conditions we have $\tau_0=-\infty$ and $\tau_{2^B}=\infty$. The algorithm iterates the above steps until convergence.

Figure \ref{fig:lm_quant} compares the quantization levels of a $7$-bit floating point (E3M3) quantizer (left) to a $7$-bit Lloyd-Max quantizer (right) when quantizing a layer of weights from the GPT3-126M model at a per-tensor granularity. As shown, the Lloyd-Max quantizer achieves substantially lower quantization MSE. Further, Table \ref{tab:FP7_vs_LM7} shows the superior perplexity achieved by Lloyd-Max quantizers for bitwidths of $7$, $6$ and $5$. The difference between the quantizers is clear at 5 bits, where per-tensor FP quantization incurs a drastic and unacceptable increase in perplexity, while Lloyd-Max quantization incurs a much smaller increase. Nevertheless, we note that even the optimal Lloyd-Max quantizer incurs a notable ($\sim 1.5$) increase in perplexity due to the coarse granularity of quantization. 

\begin{figure}[h]
  \centering
  \includegraphics[width=0.7\linewidth]{sections/figures/LM7_FP7.pdf}
  \caption{\small Quantization levels and the corresponding quantization MSE of Floating Point (left) vs Lloyd-Max (right) Quantizers for a layer of weights in the GPT3-126M model.}
  \label{fig:lm_quant}
\end{figure}

\begin{table}[h]\scriptsize
\begin{center}
\caption{\label{tab:FP7_vs_LM7} \small Comparing perplexity (lower is better) achieved by floating point quantizers and Lloyd-Max quantizers on a GPT3-126M model for the Wikitext-103 dataset.}
\begin{tabular}{c|cc|c}
\hline
 \multirow{2}{*}{\textbf{Bitwidth}} & \multicolumn{2}{|c|}{\textbf{Floating-Point Quantizer}} & \textbf{Lloyd-Max Quantizer} \\
 & Best Format & Wikitext-103 Perplexity & Wikitext-103 Perplexity \\
\hline
7 & E3M3 & 18.32 & 18.27 \\
6 & E3M2 & 19.07 & 18.51 \\
5 & E4M0 & 43.89 & 19.71 \\
\hline
\end{tabular}
\end{center}
\end{table}

\subsection{Proof of Local Optimality of LO-BCQ}
\label{subsec:lobcq_opt_proof}
For a given block $\bm{b}_j$, the quantization MSE during LO-BCQ can be empirically evaluated as $\frac{1}{L_b}\lVert \bm{b}_j- \bm{\hat{b}}_j\rVert^2_2$ where $\bm{\hat{b}}_j$ is computed from equation (\ref{eq:clustered_quantization_definition}) as $C_{f(\bm{b}_j)}(\bm{b}_j)$. Further, for a given block cluster $\mathcal{B}_i$, we compute the quantization MSE as $\frac{1}{|\mathcal{B}_{i}|}\sum_{\bm{b} \in \mathcal{B}_{i}} \frac{1}{L_b}\lVert \bm{b}- C_i^{(n)}(\bm{b})\rVert^2_2$. Therefore, at the end of iteration $n$, we evaluate the overall quantization MSE $J^{(n)}$ for a given operand $\bm{X}$ composed of $N_c$ block clusters as:
\begin{align*}
    \label{eq:mse_iter_n}
    J^{(n)} = \frac{1}{N_c} \sum_{i=1}^{N_c} \frac{1}{|\mathcal{B}_{i}^{(n)}|}\sum_{\bm{v} \in \mathcal{B}_{i}^{(n)}} \frac{1}{L_b}\lVert \bm{b}- B_i^{(n)}(\bm{b})\rVert^2_2
\end{align*}

At the end of iteration $n$, the codebooks are updated from $\mathcal{C}^{(n-1)}$ to $\mathcal{C}^{(n)}$. However, the mapping of a given vector $\bm{b}_j$ to quantizers $\mathcal{C}^{(n)}$ remains as  $f^{(n)}(\bm{b}_j)$. At the next iteration, during the vector clustering step, $f^{(n+1)}(\bm{b}_j)$ finds new mapping of $\bm{b}_j$ to updated codebooks $\mathcal{C}^{(n)}$ such that the quantization MSE over the candidate codebooks is minimized. Therefore, we obtain the following result for $\bm{b}_j$:
\begin{align*}
\frac{1}{L_b}\lVert \bm{b}_j - C_{f^{(n+1)}(\bm{b}_j)}^{(n)}(\bm{b}_j)\rVert^2_2 \le \frac{1}{L_b}\lVert \bm{b}_j - C_{f^{(n)}(\bm{b}_j)}^{(n)}(\bm{b}_j)\rVert^2_2
\end{align*}

That is, quantizing $\bm{b}_j$ at the end of the block clustering step of iteration $n+1$ results in lower quantization MSE compared to quantizing at the end of iteration $n$. Since this is true for all $\bm{b} \in \bm{X}$, we assert the following:
\begin{equation}
\begin{split}
\label{eq:mse_ineq_1}
    \tilde{J}^{(n+1)} &= \frac{1}{N_c} \sum_{i=1}^{N_c} \frac{1}{|\mathcal{B}_{i}^{(n+1)}|}\sum_{\bm{b} \in \mathcal{B}_{i}^{(n+1)}} \frac{1}{L_b}\lVert \bm{b} - C_i^{(n)}(b)\rVert^2_2 \le J^{(n)}
\end{split}
\end{equation}
where $\tilde{J}^{(n+1)}$ is the the quantization MSE after the vector clustering step at iteration $n+1$.

Next, during the codebook update step (\ref{eq:quantizers_update}) at iteration $n+1$, the per-cluster codebooks $\mathcal{C}^{(n)}$ are updated to $\mathcal{C}^{(n+1)}$ by invoking the Lloyd-Max algorithm \citep{Lloyd}. We know that for any given value distribution, the Lloyd-Max algorithm minimizes the quantization MSE. Therefore, for a given vector cluster $\mathcal{B}_i$ we obtain the following result:

\begin{equation}
    \frac{1}{|\mathcal{B}_{i}^{(n+1)}|}\sum_{\bm{b} \in \mathcal{B}_{i}^{(n+1)}} \frac{1}{L_b}\lVert \bm{b}- C_i^{(n+1)}(\bm{b})\rVert^2_2 \le \frac{1}{|\mathcal{B}_{i}^{(n+1)}|}\sum_{\bm{b} \in \mathcal{B}_{i}^{(n+1)}} \frac{1}{L_b}\lVert \bm{b}- C_i^{(n)}(\bm{b})\rVert^2_2
\end{equation}

The above equation states that quantizing the given block cluster $\mathcal{B}_i$ after updating the associated codebook from $C_i^{(n)}$ to $C_i^{(n+1)}$ results in lower quantization MSE. Since this is true for all the block clusters, we derive the following result: 
\begin{equation}
\begin{split}
\label{eq:mse_ineq_2}
     J^{(n+1)} &= \frac{1}{N_c} \sum_{i=1}^{N_c} \frac{1}{|\mathcal{B}_{i}^{(n+1)}|}\sum_{\bm{b} \in \mathcal{B}_{i}^{(n+1)}} \frac{1}{L_b}\lVert \bm{b}- C_i^{(n+1)}(\bm{b})\rVert^2_2  \le \tilde{J}^{(n+1)}   
\end{split}
\end{equation}

Following (\ref{eq:mse_ineq_1}) and (\ref{eq:mse_ineq_2}), we find that the quantization MSE is non-increasing for each iteration, that is, $J^{(1)} \ge J^{(2)} \ge J^{(3)} \ge \ldots \ge J^{(M)}$ where $M$ is the maximum number of iterations. 
%Therefore, we can say that if the algorithm converges, then it must be that it has converged to a local minimum. 
\hfill $\blacksquare$


\begin{figure}
    \begin{center}
    \includegraphics[width=0.5\textwidth]{sections//figures/mse_vs_iter.pdf}
    \end{center}
    \caption{\small NMSE vs iterations during LO-BCQ compared to other block quantization proposals}
    \label{fig:nmse_vs_iter}
\end{figure}

Figure \ref{fig:nmse_vs_iter} shows the empirical convergence of LO-BCQ across several block lengths and number of codebooks. Also, the MSE achieved by LO-BCQ is compared to baselines such as MXFP and VSQ. As shown, LO-BCQ converges to a lower MSE than the baselines. Further, we achieve better convergence for larger number of codebooks ($N_c$) and for a smaller block length ($L_b$), both of which increase the bitwidth of BCQ (see Eq \ref{eq:bitwidth_bcq}).


\subsection{Additional Accuracy Results}
%Table \ref{tab:lobcq_config} lists the various LOBCQ configurations and their corresponding bitwidths.
\begin{table}
\setlength{\tabcolsep}{4.75pt}
\begin{center}
\caption{\label{tab:lobcq_config} Various LO-BCQ configurations and their bitwidths.}
\begin{tabular}{|c||c|c|c|c||c|c||c|} 
\hline
 & \multicolumn{4}{|c||}{$L_b=8$} & \multicolumn{2}{|c||}{$L_b=4$} & $L_b=2$ \\
 \hline
 \backslashbox{$L_A$\kern-1em}{\kern-1em$N_c$} & 2 & 4 & 8 & 16 & 2 & 4 & 2 \\
 \hline
 64 & 4.25 & 4.375 & 4.5 & 4.625 & 4.375 & 4.625 & 4.625\\
 \hline
 32 & 4.375 & 4.5 & 4.625& 4.75 & 4.5 & 4.75 & 4.75 \\
 \hline
 16 & 4.625 & 4.75& 4.875 & 5 & 4.75 & 5 & 5 \\
 \hline
\end{tabular}
\end{center}
\end{table}

%\subsection{Perplexity achieved by various LO-BCQ configurations on Wikitext-103 dataset}

\begin{table} \centering
\begin{tabular}{|c||c|c|c|c||c|c||c|} 
\hline
 $L_b \rightarrow$& \multicolumn{4}{c||}{8} & \multicolumn{2}{c||}{4} & 2\\
 \hline
 \backslashbox{$L_A$\kern-1em}{\kern-1em$N_c$} & 2 & 4 & 8 & 16 & 2 & 4 & 2  \\
 %$N_c \rightarrow$ & 2 & 4 & 8 & 16 & 2 & 4 & 2 \\
 \hline
 \hline
 \multicolumn{8}{c}{GPT3-1.3B (FP32 PPL = 9.98)} \\ 
 \hline
 \hline
 64 & 10.40 & 10.23 & 10.17 & 10.15 &  10.28 & 10.18 & 10.19 \\
 \hline
 32 & 10.25 & 10.20 & 10.15 & 10.12 &  10.23 & 10.17 & 10.17 \\
 \hline
 16 & 10.22 & 10.16 & 10.10 & 10.09 &  10.21 & 10.14 & 10.16 \\
 \hline
  \hline
 \multicolumn{8}{c}{GPT3-8B (FP32 PPL = 7.38)} \\ 
 \hline
 \hline
 64 & 7.61 & 7.52 & 7.48 &  7.47 &  7.55 &  7.49 & 7.50 \\
 \hline
 32 & 7.52 & 7.50 & 7.46 &  7.45 &  7.52 &  7.48 & 7.48  \\
 \hline
 16 & 7.51 & 7.48 & 7.44 &  7.44 &  7.51 &  7.49 & 7.47  \\
 \hline
\end{tabular}
\caption{\label{tab:ppl_gpt3_abalation} Wikitext-103 perplexity across GPT3-1.3B and 8B models.}
\end{table}

\begin{table} \centering
\begin{tabular}{|c||c|c|c|c||} 
\hline
 $L_b \rightarrow$& \multicolumn{4}{c||}{8}\\
 \hline
 \backslashbox{$L_A$\kern-1em}{\kern-1em$N_c$} & 2 & 4 & 8 & 16 \\
 %$N_c \rightarrow$ & 2 & 4 & 8 & 16 & 2 & 4 & 2 \\
 \hline
 \hline
 \multicolumn{5}{|c|}{Llama2-7B (FP32 PPL = 5.06)} \\ 
 \hline
 \hline
 64 & 5.31 & 5.26 & 5.19 & 5.18  \\
 \hline
 32 & 5.23 & 5.25 & 5.18 & 5.15  \\
 \hline
 16 & 5.23 & 5.19 & 5.16 & 5.14  \\
 \hline
 \multicolumn{5}{|c|}{Nemotron4-15B (FP32 PPL = 5.87)} \\ 
 \hline
 \hline
 64  & 6.3 & 6.20 & 6.13 & 6.08  \\
 \hline
 32  & 6.24 & 6.12 & 6.07 & 6.03  \\
 \hline
 16  & 6.12 & 6.14 & 6.04 & 6.02  \\
 \hline
 \multicolumn{5}{|c|}{Nemotron4-340B (FP32 PPL = 3.48)} \\ 
 \hline
 \hline
 64 & 3.67 & 3.62 & 3.60 & 3.59 \\
 \hline
 32 & 3.63 & 3.61 & 3.59 & 3.56 \\
 \hline
 16 & 3.61 & 3.58 & 3.57 & 3.55 \\
 \hline
\end{tabular}
\caption{\label{tab:ppl_llama7B_nemo15B} Wikitext-103 perplexity compared to FP32 baseline in Llama2-7B and Nemotron4-15B, 340B models}
\end{table}

%\subsection{Perplexity achieved by various LO-BCQ configurations on MMLU dataset}


\begin{table} \centering
\begin{tabular}{|c||c|c|c|c||c|c|c|c|} 
\hline
 $L_b \rightarrow$& \multicolumn{4}{c||}{8} & \multicolumn{4}{c||}{8}\\
 \hline
 \backslashbox{$L_A$\kern-1em}{\kern-1em$N_c$} & 2 & 4 & 8 & 16 & 2 & 4 & 8 & 16  \\
 %$N_c \rightarrow$ & 2 & 4 & 8 & 16 & 2 & 4 & 2 \\
 \hline
 \hline
 \multicolumn{5}{|c|}{Llama2-7B (FP32 Accuracy = 45.8\%)} & \multicolumn{4}{|c|}{Llama2-70B (FP32 Accuracy = 69.12\%)} \\ 
 \hline
 \hline
 64 & 43.9 & 43.4 & 43.9 & 44.9 & 68.07 & 68.27 & 68.17 & 68.75 \\
 \hline
 32 & 44.5 & 43.8 & 44.9 & 44.5 & 68.37 & 68.51 & 68.35 & 68.27  \\
 \hline
 16 & 43.9 & 42.7 & 44.9 & 45 & 68.12 & 68.77 & 68.31 & 68.59  \\
 \hline
 \hline
 \multicolumn{5}{|c|}{GPT3-22B (FP32 Accuracy = 38.75\%)} & \multicolumn{4}{|c|}{Nemotron4-15B (FP32 Accuracy = 64.3\%)} \\ 
 \hline
 \hline
 64 & 36.71 & 38.85 & 38.13 & 38.92 & 63.17 & 62.36 & 63.72 & 64.09 \\
 \hline
 32 & 37.95 & 38.69 & 39.45 & 38.34 & 64.05 & 62.30 & 63.8 & 64.33  \\
 \hline
 16 & 38.88 & 38.80 & 38.31 & 38.92 & 63.22 & 63.51 & 63.93 & 64.43  \\
 \hline
\end{tabular}
\caption{\label{tab:mmlu_abalation} Accuracy on MMLU dataset across GPT3-22B, Llama2-7B, 70B and Nemotron4-15B models.}
\end{table}


%\subsection{Perplexity achieved by various LO-BCQ configurations on LM evaluation harness}

\begin{table} \centering
\begin{tabular}{|c||c|c|c|c||c|c|c|c|} 
\hline
 $L_b \rightarrow$& \multicolumn{4}{c||}{8} & \multicolumn{4}{c||}{8}\\
 \hline
 \backslashbox{$L_A$\kern-1em}{\kern-1em$N_c$} & 2 & 4 & 8 & 16 & 2 & 4 & 8 & 16  \\
 %$N_c \rightarrow$ & 2 & 4 & 8 & 16 & 2 & 4 & 2 \\
 \hline
 \hline
 \multicolumn{5}{|c|}{Race (FP32 Accuracy = 37.51\%)} & \multicolumn{4}{|c|}{Boolq (FP32 Accuracy = 64.62\%)} \\ 
 \hline
 \hline
 64 & 36.94 & 37.13 & 36.27 & 37.13 & 63.73 & 62.26 & 63.49 & 63.36 \\
 \hline
 32 & 37.03 & 36.36 & 36.08 & 37.03 & 62.54 & 63.51 & 63.49 & 63.55  \\
 \hline
 16 & 37.03 & 37.03 & 36.46 & 37.03 & 61.1 & 63.79 & 63.58 & 63.33  \\
 \hline
 \hline
 \multicolumn{5}{|c|}{Winogrande (FP32 Accuracy = 58.01\%)} & \multicolumn{4}{|c|}{Piqa (FP32 Accuracy = 74.21\%)} \\ 
 \hline
 \hline
 64 & 58.17 & 57.22 & 57.85 & 58.33 & 73.01 & 73.07 & 73.07 & 72.80 \\
 \hline
 32 & 59.12 & 58.09 & 57.85 & 58.41 & 73.01 & 73.94 & 72.74 & 73.18  \\
 \hline
 16 & 57.93 & 58.88 & 57.93 & 58.56 & 73.94 & 72.80 & 73.01 & 73.94  \\
 \hline
\end{tabular}
\caption{\label{tab:mmlu_abalation} Accuracy on LM evaluation harness tasks on GPT3-1.3B model.}
\end{table}

\begin{table} \centering
\begin{tabular}{|c||c|c|c|c||c|c|c|c|} 
\hline
 $L_b \rightarrow$& \multicolumn{4}{c||}{8} & \multicolumn{4}{c||}{8}\\
 \hline
 \backslashbox{$L_A$\kern-1em}{\kern-1em$N_c$} & 2 & 4 & 8 & 16 & 2 & 4 & 8 & 16  \\
 %$N_c \rightarrow$ & 2 & 4 & 8 & 16 & 2 & 4 & 2 \\
 \hline
 \hline
 \multicolumn{5}{|c|}{Race (FP32 Accuracy = 41.34\%)} & \multicolumn{4}{|c|}{Boolq (FP32 Accuracy = 68.32\%)} \\ 
 \hline
 \hline
 64 & 40.48 & 40.10 & 39.43 & 39.90 & 69.20 & 68.41 & 69.45 & 68.56 \\
 \hline
 32 & 39.52 & 39.52 & 40.77 & 39.62 & 68.32 & 67.43 & 68.17 & 69.30  \\
 \hline
 16 & 39.81 & 39.71 & 39.90 & 40.38 & 68.10 & 66.33 & 69.51 & 69.42  \\
 \hline
 \hline
 \multicolumn{5}{|c|}{Winogrande (FP32 Accuracy = 67.88\%)} & \multicolumn{4}{|c|}{Piqa (FP32 Accuracy = 78.78\%)} \\ 
 \hline
 \hline
 64 & 66.85 & 66.61 & 67.72 & 67.88 & 77.31 & 77.42 & 77.75 & 77.64 \\
 \hline
 32 & 67.25 & 67.72 & 67.72 & 67.00 & 77.31 & 77.04 & 77.80 & 77.37  \\
 \hline
 16 & 68.11 & 68.90 & 67.88 & 67.48 & 77.37 & 78.13 & 78.13 & 77.69  \\
 \hline
\end{tabular}
\caption{\label{tab:mmlu_abalation} Accuracy on LM evaluation harness tasks on GPT3-8B model.}
\end{table}

\begin{table} \centering
\begin{tabular}{|c||c|c|c|c||c|c|c|c|} 
\hline
 $L_b \rightarrow$& \multicolumn{4}{c||}{8} & \multicolumn{4}{c||}{8}\\
 \hline
 \backslashbox{$L_A$\kern-1em}{\kern-1em$N_c$} & 2 & 4 & 8 & 16 & 2 & 4 & 8 & 16  \\
 %$N_c \rightarrow$ & 2 & 4 & 8 & 16 & 2 & 4 & 2 \\
 \hline
 \hline
 \multicolumn{5}{|c|}{Race (FP32 Accuracy = 40.67\%)} & \multicolumn{4}{|c|}{Boolq (FP32 Accuracy = 76.54\%)} \\ 
 \hline
 \hline
 64 & 40.48 & 40.10 & 39.43 & 39.90 & 75.41 & 75.11 & 77.09 & 75.66 \\
 \hline
 32 & 39.52 & 39.52 & 40.77 & 39.62 & 76.02 & 76.02 & 75.96 & 75.35  \\
 \hline
 16 & 39.81 & 39.71 & 39.90 & 40.38 & 75.05 & 73.82 & 75.72 & 76.09  \\
 \hline
 \hline
 \multicolumn{5}{|c|}{Winogrande (FP32 Accuracy = 70.64\%)} & \multicolumn{4}{|c|}{Piqa (FP32 Accuracy = 79.16\%)} \\ 
 \hline
 \hline
 64 & 69.14 & 70.17 & 70.17 & 70.56 & 78.24 & 79.00 & 78.62 & 78.73 \\
 \hline
 32 & 70.96 & 69.69 & 71.27 & 69.30 & 78.56 & 79.49 & 79.16 & 78.89  \\
 \hline
 16 & 71.03 & 69.53 & 69.69 & 70.40 & 78.13 & 79.16 & 79.00 & 79.00  \\
 \hline
\end{tabular}
\caption{\label{tab:mmlu_abalation} Accuracy on LM evaluation harness tasks on GPT3-22B model.}
\end{table}

\begin{table} \centering
\begin{tabular}{|c||c|c|c|c||c|c|c|c|} 
\hline
 $L_b \rightarrow$& \multicolumn{4}{c||}{8} & \multicolumn{4}{c||}{8}\\
 \hline
 \backslashbox{$L_A$\kern-1em}{\kern-1em$N_c$} & 2 & 4 & 8 & 16 & 2 & 4 & 8 & 16  \\
 %$N_c \rightarrow$ & 2 & 4 & 8 & 16 & 2 & 4 & 2 \\
 \hline
 \hline
 \multicolumn{5}{|c|}{Race (FP32 Accuracy = 44.4\%)} & \multicolumn{4}{|c|}{Boolq (FP32 Accuracy = 79.29\%)} \\ 
 \hline
 \hline
 64 & 42.49 & 42.51 & 42.58 & 43.45 & 77.58 & 77.37 & 77.43 & 78.1 \\
 \hline
 32 & 43.35 & 42.49 & 43.64 & 43.73 & 77.86 & 75.32 & 77.28 & 77.86  \\
 \hline
 16 & 44.21 & 44.21 & 43.64 & 42.97 & 78.65 & 77 & 76.94 & 77.98  \\
 \hline
 \hline
 \multicolumn{5}{|c|}{Winogrande (FP32 Accuracy = 69.38\%)} & \multicolumn{4}{|c|}{Piqa (FP32 Accuracy = 78.07\%)} \\ 
 \hline
 \hline
 64 & 68.9 & 68.43 & 69.77 & 68.19 & 77.09 & 76.82 & 77.09 & 77.86 \\
 \hline
 32 & 69.38 & 68.51 & 68.82 & 68.90 & 78.07 & 76.71 & 78.07 & 77.86  \\
 \hline
 16 & 69.53 & 67.09 & 69.38 & 68.90 & 77.37 & 77.8 & 77.91 & 77.69  \\
 \hline
\end{tabular}
\caption{\label{tab:mmlu_abalation} Accuracy on LM evaluation harness tasks on Llama2-7B model.}
\end{table}

\begin{table} \centering
\begin{tabular}{|c||c|c|c|c||c|c|c|c|} 
\hline
 $L_b \rightarrow$& \multicolumn{4}{c||}{8} & \multicolumn{4}{c||}{8}\\
 \hline
 \backslashbox{$L_A$\kern-1em}{\kern-1em$N_c$} & 2 & 4 & 8 & 16 & 2 & 4 & 8 & 16  \\
 %$N_c \rightarrow$ & 2 & 4 & 8 & 16 & 2 & 4 & 2 \\
 \hline
 \hline
 \multicolumn{5}{|c|}{Race (FP32 Accuracy = 48.8\%)} & \multicolumn{4}{|c|}{Boolq (FP32 Accuracy = 85.23\%)} \\ 
 \hline
 \hline
 64 & 49.00 & 49.00 & 49.28 & 48.71 & 82.82 & 84.28 & 84.03 & 84.25 \\
 \hline
 32 & 49.57 & 48.52 & 48.33 & 49.28 & 83.85 & 84.46 & 84.31 & 84.93  \\
 \hline
 16 & 49.85 & 49.09 & 49.28 & 48.99 & 85.11 & 84.46 & 84.61 & 83.94  \\
 \hline
 \hline
 \multicolumn{5}{|c|}{Winogrande (FP32 Accuracy = 79.95\%)} & \multicolumn{4}{|c|}{Piqa (FP32 Accuracy = 81.56\%)} \\ 
 \hline
 \hline
 64 & 78.77 & 78.45 & 78.37 & 79.16 & 81.45 & 80.69 & 81.45 & 81.5 \\
 \hline
 32 & 78.45 & 79.01 & 78.69 & 80.66 & 81.56 & 80.58 & 81.18 & 81.34  \\
 \hline
 16 & 79.95 & 79.56 & 79.79 & 79.72 & 81.28 & 81.66 & 81.28 & 80.96  \\
 \hline
\end{tabular}
\caption{\label{tab:mmlu_abalation} Accuracy on LM evaluation harness tasks on Llama2-70B model.}
\end{table}

%\section{MSE Studies}
%\textcolor{red}{TODO}


\subsection{Number Formats and Quantization Method}
\label{subsec:numFormats_quantMethod}
\subsubsection{Integer Format}
An $n$-bit signed integer (INT) is typically represented with a 2s-complement format \citep{yao2022zeroquant,xiao2023smoothquant,dai2021vsq}, where the most significant bit denotes the sign.

\subsubsection{Floating Point Format}
An $n$-bit signed floating point (FP) number $x$ comprises of a 1-bit sign ($x_{\mathrm{sign}}$), $B_m$-bit mantissa ($x_{\mathrm{mant}}$) and $B_e$-bit exponent ($x_{\mathrm{exp}}$) such that $B_m+B_e=n-1$. The associated constant exponent bias ($E_{\mathrm{bias}}$) is computed as $(2^{{B_e}-1}-1)$. We denote this format as $E_{B_e}M_{B_m}$.  

\subsubsection{Quantization Scheme}
\label{subsec:quant_method}
A quantization scheme dictates how a given unquantized tensor is converted to its quantized representation. We consider FP formats for the purpose of illustration. Given an unquantized tensor $\bm{X}$ and an FP format $E_{B_e}M_{B_m}$, we first, we compute the quantization scale factor $s_X$ that maps the maximum absolute value of $\bm{X}$ to the maximum quantization level of the $E_{B_e}M_{B_m}$ format as follows:
\begin{align}
\label{eq:sf}
    s_X = \frac{\mathrm{max}(|\bm{X}|)}{\mathrm{max}(E_{B_e}M_{B_m})}
\end{align}
In the above equation, $|\cdot|$ denotes the absolute value function.

Next, we scale $\bm{X}$ by $s_X$ and quantize it to $\hat{\bm{X}}$ by rounding it to the nearest quantization level of $E_{B_e}M_{B_m}$ as:

\begin{align}
\label{eq:tensor_quant}
    \hat{\bm{X}} = \text{round-to-nearest}\left(\frac{\bm{X}}{s_X}, E_{B_e}M_{B_m}\right)
\end{align}

We perform dynamic max-scaled quantization \citep{wu2020integer}, where the scale factor $s$ for activations is dynamically computed during runtime.

\subsection{Vector Scaled Quantization}
\begin{wrapfigure}{r}{0.35\linewidth}
  \centering
  \includegraphics[width=\linewidth]{sections/figures/vsquant.jpg}
  \caption{\small Vectorwise decomposition for per-vector scaled quantization (VSQ \citep{dai2021vsq}).}
  \label{fig:vsquant}
\end{wrapfigure}
During VSQ \citep{dai2021vsq}, the operand tensors are decomposed into 1D vectors in a hardware friendly manner as shown in Figure \ref{fig:vsquant}. Since the decomposed tensors are used as operands in matrix multiplications during inference, it is beneficial to perform this decomposition along the reduction dimension of the multiplication. The vectorwise quantization is performed similar to tensorwise quantization described in Equations \ref{eq:sf} and \ref{eq:tensor_quant}, where a scale factor $s_v$ is required for each vector $\bm{v}$ that maps the maximum absolute value of that vector to the maximum quantization level. While smaller vector lengths can lead to larger accuracy gains, the associated memory and computational overheads due to the per-vector scale factors increases. To alleviate these overheads, VSQ \citep{dai2021vsq} proposed a second level quantization of the per-vector scale factors to unsigned integers, while MX \citep{rouhani2023shared} quantizes them to integer powers of 2 (denoted as $2^{INT}$).

\subsubsection{MX Format}
The MX format proposed in \citep{rouhani2023microscaling} introduces the concept of sub-block shifting. For every two scalar elements of $b$-bits each, there is a shared exponent bit. The value of this exponent bit is determined through an empirical analysis that targets minimizing quantization MSE. We note that the FP format $E_{1}M_{b}$ is strictly better than MX from an accuracy perspective since it allocates a dedicated exponent bit to each scalar as opposed to sharing it across two scalars. Therefore, we conservatively bound the accuracy of a $b+2$-bit signed MX format with that of a $E_{1}M_{b}$ format in our comparisons. For instance, we use E1M2 format as a proxy for MX4.

\begin{figure}
    \centering
    \includegraphics[width=1\linewidth]{sections//figures/BlockFormats.pdf}
    \caption{\small Comparing LO-BCQ to MX format.}
    \label{fig:block_formats}
\end{figure}

Figure \ref{fig:block_formats} compares our $4$-bit LO-BCQ block format to MX \citep{rouhani2023microscaling}. As shown, both LO-BCQ and MX decompose a given operand tensor into block arrays and each block array into blocks. Similar to MX, we find that per-block quantization ($L_b < L_A$) leads to better accuracy due to increased flexibility. While MX achieves this through per-block $1$-bit micro-scales, we associate a dedicated codebook to each block through a per-block codebook selector. Further, MX quantizes the per-block array scale-factor to E8M0 format without per-tensor scaling. In contrast during LO-BCQ, we find that per-tensor scaling combined with quantization of per-block array scale-factor to E4M3 format results in superior inference accuracy across models. 



\end{document}
