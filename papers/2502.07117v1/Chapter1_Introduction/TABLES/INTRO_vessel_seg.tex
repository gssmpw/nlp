\notsotiny

{\begin{longtable}{p{1.5cm}p{1.5cm}p{1.75cm}p{4cm}p{4cm}p{4cm}P{0.75cm}p{0.75cm}}
\toprule
Study & Method & Data & Description & Advantages & Limitations & Reproducibility & Open-source \\ \hline
\endfirsthead
\endhead
\midrule
\endfoot
\endlastfoot
Zhang, et al. \cite{zhang2012automated} & Semi-automatic & N=43, SD-OCT, None & Multiscale Hessian matrix analysis. & Addresses size-based heterogeneity of vessels. & Assumed circularity of vessel, large data loss (N=19) from poor choroid visualisation. & Yes & No \\

Kajic, et al. \cite{kajic2012automated_3d, esmaeelpour2014choroidal} & Semi-automatic & N=12 (871), EDI-OCT, Manual & 3D multi-scale edge filtering. & Addresses size-based heterogeneity of vessels. Pathological evaluation. & Vessel estimation assumed circularity. & No & No \\

Mahajan, et al. \cite{mahajan2013automated} & Semi-automatic & N=12 (12), SD-OCT, Manual & Multiple global thresholding with Frangi's filter \cite{frangi1998multiscale}. & Domain-specific method to address size/contrast heterogeneity of vessels. & Inter-dependent pipeline on global thresholding, limited evaluation. & No & No \\

Branchini, et al. \cite{branchini2013analysis} & Semi-automatic & N=42 (42), SD-OCT, None & Otsu's global thresholding after manually selecting choroidal area. & Simple application. & Sensitive to noise, limited to large vessels, requires manual selection. & No & No \\

Sonoda, et al. \cite{sonoda2014choroidal} & Semi-automatic & N=20 (20), EDI-OCT, Manual & Niblack’s local thresholding after manual selection of vessel brightness. & Open-source protocol, commonly used, assessed in pathology. & Did not specify Niblack parameters, requires manual brightness adjustment, limited to high quality choroids and strict grading protocol, slow ($\approx$ 5 min/B-scan). & Yes & Yes** \\

Vupparaboina, et al. \cite{vupparaboina2016optical} & Semi-automatic & N=2 (388), SD-OCT, None & Multi-stage enhancement with histogram equalisation and global thresholding. & Less human involvement than Sonoda, et al. \cite{sonoda2014choroidal}, less noise sensitive. & Limited and only qualitative evaluation, global thresholding not adept to multi-scale vessels. & No & No \\

Agrawal, et al. \cite{agrawal2016choroidal} & Semi-automatic & N=345 (345), EDI-OCT, Manual*  & Modified Niblack's method with morphology. & More automated than Sonoda, et al. \cite{sonoda2014choroidal} Commonly used across studies. & Lack of standardisation of Niblack’s parameters across studies, slow ($\approx$ 1 min/B-scan). & Yes & Yes\textsuperscript{\textdagger} \\

Liu, et al. \cite{liu2019robust} & Deep learning & N=10 (40), SS-OCT, Manual* & UNet based architecture with a large ResNet encoder. & No pre-processing required, assessed inter-rater variability. & High computational demand, manual vessel labelling. & No & No \\ \bottomrule

\caption[Previous semi- and fully-automatic approaches for choroid vessel segmentation in \acrshort{OCT}.]{Semi-automatic and fully-automatic approaches to choroid vessel segmentation over the last 12 years. The data column is structured to describe the number of eyes (total B-scans) used for training and/or evaluation, the type of OCT data and whether model development/training used manual labels. *: Manual labelling by 2 or more graders; ** : Measurement protocol provided in \cite{sonoda2014choroidal}; \textsuperscript{\textdagger} : \href{https://www.ocularimaging.net/land}{Website} available but membership, permission and co-author requirements \cite{betzler2022choroidal}.} \label{tab:INTRO_vessel_methods}
\end{longtable}
}
