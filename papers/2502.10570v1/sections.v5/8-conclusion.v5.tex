\section{Conclusion}

This study provides empirical evidence on the limitations of 2D gaze estimation methods in mobile contexts, particularly their vulnerability to head pose and device orientation variations. By combining motion (IMU) and vision data, we demonstrate that 2D gaze estimators struggle to handle the frequent orientation changes characteristic of real-world mobile device usage. Through two comprehensive user studies investigating the spatial dynamics of user-device interactions, we show that the assumption that models can implicitly learn to compensate for head pose variations proves inadequate in dynamic mobile scenarios. These findings highlight the need for more advanced, adaptive eye-tracking systems that can explicitly account for head movements and maintain accuracy across diverse usage contexts.  Future research should focus on developing novel approaches that address these limitations, potentially by incorporating elements of 3D gaze estimation or explicit head pose modelling into more efficient 2D frameworks that explore 2D-3D hybrid pipelines that balance real-time performance with energy efficiency.