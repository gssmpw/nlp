\section{Previous Work}
Several previous studies have laid the foundation for this research. Costescu et al. \cite{costescu2020development} focused primarily on providing ideas for such a self-adaptive platform. An important idea of such a platform is to determine the focus of attention during children's play. This can be done in the game "Mushroom Hunter" to examine the ability to sustain attention. \cite{bueno2023datasets} focused on the development and application of datasets tailored for AI research in education, especially for children with NDD. Another study \cite{thill2022modelling} explored how robotics and social AI interpret children's behaviour. Another study \cite{daehlen2024towards} explored how serious games combined with eye-tracking technology can provide insights to help teachers better support children with NDD. The study proposed here is based on research on the development of platforms to support NDD children \cite{costescu2020development}, where we focus on three objectives in particular: Processing and analyzing datasets, supporting teachers and psychologists, explainability, and privacy.