\section{Conclusion}

We conducted a study to compare feedback on experimentation protocols written by teachers, science education experts, and an LLM agent. Human raters assessed the feedback across six dimensions to evaluate its quality in both content and language-related aspects. Our findings show that, on average, there is no significant difference in overall quality between the three groups, indicating that LLMs are capable of producing valuable feedback for students.
However, the LLM agent tends to focus on the \textit{feed forward} aspect of feedback, partly neglecting the equally important \textit{feed back} component. This highlights the need for further refinement of LLMs to ensure they provide balanced and comprehensive feedback for students. Additionally, since teachers and LLMs appear to encounter challenges with different types of student errors, integrating feedback from both could offer a promising approach for enhancing educational practices in the future.