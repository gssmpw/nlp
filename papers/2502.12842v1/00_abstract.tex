
Effective feedback is essential for fostering students' success in scientific inquiry. With advancements in artificial intelligence, large language models (LLMs) offer new possibilities for delivering instant and adaptive feedback. However, this feedback often lacks the pedagogical validation provided by real-world practitioners. 
To address this limitation, our study evaluates and compares the feedback quality of LLM agents with that of human teachers and science education experts on student-written experimentation protocols.
Four blinded raters, all professionals in scientific inquiry and science education, evaluated the feedback texts generated by 1) the LLM agent, 2) the teachers and 3) the science education experts using a five-point Likert scale based on six criteria of effective feedback: Feed Up, Feed Back, Feed Forward, Constructive Tone, Linguistic Clarity, and Technical Terminology. 
Our results indicate that LLM-generated feedback shows no significant difference to that of teachers and experts in overall quality.
However, the LLM agent's performance lags in the Feed Back dimension, which involves identifying and explaining errors within the student’s work context. Qualitative analysis highlighted the LLM agent’s limitations in contextual understanding and in the clear communication of specific errors.
Our findings suggest that combining LLM-generated feedback with human expertise can enhance educational practices by leveraging the efficiency of LLMs and the nuanced understanding of educators.
