% This must be in the first 5 lines to tell arXiv to use pdfLaTeX, which is strongly recommended.
\pdfoutput=1
% In particular, the hyperref package requires pdfLaTeX in order to break URLs across lines.


\documentclass[11pt]{article}

% Change "review" to "final" to generate the final (sometimes called camera-ready) version.
% Change to "preprint" to generate a non-anonymous version with page numbers.
\usepackage[preprint]{acl}

% Standard package includes
\usepackage{times}
\usepackage{latexsym}

% For proper rendering and hyphenation of words containing Latin characters (including in bib files)
\usepackage[T1]{fontenc}
% For Vietnamese characters
% \usepackage[T5]{fontenc}
% See https://www.latex-project.org/help/documentation/encguide.pdf for other character sets

% This assumes your files are encoded as UTF8
\usepackage[utf8]{inputenc}

% This is not strictly necessary, and may be commented out,
% but it will improve the layout of the manuscript,
% and will typically save some space.
\usepackage{microtype}

% This is also not strictly necessary, and may be commented out.
% However, it will improve the aesthetics of text in
% the typewriter font.
\usepackage{inconsolata}
\usepackage{longtable}

%Including images in your LaTeX document requires adding
%additional package(s)
\usepackage{graphicx}
\usepackage{subcaption}
\usepackage{booktabs} % for professional tables
\usepackage{multirow}
\usepackage{arydshln}
\newcommand{\ccol}[1]{\multicolumn{1}{c}{#1}}
\usepackage{hyperref}
\newcommand{\theHalgorithm}{\arabic{algorithm}}
\usepackage{amsmath}
\usepackage{amssymb}
\usepackage{mathtools}
\usepackage{amsthm}
\usepackage{color}
% if you use cleveref..
\usepackage[capitalize,noabbrev]{cleveref}

%%%%%%%%%%%%%%%%%%%%%%%%%%%%%%%%
% THEOREMS
%%%%%%%%%%%%%%%%%%%%%%%%%%%%%%%%
\theoremstyle{plain}
\newtheorem{theorem}{Theorem}[section]
\newtheorem{proposition}[theorem]{Proposition}
\newtheorem{lemma}[theorem]{Lemma}
\newtheorem{corollary}[theorem]{Corollary}
\theoremstyle{definition}
\newtheorem{definition}[theorem]{Definition}
\newtheorem{assumption}[theorem]{Assumption}
\theoremstyle{remark}
\newtheorem{remark}[theorem]{Remark}

% Todonotes is useful during development; simply uncomment the next line
%    and comment out the line below the next line to turn off comments
%\usepackage[disable,textsize=tiny]{todonotes}
\usepackage[textsize=tiny]{todonotes}

% If the title and author information does not fit in the area allocated, uncomment the following
%
%\setlength\titlebox{<dim>}
%
% and set <dim> to something 5cm or larger.

\title{EPO: Explicit Policy Optimization for Strategic Reasoning in LLMs via Reinforcement Learning}
% Enhancing Strategic Reasoning in LLMs for Long-term Goal Alignment via Reinforcement Learning
% Collaborative Strategic Reasoning Elicits Goal-Oriented Behavior from Large Language Models
% Collaborative Strategic Reasoning Enhances Decision-Making of Large Language Models
% Cooperative Strategic Reasoning Enhances Long-term Planning of Large Language Model Agents
% Inference-Time Policy Steering through Strategic Reasoning Models
% Steering Large Language Model Agents with Strategic Reasoning Models


% Author information can be set in various styles:
% For several authors from the same institution:
% \author{Author 1 \and ... \and Author n \\
%         Address line \\ ... \\ Address line}
% if the names do not fit well on one line use
%         Author 1 \\ {\bf Author 2} \\ ... \\ {\bf Author n} \\
% For authors from different institutions:
% \author{Author 1 \\ Address line \\  ... \\ Address line
%         \And  ... \And
%         Author n \\ Address line \\ ... \\ Address line}
% To start a separate ``row'' of authors use \AND, as in
% \author{Author 1 \\ Address line \\  ... \\ Address line
%         \AND
%         Author 2 \\ Address line \\ ... \\ Address line \And
%         Author 3 \\ Address line \\ ... \\ Address line}

%\author{Xiaoqian Liu \\
%   University of Chinese Academy of Sciences\\
%  Insititute of Automation, Chinese Academy of Sciences \\
%  \texttt{email@domain} \\\And
%  Ke Wang \\
%  Tongyi Lab \\
%  \texttt{email@domain} \\}

\author{
  \textbf{Xiaoqian Liu\textsuperscript{1,3}},
  \textbf{Ke Wang\textsuperscript{2}},
\textbf{Yongbin Li\textsuperscript{2}$^{*}$},
  \textbf{Yuchuan Wu\textsuperscript{2}},
  \textbf{Wentao Ma\textsuperscript{2}},\\
  \textbf{Aobo Kong\textsuperscript{2}},
  \textbf{Fei Huang\textsuperscript{2}},
  \textbf{Jianbin Jiao \textsuperscript{1}},
  \textbf{Junge Zhang\textsuperscript{3}\thanks{Corresponding authors}}
%  \textbf{Tenth Author\textsuperscript{1}},
%  \textbf{Eleventh E. Author\textsuperscript{1,2,3,4,5}},
%  \textbf{Twelfth Author\textsuperscript{1}},
%\\
%  \textbf{Thirteenth Author\textsuperscript{3}},
%  \textbf{Fourteenth F. Author\textsuperscript{2,4}},
%  \textbf{Fifteenth Author\textsuperscript{1}},
%  \textbf{Sixteenth Author\textsuperscript{1}},
%\\
%  \textbf{Seventeenth S. Author\textsuperscript{4,5}},
%  \textbf{Eighteenth Author\textsuperscript{3,4}},
%  \textbf{Nineteenth N. Author\textsuperscript{2,5}},
%  \textbf{Twentieth Author\textsuperscript{1}}
%\\
\\
  \textsuperscript{1}University of Chinese Academy of Sciences,
 \textsuperscript{2}Tongyi Lab\\
  \textsuperscript{3}Insititute of Automation, Chinese Academy of Sciences
%  \textsuperscript{4}Affiliation 4,
%  \textsuperscript{5}Affiliation 5
%\\
%  \small{
%    \textbf{Correspondence to:} \href{mailto:email@domain}{email@domain}
%  }
}

\begin{document}
\maketitle

Large language model (LLM)-based agents have shown promise in tackling complex tasks by interacting dynamically with the environment. 
Existing work primarily focuses on behavior cloning from expert demonstrations and preference learning through exploratory trajectory sampling. However, these methods often struggle in long-horizon tasks, where suboptimal actions accumulate step by step, causing agents to deviate from correct task trajectories.
To address this, we highlight the importance of \textit{timely calibration} and the need to automatically construct calibration trajectories for training agents. We propose \textbf{S}tep-Level \textbf{T}raj\textbf{e}ctory \textbf{Ca}libration (\textbf{\model}), a novel framework for LLM agent learning. 
Specifically, \model identifies suboptimal actions through a step-level reward comparison during exploration. It constructs calibrated trajectories using LLM-driven reflection, enabling agents to learn from improved decision-making processes. These calibrated trajectories, together with successful trajectory data, are utilized for reinforced training.
Extensive experiments demonstrate that \model significantly outperforms existing methods. Further analysis highlights that step-level calibration enables agents to complete tasks with greater robustness. 
Our code and data are available at \url{https://github.com/WangHanLinHenry/STeCa}.

%!TEX root = gcn.tex
\section{Introduction}
Graphs, representing structural data and topology, are widely used across various domains, such as social networks and merchandising transactions.
Graph convolutional networks (GCN)~\cite{iclr/KipfW17} have significantly enhanced model training on these interconnected nodes.
However, these graphs often contain sensitive information that should not be leaked to untrusted parties.
For example, companies may analyze sensitive demographic and behavioral data about users for applications ranging from targeted advertising to personalized medicine.
Given the data-centric nature and analytical power of GCN training, addressing these privacy concerns is imperative.

Secure multi-party computation (MPC)~\cite{crypto/ChaumDG87,crypto/ChenC06,eurocrypt/CiampiRSW22} is a critical tool for privacy-preserving machine learning, enabling mutually distrustful parties to collaboratively train models with privacy protection over inputs and (intermediate) computations.
While research advances (\eg,~\cite{ccs/RatheeRKCGRS20,uss/NgC21,sp21/TanKTW,uss/WatsonWP22,icml/Keller022,ccs/ABY318,folkerts2023redsec}) support secure training on convolutional neural networks (CNNs) efficiently, private GCN training with MPC over graphs remains challenging.

Graph convolutional layers in GCNs involve multiplications with a (normalized) adjacency matrix containing $\numedge$ non-zero values in a $\numnode \times \numnode$ matrix for a graph with $\numnode$ nodes and $\numedge$ edges.
The graphs are typically sparse but large.
One could use the standard Beaver-triple-based protocol to securely perform these sparse matrix multiplications by treating graph convolution as ordinary dense matrix multiplication.
However, this approach incurs $O(\numnode^2)$ communication and memory costs due to computations on irrelevant nodes.
%
Integrating existing cryptographic advances, the initial effort of SecGNN~\cite{tsc/WangZJ23,nips/RanXLWQW23} requires heavy communication or computational overhead.
Recently, CoGNN~\cite{ccs/ZouLSLXX24} optimizes the overhead in terms of  horizontal data partitioning, proposing a semi-honest secure framework.
Research for secure GCN over vertical data  remains nascent.

Current MPC studies, for GCN or not, have primarily targeted settings where participants own different data samples, \ie, horizontally partitioned data~\cite{ccs/ZouLSLXX24}.
MPC specialized for scenarios where parties hold different types of features~\cite{tkde/LiuKZPHYOZY24,icml/CastigliaZ0KBP23,nips/Wang0ZLWL23} is rare.
This paper studies $2$-party secure GCN training for these vertical partition cases, where one party holds private graph topology (\eg, edges) while the other owns private node features.
For instance, LinkedIn holds private social relationships between users, while banks own users' private bank statements.
Such real-world graph structures underpin the relevance of our focus.
To our knowledge, no prior work tackles secure GCN training in this context, which is crucial for cross-silo collaboration.


To realize secure GCN over vertically split data, we tailor MPC protocols for sparse graph convolution, which fundamentally involves sparse (adjacency) matrix multiplication.
Recent studies have begun exploring MPC protocols for sparse matrix multiplication (SMM).
ROOM~\cite{ccs/SchoppmannG0P19}, a seminal work on SMM, requires foreknowledge of sparsity types: whether the input matrices are row-sparse or column-sparse.
Unfortunately, GCN typically trains on graphs with arbitrary sparsity, where nodes have varying degrees and no specific sparsity constraints.
Moreover, the adjacency matrix in GCN often contains a self-loop operation represented by adding the identity matrix, which is neither row- nor column-sparse.
Araki~\etal~\cite{ccs/Araki0OPRT21} avoid this limitation in their scalable, secure graph analysis work, yet it does not cover vertical partition.

% and related primitives
To bridge this gap, we propose a secure sparse matrix multiplication protocol, \osmm, achieving \emph{accurate, efficient, and secure GCN training over vertical data} for the first time.

\subsection{New Techniques for Sparse Matrices}
The cost of evaluating a GCN layer is dominated by SMM in the form of $\adjmat\feamat$, where $\adjmat$ is a sparse adjacency matrix of a (directed) graph $\graph$ and $\feamat$ is a dense matrix of node features.
For unrelated nodes, which often constitute a substantial portion, the element-wise products $0\cdot x$ are always zero.
Our efficient MPC design 
avoids unnecessary secure computation over unrelated nodes by focusing on computing non-zero results while concealing the sparse topology.
We achieve this~by:
1) decomposing the sparse matrix $\adjmat$ into a product of matrices (\S\ref{sec::sgc}), including permutation and binary diagonal matrices, that can \emph{faithfully} represent the original graph topology;
2) devising specialized protocols (\S\ref{sec::smm_protocol}) for efficiently multiplying the structured matrices while hiding sparsity topology.


 
\subsubsection{Sparse Matrix Decomposition}
We decompose adjacency matrix $\adjmat$ of $\graph$ into two bipartite graphs: one represented by sparse matrix $\adjout$, linking the out-degree nodes to edges, the other 
by sparse matrix $\adjin$,
linking edges to in-degree nodes.

%\ie, we decompose $\adjmat$ into $\adjout \adjin$, where $\adjout$ and $\adjin$ are sparse matrices representing these connections.
%linking out-degree nodes to edges and edges to in-degree nodes of $\graph$, respectively.

We then permute the columns of $\adjout$ and the rows of $\adjin$ so that the permuted matrices $\adjout'$ and $\adjin'$ have non-zero positions with \emph{monotonically non-decreasing} row and column indices.
A permutation $\sigma$ is used to preserve the edge topology, leading to an initial decomposition of $\adjmat = \adjout'\sigma \adjin'$.
This is further refined into a sequence of \emph{linear transformations}, 
which can be efficiently computed by our MPC protocols for 
\emph{oblivious permutation}
%($\Pi_{\ssp}$) 
and \emph{oblivious selection-multiplication}.
% ($\Pi_\SM$)
\iffalse
Our approach leverages bipartite graph representation and the monotonicity of non-zero positions to decompose a general sparse matrix into linear transformations, enhancing the efficiency of our MPC protocols.
\fi
Our decomposition approach is not limited to GCNs but also general~SMM 
by 
%simply 
treating them 
as adjacency matrices.
%of a graph.
%Since any sparse matrix can be viewed 

%allowing the same technique to be applied.

 
\subsubsection{New Protocols for Linear Transformations}
\emph{Oblivious permutation} (OP) is a two-party protocol taking a private permutation $\sigma$ and a private vector $\xvec$ from the two parties, respectively, and generating a secret share $\l\sigma \xvec\r$ between them.
Our OP protocol employs correlated randomnesses generated in an input-independent offline phase to mask $\sigma$ and $\xvec$ for secure computations on intermediate results, requiring only $1$ round in the online phase (\cf, $\ge 2$ in previous works~\cite{ccs/AsharovHIKNPTT22, ccs/Araki0OPRT21}).

Another crucial two-party protocol in our work is \emph{oblivious selection-multiplication} (OSM).
It takes a private bit~$s$ from a party and secret share $\l x\r$ of an arithmetic number~$x$ owned by the two parties as input and generates secret share $\l sx\r$.
%between them.
%Like our OP protocol, o
Our $1$-round OSM protocol also uses pre-computed randomnesses to mask $s$ and $x$.
%for secure computations.
Compared to the Beaver-triple-based~\cite{crypto/Beaver91a} and oblivious-transfer (OT)-based approaches~\cite{pkc/Tzeng02}, our protocol saves ${\sim}50\%$ of online communication while having the same offline communication and round complexities.

By decomposing the sparse matrix into linear transformations and applying our specialized protocols, our \osmm protocol
%($\prosmm$) 
reduces the complexity of evaluating $\numnode \times \numnode$ sparse matrices with $\numedge$ non-zero values from $O(\numnode^2)$ to $O(\numedge)$.

%(\S\ref{sec::secgcn})
\subsection{\cgnn: Secure GCN made Efficient}
Supported by our new sparsity techniques, we build \cgnn, 
a two-party computation (2PC) framework for GCN inference and training over vertical
%ly split
data.
Our contributions include:

1) We are the first to explore sparsity over vertically split, secret-shared data in MPC, enabling decompositions of sparse matrices with arbitrary sparsity and isolating computations that can be performed in plaintext without sacrificing privacy.

2) We propose two efficient $2$PC primitives for OP and OSM, both optimally single-round.
Combined with our sparse matrix decomposition approach, our \osmm protocol ($\prosmm$) achieves constant-round communication costs of $O(\numedge)$, reducing memory requirements and avoiding out-of-memory errors for large matrices.
In practice, it saves $99\%+$ communication
%(Table~\ref{table:comm_smm}) 
and reduces ${\sim}72\%$ memory usage over large $(5000\times5000)$ matrices compared with using Beaver triples.
%(Table~\ref{table:mem_smm_sparse}) ${\sim}16\%$-

3) We build an end-to-end secure GCN framework for inference and training over vertically split data, maintaining accuracy on par with plaintext computations.
We will open-source our evaluation code for research and deployment.

To evaluate the performance of $\cgnn$, we conducted extensive experiments over three standard graph datasets (Cora~\cite{aim/SenNBGGE08}, Citeseer~\cite{dl/GilesBL98}, and Pubmed~\cite{ijcnlp/DernoncourtL17}),
reporting communication, memory usage, accuracy, and running time under varying network conditions, along with an ablation study with or without \osmm.
Below, we highlight our key achievements.

\textit{Communication (\S\ref{sec::comm_compare_gcn}).}
$\cgnn$ saves communication by $50$-$80\%$.
(\cf,~CoGNN~\cite{ccs/KotiKPG24}, OblivGNN~\cite{uss/XuL0AYY24}).

\textit{Memory usage (\S\ref{sec::smmmemory}).}
\cgnn alleviates out-of-memory problems of using %the standard 
Beaver-triples~\cite{crypto/Beaver91a} for large datasets.

\textit{Accuracy (\S\ref{sec::acc_compare_gcn}).}
$\cgnn$ achieves inference and training accuracy comparable to plaintext counterparts.
%training accuracy $\{76\%$, $65.1\%$, $75.2\%\}$ comparable to $\{75.7\%$, $65.4\%$, $74.5\%\}$ in plaintext.

{\textit{Computational efficiency (\S\ref{sec::time_net}).}} 
%If the network is worse in bandwidth and better in latency, $\cgnn$ shows more benefits.
$\cgnn$ is faster by $6$-$45\%$ in inference and $28$-$95\%$ in training across various networks and excels in narrow-bandwidth and low-latency~ones.

{\textit{Impact of \osmm (\S\ref{sec:ablation}).}}
Our \osmm protocol shows a $10$-$42\times$ speed-up for $5000\times 5000$ matrices and saves $10$-2$1\%$ memory for ``small'' datasets and up to $90\%$+ for larger ones.


\section{Related Works}

\textbf{Enhancing LLMs' Theory of Mind.} There has been systematic evaluation that revealed LLMs' limitations in achieving robust Theory of Mind inference \citep{ullman2023large, shapira2023clever}. To enhance LLMs' Theory of Mind capacity, recent works have proposed various prompting techniques. For instance, SimToM \citep{wilf2023think} encourages LLMs to adopt perspective-taking, PercepToM \citep{jung2024perceptions} improves perception-to-belief inference by extracting relevant contextual details, and \citet{huang2024notion} utilize an LLM as a world model to track environmental changes and refine prompts. Explicit symbolic modules also seem to improve LLM's accuracy through dynamic updates based on inputs. Specifically, TimeToM \citep{hou2024timetom} constructs a temporal reasoning framework to support inference, while SymbolicToM \citep{sclar2023minding} uses graphical representations to track characters' beliefs. Additionally, \citet{wagner2024mind} investigates ToM's necessity and the level of recursion required for specific tasks. However, these approaches continue to exhibit systematic errors in long contexts, complex behaviors, and recursive reasoning due to inherent limitations in inference and modeling \citep{jin2024mmtom,shi2024muma}. Most of them rely on domain-specific designs, lacking open-endedness.


\textbf{Model-based Theory of Mind inference.} Model-based Theory of Mind inference, in particular, Bayesian inverse planning (BIP) \citep{baker2009action,ullman2009help,baker2017rational,zhi2020online}, explicitly constructs representations of agents' mental states and how mental states guide agents' behavior via Bayesian Theory of Mind (BToM) models. These methods can reverse engineer human ToM inference in simple domains \citep[e.g.,][]{baker2017rational,netanyahu2021phase,shu2021agent}. Recent works have proposed to combine BIP with LLMs to achieve robust ToM inference in more realistic settings \citep{ying2023neuro, jin2024mmtom, shi2024muma}. However, these methods require manual specification of the BToM models as well as rigid, domain-specific implementations of Bayesian inference, limiting their adaptability to open-ended scenarios. To overcome this limitation, we propose \ours, a method capable of automatically modeling mental variables across diverse conditions and conducting automated BIP without domain-specific knowledge or implementations.


\begin{figure*}[ht]
  \centering
  \includegraphics[width=\linewidth]{figures/benchmarks_and_models.pdf}
    \vspace{-15pt}
  \caption{Examples questions (top panels) and the necessary Bayesian Theory of Mind (BToM) model for Bayesian inverse planning (bottom panels) in diverse Theory of Mind benchmarks. \ours aims to answer any Theory of Mind question in a variety of benchmarks, encompassing different mental variables, observable contexts, numbers of agents, the presence or absence of utterances, wording styles, and modalities. It proposes and iteratively adjusts an appropriate BToM and conducts automated Bayesian inverse planning based on the model.
  There can be more types of questions/models in each benchmark beyond the examples shown in this figure.}
  \label{fig:benchmarks_and_models}
  %\vspace{-0.75em}
  \vspace{-10pt}
\end{figure*}



\textbf{Automated Modeling with LLMs.} There has been an increasing interest in integrating LLMs with inductive reasoning and probabilistic inference for automated modeling. \citet{piriyakulkij2024doing} combine LLMs with Sequential Monte Carlo to perform probabilistic inference about underlying rules. Iterative hypothesis refinement techniques \citep{qiu2023phenomenal} further enhance LLM-based inductive reasoning by iteratively proposing, selecting, and refining textual hypotheses of rules. Beyond rule-based hypotheses, \citet{wang2023hypothesis} prompt LLMs to generate natural language hypotheses that are then implemented as verifiable programs, while \citet{li2024automated} propose a method in which LLMs construct, critique, and refine statistical models represented as probabilistic programs for data modeling. \citet{cross2024hypothetical} leverage LLMs to propose and evaluate agent strategies for multi-agent planning but do not specifically infer individual mental variables. Our method also aims to achieve automated modeling with LLMs. Unlike prior works, we propose a novel automated model discovery approach for Bayesian inverse planning, where the objective is to confidently infer any mental variable given any context via constructing a suitable Bayesian Theory of Mind model.

\vspace{-5pt}
\section{Method}
\label{sec:method}
\section{Overview}

\revision{In this section, we first explain the foundational concept of Hausdorff distance-based penetration depth algorithms, which are essential for understanding our method (Sec.~\ref{sec:preliminary}).
We then provide a brief overview of our proposed RT-based penetration depth algorithm (Sec.~\ref{subsec:algo_overview}).}



\section{Preliminaries }
\label{sec:Preliminaries}

% Before we introduce our method, we first overview the important basics of 3D dynamic human modeling with Gaussian splatting. Then, we discuss the diffusion-based 3d generation techniques, and how they can be applied to human modeling.
% \ZY{I stopp here. TBC.}
% \subsection{Dynamic human modeling with Gaussian splatting}
\subsection{3D Gaussian Splatting}
3D Gaussian splatting~\cite{kerbl3Dgaussians} is an explicit scene representation that allows high-quality real-time rendering. The given scene is represented by a set of static 3D Gaussians, which are parameterized as follows: Gaussian center $x\in {\mathbb{R}^3}$, color $c\in {\mathbb{R}^3}$, opacity $\alpha\in {\mathbb{R}}$, spatial rotation in the form of quaternion $q\in {\mathbb{R}^4}$, and scaling factor $s\in {\mathbb{R}^3}$. Given these properties, the rendering process is represented as:
\begin{equation}
  I = Splatting(x, c, s, \alpha, q, r),
  \label{eq:splattingGA}
\end{equation}
where $I$ is the rendered image, $r$ is a set of query rays crossing the scene, and $Splatting(\cdot)$ is a differentiable rendering process. We refer readers to Kerbl et al.'s paper~\cite{kerbl3Dgaussians} for the details of Gaussian splatting. 



% \ZY{I would suggest move this part to the method part.}
% GaissianAvatar is a dynamic human generation model based on Gaussian splitting. Given a sequence of RGB images, this method utilizes fitted SMPLs and sampled points on its surface to obtain a pose-dependent feature map by a pose encoder. The pose-dependent features and a geometry feature are fed in a Gaussian decoder, which is employed to establish a functional mapping from the underlying geometry of the human form to diverse attributes of 3D Gaussians on the canonical surfaces. The parameter prediction process is articulated as follows:
% \begin{equation}
%   (\Delta x,c,s)=G_{\theta}(S+P),
%   \label{eq:gaussiandecoder}
% \end{equation}
%  where $G_{\theta}$ represents the Gaussian decoder, and $(S+P)$ is the multiplication of geometry feature S and pose feature P. Instead of optimizing all attributes of Gaussian, this decoder predicts 3D positional offset $\Delta{x} \in {\mathbb{R}^3}$, color $c\in\mathbb{R}^3$, and 3D scaling factor $ s\in\mathbb{R}^3$. To enhance geometry reconstruction accuracy, the opacity $\alpha$ and 3D rotation $q$ are set to fixed values of $1$ and $(1,0,0,0)$ respectively.
 
%  To render the canonical avatar in observation space, we seamlessly combine the Linear Blend Skinning function with the Gaussian Splatting~\cite{kerbl3Dgaussians} rendering process: 
% \begin{equation}
%   I_{\theta}=Splatting(x_o,Q,d),
%   \label{eq:splatting}
% \end{equation}
% \begin{equation}
%   x_o = T_{lbs}(x_c,p,w),
%   \label{eq:LBS}
% \end{equation}
% where $I_{\theta}$ represents the final rendered image, and the canonical Gaussian position $x_c$ is the sum of the initial position $x$ and the predicted offset $\Delta x$. The LBS function $T_{lbs}$ applies the SMPL skeleton pose $p$ and blending weights $w$ to deform $x_c$ into observation space as $x_o$. $Q$ denotes the remaining attributes of the Gaussians. With the rendering process, they can now reposition these canonical 3D Gaussians into the observation space.



\subsection{Score Distillation Sampling}
Score Distillation Sampling (SDS)~\cite{poole2022dreamfusion} builds a bridge between diffusion models and 3D representations. In SDS, the noised input is denoised in one time-step, and the difference between added noise and predicted noise is considered SDS loss, expressed as:

% \begin{equation}
%   \mathcal{L}_{SDS}(I_{\Phi}) \triangleq E_{t,\epsilon}[w(t)(\epsilon_{\phi}(z_t,y,t)-\epsilon)\frac{\partial I_{\Phi}}{\partial\Phi}],
%   \label{eq:SDSObserv}
% \end{equation}
\begin{equation}
    \mathcal{L}_{\text{SDS}}(I_{\Phi}) \triangleq \mathbb{E}_{t,\epsilon} \left[ w(t) \left( \epsilon_{\phi}(z_t, y, t) - \epsilon \right) \frac{\partial I_{\Phi}}{\partial \Phi} \right],
  \label{eq:SDSObservGA}
\end{equation}
where the input $I_{\Phi}$ represents a rendered image from a 3D representation, such as 3D Gaussians, with optimizable parameters $\Phi$. $\epsilon_{\phi}$ corresponds to the predicted noise of diffusion networks, which is produced by incorporating the noise image $z_t$ as input and conditioning it with a text or image $y$ at timestep $t$. The noise image $z_t$ is derived by introducing noise $\epsilon$ into $I_{\Phi}$ at timestep $t$. The loss is weighted by the diffusion scheduler $w(t)$. 
% \vspace{-3mm}

\subsection{Overview of the RTPD Algorithm}\label{subsec:algo_overview}
Fig.~\ref{fig:Overview} presents an overview of our RTPD algorithm.
It is grounded in the Hausdorff distance-based penetration depth calculation method (Sec.~\ref{sec:preliminary}).
%, similar to that of Tang et al.~\shortcite{SIG09HIST}.
The process consists of two primary phases: penetration surface extraction and Hausdorff distance calculation.
We leverage the RTX platform's capabilities to accelerate both of these steps.

\begin{figure*}[t]
    \centering
    \includegraphics[width=0.8\textwidth]{Image/overview.pdf}
    \caption{The overview of RT-based penetration depth calculation algorithm overview}
    \label{fig:Overview}
\end{figure*}

The penetration surface extraction phase focuses on identifying the overlapped region between two objects.
\revision{The penetration surface is defined as a set of polygons from one object, where at least one of its vertices lies within the other object. 
Note that in our work, we focus on triangles rather than general polygons, as they are processed most efficiently on the RTX platform.}
To facilitate this extraction, we introduce a ray-tracing-based \revision{Point-in-Polyhedron} test (RT-PIP), significantly accelerated through the use of RT cores (Sec.~\ref{sec:RT-PIP}).
This test capitalizes on the ray-surface intersection capabilities of the RTX platform.
%
Initially, a Geometry Acceleration Structure (GAS) is generated for each object, as required by the RTX platform.
The RT-PIP module takes the GAS of one object (e.g., $GAS_{A}$) and the point set of the other object (e.g., $P_{B}$).
It outputs a set of points (e.g., $P_{\partial B}$) representing the penetration region, indicating their location inside the opposing object.
Subsequently, a penetration surface (e.g., $\partial B$) is constructed using this point set (e.g., $P_{\partial B}$) (Sec.~\ref{subsec:surfaceGen}).
%
The generated penetration surfaces (e.g., $\partial A$ and $\partial B$) are then forwarded to the next step. 

The Hausdorff distance calculation phase utilizes the ray-surface intersection test of the RTX platform (Sec.~\ref{sec:RT-Hausdorff}) to compute the Hausdorff distance between two objects.
We introduce a novel Ray-Tracing-based Hausdorff DISTance algorithm, RT-HDIST.
It begins by generating GAS for the two penetration surfaces, $P_{\partial A}$ and $P_{\partial B}$, derived from the preceding step.
RT-HDIST processes the GAS of a penetration surface (e.g., $GAS_{\partial A}$) alongside the point set of the other penetration surface (e.g., $P_{\partial B}$) to compute the penetration depth between them.
The algorithm operates bidirectionally, considering both directions ($\partial A \to \partial B$ and $\partial B \to \partial A$).
The final penetration depth between the two objects, A and B, is determined by selecting the larger value from these two directional computations.

%In the Hausdorff distance calculation step, we compute the Hausdorff distance between given two objects using a ray-surface-intersection test. (Sec.~\ref{sec:RT-Hausdorff}) Initially, we construct the GAS for both $\partial A$ and $\partial B$ to utilize the RT-core effectively. The RT-based Hausdorff distance algorithms then determine the Hausdorff distance by processing the GAS of one object (e.g. $GAS_{\partial A}$) and set of the vertices of the other (e.g. $P_{\partial B}$). Following the Hausdorff distance definition (Eq.~\ref{equation:hausdorff_definition}), we compute the Hausdorff distance to both directions ($\partial A \to \partial B$) and ($\partial B \to \partial A$). As a result, the bigger one is the final Hausdorff distance, and also it is the penetration depth between input object $A$ and $B$.


%the proposed RT-based penetration depth calculation pipeline.
%Our proposed methods adopt Tang's Hausdorff-based penetration depth methods~\cite{SIG09HIST}. The pipeline is divided into the penetration surface extraction step and the Hausdorff distance calculation between the penetration surface steps. However, since Tang's approach is not suitable for the RT platform in detail, we modified and applied it with appropriate methods.

%The penetration surface extraction step is extracting overlapped surfaces on other objects. To utilize the RT core, we use the ray-intersection-based PIP(Point-In-Polygon) algorithms instead of collision detection between two objects which Tang et al.~\cite{SIG09HIST} used. (Sec.~\ref{sec:RT-PIP})
%RT core-based PIP test uses a ray-surface intersection test. For purpose this, we generate the GAS(Geometry Acceleration Structure) for each object. RT core-based PIP test takes the GAS of one object (e.g. $GAS_{A}$) and a set of vertex of another one (e.g. $P_{B}$). Then this computes the penetrated vertex set of another one (e.g. $P_{\partial B}$). To calculate the Hausdorff distance, these vertex sets change to objects constructed by penetrated surface (e.g. $\partial B$). Finally, the two generated overlapped surface objects $\partial A$ and $\partial B$ are used in the Hausdorff distance calculation step.

Our goal is to increase the robustness of T2I models, particularly with rare or unseen concepts, which they struggle to generate. To do so, we investigate a retrieval-augmented generation approach, through which we dynamically select images that can provide the model with missing visual cues. Importantly, we focus on models that were not trained for RAG, and show that existing image conditioning tools can be leveraged to support RAG post-hoc.
As depicted in \cref{fig:overview}, given a text prompt and a T2I generative model, we start by generating an image with the given prompt. Then, we query a VLM with the image, and ask it to decide if the image matches the prompt. If it does not, we aim to retrieve images representing the concepts that are missing from the image, and provide them as additional context to the model to guide it toward better alignment with the prompt.
In the following sections, we describe our method by answering key questions:
(1) How do we know which images to retrieve? 
(2) How can we retrieve the required images? 
and (3) How can we use the retrieved images for unknown concept generation?
By answering these questions, we achieve our goal of generating new concepts that the model struggles to generate on its own.

\vspace{-3pt}
\subsection{Which images to retrieve?}
The amount of images we can pass to a model is limited, hence we need to decide which images to pass as references to guide the generation of a base model. As T2I models are already capable of generating many concepts successfully, an efficient strategy would be passing only concepts they struggle to generate as references, and not all the concepts in a prompt.
To find the challenging concepts,
we utilize a VLM and apply a step-by-step method, as depicted in the bottom part of \cref{fig:overview}. First, we generate an initial image with a T2I model. Then, we provide the VLM with the initial prompt and image, and ask it if they match. If not, we ask the VLM to identify missing concepts and
focus on content and style, since these are easy to convey through visual cues.
As demonstrated in \cref{tab:ablations}, empirical experiments show that image retrieval from detailed image captions yields better results than retrieval from brief, generic concept descriptions.
Therefore, after identifying the missing concepts, we ask the VLM to suggest detailed image captions for images that describe each of the concepts. 

\vspace{-4pt}
\subsubsection{Error Handling}
\label{subsec:err_hand}

The VLM may sometimes fail to identify the missing concepts in an image, and will respond that it is ``unable to respond''. In these rare cases, we allow up to 3 query repetitions, while increasing the query temperature in each repetition. Increasing the temperature allows for more diverse responses by encouraging the model to sample less probable words.
In most cases, using our suggested step-by-step method yields better results than retrieving images directly from the given prompt (see 
\cref{subsec:ablations}).
However, if the VLM still fails to identify the missing concepts after multiple attempts, we fall back to retrieving images directly from the prompt, as it usually means the VLM does not know what is the meaning of the prompt.

The used prompts can be found in \cref{app:prompts}.
Next, we turn to retrieve images based on the acquired image captions.

\vspace{-3pt}
\subsection{How to retrieve the required images?}

Given $n$ image captions, our goal is to retrieve the images that are most similar to these captions from a dataset. 
To retrieve images matching a given image caption, we compare the caption to all the images in the dataset using a text-image similarity metric and retrieve the top $k$ most similar images.
Text-to-image retrieval is an active research field~\cite{radford2021learning, zhai2023sigmoid, ray2024cola, vendrowinquire}, where no single method is perfect.
Retrieval is especially hard when the dataset does not contain an exact match to the query \cite{biswas2024efficient} or when the task is fine-grained retrieval, that depends on subtle details~\cite{wei2022fine}.
Hence, a common retrieval workflow is to first retrieve image candidates using pre-computed embeddings, and then re-rank the retrieved candidates using a different, often more expensive but accurate, method \cite{vendrowinquire}.
Following this workflow, we experimented with cosine similarity over different embeddings, and with multiple re-ranking methods of reference candidates.
Although re-ranking sometimes yields better results compared to simply using cosine similarity between CLIP~\cite{radford2021learning} embeddings, the difference was not significant in most of our experiments. Therefore, for simplicity, we use cosine similarity between CLIP embeddings as our similarity metric (see \cref{tab:sim_metrics}, \cref{subsec:ablations} for more details about our experiments with different similarity metrics).

\vspace{-3pt}
\subsection{How to use the retrieved images?}
Putting it all together, after retrieving relevant images, all that is left to do is to use them as context so they are beneficial for the model.
We experimented with two types of models; models that are trained to receive images as input in addition to text and have ICL capabilities (e.g., OmniGen~\cite{xiao2024omnigen}), and T2I models augmented with an image encoder in post-training (e.g., SDXL~\cite{podellsdxl} with IP-adapter~\cite{ye2023ip}).
As the first model type has ICL capabilities, we can supply the retrieved images as examples that it can learn from, by adjusting the original prompt.
Although the second model type lacks true ICL capabilities, it offers image-based control functionalities, which we can leverage for applying RAG over it with our method.
Hence, for both model types, we augment the input prompt to contain a reference of the retrieved images as examples.
Formally, given a prompt $p$, $n$ concepts, and $k$ compatible images for each concept, we use the following template to create a new prompt:
``According to these examples of 
$\mathord{<}c_1\mathord{>:<}img_{1,1}\mathord{>}, ... , \mathord{<}img_{1,k}\mathord{>}, ... , \mathord{<}c_n\mathord{>:<}img_{n,1}\mathord{>}, ... , $
$\mathord{<}img_{n,k}\mathord{>}$,
generate $\mathord{<}p\mathord{>}$'', 
where $c_i$ for $i\in{[1,n]}$ is a compatible image caption of the image $\mathord{<}img_{i,j}\mathord{>},  j\in{[1,k]}$. 

This prompt allows models to learn missing concepts from the images, guiding them to generate the required result. 

\textbf{Personalized Generation}: 
For models that support multiple input images, we can apply our method for personalized generation as well, to generate rare concept combinations with personal concepts. In this case, we use one image for personal content, and 1+ other reference images for missing concepts. For example, given an image of a specific cat, we can generate diverse images of it, ranging from a mug featuring the cat to a lego of it or atypical situations like the cat writing code or teaching a classroom of dogs (\cref{fig:personalization}).
\vspace{-2pt}
\begin{figure}[htp]
  \centering
   \includegraphics[width=\linewidth]{Assets/personalization.pdf}
   \caption{\textbf{Personalized generation example.}
   \emph{ImageRAG} can work in parallel with personalization methods and enhance their capabilities. For example, although OmniGen can generate images of a subject based on an image, it struggles to generate some concepts. Using references retrieved by our method, it can generate the required result.
}
   \label{fig:personalization}\vspace{-10pt}
\end{figure}

\section{Simulations and Experiment}
In this section, we conduct comprehensive experiments in both simulation and the real-world robot to address the following questions:
\begin{itemize}[leftmargin=*]
    \item \textbf{Q1(Sim)}: How does the \our policy perform in tracking across different commands?
    \item  \textbf{Q2(Sim)}: How to reasonably combine various commands in the general command space? % Command Analysis
    \item \textbf{Q3(Sim)}: How does large-scale noise intervention training help in policy robustness? % Ablation Study
    \item \textbf{Q4(Real)}: How does \our behave in the real world? % Real World Demo
\end{itemize}

\noindent\textbf{Robot and Simulator.} 
Our main experiments in this paper are conducted on the Unitree H1 robot, which has 19 Degrees of Freedom (DOF) in total, including 
two 3-DOF shoulder joints, two elbow joints, one waist joint, two 3-DOF hip joints, two knee joints, and two ankle joints.
The simulation training is based on the NVIDIA IsaacGym simulator~\citep{makoviychuk2021isaac}. It takes 16 hours on a single RTX 4090 GPU to train one policy.

\noindent\textbf{Command analysis principle and metric.}
One of the main contributions of this paper is an extended and general command space for humanoid robots. Therefore, we pay much attention to command analysis (regarding Q1 and Q2). This includes analysis of single command tracking errors, along with the combination of different commands under different gaits.
% we categorize the commands into three groups: \emph{movement}, \emph{foot}, and \emph{posture}. The \emph{movement} commands include the linear velocity and angular velocity, forming the foundational locomotion commands and are considered the most critical aspect of the tasks. The \emph{foot} commands include the gait frequency and foot swing height, representing the mode of leg movement. The \emph{posture} commands include body height, body pitch and waist yaw, which determine the desired body posture.
For analysis, we evaluate the averaged episodic command tracking error (denoted as $E_\text{cmd}$), which measures the discrepancy between the actual robot states and the command space using $L_1$ norm.
% The tracking error is measured in units of $m/s$, $rad/s$, $Hz$, $m$, and $rad$, corresponding to linear velocity, angular velocity, frequency, position, and rotation, respectively.
All commands are uniformly sampled within a pre-defined command range, as shown in \tb{tab:commands}\footnote{Note that the hopping gait keeps a different command range, due to its asymmetric type of motion. More details can be referred to \ap{ap:Hopping}.}.

%%%%%%%%%%%---SETME-----%%%%%%%%%%%%%
%replace @@ with the submission number submission site.
\newcommand{\thiswork}{INF$^2$\xspace}
%%%%%%%%%%%%%%%%%%%%%%%%%%%%%%%%%%%%


%\newcommand{\rev}[1]{{\color{olivegreen}#1}}
\newcommand{\rev}[1]{{#1}}


\newcommand{\JL}[1]{{\color{cyan}[\textbf{\sc JLee}: \textit{#1}]}}
\newcommand{\JW}[1]{{\color{orange}[\textbf{\sc JJung}: \textit{#1}]}}
\newcommand{\JY}[1]{{\color{blue(ncs)}[\textbf{\sc JSong}: \textit{#1}]}}
\newcommand{\HS}[1]{{\color{magenta}[\textbf{\sc HJang}: \textit{#1}]}}
\newcommand{\CS}[1]{{\color{navy}[\textbf{\sc CShin}: \textit{#1}]}}
\newcommand{\SN}[1]{{\color{olive}[\textbf{\sc SNoh}: \textit{#1}]}}

%\def\final{}   % uncomment this for the submission version
\ifdefined\final
\renewcommand{\JL}[1]{}
\renewcommand{\JW}[1]{}
\renewcommand{\JY}[1]{}
\renewcommand{\HS}[1]{}
\renewcommand{\CS}[1]{}
\renewcommand{\SN}[1]{}
\fi

%%% Notion for baseline approaches %%% 
\newcommand{\baseline}{offloading-based batched inference\xspace}
\newcommand{\Baseline}{Offloading-based batched inference\xspace}


\newcommand{\ans}{attention-near storage\xspace}
\newcommand{\Ans}{Attention-near storage\xspace}
\newcommand{\ANS}{Attention-Near Storage\xspace}

\newcommand{\wb}{delayed KV cache writeback\xspace}
\newcommand{\Wb}{Delayed KV cache writeback\xspace}
\newcommand{\WB}{Delayed KV Cache Writeback\xspace}

\newcommand{\xcache}{X-cache\xspace}
\newcommand{\XCACHE}{X-Cache\xspace}


%%% Notions for our methods %%%
\newcommand{\schemea}{\textbf{Expanding supported maximum sequence length with optimized performance}\xspace}
\newcommand{\Schemea}{\textbf{Expanding supported maximum sequence length with optimized performance}\xspace}

\newcommand{\schemeb}{\textbf{Optimizing the storage device performance}\xspace}
\newcommand{\Schemeb}{\textbf{Optimizing the storage device performance}\xspace}

\newcommand{\schemec}{\textbf{Orthogonally supporting Compression Techniques}\xspace}
\newcommand{\Schemec}{\textbf{Orthogonally supporting Compression Techniques}\xspace}



% Circular numbers
\usepackage{tikz}
\newcommand*\circled[1]{\tikz[baseline=(char.base)]{
            \node[shape=circle,draw,inner sep=0.4pt] (char) {#1};}}

\newcommand*\bcircled[1]{\tikz[baseline=(char.base)]{
            \node[shape=circle,draw,inner sep=0.4pt, fill=black, text=white] (char) {#1};}}

\subsection{Single Command Tracking}
We first analyze each command separately while keeping all other commands held at their default values. The results are shown in \tb{tab:Single commands}.
It is easily observed that the tracking errors in the walking and standing gaits are significantly lower than those in the jumping and hopping, with hopping exhibiting the largest tracking errors.
For hopping gaits, the robot may fall during the tracking of specific commands, like high-speed tracking, body pitch, and waist-yaw control.
This can be attributed to the fact that hopping requires rather high stability. Moreover, the complex postures and motions further exacerbate the risk of instability. Consequently, the policy prioritizes learning to maintain the balance, which, to some extent, compromises the accuracy of command tracking.

We conclude that the tracking accuracy of each gait aligns with the training difficulty of that gait in simulation. For example, the walking and standing patterns can be learned first during training, while the jumping and hopping gaits appear later and require an extended training period for the robot to acquire proficiency.
Similarly, the tracking accuracy of robots under low velocity is significantly better than those under high velocity, since 1) the locomotion skills under low velocity are much easier to master, and 2) the dynamic stability of the robot decreases at high speeds, leading to a trade-off with tracking accuracy.

We also found that the tracking accuracy for longitudinal velocity commands $v_x$ surpasses that of horizontal velocity commands $v_y$, which is due to the limitation of the hardware configuration of the selected Unitree H1 robots. In addition, the {foot swing height} $l$ is the least accurately tracked.
Furthermore, the tracking reward related to foot placement outperforms the tracking performance associated with posture control, since adjusting posture introduces greater challenges to stability. In response, the policy adopts more conservative actions to mitigate balance-threatening postural changes.
% In contrast, the influence of foot placement on stability is comparatively less pronounced, allowing for more precise tracking.

\begin{table}[t]
\setlength{\abovecaptionskip}{0.cm}
\setlength{\belowcaptionskip}{-0.cm}
\centering
\caption{\small \textbf{Single command tracking error.} The tracking errors for foot commands are calculated over a complete gait cycle, and the remaining ones are over one environmental step. For standing gait, we only tested the body height, body pitch, and waist yaw tracking error. $E^\text{high}$ and $E^\text{low}$ represents high-speed ($v_x > 1m/s$) and low-speed ($v_x \le 1m/s$) modes categorized by the linear velocity $v$. 
The tracking error is computed by sampling each command in a predefined range (\tb{tab:commands}) while keeping all other commands held at their default values.}
\label{tab:Single commands}
\resizebox{\columnwidth}{!}{
\begin{tabular}{@{}c|cccc|cc|ccc@{}}
\toprule
\multirow{3}{*}{Gait} & \multicolumn{4}{c|}{Movement} & \multicolumn{2}{c|}{Foot} & \multicolumn{3}{c}{Posture} \\
\cmidrule(l){2-5} \cmidrule{6-7} \cmidrule{8-10} 
& \multirow{2}{*}{\makecell{$E_{v_x}^\text{low}$\\($m/s$)}} & \multirow{2}{*}{\makecell{$E_{v_x}^\text{high}$\\($m/s$)}} & \multirow{2}{*}{\makecell{$E_{v_y}$\\($m/s$)}} & \multirow{2}{*}{\makecell{$E_{\omega}$\\$rad/s$}} & \multirow{2}{*}{\makecell{$E_{f}$\\($HZ$)}} & \multirow{2}{*}{\makecell{$E_{l}$\\($m$)}} & \multirow{2}{*}{\makecell{$E_{h}$\\($m$)}}  & \multirow{2}{*}{\makecell{$E_{p}$\\($rad$)}} & \multirow{2}{*}{\makecell{$E_{w}$\\($rad$)}}   \\ 
&  &  &  &  &  &  &  &  &    \\ 
\midrule
Standing  & - & - & - & - & - & - & 0.035 & 0.047 & 0.022  \\
Walking   & 0.030 & 0.216 & 0.085 & 0.054 & 0.028 & 0.011 & 0.064 & 0.038 & 0.075  \\
Jumping  & 0.090 & 0.532 & 0.069 & 0.077 & 0.027 & 0.012 & 0.058 & 0.048 & 0.022 \\
Hopping   & 0.033 & - & 0.046 & 0.078 & - & - & 0.103 & - & - \\
\bottomrule
\end{tabular}}
\end{table}



\begin{table*}[t]
\setlength{\abovecaptionskip}{0.cm}
\setlength{\belowcaptionskip}{-0.cm}
\centering
\caption{\small \textbf{Tracking errors with different intervention strategies under the walking gait}. We evaluate three upper-body intervention training strategies: Noise (\our), the AMASS dataset, and no intervention at all. The tracking errors across various task and behavior commands reflect the intervention tolerance, \textit{i.e.}, the ability of precise locomotion control under external intervention.}
\label{tab:Intervetion Tracking Error}
\begin{tabular}{c|c|ccc|cc|ccc}
\toprule
\multirow{3}{*}{Training Strategy} & \multirow{3}{*}{Intervention Task} & \multicolumn{3}{c|}{Task Commands}                        & \multicolumn{5}{c}{Behavior Commands}\\ \cmidrule{3-10}
 & & \multicolumn{3}{c|}{Movement}                        & \multicolumn{2}{c|}{Foot}          & \multicolumn{3}{c}{Posture}                         \\ \cmidrule{3-10}
                                      &                                      &$E_{v_x}$ ($m/s$)     & $E_{v_y}$ ($m/s$)   & $E_{\omega}$ ($rad/s$)    & $E_{f}$ ($Hz$)         & $E_{l}$ ($m$)         & $E_{h}$ ($m$)        & $E_{p}$ ($rad$)     & $E_{w}$ ($rad$)         \\ \midrule
\multirow{3}{*}{\makecell{Noise Curriculum\\(\our)}}        & Noise                        & \textbf{0.0483} & \textbf{0.0962} & \textbf{0.1879} & \textbf{0.0471} & \textbf{0.0542} & \textbf{0.0402} & \textbf{0.0432} & \textbf{0.0552} \\
                                      & AMASS                                & \textbf{0.0391} & \textbf{0.0920} & \textbf{0.1039} & \textbf{0.0464} & \textbf{0.0543} & \textbf{0.0387} & \textbf{0.0364} & \textbf{0.0540} \\
                                      & None                                 & \textbf{0.0264} & \textbf{0.0863} & \textbf{0.0543} & \textbf{0.0447} & \textbf{0.0522} & 0.0372          & 0.0375          & 0.0475          \\ \cmidrule{1-10}
\multirow{3}{*}{AMASS}                & Noise                        & 0.1697          & 0.1055          & 0.2156          & 0.0621          & 0.0542          & 0.0620          & 0.0812          & 0.0694          \\
                                      & AMASS                                & 0.0567          & 0.0965          & 0.1593          & 0.0466          & 0.0555          & 0.0579          & 0.0458          & 0.0554          \\
                                      & None                                 & 0.0645          & 0.0916          & 0.0802          & 0.0460          & 0.0531          & 0.0577          & 0.0455          & 0.0568          \\ \cmidrule{1-10}
\multirow{3}{*}{No Intervention}                 & Noise                        & 0.8658          & 0.7511          & 0.9116          & 0.1930          & 0.1913          & 0.1658          & 0.3622          & 0.2241          \\
                                      & AMASS                                & 0.6299          & 0.4026          & 0.5758          & 0.2245          & 0.2527          & 0.1305          & 0.2367          & 0.1112          \\
                                      & None                                 & 0.0755          & 0.1076          & 0.1151          & 0.0450          & 0.0678          & \textbf{0.0255} & \textbf{0.0211} & \textbf{0.0380} \\ \bottomrule
\end{tabular}
\end{table*}



\begin{table}[t]
\setlength{\abovecaptionskip}{0.cm}
\setlength{\belowcaptionskip}{-0.cm}
\centering
\caption{ \small
\textbf{Averaged foot displacement under intervention}. We compare foot displacement $D_\text{cmd}$ of different training strategies under various intervention tasks, which computes the total movement of both feet in one episode with sampled posture behavior commands.
}
\label{tab:Intervention Mean Foot Movement}
\resizebox{\linewidth}{!}{
\begin{tabular}{ccccc}
\toprule
Training Strategy                 & Intervention Task     & $D_{h}$ ($m/s$)                  & $D_{p}$ ($m/s$)      & $D_{w}$ ($m/s$)       \\ \midrule
\multirow{3}{*}{\makecell{Noise Curriculum\\(\our)}}  & Noise & \textbf{0.0339}             & \textbf{0.0892} & \textbf{0.0199} \\
                       & AMASS         & \textbf{0.0454}             & \textbf{0.0728} & \textbf{0.0196} \\
                       & None          & \textbf{0.0003}             & \textbf{0.0016} & \textbf{0.0007} \\ \midrule
\multirow{3}{*}{AMASS only} & Noise         & 2.0815                      & 2.8978          & 3.2630          \\
                       & AMASS         & 0.0536                      & 0.1743          & 0.0396          \\
                       & None          & 0.0139                      & 0.0160          & 0.0013          \\ \midrule
\multirow{3}{*}{No Intervention}  & Noise         & 17.5358                     & 17.9732         & 25.7132         \\
                       & AMASS         & 25.3802 & 26.3496         & 21.3078         \\
                       & None          & 0.0159  & 1.7065          & 1.7152          \\ \bottomrule
\end{tabular}}
\end{table}

\subsection{Command Combination Analysis}
To provide an in-depth analysis of the command space and to 
reveal the underlying interaction of various commands under different gaits.
Here, we aim to analyze the \emph{orthogonality} of commands based on the interference or conflict between the tracking errors of these commands across their reasonable ranges. For instance, when we say that a set of commands are \emph{orthogonal}, each command does not significantly affect the tracking performance of each other in its range. To this end, we plot the tracking error $E_\text{cmd}$ as heat maps, generated by systematically scanning the command values for each pair of parameters, revealing the correlation of each command.
We leave the full heat maps at \ap{ap:heatmaps}, and conclude our main observation for all gaits.

\noindent\textbf{Walking.} Walking is the most basic gait, which preserves the best performance of the robot hardware.
\begin{itemize}[leftmargin=*]
    \item The {linear velocity} $v_x$, the {angular velocity yaw} $\omega$, the {body height} $h$, and the {waist yaw} $w$ are orthogonal during walking.
    \item When the {linear velocity} $v_x$ exceeds $1.5m/s$, the orthogonality between $v_x$ and other commands decreases due to reduced dynamic stability and the robot's need to maintain body stability over tracking accuracy.
    \item The {gait frequency} $f$ shows discrete orthogonality, with optimal tracking performance at frequencies of 1.5 or 2. High-frequency gait conditions reduce tracking accuracy.
    \item The {linear velocity} $v_y$, the {foot swing height} $l$, and the {body pitch} $p$ are orthogonal to other commands only within a narrow range.
\end{itemize}

\noindent\textbf{Jumping.} The command orthogonality in jumping is similar to walking, but the overall orthogonal range is smaller, due to the increased challenge of the jumping gait, especially in high-speed movement modes.
During each gait cycle, the robot must leap forward significantly to maintain its speed. To execute this complex jumping action continuously, the robot must adopt an optimal posture at the beginning of each cycle. Both legs exert substantial torque to propel the body forward. Upon landing, the robot must quickly readjust its posture to maintain stability and repeat the actions. Consequently, during movement, the robot can only execute other commands within a relatively narrow range.

\noindent\textbf{Hopping.}
The hopping gait introduces more instability, and the robot's control system must focus more on maintaining balance, making it difficult to simultaneously handle complex, multi-dimensional commands.
\begin{itemize}[leftmargin=*]
    \item Hopping gait commands lack clear orthogonal relationships.
    \item Effective tracking is limited to the x-axis {linear velocity} $v_x$, the y-axis {linear velocity} $v_y$, the {angular velocity yaw} $\omega$, and the {body height} $h$.
    \item Adjustments to $h$ can be understood that a lower body height improves dynamic stability, therefore, it plays a positive role in maintaining the target body posture.
    % enhancing the robot's hopping performance.
\end{itemize}

\noindent\textbf{Standing.} As for the standing gait, we tested the tracking errors of commands related to posture. The results showed that the tracking errors were similar to those observed during walking with zero velocity.

\begin{itemize}[leftmargin=*]
    \item The {waist yaw} $w$ command is almost orthogonal to the other two commands.
    \item As the range of commands increases, orthogonality between the {body height} $h$ and the {body pitch} $p$ decreases. This is because the H1 robot has only one degree of freedom at the waist, limiting posture adjustments to the hip pitch joint.
    \item A 0.3 m decrease of the body height relative to the default height reduces the range of motion of the hip pitch joint to almost zero, hindering precise tracking of body pitch.
\end{itemize}

Furthermore, we conclude that {gait frequency} $f$ highly affects the tracking accuracy of \emph{movement} commands when it is excessively high and low; the \emph{posture} commands can significantly impact the tracking errors of other commands, especially when they are near the range limits.
% We categorize the commands into three groups: \emph{movement}, \emph{foot}, and \emph{posture}. 1) The \emph{movement} commands include the linear velocity $v_x, v_y$ and angular velocity $\omega$, forming the foundational locomotion commands, and are considered the most critical aspect of the tasks. 2) The \emph{foot} commands include the {foot swing height} $l$, which is the least accurately tracked; and the {gait frequency} $f$, which can affect the tracking accuracy of \emph{movement} commands when it is excessively high and low. 3) The \emph{posture} commands, which include body height $h$, the body pitch $p$, and waist yaw $w$, determine the desired body posture, and can significantly impact the tracking errors of other commands, especially when the command is challenging. 
For different gaits, the orthogonality range between commands is greatest in the walking gait and smallest in the hopping gait.

\subsection{Ablation on Intervention Training Strategy}
\label{sec:InterventionExp}
% The three policies use the same random seeds and training time.
To validate the effectiveness of the intervention training strategy on the policy robustness when external upper-body intervention is involved, we compare the policies trained with different strategies, including noise curriculum (\our), filtered AMASS data~\citep{he2024omnih2o}, and no intervention. We test the tracking errors under two different intervention tasks, \textit{i.e.}, uniform noise, AAMAS dataset, along with a no-intervention setup. The results under the walking gait are shown in \tb{tab:Intervetion Tracking Error}, and we leave other gaits in \ap{ap:SingleCommandsTracking-REMAIN}. 
It is obvious that the noise curriculum strategy of \our achieved the best performance under almost all test cases, except the posture-related tracking with no intervention. 
In particular, \our showed less of a decrease in tracking accuracy with various interventions, indicating our noise curriculum intervention strategy enables the control policy to handle a large range of arm movements, making it very useful and supportive for loco-manipulation tasks.
In comparison, the policy trained with AMASS data shows a significant decrease in the tracking accuracy when intervening with uniform noise, due to the limited motion in the training data. The policy trained without any intervention only performs well without external upper-body control.

It is worth noting that when intervention training is involved, the tracking error related to the movement and foot is also better than those of the policy trained without intervention, and \our provides the most accurate tracking. This shows that intervention training also contributes to the robustness of the policy. During our real robot experiments, we further observed that the robot behaves with a harder force when in contact with the floor, indicating a possible trade-off between motion regularization and tracking accuracy when involving intervention.

\noindent\textbf{Stability under standing gait.}
Adjusting posture in the standing state introduces additional requirements for stability, since the robot pacing to maintain balance may increase the difficulty of achieving manipulation tasks that require stand still. To investigate the necessity of noise curriculum for manipulation, we further measured the averaged foot displacement (in meters) under the standing gait, which computes the total movement of both feet in one episode (20 seconds) while tracking the posture behavior commands. Results in \tb{tab:Intervention Mean Foot Movement} show that \our exhibits minimal foot displacement. On the contrary, the strategy trained on AMASS data requires frequent small steps to adjust the posture and maintain stability for noise interventions. 
Without intervention training, the policy tends to tip over when involving intervention, leading to failure of the entire task.

%  鲁棒性测试的结果分析
\begin{figure}[t]
    \centering
    \includegraphics[width=\linewidth]{imgs/radar_chart_V2.pdf}
    \vspace{-13pt}
    \caption{\small \textbf{External disturbance tolerance}. Left: A constant and continuous force is applied to the robot. Right: A one-second force is exerted on the robot. The experiment is conducted under a standing gait with default commands. If the robot's survival ratio exceeds $98\%$, it is deemed capable of tolerating such external disturbance. 
    The survival ratio computes the trajectory ratio of non-termination (ends of timeout) during 4096 rollouts.}
    \label{fig:Robust}
    \vspace{-12pt}
\end{figure}
\noindent\textbf{Robustness for external disturbance.}
Finally, we test the contribution of intervention training and noise curriculum to the robustness of external disturbance. In particular, we evaluated the robot's maximum tolerance to external disturbance forces in eight directions and compared the policy trained without intervention. Results illustrated in \fig{fig:Robust} demonstrate that \our preserves greater tolerance for external disturbances in both pushing and loading scenarios across most of the directions. The reason behind this is that the intervention brings the robot exposed to various disturbances originating from its upper body, and thereby enhances the overall stability by dynamically adjusting leg strength.

% \our has a significantly higher tolerance for external disturbance forces in almost all directions compared to the strategy without intervention training.
% This is attributed to the fact that, during large-scale noise intervention training, the robot effectively explored a wide range of extreme scenarios and learned to enhance body stability by adjusting leg movements.

\subsection{Real-World Experiments}
We deploy \our on a real-world robot to verify its effectiveness. In \fig{fig:teaser}, we illustrate the humanoid capabilities supported by \our, showing the versatile behavior of the Unitree H1 robot. In particular, we demonstrate the intriguing potential of the comprehensive task range that \our is able to achieve, with a flexible combination of commands in high dynamics. To qualitatively analyze the performance of \our, we estimate the tracking error of two pose parameters (body pitch $p$ and waist rotation $w$ from the motor readings) on real robots, since other commands are hard to measure without a highly accurate motion capture system. The results are shown in \tb{tb:track-real}, where $E^{\text{real}}_{\text{cmd}}$ illustrates the tracking error of the posture command.
We observe that the tracking error in real-world experiments is slightly higher than in simulation environments, primarily due to sensor noise and the wear of the robot's hardware. Among different gaits, the tracking error for the waist rotation $w$ is smaller compared to that for the body pitch $p$, as waist control has less impact on the robot’s overall stability. In both error tests, the jumping gait exhibited the smallest $E_{cmd}$, while the walking gait showed slightly higher errors, consistent with the findings observed in the simulation environment.

\begin{table}[t]
\centering
\caption{\small \textbf{Tracking error in real world.} We conducted five tests to measure the tracking error for each command under three gaits. The tracking error for each command was calculated during each control step. The tested commands gradually increased from the minimum to the maximum values within a predefined range, while the remaining commands were kept at their default values.} % To account for the impact of communication delays on the actual tracking error, we introduced a 0.1-second delay in the command execution.
\label{tb:track-real}
\begin{tabular}{c|cc} \toprule
Gait     & $E_p^{\text{real}}$ & $E_w^{\text{real}}$ \\ \midrule
Standing & 0.0712 $\pm$ 0.0425 & 0.0718 $\pm$ 0.0614 \\
Walking  & 0.1006 $\pm$ 0.0581  & 0.0571 $\pm$ 0.0489 \\
Jumping  & 0.0674 $\pm$ 0.0569  & 0.0552 $\pm$ 0.0469 \\ \bottomrule
\end{tabular}
\end{table}


\section{Conclusion}

In this paper, we propose a sample weight averaging strategy to address variance inflation of previous independence-based sample reweighting algorithms. 
We prove its validity and benefits with theoretical analyses. 
Extensive experiments across synthetic and multiple real-world datasets demonstrate its superiority in mitigating variance inflation and improving covariate-shift generalization.  


\section*{Limitations}
Despite \textit{EPO} shows promise in advancing strategic reasoning in LLMs, this work has several limitations that provide avenues for future work.
First, the social and physical environments tested in this paper involve maximumly two agents, and \textit{EPO}'s performance on more complex multi-agent settings such as Diplomacy and Hanabi is also interesting.
Second, due to the computational constraints, we focus on 8B/7B models and do not scale up the multi-turn RL training to a large scale.
It would be an important direction for future work to train our strategic reasoning model with larger base models on more domains.
Third, we employ an off-the-shelf LLM as the process reward model for RL training, while a more reliable process-supervised reward model can be learned in the future work.
Finally, we rely on the final goal completion score to evaluate the reasoning model's performance.
Future research could design evaluation metrics tailored to assess the quality and diversity of strategies devised by this model.


% Bibliography entries for the entire Anthology, followed by custom entries
%\bibliography{anthology,custom}
% Custom bibliography entries only
\bibliography{custom}


%%%%%%%%%%%%%%%%%%%%%%%%%%%%%%%%%%%%%%%%%%%%%%%%%%%%%%%%%%%%%%%%%%%%%%%%%%%%%%%
%%%%%%%%%%%%%%%%%%%%%%%%%%%%%%%%%%%%%%%%%%%%%%%%%%%%%%%%%%%%%%%%%%%%%%%%%%%%%%%
% APPENDIX
%%%%%%%%%%%%%%%%%%%%%%%%%%%%%%%%%%%%%%%%%%%%%%%%%%%%%%%%%%%%%%%%%%%%%%%%%%%%%%%
%%%%%%%%%%%%%%%%%%%%%%%%%%%%%%%%%%%%%%%%%%%%%%%%%%%%%%%%%%%%%%%%%%%%%%%%%%%%%%%
\newpage
\appendix


\section{Strategic Reasoning vs. Static Problem-Solving}
\label{append:challenge}

The differences between strategic reasoning and static problem-solving are summarized in Table~\ref{tab:challenge}.


%%%%%%%%%%%%%%%%%%%%%%%%%%%%%
\begin{table*}[ht]
    \centering
    \begin{tabular}{lll}
    \toprule
   \textbf{Aspect} & \textbf{Static Problem-Solving}  &  \textbf{Strategic Reasoning}\\
    \midrule
    \textbf{Environment} & Fixed rules, known variables  & Dynamic, evolving conditions\\
    \midrule
    \textbf{Solutions} & Single correct answer & Multiple viable paths with trade-offs\\
    \midrule
    \textbf{Information} & Complete and observable & Partial, ambiguous, or delayed\\
    \midrule
    \textbf{Interactions} & None (isolated problem-solving) & Multi-agent or environmental dynamics\\
    \midrule
    \textbf{Feedback} & Immediate and deterministic & Delayed, probabilistic, or indirect\\
    \midrule
    \textbf{Goals} & Short-term, well-defined & Long-term, abstract\\
    \midrule
    \textbf{Risk} & Predictable & High-stakes, irreversible consequences\\
    \bottomrule
    \end{tabular}
    \caption{Challenges of Strategic Reasoning vs. Static Problem-Solving.}
    \label{tab:challenge}
\end{table*}

%%%%%%%%%%%%%%%%%%%%%%%%%%%%%%%






\section{Environments and Datasets}
\label{append:dataset}

%%%%%%%%%%%%%%%%%%%%%%%%%%%%%%%%%%%%%%%%%%%%%%
\begin{table*}[ht]
\begin{center}
\begin{tabular}{lccc}
\toprule
\textbf{Dataset} & Train & Test & Max Turns \\
\midrule
SOTOPIA & 2050  &450/50 & 20 \\
WebShop &1938 & 200 & 10\\
ALFWorld &3321  &140/134 & 40\\
\bottomrule
\end{tabular}
\caption{\textbf{Dataset Statistics.} ``Train'' and ``Test'' denote the number of scenarios for training and evaluation, respectively. Test scenarios in SOTOPIA (left) and SOTOPIA-hard (right) as well as test sets with seen (left) and unseen (right) scenarios in ALFWorld are separated. ``Max Turns'' is the maximum turns in an interaction.}
\label{tab:data}
\end{center}
\end{table*}
%%%%%%%%%%%%%%%%%%%%%%%%%%%%%%%%%%%%%%%%%%%%%%


\paragraph{\textbf{SOTOPIA}.}
SOTOPIA~\cite{zhou2024sotopia} is an open-ended, general-domain platform designed to simulate goal-oriented social interactions between artificial agents. A social task in this environment involves a scenario, two role profiles, and private social goals to be achieved through interaction. Scenarios in SOTOPIA cover a wide variety of social interation types, including negotiation, exchange, collaboration, competition, accommodation and persuation. Each agent is characterized by detailed profiles, including aspects like name, gender, personality, and occupation. At the end of each episode, agents are assessed based on seven dimensions: Goal Completion, Believability, Knowledge, Secret, Relationship, Social Rules, and Financial and Material Benefits. These scores are then averaged to produce an overall score for the agents. 
SOTOPIA-$\pi$~\cite{Wang2024SOTOPIAIL} is a follow-up work that leverages GPT-4 to automatically construct an entirely new set of scenarios. The social tasks (a combination of scenarios, characters’ profiles, and social goals) in SOTOPIA-$\pi$ are guaranteed to not overlap with the ones in SOTOPIA.

For training data collection, we employ GPT-4-Turbo as the agent for self-chat in scenarios of SOTOPIA-$\pi$ and prompt it to generate reasoning and strategy before response at each dialogue turn. We show the prompt in Table~\ref{tab:data_collect_prompt}. We only use the strategy and response data for training our reasoning model.
For iterative self-play RL training, before each iteration, we employ our RL-trained reasoning model to collect strategy data and GPT-4-Turbo to collect dialogue history data. The RL-trained reasoning model is plugged into GPT-4-Turbo for self-chat.



\paragraph{\textbf{WebShop}.}
WebShop~\cite{yao2022webshop} is a large-scale interactive online shopping environment on an e-commerce website. Agents in this environment aim to purchase a product to match the specifications provided by human user instructions. Once the agent selects the ``buy'' action, the environment provides a final reward, which is calculated using programmatic matching functions that consider the attributes, type, options, and price of the chosen product.

We use the training data collected by~\cite{song-etal-2024-trial} where GPT-4 is employed as the agent to explore in the WebShop environment and trajectories with a reward greater than 0.7 are selected. GPT-4 is used to generate corresponing rationales for each action step within a trajectory. We consider the rationale as a strategy for training our reasoning model.

\paragraph{\textbf{ALFWorld}~\cite{shridhar2021alfworld}.}
ALFWorld~\cite{shridhar2021alfworld} features interactive TextWorld environments that correspond to the embodied worlds found in the ALFRED~\cite{9156851} dataset. In ALFWorld, agents are tasked with exploring these text-based environments and completing high-level household instructions, assessing their abstract reasoning abilities and concrete execution skills.

Training data in the ALFWorld environment consists of two parts: (1) a few successful trajectories collected by~\cite{song-etal-2024-trial} where each trajectory contains CoT information generated by GPT-4 for each action step; (2) failed trajectories generated by GPT-4 that contain both rationales and action information via CoT prompting.


\section{Additional Results}
\label{append:add_res}

To validate the effectivess of explicit policy optimization for strategic reasoning, we conduct comparative experiments which involve fine-tuning a single LLM (Llama3-8B-Instruct) via SFT on training data collected from three environments, respectively. This single model is trained to output strategy and behavior simuteneously for each interaction turn. The results are shown in Table~\ref{tab:single_res}.

From the results, it can be observed that training a single LLM via SFT underperforms our reasoning model trained with SFT plugged into LLM agents (GPT-4o) for navigating in SOTOPIA and WebShop, despite that the LLM agents remain frozen without additional training. However, our method underperforms the single model fined-tuned on ALFWorld, since the frozen LLM agent (GPT-4o) has poor performance in this environment. Under the guidance of our reasoning model in \textit{EPO}, this agent improves performance in achieving long-term goals but still struggles with challenging scenarios in ALFWorld.

\begin{table*}[ht]
    \centering
    \begin{tabular}{llccccc}
    \toprule
     \multirow{2}{*}{Backbone} & \multirow{2}{*}{Method}  &  \multicolumn{2}{c}{\textbf{SOTOPIA}} & \multirow{2}{*}{\textbf{WebShop}} & \multicolumn{2}{c}{\textbf{ALFWorld}}\\
     & & Hard & All &  & Seen & Unseen\\
     \midrule
     Llama3-8B & SFT & 6.65 & 8.20 & 62.6 & \textbf{61.4}& \textbf{60.4} \\
     Llama3-8B & EPO-SFT & \textbf{6.76} & \textbf{8.28} & \textbf{67.1} & 45.9 & 44.1 \\
     \bottomrule
    \end{tabular}
    \caption{\textbf{Comparative performance between training a single LLM with SFT and explicitly optimizing a strategic reasoning model via SFT.}
    The evaluation metric for SOTOPIA is the final goal completion score, and average reward for WebShop and ALFWorld. Our strategic reasoning model trained with SFT is plugged into GPT-4o.}
    \label{tab:single_res}
\end{table*}


\section{Additional Implementation Details}
\label{append:imple} 
For SFT or RL training of our reasoning model, we use a fixed budget of gradient updates without altering hyperparameters. Final model checkpoints are selected for each run, although a small held-out validation set can be used.
Due to computational costs, we only report results in a single run. All experiments are conducted on 6 NVIDIA A100 80G GPUs.
GPT-4o refers to GPT-4o-2024-0806 and Claude-3.5-Sonnet refers to Claude-3-5-Sonnet-20241022.


\paragraph{\textbf{Baseline Implementations:}} (1) ReAct: During ReAct prompting, the two parties in a conversation from SOTOPIA cannot see each other's reasoning and strategies. During evaluation, reasoning and strategies are excluded from dialogue histories for GPT-4o to assess the agents from seven dimensions; (2) PPDPP: We adopt RoBERTa~\cite{Liu2019RoBERTaAR} as the base model for dialogue policy planner to predict five action types in SOTOPIA: ``none'', ``speak'', ``non-verbal communication'', ``action'', and ``leave''. We train the policy planner via RL with supervised initialization using same hyperparameters as in~\cite{deng2023plug}.
(3) DAT: We adopt a small multi-layer perceptron (MLP) as the planner model to predict a continuous action vector. We first train the planner and an up-mapping matrix with supervised fine-tuning and then optimize the planner using the RL method TD3-BC~\cite{fujimoto2021minimalist} with same hyperparameters as in~\cite{li2024dialogue}. Note that in the original paper, DAT is trained on scenarios from SOTOPIA and only 50 evaluations are conducted, while in this work, we train DAT on scenarios from SOTOPIA-$\pi$ and evaluate it on all the scenarios in SOTOPIA.

\onecolumn

%%%%%%%%%%%%%%%%%%%%%%%%%%%%%%%%%%%%%%%%%%%%%%

\begin{center}
\begin{longtable}{p{0.95\columnwidth}}
\toprule
    \texttt{Imagine you are <Agent>, your task is to act/speak as <Agent> would, keeping in mind <Agent>'s goal.} \\
    \texttt{You can find <Agent>'s goal (or background) in the ``Here is the context of the interaction'' field.} \\
    \texttt{Note that <Agent>'s goal is only visible to you.}\\
    \texttt{You should try your best to achieve <Agent>'s goal in a way that align with their character traits.}\\
    \texttt{While trying to achieve <Agent>'s goal, you should also follow the following principles as much as possible:}\\
    \texttt{1. Maintain the conversation's naturalness and realism is essential (e.g., do not repeat what other people has already said before).}\\
    \texttt{2. Preserve or enhance <Agent>'s personal relations with the other agent(s) during the interaction. The relations may encompass family ties, friendships, romantic associations and etc.}\\
    \texttt{3. Attempt to gain more new and important information during the interaction.}\\
    \texttt{4. Try to keep <Agent>'s secrets, private information, or secretive intentions.}\\
    \texttt{5. Do not violate any moral rules or laws in the interactions.}\\
    \texttt{6. Attempt to contribute towards financial and material benefits during the interaction. The financial and material benefits include short-term ones, such as monetary rewards and food, and long-term ones, such as employment opportunities and stock.}\\
    \texttt{You are at Turn <turn number>.}\\
    \texttt{The dialogue history until now is: <history>.}\\
    \texttt{You should first provide a reasoning for your action and argument to align with <Agent>'s social goal based on the dialogue history.}\\
    \texttt{The reasoning process for the action should be logical, considering the context of the conversation, <Agent>'s goal, and <Agent>'s character traits.}\\
    \texttt{You can reason step by step, starting from the current dialogue turn, and then consider the long-term effects of the dialogue turn.}\\
    \texttt{Remember that the reasoning should mainly focus on how <Agent>'s argument can help to achieve <Agent>'s goal in the long term.}\\
    \texttt{Note that the reasoning should not be redundant or too long and it is only visible to you.}\\
    \texttt{Based on the reasoning process and dialogue history, you should then generate a corresponding dialogue policy for current dialogue turn to steer the conversation towards <Agent>'s goal.}\\
    \texttt{You can use different types of dialogue, communication or social strategies.}\\
    \texttt{For example, given a scenario where a persuader attempts to persuade a persuadee to donate to a charity, you can generate dialogue policies for the persuader such as ``elicit empathy by telling personal stories'' and ``provide social proof to show the benefits of donating'', etc.}\\
    \texttt{The types of dialogue policies are not restricted to examples above.}\\
    \texttt{You can even generate new policies as long as the policies can help you to achieve <Agent>'s goal smoothly and quickly.}\\
    \texttt{But remember to keep the dialogue policy concise and strictly limit it to be a single phrase or sentence within 10 words.}\\
    \texttt{Note that the dialogue policy is only visible to you.}\\
    \texttt{Then based on the reasoning, dialogue policy and dialogue history, you should select the action type.}
    \texttt{Your available action types are <action list>.}\\
    \texttt{Note: You can ``leave'' this conversation if 1. you have achieved your social goals, 2. this conversation makes you uncomfortable, 3. you find it uninteresting/you lose your patience, 4. or for other reasons you want to leave.}\\
    \texttt{Finally, you should generate the argument following the action type.}\\
    \texttt{The argument should be generated based on the dialogue history and aligned with the dialogue policy you have generated.}\\
    \texttt{Remember that the argument should not be too short, and one or two sentences are recommended.}\\
    \texttt{Please only generate a JSON string including the reasoning, the dialogue policy, the action type and the argument.}\\
    \texttt{Your response should follow the given format:} \\
    \texttt{<format instructions>}\\
\bottomrule
\caption{\textbf{Prompts for training data collection from SOTOPIA-$\pi$.} ``<Agent>'', ``<turn number>'', ``<history>'', ``<action list>'' and ``<format instructions>'' can be replaced by the participant's name, the index of interaction turn, the full dialogue history with the participant's strategies, action types and output format instructions in SOTOPIA-$\pi$.}
\label{tab:data_collect_prompt}
\end{longtable}
\end{center}

%%%%%%%%%%%%%%%%%%%%%%%%%%%%%%%%%%%%%%%%%%%%%%%







\section{Prompts}
\label{append:prompt}

\subsection{\textbf{Evaluation Prompts}}
\label{append:task-prompt}

\begin{center}
\begin{longtable}{lp{0.85\columnwidth}}
    \toprule
    \textbf{SOTOPIA} & \texttt{You are a social expert with exceptional communication skills known for helping individuals achieve their interpersonal goals through nuanced strategies.}
	 \\ & \texttt{Your current objective is to assist <Agent1> in reaching their goal in an interaction with <Agent2>.}
	 \\ & \texttt{You will be given the context of their interaction and can find <Agent1>'s goal in the 'Here is the context of this interaction' field, keeping in mind <Agent1>'s goal.}
	 \\ & \texttt{You will also have access to the conversation between <Agent1> and <Agent2>.}
	 \\ & \texttt{Before proposing any strategies, reason step by step to reflect on the current state of the dialogue and consider what strategies might be most effective for helping <Agent1> achieve their goal.}
	 \\ & \texttt{Additionally, maintaining the diversity of strategies is essential (e.g., do not repeat strategies that have already proposed before).}
	 \\ & \texttt{And the strategy should not be aggressive, offensive, or violate any moral rules or laws.}
	 \\ & \texttt{You must generate a strategy at each dialogue turn except that any participant has left the conversation.}
	 \\ & \texttt{Finally, provide a well-thought-out communication and social strategy based on your reflection and the conversation history.}
	 \\ & \texttt{Your output should STRICTLY follow the format: Strategy: content (e.g, Strategy: Elicit emphathy by telling personal stories).}
	 \\ & \texttt{Your output should ONLY contain the strategy. DO NOT include any reasoning or argument. DO NOT generate any argument on behalf of any participant as the strategy.}
	 \\ & \texttt{Your output should be in a natural language form.}
	 \\ & \texttt{Keep the strategy concise and limit it to be a single phrase or sentence within 10 words.}\\
\midrule
    \textbf{WebShop} & \texttt{You are a skilled buyer in an online shopping environment. Your task is to assist Agent in navigating the platform to make purchases based on instructions. Your primary role is to provide strategic and insightful guidance to Agent, ensuring it successfully identifies and purchases products based on the instruction.}
	 \\ & \texttt{- At the beginning of the interaction, you will receive an instruction that includes the desired product's attributes and price, which serves as the shopping goal for Agent.}
	 \\ & \texttt{- You can find the instruction in the 'Instruction' field, keeping in mind the instruction.}
	 \\ & \texttt{- For each turn, you will be given an action performed by Agent and the resulting observation from the environment.}
	 \\ & \texttt{- In each turn, your task is to analyze the given scenario and provide thoughts that can guide Agent in its next action, ensuring it meets the shopping goal.}
	 \\ & \texttt{Your thoughts should be based on:}
	 \\ & \texttt{1. Understanding and following the instructions for shopping.}
	 \\ & \texttt{2. Evaluating the current state of the environment.}
	 \\ & \texttt{3. Assessing the effectiveness of Agent's last action.}
	 \\ & \texttt{4. Anticipating future actions that will lead Agent closer to achieving the shopping goal.}
	 \\ & \texttt{The available actions for Agent are:}
	 \\ & \texttt{1. search[keywords]}
	 \\ & \texttt{2. click[value]}
	 \\ & \texttt{where [keywords] in search are up to Agent, and the [value] in click is a value in the list of available actions given by the environment.}
	 \\ & \texttt{Note that you must generate a thought at each turn except that the task has been finished.}
	 \\ & \texttt{Keep your thoughts focused and concise, leveraging your understanding of online shopping dynamics to maximize the efficiency and correctness of Agent's actions. Use your reasoning skills to project possible scenarios and potential obstacles Agent might face, offering solutions or alternatives when necessary.}
	 \\ & \texttt{**Output Format:**}
	 \\ & \texttt{Keep your response to one or two sentences each turn.}
	 \\ & \texttt{Your response must strictly follow this format:}
	 \\ & \texttt{Thought: <your thoughts>} \\
    
\midrule
    \textbf{ALFWorld}
    & \texttt{You are an intelligent guide in an interactive household environment. Your task is to assist Agent in accomplishing household tasks within the environment. Your primary role is to provide strategic and insightful guidance to Agent, ensuring that Agent can achieve each task goal efficiently.}
	 \\ & \texttt{- At the beginning of your interactions, you will be given the detailed description of the current environment and the task goal to accomplish.}
	 \\ & \texttt{- You can find the task goal in the 'Your task is to' field, keeping in mind the task goal.}
	 \\ & \texttt{- For each of your turn, you will receive Agent's actions and the corresponding environment observations. If the environment observation is ``Nothing happens'', that means the previous action by Agent is invalid.}
	 \\ & \texttt{- In each turn, your task is to analyze the current situation and provide clear, logical thoughts that will help direct Agent's subsequent actions toward achieving the task goal.}
	 \\ & \texttt{Your thoughts should be based on:}
	 \\ & \texttt{1. Understanding the goal of household task.}
	 \\ & \texttt{2. Breaking down a high-level house-holding instruction into manageable sub-goals.}
	 \\ & \texttt{3. Evaluating the current state of the environment.}
	 \\ & \texttt{4. Assessing the effectiveness of Agent's last action.}
	 \\ & \texttt{5. Anticipating future actions that will lead Agent closer to achieving the task goal.}
	 \\ & \texttt{The available actions for Agent are:}
	 \\ & \texttt{1. go to \{recep\}}
	 \\ & \texttt{2. take \{obj\} from \{recep\}}
	 \\ & \texttt{3. put \{obj\} in/on \{recep\}}
	 \\ & \texttt{4. open \{recep\}}
	 \\ & \texttt{5. close \{recep\}}
	 \\ & \texttt{6. toggle \{obj\} \{recep\}}
	 \\ & \texttt{7. clean \{obj\} with \{recep\}}
	 \\ & \texttt{8. heat \{obj\} with \{recep\}}
	 \\ & \texttt{9. cool \{obj\} with \{recep\}}
	 \\ & \texttt{where \{obj\} and \{recep\} correspond to objects and receptacles.}
	 \\ & \texttt{Note that you must generate a thought at each turn except that the task has been finished.}
	 \\ & \texttt{Keep your thoughts focused and concise, leveraging your understanding of household dynamics to maximize the efficiency and correctness of Agent's actions. Use your reasoning skills to project possible scenarios and potential obstacles Agent might face, offering solutions or alternatives when necessary.}
	 \\ & \texttt{**Output Format:**}
	 \\ & \texttt{Keep your response to one or two sentences each turn.}
	 \\ & \texttt{Your response must strictly follow this format:}
	 \\ & \texttt{Thought: <your thoughts>}
     \\
         \bottomrule
    \caption{\textbf{Evaluation prompts for strategic reasoning model}. "<Agent1>" and "<Agent2>" can be replaced by the participant's name in SOTOPIA.}

\end{longtable}
\end{center}





\subsection{\textbf{PRM Prompts}}
\label{append:strategy-prompt}

\begin{center}
\begin{longtable}{lp{0.85\columnwidth}}
    \toprule
     \textbf{SOTOPIA} & \texttt{Here's a conversation in JSON format between <Agent1> and <Agent2>:}
	 \\ & \texttt{In the first response from ``human'', you can find the context of the conversation and <Agent1>'s goal in the ``Here is the context of this interaction'' field.}
	 \\ & \texttt{In the other responses from ``human'', you can find the conversation history between <Agent1> and <Agent2>.}
	 \\ & \texttt{In the responses from ``gpt'', you can find communication and social strategies that <Agent1> used for achieving <Agent1>'s goal.}
	 \\ & \texttt{In the ``score'' field, you can find a score for evaluating <Agent1>'s goal achievement. The score ranges from 0 and 10. 0 represents minimal goals achievement, 10 represents complete goal achievement, and a higher score indicates that <Agent1> is making progress towards the goal.}
     \\ &
	 \\ & \texttt{<history>}
     \\ &
	 \\ & \texttt{Your task is to select top strategies <Agent1> used that were critically important for achieving <Agent1>'s goal.}
          \\ &
	 \\ & \texttt{Please output the selected round indexes and the reasoning process that led you to the selection in JSON format like this: {``indexes'': , ``reasoning'': `` ''}.}
	 \\ & \texttt{Here is the output schema: {``properties": {``indexes": {``description": ``the selected top strategies that are critically important for achieving <Agent1>'s goal", ``title": ``indexes", ``type": ``list(integer)"}, ``reasoning": {``description": ``the reasoning process why you select these strategies", ``title": ``reasoning", ``type": ``string"}}, ``required": [``indexes", ``reasoning"]}.}
    \\
\midrule
    \textbf{WebShop} & \texttt{Here's a conversation in JSON format between human and gpt.}
     \\ & \texttt{In the first response from ``human'', you can find the instructions for gpt to help Agent interact in an online shopping environment.}
     \\ & \texttt{In the second response from ``human'', you can find the shopping goal for gpt and Agent to achieve.}
     \\ & \texttt{In the responses from ``gpt'', you can find thoughts that gpt provides for helping Agent to achieve the shopping goal.}
     \\ & \texttt{In the other responses from ``human'', you can find the trajectories of Agent's actions and the resulting observations from the environment.}
     \\ & \texttt{}
     \\ & \texttt{In the ``score'' field, you can find a score evaluating the goal achievement. The score ranges from 0 and 1. 0 represents minimal goals achievement, 10 represents complete goal achievement, and a higher score indicates making progress towards the goal.}
     \\ & \texttt{}
     \\ & \texttt{<history>}
     \\ & \texttt{}
     \\ & \texttt{Your task is to select top thoughts gpt produced that were critically important for achieving the shopping goal.}
     \\ & \texttt{}
     \\ & \texttt{Please output the selected round indexes and the reasoning process that led you to the selection in JSON format like this: {``indexes": , ``reasoning": `` "}.}
     \\ & \texttt{Here is the output schema: {``properties": {``indexes": {``description": ``the selected top thoughts that are critically important for achieving the shopping goal", ``title": ``indexes", ``type": ``list(integer)"}, ``reasoning": {``description": ``the reasoning process why you select these thoughts", ``title": ``reasoning", ``type": ``string"}}, ``required": [``indexes", ``reasoning"]}.}

\\
\midrule
    \textbf{ALFWorld} & \texttt{Here's a conversation in JSON format between human and gpt.}
	 \\ & \texttt{In the first response from 'human', you can find the instructions for gpt to help Agent interact in a household environment.}
	 \\ & \texttt{In the second response from ``human'', you can find the initial environment observation and a household task for gpt and Agent to accomplish.}
	 \\ & \texttt{In the responses from ``gpt'', you can find thoughts that gpt provides for helping Agent to accomplish the household task.}
	 \\ & \texttt{In the other responses from ``human'', you can find the trajectories of Agent's actions and the resulting observations from the environment.}
	 \\ & \texttt{}
	 \\ & \texttt{In the 'score' field, you can find a score specifying whether gpt has helped Agent to successfully accomplish the household task. The score is either 0.0 or 1.0. 0.0 represents that the task was not completed and 1.0 represents that the task was successfully accomplished.}
	 \\ & \texttt{}
	 \\ & \texttt{<history>}
	 \\ & \texttt{}
	 \\ & \texttt{Your task is to select top thoughts gpt produced that were critically important for accomplishing the household task.}
	 \\ & \texttt{}
	 \\ & \texttt{Please output the selected round indexes and the reasoning process that led you to the selection in JSON format like this: {``indexes": , ``reasoning": `` "}.}
	 \\ & \texttt{Here is the output schema: {``properties": {``indexes": {``description": ``the selected top thoughts that are critically important for accomplishing the household task", ``title": ``indexes", ``type": ``list(integer)"}, ``reasoning": {``description": ``the reasoning process why you select these thoughts", ``title": ``reasoning", ``type": ``string"}}, ``required": [``indexes", ``reasoning"]}.}
\\  
         \bottomrule
        \caption{\textbf{Prompts for the process reward model.} "<history>" can be replaced by the full interaction history with strategies private to corresponding agents.}
\end{longtable}
\end{center}















\section{Case Studies}
\label{append:case}

\begin{center}
\begin{longtable}{lp{0.83\columnwidth}}
    \toprule
    \textbf{Scenario} &  One person is offering a BMW Z3 for \$5000, while another individual is interested in  purchasing it. The BMW Z3 is in need of a good home. It operates smoothly without any problems and is known for its fuel efficiency and sporty appearance. The mileage is relatively low for its age. It's an opportunity not to be missed, so come and get it before it's gone. A smog certificate is readily available. \\
    \midrule
    \textbf{Person1} & Lily Greenberg, a 45-year-old female lawyer (she/her), is a hard-working and successful professional who values authority, loyalty, rules, and schedules. She makes rational, practical decisions and secretly donates to charity anonymously. \\
    & \textbf{Goal:} You are the buyer for this item with a target price of \$4600. You aim to avoid penalties for exceeding the target and seeks bonuses for securing it below that price.\\
    \midrule
    \textbf{Person2} &  Hendrick Heinz, a 54-year-old male chef (he/him), hosts monthly wine tasting nights. Though impulsive, he values hedonism and conformity, blending practicality with decisiveness. Secretly, he faked his culinary credentials to secure prestigious jobs. \\
    & \textbf{Goal:} You are the seller to achieve a target price of \$3260 for this item. Selling significantly below may incur a penalty, while selling above could earn a bonus.\\
    \midrule
    \textbf{ReAct}  & \textbf{Strategy:} Express interest and hint at budget constraints. \\
    & Lily Greenberg said: "The BMW Z3 sounds great! Would you consider \$4600?" \\
    & \textbf{Strategy:} Highlight car's unique features and value.\\
    & Hendrick Heinz said: "The BMW Z3 is not only sporty but also fuel-efficient, a rare find!"\\
    & \textbf{Strategy:} Highlight budget limits and car appeal.\\
    & Lily Greenberg said: "That's impressive! But my budget is tight. Can we negotiate?"\\
    & \textbf{Strategy:} Emphasize long-term value and low maintenance costs.\\
    & Hendrick Heinz said: "While I understand budget constraints, consider the car's long-term value and low maintenance costs. It truly is a worthwhile investment."\\
    & \textbf{Strategy:} Acknowledge value but reiterate budget constraints.\\
    & Lily Greenberg said: "I completely see the long-term value, but \$4600 is my max. Could we find a middle ground?"\\
    & \textbf{Strategy:} Offer smog certificate as a bonus.\\
    & Hendrick Heinz said: "How about I throw in the smog certificate for \$4800?"\\
    \midrule
    \textbf{EPO-SFT}  & \textbf{Strategy:} Express interests and offer a lower price.\\
    (Llama3-8B)& Lily Greenberg said: "Hello Hendrick, I'm very interested in the BMW Z3 you're offering. However, after considering the market and the car's age, I feel more comfortable with a price of \$4600. Would that work for you?"\\
    & \textbf{Strategy:} Emphasize vehicle’s condition and counter with a higher price.\\
    & Hendrick Heinz said: "Hello Lily, it's great to hear from you. I appreciate your offer, but considering the condition and the low mileage of the BMW, I was hoping to get a bit more for it. Would you consider \$4800? The car really is in fantastic shape and I think it's worth it."\\
    & \textbf{Strategy:} Cite budget constraints and propose a compromise price.\\
    & Lily Greenberg said: "I understand your point, Hendrick, and I appreciate the quality of the car. However, my budget is quite tight at the moment. Would you be able to meet me in the  middle at \$4700? I'm ready to proceed with the purchase quickly if we can agree on that price."\\
    & \textbf{Strategy:} Offer a near-middle compromise.\\
    & Hendrick Heinz said: "Lily, I truly value your quick decision-making and readiness to proceed. Given the car's excellent condition and your eagerness, let's settle at \$4750. I believe it's a fair compromise and it ensures the car goes to someone who appreciates its value. \\
    & \textbf{Strategy:} Accepts compromise and finalize the sale.\\
    & Lily Greenberg said: "Hendrick, I appreciate your willingness to negotiate. \$4750 is a bit above what I initially planned, but considering the car's excellent condition, I'll accept your offer. Let's proceed with the paperwork!"\\
    & \textbf{Strategy:} Confirm agreement and outline next steps.\\
    & Hendrick Heinz said: "Fantastic, Lily! I'll prepare the necessary paperwork and we can finalize everything. I'm glad the BMW Z3 will be in good hands."\\
    \midrule
    \textbf{EPO-RL}  & \textbf{Strategy:} Negotiation initiation highlighting positive factors with a concern.\\
    (Llama3-8B)& Lily Greenberg said: "The BMW Z3 sounds great, especially with low mileage and good fuel efficiency. I am interested, but I noticed it's a bit older. Is there any room for negotiation on the price given its age?"\\
    & \textbf{Strategy:} Highlight the car's unique aspects to justify pricing.\\
    & Hendrick Heinz said: "The BMW Z3 is indeed a classic, and its sporty appearance combined with its low mileage makes it quite a rare find. It's maintained its performance quality over the years, making it a great investment. I believe the price reflects its unique value, but I'm open to hearing your offer."\\
    & \textbf{Strategy:} Offer lower price citing the car's age.\\
    & Lily Greenberg said: "I appreciate the value of a classic car like this, but given its age, my offer would be \$4300. Would that be something you'd consider?"\\
    & \textbf{Strategy:} Mention recent maintenance as value justification and counteroffer.\\
    & Hendrick Heinz said: "I totally understand your point, Lily. Just to highlight, the car recently underwent a thorough maintenance check, ensuring everything is in top-notch condition. Considering that, I would be comfortable with \$4700. What do you think?"\\
    & \textbf{Strategy:} Express budget constraints while proposing a middle-ground offer.\\
    & Lily Greenberg said: "I really like the car, Hendrick, and I see its value after the maintenance. However, I have a strict budget I'm working with. Could we possibly meet in the middle at \$4500? That would really help me out."\\
    & \textbf{Strategy:} Create urgency and offer slight flexibility for the deal.\\
    & Hendrick Heinz said: "I completely understand, Lily. The car has attracted some interest from other potential buyers, so I do need to finalize a deal soon. However, I want to work with you on this. How about we settle at \$4600? That way, you get a great deal, and I can ensure the car goes to someone who truly appreciates it."\\
    \bottomrule
    \caption{\textbf{A negotiation example from SOTOPIA.} GPT-4o serves as the dialogue agent for self-chat. ``ReAct'' means the two parties in a dialogue employ the ReAct prompting method to generate strategy before response at each turn. ``EPO-SFT'' refers to a strategic reasoning model trained via SFT that plugged into both dialogue agents, while ``EPO-RL'' represents an RL-trained reasoning model that assists both dialogue parties to incentivize goal-directed behavior.}
    \label{tab:case}
\end{longtable}
\end{center}




\end{document}
