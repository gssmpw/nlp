% This must be in the first 5 lines to tell arXiv to use pdfLaTeX, which is strongly recommended.
\pdfoutput=1
% In particular, the hyperref package requires pdfLaTeX in order to break URLs across lines.


\documentclass[11pt]{article}

% Change "review" to "final" to generate the final (sometimes called camera-ready) version.
% Change to "preprint" to generate a non-anonymous version with page numbers.
\usepackage[preprint]{acl}

% Standard package includes
\usepackage{times}
\usepackage{latexsym}

% For proper rendering and hyphenation of words containing Latin characters (including in bib files)
\usepackage[T1]{fontenc}
% For Vietnamese characters
% \usepackage[T5]{fontenc}
% See https://www.latex-project.org/help/documentation/encguide.pdf for other character sets

% This assumes your files are encoded as UTF8
\usepackage[utf8]{inputenc}

% This is not strictly necessary, and may be commented out,
% but it will improve the layout of the manuscript,
% and will typically save some space.
\usepackage{microtype}

% This is also not strictly necessary, and may be commented out.
% However, it will improve the aesthetics of text in
% the typewriter font.
\usepackage{inconsolata}
\usepackage{longtable}

%Including images in your LaTeX document requires adding
%additional package(s)
\usepackage{graphicx}
\usepackage{subcaption}
\usepackage{booktabs} % for professional tables
\usepackage{multirow}
\usepackage{arydshln}
\newcommand{\ccol}[1]{\multicolumn{1}{c}{#1}}
\usepackage{hyperref}
\newcommand{\theHalgorithm}{\arabic{algorithm}}
\usepackage{amsmath}
\usepackage{amssymb}
\usepackage{mathtools}
\usepackage{amsthm}
\usepackage{color}
% if you use cleveref..
\usepackage[capitalize,noabbrev]{cleveref}

%%%%%%%%%%%%%%%%%%%%%%%%%%%%%%%%
% THEOREMS
%%%%%%%%%%%%%%%%%%%%%%%%%%%%%%%%
\theoremstyle{plain}
\newtheorem{theorem}{Theorem}[section]
\newtheorem{proposition}[theorem]{Proposition}
\newtheorem{lemma}[theorem]{Lemma}
\newtheorem{corollary}[theorem]{Corollary}
\theoremstyle{definition}
\newtheorem{definition}[theorem]{Definition}
\newtheorem{assumption}[theorem]{Assumption}
\theoremstyle{remark}
\newtheorem{remark}[theorem]{Remark}

% Todonotes is useful during development; simply uncomment the next line
%    and comment out the line below the next line to turn off comments
%\usepackage[disable,textsize=tiny]{todonotes}
\usepackage[textsize=tiny]{todonotes}

% If the title and author information does not fit in the area allocated, uncomment the following
%
%\setlength\titlebox{<dim>}
%
% and set <dim> to something 5cm or larger.

\title{EPO: Explicit Policy Optimization for Strategic Reasoning in LLMs via Reinforcement Learning}
% Enhancing Strategic Reasoning in LLMs for Long-term Goal Alignment via Reinforcement Learning
% Collaborative Strategic Reasoning Elicits Goal-Oriented Behavior from Large Language Models
% Collaborative Strategic Reasoning Enhances Decision-Making of Large Language Models
% Cooperative Strategic Reasoning Enhances Long-term Planning of Large Language Model Agents
% Inference-Time Policy Steering through Strategic Reasoning Models
% Steering Large Language Model Agents with Strategic Reasoning Models


% Author information can be set in various styles:
% For several authors from the same institution:
% \author{Author 1 \and ... \and Author n \\
%         Address line \\ ... \\ Address line}
% if the names do not fit well on one line use
%         Author 1 \\ {\bf Author 2} \\ ... \\ {\bf Author n} \\
% For authors from different institutions:
% \author{Author 1 \\ Address line \\  ... \\ Address line
%         \And  ... \And
%         Author n \\ Address line \\ ... \\ Address line}
% To start a separate ``row'' of authors use \AND, as in
% \author{Author 1 \\ Address line \\  ... \\ Address line
%         \AND
%         Author 2 \\ Address line \\ ... \\ Address line \And
%         Author 3 \\ Address line \\ ... \\ Address line}

%\author{Xiaoqian Liu \\
%   University of Chinese Academy of Sciences\\
%  Insititute of Automation, Chinese Academy of Sciences \\
%  \texttt{email@domain} \\\And
%  Ke Wang \\
%  Tongyi Lab \\
%  \texttt{email@domain} \\}

\author{
  \textbf{Xiaoqian Liu\textsuperscript{1,3}},
  \textbf{Ke Wang\textsuperscript{2}},
\textbf{Yongbin Li\textsuperscript{2}$^{*}$},
  \textbf{Yuchuan Wu\textsuperscript{2}},
  \textbf{Wentao Ma\textsuperscript{2}},\\
  \textbf{Aobo Kong\textsuperscript{2}},
  \textbf{Fei Huang\textsuperscript{2}},
  \textbf{Jianbin Jiao \textsuperscript{1}},
  \textbf{Junge Zhang\textsuperscript{3}\thanks{Corresponding authors}}
%  \textbf{Tenth Author\textsuperscript{1}},
%  \textbf{Eleventh E. Author\textsuperscript{1,2,3,4,5}},
%  \textbf{Twelfth Author\textsuperscript{1}},
%\\
%  \textbf{Thirteenth Author\textsuperscript{3}},
%  \textbf{Fourteenth F. Author\textsuperscript{2,4}},
%  \textbf{Fifteenth Author\textsuperscript{1}},
%  \textbf{Sixteenth Author\textsuperscript{1}},
%\\
%  \textbf{Seventeenth S. Author\textsuperscript{4,5}},
%  \textbf{Eighteenth Author\textsuperscript{3,4}},
%  \textbf{Nineteenth N. Author\textsuperscript{2,5}},
%  \textbf{Twentieth Author\textsuperscript{1}}
%\\
\\
  \textsuperscript{1}University of Chinese Academy of Sciences,
 \textsuperscript{2}Tongyi Lab\\
  \textsuperscript{3}Insititute of Automation, Chinese Academy of Sciences
%  \textsuperscript{4}Affiliation 4,
%  \textsuperscript{5}Affiliation 5
%\\
%  \small{
%    \textbf{Correspondence to:} \href{mailto:email@domain}{email@domain}
%  }
}

\begin{document}
\maketitle

\begin{abstract}
Out-of-distribution (OOD) detection and OOD generalization are widely studied in Deep Neural Networks (DNNs), yet their relationship remains poorly understood. We empirically show that the degree of Neural Collapse (NC) in a network layer is inversely related with these objectives: stronger NC improves OOD detection but degrades generalization, while weaker NC enhances generalization at the cost of detection. This trade-off suggests that a single feature space cannot simultaneously achieve both tasks. To address this, we develop a theoretical framework linking NC to OOD detection and generalization. We show that entropy regularization mitigates NC to improve generalization, while a fixed Simplex Equiangular Tight Frame (ETF) projector enforces NC for better detection. Based on these insights, we propose a method to control NC at different DNN layers. In experiments, our method excels at both tasks across OOD datasets and DNN architectures. 

\begin{comment}   

Out-of-distribution (OOD) detection and OOD generalization are critical for deploying machine learning models in real-world scenarios. While substantial progress has been made in addressing these problems independently, few works have attempted to tackle them jointly. However, existing methods often rely on auxiliary OOD training data and primarily focus on covariate-shifted OOD data that share labels with in-distribution (ID) data. In contrast, we tackle the more realistic and challenging task of jointly detecting and generalizing to semantic OOD data with disjoint labels from the ID data, without auxiliary OOD training data.
Achieving both objectives simultaneously is inherently difficult due to a fundamental conflict — OOD generalization requires enhanced transferability, while OOD detection necessitates the inhibition of transfer.
To address this, we leverage insights from neural collapse (NC) — a phenomenon in deep networks where top-layer representations suppress feature variability and adopt a Simplex Equiangular Tight Frame (ETF) structure, impairing transferability. By controlling NC, we unify OOD detection and generalization: preventing NC enhances OOD transfer while inducing NC improves OOD detection.
Our proposed method excels at both tasks across various OOD datasets and architectures. 

\end{comment}


\end{abstract}

\section{Introduction}
\label{sec:intro}

Foundational models (FMs)~\cite{zhang2024data, zhou2023comprehensive} have shown remarkable progress in the healthcare domain, enabling professional-like assessment of disease diagnosis, treatment decision-making, and monitoring~\cite{zhang2023text, wang2022medclip, lu2023mi-zero}. 
Examples include LLaVA-Med~\cite{li2023llava}, Med-PaLM Multimodal~\cite{tu2024towards}, and Med-Flamingo~\cite{moor2023med}, have demonstrated their capacity on question answering, medical image analysis, and report generation.
These studies follow a predominant top-down model development strategy that requires upstream developers to collect data and train models for downstream tasks. 
Consequently, the developed model capabilities are heavily dependent on the training data, limiting their generalization performance in diverse clinical scenarios. 
For instance, Med-Gemini~\cite{yang2024advancing} reveals promising general capabilities in report generation while it lags behind state-of-the-art (SoTA) models on classification tasks, especially for out-of-domain applications. 
This indicates that while the generalizability of the foundation model is promising, more solutions are expected to meet the various specialized clinical needs.

To address these challenges, multi-center data centralization becomes essential to enhance model capacity and robustness across varied clinical scenarios~\cite{rajpurkar2022ai}. 
Centralizing distributed data can significantly improve model training and inference performance.
However, the process of medical data storage, transfer, and aggregation among centers requires extra efforts to ensure data security and system interoperability~\cite{bradford2020international}.
Moreover, a growing concern for patient privacy makes large-scale multi-center data sharing particularly challenging. 
While efforts like federated learning~\cite{wen2023survey, li2020review} can achieve good model performance on local data, the need for synchronized system coordination presents significant challenges, as clients are unable to update asynchronously. This limitation greatly restricts the practical capability of such approaches.
As a result, without a flexible collaboration, medical community still struggles to fully utilize the isolated data and local computation resources for comprehensive medical AI model development. 
To address this dilemma, open-source platforms encourage public data sharing and knowledge integration~\cite{markiewicz2021openneuro, zenodo}.
However, these platforms focus solely on raw data sharing while seldom providing collaborative model training or cooperation between different institutions.
Recently, collaborative learning has emerged as a viable approach for enhancing multi-model robustness~\cite{boulemtafes2020review}. 
For instance, software-like model development~\cite{raffel2023building} mimics software engineering practices by introducing structured workflows, enabling merging, version control, and continuous model integration.
Under this design, model ability can be strengthened with incremental knowledge updates similar to the version updating in software development. 

Although collaborative learning provides a multi-model collaboration, two key challenges remain in the leakage of raw data during collaboration~\cite{huang2023lorahub} and the synchronization of multiple collaborators~\cite{mcmahan2017communication} in the medical AI community. It is still challenging to integrate decentralized, privacy-sensitive data across institutions, leading to under-utilized insights and fragmented knowledge sharing~\cite{kaissis2020secure, rajpurkar2022ai, abdullah2021ethics}.
 To address these challenges, inspired by the collaborative software development, we propose \textbf{Med}ical \textbf{Fo}undation Models Me\textbf{rg}ing (\textbf{MedForge}), a cooperative workflow enabling continuously community-driven foundation model (FM) development.
MedForge enables a lightweight manner for individual centers to share their knowledge among multiple centers, minimizing the burden of data transmission and integration while enhancing model robustness.
Meanwhile, MedForge facilitates asynchronous and flexible collaboration, allowing individual centers to continuously update and improve medical FMs without the need for real-time synchronization.
Similar to open-source software development, MedForge incrementally updates medical knowledge and follows a sustainable model development scheme. 
This key design emphasizes a bottom-up construction of a multi-task medical FM, allowing downstream users to collaboratively build, refine, and update the upstream model according to their local resources. Our major contributions of MedForge are as below: 
\begin{enumerate}
    \item[$\bullet$] We introduce a collaborative workflow to promote the merging scheme of open-source software development. Our proposed MedForge allows distributed clinical centers to asynchronously contribute to comprehensive medical model construction while reducing transmitting costs among centers and avoiding the leakage of raw data, thus enhancing the utilization of private resources in the healthcare system. 
    \item[$\bullet$] We propose two effective knowledge-merging strategies for the asynchronous branch contribution. The MedForge-Fusion strategy updates the plugin module parameters of the main model during the merging phase, whereas the MedForge-Mixture strategy integrates the output of the plugin module by memorizing each contributor's coefficient. These strategies make MedForge more flexible and versatile. MedForge-Fusion is friendly to implement, while the MedForge-Mixture offers better performance and robustness.
    \item[$\bullet$]  We comprehensively evaluate model merging strategies to accumulate medical knowledge among multiple branch plugin modules. MedForge yields superior performance on medical classification tasks compared to other collaborative baselines across multiple datasets. We demonstrate the robustness of MedForge by shuffling the task order and evaluating various configurations of plugin modules and dataset distillation methods.
\end{enumerate}




\section{Related Work}
\label{sec:related}
\subsection{Collaborative Systems}
In the era of rapid growth in medical foundational models~\cite{huang2023visual,wang2022medclip, zhang2024data}, the top-down model development paradigm limits model capabilities by heavily relying on the resources available to the model builders. 
Such paradigm often restricts the potential of these models, as they cannot effectively utilize the diverse, private, and decentralized resources that exist within the broader medical community.
In contrast, collaborative systems present a promising alternative, offering a more flexible approach to model development.

Collaborative systems enable institutions to share knowledge by allowing distributed collaborators to contribute to a common goal~\cite{boulemtafes2020review}. 
To further protect patient privacy, federated learning (FL)~\cite{mcmahan2017communication} was proposed to alleviate such privacy concerns as server aggregating parameter updates from multiple clients without sharing their local data. 
While subsequent optimizations, such as aggregation algorithms~\cite{mcmahan2017communication, zhao2018federated, li2020federated}, secure learning~\cite{hardy2017private, xie2021crfl}, fairness improvements~\cite{sharma2022federated, zhao2022dynamic} and its application in medicine~\cite{kumar2024privacy}, have enhanced the capacity and applicability of FL, its real-world flexibility remains limited. This is primarily due to the need for synchronous updates, which require the server and clients to stay in sync, or model updates will be blocked.
This synchrony issue can be mitigated by open-source software platforms (e.g., GitHub~\cite{github}), allowing independent contributions from individual developers asynchronously. Such an asynchronous scheme enables faster iteration and the integration of specialized expertise, thus offering a more flexible and scalable approach.

Unlike synchronous collaboration, asynchronous collaboration does not require collaborators to work simultaneously and collaborators can individually complete their updates.
While the concept of asynchronous collaboration has been widely used in software development, its machine-learning applications remain limited~\cite{kandpal2023git, raffel2023building}. 
With the rise of global collaboration, large models~\cite{sahajBERT, le2023bloom} are usually co-developed by collaborators given various levels of data availability. However, this collaborative scheme requires the aggregation of local data and online synchronous cooperation of developers.
Software-like model update system~\cite{raffel2023building} alleviates the synchronous problem, where models are updated incrementally, similar to software development, by introducing merging and version control to model development.
However, the existing collaborative version control system~\cite{kandpal2023git} fails to address the complexities of medical scenarios because of the heavy dependency on plain parameter averaging across the full model without accounting for the varying requirements of different tasks.
To respond, we propose MedForge, which enables an asynchronous collaborative system and ensures strong robustness toward a continuous, community-driven enhancement of medical models while overcoming potential data leakage.

\begin{figure*}[t]
\begin{center}
\includegraphics[width=.85\linewidth]{fig_overview_v3.pdf}
\end{center}
\caption{
FastAtlas Overview: In each frame, we compute charts spanning fully or partially visible triangles (a), determine texture space bounding boxes for the visible portions of the view-space projections of each chart, and tightly pack these boxes into atlases (b, here $2K \times 2K$). We simultaneously bijectively parameterize and shade the charts into their atlas boxes, obtaining high quality texture space shading (c), and use this shading to render the shaded frames (d).}
\label{fig:overview}
\label{fig:alg_overview}
\end{figure*}

\section{Overview}
\label{sec:overview}
Our work has two core contributions: a real-time, GPU-based algorithm for tight packing of general parameterized charts into compact atlases; and a real-time TSS method that
utilizes this packing.  

\paragraph*{FastAtlas Packing.}
FastAtlas runs entirely on the GPU as a series of compute shaders. It takes the bounding boxes of parameterized charts as input, and packs them into an atlas (Fig~\ref{fig:overview}b, Sec.~\ref{sec:pack}). As such, the only input it requires are the dimensions of the bounding boxes.
Its outputs are deterministic; identical input charts are packed into identical atlases. This is critical for TSS and similar applications, as it ensures that consecutive frames taken from the same camera view have the same shading. Even minute shading differences across such frames can cause sampling jitter, leading to undesirable flicker \cite{baker2012rock}. 
While prior methods such as \cite{mueller2018shading,hladky2019tessellated,hladky2021snakebinning,Neff2022MSA} cap the dimensions of the charts that can be packed as-is for a given atlas size, and scale down all charts that exceed these dimensions, we scale all charts by the same factor, and do so only when strictly necessary to achieve packing success (Figs~\ref{fig:atlas},~\ref{fig:sas_issues}). 

\paragraph*{TSS using FastAtlas.}
Our end-to-end TSS atlas generation method combines the packing method above with a novel approach for computing seamless per-frame charts. 
We define our charts as the connected components of the visible surfaces in each frame (Fig.~\ref{fig:overview}a), and efficiently compute them using a parallel union-find algorithm (Sec.~\ref{sec:visible}). Since the boundaries of these charts coincide with the contours of the rendered surface, they are {\em invisible} to the viewer. This approach 
eliminates the artifacts caused by shading discontinuities along visible seams (Fig.~\ref{fig:seams}). 

\begin{parWithWrapFigure}
\begin{wrapfigure}{l}{.27\columnwidth}%
\includegraphics[width=\linewidth]{fig_inset_view_plane.pdf}%
\end{wrapfigure}
We bijectively parametrize the {\em visible portions} of our charts by projecting them to view space (inset). This maps a constant number of texels to each pixel in the final rendered output, evenly distributing residual undersampling error across all image pixels. While conceptually straightforward, efficiently parameterizing charts containing partially visible triangles using viewspace projection is non-trivial, as the visible portions may no longer be triangular (e.g. green triangle in the inset); applying naive projection to triangles with vertices behind the camera may produce ill-posed results. Clipping triangles before projection is both computationally expensive and significantly complicates downstream operations. We avoid explicit clipping by observing that all that is required for atlas packing is the dimensions of, potentially conservative, bounding boxes of these projected visible portions. We compute such bounding boxes without explicit chart clipping by adapting a conservative screen coverage estimator \shortcite{Blinn:CalculatingScreenCoverage} (Sec.~\ref{sec:box}). We then pack the computed boxes using FastAtlas. 
\end{parWithWrapFigure}

Finally, we shade the visible portion of each chart into its corresponding atlas bounding box (Fig~\ref{fig:overview}c). 
The resulting texture is then used during rasterization as a standard texture map (Fig. ~\ref{fig:overview}d). 
Our framework is compatible with all existing approaches for texture space shading, including forward shading, raytraced illumination, or deferred shading in texture space \cite{baker:2016}. In the examples shown, we use the standard forward shading based rendering pipeline included in the G3D Innovation Engine \cite{G3D17}, a commercial grade renderer.


\subsection{Model Merging}
In collaborative systems, proper model merging becomes increasingly vital for improving model knowledge integration from multiple sources in a resource-limited environment~\cite{li2023deep, yang2024model, goddard2024arcee}. Conceptually, model merging strategies can be categorized into entire model merging and partial model merging.

Entire model merging involves combining multiple model parameters to participate in the merging process by several means. Entire model merging can be viewed as an optimization problem~\cite{Matena_Raffel_2021, jin2022dataless, mavromatis2024packllm} or an alignment problem~\cite{ainsworth2022git, jordan2022repair, xu2024training, ainsworth2022git}, each offering unique advantages depending on the task at hand.
In the optimization-based approach, the goal is to find the best combination of multiple models to enhance performance and efficiency. For instance, using Fisher information approximation~\cite{Matena_Raffel_2021}, the optimization-based model merging can be interpreted as selecting parameters that maximize the joint likelihood of the models' posterior distributions. The optimization of model merging can also be guided by minimizing the prediction differences between the merged model and individual models~\cite{jin2022dataless}. 
With the development of large language models (LLM), optimization-based method is used to fuse multiple LLMs at test-time by minimizing perplexity over the input prompt~\cite{mavromatis2024packllm}.
To highlight, optimization-based methods are beneficial for scenarios requiring enhanced model performance and efficiency to integrate model parameters, while alignment-based methods~\cite{ainsworth2022git, jordan2022repair} are better suited for maintaining consistency and interpretability, facilitating critical information sharing across models.
For example, a training-free model merging strategy aligns relevant models by using a similarity matrix of their representations in both activation and weight spaces~\cite{xu2024training}.
Further, the alignment between the independently trained model and a reference model not only works for models with the same architecture but also for arbitrary model architectures~\cite{ainsworth2022git}.
In summary, the entire model merging methods can effectively integrate existing models into a merged model with enhanced functionality. However, they could lead to increased computational complexity and reduced flexibility, making them less scalable and harder to implement across diverse tasks.

Partial model merging refers to combining only specific components or layers of models to improve model merging efficiency and decrease the computational cost. 
Such specific components can come from the same network~\cite{kingetsu2021neural}, where the original network is divided into subnetworks for different purposes, and these subnetworks can then be recombined for new tasks.
Additionally, modules can originate from different functional networks and be merged using various strategies. For instance, arithmetic operations are applied in \cite{zhang2023composing} to fuse parameter-efficient modules.
While merging modules from different networks provides flexibility, the process also requires a selection strategy to ensure the resulting model aligns with the specific needs of the inference stage. 
The selection strategies are commonly designed based on the similarity of task~\cite{lv2023parameter} and domain clustering performance~\cite{chronopoulou2023adaptersoup}. Alternatively, the mixture-of-experts methods use a routing strategy to select appropriate component modules~\cite{ponti2023combining}. However, these strategies often require significant time and computational resources to filter through a large model pool. 
In contrast, LoRAHub~\cite{huang2023lorahub} offers a more lightweight approach, combining various LoRA modules for different tasks with minimal model training. Nevertheless, LoRAHub lacks flexibility for incremental updates, especially when handling unseen tasks.

Although the existing model merging approaches effectively combine the capabilities of individual models, these approaches often rely on raw data, leading to potential privacy risks. Our proposed MedForge emphasizes the prevention of raw data usage, which is particularly crucial in medical scenarios. Additionally, MedForge offers an extensible capability for incremental learning, enabling continuous model improvement.




\vspace{-5pt}
\section{Method}
\label{sec:method}
\begin{figure*}[t]
\begin{center}
\includegraphics[width=.85\linewidth]{fig_overview_v3.pdf}
\end{center}
\caption{
FastAtlas Overview: In each frame, we compute charts spanning fully or partially visible triangles (a), determine texture space bounding boxes for the visible portions of the view-space projections of each chart, and tightly pack these boxes into atlases (b, here $2K \times 2K$). We simultaneously bijectively parameterize and shade the charts into their atlas boxes, obtaining high quality texture space shading (c), and use this shading to render the shaded frames (d).}
\label{fig:overview}
\label{fig:alg_overview}
\end{figure*}

\section{Overview}
\label{sec:overview}
Our work has two core contributions: a real-time, GPU-based algorithm for tight packing of general parameterized charts into compact atlases; and a real-time TSS method that
utilizes this packing.  

\paragraph*{FastAtlas Packing.}
FastAtlas runs entirely on the GPU as a series of compute shaders. It takes the bounding boxes of parameterized charts as input, and packs them into an atlas (Fig~\ref{fig:overview}b, Sec.~\ref{sec:pack}). As such, the only input it requires are the dimensions of the bounding boxes.
Its outputs are deterministic; identical input charts are packed into identical atlases. This is critical for TSS and similar applications, as it ensures that consecutive frames taken from the same camera view have the same shading. Even minute shading differences across such frames can cause sampling jitter, leading to undesirable flicker \cite{baker2012rock}. 
While prior methods such as \cite{mueller2018shading,hladky2019tessellated,hladky2021snakebinning,Neff2022MSA} cap the dimensions of the charts that can be packed as-is for a given atlas size, and scale down all charts that exceed these dimensions, we scale all charts by the same factor, and do so only when strictly necessary to achieve packing success (Figs~\ref{fig:atlas},~\ref{fig:sas_issues}). 

\paragraph*{TSS using FastAtlas.}
Our end-to-end TSS atlas generation method combines the packing method above with a novel approach for computing seamless per-frame charts. 
We define our charts as the connected components of the visible surfaces in each frame (Fig.~\ref{fig:overview}a), and efficiently compute them using a parallel union-find algorithm (Sec.~\ref{sec:visible}). Since the boundaries of these charts coincide with the contours of the rendered surface, they are {\em invisible} to the viewer. This approach 
eliminates the artifacts caused by shading discontinuities along visible seams (Fig.~\ref{fig:seams}). 

\begin{parWithWrapFigure}
\begin{wrapfigure}{l}{.27\columnwidth}%
\includegraphics[width=\linewidth]{fig_inset_view_plane.pdf}%
\end{wrapfigure}
We bijectively parametrize the {\em visible portions} of our charts by projecting them to view space (inset). This maps a constant number of texels to each pixel in the final rendered output, evenly distributing residual undersampling error across all image pixels. While conceptually straightforward, efficiently parameterizing charts containing partially visible triangles using viewspace projection is non-trivial, as the visible portions may no longer be triangular (e.g. green triangle in the inset); applying naive projection to triangles with vertices behind the camera may produce ill-posed results. Clipping triangles before projection is both computationally expensive and significantly complicates downstream operations. We avoid explicit clipping by observing that all that is required for atlas packing is the dimensions of, potentially conservative, bounding boxes of these projected visible portions. We compute such bounding boxes without explicit chart clipping by adapting a conservative screen coverage estimator \shortcite{Blinn:CalculatingScreenCoverage} (Sec.~\ref{sec:box}). We then pack the computed boxes using FastAtlas. 
\end{parWithWrapFigure}

Finally, we shade the visible portion of each chart into its corresponding atlas bounding box (Fig~\ref{fig:overview}c). 
The resulting texture is then used during rasterization as a standard texture map (Fig. ~\ref{fig:overview}d). 
Our framework is compatible with all existing approaches for texture space shading, including forward shading, raytraced illumination, or deferred shading in texture space \cite{baker:2016}. In the examples shown, we use the standard forward shading based rendering pipeline included in the G3D Innovation Engine \cite{G3D17}, a commercial grade renderer.


Our goal is to increase the robustness of T2I models, particularly with rare or unseen concepts, which they struggle to generate. To do so, we investigate a retrieval-augmented generation approach, through which we dynamically select images that can provide the model with missing visual cues. Importantly, we focus on models that were not trained for RAG, and show that existing image conditioning tools can be leveraged to support RAG post-hoc.
As depicted in \cref{fig:overview}, given a text prompt and a T2I generative model, we start by generating an image with the given prompt. Then, we query a VLM with the image, and ask it to decide if the image matches the prompt. If it does not, we aim to retrieve images representing the concepts that are missing from the image, and provide them as additional context to the model to guide it toward better alignment with the prompt.
In the following sections, we describe our method by answering key questions:
(1) How do we know which images to retrieve? 
(2) How can we retrieve the required images? 
and (3) How can we use the retrieved images for unknown concept generation?
By answering these questions, we achieve our goal of generating new concepts that the model struggles to generate on its own.

\vspace{-3pt}
\subsection{Which images to retrieve?}
The amount of images we can pass to a model is limited, hence we need to decide which images to pass as references to guide the generation of a base model. As T2I models are already capable of generating many concepts successfully, an efficient strategy would be passing only concepts they struggle to generate as references, and not all the concepts in a prompt.
To find the challenging concepts,
we utilize a VLM and apply a step-by-step method, as depicted in the bottom part of \cref{fig:overview}. First, we generate an initial image with a T2I model. Then, we provide the VLM with the initial prompt and image, and ask it if they match. If not, we ask the VLM to identify missing concepts and
focus on content and style, since these are easy to convey through visual cues.
As demonstrated in \cref{tab:ablations}, empirical experiments show that image retrieval from detailed image captions yields better results than retrieval from brief, generic concept descriptions.
Therefore, after identifying the missing concepts, we ask the VLM to suggest detailed image captions for images that describe each of the concepts. 

\vspace{-4pt}
\subsubsection{Error Handling}
\label{subsec:err_hand}

The VLM may sometimes fail to identify the missing concepts in an image, and will respond that it is ``unable to respond''. In these rare cases, we allow up to 3 query repetitions, while increasing the query temperature in each repetition. Increasing the temperature allows for more diverse responses by encouraging the model to sample less probable words.
In most cases, using our suggested step-by-step method yields better results than retrieving images directly from the given prompt (see 
\cref{subsec:ablations}).
However, if the VLM still fails to identify the missing concepts after multiple attempts, we fall back to retrieving images directly from the prompt, as it usually means the VLM does not know what is the meaning of the prompt.

The used prompts can be found in \cref{app:prompts}.
Next, we turn to retrieve images based on the acquired image captions.

\vspace{-3pt}
\subsection{How to retrieve the required images?}

Given $n$ image captions, our goal is to retrieve the images that are most similar to these captions from a dataset. 
To retrieve images matching a given image caption, we compare the caption to all the images in the dataset using a text-image similarity metric and retrieve the top $k$ most similar images.
Text-to-image retrieval is an active research field~\cite{radford2021learning, zhai2023sigmoid, ray2024cola, vendrowinquire}, where no single method is perfect.
Retrieval is especially hard when the dataset does not contain an exact match to the query \cite{biswas2024efficient} or when the task is fine-grained retrieval, that depends on subtle details~\cite{wei2022fine}.
Hence, a common retrieval workflow is to first retrieve image candidates using pre-computed embeddings, and then re-rank the retrieved candidates using a different, often more expensive but accurate, method \cite{vendrowinquire}.
Following this workflow, we experimented with cosine similarity over different embeddings, and with multiple re-ranking methods of reference candidates.
Although re-ranking sometimes yields better results compared to simply using cosine similarity between CLIP~\cite{radford2021learning} embeddings, the difference was not significant in most of our experiments. Therefore, for simplicity, we use cosine similarity between CLIP embeddings as our similarity metric (see \cref{tab:sim_metrics}, \cref{subsec:ablations} for more details about our experiments with different similarity metrics).

\vspace{-3pt}
\subsection{How to use the retrieved images?}
Putting it all together, after retrieving relevant images, all that is left to do is to use them as context so they are beneficial for the model.
We experimented with two types of models; models that are trained to receive images as input in addition to text and have ICL capabilities (e.g., OmniGen~\cite{xiao2024omnigen}), and T2I models augmented with an image encoder in post-training (e.g., SDXL~\cite{podellsdxl} with IP-adapter~\cite{ye2023ip}).
As the first model type has ICL capabilities, we can supply the retrieved images as examples that it can learn from, by adjusting the original prompt.
Although the second model type lacks true ICL capabilities, it offers image-based control functionalities, which we can leverage for applying RAG over it with our method.
Hence, for both model types, we augment the input prompt to contain a reference of the retrieved images as examples.
Formally, given a prompt $p$, $n$ concepts, and $k$ compatible images for each concept, we use the following template to create a new prompt:
``According to these examples of 
$\mathord{<}c_1\mathord{>:<}img_{1,1}\mathord{>}, ... , \mathord{<}img_{1,k}\mathord{>}, ... , \mathord{<}c_n\mathord{>:<}img_{n,1}\mathord{>}, ... , $
$\mathord{<}img_{n,k}\mathord{>}$,
generate $\mathord{<}p\mathord{>}$'', 
where $c_i$ for $i\in{[1,n]}$ is a compatible image caption of the image $\mathord{<}img_{i,j}\mathord{>},  j\in{[1,k]}$. 

This prompt allows models to learn missing concepts from the images, guiding them to generate the required result. 

\textbf{Personalized Generation}: 
For models that support multiple input images, we can apply our method for personalized generation as well, to generate rare concept combinations with personal concepts. In this case, we use one image for personal content, and 1+ other reference images for missing concepts. For example, given an image of a specific cat, we can generate diverse images of it, ranging from a mug featuring the cat to a lego of it or atypical situations like the cat writing code or teaching a classroom of dogs (\cref{fig:personalization}).
\vspace{-2pt}
\begin{figure}[htp]
  \centering
   \includegraphics[width=\linewidth]{Assets/personalization.pdf}
   \caption{\textbf{Personalized generation example.}
   \emph{ImageRAG} can work in parallel with personalization methods and enhance their capabilities. For example, although OmniGen can generate images of a subject based on an image, it struggles to generate some concepts. Using references retrieved by our method, it can generate the required result.
}
   \label{fig:personalization}\vspace{-10pt}
\end{figure}

\section{Experiments}
\subsection{Experimental Setup}
We conduct a comprehensive evaluation of \textsc{CCE} across three tasks: testing preference benchmarks, judge distillation, and SFT rejection sampling. 

\begin{table*}[!t]
\centering
\small 

\resizebox{0.92\textwidth}{!}{
\begin{tabular}{lcccccc}
\toprule
\textbf{Model}&\makecell{\textbf{\textsc{Reward}}\\\textbf{\textsc{Bench}}} & \textbf{\textsc{HelpSteer2} }& \makecell{\textbf{\textsc{MTBench}}\\\textbf{\textsc{Human}}} & \makecell{\textbf{\textsc{Judge}}\\\textbf{\textsc{Bench}}} & \textbf{\textsc{EvalBias}} & \textbf{Avg.}\\

\midrule
\textbf{GPT-4o} \\
~\textit{Vanilla}&85.2&66.1&82.1&66.3&68.5&73.6\\
~\textit{LongPrompt}&86.9&67.3&81.8&63.5&70.5&74.0 \\
~\textit{EvalPlan}&88.7&65.5&81.4&62.9&74.4&74.6 \\
~\textit{16-Criteria} &87.3&69.1&82.8&66.6&73.7&75.9\\
~\textit{Maj@16} &87.9&68.9&82.4&68.6&75.5&76.7\\
~\textit{Agg@16} &88.1&68.7&82.6&67.2&77.9&76.9\\
\rowcolor{green!10}
~\textit{\textsc{CCE}-random@16} &91.2&69.5&83.1&68.9&80.1&78.6\\
\rowcolor{green!10}
~\textit{\textsc{CCE}@16} &\textbf{91.8}&\textbf{70.6}&\textbf{83.6}&\textbf{70.4}&\textbf{85.0}&\textbf{80.3}\\
\midrule
\textbf{Qwen 2.5 7B-Instruct} \\
~\textit{Vanilla}&78.2&60.7&76.1&58.3&57.4&66.1\\
\rowcolor{green!10}
~\textit{\textsc{CCE}@16}&\textbf{80.4}&\textbf{64.2}&\textbf{76.7}&\textbf{64.0}&\textbf{79.4}&\textbf{72.9}\\
\midrule
\textbf{Qwen 2.5 32B-Instruct} \\
~\textit{Vanilla}&87.4&\textbf{72.3}&79.0&68.9&71.1&75.7\\
\rowcolor{green!10}
~\textit{\textsc{CCE}@16}&\textbf{90.8}&72.1&\textbf{82.1}&\textbf{70.6}&\textbf{80.5}&\textbf{79.2}\\
\midrule
\textbf{Qwen 2.5 72B-Instruct} \\
~\textit{Vanilla}&85.2&\textbf{69.5}&79.5&68.3&68.5&74.0\\
\rowcolor{green!10}
~\textit{\textsc{CCE}@16}&\textbf{93.7}&68.5&\textbf{88.9}&\textbf{75.7}&\textbf{85.9}&\textbf{82.7}\\
\midrule
\textbf{Llama 3.3 70B-Instruct} \\
%\cdashline{1-7}
~\textit{Vanilla}&86.4&70.4&81.1&67.1&70.6&75.1\\
\rowcolor{green!10}
~\textit{\textsc{CCE}@16}&\textbf{91.7}&\textbf{71.3}&\textbf{83.5}&\textbf{69.7}&\textbf{79.2}&\textbf{79.1}\\
\bottomrule
\end{tabular}
}
\caption{Accuracy of LLM-as-a-Judge on pair-wise comparison benchmarks. \textsc{CCE} can consistently enhance the LLM-as-a-Judge's performance across 5 benchmarks, especially considerably outperforming other scaling inference strategies, like maj@16. The highest values are \textbf{bolded}. Here, \textit{\textsc{CCE}-random} refers to replacing the ``Criticizing Selection$+$Outcome-Removal Processing'' with ``Random Selection''.
}
\label{tab:main_preference}
\end{table*}




\paragraph{Preference Benchmarks and Baselines.} We adopt 5 preference benchmarks to test LLM-as-a-Judge, including \textsc{RewardBench}~\citep{lambert2024rewardbench}, \textsc{HelpSteer2}~\citep{wang2024helpsteer}, \textsc{MTBench-Human}~\citep{zheng2023mtbench}, \textsc{JudgeBench}~\citep{tan2025judgebench}, and \textsc{EvalBias}~\citep{park2024offsetbias}. These benchmarks provide general instructions across a wide range of tasks with diverse responses and use accuracy to measure their evaluation performance. They each focus on different aspects. For example, \textsc{RewardBench} covers a wider range of scenarios, while \textsc{EvalBias} focuses on various bias scenarios. We verify the generality of \textsc{CCE} on 5 LLMs and compare it against multiple baselines. In particular, we consider \textbf{Vanilla}, which uses the general LLM-as-a-Judge prompt implemented by \textsc{RewardBench}; \textbf{Maj@16}, where we independently judge a case 16 times and take a majority vote of the outcomes; \textbf{Agg@16}, where instead of majority voting, the 16 individual judgments are fed back into the LLM to aggregate a final decision; \textbf{16-Criteria}, which incorporates 16 criteria with corresponding descriptions in the prompt as designed in~\citet{hu2024arellm} and~\citet{wang2024helpsteer}; \textbf{LongPrompt}, where the LLM is explicitly directed to produce a longer CoT; and \textbf{EvalPlan}, in which an unconstrained evaluation plan is first generated based on the target case and then executed to derive the final judgment~\citep{saha2025learningplanreason}. Additional details on the preference benchmarks and baselines can be found in Appendix~\ref{sec:testing}.





\paragraph{Distilling CoT for Training Judge.} We start with a large preference dataset and evaluate it using the Vanilla LLM-as-a-Judge and \textsc{CCE} under \textit{GPT-4o-as-a-Judge}, producing two CoTs. We then pair each CoT with the original preference data to form two separate training sets, which we use to fine-tune a smaller LLM as a judge. The resulting judges’ performance clearly reflects the quality and effectiveness of each CoT. We use \textbf{TULU3-preference} data as the distillation query while the preference benchmarks for evaluating the judge remain the same as previously introduced. Details of the training implementation are provided in Appendix~\ref{sec:distilling4training}.

\paragraph{SFT Rejection Sampling.} Firstly, we generate a pool of 4 responses based on a given task instruction to serve as the rejection sampling base. We compare Crowd Rejection Sampling against Random Selection and a Vanilla Rejection Sampling method to select the best response for fine-tuning.


We select two datasets of different scales, \textbf{LIMA}~\citep{zhou2023lima} ($1$K) and \textbf{TULU3-SFT}~\citep{lambert2025tulu3} (sample $10$K), as instruction query. \textit{GPT-4o} served as the judge LLM, while \textit{Llama-3.1-8B} and \textit{Qwen-2.5-7B} are used as base models for SFT. We then evaluate the generative ability of finetuned models using \textsc{MTBench} and \textsc{AlpacaEval-2}~\citep{dubois2024lengthcontrolled}. Details of the implementation are provided in Appendix~\ref{sec:sft_data_selection}.


\begin{table*}[!t]
\centering
\small 
\resizebox{0.96\textwidth}{!}{
\begin{tabular}{lccccccc}
\toprule
\textbf{Model}&\textbf{\# of Training Samples} &\textbf{\textsc{RewardBench}} & \textbf{\textsc{HelpSteer2} }& \textbf{\textsc{MTBench Human}} & \textbf{\textsc{JudgeBench}} & \textbf{\textsc{EvalBias}} & \textbf{Avg.}\\
\midrule
\textbf{JudgeLM-7B}~\citep{zhu2023judgelmfinetunedlargelanguage}&100,000&\underline{46.4}&\underline{60.1}&64.1&32.6&\textbf{42.4}&\underline{49.1}\\
\textbf{PandaLM-7B}~\citep{wang2024pandalm}&300,000&45.7&57.6&\underline{75.0}&36.0&27.0&48.3\\
\textbf{Auto-J-13B}~\citep{li2024generative}&4,396&\textbf{47.5}&\textbf{65.1}&\textbf{75.2}&\textbf{50.9}&16.5&\textbf{51.0}\\
\textbf{Prometheus-7B}~\citep{kim2024prometheus}&100,000&34.6&30.8&52.8&9.3&11.7&27.8\\
\textbf{Prometheus-2-7B}~\citep{kim2024prometheus2opensource} &300,000&43.7&37.6&55.0&\underline{39.4}&\underline{39.8}&43.1\\
\midrule
\textbf{Llama-3.1-8B-Tuned} &&&&&&&\\
~\textit{Synthetic Judgment from Vanilla}&10,000&66.8&56.0&71.6&\underline{60.1}&34.2&57.7\\
~\textit{Synthetic Judgment from Vanilla}&30,000&\textbf{72.5}&\underline{58.6}&\underline{73.9}&50.4&\underline{46.2}&60.3\\
~\textit{Synthetic Judgment from \textsc{CCE}}&10,000&69.7&\underline{58.6}&72.7&\textbf{66.4}&38.7&\textbf{61.2}\\
~\textit{Synthetic Judgment from \textsc{CCE}}&30,000&\underline{70.0}&\textbf{60.1}&\textbf{74.3}&50.3&\textbf{50.7}&\underline{61.1}\\
\midrule
\textbf{Qwen 2.5-7B-Tuned} &&&&&&&\\
~\textit{Synthetic Judgment from Vanilla}&10,000&68.1&55.6&70.7&\underline{50.2}&38.4&56.6\\
~\textit{Synthetic Judgment from Vanilla}&30,000&71.4&56.2&75.1&48.2&54.7&61.1\\
~\textit{Synthetic Judgment from \textsc{CCE}}&10,000&68.8&56.7&71.3&49.8&40.2&57.4\\
~\textit{Synthetic Judgment from \textsc{CCE}}&30,000&\underline{73.3}&\underline{59.5}&\underline{74.9}&50.1&\underline{57.1}&\underline{63.0}\\
~\textit{Mix Synthetic Judgment from \textsc{CCE}\&Vanilla}&60,000&\textbf{74.1}&\textbf{60.7}&\textbf{76.6}&\textbf{61.6}&\textbf{60.6}&\textbf{66.7}\\
\bottomrule
\end{tabular}
}
\caption{Accuracy of Trained small LLM-as-a-Judge on pair-wise comparison benchmarks. Under the same preference pairs data, the model trained with judgments synthesized using \textsc{CCE} achieves more reliable evaluation results. The highest values are \textbf{bolded}, and the second highest is \underline{underlined}.}
\label{tab:main_distill}
\end{table*}




\subsection{Experiment Result}
In this section, we present our main results. The preference benchmark results are shown in Table~\ref{tab:main_preference}, the efficacy of distilling CoT for training smaller judges is summarized in Table~\ref{tab:main_distill}, and the training efficiency of SFT rejection sampling is reported in Table~\ref{tab:main_sft}. These three objectives are concluded across various judge LLMs and downstream tasks. Our findings for each task are as follows.



\paragraph{Performance on Preference Benchmarks.} Table~\ref{tab:main_preference} highlights \textbf{\textsc{CCE} consistently achieves state-of-the-art performance across all preference benchmarks}. First, it outperforms the Vanilla LLM-as-a-Judge, which already demonstrates reasonable reliability on multiple LLMs and benchmarks. Notably, with \textit{Qwen 2.5-72B-Instruct} as the judge, our method achieves an $8.5$ increase on \textsc{RewardBench} and an overall average gain of $8.7$. 
%



Second, \textbf{\textsc{CCE} proves considerably more effective than common scaling strategies such as \textit{Maj@16} and 16-Criteria}. Even with random selection, \textit{Maj@16} underperforms \textsc{CCE} by an average of 1.9. Although \textit{EvalPlan} offers a more response-aware reasoning process than \textit{16-Criteria}, its effectiveness remains lower $2.0$-$3.7$ than \textsc{CCE}. Simply generating longer CoT also falls short, indicating that scaling inference-time computation calls for a more nuanced approach.



\begin{table}[!thbp]
  \centering
  \resizebox{0.45\textwidth}{!}{
  \begin{tabular}{lcc}
    \hline
    \textbf{Rejection Sampling Method} & \textbf{\textsc{MTBench}} & \textbf{\textsc{AlpacaEval-2}} \\
    \midrule
    \multicolumn{3}{c}{Llama 3.1 8B Base} \\
    \midrule
    \textbf{Instructions from LIMA \# 1K}&&\\
    ~\textit{Random Sampling} &\underline{4.33}&2.89/3.29 \\
    ~\textit{Vanilla Rejection Sampling} &4.28&\underline{2.91/3.29} \\
    ~\textit{Crowd Rejection Sampling} &\textbf{4.53}&\textbf{3.02/3.31} \\
    \textbf{Instructions from Tulu 3 \# 10K}&&\\
    ~\textit{Random Sampling} &7.51&12.81/12.45 \\
    ~\textit{Vanilla Rejection Sampling}&\underline{7.56}&\underline{19.92/17.17} \\
    ~\textit{Crowd Rejection Sampling} &\textbf{7.63}&\textbf{22.23/19.74} \\
    \midrule
    \multicolumn{3}{c}{Qwen 2.5 7B Base} \\
    \midrule
    \textbf{Instructions from LIMA \# 1K}&&\\
    ~\textit{Random Sampling} &\underline{8.06}&\underline{14.52/9.40}\\
    ~\textit{Vanilla Rejection Sampling} &7.91&14.40/9.44  \\
    ~\textit{Crowd Rejection Sampling} &\textbf{8.63}&\textbf{14.86/9.59}\\
    \textbf{Instructions from Tulu 3 \# 10K}&&\\
    ~\textit{Random Sampling} &8.36&21.39/13.68 \\
    ~\textit{Vanilla Rejection Sampling} &\textbf{8.46}&\underline{22.71/16.44} \\
    ~\textit{Crowd Rejection Sampling} &\underline{8.41}&\textbf{23.78/17.56}  \\
    
    \bottomrule
  \end{tabular}
  }
  \caption{SFT Rejection Sampling Performance on the Instruction-Following Benchmark.
  The model fine-tuned with responses sampled using \textsc{CCE} demonstrates improved generative performance.}
  \label{tab:main_sft}
\end{table}






\begin{table*}[!tp]
\centering
\small 

\resizebox{0.96\textwidth}{!}{
\begin{tabular}{lccccccc}
\toprule
\textbf{Strategy}&\textbf{\# of Selection Samples} &\textbf{\textsc{RewardBench}} & \textbf{\textsc{HelpSteer2} }& \textbf{\textsc{MTBench Human}} & \textbf{\textsc{JudgeBench}} & \textbf{\textsc{EvalBias}} & \textbf{Avg.}\\

\midrule
~\textit{Random-Selection} &8&91.0&\underline{69.9}&82.6&68.7&78.4&78.1\\
~\textit{Praising-Selection} &8&86.6&64.2&81.5&67.1&77.7&75.4\\
~\textit{Criticizing-Selection} &8&\underline{91.2}&69.2&\underline{83.0}&68.9&79.1&78.3\\
~\textit{Balanced-Selection} &8&90.7&68.6&82.8&67.4&78.7&77.6\\
~\textit{Outcome-Removal Random-Selection} &8&\textbf{91.5}&\underline{69.9}&\underline{83.0}&\underline{69.4}&\underline{79.5}&\underline{78.7}\\
~\textit{Outcome-Removal Criticizing-Selection (Sota)} &8&\textbf{91.5}&\textbf{70.1}&\textbf{83.2}&\textbf{69.5}&\textbf{79.9}&\textbf{78.8}\\
\midrule
~\textit{Random-Selection} &16&91.2&69.5&83.1&68.9&80.1&78.6\\
~\textit{Praising-Selection} &16&87.0&68.4&82.0&67.1&77.9&76.5\\
~\textit{Criticizing-Selection} &16&90.8&\underline{69.7}&83.0&69.6&\underline{82.9}&\underline{79.2}\\
~\textit{Balanced-Selection} &16&90.6&69.3&82.9&68.0&79.6&78.1\\
~\textit{Outcome-Removal Random-Selection} &16&\underline{91.7}&\underline{69.7}&\underline{83.2}&\underline{70.0}&81.5&\underline{79.2}\\
~\textit{Outcome-Removal Criticizing-Selection(Sota)} &16&\textbf{91.8}&\textbf{70.6}&\textbf{83.6}&\textbf{70.4}&\textbf{85.0}&\textbf{80.3}\\

\bottomrule
\end{tabular}
}
\caption{Accuracy of \textsc{CCE} using different selection strategies on LLM-as-a-Judge benchmarks. Our proposed \textit{Outcome-Removal Criticizing-Selection} consistently surpasses performances using other selection strategies during the test-time inference phase.}
\label{tab:ablation_selection}
\end{table*}


\begin{figure*}[h]
\centering
  \includegraphics[width=0.96\linewidth]{latex/figure/scaling_inference.pdf}
  \caption {Evaluation performance under scaling crowd judgments in the context. As the number of crowd judgments grows, both accuracy and CoT length generally increase.}
  \label{fig:scaling}
\end{figure*}



Finally, \textsc{CCE} not only excels on \textsc{RewardBench}, the most general benchmark, but also \textbf{outperforms alternatives on more challenging tasks} like \textsc{JudgeBench} and \textsc{EvalBias}. Strategic crowd judgment selection further enhances performance compared to random selection. We adopt a ``Criticizing Selection + Outcome Removal'' strategy for our SOTA selection \& processing strategy, which we discuss in detail in the following analysis.





\paragraph{Distilling CoT for Training Smaller Judges.} Distilling preference evaluation capabilities from powerful LLMs to train smaller LLMs is a promising direction. Table~\ref{tab:main_distill} demonstrates that higher-quality CoT leads to more effective distillation, resulting in improved performance for smaller judge models. Fine-tuning small models (\eg, \textit{Llama 3.1-8B} and \textit{Qwen 2.5-7B}) on the CoTs generated by \textsc{CCE} yields higher accuracy on all five benchmarks than using \textit{Vanilla} CoTs. For instance, \textit{Qwen 2.5-7B} trained on \textsc{CCE}'s synthetic CoT judgments achieves up to 73.3\% on \textsc{RewardBench}, surpassing Vanilla baseline by a notable margin of 1.9. Moreover, combining both \textit{Vanilla} and \textsc{CCE} synthetic judgments further boosts performance, reaching 74.1\% on \textsc{RewardBench} and 60.6\% on \textsc{EvalBias}. This result suggests integrating diverse CoT can further enhance accuracy and generalization.

LLM-as-a-Judge can develop biases in various scenarios, such as favoring more verbose answers. This issue is particularly pronounced in smaller judge models. As shown in Table~\ref{tab:main_distill}, even after fine-tuning on over 100K samples, many baseline models struggle to exceed 50\% accuracy. This highlights the persistent challenge of evaluation bias. \textbf{Higher-quality and more comprehensive CoT distillation enhances the debiasing ability of smaller judge models}. These findings suggest that many biases stem from the model focusing on limited aspects of the responses rather than assessing them holistically.




\paragraph{Efficacy in SFT Rejection Sampling.} As we can see in Table~\ref{tab:main_sft}, Crowd Rejection Sampling proves effectiveness for both $1$K and $10$K data sizes, consistently \textbf{yielding better finetuning performances for two base LLMs}. \textsc{CCE} selects higher-quality responses compared to both Random Sampling and Vanilla Rejection Sampling, leading to consistent improvements in downstream instruction-following benchmarks on \textsc{MTBench} and \textsc{AlpacaEval-2}. For instance, with \textit{Llama 3.1-8B} and the TULU3-SFT instructions, the fine-tuned model sees performance gains of up to $22.23$/$19.74$ on \textsc{AlpacaEval-2}, compared to $19.92$/$17.17$ under the Vanilla Rejection Sampling. This underscores the reliability of \textsc{CCE} in identifying higher-quality training examples.

Overall, the experiments confirm the flexibility and effectiveness of \textsc{CCE} in three key general scenarios. By \textbf{leveraging crowd-based context, scaling inference-time computation, and strategically guiding the CoT process}, \textsc{CCE} delivers consistent improvements over strong baselines.


\subsection{Analysis Experiments}
In this section, we conduct an in-depth analysis of the two core components of our method: crowd judgment selection \& processing strategies, as well as inference scaling. We then directly examine whether the generated CoT is more comprehensive and provides a more detailed analysis of the responses under evaluation.


\paragraph{Selection \& Processing Strategy.}
We compare Random Selection, Criticizing Selection, Praising Selection, and Balanced Selection.
As shown in Table~\ref{tab:ablation_selection}, Criticizing Selection yields the best results, followed by Balanced Selection, while Praising Selection performs even worse than Random Selection. This suggests that \textbf{lose-based judgments provide deeper insights into A/B comparisons, making criticism more informative}. Additionally, the \textbf{Outcome-Removal post-processing strategy substantially improves evaluation reliability}, likely because final verdicts lack valuable details while introducing biases into LLM decision-making.




\paragraph{Inference Scaling.} 
Figure~\ref{fig:scaling} illustrates our analysis of how scaling crowd judgments influence evaluation outcomes. Measuring accuracy and the average token length of the CoT, three preference benchmarks are tested across different judgment counts and then averaged for an overall assessment. The implementation details are in Appendix~\ref{sec:infer_scal_appendix}.

As shown in Figure~\ref{fig:scaling}, \textbf{both performance and output length generally increase as crowd judgments rise from 0 to 16}. \textsc{RewardBench} displays a clear upward trend, while \textsc{HelpSteer2} dips briefly at 2 judgments before recovering. Averaging across benchmarks (rightmost panel) confirms that more crowd judgments lead to higher accuracy and longer CoT, consistent with the inference scaling observed in studies~\citep{brown2024largelanguagemonkeysscaling,snell2025scaling}.
Furthermore, we reexamine the Table~\ref{tab:main_preference} and find that \textbf{scaling test-time inference is a promising strategy for LLM-as-a-Judge}, as demonstrated by \textit{GPT-4o-as-a-Judge}. This is especially evident in bias scenarios, where the Vanilla struggles, while scaling-inference-based baselines, including \textsc{CCE}, show substantial gains.

\begin{figure}[t]
\centering
  \includegraphics[width=0.96\linewidth]{latex/figure/analysis.pdf}
  \caption {CoT Comparison. \textsc{CCE}’s CoT consistently yields a higher average number of key points and a higher coverage rate across all benchmarks.}
  \vspace{-.5em}
  \label{fig:cot_comp}
\end{figure}


\paragraph{CoT Comparison.} To more directly assess whether the CoTs generated by \textsc{CCE} are more comprehensive than those of the Vanilla approach, we perform two analyses: Key Points Counting and Coverage Rate. First, we use \textit{GPT-4o} to parse and summarize each CoT, counting the key points to measure how thoroughly the CoT is. Second, we leverage the cross-attention mechanism from \textit{Bart-base} to quantify the coverage rate--how thoroughly CoT covers details in the candidate responses. We introduce the details in the Appendix~\ref{subsec:coverage_appendix}.

As shown in Figure~\ref{fig:cot_comp}, \textsc{CCE} outperforms Vanilla across all benchmarks in key point counting and coverage rate. More key points indicate that our CoT examines the text from multiple angles, while a higher coverage rate reflects a more detailed analysis. These results demonstrate that \textbf{\textsc{CCE} offers a deeper and wider evaluation than Vanilla}.



\begin{table}[!ht]
  \centering
  \scriptsize
  \resizebox{0.46\textwidth}{!}{
    \begin{tabular}{p{26.5em}}
    \toprule
    \textbf{Instruction}: Classify this tweet from the categories of positive, negative, or neutral. Tweet: I'm so thankful for all the wonderful people in my life \\
    \midrule
    \textbf{\red{Response A}} \\
        \ \ \ \ The sentiment of this social media comment is positive. The use of the word "grateful" implies appreciation and thankfulness, while "amazing experiences" suggest that these experiences have been very fulfilling or enjoyable. \\
    \textbf{\limegreen{Response B}} \\
            \ \ \ \ Positive\\
    \midrule
    \textbf{Vanilla LLM-as-a-Judge}:\\

	\ \ \ \ Assistant A's . . . . . . \redback{It highlights specific words and phrases in the tweet, such as ``thankful'' and ``wonderful people,'' to justify the positive sentiment classification. This explanation adds depth and clarity to the response, making it more helpful and informative.}

    Assistant B, . . . . . .  the response less informative and helpful . . . 

    \ \ \ \ . . . . . .

    \ \ \ \ - Final Verdict: \red{[[A]]}
    \\
    \midrule
    \textbf{\textsc{CCE}}:\\

    \ \ \ \ \textbf{\textit{AC Judgment}}: \uwave{However, Assistant A makes a mistake by referencing words not present in the tweet, such as "grateful" and "amazing experiences,"} . . . . . . Assistant C also classifies the tweet as positive and provides a detailed explanation . . .

    \ \ \ \

    \ \ \ \ . . . . . . Assistant A . . . . .\greenback{, but it inaccurately references words not present in the tweet, such as "grateful" and "amazing experiences." This detracts from the accuracy of the response and could potentially confuse the user.} . . . . . .
    
    \ \ \ \ Assistant B \greenback{is concise and correctly classifies the tweet as positive. However, it lacks any explanation or reasoning, which limits its helpfulness and depth.} . . . . . .

    \ \ \ \ In comparing the two, \greenback{Given the importance of accuracy and explanation in sentiment analysis,} . . . . . .

    \ \ \ \ - Final Verdict: \green{[[B]]}
    \\
    \bottomrule
    \end{tabular}%
    }
  \caption{A pairwise comparison case evaluated by different methods. \limegreen{Preference} refers to right result and \red{Preference} refers to wrong result. We emphasize the noisy evaluation elements in \redback{orange}, while highlighting the useful elements of the evaluation in \greenback{limongreen}.}
  \label{tab:case-evaluation-simple}%
\vspace{-.5em}
\end{table}%




\paragraph{Case Study.} Table~\ref{tab:case-evaluation-simple} presents a representative case. The vanilla is misled by fake information in Response A, causing it to overlook the Instruction and mistakenly rate Response A as more helpful. In contrast, the crowd judgment correctly identifies the error in Response A and informs subsequent evaluations. Additionally, our method produces a more detailed CoT thereby enriching the overall evaluation process, as evidenced by statements like ``Assistant A does provide a brief explanation''.









% In this work, we propose WildLong, a novel framework for synthesizing diverse, scalable, and realistic instruction-response datasets designed for long-context tasks. Our approach addresses key challenges in dataset creation by leveraging meta-information extraction from real-world user queries, graph-based modeling of co-occurrence relationships, and adaptive instruction-response generation.
% WildLong is built on the principles of diversity, scalability, and realism, enabling it to support complex reasoning tasks such as cross-document comparison, and aggregation, which are essential for real-world applications. By integrating meta-information into the data generation process and systematically exploring new combinations through graph-based modeling, WildLong generates diverse datasets that reflect the complexity of extended contexts.
% Experimental results demonstrate that WildLong significantly improves long-context task performance, surpassing other open-source long-context-optimized models across multiple benchmarks. Importantly, this improvement is achieved without requiring supplementary short-context instruction tuning, highlighting the robustness and generalizability of our approach.
% The success of WildLong highlights the potential of structured, meta-information-driven data synthesis to enhance the capabilities of LLMs for complex, real-world tasks. By addressing the critical gaps in long-context dataset diversity and quality, WildLong sets a new standard for long-context instruction tuning and paves the way for further advancements in equipping LLMs to tackle the challenges of extended-context reasoning.
% We propose WildLong, a framework for synthesizing diverse, scalable, and realistic instruction-response datasets for long-context tasks. By leveraging meta-information extraction, graph-based modeling, and adaptive instruction generation, WildLong generates long-context instruction-tuning data with real-world complexity.
% Experiments show improved long-context task performance while retaining short-context performance without additional short-context fine-tuning, demonstrating its robustness and generalizability. We hope WildLong provides insights into generalizing instruction tuning and inspires further advancements in long-context reasoning for LLMs.
We propose WildLong, a framework for synthesizing diverse, scalable, and realistic instruction-response datasets for long-context tasks. 
It integrates meta-information extraction to ensure realistic complexity, graph-based modeling for systematic instruction expansion, and adaptive instruction generation for enhanced contextual relevance.
Our fine-tuned models consistently outperform baselines and maintain short-context performance without mixing short-context data. Notably, our finetuned Llama-3.1-8B model surpasses most open-source long-context models on Longbench-Chat and demonstrates competitive performances with even larger models across benchmarks.
WildLong enables the synthesis of instruction-tuning data that produces robust models capable of handling diverse long-context tasks. Extending beyond synthetic QA and summarization, it bridges the gap to more complex, realistic challenges, advancing the effectiveness of long-context LLMs.
We hope WildLong provides insights into generalizing synthetic data and inspires further progress in long-context reasoning for LLMs.

\section*{Limitations}
Despite \textit{EPO} shows promise in advancing strategic reasoning in LLMs, this work has several limitations that provide avenues for future work.
First, the social and physical environments tested in this paper involve maximumly two agents, and \textit{EPO}'s performance on more complex multi-agent settings such as Diplomacy and Hanabi is also interesting.
Second, due to the computational constraints, we focus on 8B/7B models and do not scale up the multi-turn RL training to a large scale.
It would be an important direction for future work to train our strategic reasoning model with larger base models on more domains.
Third, we employ an off-the-shelf LLM as the process reward model for RL training, while a more reliable process-supervised reward model can be learned in the future work.
Finally, we rely on the final goal completion score to evaluate the reasoning model's performance.
Future research could design evaluation metrics tailored to assess the quality and diversity of strategies devised by this model.


% Bibliography entries for the entire Anthology, followed by custom entries
%\bibliography{anthology,custom}
% Custom bibliography entries only
\bibliography{custom}


%%%%%%%%%%%%%%%%%%%%%%%%%%%%%%%%%%%%%%%%%%%%%%%%%%%%%%%%%%%%%%%%%%%%%%%%%%%%%%%
%%%%%%%%%%%%%%%%%%%%%%%%%%%%%%%%%%%%%%%%%%%%%%%%%%%%%%%%%%%%%%%%%%%%%%%%%%%%%%%
% APPENDIX
%%%%%%%%%%%%%%%%%%%%%%%%%%%%%%%%%%%%%%%%%%%%%%%%%%%%%%%%%%%%%%%%%%%%%%%%%%%%%%%
%%%%%%%%%%%%%%%%%%%%%%%%%%%%%%%%%%%%%%%%%%%%%%%%%%%%%%%%%%%%%%%%%%%%%%%%%%%%%%%
\newpage
\appendix


\section{Strategic Reasoning vs. Static Problem-Solving}
\label{append:challenge}

The differences between strategic reasoning and static problem-solving are summarized in Table~\ref{tab:challenge}.


%%%%%%%%%%%%%%%%%%%%%%%%%%%%%
\begin{table*}[ht]
    \centering
    \begin{tabular}{lll}
    \toprule
   \textbf{Aspect} & \textbf{Static Problem-Solving}  &  \textbf{Strategic Reasoning}\\
    \midrule
    \textbf{Environment} & Fixed rules, known variables  & Dynamic, evolving conditions\\
    \midrule
    \textbf{Solutions} & Single correct answer & Multiple viable paths with trade-offs\\
    \midrule
    \textbf{Information} & Complete and observable & Partial, ambiguous, or delayed\\
    \midrule
    \textbf{Interactions} & None (isolated problem-solving) & Multi-agent or environmental dynamics\\
    \midrule
    \textbf{Feedback} & Immediate and deterministic & Delayed, probabilistic, or indirect\\
    \midrule
    \textbf{Goals} & Short-term, well-defined & Long-term, abstract\\
    \midrule
    \textbf{Risk} & Predictable & High-stakes, irreversible consequences\\
    \bottomrule
    \end{tabular}
    \caption{Challenges of Strategic Reasoning vs. Static Problem-Solving.}
    \label{tab:challenge}
\end{table*}

%%%%%%%%%%%%%%%%%%%%%%%%%%%%%%%






\section{Environments and Datasets}
\label{append:dataset}

%%%%%%%%%%%%%%%%%%%%%%%%%%%%%%%%%%%%%%%%%%%%%%
\begin{table*}[ht]
\begin{center}
\begin{tabular}{lccc}
\toprule
\textbf{Dataset} & Train & Test & Max Turns \\
\midrule
SOTOPIA & 2050  &450/50 & 20 \\
WebShop &1938 & 200 & 10\\
ALFWorld &3321  &140/134 & 40\\
\bottomrule
\end{tabular}
\caption{\textbf{Dataset Statistics.} ``Train'' and ``Test'' denote the number of scenarios for training and evaluation, respectively. Test scenarios in SOTOPIA (left) and SOTOPIA-hard (right) as well as test sets with seen (left) and unseen (right) scenarios in ALFWorld are separated. ``Max Turns'' is the maximum turns in an interaction.}
\label{tab:data}
\end{center}
\end{table*}
%%%%%%%%%%%%%%%%%%%%%%%%%%%%%%%%%%%%%%%%%%%%%%


\paragraph{\textbf{SOTOPIA}.}
SOTOPIA~\cite{zhou2024sotopia} is an open-ended, general-domain platform designed to simulate goal-oriented social interactions between artificial agents. A social task in this environment involves a scenario, two role profiles, and private social goals to be achieved through interaction. Scenarios in SOTOPIA cover a wide variety of social interation types, including negotiation, exchange, collaboration, competition, accommodation and persuation. Each agent is characterized by detailed profiles, including aspects like name, gender, personality, and occupation. At the end of each episode, agents are assessed based on seven dimensions: Goal Completion, Believability, Knowledge, Secret, Relationship, Social Rules, and Financial and Material Benefits. These scores are then averaged to produce an overall score for the agents. 
SOTOPIA-$\pi$~\cite{Wang2024SOTOPIAIL} is a follow-up work that leverages GPT-4 to automatically construct an entirely new set of scenarios. The social tasks (a combination of scenarios, characters’ profiles, and social goals) in SOTOPIA-$\pi$ are guaranteed to not overlap with the ones in SOTOPIA.

For training data collection, we employ GPT-4-Turbo as the agent for self-chat in scenarios of SOTOPIA-$\pi$ and prompt it to generate reasoning and strategy before response at each dialogue turn. We show the prompt in Table~\ref{tab:data_collect_prompt}. We only use the strategy and response data for training our reasoning model.
For iterative self-play RL training, before each iteration, we employ our RL-trained reasoning model to collect strategy data and GPT-4-Turbo to collect dialogue history data. The RL-trained reasoning model is plugged into GPT-4-Turbo for self-chat.



\paragraph{\textbf{WebShop}.}
WebShop~\cite{yao2022webshop} is a large-scale interactive online shopping environment on an e-commerce website. Agents in this environment aim to purchase a product to match the specifications provided by human user instructions. Once the agent selects the ``buy'' action, the environment provides a final reward, which is calculated using programmatic matching functions that consider the attributes, type, options, and price of the chosen product.

We use the training data collected by~\cite{song-etal-2024-trial} where GPT-4 is employed as the agent to explore in the WebShop environment and trajectories with a reward greater than 0.7 are selected. GPT-4 is used to generate corresponing rationales for each action step within a trajectory. We consider the rationale as a strategy for training our reasoning model.

\paragraph{\textbf{ALFWorld}~\cite{shridhar2021alfworld}.}
ALFWorld~\cite{shridhar2021alfworld} features interactive TextWorld environments that correspond to the embodied worlds found in the ALFRED~\cite{9156851} dataset. In ALFWorld, agents are tasked with exploring these text-based environments and completing high-level household instructions, assessing their abstract reasoning abilities and concrete execution skills.

Training data in the ALFWorld environment consists of two parts: (1) a few successful trajectories collected by~\cite{song-etal-2024-trial} where each trajectory contains CoT information generated by GPT-4 for each action step; (2) failed trajectories generated by GPT-4 that contain both rationales and action information via CoT prompting.


\section{Additional Results}
\label{append:add_res}

To validate the effectivess of explicit policy optimization for strategic reasoning, we conduct comparative experiments which involve fine-tuning a single LLM (Llama3-8B-Instruct) via SFT on training data collected from three environments, respectively. This single model is trained to output strategy and behavior simuteneously for each interaction turn. The results are shown in Table~\ref{tab:single_res}.

From the results, it can be observed that training a single LLM via SFT underperforms our reasoning model trained with SFT plugged into LLM agents (GPT-4o) for navigating in SOTOPIA and WebShop, despite that the LLM agents remain frozen without additional training. However, our method underperforms the single model fined-tuned on ALFWorld, since the frozen LLM agent (GPT-4o) has poor performance in this environment. Under the guidance of our reasoning model in \textit{EPO}, this agent improves performance in achieving long-term goals but still struggles with challenging scenarios in ALFWorld.

\begin{table*}[ht]
    \centering
    \begin{tabular}{llccccc}
    \toprule
     \multirow{2}{*}{Backbone} & \multirow{2}{*}{Method}  &  \multicolumn{2}{c}{\textbf{SOTOPIA}} & \multirow{2}{*}{\textbf{WebShop}} & \multicolumn{2}{c}{\textbf{ALFWorld}}\\
     & & Hard & All &  & Seen & Unseen\\
     \midrule
     Llama3-8B & SFT & 6.65 & 8.20 & 62.6 & \textbf{61.4}& \textbf{60.4} \\
     Llama3-8B & EPO-SFT & \textbf{6.76} & \textbf{8.28} & \textbf{67.1} & 45.9 & 44.1 \\
     \bottomrule
    \end{tabular}
    \caption{\textbf{Comparative performance between training a single LLM with SFT and explicitly optimizing a strategic reasoning model via SFT.}
    The evaluation metric for SOTOPIA is the final goal completion score, and average reward for WebShop and ALFWorld. Our strategic reasoning model trained with SFT is plugged into GPT-4o.}
    \label{tab:single_res}
\end{table*}


\section{Additional Implementation Details}
\label{append:imple} 
For SFT or RL training of our reasoning model, we use a fixed budget of gradient updates without altering hyperparameters. Final model checkpoints are selected for each run, although a small held-out validation set can be used.
Due to computational costs, we only report results in a single run. All experiments are conducted on 6 NVIDIA A100 80G GPUs.
GPT-4o refers to GPT-4o-2024-0806 and Claude-3.5-Sonnet refers to Claude-3-5-Sonnet-20241022.


\paragraph{\textbf{Baseline Implementations:}} (1) ReAct: During ReAct prompting, the two parties in a conversation from SOTOPIA cannot see each other's reasoning and strategies. During evaluation, reasoning and strategies are excluded from dialogue histories for GPT-4o to assess the agents from seven dimensions; (2) PPDPP: We adopt RoBERTa~\cite{Liu2019RoBERTaAR} as the base model for dialogue policy planner to predict five action types in SOTOPIA: ``none'', ``speak'', ``non-verbal communication'', ``action'', and ``leave''. We train the policy planner via RL with supervised initialization using same hyperparameters as in~\cite{deng2023plug}.
(3) DAT: We adopt a small multi-layer perceptron (MLP) as the planner model to predict a continuous action vector. We first train the planner and an up-mapping matrix with supervised fine-tuning and then optimize the planner using the RL method TD3-BC~\cite{fujimoto2021minimalist} with same hyperparameters as in~\cite{li2024dialogue}. Note that in the original paper, DAT is trained on scenarios from SOTOPIA and only 50 evaluations are conducted, while in this work, we train DAT on scenarios from SOTOPIA-$\pi$ and evaluate it on all the scenarios in SOTOPIA.

\onecolumn

%%%%%%%%%%%%%%%%%%%%%%%%%%%%%%%%%%%%%%%%%%%%%%

\begin{center}
\begin{longtable}{p{0.95\columnwidth}}
\toprule
    \texttt{Imagine you are <Agent>, your task is to act/speak as <Agent> would, keeping in mind <Agent>'s goal.} \\
    \texttt{You can find <Agent>'s goal (or background) in the ``Here is the context of the interaction'' field.} \\
    \texttt{Note that <Agent>'s goal is only visible to you.}\\
    \texttt{You should try your best to achieve <Agent>'s goal in a way that align with their character traits.}\\
    \texttt{While trying to achieve <Agent>'s goal, you should also follow the following principles as much as possible:}\\
    \texttt{1. Maintain the conversation's naturalness and realism is essential (e.g., do not repeat what other people has already said before).}\\
    \texttt{2. Preserve or enhance <Agent>'s personal relations with the other agent(s) during the interaction. The relations may encompass family ties, friendships, romantic associations and etc.}\\
    \texttt{3. Attempt to gain more new and important information during the interaction.}\\
    \texttt{4. Try to keep <Agent>'s secrets, private information, or secretive intentions.}\\
    \texttt{5. Do not violate any moral rules or laws in the interactions.}\\
    \texttt{6. Attempt to contribute towards financial and material benefits during the interaction. The financial and material benefits include short-term ones, such as monetary rewards and food, and long-term ones, such as employment opportunities and stock.}\\
    \texttt{You are at Turn <turn number>.}\\
    \texttt{The dialogue history until now is: <history>.}\\
    \texttt{You should first provide a reasoning for your action and argument to align with <Agent>'s social goal based on the dialogue history.}\\
    \texttt{The reasoning process for the action should be logical, considering the context of the conversation, <Agent>'s goal, and <Agent>'s character traits.}\\
    \texttt{You can reason step by step, starting from the current dialogue turn, and then consider the long-term effects of the dialogue turn.}\\
    \texttt{Remember that the reasoning should mainly focus on how <Agent>'s argument can help to achieve <Agent>'s goal in the long term.}\\
    \texttt{Note that the reasoning should not be redundant or too long and it is only visible to you.}\\
    \texttt{Based on the reasoning process and dialogue history, you should then generate a corresponding dialogue policy for current dialogue turn to steer the conversation towards <Agent>'s goal.}\\
    \texttt{You can use different types of dialogue, communication or social strategies.}\\
    \texttt{For example, given a scenario where a persuader attempts to persuade a persuadee to donate to a charity, you can generate dialogue policies for the persuader such as ``elicit empathy by telling personal stories'' and ``provide social proof to show the benefits of donating'', etc.}\\
    \texttt{The types of dialogue policies are not restricted to examples above.}\\
    \texttt{You can even generate new policies as long as the policies can help you to achieve <Agent>'s goal smoothly and quickly.}\\
    \texttt{But remember to keep the dialogue policy concise and strictly limit it to be a single phrase or sentence within 10 words.}\\
    \texttt{Note that the dialogue policy is only visible to you.}\\
    \texttt{Then based on the reasoning, dialogue policy and dialogue history, you should select the action type.}
    \texttt{Your available action types are <action list>.}\\
    \texttt{Note: You can ``leave'' this conversation if 1. you have achieved your social goals, 2. this conversation makes you uncomfortable, 3. you find it uninteresting/you lose your patience, 4. or for other reasons you want to leave.}\\
    \texttt{Finally, you should generate the argument following the action type.}\\
    \texttt{The argument should be generated based on the dialogue history and aligned with the dialogue policy you have generated.}\\
    \texttt{Remember that the argument should not be too short, and one or two sentences are recommended.}\\
    \texttt{Please only generate a JSON string including the reasoning, the dialogue policy, the action type and the argument.}\\
    \texttt{Your response should follow the given format:} \\
    \texttt{<format instructions>}\\
\bottomrule
\caption{\textbf{Prompts for training data collection from SOTOPIA-$\pi$.} ``<Agent>'', ``<turn number>'', ``<history>'', ``<action list>'' and ``<format instructions>'' can be replaced by the participant's name, the index of interaction turn, the full dialogue history with the participant's strategies, action types and output format instructions in SOTOPIA-$\pi$.}
\label{tab:data_collect_prompt}
\end{longtable}
\end{center}

%%%%%%%%%%%%%%%%%%%%%%%%%%%%%%%%%%%%%%%%%%%%%%%







\section{Prompts}
\label{append:prompt}

\subsection{\textbf{Evaluation Prompts}}
\label{append:task-prompt}

\begin{center}
\begin{longtable}{lp{0.85\columnwidth}}
    \toprule
    \textbf{SOTOPIA} & \texttt{You are a social expert with exceptional communication skills known for helping individuals achieve their interpersonal goals through nuanced strategies.}
	 \\ & \texttt{Your current objective is to assist <Agent1> in reaching their goal in an interaction with <Agent2>.}
	 \\ & \texttt{You will be given the context of their interaction and can find <Agent1>'s goal in the 'Here is the context of this interaction' field, keeping in mind <Agent1>'s goal.}
	 \\ & \texttt{You will also have access to the conversation between <Agent1> and <Agent2>.}
	 \\ & \texttt{Before proposing any strategies, reason step by step to reflect on the current state of the dialogue and consider what strategies might be most effective for helping <Agent1> achieve their goal.}
	 \\ & \texttt{Additionally, maintaining the diversity of strategies is essential (e.g., do not repeat strategies that have already proposed before).}
	 \\ & \texttt{And the strategy should not be aggressive, offensive, or violate any moral rules or laws.}
	 \\ & \texttt{You must generate a strategy at each dialogue turn except that any participant has left the conversation.}
	 \\ & \texttt{Finally, provide a well-thought-out communication and social strategy based on your reflection and the conversation history.}
	 \\ & \texttt{Your output should STRICTLY follow the format: Strategy: content (e.g, Strategy: Elicit emphathy by telling personal stories).}
	 \\ & \texttt{Your output should ONLY contain the strategy. DO NOT include any reasoning or argument. DO NOT generate any argument on behalf of any participant as the strategy.}
	 \\ & \texttt{Your output should be in a natural language form.}
	 \\ & \texttt{Keep the strategy concise and limit it to be a single phrase or sentence within 10 words.}\\
\midrule
    \textbf{WebShop} & \texttt{You are a skilled buyer in an online shopping environment. Your task is to assist Agent in navigating the platform to make purchases based on instructions. Your primary role is to provide strategic and insightful guidance to Agent, ensuring it successfully identifies and purchases products based on the instruction.}
	 \\ & \texttt{- At the beginning of the interaction, you will receive an instruction that includes the desired product's attributes and price, which serves as the shopping goal for Agent.}
	 \\ & \texttt{- You can find the instruction in the 'Instruction' field, keeping in mind the instruction.}
	 \\ & \texttt{- For each turn, you will be given an action performed by Agent and the resulting observation from the environment.}
	 \\ & \texttt{- In each turn, your task is to analyze the given scenario and provide thoughts that can guide Agent in its next action, ensuring it meets the shopping goal.}
	 \\ & \texttt{Your thoughts should be based on:}
	 \\ & \texttt{1. Understanding and following the instructions for shopping.}
	 \\ & \texttt{2. Evaluating the current state of the environment.}
	 \\ & \texttt{3. Assessing the effectiveness of Agent's last action.}
	 \\ & \texttt{4. Anticipating future actions that will lead Agent closer to achieving the shopping goal.}
	 \\ & \texttt{The available actions for Agent are:}
	 \\ & \texttt{1. search[keywords]}
	 \\ & \texttt{2. click[value]}
	 \\ & \texttt{where [keywords] in search are up to Agent, and the [value] in click is a value in the list of available actions given by the environment.}
	 \\ & \texttt{Note that you must generate a thought at each turn except that the task has been finished.}
	 \\ & \texttt{Keep your thoughts focused and concise, leveraging your understanding of online shopping dynamics to maximize the efficiency and correctness of Agent's actions. Use your reasoning skills to project possible scenarios and potential obstacles Agent might face, offering solutions or alternatives when necessary.}
	 \\ & \texttt{**Output Format:**}
	 \\ & \texttt{Keep your response to one or two sentences each turn.}
	 \\ & \texttt{Your response must strictly follow this format:}
	 \\ & \texttt{Thought: <your thoughts>} \\
    
\midrule
    \textbf{ALFWorld}
    & \texttt{You are an intelligent guide in an interactive household environment. Your task is to assist Agent in accomplishing household tasks within the environment. Your primary role is to provide strategic and insightful guidance to Agent, ensuring that Agent can achieve each task goal efficiently.}
	 \\ & \texttt{- At the beginning of your interactions, you will be given the detailed description of the current environment and the task goal to accomplish.}
	 \\ & \texttt{- You can find the task goal in the 'Your task is to' field, keeping in mind the task goal.}
	 \\ & \texttt{- For each of your turn, you will receive Agent's actions and the corresponding environment observations. If the environment observation is ``Nothing happens'', that means the previous action by Agent is invalid.}
	 \\ & \texttt{- In each turn, your task is to analyze the current situation and provide clear, logical thoughts that will help direct Agent's subsequent actions toward achieving the task goal.}
	 \\ & \texttt{Your thoughts should be based on:}
	 \\ & \texttt{1. Understanding the goal of household task.}
	 \\ & \texttt{2. Breaking down a high-level house-holding instruction into manageable sub-goals.}
	 \\ & \texttt{3. Evaluating the current state of the environment.}
	 \\ & \texttt{4. Assessing the effectiveness of Agent's last action.}
	 \\ & \texttt{5. Anticipating future actions that will lead Agent closer to achieving the task goal.}
	 \\ & \texttt{The available actions for Agent are:}
	 \\ & \texttt{1. go to \{recep\}}
	 \\ & \texttt{2. take \{obj\} from \{recep\}}
	 \\ & \texttt{3. put \{obj\} in/on \{recep\}}
	 \\ & \texttt{4. open \{recep\}}
	 \\ & \texttt{5. close \{recep\}}
	 \\ & \texttt{6. toggle \{obj\} \{recep\}}
	 \\ & \texttt{7. clean \{obj\} with \{recep\}}
	 \\ & \texttt{8. heat \{obj\} with \{recep\}}
	 \\ & \texttt{9. cool \{obj\} with \{recep\}}
	 \\ & \texttt{where \{obj\} and \{recep\} correspond to objects and receptacles.}
	 \\ & \texttt{Note that you must generate a thought at each turn except that the task has been finished.}
	 \\ & \texttt{Keep your thoughts focused and concise, leveraging your understanding of household dynamics to maximize the efficiency and correctness of Agent's actions. Use your reasoning skills to project possible scenarios and potential obstacles Agent might face, offering solutions or alternatives when necessary.}
	 \\ & \texttt{**Output Format:**}
	 \\ & \texttt{Keep your response to one or two sentences each turn.}
	 \\ & \texttt{Your response must strictly follow this format:}
	 \\ & \texttt{Thought: <your thoughts>}
     \\
         \bottomrule
    \caption{\textbf{Evaluation prompts for strategic reasoning model}. "<Agent1>" and "<Agent2>" can be replaced by the participant's name in SOTOPIA.}

\end{longtable}
\end{center}





\subsection{\textbf{PRM Prompts}}
\label{append:strategy-prompt}

\begin{center}
\begin{longtable}{lp{0.85\columnwidth}}
    \toprule
     \textbf{SOTOPIA} & \texttt{Here's a conversation in JSON format between <Agent1> and <Agent2>:}
	 \\ & \texttt{In the first response from ``human'', you can find the context of the conversation and <Agent1>'s goal in the ``Here is the context of this interaction'' field.}
	 \\ & \texttt{In the other responses from ``human'', you can find the conversation history between <Agent1> and <Agent2>.}
	 \\ & \texttt{In the responses from ``gpt'', you can find communication and social strategies that <Agent1> used for achieving <Agent1>'s goal.}
	 \\ & \texttt{In the ``score'' field, you can find a score for evaluating <Agent1>'s goal achievement. The score ranges from 0 and 10. 0 represents minimal goals achievement, 10 represents complete goal achievement, and a higher score indicates that <Agent1> is making progress towards the goal.}
     \\ &
	 \\ & \texttt{<history>}
     \\ &
	 \\ & \texttt{Your task is to select top strategies <Agent1> used that were critically important for achieving <Agent1>'s goal.}
          \\ &
	 \\ & \texttt{Please output the selected round indexes and the reasoning process that led you to the selection in JSON format like this: {``indexes'': , ``reasoning'': `` ''}.}
	 \\ & \texttt{Here is the output schema: {``properties": {``indexes": {``description": ``the selected top strategies that are critically important for achieving <Agent1>'s goal", ``title": ``indexes", ``type": ``list(integer)"}, ``reasoning": {``description": ``the reasoning process why you select these strategies", ``title": ``reasoning", ``type": ``string"}}, ``required": [``indexes", ``reasoning"]}.}
    \\
\midrule
    \textbf{WebShop} & \texttt{Here's a conversation in JSON format between human and gpt.}
     \\ & \texttt{In the first response from ``human'', you can find the instructions for gpt to help Agent interact in an online shopping environment.}
     \\ & \texttt{In the second response from ``human'', you can find the shopping goal for gpt and Agent to achieve.}
     \\ & \texttt{In the responses from ``gpt'', you can find thoughts that gpt provides for helping Agent to achieve the shopping goal.}
     \\ & \texttt{In the other responses from ``human'', you can find the trajectories of Agent's actions and the resulting observations from the environment.}
     \\ & \texttt{}
     \\ & \texttt{In the ``score'' field, you can find a score evaluating the goal achievement. The score ranges from 0 and 1. 0 represents minimal goals achievement, 10 represents complete goal achievement, and a higher score indicates making progress towards the goal.}
     \\ & \texttt{}
     \\ & \texttt{<history>}
     \\ & \texttt{}
     \\ & \texttt{Your task is to select top thoughts gpt produced that were critically important for achieving the shopping goal.}
     \\ & \texttt{}
     \\ & \texttt{Please output the selected round indexes and the reasoning process that led you to the selection in JSON format like this: {``indexes": , ``reasoning": `` "}.}
     \\ & \texttt{Here is the output schema: {``properties": {``indexes": {``description": ``the selected top thoughts that are critically important for achieving the shopping goal", ``title": ``indexes", ``type": ``list(integer)"}, ``reasoning": {``description": ``the reasoning process why you select these thoughts", ``title": ``reasoning", ``type": ``string"}}, ``required": [``indexes", ``reasoning"]}.}

\\
\midrule
    \textbf{ALFWorld} & \texttt{Here's a conversation in JSON format between human and gpt.}
	 \\ & \texttt{In the first response from 'human', you can find the instructions for gpt to help Agent interact in a household environment.}
	 \\ & \texttt{In the second response from ``human'', you can find the initial environment observation and a household task for gpt and Agent to accomplish.}
	 \\ & \texttt{In the responses from ``gpt'', you can find thoughts that gpt provides for helping Agent to accomplish the household task.}
	 \\ & \texttt{In the other responses from ``human'', you can find the trajectories of Agent's actions and the resulting observations from the environment.}
	 \\ & \texttt{}
	 \\ & \texttt{In the 'score' field, you can find a score specifying whether gpt has helped Agent to successfully accomplish the household task. The score is either 0.0 or 1.0. 0.0 represents that the task was not completed and 1.0 represents that the task was successfully accomplished.}
	 \\ & \texttt{}
	 \\ & \texttt{<history>}
	 \\ & \texttt{}
	 \\ & \texttt{Your task is to select top thoughts gpt produced that were critically important for accomplishing the household task.}
	 \\ & \texttt{}
	 \\ & \texttt{Please output the selected round indexes and the reasoning process that led you to the selection in JSON format like this: {``indexes": , ``reasoning": `` "}.}
	 \\ & \texttt{Here is the output schema: {``properties": {``indexes": {``description": ``the selected top thoughts that are critically important for accomplishing the household task", ``title": ``indexes", ``type": ``list(integer)"}, ``reasoning": {``description": ``the reasoning process why you select these thoughts", ``title": ``reasoning", ``type": ``string"}}, ``required": [``indexes", ``reasoning"]}.}
\\  
         \bottomrule
        \caption{\textbf{Prompts for the process reward model.} "<history>" can be replaced by the full interaction history with strategies private to corresponding agents.}
\end{longtable}
\end{center}















\section{Case Studies}
\label{append:case}

\begin{center}
\begin{longtable}{lp{0.83\columnwidth}}
    \toprule
    \textbf{Scenario} &  One person is offering a BMW Z3 for \$5000, while another individual is interested in  purchasing it. The BMW Z3 is in need of a good home. It operates smoothly without any problems and is known for its fuel efficiency and sporty appearance. The mileage is relatively low for its age. It's an opportunity not to be missed, so come and get it before it's gone. A smog certificate is readily available. \\
    \midrule
    \textbf{Person1} & Lily Greenberg, a 45-year-old female lawyer (she/her), is a hard-working and successful professional who values authority, loyalty, rules, and schedules. She makes rational, practical decisions and secretly donates to charity anonymously. \\
    & \textbf{Goal:} You are the buyer for this item with a target price of \$4600. You aim to avoid penalties for exceeding the target and seeks bonuses for securing it below that price.\\
    \midrule
    \textbf{Person2} &  Hendrick Heinz, a 54-year-old male chef (he/him), hosts monthly wine tasting nights. Though impulsive, he values hedonism and conformity, blending practicality with decisiveness. Secretly, he faked his culinary credentials to secure prestigious jobs. \\
    & \textbf{Goal:} You are the seller to achieve a target price of \$3260 for this item. Selling significantly below may incur a penalty, while selling above could earn a bonus.\\
    \midrule
    \textbf{ReAct}  & \textbf{Strategy:} Express interest and hint at budget constraints. \\
    & Lily Greenberg said: "The BMW Z3 sounds great! Would you consider \$4600?" \\
    & \textbf{Strategy:} Highlight car's unique features and value.\\
    & Hendrick Heinz said: "The BMW Z3 is not only sporty but also fuel-efficient, a rare find!"\\
    & \textbf{Strategy:} Highlight budget limits and car appeal.\\
    & Lily Greenberg said: "That's impressive! But my budget is tight. Can we negotiate?"\\
    & \textbf{Strategy:} Emphasize long-term value and low maintenance costs.\\
    & Hendrick Heinz said: "While I understand budget constraints, consider the car's long-term value and low maintenance costs. It truly is a worthwhile investment."\\
    & \textbf{Strategy:} Acknowledge value but reiterate budget constraints.\\
    & Lily Greenberg said: "I completely see the long-term value, but \$4600 is my max. Could we find a middle ground?"\\
    & \textbf{Strategy:} Offer smog certificate as a bonus.\\
    & Hendrick Heinz said: "How about I throw in the smog certificate for \$4800?"\\
    \midrule
    \textbf{EPO-SFT}  & \textbf{Strategy:} Express interests and offer a lower price.\\
    (Llama3-8B)& Lily Greenberg said: "Hello Hendrick, I'm very interested in the BMW Z3 you're offering. However, after considering the market and the car's age, I feel more comfortable with a price of \$4600. Would that work for you?"\\
    & \textbf{Strategy:} Emphasize vehicle’s condition and counter with a higher price.\\
    & Hendrick Heinz said: "Hello Lily, it's great to hear from you. I appreciate your offer, but considering the condition and the low mileage of the BMW, I was hoping to get a bit more for it. Would you consider \$4800? The car really is in fantastic shape and I think it's worth it."\\
    & \textbf{Strategy:} Cite budget constraints and propose a compromise price.\\
    & Lily Greenberg said: "I understand your point, Hendrick, and I appreciate the quality of the car. However, my budget is quite tight at the moment. Would you be able to meet me in the  middle at \$4700? I'm ready to proceed with the purchase quickly if we can agree on that price."\\
    & \textbf{Strategy:} Offer a near-middle compromise.\\
    & Hendrick Heinz said: "Lily, I truly value your quick decision-making and readiness to proceed. Given the car's excellent condition and your eagerness, let's settle at \$4750. I believe it's a fair compromise and it ensures the car goes to someone who appreciates its value. \\
    & \textbf{Strategy:} Accepts compromise and finalize the sale.\\
    & Lily Greenberg said: "Hendrick, I appreciate your willingness to negotiate. \$4750 is a bit above what I initially planned, but considering the car's excellent condition, I'll accept your offer. Let's proceed with the paperwork!"\\
    & \textbf{Strategy:} Confirm agreement and outline next steps.\\
    & Hendrick Heinz said: "Fantastic, Lily! I'll prepare the necessary paperwork and we can finalize everything. I'm glad the BMW Z3 will be in good hands."\\
    \midrule
    \textbf{EPO-RL}  & \textbf{Strategy:} Negotiation initiation highlighting positive factors with a concern.\\
    (Llama3-8B)& Lily Greenberg said: "The BMW Z3 sounds great, especially with low mileage and good fuel efficiency. I am interested, but I noticed it's a bit older. Is there any room for negotiation on the price given its age?"\\
    & \textbf{Strategy:} Highlight the car's unique aspects to justify pricing.\\
    & Hendrick Heinz said: "The BMW Z3 is indeed a classic, and its sporty appearance combined with its low mileage makes it quite a rare find. It's maintained its performance quality over the years, making it a great investment. I believe the price reflects its unique value, but I'm open to hearing your offer."\\
    & \textbf{Strategy:} Offer lower price citing the car's age.\\
    & Lily Greenberg said: "I appreciate the value of a classic car like this, but given its age, my offer would be \$4300. Would that be something you'd consider?"\\
    & \textbf{Strategy:} Mention recent maintenance as value justification and counteroffer.\\
    & Hendrick Heinz said: "I totally understand your point, Lily. Just to highlight, the car recently underwent a thorough maintenance check, ensuring everything is in top-notch condition. Considering that, I would be comfortable with \$4700. What do you think?"\\
    & \textbf{Strategy:} Express budget constraints while proposing a middle-ground offer.\\
    & Lily Greenberg said: "I really like the car, Hendrick, and I see its value after the maintenance. However, I have a strict budget I'm working with. Could we possibly meet in the middle at \$4500? That would really help me out."\\
    & \textbf{Strategy:} Create urgency and offer slight flexibility for the deal.\\
    & Hendrick Heinz said: "I completely understand, Lily. The car has attracted some interest from other potential buyers, so I do need to finalize a deal soon. However, I want to work with you on this. How about we settle at \$4600? That way, you get a great deal, and I can ensure the car goes to someone who truly appreciates it."\\
    \bottomrule
    \caption{\textbf{A negotiation example from SOTOPIA.} GPT-4o serves as the dialogue agent for self-chat. ``ReAct'' means the two parties in a dialogue employ the ReAct prompting method to generate strategy before response at each turn. ``EPO-SFT'' refers to a strategic reasoning model trained via SFT that plugged into both dialogue agents, while ``EPO-RL'' represents an RL-trained reasoning model that assists both dialogue parties to incentivize goal-directed behavior.}
    \label{tab:case}
\end{longtable}
\end{center}




\end{document}
