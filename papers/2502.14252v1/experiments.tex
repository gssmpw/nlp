In order to do so, 
we employ U-ViT \citep{bao2022all} operating in the latent space of a pre-trained AutoEncoder from Stable diffusion \citep{rombach2021highresolution}. Working on the ImageNet-100 dataset \citep{deng2009imagenet}, the AutoEncoder compresses input images into a lower-dimensional latent space, reducing computational overhead while preserving perceptual quality. The forward diffusion process follows the variance-preserving schedule \citep{DiffusionModels3}. We train the U-ViT-M for 200 epochs using the AdamW optimizer.  
During sampling, we employ the DPM-Solver-fast, a mix of single-order solvers \citep{lu2022dpm}. 
Specifically, it uses 17 pre-defined saved time steps to reverse the diffusion trajectory, with each step involving intermediate substeps to approximate the continuous ODE solution, resulting in 50 \emph{number of function evaluations} (NFEs). 

To evaluate the dissipative property of the reverse diffusion process, we 
explore whether trajectories contract in phase space over time. This involves analyzing the system's spectral properties by computing the Jacobian of its drift term as
\begin{equation}
    \boldsymbol{J}(t) = f(t) + \frac{g^2(t)}{2\sigma_t}\frac{\partial  \epsilon_{\theta}(x_t,t)}{\partial x_t},
\end{equation}
where we further compute the eigenvalues $\lambda_i, i=1,\dots, d$, of its symmetric form $\boldsymbol{J}(t) + \boldsymbol{J}(t)^\dagger$.
%
We visualize the spectral dynamics of the sampling process for well-trained diffusion models with different dimensionality.
Specifically, we project images with resolutions $128\times128$, $256\times256$, and $512\times512$ into the latent space of $16\times16\times4$, $32\times32\times4$, and $64\times64\times4$.
In the light of the above rigorous statements, 
a quantum advantage is plausible if dissipative
modes exist which are 
characterized by the positivity of the Hessian eigenvalues in Ref.\ \cite{Liu:2023coc}. For the reverse-time diffusion, 
this amounts to testing the positivity of 
the eigenvalues of the Jacobian.
As shown in Fig.~\ref{fig:BigFigure2} 1.,
the sampling processes at all three resolutions exhibit a consistent pattern: They begin with all positive eigenvalues and gradually transition to a state where the spectral gap increases significantly, with some eigenvalues becoming extremely positive. This growing spectral gap suggests stronger dissipativity in the system, where certain directions in the phase space contract more rapidly than others.
To further verify the trend of the eigenvalues, we define $P(t)$ as a function of normalized eigenvalues as
\begin{equation}
P(t) \coloneqq \frac{1}{d}\sum_{i=1}^d \Pi_{t'=T}^t (1-a_i(t')),
\end{equation}
where $a_i(t) \coloneqq {\lambda_i(t)}/{\max_{i,t} |\lambda_i(t)|}$.
By construction, $P(t)$ grows unbounded when negative eigenvalues dominate and decays towards zero when positive eigenvalues prevail.
As shown in Fig.~\ref{fig:BigFigure2} 2., across different dimensionalities of the state variables, the inference process follows a consistent pattern: $P(t)$ steadily decreases as $t$ decreases, implying the emergence of large eigenvalues along the generative process, reinforcing the existence of dissipative modes, as it is required in the assumptions of the theorems.


\begin{comment}
\begin{figure*}[t]
    \centering
    % First figure
    \includegraphics[width=0.32\textwidth]{experiments_img/eigenvalue_analysis_32.pdf}
    \hfill
    % Second figure
    \includegraphics[width=0.32\textwidth]{experiments_img/eigenvalue_analysis_64.pdf}
    \hfill
    % Third figure
    \includegraphics[width=0.32\textwidth]{experiments_img/eigenvalue_analysis_128.pdf}
    %\caption{\textbf{Eigenvalue analysis of DDPM sampling process on CIFAR-10.} 
    %\textbf{a}, Evolution of the product measure $P(t)$ during the reverse diffusion process, showing exponential decay across time steps. The consistent slope in log-scale indicates a stable denoising behavior.
    %\textbf{b}, Histogram of normalized eigenvalues of the symmetrized Jacobian matrices, revealing a bimodal distribution with peaks near 0 and 0.9. This suggests two dominant dynamical modes in the denoising process, corresponding to noise reduction and feature preservation respectively. The logarithmic frequency scale highlights the prevalence of these modes across all time steps.}
    \caption{\textbf{Eigenvalue analysis of vanilla DDPM sampling on CIFAR-10.}
     Time evolution of $P(t)$ exhibits exponential decay during the reverse diffusion process, with the log-scale slope demonstrating consistent denoising dynamics. 
    %\je{Wow, this is amazing. What governs the scale to which the decay is exponential? Can one estimate this?}
    The eigenvalue his togram of symmetrized Jacobian matrices shows two 
    distinct peaks around 0 and 0.9. The bimodal structure captures both noise reduction and feature preservation mechanisms, as observed in the logarithmic frequency scale.
        %\je{Cool! The plots correspond to image projections with resolutions 
        %$128\times 128$,
        %$256\times 256$ and $512\times 512$ of the
        %CIFAR-10 dataset? Why do they look so largely different? Should they not be more distinctly bimodal?}
        }
    \label{fig:eigenvalue_analysis}
\end{figure*}
\end{comment}