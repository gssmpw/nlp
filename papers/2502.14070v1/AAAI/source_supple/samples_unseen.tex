\subsection{Experiment with DrawBench}
In the main paper, we use the DrawBench~\cite{imagen} prompt set to demonstrate the effectiveness of our method on complex prompts. Here, we further evaluate its generalization ability using 150 challenging prompts from Parti~\cite{parti} as unseen evaluation prompts. The reward curve results for the unseen prompt experiments are presented in Figure~\ref{fig:unseen_reward_curve_sup}(a). 
As shown in Figure~\ref{fig:unseen_reward_curve_sup}(a), our method achieves improved sample efficiency and higher scores compared to the baseline, demonstrating superior generalization ability even on complex prompt sets. We provide the generated images in Figure~\ref{fig:drawbench_unseen_sup}.


\subsection{Experiment with SDXL}
In the main paper, we demonstrate that our method is also effective when applied to the more advanced diffusion model, Stable Diffusion XL (SDXL)~\cite{sdxl}. 
In this section, we further verify the generalization capability of the SDXL model fine-tuned with our method by testing it on unseen prompts.
Since we fine-tune SDXL using the Aesthetic reward model, we use unseen animal names to construct unseen prompts, which are included in Table~\ref{table:unseen_prompt}.
As shown in Figure~\ref{fig:unseen_reward_curve_sup}(b), our method significantly improves the sample efficiency compared to the baseline, confirming its ability to generalize to unseen prompts even with SDXL.  We provide the generated samples in Figure~\ref{fig:sdxl_unseen_sup}.