Graphs are crucial for representing complex structures and relationships across diverse domains such as social network analysis (SNA), Internet of Things (IoT), biological networks, and transportation networks. The surge in graph data necessitates a generalized framework for graph analysis. This analysis is essential for tasks such as graph-to-graph translation, link prediction, and community detection. Graph-to-graph translation involves transforming an input graph into an output graph while preserving certain topological properties \cite{guo2019deep, lowe2014patent, van2013wu}. This process includes modifying the graph structure by adding, removing, or rearranging vertices and edges. Applications of graph-to-graph translation include graph transformation \cite{rong2020self}, graph synthesis \cite{nebli2020deep}, graph generation \cite{tang2020unbiased}, graph adaptation \cite{bai2020adaptive}, and drug discovery \cite{wu2022spatial}. Link prediction \cite{wang2020edge2vec, zitnikbiosnap} is also an important problem in the domain of graph analysis,  contributing to the understanding and utilization of graph data.

Some recent state-of-the-art (SOTA) approaches such as NEC-DGT~\cite{guo2019deep} and GT-GAN~\cite{9737289}, in graph-to-graph translation domain, have shown limitations in addressing the challenges of semi-supervised learning tasks, commonly employed in community detection task. These challenges include the resolution limit problem, where smaller communities within larger ones may remain undetected, and dynamic networks, where network structures evolve over time. 
%Moreover, current approaches demonstrate its effectiveness within specific domains; for instance, NEC-DGT~\cite{guo2019deep} yields superior results compared to IoT~\cite{guo2019deep} or Chemistry Reaction Prediction~\cite{lowe2014patent} tasks. 
Additionally, it does not perform well on the task of predicting brain functional connectivity (BFC)~\cite{van2013wu}. One prevalent explanation pertains to the intricate and dynamic nature of BFC~\cite{van2013wu} data samples, characterized by complex connectivity of brain structures that vary over time. The relationships between nodes and edges are shaped by neural activity, the functions of brain regions, and external stimuli. This intricate nature poses a challenge for simplistic NEC-DGT~\cite{guo2019deep} frameworks in capturing the complex topological structure.

Another extensively researched area in the field of graph analysis is the study of link prediction, focusing on tasks like anticipating trust or friendship connections within social networks \cite{wang2020edge2vec, Tang-etal12c, leskovec2009community}. Various established approaches in this domain are edge2vec~\cite{wang2020edge2vec}, MEIM~\cite{DBLP:conf/ijcai/TranT22}, HOGCN~\cite{kishan2021predicting}, GHRS~\cite{darban2022ghrs}. However, these frameworks are not tailored for graph-to-graph translation problems, where specific nodes or edges from the source graph might not be present in the target graph. While these techniques ~\cite{wang2020edge2vec,kishan2021predicting} demonstrate proficiency in link prediction by effectively capturing intricate relationships and structures within a single graph, leveraging both local and global features to predict missing connections, they ~\cite{wang2020edge2vec,darban2022ghrs} fall short in the context of graph-to-graph translation, which necessitates more intricate modifications, demanding a generalized model to comprehend and adjust the graph's overall structure.


To alleviate these challenges, where most of the existing frameworks do not perform well for different tasks on various domains, we propose a generalized framework namely UGN~\footnote{Our source code is available here: https://anonymous.4open.science/r/UGN} which performs significantly well on different downstream tasks present in the existing literature of graph learning and analysis.

%\end{enumerate}
%\textbf{Our Contributions:} 
We make the following contributions in the paper.
\begin{enumerate}
\item We propose a novel encoder-decoder based framework to combine the strengths of neural message passing and convolutional neural networks for various graph analysis tasks. Empirical findings demonstrate that our proposed framework, with minimal modifications, consistently achieves excellent performance, yielding  state-of-the-art (SOTA) or comparable results across various downstream tasks in both supervised and semi-supervised settings.

\item Additionally, we propose a novel strategy to initialize node feature representation for processing large graphs in low resource environment.
\end{enumerate}


%\cite{abdelnour2014network,abu2017learning} 
% Graphs are a powerful way to represent complex structures and relationships between entities in various domains such as social networks analysis (SNA), Internet of Things (IoT), biological networks, transportation networks etc. Due to the steep rise of graph data points, it has become important to develop a generalized framework for analyzing the graph data. Graph analysis plays an important role on three important tasks in this domain which are graph-to-graph translation (\cite{guo2019deep}\cite{lowe2014patent}), link prediction (\cite{wang2020edge2vec}\cite{zitnikbiosnap}) and community detection (\cite{cai2019community}\cite{girvan2002community}). Graph-to-graph translation (\cite{lowe2014patent}\cite{van2013wu}) involves the transformation of an input graph into an output graph, typically with an aim of preserving a desired topological property of the source graph while translating it into the target graph. The translation process involves modifying the topological structure of the graph by adding, removing or rearranging vertices and edges of that particular source graph. Graph-to-graph translation can be applied for a wide range of tasks such as graph transformation (\cite{rong2020self}), graph synthesis (\cite{nebli2020deep}), graph generation (\cite{tang2020unbiased}), graph adaptation (\cite{bai2020adaptive}) and also in drug discovery (\cite{wu2022spatial}) etc. 



% Moreover, these methods cannot perform semi-supervised tasks such as community detection.
% \hlt{Community detection task is commonly employed to identify groups of vertices that exhibit denser connections among themselves than with the rest of the graph. Community detection} (\cite{zachary1977information}\cite{lu2018community}\cite{DBLP:journals/debu/HamiltonYL17}\cite{cai2019community}) \hlt{can be framed as a clustering problem with a sparse distribution of nodes having community labels. Some widely used methods for community detection include DDJKM}~\cite{cai2019community}, Equation (20)~\cite{tian2018community}, \hlt{among others. But these frameworks do not perform well for the graph-to-graph translation tasks. 

% Link prediction in graphs involves predicting the presence or absence of edges between pairs of vertices, considering various scenarios such as directed edges (e.g., trust or friendship in social networks (\cite{wang2020edge2vec}\cite{Tang-etal12c}\cite{leskovec2009community})), edges of multiple types (e.g., knowledge graph completion~\cite{suchanek2007yago}, relation extraction \cite{hendrickx2010semeval}), and weighted edges (e.g., movie-rating prediction \cite{miller2003movielens}.





% We combine the power of neural message passing networks and convolutional neural networks in an unique way. \hl{Our model has an encoder-decoder based architecture. In the encoder, the node features are encoded in a latent space by aggregating neighbour features of the nodes using graph convolutional layers. Once the nodes have latent representations, an intermediate matrix representation is created for each of the nodes and edges which acts as a one-channel image. In the decoder, 2D convolutional layers, pooling layers and linear layers are applied on the intermediate matrices for further downstream tasks like classification.}
% \sudip{(Not suitable material for contribution. Move this text to the end of introduction/related work as a brief description of the model. For the contribution, you don't have to detail the model, just give a summary of the model in a sentence or two.)}
% \item Our model is applicable to supervised tasks (node classification, edge classification, link prediction) as well as semi-supervised tasks (community detection).
% \item Our model is scalable and it performs well for small as well as large graphs.
% \item Our model is generic in nature. It can be applied to any learning task where the input data is represented as graph \textcolor{magenta}{\hl{and for the graph to graph translation task,}}\hl{ there is an one-to-one or many-to-one mapping between source and target}.
% \item While working with large graphs, \hl{the RAM-size is not enough to keep the initial node features} as one-hot-encodings. We proposed a smart technique, which we named super-node, for initializing node features. Our model produces SOTA or near-SOTA performance using this technique. 
% \item Our model produces SOTA or near-SOTA results in various domains like IoT, Chemistry Reaction, Social Networks, Community detection, biomedical, etc.

% \hl{Graph-to-graph translation involves transforming an input graph into an output graph, typically with the goal of achieving a desired property or structure. This translation process can involve modifying the structure, attributes, or topology of the graph, or adding, removing, or rearranging vertices and edges. Graph-to-graph translation can be utilized for a wide range of tasks, such as graph transformation, graph synthesis, graph generation, and graph adaptation. This task has applications in various domains, such as drug discovery, where the goal is to design new molecules with desired properties, or in computer vision, where the goal is to transform an input image into an output image with a different style or content. In social network analysis, it can be used to translate a social graph into a different format for analyzing community structures or information diffusion. (\madhu{Try to give proper citations for the above claims. Because these are not our claims.}) Graph neural networks (GNNs) are a popular approach for graph-to-graph translation, which operate on the graph structure and node features to learn a mapping function between the input and output graphs. (\madhu{How you are connecting this line with prevous one. It seems like an independent statement without any connection to the discussion before it. Whenever you are trying to put such type of comments in introduction section always keep them inside the citation})}

%\hl{Link prediction in graphs involves predicting the presence or absence of edges between pairs of vertices in a graph. This task is important for applications such as recommender systems, where the goal is to recommend new items to users based on their past interactions. Link prediction can be formulated as a binary classification problem. Given a graph $G=(V, E)$ with vertices $V$ and edges $E$, the goal of link prediction is to predict whether a new edge $(u, v)$ between two vertices $u$ and $v$ should be added to the graph or not. Let $A$ be the adjacency matrix of the graph $G$, where $A_{u,v} = 1$ if there is an edge between $u$ and $v$, and $A_{u,v} = 0$ otherwise. Let $S$ be the set of all possible edges that could be added to the graph, where $S = {(u, v) | u, v \in V, u \neq v}$. The task of link prediction can then be formalized as finding a function $f: S \rightarrow [0, 1]$ that assigns a probability score to each potential edge in $S$, indicating the likelihood of the edge being present in the graph. The function $f$ can be learned using various machine learning techniques, such as logistic regression, decision trees, or neural networks. The performance of the link prediction algorithm can be evaluated using metrics such as precision, recall, and area under the receiver operating characteristic curve (AUC-ROC). Machine learning techniques, such as graph embeddings and graph convolutional networks (GCNs), are commonly used for link prediction in graphs.}

% \hl{Community detection in graphs involves identifying groups of vertices that are more densely connected among themselves than with the rest of the graph. This task has applications in various domains, such as social networks, where the goal is to identify groups of individuals with similar interests or behaviors. Community detection can be formulated as a clustering problem, where the goal is to partition the graph into disjoint communities. Given a graph $G = (V, E)$ with vertices $V$ and edges $E$, the goal of community detection is to partition the graph into $K$ disjoint communities $C = {C_1, C_2, ..., C_K}$, where each community $C_k$ is a subset of vertices in $V$. Let $A$ be the adjacency matrix of the graph $G$, where $A_{u,v} = 1$ if there is an edge between $u$ and $v$, and $A_{u,v} = 0$ otherwise. Let $z = {z_1, z_2, ..., z_N}$ be a vector of community assignments, where $z_i = k$ indicates that vertex $i$ belongs to community $C_k$. The task of community detection can then be formalized as finding the optimal community assignments $z$ that maximize the modularity score $Q$:}
% \begin{equation}
% Q = \frac{1}{2m} \sum_{u,v=1}^N \left[A_{u,v} - \frac{k_uk_v}{2m}\right] \delta(z_u, z_v)
% \end{equation}
% \hl{where $m$ is the total number of edges in the graph, $k_u$ and $k_v$ are the degrees of vertices $u$ and $v$, and $\delta(z_u, z_v) = 1$ if $z_u = z_v$ and 0 otherwise. The modularity score $Q$ measures the difference between the actual number of edges between vertices in the same community and the expected number of edges if the edges were distributed randomly. The higher the modularity score, the better the community structure of the graph. The performance of the community detection algorithm can be evaluated using metrics such as modularity score, number of communities, and stability over multiple runs. Spectral clustering, modularity optimization, and hierarchical clustering are popular approaches for community detection in graphs.}

% ...


% Similarly, in computer vision, the objective involves translating an input image into an output image with distinct styles or content. In the context of SNA, this task is leveraged to translate a social graph into an alternative format, facilitating the analysis of community structures or information propagation.


% Moreover, this task has extensive applications across various domains, including drug discovery, where the goal is to design new molecules with desired properties. 



% Additionally, this task has application in various domains such as drug discovery, where the goal is to design new molecules with desired properties, or in computer vision, where the goal is to transform an input image into an output image with a different style or content. In social network analysis, it can be used to translate a social graph into a different format for analyzing community structures or information diffusion. (\madhu{Try to give proper citations for the above claims.}) Graph neural networks (GNNs) are a popular approach for graph-to-graph translation, which operate on the graph structure and node features to learn a mapping function between the input and output graphs.
% \textcolor{violet}{Graph-to-graph translation involves transforming an input graph representation into an output graph representation, typically with an aim of maintaining the desired topological structure. The nodes and edges can be of multiple types and the adjacency matrix can be weighted or real valued.} 


% Additionally, existing approaches perform well for a particular domains, for example NEC-DGT produces better results in compare to IoT or Chemistry Reaction Prediction tasks but it does not produce good results on the brain functional connectivity prediction task due to the nature of task. These works work well for particular domains such as NEC-DGT produces better results in compare to IoT or Chemistry Reaction prediction tasks. But it does not perform well on the brain functional connectivity prediction task whereas GT-GAN perform well in case of brain graphs but it performs poorly for IoT or Chemistry reaction. Further, these frameworks have some domain specific architectures which gives the intuition that they will perform weakly outside their respective domains.

% the problem of semi-supervised learning based framework such as community detection.~\madhu{(Is there any particular reasons why these models are not working properly?)}


% \textcolor{violet}{In this work we try to generalize the problem of graph learning and analysis and propose a framework that performs well across different domain.}


% We make the following contributions in the paper. 

%\begin{enumerate}

%\item We propose a novel framework/architecture for...
%\item Our model produces state-of-the-art results on ... datasets. ...


% \MG{Another well-studied literature in graph analysis domain is link prediction such as  predicting trust or friendship in social networks (\cite{wang2020edge2vec}\cite{Tang-etal12c}\cite{leskovec2009community}), edges of multiple types where existing established methods are include edge2vec~\cite{wang2020edge2vec}, MEIM~\cite{DBLP:conf/ijcai/TranT22}, HOGCN~\cite{kishan2021predicting}, GHRS~\cite{darban2022ghrs}. However, these frameworks are not equipped for graph-to-graph translation, where certain edges or nodes in the source graph may be omitted from the target graph. The likely reason behind this performance is that such type of frameworks are highly effective for tasks like link prediction because those are designed to capture the complex relationships and patterns within a single graph. In link prediction, the goal is to infer missing links based on the existing structure, which these networks handle well by leveraging local and global graph properties. However, graph-to-graph translation involves more complex transformations where certain edges or nodes from the source graph may be omitted in the target graph. This requires the model to not only understand the existing relationships but also to learn how to modify the graph structure.}