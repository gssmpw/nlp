% Graph learning and analysis have become increasingly popular in recent years, gaining significant importance due to the growing availability of graph structured datasets across diverse domains. However, existing algorithms are generally task specific and they lack generalization power. We try to bring most graph learning problems such as link prediction, node classification, community detection, graph to graph translation, knowledge graph completion etc. under one umbrella and propose a novel generalized framework with very little task specific add-ons to generalize the problem of graph learning and analysis. Our proposed framework, UGN, utilizes an encoder-decoder architecture, employing graph convolution network (GCN) layers for the encoder and 2D Convolutional layers, pooling layers, and linear layers for the decoder. An intermediate matrix formation technique is employed between the encoder and decoder,  which aids significantly to combine the power of GCN and Conv2D layers in a novel way. We have also introduced a novel feature, namely SuperNode feature, for graphs where node feature is not available. To make UGN compatible in both supervised and semi-supervised environments, we have incorporated unsupervised loss component, and for fully connected graphs, we have created an add-on called MTCM. UGN exhibits high scalability, attributed to its optimal performance even in low-resource environment. Additionally, UGN performs well both in supervised and semi-supervised setup. We employed our model on 6 different tasks using 12 different datasets. Our empirical results demonstrate that UGN outperforms the baseline models by producing SOTA results on 10 datasets and comparable results on the rest of the datasets.

Deep neural networks have enabled researchers to create powerful generalized frameworks, such as transformers, that can be used to solve well-studied problems in various application domains, such as text and image. However, such generalized frameworks are not available for solving graph problems. Graph structures are ubiquitous in many applications around us and many graph problems have been widely studied over years. In recent times, there has been a surge in deep neural network based approaches to solve graph problems, with growing availability of graph structured datasets across diverse domains. Nevertheless, existing methods are mostly tailored to solve a specific task and lack the capability to create a generalized model leading to solutions for different downstream tasks. In this work, we propose a novel, resource-efficient framework named \emph{U}nified \emph{G}raph \emph{N}etwork (UGN) by leveraging the feature extraction capability of graph convolutional neural networks (GCN) and 2-dimensional convolutional neural networks (Conv2D). UGN unifies various graph learning tasks, such as link prediction, node classification, community detection, graph-to-graph translation, knowledge graph completion, and more, within a cohesive framework, while exercising minimal task-specific extensions (e.g., formation of supernodes for coarsening massive networks to increase scalability, use of \textit{mean target connectivity matrix} (MTCM) representation for achieving scalability in graph translation task, etc.) to enhance the generalization capability of graph learning and analysis. We test the novel UGN framework for six uncorrelated graph problems, using twelve different datasets. Experimental results show that UGN outperforms the state-of-the-art baselines by a significant margin on ten datasets, while producing comparable results on the remaining dataset.


% \hlt{Graph learning and analysis have experienced a surge in popularity in recent years, gaining considerable significance due to the growing availability of graph structured datasets across diverse domains. Nevertheless, current algorithms tend to be tailored to specific tasks and lack the ability to generalize. To this end, we propose a novel universal graph neural network (UGN) by leveraging the potential of GCN and Conv2D in a novel, resource-efficient manner. More specifically, our proposed UGN unifies various graph learning tasks, such as link prediction, node classification, graph-to-graph translation, knowledge graph completion, and more, within a cohesive framework, while introducing minimal task-specific extensions to enhance the generalization capability of graph learning and analysis.
% We employ our framework on 5 different tasks using 9 different datasets. Our empirical results demonstrate that UGN outperforms state-of-the-art baselines by a significant margin on 8 datasets, while producing comparable results on the remaining dataset.}


% The model produced SOTA results on 10
% %\textcolor{violet}{RD: Sir, I think you now considered IoT-20, 40, 60 as different datasets. Earlier they were being considered as one. So, now, SOTA result will be for 11.5 datasets. Because for IoT-20, SOTA is for edge classification only.} 
% datasets and near-SOTA or comparable results on the rest.


% In this work, we propose a novel generalize framework namely \hlt{UGN}, which shows its generalization capability by solving different graph learning tasks such as link prediction, \textcolor{blue}{node classification,} community detection, graph to graph translation, knowledge graph completion etc.