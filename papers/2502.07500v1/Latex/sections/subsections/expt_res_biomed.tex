% with batch size 1 and learning rate 0.005 and the rest (100 subjects) were taken as test set. 

\subsubsection{Human Connectome Project}
 
 
For each of the five tasks in the HCP dataset (cf. Section \ref{dataset} para \ref{data:hcp}), SC and FC pairs of 239 subjects were
taken as the training set and the
rest (100 subjects) were considered as the test set. Training was performed with a batch size of 1 and a learning rate of 0.005. For each task, an MTCM representation is formed by averaging the functional connectivity (FC) matrices in the training set (cf. Section \ref{mtcm}). For each task, for each subject, a difference matrix representation is computed by subtracting the corresponding MTCM representation from its own FC representation. The model learns to predict the difference matrix from the SC information. The predicted difference matrix representation is added to the corresponding MTCM to arrive at the predicted FC representation.
%Fig. \ref{fig:HCP_training} shows the overall working process of our model for the HCP dataset.
%\begin{figure}[htbp]
\centering
\includegraphics[width=\linewidth]{Latex/images/HCP.png}
\caption{Training process of the model on the HCP dataset}
\label{fig:HCP_training}
\end{figure}

Table~\ref{tab:HCP_compare} %showcases the efficacy of our proposed framework and 
presents a comparative performance analysis (in terms of Pearson Correlation) of our framework with other baseline models for the SC to FC prediction task on the HCP dataset. Table \ref{tab:HCP_compare} reveals that our model outperforms all the other baseline models by a significant margin, thereby positioning itself as the SOTA technique for each of the 5 tasks %pertaining to SC to FC prediction 
on the HCP dataset. 



%\subsubsection{Drug Drug Interaction (DDI)}
\subsubsection{DDI, DTI, PPI}
\label{sec:ddi}

 \begin{comment}
 % For the DDI dataset, training set had 23,514 positive edges or interactions out of total 48,514 positive edges (representing interactions), 23,514 positive edges were considered for the training set and the rest (25,000) were used for testing. We conduct 
For the DDI dataset (cf. Section \ref{dataset} Para \ref{data:ddi}), out of total 48,514 edges (representing interactions), 23,514 edges were considered for the training set and the rest (25,000) were used for testing. Negative edges were sampled using balanced negative sampling for both the training and the test set and only nodes connected to any positive edge in the corresponding set (train/test) participated in negative sampling in that set. Counting both positive edges and negative edges, training was conducted with learning rate 0.005 on total 47,028 edges and testing was carried out on total 50,000 edges. Since node feature is not available in the dataset, the initial node features are generated through supernode, as in the case of the Slashdot dataset (cf. Section \ref{sec:slashdot}). Creation of 15 supernodes from the 1,514 nodes (representing drugs) resulted in 14 supernodes of size 101 and 1 supernode consisting of 100 nodes. Thus for each node, a 15-dimensional vector is produced to encode connections with the 15 supernodes. Additionally, a 135-dimensional random vector, whose values range from 0 to 1, is concatenated with the aforementioned vector, resulting in the initial node feature being 150-dimensional. 

%The performance of our proposed model for the link prediction task on the Drug Drug Interaction dataset is presented in Table \ref{tab:DDI_PRF}. Moreover, 
%Table \ref{tab:DDI_DTI_compare} presents a comparitive evaluation of our model against other well known baseline methods on the BioSNAP-DDI dataset in terms of Area Under Precision Recall Curve (AUPRC). The results depicted in Table \ref{tab:DDI_DTI_compare} demonstrate that our model performs comparably to the SOTA models for the link prediction task on the BioSNAP-DDI dataset.





% \textcolor{magenta}{\hl{Negative edges were sampled via negative sampling with sample size same as the number of positive edges in training or test set}\sudip{(Rephrase)} and \hl{only nodes present in any positive edge in the corresponding set (train/test) participated in negative sampling in that set}. \hl{Counting negative edges and positive edges both,}} training was conducted with learning rate 0.005 on total 47,028 edges and testing was carried out on total 50,000 edges. 


\subsubsection{Drug Target Interaction (DTI)}
% \hl{In this dataset, training set had 10,638 positive edges or interactions out of total 15,138 positive edges and the rest (45,00 interactions) were used for testing. Number of negative edges sampled were same as the number of positive edges in training or test set. Considering negative edges along with the positive edges, training was conducted with learning rate 0.005 on total 21,276 edges and testing was done on total 9,000 edges. Since no node feature is available in the dataset, initial node features are created similar to the Slashdot dataset. Out of total 5,017 drug-nodes and 2,324 target-nodes, 76 super-drug-nodes and 35 super-target-nodes are made. 75 super-drug-nodes are of size 66 drug-nodes and one super-drug-node is of size 67 drug-nodes. 21 super-target-nodes are of size 66 target-nodes and 14 super-target-nodes are of size 67 target-nodes. So, 111 dimensional vector was created for encoding connections with super-drug-nodes and super-target-nodes for each node and 59 dimensional random vector, whose values lie between 0 and -1, is concatenated with it, making the initial node feature 170 dimensional.} \sudip{(This is very similar to the text in the previous section (DDI).Try to rephrase as much as possible.)}
Similar to the DDI task (cf. Section~\ref{sec:ddi}\textcolor{blue}{)}, we undertook an investigation into the efficacy of our proposed framework for another important biomedical task -- DTI. Out of the 15,138 edges
%(representing interactions) 
in the BioSNAP-DTI dataset, we considered 10,638 edges for the training set and the rest (4,500 edges) for the test set. Number of negative edges sampled were same as the number
of positive edges for both the training and test set. Considering positive edges and negative edges together, training was conducted (with learning rate 0.005) on total 21,276 edges and evaluation was carried out on total 9,000 edges. To address the unavailability of node feature information, we employed the concept of supernode. We constructed 76 super-drug-nodes and 35 super-target-nodes from the 5,017 drug-nodes and 2,324 target-nodes in the dataset. 75 super-drug-nodes are of size 66 drug-nodes and one super-drug-node contains 67 drug-nodes. Similarly, 21 super-target-nodes were of size 66 target-nodes and 14 were of size 67 target-nodes. Thus, each node was encoded with a 111-dimensional vector, describing its connections with super-drug-nodes and super-target-nodes. Additionally, a 59-dimensional vector with values ranging between 0 and \hlc[orange]{-1} was concatenated with the aforementioned vector, resulting in an initial node feature of 170 dimensions. 

%The performance of our proposed model on the link prediction task for the SNAP DTI dataset is presented in Table \ref{tab:DTI_PRF}. Furthermore, 
%Table \ref{tab:DDI_DTI_compare} provides a comparative analysis of our model's performance, in terms of AUPRC, to other established methods for the BioSNAP-DTI dataset. Our model demonstrates SOTA performance for the link prediction task on the BioSNAP-DTI dataset, as depicted in Table \ref{tab:DDI_DTI_compare}.

\subsubsection{Protein Protein Interaction (PPI)}

We also tested our model on another important graph learning task in the biomedical domain -- PPI. Out of the 52,068 positive edges (representing interactions) in the HuRI-PPI dataset, 36,048 edges were considered for the training set and the rest (15,620) were used for testing. Negative edges were sampled maintaining 1:1 ratio for both the training and test set. Thus training took place on total 72,096
edges  (with learning rate 0.005) and testing was carried out on total 31,240 edges. Because of the absence of node feature information in the dataset, we initialized the node feature by employing the concept of supernode. 103 supernodes were constructed from 8,245 nodes (proteins), 5 supernodes of size 81 and 98 supernodes of size 80. Thus, for each node, a 103-dimensional vector is created to encode connections to 103 supernodes, and a 57-dimensional random vector, whose values lie between 0 to 1, is conjoined with it to circumvent duplicate node features, resulting in 160 dimensional initial node features. 

%\madhu{The performance of our proposed model on the link prediction task for the Protein Protein Interaction dataset is presented in Table \ref{tab:PPI_PRF}. 
%Table \ref{tab:DDI_DTI_compare} compares the performance of our model (in terms of AUPRC) on the HuRI-PPI dataset against other well-known methods. It can be seen from Table \ref{tab:DDI_DTI_compare} that our model produces SOTA results for the link prediction task on the HuRI-PPI dataset.
\end{comment}

Table \ref{tab:DDI_DTI_PPI_expt_res} presents the experimental setup for the experiments on the BioSNAP-DDI, BioSNAP-DTI and HuRI-PPI datasets.
Table \ref{tab:DDI_DTI_compare} presents a comparison of the performance evaluation of our model, in terms of Area Under Precision Recall Curve (AUPRC), against other well-known baseline methods on the BioSNAP-DDI, BioSNAP-DTI and HuRI-PPI datasets. The results depicted in Table \ref{tab:DDI_DTI_compare} demonstrate that our model produces SOTA results for the link prediction task on BioSNAP-DTI and HuRI-PPI datasets and it performs comparably on the BioSNAP-DDI dataset.

% Because of the absence of node features in the dataset, initial node features are created following the same technique as for the Slashdot dataset (cf. Section \ref{sec:slashdot}).

% For the PPI dataset, out of total 52,068 positive edges (representing interactions), 36,048 positive edges were considered for the training set and the rest (15,620) were used for testing. Negative edges were created using negative sampling and number of negative edges were kept the same as the number of positive edges in training or test set. Training took place on total 72,096 edges with LR 0.005 and testing was carried out on total 31,240 edges. 

% 103 supernodes are created from the 8,245 nodes (proteins); 5 supernodes are of size 81 and 98 supernodes comprising of 80 nodes. So, 103 dimensional vector is formed to encode connections to 103 supernodes for each node and a 57 dimensional random vector, whose values lie between 0 and 1, is concatenated with it to avoid duplicate node features, making the initial node features 160 dimensional. Table \ref{tab:PPI_PRF} presents the performance of our proposed model on the link prediction task on the Protein Protein Interaction dataset. Table \ref{tab:DDI_DTI_compare} compares the performance of our model on the HuRI-PPI dataset in terms of AUPRC to other well known methods. Table \ref{tab:DDI_DTI_compare} shows that, our model produces SOTA performance on the HuRI-PPI dataset for link prediction task.
\begin{comment}
\begin{table}[htbp]
\centering
\begin{tabular}{|c|ccccc|}
\cline{2-6}
\multicolumn{1}{l|}{}        & \multicolumn{5}{c|}{Pearson Correlation}\\ \hline
\diagbox{Method}{\rotatebox{90}{Task}} & \rotatebox{90}{Res} & \rotatebox{90}{Emo} & \rotatebox{90}{Gam} & \rotatebox{90}{Lan} & \rotatebox{90}{Motor} \\
\hline\hline
\citet{galan2008network} & 0.23 & 0.14 & 0.14 & 0.14 & 0.15 \\
\citet{abdelnour2014network} & 0.23 & 0.14 & 0.14 & 0.15 & 0.15 \\
\citet{meier2016mapping} & 0.26 & 0.16 & 0.15 & 0.16 & 0.16 \\
\citet{abdelnour2018functional} & 0.23 & 0.14 & 0.14 & 0.15 & 0.15 \\
NEC-DGT \cite{guo2019deep} & 0.13 & 0.34 & 0.04 & 0.17 & 0.14 \\
GT-GAN \cite{9737289} & 0.45 & 0.34 & 0.34 & 0.35 & 0.36 \\
UGN & \textbf{0.61} & \textbf{0.67} & \textbf{0.68} & \textbf{0.66} & \textbf{0.66} \\
\hline
\end{tabular}
\caption{Performance Comparison on the HCP dataset}
\label{tab:HCP_compare}
\end{table}
\end{comment}

\begin{comment}
\begin{table}[htbp]
\centering
\begin{tabular}{|c|c|c|c|}
\cline{2-4}
\multicolumn{1}{l|}{}        & \multicolumn{3}{c|}{AUPRC}\\ \hline
Method & DDI & DTI & PPI \\
\hline
DeepWalk \cite{perozzi2014deepwalk} & 0.69 & 0.75 & 0.72 \\
node2vec \cite{grover2016node2vec} & 0.79 & 0.77 & 0.77\\
L3 \cite{kovacs2019network} & 0.85 & 0.89 & 0.90\\
VGAE \cite{kipf2016variational} & 0.83 & 0.85 & 0.88\\
GCN \cite{DBLP:conf/iclr/KipfW17} & 0.84 & 0.90 & 0.91\\
SkipGNN \cite{huang2020skipgnn} & 0.85 & 0.93 & 0.92\\
HOGCN \cite{kishan2021predicting} & \textbf{0.88} & \textbf{0.94} & 0.93 \\
UGN & 0.85 & \textbf{0.94} & \textbf{0.96}\\
\hline
\end{tabular}
\caption{Performance Comparison on the DDI, DTI and PPI datasets}
\label{tab:DDI_DTI_compare}
\end{table}
\end{comment} 

% \begin{table*}[htbp]
% \begin{tabular}{||c|c|c|c|c|c|c|c|c||}
% \hline
% \multicolumn{1}{||l|}{Dataset} &
%   \begin{tabular}[c]{@{}c@{}}\# Positive\\ Train Edges\end{tabular} &
%   \begin{tabular}[c]{@{}c@{}}\# Positive\\ Test Edges\end{tabular} &
%   \begin{tabular}[c]{@{}c@{}}Negative \\ Sampling ratio\end{tabular} &
%   \begin{tabular}[c]{@{}c@{}}SuperNode\\ Type\end{tabular} &
%   \begin{tabular}[c]{@{}c@{}}Order of \\ SuperNode\end{tabular} &
%   \begin{tabular}[c]{@{}c@{}}\# SuperNodes\end{tabular} &
%   \begin{tabular}[c]{@{}c@{}}SuperNode Feature\\ Dimension\end{tabular} &
%   \begin{tabular}[c]{@{}c@{}}Random Vector \\ Dimension\end{tabular} \\ \hline\hline
% \multirow{2}{*}{DDI} &
%   \multirow{2}{*}{23,514} &
%   \multirow{2}{*}{25,000} &
%   \multirow{2}{*}{1:1} &
%   \multirow{2}{*}{Drug} &
%   101 &
%   14 &
%   \multirow{2}{*}{15} &
%   \multirow{2}{*}{135} \\ \cline{6-7}
%  &
%    &
%    &
%    &
%    &
%   100 &
%   1 &
%    &
%    \\ \hline\hline
% \multirow{4}{*}{DTI} &
%   \multirow{4}{*}{10,638} &
%   \multirow{4}{*}{4,500} &
%   \multirow{4}{*}{1:1} &
%   \multirow{2}{*}{Drug} &
%   66 &
%   75 &
%   \multirow{4}{*}{111} &
%   \multirow{4}{*}{59} \\ \cline{6-7}
%  &
%    &
%    &
%    &
%    &
%   67 &
%   1 &
%    &
%    \\ \cline{5-7}
%  &
%    &
%    &
%    &
%   \multirow{2}{*}{Target} &
%   66 &
%   21 &
%    &
%    \\ \cline{6-7}
%  &
%    &
%    &
%    &
%    &
%   67 &
%   14 &
%    &
%    \\ \hline\hline
% \multirow{2}{*}{PPI} &
%   \multirow{2}{*}{36,048} &
%   \multirow{2}{*}{15,620} &
%   \multirow{2}{*}{1:1} &
%   \multirow{2}{*}{Target} &
%   81 &
%   5 &
%   \multirow{2}{*}{103} &
%   \multirow{2}{*}{57} \\ \cline{6-7}
%  &
%    &
%    &
%    &
%    &
%   80 &
%   98 &
%    &
%    \\ \hline
% \end{tabular}
% \caption{Experimental setup for DDI, DTI and PPI datasets}
% \label{tab:DDI_DTI_PPI_expt_res}
% \end{table*}


% \documentclass{article}
% \usepackage{graphicx}
% \usepackage{multirow}
% \usepackage{array}

\begin{table*}[htbp]
\centering
\scalebox{0.8}{
\begin{tabular}{||c|c|c|c|c|c|c|c|c||}
\hline
\multicolumn{1}{||l|}{Dataset} &
  \begin{tabular}[c]{@{}c@{}}\# Positive\\ Train Edges\end{tabular} &
  \begin{tabular}[c]{@{}c@{}}\# Positive\\ Test Edges\end{tabular} &
  \begin{tabular}[c]{@{}c@{}}Negative \\ Sampling ratio\end{tabular} &
  \begin{tabular}[c]{@{}c@{}}SuperNode\\ Type\end{tabular} &
  \begin{tabular}[c]{@{}c@{}}Order of \\ SuperNode\end{tabular} &
  \begin{tabular}[c]{@{}c@{}}\# SuperNodes\end{tabular} &
  \begin{tabular}[c]{@{}c@{}}SuperNode Feature\\ Dimension\end{tabular} &
  \begin{tabular}[c]{@{}c@{}}Random Vector \\ Dimension\end{tabular} \\ \hline\hline
\multirow{2}{*}{DDI} &
  \multirow{2}{*}{23,514} &
  \multirow{2}{*}{25,000} &
  \multirow{2}{*}{1:1} &
  \multirow{2}{*}{Drug} &
  101 &
  14 &
  \multirow{2}{*}{15} &
  \multirow{2}{*}{135} \\ \cline{6-7}
 &
   &
   &
   &
   &
  100 &
  1 &
   &
   \\ \hline\hline
\multirow{4}{*}{DTI} &
  \multirow{4}{*}{10,638} &
  \multirow{4}{*}{4,500} &
  \multirow{4}{*}{1:1} &
  \multirow{2}{*}{Drug} &
  66 &
  75 &
  \multirow{4}{*}{111} &
  \multirow{4}{*}{59} \\ \cline{6-7}
 &
   &
   &
   &
   &
  67 &
  1 &
   &
   \\ \cline{5-7}
 &
   &
   &
   &
  \multirow{2}{*}{Target} &
  66 &
  21 &
   &
   \\ \cline{6-7}
 &
   &
   &
   &
   &
  67 &
  14 &
   &
   \\ \hline\hline
\multirow{2}{*}{PPI} &
  \multirow{2}{*}{36,048} &
  \multirow{2}{*}{15,620} &
  \multirow{2}{*}{1:1} &
  \multirow{2}{*}{Target} &
  81 &
  5 &
  \multirow{2}{*}{103} &
  \multirow{2}{*}{57} \\ \cline{6-7}
 &
   &
   &
   &
   &
  80 &
  98 &
   &
   \\ \hline
\end{tabular}
}
\caption{Experimental setup for DDI, DTI and PPI datasets}
\label{tab:DDI_DTI_PPI_expt_res}
\end{table*}


\begin{comment}
\begin{table*}[htbp]
\centering
\begin{tabular}{||c|c|c|c|c|c|c|c||}
\hline
Method & \makecell{DihEdral \\\cite{xu2019relation}} & \makecell{InteractE \\\cite{vashishth2020interacte}} & \makecell{RefE \\\cite{chami2020low}} & \makecell{MEI \\\cite{nghiep2020multi}} & \makecell{ComplEx-DURA \\\cite{zhang2020duality}} & \makecell{MEIM \\\cite{DBLP:conf/ijcai/TranT22}} & UGN \\
\hline
\hline
HITS@1 & 0.38 & 0.46 & 0.50 & 0.51 & 0.51 & 0.51 & \textbf{0.61}\\
\hline
\end{tabular}
\caption{Performance Comparison on the YAGO dataset}
\label{tab:YAGO_compare}
\end{table*}
\end{comment}

\begin{comment}
\begin{table*}[htbp]
\centering
\begin{tabular}{||c|c||}
\hline
Method & HITS@1\\
\hline\hline
\makecell{DihEdral \cite{xu2019relation}} & 0.38\\
\makecell{InteractE \cite{vashishth2020interacte}} & 0.46\\
\makecell{RefE \cite{chami2020low}} & 0.50\\
\makecell{MEI \cite{nghiep2020multi}} & 0.51\\
\makecell{ComplEx-DURA \cite{zhang2020duality}} & 0.51\\
\makecell{MEIM \cite{DBLP:conf/ijcai/TranT22}} & 0.51\\
UGN & \textbf{0.61}\\
\hline
\end{tabular}
\caption{Performance Comparison on the YAGO dataset}
\label{tab:YAGO_compare}
\end{table*}
\end{comment}

\begin{table*}[htbp]
    \begin{minipage}{.3\linewidth}
      \centering
      \begin{tabular}{||c|c||}
        \hline
        Method & HITS@1\\
        \hline\hline
        \makecell{DihEdral \cite{xu2019relation}} & 0.38\\
        \makecell{InteractE \cite{vashishth2020interacte}} & 0.46\\
        \makecell{RefE \cite{chami2020low}} & 0.50\\
        \makecell{MEI \cite{nghiep2020multi}} & 0.51\\
        \makecell{ComplEx-DURA \cite{zhang2020duality}} & 0.51\\
        \makecell{MEIM \cite{DBLP:conf/ijcai/TranT22}} & 0.51\\
        UGN & \textbf{0.61}\\
        \hline
        \end{tabular}
        \caption{Performance Comparison on the YAGO dataset}
        \label{tab:YAGO_compare}
    \end{minipage}%
    \begin{minipage}{.3\linewidth}
    \centering
    \begin{tabular}{||c|c|c|c||}
        \hline
        Train & Test & AUPRC & Acc.\\
        \hline\hline
        \rowcolor{Gray}
        DDI & DDI & 0.85 & 0.78\\
        PPI & DDI & \textit{\textbf{0.86}} & \textit{0.78}\\
        DTI & DDI & 0.82 & \textit{\textbf{0.80}}\\
        \hline\hline
        \rowcolor{Gray}
        PPI & PPI & 0.96 & 0.90\\
        DDI & PPI & \textit{\textbf{0.97}} & \textit{0.91}\\
        DTI & PPI & \textit{\textbf{0.97}} & \textit{\textbf{0.92}}\\
        \hline\hline
        \rowcolor{Gray}
        DTI & DTI & \textbf{0.94} & \textbf{0.84}\\
        DDI & DTI & 0.92 & \textit{\textbf{0.84}}\\
        PPI & DTI & 0.90 & 0.82\\
        \hline
        \end{tabular}
        \caption{Zero-Shot learning on \\the biomedical datasets}      \label{tab:biomedical_zeroshot}
    \end{minipage}%
    \begin{minipage}{.3\linewidth}
    \centering
    \begin{tabular}{||c|c|c||}
        \hline
        \makecell{Max. train samples\\per relation type} & \makecell{Test\\accuracy} & \makecell{Number of\\train edges}\\
        \hline\hline
        10 & 0.10 & 370\\
        50 & 0.14 & 1,819\\
        100 & 0.16 & 3,587\\
        500 & 0.28 & 16,504\\
        1,000 & 0.36 & 30,656\\
        2,000 & 0.42 & 55,958\\
        4,000 & 0.48 & 99,351\\
        \hdashline
        8,000 & 0.53 & 156,184\\
        16,000 & 0.57 & 224,142\\
        \hline
        \end{tabular}
        \caption{Few-Shot learning on the YAGO dataset}
        \label{tab:YAGO_fewshot}
    \end{minipage}%
\end{table*}
    