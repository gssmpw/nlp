
Community detection is a node-level multi-class classification task. Since it is a semi-supervised task, the required loss function in the training process has an unsupervised loss component ($L_u$) as mentioned in Section ~\ref{unsupervised} (cf. Equation (\ref{eqn:unsup_loss})) along with the categorical cross-entropy loss ($L_s$, cf. Equation (\ref{eqn:sup_loss})) of the nodes (one from each community) whose label (community) is revealed during the training process. We carried out training for this task by using a learning rate of 0.005 employing a combination of the unsupervised and supervised losses as mentioned in Section \ref{sec:SpecialCases}.
% Let, $E$ be the edge set of the graph and $O$ is
% a
% an array-like of size $N\times C$, where $N$ is the number of nodes in the graph and $C$ is the number of communities. $O[i]$ denotes the $C$ dimensional transformer output-vector for $node_i$. $L_u$ is defined as in Equation \ref{loss_function}. 
% \begin{equation}
% \label{loss_function}   
% L_u=\sum_{(u,v) \in E}MSE(O[u],O[v])=\sum_{(u,v) \in E}\sum_{c=0}^{C-1}\frac{1}{C}(O[u][c]-O[v][c])^2 
% \end{equation}

%Fig. \ref{fig:karate2c}, \ref{fig:karate4c}, \ref{fig:football}, and \ref{fig:polbooks} visually present the performance of our model on the 3 datasets where spatial clustering represents the original communities and the color-coded nodes signify the communities predicted by our model.

% We used a learning rate of 0.005 for training on all the 3 Community Detection datasets.

%\subsubsection{Zachary's karate club}
%Additionally, Table \ref{tab:karate2_PRF} and \ref{tab:karate4_PRF} present the performance of our model on the community detection task on the Zachary's karate club dataset with 2 communities and 4 communities respectively. \sout{Figure \ref{fig:karate} pictorially shows the performance of our model on the 2 versions of the Zachary's karate club dataset.}

%\begin{figure}
\begin{center}
\begin{tikzpicture}[scale = .6]
\tikzstyle{lm1} = [fill = yafcolor5!50!black, inner sep = 1.2pt]
\tikzstyle{lm2} = [circle, fill = yafcolor4!50!black, inner sep = 1pt]
\tikzstyle{lm3} = [circle, fill = gray!50, inner sep = 1pt]
\tikzstyle{ll} = []
%\tikzstyle{le1} = [yafcolor2, thick, bend left = 10, opacity = 0.5]
%\tikzstyle{le2} = [gray, bend left = 10, opacity = 0.5]
\tikzstyle{le1} = [yafcolor5, thick, bend left = 10]
\tikzstyle{le2} = [yafcolor4, thick, bend left = 10]
\tikzstyle{le3} = [gray!50, bend left = 10]
\input{karate_graph/karate_sintos}
\node[anchor=south west] at (-0.5, -3) {(a)};
\end{tikzpicture}
\hspace{0.5cm}
\begin{tikzpicture}[scale = .6]
\tikzstyle{lm1} = [fill = yafcolor3!50!black, inner sep = 1.2pt]
\tikzstyle{lm2} = [circle, fill = yafcolor4!50!black, inner sep = 1pt]
\tikzstyle{lm3} = [circle, fill = gray!50, inner sep = 1pt]
\tikzstyle{ll} = []
%\tikzstyle{le1} = [yafcolor2, thick, bend left = 10, opacity = 0.5]
%\tikzstyle{le2} = [gray, bend left = 10, opacity = 0.5]
\tikzstyle{le1} = [yafcolor4, thick, bend left = 10]
\tikzstyle{le2} = [yafcolor5, thick, bend left = 10]
\tikzstyle{le3} = [gray!50, bend left = 10]
\input{karate_graph/karate_greedy}
\node[anchor=south west] at (-0.5, -3) {(b)};
\end{tikzpicture}
\hspace{0.5cm}
\begin{tikzpicture}[scale = .6]
\tikzstyle{lm1} = [fill = yafcolor3!50!black, inner sep = 1.2pt]
\tikzstyle{lm2} = [circle, fill = yafcolor4!50!black, inner sep = 1pt]
\tikzstyle{lm3} = [circle, fill = gray!50, inner sep = 1pt]
\tikzstyle{ll} = []
%\tikzstyle{le1} = [yafcolor2, thick, bend left = 10, opacity = 0.5]
%\tikzstyle{le2} = [gray, bend left = 10, opacity = 0.5]
\tikzstyle{le1} = [yafcolor4, thick, bend left = 10]
\tikzstyle{le2} = [yafcolor5, thick, bend left = 10]
\tikzstyle{le3} = [gray!50, bend left = 10]

\node[lm2] (n0) at (-0.39521, -0.90369) {};
\node[lm2] (n1) at (-0.1494, -0.63131) {};
\node[lm2] (n2) at (0.089756, -0.058231) {};
\node[lm2] (n3) at (0.32331, -1.0532) {};
\node[lm3] (n4) at (0.023263, -1.9859) {};
\node[lm3] (n5) at (-1.2108, -1.6373) {};
\node[lm3] (n6) at (-0.73841, -1.9544) {};
\node[lm3] (n7) at (0.69671, -1.0172) {};
\node[lm2] (n8) at (0.45414, -0.064462) {};
\node[lm3] (n9) at (1.35, 0.053898) {};
\node[lm3] (n10) at (-0.47034, -1.7978) {};
\node[lm3] (n11) at (-1.6432, -0.89285) {};
\node[lm3] (n12) at (0.52145, -1.7831) {};
\node[lm2] (n13) at (0.59789, -0.39239) {};
\node[lm3] (n14) at (1.15, 1.6278) {};
\node[lm3] (n15) at (1.644, 0.73643) {};
\node[lm3] (n16) at (-1.4464, -2.6101) {};
\node[lm3] (n17) at (-1.278, -0.60231) {};
\node[lm3] (n18) at (0.51654, 1.9636) {};
\node[lm3] (n19) at (-0.30691, 0.035621) {};
\node[lm3] (n20) at (0.10109, 1.891) {};
\node[lm3] (n21) at (-1.1086, -1.0869) {};
\node[lm3] (n22) at (1.4907, 1.075) {};
\node[lm3] (n23) at (-0.2767, 1.6423) {};
\node[lm3] (n24) at (-1.5541, 0.77017) {};
\node[lm3] (n25) at (-1.2836, 1.3377) {};
\node[lm3] (n26) at (1.5531, 1.5267) {};
\node[lm3] (n27) at (-0.5343, 0.9102) {};
\node[lm3] (n28) at (-0.16321, 0.83704) {};
\node[lm3] (n29) at (0.70218, 1.7305) {};
\node[lm2] (n30) at (0.93024, 0.14795) {};
\node[lm3] (n31) at (-0.5944, 0.37166) {};
\node[lm2] (n32) at (0.42538, 0.99008) {};
\node[lm2] (n33) at (0.58393, 0.82341) {};
\begin{pgfonlayer}{background}

% strong edges

\draw (n0) edge[le1] (n1);
\draw (n0) edge[le1] (n2);
\draw (n0) edge[le1] (n3);
\draw (n0) edge[le1] (n13);

\draw (n1) edge[le1] (n2);
\draw (n1) edge[le1] (n3);
\draw (n1) edge[le1] (n13);

\draw (n2) edge[le1] (n3);
\draw (n2) edge[le1] (n13);

\draw (n3) edge[le1] (n13);

\draw (n8) edge[le1] (n32);
\draw (n8) edge[le1] (n30);
\draw (n8) edge[le1] (n33);



\draw (n30) edge[le1] (n33);
\draw (n30) edge[le1] (n32);
\draw (n32) edge[le1] (n33);


% weak edges

\draw (n0) edge[le2] (n8);
\draw (n1) edge[le2] (n30);
\draw (n2) edge[le2] (n8);
\draw (n2) edge[le2] (n32);
\draw (n13) edge[le2] (n33);





%  other edges


\draw (n0) edge[le3] (n31);
\draw (n2) edge[le3] (n28);
\draw (n28) edge[le3] (n33);
\draw (n31) edge[le3] (n32);
\draw (n31) edge[le3] (n33);
\draw (n28) edge[le3] (n31);

\draw (n0) edge[le3] (n4);
\draw (n0) edge[le3] (n5);
\draw (n0) edge[le3] (n6);
\draw (n0) edge[le3] (n7);

\draw (n0) edge[le3] (n10);
\draw (n0) edge[le3] (n11);
\draw (n0) edge[le3] (n12);
\draw (n0) edge[le3] (n17);
\draw (n0) edge[le3] (n19);
\draw (n0) edge[le3] (n21);
\draw (n1) edge[le3] (n7);
\draw (n1) edge[le3] (n17);
\draw (n1) edge[le3] (n19);
\draw (n1) edge[le3] (n21);



\draw (n2) edge[le3] (n7);

\draw (n2) edge[le3] (n9);

\draw (n2) edge[le3] (n27);

\draw (n3) edge[le3] (n7);
\draw (n3) edge[le3] (n12);
\draw (n4) edge[le3] (n6);
\draw (n4) edge[le3] (n10);
\draw (n5) edge[le3] (n6);
\draw (n5) edge[le3] (n10);
\draw (n5) edge[le3] (n16);
\draw (n6) edge[le3] (n16);

\draw (n9) edge[le3] (n33);

\draw (n14) edge[le3] (n32);
\draw (n14) edge[le3] (n33);
\draw (n15) edge[le3] (n32);
\draw (n15) edge[le3] (n33);
\draw (n18) edge[le3] (n32);
\draw (n18) edge[le3] (n33);
\draw (n19) edge[le3] (n33);
\draw (n20) edge[le3] (n32);
\draw (n20) edge[le3] (n33);
\draw (n22) edge[le3] (n32);
\draw (n22) edge[le3] (n33);
\draw (n23) edge[le3] (n25);
\draw (n23) edge[le3] (n27);
\draw (n23) edge[le3] (n29);
\draw (n23) edge[le3] (n32);
\draw (n23) edge[le3] (n33);
\draw (n24) edge[le3] (n25);
\draw (n24) edge[le3] (n27);
\draw (n24) edge[le3] (n31);
\draw (n25) edge[le3] (n31);
\draw (n26) edge[le3] (n29);
\draw (n26) edge[le3] (n33);
\draw (n27) edge[le3] (n33);


\draw (n29) edge[le3] (n32);
\draw (n29) edge[le3] (n33);




\end{pgfonlayer}





%%%%%%%%%%%%%%%%%%
% NEW IP implemenation 
% 0.7

% strong 
% [(0, 1), (0, 2), (0, 3), (0, 7), (1, 2), (1, 3), (1, 7), (2, 3), (2, 7), (3, 7), (8, 30), (8, 32), (8, 33), (28, 31), (30, 32), (30, 33), (32, 33)]
% obj:  2.2333333333333325 subgraph :  12, st 17.0 wk 14.0, itr 10
% IP based : lam: 0.7 density : 2.2333333333333325 size: 12


% [0, 1, 2, 3, 7, 8, 13, 28, 30, 31, 32, 33]

% weak
% [(0, 8), (0, 13), (0, 31), (1, 13), (1, 30), (2, 8), (2, 13), (2, 28), (2, 32), (3, 13), (13, 33), (28, 33), (31, 32), (31, 33)]


%%%%%%%%%%%%%%%%%%
%0.5

% [(0, 1), (0, 2), (0, 3), (0, 13), (1, 2), (1, 3), (1, 13), (2, 3), (2, 13), (3, 13), (8, 30), (8, 32), (8, 33), (30, 32), (30, 33), (32, 33)]

% [(0, 8), (1, 30), (2, 8), (2, 32), (13, 33)]

% [0, 1, 2, 3, 8, 13, 30, 32, 33]
% obj:  2.0555555555555554 subgraph :  9, st 16.0 wk 5.0, itr 10
% IP based : lam: 0.5 density : 2.0555555555555554 size: 9
% 0.22390294075012207 seconds ---


%%%%%%%%%%%%%%%%%%Full-graph
% [(0, 1), (0, 2), (0, 3), (0, 7), (1, 2), (1, 3), (1, 7), (2, 3), (2, 7), (3, 7), (4, 10), (5, 6), (5, 16), (6, 16), (8, 30), (8, 32), (8, 33), (23, 27), (24, 25), (24, 31), (25, 31), (26, 29), (30, 32), (30, 33), (32, 33)]

% SUBGRPAH OF [0, 1, 2, 3, 8, 13, 30, 32, 33]
% [(0, 1), (0, 2), (0, 3), (1, 2), (1, 3), (2, 3), , (8, 30), (8, 32), (8, 33),, (30, 32), (30, 33), (32, 33)]

% (0,13), (1, 13), (2, 13), (3,13) : w to s
\node[anchor=south west] at (-0.5, -3) {(c)};
\end{tikzpicture}


\caption{
Strong~(Red) and weak~(Blue) edges of the Karate club dataset maximizing the number of strong edges (a), $\lambda = 0$ (b), and $\lambda = 0.5$ (c) using our integer linear program based algorithm~(\algip). 
% Strong edge density is given by $0.74$ for (a). alg:greedy
We define our score as the sum of the number of strong and weak edges weighted by a parameter $\lambda$, divided by the size of the subgraph.
The scores are $2.0$ and $2.06$ for (b) and (c), respectively. We see that (b) is a clique of size $5$.}
\label{fig:karate}
\end{center}
\end{figure}
 

%\subsubsection{American College football}
%Table \ref{tab:football_PRF} and Figure \ref{fig:football} present the performance of our model on the community detection task on the American College football dataset.

%\begin{figure}[htbp]
\centering
\includegraphics[width=\linewidth]{Latex/images/football.png}
\caption{American College football}
\label{fig:football}
\end{figure}

\begin{figure}[htbp]
\centering
\includegraphics[width=\linewidth]{Latex/images/polbooks.png}
\caption{Books about US politics}
\label{fig:polbooks}
\end{figure} 

%\subsubsection{Books about US politics}
%Table \ref{tab:polbooks_PRF} and Figure \ref{fig:polbooks} present the performance of our model on the community detection task on the Books about US politics dataset.

We carried out experiments on 3 popular Community Detection datasets (cf. Section \ref{dataset} Para \ref{data:community}) and Table \ref{tab:ComDec_compare} compares the performance of our model on these three datasets
%on the Zachary’s karate club (with 2 communities), American College football, and Books about US politics datasets, 
against well-known methods on this task. Our model produces SOTA results on all three datasets on community detection.

\begin{comment}
\begin{table*}[htbp]
\centering
\begin{tabular}{||c|c|c|c|c|c|c|c|c||}
\hline
\diagbox{Node Acc.}{Method} & \makecell{GN\\ \cite{girvan2002community}} & \makecell{FG\\ \cite{clauset2004finding}} & \makecell{MIGA\\ \cite{shang2013community}} & \makecell{SLC\\ \cite{mahmood2015subspace}} & \makecell{Equation (20)\\ \cite{tian2018community}} & \makecell{$k$-means\\ \cite{cai2019community}} & \makecell{DDJKM\\ \cite{cai2019community}} & UGN \\ [1ex]
\hline\hline
Karate --2 Community & 0.97 & 0.97 & \textbf{1.00} & 0.97 & \textbf{1.00} & 0.88 & \textbf{1.00} & \textbf{1.00} \\ [0.5ex]
\hline
Football & 0.80 & 0.53 & 0.86 & 0.85 & 0.86 & 0.73 & 0.89 & \textbf{0.92} \\ [0.5ex]
\hline
US Politics Books & 0.81 & 0.73 & 0.80 & 0.80 & 0.83 & 0.66 & 0.73 & \textbf{0.88} \\ [0.5ex]
\hline
\end{tabular}
\caption{Performance Comparison on the Community Detection datasets}
\label{tab:ComDec_compare}
\end{table*}
\end{comment}

\begin{comment}
\begin{table}[htbp]
\centering
\begin{tabular}{c|c|c|c|}
\cline{2-4}
\multicolumn{1}{l|}{}        & \multicolumn{3}{c|}{Node Acc.}\\ \cline{1-4}
\multicolumn{1}{|c|}{Method} & \rotatebox{90}{\makecell{Karate\\Club}} & \rotatebox{90}{Football} & \rotatebox{90}{\makecell{US Politics\\Books}}\\
\hline
\multicolumn{1}{|c|}{GN \cite{girvan2002community}}  & 0.97 & 0.80 & 0.81\\
\multicolumn{1}{|c|}{FG \cite{clauset2004finding}} & 0.97 & 0.53  & 0.73\\
\multicolumn{1}{|c|}{MIGA \cite{shang2013community}} & \textbf{1.00} & 0.86 & 0.80\\
\multicolumn{1}{|c|}{SLC \cite{mahmood2015subspace}} & 0.97 & 0.85 & 0.80\\ 
\multicolumn{1}{|c|}{Equation (20) \cite{tian2018community}}  & \textbf{1.00} & 0.86 & 0.83\\
\multicolumn{1}{|c|}{$k$-means \cite{cai2019community}} & 0.88 & 0.73 & 0.66\\
\multicolumn{1}{|c|}{DDJKM \cite{cai2019community}} & \textbf{1.00} & 0.89  & 0.73\\
\multicolumn{1}{|c|}{UGN} & \textbf{1.00} &  \textbf{0.92}  & \textbf{0.88}\\
\hline
\end{tabular}
\caption{Community Detection}
\label{tab:ComDec_compare}
\end{table}
\end{comment}

\begin{comment}
\begin{table*}[htbp]
    \begin{subtable}{.3\linewidth}
      \centering
        \begin{tabular}{c|c|c|c|}
        \cline{2-4}
        \multicolumn{1}{l|}{}        & \multicolumn{3}{c|}{Node Acc.}\\ \cline{1-4}        
        \multicolumn{1}{|c|}{Method} & \rotatebox{90}{\makecell{Karate\\Club}} & \rotatebox{90}{Football} & \rotatebox{90}{\makecell{US Politics\\Books}}\\
        \hline
        \multicolumn{1}{|c|}{GN \cite{girvan2002community}}  & 0.97 & 0.80 & 0.81\\
        \multicolumn{1}{|c|}{FG \cite{clauset2004finding}} & 0.97 & 0.53  & 0.73\\
        \multicolumn{1}{|c|}{MIGA \cite{shang2013community}} & \textbf{1.00} & 0.86 & 0.80\\
        \multicolumn{1}{|c|}{SLC \cite{mahmood2015subspace}} & 0.97 & 0.85 & 0.80\\ 
        \multicolumn{1}{|c|}{Eqn (20) \cite{tian2018community}}  & \textbf{1.00} & 0.86 & 0.83\\
        \multicolumn{1}{|c|}{$k$-means \cite{cai2019community}} & 0.88 & 0.73 & 0.66\\
        \multicolumn{1}{|c|}{DDJKM \cite{cai2019community}} & \textbf{1.00} & 0.89  & 0.73\\
        \multicolumn{1}{|c|}{UGN} & \textbf{1.00} &  \textbf{0.92}  & \textbf{0.88}\\
        \hline
        \end{tabular}
        \caption{Community Detection}
        \label{tab:ComDec_compare}
    \end{subtable}
    \begin{subtable}{.4\linewidth}
      \centering
        \begin{tabular}{|c|ccccc|}
        \cline{2-6}
        \multicolumn{1}{l|}{}        & \multicolumn{5}{c|}{Pearson Correlation}\\ \hline
        Method & \rotatebox{90}{Res} & \rotatebox{90}{Emo} & \rotatebox{90}{Gam} & \rotatebox{90}{Lan} & \rotatebox{90}{Motor} \\
        \hline\hline
        \cite{galan2008network} & 0.23 & 0.14 & 0.14 & 0.14 & 0.15 \\
        \cite{abdelnour2014network} & 0.23 & 0.14 & 0.14 & 0.15 & 0.15 \\
        \cite{meier2016mapping} & 0.26 & 0.16 & 0.15 & 0.16 & 0.16 \\
        \cite{abdelnour2018functional} & 0.23 & 0.14 & 0.14 & 0.15 & 0.15 \\
        NEC-DGT \cite{guo2019deep} & 0.13 & 0.34 & 0.04 & 0.17 & 0.14 \\
        GT-GAN \cite{9737289} & 0.45 & 0.34 & 0.34 & 0.35 & 0.36 \\
        UGN & \textbf{0.61} & \textbf{0.67} & \textbf{0.68} & \textbf{0.66} & \textbf{0.66} \\
        \hline
        \end{tabular}
        \caption{HCP}
        \label{tab:HCP_compare}
    \end{subtable}
    \begin{subtable}{.27\linewidth}
      \centering
        \begin{tabular}{|c|c|c|c|}
        \cline{2-4}
        \multicolumn{1}{l|}{}        & \multicolumn{3}{c|}{AUPRC}\\ \hline
        Method & DDI & DTI & PPI \\
        \hline
        DeepWalk \cite{perozzi2014deepwalk} & 0.69 & 0.75 & 0.72 \\
        node2vec \cite{grover2016node2vec} & 0.79 & 0.77 & 0.77\\
        L3 \cite{kovacs2019network} & 0.85 & 0.89 & 0.90\\
        VGAE \cite{kipf2016variational} & 0.83 & 0.85 & 0.88\\
        GCN \cite{DBLP:conf/iclr/KipfW17} & 0.84 & 0.90 & 0.91\\
        SkipGNN \cite{huang2020skipgnn} & 0.85 & 0.93 & 0.92\\
        HOGCN \cite{kishan2021predicting} & \textbf{0.88} & \textbf{0.94} & 0.93 \\
        UGN & 0.85 & \textbf{0.94} & \textbf{0.96}\\
        \hline
        \end{tabular}
        \caption{DDI, DTI and PPI}
        \label{tab:DDI_DTI_compare}
    \end{subtable}
\caption{Performance comparison on (a) IoT, (b) HCP, and (c) DDI, DTI and PPI datasets}
\end{table*}
\end{comment}



% In Figure \ref{fig:karate}, \ref{fig:football}, and \ref{fig:polbooks}, spatial clustering shows the original communities and colouring of the nodes denote the communities predicted by our model.