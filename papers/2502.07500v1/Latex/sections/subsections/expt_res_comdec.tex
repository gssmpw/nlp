
Community detection is a node-level multi-class classification task. Since it is a semi-supervised task, the required loss function in the training process has an unsupervised loss component ($L_u$) as mentioned in Section ~\ref{unsupervised} (cf. Equation (\ref{eqn:unsup_loss})) along with the categorical cross-entropy loss ($L_s$, cf. Equation (\ref{eqn:sup_loss})) of the nodes (one from each community) whose label (community) is revealed during the training process. We carried out training for this task by using a learning rate of 0.005 employing a combination of the unsupervised and supervised losses as mentioned in Section \ref{sec:SpecialCases}.
% Let, $E$ be the edge set of the graph and $O$ is
% a
% an array-like of size $N\times C$, where $N$ is the number of nodes in the graph and $C$ is the number of communities. $O[i]$ denotes the $C$ dimensional transformer output-vector for $node_i$. $L_u$ is defined as in Equation \ref{loss_function}. 
% \begin{equation}
% \label{loss_function}   
% L_u=\sum_{(u,v) \in E}MSE(O[u],O[v])=\sum_{(u,v) \in E}\sum_{c=0}^{C-1}\frac{1}{C}(O[u][c]-O[v][c])^2 
% \end{equation}

%Fig. \ref{fig:karate2c}, \ref{fig:karate4c}, \ref{fig:football}, and \ref{fig:polbooks} visually present the performance of our model on the 3 datasets where spatial clustering represents the original communities and the color-coded nodes signify the communities predicted by our model.

% We used a learning rate of 0.005 for training on all the 3 Community Detection datasets.

%\subsubsection{Zachary's karate club}
%Additionally, Table \ref{tab:karate2_PRF} and \ref{tab:karate4_PRF} present the performance of our model on the community detection task on the Zachary's karate club dataset with 2 communities and 4 communities respectively. \sout{Figure \ref{fig:karate} pictorially shows the performance of our model on the 2 versions of the Zachary's karate club dataset.}

%\begin{figure}[htbp]
\centering
\includegraphics[width=\linewidth]{Latex/images/karate2.png}
\caption{Zachary's karate club with two communities}
\label{fig:karate2c}
\end{figure}

\begin{figure}[htbp]
\centering
\includegraphics[width=\linewidth]{Latex/images/karate4.png}
\caption{Zachary's karate club with four communities}
\label{fig:karate4c}
\end{figure} 

%\subsubsection{American College football}
%Table \ref{tab:football_PRF} and Figure \ref{fig:football} present the performance of our model on the community detection task on the American College football dataset.

%\begin{figure}[htbp]
\centering
\includegraphics[width=\linewidth]{Latex/images/football.png}
\caption{American College football}
\label{fig:football}
\end{figure}

\begin{figure}[htbp]
\centering
\includegraphics[width=\linewidth]{Latex/images/polbooks.png}
\caption{Books about US politics}
\label{fig:polbooks}
\end{figure} 

%\subsubsection{Books about US politics}
%Table \ref{tab:polbooks_PRF} and Figure \ref{fig:polbooks} present the performance of our model on the community detection task on the Books about US politics dataset.

We carried out experiments on 3 popular Community Detection datasets (cf. Section \ref{dataset} Para \ref{data:community}) and Table \ref{tab:ComDec_compare} compares the performance of our model on these three datasets
%on the Zachary’s karate club (with 2 communities), American College football, and Books about US politics datasets, 
against well-known methods on this task. Our model produces SOTA results on all three datasets on community detection.

\begin{comment}
\begin{table*}[htbp]
\centering
\begin{tabular}{||c|c|c|c|c|c|c|c|c||}
\hline
\diagbox{Node Acc.}{Method} & \makecell{GN\\ \cite{girvan2002community}} & \makecell{FG\\ \cite{clauset2004finding}} & \makecell{MIGA\\ \cite{shang2013community}} & \makecell{SLC\\ \cite{mahmood2015subspace}} & \makecell{Equation (20)\\ \cite{tian2018community}} & \makecell{$k$-means\\ \cite{cai2019community}} & \makecell{DDJKM\\ \cite{cai2019community}} & UGN \\ [1ex]
\hline\hline
Karate --2 Community & 0.97 & 0.97 & \textbf{1.00} & 0.97 & \textbf{1.00} & 0.88 & \textbf{1.00} & \textbf{1.00} \\ [0.5ex]
\hline
Football & 0.80 & 0.53 & 0.86 & 0.85 & 0.86 & 0.73 & 0.89 & \textbf{0.92} \\ [0.5ex]
\hline
US Politics Books & 0.81 & 0.73 & 0.80 & 0.80 & 0.83 & 0.66 & 0.73 & \textbf{0.88} \\ [0.5ex]
\hline
\end{tabular}
\caption{Performance Comparison on the Community Detection datasets}
\label{tab:ComDec_compare}
\end{table*}
\end{comment}

\begin{comment}
\begin{table}[htbp]
\centering
\begin{tabular}{c|c|c|c|}
\cline{2-4}
\multicolumn{1}{l|}{}        & \multicolumn{3}{c|}{Node Acc.}\\ \cline{1-4}
\multicolumn{1}{|c|}{Method} & \rotatebox{90}{\makecell{Karate\\Club}} & \rotatebox{90}{Football} & \rotatebox{90}{\makecell{US Politics\\Books}}\\
\hline
\multicolumn{1}{|c|}{GN \cite{girvan2002community}}  & 0.97 & 0.80 & 0.81\\
\multicolumn{1}{|c|}{FG \cite{clauset2004finding}} & 0.97 & 0.53  & 0.73\\
\multicolumn{1}{|c|}{MIGA \cite{shang2013community}} & \textbf{1.00} & 0.86 & 0.80\\
\multicolumn{1}{|c|}{SLC \cite{mahmood2015subspace}} & 0.97 & 0.85 & 0.80\\ 
\multicolumn{1}{|c|}{Equation (20) \cite{tian2018community}}  & \textbf{1.00} & 0.86 & 0.83\\
\multicolumn{1}{|c|}{$k$-means \cite{cai2019community}} & 0.88 & 0.73 & 0.66\\
\multicolumn{1}{|c|}{DDJKM \cite{cai2019community}} & \textbf{1.00} & 0.89  & 0.73\\
\multicolumn{1}{|c|}{UGN} & \textbf{1.00} &  \textbf{0.92}  & \textbf{0.88}\\
\hline
\end{tabular}
\caption{Community Detection}
\label{tab:ComDec_compare}
\end{table}
\end{comment}

\begin{comment}
\begin{table*}[htbp]
    \begin{subtable}{.3\linewidth}
      \centering
        \begin{tabular}{c|c|c|c|}
        \cline{2-4}
        \multicolumn{1}{l|}{}        & \multicolumn{3}{c|}{Node Acc.}\\ \cline{1-4}        
        \multicolumn{1}{|c|}{Method} & \rotatebox{90}{\makecell{Karate\\Club}} & \rotatebox{90}{Football} & \rotatebox{90}{\makecell{US Politics\\Books}}\\
        \hline
        \multicolumn{1}{|c|}{GN \cite{girvan2002community}}  & 0.97 & 0.80 & 0.81\\
        \multicolumn{1}{|c|}{FG \cite{clauset2004finding}} & 0.97 & 0.53  & 0.73\\
        \multicolumn{1}{|c|}{MIGA \cite{shang2013community}} & \textbf{1.00} & 0.86 & 0.80\\
        \multicolumn{1}{|c|}{SLC \cite{mahmood2015subspace}} & 0.97 & 0.85 & 0.80\\ 
        \multicolumn{1}{|c|}{Eqn (20) \cite{tian2018community}}  & \textbf{1.00} & 0.86 & 0.83\\
        \multicolumn{1}{|c|}{$k$-means \cite{cai2019community}} & 0.88 & 0.73 & 0.66\\
        \multicolumn{1}{|c|}{DDJKM \cite{cai2019community}} & \textbf{1.00} & 0.89  & 0.73\\
        \multicolumn{1}{|c|}{UGN} & \textbf{1.00} &  \textbf{0.92}  & \textbf{0.88}\\
        \hline
        \end{tabular}
        \caption{Community Detection}
        \label{tab:ComDec_compare}
    \end{subtable}
    \begin{subtable}{.4\linewidth}
      \centering
        \begin{tabular}{|c|ccccc|}
        \cline{2-6}
        \multicolumn{1}{l|}{}        & \multicolumn{5}{c|}{Pearson Correlation}\\ \hline
        Method & \rotatebox{90}{Res} & \rotatebox{90}{Emo} & \rotatebox{90}{Gam} & \rotatebox{90}{Lan} & \rotatebox{90}{Motor} \\
        \hline\hline
        \cite{galan2008network} & 0.23 & 0.14 & 0.14 & 0.14 & 0.15 \\
        \cite{abdelnour2014network} & 0.23 & 0.14 & 0.14 & 0.15 & 0.15 \\
        \cite{meier2016mapping} & 0.26 & 0.16 & 0.15 & 0.16 & 0.16 \\
        \cite{abdelnour2018functional} & 0.23 & 0.14 & 0.14 & 0.15 & 0.15 \\
        NEC-DGT \cite{guo2019deep} & 0.13 & 0.34 & 0.04 & 0.17 & 0.14 \\
        GT-GAN \cite{9737289} & 0.45 & 0.34 & 0.34 & 0.35 & 0.36 \\
        UGN & \textbf{0.61} & \textbf{0.67} & \textbf{0.68} & \textbf{0.66} & \textbf{0.66} \\
        \hline
        \end{tabular}
        \caption{HCP}
        \label{tab:HCP_compare}
    \end{subtable}
    \begin{subtable}{.27\linewidth}
      \centering
        \begin{tabular}{|c|c|c|c|}
        \cline{2-4}
        \multicolumn{1}{l|}{}        & \multicolumn{3}{c|}{AUPRC}\\ \hline
        Method & DDI & DTI & PPI \\
        \hline
        DeepWalk \cite{perozzi2014deepwalk} & 0.69 & 0.75 & 0.72 \\
        node2vec \cite{grover2016node2vec} & 0.79 & 0.77 & 0.77\\
        L3 \cite{kovacs2019network} & 0.85 & 0.89 & 0.90\\
        VGAE \cite{kipf2016variational} & 0.83 & 0.85 & 0.88\\
        GCN \cite{DBLP:conf/iclr/KipfW17} & 0.84 & 0.90 & 0.91\\
        SkipGNN \cite{huang2020skipgnn} & 0.85 & 0.93 & 0.92\\
        HOGCN \cite{kishan2021predicting} & \textbf{0.88} & \textbf{0.94} & 0.93 \\
        UGN & 0.85 & \textbf{0.94} & \textbf{0.96}\\
        \hline
        \end{tabular}
        \caption{DDI, DTI and PPI}
        \label{tab:DDI_DTI_compare}
    \end{subtable}
\caption{Performance comparison on (a) IoT, (b) HCP, and (c) DDI, DTI and PPI datasets}
\end{table*}
\end{comment}



% In Figure \ref{fig:karate}, \ref{fig:football}, and \ref{fig:polbooks}, spatial clustering shows the original communities and colouring of the nodes denote the communities predicted by our model.