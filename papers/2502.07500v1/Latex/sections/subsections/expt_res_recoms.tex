 
% The task of rating prediction on the MovieLens-100k dataset is presented as a regression problem because of the fact that the rating ``4" is more similar to rating ``5" than to rating ``1", but classification problems cannot consider this fact. Commonly used performance metric for model evaluation on this dataset is RMSE loss. 80,000 edges were used as training set with LR 0.005 and the rest (20,000 edges) were kept for testing. 

To showcase the generalization ability of our proposed framework, we also investigate its performance on the movie recommendation task on the MovieLens-100k dataset. It is to be noted that in literature  movie recommendation task is considered as a regression problem since the rating ``4" is more similar to rating ``5" than to rating ``1", but classification problems do not consider this fact. Commonly used performance metric for model evaluation on this
dataset is RMSE loss. MovieLens-100k dataset contains 100,000 edges (user ratings), out of which 80,000 edges were used as training set with learning rate 0.005
and the rest (20,000) were treated as test instances. Here, the initial node features were comprised of a concatenation of various one-hot-encoded user characteristics, including user occupation (21 dimensions), user gender (2 dimensions), and user age (1 dimension). Additionally, the initial node features included information on the movie genre (19 dimensions), movie release year (1 dimension), and ratings (1-5) received from users (943 dimensions) (with a default value of 0). Furthermore, the initial node features contained ratings (1-5) given to movies (1682 dimensions) (with a default value of 0). As a result, the initial node features were quite extensive, totaling 2669 dimensions. 

Table \ref{tab:MovieLens_compare} compares the performance of our model in terms of RMSE loss to other well known models and shows that our model produces comparable results in the movie recommendation task on the MovieLens-100k dataset.
\begin{table*}[htbp]
\centering
\begin{tabular}{||c|c|c|c|c|c|c|c|c||}
\hline
Method & \makecell{GHRS \\\cite{darban2022ghrs}} & \makecell{GLocal-K \\\cite{han2021glocal}} & \makecell{GraphRec \\\cite{rashed2019attribute}} & \makecell{IGMC \\\cite{zhanginductive}} & \makecell{GC-MC \\\cite{berg2017graph}} & \makecell{F-EAE \\\cite{hartford2018deep}} & \makecell{PinSage \\\cite{ying2018graph}} & Our Model \\
\hline\hline
RMSE & \textbf{0.89} & \textbf{0.89} & 0.90 & 0.91 & 0.91 & 0.92 & 0.95 & 1.01\\
\hline
\end{tabular}
\caption{Performance Comparison on the MovieLens-100k dataset}
\label{tab:MovieLens_compare}
\end{table*}