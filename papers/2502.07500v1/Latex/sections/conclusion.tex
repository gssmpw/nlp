In this paper, we presented a thorough investigation of the feasibility of applying encoder-decoder based frameworks for graph analysis tasks by providing valuable insights into its applications, challenges and potential solutions. Through an in-depth analysis, we can conclude concretely that our novel UGN framework performs relatively well by producing SOTA results on few datasets and produce comparable results on the rest of the datasets. In this work, we have developed a novel generalized framework and evaluated it on six different tasks by employing twelve datasets. We have introduced some task specific add-ons like supernode feature for graphs, where node feature is not available, unsupervised loss component for semi-supervised settings, and MTCM for complete graphs. Our extensive empirical analysis suggests that our proposed UGN produces SOTA results on ten datasets and comparable results on the remaining two datasets. Additionally, our work has contributed to the field of graph learning and analysis by providing a comprehensive overview of the existing state-of-the-art techniques, identifying challenges, and proposing novel solutions. The findings of this research serves as a valuable resource for researchers, practitioners, and stakeholders interested in leveraging the power of graph data for various applications, e.g., Supply Chain Optimization, Fraud Detection, Recommendation Systems, Cyber-security, etc. As the field continues to evolve, future research has the opportunity to extend the foundations established in this work, thereby advancing the comprehension, scalability, and applicability of graph learning and analysis techniques.

% \hl{Notably, our UGN introduces certain constraints for the graph translation task. Specifically, it requires the schema of the source and target graphs to be identical, and mandates a one-to-one or many-to-one mapping from source to target nodes.}


% the   shed light on the complex nature of graph data and the importance of effective analysis for extracting meaningful information. 

