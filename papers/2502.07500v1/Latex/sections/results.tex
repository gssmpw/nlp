%\subsection{Dataset}
%\label{ss:dataset}
\begin{table*}[ht]
    \footnotesize
    \centering
    \renewcommand{\arraystretch}{1.1} % Adjusts the row spacing
    \resizebox{16cm}{!} 
    { 
    \begin{tblr}{hline{1,2,Z} = 0.8pt, hline{3-Y} = 0.2pt,
                 colspec = {Q[l,m, 13em] Q[l,m, 6em] Q[c,m, 8em] Q[c,m, 5em] Q[l,m, 14em]},
                 colsep  = 4pt,
                 row{1}  = {0.4cm, font=\bfseries, bg=gray!30},
                 row{2-Z} = {0.2cm},
                 }
\textbf{Dataset}       & \textbf{Table Source} & \textbf{\# Tables / Statements} & \textbf{\# Words / Statement} & \textbf{Explicit Control}\\ 
\SetCell[c=5]{c} \textit{Single-sentence Table-to-Text}\\
ToTTo \cite{parikh2020tottocontrolledtabletotextgeneration}   & Wikipedia        & 83,141 / 83,141                  & 17.4                          & Table region      \\
LOGICNLG \cite{chen2020logicalnaturallanguagegeneration} & Wikipedia        & 7,392 / 36,960                  & 14.2                          & Table regions      \\ 
HiTab \cite{cheng-etal-2022-hitab}   & Statistics web   & 3,597 / 10,672                  & 16.4                          & Table regions \& reasoning operator \\ 
\SetCell[c=5]{c} \textit{Generic Table Summarization}\\
ROTOWIRE \cite{wiseman2017challengesdatatodocumentgeneration} & NBA games      & 4,953 / 4,953                   & 337.1                         & \textbf{\textit{X}}                   \\
SciGen \cite{moosavi2021scigen} & Sci-Paper      & 1,338 / 1,338                   & 116.0                         & \textbf{\textit{X}}                   \\
NumericNLG \cite{suadaa-etal-2021-towards} & Sci-Paper   & 1,355 / 1,355                   & 94.2                          & \textbf{\textit{X}}                    \\
\SetCell[c=5]{c} \textit{Table Question Answering}\\
FeTaQA \cite{nan2021fetaqafreeformtablequestion}     & Wikipedia      & 10,330 / 10,330                 & 18.9                          & Queries rewritten from ToTTo \\
\SetCell[c=5]{c} \textit{Query-Focused Table Summarization}\\
QTSumm \cite{zhao2023qtsummqueryfocusedsummarizationtabular}                        & Wikipedia      & 2,934 / 7,111                   & 68.0                          & Queries from real-world scenarios\\ 
\textbf{eC-Tab2Text} (\textit{ours})                           & e-Commerce products      & 1,452 / 3,354                   & 56.61                          & Queries from e-commerce products\\
    \end{tblr}
    }
\caption{Comparison between \textbf{eC-Tab2Text} (\textit{ours}) and existing table-to-text generation datasets. Statements and queries are used interchangeably. Our dataset specifically comprises tables from the e-commerce domain.}
\label{tab:datasets}
\end{table*}


\begin{comment}
\subsection{Research Questions}
\label{ss:rqs}
As discussed in Section~\ref{ss:intro}, the rapid growth of graph data in different domains such as social networks, chemistry, biomedical, IoT, etc. has necessitated a generalized graph analysis framework that can work well with data belonging to different domains. In this context, we raise our first research question.

\begin{itemize}
    \item \madhu{\textbf{RQ-1}: How effectively can our proposed novel encoder-decoder based framework generalize its performance across diverse domains of graph datasets?}
\end{itemize}

\madhu{As mentioned in paragraph~\ref{super}, when dealing with large graphs in graph analysis tasks, a typical issue arises.  The utilization of one hot encoding as initial node attributes becomes unfeasible due to restricted resources, such as insufficient RAM capacity, that are required to process the entire computational graph. In context to that we investigate our second research question,}

\begin{itemize}
    \item \textbf{RQ-2}:\madhu{How efficiently can we integrate our novel supernode initialization strategy into our proposed framework to conduct experiments in a low-resource scenario, thereby achieving SOTA performance?}
\end{itemize}

\madhu{It is an important thing to discuss that how well a generalized framework, performs in constraint scenario . In context to that we investigate our third research question,}

\begin{itemize}
    \item \textbf{RQ-3}: \madhu{How well does our proposed framework perform in zero-shot scenario or in cross-domain scenario?}
\end{itemize}
\end{comment}