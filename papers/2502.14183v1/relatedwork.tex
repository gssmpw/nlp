\section{Related Work}
Advancements in forecasting future glucose levels have been crucial for managing patients with T1D, enabling them to proactively respond to glucose fluctuations and significantly improving glucose control. Marigliano et al. \cite{marigliano2024glucose} demonstrated that integrating predictive alarms with CGM technology reduced hypoglycemic events in adolescents by 40\% and severe hypoglycemia by 60\%, highlighting the tangible benefits of predictive alerts in real-world settings. Vettoretti et al. \cite{vettoretti2020advanced} further supported these findings by showcasing how artificial intelligence-based diabetes management systems can enhance patient outcomes through early glucose predictions, prompting timely interventions such as insulin dose adjustments or dietary changes to maintain stable glucose levels and mitigate risks like cardiovascular disease and nerve damage. Additionally, Shroff et al. \cite{shroff2023glucoseassist} emphasized the shift from reactive to proactive diabetes care with personalized prediction systems that learn individual response patterns, offering tailored alerts to meet each patient's unique physiological needs. These predictive technologies not only set a new standard in diabetes care by focusing on prevention over treatment but also significantly reduce the daily management burden. Arefeen et al. \cite{arefeen2022forewarning} suggest that machine learning algorithms can effectively predict hyperglycemia events using data from controlled feeding trials. In the following sections, we categorize these algorithms based on their architecture and discuss their approaches to predicting abnormal glucose levels, providing early warnings for timely interventions.

\subsection{Evidential Deep Learning and Meta-Learning}
Machine learning techniques, particularly deep learning algorithms, have achieved reliable glucose-level prediction performance with minimal feature engineering required. Zhu et al. \cite{zhu2022personalized} demonstrated the use of evidential deep learning combined with meta-learning to create a model that adapts to individual patient data. This approach significantly enhances prediction accuracy by considering the uncertainty in predictions and personalizing the model to each patient's unique glucose response patterns. This method's strength lies in its ability to provide precise predictions with fewer input features, simplifying the data collection process for patients.

\subsection{Convolutional Recurrent Neural Networks}
Another innovative approach is the use of Convolutional Recurrent Neural Networks (CRNN) \cite{keren2016convolutional} to estimate glucose levels for up to a 60-minute prediction horizon (PH) based on prior CGM data and information on meal and insulin intakes. Li et al. \cite{li2019convolutional} introduced this model, which combines the feature extraction capabilities of convolutional neural networks (CNN) \cite{li2021survey} with the temporal learning capabilities of recurrent neural networks (RNN) \cite{medsker2001recurrent}. The CRNN model demonstrated superior performance in both simulated and real patient data, providing accurate short-term glucose predictions that are essential for proactive diabetes management.

\subsection{Long Short-Term Memory Networks}
Long Short-Term Memory (LSTM) \cite{hochreiter1997long} units have also been employed by Aliberti et al. \cite{aliberti2019multi} to predict glucose levels. In this study, they developed a predictive model for blood glucose levels using a multi-patient dataset, focusing on leveraging the strengths of LSTM networks. The researchers compared the performance of LSTM networks with other models like Non-Linear Autoregressive (NAR) neural networks \cite{billings2013nonlinear} and found that the LSTM model significantly outperformed others in both short- and long-term predictions. The LSTM model demonstrated superior accuracy due to its ability to handle long-term dependencies and mitigate issues like the vanishing gradient problem that commonly affects traditional RNNs. This study's findings underscore the potential of LSTM networks in enhancing predictive accuracy and clinical outcomes for diabetes management.

\subsection{Bi-Directional LSTM Variants}
Further extending the capabilities of LSTM networks, researchers have explored bi-directional LSTM variants for glucose prediction \cite{sun2018predicting}. Butt et al. \cite{butt2023feature} investigated how feature transformation techniques could enhance the efficiency of blood glucose prediction models. By employing bi-directional LSTM units, the model can consider both past and future data points, providing a more comprehensive understanding of glucose trends and improving prediction accuracy.

\subsection{Linear Regression}
The Tandem t:slim X2 with Control-IQ technology employs simple linear regression to forecast blood glucose levels 30 minutes ahead based on previous CGM data, highlighting the importance of understanding these algorithms and their accuracy. To address the challenges of applying linear regression to time-series data and multi-step predictions, Zhang et al. \cite{zhang2021deep} developed a multiple linear regression (MLR) model that predicts each future time step separately. This MLR approach combines \( k \) individual linear regression models, denoted as \( L_i \), each trained to relate the training data \( X_{\text{train}} \) to the CGM values at future time points \( y_{\text{train}}(t + i) \) for \( i = 1 \) to \( k \). For instance, setting \( k = 6 \) or \( 12 \) corresponds to predicting 30 or 60 minutes into the future, respectively, as illustrated in Fig. \ref{fig:multiple_linear_regression}. During the prediction phase, the trained models utilize their respective coefficients and intercepts to forecast glucose levels for the next \( k \) time steps by applying the models to the testing data one time step prior to the first test point. This iterative process is repeated for each row of the testing matrix, enabling the generation of multi-horizon predictions. Despite the inherent difficulties of using linear regression for time-series forecasting, the structured MLR approach enhances the accuracy of glucose level predictions at various future points, thereby improving the reliability of automated insulin delivery systems. Additionally, this method allows for scalability and adaptability in different clinical settings, making it a valuable tool in the ongoing efforts to optimize diabetes management.

\begin{figure}[tb]
\centerline{\includegraphics[scale=0.25]{figures/multiple_linear_regression.png}}
\caption{Block diagram of the MLR model using multi PH \cite{zhang2021deep}.}
\label{fig:multiple_linear_regression}
\end{figure}