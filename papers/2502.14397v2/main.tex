% CVPR 2024 Paper Template; see https://github.com/cvpr-org/author-kit

\documentclass[10pt,twocolumn,letterpaper]{article}
\usepackage{cuted}
%%%%%%%%% PAPER TYPE  - PLEASE UPDATE FOR FINAL VERSION
\usepackage{cvpr}              % To produce the CAMERA-READY version
% \usepackage[review]{cvpr}      % To produce the REVIEW version
% \usepackage[pagenumbers]{cvpr} % To force page numbers, e.g. for an arXiv version
\usepackage{CJKutf8}
\usepackage{float}

% Import additional packages in the preamble file, before hyperref
\newcommand{\CG}{\mathcal{G}\xspace}
\newcommand{\CV}{\mathcal{V}\xspace}
\newcommand{\CE}{\mathcal{E}\xspace}
\newcommand{\CA}{\mathcal{A}\xspace}
\newcommand{\CF}{\mathcal{F}\xspace}
\newcommand{\CR}{\mathcal{R}\xspace}
\newcommand{\CB}{\mathcal{B}\xspace}
\newcommand{\CX}{\mathcal{X}\xspace}
\newcommand{\CK}{\mathcal{K}\xspace}
\newcommand{\CM}{\mathcal{M}\xspace}
\newcommand{\CC}{\mathcal{C}\xspace}
\newcommand{\CL}{\mathcal{L}\xspace}
\newcommand{\CI}{\mathcal{I}\xspace}
\newcommand{\CQ}{\mathcal{Q}\xspace}
\newcommand{\CO}{\mathcal{O}\xspace}
\newcommand{\CP}{\mathcal{P}\xspace}
\newcommand{\CS}{\mathcal{S}\xspace}
\newcommand{\CT}{\mathcal{T}\xspace}
\newcommand{\CJ}{\mathcal{J}\xspace}
\usepackage[para]{footmisc}
\usepackage{subfig}
% \usepackage{subcaption}
% \usepackage{array}
% \usepackage{colortbl}



% It is strongly recommended to use hyperref, especially for the review version.
% hyperref with option pagebackref eases the reviewers' job.
% Please disable hyperref *only* if you encounter grave issues, 
% e.g. with the file validation for the camera-ready version.
%
% If you comment hyperref and then uncomment it, you should delete *.aux before re-running LaTeX.
% (Or just hit 'q' on the first LaTeX run, let it finish, and you should be clear).
\definecolor{cvprblue}{rgb}{0.21,0.49,0.74}
\usepackage[pagebackref,breaklinks,colorlinks,citecolor=cvprblue]{hyperref}
\usepackage{amssymb}
\usepackage{amsmath}
\usepackage{booktabs}
\usepackage{graphicx}
\usepackage{tabularx} 
\usepackage{multirow}
\usepackage{xcolor}

%%%%%%%%% PAPER ID  - PLEASE UPDATE
\def\paperID{1898} % *** Enter the Paper ID here
\def\confName{CVPR}
\def\confYear{2025}

%%%%%%%%% TITLE - PLEASE UPDATE
\title{PhotoDoodle: Learning Artistic Image Editing from Few-Shot Examples}

%%%%%%%%% AUTHORS - PLEASE UPDATE
% \author{First Author\\
% Institution1\\
% Institution1 address\\
% {\tt\small firstauthor@i1.org}
% % For a paper whose authors are all at the same institution,
% % omit the following lines up until the closing }''.
% % Additional authors and addresses can be added with \and'',
% % just like the second author.
% % To save space, use either the email address or home page, not both
% \and
% Second Author\\
% Institution2\\
% First line of institution2 address\\
% {\tt\small secondauthor@i2.org}
% }



% \author{
% Shijie Huang\thanks{Equal contribution.} \quad Yiren Song\footnotemark[1] \quad Yuxuan Zhang\footnotemark[2] \quad Hailong Guo\footnotemark[3] \quad Xueyin Wang\footnotemark[4]  \quad Mike Zheng Shou\thanks{Corresponding author.} \quad Jiaming Liu\footnotemark[5] \\
% National University of Singapore \\
% Shanghai Jiao Tong University \\
% Beijing University of Posts and Telecommunications \\
% Byte Dance \\
% Tiamat \\
% }

\author{
Shijie Huang\textsuperscript{1,5}\thanks{Equal contribution.} \quad Yiren Song\textsuperscript{1,5}\footnotemark[1] \quad Yuxuan Zhang\textsuperscript{2,5} \quad Hailong Guo\textsuperscript{3,5} \\
Xueyin Wang\textsuperscript{4} \quad Mike Zheng Shou\textsuperscript{1}\thanks{Corresponding author.} \quad Jiaming Liu\textsuperscript{5}\thanks{Project leader.} \\
\textsuperscript{1}National University of Singapore \quad \textsuperscript{2}Shanghai Jiao Tong University \\
\textsuperscript{3}Beijing University of Posts and Telecommunications \quad \textsuperscript{4}Byte Dance \\ \textsuperscript{5}Tiamat \\
% \footnotemark{*}Equal contribution. \quad \footnotemark{\dag}Corresponding author. \quad \footnotemark{\ddag}Project leader.
}

% \author{
% Yiren Song \quad XiaoKang Liu \quad Mike Zheng Shou\textsuperscript{\textdagger} \\
% Show Lab, National University of Singapore \\
% \texttt{yiren@nus.edu.sg, xliu@u.nus.edu, mike.zheng.shou@gmail.com}
% }
% \thanks{\textsuperscript{\textdagger}Corresponding author.}


% \renewcommand{\thefootnote}{\textdagger} % 定义脚注为十字号
% \footnotetext{Corresponding author.}

\begin{document}
\maketitle

\begin{strip}
\centering
    \vspace{-1.5cm}
    \includegraphics[width=\linewidth]{image/teaser.pdf}
    \vspace{-7pt}
    \captionof{figure}{PhotoDoodle can mimic the styles and techniques of human artists in creating photo doodles, adding decorative elements to photos while maintaining perfect consistency between the pre- and post-edit states.}
    \label{fig:teaser}
\end{strip}





\begin{abstract}

% Recent works to jointly reconstruct 3D human and object from a single RGB image, are mostly model-based, that fail to capture the fine details of the clothed human body and object surface. In this paper, we introduce ReCHOR, a novel, model-free, first-method to produce realistic clothed human-object reconstructions from a monocular view. This is extremely challenging due to human-object occlusions, diverse interactions and depth ambiguity, as it needs to infer both 3D spatial awareness and high resolution details. Our core idea is based on estimating neural implicit representations for human and object respectively by an attention-based neural implicit model that attends to pixel-aligned features from both the global human-object image for spatial awareness and  the local separate view of human and object images for high quality details. Additionally, the network is conditioned on semantic features from an initial estimated human-object pose prior and a generative diffusion model that inpaints occluded regions, thus enabling the retrieval of details from them.
% We also propose a synthetic dataset with rendered scenes of diverse, inter-occluded 3D human and object scans, to train our network. We evaluate our method on the synthetic and real world BEHAVE dataset. Our experiments show that our method outperforms the SOTA in achieving realistic clothed human-object reconstructions.
Recent approaches to jointly reconstruct 3D humans and objects from a single RGB image represent 3D shapes with template-based or coarse models, which fail to capture details of loose clothing on human bodies. In this paper, we introduce a novel implicit approach for jointly reconstructing realistic 3D clothed humans and objects from a monocular view. For the first time, we model both the human and the object with an implicit representation, allowing to capture more realistic details such as clothing. This task is extremely challenging due to human-object occlusions and the lack of 3D information in 2D images, often leading to poor detail reconstruction and depth ambiguity. To address these problems, we propose a novel attention-based neural implicit model that leverages image pixel alignment from both the input human-object image for a global understanding of the human-object scene and from local separate views of the human and object images to improve realism with, for example, clothing details. Additionally, the network is conditioned on semantic features derived from an estimated human-object pose prior, which provides 3D spatial information about the shared space of humans and objects. To handle human occlusion caused by objects, we use a generative diffusion model that inpaints the occluded regions, recovering otherwise lost details. For training and evaluation, we introduce a synthetic dataset featuring rendered scenes of inter-occluded 3D human scans and diverse objects. Extensive evaluation on both synthetic and real-world datasets demonstrates the superior quality of the proposed human-object reconstructions over competitive methods.
\end{abstract}    
\section{Introduction}
\label{sec:intro}
% Image editing methods in diffusion models depend on user-defined control directions - users can unlock their creativity using these methods by specifying the desired manipulation through prompts~\cite{gandikota2023concept}, reference images~\cite{ruiz2022dreambooth, kumari2022customdiffusion, gal2022image, chen2024trainingfreeregionalpromptingdiffusion}, or attribute vectors~\cite{parmar2023zero,hertz2022prompt}. In this work, we ask a fundamentally different question: \emph{Can we automatically discover the underlying visual structure of a concept within diffusion model's knowledge?} %Rather than requiring user-specified controls, we aim to decompose the model's internal knowledge into meaningful directions.

% This question touches on a fundamental limitation in how we interact with diffusion models. Current control methods ~\cite{zhang2023addingconditionalcontroltexttoimage, gandikota2023concept, ye2023ipadaptertextcompatibleimage,ye2023ipadaptertextcompatibleimage, hertz2024stylealignedimagegeneration, li2023photomaker, shi2024instantbooth, chen2024trainingfreeregionalpromptingdiffusion} require users to specify their desired manipulations in advance, limiting interactive creativity. This contrasts with natural human artistic workflows, where creators dynamically explore creative ideas while jointly refining them toward meaningful artistic outcomes~\cite{hoffmann2016modeling}. This synergy between specification and exploration is not new to generative models. Early GAN architectures naturally developed disentangled latent spaces that enabled continuous\cite{harkonen2020ganspace,radford2015unsupervised, wu2021stylespace, shen2020interfacegan}, compositional control over generated images. Users could explore these spaces to discover interesting variations that would be difficult to describe in words~\cite{wu2021stylespace}, then combine them to achieve their creative goals~\cite{grabe2022towards}. 


% While diffusion models have largely superseded GANs in conditional image synthesis~\cite{dhariwal2021diffusion},  their underlying structure remains less understood. Diffusion models achieve remarkable diversity through high-dimensional latents, unlike GANs' compact latent spaces.  With a single prompt, diffusion models can generate radically different variations through different random initializations of input noise. We ask - Is it possible to discover interpretable structure within this vast space of variations?

Text-to-image diffusion models are capable of generating remarkable visual variations from a single prompt through different random initializations. However, this vast creative potential remains largely opaque to users---while we can generate diverse images, we lack understanding of the underlying structure of these variations. This presents a fundamental challenge: how can we discover and expose the latent visual capabilities encoded within these models?

\let\thefootnote\relax \footnote{$^{*}$Correspondence to \texttt{gandikota.ro@northeastern.edu}}

The challenge touches on a key limitation in how we interact with diffusion models today. Current control methods require users to explicitly specify their desired edits in advance through prompts~\cite{gandikota2023concept}, reference images~\cite{zhang2023addingconditionalcontroltexttoimage, chen2024trainingfreeregionalpromptingdiffusion, ruiz2022dreambooth,kumari2022customdiffusion, Ryu_lora, hu2021lora}, or attribute vectors~\cite{ye2023ipadaptertextcompatibleimage, hertz2024stylealignedimagegeneration, li2023photomaker, shi2024instantbooth,parmar2023zero,hertz2022prompt}. That contrasts sharply with natural human creative workflows, where artists dynamically explore creative ideas and jointly refine them toward meaningful artistic outcomes~\cite{hoffmann2016modeling}. The need for pre-specified controls creates a barrier between users and the full creative potential of these models.

Interestingly, earlier generative models like GANs~\cite{gans,karras2019style,brock2018large} naturally developed more interpretable internal structures. Their compact latent spaces often exhibited emergent disentanglement~\cite{harkonen2020ganspace,radford2015unsupervised, wu2021stylespace, shen2020interfacegan}, enabling continuous and compositional control over generated images. Users could explore these spaces to discover interesting variations that would be difficult to describe in words~\cite{wu2021stylespace}, then combine them to achieve their creative goals~\cite{grabe2022towards}.

Diffusion models have largely superseded GANs in conditional image synthesis~\cite{dhariwal2021diffusion}, achieving greater diversity through much higher-dimensional latents. And yet an understanding of the underlying structure of these larger latent spaces has remained elusive. In this work, we ask a fundamental question: \emph{Can we automatically discover the visual structure within a diffusion model's knowledge of a concept?} Rather than requiring user-specified controls, we aim to decompose the model's internal representations into expressive directions that users can explore and combine.

To address these needs, we present \textbf{SliderSpace}, a framework that brings systematic explorability to diffusion models. Given just a text prompt, SliderSpace discovers a canonical set of meaningful, diverse, and controllable directions within the model's knowledge of that concept. Each direction is implemented as a low-rank adapter~\cite{hu2021lora} that can be scaled and composed with others, allowing users to explore and smoothly combine different aspects of variation, as shown in Figure~\ref{fig:intro}.

We ground SliderSpace discovery in three key requirements for meaningful decomposition of a diffusion model's visual manifold: 
\begin{enumerate}
    \item \textbf{Unsupervised Discovery:} The decomposition process should emerge from the intrinsic structure of the model's learned representation, rather than being guided by predefined attributes. This ensures we capture the true topology of the model's knowledge space rather than projecting our assumptions onto it.
    
    \item \textbf{Semantic Orthogonality:} Each discovered control must represent a distinct semantic direction. This is enforced in a semantic feature space, like CLIP, where every slider has an orthogonal effect in embeddings. This prevents discovering multiple controls that create similar semantic effects, making the system more efficient and easier.
    
    \item \textbf{Distribution Consistency:} Directions must induce consistent transformations across both random seeds and prompt variations. 
\end{enumerate}

These requirements naturally lead to our proposed framework, which we formalize in Section~\ref{sec:method}. As we show in our experiments, SliderSpace is architecture-agnostic, working with both conventional U-Net based models like Stable Diffusion~\cite{rombach2022high, rombach2022sd20, podell2023sdxl, turbo, dmd} and recent transformer-based architectures like Flux~\cite{flux}.

We demonstrate the expressiveness of SliderSpace through three applications: First, we show how SliderSpace can decompose high-level concepts into diverse and expressive components, revealing the natural axes of variation in the model's understanding. Second, we explore artistic style variation, where SliderSpace discovers directions that match or exceed the diversity of manually curated artist lists while being judged more useful by human evaluators. Finally, we show how SliderSpace can help reverse the mode collapse commonly observed in distilled diffusion models, restoring diversity while maintaining generation speed.

Beyond providing practical creative control, SliderSpace opens new avenues for understanding and utilizing the latent capabilities of diffusion models. By mapping these models' visual potential into intuitive, composable directions, we take a step toward making their creative possibilities more accessible and interpretable to users.

% Image editing methods in diffusion models unlock the creativity of users. In this work we ask an alternate question: \emph{Can we organize and expose what of the diffusion model is already capable of?}.
% Existing methods for controlling image generation typically require users to manually specify edit directions for desired changes. This process is time-consuming, requires technical expertise, and limits the spontaneity of the creative process. For instance, if a user wants to adjust the smile of a generated person, they must explicitly request this edit, often through imprecise prompt engineering or model fine-tuning. This approach of predefined controls or manual specifications restricts users from fully exploring the latent capabilities of the model. There may be interesting stylistic variations or attributes that the model can generate, but users have no easy way to discover or utilize these.

% Natural visual disentanglement was an emergent property in the latent space of Generative Adversarial Models (GANs) \cite{harkonen2020ganspace,radford2015unsupervised, wu2021stylespace, shen2020interfacegan}. In particular, it has been observed that StyleGAN~\cite{karras2019style} stylespace neurons offer detailed control over many meaningful aspects of images that would be difficult to describe in words~\cite{wu2021stylespace}. However, diffusion models do not share such a compact latent space~\cite{park2023unsupervised}; and efforts to uncover such a space in the semantic embeddings of the text conditioning have met with limited success \nik{Nick - is there a specific citation you were thinking about?}.

% In this work we introduce \textbf{SliderSpace}, which takes a step towards uncovering an analogous low dimensional representation of diffusion models' visual breadth; in essence treating the diffusion model as many generators sharing parameters, where a particular generator is defined by a specific prompt. For a given prompt we sample many random seeds (and optionally prompt expansions using an LLM), generate the corresponding images, and apply an off the shelf feature extractor (in this work CLIP, but our method can be applied to any differentiable feature extractor). We use PCA to analyze these features, and for each of the leading $k$ principal components we train a LoRA \cite{} which causes the diffusion model to produces images which increase the feature magnitude along that component when passed back through the same feature extractor. This leads to a 'Slider' for each principal component, because each LoRA can be scaled and applied to the original diffusion model, continuously varying those visual features in the generated results (as measured, in our case, by CLIP).

% There are many other works that enhance the controllability of diffusion models. One common approach is enabling users to add spatial constraints to a generation either manually, or via a reference image \cite{zhang2023addingconditionalcontroltexttoimage, chen2024trainingfreeregionalpromptingdiffusion}, a second is leveraging more abstract embeddings (e.g. identity, style) extracted from a reference image \cite{ye2023ipadaptertextcompatibleimage, hertz2024stylealignedimagegeneration, li2023photomaker, shi2024instantbooth}, a third is finetuning a foundation model to better generate a concept important to the user \cite{ruiz2022dreambooth, kumari2022customdiffusion, Ryu_lora, hu2021lora}, and a fourth (most relevant to this work) is finding low-rank adaptors of the model based on a prompt or small training set which can be scaled to provide continous control over one aspect of generated image (e.g. night vs day, basic vs luxury, etc.) \cite{gandikota2023concept}. SliderSpace is complementary to all of these methods and offers something distinct. All of the other methods we are aware require the user (and / or model designer) to know in advance what type of control they want. In contrast SliderSpace assists users in discovering and controlling hidden capabilities present in the diffusion model's distribution of possible generations.

%We propose that truly intuitive creative control in a text-to-image model should meet three key criteria: \emph{discoverability}, \emph{intuitiveness}, and \emph{specificity}. The model should reveal controllable attributes that may not be immediately obvious, offer controls that are easy to understand and manipulate, and ensure each control affects a distinct attribute of the generated image.

% We demonstrate the utility and power of SliderSpace using three applications built on top of SDXL-DMD \cite{dmd}, because its fast generation speed lends itself well to the continuous control offered by SliderSpace.

% First, we study concept decomposition (Section \ref{sec:concept_exp}), where we learn sliders for a specific concept (e.g. 'monster', 'waterfall', 'car'). Through quantitative metrics of diversity and text alignment we demonstrate that the learned sliders dramatically boost the diversity of generations when randomly applied without harming text alignment; we also ask humans to qualitatively judge these results in a user study where they find the SliderSpace results to be more 'Diverse', 'Useful', and 'Creative' than our baselines.

% Second, we attempt to compare the automatic discoveries of SliderSpace to a large scale manual study of artistic styles (Section \ref{sec:art_exp}), open-sourced by ParrotZone \cite{parrotzone}. In this study SDXL was prompted with over 4300 artist names,  and based on visual inspection the cases of successful stylistic mimicry recorded. Quantitatively SliderSpace more closely matches the distribution of artistic variation discovered by ParrotZone than other baselines, and in our user studies was judged to be significantly more 'Diverse' and 'Useful' than the baselines. To our surprise humans even judged SliderSpace results to be slightly more 'Diverse' than the results generated by the manually discovered artist names of \cite{parrotzone}.

% Third, we attempt to use SliderSpace to reverse the mode collapse commonly observed in distilled few-step diffusion models relative to the original teacher model (Section \ref{sec:diverse_exp}). We quantitatively demonstrate that applying SliderSpace to SDXL-DMD leads to more closely matching the distribution of images by the original teacher, SDXL.

%Through extensive experiments on various state-of-the-art text-to-image models, we demonstrate that SliderSpace significantly enhances user control and creative expression in AI-assisted image generation tasks. Our method enables a range of applications, including concept decomposition and control, diversity improvement in generated images, customization dissection and edits, and the exploration of artistic styles inherent in the model.

% SliderSpace goes beyond providing a practical tool for enhanced creative control. By mapping the visual potential of diffusion models it can open new avenues for generative creativity and deepens our understanding of each model's hidden potential.
% !TEX root = ../main.tex

\section{Scene Graph Construction for Videos}
\label{sec:scene}
\begin{figure*}[t]
    \centering
    \includegraphics[width=\textwidth]{figures/overview.pdf}
    \vspace{-4mm}
    \caption{An overview of our zero-shot video caption generation pipeline. The pipeline consists of (a) frame-level caption generation using image VLMs, (b) textual scene graph parsing for each frame caption, (c) merging of scene graphs into a unified graph, and (d) video-level caption generation through our graph-to-text model. Our proposed framework leverages frame-level scene graphs to produce detailed and coherent video captions.
    }
    \label{fig:framework}
\end{figure*}

Our objective is to effectively extend the capabilities of image-based vision-language models (VLMs) to the video domain without relying on video-text training. 
To this end, we introduce a novel video captioning framework that combines image VLMs with scene graph structures, as shown in Figure~\ref{fig:framework}.
The proposed method consists of four key steps: 1) generating captions for each frame using an image VLM, 2) converting these captions into scene graphs, 3) consolidating the scene graphs from all frames into a unified graph, and 4) generating comprehensive descriptions from this unified graph. 
This algorithm enables the generation of coherent and detailed video captions, bridging the gap between image and video understanding.

\subsection{Generating image-level captions}
\label{sub:generating}
We obtain image-level captions from a set of sparsely sampled frames using the open-source image VLM, LLAVA-NEXT-7B~\cite{liu2024llavanext}.
This model is selected for its strong performance across multiple benchmarks.
Our approach, however, is flexible and can incorporate any image-based VLM, including proprietary, closed-source models, as long as APIs are accessible.
The model is prompted to generate sentences optimized for scene graph construction, which are subsequently parsed into scene graphs.

\subsection{Parsing captions into scene graphs}
A scene graph $G = (\mathcal{O}, \mathcal{E})$ is defined by a set of objects, $\mathcal{O} = \{o_1, o_2, \ldots \}$, and a set of edges, $\mathcal{E}$.
Each object $o_i = (c_i, \mathcal{A}_i)$ consists of an object class $c_i \in \mathcal{C}$ and a set of attributes $\mathcal{A}_i \subseteq A$, where $\mathcal{C}$ is a set of object classes and $\mathcal{A}$ is a set of all possible attributes.
A directed edge, $e_{i,j} \equiv (o_i, o_j) \in \mathcal{E}$, has a label $r \in \mathcal{R}$, specifying the relationship from one object to the other.
All the values of object classes, attributes, and relationship labels, are text strings.

We convert the generated caption from each frame into a scene graph, providing more structured understanding of individual frames. 
By expressing the visual content in each frame using a graph based on detected objects and their relationships, we can apply a graph merging technique to produce a holistic representation of the entire input video.
We parse a caption into a scene graph using a textual scene graph parser, specifically the FACTUAL-MR parser~\cite{li-etal-2023-factual} in our implementation.

\subsection{Scene graph consolidation}
\label{sub:scene}
% !TEX root = ../main.tex

\begin{algorithm}[t]
\caption{Scene graph merging}
\label{alg:hierarchical_graph_merge}
\begin{algorithmic}[1]

  \STATE \textbf{Input:} 
  \STATE \quad $\mathcal{Q} = [ G_1, G_2, \dots, G_n ]$: a priority queue with frame-level scene graphs
  \STATE \quad $\phi(\cdot)$: a graph encoder
  \STATE \quad $\psi_i(\cdot)$: a function returning the $i^\text{th}$ object in a graph
  \STATE \quad $\pi$: a permutation function
  \STATE \quad $\tau$: a threshold

  \STATE \textbf{Output:} $G_{\text{video}}$: a video-level scene graph

  \WHILE{$|\mathcal{Q}| > 1$}
    \STATE $G^s = (\mathcal{O}^s, \mathcal{E}^s) \gets \text{dequeue}(\mathcal{Q})$
    \STATE $G^t = (\mathcal{O}^t, \mathcal{E}^t) \gets \text{dequeue}(\mathcal{Q})$
    \STATE $G^m = (\mathcal{O}^m, \mathcal{E}^m) \gets (\mathcal{O}^s \cup \mathcal{O}^t, \mathcal{E}^s \cup \mathcal{E}^t)$

    \STATE $\pi^* \gets \displaystyle \arg\max_{\pi \in \Pi} \sum_{i} 
      \frac{\psi_i(\phi(G^s))}{\lVert \psi_i(\phi(G^s)) \rVert} \; \cdot \;
      \frac{\psi_i(\phi(G_{\pi}^t))}{\lVert \psi_i(\phi(G_{\pi}^t)) \rVert}$

    \FOR{$(p, q) \in \mathcal{M}$ such that $s_{p, q} > \tau$}
      \STATE $\hat{c} \gets \text{update\_class}(c^s_p, c^t_q)$
      \STATE $\hat{o} \gets (\hat{c}, \mathcal{A}^s_p \cup \mathcal{A}^t_q)$
      \STATE $\mathcal{O}^m \gets \{\hat{o}\} \cup \bigl(\mathcal{O}^m \setminus \{o^s_p, o^t_q\}\bigr)$
      \STATE \textbf{for each} $(o_x, o_y) \in \mathcal{E}^m$:
      \STATE \quad $(o_x, o_y) \mapsto 
        \begin{cases}
           (\hat{o}, o_y), & \text{if } o_x \in \{ o_p^s, o_q^t \}; \\
           (o_x, \hat{o}), & \text{if } o_y \in \{ o_p^s, o_q^t \}; \\
           (o_x, o_y), & \text{otherwise.}
        \end{cases}$
    \ENDFOR

    \STATE $\mathcal{Q} \gets \text{enqueue}(\mathcal{Q}, G^m)$
  \ENDWHILE

  \STATE $G_{\text{video}} \gets \text{dequeue}(\mathcal{Q})$
  \STATE \textbf{return} $G_{\text{video}}$

\end{algorithmic}
\end{algorithm} 
The scene graph consolidation step combines all frame-level scene graphs into a single graph that captures the overall visual content of the video. 
We outline our graph merging procedure, followed by a subgraph extraction technique for more focused video caption generation.

\subsubsection{Video-level graph integration}

Given two scene graphs, $G^s = (\mathcal{O}^s, \mathcal{E}^s)$ and $G^t = (\mathcal{O}^t, \mathcal{E}^t)$, constructed from two different frames, we perform the Hungarian matching between their object sets, $\mathcal{O}^s$ and $\mathcal{O}^t$.
The Hungarian algorithm aims to find the maximum matching between the objects in $\mathcal{O}^s$ and $\mathcal{O}^t$, which is given by
%
\begin{equation}
	\pi^* = \underset{\pi \in \Pi}{\arg\max} \sum_{i} \frac{ \psi_i(\phi(G^s))}{\| \psi_i(\phi(G^s)) \|} \cdot \frac{\psi_i(\phi(G_\pi^t)) }{\| \psi_i(\phi(G_\pi^t)) \|},
\end{equation}
%
where $\phi(\cdot)$ denotes the graph encoder, $\psi_i(\cdot)$ is the function to extract the $i^\text{th}$ object from an embedded graph, and $\pi \in \Pi$ indicates a permutation of objects in a graph.
Note that we introduce dummy objects  to deal with different numbers of objects for matching.

After identifying a set of matching object pairs, $\mathcal{M}$, \eg, $(p, q)$, where $o_p^s \in \mathcal{O}^s$ and $o_q^t \in \mathcal{O}^t$, using their cosine similarity with a predefined threshold, $\tau$, we merge the matched objects into a new one $\hat{o} \in \hat{\mathcal{O}}$, which is given by
%
\begin{equation}
    \hat{o} = (\hat{c} , \mathcal{A}^s_p \cup \mathcal{A}^t_q) \in \hat{\mathcal{O}},
\end{equation}
%
where $\hat{c}$ represents a class of the merged objects and $\hat{\mathcal{O}}$ denotes a set of new objects from all legitimate matching pairs.

Using this, we construct a new merged scene graph, $G^m$, which replaces each pair of merged objects with a new object $\hat{o}$, as follows:
%
\begin{equation}
	G^m = (\mathcal{O}^m, \mathcal{E}^m),
\end{equation}
%
where $\mathcal{O}^{m} =\mathcal{O}^s \cup \mathcal{O}^t \cup \hat{\mathcal{O}} ~ \setminus \bigcup_{(p, q) \in \mathcal{M}} \{o^s_p, o^t_q\}$, and the edge set $\mathcal{E}^m$ is also updated to reflect the changes in the object configuration.
Formally, each matching pair $(p, q) \in \mathcal{M}$ incurs the merge of the two objects and the construction of a new object $\hat{o}$, which results in the update of the edge set as $\mathcal{E}^m \equiv \mathcal{E}^s \cup \mathcal{E}^t$, which is formally given by
%
\begin{equation}
	(o_x, o_y) \in \mathcal{E}^m \rightarrow 
	\begin{cases}
		(\hat{o}, o_y) & \text{if } o_x \in \{o_p^s, o_q^t \}, \\
		(o_x, \hat{o}) & \text{if } o_y \in \{o_p^s, o_q^t \}, \\
		(o_x, o_y) & \text{otherwise.}
	\end{cases}
\end{equation}

We perform graph merging using a priority queue, where pairs of graphs are prioritized for merging based on their embedding similarity. 
In each iteration, the two most similar graphs are dequeued, merged, and the resulting graph is enqueued back into the priority queue.
This process is repeated until only one scene graph remains.
The final scene graph provides a comprehensive representation of the video, preserving frame-level details often overlooked by standard captioning models.
Algorithm~\ref{alg:hierarchical_graph_merge} describes the detailed procedure of our graph merging strategy.
  
\subsubsection{Prioritized subgraph extraction}
To generate concise and focused video captions, we apply subgraph extraction to retain only the most contextually relevant information. 
During the graph merging process, we track each node's merge count as a measure of its significance within the consolidated graph. 
We then identify the top $k$ nodes with the highest merge counts and extract their corresponding subgraphs. 
This approach prioritizes objects that consistently appear across multiple frames, as they often represent key entities in the scene. 
By emphasizing these essential elements and filtering out less relevant details, our method constructs a compact scene graph to generate a more focused video caption.

\section{Video Captioning}
\label{sec:videocaption}
To generate video-level descriptions that accurately reflect visual content, we developed a model that takes scene graphs as input and produce natural language descriptions.
This model is designed to effectively capture key components and relationships within the scene graph in generated text.

\vspace{-2mm}
\paragraph{Architecture}
We employ a modified encoder-decoder transformer architecture.
To prepare the input sequence for the graph encoder, each node, edge, and attribute in the graph, represented as a word or phrase, is tokenized into NLP tokens. 
These tokens are mapped to their embeddings via an embedding lookup.
For nodes consisting of multiple NLP tokens, their embeddings are averaged to form a single vector representation.
Additionally, a [CLS] token is appended as a global node to prevent isolation among disconnected components and ensure coherence. 
The adjacency matrix serves as an attention mask, incorporating graph topology into the attention mechanism. 
The graph encoder's output is then used as key and value inputs for the cross-attention layers of the text decoder, which generates the final outputs.

\vspace{-2mm}
\paragraph{Dataset}
For training, we collected approximately 2.5M text corpora that cover diverse visual scene contexts from various sources, including image caption datasets such as  MS-COCO~\cite{chen2015microsoft}, Flickr30k~\cite{young2014image}, TextCaps~\cite{sidorov2020textcaps}, Visual Genome~\cite{krishna2017visual}, and Visual Genome paragraph captioning~\cite{krause2016paragraphs}.
To further enhance the dataset, we incorporated model-generated captions for Kinetics-400~\cite{kay2017kinetics} dataset, with four uniformly sampled frames per video.
Note that neither the datasets nor the image VLMs used for generating frame captions are related to the target video captioning benchmarks.


\vspace{-2mm}
\paragraph{Training}
The model is trained using a next-token prediction objective, aiming to reconstruct the source text conditioned on the scene graph:
%
\begin{equation}
\mathcal{L}(\theta) = \sum_{i=1}^{N} \log P_{\theta}(t_i \mid t_{1:i-1}, G),
\end{equation}  
%
where $t_i$ represents the $i^\text{th}$ token in the source text, and $N$ denotes the total number of tokens.


\vspace{-2mm}
\paragraph{Video caption generation}
After constructing the video-level scene graph as described in Section~\ref{sec:scene}, we generate a video caption using the trained graph-to-text decoder, which conveys the overall narrative of the video.

\section{Experiments}
\label{sec:exp}

We conduct extensive experiments to address the following research questions:
1) How effectively can our retrieval mechanism enhance LLM-based TabICL in leveraging large-scale datasets?
2) How does LLM-based TabICL perform in comparison to numeric-based TabICL models and classic tabular models that are well-tuned on a case-by-case basis?
3) What are the unique strengths, current limitations, and potential future directions for LLM-based TabICL?


% \begin{figure*}[t]
% \vskip 0.2in
% \begin{center}
% \centerline{\includegraphics[width=\linewidth]{main_figures/main_scaling_pool_and_context.pdf}}
% \caption{
% We investigate the effects of increasing the number of training instances ($|D_{\text{train}}^{T'}|$) and the number of in-context instances per test example ($N^C$) on the TabICL performance of Phi3-GTL models.
% In each subplot, we compare the scaling effects of two Phi3-GTL models with different retrieval policies: one that randomly selects in-context instances, denoted as "Random," and the other employing our default \texttt{TabRAG} module, denoted as "RAG".
% We use violin plots to visualize the performance distribution across multiple held-out datasets. Additionally, dashed lines are used to emphasize that the median prediction error of our approach follows a power-law relationship with the number of training instances.
% }
% \label{fig:scaling_pool_ctx}
% \end{center}
% \vskip -0.2in
% \end{figure*}


\subsection{Experimental Setups}
\label{sec:exp_setup}

\paragraph{LLM Post-Training}
We use real-world tabular datasets to post-train a base LLM using generative tabular learning (GTL) objective as did in~\citep{wen2024GTL}.
However, unlike their approach, we adopt Phi-3~\citep{abdin2024phi3} as the base LLM, extending the effective context length from 4K to 128K and aligning it with our default retrieval policy.
Details of this post-training process are provided in Appendix~\ref{app:method_align_rag_llm}. For brevity and clearness, we denote our post-trained model as Phi3-GTL and refer to our approach as RAG+Phi3-GTL throughout the remainder of this paper.

\paragraph{Held-out Datasets}
We compile a comprehensive benchmark from the literature~\citep{gorishniy2021revisit_tab_dnn,grinsztajn2022tree_gt_tab_nn,gorishniy2024TabR,wen2024GTL}, ensuring diverse datasets that may favor different learning paradigms.
To avoid data leakage, we carefully examine and exclude any datasets used during the training of the Phi3-GTL model.
This process results in 29 classification datasets and 40 regression datasets for held-out evaluation, covering a wide range of domains, feature dimensions, types, and distributions.
Details of the data construction process are included in Appendix~\ref{app:data_constr}.

\begin{figure*}[t]
\vskip 0.2in
\begin{center}
\centerline{\includegraphics[width=0.98\linewidth]{main_figures/main_overall_comp_raw_metric.pdf}}
\caption{
An overall performance comparison of all models. In the left subplot, we use violin plots to show the AUROC scores of different models across 29 classification tasks, while the right subplot displays the NMAE scores for 40 regression tasks. Models are sorted by their median metric score across the held-out datasets, with dashed lines indicating these median scores in each subplot. Our approach, RAG+Phi3-GTL, is prefixed with a marker (*), for quick identification.
}
\label{fig:overall_comp}
\end{center}
\vskip -0.2in
\end{figure*}

\begin{figure*}[t]
\vskip 0.2in
\begin{center}
\centerline{\includegraphics[width=0.90\linewidth]{main_figures/main_ensemble_norm_metric.pdf}}
\caption{
    Ensemble performance comparisons of RAG+Phi3-GTL, TabPFN-v2, LightGBM, and CatBoost are presented, where normalized AUROC or NMAE scores (min-max normalized across methods for each dataset) are plotted to highlight their relative strengths across multiple datasets, while omitting absolute metric differences.
}
\label{fig:ensemble_res}
\end{center}
\vskip -0.2in
\end{figure*}


\paragraph{Baselines}
We include Phi3-GTL and RAG+KNN as two ablated variants of RAG+Phi3-GTL: the former uses randomly selected in-context instances, while the latter employs the same default retrieval policy but relies on the K-Nearest Neighbors (KNN) algorithm~\citep{fix1951knn,cover1967knn_cls} for prediction.
We compare against TabPFN-v1~\citep{hollmann2023TabPFN}, which supports only classification tasks, and TabPFN-v2~\citep{hollmann2025TabPFNv2}, the state-of-the-art TabICL model utilizing numeric representations.
In addition, our baselines include other representative tabular models such as XGBoost~\citep{chen2016XGBoost}, LightGBM~\citep{ke2017LightGBM}, CatBoost~\citep{prokhorenkova2018catboost}, MLP, FTT~\citep{gorishniy2021revisit_tab_dnn}, and TabR~\citep{gorishniy2024TabR}, all of which are extensively tuned via hyperparameter search for each dataset.
We also include several ``RAG + X'' baselines, where "X" represents models trained and inferred on the selected in-context instances using the same retrieval policy as RAG+Phi3-GTL. These include Logistic Regression (LR), TabPFN-v1, TabPFN-v2, and XGBoost.
These baselines are designed to highlight the TabICL capability of LLMs given limited in-context instances.

\paragraph{Metrics}
For classification tasks, we use the Area Under the Receiver Operating Characteristic curve (AUROC) as the primary evaluation metric. For regression tasks, we employ the Mean Absolute Error normalized by the label mean (NMAE).
Additionally, to compare the relative performance across a group of methods, we utilize group-wise min-max normalized AUROC and NMAE metrics.


\begin{figure*}[t]
\vskip 0.2in
\begin{center}
\centerline{\includegraphics[width=\linewidth]{main_figures/main_per_dataset_comp.pdf}}
\caption{
    Per-dataset performance comparisons between RAG+Phi3-GTL and the two most competitive baselines, TabPFN-v2 and CatBoost, are presented, with dataset IDs sorted by performance gaps. Dashed lines and annotations are used to indicate the proportion of datasets where RAG+Phi3-GTL outperforms these baselines and where it significantly lags behind.
}
\label{fig:per_dataset_comp}
\end{center}
\vskip -0.2in
\end{figure*}


\begin{figure*}[t]
\vskip 0.2in
\begin{center}
% \centerline{\includegraphics[width=0.9\linewidth]{main_figures/main_decision_boundary.pdf}}
\centerline{\includegraphics[width=1.0\linewidth]{main_figures/main_decision_boundary.pdf}}
\caption{
Decision boundary comparisons of various models, where each row corresponds to a specific set of training instances generated from a given data distribution. The first column visualizes these training instances, while the subsequent columns illustrate the decision boundaries of different models. The top two rows represent the same data distribution but with varying numbers of training instances, whereas the bottom two rows depict a different data distribution.
}
\label{fig:decision_boundary}
\end{center}
\vskip -0.2in
\end{figure*}


\subsection{Scaling with Available Training Instances}
\label{sec:exp_scaling_pool_ctx}

Figure~\ref{fig:scaling_pool_ctx} illustrates the performance variations as the size of the training data and the number of in-context instances increase for our approach under two retrieval policies: Random and RAG (our default retrieval policy).
It is evident that the RAG policy enables Phi3-GTL to effectively leverage larger training datasets, while the Random policy lacks this capability. Specifically, with the RAG policy, the median prediction error demonstrates a power-law relationship with the number of training instances, expressed as $L(D) = (D_c / D)^{\alpha}$. For classification tasks, $L=1-\text{AUROC}$, $D_c \sim 6.05e^{-5}$, and $\alpha \sim 0.102$, whereas for regression tasks, $L=\text{NMAE}$, $D_c \sim 8.05e^{-8}$, and $\alpha \sim 0.053$. This finding highlights a favorable statistical learning characteristic: given a distinguishable feature space and sufficient training instances, the expected prediction error approaches zero.

Moreover, the RAG policy reduces the number of in-context instances required for accurate predictions. As shown in the right two subplots of Figure~\ref{fig:scaling_pool_ctx}, the model using the Random policy benefits significantly from an increased number of in-context instances. In contrast, with the RAG policy, performance often saturates as the number of adaptive in-context instances increases. This indicates that, for most datasets, tens of training instances are sufficient to form a supportive context for inferring the label of a test instance.


\subsection{Overall Comparison}
\label{sec:exp_overall_comp}


Figure~\ref{fig:overall_comp} presents an overall comparison of all models by illustrating the error distributions across held-out datasets. TabPFN-v2 emerges as the most competitive baseline in terms of the median prediction error. However, its wider error bars indicate sub-optimal performance in certain cases. In contrast, well-tuned tree-based models such as LightGBM and CatBoost, as well as neural models like FTT and TabR, demonstrate more robust performance with narrower error distributions.

When comparing our approach, RAG+Phi3-GTL, with these baselines, we observe significant improvements over its ablated variants, Phi3-GTL and RAG+KNN, underscoring the importance of both retrieval and TabICL components. Furthermore, RAG+Phi3-GTL is among the top-performing of ``RAG + X'' baselines, even surpassing RAG+TabPFN-v2 in terms of median prediction performance across held-out datasets.
This highlights the potential of TabICL based on text representations, which can uncover novel and highly effective ICL algorithms by operating in a text space.
Besides, our approach achieves zero NMAE for a specific integer regression task without requiring explicit programming, whereas all numeric models, by default, produce float outputs.
Lastly, RAG+Phi3-GTL still lags behind well-tuned baseline models and TabPFN-v2 in overall performance.


\subsection{Ensemble Results}
\label{sec:exp_ensemble_res}

We further explore the potential of RAG+Phi3-GTL by investigating its contribution to ensemble diversity, as shown in Figure~\ref{fig:ensemble_res}.
When comparing RAG+Phi3-GTL with TabPFN-v2, LightGBM, and CatBoost, we observe that although RAG+Phi3-GTL underperforms the top-performing baselines overall, it exhibits unique strengths in certain scenarios.
Moreover, a comparison of Ensemble-All with TabPFN+LightGBM and TabPFN+CatBoost reveals that these unique strengths translate into ensemble diversity, enhancing the robustness of overall ensemble performance.
These findings highlight the potential of leveraging language as an alternative interface for tabular data learning, complementing existing tabular learning algorithms.

\subsection{Per-dataset Comparisons}
\label{sec:exp_per_ds_comp}

These findings further motivate us to conduct per-dataset comparisons to identify datasets where RAG+Phi3-GTL excels and those where it still underperforms.
Figure~\ref{fig:per_dataset_comp} summarizes these results.
We observe that on approximately 17\%-20\% of datasets, RAG+Phi3-GTL outperforms the state-of-the-art TabICL model, TabPFN-v2, as well as the classic tree-based model, CatBoost, which has been carefully tuned for each dataset.
Furthermore, on over 80\% of datasets, the performance of RAG+Phi3-GTL falls within a small gap of these two competitive baselines.
These results indicate that RAG+Phi3-GTL is already a strong prediction model for most tabular datasets, suggesting that we could not only engage with a well-prepared LLM conversationally but also leverage it to understand tabular data, provide accurate predictions, and offer potential explanations.


\paragraph{Analysis of Failure Cases}
Per-dataset comparison results also prompt an investigation into why the performance of RAG+Phi3-GTL falls short in certain cases. Case studies detailed in Appendix~\ref{app:case_study} provide insights into these failure scenarios.
In summary, most failure cases are attributed to the limitations of the default retrieval policy, which struggles to extract effective in-context instances due to specific data characteristics (e.g., datasets R-25, R-33, and R-27). In these cases, we find that slight adjustments to the retrieval strategy—such as applying alternative numerical normalization methods for feature similarity calculations or leveraging prior knowledge to define instance similarities—can lead to significant performance improvements for RAG+Phi3-GTL.
These findings suggest that a non-parametric, default retrieval mechanism may be insufficient in certain scenarios. Practitioners could potentially achieve better performance by employing "retrieval engineering" (as discussed in Section~\ref{sec:method}).
Additionally, some failure cases, such as dataset C-17, reveal limitations in Phi3-GTL’s ability to perform effective TabICL on specific data distributions, where TabPFN-v2 significantly outperforms. We hypothesize that this gap arises from the limited coverage of data patterns during Phi3-GTL’s post-training phase, which utilized approximately 300 real-world datasets from~\citeauthor{wen2024GTL}. In contrast, TabPFN-v2 likely benefits from pre-training on a much broader family of synthesized datasets.


\subsection{Decision Boundary Analysis}
\label{sec:exp_dec_bound}

Figure~\ref{fig:decision_boundary} compares the decision boundaries of various models across four groups of synthetic instances.
We observe that RAG+Phi3-GTL produces a distinctive, non-smooth decision boundary, which is entirely different from models relying on numeric representations. This boundary reflects case-by-case generalization from known training instances to unseen regions, leaving more uncertain areas when data samples are sparse.

In terms of shape, the decision boundary of RAG+Phi3-GTL bears some resemblance to that of Nearest Neighbors. However, LLM-based TabICL generalizes far beyond a simple rule-based average of neighboring training instances. We hypothesize that this unique behavior arises from the text-based representation of tabular data and the ICL capability of LLMs.
These findings also highlight opportunities for further improving LLM-based TabICL. Specifically, to encourage smoother decision boundaries when sufficient training data is available, one approach could involve generating large-scale synthetic data and fine-tuning LLMs to emulate such behaviors.



% \subsection{Case Analysis}
% \label{sec:exp_case_study}

% Winning Case analysis against TabPFN v2:
% regression-cat-medium-0-analcatdata_supreme

% Failure Case analysis:
% lack of learning certain feature distributions
% fail to select the best in-context examples
% \subsection{Experiment settints}

% \textbf{Data preparation.} For evaluating the performance of tree-based models versus large language models (LLMs) and deep learning approaches on tabular data tasks, we prepare datasets drawn from three research branches—GTL~\cite{wen2024GTL}, TabR~\cite{gorishniy2024TabR}, and Tree~\cite{grinsztajn2022tree_gt_tab_nn}, encompassing 32 regression and 23 classification tasks. This diverse dataset collection enables robust and transparent comparisons across model classes. For GTL, we filtered 50 test datasets from an initial pool of 350, focusing on feature validation to avoid data leakage and ensure accurate classification of features. Only datasets meeting size and integrity criteria were retained, ensuring robust evaluations across diverse tabular tasks. In TabR, we reused GTL’s pretrained checkpoints, eliminating redundancy by de-duplicating datasets from GTL’s collection. Large datasets were downsampled to 100K samples to ensure computational efficiency without sacrificing representativeness, as detailed in Appendix~\ref{}. For Tree, datasets with numerical and categorical features were curated with a focus on eliminating overlap with GTL and TabR datasets. Multiple versions or splits were consolidated to a single representative version, ensuring a fair and unbiased comparison across model classes, and preserving dataset diversity.

% \textbf{Baselines.}  We provide below a list of our used baselines. For most baselines, a hyperparameter search space is carefully prepared by the literature and we just follow this space setting and perform hyperparameter optimization on this then ensemble the results to boost the performance.

% \begin{itemize}
%     \item \textbf{LR} (Linear/Logistic Regression): A simple, linear baseline for tabular tasks that struggles with nonlinearity, limiting its performance on more complex datasets.
%     \item \textbf{KNN} (K-Nearest-Neighbor): A non-parametric model that makes predictions based on nearby samples. We use the same context samples as retrieved by TabRAG, providing a simple baseline for context ensemble
%     \item \textbf{GBDT} (XGBoost~\cite{chen2016XGBoost}, LightGBM~\cite{ke2017LightGBM}, CatBoost~\cite{prokhorenkova2018catboost}): Gradient boosting frameworks that build sequential decision tree models to optimize performance, particularly excelling in large, imbalanced, or missing data. Extensive hyperparameter tuning is applied for optimal performance.
%     \item \textbf{MLP} (Multi-layer Perceptron): A basic neural network used to evaluate the effectiveness of deep learning on tabular data. We applied a range of hyperparameters and ensemble strategies to maximize its performance.
%     \item \textbf{SAINT}~\cite{somepalli2021SAINT}: A novel architecture that uses self-attention and contrastive learning to capture feature relationships within rows of tabular data. It enhances performance by focusing on important features and pre-training on large tabular datasets.
%     \item \textbf{FT-Transformer}~\cite{gorishniy2021revisit_tab_dnn}: Adapts transformer architecture for tabular data by tokenizing both categorical and numerical features. It uses learned embeddings and positional encodings to process features, enabling the transformer to handle tabular structures effectively.
%     \item \textbf{TabPFN}~\cite{hollmann2022TabPFN}: A probabilistic model that uses in-context learning, pretrained on a wide variety of small tabular datasets to learn common feature interactions, allowing predictions without explicit retraining. It struggles with large datasets and regression tasks.
%     \item \textbf{TabR}~\cite{gorishniy2024TabR}: A retrieval-based foundation model that encodes query datasets and searches for similar datasets in latent space, enabling it to generalize to new tasks by leveraging historical data. Its predictions benefit from learned embeddings and in-context retrieval.
% \end{itemize}

{
    \small
    \bibliographystyle{ieeenat_fullname}
    \bibliography{main}
}


\clearpage
% \setcounter{page}{1}
% \maketitlesupplementary
\begin{center}
Supplementary Material
\end{center}

% {
%     \onecolumn
%     \centering
%     \Large
%     \textbf{\thetitle}\\
%     \vspace{0.5em}Supplementary Material \\
%     \vspace{1.0em}
% }

\section{Proof of \cref{theorem:dr}}
We require some additional regularity assumptions:
\begin{assumption} 1) The number of classes $C$ is bounded w.r.t the number of samples $N$, 2) the missingness mechanism $P(A=1|Y,\theta)$, as well as its estimated counterpart $P(A=1|Y,\theta)$, are bounded below by some constant $\epsilon > 0$, 3) the quantities $P(Y|X,\theta)$ and $P(A|Y,\theta)$ are estimated using auxiliary samples independent of samples used for the sample averaging.
\label{assumption:extra}
\end{assumption}
Assumptions 1 and 2 are natural. For the missingness mechanism, the ground truth being bounded means that there is a non-vanishing proportion of samples for every class. The boundedness of the estimate can be enforced by clipping the estimate. Assumption 3 is called sample splitting in \cite{kennedy-dr}.

For convenience we use operator $\E_N$ to denote the average of $N$ samples i.e. $\frac{1}{N}\sum_{i=1}^N$. Note that this is by itself a random variable, in contrast to $\E$ which is a fixed number.

\begin{proof}[Proof of \cref{theorem:dr}] Because $C$ is bounded (assumption \ref{assumption:extra}), we can fix a class $c$ and prove the theorem.
Let us define the influence function $\phi$, parameterized by $\theta$, as
\begin{equation}
\phi(O | \theta)(c) = P(Y=c|X,\theta) + \frac{\one(A=1)}{P(A=1|Y,\theta)} (\one(Y=c) - P(Y=c|X,\theta)) - P(Y=c)
\end{equation}
As we have done in the main text, we use $\phi(O)$ to denote the same function but all estimated quantities are replaced with their truths. In other words, we use $\phi(O)$ for $\phi(O|\theta_0)$ where $\theta_0$ is the truth, given that our model contains $\theta_0$ e.g. when the model is consistent.

Recall that:
\begin{equation}
\begin{aligned}
\Psi_{dr}(\theta)(c) &= \frac{1}{N}\sum_{i=1}^N \left\{P(Y=c|X,\theta) + \frac{\one(A=1)}{P(A=1|Y,\theta)} (\one(Y=c) - P(Y=c|X,\theta))\right\}\\
&= \E_N [\phi(O|\theta)(c)] + P(Y=c)
\end{aligned}
\end{equation}

We will show that:
\begin{equation}
\Psi_{dr}(\theta)(c) - P(Y=c) = (\E_N - \E)[\phi(O)(c)] + o_P(N^{-1/2})
\label{eq:proof-linearity}
\end{equation}
To do that, we use the following decomposition
\begin{equation}
\begin{aligned}
\Psi_{dr}(\theta)(c) - P(Y=c) &= \E_N [\phi(O|\theta)(c)] \\
&= (\E_N - \E)[\phi(O)(c)] + (\E_N - \E)[\phi(O|\theta)(c) - \phi(O)(c)] + \E[\phi(O|\theta)(c)]
% &+ (\E_n - \E)[\phi(O;\theta) - \phi(O)]\\
% &+ \E[P(Y=c|X,\theta)] - \E[P(Y=c|X)] + \E[\phi(O,\theta)]
\end{aligned}
\end{equation}
and analyze the second and third term. The third term is:
\begin{equation}
\begin{aligned}
\E[\phi(O|\theta)(c)] &= \E[P(Y=c|X,\theta)] + \E\left[\frac{\one(A=1)}{P(A=1|Y,\theta)}(\one(Y=c) - P(Y=c|X,\theta))\right]- P(Y=c) \\
&= \E\left[P(Y=c|X,\theta) + \frac{P(A=1|Y)}{P(A=1|Y,\theta)}(P(Y=c|X) - P(Y=c|X,\theta))\right] - \E[P(Y=c|X)]\\
&= \E\left[(P(Y=c|X,\theta) - P(Y=c|X)) (P(A=1|Y,\theta) -P(A=1|Y)) \frac{1}{P(A=1|Y,\theta)}\right]\\
\end{aligned}
\end{equation}
by Cauchy-Schwarz inequality:
\begin{equation}
\begin{aligned}
\E[\phi(O|\theta)(c)] &\le \frac{1}{\epsilon} \|P(A=1|Y,\theta) - P(A=1|Y)\|_2 \|P(Y=c|X,\theta) - P(Y=c|X)\|_{L_2(P)}\\
&= \frac{1}{\epsilon} o_P(N^{-1/4} N^{-1/4}) = o_P(N^{-1/2})
\end{aligned}
\end{equation}
by assumption \ref{assumption:4th-root-n} and that $P(A=1|Y,\theta) > \epsilon$ (assumption \ref{assumption:extra}). The second term can be bounded by Chebyshev inequality
% \begin{equation}
% \begin{aligned}
% \E[\E_N[\phi(O|\theta)(c) - \phi(O)(c)]] &= \E[\phi(O|\theta)(c) - \phi(O)(c)]\\
% \var[\E_N[\phi(O|\theta)(c) - \phi(O)(c)]] &= \frac{1}{N}\var[\phi(O|\theta)(c) - \phi(O)(c)] \le 
% \end{aligned}
% \end{equation}
\begin{equation}
P(|(\E_N - \E)[\phi(O|\theta)(c) - \phi(O)(c)]| \ge t) \le \frac{\var[\E_N[\phi(O|\theta)(c) - \phi(O)(c)]]}{t^2} = \frac{\var[\phi(O|\theta)(c) - \phi(O)(c)]}{Nt^2}
\end{equation}
note here that $\theta$ is independent of the samples used for $\E_N$ by assumption \ref{assumption:extra}. For any $\varepsilon > 0$, by picking $t = \frac{1}{\sqrt{N\varepsilon}}$ we get
\begin{equation}
P\left(\left|\frac{(\E_N - \E)[\phi(O|\theta)(c) - \phi(O)(c)]}{N^{-1/2}}\right| \ge \frac{1}{\sqrt{\varepsilon}}\right) \le \varepsilon \var[\phi(O|\theta)(c) - \phi(O)(c)]
\end{equation}
by the definition of $O_P$, we then get
\begin{equation}
(\E_N - \E)[\phi(O|\theta)(c) - \phi(O)(c)] = O_P(N^{-1/2}\var[\phi(O|\theta)(c) - \phi(O)(c)])
\end{equation}
Because $\phi$ is a continuous function of $P(Y|X,\theta)$ and $P(A|Y,\theta)$ (given $P(A|Y,\theta) > \epsilon$, assumption \ref{assumption:extra}), by the continuous mapping theorem and the fact that $P(Y|X,\theta)$ and $P(A|Y,\theta)$ are convergent in probability (assumption \ref{assumption:4th-root-n}), we get $\var[\phi(O|\theta)(c) - \phi(O)(c)] = o_P(1)$. This gives
\begin{equation}
(\E_N - \E)[\phi(O|\theta)(c) - \phi(O)(c)] = o_P(N^{-1/2})
\end{equation}
Therefore, we have shown that the second and third term are both $o_P(N^{-1/2})$, proving \cref{eq:proof-linearity}. As the final step, multiply both sides of this equation by $\sqrt{N}$ we get:
\begin{equation}
\sqrt{N}(\Psi_{dr}(\theta)(c) - P(Y=c)) = \sqrt{N} (\E_N - \E)[\phi(O)(c)] + o_P(1) \rightsquigarrow \mathcal{N}(0, \var[\phi(O)(c)])
\end{equation}
by the central limit theorem, and $\var[\phi(O)(c)] = \E[\phi(O)(c)^2]$ because $\E[\phi(O)(c)] = 0$.
\end{proof}

While we started with the definition of $\phi$, \cref{eq:proof-linearity} shows that $\phi$ is indeed an influence function. Now we show that $\phi$ is also the efficient influence function, by using the characterization of the model's tangent space \cite{tsiatis-missingdata}. Note that the joint probability factorizes as $P(X,A,Y) = P(X)P(Y|X)P(A|Y)$, therefore the tangent space $\mathcal{T}$ factorizes as $\mathcal{T} = \mathcal{T}_{X} \oplus \mathcal{T}_{Y|X} \oplus \mathcal{T}_{A|Y}$ where $\mathcal{T}_X = \{h(X): \E[h] = 0\}$, $\mathcal{T}_{Y|X} = \{h(X,Y): \E[h|X] = 0\}$, $\mathcal{T}_{A|Y} = \{h(A,Y): \E[h|Y] = 0\}$, and the 3 subspaces are pairwise orthogonal. All influence functions are orthogonal to the tangent space, but the influence function that is also in the tangent space has the smallest variance and is called the efficient influence function. As $\phi$ is already an influence function, we need only show that $\phi$ is in $\mathcal{T}$. We write $\phi$ as
\begin{equation}
\phi(O)(c) = (P(Y=c|X) - P(Y=c)) + \left[\frac{\one(A=1)}{P(A=1|Y)} - 1\right](\one(Y=c) - P(Y=c|X)) + (\one(Y=c) - P(Y=c|X))
\end{equation}
and note that the first, second and third term are in $\mathcal{T}_X$, $\mathcal{T}_{A|Y}$ and $\mathcal{T}_{Y|X}$ respectively. Therefore, $\phi$ is indeed in $\mathcal{T}$. The efficient influence function has the smallest variance of all influence function, and therefore our estimator being asymptotically linear in $\phi$ (\cref{eq:proof-linearity}) has the smallest mean squared error in a local asymptotic minimax sense \cite{kennedy-dr, asymptoticstatistics}

\section{Further background and related work}
\paragraph{Discussion on semi-supervised EM.}
It appears that semi-supervised EM was first used for parameter estimation when the missingness mechanism is non-ignorable in \cite{ibrahim1996parameter}, but has not been used for label shift estimation.
Perhaps this is because the semi-supervised situation where additional unlabeled data is available during training is rarer than the test-time adaptation case. EM is well suited to take advantage of the extra unlabeled data to improve the classifier under very scarce and long-tailed labeled data. While the connection between pseudo-labeling and EM has been explored before \cite{entropyminimization}, the situation with label shift has not until recently \cite{simpro}. Here the application of EM is much more interesting, because other than simply giving pseudo-labeling a rigorous formulation, EM also estimates the missingness mechanism (equivalently the label distribution shift), which is important for shift correction and thus high-quality pseudo-labels \cite{acr}. The application of confidence thresholding can be seen as a sparse variant of EM \cite{neal1998view}.

\paragraph{The doubly-robust risk.} 
\label{subsec:dr-risk}
A technique that also derives from the theory of semi-parametric efficiency is orthogonal statistical learning \citep{foster2023orthogonal}. The idea is to minimize the doubly-robust risk:
\label{subsec:method-dr-risk}
\begin{equation}
\label{eq:dr-risk}
\mathcal{R}(\theta_2) = \frac{1}{N} \sum_{i=1}^N \Bigg[ l(x_i, \hat y_i|\theta_2) + \frac{\one(a_i=1)}{P(A=a_i|Y=y_i, \theta_1)} (l(x_i, y_i | \theta_2) - l(x_i, \hat y_i | \theta_2))\Bigg]
\end{equation}
where $l(x,y|\theta) = -\sum_{c=1}^C [y]_c \log P(Y=c|X=x,\theta)$ is the negative cross-entropy. 
The notation $[y]_c$ means that we are using the $c$-entry in a C-dimension probability vector $y$. 
Thus, $y_i$ denotes the one-hot label of observation $i$, while $\hat y_i$ denotes the pseudo-label, which can be one-hot or all-zero. 
Finally, we use $\theta_1$ to denote that $P(a|y,\theta_1)$ is an estimation from a previous stage, but it can be estimated with $\theta_2$ as well. 
The risk $\mathcal{R}(\theta_2)$ can be used as a training loss in a straightforward fashion. 
Similar to the doubly robust estimation of $P(Y)$, the doubly robust risk provides approximately unbiased estimation of the risk. 
This property has been used in \citep{arelabelsinformative, onnonrandommissinglabels, drst} also in the semi-supervised learning setting.
More broadly, it is at the heart of one of the core techniques in heterogenous treatment effect estimation in causal estimation \cite{kennedy2023towards, foster2023orthogonal, wager2018estimation}. 
The focus here is not the estimation of $\mathcal{R}(\theta_2)$ per se, but the quality of the learned model \cite{foster2023orthogonal}.
By using the doubly-robust risk, we can achieve an optimality result similar in spirit to our theorem \cref{theorem:dr}, but for the generalization error.
While this is appealing, in practice there are 2 problems with this approach. First, the inverse probability weight $P(A=a_i|Y=y_i,\theta_1)$ can be very large if the class ratio is highly unlabeled, making training unstable \cite{kallus2020deepmatch, pham2023stable}. 
This problem exists for our estimation as well. However, it is much easier to control for estimation than for training because of the iterative nature of model update. Secondly, we can further write $\mathcal{R}$ as:
\begin{equation}
\mathcal{R}(\theta_2) = \frac{1}{N}\sum_{i=1}^N l\left(x_i, \hat y_i + \frac{\one(a_i=1)}{P(A=a_i|Y=y_i,\theta_1)} (y_i - \hat y_i)\Bigg\vert\theta_2\right)
\end{equation}
which is a cross-entropy loss with new meta-pseudo-labels. However, these labels are not meant to be learned exactly, and furthermore they can be negative. Thus, theoretical works have to put stringent assumptions on the models. In \cref{subsec:ablation-1}, we show that experimentally that the instability problem makes doubly-robust risk performance worse than our 2-stage approach.

\section{Training and hyperparameter settings.}
\label{subsec:training-setting}
For neural network training, we follow the implementation and hyperparameter settings of \cite{simpro}. In particular, we adapt the core code of SimPro for Supervised, MLE and EM. For MLE, we update $P(A|Y)$ using the Adam optimizer with learning rate 1e-3, while for EM we use a momentum update similar to SimPro's update of $P(Y|A)$ because it has a a closed-form solution at each mini-batch. We use Wide ResNet-28-2 on all methods and all datasets in this section, including Imagenet-127, because we are motivated by the fact that stage-1's goal is not classification accuracy but the estimation of a finite-dimensional parameter. When using Wide ResNet-28-2 for Imagenet-127, we use the hyperparameters of CIFAR-100, except we lower the batch size of unlabeled data to 2 times that of labeled data instead of 8 for memory reason. We do not perform additional hyperparameter tuning. All experiments can be performed on 1 A6000 RTX GPU, and are run 3 times. We report the total variation distance between the estimated and the ground truth unlabeled class distribution, similar to its usage in Theorem 3.1 of \cite{lsc}, and the top-1 classification accuracy.

In the second stage of our algorithm, we freeze our estimation and plug it in SimPro and BOAT.
We keep exactly the same hyperparameter settings that SimPro and BOAT use. In particular, for Imagenet-127, we now use ResNet-50 and run each experiment once.
In SimPro, we set the unlabeled class distribution $P(Y|A=0)$ at the E-step;  however, we still keep a running estimate of the class distribution $P(Y)$ in the logit adjustment loss \cref{eq:simpro-la-loss}. While it is possible to use the first stage estimate in the logit adjustment loss, we observe that doing so results in lower accuracy than using the the running average. This is conceptually consistent with the role of the running average - serving not as an accurate estimate of $P(Y)$ but to make the classifier's class distribution uniform through the logit adjustment loss, which is good for the test set. Similarly, in BOAT, we only replace $\Delta_c = \log P(Y|A=1) - \log P(Y|A=0)$ in equation (4) of \cite{boat}, which is adjusting a classifier's predictions from the labeled to the unlabeled class distribution, with our SimPro + DR estimate instead of their on-the-fly estimate. 


% \section{Additional experiments}
% % \begin{table*}[t]
\centering
\caption{Total Variation Distance on CIFAR-10-LT ($N_l = 500$, $M_l = 4000$) with different class imbalance ratios $\gamma_l$ and $\gamma_u$ under five different unlabeled class distributions.}
\label{tab:cifar10-tv}
\resizebox{\textwidth}{!}{
\begin{tabular}{lccccccccccc}
\toprule
& & \multicolumn{2}{c}{consistent} & \multicolumn{2}{c}{uniform} & \multicolumn{2}{c}{reversed} & \multicolumn{2}{c}{middle} & \multicolumn{2}{c}{head-tail} \\
\cmidrule(lr){3-4} \cmidrule(lr){5-6} \cmidrule(lr){7-8} \cmidrule(lr){9-10} \cmidrule(lr){11-12}
& & $\gamma_l = 150$ & $\gamma_l = 100$ & $\gamma_l = 150$ & $\gamma_l = 100$ & $\gamma_l = 150$ & $\gamma_l = 100$ & $\gamma_l = 150$ & $\gamma_l = 100$ & $\gamma_l = 150$ & $\gamma_l = 100$ \\
Model & Estimator & $\gamma_u = 150$ & $\gamma_u = 100$ & $\gamma_u = 1$ & $\gamma_u = 1$ & $\gamma_u = 1/150$ & $\gamma_u = 1/100$ & $\gamma_u = 150$ & $\gamma_u = 100$ & $\gamma_u = 150$ & $\gamma_u = 100$ \\
\midrule
Supervised & MLLS & 0.269 ± 0.252 & 0.038 ± 0.006 & 0.251 ± 0.046 & 0.255 ± 0.060 & 0.429 ± 0.028 & 0.493 ± 0.050 & 0.333 ± 0.042 & 0.320 ± 0.009 & 0.457 ± 0.034 & 0.444 ± 0.043 \\
Supervised & RLLS & 0.043 ± 0.001 & 0.044 ± 0.010 & 0.348 ± 0.034 & 0.305 ± 0.068 & 0.769 ± 0.016 & 0.678 ± 0.028 & 0.430 ± 0.008 & 0.368 ± 0.013 & 0.539 ± 0.018 & 0.503 ± 0.020 \\
\midrule
MLE & IPW & 0.027 ± 0.001 & 0.027 ± 0.000 & 0.319 ± 0.072 & 0.243 ± 0.010 & 0.674 ± 0.020 & 0.646 ± 0.041 & 0.438 ± 0.020 & 0.454 ± 0.026 & 0.547 ± 0.049 & 0.491 ± 0.059 \\
MLE & OR & 0.045 ± 0.004 & 0.042 ± 0.000 & 0.215 ± 0.026 & 0.203 ± 0.032 & 0.433 ± 0.017 & 0.395 ± 0.033 & 0.193 ± 0.006 & 0.209 ± 0.037 & 0.307 ± 0.147 & 0.249 ± 0.130 \\
MLE & DR & 0.090 ± 0.002 & 0.079 ± 0.000 & 0.407 ± 0.027 & 0.360 ± 0.007 & 0.425 ± 0.007 & 0.421 ± 0.029 & 0.256 ± 0.001 & 0.286 ± 0.031 & 0.435 ± 0.136 & 0.362 ± 0.122 \\
\midrule
EM & IPW & 0.035 ± 0.002 & 0.040 ± 0.001 & 0.021 ± 0.001 & 0.029 ± 0.015 & 0.303 ± 0.187 & 0.091 ± 0.010 & 0.119 ± 0.011 & 0.105 ± 0.022 & 0.104 ± 0.026 & 0.104 ± 0.051 \\
EM & OR & 0.037 ± 0.003 & 0.042 ± 0.002 & 0.016 ± 0.001 & 0.024 ± 0.012 & 0.269 ± 0.183 & 0.090 ± 0.008 & 0.122 ± 0.012 & 0.103 ± 0.022 & 0.072 ± 0.012 & 0.073 ± 0.024 \\
EM & DR & 0.034 ± 0.004 & 0.037 ± 0.001 & 0.014 ± 0.001 & 0.027 ± 0.020 & 0.264 ± 0.191 & 0.092 ± 0.005 & 0.111 ± 0.019 & 0.097 ± 0.026 & 0.077 ± 0.016 & 0.073 ± 0.028 \\
\midrule
SimPro & IPW & 0.070 ± 0.011 & 0.058 ± 0.000 & 0.046 ± 0.001 & 0.049 ± 0.005 & 0.254 ± 0.074 & 0.223 ± 0.098 & 0.097 ± 0.025 & 0.067 ± 0.002 & 0.105 ± 0.066 & 0.110 ± 0.079 \\
SimPro & OR & 0.071 ± 0.012 & 0.058 ± 0.000 & 0.045 ± 0.001 & 0.049 ± 0.006 & 0.040 ± 0.003 & 0.059 ± 0.017 & 0.074 ± 0.006 & 0.075 ± 0.002 & 0.033 ± 0.003 & 0.033 ± 0.003 \\
SimPro & DR & 0.017 ± 0.004 & 0.026 ± 0.001 & 0.019 ± 0.002 & 0.018 ± 0.003 & 0.039 ± 0.003 & 0.058 ± 0.025 & 0.091 ± 0.007 & 0.031 ± 0.001 & 0.015 ± 0.003 & 0.019 ± 0.007 \\
\bottomrule
\end{tabular}
}
\end{table*}
% 

\begin{table*}[t]
\centering
\caption{Total Variation Distance on CIFAR-100-LT ($N_l = 50$, $M_l = 400$) with different class imbalance ratios $\gamma_l$ and $\gamma_u$ under five different unlabeled class distributions.}
\label{tab:cifar100-tv}
\resizebox{\textwidth}{!}{
\begin{tabular}{lccccccccccc}
\toprule
& & \multicolumn{2}{c}{consistent} & \multicolumn{2}{c}{uniform} & \multicolumn{2}{c}{reversed} & \multicolumn{2}{c}{middle} & \multicolumn{2}{c}{head-tail} \\
\cmidrule(lr){3-4} \cmidrule(lr){5-6} \cmidrule(lr){7-8} \cmidrule(lr){9-10} \cmidrule(lr){11-12}
& & $\gamma_l = 20$ & $\gamma_l = 10$ & $\gamma_l = 20$ & $\gamma_l = 10$ & $\gamma_l = 20$ & $\gamma_l = 10$ & $\gamma_l = 20$ & $\gamma_l = 10$ & $\gamma_l = 20$ & $\gamma_l = 10$ \\
Model & Estimator & $\gamma_u = 20$ & $\gamma_u = 10$ & $\gamma_u = 1$ & $\gamma_u = 1$ & $\gamma_u = 1/20$ & $\gamma_u = 1/10$ & $\gamma_u = 20$ & $\gamma_u = 10$ & $\gamma_u = 20$ & $\gamma_u = 10$ \\
\midrule
Supervised & MLLS & 0.707 ± 0.016 & 0.313 ± 0.100 & 0.445 ± 0.172 & 0.309 ± 0.119 & 0.383 ± 0.075 & 0.397 ± 0.006 & 0.570 ± 0.001 & 0.373 ± 0.107 & 0.543 ± 0.009 & 0.231 ± 0.057 \\
Supervised & RLLS & 0.520 ± 0.007 & 0.133 ± 0.003 & 0.337 ± 0.125 & 0.253 ± 0.082 & 0.424 ± 0.060 & 0.463 ± 0.003 & 0.454 ± 0.021 & 0.306 ± 0.074 & 0.460 ± 0.028 & 0.241 ± 0.040 \\
\midrule
MLE & IPW & 0.075 ± 0.000 & 0.071 ± 0.001 & 0.229 ± 0.001 & 0.167 ± 0.002 & 0.565 ± 0.005 & 0.443 ± 0.007 & 0.415 ± 0.000 & 0.311 ± 0.005 & 0.343 ± 0.000 & 0.280 ± 0.001 \\
MLE & OR & 0.065 ± 0.002 & 0.061 ± 0.001 & 0.200 ± 0.007 & 0.143 ± 0.001 & 0.526 ± 0.011 & 0.399 ± 0.023 & 0.360 ± 0.003 & 0.256 ± 0.012 & 0.328 ± 0.003 & 0.266 ± 0.005 \\
MLE & DR & 0.149 ± 0.019 & 0.145 ± 0.010 & 0.243 ± 0.004 & 0.214 ± 0.019 & 0.568 ± 0.005 & 0.464 ± 0.014 & 0.403 ± 0.014 & 0.309 ± 0.012 & 0.365 ± 0.007 & 0.320 ± 0.004 \\
\midrule
EM & IPW & 0.097 ± 0.008 & 0.092 ± 0.004 & 0.239 ± 0.007 & 0.179 ± 0.003 & 0.478 ± 0.012 & 0.329 ± 0.020 & 0.262 ± 0.016 & 0.202 ± 0.003 & 0.312 ± 0.002 & 0.227 ± 0.001 \\
EM & OR & 0.121 ± 0.007 & 0.108 ± 0.005 & 0.261 ± 0.007 & 0.189 ± 0.004 & 0.489 ± 0.013 & 0.335 ± 0.020 & 0.274 ± 0.016 & 0.211 ± 0.004 & 0.336 ± 0.003 & 0.235 ± 0.001 \\
EM & DR & 0.125 ± 0.005 & 0.111 ± 0.004 & 0.269 ± 0.007 & 0.194 ± 0.005 & 0.497 ± 0.010 & 0.336 ± 0.024 & 0.281 ± 0.019 & 0.219 ± 0.008 & 0.336 ± 0.007 & 0.233 ± 0.004 \\
\midrule
SimPro & IPW & 0.125 ± 0.001 & 0.100 ± 0.005 & 0.166 ± 0.007 & 0.141 ± 0.009 & 0.353 ± 0.023 & 0.261 ± 0.008 & 0.202 ± 0.003 & 0.158 ± 0.005 & 0.277 ± 0.009 & 0.197 ± 0.003 \\
SimPro & OR & 0.133 ± 0.005 & 0.100 ± 0.004 & 0.160 ± 0.007 & 0.138 ± 0.010 & 0.322 ± 0.014 & 0.253 ± 0.008 & 0.202 ± 0.003 & 0.156 ± 0.005 & 0.269 ± 0.006 & 0.191 ± 0.004 \\
SimPro & DR & 0.122 ± 0.003 & 0.106 ± 0.006 & 0.188 ± 0.009 & 0.149 ± 0.006 & 0.343 ± 0.023 & 0.257 ± 0.007 & 0.219 ± 0.010 & 0.172 ± 0.002 & 0.279 ± 0.007 & 0.198 ± 0.004 \\
\bottomrule
\end{tabular}
}
\end{table*}

% WARNING: do not forget to delete the supplementary pages from your submission 
% 
\clearpage
% \setcounter{page}{1}
% \maketitlesupplementary
\begin{center}
Supplementary Material
\end{center}

% {
%     \onecolumn
%     \centering
%     \Large
%     \textbf{\thetitle}\\
%     \vspace{0.5em}Supplementary Material \\
%     \vspace{1.0em}
% }

\section{Proof of \cref{theorem:dr}}
We require some additional regularity assumptions:
\begin{assumption} 1) The number of classes $C$ is bounded w.r.t the number of samples $N$, 2) the missingness mechanism $P(A=1|Y,\theta)$, as well as its estimated counterpart $P(A=1|Y,\theta)$, are bounded below by some constant $\epsilon > 0$, 3) the quantities $P(Y|X,\theta)$ and $P(A|Y,\theta)$ are estimated using auxiliary samples independent of samples used for the sample averaging.
\label{assumption:extra}
\end{assumption}
Assumptions 1 and 2 are natural. For the missingness mechanism, the ground truth being bounded means that there is a non-vanishing proportion of samples for every class. The boundedness of the estimate can be enforced by clipping the estimate. Assumption 3 is called sample splitting in \cite{kennedy-dr}.

For convenience we use operator $\E_N$ to denote the average of $N$ samples i.e. $\frac{1}{N}\sum_{i=1}^N$. Note that this is by itself a random variable, in contrast to $\E$ which is a fixed number.

\begin{proof}[Proof of \cref{theorem:dr}] Because $C$ is bounded (assumption \ref{assumption:extra}), we can fix a class $c$ and prove the theorem.
Let us define the influence function $\phi$, parameterized by $\theta$, as
\begin{equation}
\phi(O | \theta)(c) = P(Y=c|X,\theta) + \frac{\one(A=1)}{P(A=1|Y,\theta)} (\one(Y=c) - P(Y=c|X,\theta)) - P(Y=c)
\end{equation}
As we have done in the main text, we use $\phi(O)$ to denote the same function but all estimated quantities are replaced with their truths. In other words, we use $\phi(O)$ for $\phi(O|\theta_0)$ where $\theta_0$ is the truth, given that our model contains $\theta_0$ e.g. when the model is consistent.

Recall that:
\begin{equation}
\begin{aligned}
\Psi_{dr}(\theta)(c) &= \frac{1}{N}\sum_{i=1}^N \left\{P(Y=c|X,\theta) + \frac{\one(A=1)}{P(A=1|Y,\theta)} (\one(Y=c) - P(Y=c|X,\theta))\right\}\\
&= \E_N [\phi(O|\theta)(c)] + P(Y=c)
\end{aligned}
\end{equation}

We will show that:
\begin{equation}
\Psi_{dr}(\theta)(c) - P(Y=c) = (\E_N - \E)[\phi(O)(c)] + o_P(N^{-1/2})
\label{eq:proof-linearity}
\end{equation}
To do that, we use the following decomposition
\begin{equation}
\begin{aligned}
\Psi_{dr}(\theta)(c) - P(Y=c) &= \E_N [\phi(O|\theta)(c)] \\
&= (\E_N - \E)[\phi(O)(c)] + (\E_N - \E)[\phi(O|\theta)(c) - \phi(O)(c)] + \E[\phi(O|\theta)(c)]
% &+ (\E_n - \E)[\phi(O;\theta) - \phi(O)]\\
% &+ \E[P(Y=c|X,\theta)] - \E[P(Y=c|X)] + \E[\phi(O,\theta)]
\end{aligned}
\end{equation}
and analyze the second and third term. The third term is:
\begin{equation}
\begin{aligned}
\E[\phi(O|\theta)(c)] &= \E[P(Y=c|X,\theta)] + \E\left[\frac{\one(A=1)}{P(A=1|Y,\theta)}(\one(Y=c) - P(Y=c|X,\theta))\right]- P(Y=c) \\
&= \E\left[P(Y=c|X,\theta) + \frac{P(A=1|Y)}{P(A=1|Y,\theta)}(P(Y=c|X) - P(Y=c|X,\theta))\right] - \E[P(Y=c|X)]\\
&= \E\left[(P(Y=c|X,\theta) - P(Y=c|X)) (P(A=1|Y,\theta) -P(A=1|Y)) \frac{1}{P(A=1|Y,\theta)}\right]\\
\end{aligned}
\end{equation}
by Cauchy-Schwarz inequality:
\begin{equation}
\begin{aligned}
\E[\phi(O|\theta)(c)] &\le \frac{1}{\epsilon} \|P(A=1|Y,\theta) - P(A=1|Y)\|_2 \|P(Y=c|X,\theta) - P(Y=c|X)\|_{L_2(P)}\\
&= \frac{1}{\epsilon} o_P(N^{-1/4} N^{-1/4}) = o_P(N^{-1/2})
\end{aligned}
\end{equation}
by assumption \ref{assumption:4th-root-n} and that $P(A=1|Y,\theta) > \epsilon$ (assumption \ref{assumption:extra}). The second term can be bounded by Chebyshev inequality
% \begin{equation}
% \begin{aligned}
% \E[\E_N[\phi(O|\theta)(c) - \phi(O)(c)]] &= \E[\phi(O|\theta)(c) - \phi(O)(c)]\\
% \var[\E_N[\phi(O|\theta)(c) - \phi(O)(c)]] &= \frac{1}{N}\var[\phi(O|\theta)(c) - \phi(O)(c)] \le 
% \end{aligned}
% \end{equation}
\begin{equation}
P(|(\E_N - \E)[\phi(O|\theta)(c) - \phi(O)(c)]| \ge t) \le \frac{\var[\E_N[\phi(O|\theta)(c) - \phi(O)(c)]]}{t^2} = \frac{\var[\phi(O|\theta)(c) - \phi(O)(c)]}{Nt^2}
\end{equation}
note here that $\theta$ is independent of the samples used for $\E_N$ by assumption \ref{assumption:extra}. For any $\varepsilon > 0$, by picking $t = \frac{1}{\sqrt{N\varepsilon}}$ we get
\begin{equation}
P\left(\left|\frac{(\E_N - \E)[\phi(O|\theta)(c) - \phi(O)(c)]}{N^{-1/2}}\right| \ge \frac{1}{\sqrt{\varepsilon}}\right) \le \varepsilon \var[\phi(O|\theta)(c) - \phi(O)(c)]
\end{equation}
by the definition of $O_P$, we then get
\begin{equation}
(\E_N - \E)[\phi(O|\theta)(c) - \phi(O)(c)] = O_P(N^{-1/2}\var[\phi(O|\theta)(c) - \phi(O)(c)])
\end{equation}
Because $\phi$ is a continuous function of $P(Y|X,\theta)$ and $P(A|Y,\theta)$ (given $P(A|Y,\theta) > \epsilon$, assumption \ref{assumption:extra}), by the continuous mapping theorem and the fact that $P(Y|X,\theta)$ and $P(A|Y,\theta)$ are convergent in probability (assumption \ref{assumption:4th-root-n}), we get $\var[\phi(O|\theta)(c) - \phi(O)(c)] = o_P(1)$. This gives
\begin{equation}
(\E_N - \E)[\phi(O|\theta)(c) - \phi(O)(c)] = o_P(N^{-1/2})
\end{equation}
Therefore, we have shown that the second and third term are both $o_P(N^{-1/2})$, proving \cref{eq:proof-linearity}. As the final step, multiply both sides of this equation by $\sqrt{N}$ we get:
\begin{equation}
\sqrt{N}(\Psi_{dr}(\theta)(c) - P(Y=c)) = \sqrt{N} (\E_N - \E)[\phi(O)(c)] + o_P(1) \rightsquigarrow \mathcal{N}(0, \var[\phi(O)(c)])
\end{equation}
by the central limit theorem, and $\var[\phi(O)(c)] = \E[\phi(O)(c)^2]$ because $\E[\phi(O)(c)] = 0$.
\end{proof}

While we started with the definition of $\phi$, \cref{eq:proof-linearity} shows that $\phi$ is indeed an influence function. Now we show that $\phi$ is also the efficient influence function, by using the characterization of the model's tangent space \cite{tsiatis-missingdata}. Note that the joint probability factorizes as $P(X,A,Y) = P(X)P(Y|X)P(A|Y)$, therefore the tangent space $\mathcal{T}$ factorizes as $\mathcal{T} = \mathcal{T}_{X} \oplus \mathcal{T}_{Y|X} \oplus \mathcal{T}_{A|Y}$ where $\mathcal{T}_X = \{h(X): \E[h] = 0\}$, $\mathcal{T}_{Y|X} = \{h(X,Y): \E[h|X] = 0\}$, $\mathcal{T}_{A|Y} = \{h(A,Y): \E[h|Y] = 0\}$, and the 3 subspaces are pairwise orthogonal. All influence functions are orthogonal to the tangent space, but the influence function that is also in the tangent space has the smallest variance and is called the efficient influence function. As $\phi$ is already an influence function, we need only show that $\phi$ is in $\mathcal{T}$. We write $\phi$ as
\begin{equation}
\phi(O)(c) = (P(Y=c|X) - P(Y=c)) + \left[\frac{\one(A=1)}{P(A=1|Y)} - 1\right](\one(Y=c) - P(Y=c|X)) + (\one(Y=c) - P(Y=c|X))
\end{equation}
and note that the first, second and third term are in $\mathcal{T}_X$, $\mathcal{T}_{A|Y}$ and $\mathcal{T}_{Y|X}$ respectively. Therefore, $\phi$ is indeed in $\mathcal{T}$. The efficient influence function has the smallest variance of all influence function, and therefore our estimator being asymptotically linear in $\phi$ (\cref{eq:proof-linearity}) has the smallest mean squared error in a local asymptotic minimax sense \cite{kennedy-dr, asymptoticstatistics}

\section{Further background and related work}
\paragraph{Discussion on semi-supervised EM.}
It appears that semi-supervised EM was first used for parameter estimation when the missingness mechanism is non-ignorable in \cite{ibrahim1996parameter}, but has not been used for label shift estimation.
Perhaps this is because the semi-supervised situation where additional unlabeled data is available during training is rarer than the test-time adaptation case. EM is well suited to take advantage of the extra unlabeled data to improve the classifier under very scarce and long-tailed labeled data. While the connection between pseudo-labeling and EM has been explored before \cite{entropyminimization}, the situation with label shift has not until recently \cite{simpro}. Here the application of EM is much more interesting, because other than simply giving pseudo-labeling a rigorous formulation, EM also estimates the missingness mechanism (equivalently the label distribution shift), which is important for shift correction and thus high-quality pseudo-labels \cite{acr}. The application of confidence thresholding can be seen as a sparse variant of EM \cite{neal1998view}.

\paragraph{The doubly-robust risk.} 
\label{subsec:dr-risk}
A technique that also derives from the theory of semi-parametric efficiency is orthogonal statistical learning \citep{foster2023orthogonal}. The idea is to minimize the doubly-robust risk:
\label{subsec:method-dr-risk}
\begin{equation}
\label{eq:dr-risk}
\mathcal{R}(\theta_2) = \frac{1}{N} \sum_{i=1}^N \Bigg[ l(x_i, \hat y_i|\theta_2) + \frac{\one(a_i=1)}{P(A=a_i|Y=y_i, \theta_1)} (l(x_i, y_i | \theta_2) - l(x_i, \hat y_i | \theta_2))\Bigg]
\end{equation}
where $l(x,y|\theta) = -\sum_{c=1}^C [y]_c \log P(Y=c|X=x,\theta)$ is the negative cross-entropy. 
The notation $[y]_c$ means that we are using the $c$-entry in a C-dimension probability vector $y$. 
Thus, $y_i$ denotes the one-hot label of observation $i$, while $\hat y_i$ denotes the pseudo-label, which can be one-hot or all-zero. 
Finally, we use $\theta_1$ to denote that $P(a|y,\theta_1)$ is an estimation from a previous stage, but it can be estimated with $\theta_2$ as well. 
The risk $\mathcal{R}(\theta_2)$ can be used as a training loss in a straightforward fashion. 
Similar to the doubly robust estimation of $P(Y)$, the doubly robust risk provides approximately unbiased estimation of the risk. 
This property has been used in \citep{arelabelsinformative, onnonrandommissinglabels, drst} also in the semi-supervised learning setting.
More broadly, it is at the heart of one of the core techniques in heterogenous treatment effect estimation in causal estimation \cite{kennedy2023towards, foster2023orthogonal, wager2018estimation}. 
The focus here is not the estimation of $\mathcal{R}(\theta_2)$ per se, but the quality of the learned model \cite{foster2023orthogonal}.
By using the doubly-robust risk, we can achieve an optimality result similar in spirit to our theorem \cref{theorem:dr}, but for the generalization error.
While this is appealing, in practice there are 2 problems with this approach. First, the inverse probability weight $P(A=a_i|Y=y_i,\theta_1)$ can be very large if the class ratio is highly unlabeled, making training unstable \cite{kallus2020deepmatch, pham2023stable}. 
This problem exists for our estimation as well. However, it is much easier to control for estimation than for training because of the iterative nature of model update. Secondly, we can further write $\mathcal{R}$ as:
\begin{equation}
\mathcal{R}(\theta_2) = \frac{1}{N}\sum_{i=1}^N l\left(x_i, \hat y_i + \frac{\one(a_i=1)}{P(A=a_i|Y=y_i,\theta_1)} (y_i - \hat y_i)\Bigg\vert\theta_2\right)
\end{equation}
which is a cross-entropy loss with new meta-pseudo-labels. However, these labels are not meant to be learned exactly, and furthermore they can be negative. Thus, theoretical works have to put stringent assumptions on the models. In \cref{subsec:ablation-1}, we show that experimentally that the instability problem makes doubly-robust risk performance worse than our 2-stage approach.

\section{Training and hyperparameter settings.}
\label{subsec:training-setting}
For neural network training, we follow the implementation and hyperparameter settings of \cite{simpro}. In particular, we adapt the core code of SimPro for Supervised, MLE and EM. For MLE, we update $P(A|Y)$ using the Adam optimizer with learning rate 1e-3, while for EM we use a momentum update similar to SimPro's update of $P(Y|A)$ because it has a a closed-form solution at each mini-batch. We use Wide ResNet-28-2 on all methods and all datasets in this section, including Imagenet-127, because we are motivated by the fact that stage-1's goal is not classification accuracy but the estimation of a finite-dimensional parameter. When using Wide ResNet-28-2 for Imagenet-127, we use the hyperparameters of CIFAR-100, except we lower the batch size of unlabeled data to 2 times that of labeled data instead of 8 for memory reason. We do not perform additional hyperparameter tuning. All experiments can be performed on 1 A6000 RTX GPU, and are run 3 times. We report the total variation distance between the estimated and the ground truth unlabeled class distribution, similar to its usage in Theorem 3.1 of \cite{lsc}, and the top-1 classification accuracy.

In the second stage of our algorithm, we freeze our estimation and plug it in SimPro and BOAT.
We keep exactly the same hyperparameter settings that SimPro and BOAT use. In particular, for Imagenet-127, we now use ResNet-50 and run each experiment once.
In SimPro, we set the unlabeled class distribution $P(Y|A=0)$ at the E-step;  however, we still keep a running estimate of the class distribution $P(Y)$ in the logit adjustment loss \cref{eq:simpro-la-loss}. While it is possible to use the first stage estimate in the logit adjustment loss, we observe that doing so results in lower accuracy than using the the running average. This is conceptually consistent with the role of the running average - serving not as an accurate estimate of $P(Y)$ but to make the classifier's class distribution uniform through the logit adjustment loss, which is good for the test set. Similarly, in BOAT, we only replace $\Delta_c = \log P(Y|A=1) - \log P(Y|A=0)$ in equation (4) of \cite{boat}, which is adjusting a classifier's predictions from the labeled to the unlabeled class distribution, with our SimPro + DR estimate instead of their on-the-fly estimate. 


% \section{Additional experiments}
% % \begin{table*}[t]
\centering
\caption{Total Variation Distance on CIFAR-10-LT ($N_l = 500$, $M_l = 4000$) with different class imbalance ratios $\gamma_l$ and $\gamma_u$ under five different unlabeled class distributions.}
\label{tab:cifar10-tv}
\resizebox{\textwidth}{!}{
\begin{tabular}{lccccccccccc}
\toprule
& & \multicolumn{2}{c}{consistent} & \multicolumn{2}{c}{uniform} & \multicolumn{2}{c}{reversed} & \multicolumn{2}{c}{middle} & \multicolumn{2}{c}{head-tail} \\
\cmidrule(lr){3-4} \cmidrule(lr){5-6} \cmidrule(lr){7-8} \cmidrule(lr){9-10} \cmidrule(lr){11-12}
& & $\gamma_l = 150$ & $\gamma_l = 100$ & $\gamma_l = 150$ & $\gamma_l = 100$ & $\gamma_l = 150$ & $\gamma_l = 100$ & $\gamma_l = 150$ & $\gamma_l = 100$ & $\gamma_l = 150$ & $\gamma_l = 100$ \\
Model & Estimator & $\gamma_u = 150$ & $\gamma_u = 100$ & $\gamma_u = 1$ & $\gamma_u = 1$ & $\gamma_u = 1/150$ & $\gamma_u = 1/100$ & $\gamma_u = 150$ & $\gamma_u = 100$ & $\gamma_u = 150$ & $\gamma_u = 100$ \\
\midrule
Supervised & MLLS & 0.269 ± 0.252 & 0.038 ± 0.006 & 0.251 ± 0.046 & 0.255 ± 0.060 & 0.429 ± 0.028 & 0.493 ± 0.050 & 0.333 ± 0.042 & 0.320 ± 0.009 & 0.457 ± 0.034 & 0.444 ± 0.043 \\
Supervised & RLLS & 0.043 ± 0.001 & 0.044 ± 0.010 & 0.348 ± 0.034 & 0.305 ± 0.068 & 0.769 ± 0.016 & 0.678 ± 0.028 & 0.430 ± 0.008 & 0.368 ± 0.013 & 0.539 ± 0.018 & 0.503 ± 0.020 \\
\midrule
MLE & IPW & 0.027 ± 0.001 & 0.027 ± 0.000 & 0.319 ± 0.072 & 0.243 ± 0.010 & 0.674 ± 0.020 & 0.646 ± 0.041 & 0.438 ± 0.020 & 0.454 ± 0.026 & 0.547 ± 0.049 & 0.491 ± 0.059 \\
MLE & OR & 0.045 ± 0.004 & 0.042 ± 0.000 & 0.215 ± 0.026 & 0.203 ± 0.032 & 0.433 ± 0.017 & 0.395 ± 0.033 & 0.193 ± 0.006 & 0.209 ± 0.037 & 0.307 ± 0.147 & 0.249 ± 0.130 \\
MLE & DR & 0.090 ± 0.002 & 0.079 ± 0.000 & 0.407 ± 0.027 & 0.360 ± 0.007 & 0.425 ± 0.007 & 0.421 ± 0.029 & 0.256 ± 0.001 & 0.286 ± 0.031 & 0.435 ± 0.136 & 0.362 ± 0.122 \\
\midrule
EM & IPW & 0.035 ± 0.002 & 0.040 ± 0.001 & 0.021 ± 0.001 & 0.029 ± 0.015 & 0.303 ± 0.187 & 0.091 ± 0.010 & 0.119 ± 0.011 & 0.105 ± 0.022 & 0.104 ± 0.026 & 0.104 ± 0.051 \\
EM & OR & 0.037 ± 0.003 & 0.042 ± 0.002 & 0.016 ± 0.001 & 0.024 ± 0.012 & 0.269 ± 0.183 & 0.090 ± 0.008 & 0.122 ± 0.012 & 0.103 ± 0.022 & 0.072 ± 0.012 & 0.073 ± 0.024 \\
EM & DR & 0.034 ± 0.004 & 0.037 ± 0.001 & 0.014 ± 0.001 & 0.027 ± 0.020 & 0.264 ± 0.191 & 0.092 ± 0.005 & 0.111 ± 0.019 & 0.097 ± 0.026 & 0.077 ± 0.016 & 0.073 ± 0.028 \\
\midrule
SimPro & IPW & 0.070 ± 0.011 & 0.058 ± 0.000 & 0.046 ± 0.001 & 0.049 ± 0.005 & 0.254 ± 0.074 & 0.223 ± 0.098 & 0.097 ± 0.025 & 0.067 ± 0.002 & 0.105 ± 0.066 & 0.110 ± 0.079 \\
SimPro & OR & 0.071 ± 0.012 & 0.058 ± 0.000 & 0.045 ± 0.001 & 0.049 ± 0.006 & 0.040 ± 0.003 & 0.059 ± 0.017 & 0.074 ± 0.006 & 0.075 ± 0.002 & 0.033 ± 0.003 & 0.033 ± 0.003 \\
SimPro & DR & 0.017 ± 0.004 & 0.026 ± 0.001 & 0.019 ± 0.002 & 0.018 ± 0.003 & 0.039 ± 0.003 & 0.058 ± 0.025 & 0.091 ± 0.007 & 0.031 ± 0.001 & 0.015 ± 0.003 & 0.019 ± 0.007 \\
\bottomrule
\end{tabular}
}
\end{table*}
% 

\begin{table*}[t]
\centering
\caption{Total Variation Distance on CIFAR-100-LT ($N_l = 50$, $M_l = 400$) with different class imbalance ratios $\gamma_l$ and $\gamma_u$ under five different unlabeled class distributions.}
\label{tab:cifar100-tv}
\resizebox{\textwidth}{!}{
\begin{tabular}{lccccccccccc}
\toprule
& & \multicolumn{2}{c}{consistent} & \multicolumn{2}{c}{uniform} & \multicolumn{2}{c}{reversed} & \multicolumn{2}{c}{middle} & \multicolumn{2}{c}{head-tail} \\
\cmidrule(lr){3-4} \cmidrule(lr){5-6} \cmidrule(lr){7-8} \cmidrule(lr){9-10} \cmidrule(lr){11-12}
& & $\gamma_l = 20$ & $\gamma_l = 10$ & $\gamma_l = 20$ & $\gamma_l = 10$ & $\gamma_l = 20$ & $\gamma_l = 10$ & $\gamma_l = 20$ & $\gamma_l = 10$ & $\gamma_l = 20$ & $\gamma_l = 10$ \\
Model & Estimator & $\gamma_u = 20$ & $\gamma_u = 10$ & $\gamma_u = 1$ & $\gamma_u = 1$ & $\gamma_u = 1/20$ & $\gamma_u = 1/10$ & $\gamma_u = 20$ & $\gamma_u = 10$ & $\gamma_u = 20$ & $\gamma_u = 10$ \\
\midrule
Supervised & MLLS & 0.707 ± 0.016 & 0.313 ± 0.100 & 0.445 ± 0.172 & 0.309 ± 0.119 & 0.383 ± 0.075 & 0.397 ± 0.006 & 0.570 ± 0.001 & 0.373 ± 0.107 & 0.543 ± 0.009 & 0.231 ± 0.057 \\
Supervised & RLLS & 0.520 ± 0.007 & 0.133 ± 0.003 & 0.337 ± 0.125 & 0.253 ± 0.082 & 0.424 ± 0.060 & 0.463 ± 0.003 & 0.454 ± 0.021 & 0.306 ± 0.074 & 0.460 ± 0.028 & 0.241 ± 0.040 \\
\midrule
MLE & IPW & 0.075 ± 0.000 & 0.071 ± 0.001 & 0.229 ± 0.001 & 0.167 ± 0.002 & 0.565 ± 0.005 & 0.443 ± 0.007 & 0.415 ± 0.000 & 0.311 ± 0.005 & 0.343 ± 0.000 & 0.280 ± 0.001 \\
MLE & OR & 0.065 ± 0.002 & 0.061 ± 0.001 & 0.200 ± 0.007 & 0.143 ± 0.001 & 0.526 ± 0.011 & 0.399 ± 0.023 & 0.360 ± 0.003 & 0.256 ± 0.012 & 0.328 ± 0.003 & 0.266 ± 0.005 \\
MLE & DR & 0.149 ± 0.019 & 0.145 ± 0.010 & 0.243 ± 0.004 & 0.214 ± 0.019 & 0.568 ± 0.005 & 0.464 ± 0.014 & 0.403 ± 0.014 & 0.309 ± 0.012 & 0.365 ± 0.007 & 0.320 ± 0.004 \\
\midrule
EM & IPW & 0.097 ± 0.008 & 0.092 ± 0.004 & 0.239 ± 0.007 & 0.179 ± 0.003 & 0.478 ± 0.012 & 0.329 ± 0.020 & 0.262 ± 0.016 & 0.202 ± 0.003 & 0.312 ± 0.002 & 0.227 ± 0.001 \\
EM & OR & 0.121 ± 0.007 & 0.108 ± 0.005 & 0.261 ± 0.007 & 0.189 ± 0.004 & 0.489 ± 0.013 & 0.335 ± 0.020 & 0.274 ± 0.016 & 0.211 ± 0.004 & 0.336 ± 0.003 & 0.235 ± 0.001 \\
EM & DR & 0.125 ± 0.005 & 0.111 ± 0.004 & 0.269 ± 0.007 & 0.194 ± 0.005 & 0.497 ± 0.010 & 0.336 ± 0.024 & 0.281 ± 0.019 & 0.219 ± 0.008 & 0.336 ± 0.007 & 0.233 ± 0.004 \\
\midrule
SimPro & IPW & 0.125 ± 0.001 & 0.100 ± 0.005 & 0.166 ± 0.007 & 0.141 ± 0.009 & 0.353 ± 0.023 & 0.261 ± 0.008 & 0.202 ± 0.003 & 0.158 ± 0.005 & 0.277 ± 0.009 & 0.197 ± 0.003 \\
SimPro & OR & 0.133 ± 0.005 & 0.100 ± 0.004 & 0.160 ± 0.007 & 0.138 ± 0.010 & 0.322 ± 0.014 & 0.253 ± 0.008 & 0.202 ± 0.003 & 0.156 ± 0.005 & 0.269 ± 0.006 & 0.191 ± 0.004 \\
SimPro & DR & 0.122 ± 0.003 & 0.106 ± 0.006 & 0.188 ± 0.009 & 0.149 ± 0.006 & 0.343 ± 0.023 & 0.257 ± 0.007 & 0.219 ± 0.010 & 0.172 ± 0.002 & 0.279 ± 0.007 & 0.198 ± 0.004 \\
\bottomrule
\end{tabular}
}
\end{table*}

\end{document}


