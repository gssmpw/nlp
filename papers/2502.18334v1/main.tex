%%%%%%%% ICML 2025 EXAMPLE LATEX SUBMISSION FILE %%%%%%%%%%%%%%%%%

\documentclass{article}
\pdfoutput=1
% Recommended, but optional, packages for figures and better typesetting:
\usepackage{microtype}
\usepackage{graphicx}
% \usepackage{subfigure}
\usepackage{booktabs} % for professional tables
\usepackage{enumitem}
\usepackage{subfig}
\usepackage[inkscapelatex=false]{svg}
\usepackage{multirow}
\usepackage{colortbl} % For coloring tables
\usepackage{xcolor}   % For color management
\usepackage{booktabs} % For professional table formatting
\usepackage{adjustbox} % Optional for advanced resizing
\usepackage{float}
\usepackage{xspace}
\definecolor{rowcolor1}{HTML}{FBE4E4} % Light pink
\newcommand{\INDSTATE}[1][1]{\STATE\hspace{#1em}}
\newcolumntype{C}{>{\cellcolor{red!10}}c}
% \usepackage{algpseudocode}

% hyperref makes hyperlinks in the resulting PDF.
% If your build breaks (sometimes temporarily if a hyperlink spans a page)
% please comment out the following usepackage line and replace
% \usepackage{icml2025} with \usepackage[nohyperref]{icml2025} above.
\usepackage{hyperref}
%%%%% NEW MATH DEFINITIONS %%%%%

\usepackage{amsmath,amsfonts,bm}
\usepackage{derivative}
% Mark sections of captions for referring to divisions of figures
\newcommand{\figleft}{{\em (Left)}}
\newcommand{\figcenter}{{\em (Center)}}
\newcommand{\figright}{{\em (Right)}}
\newcommand{\figtop}{{\em (Top)}}
\newcommand{\figbottom}{{\em (Bottom)}}
\newcommand{\captiona}{{\em (a)}}
\newcommand{\captionb}{{\em (b)}}
\newcommand{\captionc}{{\em (c)}}
\newcommand{\captiond}{{\em (d)}}

% Highlight a newly defined term
\newcommand{\newterm}[1]{{\bf #1}}

% Derivative d 
\newcommand{\deriv}{{\mathrm{d}}}

% Figure reference, lower-case.
\def\figref#1{figure~\ref{#1}}
% Figure reference, capital. For start of sentence
\def\Figref#1{Figure~\ref{#1}}
\def\twofigref#1#2{figures \ref{#1} and \ref{#2}}
\def\quadfigref#1#2#3#4{figures \ref{#1}, \ref{#2}, \ref{#3} and \ref{#4}}
% Section reference, lower-case.
\def\secref#1{section~\ref{#1}}
% Section reference, capital.
\def\Secref#1{Section~\ref{#1}}
% Reference to two sections.
\def\twosecrefs#1#2{sections \ref{#1} and \ref{#2}}
% Reference to three sections.
\def\secrefs#1#2#3{sections \ref{#1}, \ref{#2} and \ref{#3}}
% Reference to an equation, lower-case.
\def\eqref#1{equation~\ref{#1}}
% Reference to an equation, upper case
\def\Eqref#1{Equation~\ref{#1}}
% A raw reference to an equation---avoid using if possible
\def\plaineqref#1{\ref{#1}}
% Reference to a chapter, lower-case.
\def\chapref#1{chapter~\ref{#1}}
% Reference to an equation, upper case.
\def\Chapref#1{Chapter~\ref{#1}}
% Reference to a range of chapters
\def\rangechapref#1#2{chapters\ref{#1}--\ref{#2}}
% Reference to an algorithm, lower-case.
\def\algref#1{algorithm~\ref{#1}}
% Reference to an algorithm, upper case.
\def\Algref#1{Algorithm~\ref{#1}}
\def\twoalgref#1#2{algorithms \ref{#1} and \ref{#2}}
\def\Twoalgref#1#2{Algorithms \ref{#1} and \ref{#2}}
% Reference to a part, lower case
\def\partref#1{part~\ref{#1}}
% Reference to a part, upper case
\def\Partref#1{Part~\ref{#1}}
\def\twopartref#1#2{parts \ref{#1} and \ref{#2}}

\def\ceil#1{\lceil #1 \rceil}
\def\floor#1{\lfloor #1 \rfloor}
\def\1{\bm{1}}
\newcommand{\train}{\mathcal{D}}
\newcommand{\valid}{\mathcal{D_{\mathrm{valid}}}}
\newcommand{\test}{\mathcal{D_{\mathrm{test}}}}

\def\eps{{\epsilon}}


% Random variables
\def\reta{{\textnormal{$\eta$}}}
\def\ra{{\textnormal{a}}}
\def\rb{{\textnormal{b}}}
\def\rc{{\textnormal{c}}}
\def\rd{{\textnormal{d}}}
\def\re{{\textnormal{e}}}
\def\rf{{\textnormal{f}}}
\def\rg{{\textnormal{g}}}
\def\rh{{\textnormal{h}}}
\def\ri{{\textnormal{i}}}
\def\rj{{\textnormal{j}}}
\def\rk{{\textnormal{k}}}
\def\rl{{\textnormal{l}}}
% rm is already a command, just don't name any random variables m
\def\rn{{\textnormal{n}}}
\def\ro{{\textnormal{o}}}
\def\rp{{\textnormal{p}}}
\def\rq{{\textnormal{q}}}
\def\rr{{\textnormal{r}}}
\def\rs{{\textnormal{s}}}
\def\rt{{\textnormal{t}}}
\def\ru{{\textnormal{u}}}
\def\rv{{\textnormal{v}}}
\def\rw{{\textnormal{w}}}
\def\rx{{\textnormal{x}}}
\def\ry{{\textnormal{y}}}
\def\rz{{\textnormal{z}}}

% Random vectors
\def\rvepsilon{{\mathbf{\epsilon}}}
\def\rvphi{{\mathbf{\phi}}}
\def\rvtheta{{\mathbf{\theta}}}
\def\rva{{\mathbf{a}}}
\def\rvb{{\mathbf{b}}}
\def\rvc{{\mathbf{c}}}
\def\rvd{{\mathbf{d}}}
\def\rve{{\mathbf{e}}}
\def\rvf{{\mathbf{f}}}
\def\rvg{{\mathbf{g}}}
\def\rvh{{\mathbf{h}}}
\def\rvu{{\mathbf{i}}}
\def\rvj{{\mathbf{j}}}
\def\rvk{{\mathbf{k}}}
\def\rvl{{\mathbf{l}}}
\def\rvm{{\mathbf{m}}}
\def\rvn{{\mathbf{n}}}
\def\rvo{{\mathbf{o}}}
\def\rvp{{\mathbf{p}}}
\def\rvq{{\mathbf{q}}}
\def\rvr{{\mathbf{r}}}
\def\rvs{{\mathbf{s}}}
\def\rvt{{\mathbf{t}}}
\def\rvu{{\mathbf{u}}}
\def\rvv{{\mathbf{v}}}
\def\rvw{{\mathbf{w}}}
\def\rvx{{\mathbf{x}}}
\def\rvy{{\mathbf{y}}}
\def\rvz{{\mathbf{z}}}

% Elements of random vectors
\def\erva{{\textnormal{a}}}
\def\ervb{{\textnormal{b}}}
\def\ervc{{\textnormal{c}}}
\def\ervd{{\textnormal{d}}}
\def\erve{{\textnormal{e}}}
\def\ervf{{\textnormal{f}}}
\def\ervg{{\textnormal{g}}}
\def\ervh{{\textnormal{h}}}
\def\ervi{{\textnormal{i}}}
\def\ervj{{\textnormal{j}}}
\def\ervk{{\textnormal{k}}}
\def\ervl{{\textnormal{l}}}
\def\ervm{{\textnormal{m}}}
\def\ervn{{\textnormal{n}}}
\def\ervo{{\textnormal{o}}}
\def\ervp{{\textnormal{p}}}
\def\ervq{{\textnormal{q}}}
\def\ervr{{\textnormal{r}}}
\def\ervs{{\textnormal{s}}}
\def\ervt{{\textnormal{t}}}
\def\ervu{{\textnormal{u}}}
\def\ervv{{\textnormal{v}}}
\def\ervw{{\textnormal{w}}}
\def\ervx{{\textnormal{x}}}
\def\ervy{{\textnormal{y}}}
\def\ervz{{\textnormal{z}}}

% Random matrices
\def\rmA{{\mathbf{A}}}
\def\rmB{{\mathbf{B}}}
\def\rmC{{\mathbf{C}}}
\def\rmD{{\mathbf{D}}}
\def\rmE{{\mathbf{E}}}
\def\rmF{{\mathbf{F}}}
\def\rmG{{\mathbf{G}}}
\def\rmH{{\mathbf{H}}}
\def\rmI{{\mathbf{I}}}
\def\rmJ{{\mathbf{J}}}
\def\rmK{{\mathbf{K}}}
\def\rmL{{\mathbf{L}}}
\def\rmM{{\mathbf{M}}}
\def\rmN{{\mathbf{N}}}
\def\rmO{{\mathbf{O}}}
\def\rmP{{\mathbf{P}}}
\def\rmQ{{\mathbf{Q}}}
\def\rmR{{\mathbf{R}}}
\def\rmS{{\mathbf{S}}}
\def\rmT{{\mathbf{T}}}
\def\rmU{{\mathbf{U}}}
\def\rmV{{\mathbf{V}}}
\def\rmW{{\mathbf{W}}}
\def\rmX{{\mathbf{X}}}
\def\rmY{{\mathbf{Y}}}
\def\rmZ{{\mathbf{Z}}}

% Elements of random matrices
\def\ermA{{\textnormal{A}}}
\def\ermB{{\textnormal{B}}}
\def\ermC{{\textnormal{C}}}
\def\ermD{{\textnormal{D}}}
\def\ermE{{\textnormal{E}}}
\def\ermF{{\textnormal{F}}}
\def\ermG{{\textnormal{G}}}
\def\ermH{{\textnormal{H}}}
\def\ermI{{\textnormal{I}}}
\def\ermJ{{\textnormal{J}}}
\def\ermK{{\textnormal{K}}}
\def\ermL{{\textnormal{L}}}
\def\ermM{{\textnormal{M}}}
\def\ermN{{\textnormal{N}}}
\def\ermO{{\textnormal{O}}}
\def\ermP{{\textnormal{P}}}
\def\ermQ{{\textnormal{Q}}}
\def\ermR{{\textnormal{R}}}
\def\ermS{{\textnormal{S}}}
\def\ermT{{\textnormal{T}}}
\def\ermU{{\textnormal{U}}}
\def\ermV{{\textnormal{V}}}
\def\ermW{{\textnormal{W}}}
\def\ermX{{\textnormal{X}}}
\def\ermY{{\textnormal{Y}}}
\def\ermZ{{\textnormal{Z}}}

% Vectors
\def\vzero{{\bm{0}}}
\def\vone{{\bm{1}}}
\def\vmu{{\bm{\mu}}}
\def\vtheta{{\bm{\theta}}}
\def\vphi{{\bm{\phi}}}
\def\va{{\bm{a}}}
\def\vb{{\bm{b}}}
\def\vc{{\bm{c}}}
\def\vd{{\bm{d}}}
\def\ve{{\bm{e}}}
\def\vf{{\bm{f}}}
\def\vg{{\bm{g}}}
\def\vh{{\bm{h}}}
\def\vi{{\bm{i}}}
\def\vj{{\bm{j}}}
\def\vk{{\bm{k}}}
\def\vl{{\bm{l}}}
\def\vm{{\bm{m}}}
\def\vn{{\bm{n}}}
\def\vo{{\bm{o}}}
\def\vp{{\bm{p}}}
\def\vq{{\bm{q}}}
\def\vr{{\bm{r}}}
\def\vs{{\bm{s}}}
\def\vt{{\bm{t}}}
\def\vu{{\bm{u}}}
\def\vv{{\bm{v}}}
\def\vw{{\bm{w}}}
\def\vx{{\bm{x}}}
\def\vy{{\bm{y}}}
\def\vz{{\bm{z}}}

% Elements of vectors
\def\evalpha{{\alpha}}
\def\evbeta{{\beta}}
\def\evepsilon{{\epsilon}}
\def\evlambda{{\lambda}}
\def\evomega{{\omega}}
\def\evmu{{\mu}}
\def\evpsi{{\psi}}
\def\evsigma{{\sigma}}
\def\evtheta{{\theta}}
\def\eva{{a}}
\def\evb{{b}}
\def\evc{{c}}
\def\evd{{d}}
\def\eve{{e}}
\def\evf{{f}}
\def\evg{{g}}
\def\evh{{h}}
\def\evi{{i}}
\def\evj{{j}}
\def\evk{{k}}
\def\evl{{l}}
\def\evm{{m}}
\def\evn{{n}}
\def\evo{{o}}
\def\evp{{p}}
\def\evq{{q}}
\def\evr{{r}}
\def\evs{{s}}
\def\evt{{t}}
\def\evu{{u}}
\def\evv{{v}}
\def\evw{{w}}
\def\evx{{x}}
\def\evy{{y}}
\def\evz{{z}}

% Matrix
\def\mA{{\bm{A}}}
\def\mB{{\bm{B}}}
\def\mC{{\bm{C}}}
\def\mD{{\bm{D}}}
\def\mE{{\bm{E}}}
\def\mF{{\bm{F}}}
\def\mG{{\bm{G}}}
\def\mH{{\bm{H}}}
\def\mI{{\bm{I}}}
\def\mJ{{\bm{J}}}
\def\mK{{\bm{K}}}
\def\mL{{\bm{L}}}
\def\mM{{\bm{M}}}
\def\mN{{\bm{N}}}
\def\mO{{\bm{O}}}
\def\mP{{\bm{P}}}
\def\mQ{{\bm{Q}}}
\def\mR{{\bm{R}}}
\def\mS{{\bm{S}}}
\def\mT{{\bm{T}}}
\def\mU{{\bm{U}}}
\def\mV{{\bm{V}}}
\def\mW{{\bm{W}}}
\def\mX{{\bm{X}}}
\def\mY{{\bm{Y}}}
\def\mZ{{\bm{Z}}}
\def\mBeta{{\bm{\beta}}}
\def\mPhi{{\bm{\Phi}}}
\def\mLambda{{\bm{\Lambda}}}
\def\mSigma{{\bm{\Sigma}}}

% Tensor
\DeclareMathAlphabet{\mathsfit}{\encodingdefault}{\sfdefault}{m}{sl}
\SetMathAlphabet{\mathsfit}{bold}{\encodingdefault}{\sfdefault}{bx}{n}
\newcommand{\tens}[1]{\bm{\mathsfit{#1}}}
\def\tA{{\tens{A}}}
\def\tB{{\tens{B}}}
\def\tC{{\tens{C}}}
\def\tD{{\tens{D}}}
\def\tE{{\tens{E}}}
\def\tF{{\tens{F}}}
\def\tG{{\tens{G}}}
\def\tH{{\tens{H}}}
\def\tI{{\tens{I}}}
\def\tJ{{\tens{J}}}
\def\tK{{\tens{K}}}
\def\tL{{\tens{L}}}
\def\tM{{\tens{M}}}
\def\tN{{\tens{N}}}
\def\tO{{\tens{O}}}
\def\tP{{\tens{P}}}
\def\tQ{{\tens{Q}}}
\def\tR{{\tens{R}}}
\def\tS{{\tens{S}}}
\def\tT{{\tens{T}}}
\def\tU{{\tens{U}}}
\def\tV{{\tens{V}}}
\def\tW{{\tens{W}}}
\def\tX{{\tens{X}}}
\def\tY{{\tens{Y}}}
\def\tZ{{\tens{Z}}}


% Graph
\def\gA{{\mathcal{A}}}
\def\gB{{\mathcal{B}}}
\def\gC{{\mathcal{C}}}
\def\gD{{\mathcal{D}}}
\def\gE{{\mathcal{E}}}
\def\gF{{\mathcal{F}}}
\def\gG{{\mathcal{G}}}
\def\gH{{\mathcal{H}}}
\def\gI{{\mathcal{I}}}
\def\gJ{{\mathcal{J}}}
\def\gK{{\mathcal{K}}}
\def\gL{{\mathcal{L}}}
\def\gM{{\mathcal{M}}}
\def\gN{{\mathcal{N}}}
\def\gO{{\mathcal{O}}}
\def\gP{{\mathcal{P}}}
\def\gQ{{\mathcal{Q}}}
\def\gR{{\mathcal{R}}}
\def\gS{{\mathcal{S}}}
\def\gT{{\mathcal{T}}}
\def\gU{{\mathcal{U}}}
\def\gV{{\mathcal{V}}}
\def\gW{{\mathcal{W}}}
\def\gX{{\mathcal{X}}}
\def\gY{{\mathcal{Y}}}
\def\gZ{{\mathcal{Z}}}

% Sets
\def\sA{{\mathbb{A}}}
\def\sB{{\mathbb{B}}}
\def\sC{{\mathbb{C}}}
\def\sD{{\mathbb{D}}}
% Don't use a set called E, because this would be the same as our symbol
% for expectation.
\def\sF{{\mathbb{F}}}
\def\sG{{\mathbb{G}}}
\def\sH{{\mathbb{H}}}
\def\sI{{\mathbb{I}}}
\def\sJ{{\mathbb{J}}}
\def\sK{{\mathbb{K}}}
\def\sL{{\mathbb{L}}}
\def\sM{{\mathbb{M}}}
\def\sN{{\mathbb{N}}}
\def\sO{{\mathbb{O}}}
\def\sP{{\mathbb{P}}}
\def\sQ{{\mathbb{Q}}}
\def\sR{{\mathbb{R}}}
\def\sS{{\mathbb{S}}}
\def\sT{{\mathbb{T}}}
\def\sU{{\mathbb{U}}}
\def\sV{{\mathbb{V}}}
\def\sW{{\mathbb{W}}}
\def\sX{{\mathbb{X}}}
\def\sY{{\mathbb{Y}}}
\def\sZ{{\mathbb{Z}}}

% Entries of a matrix
\def\emLambda{{\Lambda}}
\def\emA{{A}}
\def\emB{{B}}
\def\emC{{C}}
\def\emD{{D}}
\def\emE{{E}}
\def\emF{{F}}
\def\emG{{G}}
\def\emH{{H}}
\def\emI{{I}}
\def\emJ{{J}}
\def\emK{{K}}
\def\emL{{L}}
\def\emM{{M}}
\def\emN{{N}}
\def\emO{{O}}
\def\emP{{P}}
\def\emQ{{Q}}
\def\emR{{R}}
\def\emS{{S}}
\def\emT{{T}}
\def\emU{{U}}
\def\emV{{V}}
\def\emW{{W}}
\def\emX{{X}}
\def\emY{{Y}}
\def\emZ{{Z}}
\def\emSigma{{\Sigma}}

% entries of a tensor
% Same font as tensor, without \bm wrapper
\newcommand{\etens}[1]{\mathsfit{#1}}
\def\etLambda{{\etens{\Lambda}}}
\def\etA{{\etens{A}}}
\def\etB{{\etens{B}}}
\def\etC{{\etens{C}}}
\def\etD{{\etens{D}}}
\def\etE{{\etens{E}}}
\def\etF{{\etens{F}}}
\def\etG{{\etens{G}}}
\def\etH{{\etens{H}}}
\def\etI{{\etens{I}}}
\def\etJ{{\etens{J}}}
\def\etK{{\etens{K}}}
\def\etL{{\etens{L}}}
\def\etM{{\etens{M}}}
\def\etN{{\etens{N}}}
\def\etO{{\etens{O}}}
\def\etP{{\etens{P}}}
\def\etQ{{\etens{Q}}}
\def\etR{{\etens{R}}}
\def\etS{{\etens{S}}}
\def\etT{{\etens{T}}}
\def\etU{{\etens{U}}}
\def\etV{{\etens{V}}}
\def\etW{{\etens{W}}}
\def\etX{{\etens{X}}}
\def\etY{{\etens{Y}}}
\def\etZ{{\etens{Z}}}

% The true underlying data generating distribution
\newcommand{\pdata}{p_{\rm{data}}}
\newcommand{\ptarget}{p_{\rm{target}}}
\newcommand{\pprior}{p_{\rm{prior}}}
\newcommand{\pbase}{p_{\rm{base}}}
\newcommand{\pref}{p_{\rm{ref}}}

% The empirical distribution defined by the training set
\newcommand{\ptrain}{\hat{p}_{\rm{data}}}
\newcommand{\Ptrain}{\hat{P}_{\rm{data}}}
% The model distribution
\newcommand{\pmodel}{p_{\rm{model}}}
\newcommand{\Pmodel}{P_{\rm{model}}}
\newcommand{\ptildemodel}{\tilde{p}_{\rm{model}}}
% Stochastic autoencoder distributions
\newcommand{\pencode}{p_{\rm{encoder}}}
\newcommand{\pdecode}{p_{\rm{decoder}}}
\newcommand{\precons}{p_{\rm{reconstruct}}}

\newcommand{\laplace}{\mathrm{Laplace}} % Laplace distribution

\newcommand{\E}{\mathbb{E}}
\newcommand{\Ls}{\mathcal{L}}
\newcommand{\R}{\mathbb{R}}
\newcommand{\emp}{\tilde{p}}
\newcommand{\lr}{\alpha}
\newcommand{\reg}{\lambda}
\newcommand{\rect}{\mathrm{rectifier}}
\newcommand{\softmax}{\mathrm{softmax}}
\newcommand{\sigmoid}{\sigma}
\newcommand{\softplus}{\zeta}
\newcommand{\KL}{D_{\mathrm{KL}}}
\newcommand{\Var}{\mathrm{Var}}
\newcommand{\standarderror}{\mathrm{SE}}
\newcommand{\Cov}{\mathrm{Cov}}
% Wolfram Mathworld says $L^2$ is for function spaces and $\ell^2$ is for vectors
% But then they seem to use $L^2$ for vectors throughout the site, and so does
% wikipedia.
\newcommand{\normlzero}{L^0}
\newcommand{\normlone}{L^1}
\newcommand{\normltwo}{L^2}
\newcommand{\normlp}{L^p}
\newcommand{\normmax}{L^\infty}

\newcommand{\parents}{Pa} % See usage in notation.tex. Chosen to match Daphne's book.

\DeclareMathOperator*{\argmax}{arg\,max}
\DeclareMathOperator*{\argmin}{arg\,min}

\DeclareMathOperator{\sign}{sign}
\DeclareMathOperator{\Tr}{Tr}
\let\ab\allowbreak



% Attempt to make hyperref and algorithmic work together better:
\newcommand{\theHalgorithm}{\arabic{algorithm}}
\newcommand{\proj}{TSA\xspace}
% Use the following line for the initial blind version submitted for review:
\usepackage[preprint]{icml2025}

% If accepted, instead use the following line for the camera-ready submission:
% \usepackage[accepted]{icml2025}

% For theorems and such
\usepackage{amsmath}
\usepackage{amssymb}
\usepackage{mathtools}
\usepackage{amsthm}
\usepackage{thmtools, thm-restate}
\usepackage{subcaption} % For subfigures

% if you use cleveref..
\usepackage[capitalize,noabbrev]{cleveref}

%%%%%%%%%%%%%%%%%%%%%%%%%%%%%%%%
% THEOREMS
%%%%%%%%%%%%%%%%%%%%%%%%%%%%%%%%
\theoremstyle{plain}
\newtheorem{theorem}{Theorem}[section]
\newtheorem{proposition}[theorem]{Proposition}
\newtheorem{lemma}[theorem]{Lemma}
\newtheorem{corollary}[theorem]{Corollary}
\theoremstyle{definition}
\newtheorem{definition}[theorem]{Definition}
\newtheorem{assumption}[theorem]{Assumption}
\theoremstyle{remark}
\newtheorem{remark}[theorem]{Remark}
\newcommand{\pan}[1]{{\color{red}{PL: #1}}}
\newcommand{\han}[1]{{\color{magenta}{Han: #1}}}
\newcommand{\shikun}[1]{{\color{blue}{SL: #1}}}
\newcommand{\hans}[1]{{\color{brown}{HH: #1}}}
% Todonotes is useful during development; simply uncomment the next line
%    and comment out the line below the next line to turn off comments
%\usepackage[disable,textsize=tiny]{todonotes}
\usepackage[textsize=tiny]{todonotes}
\renewcommand{\paragraph}[1]{\textbf{#1}~~}


% The \icmltitle you define below is probably too long as a header.
% Therefore, a short form for the running title is supplied here:
% \icmltitlerunning{Submission and Formatting Instructions for ICML 2025}
\icmltitlerunning{Structural Alignment Improves Graph Test-Time Adaptation}

\begin{document}

\twocolumn[
\icmltitle {Structural Alignment Improves Graph Test-Time Adaptation}

% It is OKAY to include author information, even for blind
% submissions: the style file will automatically remove it for you
% unless you've provided the [accepted] option to the icml2025
% package.

% List of affiliations: The first argument should be a (short)
% identifier you will use later to specify author affiliations
% Academic affiliations should list Department, University, City, Region, Country
% Industry affiliations should list Company, City, Region, Country

% You can specify symbols, otherwise they are numbered in order.
% Ideally, you should not use this facility. Affiliations will be numbered
% in order of appearance and this is the preferred way.
\icmlsetsymbol{equal}{*}

\begin{icmlauthorlist}
\icmlauthor{Hans Hao-Hsun Hsu}{equal,tum}
\icmlauthor{Shikun Liu}{equal,gatech}
\icmlauthor{Han Zhao}{uiuc}
\icmlauthor{Pan Li}{gatech}
%\icmlauthor{}{sch}
%\icmlauthor{}{sch}
%\icmlauthor{}{sch}
\end{icmlauthorlist}

% \icmlaffiliation{tum}{School of Computation, Information and Technology, Technical University of Munich, Munich, Germany}
% \icmlaffiliation{gatech}{Department of Electrical and Computer Engineering, Georgia Institute of Technology, Georgia, USA}
% \icmlaffiliation{uiuc}{Department of Computer Science, University of Illinois Urbana-Champaign, Champaign, USA}
\icmlaffiliation{tum}{Technical University of Munich}
\icmlaffiliation{gatech}{Georgia Institute of Technology}
\icmlaffiliation{uiuc}{University of Illinois Urbana-Champaign}

\icmlcorrespondingauthor{Hans Hao-Hsun Hsu}{hans.hsu@tum.de}
\icmlcorrespondingauthor{Shikun Liu}{shikun.liu@gatech.edu}
\icmlcorrespondingauthor{Pan Li}{panli@gatech.edu}

% You may provide any keywords that you
% find helpful for describing your paper; these are used to populate
% the "keywords" metadata in the PDF but will not be shown in the document
\icmlkeywords{Machine Learning, ICML}

\vskip 0.3in
]

% this must go after the closing bracket ] following \twocolumn[ ...

% This command actually creates the footnote in the first column
% listing the affiliations and the copyright notice.
% The command takes one argument, which is text to display at the start of the footnote.
% The \icmlEqualContribution command is standard text for equal contribution.
% Remove it (just {}) if you do not need this facility.

% \printAffiliationsAndNotice{}  % leave blank if no need to mention equal contribution
\printAffiliationsAndNotice{\icmlEqualContribution} % otherwise use the standard text.
%%%%%%%%%%%%%%%%% Change icml2025.sty %%%%%%%%%%%%%%%%%

\begin{abstract}
Graph-based learning has achieved remarkable success in domains ranging from recommendation to fraud detection and particle physics by effectively capturing underlying interaction patterns. However, it often struggles to generalize when distribution shifts occur, particularly those involving changes in network connectivity or interaction patterns. Existing approaches designed to mitigate such shifts typically require retraining with full access to source data, rendering them infeasible under strict computational or privacy constraints. To address this limitation, we propose a test-time structural alignment (TSA) algorithm for Graph Test-Time Adaptation (GTTA), a novel method that aligns graph structures during inference without revisiting the source domain. Built upon a theoretically grounded treatment of graph data distribution shifts, TSA integrates three key strategies: an uncertainty-aware neighborhood weighting that accommodates structure shifts, an adaptive balancing of self-node and neighborhood-aggregated representations driven by node representations’ signal-to-noise ratio, and a decision boundary refinement that corrects remaining label and feature shifts. Extensive experiments on synthetic and real-world datasets demonstrate that TSA can consistently outperform both non-graph TTA methods and state-of-the-art GTTA baselines.
\end{abstract}



\section{Introduction}


\begin{figure}[t]
\centering
\includegraphics[width=0.6\columnwidth]{figures/evaluation_desiderata_V5.pdf}
\vspace{-0.5cm}
\caption{\systemName is a platform for conducting realistic evaluations of code LLMs, collecting human preferences of coding models with real users, real tasks, and in realistic environments, aimed at addressing the limitations of existing evaluations.
}
\label{fig:motivation}
\end{figure}

\begin{figure*}[t]
\centering
\includegraphics[width=\textwidth]{figures/system_design_v2.png}
\caption{We introduce \systemName, a VSCode extension to collect human preferences of code directly in a developer's IDE. \systemName enables developers to use code completions from various models. The system comprises a) the interface in the user's IDE which presents paired completions to users (left), b) a sampling strategy that picks model pairs to reduce latency (right, top), and c) a prompting scheme that allows diverse LLMs to perform code completions with high fidelity.
Users can select between the top completion (green box) using \texttt{tab} or the bottom completion (blue box) using \texttt{shift+tab}.}
\label{fig:overview}
\end{figure*}

As model capabilities improve, large language models (LLMs) are increasingly integrated into user environments and workflows.
For example, software developers code with AI in integrated developer environments (IDEs)~\citep{peng2023impact}, doctors rely on notes generated through ambient listening~\citep{oberst2024science}, and lawyers consider case evidence identified by electronic discovery systems~\citep{yang2024beyond}.
Increasing deployment of models in productivity tools demands evaluation that more closely reflects real-world circumstances~\citep{hutchinson2022evaluation, saxon2024benchmarks, kapoor2024ai}.
While newer benchmarks and live platforms incorporate human feedback to capture real-world usage, they almost exclusively focus on evaluating LLMs in chat conversations~\citep{zheng2023judging,dubois2023alpacafarm,chiang2024chatbot, kirk2024the}.
Model evaluation must move beyond chat-based interactions and into specialized user environments.



 

In this work, we focus on evaluating LLM-based coding assistants. 
Despite the popularity of these tools---millions of developers use Github Copilot~\citep{Copilot}---existing
evaluations of the coding capabilities of new models exhibit multiple limitations (Figure~\ref{fig:motivation}, bottom).
Traditional ML benchmarks evaluate LLM capabilities by measuring how well a model can complete static, interview-style coding tasks~\citep{chen2021evaluating,austin2021program,jain2024livecodebench, white2024livebench} and lack \emph{real users}. 
User studies recruit real users to evaluate the effectiveness of LLMs as coding assistants, but are often limited to simple programming tasks as opposed to \emph{real tasks}~\citep{vaithilingam2022expectation,ross2023programmer, mozannar2024realhumaneval}.
Recent efforts to collect human feedback such as Chatbot Arena~\citep{chiang2024chatbot} are still removed from a \emph{realistic environment}, resulting in users and data that deviate from typical software development processes.
We introduce \systemName to address these limitations (Figure~\ref{fig:motivation}, top), and we describe our three main contributions below.


\textbf{We deploy \systemName in-the-wild to collect human preferences on code.} 
\systemName is a Visual Studio Code extension, collecting preferences directly in a developer's IDE within their actual workflow (Figure~\ref{fig:overview}).
\systemName provides developers with code completions, akin to the type of support provided by Github Copilot~\citep{Copilot}. 
Over the past 3 months, \systemName has served over~\completions suggestions from 10 state-of-the-art LLMs, 
gathering \sampleCount~votes from \userCount~users.
To collect user preferences,
\systemName presents a novel interface that shows users paired code completions from two different LLMs, which are determined based on a sampling strategy that aims to 
mitigate latency while preserving coverage across model comparisons.
Additionally, we devise a prompting scheme that allows a diverse set of models to perform code completions with high fidelity.
See Section~\ref{sec:system} and Section~\ref{sec:deployment} for details about system design and deployment respectively.



\textbf{We construct a leaderboard of user preferences and find notable differences from existing static benchmarks and human preference leaderboards.}
In general, we observe that smaller models seem to overperform in static benchmarks compared to our leaderboard, while performance among larger models is mixed (Section~\ref{sec:leaderboard_calculation}).
We attribute these differences to the fact that \systemName is exposed to users and tasks that differ drastically from code evaluations in the past. 
Our data spans 103 programming languages and 24 natural languages as well as a variety of real-world applications and code structures, while static benchmarks tend to focus on a specific programming and natural language and task (e.g. coding competition problems).
Additionally, while all of \systemName interactions contain code contexts and the majority involve infilling tasks, a much smaller fraction of Chatbot Arena's coding tasks contain code context, with infilling tasks appearing even more rarely. 
We analyze our data in depth in Section~\ref{subsec:comparison}.



\textbf{We derive new insights into user preferences of code by analyzing \systemName's diverse and distinct data distribution.}
We compare user preferences across different stratifications of input data (e.g., common versus rare languages) and observe which affect observed preferences most (Section~\ref{sec:analysis}).
For example, while user preferences stay relatively consistent across various programming languages, they differ drastically between different task categories (e.g. frontend/backend versus algorithm design).
We also observe variations in user preference due to different features related to code structure 
(e.g., context length and completion patterns).
We open-source \systemName and release a curated subset of code contexts.
Altogether, our results highlight the necessity of model evaluation in realistic and domain-specific settings.






\section{Preliminaries and Related Works}

\subsection{Notations and Problem Setup}

We use upper-case letters, such as $Y$ to represent random variables, lower-case letters, such as $y$ to represent their realization.
The calligraphic letters, such as $\gY$ denote the domain of random variables.
We use bold capital such as $\rmY$ to represent the vectorized corresponds, i.e., a collections of random variables.
The probability distribution of a random variable $Y$ for a realization is expressed as $\prob(Y=y)$.

% CheatSheet
% \rx random variable x
% \rvx random vector x
% \ervx element of random vector x
% \rmX random matrix X
% \ermX element of random matrix X
% \vx vector x
% \evx elements of vector x
% \mX matrix X


% \paragraph{Test-Time Adaptation (TTA).}
% Assume the model consists of a feature encoder $\phi:\gX\rightarrow \gH$ and a a classifier $g: \gH \rightarrow \gY$ and each domain $\gU \in \{\gS,\gT\}$ consists of a joint feature and label distribution $\probu(X,Y)$.
% In TTA, the model is first trained on the source domain $\probs(X,Y)$. The objective is to adapt the model to minimize the target test error $\errort(g\circ\phi)=\probt(g(\phi(X))\neq Y)$ \emph{without} accessing the source domain.

\paragraph{Graph Neural Networks (GNNs).} 
We let $\gG = (\gV,\gE)$ denote an undirected and unweighted graph with the symmetric adjacency matrix $\mA\in\R^{N\times N}$ and the node feature matrix $\mX=[x_1,\dots,x_N]^T$.
GNNs utilize neighborhood information by encoding $\mA$ and $\mX$ into node representations $\{h_{v}^{(k)}, v \in \neighbor_u\}$.
With $h_u^{(1)}=x_u$, the message passing in standard GNNs for node $v$ and each layer $k \in [L] \coloneqq \{1,\dots,L\}$ can be written as
\begin{equation}
    h_u^{(k+1)} = \text{UPT}(h_u^{(k)}, \text{AGG}(\{h_{v}^{(k)}, v \in \neighbor_u\})) \label{eq:gnn}
\end{equation}
where $\neighbor_u$ denotes the set of neighbors of node $u$, which $|\neighbor_u|$ represent the node degree $d_u$.
The AGG function aggregates messages
from the neighbors, and the UPT function updates the node
representations. 

\paragraph{Graph Test-Time Adaptation (GTTA).} 

% Assume the model consists of a feature encoder $\phi:\gX\rightarrow \gH$ and a a classifier $g: \gH \rightarrow \gY$ and each domain $\gU \in \{\gS,\gT\}$ consists of a joint feature and label distribution $\probu(X,Y)$.
% In TTA, the model is first trained on the source domain $\probs(X,Y)$. The objective is to adapt the model to minimize the target test error $\errort(g\circ\phi)=\probt(g(\phi(X))\neq Y)$ \emph{without} accessing the source domain.

% We focus on the node classification tasks in GTTA. 
% Similar to TTA, the model is trained on source graph $\gG^{\gS}=(\gV^{\gS}, \gE^{\gS})$ and the goal is to enhance model performance on test graph $\gG^{\gT}=(\gV^{\gT}, \gE^{\gT})$.
% The encoder $\phi$ is a switched to graph based method such as a GNN.
% For node classification tasks, we aim to minimize the test error $\errort(g\circ\phi)=\probt(g(\phi(X_u, \rmA))\neq Y_u)$ \emph{without} accessing $\gG^{\gS}$.

Assume the model consists of a GNN feature encoder $\phi:\gX\rightarrow \gH$ and a classifier $g: \gH \rightarrow \gY$. The model is trained on source graph $\gG^{\gS}=(\gV^{\gS}, \gE^{\gS})$ with node labels $\mathbf{y}^\gS$ and the goal is to enhance model performance on test graph $\gG^{\gT}=(\gV^{\gT}, \gE^{\gT})$ with distribution shifts that will be defined in Sec.~\ref{sec:shift}. For node classification tasks, we aim to minimize the test error $\errort(g\circ\phi)=\probt(g(\phi(X_u, \rmA))\neq Y_u)$ \emph{without} accessing $\gG^{\gS}$.
%We use the prime $(')$ symbol to denote prediction from TTA and the hat $(\, \hat{}\,)$ symbol to denote the soft prediction.
%For example, $\hat{y}'$ represents the soft prediction from TTA.


\subsection{Related Works}
%TTA aims to adapt a pre-trained model from the source domain to unlabeled target domain without accessing the source domain during adaptation \cite{liang2024comprehensive}.
\textbf{Test-time Adaptation}
\emph{Test-time training} \cite{sun2020test, liu2021ttt, bartler2022mt3} adapts the source model to the target domain but requires to first add a customized self-supervised losses in model pretraining.
In contrast, our setup falls into the category of \emph{fully test-time adaptation} \cite{wang2020tent,liang2024comprehensive}, where we do not alter the model training pipeline.  %agnois arbitrary pretrained model. 
 Tent \cite{wang2020tent} adapts the batch normalization parameters by minimizing the entropy, which motivated following-up studies on normalization layer adaptation
 %Other works were motivated by this and perform adaptation in normalizing layers 
 ~\cite{gong2022note, zhao2023delta, lim2023ttn, niu2023towards}. 
%In addition to normalization layer adaptation, 
Some TTA works directly modify the classifier's prediction, such as
 LAME \cite{boudiaf2022parameter} that directly refines the model's soft prediction by applying a regularization %that enforces  
 %label consistency for neighboring points 
 in the feature space, 
T3A \cite{iwasawa2021test} that %updates the classifier's last layer using prototypes from pseudo-labels and 
classifies test data based on the distance to the pseudo-prototypes derived from pseudo-labels, 
TAST \cite{jang2022test} and PROGRAM \cite{sunprogram} that extend T3A through constructing more reliable %pseudolabel through  nearest-neighbor and 
prototype graphs.
% self-training a new classifier with nearest neighbor information and construct pseudo label through ensemble.
However, the above methods are designed for image-based applications and cannot handle the shifts in neighborhood information of graph data. %do not explicitly mitigate discrepancies in aggregated node labels induced by shifts in neighborhood information.



% \paragraph{Test-Time Adaptation on Label Shifts.}
% Label shift studies cases where the marginal label distributions differ $\probs(Y)\neq\probt(Y)$ but the conditional distributions given class remain fixed $\probs(X|Y)=\probt(X|Y)$. Based on the assumption, the prediction of the model in target domain can be adapted as $\hat{Y}=\argmax_{y\in\gY}g(\phi(X)) + \log (\frac{\probt(Y)}{\probs(Y)})$. Exiting works that are originally developed for domain adaptation are applicable to the TTA framework by having the model saved with source label distribution or the confusion matrix. 
% However the assumption of fixed class-conditional distributions do not hold on graphs, as structure shifts can can affect the aggregated node labels, thereby causing attribution shifts on node representations. \pan{this should be put in the intro to argue against the non-graph test-time works}

% \paragraph{Test-Time Adaptation on Graphs} 
\textbf{Graph Test-time Adaptation} Studies on GTTA are in two categories - node and graph classification.
% Existing work on graph classification 
Graph classification problems can treat each graph as an i.i.d.\ input, allowing more direct extension of image-based TTA techniques to graphs \cite{chen2022graphtta, wang2022test}.
Our work focuses on node classification. %, where the connection patterns introduce unique challenges.
GTrans \cite{jin2022empowering} proposes to augment the target graph at the test time by optimizing a contrastive loss by generating positive views from DropEdge \cite{rong2019dropedge} and negative samples from the features of the shuffling nodes \cite{velivckovic2018deep}. 
GraphPatcher \cite{ju2024graphpatcher} learns to generate virtual neighbors to improve low-degree nodes classification. 
SOGA \cite{mao2024source} designs a self-supervised loss that relies on  mutual information maximization and homophily assumption.
These works are mostly built upon heuristics and may not address  structure shifts in principle.

% Recently, AdaRC \cite{bao2024adarc} partially tackles degree and homophily shifts by adapting the hop-aggregation parameters to restore representation. However, AdaRC still does not fully handle structure shifts.
% We provide a detailed comparison with AdaRC in Sec. \ref{subsec:TSA_CSS}. 
% Notably, the aforementioned studies overlook the aspect of label shift on graphs which entails structure shift.
% To the best of our knowledge, this work is the first to study structure shift, including label shift, in a principled framework in GTTA. \pan{the last sentence may not be needed. Still the importance of label shift is overemphasized unless we have algorithms dedicated to it.}

\section{Test Error Analysis}
\label{Sec:Test Error Analysis}

% In this section, we first formally define the distribution shift on graphs, categorizing it into feature shift, label shift, and conditional structure shift (CSS).
% Addressing CSS can be reduced to addressing neighborhood shift in terms of first-order alignment \pan{too technical, in the expectation sense?, or not saying this. Later, say the missing part} under GNN mean pooling.
% Next, we examine how these shifts impact the generalization gap in GTTA given a pretrained source model.
% Lastly, we conduct an empirical investigation to corroborate and provide deeper insights into our theoretical findings, discussing additional second-order adjustment \pan{variance} strategies that are necessary but not covered in the theoretical analysis.
In this section, we characterize the generalization error between source and target graphs and explicitly attribute it to three different kinds of shifts: label shift, feature shift, as well as neighborhood shift. Motivated by our theoretical analysis on the generalization error, we then propose \proj algorithm to minimize the across-graph generalization error. 
% how different shifts between the source and target graph can impact the generalization gap. Our error decomposition echoes the findings in previous GDA works, but in a more formal bound \han{Bound is always formal. Can we claim our bound to be tighter? More general?}, which further motivates the \proj algorithm design. 

\begin{figure*}[t]
\centering
\begin{tabular}{c|cc|c}
(a) Reliability of $\gamma$ 
% &\multicolumn{2}{|c|}{ (b) Decision Boundary Refinement}
&(b) Source &(c) Nbr. Align 
& (d) SNR $\alpha$  \\
\includegraphics[width=0.23\textwidth]{icml2025/figures/uncertainty_nbr.pdf} &
\includegraphics[width=0.22\textwidth]{icml2025/figures/aggre_src.pdf} &
% \includegraphics[width=0.18\textwidth]{icml2025/figures/aggre_ss.pdf} &
\includegraphics[width=0.22\textwidth]{icml2025/figures/aggre_no_nbr.pdf} &
\includegraphics[width=0.23\textwidth]{icml2025/figures/uscn_alpha.pdf} \\

\includegraphics[width=0.23\textwidth]{icml2025/figures/uncertainty_str.pdf} &
\includegraphics[width=0.22\textwidth]{icml2025/figures/output_src.pdf} &
% \includegraphics[width=0.18\textwidth]{icml2025/figures/output_ss.pdf} &
\includegraphics[width=0.22\textwidth]{icml2025/figures/output_no_nbr.pdf} &
\includegraphics[width=0.23\textwidth]{icml2025/figures/cnus_alpha.pdf} 
\end{tabular}
\vspace{-4mm}
\caption{(a) Comparison of neighborhood alignment with $\mgamma$ from model prediction and Oracle on the CSBM graphs~\cite{deshpande2018contextual}. The top (or bottom) subfigures represents the assignment under neighbor shift (or neighbor shift plus label shift, respectively). Nodes are grouped by the entropy of their soft pseudo labels and the y axis shows the accuracy after assigning $\mgamma$. Ideally, a correct assignment (red) would lead to near-perfect accuracy.
However, the assignment based on pseudo labels is far from optimal (blue).
%(b) and (c) top \pan{} represent the t-SNE visualization of the node representations. (b) and (c) bottom visualize the node representations before passing through the classifier with its decision boundary on CSBM. 
From (b) to (c), the figure with the t-SNE visualization of node representations indicates a model trained from the source domain (CSBM with a label distribution $[0.1, 0.3, 0.6]$) to the target domain (CSBM with a label distribution $[0.3, 0.3, 0.3]$). The color of the nodes represents the ground-truth labels. The top subfigures of (b) and (c) show the output given by the GNN encoder while the bottom subfigures show the classifier decision boundaries.  
%Detail discussed in Sec. \ref{subsec:boundary}.
(d) Analysis of SNR adjustment $\alpha$ with respect to different layers and node degrees on MAG.
Detail discussed in Sec. \ref{subsec:alpha}.
}
\label{fig:pa_vis}
\vspace{-2mm}
\end{figure*}

\subsection{Distribution Shifts on Graphs} 
\label{sec:shift}
Distribution shifts on graphs were formally studied in previous GDA works \cite{wu2020unsupervised, liao2021information, zhu2021shift, wu2022handling, you2023graph, zhu2023explaining}.  
Following their definition, we categorize shifts in graphs into two types: feature shift and structure shift. For simplicity, our analysis is based on a data generation process: $\rmX \leftarrow \rmY \rightarrow \rmA$,  
where graph structure and node features are both conditioned on node labels.

\begin{definition}
\label{def:attrshift}
(Feature Shift). Assume node features $x_u, u \in \gV$ are i.i.d sampled given labels $y_u$, then we have $\prob(\rmX|\rmY)=\prod_{u\in\gV}\prob(X_u|Y_u)$. We then define the feature shift as $\probs(X_u|Y_u)\neq \probt(X_u|Y_u)$.
\end{definition}


\begin{definition}
\label{def:strucshift}
(Structure Shift). As graph structure involves the connecting pattern between labels, we consider the joint distribution of the adjacency matrix and labels $\prob(\rmA,\rmY)$, where \emph{Structure shift}, denoted by $\probs(\rmA,\rmY) \neq \probt(\rmA,\rmY)$,  can be decomposed into as \emph{conditional structure shift (CSS)} $\probs(\rmA|\rmY) \neq \probt(\rmA|\rmY)$ and \emph{label shift (LS)} $\probs(\rmY) \neq \probt(\rmY)$. 

\end{definition}

% While feature shift and label shift have been widely studied  in non-graph literature, CSS is specific to graph data as it manifest the connection discrepancy given the same label. 
% Notably, in real-world datasets, these shifts often coexist \cite{li2022out, zhang2024survey}.

% According to Theorem 3.3 in \cite{liu2024pairwise}, the sufficient condition for mitigating the influence of CSS under GNNs lies in aligning \emph{degree shift} $\probs(D_u | Y_u)\neq\probt(D_u | Y_u)$, where the node degrees within the
% same label differs, and \emph{neighborhood shift} $\probs(Y_v|Y_u, v \in\neighbor_u)\neq\probt(Y_v|Y_u, v \in\neighbor_u)$, where the neighboring node label connections
% within the same class differs.

\subsection{Theoretical Analysis}

% Among the defined shifts, \textit{feature shift} and \textit{label shift} have been studied for TTA in non-graph settings.  CSS has been studied for training-time GDA~\cite{liu2024pairwise}. % showing the impacts from both \textit{neighborhood shift:} $\probs(Y_v|Y_u, v \in\neighbor_u)\neq\probt(Y_v|Y_u, v \in\neighbor_u)$ and \textit{degree shift:} $\probs(|\neighbor_u||Y_u) =\probt(|\neighbor_u||Y_u)$. 
% % However, there is no previous works investigating the overall generalization gap in GTTA. 
% In this works, we aim to extend the scope to GTTA by characterizing how above shifts impact the target domain error. The analysis  
%  further inspires our later algorithm framework. %respectively given the pretrained source model in GTTA. 

% In practice, mean pooling is commonly preferred in GNNs for node
% classification tasks, as it smoothens the graph signal by
% reducing variance within the pooling, which shows superior performance.
% Since mean pooling directly addresses the variation of magnitude in neighborhood information, first-order expectation of degree distribution is naturally aligned. \shikun{naturally aligned?}
% As a result, we first focus on addressing neighborhood shift $\probs(Y_v|Y_u, v \in\neighbor_u)\neq\probt(Y_v|Y_u, v \in\neighbor_u)$ to mitigate CSS in our analysis.



% \shikun{For the bound, we need to revise the proof later on to provide more clear indication of mean pooling assumption, what we mean by consider the first-order information and how that reduce the term. The original degree term expansion is not needed, we can just start from the very beginning to expand over $\pi$ and $\omega$ and be clear about why $\omega$ is not considered. Need some revision on the bound and the corresponding discussion}
% \shikun{also, the notation here is not consistent, the subscript $v$ or $v_t$. Check the consistency again when modifying the proof and the bound}

Let $\errors(g \circ \phi)$ denote the error of a pretrained GNN on the source domain and $\errort(g \circ \phi)$ the error of the model when applied to a test graph in the target domain.
Inspired by \citet{tachet2020domain}, we provide an upper bound on the error gap between source and target domains, showing how a pretrained GNN (e.g., ERM) can be influenced. 
% \vspace{-1mm}
%\begin{definition} 
We denote the \emph{balanced error rate} of a pretrained node predictor $\predy_u$ on the source domain as
%\label{sber definition}
    %\begin{align*}
        $\balerrorrate{\predy_u}{\rvY_u} \coloneq \max_{j \in \mathcal{Y}} \probs(\predy_u\neq\rvY_u | \rvY_u=j).$
    %\end{align*}
%\end{definition}

%\vspace{-1mm}
\begin{restatable}[Error Decomposition Theorem]{theorem}{decomposeerror}
\label{theory:decomposeerror}
Suppose $\gG^{\gS}$ and $\gG^{\gT}$, we can decouple both graphs into independent ego-networks (center nodes and 1-hop neighbors).
For any classifier $g$ with a mean pooling GNN encoder $\phi$ in node classification tasks, we have the following upper bound for on the error gap between source and target under feature shift and structure shift: 
%{\small
\begin{align*}
&\abs{\errors(g \circ \phi) - \errort(g \circ \phi)} \\ 
&\leq   
\balerrorrate{\predy_u}{\rvY_u}\cdot
    \cbrace{\underbrace{
    \rpar{2\text{TV}(\probs(\rvY_u), \probt(\rvY_u)}
    }_{\text{Label Shift}}
   \\
   % &+
   % \underbrace{
   % \E_{\rvY}\spar{
   %  \rpar{
   %  \max_{k \in \mathcal{Y}}
   %  \abs{1 -\frac{\probt(\rvY_{v_t}=k | \rvY_u=j, v_t\in\neighbor_u)}{\probs(\rvY_{v_t}=k | \rvY_u=j, v_t\in\neighbor_u)} }
   %  }^{d_{m,y}}
   %  }
   %  }_{\text{Neighborhood Shift}}
   %  }\\
   %  &+\underbrace{\max_{i\
   %  \neq j}\E_{\{Y_v \}}\spar*{|\probs_k - \probt_k |}}_{\text{Feature shift}}
    &+\underbrace{\E_{Y_u}^\gT\spar{\max_{k\in \gY} \abs{1-\frac{\probt(Y_v=k|Y_u,v\in \neighbor_u)}{\probs(Y_v=k|Y_u,v\in \neighbor_u)}}}  }_{\textbf{Neighborhood shift}}}
    + \underbrace{\Delta_{CE}}_{\text{Feature shift}}
\end{align*}
%}
where 
    $TV(\probs(\rvY_u), \probt(\rvY_u))$ % \coloneq \frac{1}{2}\sum_{j\in\gY}\abs{\probs(\rvY_u=j) - \probt(\rvY_u=j)}$ 
    is the total variation distance between the source and target label distributions. % of source and target
    and 
    $\Delta_{CE}$ is the error gap that exists if and only if feature shift exists.
\end{restatable}

% \pan{first term should be in total variation; it is not good to mix sum and expectation; why there is a node association between source and target domain; the degree is a distribution, does max over d make sense? what is the neighbor shift here?; where is i,j in the feature shift?; Does the above bound assume mean pooling or anything?}

Our bound aligns with the findings of \citet{liu2024pairwise}, which highlight the impact of neighborhood shift, label shift, and feature shift on generalization to the target graph. We extend this understanding by deriving an explicit error bound. Notably, neighborhood shift is reduced from conditional structure shift given the assumptions  in Thm.\ref{theory:decomposeerror}.

% The error decomposition gives a way to analyze how node predictions in graphs are influenced by feature shift, label shift, degree shift and neighborhood shift \pan{no degree shift above} \pan{here define neighborhood shift and explain it, and cite Liu et al. here. And point out to fig5 c}, where the last three manifest as structure shift. The bound provide several insights: 1) Source model worst-class accuracy determines the test error upper bound.
% 2) In neighborhood shift, low density of the neighboring 
%  node label in the source domain contributes to the error.
% 3) Degree shift \pan{no label shift} and neighborhood shift of the majority classes in target domain contribute primarily to the error.
% 4) If the source neighbor label density is two times less than the target neighbor label density $2<\max\frac{\pit{t_k|u_j}}{\pis{t_k|u_j}}$, $d_{m,y}$ should be the node that contains the highest degree given the class label.
% Otherwise, $0<\max\frac{\pit{t_k|u_j}}{\pis{t_k|u_j}}<2$,  low-degree nodes contribute primarily to the error and $d_{m,y}$ should be the node that contain the lowest degree given the class label. \pan{the degree shift notation here looks weird}

% Following the arguments in \cite{liu2024pairwise} regarding the handling of CSS, it is important to note that in practical applications, mean pooling is commonly preferred in GNNs for node classification tasks. This is because it smoothens the graph signal by reducing noise in node representations, % within the pooling, 
% leading to superior performance. Aligning the first-order expectations of \pan{do we need to emphasize first order here?} the source and target neighborhood distributions should, therefore, result in satisfactory performance \cite{xu2018powerful}. Our proposed error gap primarily focuses on demonstrating the impact of \textit{neighborhood shift} \pan{why not CSS? neighorhood shift was not defined} on first-order statistics \pan{ here?}. Furthermore, we consider the impact of \textit{degree shift} as a change in the signal-to-noise ratio (SNR) of the aggregated neighborhood message, which will be discussed in Sec.~\ref{subsec:empirical_invest}.

 

% \subsection{Empirical Investigation}

% \label{subsec:empirical_invest}

% % To provide empirical validation to support our theoretical findings, we conduct a fine-grained study on synthetic datasets using contextual stochastic block model (CSBM) \cite{deshpande2018contextual}.
% % We use a one-layer GraphSAGE \cite{hamilton2017inductive} as the GNN encoder $\phi$ and a one-layer MLP with batch normalization \cite{batchnorm} as the classifier $g$. 
% % The model is trained on an imbalanced source graph with a label ratio of $[0.1, 0.4, 0.5]$ and analyzed for adaptation on two domains:
% % 1) Under CSS and label shift, where label ratio becomes $[1/3, 1/3, 1/3]$.
% % 2) SNR decreases from degree shift, where no neighborhood or label shift occurs.
% % See Appendix for more experimental details.

% % Below we summarize the main insights of our results:


% % To provide empirical validation to support our theoretical findings, we conduct a fine-grained study on synthetic datasets using contextual stochastic block model (CSBM) \cite{deshpande2018contextual} to mimic different types of shifts in our bound. We use a one-layer GraphSAGE \cite{hamilton2017inductive} as the GNN encoder $\phi$ and a one-layer MLP with batch normalization \cite{batchnorm} as the classifier $g$. 
% % The model is trained on a source graph as in Fig.\ref{fig:pa_vis} top (a). 
% \pan{if just visualization, no need to be a new section} \pan{also too long}
% \pan{better thing is to show why naively extending pairalign cannot solve the issue.}
% In this section, we (aim to demonstrate empirically) \pan{visualize} how the shifts identified in Thm.~\ref{theory:decomposeerror}, which contribute to target test time error, negatively impact model adaptation. We conduct a detailed study using synthetic datasets generated by the contextual stochastic block model (CSBM) \cite{deshpande2018contextual} to simulate various types of shifts outlined in our bound. For this analysis, we employ a one-layer GraphSAGE \cite{hamilton2017inductive} as the GNN encoder $\phi$ and a one-layer MLP with batch normalization \cite{batchnorm} as the classifier $g$. 

% The model is initially trained on a fixed source graph, and then adapted to target graphs that exhibit distinct types of shifts. We present the final node representation space after mean pooling aggregation and the corresponding classifier decision boundary for both the source and target models in Fig~\ref{fig:pa_vis}, illustrating the reasons for performance degradation.

% Based on the error bound, we construct three target graphs, each constructed by one of the following shifts: \textit{feature shift}, \textit{neighborhood shift}, and \textit{label shift}. Detailed experimental settings are provided in Appendix~\ref{}.

% \paragraph{Feature Shift} \pan{use \textbf{Feature shift} not paragraph} As illustrated in Fig~\ref{fig:pa_vis} top (b), feature shift in graph-structured data also leads to misalignment in the latent representation space, a phenomenon well-explored in the non-graph TTA literature \cite{wang2020tent, boudiaf2022parameter, iwasawa2021test}. This issue can be addressed by adapting existing techniques. Therefore, this paper focuses on addressing the remaining structure shifts, specifically neighborhood shift and label shift, in the context of graph test-time adaptation (GTTA). \pan{can be substantially trimmed}

% \paragraph{Neighborhood Shift} From Fig. \ref{fig:pa_vis}, top (a) to (c), we observe a misalignment of aggregated node representations caused by discrepancies in neighborhood distribution, even in the absence of feature shift. Notably, neighborhood shift results in aggregated node representations that are less distinguishable, in contrast to the misaligned yet distinguishable representation space caused by feature shift. Consequently, merely aligning the feature distribution is insufficient to address the neighborhood shift. Alignment over the neighborhood distribution is the key to mitigate the impact of neighborhood shift as indicated in the bound. 
% From Fig. \ref{fig:pa_vis} top (a) to (b), we can observe that the neighborhood information degrades as structure shift occurs. 
% Based on the conclusion from Theorem \ref{theory:decomposeerror}, we employ Oracle values to bootstrap neighboring nodes for first-order alignment (detailed methodology in Sec. \ref{subsec:1stalign}). 
% Fig. \ref{fig:pa_vis} top (c) shows the neighborhood hidden representations after alignment, which match those in the source domain.
% As a result, we achieve domain-invariant representations across the source and target graphs, i.e., $\probs(H_u|Y_u) = \probt(H_u|Y_u)$.






% Note that SNR refers to the second moment, providing an additional perspective on aligning neighborhood information. 
% In contrast, our theory focuses on the first moment of the generalization error.

% \hans{not final node representations. did not contain self representation only aggregated neighborhood representation}






\section{Test-Time Structural Alignment  (\proj)}


Motivated by our theoretical analysis, we propose \proj to address GTTA in a principled way. % mitigate variations in neighborhood information for GTTA principally.
To address neighborhood shift, \proj first conducts neighborhood alignment via weighted aggregation and then properly balances self and neighboring node information combination based on the SNRs of these representations. Lastly, as a generic approach for  neighborhood shift, \proj can be combined with non-graph TTA methods to refine the decision boundary to address the remaining feature and label shift. 

% \hans{neighborhood shift: difference compared to pairalign
% 1.compute inverse (1st order)
% 2.trainable $\rightarrow$ TTA is not trainable $\rightarrow$ adjust classifier, SNR (2nd order)
% 3.stable usage $\rightarrow$ why gamma not accurate $\rightarrow$ entropy
% }


\subsection{Neighborhood Alignment}
\label{subsec:1stalign}


Neighborhood shift alters the node label ratios in the aggregated neighborhood information, causing the shift in the GNN-encoded representations. 
In training-time GDA, PairAlign \cite{liu2024pairwise} leverages such a technique by assigning edge weights to align the distribution of the source neighborhood with the target domain.

%\pan{the original version of the next paragraph}

Inspired by this idea, for GTTA, we aim to achieve a similar goal but in a different direction. Based on Theorem \ref{theory:decomposeerror}, we align the target neighborhood distribution with the source domain, leveraging the fact that the pre-trained model is optimized for the source distribution. Specifically, the neighborhood distribution determines the ratio of messages passed from a neighboring class $j$ to center class $i$. To adjust for distributional differences, this ratio can be rescaled by assigning weights to edges from class $j$ to class $i$, effectively acting as an up/down-sampling mechanism during message aggregation. To ensure that message aggregation in the target domain aligns with the expected behavior in the source domain, \proj incorporates the following adjustment:










% This ratio can be rescaled by assigning weights on $j$ to $i$ edges, which can be interpreted as up/down-sampling of messages during message aggregation.  

% Hence, \proj incorporates defined as follows to make the target domain message aggregation match the source domain in expectation:


\begin{definition}
Let $\probt(Y_v=j|Y_u=i,v\in \neighbor_u)>0, \forall i,j \in \gY$, we have $\mgamma \in \R^{|\gY|\times|\gY|}$ as:
% \label{def:gamma}
\begin{equation} 
    [\mgamma]_{i,j} = \frac{\probs(Y_v=j|Y_u=i,v\in \neighbor_u)}{\probt(Y_v=j|Y_u=i,v\in \neighbor_u)}, \: \forall i,j \in \gY
\label{eq:gamma_ratio}
\end{equation}
\end{definition}
\vspace{-2mm}

% Though it may seem trivial that applying the same alignment strategy with the ratio is simply the inverse of the ratio used in GDA.
To estimate $\mgamma$, we assume that the source summary statistics $\probs(Y_v=j|Y_u=i,v\in \neighbor_u)$ are recorded and available at test time; otherwise, alignment would not be feasible. Storing  $\probs(Y_v=j|Y_u=i,v\in \neighbor_u)$ incurs minimal cost, as it is merely an $|\gY|\times|\gY|$ matrix. Beyond this, no additional information from the source domain is required. For $\probt(Y_v=j|Y_u=i,v\in \neighbor_u)$, we estimate it based on target pseudo labels. Note that PairAlign~\cite{liu2024pairwise} enhances estimation accuracy that relies pseudo labels~\cite{liu2023structural} by leveraging a least-squares constrained optimization. However, in GTTA, the absence of source graphs and the potential need for real-time adaptation render this approach impractical.




While it seems direct to apply a similar alignment strategy by using the inverse of the ratio employed in PairAlign, assigning $\gamma$ requires knowledge of node labels, which is missing in the targe graph, making it challenging in GTTA.


\paragraph{Reliable assignment of $\mgamma$.} 
The ratio $\gamma$ should be assigned to edge weights based on the label pairs of the central and neighboring nodes. 
In training-time GDA, this assignment is straightforward as it relies on the ground-truth labels of the source graph. However, in GTTA, this information is unavailable. A naive approach is to use target pseudo labels, but this often results in significant mismatches. \proj addresses this by quantifying the uncertainty of target pseudo labels~\cite{zhang2019bayesian,stadler2021graph,hsu2022graph}. In particular, TSA assigns $[\mgamma]_{i,j}$ only to node pairs $v\rightarrow u$ where both of their soft target pseudo labels $\hat{y}$ have low entropy $H(\hat{y})=-\sum_{i\in \gY}[\hat{y}]_i\ln([\hat{y}]_i)\leq\rho_1 \cdot \ln ( |\gY|)$. Here, $\rho_1$ is a hyperparameter and $ \ln( |\gY|)$ is the maximum entropy in $|\gY|$ class prediction. In Fig.\ref{fig:pa_vis} (a), nodes with low-entropy soft predictions are more reliable, resulting in higher accuracy after the assignment of $\mgamma$. 

% In GTTA, the assignment is challenging as the label is missing when assigning $[\mgamma]_{i,j}$ to the target graph.
% Furthermore,  $\probt(Y_v=j|Y_u=i,v\in \neighbor_u)$ can not be directly computed without node labels.
% The bootstrapping ratio should be assigned to edge weight based on the center and neighboring node label pairs. 
%Fig. \ref{fig:pa_vis} (a) demonstrates the challenge of correctly assigning $\gamma$ to edge weight in GTTA.
%The (top) represents the assignment under neighbor shift the (bottom) indicates an structure shift.
% An additional label shift also lowers its assigned accuracy.

% The challenge arises when estimating $\probt(Y_v=j|Y_u=i,v\in \neighbor_u)$, where no ground-truth labels are available. We may adopt target pseudo labels to compute this distribution. However, we found that the obtained $\mgamma$ might be inaccurate, as illustrated in Fig.\ref{fig:pa_vis}(a) \pan{(reference incorrect here)}. This observation aligns with prior findings in training-time GDA~\cite{liu2023structural}. PairAlign~\cite{liu2024pairwise} improves estimation accuracy by leveraging a least-squares constrained optimization to align \emph{soft} source and target pseudo-label predictions. However, in GTTA, the absence of source graphs and the requirement for real-time adaptation, even if source graphs were available, make this strategy infeasible.

%In GDA, we assume access to ground truth source data, and in GTTA, we assume the availability of source summary statistics $\probs(Y_v=j|Y_u=i,v\in \neighbor_u)$.

%However, target labels are absent in both scenarios, causing $\mgamma$ estimation to be unreliable due to unstable target pseudo-labels, as noted in initial GDA studies~\cite{liu2023structural}. Later, PairAlign enhanced the robustness of $\mgamma$ estimation by employing a least square constrained optimization aimed at aligning soft source and target pseudo-label predictions. However, In GTTA, the absence of source data hinders the use of such robust estimation strategies. Relying solely on target hard pseudo-label predictions for $\mgamma$ estimation is proven to be unsatisfactory when compared to oracle results, as shown in Fig.~\ref{fig:pa_vis}(a). Thus, developing a robust estimation method for $\mgamma$ without source data access remains a significant challenge.


% In GDA, obtaining $\mgamma$ is easier when we have the access to the source graph ground truth information, and is relatively stable through solving a constrained least square optimization. However, in GTTA, we need to assign $[\mgamma]_{i,j}$ based on only summary statistics of source graph neighborhood distribution and the unstable target pseudo-label predictions. 

%\proj addresses this by quantifying the uncertainty of target pseudo labels~\cite{zhang2019bayesian,stadler2021graph,hsu2022graph}. In particular, TSA assigns $[\mgamma]_{i,j}$ only to node pairs $v\rightarrow u$ where both of their soft target pseudo labels have low entropy $H(\hat{y}')=-\sum_{i\in \gY}[\hat{y}']_i\ln([\hat{y}']_i)\leq\rho_1 \cdot \ln ( |\gY|)$. Here, $\rho_1$ is a hyperparameter and $ \ln( |\gY|)$ is the maximum entropy in $|\gY|$ class prediction. In Fig.\ref{fig:pa_vis}(a), nodes with low-entropy soft predictions are more reliable, resulting in higher accuracy after the assignment of $\mgamma$. 


%^by assigning  $[\mgamma]_{i,j}$ based on the uncertainty of the node pair prediction \cite{zhang2019bayesian,stadler2021graph,hsu2022graph}.
%In particular, we estimate $\probt(Y_v=j|Y_u=i,v\in \neighbor_u)$ using refined soft prediction $\hat{y}'_u$ from the classifier adapted TTA.
%The assignment relies on the accuracy of pseudo-label $y'_u=\argmax \hat{y}'_u$, hence TSA filter out unconfident node label pairs using entropy to estimate uncertainty.
%In particular, TSA assigns $[\mgamma]_{i,j}$ to node pairs where both predictions have entropy $H(\hat{y}_u')=-\sum_{i\in \gY}[\hat{y}_u']_i\ln([\hat{y}_u']_i)\leq\rho_1 \cdot \ln ( |\gY|)$. Here, $\rho_1$ is a hyperparameter and $ \ln( |\gY|)$ is the maximum entropy in $|\gY|$ class prediction.

% \paragraph{Neighborhood Information SNR and Classifier Boundary Bias from Source Training.}
% GTTA does not train the model from scratch; the combination of self and neighborhood features in the GNN as well as the classifier's decision boundary are predetermined based on the source domain. 
% On one hand, in the target domain, the optimal combination from the source domain may be suboptimal, as the variance of neighborhood information changes (see Fig. \ref{fig:pa_vis} (e)).
% Furthermore, even within the same domain, the SNR of neighborhood information varies across different GNN layers and node degrees. On the other hand, a decision boundary obtained from label-imbalanced source training often lies in high density regions, which may further hinder test-time accuracy under label shift. 

% We explain how these two issues are addressed in Sec. \ref{subsec:2ndSNR} and \ref{subsec:boundary_refine}.



\begin{algorithm}[t]
   \caption{Test-Time Structural Alignment (TSA)}
   \label{alg:example}
\begin{algorithmic}[1]
   \STATE {\bfseries Input:} A GNN $\phi$ and a classifier $g$ pretrained on source graph $\gG^{\gS}$; Test-time target graph $\gG^{\gT}$; Source statistics $\probs(Y_v|Y_u,v\in \neighbor_u)\in \R_+^{|\gY|\times|\gY|}$
   \STATE Initialize $b^{(k)}\leftarrow 1$ and $\text{MLP}^{(k)}\text{ parameters} \leftarrow0$ for the $k$-th layer.
    \STATE Perform boundary refinement based on embeddings from $g\circ \phi(\gG^{\gT})$ and get soft pseudo-labels $\hat{y}$
    \STATE {\bfseries Get} $\gG^{\gT}_{new}$ {\bfseries by assigning edge weights:}
    \INDSTATE Compute $\gamma$ using  $\hat{y}$ via Eq. \ref{eq:gamma_ratio}
    \INDSTATE  Assign $\gamma$ only if node pairs $H(\hat{y})\leq \rho_1 \cdot \ln(|\mathcal{Y}|)$
        \INDSTATE Assign the parameterized $\alpha$ via Eq. \ref{eq:alpha_ratio}
    \STATE Update $\alpha$'s parameters $b^{(k)}$ and $\text{MLP}^{(k)}$ via Eq. \ref{eq:loss}
    %\STATE Perform boundary refinement again %using $\hat{y}'$ where $H(\hat{y}')\leq \rho_2 \cdot \ln(|\mathcal{Y}|)$ as supervision 
   % \ENDFOR
   \STATE {\bfseries return} $\hat{y}_{\mathrm{final}}$ after another boundary refinement
   % $\hat{Y}_{final}\leftarrow\mathit{ClsTTA}(g(\phi(\gG^{\gT}_{new})))$
   % \REPEAT
   % \STATE Initialize $noChange = true$.
   % \FOR{$i=1$ {\bfseries to} $m-1$}
   % \IF{$x_i > x_{i+1}$}
   % \STATE Swap $x_i$ and $x_{i+1}$
   % \STATE $noChange = false$
   % \ENDIF
   % \ENDFOR
   % \UNTIL{$noChange$ is $true$}
\end{algorithmic}
\end{algorithm}



% On top of the neighborhood alignment, we introduce additional techniques that help better adaptation under the GTTA setting, taking into account the constraints of starting with a pretrained source model and lacking the opportunity for substantial retraining.
\subsection{SNR-Inspired Adjustment}
\label{subsec:snr}
Building on neighborhood alignment, we further optimize the signal-to-noise ratio (SNR) of node representations to enhance performance. Specifically, SNR is characterized by the ratio of inter-class representation distances to intra-class representation distances. %, as illustrated in Fig.~\ref{}. 
A higher SNR indicates more informative and well-separated representations, which benefit classification.

Optimizing SNR complements the neighborhood alignment approach. Even if neighborhood label distributions are perfectly aligned, variations in neighbor set sizes between the source and target graphs can impact the SNR of aggregated neighboring node representations. Consequently, the combination of these aggregated representations with self-node representations in a typical GNN pipeline (Eq.~\ref{eq:gnn}) should be adjusted accordingly across source and target domains, particularly when the source and target graphs exhibit very different degree distributions. Furthermore, the SNR of node self-representations may vary across GNN layers, as deeper layers generally reduce variance. As a result, node self-representations in deeper layers tend to have higher SNR and should be assigned greater emphasis.


To implement the SNR-inspired adjustment, we introduce a parameter to perform the weighted combination of self-node representations and neighborhood-aggregated representations at each layer, adapting to node degrees as follows:

% \hans{is this appropriate to use definition as it is not something derived from principle?}
\begin{definition}
Let $\tilde{d_u}=\left(\frac{\ln(d_u +1)}{\ln(\max_{v\in\node} d_v+1)}\right)$ denote log-normalized degree of node $u$ and let $\text{MLP}^{(k)}$ and $b^{(k)}$ to be learnable parameters for adjusting $k$-th layer combination, define the weights for combination $\alpha \in \R^{L}$ as:
\begin{equation}
    [\alpha]_{k} =
\sigma(\text{MLP}^{(k)}(\tilde{d_u}))-0.5+ b^{(k)}, \: \forall k \in [L]
\label{eq:alpha_ratio}
\end{equation}
\end{definition}

where $\sigma$ is the sigmoid function. 
%The term $b^{(k)}$ accounts for the global tendency, while $\text{MLP}^{(k)}$ gives a local interpretation with respect to node degrees, both at the $k$-th layer. 
During initialization $[\alpha]_{k}$ is set to $1$.
Degree values are taken in the logarithmic domain to handle their often long-tailed distribution. 

Combined with $\mgamma$, $[\alpha]_{k}\cdot[\mgamma]_{i,j}$ is used to reweight the GNN message for non-self-loop node pairs, adjusting the influence from the node with a highly certain pseudo label $j$ to the node with a highly certain pseudo label $i$.
%Since $[\alpha]_{k}$ is constant with respect to the GNN aggregation function, it can be considered as a bootstrapping ratio for non-self-loop pairs.

% Moreover, the SNR of node self representations may also vary across different GNN layers, since in general deep layers help with reducing the variance. 

\textbf{Remark.} Neighborhood alignment alone does not address potential shifts in SNR. This is because the alignment approach, inspired by Thm.~\ref{theory:decomposeerror}, focuses solely on expectation-based (first-order statistical) performance, whereas SNR also incorporates variance (second-order statistics). Thus, these two aspects complement each other. 

% In the context of GNN, the formation of final node representations combines the neighborhood representations after aggregation and self-features. SNR of final node representations will be affected by the SNR of each component. Firstly, when aggregating the neighborhood messages, the density of graph that affects the neighborhood representation SNR. Given the same neighborhood label distribution, the variance will change depending on different number of neighbors. Namely, we have higher SNR if we are aggregating over graph with higher degrees since the aggregated representations have lower variance. Secondly, self-node representation SNR can vary across different layers of GNN. GNN layers have smoothing effect that reduces variances when we aggregate more neighbors. Therefore, node representations in later layers have higher SNR. 

% % In GNN aggregation, self-features should be properly combined with the neighboring aggregated features. 
% Consequently, the key is to identify the optimal combination to form final node representations based on the SNR of both self and neighborhood representations. Under GTTA, the predetermined combination provided by the source model may be suboptimal in terms of SNR in the target graph. Consequently, we introduce SNR-inspired adjustments to enhance the quality of the node representation space. Additionally, note that considering SNR implies the second-order statistics adjustment in terms of the variance which compensates our theoretical analysis that primarily focuses on demonstrating the impact of neighborhood shift on first-order expectation under mean pooling following~\cite{liu2024pairwise}. 

% However, it is important to note that node feature distribution and neighborhood distribution naturally incorporates second-order information, like variance ($\sigma^2$). This can result in different Signal-to-Noise Ratios (SNR), expressed as $\frac{\mu^2}{\sigma^2}$ even when the expectation ($\mu$) remains aligned after addressing neighborhood and feature shifts. In practice, SNR of the node representations can vary depending on node degrees and different GNN layers \pan{why?}. Specifically, below we show the empirical impact from different SNR of the aggregated representation space caused by different graph sparsity in the source and target graph. 

% \paragraph{Signal-to-noise Ratio (SNR) Shift}
% In Fig. \ref{fig:pa_vis} top (a) and top (e), the SNR of aggregated node representation decreases in the target domain as node connectivity decreases while with the same expectations.
% In this case, target node representations, especially minority class, become noisier and scatter among other classes.
% This causes the node representations in Fig. \ref{fig:pa_vis} bottom (e) to spread into regions of other classes, leading to misclassification. Therefore, the observation highlights the necessity of designing a method that optimizes the formation of final node representations by considering the SNR of both the neighborhood representations and self-node features \pan{why should balance these two parts according to SNR}. \pan{the better way to show snr is via different layer mu/sigma; and different density mu/sigma;} \pan{having a proposition and the best balance}


% As discussed in Sec.~\ref{subsec:empirical_invest}, achieving a higher SNR ratio for the final node representations can lead to empirical performance gains. PairAlign does not account for this adjustment during training-time adaptation because it has the flexibility to train the model on a transformed source graph that mimics the target graph, allowing it to learn a more generalizable model for the target graph. 
% % \hans{I think it should be PairAlign access the target graph during training, so the model can adapt to the SNR in target graph through the estimated pseudo-label distribution.}





% Specifically, we adopt a scaling parameter to adjust the optimal combination at each layer as follows:

% % \hans{is this appropriate to use definition as it is not something derived from principle?}
% \begin{definition}
% Let $\tilde{d_u}=\left(\frac{\ln(d_u +1)}{\ln(\max_{v\in\node} d_v+1)}\right),  \forall d_u, d_v \in \gD$ and let $\text{MLP}^{(k)}$ and $b^{(k)}$ to be learnable parameters for adjusting $k$-th layer combination, we have $\alpha \in \R^{L}$ as:
% \begin{equation}
%     [\alpha]_{k} =
% \sigma(\text{MLP}^{(k)}(\tilde{d_u}))-0.5+ b^{(k)}, \: \forall k \in [L]
% \label{eq:alpha_ratio}
% \end{equation}
% \end{definition}

% where $\sigma$ is the sigmoid function. 
% The term $b^{(k)}$ accounts for the global tendency, while $\text{MLP}^{(k)}$ gives a local interpretation with respect to node degrees, both at the $k$-th layer. 
% During initialization $[\alpha]_{k}$ is set to $1$.
% Degree values are taken in the logarithmic domain to handle their often long-tailed distribution.
% Since $[\alpha]_{k}$ is constant with respect to the GNN aggregation function, it can be considered as a bootstrapping ratio for non-self-loop pairs.

\subsection{Decision Boundary Refinement}
\label{subsec:boundary}


% Although Fig. \ref{fig:pa_vis} top (d) shows an aligned latent node representation space, label shift results in a mismatch of the decision boundary between the source and target domains, 
In GTTA, label shift can result in a mismatch of the decision boundary, even after addressing neighborhood shift and obtaining high-SNR node representations. This is illustrated in  Fig.~\ref{fig:pa_vis} (b) and (c). 
% The primary issue arises due to data imbalance during source training. 
%As shown in Fig.~\ref{fig:pa_vis}(b), when the source model is trained on label-imbalanced data with class proportions $[0.1, 0.3, 0.6]$, the decision boundary for the minority class (class $0$) tends to be located in high-density regions. 
%Consequently, even when the neighborhood distribution is perfectly aligned with Oracle assignment, the predefined boundary from the source model can still be affected by label shift. % where the target class ratio is $[0.3, 0.3, 0.3]$ (Fig.~\ref{fig:pa_vis}(c)).


% \label{subsec:boundary_refine}
% Section \ref{Sec:Test Error Analysis} theoretically and empirically shows that simply addressing neighborhood shift in GNNs leads to suboptimal target risk $\errort(g\circ\phi)$. The predefined decision boundary in the source model can easily suffer from the shift in label distribution during test time. Recall the example that label imbalance in the source domain causes the boundary to lie in high density regions for small portion classes, which cannot be solved by simply aligning the neighborhood distribution.
% Under label shift, accuracy may be further hindered as the portion of small-portion classes increases, leading to a decrease in overall accuracy (see Fig. \ref{fig:pa_vis} (c)).
A straightforward approach to refining the decision boundary at test time is to adjust the classifier's batch normalization parameters (TENT~\cite{wang2020tent}) or directly modify its output (T3A~\cite{iwasawa2021test} and LAME~ \cite{boudiaf2022parameter}). 
We integrate these techniques into our framework for two folds: (1) their refined pseudo-labels provide a more reliable assignment of $\mgamma$ and can supervise the update of SNR adjustment. (2) Reciprocally, better alignment of neighborhood information can further refine the decision boundary.
% The final adaptation from the the classifier-adapting TTA further enhances prediction accuracy by reducing CSS discrepancy.

% We integrate these techniques into our framework. \pan{after classifier, y , y}

% leverage non-graph TTA 

% without changing the GNN encoder. 
% Non-graph TTA has widely explored this by adjusting the classifier's batch norm parameters \cite{wang2020tent} or its output directly \cite{iwasawa2021test,boudiaf2022parameter}.
% By integrating with non-graph TTA, TSA effectively handles both label shift and feature shift.

% \shikun{we may want to include more insights in how and why previous works can adjust decision boundary. how they are trained, provide some brief summarization, list two works in 3-4 sentences. }


%%%%%%%%%%%%%%%%%%%%%%%%%%%%%%%%%%%%%%%%%%%%%%%%%%%%%%%%%%%%%%%%%%%%%%%%%%%%%%%%%%%%%%%%%%%%%%%%%%%%%%%%%%%%%%%%%%%%%%%%%%%%%%%%%%%%%%%%%%%%%%%%%%%%%%%%%%%%%%%%%%%%%%%%%%%%%%%%%%%%%%%



\subsection{TSA Overview.}
\label{Sec:overview}
 %Since $\bar{\gamma}_{u,v}$ relies on the quality of the predictions, TSA utilizes more accurate prediction from a classifier-adapting TTA to get a more reliable estimation on $\probt(Y_v,Y_u | v\in \neighbor_u)$ and pseudo-label assignments.
Note that $\mgamma$ is obtained from the initial pseudo labels.
Only the parameters in $\text{MLP}^{(k)}$ and $b^{(k)}$ for $\alpha$ estimation in Eq.~\ref{eq:alpha_ratio} need to be optimized in the test time according to Eq.~\ref{eq:loss}.

% The learnable bootstrapping $\alpha$ and parameters in the MLP are optimized by the cross-entropy loss:

% The parameters in $\text{MLP}^{(k)}$ and $b^{(k)}$ will be optimized in the test time via Eq.~\ref{eq:loss} supervised by hard pseudo labels.


\vspace{-5mm}
\begin{equation}
    \gL_{CE} = \frac{1}{|\gV^{\gT}|}\sum_{u\in\gV^{\gT}}\text{cross-entropy}(y_u',\hat{y}_u) 
    \label{eq:loss}
\end{equation}
\vspace{-2mm}

where $y_u'$ is the hard pseudo label refined by procedure in Sec.~\ref{subsec:boundary} and $\hat{y}_u$ is the soft prediction from the original model.
After updating $\alpha$, TSA makes predictions on the newly weighted graphs and then further adapts the boundaries as described in Sec. \ref{subsec:boundary}. %This procedure will iterate. 
Our proposed algorithm is summarized in Alg. \ref{alg:example}. % to adapt predictions on the newly weighted graph.


\textbf{Comparison to AdaRC \cite{bao2024adarc}}
AdaRC is a recent work on GTTA that also integrates non-graph TTA methods with an approach for addressing structure shifts. However, it considers only degree shift and homophily shift, where homophily shift is merely a special case of neighborhood shift in our context. Consequently, AdaRC does not introduce the parameter $\mgamma$ and, therefore, lacks the ability to properly align neighborhood aggregated representations when the neighboring label distribution shifts. 
%Their idea is that these two shifts will yield the neighborhood information degraded and hence learn to adapt the k-hop aggregation parameters in order to restore better node representation from the combination of self and neighboring feature.  
%However, AdaRC 
%and thus cannot properly align \emph{neighborhood representations} when neighboring node label shifts. %that have already been mismatched due to neighborhood shift.
%Also, AdaRC does not distinguish between degree shift and homophily shift when addressing them. 
%On the other hand, TSA learns a function that takes node degree as input, providing better interpretability on the emphasis between center and neighbor information with respect to node degree.
% Our work is also the first work to systematically explore the idea of domain invariant learning in GTTA \cite{kurmi2021domain, liang2021source, mirza2023actmad, abdul2023align}.\shikun{what does it mean by this sentence?}















% \begin{restatable}{proposition}{scaledbootstrap}
% \label{prop:scaledbootstrap}If $f^{(k)}_{u,v}(\cdot)$ is a constant with respect to the aggregation function and is a scalar used to scale the aggregated neighboring nodes’ representations:  $h_u^{(k+1)} = \text{UPT}(h_u^{(k)}, f^{(k)}_{u,v}(\cdot)\text{AGG}(\{h_{v}^{(k)}, v \in \neighbor_u\}))$. Then, the scaled aggregation is equivalent to bootstrapping the neighboring nodes by a ratio of $f^{(k)}_{u,v}(\cdot)$. \pan{do we really want to add this proposition? }
% \end{restatable}

% Proposition \ref{prop:scaledbootstrap} demonstrates a design that enables bootstrapping to balance  the information from the center node and its neighborhood.
% Note that this can be combined with the neighborhood shift mitigation strategy by assigning edge weights as $\bar{\gamma}_{u,v}\cdot f^{(k)}_{u,v}(\cdot)$, together TSA addresses the CSS.

\section{Experiments}  

\begin{table*}[t]
\vspace{-2mm}
\scriptsize
\caption{MAG results (accuracy). \textbf{Bold} indicates improvements in comparison to the corresponding non-graph TTA baselines. \underline{Underline} indicates the best model.}
\vspace{-2mm}
\label{table:mag}
\begin{center}
\begin{adjustbox}{width = 1\textwidth}
\begin{tabular}{lcccccccccc}
\toprule
\textbf{Method} & US$\rightarrow$CN & US$\rightarrow$DE & US$\rightarrow$JP & US$\rightarrow$RU & US$\rightarrow$FR & CN$\rightarrow$US &CN$\rightarrow$DE & CN$\rightarrow$JP & CN$\rightarrow$RU & CN$\rightarrow$FR\\
\midrule
ERM &31.86$\pm$0.83 &32.22$\pm$1.16 &41.77$\pm$1.27 &29.22$\pm$1.64 &24.80$\pm$0.88 &37.41$\pm$1.01 &21.54$\pm$0.79 &30.12$\pm$0.72 &19.19$\pm$1.12 &16.92$\pm$0.58 
\\
GTrans  &31.77$\pm$0.91 &32.14$\pm$1.05 &41.55$\pm$1.23 &29.74$\pm$1.57 &25.03$\pm$0.85 &36.17$\pm$0.89 &21.07$\pm$0.93 &29.08$\pm$0.82 &19.68$\pm$1.14 &16.78$\pm$0.62  \\
SOGA &21.54$\pm$2.52 &25.48$\pm$0.93 &36.24$\pm$3.31 &29.07$\pm$4.14 &24.34$\pm$0.91 &38.95$\pm$3.35 &25.75$\pm$1.14 &38.25$\pm$1.42 &29.86$\pm$1.71 &23.50$\pm$0.67  \\
TENT &26.72$\pm$1.33 &32.73$\pm$0.63 &40.80$\pm$0.91 &32.26$\pm$0.95 &28.32$\pm$0.66 &27.21$\pm$0.88 &15.66$\pm$0.86 &24.62$\pm$0.48 &21.37$\pm$0.73 &13.84$\pm$0.62 
\\ 
LAME &35.75$\pm$0.85 &33.64$\pm$1.70 &44.97$\pm$1.15 &30.19$\pm$1.64 &24.17$\pm$1.84 &40.08$\pm$1.13 &22.64$\pm$1.14 &33.00$\pm$1.48 &17.80$\pm$0.55 &17.43$\pm$0.93 
 \\
T3A  &41.47$\pm$1.15 &45.36$\pm$2.15 &50.34$\pm$0.94 &46.41$\pm$0.84 &40.26$\pm$1.69 &46.50$\pm$1.26 &38.62$\pm$1.03 &46.10$\pm$0.38 &43.11$\pm$0.76 &29.95$\pm$1.36  
\\
% % \midrule
% % AdaRC-TENT &13.90$\pm$0.71 &25.87$\pm$0.47 &29.30$\pm$0.82 &18.14$\pm$0.19 &21.03$\pm$0.26 &16.38$\pm$0.80 &7.62$\pm$0.73 &13.84$\pm$0.74 &7.85$\pm$0.81 &5.97$\pm$0.36
% % \\
% % AdaRC-LAME &8.93$\pm$4.76 &8.18$\pm$0.78 &15.46$\pm$1.34 &6.25$\pm$1.43 &6.69$\pm$0.81 &1.35$\pm$2.69 &17.60$\pm$8.89 &33.42$\pm$2.41 &18.81$\pm$1.42 &4.23$\pm$8.02 
% % \\
% % AdaRC-T3A &22.34$\pm$18.21 &32.18$\pm$1.70 &42.85$\pm$1.15 &34.77$\pm$2.62 &25.75$\pm$0.62 &45.27$\pm$2.57 &33.01$\pm$1.17 &42.08$\pm$0.82 &38.57$\pm$1.50 &29.30$\pm$0.47 
% % \\
% \midrule
% Ours-TENT &27.16$\pm$1.32 &32.81$\pm$0.73 &40.72$\pm$0.97 &32.32$\pm$0.77 &28.57$\pm$0.52 &27.81$\pm$0.89 &15.95$\pm$0.86 &24.81$\pm$0.46 &21.67$\pm$0.70 &14.11$\pm$0.73
% \\
% Ours-LAME &36.71$\pm$0.99 &34.89$\pm$1.60 &45.73$\pm$0.95 &31.37$\pm$1.72 &25.44$\pm$1.79 &42.41$\pm$0.76 &24.78$\pm$0.69
% &35.78$\pm$1.10 &19.92$\pm$0.42 &18.98$\pm$0.53
% \\
% Ours-T3A &41.59$\pm$0.82 &46.39$\pm$1.85 &51.38$\pm$0.96 &46.54$\pm$0.92 &42.40$\pm$0.85 &47.51$\pm$0.46 &39.21$\pm$1.38
% &46.35$\pm$0.45 &43.33$\pm$0.96 &30.30$\pm$1.65
% \\
% \midrule
% Ours-TENT &27.27$\pm$1.73 &32.87$\pm$0.76 &40.81$\pm$1.11 &32.25$\pm$0.81 &28.56$\pm$0.54 &27.85$\pm$0.92 &16.21$\pm$1.17 &24.88$\pm$0.54 &22.09$\pm$0.83 &14.14$\pm$0.81
% \\
% Ours-LAME 
% &37.92$\pm$1.14 &36.31$\pm$1.71 &46.98$\pm$1.01 &32.84$\pm$1.94 &26.64$\pm$1.63 &44.61$\pm$0.81 &28.21$\pm$0.38 &39.59$\pm$1.40 &24.09$\pm$0.85 &22.01$\pm$0.27
% \\
% Ours-T3A 
% &41.76$\pm$1.02 &46.93$\pm$2.04 &51.46$\pm$1.10 &46.48$\pm$1.11 &42.88$\pm$0.62 &47.70$\pm$0.61 &39.21$\pm$1.90 &46.30$\pm$0.62 &43.43$\pm$0.73 &30.90$\pm$2.10
% \\
% TSA-TENT
% &27.30$\pm$1.61 &32.84$\pm$0.78 &40.82$\pm$0.99 &32.53$\pm$0.91 &28.62$\pm$0.63 &27.89$\pm$0.98 &16.22$\pm$1.17 &24.87$\pm$0.56 &22.05$\pm$0.84 &14.20$\pm$0.71
% \\
% TSA-LAME
% &37.95$\pm$0.97 &36.29$\pm$1.65 &46.86$\pm$1.13 &32.86$\pm$2.23 &27.22$\pm$1.48 &44.83$\pm$0.88 &28.51$\pm$0.44 &39.80$\pm$0.99 &24.54$\pm$0.87 &22.39$\pm$0.30
% \\
% TSA-T3A
% &\textbf{41.65$\pm$0.99} &\textbf{47.01$\pm$2.08} &\textbf{51.65$\pm$0.90} &\textbf{46.61$\pm$0.88} &\textbf{43.45$\pm$0.81} &\textbf{48.09$\pm$0.60} &\textbf{39.18$\pm$1.87} &\textbf{46.50$\pm$0.25} &\textbf{43.70$\pm$1.38} &\textbf{30.89$\pm$2.13} 
% \\
% \midrule
% TSA-TENT
% & \underline{27.30$\pm$1.61} & \underline{32.84$\pm$0.78} & \underline{40.82$\pm$0.99} & \underline{32.53$\pm$0.91} & \underline{28.62$\pm$0.63} & \underline{27.89$\pm$0.98} & \underline{16.22$\pm$1.17} & \underline{24.87$\pm$0.56} & \underline{22.05$\pm$0.84} & \underline{14.20$\pm$0.71}
% \\
% TSA-LAME
% & \underline{37.95$\pm$0.97} & \underline{36.29$\pm$1.65} & \underline{46.86$\pm$1.13} & \underline{32.86$\pm$2.23} & \underline{27.22$\pm$1.48} & \underline{44.83$\pm$0.88} & \underline{28.51$\pm$0.44} & \underline{39.80$\pm$0.99} & \underline{24.54$\pm$0.87} & \underline{22.39$\pm$0.30}
% \\
% TSA-T3A
% & \underline{\textbf{41.65$\pm$0.99}} & \underline{\textbf{47.01$\pm$2.08}} & \underline{\textbf{51.65$\pm$0.90}} & \underline{\textbf{46.61$\pm$0.88}} & \underline{\textbf{43.45$\pm$0.81}} & \underline{\textbf{48.09$\pm$0.60}} & \underline{\textbf{39.18$\pm$1.87}} & \underline{\textbf{46.50$\pm$0.25}} & \underline{\textbf{43.70$\pm$1.38}} & \underline{\textbf{30.89$\pm$2.13}}
% \\
\midrule
TSA-TENT
& \textbf{27.30$\pm$1.61} & \textbf{32.84$\pm$0.78} & \textbf{40.82$\pm$0.99} & \textbf{32.53$\pm$0.91} & \textbf{28.62$\pm$0.63} & \textbf{27.89$\pm$0.98} & \textbf{16.22$\pm$1.17} & \textbf{24.87$\pm$0.56} & \textbf{22.05$\pm$0.84} & \textbf{14.20$\pm$0.71}
\\
TSA-LAME
& \textbf{37.95$\pm$0.97} & \textbf{36.29$\pm$1.65} & \textbf{46.86$\pm$1.13} & \textbf{32.86$\pm$2.23} & \textbf{27.22$\pm$1.48} & \textbf{44.83$\pm$0.88} & \textbf{28.51$\pm$0.44} & \textbf{39.80$\pm$0.99} & \textbf{24.54$\pm$0.87} & \textbf{22.39$\pm$0.30}
\\
TSA-T3A
& \underline{\textbf{41.65$\pm$0.99}} & \underline{\textbf{47.01$\pm$2.08}} & \underline{\textbf{51.65$\pm$0.90}} & \underline{\textbf{46.61$\pm$0.88}} & \underline{\textbf{43.45$\pm$0.81}} & \underline{\textbf{48.09$\pm$0.60}} & \underline{\textbf{39.18$\pm$1.87}} & \underline{\textbf{46.50$\pm$0.25}} & \underline{\textbf{43.70$\pm$1.38}} & \underline{\textbf{30.89$\pm$2.13}}
\\
\bottomrule
\end{tabular}
\end{adjustbox}
\end{center}
\vspace{-2mm}
\end{table*}


\begin{table*}[t]
\vspace{-5mm}
\caption{Synthetic CSBM results (accuracy). \textbf{Bold} indicates improvements in comparison to the corresponding non-graph TTA baselines. \underline{Underline} indicates the best model. First six: imbalanced source training. Last two: balanced source training. }
\label{table:syn}
\begin{center}
\resizebox{0.9\textwidth}{!}{%
\small
\begin{tabular}{lcccccccc}
\toprule
& \multicolumn{2}{c}{Nbr. Shift} &\multicolumn{2}{c}{Nbr.+ SNR Shift} &\multicolumn{2}{c}{Struct. Shift (Imbal.$\rightarrow$ Bal.)} &\multicolumn{2}{c}{Struct. Shift (Bal.$\rightarrow$ Imbal.)}\\
% \textbf{Method} &2014-2016 &2016-2018 &2014-2016 &2016-2018 &2014-2016 &2016-2018 & D$\rightarrow$A & A$\rightarrow$D\\
\midrule
ERM &82.70$\pm$4.45 &61.11$\pm$10.81 &77.03$\pm$3.99 &61.93$\pm$6.44 &50.41$\pm$4.88 &39.12$\pm$4.71 &68.27$\pm$5.00 &61.39$\pm$1.89 
\\
GTrans &86.67$\pm$3.59 &72.37$\pm$4.06 &79.55$\pm$1.23 &68.69$\pm$3.27 &59.24$\pm$2.12 &47.42$\pm$3.73 &79.29$\pm$2.71 &66.77$\pm$2.52  

\\
SOGA &86.09$\pm$3.89 &70.39$\pm$7.96 &79.75$\pm$3.20 &69.00$\pm$5.28 &56.60$\pm$3.88 &44.09$\pm$5.32 &73.52$\pm$5.30 &63.76$\pm$3.45  
\\
TENT
&87.48$\pm$2.86 &77.14$\pm$4.64 &81.04$\pm$2.72 &72.51$\pm$3.39 &76.48$\pm$6.21 &59.12$\pm$5.06 &77.21$\pm$5.53 &62.36$\pm$6.83  

\\
LAME &83.96$\pm$5.35 &61.44$\pm$11.33 &77.56$\pm$4.92 &62.58$\pm$7.24 &50.33$\pm$4.78 &39.05$\pm$4.67  &68.33$\pm$5.02 &61.26$\pm$1.97 

\\
T3A &77.05$\pm$7.10 &59.83$\pm$10.50 &71.44$\pm$6.11 &56.56$\pm$7.50 &48.13$\pm$5.64 &38.19$\pm$3.72 &68.50$\pm$4.81 &61.63$\pm$1.81  


\\
% \midrule
% TSA-TENT  &88.78$\pm$1.37 &80.51$\pm$2.39 &83.19$\pm$1.46 &76.41$\pm$1.25 &\textbf{88.68$\pm$4.99} &\textbf{66.25$\pm$7.75} &81.20$\pm$8.18 &\textbf{70.15$\pm$2.30} 
% \\
% TSA-LAME &88.96$\pm$1.66 &80.02$\pm$5.44 &83.51$\pm$0.55 &\textbf{79.56$\pm$1.82}  &65.09$\pm$2.34 &52.90$\pm$6.11 &82.20$\pm$5.17 &63.15$\pm$2.58
% \\
% TSA-T3A  &\textbf{89.96$\pm$1.33} &\textbf{81.08$\pm$2.73} &\textbf{84.23$\pm$1.24} &76.89$\pm$2.02 &65.59$\pm$2.57 &52.34$\pm$7.19 &\textbf{82.55$\pm$5.06} &64.88$\pm$3.30  
% \\
\midrule
TSA-TENT  & \textbf{88.78$\pm$1.37} & \textbf{80.51$\pm$2.39} & \textbf{83.19$\pm$1.46} & \textbf{76.41$\pm$1.25} & \underline{\textbf{88.68$\pm$4.99}} & \underline{\textbf{66.25$\pm$7.75}} & \textbf{81.20$\pm$8.18} & \underline{\textbf{70.15$\pm$2.30}} 
\\
TSA-LAME & \textbf{88.96$\pm$1.66} & \textbf{80.02$\pm$5.44} & \textbf{83.51$\pm$0.55} & \underline{\textbf{79.56$\pm$1.82}} & \textbf{65.09$\pm$2.34} & \textbf{52.90$\pm$6.11} & \textbf{82.20$\pm$5.17} & \textbf{63.15$\pm$2.58}
\\
TSA-T3A  & \underline{\textbf{89.96$\pm$1.33}} & \underline{\textbf{81.08$\pm$2.73}} & \underline{\textbf{84.23$\pm$1.24}} & \textbf{76.89$\pm$2.02} & \textbf{65.59$\pm$2.57} & \textbf{52.34$\pm$7.19} & \underline{\textbf{82.55$\pm$5.06}} & \textbf{64.88$\pm$3.30} 
\\
\bottomrule
\end{tabular}
}
\end{center}
\vspace{-3mm}
\end{table*}


%LAME Class0:97.59$\pm$1.16 Class1:93.34$\pm$3.90 Class2:77.00$\pm$10.40 T3A Class0:98.21$\pm$1.05 Class1:90.37$\pm$4.94 Class2:66.86$\pm$14.19 TENT Class0:97.49$\pm$0.89 Class1:93.70$\pm$2.28 Class2:82.70$\pm$5.34


%LAME
%LAME
%LAME

%LAME Class0:99.98$\pm$0.03 Class1:50.62$\pm$14.09 Class2:0.39$\pm$0.69 
% T3A Class0:99.98$\pm$0.03 Class1:42.18$\pm$15.17 Class2:2.23$\pm$3.49 
% TENT Class0:94.32$\pm$4.10 Class1:45.29$\pm$19.91 Class2:89.85$\pm$2.50


%LAME

%LAME Class0:0.00$\pm$0.00 Class1:27.75$\pm$16.73 Class2:100.00$\pm$0.00 
% T3A Class0:0.86$\pm$1.12 Class1:28.05$\pm$15.98 Class2:100.00$\pm$0.00 
% TENT Class0:63.20$\pm$11.75 Class1:45.70$\pm$17.58 Class2:95.29$\pm$1.79


%LAME

\begin{table*}[t]
\vspace{-3mm}
\caption{Pileup results (f1-scores). \textbf{Bold} indicates improvements in comparison to the corresponding non-graph TTA baselines. \underline{Underline} indicates the best model.}
\label{table:pileup}
\begin{center}
\resizebox{0.9\textwidth}{!}{%
\small
\begin{tabular}{lcccccccc}
\toprule
\textbf{Method} & PU10$\rightarrow$30 & PU30$\rightarrow$10 & PU10$\rightarrow$50 & PU50$\rightarrow$10 & PU30$\rightarrow$140 & PU140$\rightarrow$30 &gg$\rightarrow$qq & qq$\rightarrow$gg \\
\midrule
ERM &57.98$\pm$0.66 &65.40$\pm$2.17 &47.66$\pm$1.47 &67.81$\pm$1.70 &19.42$\pm$2.59 &57.49$\pm$3.02 &69.35$\pm$0.81 &67.90$\pm$0.46 
\\
GTrans  &57.37$\pm$1.49 &63.66$\pm$2.43 &48.13$\pm$2.11 &65.74$\pm$1.95 &28.41$\pm$4.01 &57.65$\pm$1.97 &69.17$\pm$0.82 &67.37$\pm$0.56  
\\
SOGA &60.13$\pm$0.75 &69.71$\pm$0.92 &51.68$\pm$0.83 &67.47$\pm$1.42 &37.16$\pm$1.15 &56.59$\pm$3.86 &\underline{70.83$\pm$0.66}
&\underline{68.84$\pm$0.97}  
\\
TENT &58.25$\pm$1.85 &58.73$\pm$4.80 &48.27$\pm$2.58 &58.41$\pm$7.89 &30.59$\pm$3.07 &0.04$\pm$0.08 &68.72$\pm$0.48 &68.00$\pm$0.60 
\\
LAME &57.77$\pm$0.66 &66.29$\pm$2.42 &47.31$\pm$1.58 &68.24$\pm$1.61 &11.96$\pm$3.08 &58.81$\pm$2.00 &69.25$\pm$0.52 &67.94$\pm$0.66 
\\
T3A &58.74$\pm$1.09 &70.02$\pm$2.33 &49.61$\pm$1.03 &70.85$\pm$0.64 &30.33$\pm$3.66 &54.79$\pm$1.52 &69.45$\pm$0.97 &68.45$\pm$0.61  
\\
% % \midrule
% % AdaRC-TENT &32.76$\pm$3.89 &2.01$\pm$0.28 &27.01$\pm$0.06 &2.02$\pm$0.38 &15.44$\pm$1.60 &0.00$\pm$0.00 &2.59$\pm$0.24 &2.35$\pm$0.35 
% % \\
% % AdaRC-LAME &14.60$\pm$2.86 &32.41$\pm$25.54 &37.50$\pm$15.91 &56.76$\pm$28.39 &35.10$\pm$2.14 &42.99$\pm$12.99 &1.28$\pm$0.37 &0.83$\pm$0.33 

% % \\
% % AdaRC-T3A &11.24$\pm$6.46 &23.79$\pm$7.94 &4.30$\pm$6.58 &51.85$\pm$9.61 &36.16$\pm$3.57 &36.79$\pm$16.46 &1.56$\pm$0.64 &1.27$\pm$0.77 
% % \\
% \midrule
% Ours-TENT &54.26$\pm$0.39 &57.72$\pm$3.89 &48.38$\pm$2.55 &58.71$\pm$7.13 &36.11$\pm$2.06 &6.84$\pm$7.48 &68.98$\pm$0.54 &68.57$\pm$0.65 
% \\
% Ours-LAME &55.07$\pm$0.19 &61.65$\pm$1.44 &51.01$\pm$1.50 &68.50$\pm$1.37 &36.89$\pm$1.09 &59.60$\pm$1.71 &69.58$\pm$0.64 &67.99$\pm$0.50
% \\
% Ours-T3A &54.07$\pm$1.56 &65.39$\pm$2.25 &52.27$\pm$0.85 &70.91$\pm$0.63 &36.54$\pm$2.13 &56.34$\pm$1.52 &69.53$\pm$1.00
% &68.72$\pm$0.59
% \\
% \midrule
% Ours-TENT &54.34$\pm$0.43 &56.61$\pm$4.49 &48.05$\pm$2.52 &56.68$\pm$8.15 &37.17$\pm$1.52 &6.65$\pm$7.81 &68.94$\pm$0.63 &68.64$\pm$0.56 
% \\ %**
% Ours-LAME &55.04$\pm$0.25 &61.65$\pm$1.47 &51.28$\pm$1.79 &68.87$\pm$1.21 &36.87$\pm$0.95 &59.86$\pm$1.75 &69.77$\pm$0.69 &67.92$\pm$0.51
% \\
% Ours-T3A &53.86$\pm$1.33 &65.52$\pm$2.42 &52.05$\pm$1.10 &70.94$\pm$0.64 &38.28$\pm$1.08 &56.11$\pm$0.80 &69.75$\pm$1.09 &68.70$\pm$0.76
% \\
% \midrule
% TSA-TENT &58.27$\pm$1.84 &60.11$\pm$3.67 &48.44$\pm$2.67 &58.79$\pm$7.01 &36.22$\pm$2.04 &6.13$\pm$5.93 &69.26$\pm$0.54 &68.80$\pm$0.78
% \\
% TSA-LAME &60.76$\pm$0.71 &66.94$\pm$1.62 &50.72$\pm$1.49 &68.53$\pm$1.55 &36.43$\pm$2.57 &\textbf{59.36$\pm$2.53} &69.85$\pm$0.66 &68.31$\pm$0.54
% \\
% TSA-T3A
% &\textbf{61.03$\pm$1.19} &\textbf{70.42$\pm$1.66} &\textbf{52.39$\pm$0.86} &\textbf{70.92$\pm$0.64} &\textbf{37.33$\pm$2.59} &57.73$\pm$1.98 &69.73$\pm$1.09 &68.75$\pm$0.66
% \\
\midrule
TSA-TENT & \textbf{58.27$\pm$1.84} & \textbf{60.11$\pm$3.67} & \textbf{48.44$\pm$2.67} & \textbf{58.79$\pm$7.01} & \textbf{36.22$\pm$2.04} & \textbf{6.13$\pm$5.93} & \textbf{69.26$\pm$0.54} & \textbf{68.80$\pm$0.78}
\\
TSA-LAME & \textbf{60.76$\pm$0.71} & \textbf{66.94$\pm$1.62} & \textbf{50.72$\pm$1.49} & \textbf{68.53$\pm$1.55} & \textbf{36.43$\pm$2.57} & \underline{\textbf{59.36$\pm$2.53}} & \textbf{69.85$\pm$0.66} & \textbf{68.31$\pm$0.54}
\\
TSA-T3A
& \underline{\textbf{61.03$\pm$1.19}} & \underline{\textbf{70.42$\pm$1.66}} & \underline{\textbf{52.39$\pm$0.86}} & \underline{\textbf{70.92$\pm$0.64}} & \underline{\textbf{37.33$\pm$2.59}} & \textbf{57.73$\pm$1.98} & \textbf{69.73$\pm$1.09} & \textbf{68.75$\pm$0.66}
\\
\bottomrule
\end{tabular}
}
\end{center}
\vspace{-2mm}
\end{table*}


\begin{table*}[t]
\vspace{-5mm}
\captionsetup{justification=centering}  % Center the caption text
\caption{Arxiv and DBLP/ACM no balanced training (accuracy). \textbf{Bold} indicates improvements in comparison to the corresponding non-graph TTA baselines. \underline{Underline} indicates the best model.}
\label{table:arxiv}
\begin{center}
\resizebox{0.9\textwidth}{!}{%
\small
\begin{tabular}{lcccccccc}
\toprule
& \multicolumn{2}{c}{1950-2007} &\multicolumn{2}{c}{1950-2009} &\multicolumn{2}{c}{1950-2011} &\multicolumn{2}{c}{DBLP \& ACM}\\
\textbf{Method} &2014-2016 &2016-2018 &2014-2016 &2016-2018 &2014-2016 &2016-2018 & D$\rightarrow$A & A$\rightarrow$D\\
\midrule
ERM &41.04$\pm$0.50 &40.48$\pm$1.45 &44.80$\pm$1.96 &42.38$\pm$3.64 &53.40$\pm$1.14 &51.68$\pm$1.76 
&28.95$\pm$4.50 &52.45$\pm$6.81 

\\
GTrans &40.92$\pm$0.32 &40.25$\pm$1.69 &45.31$\pm$1.99 &43.83$\pm$3.15 &53.68$\pm$0.95 &52.57$\pm$1.23 
&34.47$\pm$2.99
&49.68$\pm$5.44 
\\
SOGA &34.11$\pm$2.91 &28.94$\pm$4.68 &41.59$\pm$2.03 &39.61$\pm$3.22 &50.12$\pm$1.38 &43.03$\pm$3.70 
&35.74$\pm$5.20 &\underline{58.59$\pm$8.35}
\\
TENT&40.77$\pm$0.32 &39.58$\pm$1.30 &45.74$\pm$1.26 &43.98$\pm$1.90 &54.45$\pm$1.03 &53.19$\pm$1.31 
&36.59$\pm$4.26 &53.38$\pm$6.60 
\\
LAME  &40.91$\pm$0.88 &41.02$\pm$1.81 &45.13$\pm$2.49 &43.23$\pm$4.42 &53.63$\pm$1.21 &52.07$\pm$1.62 
&28.02$\pm$5.08 &52.74$\pm$7.08 
\\
T3A &39.57$\pm$1.06 &38.98$\pm$1.78 &43.34$\pm$1.51 &41.17$\pm$2.88 &50.10$\pm$1.70 &48.00$\pm$2.82 
&35.29$\pm$5.58 &52.50$\pm$6.83 
\\
% \midrule
% % AdaRC-TENT &18.77$\pm$22.84 &24.73$\pm$20.36 &0.20$\pm$0.04 &0.19$\pm$0.03 &0.18$\pm$0.04 &0.16$\pm$0.04 &64.39$\pm$0.94 &78.61$\pm$1.47\\
% % AdaRC-LAME &0.14$\pm$0.03 &0.08$\pm$0.02 &0.13$\pm$0.01 &32.48$\pm$26.48 &0.15$\pm$0.02 &0.10$\pm$0.02 &67.81$\pm$1.75 &70.43$\pm$2.12\\
% % AdaRC-T3A &0.05$\pm$0.02 &0.06$\pm$0.02 &0.07$\pm$0.02 &0.05$\pm$0.03 &0.04$\pm$0.02 &9.63$\pm$19.19 &67.62$\pm$1.86 &73.03$\pm$2.54\\
% % \midrule
% Ours-TENT &40.89$\pm$0.38 &39.81$\pm$1.23 &46.21$\pm$1.25 &44.52$\pm$1.79 &54.60$\pm$1.04 &53.44$\pm$1.29
% &36.76$\pm$4.05 &53.76$\pm$6.61  
% \\
% Ours-LAME &41.27$\pm$0.77 &41.02$\pm$1.65 &45.43$\pm$2.44 &43.65$\pm$4.59 &54.14$\pm$0.90 &52.54$\pm$1.25 
% &28.52$\pm$5.08 &52.57$\pm$7.21 
% \\
% Ours-T3A &40.07$\pm$0.98 &39.65$\pm$1.64 &44.11$\pm$1.50 &42.16$\pm$2.79 &50.91$\pm$1.55 &49.03$\pm$2.64 
% &32.99$\pm$4.73 &52.51$\pm$6.95 
% \\
% \midrule
% Ours-TENT &40.89$\pm$0.41 &39.80$\pm$1.22 &46.30$\pm$0.85 &44.89$\pm$1.34 &54.83$\pm$0.84 &53.63$\pm$1.16 &36.99$\pm$3.93 &53.46$\pm$6.41
% \\
% Ours-LAME &41.29$\pm$0.71 &41.02$\pm$1.77 &45.54$\pm$2.73 &43.65$\pm$4.26 &54.46$\pm$0.94 &52.93$\pm$1.46  &28.51$\pm$5.06 &52.78$\pm$7.78 
% \\
% Ours-T3A &39.91$\pm$1.04 &39.56$\pm$1.59 &44.02$\pm$1.54 &41.99$\pm$2.79 &50.67$\pm$1.69 &48.58$\pm$2.77 &32.49$\pm$5.10 &52.10$\pm$6.88
% \\
\midrule
TSA-TENT & \textbf{40.92$\pm$0.34} & \textbf{39.78$\pm$1.21} & \underline{\textbf{46.68$\pm$1.24}} & \underline{\textbf{45.08$\pm$1.71}} & \underline{\textbf{54.78$\pm$0.80}} & \underline{\textbf{53.61$\pm$1.24}} & \underline{\textbf{37.06$\pm$3.93}} & \textbf{54.06$\pm$6.63}
\\
TSA-LAME & \underline{\textbf{41.34$\pm$0.83}} & \underline{\textbf{41.23$\pm$1.70}} & \textbf{45.42$\pm$2.45} & \textbf{43.71$\pm$4.99} & \textbf{54.05$\pm$1.05} & \textbf{52.76$\pm$1.47} & \textbf{28.66$\pm$5.27} & \textbf{52.90$\pm$7.30}
\\
TSA-T3A & \textbf{40.03$\pm$1.03} & \textbf{39.70$\pm$1.49} & \textbf{44.09$\pm$1.53} & \textbf{42.17$\pm$2.91} & \textbf{50.96$\pm$1.68} & \textbf{49.21$\pm$2.67} & \textbf{35.30$\pm$5.57} & \textbf{52.55$\pm$6.92}
\\
\bottomrule
\end{tabular}
}
\end{center}
\vspace{-4mm}
\end{table*}




We evaluate \proj %by comparing with non-graph TTA methods and existing GTTA methdos the combination with the three non-graph TTA methods and compared to basedlines from TTA
on synthetic datasets and 5 real-world datasets. More discussions and results in experiments can be found in Appendix~\ref{app:exp}.

\subsection{Datasets and Baselines}

\textbf{Synthetic Data.} 
We use the CSBM model~\cite{deshpande2018contextual} to generate three-class datasets, focusing on structure shifts while keeping feature distributions unchanged. Specifically, we evaluate performance under three conditions: (1) neighborhood shift, (2) neighborhood shift plus SNR (induced by degree changes) shift, and (3) neighborhood shift combined with both SNR shift and label shift. This experimental setup is motivated by Thm. \ref{theory:decomposeerror} and the observations in Sec.\ref{subsec:snr}. For each shift scenario, we examine two levels of severity, with the left column in later Table~\ref{table:syn} corresponding to the smaller shift.

\textbf{MAG~\cite{liu2024pairwise}} is a citation network extracted by OGB-MAG \cite{hu2020open}.
Distribution shifts arise from partitioning papers into distinct graphs based on their countries of publication. The task is to classify the publication venue of each paper. Our model is pretrained on graphs from the US and China and subsequently adapted to graphs from other countries.

\textbf{Pileup Mitigation~\cite{liu2023structural}} is a dataset curated for the data-denoising step in high energy physics~\cite{bertolini2014pileup}.
Particles are generated by proton-proton collisions in LHC experiments. The task is to classify leading-collision (LC) neutral particles from other-collision (OC) particles. Particles are connected via kNN graphs if they are close in the $\eta-\phi$ space shown in Fig. \ref{fig:pileup_example}.
Distribution shifts arise from pile-up (PU) conditions (primarily structure shift), where PU level indicates the number of OC in the beam, and from the particle generation processes gg$\rightarrow$qq and qq$\rightarrow$gg (primarily feature shift).


\textbf{Arxiv~\cite{hu2020open}} is a citation network between all Computer Science (CS) arXiv papers. 
Distribution shifts originate from different time.
Our model is pretrained on the earlier time span 1950 to 2007/ 2009/ 2011 and test on later 2014 to 2016 and 2016 to 2018.


\textbf{DBLP and ACM~\cite{tang2008arnetminer, wu2020unsupervised}} are two citation networks. 
The model is trained on one network and adapted to the other to predict the research topic of a paper (node).
% \hans{I was thinking to remove DMLP ACM because the variance it too large same in PairAlign and SOGA did pretty well on A to D}

\textbf{Baselines} We compare TSA with six baselines. For non-graph TTA methods, we include TENT \cite{wang2020tent}, LAME \cite{boudiaf2022parameter}, and T3A \cite{iwasawa2021test}. Note that TENT is only applied to the classifier, as GNNs typically do not include batch normalization layers. GTrans \cite{jin2022empowering}, SOGA \cite{mao2024source}, and AdaRC \cite{bao2024adarc} are direct comparisons in GTTA.
AdaRC is limited to GPRGNN due to its design. We present the results for GraphSAGE \cite{hamilton2017inductive} in the main paper, while the results for GPRGNN \cite{chien2020adaptive} (including AdaRC) are provided in the Appendix.
% TSA results are presented for one epoch of optimization for Eq. \ref{eq:loss}.

% \vspace{-3mm}
\subsection{Result Analysis}
From the MAG dataset results in Table~\ref{table:mag}, TSA combined with boundary refinement techniques consistently achieves top performance among all baselines. Compared to non-graph TTA baselines, such as TENT vs. TSA-TENT, TSA also provides considerable improvements over the original adaptation results. Notably, the benefits of TSA after neighborhood alignment are more pronounced in scenarios with larger neighborhood shifts, such as adaptation from the US to FR, as reflected in the dataset shift metrics (Table~\ref{table:magstats}). 

Among all baselines, T3A consistently outperforms as a representative non-gradient-based method that refines the decision boundary, while TENT struggles in most scenarios. This is primarily due to the relatively imbalanced class distribution in the MAG datasets, where evaluation is conducted on the top 19 classes. Since TENT learns feature-wise transformations through entropy optimization on the target graph, it tends to be biased toward the dominant class. Other GTTA methods, such as GTrans, achieve performance comparable to ERM, whereas SOGA sometimes underperforms due to its strong reliance on the graph homophily assumption, which is often weak in sparse graphs with many classes, such as MAG and Arxiv.

The results for the synthetic datasets are presented in Table~\ref{table:syn}.
The first six columns correspond to experiments conducted under imbalanced source training.
Compared to real datasets, we observe more significant advantages from TSA. TENT outperforms LAME and T3A, as it directly maximizes the most probable class through entropy minimization. In contrast, LAME incorporates regularization, while T3A relies on distance to prototypes, making it less sensitive to majority-class dominance \cite{zhang2022divide}. When combined with \proj, we observe similar results across all three variants, indicating that the primary adaptation ability stems from \proj rather than non-graph TTA methods. While GTTA baselines perform better than ERM, they still fall short of our TSA-based methods. \proj also demonstrates strong performance under additional shifts in SNR caused by degree shifts. 
Additionally, we consider training a pretrained model on a balanced source domain under the same edge connection probability as the previous two columns.
The last two columns demonstrate better performance than the previous two under balanced training, even though they encounter the same shift in label connection.
This showcases that dataset imbalance is a severe issue in source training, which aligns with the worst-case error described in Theorem \ref{theory:decomposeerror}. 

The pileup results in Table~\ref{table:pileup} demonstrate the effectiveness of TSA-based methods under both neighborhood shift and label shift, with each module contributing in different ways. The neighborhood alignment module provides significant improvements when generalizing from a low PU level to a high PU level, as this scenario is primarily dominated by neighborhood shift. Conversely, when adapting from a high PU level to a low PU level, label shift becomes the dominant factor. In this case, incorporating non-graph TTA techniques is crucial, while neighborhood alignment offers an additional but smaller benefit. Moreover, TENT can be vulnerable when generalizing from an imbalanced source graph to a balanced target graph. 
This is because source training on high PU graphs suffers from class imbalance, making entropy minimization non-robust due to a mismatched decision boundary when adapting to low PU graphs. The last two columns, which represent the same PU level but different physical signals, show minimal benefits from TSA. This is expected, as these scenarios primarily exhibit feature shifts rather than structure or label shifts.

The results from Arxiv and DBLP/ACM in Table~\ref{table:arxiv} confirm that TSA can be effectively integrated with non-graph TTA methods to mitigate feature shift while also facilitating reasonable adjustments for structure shifts. These citation networks exhibit relatively mild structure shifts, leading to closely comparable performance across all baselines. TENT and TSA-TENT stand out by maximizing the majority classes through entropy minimization. Nonetheless, we still observe meaningful improvements with the addition of TSA when compared to the corresponding baselines, demonstrating its effectiveness in enhancing adaptation.




% Neighborhood alignment are essential in handling neighborhood shift, SNR-based representation adjustment can alleviate SNR shift and decision boundary refinement should be added on top of the other techniques to compensate additional label and feature shift.



\subsection{Analysis of $\alpha$ for SNR-induced refinement}
\label{subsec:alpha}
As defined in Eq.\ref{eq:alpha_ratio}, $\alpha$ is learned, depending on node degrees and GNN layer depth. The trend of $\alpha$ in Fig.~\ref{fig:pa_vis} (d) aligns with our expectations: earlier GNN layers require less attention to self-node representations (small $\alpha$), while deeper layers increasingly rely on them. Additionally, the plot reveals a correlation with node degrees, where nodes with higher degrees (and thus higher SNR) place greater weight on neighborhood-aggregated messages. However, the overall effect of different GNN layers appears to be greater than that of varying node degrees. % as we observed that $\alpha$ for does not goes up even when the degree becomes larger. 

\subsection{Ablation Studies}
In Table~\ref{table:ablation} (Appendix), we examine the effects of neighborhood alignment and SNR-based representation adjustments. The results consistently show that incorporating both modules yields the best performance. In some cases, removing SNR adjustments leads to a greater performance drop compared to removing neighborhood alignment. However, this does not necessarily indicate that neighborhood alignment is less important. Instead, we think this effect arises because the additional backpropagation training for SNR adjustments amplifies their contributions. 

% \begin{figure}[t]
%     \centering
%         \centering
%         \includegraphics[width=0.45\columnwidth]{icml2025/figures/uscn_alpha.pdf}
%         \centering
%         \includegraphics[width=0.45\columnwidth]{icml2025/figures/cnus_alpha.pdf}
%     \caption{Analysis of SNR adjustment $\alpha$ with respect to different layers and node degrees on MAG $US\rightarrow CN$ and $CN \rightarrow US$.}
%     \label{fig:side_by_side}
% \end{figure}







\section{Conclusion}
In this work, we propose a simple yet effective approach, called SMILE, for graph few-shot learning with fewer tasks. Specifically, we introduce a novel dual-level mixup strategy, including within-task and across-task mixup, for enriching the diversity of nodes within each task and the diversity of tasks. Also, we incorporate the degree-based prior information to learn expressive node embeddings. Theoretically, we prove that SMILE effectively enhances the model's generalization performance. Empirically, we conduct extensive experiments on multiple benchmarks and the results suggest that SMILE significantly outperforms other baselines, including both in-domain and cross-domain few-shot settings.




% Note use of \abovespace and \belowspace to get reasonable spacing
% above and below tabular lines.

% \begin{table}[t]
% \caption{Classification accuracies for naive Bayes and flexible
% Bayes on various data sets.}
% \label{sample-table}
% \vskip 0.15in
% \begin{center}
% \begin{small}
% \begin{sc}
% \begin{tabular}{lcccr}
% \toprule
% Data set & Naive & Flexible & Better? \\
% \midrule
% Breast    & 95.9$\pm$ 0.2& 96.7$\pm$ 0.2& $\surd$ \\
% Cleveland & 83.3$\pm$ 0.6& 80.0$\pm$ 0.6& $\times$\\
% Glass2    & 61.9$\pm$ 1.4& 83.8$\pm$ 0.7& $\surd$ \\
% Credit    & 74.8$\pm$ 0.5& 78.3$\pm$ 0.6&         \\
% Horse     & 73.3$\pm$ 0.9& 69.7$\pm$ 1.0& $\times$\\
% Meta      & 67.1$\pm$ 0.6& 76.5$\pm$ 0.5& $\surd$ \\
% Pima      & 75.1$\pm$ 0.6& 73.9$\pm$ 0.5&         \\
% Vehicle   & 44.9$\pm$ 0.6& 61.5$\pm$ 0.4& $\surd$ \\
% \bottomrule
% \end{tabular}
% \end{sc}
% \end{small}
% \end{center}
% \vskip -0.1in
% \end{table}


% \subsection{Theorems and such}
% The preferred way is to number definitions, propositions, lemmas, etc. consecutively, within sections, as shown below.
% \begin{definition}
% \label{def:inj}
% A function $f:X \to Y$ is injective if for any $x,y\in X$ different, $f(x)\ne f(y)$.
% \end{definition}
% Using \cref{def:inj} we immediate get the following result:
% \begin{proposition}
% If $f$ is injective mapping a set $X$ to another set $Y$, 
% the cardinality of $Y$ is at least as large as that of $X$
% \end{proposition}
% \begin{proof} 
% Left as an exercise to the reader. 
% \end{proof}
% \cref{lem:usefullemma} stated next will prove to be useful.
% \begin{lemma}
% \label{lem:usefullemma}
% For any $f:X \to Y$ and $g:Y\to Z$ injective functions, $f \circ g$ is injective.
% \end{lemma}
% \begin{theorem}
% \label{thm:bigtheorem}
% If $f:X\to Y$ is bijective, the cardinality of $X$ and $Y$ are the same.
% \end{theorem}
% An easy corollary of \cref{thm:bigtheorem} is the following:
% \begin{corollary}
% If $f:X\to Y$ is bijective, 
% the cardinality of $X$ is at least as large as that of $Y$.
% \end{corollary}
% \begin{assumption}
% The set $X$ is finite.
% \label{ass:xfinite}
% \end{assumption}
% \begin{remark}
% According to some, it is only the finite case (cf. \cref{ass:xfinite}) that is interesting.
% \end{remark}
%restatable



% In the unusual situation where you want a paper to appear in the
% references without citing it in the main text, use \nocite
% \nocite{langley00}

% \newpage
\section{Broader Impact}
This paper presents work whose goal is to advance the field of Machine Learning. There are many potential societal consequences of our work, none which we feel must be specifically highlighted here.
\section{Acknowledgment}
H. Hsu, S. Liu, and P. Li are partially supported by NSF awards PHY-2117997, IIS-2239565, IIS-2428777, and CCF-2402816; DOE award DE-FOA-0002785; JPMC faculty awards; and Microsoft Azure Research Credits for Generative AI.


\bibliography{main}
\bibliographystyle{icml2025}


%%%%%%%%%%%%%%%%%%%%%%%%%%%%%%%%%%%%%%%%%%%%%%%%%%%%%%%%%%%%%%%%%%%%%%%%%%%%%%%
%%%%%%%%%%%%%%%%%%%%%%%%%%%%%%%%%%%%%%%%%%%%%%%%%%%%%%%%%%%%%%%%%%%%%%%%%%%%%%%
% APPENDIX
%%%%%%%%%%%%%%%%%%%%%%%%%%%%%%%%%%%%%%%%%%%%%%%%%%%%%%%%%%%%%%%%%%%%%%%%%%%%%%%
%%%%%%%%%%%%%%%%%%%%%%%%%%%%%%%%%%%%%%%%%%%%%%%%%%%%%%%%%%%%%%%%%%%%%%%%%%%%%%%
\newpage
\appendix
\onecolumn
% \section{You \emph{can} have an appendix here.}


% You can have as much text here as you want. The main body must be at most $8$ pages long.
% For the final version, one more page can be added.
% If you want, you can use an appendix like this one.  

% The $\mathtt{\backslash onecolumn}$ command above can be kept in place if you prefer a one-column appendix, or can be removed if you prefer a two-column appendix.  Apart from this possible change, the style (font size, spacing, margins, page numbering, etc.) should be kept the same as the main body.

\subsection{Lloyd-Max Algorithm}
\label{subsec:Lloyd-Max}
For a given quantization bitwidth $B$ and an operand $\bm{X}$, the Lloyd-Max algorithm finds $2^B$ quantization levels $\{\hat{x}_i\}_{i=1}^{2^B}$ such that quantizing $\bm{X}$ by rounding each scalar in $\bm{X}$ to the nearest quantization level minimizes the quantization MSE. 

The algorithm starts with an initial guess of quantization levels and then iteratively computes quantization thresholds $\{\tau_i\}_{i=1}^{2^B-1}$ and updates quantization levels $\{\hat{x}_i\}_{i=1}^{2^B}$. Specifically, at iteration $n$, thresholds are set to the midpoints of the previous iteration's levels:
\begin{align*}
    \tau_i^{(n)}=\frac{\hat{x}_i^{(n-1)}+\hat{x}_{i+1}^{(n-1)}}2 \text{ for } i=1\ldots 2^B-1
\end{align*}
Subsequently, the quantization levels are re-computed as conditional means of the data regions defined by the new thresholds:
\begin{align*}
    \hat{x}_i^{(n)}=\mathbb{E}\left[ \bm{X} \big| \bm{X}\in [\tau_{i-1}^{(n)},\tau_i^{(n)}] \right] \text{ for } i=1\ldots 2^B
\end{align*}
where to satisfy boundary conditions we have $\tau_0=-\infty$ and $\tau_{2^B}=\infty$. The algorithm iterates the above steps until convergence.

Figure \ref{fig:lm_quant} compares the quantization levels of a $7$-bit floating point (E3M3) quantizer (left) to a $7$-bit Lloyd-Max quantizer (right) when quantizing a layer of weights from the GPT3-126M model at a per-tensor granularity. As shown, the Lloyd-Max quantizer achieves substantially lower quantization MSE. Further, Table \ref{tab:FP7_vs_LM7} shows the superior perplexity achieved by Lloyd-Max quantizers for bitwidths of $7$, $6$ and $5$. The difference between the quantizers is clear at 5 bits, where per-tensor FP quantization incurs a drastic and unacceptable increase in perplexity, while Lloyd-Max quantization incurs a much smaller increase. Nevertheless, we note that even the optimal Lloyd-Max quantizer incurs a notable ($\sim 1.5$) increase in perplexity due to the coarse granularity of quantization. 

\begin{figure}[h]
  \centering
  \includegraphics[width=0.7\linewidth]{sections/figures/LM7_FP7.pdf}
  \caption{\small Quantization levels and the corresponding quantization MSE of Floating Point (left) vs Lloyd-Max (right) Quantizers for a layer of weights in the GPT3-126M model.}
  \label{fig:lm_quant}
\end{figure}

\begin{table}[h]\scriptsize
\begin{center}
\caption{\label{tab:FP7_vs_LM7} \small Comparing perplexity (lower is better) achieved by floating point quantizers and Lloyd-Max quantizers on a GPT3-126M model for the Wikitext-103 dataset.}
\begin{tabular}{c|cc|c}
\hline
 \multirow{2}{*}{\textbf{Bitwidth}} & \multicolumn{2}{|c|}{\textbf{Floating-Point Quantizer}} & \textbf{Lloyd-Max Quantizer} \\
 & Best Format & Wikitext-103 Perplexity & Wikitext-103 Perplexity \\
\hline
7 & E3M3 & 18.32 & 18.27 \\
6 & E3M2 & 19.07 & 18.51 \\
5 & E4M0 & 43.89 & 19.71 \\
\hline
\end{tabular}
\end{center}
\end{table}

\subsection{Proof of Local Optimality of LO-BCQ}
\label{subsec:lobcq_opt_proof}
For a given block $\bm{b}_j$, the quantization MSE during LO-BCQ can be empirically evaluated as $\frac{1}{L_b}\lVert \bm{b}_j- \bm{\hat{b}}_j\rVert^2_2$ where $\bm{\hat{b}}_j$ is computed from equation (\ref{eq:clustered_quantization_definition}) as $C_{f(\bm{b}_j)}(\bm{b}_j)$. Further, for a given block cluster $\mathcal{B}_i$, we compute the quantization MSE as $\frac{1}{|\mathcal{B}_{i}|}\sum_{\bm{b} \in \mathcal{B}_{i}} \frac{1}{L_b}\lVert \bm{b}- C_i^{(n)}(\bm{b})\rVert^2_2$. Therefore, at the end of iteration $n$, we evaluate the overall quantization MSE $J^{(n)}$ for a given operand $\bm{X}$ composed of $N_c$ block clusters as:
\begin{align*}
    \label{eq:mse_iter_n}
    J^{(n)} = \frac{1}{N_c} \sum_{i=1}^{N_c} \frac{1}{|\mathcal{B}_{i}^{(n)}|}\sum_{\bm{v} \in \mathcal{B}_{i}^{(n)}} \frac{1}{L_b}\lVert \bm{b}- B_i^{(n)}(\bm{b})\rVert^2_2
\end{align*}

At the end of iteration $n$, the codebooks are updated from $\mathcal{C}^{(n-1)}$ to $\mathcal{C}^{(n)}$. However, the mapping of a given vector $\bm{b}_j$ to quantizers $\mathcal{C}^{(n)}$ remains as  $f^{(n)}(\bm{b}_j)$. At the next iteration, during the vector clustering step, $f^{(n+1)}(\bm{b}_j)$ finds new mapping of $\bm{b}_j$ to updated codebooks $\mathcal{C}^{(n)}$ such that the quantization MSE over the candidate codebooks is minimized. Therefore, we obtain the following result for $\bm{b}_j$:
\begin{align*}
\frac{1}{L_b}\lVert \bm{b}_j - C_{f^{(n+1)}(\bm{b}_j)}^{(n)}(\bm{b}_j)\rVert^2_2 \le \frac{1}{L_b}\lVert \bm{b}_j - C_{f^{(n)}(\bm{b}_j)}^{(n)}(\bm{b}_j)\rVert^2_2
\end{align*}

That is, quantizing $\bm{b}_j$ at the end of the block clustering step of iteration $n+1$ results in lower quantization MSE compared to quantizing at the end of iteration $n$. Since this is true for all $\bm{b} \in \bm{X}$, we assert the following:
\begin{equation}
\begin{split}
\label{eq:mse_ineq_1}
    \tilde{J}^{(n+1)} &= \frac{1}{N_c} \sum_{i=1}^{N_c} \frac{1}{|\mathcal{B}_{i}^{(n+1)}|}\sum_{\bm{b} \in \mathcal{B}_{i}^{(n+1)}} \frac{1}{L_b}\lVert \bm{b} - C_i^{(n)}(b)\rVert^2_2 \le J^{(n)}
\end{split}
\end{equation}
where $\tilde{J}^{(n+1)}$ is the the quantization MSE after the vector clustering step at iteration $n+1$.

Next, during the codebook update step (\ref{eq:quantizers_update}) at iteration $n+1$, the per-cluster codebooks $\mathcal{C}^{(n)}$ are updated to $\mathcal{C}^{(n+1)}$ by invoking the Lloyd-Max algorithm \citep{Lloyd}. We know that for any given value distribution, the Lloyd-Max algorithm minimizes the quantization MSE. Therefore, for a given vector cluster $\mathcal{B}_i$ we obtain the following result:

\begin{equation}
    \frac{1}{|\mathcal{B}_{i}^{(n+1)}|}\sum_{\bm{b} \in \mathcal{B}_{i}^{(n+1)}} \frac{1}{L_b}\lVert \bm{b}- C_i^{(n+1)}(\bm{b})\rVert^2_2 \le \frac{1}{|\mathcal{B}_{i}^{(n+1)}|}\sum_{\bm{b} \in \mathcal{B}_{i}^{(n+1)}} \frac{1}{L_b}\lVert \bm{b}- C_i^{(n)}(\bm{b})\rVert^2_2
\end{equation}

The above equation states that quantizing the given block cluster $\mathcal{B}_i$ after updating the associated codebook from $C_i^{(n)}$ to $C_i^{(n+1)}$ results in lower quantization MSE. Since this is true for all the block clusters, we derive the following result: 
\begin{equation}
\begin{split}
\label{eq:mse_ineq_2}
     J^{(n+1)} &= \frac{1}{N_c} \sum_{i=1}^{N_c} \frac{1}{|\mathcal{B}_{i}^{(n+1)}|}\sum_{\bm{b} \in \mathcal{B}_{i}^{(n+1)}} \frac{1}{L_b}\lVert \bm{b}- C_i^{(n+1)}(\bm{b})\rVert^2_2  \le \tilde{J}^{(n+1)}   
\end{split}
\end{equation}

Following (\ref{eq:mse_ineq_1}) and (\ref{eq:mse_ineq_2}), we find that the quantization MSE is non-increasing for each iteration, that is, $J^{(1)} \ge J^{(2)} \ge J^{(3)} \ge \ldots \ge J^{(M)}$ where $M$ is the maximum number of iterations. 
%Therefore, we can say that if the algorithm converges, then it must be that it has converged to a local minimum. 
\hfill $\blacksquare$


\begin{figure}
    \begin{center}
    \includegraphics[width=0.5\textwidth]{sections//figures/mse_vs_iter.pdf}
    \end{center}
    \caption{\small NMSE vs iterations during LO-BCQ compared to other block quantization proposals}
    \label{fig:nmse_vs_iter}
\end{figure}

Figure \ref{fig:nmse_vs_iter} shows the empirical convergence of LO-BCQ across several block lengths and number of codebooks. Also, the MSE achieved by LO-BCQ is compared to baselines such as MXFP and VSQ. As shown, LO-BCQ converges to a lower MSE than the baselines. Further, we achieve better convergence for larger number of codebooks ($N_c$) and for a smaller block length ($L_b$), both of which increase the bitwidth of BCQ (see Eq \ref{eq:bitwidth_bcq}).


\subsection{Additional Accuracy Results}
%Table \ref{tab:lobcq_config} lists the various LOBCQ configurations and their corresponding bitwidths.
\begin{table}
\setlength{\tabcolsep}{4.75pt}
\begin{center}
\caption{\label{tab:lobcq_config} Various LO-BCQ configurations and their bitwidths.}
\begin{tabular}{|c||c|c|c|c||c|c||c|} 
\hline
 & \multicolumn{4}{|c||}{$L_b=8$} & \multicolumn{2}{|c||}{$L_b=4$} & $L_b=2$ \\
 \hline
 \backslashbox{$L_A$\kern-1em}{\kern-1em$N_c$} & 2 & 4 & 8 & 16 & 2 & 4 & 2 \\
 \hline
 64 & 4.25 & 4.375 & 4.5 & 4.625 & 4.375 & 4.625 & 4.625\\
 \hline
 32 & 4.375 & 4.5 & 4.625& 4.75 & 4.5 & 4.75 & 4.75 \\
 \hline
 16 & 4.625 & 4.75& 4.875 & 5 & 4.75 & 5 & 5 \\
 \hline
\end{tabular}
\end{center}
\end{table}

%\subsection{Perplexity achieved by various LO-BCQ configurations on Wikitext-103 dataset}

\begin{table} \centering
\begin{tabular}{|c||c|c|c|c||c|c||c|} 
\hline
 $L_b \rightarrow$& \multicolumn{4}{c||}{8} & \multicolumn{2}{c||}{4} & 2\\
 \hline
 \backslashbox{$L_A$\kern-1em}{\kern-1em$N_c$} & 2 & 4 & 8 & 16 & 2 & 4 & 2  \\
 %$N_c \rightarrow$ & 2 & 4 & 8 & 16 & 2 & 4 & 2 \\
 \hline
 \hline
 \multicolumn{8}{c}{GPT3-1.3B (FP32 PPL = 9.98)} \\ 
 \hline
 \hline
 64 & 10.40 & 10.23 & 10.17 & 10.15 &  10.28 & 10.18 & 10.19 \\
 \hline
 32 & 10.25 & 10.20 & 10.15 & 10.12 &  10.23 & 10.17 & 10.17 \\
 \hline
 16 & 10.22 & 10.16 & 10.10 & 10.09 &  10.21 & 10.14 & 10.16 \\
 \hline
  \hline
 \multicolumn{8}{c}{GPT3-8B (FP32 PPL = 7.38)} \\ 
 \hline
 \hline
 64 & 7.61 & 7.52 & 7.48 &  7.47 &  7.55 &  7.49 & 7.50 \\
 \hline
 32 & 7.52 & 7.50 & 7.46 &  7.45 &  7.52 &  7.48 & 7.48  \\
 \hline
 16 & 7.51 & 7.48 & 7.44 &  7.44 &  7.51 &  7.49 & 7.47  \\
 \hline
\end{tabular}
\caption{\label{tab:ppl_gpt3_abalation} Wikitext-103 perplexity across GPT3-1.3B and 8B models.}
\end{table}

\begin{table} \centering
\begin{tabular}{|c||c|c|c|c||} 
\hline
 $L_b \rightarrow$& \multicolumn{4}{c||}{8}\\
 \hline
 \backslashbox{$L_A$\kern-1em}{\kern-1em$N_c$} & 2 & 4 & 8 & 16 \\
 %$N_c \rightarrow$ & 2 & 4 & 8 & 16 & 2 & 4 & 2 \\
 \hline
 \hline
 \multicolumn{5}{|c|}{Llama2-7B (FP32 PPL = 5.06)} \\ 
 \hline
 \hline
 64 & 5.31 & 5.26 & 5.19 & 5.18  \\
 \hline
 32 & 5.23 & 5.25 & 5.18 & 5.15  \\
 \hline
 16 & 5.23 & 5.19 & 5.16 & 5.14  \\
 \hline
 \multicolumn{5}{|c|}{Nemotron4-15B (FP32 PPL = 5.87)} \\ 
 \hline
 \hline
 64  & 6.3 & 6.20 & 6.13 & 6.08  \\
 \hline
 32  & 6.24 & 6.12 & 6.07 & 6.03  \\
 \hline
 16  & 6.12 & 6.14 & 6.04 & 6.02  \\
 \hline
 \multicolumn{5}{|c|}{Nemotron4-340B (FP32 PPL = 3.48)} \\ 
 \hline
 \hline
 64 & 3.67 & 3.62 & 3.60 & 3.59 \\
 \hline
 32 & 3.63 & 3.61 & 3.59 & 3.56 \\
 \hline
 16 & 3.61 & 3.58 & 3.57 & 3.55 \\
 \hline
\end{tabular}
\caption{\label{tab:ppl_llama7B_nemo15B} Wikitext-103 perplexity compared to FP32 baseline in Llama2-7B and Nemotron4-15B, 340B models}
\end{table}

%\subsection{Perplexity achieved by various LO-BCQ configurations on MMLU dataset}


\begin{table} \centering
\begin{tabular}{|c||c|c|c|c||c|c|c|c|} 
\hline
 $L_b \rightarrow$& \multicolumn{4}{c||}{8} & \multicolumn{4}{c||}{8}\\
 \hline
 \backslashbox{$L_A$\kern-1em}{\kern-1em$N_c$} & 2 & 4 & 8 & 16 & 2 & 4 & 8 & 16  \\
 %$N_c \rightarrow$ & 2 & 4 & 8 & 16 & 2 & 4 & 2 \\
 \hline
 \hline
 \multicolumn{5}{|c|}{Llama2-7B (FP32 Accuracy = 45.8\%)} & \multicolumn{4}{|c|}{Llama2-70B (FP32 Accuracy = 69.12\%)} \\ 
 \hline
 \hline
 64 & 43.9 & 43.4 & 43.9 & 44.9 & 68.07 & 68.27 & 68.17 & 68.75 \\
 \hline
 32 & 44.5 & 43.8 & 44.9 & 44.5 & 68.37 & 68.51 & 68.35 & 68.27  \\
 \hline
 16 & 43.9 & 42.7 & 44.9 & 45 & 68.12 & 68.77 & 68.31 & 68.59  \\
 \hline
 \hline
 \multicolumn{5}{|c|}{GPT3-22B (FP32 Accuracy = 38.75\%)} & \multicolumn{4}{|c|}{Nemotron4-15B (FP32 Accuracy = 64.3\%)} \\ 
 \hline
 \hline
 64 & 36.71 & 38.85 & 38.13 & 38.92 & 63.17 & 62.36 & 63.72 & 64.09 \\
 \hline
 32 & 37.95 & 38.69 & 39.45 & 38.34 & 64.05 & 62.30 & 63.8 & 64.33  \\
 \hline
 16 & 38.88 & 38.80 & 38.31 & 38.92 & 63.22 & 63.51 & 63.93 & 64.43  \\
 \hline
\end{tabular}
\caption{\label{tab:mmlu_abalation} Accuracy on MMLU dataset across GPT3-22B, Llama2-7B, 70B and Nemotron4-15B models.}
\end{table}


%\subsection{Perplexity achieved by various LO-BCQ configurations on LM evaluation harness}

\begin{table} \centering
\begin{tabular}{|c||c|c|c|c||c|c|c|c|} 
\hline
 $L_b \rightarrow$& \multicolumn{4}{c||}{8} & \multicolumn{4}{c||}{8}\\
 \hline
 \backslashbox{$L_A$\kern-1em}{\kern-1em$N_c$} & 2 & 4 & 8 & 16 & 2 & 4 & 8 & 16  \\
 %$N_c \rightarrow$ & 2 & 4 & 8 & 16 & 2 & 4 & 2 \\
 \hline
 \hline
 \multicolumn{5}{|c|}{Race (FP32 Accuracy = 37.51\%)} & \multicolumn{4}{|c|}{Boolq (FP32 Accuracy = 64.62\%)} \\ 
 \hline
 \hline
 64 & 36.94 & 37.13 & 36.27 & 37.13 & 63.73 & 62.26 & 63.49 & 63.36 \\
 \hline
 32 & 37.03 & 36.36 & 36.08 & 37.03 & 62.54 & 63.51 & 63.49 & 63.55  \\
 \hline
 16 & 37.03 & 37.03 & 36.46 & 37.03 & 61.1 & 63.79 & 63.58 & 63.33  \\
 \hline
 \hline
 \multicolumn{5}{|c|}{Winogrande (FP32 Accuracy = 58.01\%)} & \multicolumn{4}{|c|}{Piqa (FP32 Accuracy = 74.21\%)} \\ 
 \hline
 \hline
 64 & 58.17 & 57.22 & 57.85 & 58.33 & 73.01 & 73.07 & 73.07 & 72.80 \\
 \hline
 32 & 59.12 & 58.09 & 57.85 & 58.41 & 73.01 & 73.94 & 72.74 & 73.18  \\
 \hline
 16 & 57.93 & 58.88 & 57.93 & 58.56 & 73.94 & 72.80 & 73.01 & 73.94  \\
 \hline
\end{tabular}
\caption{\label{tab:mmlu_abalation} Accuracy on LM evaluation harness tasks on GPT3-1.3B model.}
\end{table}

\begin{table} \centering
\begin{tabular}{|c||c|c|c|c||c|c|c|c|} 
\hline
 $L_b \rightarrow$& \multicolumn{4}{c||}{8} & \multicolumn{4}{c||}{8}\\
 \hline
 \backslashbox{$L_A$\kern-1em}{\kern-1em$N_c$} & 2 & 4 & 8 & 16 & 2 & 4 & 8 & 16  \\
 %$N_c \rightarrow$ & 2 & 4 & 8 & 16 & 2 & 4 & 2 \\
 \hline
 \hline
 \multicolumn{5}{|c|}{Race (FP32 Accuracy = 41.34\%)} & \multicolumn{4}{|c|}{Boolq (FP32 Accuracy = 68.32\%)} \\ 
 \hline
 \hline
 64 & 40.48 & 40.10 & 39.43 & 39.90 & 69.20 & 68.41 & 69.45 & 68.56 \\
 \hline
 32 & 39.52 & 39.52 & 40.77 & 39.62 & 68.32 & 67.43 & 68.17 & 69.30  \\
 \hline
 16 & 39.81 & 39.71 & 39.90 & 40.38 & 68.10 & 66.33 & 69.51 & 69.42  \\
 \hline
 \hline
 \multicolumn{5}{|c|}{Winogrande (FP32 Accuracy = 67.88\%)} & \multicolumn{4}{|c|}{Piqa (FP32 Accuracy = 78.78\%)} \\ 
 \hline
 \hline
 64 & 66.85 & 66.61 & 67.72 & 67.88 & 77.31 & 77.42 & 77.75 & 77.64 \\
 \hline
 32 & 67.25 & 67.72 & 67.72 & 67.00 & 77.31 & 77.04 & 77.80 & 77.37  \\
 \hline
 16 & 68.11 & 68.90 & 67.88 & 67.48 & 77.37 & 78.13 & 78.13 & 77.69  \\
 \hline
\end{tabular}
\caption{\label{tab:mmlu_abalation} Accuracy on LM evaluation harness tasks on GPT3-8B model.}
\end{table}

\begin{table} \centering
\begin{tabular}{|c||c|c|c|c||c|c|c|c|} 
\hline
 $L_b \rightarrow$& \multicolumn{4}{c||}{8} & \multicolumn{4}{c||}{8}\\
 \hline
 \backslashbox{$L_A$\kern-1em}{\kern-1em$N_c$} & 2 & 4 & 8 & 16 & 2 & 4 & 8 & 16  \\
 %$N_c \rightarrow$ & 2 & 4 & 8 & 16 & 2 & 4 & 2 \\
 \hline
 \hline
 \multicolumn{5}{|c|}{Race (FP32 Accuracy = 40.67\%)} & \multicolumn{4}{|c|}{Boolq (FP32 Accuracy = 76.54\%)} \\ 
 \hline
 \hline
 64 & 40.48 & 40.10 & 39.43 & 39.90 & 75.41 & 75.11 & 77.09 & 75.66 \\
 \hline
 32 & 39.52 & 39.52 & 40.77 & 39.62 & 76.02 & 76.02 & 75.96 & 75.35  \\
 \hline
 16 & 39.81 & 39.71 & 39.90 & 40.38 & 75.05 & 73.82 & 75.72 & 76.09  \\
 \hline
 \hline
 \multicolumn{5}{|c|}{Winogrande (FP32 Accuracy = 70.64\%)} & \multicolumn{4}{|c|}{Piqa (FP32 Accuracy = 79.16\%)} \\ 
 \hline
 \hline
 64 & 69.14 & 70.17 & 70.17 & 70.56 & 78.24 & 79.00 & 78.62 & 78.73 \\
 \hline
 32 & 70.96 & 69.69 & 71.27 & 69.30 & 78.56 & 79.49 & 79.16 & 78.89  \\
 \hline
 16 & 71.03 & 69.53 & 69.69 & 70.40 & 78.13 & 79.16 & 79.00 & 79.00  \\
 \hline
\end{tabular}
\caption{\label{tab:mmlu_abalation} Accuracy on LM evaluation harness tasks on GPT3-22B model.}
\end{table}

\begin{table} \centering
\begin{tabular}{|c||c|c|c|c||c|c|c|c|} 
\hline
 $L_b \rightarrow$& \multicolumn{4}{c||}{8} & \multicolumn{4}{c||}{8}\\
 \hline
 \backslashbox{$L_A$\kern-1em}{\kern-1em$N_c$} & 2 & 4 & 8 & 16 & 2 & 4 & 8 & 16  \\
 %$N_c \rightarrow$ & 2 & 4 & 8 & 16 & 2 & 4 & 2 \\
 \hline
 \hline
 \multicolumn{5}{|c|}{Race (FP32 Accuracy = 44.4\%)} & \multicolumn{4}{|c|}{Boolq (FP32 Accuracy = 79.29\%)} \\ 
 \hline
 \hline
 64 & 42.49 & 42.51 & 42.58 & 43.45 & 77.58 & 77.37 & 77.43 & 78.1 \\
 \hline
 32 & 43.35 & 42.49 & 43.64 & 43.73 & 77.86 & 75.32 & 77.28 & 77.86  \\
 \hline
 16 & 44.21 & 44.21 & 43.64 & 42.97 & 78.65 & 77 & 76.94 & 77.98  \\
 \hline
 \hline
 \multicolumn{5}{|c|}{Winogrande (FP32 Accuracy = 69.38\%)} & \multicolumn{4}{|c|}{Piqa (FP32 Accuracy = 78.07\%)} \\ 
 \hline
 \hline
 64 & 68.9 & 68.43 & 69.77 & 68.19 & 77.09 & 76.82 & 77.09 & 77.86 \\
 \hline
 32 & 69.38 & 68.51 & 68.82 & 68.90 & 78.07 & 76.71 & 78.07 & 77.86  \\
 \hline
 16 & 69.53 & 67.09 & 69.38 & 68.90 & 77.37 & 77.8 & 77.91 & 77.69  \\
 \hline
\end{tabular}
\caption{\label{tab:mmlu_abalation} Accuracy on LM evaluation harness tasks on Llama2-7B model.}
\end{table}

\begin{table} \centering
\begin{tabular}{|c||c|c|c|c||c|c|c|c|} 
\hline
 $L_b \rightarrow$& \multicolumn{4}{c||}{8} & \multicolumn{4}{c||}{8}\\
 \hline
 \backslashbox{$L_A$\kern-1em}{\kern-1em$N_c$} & 2 & 4 & 8 & 16 & 2 & 4 & 8 & 16  \\
 %$N_c \rightarrow$ & 2 & 4 & 8 & 16 & 2 & 4 & 2 \\
 \hline
 \hline
 \multicolumn{5}{|c|}{Race (FP32 Accuracy = 48.8\%)} & \multicolumn{4}{|c|}{Boolq (FP32 Accuracy = 85.23\%)} \\ 
 \hline
 \hline
 64 & 49.00 & 49.00 & 49.28 & 48.71 & 82.82 & 84.28 & 84.03 & 84.25 \\
 \hline
 32 & 49.57 & 48.52 & 48.33 & 49.28 & 83.85 & 84.46 & 84.31 & 84.93  \\
 \hline
 16 & 49.85 & 49.09 & 49.28 & 48.99 & 85.11 & 84.46 & 84.61 & 83.94  \\
 \hline
 \hline
 \multicolumn{5}{|c|}{Winogrande (FP32 Accuracy = 79.95\%)} & \multicolumn{4}{|c|}{Piqa (FP32 Accuracy = 81.56\%)} \\ 
 \hline
 \hline
 64 & 78.77 & 78.45 & 78.37 & 79.16 & 81.45 & 80.69 & 81.45 & 81.5 \\
 \hline
 32 & 78.45 & 79.01 & 78.69 & 80.66 & 81.56 & 80.58 & 81.18 & 81.34  \\
 \hline
 16 & 79.95 & 79.56 & 79.79 & 79.72 & 81.28 & 81.66 & 81.28 & 80.96  \\
 \hline
\end{tabular}
\caption{\label{tab:mmlu_abalation} Accuracy on LM evaluation harness tasks on Llama2-70B model.}
\end{table}

%\section{MSE Studies}
%\textcolor{red}{TODO}


\subsection{Number Formats and Quantization Method}
\label{subsec:numFormats_quantMethod}
\subsubsection{Integer Format}
An $n$-bit signed integer (INT) is typically represented with a 2s-complement format \citep{yao2022zeroquant,xiao2023smoothquant,dai2021vsq}, where the most significant bit denotes the sign.

\subsubsection{Floating Point Format}
An $n$-bit signed floating point (FP) number $x$ comprises of a 1-bit sign ($x_{\mathrm{sign}}$), $B_m$-bit mantissa ($x_{\mathrm{mant}}$) and $B_e$-bit exponent ($x_{\mathrm{exp}}$) such that $B_m+B_e=n-1$. The associated constant exponent bias ($E_{\mathrm{bias}}$) is computed as $(2^{{B_e}-1}-1)$. We denote this format as $E_{B_e}M_{B_m}$.  

\subsubsection{Quantization Scheme}
\label{subsec:quant_method}
A quantization scheme dictates how a given unquantized tensor is converted to its quantized representation. We consider FP formats for the purpose of illustration. Given an unquantized tensor $\bm{X}$ and an FP format $E_{B_e}M_{B_m}$, we first, we compute the quantization scale factor $s_X$ that maps the maximum absolute value of $\bm{X}$ to the maximum quantization level of the $E_{B_e}M_{B_m}$ format as follows:
\begin{align}
\label{eq:sf}
    s_X = \frac{\mathrm{max}(|\bm{X}|)}{\mathrm{max}(E_{B_e}M_{B_m})}
\end{align}
In the above equation, $|\cdot|$ denotes the absolute value function.

Next, we scale $\bm{X}$ by $s_X$ and quantize it to $\hat{\bm{X}}$ by rounding it to the nearest quantization level of $E_{B_e}M_{B_m}$ as:

\begin{align}
\label{eq:tensor_quant}
    \hat{\bm{X}} = \text{round-to-nearest}\left(\frac{\bm{X}}{s_X}, E_{B_e}M_{B_m}\right)
\end{align}

We perform dynamic max-scaled quantization \citep{wu2020integer}, where the scale factor $s$ for activations is dynamically computed during runtime.

\subsection{Vector Scaled Quantization}
\begin{wrapfigure}{r}{0.35\linewidth}
  \centering
  \includegraphics[width=\linewidth]{sections/figures/vsquant.jpg}
  \caption{\small Vectorwise decomposition for per-vector scaled quantization (VSQ \citep{dai2021vsq}).}
  \label{fig:vsquant}
\end{wrapfigure}
During VSQ \citep{dai2021vsq}, the operand tensors are decomposed into 1D vectors in a hardware friendly manner as shown in Figure \ref{fig:vsquant}. Since the decomposed tensors are used as operands in matrix multiplications during inference, it is beneficial to perform this decomposition along the reduction dimension of the multiplication. The vectorwise quantization is performed similar to tensorwise quantization described in Equations \ref{eq:sf} and \ref{eq:tensor_quant}, where a scale factor $s_v$ is required for each vector $\bm{v}$ that maps the maximum absolute value of that vector to the maximum quantization level. While smaller vector lengths can lead to larger accuracy gains, the associated memory and computational overheads due to the per-vector scale factors increases. To alleviate these overheads, VSQ \citep{dai2021vsq} proposed a second level quantization of the per-vector scale factors to unsigned integers, while MX \citep{rouhani2023shared} quantizes them to integer powers of 2 (denoted as $2^{INT}$).

\subsubsection{MX Format}
The MX format proposed in \citep{rouhani2023microscaling} introduces the concept of sub-block shifting. For every two scalar elements of $b$-bits each, there is a shared exponent bit. The value of this exponent bit is determined through an empirical analysis that targets minimizing quantization MSE. We note that the FP format $E_{1}M_{b}$ is strictly better than MX from an accuracy perspective since it allocates a dedicated exponent bit to each scalar as opposed to sharing it across two scalars. Therefore, we conservatively bound the accuracy of a $b+2$-bit signed MX format with that of a $E_{1}M_{b}$ format in our comparisons. For instance, we use E1M2 format as a proxy for MX4.

\begin{figure}
    \centering
    \includegraphics[width=1\linewidth]{sections//figures/BlockFormats.pdf}
    \caption{\small Comparing LO-BCQ to MX format.}
    \label{fig:block_formats}
\end{figure}

Figure \ref{fig:block_formats} compares our $4$-bit LO-BCQ block format to MX \citep{rouhani2023microscaling}. As shown, both LO-BCQ and MX decompose a given operand tensor into block arrays and each block array into blocks. Similar to MX, we find that per-block quantization ($L_b < L_A$) leads to better accuracy due to increased flexibility. While MX achieves this through per-block $1$-bit micro-scales, we associate a dedicated codebook to each block through a per-block codebook selector. Further, MX quantizes the per-block array scale-factor to E8M0 format without per-tensor scaling. In contrast during LO-BCQ, we find that per-tensor scaling combined with quantization of per-block array scale-factor to E4M3 format results in superior inference accuracy across models. 


%%%%%%%%%%%%%%%%%%%%%%%%%%%%%%%%%%%%%%%%%%%%%%%%%%%%%%%%%%%%%%%%%%%%%%%%%%%%%%%
%%%%%%%%%%%%%%%%%%%%%%%%%%%%%%%%%%%%%%%%%%%%%%%%%%%%%%%%%%%%%%%%%%%%%%%%%%%%%%%


\end{document}


% This document was modified from the file originally made available by
% Pat Langley and Andrea Danyluk for ICML-2K. This version was created
% by Iain Murray in 2018, and modified by Alexandre Bouchard in
% 2019 and 2021 and by Csaba Szepesvari, Gang Niu and Sivan Sabato in 2022.
% Modified again in 2023 and 2024 by Sivan Sabato and Jonathan Scarlett.
% Previous contributors include Dan Roy, Lise Getoor and Tobias
% Scheffer, which was slightly modified from the 2010 version by
% Thorsten Joachims & Johannes Fuernkranz, slightly modified from the
% 2009 version by Kiri Wagstaff and Sam Roweis's 2008 version, which is
% slightly modified from Prasad Tadepalli's 2007 version which is a
% lightly changed version of the previous year's version by Andrew
% Moore, which was in turn edited from those of Kristian Kersting and
% Codrina Lauth. Alex Smola contributed to the algorithmic style files.
