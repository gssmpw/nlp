\section{Discussion and Limitations}
In this work, we theoretically analyze the impact of various graph shifts on the generalization gap in GTTA and highlight their empirical effects on test-time performance degradation. Based on these insights, we propose the \proj algorithm, which addresses neighborhood shift through extended and improved neighborhood alignment strategies tailored to the unique challenges of GTTA, and handles label shift via classifier adjustments. %TSA is model-agnostic, with neighborhood alignment implemented through edge weight assignments in practice.

Despite its effectiveness, we observe that \proj’s benefits can be marginal when feature shift dominates, as evidenced in the last two columns of Pileup, Arxiv, and DBLP/ACM. This is primarily because the neighborhood alignment strategy is not robust in such scenarios. Moreover, our current estimation of $\mgamma$ heavily depends on the accuracy of refined soft pseudo-labels. While confusion-matrix-based methods~\cite{lipton2018detecting, azizzadenesheli2019regularized, alexandari2020maximum} can provide more robust estimations, they often require additional adversarial training, which is not compatible with GTTA constraints. We encourage future work to explore more robust approaches for estimating $\mgamma$ to further enhance adaptation under severe feature shifts.



% We observe that the benefits of TSA are sometimes marginal when feature shift increases (as seen in the last two columns of Pileup, Arxiv, and DBLP/ACM).
% This is because the neighborhood alignment strategy is not robust enough.
% The current estimation of $\gamma$ heavily relies on the accuracy of the refined soft pseudo-labels.
% Although there is an approach using a confusion matrix \cite{lipton2018detecting, azizzadenesheli2019regularized, alexandari2020maximum} for robust estimation, such an approach often involves additional adversarial training, which is not applicable under GTTA.
% We encourage the development of more robust estimation of $\gamma$ in future work.







% \section{Conclusion}
% In summary, we theoretically analyze the impact of various graph shifts on the generalization gap in GTTA and highlight their empirical effects on test-time performance degradation. Subsequently, we propose the TSA algorithm, which addresses neighborhood shift through extended and improved neighborhood alignment strategies, specifically designed to accommodate the challenges in GTTA compared to previous GDA techniques, and addresses label shift through classifier adjustments. TSA is model-agnostic, with neighborhood alignment implemented via edge weight assignments in practice. 
