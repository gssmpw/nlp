\section{Related Work\label{sec:Related-Work}
}
%\subsection{Research of SFCs in terrestrial or physical network}

% In \cite{ZHANG2023109379}, the authors aimed to improve the resilience of space information networks against cascading failures. They introduced a failure model, identify satellite load imbalance as a key issue, and developed a gateway layout optimization model using a cuckoo search algorithm.

% In \cite{10230313}, the authors explored how cascading failure theory can enhance LEO satellite network resilience to cyber attacks, focusing on how initial load and real-time routing affect network stability.

There exist several studies exploring SFC or VNF deployment in SAGIN. For example, the authors in \cite{9951143} applied federated learning algorithms to address the SFC embedding challenge in SAGIN and adjusted SFC configurations to minimize service blocking rates. The authors in \cite{AKYILDIZ2019134} introduced a cyber-physical system that integrates ground, air, and space layers using software-defined networking (SDN) and NFV techniques. In \cite{fi16010027}, the authors proposed a hierarchical resource management structure for SAGIN, which integrated SDN, NFV, and multi-access edge computing to manage heterogeneous network resources. In \cite{9062531}, the authors proposed a heuristic greedy algorithm to tackle the SFC planning issue in a reconfigurable service provisioning framework for SAGIN. The work of \cite{10398221} proposed a SAGIN architecture with edge intelligence to improve the capabilities of communication, computing, sensing, and storage. It introduced a new deep reinforcement learning-based algorithm for resource allocation and computation offloading. In \cite{9749937}, the authors developed a service model by segmenting network slices and suggested an SFC mapping approach based on delay prediction. The authors in \cite{9351537} focused on dynamic VNF mapping and scheduling within SAGIN, and proposed two Tabu search-based algorithms for achieving near-optimal solutions. In \cite{10123085}, the authors proposed a dynamic network architecture for integrating terrestrial and non-terrestrial networks, addressing challenges of satellite mobility and communication delays. It optimized the allocation of VNFs among LEO CubeSats using three kinds of heuristic algorithms, achieving the near-optimal performance in simulations. These papers considered various models and algorithms in SAGIN to solve the deployment problems of SFCs or VNFs. However, the resource failures are not well considered, which is an important problem in dynamic SAGIN. In the dynamic network environment, nodes and links are easy to fail due to the environmental uncertainty, resulting in the tasks on nodes and links can no longer be effectively transmitted and processed, respectively.
% However, these studies often overlook the dynamic nature of SAGIN's topology over time, which is a critical aspect that requires further attention.

\begin{table*}[!t]
    \renewcommand\arraystretch{1.3}
	\begin{center}
		\caption{EVALUATION OF THE RELATED WORK} \label{evaluation of the related work}
        \begin{tabular}{|c|c|c|c|c|c|}
            \hline
            \multirow{2}{*}{References} & \multicolumn{5}{c|}{Requirements} \\ \cline{2-6} & Deployment & Scheduling & Resource failure & Terrestrial networks & Non-terrestrial networks\\
            \hline
            \hline
            \cite{9951143}, \cite{9351537} & \checkmark & \checkmark & - & - & \checkmark \\
            \hline
            \cite{AKYILDIZ2019134}, \cite{fi16010027}, \cite{10398221}, \cite{9749937}, \cite{10123085} & \checkmark & - & - & - & \checkmark \\
            \hline
            \cite{9062531} & - & \checkmark & - & - & \checkmark \\
            \hline
            \cite{8463632}, \cite{9585385}, \cite{10179962}, \cite{8923413}, \cite{9296232}, \cite{10533654} & - & - & \checkmark & \checkmark & - \\
            \hline
            \cite{7987282}, \cite{7898396} & \checkmark & - & \checkmark & \checkmark & -\\
            \hline
            Our work & \checkmark & \checkmark & \checkmark & - & \checkmark \\
            \hline
		\end{tabular}
	\end{center}
\end{table*}

\begin{table}[!t]
    \renewcommand\arraystretch{1.3}
	\begin{center}
		\caption{KEY NOTATIONS} \label{key notations}
		\begin{tabular}{|p{2cm}|p{6cm}|}
			\hline
			Symbol& Description \\
			\hline
            \hline
			$\mathcal{G} =(\mathcal{E},\mathcal{Y})$ & SAGIN graph composed of nodes $\mathcal{E}$ and links $\mathcal{Y}$.\\
            \hline
			$T$, $t$, $\mathscr{T}$, $\tau$ & Set of time slots, order number of time slots, total number of time slots, and time slot length.  \\
            \hline
            $\mathcal{V}_k$, $\mathcal{K}$, $l_k$ & SFC of the $k$-th task, the total number of tasks, and the total number of VNFs in the $k$-th SFC. \\
            \hline
            $i^t$, $v_k^m$ & The node $i$ in time slot $t$, and the $m$-th VNF of SFC $\mathcal{V}_k$. \\
			\hline
            $x_{v^m_k, i^t} $ & Binary variable indicating whether VNF $v^m_k$ of SFC $\mathcal{V}_k$ is deployed on node $i^t$.\\
            \hline
            $y^k_{(i^t,j^t)}$ & Binary variable indicating whether SFC $\mathcal{V}_k$ is deployed on link $(i^t_i,j^t)$.  \\
            \hline
            $z_{(i^t,i^{t+1})}^k $ & Binary variable indicating whether SFC $\mathcal{V}_k$ is stored on $i^t$ from $t$ to $t+1$.\\
            \hline
            $w^m_k$ & Binary variable indicating whether VNF $v^m_k$ of SFC $\mathcal{V}_k$ needs to be redeployed.\\
            \hline
            $\varphi _{i^t}$ & Computation ability of node $i^t$.\\
            \hline
            $\Delta_k$ & Data amount of SFC $\mathcal{V}_k$.\\
            \hline
            $\sigma _{v_k^m}$ & Computing resource consumed by VNF $v_k^m$.\\
            \hline
            $e^c_{i^t}$ & Energy consumption per unit of computing resource on node $i^t$.\\
            \hline
		\end{tabular}
	\end{center}
\end{table}

The resource failure and solutions based on NFV in terrestrial networks have been studied. For example, the authors in \cite{8463632} presented a framework for provisioning SFC request availability in multi-layered, heterogeneous-failure environment of a data center, and optimize resource usage. The authors in \cite{9585385} presented a dynamic virtual resource allocation mechanism for NFV-enabled vehicular and the fifth generation mobile communication technology (5G) networks. The mechanism was designed to maintain service performance despite network element failures. In \cite{10179962}, the authors introduced a digital twin-based scheme for SFC failure localization, involving failure classification and root cause analysis. In \cite{8923413}, the authors investigated quick VNF recovery using diversity coding in 5G networks, which avoided retransmissions and reduce capacity costs. It improved the reliability and reduced delays. The authors in \cite{9296232} tackled the service chain composition in NFV systems, considering user competition and resource failures. In \cite{10533654}, the authors introduced the VNF restoration problem in NFV and proposed the online recovery algorithm, which aimed to maximize the total weight of recovered services, prioritizing the restoration of the most important services during failures. In \cite{7987282}, the authors introduced a decision tree-based algorithm for VNF placement and chaining to mitigate penalties from link failures by Monte-Carlo Tree Search. The authors in \cite{7898396} presented a recovery approach for virtual networks affected by substrate node failures. They offered fair and priority-based recovery methods, and proposed a heuristic algorithm to solve the problem. However, the above papers only studied the fixed and stable physical terrestrial network, without considering the dynamic and heterogeneous nature of SAGIN.

As analyzed above, the problems of SFC deployment in SAGIN, and resource failure and recovery faced by NFV and SFCs in terrestrial networks are well studied. However, as far as the authors' knowledge, the research on recovery of SFCs in SAGIN is not comprehensive. Therefore, in this paper, we consider the dynamic resource instability of multi-layer SAGIN, propose the corresponding SFC deployment and recovery model in the case of resource failure, and design the algorithm based on matching game. The comparisons between our work and the related works are shown in Table \ref{evaluation of the related work}.

\begin{figure}[!t]
    \centerline{\includegraphics[width=9.9cm]{work3_fig1.eps}}
   \caption{Scenario of SFC deployments in SAGIN.}\label{fig1}
\end{figure}

%