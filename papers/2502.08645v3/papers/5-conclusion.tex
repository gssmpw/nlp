\section{Conclusion}
\label{sec:conclusion}

We introduced \our, a novel Real-to-Sim-to-Real pipeline that integrates Gaussian splatting with NVIDIA Isaac Sim's PhysX engine, improving scene reconstruction and sim-to-real transfer for robotic manipulation tasks. Extensive experiments demonstrate its efficacy and scalability, showing that \our significantly reduces the sim-to-real gap and enables reliable real-world task performance. Models trained on large simulated datasets achieved competitive success rates compared with those trained on real-world data, highlighting the potential \our to generate diverse, high-quality data for pre-training large-scale robot models.
% Our sim-to-real transfer experiments showed that \our significantly reduces the sim-to-real gap, enabling trained policy models to perform reliably in real-world scenarios. Models trained on extensive simulated datasets achieved competitive success rates, often surpassing mesh-based rendering methods such as IL-RialTo and matching the performance of models trained with real-world data and the AnyGrasp-Rule method. These results highlight \our's ability to generate high-quality and diverse data essential for training robust imitation learning models.

 % and advancing robotic applications.

\noindent\textbf{Limitations.} 
 \our is currently limited to rigid objects and has not yet been extended to reconstruct articulated, deformable objects, or liquids. Additionally, it relies on manually defined physics parameters rather than system identification methods. Rule-based policies for data collection become challenging as task complexity increases. We leave addressing these limitations to future work. 

% Our method is currently limited to rigid objects and cannot reconstruct articulated, deformable objects, or liquids. Furthermore, we do not utilize system identification methods, instead relying on default values for physics parameters
% , which are inaccurate for objects with unique physical characteristics such as irregular centers of mass. 
% Additionally, rule-based policies for data collection become challenging as task complexity increases. 
% Future work could focus on developing a unified pipeline for high-fidelity articulation reconstruction and exploring more user-friendly data generation methods to address these limitations.


\section*{Impact Statement}

The proposed \our system provides an intriguing way of generating simulated robot demonstrations for real-world tasks. This work is meant to release the human burden of collecting large-scale real-world data, improve efficiency, and push the limit of generalist robot policies with data scaling law. \our can be applied across various robot applications and thus may cause potential unemployment issues.