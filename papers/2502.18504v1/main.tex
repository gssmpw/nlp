\pdfoutput=1

\documentclass[11pt]{article}




\usepackage[preprint]{acl}

\usepackage{times}
\usepackage{latexsym}

\usepackage[T1]{fontenc}

\usepackage[utf8]{inputenc}

\usepackage{microtype}

\usepackage{inconsolata}

\usepackage{graphicx}
\graphicspath{{./figs/}{./}}

\usepackage{amsfonts}
\usepackage{multirow} 
\usepackage{booktabs}
\usepackage{enumitem}
\usepackage{amsmath}

\usepackage{algorithm}
\usepackage[noend]{algpseudocode}
\usepackage{xspace}
\usepackage{stfloats}
\usepackage{xcolor} %
\usepackage{mdframed}
\newmdenv[
backgroundcolor=hlcolor,
topline=false,
bottomline=false,
leftline=false,
rightline=false,
]{shaded}


\newcommand{\qnote}[1]{[\textcolor{blue}{Q-note: #1}]}




\definecolor{color1}{HTML}{f3f3f3}
\definecolor{color2}{HTML}{000000}
\newmdenv[
backgroundcolor=color1,
fontcolor=color2,
topline=true,
bottomline=true,
leftline=true,
rightline=true,
]{shaded1}

\newcommand{\oldalglinenumber}{}%
\NewDocumentCommand{\NoLNElse}{ m }{%
  \RenewCommandCopy{\oldalglinenumber}{\alglinenumber}%
  \RenewDocumentCommand{\alglinenumber}{ m }{}%
  \Else #1%
  \addtocounter{ALG@line}{-1}%
  \RenewCommandCopy{\alglinenumber}{\oldalglinenumber}%
}
\NewDocumentCommand{\NoLNState}{}{%
  \RenewCommandCopy{\oldalglinenumber}{\alglinenumber}%
  \RenewDocumentCommand{\alglinenumber}{ m }{}%
  \State%
  \addtocounter{ALG@line}{-1}%
  \RenewCommandCopy{\alglinenumber}{\oldalglinenumber}%
}

\newcommand{\todo}{\textcolor{red}{TODO}}
\newcommand{\aman}[1]{\textcolor{orange}{Aman: #1}}



\newcommand{\bedrockfuzz}{{\sc TurboFuzzLLM}\xspace}





















\title{\bedrockfuzz: Turbocharging Mutation-based Fuzzing for Effectively Jailbreaking Large Language Models in Practice}


\author{Aman Goel$^*$, Xian Carrie Wu, Zhe Wang, Dmitriy Bespalov, Yanjun Qi$^*$  \\
  Amazon Web Services, USA \\
  \texttt{\{goelaman, xianwwu, zhebeta, dbespal, yanjunqi\}@amazon.com} \\}


\begin{document}
\maketitle
\begingroup\def\thefootnote{*}\footnotetext{Corresponding authors}\endgroup
\begin{abstract}
Jailbreaking large-language models (LLMs) involves testing their robustness against adversarial prompts and evaluating their ability to withstand prompt attacks that could elicit unauthorized or malicious responses. In this paper, we present \bedrockfuzz, a mutation-based fuzzing technique for efficiently finding a collection of effective jailbreaking templates that, when combined with harmful questions, can lead a target LLM to produce harmful responses through black-box access via user prompts. We describe the limitations of directly applying existing template-based attacking techniques in practice, and present functional and efficiency-focused upgrades we added to mutation-based fuzzing to generate effective jailbreaking templates automatically. \bedrockfuzz achieves $\geq$ 95\% attack success rates (ASR) on public datasets for leading LLMs (including GPT-4o \& GPT-4 Turbo), shows impressive generalizability to unseen harmful questions, and helps in improving model defenses to prompt attacks.\footnote{\textcolor{red}{Warning: This paper contains techniques to generate unfiltered content by LLMs that may be offensive to readers.}}
\end{abstract}

\section{Introduction}
\label{sec:introduction}
The business processes of organizations are experiencing ever-increasing complexity due to the large amount of data, high number of users, and high-tech devices involved \cite{martin2021pmopportunitieschallenges, beerepoot2023biggestbpmproblems}. This complexity may cause business processes to deviate from normal control flow due to unforeseen and disruptive anomalies \cite{adams2023proceddsriftdetection}. These control-flow anomalies manifest as unknown, skipped, and wrongly-ordered activities in the traces of event logs monitored from the execution of business processes \cite{ko2023adsystematicreview}. For the sake of clarity, let us consider an illustrative example of such anomalies. Figure \ref{FP_ANOMALIES} shows a so-called event log footprint, which captures the control flow relations of four activities of a hypothetical event log. In particular, this footprint captures the control-flow relations between activities \texttt{a}, \texttt{b}, \texttt{c} and \texttt{d}. These are the causal ($\rightarrow$) relation, concurrent ($\parallel$) relation, and other ($\#$) relations such as exclusivity or non-local dependency \cite{aalst2022pmhandbook}. In addition, on the right are six traces, of which five exhibit skipped, wrongly-ordered and unknown control-flow anomalies. For example, $\langle$\texttt{a b d}$\rangle$ has a skipped activity, which is \texttt{c}. Because of this skipped activity, the control-flow relation \texttt{b}$\,\#\,$\texttt{d} is violated, since \texttt{d} directly follows \texttt{b} in the anomalous trace.
\begin{figure}[!t]
\centering
\includegraphics[width=0.9\columnwidth]{images/FP_ANOMALIES.png}
\caption{An example event log footprint with six traces, of which five exhibit control-flow anomalies.}
\label{FP_ANOMALIES}
\end{figure}

\subsection{Control-flow anomaly detection}
Control-flow anomaly detection techniques aim to characterize the normal control flow from event logs and verify whether these deviations occur in new event logs \cite{ko2023adsystematicreview}. To develop control-flow anomaly detection techniques, \revision{process mining} has seen widespread adoption owing to process discovery and \revision{conformance checking}. On the one hand, process discovery is a set of algorithms that encode control-flow relations as a set of model elements and constraints according to a given modeling formalism \cite{aalst2022pmhandbook}; hereafter, we refer to the Petri net, a widespread modeling formalism. On the other hand, \revision{conformance checking} is an explainable set of algorithms that allows linking any deviations with the reference Petri net and providing the fitness measure, namely a measure of how much the Petri net fits the new event log \cite{aalst2022pmhandbook}. Many control-flow anomaly detection techniques based on \revision{conformance checking} (hereafter, \revision{conformance checking}-based techniques) use the fitness measure to determine whether an event log is anomalous \cite{bezerra2009pmad, bezerra2013adlogspais, myers2018icsadpm, pecchia2020applicationfailuresanalysispm}. 

The scientific literature also includes many \revision{conformance checking}-independent techniques for control-flow anomaly detection that combine specific types of trace encodings with machine/deep learning \cite{ko2023adsystematicreview, tavares2023pmtraceencoding}. Whereas these techniques are very effective, their explainability is challenging due to both the type of trace encoding employed and the machine/deep learning model used \cite{rawal2022trustworthyaiadvances,li2023explainablead}. Hence, in the following, we focus on the shortcomings of \revision{conformance checking}-based techniques to investigate whether it is possible to support the development of competitive control-flow anomaly detection techniques while maintaining the explainable nature of \revision{conformance checking}.
\begin{figure}[!t]
\centering
\includegraphics[width=\columnwidth]{images/HIGH_LEVEL_VIEW.png}
\caption{A high-level view of the proposed framework for combining \revision{process mining}-based feature extraction with dimensionality reduction for control-flow anomaly detection.}
\label{HIGH_LEVEL_VIEW}
\end{figure}

\subsection{Shortcomings of \revision{conformance checking}-based techniques}
Unfortunately, the detection effectiveness of \revision{conformance checking}-based techniques is affected by noisy data and low-quality Petri nets, which may be due to human errors in the modeling process or representational bias of process discovery algorithms \cite{bezerra2013adlogspais, pecchia2020applicationfailuresanalysispm, aalst2016pm}. Specifically, on the one hand, noisy data may introduce infrequent and deceptive control-flow relations that may result in inconsistent fitness measures, whereas, on the other hand, checking event logs against a low-quality Petri net could lead to an unreliable distribution of fitness measures. Nonetheless, such Petri nets can still be used as references to obtain insightful information for \revision{process mining}-based feature extraction, supporting the development of competitive and explainable \revision{conformance checking}-based techniques for control-flow anomaly detection despite the problems above. For example, a few works outline that token-based \revision{conformance checking} can be used for \revision{process mining}-based feature extraction to build tabular data and develop effective \revision{conformance checking}-based techniques for control-flow anomaly detection \cite{singh2022lapmsh, debenedictis2023dtadiiot}. However, to the best of our knowledge, the scientific literature lacks a structured proposal for \revision{process mining}-based feature extraction using the state-of-the-art \revision{conformance checking} variant, namely alignment-based \revision{conformance checking}.

\subsection{Contributions}
We propose a novel \revision{process mining}-based feature extraction approach with alignment-based \revision{conformance checking}. This variant aligns the deviating control flow with a reference Petri net; the resulting alignment can be inspected to extract additional statistics such as the number of times a given activity caused mismatches \cite{aalst2022pmhandbook}. We integrate this approach into a flexible and explainable framework for developing techniques for control-flow anomaly detection. The framework combines \revision{process mining}-based feature extraction and dimensionality reduction to handle high-dimensional feature sets, achieve detection effectiveness, and support explainability. Notably, in addition to our proposed \revision{process mining}-based feature extraction approach, the framework allows employing other approaches, enabling a fair comparison of multiple \revision{conformance checking}-based and \revision{conformance checking}-independent techniques for control-flow anomaly detection. Figure \ref{HIGH_LEVEL_VIEW} shows a high-level view of the framework. Business processes are monitored, and event logs obtained from the database of information systems. Subsequently, \revision{process mining}-based feature extraction is applied to these event logs and tabular data input to dimensionality reduction to identify control-flow anomalies. We apply several \revision{conformance checking}-based and \revision{conformance checking}-independent framework techniques to publicly available datasets, simulated data of a case study from railways, and real-world data of a case study from healthcare. We show that the framework techniques implementing our approach outperform the baseline \revision{conformance checking}-based techniques while maintaining the explainable nature of \revision{conformance checking}.

In summary, the contributions of this paper are as follows.
\begin{itemize}
    \item{
        A novel \revision{process mining}-based feature extraction approach to support the development of competitive and explainable \revision{conformance checking}-based techniques for control-flow anomaly detection.
    }
    \item{
        A flexible and explainable framework for developing techniques for control-flow anomaly detection using \revision{process mining}-based feature extraction and dimensionality reduction.
    }
    \item{
        Application to synthetic and real-world datasets of several \revision{conformance checking}-based and \revision{conformance checking}-independent framework techniques, evaluating their detection effectiveness and explainability.
    }
\end{itemize}

The rest of the paper is organized as follows.
\begin{itemize}
    \item Section \ref{sec:related_work} reviews the existing techniques for control-flow anomaly detection, categorizing them into \revision{conformance checking}-based and \revision{conformance checking}-independent techniques.
    \item Section \ref{sec:abccfe} provides the preliminaries of \revision{process mining} to establish the notation used throughout the paper, and delves into the details of the proposed \revision{process mining}-based feature extraction approach with alignment-based \revision{conformance checking}.
    \item Section \ref{sec:framework} describes the framework for developing \revision{conformance checking}-based and \revision{conformance checking}-independent techniques for control-flow anomaly detection that combine \revision{process mining}-based feature extraction and dimensionality reduction.
    \item Section \ref{sec:evaluation} presents the experiments conducted with multiple framework and baseline techniques using data from publicly available datasets and case studies.
    \item Section \ref{sec:conclusions} draws the conclusions and presents future work.
\end{itemize}
\section{Method: \bedrockfuzz}
\label{sec:bedrockfuzz}
Figure~\ref{fig:workflow} presents an overview of \bedrockfuzz. Except of a collection of functional (\S\ref{sec:functional}), efficiency-focused (\S\ref{sec:efficiency}), and engineering upgrades (Appendix~\ref{sec:engineering}), the overall workflow of \bedrockfuzz is the same as GPTFuzzer.

Given a set of original templates $O = \{ o_1, o_2, \dots, o_{|O|} \}$, a set of harmful questions $Q = \{ q_1, q_2, \dots, q_{|Q|} \}$, and a target model $T$, \bedrockfuzz performs black-box mutation-based fuzzing to iteratively generate new jailbreaking templates $G = \{ g_1, g_2, \dots, g_{|G|} \}$.
In each fuzzing iteration, \bedrockfuzz selects a template $t$ from the current population $P = O \cup G$ (initially $G = \emptyset$) and a mutation $m$ from the set of all mutations $M$ to generate a new mutant $m(t)$.
Next, the effectiveness of this new template $m(t)$ is evaluated by attacking the target model $T$ using $Q$, i.e., $m(t)$ is combined with questions $q_i \in Q$ to formulate attack prompts $A_{m(t)} = \{ a_{q_1}, a_{q_2}, \dots, a_{q_{|Q|}} \}$, which are queried to $T$ to get a set of responses $R_{m(t)} = \{ r_{q_1}, r_{q_2}, \dots, r_{q_{|Q|}} \}$.
Each response $r_{q_i}$ from $T$ is sent to a judge model to evaluate whether or not $r_{q_i}$ represents a successful jailbreak for question $q_i$, to get the subset of successful jailbreak responses $R_{m(t)}^{success} \subseteq R_{m(t)}$.
If $m(t)$ jailbreaks at least one question (i.e., $R_{m(t)}^{success} \neq \emptyset$), then $m(t)$ is added to $G$, or else, $m(t)$ is discarded.
Fuzzing iterations end when a stopping criteria, such as all questions got jailbroken or the target model query budget, is reached.

\subsection{Functional Upgrades}
\label{sec:functional}
\bedrockfuzz implements two groups of functional upgrades over GPTFuzzer: i) new mutations, and ii) new selection policies to improve the mutant space explored during the search.

\subsubsection{New Mutations}
\label{sec:mutations}
In addition to the 5 mutations from GPTFuzzer~\cite{yu2023gptfuzzer}, we added 2 syntactic and 3 LLM-based new mutations to \bedrockfuzz. 
 
\begin{itemize}
    \item \textit{Refusal Suppression}. This is a static mutation, inspired from~\cite{wei2024jailbroken}, that instructs the model to respond under constraints that rule out common refusal responses, thus making unsafe responses more likely. Figure~\ref{fig:refusal} in Appendix~\ref{app:mutations} details the mutant template generated on applying refusal suppression mutation to a given template.
    
    \item \textit{Inject Prefix}. This is a static mutation, inspired from~\cite{wei2024jailbroken,jiang2024chatbug}, that appends the fixed string ``Sure, here is'' to a given template. This can make the model to heavily penalize refusing and continue answering the unsafe prompt with a jailbreaking response.
    
    \item \textit{Expand After}. This is a LLM-based mutation, inspired from the \textit{Expand} mutation from GPTFuzzer~\cite{yu2023gptfuzzer}, designed to append the new content at the end of the given template (instead of adding new content to the beginning as in \textit{Expand}).

    \item \textit{Transfer Mutation}. This is a LLM-based mutation that transforms a given template $y$ using another template-mutant pair $\left( x, m^*(x) \right)$ as an example, instructing the LLM to infer the (compounded) mutation $m^*$ and return $m^*(y)$. The example mutant $m^*(x)$ is selected randomly from among the top 10 jailbreaking mutants generated so far during fuzzing and $x$ is its corresponding root parent template, i.e., $x \in O$ and $m^*(x) = m_k(\dots m_2(m_1(x))\dots)$. The key idea here is to apply in-context learning to transfer the series of mutations $m_1, m_2, \dots, m_k$ applied to an original template $x$ to derive one of the top ranking mutants $m^*(x)$ identified so far to the given template $y$ in a single fuzzing iteration. Figure~\ref{fig:transfer_mutation} in Appendix~\ref{app:mutations} details the prompt used to apply this mutation to a given template.
    
    \item \textit{Few Shots}. This is a LLM-based mutation that transforms a given template $y$ using a fixed set of mutants $[g_1, g_2, \dots, g_k]$ as in-context examples. These few-shot examples are selected as the top 3 jailbreaking mutants generated so far from the same sub tree as $y$ (i.e., $root(y) = root(g_i)$ for 1 $\leq$ $i$ $\leq$ $k$). The key idea here is to apply few-shot in-context learning to transfer to the given template $y$ a hybrid combination of top ranking mutants identified so far and originating from the same original template as $y$. Figure~\ref{fig:few_shots} in Appendix~\ref{app:mutations} details the prompt used to apply this mutation to a given template.
\end{itemize}

\subsubsection{New Selection Policies}
\label{sec:selection}
\bedrockfuzz introduces new template and mutation selection policies based on reinforcement learning to learn from previous fuzzing iterations which template or mutation could work better than the others in a given fuzzing iteration.

\begin{itemize}
    \item \textit{Mutation selection using Q-learning}.
    \bedrockfuzz utilizes a Q-learning based technique to learn over time which mutation works the best for a given template $t$.
    \bedrockfuzz maintains a Q-table $\mathcal{Q} : \mathit{S} \times \mathit{A} \rightarrow \mathbb{R}$ where $\mathit{S}$ represents the current state of the environment and $\mathit{A}$ represents the possible actions to take at a given state.
    Given a template $t$ selected in a fuzzing iteration, \bedrockfuzz tracks the original root parent $root(t) \in O$ corresponding to $t$ and uses it as the state for Q-learning. The set of possible mutations $M$ are used as the actions set $\mathit{A}$ for any given state. The selected mutation $m$ is rewarded based on the attack success rate of the mutant $m(t)$. Algorithm~\ref{alg:mutation_selection_ql} in Appendix~\ref{app:mutation_selection} provides the pseudo code of Q-learning based mutation selection.
    
    \item \textit{Template selection using multi-arm bandits}. This template selection method is basically the same as Q-learning based mutation selection, except that there is no environment state that is tracked, making it similar to a multi-arm bandits selection~\cite{slivkins2019introduction}.
    Algorithm~\ref{alg:template_selection_ql} in Appendix~\ref{app:template_selection} provides the pseudo code in detail.
    \end{itemize}

\subsection{Efficiency Upgrades}
\label{sec:efficiency}
\bedrockfuzz implements two efficiency-focused upgrades with the objective of jailbreaking more harmful questions with fewer queries to the target model.


\subsubsection{Early-exit Fruitless Templates}
\label{sec:early_exit}
Given a mutant $m(t)$ generated in a fuzzing iteration, \bedrockfuzz exits the fuzzing iteration early before all questions $Q$ are combined with $m(t)$ if $m(t)$ is determined as fruitless. To determine whether or not $m(t)$ is fruitless without making $|Q|$ queries to the target model, \bedrockfuzz utilizes a simple heuristic that iterates over $Q$ in a random order and if any 10\% of the corresponding attack prompts serially evaluated do not result in a jailbreak, $m(t)$ is classified as fruitless. In such a scenario, the remaining questions are skipped, i.e., not combined with $m(t)$ into attack prompts, and the fuzzing iteration is terminated prematurely.

Using such a heuristic significantly reducing the number of queries sent to the target model that are likely futile. However, this leaves the possibility that a mutant $m(t)$ is never combined with a question $q_k \in Q$, even though it might result in a jailbreak. To avoid such a case, we added a new identity/noop mutation such that $m_{identity}(t) = t$. Thus, even if a mutant $m(t)$ is determined as fruitless in a fuzzing iteration $k$, questions skipped in iteration $k$ can still be combined with $m(t)$ in a possible future iteration $l$ ($l > k$) that applies identity mutation on $m(t)$.

\subsubsection{Warmup Stage}
\label{sec:warmup}
\bedrockfuzz adds an initial warmup stage that uses original templates $O$ directly to attack the target model, before beginning the fuzzing stage. The benefits of warmup stage are two-fold: i) it identifies questions that can be jailbroken with original templates directly, and ii) it warms up the Q-table for mutation/template selectors (\S\ref{sec:selection}). Note that the early-exit fruitless templates heuristic (\S\ref{sec:early_exit}) ensures that only a limited number of queries are spent in the warmup stage if the original templates as is are ineffective/fruitless. 








\section{Experiments}
\label{sec:experiments}
The experiments are designed to address two key research questions.
First, \textbf{RQ1} evaluates whether the average $L_2$-norm of the counterfactual perturbation vectors ($\overline{||\perturb||}$) decreases as the model overfits the data, thereby providing further empirical validation for our hypothesis.
Second, \textbf{RQ2} evaluates the ability of the proposed counterfactual regularized loss, as defined in (\ref{eq:regularized_loss2}), to mitigate overfitting when compared to existing regularization techniques.

% The experiments are designed to address three key research questions. First, \textbf{RQ1} investigates whether the mean perturbation vector norm decreases as the model overfits the data, aiming to further validate our intuition. Second, \textbf{RQ2} explores whether the mean perturbation vector norm can be effectively leveraged as a regularization term during training, offering insights into its potential role in mitigating overfitting. Finally, \textbf{RQ3} examines whether our counterfactual regularizer enables the model to achieve superior performance compared to existing regularization methods, thus highlighting its practical advantage.

\subsection{Experimental Setup}
\textbf{\textit{Datasets, Models, and Tasks.}}
The experiments are conducted on three datasets: \textit{Water Potability}~\cite{kadiwal2020waterpotability}, \textit{Phomene}~\cite{phomene}, and \textit{CIFAR-10}~\cite{krizhevsky2009learning}. For \textit{Water Potability} and \textit{Phomene}, we randomly select $80\%$ of the samples for the training set, and the remaining $20\%$ for the test set, \textit{CIFAR-10} comes already split. Furthermore, we consider the following models: Logistic Regression, Multi-Layer Perceptron (MLP) with 100 and 30 neurons on each hidden layer, and PreactResNet-18~\cite{he2016cvecvv} as a Convolutional Neural Network (CNN) architecture.
We focus on binary classification tasks and leave the extension to multiclass scenarios for future work. However, for datasets that are inherently multiclass, we transform the problem into a binary classification task by selecting two classes, aligning with our assumption.

\smallskip
\noindent\textbf{\textit{Evaluation Measures.}} To characterize the degree of overfitting, we use the test loss, as it serves as a reliable indicator of the model's generalization capability to unseen data. Additionally, we evaluate the predictive performance of each model using the test accuracy.

\smallskip
\noindent\textbf{\textit{Baselines.}} We compare CF-Reg with the following regularization techniques: L1 (``Lasso''), L2 (``Ridge''), and Dropout.

\smallskip
\noindent\textbf{\textit{Configurations.}}
For each model, we adopt specific configurations as follows.
\begin{itemize}
\item \textit{Logistic Regression:} To induce overfitting in the model, we artificially increase the dimensionality of the data beyond the number of training samples by applying a polynomial feature expansion. This approach ensures that the model has enough capacity to overfit the training data, allowing us to analyze the impact of our counterfactual regularizer. The degree of the polynomial is chosen as the smallest degree that makes the number of features greater than the number of data.
\item \textit{Neural Networks (MLP and CNN):} To take advantage of the closed-form solution for computing the optimal perturbation vector as defined in (\ref{eq:opt-delta}), we use a local linear approximation of the neural network models. Hence, given an instance $\inst_i$, we consider the (optimal) counterfactual not with respect to $\model$ but with respect to:
\begin{equation}
\label{eq:taylor}
    \model^{lin}(\inst) = \model(\inst_i) + \nabla_{\inst}\model(\inst_i)(\inst - \inst_i),
\end{equation}
where $\model^{lin}$ represents the first-order Taylor approximation of $\model$ at $\inst_i$.
Note that this step is unnecessary for Logistic Regression, as it is inherently a linear model.
\end{itemize}

\smallskip
\noindent \textbf{\textit{Implementation Details.}} We run all experiments on a machine equipped with an AMD Ryzen 9 7900 12-Core Processor and an NVIDIA GeForce RTX 4090 GPU. Our implementation is based on the PyTorch Lightning framework. We use stochastic gradient descent as the optimizer with a learning rate of $\eta = 0.001$ and no weight decay. We use a batch size of $128$. The training and test steps are conducted for $6000$ epochs on the \textit{Water Potability} and \textit{Phoneme} datasets, while for the \textit{CIFAR-10} dataset, they are performed for $200$ epochs.
Finally, the contribution $w_i^{\varepsilon}$ of each training point $\inst_i$ is uniformly set as $w_i^{\varepsilon} = 1~\forall i\in \{1,\ldots,m\}$.

The source code implementation for our experiments is available at the following GitHub repository: \url{https://anonymous.4open.science/r/COCE-80B4/README.md} 

\subsection{RQ1: Counterfactual Perturbation vs. Overfitting}
To address \textbf{RQ1}, we analyze the relationship between the test loss and the average $L_2$-norm of the counterfactual perturbation vectors ($\overline{||\perturb||}$) over training epochs.

In particular, Figure~\ref{fig:delta_loss_epochs} depicts the evolution of $\overline{||\perturb||}$ alongside the test loss for an MLP trained \textit{without} regularization on the \textit{Water Potability} dataset. 
\begin{figure}[ht]
    \centering
    \includegraphics[width=0.85\linewidth]{img/delta_loss_epochs.png}
    \caption{The average counterfactual perturbation vector $\overline{||\perturb||}$ (left $y$-axis) and the cross-entropy test loss (right $y$-axis) over training epochs ($x$-axis) for an MLP trained on the \textit{Water Potability} dataset \textit{without} regularization.}
    \label{fig:delta_loss_epochs}
\end{figure}

The plot shows a clear trend as the model starts to overfit the data (evidenced by an increase in test loss). 
Notably, $\overline{||\perturb||}$ begins to decrease, which aligns with the hypothesis that the average distance to the optimal counterfactual example gets smaller as the model's decision boundary becomes increasingly adherent to the training data.

It is worth noting that this trend is heavily influenced by the choice of the counterfactual generator model. In particular, the relationship between $\overline{||\perturb||}$ and the degree of overfitting may become even more pronounced when leveraging more accurate counterfactual generators. However, these models often come at the cost of higher computational complexity, and their exploration is left to future work.

Nonetheless, we expect that $\overline{||\perturb||}$ will eventually stabilize at a plateau, as the average $L_2$-norm of the optimal counterfactual perturbations cannot vanish to zero.

% Additionally, the choice of employing the score-based counterfactual explanation framework to generate counterfactuals was driven to promote computational efficiency.

% Future enhancements to the framework may involve adopting models capable of generating more precise counterfactuals. While such approaches may yield to performance improvements, they are likely to come at the cost of increased computational complexity.


\subsection{RQ2: Counterfactual Regularization Performance}
To answer \textbf{RQ2}, we evaluate the effectiveness of the proposed counterfactual regularization (CF-Reg) by comparing its performance against existing baselines: unregularized training loss (No-Reg), L1 regularization (L1-Reg), L2 regularization (L2-Reg), and Dropout.
Specifically, for each model and dataset combination, Table~\ref{tab:regularization_comparison} presents the mean value and standard deviation of test accuracy achieved by each method across 5 random initialization. 

The table illustrates that our regularization technique consistently delivers better results than existing methods across all evaluated scenarios, except for one case -- i.e., Logistic Regression on the \textit{Phomene} dataset. 
However, this setting exhibits an unusual pattern, as the highest model accuracy is achieved without any regularization. Even in this case, CF-Reg still surpasses other regularization baselines.

From the results above, we derive the following key insights. First, CF-Reg proves to be effective across various model types, ranging from simple linear models (Logistic Regression) to deep architectures like MLPs and CNNs, and across diverse datasets, including both tabular and image data. 
Second, CF-Reg's strong performance on the \textit{Water} dataset with Logistic Regression suggests that its benefits may be more pronounced when applied to simpler models. However, the unexpected outcome on the \textit{Phoneme} dataset calls for further investigation into this phenomenon.


\begin{table*}[h!]
    \centering
    \caption{Mean value and standard deviation of test accuracy across 5 random initializations for different model, dataset, and regularization method. The best results are highlighted in \textbf{bold}.}
    \label{tab:regularization_comparison}
    \begin{tabular}{|c|c|c|c|c|c|c|}
        \hline
        \textbf{Model} & \textbf{Dataset} & \textbf{No-Reg} & \textbf{L1-Reg} & \textbf{L2-Reg} & \textbf{Dropout} & \textbf{CF-Reg (ours)} \\ \hline
        Logistic Regression   & \textit{Water}   & $0.6595 \pm 0.0038$   & $0.6729 \pm 0.0056$   & $0.6756 \pm 0.0046$  & N/A    & $\mathbf{0.6918 \pm 0.0036}$                     \\ \hline
        MLP   & \textit{Water}   & $0.6756 \pm 0.0042$   & $0.6790 \pm 0.0058$   & $0.6790 \pm 0.0023$  & $0.6750 \pm 0.0036$    & $\mathbf{0.6802 \pm 0.0046}$                    \\ \hline
%        MLP   & \textit{Adult}   & $0.8404 \pm 0.0010$   & $\mathbf{0.8495 \pm 0.0007}$   & $0.8489 \pm 0.0014$  & $\mathbf{0.8495 \pm 0.0016}$     & $0.8449 \pm 0.0019$                    \\ \hline
        Logistic Regression   & \textit{Phomene}   & $\mathbf{0.8148 \pm 0.0020}$   & $0.8041 \pm 0.0028$   & $0.7835 \pm 0.0176$  & N/A    & $0.8098 \pm 0.0055$                     \\ \hline
        MLP   & \textit{Phomene}   & $0.8677 \pm 0.0033$   & $0.8374 \pm 0.0080$   & $0.8673 \pm 0.0045$  & $0.8672 \pm 0.0042$     & $\mathbf{0.8718 \pm 0.0040}$                    \\ \hline
        CNN   & \textit{CIFAR-10} & $0.6670 \pm 0.0233$   & $0.6229 \pm 0.0850$   & $0.7348 \pm 0.0365$   & N/A    & $\mathbf{0.7427 \pm 0.0571}$                     \\ \hline
    \end{tabular}
\end{table*}

\begin{table*}[htb!]
    \centering
    \caption{Hyperparameter configurations utilized for the generation of Table \ref{tab:regularization_comparison}. For our regularization the hyperparameters are reported as $\mathbf{\alpha/\beta}$.}
    \label{tab:performance_parameters}
    \begin{tabular}{|c|c|c|c|c|c|c|}
        \hline
        \textbf{Model} & \textbf{Dataset} & \textbf{No-Reg} & \textbf{L1-Reg} & \textbf{L2-Reg} & \textbf{Dropout} & \textbf{CF-Reg (ours)} \\ \hline
        Logistic Regression   & \textit{Water}   & N/A   & $0.0093$   & $0.6927$  & N/A    & $0.3791/1.0355$                     \\ \hline
        MLP   & \textit{Water}   & N/A   & $0.0007$   & $0.0022$  & $0.0002$    & $0.2567/1.9775$                    \\ \hline
        Logistic Regression   &
        \textit{Phomene}   & N/A   & $0.0097$   & $0.7979$  & N/A    & $0.0571/1.8516$                     \\ \hline
        MLP   & \textit{Phomene}   & N/A   & $0.0007$   & $4.24\cdot10^{-5}$  & $0.0015$    & $0.0516/2.2700$                    \\ \hline
       % MLP   & \textit{Adult}   & N/A   & $0.0018$   & $0.0018$  & $0.0601$     & $0.0764/2.2068$                    \\ \hline
        CNN   & \textit{CIFAR-10} & N/A   & $0.0050$   & $0.0864$ & N/A    & $0.3018/
        2.1502$                     \\ \hline
    \end{tabular}
\end{table*}

\begin{table*}[htb!]
    \centering
    \caption{Mean value and standard deviation of training time across 5 different runs. The reported time (in seconds) corresponds to the generation of each entry in Table \ref{tab:regularization_comparison}. Times are }
    \label{tab:times}
    \begin{tabular}{|c|c|c|c|c|c|c|}
        \hline
        \textbf{Model} & \textbf{Dataset} & \textbf{No-Reg} & \textbf{L1-Reg} & \textbf{L2-Reg} & \textbf{Dropout} & \textbf{CF-Reg (ours)} \\ \hline
        Logistic Regression   & \textit{Water}   & $222.98 \pm 1.07$   & $239.94 \pm 2.59$   & $241.60 \pm 1.88$  & N/A    & $251.50 \pm 1.93$                     \\ \hline
        MLP   & \textit{Water}   & $225.71 \pm 3.85$   & $250.13 \pm 4.44$   & $255.78 \pm 2.38$  & $237.83 \pm 3.45$    & $266.48 \pm 3.46$                    \\ \hline
        Logistic Regression   & \textit{Phomene}   & $266.39 \pm 0.82$ & $367.52 \pm 6.85$   & $361.69 \pm 4.04$  & N/A   & $310.48 \pm 0.76$                    \\ \hline
        MLP   &
        \textit{Phomene} & $335.62 \pm 1.77$   & $390.86 \pm 2.11$   & $393.96 \pm 1.95$ & $363.51 \pm 5.07$    & $403.14 \pm 1.92$                     \\ \hline
       % MLP   & \textit{Adult}   & N/A   & $0.0018$   & $0.0018$  & $0.0601$     & $0.0764/2.2068$                    \\ \hline
        CNN   & \textit{CIFAR-10} & $370.09 \pm 0.18$   & $395.71 \pm 0.55$   & $401.38 \pm 0.16$ & N/A    & $1287.8 \pm 0.26$                     \\ \hline
    \end{tabular}
\end{table*}

\subsection{Feasibility of our Method}
A crucial requirement for any regularization technique is that it should impose minimal impact on the overall training process.
In this respect, CF-Reg introduces an overhead that depends on the time required to find the optimal counterfactual example for each training instance. 
As such, the more sophisticated the counterfactual generator model probed during training the higher would be the time required. However, a more advanced counterfactual generator might provide a more effective regularization. We discuss this trade-off in more details in Section~\ref{sec:discussion}.

Table~\ref{tab:times} presents the average training time ($\pm$ standard deviation) for each model and dataset combination listed in Table~\ref{tab:regularization_comparison}.
We can observe that the higher accuracy achieved by CF-Reg using the score-based counterfactual generator comes with only minimal overhead. However, when applied to deep neural networks with many hidden layers, such as \textit{PreactResNet-18}, the forward derivative computation required for the linearization of the network introduces a more noticeable computational cost, explaining the longer training times in the table.

\subsection{Hyperparameter Sensitivity Analysis}
The proposed counterfactual regularization technique relies on two key hyperparameters: $\alpha$ and $\beta$. The former is intrinsic to the loss formulation defined in (\ref{eq:cf-train}), while the latter is closely tied to the choice of the score-based counterfactual explanation method used.

Figure~\ref{fig:test_alpha_beta} illustrates how the test accuracy of an MLP trained on the \textit{Water Potability} dataset changes for different combinations of $\alpha$ and $\beta$.

\begin{figure}[ht]
    \centering
    \includegraphics[width=0.85\linewidth]{img/test_acc_alpha_beta.png}
    \caption{The test accuracy of an MLP trained on the \textit{Water Potability} dataset, evaluated while varying the weight of our counterfactual regularizer ($\alpha$) for different values of $\beta$.}
    \label{fig:test_alpha_beta}
\end{figure}

We observe that, for a fixed $\beta$, increasing the weight of our counterfactual regularizer ($\alpha$) can slightly improve test accuracy until a sudden drop is noticed for $\alpha > 0.1$.
This behavior was expected, as the impact of our penalty, like any regularization term, can be disruptive if not properly controlled.

Moreover, this finding further demonstrates that our regularization method, CF-Reg, is inherently data-driven. Therefore, it requires specific fine-tuning based on the combination of the model and dataset at hand.

\begin{table*}[t]
\resizebox{\linewidth}{!}{%
\centering
\small
\begin{tabular}{l|r|r|r}\toprule
\multirow{2}{*}{\textbf{Model}} &\multicolumn{1}{c|}{\textbf{ASR (\%)}} &\multicolumn{1}{c|}{\textbf{Average Queries Per Jailbreak}} &\multicolumn{1}{c}{\textbf{Number of Jailbreaking Templates}} \\
&\multicolumn{1}{c|}{(higher is better)} &\multicolumn{1}{c|}{(lower is better)} &\multicolumn{1}{c}{(higher is better)} \\\cmidrule{1-4}
Gemma 7B (Original) &100 &6.88 &30 \\
Gemma 7B (Fine-tuned) &26 &75.88 &26 \\
\bottomrule
\end{tabular}
}
\caption{\bedrockfuzz attack performance on Gemma 7B before and after fine-tuning evaluated on 200 harmful behaviors from HarmBench~\cite{mazeika2024harmbench} text standard dataset with a target model query budget of 4000.}
\label{tab:rq1_ft}
\end{table*}



\subsection{Improving In-built Defenses with Supervised Adversarial Training}
\label{sec:defense}
Jailbreaking artifacts generated by \bedrockfuzz represent high-quality data that can be utilized to develop effective defensive and mitigation techniques. One defensive technique is to adapt jailbreaking data to perform supervised fine tuning with the objective of improving in-built safety mitigation in the fine-tuned model.




We performed instruction fine tuning for Gemma 7B using HuggingFace SFTTrainer\footnote{\url{https://huggingface.co/docs/trl/sft_trainer}} with QLoRA~\cite{dettmers2023qlora} and FlashAttention~\cite{dao2022flashattention}.
We collected a total of 1171 attack prompts that were successful in jailbreaking Gemma 7B (200 from Table~\ref{tab:rq1} and 971 from Table~\ref{tab:rq3}), paired each one of them with sampled safe responses generated by Gemma 7B for the corresponding question, and used these $( \text{successful attack prompt}, \text{safe response} )$ pairs as the fine-tuning dataset.



\begin{table}[!htp]
\centering
\small
\begin{tabular}{l|rr}\toprule
\multirow{2}{*}{Metric (\%)} &\multicolumn{2}{c}{Gemma 7B} \\\cmidrule{2-3}
&Original &Fine-tuned \\\midrule
ASR &100 &35 \\\midrule
Top-1 Template ASR &75 &16 \\
Top-5 Template ASR &98 &30 \\
\bottomrule
\end{tabular}
\caption{Templates learnt with \bedrockfuzz in \textit{RQ1} (Table~\ref{tab:rq1}) evaluated on 100 harmful questions from JailBreakBench~\cite{chao2024jailbreakbench} for attacking Gemma 7B before and after fine tuning.}
\label{tab:rq3_ft}
\end{table}


Tables~\ref{tab:rq1_ft} \&~\ref{tab:rq3_ft} present the comparison of the original versus fine-tuned Gemma 7B.
We found attacking the fine-tuned model by \bedrockfuzz to generate new successful templates to become much more difficult, reaching a much lower ASR and requiring many more queries per jailbreak (Table~\ref{tab:rq1_ft}).
Similarly, the fine-tuned model showed significantly lower attack success rates when evaluated on the previously-successful templates (Table~\ref{tab:rq3_ft}).

\section{Conclusion}
In this work, we propose a simple yet effective approach, called SMILE, for graph few-shot learning with fewer tasks. Specifically, we introduce a novel dual-level mixup strategy, including within-task and across-task mixup, for enriching the diversity of nodes within each task and the diversity of tasks. Also, we incorporate the degree-based prior information to learn expressive node embeddings. Theoretically, we prove that SMILE effectively enhances the model's generalization performance. Empirically, we conduct extensive experiments on multiple benchmarks and the results suggest that SMILE significantly outperforms other baselines, including both in-domain and cross-domain few-shot settings.
\section{Acknowledgements}

Our study was approved by the IRB of our institution.
Participants electronically signed a consent form describing the nature of our study and the data we would collect: their answers to the questionnaires, their demographic information provided by the platform, and their interactions with the study platform. All data was stored pseudonymously.
While our initial study description did not explicitly mention participants they would be exposed to phishing, this is a commonly used method in most phishing studies~\cite{resnik2018ethics,thomopoulos2023methodologies} to avoid excessive priming.
The participants were debriefed after completing the study with the full description, and is confirmed to incur only minimal risks~\cite{finn2007designing}, also confirmed by our IRB classifying our study as minimal risk.
Participants were appropriately remunerated for their time with a payment matching the highest minimum wage in their country.

We took further countermeasures to ensure participants' safety: the discomfort of being exposed to phishing emails was mitigated by the roleplay setting and their assigned fictitious identity.
Furthermore, their task was limited to clicking on links---there was no interaction with simulated phishing websites or other potentially harmful content.
Additionally, the phishing URLs we provided did not offer an easy way for participants to actually visit them (as our environment was preventing navigation); however, to protect participants that might transcribe or copy-paste them into their browsers, we constantly monitored all URLs to ensure they were offline during the duration of the study.



\bibliography{main}

\vspace*{\fill} \pagebreak
\appendix
% --------------------------------------------------------------------------
% --------------------------------------------------------------------------
\section{Supplementary Material}\label{sec:appendix}

% --------------------------------------------------------------------------
% --------------------------------------------------------------------------

% \subsection{Western United States Learned Masks}\label{sec:masks}

\begin{figure}[h!]
    \centering
    \noindent\includegraphics[width=0.95\textwidth]{supplemental_figures/wus_masks_ltall.png}
    \caption[Western United States learned masks, 1-10 years]{
        Model learned masks for the western United States for lead times 1-10 years.
    }
    \label{fig:wus_all_learned_masks}
\end{figure}

\begin{figure}[h!]
    \centering
    \noindent\includegraphics[width=0.95\textwidth]{supplemental_figures/wus_tlmasks_ltall.png}
    \caption[Western United States transfer learned masks, 1-10 years]{
        Transfer learned masks for the western United States for lead times 1-10 years.
    }
    \label{fig:wus_all_tl_masks}
\end{figure}

% --------------------------------------------------------------------------
% --------------------------------------------------------------------------

% \subsection{Metrics Versus Number of Analogs}\label{sec:number_analogs}

\begin{figure}[h!]
    \centering
    \noindent\includegraphics[width=0.95\textwidth]{supplemental_figures/wus_mse_nana_full.png}
    \caption[MSE versus number of analogs]{
        MSE versus number analogs used to calculate the mean prediction, for all lead times.
        Also shown are the global and regional mask results.
    }
    \label{fig:mse_nana_full}
\end{figure}

\begin{figure}[h!]
    \centering
    \noindent\includegraphics[width=0.95\textwidth]{supplemental_figures/wus_crps_nana_full.png}
    \caption[CRPS versus number of analogs]{
        Same as Figure \ref{fig:mse_nana_full}, but for CRPS.
    }
    \label{fig:crps_nana_full}
\end{figure}

% --------------------------------------------------------------------------
% --------------------------------------------------------------------------

% \subsection{Extra Metrics: EMD and Class Accuracy}\label{sec:extra_metrics}

\begin{figure}[h!]
    \centering
    \noindent\includegraphics[width=0.8\textwidth]{supplemental_figures/wus_emd_lead.png}
    \caption[Earth mover's distance western United States results]{
        EMD versus lead time for the western United States for mean predictions (line plots) and individual analogs (distributions).
        The top two panels show EMD for the time period $1864$-$2023$, while the bottom two panels show EMD for $2009$-$2018$.
    }
    \label{fig:emd_lead}
\end{figure}

\begin{figure}[h!]
    \centering
    \noindent\includegraphics[width=0.8\textwidth]{supplemental_figures/wus_acc4_lead.png}
    \caption[4 class accuracy for western United States]{
        Same as Figure \ref{fig:emd_lead}, but for class accuracy, with four classes.
    }
    \label{fig:acc4_lead}
\end{figure}

\begin{figure}[h!]
    \centering
    \includegraphics[width=0.6\textwidth]{supplemental_figures/regions_emd.png}
    \includegraphics[width=0.6\textwidth]{supplemental_figures/regions_acc4.png}
    \caption[EMD and class accuracy for 5 regions]{
        EMD (top) and class accuracy (bottom, 4 classes).
        The circles are the mean metric covering the time period $1956$-$2023$, the squares cover $1999$-$2018$.
    }
    \label{fig:emdacc_region}
\end{figure}


% --------------------------------------------------------------------------
% --------------------------------------------------------------------------

% \subsection{Prediction Year Versus Analog Year}\label{sec:year_comp}

\begin{figure}[h!]
    \centering
    \includegraphics[width=0.48\textwidth]{supplemental_figures/year_comp_wus5.png}
    \includegraphics[width=0.48\textwidth]{supplemental_figures/year_comp_regional_wus5.png}
    \caption[Analog year versus prediction year]{
        For each prediction year, the range of years for the top ten analogs (top panel) as well as the error over that time (bottom).
        While the model learned mask (left) begins following the one-to-one line (indicating that the selected analog year is the same as the prediction year) by the $1970$s, the regional mask (right) only does so in the final few years.
        In the bottom panels, it can be seen that the model learned mask predictions are not biased toward underestimating or overestimating at any point, while the regional mask predictions tends towards underestimation starting in the $1970$s.
        This is likely due to the regional mask not being able to pick up on the forced warming signal, which may be most apparent outside the target region.
    }
    \label{fig:year_comp}
\end{figure}



\end{document}
