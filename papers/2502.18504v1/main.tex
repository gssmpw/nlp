\pdfoutput=1

\documentclass[11pt]{article}




\usepackage[preprint]{acl}

\usepackage{times}
\usepackage{latexsym}

\usepackage[T1]{fontenc}

\usepackage[utf8]{inputenc}

\usepackage{microtype}

\usepackage{inconsolata}

\usepackage{graphicx}
\graphicspath{{./figs/}{./}}

\usepackage{amsfonts}
\usepackage{multirow} 
\usepackage{booktabs}
\usepackage{enumitem}
\usepackage{amsmath}

\usepackage{algorithm}
\usepackage[noend]{algpseudocode}
\usepackage{xspace}
\usepackage{stfloats}
\usepackage{xcolor} %
\usepackage{mdframed}
\newmdenv[
backgroundcolor=hlcolor,
topline=false,
bottomline=false,
leftline=false,
rightline=false,
]{shaded}


\newcommand{\qnote}[1]{[\textcolor{blue}{Q-note: #1}]}




\definecolor{color1}{HTML}{f3f3f3}
\definecolor{color2}{HTML}{000000}
\newmdenv[
backgroundcolor=color1,
fontcolor=color2,
topline=true,
bottomline=true,
leftline=true,
rightline=true,
]{shaded1}

\newcommand{\oldalglinenumber}{}%
\NewDocumentCommand{\NoLNElse}{ m }{%
  \RenewCommandCopy{\oldalglinenumber}{\alglinenumber}%
  \RenewDocumentCommand{\alglinenumber}{ m }{}%
  \Else #1%
  \addtocounter{ALG@line}{-1}%
  \RenewCommandCopy{\alglinenumber}{\oldalglinenumber}%
}
\NewDocumentCommand{\NoLNState}{}{%
  \RenewCommandCopy{\oldalglinenumber}{\alglinenumber}%
  \RenewDocumentCommand{\alglinenumber}{ m }{}%
  \State%
  \addtocounter{ALG@line}{-1}%
  \RenewCommandCopy{\alglinenumber}{\oldalglinenumber}%
}

\newcommand{\todo}{\textcolor{red}{TODO}}
\newcommand{\aman}[1]{\textcolor{orange}{Aman: #1}}



\newcommand{\bedrockfuzz}{{\sc TurboFuzzLLM}\xspace}





















\title{\bedrockfuzz: Turbocharging Mutation-based Fuzzing for Effectively Jailbreaking Large Language Models in Practice}


\author{Aman Goel$^*$, Xian Carrie Wu, Zhe Wang, Dmitriy Bespalov, Yanjun Qi$^*$  \\
  Amazon Web Services, USA \\
  \texttt{\{goelaman, xianwwu, zhebeta, dbespal, yanjunqi\}@amazon.com} \\}


\begin{document}
\maketitle
\begingroup\def\thefootnote{*}\footnotetext{Corresponding authors}\endgroup
\begin{abstract}
Jailbreaking large-language models (LLMs) involves testing their robustness against adversarial prompts and evaluating their ability to withstand prompt attacks that could elicit unauthorized or malicious responses. In this paper, we present \bedrockfuzz, a mutation-based fuzzing technique for efficiently finding a collection of effective jailbreaking templates that, when combined with harmful questions, can lead a target LLM to produce harmful responses through black-box access via user prompts. We describe the limitations of directly applying existing template-based attacking techniques in practice, and present functional and efficiency-focused upgrades we added to mutation-based fuzzing to generate effective jailbreaking templates automatically. \bedrockfuzz achieves $\geq$ 95\% attack success rates (ASR) on public datasets for leading LLMs (including GPT-4o \& GPT-4 Turbo), shows impressive generalizability to unseen harmful questions, and helps in improving model defenses to prompt attacks.\footnote{\textcolor{red}{Warning: This paper contains techniques to generate unfiltered content by LLMs that may be offensive to readers.}}
\end{abstract}

\documentclass[../main.tex]{subfiles}
\graphicspath{{../images/}}
\makeatletter
\def\input@path{{../images/}}
\makeatother
\begin{document}
\section{Introduction}
\begin{figure}
\centering
\begin{tikzpicture}
\node[inner sep=0pt] (ws) at (0, 0) {
\includegraphics[height=.4\textwidth, trim={10cm 0 10cm 0},clip]{world_space.png}};
\node[inner sep=0pt] (cs) at (6,0) {\includegraphics[height=.4\textwidth, trim={10cm 1cm 10cm 4cm},clip]{conf_space.png}};
\end{tikzpicture}
\vspace{-5pt}
\label{fig:pbrm_intro}
\caption{\textbf{Left}: Shows world space obstacles as grey spheres. Robots start and goal configuration is colored red and green, respectively. Configurations along the computed path are colored transparent blue. \textbf{Right:} Mapped world space scenario to configuration space. Obstacle region is the grey mesh. Red spheres are collision-free regions computed by the neural SCDF. The optimized shortest path in the convex corridor is the blue curve.}
\vspace{-25pt}
\end{figure}
Motion planning is the problem of finding a collision-free trajectory that connects a given start and goal configuration. The planning takes place in the configuration space of the robot. For single body robots, like mobile robots or drones, the configuration space and the world space are usually the same. This simplifies the planning, since explicit obstacle representations are available which enables geometrical tools like separating hyperplanes, smallest distance to obstacles etc., to be used when designing motion planning algorithms. For multi-body robots like manipulators, the situation is completely different. The world space obstacles are usually mapped to non-convex regions, and to make the problem even harder, the mapping is usually not known. Forming explicit representations of the obstacle region in the configuration space is usually too expensive or intractable. Despite all of this, sampling based planners are used with great success, which mainly is due to their use of implicit representations of the obstacle region. The basic idea is to construct a graph in the configuration space that covers and connects the collision-free region. From this graph, a path can be extracted that connects a given start and goal configuration. The approach is computationally expensive, since the graph is constructed with the smallest geometrical building block available, points, which represents a collision-check. Furthermore, the extracted paths from the graph are non-smooth and jagged due to the stochastic nature of the approach. This adds an additional post-processing step to the process, where the paths are shortcutted and smoothened, before the path can be used for tracking. Clearly a lot of time is invested to form this graph and produce smooth paths. Thus, if the obstacles start to move, then all of this work is done in no use, since all points that make up this graph need to be re-verified, which is simply too time consuming to be done in real time.
\\\\
In this work, we want to address the existing drawbacks of the sampling based planners. Our main contribution is an improved motion planner where each vertex in the graph covers a collision-free region in the form of a sphere instead of a point and where the edges are formed with neighboring intersecting spheres. This representation has the advantage of instead of returning piecewise linear paths, returning a sequence of overlapping spheres, i.e. a convex corridor, that connects a given start and goal configuration, illustrated in Figure \ref{fig:pbrm_intro}. This convex corridor allows us to use convex optimization to produce smooth trajectories, instead of computationally expensive post-processing methods. The representation further allows us to estimate the coverage of the collision-free space, which gives us awareness and feedback in the offline roadmap construction phase. Finally, our representation is simple to adapt to moving obstacles, simply requery for the new radii and recheck for intersections. 
\\\\
The spherical collision-free regions are formed using a signed distance function (SDF), which is a function that returns the smallest distance from an arbitrary point to the boundary of an obstacle. As the name implies, the distance is signed, thus if the point is inside the obstacle it is negative otherwise positive. If the distance is positive, a sphere with radius equal to the distance is guaranteed to cover a collision-free region. Using an SDF in motion planning is not new, but what is novel about our approach is that we express the distance in the configuration space instead of the world space and by doing so allows us to form these convex collision-free regions. We refer to the resulting SDF as a signed configuration distance function (SCDF). Computing an SCDF analytically is non-trivial, our approach is therefore to parameterize the SCDF with a deep neural network and learn the mapping by supervised learning. Our resulting neural SCDF can compute distances for different parameter values of obstacle shapes and we also show how multiple distances can be combined, thus making our approach flexible.
\section{Related work}
Motion planning algorithms can roughly be divided into three families, grid-based, sampling based and optimization based methods. Grid-based methods (GBM) discretize the planning space from which a graph is then compiled. A standard search method is A$^\star$ \citep{a_star}, which is classified as an \textit{informed} search method, since it employs a heuristic function to speed up the search. A$^\star$ guarantees to return an optimal path at the level of discretization used. GBMs usually discretize the planning space by a regular lattice and this limits the GBMs to problems with low dimensionality due to the curse of dimensionality. Thus, GBMs are usually limited to single-body robots where the degrees of freedom (DOF) are low. To overcome the inherent scaling problem with the GBMs, stochastic methods are usually used for multi-body robots. These methods are termed as sampling-based methods (SBM) and core members within this family are the rapidly-exploring random trees (RRT) \citep{rrt} and the probabilistic roadmap (PRM) \citep{prm}. RRT grows a tree from the start configuration and explores the collision-free region in a rapid way until it is able to connect to the goal region. RRT is usually improved by bi-directional planning \citep{rrt_connect}, i.e. an additional tree is grown from the goal configuration and the trees are tested for connection after any tree has been expanded. RRT is a single-query method, thus it searches for a path from scratch each time it is queried. Contrary to this, PRM is a multi-query method, which solves for multiple queries without starting from scratch. PRM does this by creating a roadmap (graph) that covers the collision-free space as an offline step. The graph is then used to solve for multiple queries. PRMs are used in cases where the environment does not change since the extra offline step is too computationally costly and needs to be re-done if the environment is changed. In our work, we address this inherent issue by using a different roadmap representation. Our vertices in the graph cover a collision-free region in the form of spheres and we form the edges by checking for intersecting spheres. If something in the environment changes, we recompute the spheres radii and recheck the intersections, without relying on collision detection. We use a trained neural network to compute the sphere radius, therefore querying for the radius can be done fast, hence our representation enables the PRM for dynamic environments.
\\\\
In the recent decades, optimization based methods (OBM) \citep{chomp, schulman, itomp, stomp} have been introduced as an alternative to SBM for multi-body robots. Like the SBM, the OBMs scale well to higher dimensional problems and produce smoother motion. It is common to use a SDF in the optimization since it is a smooth function, thus enabling gradient-based methods. However, the standard way of expressing the SDF is in world space. The distance therefore needs to be mapped to the configuration space by the forward kinematics. This mapping makes the optimization problem a non-linear program (NLP), which is computationally expensive to solve. Recently, a different approach has been proposed. In \cite{mp_gcs} motion planning is formulated as a convex optimization problem by using the graph of convex sets framework \citep{gcs}. The underlying idea is to decompose the collision-free space into intersecting convex sets from which a convex optimization problem is formulated. In cases where an explicit representation of the obstacles in the configuration space exists, like for single-body robots, creating collision-free convex regions can be done fast \citep{iris}. For multi-body robots, this is non-trivial. Existing work does this successfully \citep{iris_nlp, iris_c} by an optimization based approach, but the methods are still too time consuming to be used in the presence of moving obstacles. Our approach is instead to use deep learning to learn an SDF expressed in the configuration space. With this, we can query for shortest distances to the collision boundary, which allows us to expand spherical regions which are collision-free. Our approach is fast and therefore enables our suggested roadmap planner to be used in dynamic environments.
\\\\
Recent research has focused on learning collision detection \citep{fk_kernel_distance, diffco, graphdistnet} by predicting the signed distance between the robot links and the surrounding obstacles in the world space. The learned SDF is used in trajectory optimization but since the distance is expressed in the world space, the problem becomes an NLP and therefore takes a long time to solve. We take a novel approach and suggest to instead express the signed distance in the configuration space. This allows us to improve the PRM at the same time as it enables convex optimization for trajectory optimization, which runs faster and is more reliable than NLP solvers. In \cite{cspf} a learned signed distance function in the configuration space is proposed similar to our approach. However, their approach is restricted to point cloud representations, while we propose to represent the obstacles as parameterized geometric shapes, e.g. spheres. Furthermore, we also show how to use our learned SCDF to improve an existing roadmap planner.
\section{Problem formulation}
A robot is located in the world space, $\W \subset \R^3 $. The unique location of the robot is given by its configuration $\q \in \C$, where $\C$ is the configuration space. The set of points covered by the robots bodies at a certain configuration is expressed as $\B(\q) \subset \W$. The robot is surrounded by $\NrObst$ obstacles $\O = \bigcup_{i=1}^{\NrObst} \O_i$, where  $\O_i \subset \W$. The representation of the obstacle in the configuration space is the set $\C\O_i = \{\q \in \C \: |\: \B(\q) \cap \O_i \neq \emptyset \}$. The obstacle space is formed as $\Co = \bigcup_{i=1}^{\NrObst} \C \O_i$. The complement is referred to as the free space, $\Cf = \C \setminus \Co$. The path planning problem is a tuple, ($\Cf$, $\qStart$, $\qGoal$), where we want to connect a query pair, consisting of a start, $\qStart$, and goal configuration, $\qGoal$, with a geometric path, $\q(s): [0, 1] \mapsto \Cf$, such that $\q(0)=\qStart$ and $\q(1)=\qGoal$, or report correctly when such a path does not exist.
\end{document}

\section{Method: \bedrockfuzz}
\label{sec:bedrockfuzz}
Figure~\ref{fig:workflow} presents an overview of \bedrockfuzz. Except of a collection of functional (\S\ref{sec:functional}), efficiency-focused (\S\ref{sec:efficiency}), and engineering upgrades (Appendix~\ref{sec:engineering}), the overall workflow of \bedrockfuzz is the same as GPTFuzzer.

Given a set of original templates $O = \{ o_1, o_2, \dots, o_{|O|} \}$, a set of harmful questions $Q = \{ q_1, q_2, \dots, q_{|Q|} \}$, and a target model $T$, \bedrockfuzz performs black-box mutation-based fuzzing to iteratively generate new jailbreaking templates $G = \{ g_1, g_2, \dots, g_{|G|} \}$.
In each fuzzing iteration, \bedrockfuzz selects a template $t$ from the current population $P = O \cup G$ (initially $G = \emptyset$) and a mutation $m$ from the set of all mutations $M$ to generate a new mutant $m(t)$.
Next, the effectiveness of this new template $m(t)$ is evaluated by attacking the target model $T$ using $Q$, i.e., $m(t)$ is combined with questions $q_i \in Q$ to formulate attack prompts $A_{m(t)} = \{ a_{q_1}, a_{q_2}, \dots, a_{q_{|Q|}} \}$, which are queried to $T$ to get a set of responses $R_{m(t)} = \{ r_{q_1}, r_{q_2}, \dots, r_{q_{|Q|}} \}$.
Each response $r_{q_i}$ from $T$ is sent to a judge model to evaluate whether or not $r_{q_i}$ represents a successful jailbreak for question $q_i$, to get the subset of successful jailbreak responses $R_{m(t)}^{success} \subseteq R_{m(t)}$.
If $m(t)$ jailbreaks at least one question (i.e., $R_{m(t)}^{success} \neq \emptyset$), then $m(t)$ is added to $G$, or else, $m(t)$ is discarded.
Fuzzing iterations end when a stopping criteria, such as all questions got jailbroken or the target model query budget, is reached.

\subsection{Functional Upgrades}
\label{sec:functional}
\bedrockfuzz implements two groups of functional upgrades over GPTFuzzer: i) new mutations, and ii) new selection policies to improve the mutant space explored during the search.

\subsubsection{New Mutations}
\label{sec:mutations}
In addition to the 5 mutations from GPTFuzzer~\cite{yu2023gptfuzzer}, we added 2 syntactic and 3 LLM-based new mutations to \bedrockfuzz. 
 
\begin{itemize}
    \item \textit{Refusal Suppression}. This is a static mutation, inspired from~\cite{wei2024jailbroken}, that instructs the model to respond under constraints that rule out common refusal responses, thus making unsafe responses more likely. Figure~\ref{fig:refusal} in Appendix~\ref{app:mutations} details the mutant template generated on applying refusal suppression mutation to a given template.
    
    \item \textit{Inject Prefix}. This is a static mutation, inspired from~\cite{wei2024jailbroken,jiang2024chatbug}, that appends the fixed string ``Sure, here is'' to a given template. This can make the model to heavily penalize refusing and continue answering the unsafe prompt with a jailbreaking response.
    
    \item \textit{Expand After}. This is a LLM-based mutation, inspired from the \textit{Expand} mutation from GPTFuzzer~\cite{yu2023gptfuzzer}, designed to append the new content at the end of the given template (instead of adding new content to the beginning as in \textit{Expand}).

    \item \textit{Transfer Mutation}. This is a LLM-based mutation that transforms a given template $y$ using another template-mutant pair $\left( x, m^*(x) \right)$ as an example, instructing the LLM to infer the (compounded) mutation $m^*$ and return $m^*(y)$. The example mutant $m^*(x)$ is selected randomly from among the top 10 jailbreaking mutants generated so far during fuzzing and $x$ is its corresponding root parent template, i.e., $x \in O$ and $m^*(x) = m_k(\dots m_2(m_1(x))\dots)$. The key idea here is to apply in-context learning to transfer the series of mutations $m_1, m_2, \dots, m_k$ applied to an original template $x$ to derive one of the top ranking mutants $m^*(x)$ identified so far to the given template $y$ in a single fuzzing iteration. Figure~\ref{fig:transfer_mutation} in Appendix~\ref{app:mutations} details the prompt used to apply this mutation to a given template.
    
    \item \textit{Few Shots}. This is a LLM-based mutation that transforms a given template $y$ using a fixed set of mutants $[g_1, g_2, \dots, g_k]$ as in-context examples. These few-shot examples are selected as the top 3 jailbreaking mutants generated so far from the same sub tree as $y$ (i.e., $root(y) = root(g_i)$ for 1 $\leq$ $i$ $\leq$ $k$). The key idea here is to apply few-shot in-context learning to transfer to the given template $y$ a hybrid combination of top ranking mutants identified so far and originating from the same original template as $y$. Figure~\ref{fig:few_shots} in Appendix~\ref{app:mutations} details the prompt used to apply this mutation to a given template.
\end{itemize}

\subsubsection{New Selection Policies}
\label{sec:selection}
\bedrockfuzz introduces new template and mutation selection policies based on reinforcement learning to learn from previous fuzzing iterations which template or mutation could work better than the others in a given fuzzing iteration.

\begin{itemize}
    \item \textit{Mutation selection using Q-learning}.
    \bedrockfuzz utilizes a Q-learning based technique to learn over time which mutation works the best for a given template $t$.
    \bedrockfuzz maintains a Q-table $\mathcal{Q} : \mathit{S} \times \mathit{A} \rightarrow \mathbb{R}$ where $\mathit{S}$ represents the current state of the environment and $\mathit{A}$ represents the possible actions to take at a given state.
    Given a template $t$ selected in a fuzzing iteration, \bedrockfuzz tracks the original root parent $root(t) \in O$ corresponding to $t$ and uses it as the state for Q-learning. The set of possible mutations $M$ are used as the actions set $\mathit{A}$ for any given state. The selected mutation $m$ is rewarded based on the attack success rate of the mutant $m(t)$. Algorithm~\ref{alg:mutation_selection_ql} in Appendix~\ref{app:mutation_selection} provides the pseudo code of Q-learning based mutation selection.
    
    \item \textit{Template selection using multi-arm bandits}. This template selection method is basically the same as Q-learning based mutation selection, except that there is no environment state that is tracked, making it similar to a multi-arm bandits selection~\cite{slivkins2019introduction}.
    Algorithm~\ref{alg:template_selection_ql} in Appendix~\ref{app:template_selection} provides the pseudo code in detail.
    \end{itemize}

\subsection{Efficiency Upgrades}
\label{sec:efficiency}
\bedrockfuzz implements two efficiency-focused upgrades with the objective of jailbreaking more harmful questions with fewer queries to the target model.


\subsubsection{Early-exit Fruitless Templates}
\label{sec:early_exit}
Given a mutant $m(t)$ generated in a fuzzing iteration, \bedrockfuzz exits the fuzzing iteration early before all questions $Q$ are combined with $m(t)$ if $m(t)$ is determined as fruitless. To determine whether or not $m(t)$ is fruitless without making $|Q|$ queries to the target model, \bedrockfuzz utilizes a simple heuristic that iterates over $Q$ in a random order and if any 10\% of the corresponding attack prompts serially evaluated do not result in a jailbreak, $m(t)$ is classified as fruitless. In such a scenario, the remaining questions are skipped, i.e., not combined with $m(t)$ into attack prompts, and the fuzzing iteration is terminated prematurely.

Using such a heuristic significantly reducing the number of queries sent to the target model that are likely futile. However, this leaves the possibility that a mutant $m(t)$ is never combined with a question $q_k \in Q$, even though it might result in a jailbreak. To avoid such a case, we added a new identity/noop mutation such that $m_{identity}(t) = t$. Thus, even if a mutant $m(t)$ is determined as fruitless in a fuzzing iteration $k$, questions skipped in iteration $k$ can still be combined with $m(t)$ in a possible future iteration $l$ ($l > k$) that applies identity mutation on $m(t)$.

\subsubsection{Warmup Stage}
\label{sec:warmup}
\bedrockfuzz adds an initial warmup stage that uses original templates $O$ directly to attack the target model, before beginning the fuzzing stage. The benefits of warmup stage are two-fold: i) it identifies questions that can be jailbroken with original templates directly, and ii) it warms up the Q-table for mutation/template selectors (\S\ref{sec:selection}). Note that the early-exit fruitless templates heuristic (\S\ref{sec:early_exit}) ensures that only a limited number of queries are spent in the warmup stage if the original templates as is are ineffective/fruitless. 








\section{Experiments: Planning outperforms Heuristics}
\label{sec:experiment}

We begin our empirical demonstrations by showcasing the effectiveness of our planning framework on both synthetic and real datasets. We focus on the simplest planning algorithm, 1-step lookaheads (Algorithm~\ref{alg:complete}), and show that even basic planning can hold great promise. 
We illustrate our framework using two uncertainty quantification modules---GPs and 
\ensembles/ \ensembleplus. 

Throughout this section, we focus on evaluating the mean squared error of 
a regression model $\model$,  and develop adaptive policies that minimize uncertainty on $g(f)$ defined in~\eqref{eqn:l2-g-f}.
When GPs provide a valid model of uncertainty, 
our experiments show that our planning framework significantly outperforms other baselines. 
We further demonstrate that our conceptual framework extends to deep learning-based uncertainty quantification methods such as  \ensembleplus while highlighting computational challenges that need to be resolved in order to scale our ideas. 
For simplicity, we assume a naive predictor, i.e., $\psi(\cdot) \equiv 0$. However, we emphasize that this problem is just as complex as if we were using a sophisticated model $\psi(.)$. The performance gap between the algorithms 
primarily depends
on the level  of uncertainty in our prior beliefs.

To evaluate the performance of our algorithm, we benchmark it against several baselines. 
%Active learning baselines use an acquisition function $\ac$ to select points that have the highest   function value: $X\opt_t \in \argmax_{X \in \xpoolj{t}} \ac({X})$ at every step $t$. These methods may also need an UQ module, which we simply use the same UQ module as in our algorithm, and it  outputs $V(X)$ that measures the the uncertainty of each point $X \in \xpoolj{t}$.
Our first set of baselines are from active learning~\citep{AggarwalKoGuHaPh14}:
\\ % \noindent\textbf{Active Learning Heuristics:} 
\textbf{(1)} 
\textsf{Uncertainty Sampling (Static):}  In this approach, we query the samples for which the model is least certain about. Specifically, we estimate the variance of the latent output $f(X)$ for each $X \in \xpool$ using the UQ module and select the top-$K$ points with the highest uncertainty. \\
\textbf{(2)} \textsf{Uncertainty Sampling (Sequential):} This is a greedy heuristic that sequentially selects the points with the highest uncertainty within a batch, while updating the posterior beliefs using pseudo labels from the current posterior state. Unlike \textsf{Uncertainty Sampling (Static)}, this method takes into account the information gained from each point within batch, and hence tries to diversify the selected points within a batch. 

 
We also compare our approach to the  \textbf{(3)} \textsf{Random Sampling}, which selects each batch uniformly at random from the pool. Additionally, we compare solving the planning problem using  \textsf{REINFORCE}-based policy gradients with   $\mathsf{Smoothed\text{-}Autodiff}$ policy gradients.\footnote{Our code repository is available at
  \url{https://github.com/namkoong-lab/adaptive-labeling}.}
%Detailed experimental setups are provided in Section \ref{sec:details-experiments}.

%We repeat all experiments with 10 random seeds.




\begin{figure}[t]
\centering
\begin{minipage}[b]{0.49\textwidth}
\centering
\includegraphics[width=\textwidth, height=5cm]{figures/original_scale/Var_of_l_2_loss.pdf}
\caption{(Synthetic data) Variance of mean squared loss evaluated through the posterior belief $\mu_t$ at each horizon $t$. This is the objective that policy gradient methods like \textsf{REINFORCE} and $\ouralgo$ optimizes. 1-step lookaheads are surprisingly effective even in long horizons.}
\label{fig:var-l2-sim}
\end{minipage}
\hfill
\begin{minipage}[b]{0.49\textwidth}
\centering \includegraphics[width=\textwidth, height=5cm]{figures/original_scale/Error_of_estimated_model_l_2_loss.pdf}
\caption{(Synthetic data) Error between MSE calculated based on collected data $\mc{D}^{0:T}$ vs. population oracle MSE over $\mc{D}_{\rm eval} \sim P_X$. Reducing uncertainty over posteriors directly leads to better OOD evaluations. 1-step lookaheads significantly outperform active learning heuristics in small horizons.}
\label{fig:mean-l2-sim}
\end{minipage}
%\caption{Simulated data for GPs}
%\label{fig:both_plots}
\end{figure}

\subsection{Planning with Gaussian processes}
\label{sec:experiment-plan-GP}
We now briefly describe the data generation process for the GP experiments,  deferring a more detailed discussion of the dataset generation to Section~\ref{sec:details-experiments}. 
We use both the synthetic data and the real data to test our methodology.
For the \emph{simulated data},  we construct a setting where the general population is distributed across \emph{51 non-overlapping clusters} while the initial labeled data $\dtrain$ just comes from one cluster. In contrast, both $\dpool \defeq (\xpool,\ypool),\deval \defeq (\xeval,\yeval)$ are generated   from all the clusters. 
We begin with a low-dimensional scenario, generating a one-dimensional regression setting using a GP. %Gaussian Process (GP).
Although the data-generating process is not known to the algorithms,  we assume that the GP hyperparameters are known to all the algorithms
to ensure fair comparisons. This can be viewed as a setting where our prior is well-specified, allowing us to isolate the effects
of different policy optimization approaches
 without any concerns about the misspecified priors. We select $10$ batches, each of size $K=5$ across $T = 10$ time horizons.

To examine the robustness of our method against the distributional assumptions made  in the simulated case, we then move to a real dataset where the correct prior is not known. We simulate selection bias from the eICU dataset~\citep{PollardJoRaCeMaBa18}, which contains real-world patient data with in-hospital mortality outcomes. 
We conduct a $k$-means clustering to generate 51 clusters and then select data from those clusters. We view this to be a credible replication of practice, as severe distribution shifts are common due to selection bias in clinical labels.  To convert the binary mortality labels into a regression setting, we train a  random forest classifier and fit a GP on predicted scores, which serves as the UQ module for all the algorithms. As before, the task is to select 10 batches, each consisting of 5 samples, across 10 time horizons.

 In Figures~\ref{fig:var-l2-sim} and~\ref{fig:mean-l2-sim}, we present results for the simulated data. 
Figure~\ref{fig:var-l2-sim} shows the variance of $\ell_2$ loss, and Figure~\ref{fig:mean-l2-sim} presents the error in the estimated $\ell_2$ loss using $\mu_t$ (relative to true $\ell_2$ loss, that is unknown to the algorithm). 
As we can see from these plots, our method one-step lookahead  gives substantial improvements  over active learning baselines and random sampling. In addition,
compared to the one-step lookahead planning approach using \textsf{REINFORCE}-based policy gradients, 
we observe that $\mathsf{Smoothed\text{-}Autodiff}$-based policy gradients provide significantly more robust performance over all horizons.

In Figures~\ref{fig:var-l2-real}~and~\ref{fig:mean-l2-real}, we observe similar findings on the eICU data. We see that planning policies (\textsf{REINFORCE} and $\mathsf{Smoothed\text{-}Autodiff}$) consistently outperform other heuristics by a large margin.  Active learning baselines perform poorly in these small-horizon batched problems and can sometimes be even worse than the random search baselines.  Overall, our results show the importance of careful planning in adaptive labeling for reliable model evaluation. 

We offer some intuition as to why one-step lookahead planning may outperform other heuristic algorithms. 
 First,  \textsf{Uncertainty sampling (Static)} while myopically selects the
 top-$K$ inputs with the highest uncertainty, it fails to consider 
the overlap in information content among the ``best” instances; see \citep{AggarwalKoGuHaPh14} for more details. 
In other words,  it might acquire points from the same region with high uncertainty while failing to induce diversity among the batch.
Although \textsf{Uncertainty Sampling (Sequential)} somewhat addresses the issue of information overlap, a significant drawback of 
this algorithm
is the disconnect between the objective we aim to optimize and the algorithm. For example, it might sample from a region with high uncertainty but very low density. 

\begin{figure}[t]
\centering
\begin{minipage}[b]{0.48\textwidth}
\centering
\includegraphics[width=\textwidth, height=5cm]{figures/original_scale/Var_of_l_2_loss_real.pdf}
\caption{(Real-world eICU data) Variance of mean squared loss evaluated through the posterior belief $\mu_t$ at each horizon $t$. Even 1-step lookaheads are extremely effective planners, and auto-differentiation-based pathwise policy gradients provide a reliable optimization algorithm based on low-variance gradient estimates.}
\label{fig:var-l2-real}
\end{minipage}
\hfill
\begin{minipage}[b]{0.48\textwidth}
\centering \includegraphics[width=\textwidth, height=5cm]{figures/original_scale/Error_of_estimated_model_l_2_loss_real.pdf}
\caption{(Real-world eICU data) Error between MSE calculated based on collected data $\mc{D}^{0:T}$ vs. population oracle MSE over $\mc{D}_{\rm eval} \sim P_X$. Reducing uncertainty over posteriors directly leads to better OOD evaluations. Our method significantly outperforms active learning-based heuristics, and random sampling.}
\label{fig:mean-l2-real}
\end{minipage}
%\caption{Real data for GPs}
\end{figure}
 
%\vspace{-1.5cm}
% \begin{wrapfigure}{r}{.32\columnwidth}
%   \vspace{-.5cm} 
%   \centering
% \includegraphics[scale=.29]{figures/Var of l2l_2 loss.pdf}
%   \vspace{-0.2cm}
%   \caption{Results of GP}
% \label{fig:var-l2-gp}
%   \vspace{-0.1cm}
% \end{wrapfigure}


% Attempts have been made  in the past to address these  drawbacks heuristically  (see \citep{AggarwalKoGuHaPh14}). We give a unified computational framework while approaching the problem in a more principled manner and solving it more optimally.




\subsection{Planning with  neural network-based uncertainty quantification methods ($\ensembleplus$)}


We now provide a proof-of-concept that shows the generalizability of our conceptual framework  to the deep learning-based UQ modules, specifically focusing on $\ensembleplus$ due to their previously observed superior performance~\citep{OsbandWenAsDwIbLuRo23}. Recall that implementing our framework with deep learning-based UQ modules  requires us to retrain the model across multiple possible random actions $\bm{a}(\theta)$ sampled from the current policy $\pi_\theta$.
This requires significant computational resources, in sharp contrast to the GPs where the posteriors are in closed form and can be readily updated and differentiated. 

Due to the computational constraints, we test $\ensembleplus$ on a toy setting to demonstrate the generalizability of our framework. We consider a setting where the general population consists of four clusters, while the initial labeled data only comes from one cluster. Again we generate data using GPs.  The task is to select a batch of 2 points in one horizon. We detail the $\ensembleplus$ architecture in Section \ref{sec:details-experiments}, and we assume prior uncertainty to be large (depends on the scaling of the prior generating functions). 
The results are summarized in the Table~\ref{tab:UQ_ensemble}.

% \begin{table}[H]
% \vspace{-10pt}
% \caption{Performance under \ensembleplus as UQ module}
%     \centering
%     \begin{tabular}{|m{3cm}|m{2.5cm}|m{2cm}|} 
%     \hline
%       Algorithm   & Variance of $\loss_2$ loss estimate & Error of $\loss_2$ loss estimate  \\ \hline Random Sampling 
%          & $1710.9 \pm 1352.1$ & $8.67\pm6.62$ 
%       \\ \hline \ouralgo & $1.30 \pm 0.68$ & $0.91\pm0.25$ \\ \hline
%     \end{tabular}
%     \label{tab:UQ_ensemble}
%     %\vspace{-10pt}
% \end{table}




\begin{table}[h]
\vspace{-10pt}
\caption{Performance under \ensembleplus as the UQ module}
\centering
\begin{tabular}{|l|l|l|}
\hline
Algorithm   & Variance of $\loss_2$ loss estimate & Error of $\loss_2$ loss estimate  \\
\hline
\textsf{Random sampling} & 7129.8 $\pm$ 1027.0 & 136.2 $\pm$ 8.28 \\ \hline
\textsf{Uncertainty sampling (Static)} & 10852 $\pm$ 0.0 & 162.156 $\pm$ 0.0 \\ \hline
\textsf{Uncertainty sampling (Sequential)} & 8585.5 $\pm$ 898.9 & 144 $\pm$ 6.93 \\ \hline
\textsf{REINFORCE} & 1697.1 $\pm$ 0.0 & 45.27 $\pm$ 0.0 \\ \hline
\ouralgo & 1697.1 $\pm$ 0.0 & 45.27 $\pm$ 0.0 \\ \hline
\end{tabular}
%\caption{Comparison of different algorithms based on variance   and   error in $\ell_2$ loss estimation with Ensemble $+$ as the UQ module. Our results demonstrate that {\ouralgo} and REINFORCE outperformthe other active learning based heuristics, confirming the benefits of our MDP formulation for the adaptive labeling problem, as also demonstrated in Section 4.\\
%\footnotesize{Experimental details: We use Gaussian Processes as our data generating process, GP parameters are the same as in Section D.3.  The task is to select a batch of 2 points along one horizon.The marginal distribution $p_X$ has 4 \textit{non-overlapping} clusters. Initial data comes from one cluster, while pool and evaluation points comes from all the clusters. We have $20$ initial labeled data points, $10$ pool points, and $252$ evaluation points.  Training procedures are similar to the one in Section D.3.} }
\label{tab:UQ_ensemble}
\end{table}



% We faced  issues in scaling up these experiments which will be our focus in the future. 





% \begin{itemize}
%     \item Posteriors should be consistent. Two dimensions: even with less training,  
%     \item the inference should be  fast enough
% \end{itemize}


% Potential research directions for uncertainty quantification

% In this section we consider a simple setting We consider a simpler setting and 


% For synthetic dataset generation, we use ...... For real datasets, we use ...... We compare our methodolgy to several baselines ()    This Section is structured as follows:
% \begin{itemize}
%     \item \textbf{GPs, square loss objective} (Section \ref{}): 
%     %the broad aim of the experiments  in this section is to isolate the performance of our methodology without any concerns for the inefficiencies induced due to a mis-specified prior or imperfect posterior inference. To accomplish this we generate synthetic datasets using GPs (detailed later). We use the well specified prior (GPs - with same hyperparameter setting) as our UQ module.   
%      As GPs provide differentaible posterior inference - any errors induced due to imperfect posterior updates are also isolated. We note that under this setting
%      \item In Section\ref{} we demonstrate why our methodology performs better than other baselines - by devising various synthetic experiments ()
%     \item  \textbf{UQ Benchmarking }(Section \ref{}): Before diving into the experiments using $\ensembleplus$ and ENNs,  we showcase our benchmarking experiments in Section \ref{}. We use real datasets We observe that ENNs perform better
%      \item \textbf{Ensemble $+$}, objective: recall, accuracy
%     \item \textbf{ENN}, objective: recall, accuracy
% \end{itemize}




% In Section {}, we test 
% \subsection{Experimental details}

% \begin{itemize}
%     \item UQ methodologies - GPs, ENNs
%     \item Objectives - Recall,  ATE
%     \item Datasets - ATE-synthetic datasets, Recall-synthetic, real datasets
%     \item Baselines - 
%     \begin{itemize}
%         \item Random sampling
%         \item Active learning - Uncertainty based sampling - In regression setting almost all of the 
%         \item Myopic greedy - Greedy Batch based sampling
%         \item Policy Gradient
%     \end{itemize}
    
% \end{itemize}

% \subsection{Experiments}
%     \begin{itemize}
%     \item GPs with square loss
%     \item Benchmarking ENN
%         \item ENNs with ATE
%         \item ENNs with Recall
%     \end{itemize}

% \subsection{Benefits over other algorithms - intuition and experiments}

%Active learning - Myopic greedy / Don't rely on the objective rather some entropy version.


%%% Local Variables:
%%% mode: latex
%%% TeX-master: "main"
%%% End:


\begin{table*}[t]
\resizebox{\linewidth}{!}{%
\centering
\small
\begin{tabular}{l|r|r|r}\toprule
\multirow{2}{*}{\textbf{Model}} &\multicolumn{1}{c|}{\textbf{ASR (\%)}} &\multicolumn{1}{c|}{\textbf{Average Queries Per Jailbreak}} &\multicolumn{1}{c}{\textbf{Number of Jailbreaking Templates}} \\
&\multicolumn{1}{c|}{(higher is better)} &\multicolumn{1}{c|}{(lower is better)} &\multicolumn{1}{c}{(higher is better)} \\\cmidrule{1-4}
Gemma 7B (Original) &100 &6.88 &30 \\
Gemma 7B (Fine-tuned) &26 &75.88 &26 \\
\bottomrule
\end{tabular}
}
\caption{\bedrockfuzz attack performance on Gemma 7B before and after fine-tuning evaluated on 200 harmful behaviors from HarmBench~\cite{mazeika2024harmbench} text standard dataset with a target model query budget of 4000.}
\label{tab:rq1_ft}
\end{table*}



\subsection{Improving In-built Defenses with Supervised Adversarial Training}
\label{sec:defense}
Jailbreaking artifacts generated by \bedrockfuzz represent high-quality data that can be utilized to develop effective defensive and mitigation techniques. One defensive technique is to adapt jailbreaking data to perform supervised fine tuning with the objective of improving in-built safety mitigation in the fine-tuned model.




We performed instruction fine tuning for Gemma 7B using HuggingFace SFTTrainer\footnote{\url{https://huggingface.co/docs/trl/sft_trainer}} with QLoRA~\cite{dettmers2023qlora} and FlashAttention~\cite{dao2022flashattention}.
We collected a total of 1171 attack prompts that were successful in jailbreaking Gemma 7B (200 from Table~\ref{tab:rq1} and 971 from Table~\ref{tab:rq3}), paired each one of them with sampled safe responses generated by Gemma 7B for the corresponding question, and used these $( \text{successful attack prompt}, \text{safe response} )$ pairs as the fine-tuning dataset.



\begin{table}[!htp]
\centering
\small
\begin{tabular}{l|rr}\toprule
\multirow{2}{*}{Metric (\%)} &\multicolumn{2}{c}{Gemma 7B} \\\cmidrule{2-3}
&Original &Fine-tuned \\\midrule
ASR &100 &35 \\\midrule
Top-1 Template ASR &75 &16 \\
Top-5 Template ASR &98 &30 \\
\bottomrule
\end{tabular}
\caption{Templates learnt with \bedrockfuzz in \textit{RQ1} (Table~\ref{tab:rq1}) evaluated on 100 harmful questions from JailBreakBench~\cite{chao2024jailbreakbench} for attacking Gemma 7B before and after fine tuning.}
\label{tab:rq3_ft}
\end{table}


Tables~\ref{tab:rq1_ft} \&~\ref{tab:rq3_ft} present the comparison of the original versus fine-tuned Gemma 7B.
We found attacking the fine-tuned model by \bedrockfuzz to generate new successful templates to become much more difficult, reaching a much lower ASR and requiring many more queries per jailbreak (Table~\ref{tab:rq1_ft}).
Similarly, the fine-tuned model showed significantly lower attack success rates when evaluated on the previously-successful templates (Table~\ref{tab:rq3_ft}).

\section*{Conclusion}
This paper aims to enhance our understanding of the computational complexity of computing various Shapley value variants. We found that for various ML models --- including decision trees, regression tree ensembles, weighted automata, and linear regression --- both local and global interventional and baseline SHAP can be computed in polynomial time under HMM modeled distributions. This extends popular algorithms, such as TreeSHAP, beyond their empirical distributional scope. We also establish strict complexity gaps between the various SHAP variants (baseline, interventional, and conditional) and prove the intractability of computing SHAP for tree ensembles and neural networks in simplified scenarios. Overall, we present SHAP as a versatile framework whose complexity depends on four key factors: \begin{inparaenum}[(i)] \item model type, \item SHAP variant, \item distribution modeling approach, \item and local vs. global explanations\end{inparaenum}. We believe this perspective provides deeper insight into the computational complexity of SHAP, paving the way for future work.




%We believe that our framework provides a more intricate understanding of SHAP computation complexity across different models, distributions, and variants, paving the way for further research.

Our work opens promising directions for future research. First, expanding our computational analysis to other SHAP-related metrics, such as asymmetric SHAP~\citep{frye20} and SAGE~\citep{covert2020understanding}, would be valuable. Additionally, we aim to explore more expressive distribution classes and relaxed assumptions beyond those in Section \ref{sec:tractable} while maintaining tractable SHAP computation. Finally, when exact computation is intractable (Section \ref{sec:intractable}), investigating the approximability of SHAP metrics through approximation and parameterized complexity theory~\citep{downey2012parameterized} is an important direction.

%Our work opens several promising avenues for future research on the computational properties of explainable AI methods, with a particular focus on SHAP. First, it would be interesting to broaden the computational analysis conducted in this work to include other popular SHAP-related metrics in the literature, such as asymmetric SHAP \cite{frye20} and SAGE \cite{covert2020understanding}. Also, in the future, we aim to explore more expressive distribution classes and relaxed distributional assumptions—extending beyond those examined in Section \ref{sec:tractable} —that still yield tractable SHAP computation. Finally, when exact computation proves intractable (Section \ref{sec:intractable}), it is worthwhile to theoretically investigate the question of the approximability of computing the SHAP metrics across various configurations, through the lens of approximation and parametrized complexity theory \cite{arora2009computational}.

%This paper aims to deepen our understanding of the computational complexity involved in obtaining different Shapley value variants. We found that for a variety of ML models, including decision trees, tree ensembles for regression, weighted automata, and linear regression models — computing both local and global interventional and baseline SHAP can be done in polynomial time when distributions are modeled by HMMs. This extends the distributional scope of popular algorithms like TreeSHAP, which is limited to empirical distributions. Additionally, we demonstrate a strict complexity gap between SHAP variants, showing that interventional and baseline SHAP can be strictly easier to compute than conditional SHAP. Despite these positive results, we uncovered intractability for various SHAP variants in neural networks and tree ensembles. Finally, we provided generalized complexity relations across SHAP variants. We believe that our framework offers a deeper understanding of the complexity involved in computing SHAP across various variants, models, distributions, as well as in both local and global computations, laying the groundwork for future research.
\section*{Acknowledgments}
{\textcopyright}2025 All rights reserved. The research described in this paper was carried out at the Jet Propulsion Laboratory, California Institute of Technology, under a contract with the National Aeronautics and Space Administration (80NM0018D0004).
%In this paper we emphasize the importance of AI technologies that reflect linguistic diversity and are inclusive, particularly of minority populations such as Black Americans in the United States. Specifically, we explore if large language model-based generative AI technologies support the unique needs of Black Americans with respect to communications in AAE. In doing so,
We recruited Black American study participants to provide their opinions about the generation of AAE in AI technologies, such as AI assistants, and make judgments about how effectively these systems produce AAE. We do not believe our study participants were exposed to any meaningful risks through this process, and we ensured that their remuneration was fair and above average (two and a half times the U.S. federal minimum wage) for their time. Any minor risks that our participants might have been exposed to were delineated in our application to the Institutional Review Board of \emph{redacted}, which was approved with a status of ``Exempt'' on \emph{redacted}. All study participants provided informed consent for their participation. All data utilized by the large language models in this study was anonymized; specifically, we used publicly available transcriptions of interviews with Black Americans from the CORAAL corpus, which was anonymous when we retrieved it online. Finally, we utilized AI code-writing assistance to develop our code used to prepare our data sets.



\bibliography{main}

\vspace*{\fill} \pagebreak
\appendix
\begin{appendices}
\renewcommand{\thetable}{A\arabic{table}}
\counterwithin{table}{section}
\section{Appendices}
The full list of examined primary publications is enumerated in Table\ref{tab:flep}.

\begin{table}[ht!]
\caption{Full List Of Examined Publications
\label{tab:flep}}
\begin{adjustbox}{width=\linewidth, center}
\centering
\begin{tabular}{llll}
\toprule
\textbf{Year} & \textbf{Venue Type} & \textbf{Venue}               & \textbf{Title}              \\

\textsc{Conference} & & &\\   
\midrule
2023          & Conference          & AINA                         & A Vulnerability Detection Method for SDN with Optimized Fuzzing\citep{10.1007/978-3-031-28451-9_46}                                                                            \\
2022          & Conference          & ICAIS                        & Test Traffic Control Based on REST API for Software-Defined Networking\citep{10.1007/978-3-031-06788-4_40}                                                                     \\
2021          & Conference          & SOSR                         & Tardis: A Fault-Tolerant Design for Network Control Planes\citep{10.1145/3482898.3483355}                                                                                      \\
2021          & Conference          & NISS                         & SDN Control Plane Security: Attacks and Mitigation Techniques\citep{10.1145/3454127.3456612}                                                                                   \\
2021          & Conference          & DSN                          & A Comprehensive Study of Bugs in Software Defined Networks\citep{9505089}                                                                                                      \\
2021          & Conference          & NTMS                         & SDN Security through System Call Learning\citep{9432640}                                                                                                                       \\
2020          & Conference          & CIC                          & \begin{tabular}[c]{@{}l@{}}ParaSDN: An Access Control Model for SDN Applications based on Parameterized Roles and \\ Permissions\citep{9319021}\end{tabular}                   \\
2020          & Conference          & SP                           & Unexpected Data Dependency Creation and Chaining: A New Attack to SDN\citep{9152642}                                                                                           \\
2020          & Conference          & CSPS                         & Deep Learning Based Detection Method for SDN Malicious Applications\citep{10.1007/978-981-13-6508-9_13}                                                                        \\
2020          & Conference          & INFOCOM                      & AudiSDN: Automated Detection of Network Policy Inconsistencies in Software-Defined Networks\citep{9155378}                                                                     \\
2020          & Conference          & WISA                         & FSF: Code Coverage-Driven Fuzzing for Software-Defined Networking\citep{10.1007/978-3-030-39303-8_4}                                                                           \\
2020          & Conference          & CAV                          & Towards Model Checking Real-World Software-Defined Networks\citep{10.1007/978-3-030-53291-8_8}                                                                                 \\
2019          & Conference          & IM                           & Mining Software Repositories for Predictive Modelling of Defects in SDN Controller\citep{8717837}                                                                              \\
2019          & Conference          & IM                           & Thinking inside the Box: Differential Fault Localization for SDN Control Plane\citep{8717815}                                                                                  \\
2018          & Conference          & NFV-SDN                      & RE-CHECKER: Towards Secure RESTful Service in Software-Defined Networking\citep{8725649}                                                                                       \\
2018          & Conference          & ICNP                         & INDAGO: A New Framework For Detecting Malicious SDN Applications\citep{8526819}                                                                                                \\
% 2018          & Chapter             &                              & Security Analysis of FloodLight, ZeroSDN, Beacon and POX SDN Controllers\citep{Ilyas2018}                                                                                      \\
2018          & Conference          & CCS                          & AIM-SDN: Attacking Information Mismanagement in SDN-datastores\citep{10.1145/3243734.3243799}                                                                                  \\
2018          & Conference          & ICC                          & SENAD: Securing Network Application Deployment in Software Defined Networks\citep{8422405}                                                                                     \\
% 2018          & Chapter             &                              & Security Analysis of SDN Routing Applications\citep{Sagare2018}                                                                                                                \\
2017          & Conference          & NSDI                         & Automated Bug Removal for Software-Defined Networks\citep{10.5555/3154630.3154688}                                                                                             \\
2017          & Conference          & MASCOTS                      & Testing Black-Box SDN Applications with Formal Behavior Models\citep{8107437}                                                                                                  \\
2017          & Conference          & ICC                          & Controller DAC: Securing SDN controller with dynamic access control\citep{7997249}                                                                                             \\
2017          & Conference          & ISSREW                       & Analytics-Enhanced Automated Code Verification for Dependability of Software-Defined Networks\citep{8109275}                                                                   \\
2017          & Conference          & SIGCOMM                      & BigBug: Practical Concurrency Analysis for SDN\citep{10.1145/3050220.3050230}                                                                                                  \\
2017          & Conference          & ICCNT                        & Requirement analysis for abstracting security in software-defined network\citep{8204161}                                                                                       \\
2017          & Conference          & CNSM                         & An empirical study of software reliability in SDN controllers\citep{8256002}                                                                                                   \\
2017          & Conference          & BWCCA                        & A Comprehensive Security Analysis Checksheet for OpenFlow Networks\citep{10.1007/978-3-319-49106-6_22}                                                                         \\
2017          & Conference          & RAID                         & BEADS: Automated Attack Discovery in OpenFlow-Based SDN Systems\citep{10.1007/978-3-319-66332-6_14}                                                                            \\
2017          & Conference          & MIWAI                        & DREAD-R: Severity Assessment of ONOS SDN Controller\citep{10.1007/978-3-319-69456-6_27}                                                                                        \\
2016          & Conference          & CODASPY                      & SHIELD: An Automated Framework for Static Analysis of SDN Applications\citep{10.1145/2876019.2876026}                                                                          \\
2016          & Conference          & CODASPY                      & The Smaller, the Shrewder: A Simple Malicious Application Can Kill an Entire SDN Environment\citep{10.1145/2876019.2876024}                                                    \\
2016          & Conference          & ISSREW                       & Programming the Network: Application Software Faults in Software-Defined Networks\citep{7789391}                                                                               \\
2016          & Conference          & ICSP                         & BuDDI: Bug detection, debugging, and isolation middlebox for software-defined network controllers\citep{7818438}                                                               \\
2016          & Conference          & ICIEV                        & SDN testing and debugging tools: A survey\citep{7760078}                                                                                                                       \\
2015          & Conference          & ICUFN                        & Secure your Northbound SDN API\citep{7182679}                                                                                                                                  \\
2015          & Conference          & TSA                          & Model-Based Testing of SDN Firewalls: A Case Study\citep{7335947}                                                                                                              \\
2015          & Conference          & SIGCOMM                          & Automated Network Repair with Meta Provenance\citep{10.1145/2834050.2834112}                                                                                               \\
2015          & Conference          & GPCE                         & Safer SDN programming through Arbiter\citep{10.1145/2936314.2814218}                                                                                                           \\
2015          & Conference          & NetSoft                      & Design and deployment of secure, robust, and resilient SDN controllers\citep{7258233}                                                                                          \\
2014          & Conference          & CONEXT                       & Controller-agnostic SDN Debugging\citep{10.1145/2674005.2674993}                                                                                                               \\
2014          & Conference          & PLDI                         & VeriCon: towards verifying controller programs in software-defined networks\citep{10.1145/2666356.2594317}                                                                     \\
2014          & Conference          & NSDI                         & Tierless programming and reasoning for software-defined networks\citep{10.5555/2616448.2616496}                                                                                \\
2014          & Conference          & CONEXT                       & Model Based Black-Box Testing of SDN Applications\citep{10.1145/2680821.2680828}                                                                                               \\
2014          & Conference          & SIGCOMM                      & Troubleshooting blackbox SDN control software with minimal causal sequences\citep{10.1145/2740070.2626304}                                                                     \\
2013          & Conference          & PLDI                         & Machine-verified network controllers\citep{10.1145/2499370.2462178}                                                                                                            \\
2012          & Conference          & NSDI                         & A NICE Way to Test OpenFlow Applications\citep{180591}                                                                                                                         \\
              &                     &                              &                                                                                                                                                                                \\
\textsc{Journal} & & &\\ \hline
2023          & Journal             & WWW                          & DACAS: integration of attribute-based access control for northbound interface security in SDN\citep{Liu2023}                                                                   \\
2022          & Journal             & ToN                          & A Framework for Policy Inconsistency Detection in Software-Defined Networks\citep{9681706}                                                                                     \\
2021          & Journal             & CSUR                         & Application Threats to Exploit Northbound Interface Vulnerabilities in Software Defined Networks\citep{10.1145/3453648}                                                        \\
2021          & Journal             & JPDC                         & \begin{tabular}[c]{@{}l@{}}SEAPP: A secure application management framework based on REST API access control in SDN-enabled \\ cloud environment\citep{HU2021108}\end{tabular} \\
2020          & Journal             & IJCNA                        & Security in SDN: A comprehensive survey\citep{CORREACHICA2020102595}                                                                                                           \\
2019          & Journal             & COMST                        & Fault Management in Software-Defined Networking: A Survey\citep{8456508}                                                                                                       \\
2019          & Journal             & COMST                        & A Survey on Network Verification and Testing With Formal Methods: Approaches and Challenges\citep{8453007}                                                                     \\
2019          & Journal             & JAMT                         & Modeling and Verifying Basic Modules of Floodlight\citep{Xiang2019}                                                                                                            \\
2018          & Journal             & TNSM                         & Assessing the Maturity of SDN Controllers With Software Reliability Growth Models\citep{8386840}                                                                               \\
2018          & Journal             & JSAC                         & MORPH: An Adaptive Framework for Efficient and Byzantine Fault-Tolerant SDN Control Plane\citep{8490892}                                                                       \\
2017          & Journal             & COMST                        & A Survey on Fault Management in Software-Defined Networks\citep{7959044}                                                                                                       \\
2016          & Journal             & IJCNA                        & Secure and dependable software defined networks\citep{AKHUNZADA2016199}                                                                                                        \\
2016          & Journal             & MONET                        & Security in Software-Defined Networking: Threats and Countermeasures\citep{Shu2016}                                                                                            \\

\bottomrule

\end{tabular}
\end{adjustbox}
\end{table}
\end{appendices}


\end{document}
