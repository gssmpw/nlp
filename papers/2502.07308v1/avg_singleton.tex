\section{Average radius generalized Singleton bound}\label{sec:avg-singleton}
%
We now prove that AEL codes satisfy the average-radius generalized Singleton bound. 
%
We will actually prove a more general statement involving a received word where some of the coordinates are erasures, marked with a special symbol $\bot$. We will denote such partially erased received words as $\erase{g}$ instead of $g$ to mark the distinction.
%
For $\erase{g} \in (\Sigma \cup \{\bot\})^n$ and $f \in \Sigma^n$, we define the distance using only the non-erasure coordinates as
\[
\Delta(\erase{g},h) ~\defeq~ 
\frac{1}{n} \cdot \abs{\inbraces{i \in [n] ~\mid~ \erase{g}_i \in \Sigma ~~\text{and}~~ h_i \neq \erase{g}_i}} \mper
\]
%
Note that if $s$ denotes the fraction of erasures in $\erase{g} \in (\Sigma \cup \{\bot\})^n$ and $h_1, h_2$ are two codewords from a code with distance $\delta$, then the triangle inequality implies $\Delta(\erase{g},h_1)+\Delta(\erase{g},h_2) \geq \delta - s$. 
%
The following definition generalizes this to any set of (at most) $k$ distinct codewords.

\begin{definition}[Average-radius list decodable with erasures]\label{def:gen_singleton}
A code $\calC \sub \Sigma^n$ is $(\delta, k,\eps)$ {\deffont average-radius
  list decodable with erasures} if for any $\erase{g} \in (\Sigma \cup \{\bot\})^n$ with (say) $s$ fraction of erasures, and for any set of codewords $\calH \sub \calC$ with $|\calH| \leq k$, it holds that 
\[
\sum_{h \in \calH} \Delta(\erase{g},h) ~\geq~ (\abs{\calH} - 1) \cdot (\delta - s - \eps) \mper
\]
\end{definition}
%
Note that the above definition also implies a lower bound on the distance of the code $\calC$, since for
any two distinct codewords $h_1, h_2$, we can take $g = h_1$ to get $\Delta(h_1, h_2) \geq \delta - \eps$.
%
Moreover, a code $\calC$ satisfying the above definition (even with $s=0$) must have the property that an open ball around any $g \in \Sigma^n$ of radius $(\frac{k-1}{k}) \cdot (\delta - \eps)$ contains at most $k-1$ codewords. For $k = 1/\eps$, this yields a list size of $1/\eps$ at radius $\delta - 2\eps$. 
%\tnote{should be open ball here? there can be k codewords with distance exactly $k-1/k(\delta-\eps)$ from $g$}

We show that the stronger property above of being average-radius list decodable \emph{with erasures}
interacts nicely with the AEL construction, which yields a local-to-global result for this property
\ie if the (constant-sized) inner code $\calC_{\inn} \subseteq \Sigma_{\inn}^d$ used in the AEL
construction is $(\delta, k,\eps)$ average-radius list decodable with erasures, then so is the resulting (global) code $\AELC \subseteq (\Sigma_{\inn}^d)^n$.
%

\begin{theorem}\label{thm:main_technical_avg}
%
Let $k\geq 1$ be an integer and let $\eps > 0$. Let $\AELC$ be a
code obtained using the AEL construction using  $(G, \calC_{\out}, \calC_{\inn})$, where $\calC_{\inn}$ is $(\delta_0, k_0, \eps/2)$ average-radius list decodable with erasures, and $G$ is a $(n,d,\lambda)$-expander for $\lambda \leq \frac{\delta_{\out}}{6{k_0}^{k_0}} \cdot \eps$. 
%
Then, $\AELC$ is $(\delta_{0}, k_0,\eps)$ average-radius list decodable with erasures.
\end{theorem}
%
% It follows from \cref{thm:ael_distance} that $\Delta(\AELC) \geq \delta_{\inn} - \lambda/\delta_{\out}
%   \geq \delta_{\inn} - \eps$. We will denote the quantity $\delta_{\inn} - \eps$ simply as $\Delta$ in the rest of the proof.
%
We need to prove that for any collection $\calH = \inbraces{ h_1,\cdots , h_k} \subseteq \AELC$, and any $\erase{g} \in (\Sigma_{\inn}^d \cup \{\bot\})^R$ with fraction of erasures (say) $s$, we must have 
\[
\sum_{h \in \calH} \Delta_R(\erase{g}, h) ~\geq~ (k-1) \cdot (\delta_0 - s - \eps) \mper
\]
We will prove this by induction on the size $k$ of the collection $\calH$. 
%
Note that the case $k=1$ is trivial since distances are non-negative. 
Before proceeding to the induction step, we first need to understand the ``local views" of codewords $h \in \calH$ from each vertex $\ell \in L$. We refer to them as local projections and develop some inequalities for them below.

%For the inductive step, we assume that the conclusion of the theorem is true for any $\erase{g}$ and any $\calH' \subseteq \AELC$ with $\abs{\calH'} \leq k-1$. \tnote{I feel there is little value in having this para here.}
%

\vspace{-5 pt}
\paragraph{Local projections and induced partitions.}
%
To use the inequalities for the local codes $\calC_{\inn}$, we will need to consider the ``local projections" of codewords $h \in \calH$ for each vertex $\ell \in L$, which are codewords in $\calC_{\inn} \subseteq \Sigma_{\inn}^d$. 
%
For a vertex $\ell \in L$, and $h \in \calH$, let $\hl$ be the local codeword in $\calC_{\inn}$ given by the values in $h$ for the edges incident on $\ell$.
%
We know that the codewords in $\calH$ are pairwise distinct, however, this need not be true for their local projections $\hil{1},\cdots \hil{k}$. 
%
We say that a left vertex $\ell$ \textit{induces a partition} $\tau_{\ell} = (\calH_{1,\ell}, \ldots, \calH_{p_{\ell},\ell})$ of $\calH$, wherein $h_i, h_j$ are in the same part if and only if $\hil{i} = \hil{j}$. 
%
Since the number of partitions if bounded by $k^k$, many left vertices must induce the same partition. We will need the additional fact that this must be a non-trivial partition with the number of parts $p \geq 2$.

\begin{claim}\label{lem:type_arg}
For any set of codewords $\calH = \inbraces{ h_1,\cdots , h_k}$, there exists a partition $\tau^*$
of it with at least 2 non-empty parts, and a set $L^*\subseteq L$ such that $\tau_{\ell} = \tau^*$
for all $\ell \in L^*$, and $|L^*| \geq \delta_\out \cdot n / {k^k}$.
\end{claim}
%	
\begin{proof}
%
Let $L'$ be the set of vertices in $L$ which induces a non-trivial partition, \ie for which not all local projections of the codewords in $\calH$ are identical:
\[
L' ~=~ \{\ell \in L \mid \exists\, i,j \in [k], \; \hil{i} \neq \hil{j} \}.
\]
%
Since the codewords in $\calH$ are distinct, we know that $|L'| \geq \delta_{\out} \cdot n$. 
%
The total number of partitions of $\calH$ is at most $k^k$, and thus among the vertices in $L'$, there must be at least $\nfrac{|L'|}{k^k}$ many vertices that induce a common partition of $\calH$, which by definition of $L'$ will be non-trivial.	
%
\end{proof}
	
	
Throughout the proof, we will work with one such fixed partition $\tau^* = (\calH_1, \ldots, \calH_p)$, and the corresponding set $L^*$. Fix an $\ell \in L^*$ and $j\in [p]$. 
%
By definition, the local codeword $\hl$ is the same for all $h\in \calH_j$. We denote this common codeword by $\fjl$ and the local projection of $\erase{g}$ by $\gl \subseteq (\Sigma_{\inn} \cup \{\bot\})^d$, where if $\erase{g}_r = \bot$ for a vertex $r \in R$, we take projection to be $\bot$ on all edges incident on $r$. 
%
Let $\sl$ denote the fraction of erasures ($\bot$ symbols) in $\gl$. We need the following inequality for local projections.
%
\begin{claim}\label{claim:local-bound}
Let $\tau^*$ and $L^*$ be as above, and  $\gl, \sl$ and $\fjl$ be defined as above for each $\ell \in L^*$. Then,
\[
\sum_{j\in [p]} \Delta(\gl, \fjl) ~\geq~ (p-1) \cdot (\delta_{0} - \sl - \eps/2) \mper
\]
\end{claim}
%
\begin{proof}
The claim follows from the fact that the inner code is $(\delta_0, k_0,\eps/2)$ average-radius list decodable with erasures and $p < k \leq k_0$.
\end{proof}

We will also need the following claim regarding the local erasure fractions $\sl$.
%
\begin{claim}[Sampling bound for erasures]\label{claim:sampling_erasure}
For the set $L^*$ defined as above with $\abs{L^*} \geq \delta_{\out} \cdot n / k^k$, and local erasure fractions $\sl$,
\[
		\Ex{\ell\in L^*}{\sl} ~\leq~ s + \frac{\eps}{6} \mper
\]
\end{claim}
%
\begin{proof}
Let $S = \inbraces{r \in R ~|~ \erase{g}_r = \bot}$ be the set of erasure vertices on the right with $\abs{S} = s \cdot n$. 
Then, applying the expander mixing lemma yields
\[
\sum_{\ell \in L^*} \sl \cdot d ~=~ \abs{E(L^*,S)} ~\leq~ \frac{d}{n} \cdot \abs{L^*} \cdot \abs{S} + \lambda \cdot d \cdot n 
\quad\implies\quad
\Ex{\ell\in L^*}{\sl} ~\leq~ s + \lambda \cdot \frac{n}{\abs{L^*}} \mper
\]
Using $\abs{L^*} \geq \delta_{\out} \cdot n / k^k$ and $\lambda \leq (\delta_{\out}/k^k) \cdot (\eps/6)$ then proves the claim. 
\end{proof}
%

\vspace{-5 pt}
\paragraph{Completing the induction step.} 
%
Recall that using the induction hypothesis, we can say that for any $\erase{g'}$ (with say $s'$ fraction of erasures) and any $\calH' \subseteq \AELC$ with $\abs{\calH'} \leq k-1$, we must have $\sum_{h \in \calH'} \Delta_R(\erase{g'},h) \geq (\abs{\calH'}-1) \cdot (\Delta - s' - \eps)$.
%
%
Since the partition $\tau^* = (\calH_1, \ldots, \calH_p)$ is nontrivial, the cardinality $\abs{\calH_j}$ of each part is at most $k-1$ and we can claim by induction with the given $\erase{g}$ that
\[
\forall j \in [p] \qquad \sum_{h \in \calH_j} \Delta_R(\erase{g}, h) \geq (\abs{\calH_j}-1) \cdot (\Delta - s - \eps) \mper
\]
% 
The following key lemma yields a strengthening of this bound by applying the induction hypothesis with a \emph{different} center $\gj$ for each $\calH_j$.
%


\begin{lemma}[Inductive bound on distances]\label{lemma:inductive}
Let the partition $\tau^* = (\calH_1, \ldots, \calH_p)$ and the set $L^*$ be as above, and let the local projections $\gl$ and $\fjl$ be also as defined above. If the code $\AELC$ is $(\delta_0,k-1,\eps)$ average-radius list decodable with erasures, then for every $j\in [p]$,
\[			
\sum_{h\in \calH_j} \Delta_R(\erase{g},h) ~\geq~  (|\calH_j|-1) \cdot \inparen{\Delta - s - \eps} + \Ex{\ell \in L^*}{\Delta(\gl, \fjl)} - \frac{\eps}{6} \mper
\]
\end{lemma}
%
\begin{proof}
%
By definition of $\fjl$, we have that for all $h \in \calH_j$, $\hl = \fjl$ for all $\ell \in L^*$. Thus, if $\gl$ and $\fjl$ differ on edge $(\ell,r)$ with $\erase{g} \neq \bot$, then $r$ is a \emph{common error location} for all $h \in \calH_j$. We define the set 
\[
S_j ~\defeq~ \inbraces{r \in R ~\mid~ \erase{g}_r \neq \bot ~\text{and}~ \exists \ell \in L^*, e = (\ell,r) ~\text{such that}~ (\gl)_e \neq (\fjl)_e } \mper
\]
Let $s_j = \abs{S_j}/n$ and let $\gj$ be obtained from $\erase{g}$ by replacing symbols in $S_j$ by $\bot$. 
%
The total fraction of erasures in $\gj$ is $(\abs{S}+\abs{S_j})/n = s+s_j$. Also, $\Delta_R\parens[\big]{\gj,h} = \Delta_R(\erase{g},h) - s_j$ for all $h \in \calH_j$, since all vertices in $S_j$ are known to be error locations which are erased in $\gj$. Applying the inductive hypothesis with $\gj$ now gives
\begin{align*}
&\sum_{h \in \calH_j} (\Delta_R(\erase{g},h) - s_j)
~=~ \sum_{h \in \calH_j} \Delta_R\parens[\big]{\gj,h} 
~\geq~ (\abs{\calH_j}-1) \cdot (\Delta - s - s_j - \eps) \\
\implies~~
&\sum_{h \in \calH_j} \Delta_R(\erase{g},h) ~\geq~ (\abs{\calH_j}-1) \cdot (\Delta - s - \eps) + s_j \mper
\end{align*}
%
To obtain a bound on $s_j$, we again use expander mixing lemma to deduce
\[
\sum_{\ell \in L^*} \Delta(\gl,\fjl) \cdot d ~=~ \abs{E(L^*, S_j)} ~\leq~ \frac{d}{n} \cdot \abs{L^*} \cdot \abs{S_j} + \lambda \cdot d \cdot n 
\quad \implies \quad
\Ex{\ell \in L^*}{\Delta(\gl, \fjl)} ~\leq~ s_j + \lambda \cdot \frac {n}{\abs{L^*}} 
\mper 
\]
Using $\abs{L^*} \geq \delta_{\out} \cdot n / k^k$ and $\lambda \leq (\delta_{\out}/k^k) \cdot (\eps/6)$ gives the required bound.
%
\end{proof}

%
We can now prove the induction step for the set $\calH = \inbraces{h_1, \ldots, h_k}$.
%
\begin{proof}[Proof of \cref{thm:main_technical_avg}]
%
%
The proof, as mentioned earlier, is by induction on $k$.
%Note that the case $k=1$ is trivial since distances are non-negative. For the inductive step, we assume that the conclusion of the theorem is true for any $\erase{g}$ and any $\calH' \subseteq \AELC$ with $\abs{\calH'} \leq k-1$. \tnote{Added the para from before here.}

Let $L^*$ and $\tau^* = (\calH_1, \ldots, \calH_p)$ be as above. We use the induction hypothesis to apply the bound from \cref{lemma:inductive} to each part $\calH_j$ which has size at most $k-1$ as $\tau^*$ is non-trivial. This gives,
%
\begin{align*}
\sum_{h\in \calH} \Delta_R(\erase{g},h) 
~=~ \sum_{j\in [p]} \sum_{h\in \calH_j} \Delta_R(\erase{g},h) 
&~\geq~ \sum_{j\in [p]} \inparen{(|\calH_j|-1) \cdot \inparen{\Delta - s - \eps} + \Ex{\ell \in L^*}{\Delta(\gl, \fjl)} - \frac{\eps}{6}}\\
&~=~  (k-p) \cdot \inparen{\Delta - s - \eps} + \sum_{j\in [p]} \Ex{\ell \in L^*}{\Delta(\gl, \fjl)} - \frac{p\eps}{6}.
\end{align*}
%
Using local distance inequality from \cref{claim:local-bound} and the sampling bound from \cref{claim:sampling_erasure}, we can bound the second term as
\[
\sum_{j\in [p]} \Ex{\ell \in L^*}{\Delta(\gl, \fjl)} 
~\geq~ (p-1) \cdot \parens[\Big]{\delta_{\inn} - \Ex{\ell \in L^*}{s_{\ell}} - \frac{\eps}{2}}
~\geq~ (p-1) \cdot  \parens[\Big]{\delta_{\inn} - s - \frac{2\eps}{3}} \mper
\]
Combining the above bounds and using $\delta_{\inn} \geq \Delta$ gives,
\begin{align*}
\sum_{h\in \calH} \Delta_R(\erase{g},h) 
&~\geq~ (k-p) \cdot (\Delta - s - \eps) + (p-1) \cdot  \parens[\Big]{\Delta - s - \frac{2\eps}{3}} - \frac{p\eps}{6} \\
&~=~ (k-1) \cdot (\Delta - s - \eps) + \frac{(p-1)\eps}{3} - \frac{p\eps}{6} \mcom
\end{align*}
which completes the proof since $p \geq 2$.
%
\end{proof}
%
As we will prove in \cref{sec:inner-code}, it is easy to observe using known results by Alrabiah, Guruswami and Li~\cite{AGL24} that a random
linear code satisfies \cref{def:gen_singleton} with high probability, and can thus be used as the
inner code $\calC_{\inn}$. 
%
Since the inner code is a constant-sized object, we can search over all linear codes in $(\F_q)^d$
of dimension $\rho \cdot d$ for a given rate $\rho$, and the code $\AELC$ then yields an explicit
construction of codes achieving the generalized Singleton bound.
%
Moreover, if the inner code is required to be fully explicit, it can also be obtained from folded Reed-Solomon codes, using the results by Chen and Zhang~\cite{CZ24}.
%
\begin{corollary}\label{cor:ael_instantiation}
For every $\rho, \eps \in (0,1)$ and $k \in \N$, there exist explicit inner codes $\calC_{\inn}$ and an infinite family of explicit codes $\AELC \subseteq (\F_q^d)^n$ obtained via the AEL construction that satisfy: 
\begin{enumerate}
\item $\rho(\AELC) \geq \rho$.
\item For any $g \in  (\F_q^d)^n$ and any $\calH \subseteq \AELC$ with $\abs{\calH} \leq k$ that
\[
\sum_{h \in \calH} \Delta(g,h) ~\geq~ (\abs{\calH}-1) \cdot (1 - \rho - \eps) \mper
\]
\item The alphabet size $q^d$ of the code $\AELC$ can be taken to be $2^{O(k^{3k}/\eps^9)}$.
\item $\AELC$ is characterized by parity checks of size $O(k^{2k}/\eps^{11})$ over the field $\F_q$.
\end{enumerate}
\end{corollary}
% 
\begin{proof}
Let $d = O(k^{2k}/\eps^8)$ be such that there exist explicit families of $(n,d,\lambda)$-expander graphs (for arbitrarily large $n$) with $\lambda \leq \eps^4/(2^{18} k^k)$. 
%
Let $\calC_{\inn} \subseteq \F_q^d$ be a code given by \cref{cor:random-code}, with rate $\rho_{\inn} = \rho + \eps/4$, which is $(1 - \rho, k, \eps/2)$ average-radius list decodable with erasures. Note that the alphabet size $q$ for $\calC_{\inn}$ can be taken to be $2^{O(k + 1/\eps)}$.
%
Finally, let $\calC_{\out} \subseteq (\F_q^{\rho_{\inn} \cdot d})^n$ be an outer (linear) code with rate $\rho_{\out} = 1 - \eps/4$ and distance (say) $\delta_{\out} = \eps^3/2^{15}$. Explicit families of such codes can be also be obtained (for example) via expander-based Tanner code constructions (see Theorem 11.4.6 and Corollary 11.4.8 in \cite{GRS23}). Using Tanner codes also gives that $\calC_{\out}$ has parity checks of size at most $O(1/\eps^3)$.
%

Given the above parameters, we have 
$\rho(\AELC) ~\geq~ \rho_{\out} \cdot \rho_{\inn} ~=~ (1-\eps/4) \cdot (\rho + \eps/4) ~\geq~ \rho$.
%
Since $\lambda \leq \eps \cdot \delta_{\out}/(6k^k)$ and $\calC_{\inn}$ is $(1-\rho, k, \eps/2)$ average-radius list decodable with erasures, we can use \cref{thm:main_technical_avg} to conclude that $\AELC$ is $(1-\rho, k, \eps)$ average-radius list decodable (with erasures) which yields the second condition. 
%
Since $\calC_{\out}$ has parity checks of $O(1/\eps^3)$, $\calC_{\inn} \subseteq \F_q^d$, and each symbol of $\AELC$ is a function of at most $d$ symbols from $\calC_{\out}$ (encoded via $\calC_{\inn}$), $\AELC$ can be taken to have parity checks of size at most $O(d \cdot (1/\eps^3)) = O(k^{2k}/\eps^{11})$.
%
Finally, we note that the alphabet size of the code $\AELC$ is $q^d = \exp\inparen{O((k + 1/\eps) \cdot (k^{2k}/\eps^8))} = \exp\inparen{k^{3k}/\eps^9}$, which proves the claim. 
\end{proof}

\vspace{-5 pt}
\paragraph{A weaker consequence of \cref{def:gen_singleton}.}
%
We also state the following consequence of \cref{def:gen_singleton}, which still yields a
strengthening of the average distance inequality and the generalized Singleton bound (when
instantiated with the appropriate code), and may be of independent interest. 
%
Note that this statement also yields a corollary of \cref{thm:main_technical_avg} which is simply in
terms of center $g \in \Sigma^n$ with no erasure symbols, and yields an advantage over the distance
inequality, in terms of the error locations which are \emph{common} to all the codewords $h_1, \ldots, h_k$.
%
\begin{lemma}\label{lemma:common-error-bound}
%
Let $\calC \subseteq \Sigma^n$ be $(\delta_0, k_0, \eps)$ average radius list-decodable with
erasures. Then, for any $g \in \Sigma^n$, any $k \leq k_0$ and $h_1, \ldots, h_k \in \calC$, we have
that
\[ \sum_{i \in [k]} \Delta(g,h_i) ~\geq~ (k-1) \cdot (\delta_0 - \eps) + \Ex{r \in [n]}{\prod_{i \in
    [k]}\indi{h_{i,r} \neq g_r}} \mper
\]
%
\end{lemma}
%
\begin{proof}
Define $\erase{g} \in \Sigma^n$ as 
\[
\erase{g}_r ~=~ 
\begin{cases}
\bot & \text{if} ~g_r \neq h_{i,r} ~\forall i \in [k] \\
g_r &\text{otherwise} \mper
\end{cases}
\]
Note that the fraction of erasure symbols is $s = \Ex{r \in [n]}{\prod_{i \in [k]}\indi{h_{i,r} \neq
    g_r}}$ and $\Delta(\erase{g},h_i) = \Delta(g,h) - s$ for all $i \in [k]$. Applying
\cref{def:gen_singleton} with $\erase{g}$ gives
\[
\sum_{i \in [k]} (\Delta(g,h_i) - s) 
~=~ \sum_{i \in [k]} \Delta(\erase{g},h_i) 
~\geq~ (k-1) \cdot (\delta_0 - s - \eps) \mcom
\]
and rearranging proves the claim.
\end{proof}
% 
%
\begin{remark}
A reader might notice that the definition of the $\erase{g}$ is the same as used in the proof of
\cref{lemma:inductive}. 
%
In fact, it is easy to see that the consequence \cref{lemma:common-error-bound} can directly be
proved via induction using the same proof as \cref{thm:main_technical_avg}, which avoids using a
$\erase{g}$ with erasures as part of the induction (although one still needs the list decodability
with erasures for the inner code $\calC_{\inn}$).
%
While we chose to prove the stronger local-to-global statement as \cref{thm:main_technical_avg}
above, for the algorithmic application we will only prove an algorithmic analogue of
\cref{lemma:common-error-bound} to avoid technical issues with keeping track of arbitrary erasure patterns.
%
\end{remark}
%



%!TEX root=main.tex

%%% Local Variables:
%%% mode: latex
%%% TeX-master: "main"
%%% End:
