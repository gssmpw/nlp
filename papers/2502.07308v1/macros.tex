%%%%%%%%%%%% Packages
\usepackage{xspace,enumerate}
\usepackage{amsmath,amssymb}
\usepackage{amsthm}
\ifnum\confversion=0
	\usepackage[toc,page]{appendix}
\fi
\usepackage{thmtools}
\usepackage{thm-restate}
\usepackage{graphicx}
\usepackage{boxedminipage}
\usepackage{csquotes}
\usepackage{makecell}
\usepackage{tabularx}
\usepackage{relsize}
\usepackage{scalerel}
\usepackage[dvipsnames]{xcolor} %% Load before TikZ
\usepackage{tikz}
%\usepackage{prettyref}
% \usepackage[varg]{txfonts} % varg - uses nicer g,v,y,w

\ifnum\showkeys=1
\usepackage[color]{showkeys}
\fi


\definecolor{darkred}{rgb}{0.5,0,0}
\definecolor{darkgreen}{rgb}{0,0.35,0}
\definecolor{darkblue}{rgb}{0,0,0.55}


%\usepackage[pdfstartview=FitH,pdfpagemode=UseNone,colorlinks=false,linkcolor=darkblue,filecolor=darkred,citecolor=darkgreen,urlcolor=darkred,pagebackref,bookmarks=false]{hyperref}

%Commenting as color names missing
\ifnum\confversion=1
\usepackage[pdfstartview=FitH,pdfpagemode=UseNone,colorlinks=false,linkcolor=darkblue,filecolor=darkred,citecolor=darkgreen,urlcolor=darkred,pagebackref,bookmarks=false]{hyperref}
\else
\usepackage[pdfstartview=FitH,pdfpagemode=UseNone,colorlinks,linkcolor=NavyBlue,filecolor=blue,citecolor=OliveGreen,urlcolor=NavyBlue,pagebackref]{hyperref}
\fi

\usepackage[capitalise,nameinlink]{cleveref}
\usepackage[T1]{fontenc}
% \usepackage{mathtools,dsfont,bbm}

\usepackage{mathtools,dsfont}

%\usepackage{palatino}
%\usepackage[scaled=.95]{helvet}
%\usepackage{eulerpx}


\usepackage{mathpazo}
\usepackage{microtype}
\ifnum\widemargin=0
\usepackage[top=1in, bottom=1in, left=1in, right=1in]{geometry}
\else
\usepackage[top=1in, bottom=1in, left=1.25in, right=1.25in]{geometry}
\fi
% \usepackage{fullpage}


%%%%%%%%%%%%%%% Lengths
% \setlength{\parindent}{0 in} 
\setlength{\parskip}{0.05 in}
\setlength{\parindent}{3 ex} 
% \setlength{\parskip}{0.5 ex}

%%%%%%%%%%%%%%% Author Notes
\ifnum\showauthornotes=1
\newcommand{\Authornote}[3]{{\sf\small\color{#3}{[#1: #2]}}}
\newcommand{\Authorcomment}[2]{{\sf \small\color{gray}{[#1: #2]}}}
\newcommand{\Authorfnote}[2]{\footnote{\color{red}{#1: #2}}}
\else
\newcommand{\Authornote}[3]{}
\newcommand{\Authorcomment}[2]{}
\newcommand{\Authorfnote}[2]{}
\fi

%%%%%%%%%%%%%%%%% Draftbox
%\ifnum\showdraftbox=1
%\newcommand{\draftbox}{\begin{center}
%  \fbox{%
%    \begin{minipage}{2in}%
%      \begin{center}%
%        \begin{Large}%
%          \textsc{Working Draft}%
%        \end{Large}\\
%        Please do not distribute%
%      \end{center}%
%    \end{minipage}%
%  }%
%\end{center}
%\vspace{0.2cm}}
%\else
%\newcommand{\draftbox}{}
%\fi
%
%%%%%%%%%%%%%%% No bullets
%\renewcommand{\labelitemi}{-}


%%%%%%%%%%%%%%%%%% Theorem Environments
%%%%%%%%%%%%%%%%%% Theorem Environments (cleaner version due to Ryan O'Donnell )

\declaretheorem[numberwithin=section]{theorem}
%\declaretheorem{theorem}
\declaretheorem[sibling=theorem]{lemma}
\declaretheorem[sibling=theorem]{claim}
\declaretheorem[sibling=theorem]{proposition}
\declaretheorem[sibling=theorem]{fact}
\declaretheorem[sibling=theorem]{corollary}
\declaretheorem[sibling=theorem]{conjecture}
\declaretheorem[sibling=theorem]{question}
\declaretheorem[sibling=theorem]{answer}
\declaretheorem[sibling=theorem]{solution}
\declaretheorem[sibling=theorem]{hypothesis}
\declaretheorem[sibling=theorem]{exercise}
\declaretheorem[sibling=theorem]{summary}

\theoremstyle{definition}
\declaretheorem[sibling=theorem]{definition}
\declaretheorem[sibling=theorem]{remark}
\declaretheorem[sibling=theorem]{notation}
\declaretheorem[sibling=theorem]{observation}
\declaretheorem[sibling=theorem]{example}

\newtheorem{apdxlemma}{Lemma}
%\newtheorem{algorithm}[theorem]{Algorithm}
\newtheorem{algo}[theorem]{Algorithm}

\newenvironment{algorithm}[3]
        {\noindent\begin{boxedminipage}{\textwidth}\begin{algo}[#1]\ \par
        {\begin{tabular}{r l}
        \textbf{Input} & #2\\
        \textbf{Output} & #3
        \end{tabular}\par\enskip}}
        {\end{algo}\end{boxedminipage}}

%%%%%%%%%%%%%%%%% Proof Environments

\def\FullBox{\hbox{\vrule width 6pt height 6pt depth 0pt}}

\def\qed{\ifmmode\qquad\FullBox\else{\unskip\nobreak\hfil
\penalty50\hskip1em\null\nobreak\hfil\FullBox
\parfillskip=0pt\finalhyphendemerits=0\endgraf}\fi}

\def\qedsketch{\ifmmode\Box\else{\unskip\nobreak\hfil
\penalty50\hskip1em\null\nobreak\hfil$\Box$
\parfillskip=0pt\finalhyphendemerits=0\endgraf}\fi}

%\ifnotconf
%\renewenvironment{proof}{\begin{trivlist} \item {\bf Proof:~~}}
%   {\qed\end{trivlist}}
%\fi

\newenvironment{proofsketch}{\begin{trivlist} \item {\bf
Proof Sketch:~~}}
  {\qedsketch\end{trivlist}}

\newenvironment{proofof}[1]{\begin{trivlist} \item {\bf Proof of
#1:~~}}
  {\qed\end{trivlist}}

\newenvironment{claimproof}{\begin{quotation} \noindent
{\bf Proof of claim:~~}}{\qedsketch\end{quotation}}


%%%%%%%%%% Symbols and Fonts
\def\from{:}
\def\to{\rightarrow}
\def\eps{\varepsilon}
\def\epsilon{\varepsilon}
\def\e{\epsilon}
\def\eps{\epsilon}
\def\d{\delta}
\def\phi{\varphi}
\def\cal{\mathcal}
\def\xor{\oplus}
\def\ra{\rightarrow}
\def\implies{\Rightarrow}
\newcommand{\sub}{\ensuremath{\subseteq}}
\def\psdgeq{\succeq} 
% \newcommand{\defeq}{\stackrel{\mathrm{def}}=}     
\newcommand{\defeq}{:=}     
\providecommand{\cod}{\mathop{\sf cod}\nolimits}


\renewcommand{\bar}{\overline} 
\newcommand{\ol}{\overline}
\newcommand{\given}{\;\ifnum\currentgrouptype=16 \middle\fi \vert\;}
% \renewcommand{\mathbb}{\varmathbb}
% \renewcommand{\Bbbk}{\varBbbk}

\def\bull{\vrule height .9ex width .8ex depth -.1ex }

%%%%%%%%%%%%%%%%%%%%%% Text Macros
\newcommand{\ie}{i.e.,\xspace}
\newcommand{\eg}{e.g.,\xspace}
\newcommand{\etal}{et al.\xspace}
\newcommand{\cf}{{\it cf.,}}

%%%%%%%%%%%%%%%%%%%%% Punctuation at the end of a displayed formula
\newcommand{\mper}{\,.}
\newcommand{\mcom}{\,,}

%%%%%%%%%%%%%%%%%%%%%% Number Sets

%
%%%%%%%%%%% Standard Normal Distribution
%
%\newcommand{\gauss}[2]{{\cal N(#1, #2)}}
%\newcommand{\gaussian}[2]{{\cal N(#1, #2)}}

%%%%%%%%%%%% Fractions
% commands for fractions 
\usepackage{nicefrac}
% poor man's fraction
\newcommand{\flatfrac}[2]{#1/#2}
\newcommand{\varflatfrac}[2]{#1\textfractionsolidus#2}

\let\nfrac=\nicefrac
\let\ffrac=\flatfrac
% similar commands: tfrac,dfrac

\newcommand{\half}{\nicefrac12}
\newcommand{\onequarter}{\nicefrac14}
\newcommand{\threequarter}{\nicefrac34}

%%%%%%%%%%%%%%%%%% Delimiters
\DeclarePairedDelimiter\parens{\lparen}{\rparen}
\DeclarePairedDelimiter\abs{\lvert}{\rvert}
\DeclarePairedDelimiter\norm{\lVert}{\rVert}
\DeclarePairedDelimiter\floor{\lfloor}{\rfloor}
\DeclarePairedDelimiter\ceil{\lceil}{\rceil}
\DeclarePairedDelimiter\braces{\lbrace}{\rbrace}
\DeclarePairedDelimiter\brackets{\lbrack}{\rbrack}
\DeclarePairedDelimiter\angles{\langle}{\rangle}
\DeclarePairedDelimiterXPP\lnorm[1]{}\lVert\rVert{_2}{#1}

%%%%%%%%%%%%%%%%%%%%%%%%%%% Enclosures
\newcommand{\inparen}[1]{\left(#1\right)}             %\inparen{x+y}  is (x+y)
\newcommand{\inbraces}[1]{\left\{#1\right\}}           %\inbrace{x+y}  is {x+y}
\newcommand{\insquare}[1]{\left[#1\right]}             %\insquare{x+y}  is [x+y]
\newcommand{\inangle}[1]{\left\langle#1\right\rangle} %\inangle{A}    is <A>

%
%------------------------------------ Norms -------------------
%\newcommand{\norm}[1]{\ensuremath{\left\lVert #1 \right\rVert}}
%\newcommand{\smallnorm}[1]{\ensuremath{\lVert #1 \rVert}}
%
%\newcommand{\mydot}[2]{\ensuremath{\left\langle #1, #2 \right\rangle}}
%\newcommand{\mysmalldot}[2]{\ensuremath{\langle #1, #2 \rangle}}
%%\newcommand{\ip}[1]{\left\langle #1 \right\rangle}
%\newcommand{\ip}[2] {\ensuremath{\left\langle #1 , #2 \right\rangle}}
%
%\newcommand{\emysmalldot}[2]{\ensuremath{\langle #1, #2 \rangle}_E} %% Euiwoong 20171030
%\newcommand{\cmysmalldot}[2]{\ensuremath{\langle #1, #2 \rangle}_C} %% Euiwoong 20171030
%
%\newcommand{\cip}[1]{\left\langle #1 \right\rangle}  %% Euiwoong 20171030
%\newcommand{\eip}[1]{\left\langle #1 \right\rangle} %% Euiwoong 20171030

%%%%%%%%%%% Vectors
\def\bfx {{\bf x}}
\def\bfy {{\bf y}}
\def\bfz {{\bf z}}
%
\def\bfu{{\bf u}}
\def\bfv{{\bf v}}
\def\bfw{{\bf w}}
%
\def\bfa{{\bf a}}
\def\bfb{{\bf b}}
\def\bfc{{\bf c}}
%
\def\bfg{{\bf g}}

\newcommand{\zero}{\mathbf 0}
\newcommand{\one}{{\mathbf{1}}}
\newcommand{\zeroone}{{0/1}\xspace}
\newcommand{\minusoneone}{{-1/1}\xspace}

\newcommand{\yes}{{\sf Yes}\xspace}
\newcommand{\no}{{\sf No}\xspace}

%%%%%%%%%%%%%%%%%%%%%%%% Lasserre Variables
\newcommand{\vone}[1]{\mathbf{u}_{#1}}
\newcommand{\voneempty}{\mathbf{u}_{\emptyset}}
\newcommand{\vtwo}[2]{\mathbf{v}_{(#1, #2)}}
\newcommand{\vtwoempty}{\mathbf{v}_{(\emptyset, \emptyset)}}

%%%%%%%%%%%%%%%% Sherali-Adams
\newcommand{\qary}{[q]}
\newcommand{\varone}[1]{y_{#1}}
\newcommand{\varoneempty}{y_{\emptyset}}
\newcommand{\vartwo}[2]{x_{(#1, #2)}}
\newcommand{\vartwoempty}{x_{(\emptyset,\emptyset)}}


%%%%%%%%%%%%%%%%%%%%% Random Variables
\newcommand{\Esymb}{\mathbb{E}}
\newcommand{\Psymb}{\mathbb{P}}
\newcommand{\Vsymb}{\mathbb{V}}
\newcommand{\Varsymb}{\mathrm{Var}}
\DeclareMathOperator*{\ExpOp}{\Esymb}
\DeclareMathOperator*{\ProbOp}{\Psymb}

%%%% Variable arguments \Pr, \Ex, \Var, \tildeEx
%%%%  changes its behavior according to number of arguments as follows:
%%%%    - 1: <symbol> [ #1 ]
%%%%    - 2: <symbol>_[#1] [ #2 ] -- this behaves like \max, i.e., subscript is placed
%%%%                underneath the symbol if in displaystyle environment
%%%%    - 4: <symbol>_[#1]  #3 #2 #4 -- here #3 and #4 are the opening and closing
%%%%                braces symbol
\makeatletter
\def\Pr#1{%
    \ProbabilityRender{\Psymb}{#1}%
}

\def\Ex#1{%
    \ProbabilityRender{\Esymb}{#1}%
}

\def\tildeEx#1{%
    \ProbabilityRender{\widetilde{\Esymb}}{#1}%
}

\def\condPE#1#2{%
	\@ifnextchar\bgroup
	{\ConditionalProbabilityRender{\widetilde{\Esymb}}{#1}{#2}}
	{\ProbabilityRender{\widetilde{\Esymb}}{#1 \given #2}}
}

%\def\condPE#1#2#3{%
%    \underset{#1}{\widetilde{\Esymb}}{\left[ #2|#3 \right]}%
%}

\def\Var#1{%
    \ProbabilityRender{\Varsymb}{#1}%
}

\def\tildeVar#1{
	\ProbabilityRender{\widetilde{\Varsymb}}{#1}
}

\def\Cov#1#2{
	\ProbabilityRender{\cov}{#1,#2}%
}

\def\tildeCov#1#2{
	\ProbabilityRender{\tildecov}{#1,#2}
}

\def\ConditionalProbabilityRender#1#2#3#4{
	\renderwithdist{#1}{#2}{#3 \given #4}	
}

\def\ProbabilityRender#1#2{%fancy probability command
  \@ifnextchar\bgroup%
  {\renderwithdist{#1}{#2}}
   {\singlervrender{#1}{#2}}
}
\def\singlervrender#1#2{%
   \ensuremath{\mathchoice
       {{#1}\insquare{ #2 }}
       {{#1}\insquare{ #2 }}
       {{#1}\insquare{ #2 }}
       {{#1}\insquare{ #2 }}
   }
}
\def\renderwithdist#1#2#3{%
   \@ifnextchar\bgroup
   {\superfancyrender{#1}{#2}{#3}}
   {\ensuremath{\mathchoice
      {\underset{#2}{#1}\insquare{ #3 }}
      {{#1}_{#2}\insquare{ #3 }}
      {{#1}_{#2}\insquare{ #3 }}
      {{#1}_{#2}\insquare{ #3 }}
     }
   }
}
\def\superfancyrender#1#2#3#4#5{
   \ensuremath{\mathchoice
      {\underset{#1}{{#1}}\left#4 #3 \right#5}
      {{#1}_{#2}#4 #3 #5}
      {{#1}_{#2}#4 #3 #5}
      {{#1}_{#2}#4 #3 #5}
   }
}
\makeatother


\def\expop{\ExpOp}
\def\probop{\ProbOp}

\newcommand\restrict[1]{\raisebox{-.5ex}{$|$}_{#1}}

\newcommand{\conv}[1]{\mathrm{conv}\inparen{#1}}


%\newcommand{\varex}[1]{\E\paren{#1}}
%\newcommand{\varEx}[1]{\E\Paren{#1}}



% \newcommand{\maximize}{\mathop{\textrm{maximize}}}
% \newcommand{\minimize}{\mathop{\textrm{minimize}}}
% \newcommand{\subjectto}{\mathop{\textrm{subject to}}}


%\newcommand{\poly}{{\mathrm{poly}}}
%\newcommand{\polylog}{{\mathrm{polylog}}}
%\newcommand{\loglog}{{\mathop{\mathrm{loglog}}}}
%\newcommand{\suchthat}{{\;\; : \;\;}}
%\newcommand{\getsr}{\mathbin{\stackrel{\mbox{\tiny R}}{\gets}}}

%
%%%%%%%%%%% Operators
%\DeclareMathOperator\supp{Supp}
%
%%%%%%%%%%% Asymptotic and math symbols borrowed from Laci
%\DeclareMathOperator{\Div}{\operatorname {Div}}
%\DeclareMathOperator{\cov}{\operatorname {Cov}}
%\DeclareMathOperator{\period}{\operatorname {per}}
%\DeclareMathOperator{\rank}{\operatorname {rk}}
%\DeclareMathOperator{\tr}{\operatorname {tr}}
%\DeclareMathOperator{\Span}{\operatorname {span}}
%\DeclareMathOperator{\sgn}{\operatorname{sgn}}
%\DeclareMathOperator{\inv}{\operatorname{Inv}}
%\DeclareMathOperator{\sym}{\operatorname{Sym}}
%\DeclareMathOperator{\alt}{\operatorname{Alt}}
%\DeclareMathOperator{\aut}{\operatorname{Aut}}
%\DeclareMathOperator{\ord}{\operatorname{ord}}
%\DeclareMathOperator{\lcm}{\operatorname{lcm}}
%\DeclareMathOperator*{\argmin}{\arg\!\min}
%\DeclareMathOperator*{\argmax}{\arg\!\max}
%\DeclareMathOperator{\Paley}{\operatorname {Paley}}
%\DeclareMathOperator{\Vtx}{\operatorname V}
%\DeclareMathOperator{\Edg}{\operatorname E}
%\DeclareMathOperator{\im}{\operatorname{im}}
%\DeclareMathOperator{\ran}{\operatorname{ran}}
%\DeclareMathOperator{\diag}{\operatorname{diag}}
%\DeclareMathOperator{\erf}{\operatorname{erf}}
%\newcommand{\ee}{\ensuremath{\mathrm e}}
%\newcommand{\w}{\ensuremath{\omega}}
%\newcommand{\vp}{\ensuremath{\varphi}}
%\newcommand{\ve}{\ensuremath{\varepsilon}}
%\newcommand{\el}{\ensuremath{\ell} }
%\newcommand{\CCC}{\ensuremath{\mathbb{C}}}
%\newcommand{\EEE}{\ensuremath{\mathbb{E}}}
%\newcommand{\FFF}{\ensuremath{\mathbb{F}}}
%\newcommand{\QQQ}{\ensuremath{\mathbb{Q}}}
%\newcommand{\RRR}{\ensuremath{\mathbb{R}}}
%\newcommand{\NNN}{\ensuremath{\mathbb{N}}}
%\newcommand{\ZZZ}{\ensuremath{\mathbb{Z}}}
%\renewcommand{\iff}{\ensuremath{\Leftrightarrow}}
%\renewcommand{\bar}[1]{\ensuremath{\overline{#1}}}
%\newcommand{\1}[1]{\mathds{1}\left[#1\right]}
%
%\newcommand{\bigoh}{\operatorname{O}}
%\newcommand{\bigtheta}{\mathop{\Theta}}
%\newcommand{\bigomega}{\mathop{\Omega}}
%\newcommand{\littleoh}{\operatorname{o}}
%
