\section{Preliminaries}\label{sec:prelims}


\subsection{Expanders and Codes}



%\begin{definition}[$(\beta,\gamma)$-expander]
%	Let $\beta,\gamma\in (0,1)$. A $d$-regular bipartite graph $G =(L,R,E)$ with $|L|=|R|=n$ is a $(\beta,\gamma)$-expander if the following is true for every $\alpha \in (\gamma,1)$ and for every $S\sub L$ with $|S|\geq \beta\cdot n$: if for every vertex in $S$, $\alpha d$ edges among its neighborhood are colored red, then at least $(\alpha-\gamma)n$ vertices in $R$ have one or more red edges incident on it.
%\end{definition}
%
%\begin{proposition}
%	The complete bipartite graph $K_{n,n}$ is a $(\frac{1}{n},0)$-expander.
%\end{proposition}
%
%\begin{proposition}
%	The bipartite spectral expander with second largest normalized singular value $\lambda$ is a $(\beta,\gamma)$-expander if $\lambda \leq \beta \gamma$.
%\end{proposition}


For a bipartite graph $G=(L,R,E)$, let $L$ be the set of left vertices, and $R$ be the set of right vertices. Let $A_G$ denote the $L\times R$ biadjacency matrix, and $\sigma_2(A_G)$ be its second largest singular value.
 
 
\begin{definition}[$(n,d,\lambda)$-expander]
A $d$-regular bipartite graph $G(L,R,E)$ with $|L|=|R|=n$ is said to be an $(n,d,\lambda)$--expander if $\sigma_2(A_G) ~\leq~ \lambda \cdot d$.
\end{definition}

Infinite families of $(n,d,\lambda)$--expanders, with growing $n$ as $d$ and $\lambda$ are constant, can be derived based on double covers of Ramanujan graphs of~\cite{LPS88} as long as $\lambda \geq \frac{2\sqrt{d-1}}{d}$.

\begin{lemma}[Expander Mixing Lemma]\label{lem:eml}
	Given an $(n,d,\lambda)$-expander $G=(L,R,E)$ and functions $f: L \rightarrow \R$ and $g: R \rightarrow \R$, the following well--known property is a simple consequence of definition of $(n,d,\lambda)$--expanders:
	\[ \Big\vert{\Ex{(\li,\ri) \sim E}{ f(\li) \cdot g(\ri)} - \Ex{\li\sim L}{f(\li)} \cdot \Ex{\ri\sim R}{g(\ri)}}
                \Big\vert ~\leq~ \lambda \cdot \norm{f}_2\norm{g}_2 \mper
	\]
	When $f = \mathbb{1}_S,g = \mathbb{1}_T$ are indicators of sets, denote by $E(S,T)$ the number of edges between $S$ and $T$.  Then,
	\[ \Big\vert{ {E(S,T)} -\frac{d\cdot |S||T|}{n}}
                \Big\vert ~\leq~ \lambda \cdot d\cdot  \sqrt{|S||T|} ~\leq~ \lambda \cdot d\cdot n \mper
	\]	
\end{lemma}

\paragraph{Notation and Edge orderings.} 
For a bipartite graph $G=(L,R,E)$, we index the left set by $\li$ and the right set by $\ri$. For a vertex $\li \in L$, 
we denote the set of edges incident to it by $N(\li)$ (left neighborhood), and the set of edges
incident to  $\ri\in R$ is denoted by $N(\ri)$(right neighborhood). 
%
We use $\li \sim \ri$ to denote that the vertex $\li \in L$ is adjacent to the vertex $\ri \in R$, that is, $(\li,\ri)\in E$. 

Fix an arbitrary ordering of the edges. Then there are bijections between the sets $E$, $L \times
[d]$, and $R \times [d]$, 
given by taking $(\li,i)$ to be the $i^{th}$ edge incident on $\li$, and similarly for $R \times [d]$.
Henceforth, we will implicitly assume such an ordering of the edges is fixed, and use the resulting bijections.


\paragraph{Codes.} Now we define the most basic notions associated with a code. 
% We will only work with linear codes but give the general definition here.
\begin{definition}[Fractional Hamming Distance]
	Let $\Sigma$ be a finite alphabet and let $f,g\in \Sigma^n$. Then the (fractional) distance
        between $f,g$ is defined as \[ \dis(f,g) = \Ex{i\in [n]}{ \indi{f_i \neq g_i}} \mper \]
\end{definition}

\begin{definition}[Code, distance and rate]
	A code $\calC$ of block length $n$, distance $\delta$ and rate $\rho$ over the alphabet size $\Sigma$ is a set $\calC \subseteq \Sigma^n$ such that,
	\[
	\rho = \frac{\log_{|\Sigma|} |\calC|}{n} \;\;\text{and }\; \delta = \min_{h_1,h_2\in \calC \colon h_1 \ne h_2} \dis(h_1,h_2).
	\]	
%	\begin{enumerate}[(i)]
%		\item $\rho = \frac{\log_{|\Sigma|} |\calC|}{n}$
%		\item $\delta = \min_{h_1,h_2\in \calC} \dis(h_1,h_2)$
%	\end{enumerate}
	Such codes are succinctly represented as $[n,\rho,\delta]_\Sigma$.
	We say $\calC$ is a \emph{linear code} if $\Sigma$ can be identified with a finite field $\F_q$, and $\calC$ is a linear subspace of $\F_q^n$.
\end{definition}

% Sometimes, we will view $\calC$ as a bijection $\calC: [|\calC|] \rightarrow [q]^n$, or $\calC: [q^{\rho \cdot n}] \rightarrow [q]^n$.

%\snote{Define list of codewords $\calL(g,\delta)$ here.}
%\tnote{Should we?}

\subsection{Alon-Edmonds-Luby distance amplification}\label{sec:AEL_prelims}


Alon, Bruck, Naor, Naor and Roth \cite{ABNNR92} introduced a graph-based distance amplification scheme which was generalized by Alon, Edmonds and Luby \cite{AEL95}, and used by Guruswami and Indyk \cite{GI05} to design linear-time unique decodable near-MDS codes.

The scheme is a three-step process involving an outer code  ($\calC_\out$), an inner code ($\calC_\inn$), and a bipartite expander $G$: (i) concatenate the outer code $\cC_\out$ with inner code $\cC_\inn$ (ii) shuffle the symbols of concatenated code via edges on a $d$-regular bipartite expander graph $G$, and (iii) collect $d$-symbols on the right vertices and fold them back to produce the final code, $\AELC$. We now formally define this procedure.

%The graph $G$ is chosen such that the size of $L$ and $R$ match the blocklength of the outer code, and the degree matches the blocklength of the inner code. 
%
%Let $\cC_\inn \subseteq \F_q^d$ be a code of blocklength $d$ and rate $r$, $\cC_\out \subseteq \F_{q^r}^n$ be a code of blocklength $n$, and $G = (n,d, \lambda)$- biprartite expander.
%
%
%\begin{definition}[Concatenated codes]
%Let $\calC_{\out}$ be an $[n,\delta_{\out},r_{\out}]_{\Sigma_\out}$ code and let $\calC_{\inn}$ be a $[d,\delta_{\inn},r_{\inn}]_{\Sigma_\inn}$ code with $\abs{\Sigma_\out} = |\calC_\inn|$. 	Fix an $(n,d,\lambda)$-expander $G(L,R,E)$. 
%	
%	We define the concatenation of $f\in \Sigma_\out^L$ with $\phi: \Sigma_\out\rightarrow \calC_\inn\subseteq \Sigma_\inn^d$ as $f^*: E \rightarrow \Sigma_\inn$, defined as
%	\[
%		f^{*}_{\calC_{0}}(e) ~=~ \phi(f(\li))(j),
%	\]
%	where $\li$ is the left endpoint of edge $e$ and $e$ is the $j^{th}$ edge incident on $\li$.
%	The concatenated code $\calC^{*}_{\calC_{0}}(\calC_1)$ of block length $n\cdot d$ and alphabet $[q_0]$ is defined to be,
%	\[
%		\calC^{*}_{\calC_{0}}(\calC_{1}) ~=~ \braces[\big]{\; f^{*}_{\calC_{0}} \mid f \in \calC_{1}\,} .
%	\]
%	When clear from context, we will omit $\calC_{0},\calC_1$ in the above notation to call the concatenated code $\calC^*$.
%\end{definition}

%\begin{claim}
%	$\dis(f^{*}_{\calC_{0}},g^{*}_{\calC_{0}}) ~\geq~ \delta_{0}\cdot \dis(f,g)$, which also
%        implies $\Delta(\calC^{*}_{\calC_{0}}(\calC_{1})) \geq \delta_{0} \cdot \delta_{1}$.
%\end{claim}


% \begin{corollary}
% 		The distance of the concatenated code $\calC^{*}_{\calC_{0}}(\calC_{1})$ is at least $ \delta_{0} \cdot \delta_{1}$.
% \end{corollary}

Fix an $(n,d,\lambda)$-expander $G = (L,R,E)$. Let $\calC_{\out}$ be an $[n,r_{\out},\delta_{\out}]_{\Sigma_\out}$ code and let $\calC_{\inn}$ be a $[d,r_{\inn},\delta_{\inn}]_{\Sigma_\inn}$ code with $\abs{\Sigma_\out} = |\calC_\inn|$, and a bijection $\phi: \Sigma_\out\rightarrow \calC_\inn\subseteq \Sigma_\inn^d$,  realizing this.  
%\fnote{The convention $[d,r_{\inn}, \delta_{\inn}]_{\Sigma_\inn}$ with dimension/rate before distance seems more common. Not sure if it is worth changing.}\tnote{agreed and done.}

A codeword $f$ of $\AELC$ technically belongs to the space $(\Sigma_\inn^d)^R$ but by our choice of parameters, $(\Sigma_\inn^d)^R$ is in bijection with the set $\Sigma_\inn^E$ and one can just \emph{fold} or \emph{unfold} the symbols to move from one space to the other. Choosing $f$ to be in $\Sigma_\inn^E$ allows us to talk about $f$ viewed from left vertex set $L$ or right vertex set $R$. Formally, if $f_\ell \defeq f\restrict{N(\ell)} \in \cC_\inn$, we have $ \inv{\varphi}(f_\ell) \in \Sigma_\out$. 

%We refer to  which is the view of $f$ from left after converting the symbols.

%For $f: L \to \Sigma_\out$, define $f^*: E\to \Sigma_\inn$ such that $f^*\restrict{N(\ell)} := \phi(f(\ell))$
%
% 
%
%%we have ${\Sigma_\out}^L \cong \cC_\inn^L \cong {\Sigma_\inn}^E \cong ({\Sigma_\inn^d})^R$, 
%
%
%$f \in \calC_{\out} \subseteq {\Sigma_\out}^L$
%define $f_\ell:= f\restrict{N(\ell)}$.  


\begin{definition}[AEL Codes]

%
Given inner code $\cC_\inn$ an outer code $\cC_\out$, a map $\varphi$, and a graph $G$ as above, the AEL code is defined as,
\[
\AELC ~=~ \braces[\big]{f: R\to\Sigma_\inn^d \mid \forall \ell \in L, f_\ell \in \cC_\inn,\;\text{and } \parens[\big]{\inv{\varphi}(f_\ell)}_\ell \in \cC_\out }.
\]
%	\[
%		f^{\AEL} (\ri) ~=~ \left( f^{*}_{\calC_{0}}(e_1),f^{*}_{\calC_{0}}(e_2),\cdots,f^{*}_{\calC_{0}}(e_d) \right)
%	\]
%	where $e_1,e_2,\cdots,e_d$ are the $d$ edges incident on $\ri$.
%	The AEL code $\calC^{\AEL}_{\calC_{0}}(\calC_{1}) \subseteq [q_0^d]^{n}$ is defined as 
%	\[
%		\calC^{\AEL}_{\calC_{0}}(\calC_{1}) ~=~ \{ f^{\AEL}_{\calC_{0}} \mid f \in \calC_{1}\}
%	\]
%	When clear from context, we will omit $\calC_{0},\calC_1$ in the above notation to call the AEL code $\AELC$.
\end{definition}
%

One can also think of the code procedurally as, starting with $f^* \in \cC_\out$, and defining $f : E\to \Sigma_\inn$ as the function such that $f_\ell := \varphi(f^*(\ell))$. This function $f$ can then be folded on the right vertices to obtain an AEL codeword. 

%A pictorial description of this process is as follows:
%
%\begin{center}	
%\begin{tikzpicture} 
%\begin{scope}
%\draw[thick] (0,0) ellipse (1.5cm and 3cm);
%\node[below] at (0,-3.1) {$L$};
%\draw[thick] (5,0) ellipse (1.5cm and 3cm);
%\node[below] at (5,-3.1) {$R$};
%
%%\rotatebox{90}{$\,=$}\cC_\inn \ni \varphi(f^*(\ell))=  
%
%\node[fill,circle] (a1) at (0,2) {};
%\node[fill,circle] (a2) at (0,1) {};
%\node[left] at (-0.29,1) {$ \parens[\big]{f(e_1),f(e_1), f(e_3)} = f_{\ell}  $};
%\node[left] at (-1.9,0.55) {$\Large \rotatebox{90}{\,=} $};
%\node[left] at (-1.2,0.2) {$\cC_\inn \ni \varphi(f^*(\ell))  $};
%\node[fill,circle] (a3) at (0,0) {};
%\node[fill,circle] (a4) at (0,-1) {};
%\node[fill,circle] (a5) at (0,-2) {};
%%label={$(f_{\ell}(e_2),f_{\ell'}(e_3), f_{\ell''}(e_2))$}
%\node[fill,circle] (b1) at (5,2) {};
%\node[fill,circle] (b2) at (5,1) {};
%\node[fill,circle] (b3) at (5,0) {};
%\node[below] at (8,0.45) {{$f_r = \parens[\Big]{f(e_2),f(e_4), f(e_5)} \in \Sigma_\inn^d$}};
%\node[fill,circle] (b4) at (5,-1) {};
%\node[fill,circle] (b5) at (5,-2) {};
%
%\draw[thick](a2)--node[pos=0.4,sloped, above] {$f(e_1)$}(b1);
%\draw[thick](a2)--node[pos=0.4,sloped, above] {$f(e_2)$}(b3);
%\draw[thick](a2)--node[pos=0.4,sloped, above] {$f(e_3)$}(b5);
%\draw[thick] (a4)--node[pos=0.2,sloped, above] {$f(e_4)$}(b3);
%\draw[thick] (a5)--node[pos=0.2,sloped, above] {$f(e_5)$}(b3);
%
%\node[below] at (2.8,-3.8) {Illustration of the AEL procedure};
%\end{scope}
%\end{tikzpicture}
%\end{center}

%\fnote{Is this picture consistent with the previous notation?}\tnote{Yes as i just changed notation. I will fix it}

% By choosing $\delta_{0}$ large and $\lambda$ small enough, we can amplify the distance between strings via this encoding.

% \begin{corollary}
% 	The distance of AEL code $\calC^{\AEL}_{\calC_{0}}(\calC_{1})$ is at least $\delta_{0} - \frac{\lambda}{\delta_{1}}$.
% \end{corollary}

\paragraph{Distance metrics for AEL Codes.}
Let $f,g \in \AELC$. As we have seen, one can view these as functions on $L$, or on $R$. Associated with each of these viewpoints, we can define the following two distance functions:
\begin{align*}
	\dis_L(f,g) &\defeq \Ex{\li\in L}{ \indi{ f_{\ell} \neq g_{\ri}}} \\
%	\dis(f,g) &\defeq \Ex{e\in E}{\indi{ f_e \neq g_e}} = \Ex{\li \in L}{\dis(f_{N(\li)},g_{N(\li)})} = \Ex{\ri \in R}{\dis(f_{N(\ri)},g_{N(\ri)})}\\
	\dis_R(f,g) &\defeq \Ex{\ri\in R}{ \indi{ f_{\ri} \neq g_{\ri}}} .
\end{align*}

Note that for the AEL code, the distance metric is the right distance $\Delta_R(\cdot, \cdot)$. Alon, Edmonds, and Luby, proved the following result, which shows that the construction can be used to amplify the distance to $\delta_0$ by choosing $\lambda$ sufficiently small.
%
\begin{theorem}[\cite{AEL95}]\label{thm:ael_distance}
	Let $f,g \in \AELC$ be distinct. Then, $\dis_R(f,g)~\geq~ \delta_{\inn}-
        \frac{\lambda}{\dis_L(f,g)} \geq~ \delta_{\inn} - \frac{\lambda}{\delta_{\out}} $.
%        , which also implies $\Delta(\calC^{\AEL})        ~\geq~ \delta_{\inn} - \frac{\lambda}{\delta_{\out}}$.
\end{theorem}
We wait to give a proof as we will prove a much more general version of this claim in the next section. For the rest of the paper, the alphabet of the AEL code, $\Sigma_\inn^d$ will be denoted simply as $\Sigma$.



