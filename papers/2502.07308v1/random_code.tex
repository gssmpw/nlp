\section{Inner Codes meeting generalized Singleton Bound}\label{sec:inner-code}

In this section, we will look at two constructions of inner codes that are average-radius list decodable with erasures, the property we need to instantiate our construction. These will be a random linear code, and folded Reed-Solomon codes.

In the literature, the property of being average-radius list decodable is usually defined without reference to any erasures (\ie $s=0$ in \cref{def:gen_singleton}). Formally, we say that $\calC \subseteq \Sigma^n$ is $(\delta,k,\eps)$ {\deffont average-radius list decodable} if for all $g \in \Sigma^n$ and all $\calH \subseteq \calC$ with $\abs{\calH} \leq k$, we have
%
\[
\sum_{h \in \calH}{\Delta(g,h)} ~\geq~ \inparen{\abs{\calH}-1} \cdot (\delta - \eps) \mper
\]
%
The other way of looking at erasures is via puncturings, and we will now see that if a code and its puncturings are average-radius list decodable in the above sense, then it is average-radius list decodable with erasures.  


\vspace{-5 pt}
\paragraph{Erasures and Puncturings.}
Let $C \subseteq \F_q^n$ be a code, $S\subseteq [n]$, and denote $s:= \nfrac{|S|}{n}$. Let $\rho =
\rho(C)$ denote the rate of $C$. Define $C_S  \subseteq \F_q^{(1-s)n}$ as the punctured code obtained by removing the coordinates in $S$. 
\begin{claim}\label{claim:puncture}
If for each $S\subseteq [n]$ with  $s \leq 1-\rho$, $C_S$  is $(1-\rho,L, \frac{\varepsilon}{1-s} )$ average-radius list decodable, then $C$ is  $(1-\rho, L, \eps )$ average-radius list decodable with erasures.	
\end{claim}
\begin{proof}
	
Let $\erase{g}$ have erasures in a set $S\subseteq [n]$, and denote by $g_S$, the punctured vectored with these erasure locations removed. Similarly for any list of $L$ codewords $\calH \subseteq C$, denote the punctured list by $\calH_S$. If the code $C_S$ is $(L, \frac{\eps}{1-s} )$ average-radius list decodable (without erasures), then, 

\[
\sum_{h_S \in \calH_S} \Delta(g_S,h_S) ~\geq~ (L - 1) \cdot \parens[\Big]{1 - \rho(C_S) - \frac{\eps}{1-s} }\mper
\]
Observe that if $C$ is a rate $\rho$ code, then $\rho(C_S) \leq \frac{\rho}{1-s}$ (in fact,
$\rho(C_S) = \rho$ if the distance of $C_S$ is greater than 0, but we only need the one-sided inequality). Also $\Delta(\erase{g},h) \geq \Delta(g_S,h_S) \cdot (1-s)$. Multiplying the entire equation by $(1-s)$ and plugging this in, we get,
\[
\sum_{h_S \in \calH_S} \Delta(\erase{g},h) ~\geq~ (L - 1) \cdot \parens[\Big]{1-s - \rho - \eps }\mper \qedhere
\]% which is the requirement in \cref{def:gen_singleton}. 
%Observe that $\Delta(C_S) \geq \frac{\Delta-s}{1-s}$, and $\Delta_S(\erase{g},h) = \Delta_S(g_S,h_S) \cdot (1-s)$. Plugging this in the above equation yields the requirement in \cref{def:gen_singleton}.  
\end{proof}

%. Note that a code $C$ is average-radius list decodable with erasures (\cref{def:gen_singleton}) if $C_S$ is $(L,\eta(1-|S|/n))$\tnote{check this $\eta$ normalization thing} average-radius list decodable (without any erasures) for all subsets of size at most $\Delta n$, the set of such $S$ captures the possible set of erasures. 

\vspace{-5 pt}
\paragraph{Random Linear Codes.} Let $\cG_{n,\rho n,q}$ be the uniform distribution over $n\times \rho n$-matrices over $\F_q$, \ie where each entry of the matrix is picked uniformly at random from $\F_q$. Let $C = \im(G)$ where $G\sim \cG_{n,\rho n,q}$. Then, for any fixed $S\subseteq [n]$,  $C_S = \im(G_S) $ where $G_S\sim \cG_{n-|S|,\rho n,q}$. 

\begin{theorem}[{\cite[Thm. 1.3]{AGL24}}]
Fix an integer $L \geq 1$, $q \geq 2\cdot 2^{10L/\epsilon}$, and $\rho,\epsilon \in (0,1)$.
Then for sufficiently large $n$, a random linear code $C = \im(G)$ where $G\sim \cG_{n,\,\rho n, q}$, is $(1-\rho, L,\epsilon)$ average-radius list decodable with probablity at least $1- \kappa$ , where, 
\[\kappa ~=~  \parens[\Big]{\frac{c_{L,\epsilon}}{q}}^{\lfloor \frac{\epsilon n}{2} \rfloor}, \;\text{ for } c_{L,\epsilon} < 2\cdot 2^{10L/\epsilon}  .\] 
%\[\tau ~\leq~ 2+^{(L+2)n} \cdot {n \choose r} \cdot2^{(L+1)r}\cdot \parens[\bigg]{\frac{L}{q}}^r, \;\text{ for } r = \Big\lfloor \frac{\epsilon n}{2} \Big\rfloor .\] 
\end{theorem}
\begin{proof}
	Their definition of average-radius list decodable gives an inequality for exactly $\ell$ distinct codewords. Thus, we obtain this bound by taking a union over their bounds for $\ell = 1, \cdots, L$. 
\end{proof}


\begin{corollary}\label{cor:random-code}
Let $C$ be a random linear code as generated above for $q \geq 2^{\nfrac{(10L+2)}{\epsilon}}$. Then,
with probability at least $(1-2^{-n/3})$, $C$ is $(1-\rho,L,\eps)$ average-radius list decodable with erasures.\end{corollary}
%\begin{corollary}
%Let $C$ be a random linear code as generated above for $q \gg 2^{10L/\epsilon}\cdot 2^{\frac{3(1-\gamma)}{\epsilon}}$ where $\gamma \in (0,1)$. Then, with high probability, for all $S \subseteq [n]$ of size at most $|S| \leq \gamma n$, the punctured code $C_S$ is $ (L, \epsilon)$ average-radius list decodable. 
%\end{corollary}
\begin{proof}
We will show that with high probability, for all $S \subseteq [n]$ of size at most $|S| \leq (1-\rho)\cdot n$, the punctured code $C_S$ is $ (1-\rho(C_S), L, \epsilon)$ average-radius list decodable. 

For a fixed $S$ of fractional size  $s \leq 1-\rho$, we have that $C_S$ is not
$(1-\rho(C_S),L,\frac{\epsilon}{1-s})$ average-radius list-decodable with probablity at most, 
%
\[
\leq \parens[\Big]{\frac{c_{L,\frac{\epsilon}{1-s}}}{q}}^{\Big\lfloor \frac{ \frac{\epsilon}{1-s}
      (1- s) n}{2}\Big\rfloor} \leq \parens[\Big]{\frac{c_{L,\eps}}{q}}^{\lfloor\frac{ \eps
      n}{2}\rfloor} = \kappa .
\] 
There are at most $2^n$ choices of the subsets $S$, and thus by a union bound, the probability that
all the punctured codes are average-radius list-decodable is at least 
\[
1 - 2^{n} \kappa ~\geq~ 1 - 2^n \cdot 2^{(10L/\eps + 1) \cdot \eps n/2} \cdot q^{-\lfloor\eps
  n/2\rfloor } ~\geq~ 1 - 2^{-n/3} \mcom
\]
for $q \geq 2^{2/\epsilon}\cdot 2^{10L/\epsilon}$ and $n \geq 60L+12$. 	
\end{proof}
%	 where, 
%\[\tau ~\leq~ 2^{(L+2)n} \cdot {n \choose r} \cdot2^{(L+1)r}\cdot \parens[\bigg]{\frac{L}{q}}^r, \;\text{ for } r = \Big\lfloor \frac{\epsilon n}{2} \Big\rfloor .\] 


%Plugging in $\gamma = \Delta$, we get that 

Thus, for a large enough alphabet size, random linear codes are average-radius list-decodable with erasures.


\vspace{-5 pt}
\paragraph{Folded Reed--Solomon Codes.} A recent work of Chen and Zhang~\cite{CZ24} shows that explicit folded Reed--Solomon (RS) codes are also average-radius list-decodable. Let $\F_q[x]$ denote the set of polynomials with $\F_q$-coefficients, and $\F_q^*$ be the multiplicative cyclic group of non-zero elements. 

\begin{definition}[Folded RS Codes]
Fix $n,b >0$, $\rho \in (0,1)$, and $q \geq bn$. Let $\gamma$ be a generator of $\F_q^*$, and pick $\vec{\alpha} = (\alpha_1,\cdots, \alpha_n) \in \F_q^n$. For $f \in \F_q[x]$, let $\Gamma_i = (f(\alpha_i), f(\gamma\alpha_i), \cdots, f(\gamma^{b-1}\alpha_i))$
Then,
\[
\mathrm{FRS}^{b,\gamma}_{n,\rho}(\vec{\alpha}) = \braces{(\Gamma_1, \cdots, \Gamma_n) \mid \deg(f) < \rho b n} \subseteq (\F_q^{b})^n.
\]	
The code is called \textit{appropriate} if $\{ \gamma^i\alpha_j \mid 0\leq i \leq b-1, j \in [n] \}$ has size $bn$, \ie all values are distinct.
\end{definition}

\begin{theorem}[{\cite[Thm. 1.3]{CZ24}}]
For any integer $L \geq 1$ and $\epsilon \in (0,1)$, and $b \geq L/\epsilon$. Then, an appropriate folded Reed-Solomon code, $\mathrm{FRS}^{b,\gamma}_{n,\rho}(\vec{\alpha})$, is $(1- \rho,L,\varepsilon)$ average-radius list decodable where $\rho$ is the rate of this folded code.
 \end{theorem}
% $1- \frac{s\rho}{s-L+1}
%\tnote{Is plugging in $s = L/\epsilon$ okay?}

Note that puncturing the folded Reed--Solomon code is equivalent to the FRS code over a puncturing of $\vec{\alpha}$. Clearly, a puncturing of an appropriate folded Reed--Solomon code is also an appropriate Reed--Solomon code, and thus using \cref{claim:puncture}, one obtains:  

\begin{corollary}
	For any integer $L \geq 1$ and $\epsilon \in (0,1)$, set $b = L/\epsilon$. Then, an appropriate folded Reed-Solomon code, $\mathrm{FRS}^{s,\gamma}_{n,k}(\vec{\alpha})$, is $(1- \rho,L,\varepsilon)$ average-radius list decodable with erasures.
\end{corollary}

%\tnote{Should I change L to $k$? Not doing that as the papers we cite use $k$ for something else and it would be confusing when someone go reads the reference.}



%For a Reed--Solomon code, puncturing is the same as shrinking the evaluation set. Since the notion of \enquote{appropriate evaluation points} is closed under taking subsets, we immediately obtain that puncturings of the Folded Reed--Solomon code are also average-radius list-decodable. Therefore, by \cref{claim:puncture} 
%For a Reed--Solomon code, puncturing is the same as shrinking the evaluation set. Since the notion of \enquote{appropriate evaluation points} is closed under taking subsets, we immediately obtain that puncturings of the Folded Reed--Solomon code are also average-radius list-decodable.  

%!TEX root=main.tex

%%% Local Variables:
%%% mode: latex
%%% TeX-master: "main"
%%% End:
