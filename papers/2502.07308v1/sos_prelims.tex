
The sum-of-squares hierarchy of semidefinite programs (SDPs) provides a family of increasingly
powerful convex relaxations for several optimization problems. 
%
Each ``level" $t$ of the hierarchy is parametrized by a set of constraints corresponding to
polynomials of degree at most $t$ in the optimization variables. While the relaxations in the
hierarchy can be viewed as  semidefinite programs of size $n^{O(t)}$ \cite{BS14, FKP19}, 
it is often convenient to view the solution as a linear operator, called the ``pseudoexpectation" operator.
%
% It is well-known that such constrained pseudoexpectation operators of SoS-degree $t$ can be described as solutions to semidefinite programs of size $n^{O(t)}$ \cite{BS14, Laurent09}. This hierarchy of semidefinite programs for increasing $t$ is known as the SoS hierarchy.


%
%
%\fnote{Someone familiar with SoS will likely want to skip most of this paragraph (only use it as a reference as needed) and jump to the AEL part of the preliminaries.}
%\vspace{-5 pt}
%
\paragraph{Pseudoexpectations}
 
%
Let $t$ be a positive even integer and fix an alphabet $\Sigma$ of size $s$. Let $\zee = \{Z_{i,j}\}_{i\in[m],j\in\Sigma}$ be a collection of variables and $\R[\zee]^{\leq t}$ be the vector space of polynomials of degree at most $t$ in the variables $\zee$ (including the constants).


\begin{definition}[Constrained Pseudoexpectations]\label{def:constraints_on_sos}
%	 
Let $\calS = \inbraces{f_1 = 0, \ldots, f_m = 0, g_1 \geq 0, \ldots, g_r \geq 0}$ be a system of
polynomial constraints in $\zee$, with each polynomial in $\calS$ of degree at most $t$. We say $\tildeEx{\cdot}$ is a pseudoexpectation operator of SoS-degree $t$, over the variables $\zee$  respecting $\calS$, if it is a linear operator $ \tildeEx{\cdot}: \R[\zee]^{\leq t} \rightarrow \R$ such that:
	%
	\begin{enumerate}
	\item $\tildeEx{1} = 1$.
    \item $\tildeEx{p^2} \geq 0$ if $p$ is a polynomial in $\zee = \{Z_{i,j}\}_{i\in [m],j\in \Sigma}$ of degree $\leq t/2$.
	\item $\tildeEx{p \cdot f_i} = 0$,  $\forall\, i \in [m]$ and $\forall\, p$ such that $\deg(p \cdot f_i) \leq t$.
	\item $\tildeEx{p^2 \cdot \prod_{i \in S} g_i} \geq 0$, $\forall\, S \subseteq [r]$ and $\forall\, p$ such that $\deg(p^2\cdot \prod_{i \in S} g_i) \leq t$.
	\end{enumerate}
\end{definition}
~


%\tnote{Suggestion -- directly define the constrained version. Attempting to make the connection with assignments a bit more explicit below.}

% An SoS solution of degree $t$, or a pseudoexpectation of SoS-degree $t$, over the variables $\zee$ is represented by a linear operator $ \tildeEx{\cdot}: \R[\zee]^{\leq t} \rightarrow \R$ such that:
%%
%\vspace{-5 pt}
%%
%\begin{enumerate}[(i)]
%    \item $\tildeEx{1} = 1$.
%    \item $\tildeEx{p^2} \geq 0$ if $p$ is a polynomial in $\zee = \{Z_{i,j}\}_{i\in [m],j\in [q]}$ of degree $\leq t/2$.
%\end{enumerate}
%%
%\vspace{-5 pt}
%%
%\tnote{Is the note needed? It is reiterating that it is a linear operator}
% Note that linearity implies $\tildeEx{p_1} + \tildeEx{p_2} = \tildeEx{p_1+p_2}$ and $\tildeEx{c\cdot
%  p_1} = c \cdot \tildeEx{p_1}$ for $c\in \R$, for $p_1, p_2 \in \R[\zee]^{\leq t}$.
%%
%This also allows for a succinct representation of $\tildeEx{\cdot}$ using any basis for $\R[\zee]^{\leq t}$.
%
%\tnote{ }

% to $m$ variables in alphabet $[q]$.

Let $\mu$ be a distribution over the set of assignments, $\Sigma^m$.  Define the following collection of random variables,  \[\zee = \braces[\big]{ \, Z_{i,j}  = \indi{ i \mapsto j}\, \mid \, i\in[m],\, j\in\Sigma } .\] Then, setting $\tildeEx{p(\zee)} = \Ex{\mu}{p(\zee)} $ for any polynomial $p(\cdot)$ defines an (unconstrained) pseudoexpectation operator. However, the reverse is not true when $t < m$, and there can be degree-$t$ pseudoexpectations that do not correspond to any genuine distribution, $\mu$. Therefore, the set of all pseudoexpectations should be seen as a relaxation for the set of all possible distributions over such assignments. The upshot of this relaxation is that it is possible to optimize over the set. Under certain conditions on the bit-complexity of solutions~\cite{OD16, RW17:sos}, one can optimize over the set of degree-$t$ pseudoexpectations in time $m^{O(t)}$ via SDPs.

%   setting the true expectation operator under this distribution is also a . 
%
%
%Given an assignment $f: [m] \rightarrow [q]$, the operator $\tildeEx{Z_{i_1,j_1}\cdots Z_{i_k,j_k}} = \indi{ j_t = f(i_t) \; \forall\, t }$
%\[
%f: [m] \rightarrow [q] ~\mapsto~  \tildeEx{Z_{i_1,j_1}\cdots Z_{i_k,j_k}} = \indi{ j_t = f(i_t) \; \forall\, t }
%\]
%can be seen as a pseudoexpectation that assigns the value $1$ to a monomial consistent with $f$ and $0$ otherwise. 

%
%This can be extended via linearity to all
%polynomials, and then by convexity of the constraints to all distributions over assignments.
%


 
%
%We next define what it means for pseudoexpectations to satisfy some problem-specific constraints.
%\[
% \left\{ \substack{\text{ True expectations} \\
%\text{ \ie for an assignment } f: [m] \rightarrow [q] \\\tildeEx{Z_{} \cdots Z_{}} =  } \right\} \subset  \{ \text{ SoS Pseudoexpectations }  \}
%\]

% Indeed, any assignment $f: [m] \rightarrow [q]$ has a corresponding pseudoexpectation described below.

% Let $z^{(f)}_{i,j} = \indi{f(i) = j}$, where $\indi{\cdot}$ is the $0/1$ indicator function. Then the pseudoexpectation corresponding to assignment $f$ is given by
% \begin{align*}
% 	\PExp^{(f)}[{p(\zee)}] = p\inparen{\inbraces{z^f_{i,j}}_{i\in[m],j\in[q]}}
% \end{align*}

% for every polynomial $p$ of degree at most $t$.

% As can be verified easily, the set of pseudoexpectations is convex, and so we can also extend this correspondence to distributions over assignments in a natural way. If $\PExp^{(\calD)}[{\cdot}]$ is the pseudoexpectation corresponding to a distribution $\calD$ over assignments, it holds that
% \[
% 	\PExp^{(\calD)}[{p(\zee)}] = \Ex{f\sim \calD}{p\inparen{\inbraces{z^f_{i,j}}_{i\in[m],j\in[q]}}}
% \]
% which explains the term pseudo-expectation.

% However, pseudoexpectations only exist for low-degree polynomials, and there can be pseudoexpectation operators that do not correspond to any genuine distribution over assignments. The main reason for working with pseudoexpectations instead of genuine distributions is that the set of SoS-degree $t$ can be optimized over in time $m^{O(t)}$ via SDPs.

% As a relaxation, pseudoexpectations are used in efficient algorithm design when coupled with
% suitable rounding algorithms. 
% %
% For such applications (including ours), it is important to look at relaxations that satisfy certain problem-specific constraints. We define next what it means for pseudoexpectations to satisfy constraints.

% \paragraph{Constrained Pseudoexpectations}
% 

%
%
\paragraph{Local constraints and local functions.}
%
Any constraint that involves at most $k$ variables from $\zee$, with $k\leq t$, can be written as a degree-$k$ polynomial, and such constraints may be enforced into the SoS solution.
%
% \paragraph{Canonical usage}
%
In particular, we will always consider the following canonical constraints on the variables $\zee$.
\ifnum\confversion=1
\begin{align*}
&Z_{i,j}^2 = Z_{i,j},\ \forall i\in[m],j\in[s] \\
\text{and} \quad &\sum_j Z_{i,j} = 1,\ \forall i\in[m] \mper
\end{align*}
\else
\[
Z_{i,j}^2 = Z_{i,j},\ \forall \,i\in[m],j\in\Sigma
\quad \text{and} \quad 
\sum_j Z_{i,j} = 1,\ \forall\, i\in[m] \mper
\]
\fi
% As shown in the previous section, an assignment $f:[m]\rightarrow [q]$ is encoded using $m$ characteristic vectors, and so we wish to impose the following local constraints on the variables $\zee$:
%
% \begin{enumerate}[(i)]
% 	\item $Z_{i,j}^2 = Z_{i,j},\ \forall i\in[m],j\in[q]$.
% 	\item $\sum_j Z_{i,j} = 1,\ \forall i\in[m]$.
% \end{enumerate}
%
% These constraints are enforced as described in \cref{def:constraints_on_sos}, and we will henceforth not explicitly mention it. \snote{See Madhur's comment.}
%
We will also consider additional constraints and corresponding polynomials, defined by ``local" functions. For any $f\in \Sigma^m$ and $M\sub [m]$, we use $f_M$ to denote the restriction $f|_M$, and $f_i$ to denote $f_{\{i\}}$ for convenience.
%
\begin{definition}[$k$-local function]
	A function $\mu: \Sigma^m \rightarrow \R$ is called $k$-local if there is a set $M\subseteq [m]$ of size $k$ such that $\mu(f)$ only depends on $\inbraces{f(i)}_{i\in M}$, or equivalently, $\mu(f)$ only depends on $f|_M$.
	
	If $\mu$ is $k$-local, we abuse notation to also use $\mu: \Sigma^M \rightarrow \R$ with $\mu(\alpha) = \mu(f)$ for any $f$ such that $f|_M=\alpha$. It will be clear from the input to the function $\mu$ whether we are using $\mu$ as a function on $\Sigma^m$ or $\Sigma^M$.
\end{definition}

Let $\mu:\Sigma^m\rightarrow \R$ be a $k$-local function that depends on coordinates $M\subseteq [m]$ with $|M|=k$. Then $\mu$ can be written as a degree-$k$ polynomial $P_{\mu}$ in $\zee$:
\[
	P_{\mu}(\zee) = \sum_{\alpha \in \Sigma^M} \parens[\Big]{\mu(\alpha) \cdot\prod_{i\in M} Z_{i,\alpha_i}}
\]

% To see how $p_{\mu}$ is related to the $k$-local function $\mu$, observe that $p_{\mu}\inparen{\zee = \inbraces{z^{(f)}_{i,j}}} = \mu(f)$.

With some abuse of notation, we let $\mu(\zee)$ denote $P_{\mu}(\zee)$. We will use such $k$-local
functions inside $\tildeEx{\cdot}$ freely without worrying about their polynomial
representation. For example, $\tildeEx{ \indi{\zee_{i} \neq j}}$ denotes $\tildeEx{ 1- Z_{i,j}}$. Likewise, sometimes we will say we set $\zee_i = j$ to mean that we set $Z_{i,j} = 1$ and $Z_{i,j'} = 0$ for all $j'\in \Sigma \backslash \{j\}$.

% $\tildeEx{\cdot}$ operator applied to the polynomial corresponding to the $1$-local function $\mu: [q]^m \rightarrow \R$ which is defined as: $\mu(f)$ is $1$ if $f_i = 0$ and $\mu(f) = 0$ otherwise.
% %
The notion of $k$-local functions can also be extended from real-valued functions to vector-valued functions straightforwardly.

\begin{definition}[Vector-valued local functions]
A function $\mu: \Sigma^m \rightarrow \R^N$ is $k$-local if the $N$ real valued functions corresponding to the $N$ coordinates are all $k$-local. Note that these different coordinate functions may depend on different sets of variables, as long as the number is at most $k$ for each of the functions.
\end{definition}
%
%
\paragraph{Local distribution view of SoS}

It will be convenient to use a shorthand for the function $\indi{\zee_A = \alpha}$, and we will use $\zee_{A,\alpha}$. Likewise, we use $\zee_{i,j}$ as a shorthand for the function $\indi{\zee_i = j}$. That is, henceforth,
\ifnum\confversion=1
\begin{align*}
	&\tildeEx{\zee_{A,\alpha}} = \tildeEx{\indi{\zee_A = \alpha}} = \tildeEx{ \prod_{a\in A}
                             Z_{a,\alpha_a}}
                             \\
	\text{and } \quad & \tildeEx{\zee_{i,j}} = \tildeEx{\indi{\zee_i = j}} = \tildeEx{ Z_{i,j}}.
\end{align*}
\else
\begin{align*}
	\tildeEx{\zee_{A,\alpha}} ~=~ \tildeEx{\indi{\zee_A = \alpha}} ~=~ \tildeE\brackets[\Big]{\prod_{a\in A}
                      Z_{a,\alpha_a}
                                    }
\qquad \text{and} \qquad
	\tildeEx{\zee_{i,j}} ~=~ \tildeEx{\indi{\zee_i = j}} = \tildeEx{ Z_{i,j}}
\end{align*}
\fi

% Note that for any degree-$t$ pseudoexpectation operator $\tildeEx{\cdot}$ with $t\geq 2$,
% \[
% 	\sum_{j\in [q]} \tildeEx{\zee_{i,j}} = \sum_{j} \tildeEx{Z_{i,j}} = \tildeEx{\sum_j Z_{i,j}}= 1
% \qquad \text{and} \qquad
% 	\tildeEx{\zee_{i,j}} = \tildeEx{Z_{i,j}} = \tildeEx{Z_{i,j}^2} \geq 0
% \]
% Thus, the real values $\inbraces{\tildeEx{\zee_{i,j}}}_{j\in [q]}$ define a distribution over $[q]$
% , which we will sometimes call local distribution for $\zee_i$. 

% In fact, this argument can be extended to define local distributions for $\zee_S$ for $|S|\leq
% t/2$. Let $S \subseteq [m]$ be such that $|S|=k\leq t/2$,
Note that for any $A \subseteq [m]$ with $\abs*{A} = k \leq t/2$,
\ifnum\confversion=1
\begin{gather*}
	\sum_{ \alpha \in \Sigma^{k}} \tildeEx{\zee_{A,\alpha}} = 
% \sum_{ \alpha \in [q]^{k}} \tildeEx{ \prod_{s\in S} Z_{s,\alpha_s}} 
 \tildeEx{ \prod_{a\in A} \inparen{ \sum_{j\in \Sigma} Z_{a,j}} } = 1
\\
	\tildeEx{\zee_{A,\alpha}} = \tildeEx{ \prod_{a\in A} Z_{a,\alpha_a}} = \tildeEx{ \prod_{a\in
            A} Z^2_{a,\alpha_a}} \geq 0 \mper
\end{gather*}
\else
\[
	\sum_{ \alpha \in \Sigma^{k}} \tildeEx{\zee_{A,\alpha}} = 
% \sum_{ \alpha \in [q]^{k}} \tildeEx{ \prod_{s\in S} Z_{s,\alpha_s}} 
 \tildeE\brackets[\bigg]{ \prod_{a\in A} \parens[\bigg]{\sum_{j\in \Sigma} Z_{a,j} } } = 1
\qquad \text{and} \qquad
	\tildeEx{\zee_{A,\alpha}} = \tildeE\brackets[\bigg]{ \prod_{a\in A} Z_{a,\alpha_a}} = \tildeE\brackets[\bigg]{ \prod_{a\in
            A} Z^2_{a,\alpha_a}} \geq 0 \mper
\]
\fi

Thus, the values $\inbraces{\tildeEx{\zee_{A, \alpha}}}_{\alpha\in \Sigma^A}$ define a distribution
over $\Sigma^k$. We call this the local distribution for $\zee_A$, or simply for $A$.
% which we think of as the local distribution for $\zee_S$.
%
% Given a distribution $\calD$ over assignments in $[q]^m$, the local distribution induced by $\PExp^{(\calD)}[\cdot]$ on $\zee_S$ is precisely the marginal distribution induced by $\calD$ for the set $S$. In this view, degree-$t$ pseudoexpectations give us consistent marginal distributions over sets of size at most $t/2$ that may not correspond to any global distribution, and allow us to optimize over this set in time $m^{\calO(t)}$.
%
Let $\mu: \Sigma^m \rightarrow\R$ be a $k$-local function for $k\leq t/2$, depending on $M \subseteq
[m]$. Then,
%
\ifnum\confversion=1
\begin{align*}
	\tildeEx{\mu(\zee)} 
~=~& \tildeEx{\sum_{\alpha\in \Sigma^M} \inparen{\mu(\alpha) \cdot\prod_{i\in M} Z_{i,\alpha_i}}}\\
~=~& \sum_{\alpha\in \Sigma^M} \mu(\alpha) \cdot \tildeEx{\prod_{i\in M} Z_{i,\alpha_i}}\\
~=~& \sum_{\alpha\in \Sigma^M} \mu(\alpha) \cdot \tildeEx{\zee_{M,\alpha}}
\end{align*}
\else
\begin{align*}
	\tildeEx{\mu(\zee)} 
~=~ \tildeE\brackets[\bigg]{\sum_{\alpha\in \Sigma^M} \parens[\bigg]{\mu(\alpha) \cdot\prod_{i\in M} Z_{i,\alpha_i}}}
~=~ \sum_{\alpha\in \Sigma^M} \mu(\alpha) \cdot \tildeE\brackets[\Big]{\prod_{i\in M} Z_{i,\alpha_i}}
~=~ \sum_{\alpha\in \Sigma^M} \mu(\alpha) \cdot \tildeEx{\zee_{M,\alpha}}
\end{align*}
\fi
%

That is, $\tildeEx{\mu(\zee)}$ can be seen as the expected value of the function $\mu$ under the local distribution for $M$.

\begin{claim}\label{claim:sos_domination}
	Let $\tildeEx{\cdot}$ be a degree-$t$ pseudoexpectation. For $k \leq t/2$, let $\mu_1,\mu_2$
        be two $k$-local functions on $\Sigma^m$, depending on the same set of coordinates $M$, and
        $\mu_1(\alpha) \leq \mu_2(\alpha) ~~\forall \alpha \in \Sigma^M$. Then $\tildeEx{\mu_1(\zee)} \leq \tildeEx{\mu_2(\zee)}$.
%
 % Suppose that for any $\alpha\in [q]^M$, $\mu_1(\alpha) \leq \mu_2(\alpha)$. Then
 %        \[
 %        	\tildeEx{\mu_1(\zee)} \leq \tildeEx{\mu_2(\zee)}
 %        \]
\end{claim}

\begin{proof}
Let $\calD_M$ be the local distribution induced by $\tildeEx{\cdot}$ for $\zee_M$. Then
$\tildeEx{\mu_1(\zee)} = \Ex{\alpha \sim \calD_M}{\mu_1(\alpha)}$, and $\tildeEx{\mu_2(\zee)} =
\Ex{\alpha\sim \calD_M}{\mu_2(\alpha)}$, which implies $\tildeEx{\mu_1(\zee)} \leq \tildeEx{\mu_2(\zee)}$.
%
% Since $\mu_1(\alpha) \leq \mu_2(\alpha)$ for every $\alpha\in [q]^M$, 
% \[
% 	\Ex{\alpha \sim \calD_M}{\mu_1(\alpha)} \leq \Ex{\alpha\sim \calD_M}{\mu_2(\alpha)}
% \]
% and so,
% \[
% 	\tildeEx{\mu_1(\zee)} \leq \tildeEx{\mu_2(\zee)}
% \]
\end{proof}
%
The previous claim allows us to replace any local function inside $\tildeEx{\cdot}$ by another local function that dominates it. We will make extensive use of this fact.
%
%
\vspace{-5 pt}
%\tnote{Move covariance, conditioning stuff to appendix 1? It seems to disrupt the flow and we only need to cite it in section 6 for Lemma 6.1?}
\paragraph{Covariance for SoS solutions}
Given two sets $S,T \sub [m]$ with $|S|,|T|\leq k/4$, we can define the covariance between indicator random variables of $\zee_S$ and $\zee_T$ taking values $\alpha$ and $\beta$ respectively, according to the local distribution over $S \cup T$. This is formalized in the next definition.
\begin{definition}
Let $\tildeEx{\cdot}$ be a pseudodistribution operator of SoS-degree-$t$, and $S,T$ are two sets of
size at most $t/4$, and $\alpha\in \Sigma^S$, $\beta\in \Sigma^T$, we define the pseudo-covariance and
pseudo-variance,
%
\ifnum\confversion=1
\small
\begin{gather*}
\tildecov(\zee_{S,\alpha},\zee_{T,\beta}) 
= \tildeEx{ \zee_{S,\alpha} \cdot \zee_{T,\beta} } - \tildeEx{\zee_{S,\alpha}} \tildeEx{\zee_{T,\beta}} \\	
\tildeVar{\zee_{S,\alpha}} ~=~ \tildecov(\zee_{S,\alpha},\zee_{S,\alpha})
\end{gather*}
\normalsize
\else
\begin{align*}
\tildecov(\zee_{S,\alpha},\zee_{T,\beta}) 
~&=~ \tildeEx{ \zee_{S,\alpha} \cdot \zee_{T,\beta} } - \tildeEx{\zee_{S,\alpha}} \,\tildeEx{\zee_{T,\beta}}\\ 		
\tildeVar{\zee_{S,\alpha}} ~&=~ \tildecov(\zee_{S,\alpha},\zee_{S,\alpha})
\end{align*}
\fi
%
The above definition is extended to pseudo-covariance and pseudo-variance for pairs of sets $S,T$, 
as the sum of absolute value of pseudo-covariance for all pairs $\alpha,\beta$ :
%
\ifnum\confversion=1
\begin{gather*}
\tildecov(\zee_S,\zee_T) 
~=~ \sum_{\alpha\in \Sigma^S \atop \beta\in \Sigma^T} \abs*{ \tildecov(\zee_{S,\alpha},\zee_{T,\beta}) } \\
\tildeVar{\zee_S} ~=~ \sum_{\alpha\in \Sigma^S} \abs*{ \tildeVar{\zee_{S,\alpha} } }
\end{gather*}
\else
\begin{align*}
\tildecov(\zee_S,\zee_T) 
~&=~ \sum_{\alpha\in \Sigma^S, \beta\in \Sigma^T} \abs*{ \tildecov(\zee_{S,\alpha},\zee_{T,\beta}) }\\[3pt]
\tildeVar{\zee_S} ~&=~ \sum_{\alpha\in \Sigma^S} \abs*{ \tildeVar{\zee_{S,\alpha} } }
\end{align*}
\fi
%
\end{definition}

% These definitions can be extended to pseudocovariances and pseudo-variances for pairs of sets $S,T$ with $|S|,|T|\leq t/4$ as the sum of absolute value of pseudocovariance for all pairs $\alpha,\beta$.
% \begin{definition}
% Let $\tildeEx{\cdot}$ be a pseudodistribution operator of SoS-degree-$t$, and $S,T$ are two sets of size at most $t/4$, we define the pseudo-covariance between $\zee_S$ and $\zee_T$ as,
% 	\begin{align*}
% 		\tildecov(\zee_S,\zee_T) = \sum_{\alpha\in [q]^S,\beta\in [q]^T} \abs*{ \tildecov(\zee_{S,\alpha},\zee_{T,\beta}) }
% 	\end{align*}
% We also define the analogous pseudovariance as,
% 	\begin{align*}
% 		\tildeVar{\zee_S} = \sum_{\alpha\in [q]^S} \abs*{ \tildeVar{\zee_{S,\alpha} } }
% 	\end{align*}
% \end{definition}
%
We will need the fact that $\tildeVar{\zee_S}$ is bounded above by 1, since,
%
\ifnum\confversion=1
\begin{align*}
\tildeVar{\zee_S} 
&~=~ \sum_{\alpha} \abs*{\tildeVar{\zee_{S,\alpha}}} \\
&~=~ \sum_{\alpha}\inparen{
          \tildeEx{\zee_{S,\alpha}^2} - \tildeEx{\zee_{S,\alpha}}^2} \\
&~\leq~ \sum_{\alpha} \tildeEx{\zee_{S,\alpha}^2} \\
&~=~ \sum_{\alpha} \tildeEx{\zee_{S,\alpha}} 
~=~ 1.
\end{align*}
\else
\[
\tildeVar{\zee_S} 
~=~ \sum_{\alpha} \abs*{\tildeVar{\zee_{S,\alpha}}} 
~=~ \sum_{\alpha}\inparen{
          \tildeEx{\zee_{S,\alpha}^2} - \tildeEx{\zee_{S,\alpha}}^2} 
~\leq~ \sum_{\alpha} \tildeEx{\zee_{S,\alpha}^2} 
~=~ \sum_{\alpha} \tildeEx{\zee_{S,\alpha}} 
~=~ 1.
\]
\fi

% \begin{claim}\quad	$\tildeVar{\zee_S} \leq 1$.
% \end{claim}
% %
% \begin{proof}
% \begin{align*}
% 	\tildeVar{\zee_S} &= \sum_{\alpha} \abs*{\tildeVar{\zee_{S,\alpha}}} = \sum_{\alpha}\inparen{ \tildeEx{\zee_{S,\alpha}^2} - \tildeEx{\zee_{S,\alpha}}^2} \leq \sum_{\alpha} \tildeEx{\zee_{S,\alpha}^2} = \sum_{\alpha} \tildeEx{\zee_{S,\alpha}} = 1
% \end{align*}
% \end{proof}
%
\vspace{-5 pt}
\paragraph{Conditioning SoS solutions.}
%
We will also make use of conditioned pseudoexpectation operators, which may be defined in a way
similar to usual conditioning for true expectation operators, as long as the event we condition on
is local. 
%
The conditioned SoS solution is of a smaller degree but continues to respect the constraints that the original solution respects.

\begin{definition}[Conditioned SoS Solution] Let $F \subseteq \Sigma^m$ be subset (to be thought of as an event) such that $\one_F:\Sigma^m \rightarrow \{0,1\}$ is a $k$-local function. Then for every $t>2k$, we can condition a pseudoexpectation operator of SoS-degree $t$ on $F$ to obtain a new conditioned pseudoexpectation operator $\condPE{\cdot}{F}$ of SoS-degree $t-2k$, as long as $\tildeEx{\one^2_F(\zee)}>0$. The conditioned SoS solution is given by
\[
	\condPE{ p(\zee)}{F(\zee) } \defeq \frac{\tildeEx{p(\zee) \cdot \one^2_{F}(\zee)}}{\tildeEx{\one^2_{F}(\zee)}}
\]
where $p$ is any polynomial of degree at most $t-2k$.
\end{definition}

%\snote{Mention that constraints \emph{respected} by SoS remain true after conditioning.}

We can also define pseudocovariances and pseudo-variances for the conditioned SoS solutions.
\begin{definition}[Pseudocovariance]
	Let $F\sub \Sigma^m$ be an event such that $\one_F$ is $k$-local, and let $\tildeEx{\cdot}$ be a pseudoexpectation operator of degree $t$, with $t>2k$. Let $S,T$ be two sets of size at most $\frac{t-2k}{2}$ each. Then the pseudocovariance between $\zee_{S,\alpha}$ and $\zee_{T,\beta}$ for the solution conditioned on event $F$ is defined as,
\ifnum\confversion=1
\begin{multline*}
\tildecov(\zee_{S,\alpha},\zee_{T,\beta} \vert F) = \\
\tildeEx{ \zee_{S,\alpha} \zee_{T,\beta} \vert F} - \tildeEx{\zee_{S,\alpha} \vert F} \tildeEx{\zee_{T,\beta} \vert F}
\end{multline*}
\else
\begin{align*}
\tildecov(\zee_{S,\alpha},\zee_{T,\beta} \vert F) 
~=~ \tildeEx{ \zee_{S,\alpha} \zee_{T,\beta} \vert F} - \tildeEx{\zee_{S,\alpha} \vert F} ~ \tildeEx{\zee_{T,\beta} \vert F}
\end{align*}
\fi
\end{definition}

We also define the pseudocovariance between $\zee_{S,\alpha}$ and $\zee_{T,\beta}$ after
conditioning on a random assignment for some $\zee_V$ with $V\sub [m]$. 
%
Note that here the random assignment for $\zee_V$ is chosen according to the local distribution for
the set $V$.

\begin{definition}[Pseudocovariance for conditioned pseudoexpectation operators]
\ifnum\confversion=1
\begin{multline*}
\tildecov(\zee_{S,\alpha},\zee_{T,\beta} \vert \zee_V) 
= \\ \sum_{\gamma \in \Sigma^V}   \tildecov(\zee_{S,\alpha},\zee_{T,\beta} \vert \zee_V = \gamma) \cdot \tildeEx{\zee_{V,\gamma}}
	\end{multline*}
\else
\begin{align*}
\tildecov(\zee_{S,\alpha},\zee_{T,\beta} \vert \zee_V) 
~=~ \sum_{\gamma \in \Sigma^V}   \tildecov(\zee_{S,\alpha},\zee_{T,\beta} \vert \zee_V = \gamma) \cdot \tildeEx{\zee_{V,\gamma}}
\end{align*}
\fi
\end{definition}

And we likewise define $\tildeVar{\zee_{S,\alpha} \vert \zee_V}$, $\tildecov(\zee_S, \zee_T \vert \zee_V)$ and $\tildeVar{\zee_S \vert \zee_V}$.






%%% Local Variables:
%%% mode: latex
%%% TeX-master: "main"
%%% End:
