In this subsection, we describe the final decoding algorithm. We will mainly rely on the main result of the previous subsection: that for appropriately instantiated AEL codes and for $\eta$ small enough, $\eta$-good $k$-tuple of pseudocodewords satisfy the generalized Singleton bound.

\subsubsection{Decoding from Distributions}
Before describing our main algorithm, we argue that a simple idea based on randomized rounding can be used to extend the unique decoder of $\calC_{\out}$ to decode not only from integral strings close to a codeword, but an ensemble of distributions - one for each coordinate - that is close to a codeword in average sense. It is also standard to derandomize this process via threshold rounding, which is what we describe next. We will be needing this strengthening of the unique decoder of $\calC_{\out}$ as a subroutine in the main algorithm.

It will be helpful to index the codewords in $\calC_{\inn}$ by integers. Let $\abs*{\calC_{\inn}} = M$, and let its codewords be $\alpha_1,\alpha_2,\cdots ,\alpha_M$. Recall that for any $k$-tuple of pseudocodewords $\tildeEx{\cdot}$, for any $j\in [k]$ and $\li \in L$, the set of values
\[
	\inbraces{ \tildeEx{\indi{\zee_{j,\li} = \alpha_i}} }_{i\in [M]}
\]
correspond to a probability distribution over $\calC_{\inn}$. Therefore, the set of intervals
\[
	\inbraces{ \left[\sum_{i=1}^{m-1} \tildeEx{\indi{\zee_{j,\li} = \alpha_i}}, \sum_{i=1}^{m} \tildeEx{\indi{\zee_{j,\li} = \alpha_i}} \right) }_{m\in [M]}
\]
partitions the interval $[0,1)$ into at most $M$ parts.

\begin{lemma}\label{lem:decode_from_distrib}
	Suppose the code $\calC_{\out}$ is unique decodable up to radius $\delta_{\out}^{\dec}$, and let $h$ be a codeword in $\AELC$. Given a collection of distributions $\{\calD_{\li}\}_{\li \in L}$, with each distribution over $\calC_{\inn}$ such that
	\[
		\Ex{\li}{\Ex{f\sim \calD_{\li}}{{\indi{f\neq h_{\li}}}}} \leq \delta_{\out}^{\dec},
	\]
	the \cref{algo:unique-decoding} finds $h$.
\end{lemma}

\begin{figure}[!ht]
\begin{algorithm}{\DECODE}{$\{\calD_{\li}\}_{\li\in L}$}{Codeword $h\in \AELC$ such that $\Ex{\li}{\Ex{f\sim \calD_{\li}}{\indi{f\neq h_{\li}}}} \leq \delta_{\out}^{\dec}$}
\label{algo:unique-decoding}
%
\begin{itemize}
%
%	\item For $p=1$ to $k$:\label{step2}
%	\begin{enumerate}[(i)]
%%		\item Find a $p$-tuple of pseudocodewords $\pcod{p}{\cdot}$ of degree $t = \cdots$ that respects the following constraints:
%%		\begin{itemize}
%%			\item For every $i,i' \in [p]$ with $i\neq i'$, $~\pcod{p}{\Delta_L(\zee_i,\zee_{i'})} > \delta_{\out}^{\dec}$.
%%			\item For every $i\in [p]$, $\pcod{p}{\Delta_R(g,\zee_i)} \leq \frac{k-1}{k}(1-\rho-\eps)$.
%%		\end{itemize}
%%		\item If no such $\pcod{p}{\cdot}$ exists:
%%		\begin{itemize}
%%		\item $p^* = p-1$
%%		\item Exit loop.
%%		\end{itemize}
%	\end{enumerate}
	\item Let $w_{\li,j}$ be the weight on codeword $\alpha_i$ according to distribution $\calD_{\li}$.
	\item For every threshold $\theta \in [0,1]$:\footnote{As written, this involves trying uncountably many thresholds. However, we only need to try at most $M \cdot |L|$ thresholds since the algorithm only depends on which intervals $\theta$ belongs to. Since the number of endpoints of these intervals is at most $M\cdot |L|$ in total, it suffices to try only $M\cdot |L| = M\cdot n$ many distinct thresholds. This is a standard method called threshold rounding.}
			\begin{enumerate}[(i)]
				\item Construct an $f_{\theta} \in \calC_{\inn}^L$ by assigning 
				\[(f_{\theta})_{\li} = \alpha_m  \iff \theta \in \left[ \sum_{i=1}^{m-1} w_{\li, i}, \sum_{i=1}^{m} w_{\li,i} \right)\]
				for every $\li \in L$.
				\item Let $f_\theta^* \in \Sigma_{\out}^L$ defined as $(f_{\theta}^*)_{\li} = \phi^{-1}((f_{\theta})_{\li})$. 
				\item Use the unique decoder of $\calC_{\out}$ to find an $h^* \in \calC_{\out}$ whose distance from $f_{\theta}^*$ is at most $\delta_{\out}^{\dec}$, if such an $h^*$ exists. That is, $h^* \leftarrow \mathrm{Dec}_{\calC_{\out}}(f_{\theta}^*,\delta_{\out}^{\dec})$.
				\item Let $h\in \AELC$ be the codeword corresponding to $h^* \in \calC_{\out}$. Return $h$.
				\end{enumerate}
		\end{itemize}
%		\item For every threshold $\theta \in [0,1]$:\footnote{As written, this involves trying uncountably many thresholds. However, we only need to try at most $M \cdot |L|$ thresholds since the algorithm only depends on which intervals $\theta$ belongs to. Since the number of thresholds is at most $M\cdot |L|$, it suffices to try only $M\cdot |L|$ many distinct thresholds. This is a standard method called threshold rounding.}
%			\begin{enumerate}[(i)]
%				\item Construct an $f_{\theta} \in \calC_{\inn}^L$ by assigning 
%				\[f_{\li} = \alpha_m  \iff \theta \in \left[ \sum_{i=1}^{m-1} \pcod{p^*}{\indi{\zee_{j,\li} = \alpha_i}}, \sum_{i=1}^{m} \pcod{p^*}{\indi{\zee_{j,\li} = \alpha_i}} \right)\]
%				for every $\li \in L$.
%				\item Let $f_\theta^* \in \Sigma_{\out}^L$ defined as $(f_{\theta}^*)_{\li} = \phi^{-1}((f_{\theta})_{\li})$. 
%				\item Use the unique decoder of $\calC_{\out}$ to find an $h^* \in \calC_{\out}$ whose distance from $f_{\theta}^*$ is at most $\delta_{\out}^{\dec}$, if such an $h^*$ exists. That is, $h^* \leftarrow \mathrm{Dec}(f_{\theta}^*,\delta_{\out}^{\dec})$.
%				\item Let $h$ be the codeword of $\AELC$ corresponding to $h^* \in \calC_{\out}$. If $\Delta_R(g,h) < \frac{k-1}{k} (1-\rho-\eps)$, add $h$ to $\calL$.
%			\end{enumerate}
%
\vspace{5pt}
%
\end{algorithm}
\end{figure}

\begin{proof}
	It suffices to show that there exists a threshold $\theta \in [0,1)$ for which the distance between $f_{\theta}^*$ constructed by \cref{algo:unique-decoding} and $h^*$ is at most $\delta_{\out}^{\dec}$. In fact, we will show that this is true for an average $\theta$.
	\begin{align*}
		\Ex{\theta \in [0,1)}{ \Delta(f_{\theta}^*,h^*)} &= \Ex{\theta \in [0,1)}{ \Delta_L(f_{\theta},h)} = \Ex{\theta \in [0,1)}{ \Ex{\li}{\indi{(f_{\theta})_{\li} \neq h_{\li}}}}  = \Ex{\theta \in [0,1)}{ \Ex{\li}{\sum_{i\in [M] : \alpha_i \neq h_{\li}}\indi{(f_{\theta})_{\li} = \alpha_i}}} 
	\end{align*}
	We can move the expectation over $\theta$ inside and use the fact that $(f_{\theta})_{\li}$ is $\alpha_i$ with probability exactly $w_{\li,i}$ to get
	\begin{align*}
		\Ex{\li}{\sum_{i\in [M] : \alpha_i \neq h_{\li}} \Ex{\theta \in [0,1)}{\indi{(f_{\theta})_{\li} = \alpha_i}}} =  \Ex{\li}{\sum_{i\in [M] : \alpha_i \neq h_{\li}} w_{\li,i}} = \Ex{\li}{\Ex{f\sim \calD_{\li}}{{\indi{f\neq h_{\li}}}}} \leq \delta_{\out}^{\dec}. \qedhere
	\end{align*}
\end{proof}

\subsubsection{Decoding Algorithm}
\begin{table}[h]
\hrule
\vline
\begin{minipage}[t]{0.99\linewidth}
\vspace{-5 pt}
{\small
\begin{align*}
    &\mbox{find}\quad ~~ \tildeEx{\cdot} ~\text{on}~ \zee_{[p],E} \text{ and alphabet }\Sigma_{\inn}%\tag{List Decoding Program}\label{sos:list_dec}
    \\
&\mbox{subject to}\quad \quad ~\\
	&\qquad \text{(i)}~~ \tildeEx{\cdot} \text{is a }p\text{-tuple of pseudocodewords of SoS-degree } t \\
    &\qquad \text{(ii)}~~ \forall i \in [p],~ \text{the constraint }\Delta_R(g,\zee_i) < \frac{k-1}{k}(1-\rho-\eps) \text{ is respected by }\tildeEx{\cdot}\label{cons:agreement-ld}    \\
&\qquad \text{(iii)}~ \forall F \sub E \text{ such that } |F|\leq \frac{t-2pd}{2}, \sigma \in \Sigma_{\inn}^{[p]\times F} \text{ and }\forall~i,i' \in [p] \text{ with } i\neq i', \\
&\qquad \qquad \qquad \qquad \tildeEx{\inparen{\Delta_L(\zee_i,\zee_{i'}) - \delta_{\out}^{\dec}}\cdot  \indi{\zee_{[p],F} = \sigma}^2 } \geq 0
\end{align*}}
\vspace{-10 pt}
\end{minipage}
\hfill\vline
\hrule
\caption{$\mathrm{SDP}(p, t)$}
\label{tab:SDP_for_feasibility}
\end{table}

\begin{theorem}\label{thm:sos_technical}
%
Let $k\geq 1$ be an integer and let $\eps > 0$. Let $\AELC$ be a
code obtained using the AEL construction using  $(G, \calC_{\out}, \calC_{\inn})$, where $\calC_{\inn}$ is $(\delta_0, k, \eps/2)$ average-radius list decodable with erasures, and $G$ is a $(n,d,\lambda)$-expander. 
%
Suppose that $\calC_{\out}$ is unique decodable from radius $\delta_{\out}^{\dec}$ in time $\calT(n)$.

If $\lambda \leq \frac{\delta_{\out}^{\dec}}{12{k}^{k}} \cdot \eps$, then there is a deterministic algorithm that takes as input $g\in (\Sigma_{\inn}^d)^R$, runs in time $\calT(n) + n^{O\inparen{ \frac{d \cdot k^{3k}|\Sigma_{\inn}|^{3kd}}{(\delta_{\out}^{\dec})^2\cdot \eps^2}}}$, and outputs the list $\calL(g,\frac{k-1}{k}(\delta_0-\eps))$.
\end{theorem}

\begin{figure}[!ht]
\begin{algorithm}{List Decoding algorithm up to $\frac{k-1}{k}(\delta_0-\eps)$}{$k$, $g \in (\Sigma_{\inn}^d)^R$}{List of codewords $\calL\inparen{g, \frac{k-1}{k}\inparen{\delta_0-\eps}}$}\label{algo:sos-decoding}
%
\begin{enumerate}
%
	\item Initialize $\calL = \{ \}$, $t = 2kd\cdot \inparen{1+\frac{144k^{2k}|\Sigma_{\inn}|^{3kd}}{(\delta_{\out}^{\dec})^2\cdot \eps^2}}$.
	\item Let $p^*$ be the largest $p\in [k]$ such that $\mathrm{SDP}(p,t)$ is feasible.
%	\item For $p=1$ to $k$:\label{step2}
%	\begin{enumerate}[(i)]
%%		\item Find a $p$-tuple of pseudocodewords $\pcod{p}{\cdot}$ of degree $t = \cdots$ that respects the following constraints:
%%		\begin{itemize}
%%			\item For every $i,i' \in [p]$ with $i\neq i'$, $~\pcod{p}{\Delta_L(\zee_i,\zee_{i'})} > \delta_{\out}^{\dec}$.
%%			\item For every $i\in [p]$, $\pcod{p}{\Delta_R(g,\zee_i)} \leq \frac{k-1}{k}(1-\rho-\eps)$.
%%		\end{itemize}
%%		\item If no such $\pcod{p}{\cdot}$ exists:
%%		\begin{itemize}
%%		\item $p^* = p-1$
%%		\item Exit loop.
%%		\end{itemize}
%	\end{enumerate}
	\item For every $F \sub E$ with $|F|\leq \frac{t-2kd}{2}$ and $\sigma\in \Sigma_{\inn}^F$ with $\pcod{p^*}{\indi{\zee_{[p^*],F} = \sigma}} > 0$:\label{step3}
		\begin{itemize}
		\item For $j=1$ to $p^*$:
		\begin{itemize}
			\item Construct an ensemble of distributions $\calD = \{\calD_{\li}\}_{\li\in L}$, with each $\calD_{\li}$ over $\calC_{\inn}$ by using the local distribution induced by the conditioned $\pcod{p^*}{ ~\cdot \given \zee_{[p^*],F} = \sigma}$ over the set $\{j\} \times N(\li)$, or equivalently, over variables $\zee_{j,\li}$.
			\item $h\leftarrow \text{\DECODE} \inparen{\calD}$.
			\item If $\Delta_R(g,h) < \frac{k-1}{k} (\delta_0-\eps)$, add $h$ to $\calL$.
		\end{itemize}
%		\item For every threshold $\theta \in [0,1]$:\footnote{As written, this involves trying uncountably many thresholds. However, we only need to try at most $M \cdot |L|$ thresholds since the algorithm only depends on which intervals $\theta$ belongs to. Since the number of thresholds is at most $M\cdot |L|$, it suffices to try only $M\cdot |L|$ many distinct thresholds. This is a standard method called threshold rounding.}
%			\begin{enumerate}[(i)]
%				\item Construct an $f_{\theta} \in \calC_{\inn}^L$ by assigning 
%				\[f_{\li} = \alpha_m  \iff \theta \in \left[ \sum_{i=1}^{m-1} \pcod{p^*}{\indi{\zee_{j,\li} = \alpha_i}}, \sum_{i=1}^{m} \pcod{p^*}{\indi{\zee_{j,\li} = \alpha_i}} \right)\]
%				for every $\li \in L$.
%				\item Let $f_\theta^* \in \Sigma_{\out}^L$ defined as $(f_{\theta}^*)_{\li} = \phi^{-1}((f_{\theta})_{\li})$. 
%				\item Use the unique decoder of $\calC_{\out}$ to find an $h^* \in \calC_{\out}$ whose distance from $f_{\theta}^*$ is at most $\delta_{\out}^{\dec}$, if such an $h^*$ exists. That is, $h^* \leftarrow \mathrm{Dec}(f_{\theta}^*,\delta_{\out}^{\dec})$.
%				\item Let $h$ be the codeword of $\AELC$ corresponding to $h^* \in \calC_{\out}$. If $\Delta_R(g,h) < \frac{k-1}{k} (1-\rho-\eps)$, add $h$ to $\calL$.
%			\end{enumerate}
		\end{itemize}
	\item Return $\calL$.
\end{enumerate}
%
\vspace{5pt}
%
\end{algorithm}
\end{figure}

\begin{proof}
	\cref{algo:sos-decoding} describes this algorithm. In the rest of the proof, we argue the correctness of this algorithm.
	
	We first start with a lemma that readily follows from \cref{thm:sos_main}.
	
	\begin{lemma}\label{lem:pseudocodeword_list_size}
	If $\lambda \leq  \frac{\delta_{\out}^{\dec}}{12k^k} \cdot \eps$, and $t \geq 2kd\cdot \inparen{1+\frac{144k^{2k}|\Sigma_{\inn}|^{3kd}}{(\delta_{\out}^{\dec})^2\cdot \eps^2}}$, then the $\mathrm{SDP}(k,t)$ is infeasible.
	\end{lemma}

	In other words, the lemma says that $p^* < k$, and that $\mathrm{SDP}(p^*+1,t)$ is infeasible. Let $\pcod{p^*}{\cdot}$ be the $p^*$-tuple of pseudocodewords found by $\mathrm{SDP}(p^*,t)$.
%	
	\begin{lemma}\label{lem:covering}
	For any $h \in \calL\inparen{g,\frac{k-1}{k}(1-\rho-\eps)}$,
%	 and its corresponding $h^* \in \calC_{\out}$, 
there exists an $i\in [p^*]$, a set $F\sub E$ with $|F|\leq \frac{t-2kd}{2}$, and a $\sigma \in \Sigma_{\inn}^{[p^*]\times F}$ with $\pcod{p^*}{\indi{\zee_{[p^*],F} = \sigma}} > 0$, such that 
	\[
	\pcod{p^*}{\Delta_L(h, \zee_i) \given \zee_{[p^*],F} = \sigma} \leq \delta_{\out}^{\dec}.
	\] 
	\end{lemma}
	This lemma proves the correctness of the algorithm, since then $h$ will be added to $\calL$ in \hyperref[step3]{Step~\ref*{step3}}, by \cref{lem:decode_from_distrib}.
\end{proof}

\begin{proof}[Proof of \cref{lem:covering}]
	To show this, we will construct a $(p^*+1)$-tuple of pseudocodewords, say $\pcod{p^*+1}{\cdot}$, and use its infeasibility for $\mathrm{SDP}(p^*+1,t)$.
	
	Recall that $\pcod{p^*}{\cdot}$ is an SoS relaxation over the variables $\zee_{[p^*] \times E }$,
% = \{ Z_{i,e,s} \}_{i\in [p^*], e\in E,s \in \Sigma_{\inn}}$
while $\pcod{p^*+1}{\cdot}$ will be an SoS relaxation over the variables $\zee_{[p^*+1]\times E }$.
% = \{ Z_{i,e,s} \}_{i\in [p^*+1], e\in E,s\in \Sigma_{\inn}}$. 
	
	To describe the new $(p^*+1)$-tuple of pseudocodewords, we explicitly specify the corresponding pseudoexpectation operator $\pcod{p^*+1}{\cdot}$.
	Let $P$ be an arbitrary polynomial of degree $t$ over $\zee_{[p^*+1], E}$. We define a new polynomial $P_h$ over $\zee_{[p^*], E}$ by assigning the variables $\zee_{p^*+1,e}$ to be $h_e$ for every $e\in E$.
%	\begin{align*}
%		Z_{p^*+1,e,s} = \begin{cases}
%		1 \quad & h_e = s \\
%		0 \quad & h_e \neq s
%		\end{cases}
%	\end{align*}	
	Then, we define $\pcod{p^*+1}{\cdot}$ using the following
	\[
		\pcod{p^*+1}{P(\zee_{[p^*+1],E})} = \pcod{p^*}{P_h(\zee_{[p^*],E})}
	\]
	This is well-defined since the degree of $P_h$ cannot be more than $t$.
	
	By optimality of $p^*$, $\pcod{p^*+1}{\cdot}$ must be infeasible for $\mathrm{SDP}(p^*+1,t)$. It can be verified that $\pcod{p^*+1}{\cdot}$ is a valid $(p^*+1)$-tuple of pseudocodewords. Further, the constraints 
	\[
		\Delta_R(g,\zee_i) < \frac{k-1}{k}(\delta_0 - \eps)
	\]
	are respected for all $i\in [p^*]$. This is because 
	\[
		\pcod{p^*+1}{P(\zee)^2 \cdot \inparen{\Delta_R(g,\zee_i) - \frac{k-1}{k}(\delta_0 - \eps)}} = \pcod{p^*}{P_h(\zee)^2 \cdot \inparen{\Delta_R(g,\zee_i) - \frac{k-1}{k}(\delta_0 - \eps)}} < 0
	\]
	Further, for $i=p^*+1$,
	\begin{align*}
		\pcod{p^*+1}{P(\zee)^2 \cdot \inparen{\Delta_R(g,\zee_i) - \frac{k-1}{k}(\delta_0 - \eps)}} &= \pcod{p^*}{P_h(\zee)^2 \cdot \inparen{\Delta_R(g,h) - \frac{k-1}{k}(\delta_0 - \eps)}} \\
		&= \inparen{\Delta_R(g,h) - \frac{k-1}{k}(\delta_0 - \eps)} \cdot \pcod{p^*}{P_h(\zee)^2} \\
		&< 0
	\end{align*}
	Therefore, the infeasibility must be due to constraints of type (iii) in the SDP. In other words, there exist
	\begin{enumerate}
	\item a set $F\sub E$ with $|F|\leq \frac{t-2(p^*+1)d}{2}$,
	\item a $\sigma \in \Sigma_{\inn}^{[p^*+1]\times F}$ 
	%with $\pcod{p^*+1}{\zee_{[p^*+1],F,\sigma}} >0$
	, and 
	\item a pair $i,i'\in [p^*+1]$ with $i\neq i'$
	\end{enumerate}
	such that
	\[
		\pcod{p^*+1}{ \inparen{\Delta_L(\zee_i,\zee_{i'}) - \delta_{\out}^{\dec}} \cdot \indi{\zee_{[p^*+1],F} = \sigma}^2} < 0
	\]
	Case I. $i,i' \in [p^*]$:
	
	In this case,
	\begin{align*}
		0~>&~ \pcod{p^*+1}{ \inparen{\Delta_L(\zee_i,\zee_{i'}) - \delta_{\out}^{\dec}} \cdot \indi{\zee_{[p^*+1],F} = \sigma}} \\
		=~~ & \pcod{p^*+1}{ \inparen{\Delta_L(\zee_i,\zee_{i'}) - \delta_{\out}^{\dec}} \cdot \indi{\zee_{[p^*],F} = \sigma_1} \cdot \indi{\zee_{p^*+1,F} = \sigma_2}} \\
		=~~ & \pcod{p^*}{ \inparen{\Delta_L(\zee_i,\zee_{i'}) - \delta_{\out}^{\dec}} \cdot \indi{\zee_{[p^*],F} = \sigma_1} \cdot \indi{h_F = \sigma_2}} \\
		=~~ & \indi{h_F = \sigma_2} \cdot \pcod{p^*}{ \inparen{\Delta_L(\zee_i,\zee_{i'}) - \delta_{\out}^{\dec}} \cdot \indi{\zee_{[p^*],F} = \sigma_1} }
	\end{align*}
Because of the strict inequality with $0$, it must be the case that $\indi{h_F = \sigma_2} =1$, and the resulting equation contradicts the feasibility of $\pcod{p^*}{\cdot}$. So this case cannot happen.

\medskip
%	
\noindent Case II. Without loss of generality, $i'=p^*+1$.
	
%	We may assume $\pcod{p^*+1}{\indi{\zee_{[p^*+1],F} = \sigma}^2} > 0$. If not, 
%	\begin{align*}
%		&\pcod{p^*+1}{\indi{\zee_{[p^*+1],F} = \sigma}^2} = 0 \\
%		\implies & \pcod{p^*+1}{ \Delta_L(\zee_i,\zee_{i'}) \cdot \indi{\zee_{[p^*+1],F} = \sigma}^2} < 0
%	\end{align*}
%	
%	Further, it can be checked that both $i$ and $i'$ cannot be $\leq p^*$, since $\pcod{p^*}{\cdot}$ is feasible for $\mathrm{SDP}(p^*,t)$. Without loss of generality, let $i' = p^*+1$. Then,
In this case,
	\begin{align*}
		0 &~>~  \pcod{p^*+1}{ \inparen{\Delta_L(\zee_i,\zee_{p^*+1}) - \delta_{\out}^{\dec}} \cdot \indi{\zee_{[p^*+1],F} = \sigma}} \\
		&=  \pcod{p^*+1}{ \inparen{\Delta_L(\zee_i,\zee_{p^*+1}) - \delta_{\out}^{\dec}} \cdot \indi{\zee_{[p^*],F} = \sigma_1} \cdot \indi{\zee_{p^*+1,F} = \sigma_2}} \\
		&=\pcod{p^*}{ \inparen{\Delta_L(\zee_i,h) - \delta_{\out}^{\dec}} \cdot \indi{\zee_{[p^*],F} = \sigma_1} \cdot \indi{h_F = \sigma_2}}\\
		&=\indi{h_F = \sigma_2} \cdot \pcod{p^*}{ \inparen{\Delta_L(\zee_i,h) - \delta_{\out}^{\dec}} \cdot \indi{\zee_{[p^*],F} = \sigma_1}}\\
%		&=\frac{\pcod{p^*}{ \inparen{\Delta_L(\zee_i,h) - \delta_{\out}^{\dec}} \cdot \indi{\zee_{[p^*],F} = \sigma_1}}}{\pcod{p^*}{\indi{\zee_{[p^*],F} = \sigma_1}}} \\
%		&=\pcod{p^*}{ \inparen{\Delta_L(\zee_i,h) - \delta_{\out}^{\dec}} \given \zee_{[p^*],F} = \sigma_1}
	\end{align*}
	Again, as before we must have $\indi{h_F = \sigma_2} =1$, so that
	\begin{align}\label{eq:without_h}
		\pcod{p^*}{ \inparen{\Delta_L(\zee_i,h) - \delta_{\out}^{\dec}} \cdot \indi{\zee_{[p^*],F} = \sigma_1}} < 0
	\end{align}
	Finally, we may also assume $\pcod{p^*}{\indi{\zee_{[p^*+1],F} =\sigma}} >0$. If not, then
	\[
		\pcod{p^*}{ \Delta_L(\zee_i,h) \cdot \indi{\zee_{[p^*],F} = \sigma_1}} < 0
	\]
	which is impossible since $\pcod{p^*}{P(\zee)} \geq 0$ for all polynomials that are sum of squares of polynomials. It is easy to verify that $\Delta_L(\zee_i,h) \cdot \indi{\zee_{[p^*],F} = \sigma_1}$ is a sum of squares of polynomials.
	
	Therefore, dividing \cref{eq:without_h} by $\pcod{p^*}{\indi{\zee_{[p^*+1],F} =\sigma}}$, we get that the set $F$ and the assignment $\sigma_1 \in \Sigma_{\inn}^{[p^*] \times F}$ have the property that
	\[
		\pcod{p^*}{\Delta_L(\zee_i,h) \given \zee_{[p^*],F} = \sigma_1} \leq \delta_{\out}^{\dec}
	\]
	as needed.
\end{proof}

\begin{proof}[Proof of \cref{lem:pseudocodeword_list_size}]
	Suppose $\mathrm{SDP}(k,t)$ is feasible with $t = 2kd\cdot \inparen{1+\frac{144k^{2k}|\Sigma_{\inn}|^{3kd}}{(\delta_{\out}^{\dec})^2\cdot \eps^2}}$, so that there exists a $k$-tuple of pseudocodewords $\tildeEx{\cdot}$ that satisfies all the constraints in the SDP. 
	Let $\eta = \frac{\delta_{\out}^{\dec}}{12k^k} \cdot \eps$.
	
	 Then $t\geq 2kd\cdot \inparen{1+\frac{|\Sigma_{\inn}|^{3kd}}{\eta^2}}$, so that \cref{lem:condition_for_eta_good} says that there exists a set $S \sub E$ of size at most $d\cdot \frac{|\Sigma_{\inn}|^{3kd}}{\eta^2}$ such that \[\pcod{S}{\cdot} \defeq \tildeEx{~\cdot \given \zee_{[k],S}} \] is $\eta$-good.
	
	Further, since the constraint $\Delta_R(g,\zee_i) < \frac{k-1}{k}(1-\rho-\eps)$ is \emph{respected} by $\tildeEx{\cdot}$, we can also conclude that for all $i\in [k]$,
	\begin{align}\label{eqn:zee_i_in_ball}
		\pcod{S}{\Delta_R(g,\zee_i)} < \frac{k-1}{k}(1-\rho-\eps)
	\end{align}
	Moreover, for any $i,i'\in [k]$ with $i\neq i'$,
	\[
		\pcod{S}{\inparen{\Delta_L(\zee_i,\zee_{i'}) - \delta_{\out}^{\dec}}} = \Ex{\sigma \sim \calD_{[k]\times S}}{\frac{\tildeEx{ \inparen{\Delta_L(\zee_i,\zee_{i'}) - \delta_{\out}^{\dec}} \cdot \indi{\zee_{[k], S} = \sigma}^2}}{\tildeEx{\indi{\zee_{[k], S}}^2}}} \geq 0
	\]
	where $\calD_{[k]\times S}$ is the local distribution on $\zee_{[k]\times S}$ according to $\tildeEx{\cdot}$. This means that
	\[
		\pcod{S}{\Delta_L(\zee_i,\zee_{i'})} \geq \delta_{\out}^{\dec}
	\]
	Since $\lambda \leq \frac{\delta_{\out}^{\dec}}{12k^k} \cdot \eps$ and $\eta \leq \frac{\delta_{\out}^{\dec}}{12k^k} \cdot \eps$, we can apply \cref{thm:sos_main} to $\pcod{S}{\cdot}$ with $\beta = \delta_{\out}^{\dec}$ and get
	\[
		\sum_{i\in [k]} \pcod{S}{\Delta_R(g,\zee_i)} \geq \frac{k-1}{k}(1-\rho-\eps) + \pcod{S}{\Delta_R(g,\zee_{[k]})} \geq \frac{k-1}{k}(1-\rho-\eps)
	\]
	which is contradicted by \cref{eqn:zee_i_in_ball}.
\end{proof}

Finally, we instantiate \cref{thm:sos_technical} with unique decodable outer codes to obtain codes that can be efficiently decoded up to the list decoding capacity.
\begin{corollary}\label{cor:algo-main}
For every $\rho, \eps \in (0,1)$ and $k \in \N$, there exist explicit inner codes $\calC_{\inn}$ and an infinite family explicit codes $\AELC \subseteq (\F_q^d)^n$ obtained via the AEL construction that satisfy: 
\begin{enumerate}
\item $\rho(\AELC) \geq \rho$
\item For any $g \in  (\F_q^d)^n$ and any $\calH \subseteq \AELC$ with $\abs{\calH} \leq k$ that
\[
\sum_{h \in \calH} \Delta(g,h) ~\geq~ (\abs{\calH}-1) \cdot (1 - \rho - \eps) \mper
\]
\item The alphabet size $q^d$ of the code $\AELC$ can be taken to be $2^{O(k^{3k}/\eps^9)}$.
\item $\AELC$ can be decoded from radius $\frac{k-1}{k}(1-\rho-\eps)$ in time $n^{2^{O(k^{4k}/\eps^{10})}}$ with a list of size at most $k-1$.
\end{enumerate}
\end{corollary}

\begin{proof}
	The first three properties can be ensured by instantiating the AEL amplification procedure as in \cref{cor:ael_instantiation}. We briefly recall the choices in that instantiation.
	
	\begin{itemize}
		\item We picked a graph $G$ with $d = O(k^{2k}/\eps^8)$ and $\lambda \leq \eps^4/(2^{21} k^k)$. 
%
		\item We picked $\calC_{\inn} \subseteq \Sigma_{\inn}^d$ to be a code with rate $\rho_{\inn} = \rho + \eps/4$, which is $(1 - \rho, k, \eps/2)$ average-radius list decodable with erasures. The alphabet size $|\Sigma_{\inn}|$ for $\calC_{\inn}$ was $2^{O(k + 1/\eps)}$.
%
		\item Finally, we picked $\calC_{\out} \subseteq (\Sigma_{\inn}^{\rho_{\inn} \cdot d})^n$ be an outer code with rate $\rho_{\out} = 1 - \eps/4$ and distance (say) $\delta_{\out} = \eps^3/2^{15}$. We note that this code can also be decoded from a radius $\delta_{\out}^{\dec} = \frac{\eps^3}{2^{17}}$ \cite{Zemor01, GRS23} in linear time.
	\end{itemize}
%

The choice of $\lambda$ above is sufficient to ensure $\lambda \leq \frac{\delta_{\out}^{\dec}}{12k^k}\cdot \eps$, and so we can use \cref{thm:sos_technical} to conclude that there is a deterministic algorithm, that takes as input $g\in (\Sigma_{\inn}^d)^R$, runs in time $n^{O\inparen{ \frac{d \cdot k^{3k}|\Sigma_{\inn}|^{3kd}}{(\delta_{\out}^{\dec})^2\cdot \eps^2}}}$, and outputs the list $\calL\inparen{g,\frac{k-1}{k}\inparen{1-\rho-\eps}}$. With the above choice of parameters, the exponent of $n$ in the runtime becomes $2^{O(\frac{k^{4k}}{\eps^{10}})}$.
%\begin{align*}
%	\frac{d \cdot k^{3k}|\Sigma_{\inn}|^{3kd}}{(\delta_{\out}^{\dec})^2\cdot \eps^2} &= \frac{k^{2k}}{6} \cdot k^{3k} \cdot \frac{1}{\eps^2} \frac{1024}{\eps^4} 2^{O(kd(k+\frac{1}{\eps}))} \\
%	&= \frac{k^{2k}}{6} \cdot k^{3k} \cdot \frac{1}{\eps^2} \frac{1024}{\eps^4} 2^{O(\frac{k^{3k}}{\eps^7})} \\
%	 &= 2^{O(\frac{k^{4k}}{\eps^8})}
%\end{align*}
%
%Given the above parameters, we have 
%$\rho(\AELC) ~\geq~ \rho_{\out} \cdot \rho_{\inn} ~=~ (1-\eps/4) \cdot (\rho + \eps/4) ~\geq~ \rho$.
%%
%Since $\lambda \leq \eps \cdot \delta_{\out}/(6k^k)$ and $\calC_{\inn}$ is $(1-\rho, k, \eps/2)$ average-radius list decodable with erasures, we can use \cref{thm:main_technical_avg} to conclude that $\AELC$ is $(1-\rho, k, \eps)$ average-radius list decodable (with erasures) which yields the second condition. Finally, we note that the alphabet size of the code $\AELC$ is $q^d = \exp\inparen{O((k + 1/\eps) \cdot (k^{2k}/\eps^6))} = \exp\inparen{k^{3k}/\eps^7}$, which proves the claim. 
%
%We only need to argue that the codes constructed there can be decoded efficiently.
\end{proof}