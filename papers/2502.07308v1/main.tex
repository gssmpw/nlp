\documentclass{article}

\usepackage[utf8]{inputenc}
\usepackage{pdfsync}

\def\showauthornotes{0}
\def\showkeys{0}
\def\showdraftbox{0}
\def\confversion{0}
\def\widemargin{0}

\newcommand{\thought}[1]{{\color[rgb]{0.2,0.39,0.66}(#1)}}
\newcommand{\todo}[1]{{\color[rgb]{1.0,0.0,0.0}(#1)}}
\newcommand{\hsh}[1]{{\color{green!50!black} Henrik: #1}}
\newcommand{\st}[1]{{\color{red!50!black} Sebastian: #1}}

\newcommand{\ulm}[1]{_{\scaleto{\mathrm{#1}}{3pt}}}
\newcommand\at[2]{\left.#1\right|_{#2}}











\newtheorem{assumption}{Assumption}

\DeclareMathOperator*{\argmax}{arg\,max}
\DeclareMathOperator*{\argmin}{arg\,min}

\newcommand{\swname}[1]{\texttt{#1}}
\newcommand{\ie}{i\/.\/e\/.,\/~}
\newcommand{\eg}{e\/.\/g\/.,\/~}
\newcommand{\cf}{cf\/.\/~}

\newcommand{\fig}{Fig\/.\/~}
\newcommand{\defn}{Def\/.\/~}
\newcommand{\sect}{Sec\/.\/~}
\newcommand{\tabl}{Tab\/.\/~}
\newcommand{\algo}{Algorithm~}
\newcommand{\theo}{Theorem~}

\newcommand{\bnnl}{3 hidden layers}
\newcommand{\bnnn}{50 neurons}
\newcommand{\bnna}{tanh activations}

\newcommand{\capt}[1]{\mdseries{\emph{#1}}}

\newcommand{\videolink}{at \url{https://youtu.be/_d7AqTRjz6g}}
\newcommand{\codelink}{\url{https://github.com/wheelbot/mini-wheelbot}}

\newcommand{\fakepar}[1]{\vspace{0mm}\noindent\textbf{#1.}}

\newcommand{\needref}{\textcolor{red}{[REF]}}

\newcommand{\plotfontsize}{9pt}

\newcommand{\softmax}{{\textsc{Softmax}}}
\newcommand{\layernorm}{{\textsc{LayerNorm}}}
\newcommand{\gru}{{\textsc{GRU}}}
\newcommand{\mlp}{{\textsc{MLP}}}
\newcommand{\relu}{{\textsc{ReLU}}}
\newcommand{\transformerdecoder}{{\textsc{Decoder}}}
\newcommand{\bos}{{\textsc{[BOS]}}}

\newcommand{\head}[1]{{\smallskip\noindent\textbf{#1}}}
\newcommand{\alert}[1]{{\color{red}{#1}}}
\newcommand{\sm}{\scriptsize}
\newcommand{\eq}[1]{(\ref{eq:#1})}

\newcommand{\Th}[1]{\textsc{#1}}
\newcommand{\mr}[2]{\multirow{#1}{*}{#2}}
\newcommand{\mc}[2]{\multicolumn{#1}{c}{#2}}
\newcommand{\tb}[1]{\textbf{#1}}
\newcommand{\ch}{\checkmark}

\newcommand{\red}[1]{{\textcolor{red}{#1}}}
\newcommand{\blue}[1]{{\textcolor{blue}{#1}}}
\newcommand{\green}[1]{{\textcolor{green}{#1}}}
\newcommand{\gray}[1]{{\textcolor{gray}{#1}}}

\newcommand{\citeme}[1]{\red{[XX]}}
\newcommand{\refme}[1]{\red{(XX)}}

\newcommand{\fig}[2][1]{\includegraphics[width=#1\linewidth]{fig/#2}}
\newcommand{\figh}[2][1]{\includegraphics[height=#1\linewidth]{fig/#2}}
\newcommand{\figa}[2][1]{\includegraphics[width=#1]{fig/#2}}
\newcommand{\figah}[2][1]{\includegraphics[height=#1]{fig/#2}}

\newcommand{\tran}{^\top}
\newcommand{\mtran}{^{-\top}}
\newcommand{\zcol}{\mathbf{0}}
\newcommand{\zrow}{\zcol\tran}

\newcommand{\ind}{\mathbbm{1}}
\newcommand{\expect}{\mathbb{E}}
\newcommand{\nat}{\mathbb{N}}
\newcommand{\zahl}{\mathbb{Z}}
\newcommand{\real}{\mathbb{R}}
\newcommand{\proj}{\mathbb{P}}
\newcommand{\prob}{\operatorname{P}}
\newcommand{\normal}{\mathcal{N}}

\newcommand{\mif}{\textrm{if}\ }
\newcommand{\other}{\textrm{otherwise}}
\newcommand{\minimize}{\textrm{minimize}\ }
\newcommand{\maximize}{\textrm{maximize}\ }
\newcommand{\st}{\textrm{subject\ to}\ }

\newcommand{\id}{\operatorname{id}}
\newcommand{\const}{\operatorname{const}}
\newcommand{\sgn}{\operatorname{sgn}}
\newcommand{\var}{\operatorname{Var}}
\newcommand{\mean}{\operatorname{mean}}
\newcommand{\trace}{\operatorname{tr}}
\newcommand{\diag}{\operatorname{diag}}
\newcommand{\vect}{\operatorname{vec}}
\newcommand{\cov}{\operatorname{cov}}
\newcommand{\sign}{\operatorname{sign}}
\newcommand{\prj}{\operatorname{proj}}

\newcommand{\defn}{\mathrel{:=}}
\newcommand{\peq}{\mathrel{+\!=}}
\newcommand{\meq}{\mathrel{-\!=}}

\newcommand{\paren}[1]{\left({#1}\right)}
\newcommand{\mat}[1]{\left[{#1}\right]}
\newcommand{\set}[1]{\left\{{#1}\right\}}
\newcommand{\floor}[1]{\left\lfloor{#1}\right\rfloor}
\newcommand{\ceil}[1]{\left\lceil{#1}\right\rceil}
\newcommand{\inner}[1]{\left\langle{#1}\right\rangle}
\newcommand{\norm}[1]{\left\|{#1}\right\|}
\newcommand{\abs}[1]{\left|{#1}\right|}
\newcommand{\frob}[1]{\norm{#1}_F}
\newcommand{\card}[1]{\left|{#1}\right|\xspace}

\newcommand{\diff}{\mathrm{d}}
\newcommand{\der}[3][]{\frac{\diff^{#1}#2}{\diff#3^{#1}}}
\newcommand{\ider}[3][]{\diff^{#1}#2/\diff#3^{#1}}
\newcommand{\pder}[3][]{\frac{\partial^{#1}{#2}}{\partial{{#3}^{#1}}}}
\newcommand{\ipder}[3][]{\partial^{#1}{#2}/\partial{#3^{#1}}}
\newcommand{\dder}[3]{\frac{\partial^2{#1}}{\partial{#2}\partial{#3}}}

\newcommand{\wb}[1]{\overline{#1}}
\newcommand{\wt}[1]{\widetilde{#1}}

\def\xssp{\hspace{1pt}}
\def\ssp{\hspace{3pt}}
\def\msp{\hspace{5pt}}
\def\lsp{\hspace{12pt}}

\newcommand{\cA}{\mathcal{A}}
\newcommand{\cB}{\mathcal{B}}
\newcommand{\cC}{\mathcal{C}}
\newcommand{\cD}{\mathcal{D}}
\newcommand{\cE}{\mathcal{E}}
\newcommand{\cF}{\mathcal{F}}
\newcommand{\cG}{\mathcal{G}}
\newcommand{\cH}{\mathcal{H}}
\newcommand{\cI}{\mathcal{I}}
\newcommand{\cJ}{\mathcal{J}}
\newcommand{\cK}{\mathcal{K}}
\newcommand{\cL}{\mathcal{L}}
\newcommand{\cM}{\mathcal{M}}
\newcommand{\cN}{\mathcal{N}}
\newcommand{\cO}{\mathcal{O}}
\newcommand{\cP}{\mathcal{P}}
\newcommand{\cQ}{\mathcal{Q}}
\newcommand{\cR}{\mathcal{R}}
\newcommand{\cS}{\mathcal{S}}
\newcommand{\cT}{\mathcal{T}}
\newcommand{\cU}{\mathcal{U}}
\newcommand{\cV}{\mathcal{V}}
\newcommand{\cW}{\mathcal{W}}
\newcommand{\cX}{\mathcal{X}}
\newcommand{\cY}{\mathcal{Y}}
\newcommand{\cZ}{\mathcal{Z}}

\newcommand{\vA}{\mathbf{A}}
\newcommand{\vB}{\mathbf{B}}
\newcommand{\vC}{\mathbf{C}}
\newcommand{\vD}{\mathbf{D}}
\newcommand{\vE}{\mathbf{E}}
\newcommand{\vF}{\mathbf{F}}
\newcommand{\vG}{\mathbf{G}}
\newcommand{\vH}{\mathbf{H}}
\newcommand{\vI}{\mathbf{I}}
\newcommand{\vJ}{\mathbf{J}}
\newcommand{\vK}{\mathbf{K}}
\newcommand{\vL}{\mathbf{L}}
\newcommand{\vM}{\mathbf{M}}
\newcommand{\vN}{\mathbf{N}}
\newcommand{\vO}{\mathbf{O}}
\newcommand{\vP}{\mathbf{P}}
\newcommand{\vQ}{\mathbf{Q}}
\newcommand{\vR}{\mathbf{R}}
\newcommand{\vS}{\mathbf{S}}
\newcommand{\vT}{\mathbf{T}}
\newcommand{\vU}{\mathbf{U}}
\newcommand{\vV}{\mathbf{V}}
\newcommand{\vW}{\mathbf{W}}
\newcommand{\vX}{\mathbf{X}}
\newcommand{\vY}{\mathbf{Y}}
\newcommand{\vZ}{\mathbf{Z}}

\newcommand{\va}{\mathbf{a}}
\newcommand{\vb}{\mathbf{b}}
\newcommand{\vc}{\mathbf{c}}
\newcommand{\vd}{\mathbf{d}}
\newcommand{\ve}{\mathbf{e}}
\newcommand{\vf}{\mathbf{f}}
\newcommand{\vg}{\mathbf{g}}
\newcommand{\vh}{\mathbf{h}}
\newcommand{\vi}{\mathbf{i}}
\newcommand{\vj}{\mathbf{j}}
\newcommand{\vk}{\mathbf{k}}
\newcommand{\vl}{\mathbf{l}}
\newcommand{\vm}{\mathbf{m}}
\newcommand{\vn}{\mathbf{n}}
\newcommand{\vo}{\mathbf{o}}
\newcommand{\vp}{\mathbf{p}}
\newcommand{\vq}{\mathbf{q}}
\newcommand{\vr}{\mathbf{r}}
\newcommand{\Vs}{\mathbf{s}}
\newcommand{\vt}{\mathbf{t}}
\newcommand{\vu}{\mathbf{u}}
\newcommand{\vv}{\mathbf{v}}
\newcommand{\vw}{\mathbf{w}}
\newcommand{\vx}{\mathbf{x}}
\newcommand{\vy}{\mathbf{y}}
\newcommand{\vz}{\mathbf{z}}

\newcommand{\vone}{\mathbf{1}}
\newcommand{\vzero}{\mathbf{0}}

\newcommand{\valpha}{{\boldsymbol{\alpha}}}
\newcommand{\vbeta}{{\boldsymbol{\beta}}}
\newcommand{\vgamma}{{\boldsymbol{\gamma}}}
\newcommand{\vdelta}{{\boldsymbol{\delta}}}
\newcommand{\vepsilon}{{\boldsymbol{\epsilon}}}
\newcommand{\vzeta}{{\boldsymbol{\zeta}}}
\newcommand{\veta}{{\boldsymbol{\eta}}}
\newcommand{\vtheta}{{\boldsymbol{\theta}}}
\newcommand{\viota}{{\boldsymbol{\iota}}}
\newcommand{\vkappa}{{\boldsymbol{\kappa}}}
\newcommand{\vlambda}{{\boldsymbol{\lambda}}}
\newcommand{\vmu}{{\boldsymbol{\mu}}}
\newcommand{\vnu}{{\boldsymbol{\nu}}}
\newcommand{\vxi}{{\boldsymbol{\xi}}}
\newcommand{\vomikron}{{\boldsymbol{\omikron}}}
\newcommand{\vpi}{{\boldsymbol{\pi}}}
\newcommand{\vrho}{{\boldsymbol{\rho}}}
\newcommand{\vsigma}{{\boldsymbol{\sigma}}}
\newcommand{\vtau}{{\boldsymbol{\tau}}}
\newcommand{\vupsilon}{{\boldsymbol{\upsilon}}}
\newcommand{\vphi}{{\boldsymbol{\phi}}}
\newcommand{\vchi}{{\boldsymbol{\chi}}}
\newcommand{\vpsi}{{\boldsymbol{\psi}}}
\newcommand{\vomega}{{\boldsymbol{\omega}}}

\newcommand{\rLambda}{\mathrm{\Lambda}}
\newcommand{\rSigma}{\mathrm{\Sigma}}

\newcommand{\vLambda}{\bm{\rLambda}}
\newcommand{\vSigma}{\bm{\rSigma}}

\newcommand{\mbo}{\Th{mBO}\xspace}%
\newcommand{\mboi}{\Th{mBO$^{i}$}\xspace}%
\newcommand{\mboc}{\Th{mBO$^{c}$}\xspace}%
\newcommand{\fgari}{\Th{FG-ARI}\xspace}%
\newcommand{\fgiou}{\Th{mIoU}\xspace}%
\newcommand{\miou}{\Th{mIoU}\xspace}%

\newcommand{\decoder}{\Th{Decoder}\xspace}
\newcommand{\encoder}{\Th{Slot Attention}\xspace}
\newcommand{\maxdecenc}{\Th{Max$($Dec, Slot Att$)$}\xspace}


\makeatletter
\newcommand*\bdot{\mathpalette\bdot@{.7}}
\newcommand*\bdot@[2]{\mathbin{\vcenter{\hbox{\scalebox{#2}{$\m@th#1\bullet$}}}}}
\makeatother

\makeatletter
\DeclareRobustCommand\onedot{\futurelet\@let@token\@onedot}
\def\@onedot{\ifx\@let@token.\else.\null\fi\xspace}

\def\eg{\emph{e.g}\onedot} \def\Eg{\emph{E.g}\onedot}
\def\ie{\emph{i.e}\onedot} \def\Ie{\emph{I.e}\onedot}
\def\cf{\emph{cf}\onedot} \def\Cf{\emph{Cf}\onedot}
\def\etc{\emph{etc}\onedot} \def\vs{\emph{vs}\onedot}
\def\wrt{w.r.t\onedot} \def\dof{d.o.f\onedot} \def\aka{a.k.a\onedot}
\def\etal{\emph{et al}\onedot}
\makeatother

\newcommand{\dit}{\texttt{DiT}\xspace}
\newcommand{\sit}{\texttt{SiT}\xspace}
\newcommand{\ditbtwo}{\texttt{DiT-B/2}\xspace}
\newcommand{\sitbtwo}{\texttt{SiT-B/2}\xspace}
\newcommand{\ditltwo}{\texttt{DiT-L/2}\xspace}
\newcommand{\sitltwo}{\texttt{SiT-L/2}\xspace}
\newcommand{\ditxl}{\texttt{DiT-XL}\xspace}
\newcommand{\ditxltwo}{\texttt{DiT-XL/2}\xspace}
\newcommand{\sitxl}{\texttt{SiT-XL}\xspace}
\newcommand{\sitxltwo}{\texttt{SiT-XL/2}\xspace}
\newcommand{\sdvae}{\texttt{SD-VAE}\xspace}
\newcommand{\sdxlvae}{\texttt{SDXL-VAE}\xspace}
\newcommand{\stablediff}{\texttt{Stable Diffusion}\xspace}
\newcommand{\maskgit}{\texttt{MaskGIT}\xspace}


\newcommand{\our}{\texttt{EQ-VAE}\xspace}
\newcommand{\ours}{Equivariant-VAE\xspace}

\def\DECODE{Decode-from-distributions}

\newcommand{\fnote}[1]{\Authornote{Fernando}{#1}{darkgreen}}
\newcommand{\snote}[1]{\Authornote{Shashank}{#1}{blue}}
\newcommand{\mnote}[1]{\Authornote{Madhur}{#1}{BrickRed}}
\newcommand{\tnote}[1]{\Authornote{Tushant}{#1}{magenta}}

\title{Explicit Codes approaching Generalized Singleton Bound using Expanders}
\author{
Fernando Granha Jeronimo\thanks{{\tt University of Illinois Urbana-Champaign}. {\tt granha@illinois.edu}. }
\and
Tushant Mittal\thanks{{\tt Stanford University}. {\tt tushant@stanford.edu}. Partly supported by NSF grant CCF-2326685.}
\and
Shashank Srivastava\thanks{{\tt DIMACS (Rutgers) \& Institute for Advanced Study}. {\tt shashanks@ias.edu}. Partly supported by the NSF grant CCF-2326685.}
\and
Madhur Tulsiani\thanks{{\tt Toyota Technological Institute at Chicago}. {\tt madhurt@ttic.edu}. Supported by the NSF grant CCF-2326685.} 
}

\allowdisplaybreaks
\begin{document}

\date{}
\maketitle

\thispagestyle{empty}

\begin{abstract}
We construct a new family of explicit codes that are list decodable to capacity and achieve an optimal list size of $O(\nfrac{1}{\eps})$. In contrast to existing explicit constructions of codes achieving list decoding capacity, our arguments do not rely on algebraic structure but utilize simple combinatorial properties of expander graphs. 
\medskip


% Our construction is based on a celebrated distance amplification procedure due to Alon, Edmonds, and Luby [FOCS'95], and our result can be interpreted as a ``local-to-global'' phenomenon for (a slight strengthening of) the generalized Singleton bound. 
%
Our construction is based on a celebrated distance amplification procedure due to Alon, Edmonds, and Luby [FOCS'95], which transforms any high-rate code into one with near-optimal rate-distance tradeoff. We generalize it to show that the same procedure can be used to transform any high-rate code into one that achieves list decoding capacity. Our proof can be interpreted as a ``local-to-global’' phenomenon for (a slight strengthening of) the generalized Singleton bound.
%
Using this construction,  for every $R, \eps \in (0,1)$ and $k \in \N^+$, we obtain an \emph{explicit} family of codes $\mathcal{C} \subseteq \Sigma^n$, with rate $R$ such that,\begin{itemize}
	\item They achieve the $\eps$-relaxed generalized Singleton bound: for any $g \in  \Sigma^n$ and any list $\calH$ of at most $k$ codewords, we have, 
\[ \Ex{h \in \calH}{\Delta(g,h)} ~\geq~ \frac{\abs{\calH}-1}{\abs{\calH}} \cdot (1 - R - \eps) \mper
\]

	\item The alphabet size is a constant depending only on $\eps$ and $k$.
	\item They can be list decoded up to radius $\frac{k-1}{k}(1-R-\eps)$, in time $n^{O_{k,\eps}(1)}$.
\end{itemize}
\medskip

As a corollary of our result, we also obtain the first explicit construction of LDPC codes achieving list decoding capacity, and in fact arbitrarily close to the generalized Singleton bound.


\medskip

% The only known explicit construction of codes achieving the generalized Singleton bound are folded Reed--Solomon code and univariate multiplicity codes [Chen, Zhang 2024], both of which require an alphabet size growing with $n$. 


%the alphabet size $\abs{\Sigma}$ of the code $\mathcal{C}$ can be taken to be $2^{(k^k/\eps)^{O(1)}}$.
%
%For any choice $R$, we give an explicit construction of codes with rate $R$, such that  for every fixed list of code $k$, with a constant alphabet size depending only on $\eps$ and $k$. (stuff about ldpc/ additive/linear). Decoding in polynomial time.
\end{abstract}

% \newpage


% \pagenumbering{roman}
% \tableofcontents
% \clearpage


\newpage
\pagenumbering{arabic}
\setcounter{page}{1}


\section{Introduction}
\label{sec:intro}

\begin{figure*}[tb]
    \centering
    \includegraphics[width=0.848\linewidth]{figs/circuitnn.pdf} 
    \caption{Illustration of differentiable CircuitNN. CircuitNN is designed based on differentiable NAND gates. After DAS is guided by PI and PO pairs of the truth table, CircuitNN can get the precise circuit architecture logic equivalent to the truth table.}
    \label{fig:circuitnn}
\end{figure*}

% 1. Describe the importance of logic synthesis
% 2. Existing Problems
% (a) Neural Architecture Search: Unstable, Predefined Setting, etc.
% (b) Circuit Generation: Probabilistic Model, Logic Equivalence

With the rapid advancement of technology, the scale of integrated circuits (ICs) has expanded exponentially. 
This expansion has introduced significant challenges in chip manufacturing, particularly concerning power and area metrics.
A primary objective in IC design is achieving the same circuit function with fewer transistors, thereby reducing power usage and area occupancy.

Logic synthesis~\cite{hachtel2005logicsynth}, a critical step in electronic design automation (EDA), transforms behavioral-level circuit designs into optimized gate-level circuits, ultimately yielding the final IC layout. 
The primary goal of logic synthesis is to identify the physical implementation with the fewest gates for a given circuit function. 
This task constitutes a challenging NP-hard combinatorial optimization problem. 
Current logic synthesis tools~\cite{brayton2010abc, wolf2013yosys} rely on human-designed heuristics, often leading to sub-optimal outcomes.

Differentiable architecture search (DAS) techniques~\cite{liu2018darts, chu2020darts} offer novel perspectives on addressing challenges in this problem.
Circuit functions can be represented through truth tables, which map binary inputs to their corresponding outputs. 
Truth tables provide a precise representation of input-output relationships, ensuring the design of functionally equivalent circuits.
Inspired by this, researchers~\cite{deepmind2024ai4sys, wang2024tnet} have begun exploring the application of DAS to synthesize circuits directly from truth tables.
Specifically, \citet{deepmind2024ai4sys} proposed CircuitNN, a framework that learns differentiable connection structures with logic gates, enabling the automatic generation of logic circuits from truth tables.
This approach significantly reduces the complexity of traditional circuit generation. 
Building on this, \citet{wang2024tnet} introduced T-Net, a triangle-shaped variant of CircuitNN, incorporating regularization techniques to enhance the efficiency of DAS.

Despite these advancements, several challenges remain. 
The computational complexity of DAS grows quadratically with the number of gates, posing scalability issues.
Although triangle-shaped architecture~\cite{wang2024tnet} partially mitigates this problem, redundancy persists. 
%Additionally, DAS is susceptible to converging to local optima, limiting the ability to search architectures that satisfy the given truth tables~\cite{liu2018darts}. 
%Furthermore, hyperparameters (network depth and layer width) require extensive searches, introducing complexity and prolonging the synthesis process. 
Additionally, DAS is susceptible to converging to local optima~\cite{liu2018darts} and hyperparameters (network depth and layer width) require extensive searches. 
The challenges arise from the vast search space in DAS. 
% Even with predefined settings for CircuitNN, finding a configuration that meets the truth table requires extensive trial and error during the DAS process. 
Intuitively, limiting the search space through predefined parameters (network depth, gates per layer, and connection probabilities) can significantly reduce the complexity.

Recent advances~\cite{openai2023gpt4, abramson2024alphafold3, esser2024sd3, li2024mar} in conditional generative models have demonstrated remarkable performance across language, vision, and graph generation tasks. 
Motivated by these developments, we propose a novel approach to circuit generation that generates preliminary circuit structures to guide DAS in generating refined circuits matching specified truth tables. 
Firstly, we introduce CircuitVQ, a tokenizer with a discrete codebook for circuit tokenization. 
Built upon our Circuit AutoEncoder framework~\cite{hou2022graphmae,li2023maskgae,wu2025mgvga}, CircuitVQ is trained through a circuit reconstruction task. 
Specifically, the CircuitVQ encoder encodes input circuits into discrete tokens using a learnable codebook, while the decoder reconstructs the circuit adjacency matrix based on these tokens.
Subsequently, the CircuitVQ encoder serves as a circuit tokenizer for CircuitAR pretraining, which employs a masked autoregressive modeling paradigm~\cite{chang2022maskgit, li2023mage}. 
In this process, the discrete codes function as supervision signals. 
After training, CircuitAR can generate discrete tokens progressively, which can be decoded into initial circuit structures by the decoder of the CircuitVQ. 
These prior insights can guide DAS in producing refined circuits that match the target truth tables precisely.

Our key contributions can be summarized as follows:
\begin{itemize}
\item We introduce CircuitVQ, a circuit tokenizer that facilitates graph autoregressive modeling for circuit generation, based on our Circuit AutoEncoder framework;
\item Develop CircuitAR, a model trained using masked autoregressive modeling, which generates initial circuit structures conditioned on given truth tables;
\item Propose a refinement framework that integrates differentiable architecture search to produce functionally equivalent circuits guided by target truth tables;
\item Comprehensive experiments demonstrating the scalability and capability emergence of our CircuitAR and the superior performance of the proposed circuit generation approach.
\end{itemize}

% Motivation
% (a) Diffusion (Vision, Graph), Autoregressive (Language, Vision)
% (b) Circuit Generation for Predefined Setting
% (c) Neural Architecture Search for Strict Logic Equivalence

% Contribution
% (a) Circuit Tokenizer (new transformer arch, training strategy)
% (b) CircuitAR (train and gen strategies, post-ar strategy)
% (c) Extensive Evaluation including BitD (Bit Distance) for Scalability



\section{Notation and Preliminaries}\label{sec:prelims}
This section fixes the notation and relevant notions for fair division of goods; the notation specific to division of chores is relegated to Section \ref{sec:chores}. 
 
\paragraph{Fair Division Instances.} A {fair division instance} is given by a tuple $\langle [n], [m], \{v_i\}_{i=1}^n \rangle$, where $[n]=\{1,2,.\dots,n\}$ is the set of $n\in\mathbb{Z}_+$ agents, $[m]=\{1,2, \dots, m\}$ the set of $m\in \mathbb{Z}_+$ indivisible goods, and for each agent $i\in[n]$, the set function $v_i: 2^{[m]} \to \mathbb{R}_+$ denotes the valuation of agent $i$ over subsets of goods. Specifically, $v_i(S) \in \mathbb{R}_+$ denotes the value that agent $i$ derives from the subset $S \subseteq [m]$ of goods. For subsets $S \subseteq [m]$ and $g \in [m]$, we will write $S + g$ to denote the union $S \cup \{ g\}$. 

A valuation $v_i$ is said to be monotone if the inclusion of goods into any subset does not decrease its value, under $v_i$, i.e., $v_i(S)\leq v_i(T)$ for every pair of subsets $S \subseteq T \subseteq[m]$. We will assume throughout that the agents' valuations are monotone and normalized: $v_i(\emptyset)=0$ for all agents $i$. 

We will additionally consider instances with identically ordered valuations. Here, we have an indexing of the $m$ goods, $\{g_1, \ldots g_m\}$, such that for each pair of goods $g_s, g_t$, with index $s < t$, and all agents $i \in [n]$, the inequality $v_i(S + g_s) \geq  v_i(S + g_t)$ holds for each subset $S \subset [m]$ that does not contain $g_s$ and $g_t$; see Example \ref{ex:sqrt-ordered} in Section \ref{subsec:additive-ordered}. 

This work also establishes improved bounds for the specific case of additive valuations. Recall that a valuation $v_i$ is said to be additive if, for every subset $S\subseteq[m]$ of goods, $v_i(S)=\sum_{g\in S} v_i(\{g\})$. We will use the shorthand $v_i(g)$---instead of $v_i(\{g\}) \in \mathbb{R}_+$---to denote agent $i$'s value for any good $g \in [m]$.  


\paragraph{Allocations and Multi-Allocations.} An allocation $\calB=(B_1,B_2,\ldots, B_n)$ of the goods among the $n$ agents is a partition of $[m]$ into $n$ pairwise disjoint subsets $B_1,\ldots, B_n \subseteq [m]$. Here, the subset of goods $B_i$ is assigned to agent $i \in [n]$ and is referred to as $i$'s bundle. In addition, write $\Pi_n([m])$ to denote the collection of all $n$-partitions of $[m]$. Note that for any allocation $\calB =(B_1,\ldots, B_n)$ we have, by definition, $\cup_{i=1}^n B_i = [m]$ and $B_i \cap B_j = \emptyset$, for all $i \neq j$, and hence $\calB \in \Pi_n([m])$.

 
A \textit{multi-allocation} is a tuple $\calA=(A_1,A_2\dots,A_n)$ of $n$ subsets, wherein subset $A_i \subseteq [m]$ denotes the bundle assigned to agent $i$. In contrast to allocations, in a multi-allocation, we do not require that the assigned bundles $A_i$ are pairwise disjoint and that they partition $[m]$.\footnote{Note that $A_i$s are still subsets of goods and not multisets.} Hence, in a multi-allocation, a single good may be present in multiple bundles or even in none. 

Though, when in a multi-allocation $\calA$, each good $g$ is assigned to exactly one agent, we refer to $\calA$ as an {\it exact allocation}; this is to emphasize that the bundles of such a multi-allocation do partition $[m]$. 

We associate with each bundle $A_i \subseteq [m]$ an $m$-dimensional characteristic vector $\rmchar(A_i) \in \{0,1\}^m$. For each good $g\in [m]$, the $g$th component of the characteristic vector---denoted as $\rmchar(A_i)_g$---is equal to one if $g \in A_i$, otherwise the $g$th component is zero. That is, 
\begin{align*}
\rmchar(A_i)_g \coloneqq \begin{cases}
    1 & \text{if } g\in A_i \\
    0 & \text{otherwise}.
\end{cases}
\end{align*}

For any multi-allocation $\calA=(A_1, \ldots, A_n)$, we will use $\chi^\calA \in \mathbb{Z}^m_+$ to denote the vector sum of the characteristic vectors of its bundles, $\chi^\calA \coloneqq \sum_{i=1}^n\rmchar(A_i)$. We will refer to $\chi^\calA$ as the \textit{characteristic vector} of the multi-allocation $\calA$. When there is no ambiguity, we will omit the notational dependence in the superscript and simply write $\chi$ for $\chi^\calA$.

Note that for any good $g\in [m]$ and multi-allocation $\calA$, the $g^{th}$ component of the characteristic vector $\chi^\calA_g$ is equal to the number of bundles in $\calA$ that contain $g$. Conceptually, we think of this setting as one in which $\chi^A_g$ identical copies of the good $g$ are assigned among different agents. 

Write $\ellone{\chi^\calA}$ and $\ellinfty{\chi^\calA}$ to denote the $\ell_1$ and $\ell_\infty$ norm, respectively, of the characteristic vector. Hence,  $\ellone{\chi^\calA} = \sum_{g=1}^m \chi^\calA_g$ and $\ellinfty{\chi^\calA} = \max_{g\in[m]} \chi^\calA_g$. It is relevant to note that $\ellone{\chi^\calA}$ captures the total number of goods, with copies, assigned among the agents,  $\ellone{\chi^\calA} = \sum_{i=1}^n |A_i|$. Further, $\ellinfty{\chi^\calA}$ captures the maximum number of copies of any one good $g$ assigned under $\calA$.

In particular, if $\calA$ is an {\it exact} allocation, then $\chi^\calA$ is equal to the all-ones vector and we have $\ellone{\chi^\calA} =m$ and $\ellinfty{\chi^\calA} =1$.
 
\noindent
The shared-based fairness criterion we study in this work is defined using maximin shares; these shares are defined next.
\begin{definition}[Maximin Share (MMS)]\label{def:mms}
    Given any fair division instance $\langle [n], [m], \{v_i\}_{i=1}^n \rangle$ with goods, the {maximin share}, $\mu_i \in \mathbb{R}_+$, of each agent $i \in [n]$ is defined as 
    \begin{align*}
    \mu_i \coloneqq  \max_{(X_1,\dots, X_n) \in \Pi_n([m])} \ \ \min_{j\in[n]} v_i(X_{j}).
    \end{align*}
Further, for each agent $i$, let $\calM^i=(M^i_1, M^i_2, \ldots, M^i_n) \in \Pi_n([m])$ denote an {MMS-inducing partition}:
\begin{align*}
\calM^i \in \argmax_{(X_1,\dots, X_n) \in \Pi_n([m])} \ \ \min_{j\in[n]} v_i(X_{j})
\end{align*}
\end{definition}

Note that in Definition \ref{def:mms} the maximum is taken over all $n$-partitions of $[m]$. Also, by definition, the partition $\calM^i =(M^i_1, \ldots, M^i_n)$ satisfies $v_i(M^i_j) \geq \mu_i$, for each index $j \in [n]$. 

\paragraph{Fair Multi-Allocations.} A multi-allocation $\calA=(A_1,\dots,A_n)$ is said to be an \emph{MMS multi-allocation} (i.e., it is deemed to be fair) if under it each agent receives a bundle of value at least its maximin share:  $v_i(A_i)\geq \mu_i$ for all agents $i \in [n]$.
 
To establish existential guarantees for MMS multi-allocations $\calA$, we will assume that, for all the agents, we are given the MMS-inducing partitions $\calM^i$, which in turn are guaranteed to exist (see Definition \ref{def:mms}).  

\section{Average radius generalized Singleton bound}\label{sec:avg-singleton}
%
We now prove that AEL codes satisfy the average-radius generalized Singleton bound. 
%
We will actually prove a more general statement involving a received word where some of the coordinates are erasures, marked with a special symbol $\bot$. We will denote such partially erased received words as $\erase{g}$ instead of $g$ to mark the distinction.
%
For $\erase{g} \in (\Sigma \cup \{\bot\})^n$ and $f \in \Sigma^n$, we define the distance using only the non-erasure coordinates as
\[
\Delta(\erase{g},h) ~\defeq~ 
\frac{1}{n} \cdot \abs{\inbraces{i \in [n] ~\mid~ \erase{g}_i \in \Sigma ~~\text{and}~~ h_i \neq \erase{g}_i}} \mper
\]
%
Note that if $s$ denotes the fraction of erasures in $\erase{g} \in (\Sigma \cup \{\bot\})^n$ and $h_1, h_2$ are two codewords from a code with distance $\delta$, then the triangle inequality implies $\Delta(\erase{g},h_1)+\Delta(\erase{g},h_2) \geq \delta - s$. 
%
The following definition generalizes this to any set of (at most) $k$ distinct codewords.

\begin{definition}[Average-radius list decodable with erasures]\label{def:gen_singleton}
A code $\calC \sub \Sigma^n$ is $(\delta, k,\eps)$ {\deffont average-radius
  list decodable with erasures} if for any $\erase{g} \in (\Sigma \cup \{\bot\})^n$ with (say) $s$ fraction of erasures, and for any set of codewords $\calH \sub \calC$ with $|\calH| \leq k$, it holds that 
\[
\sum_{h \in \calH} \Delta(\erase{g},h) ~\geq~ (\abs{\calH} - 1) \cdot (\delta - s - \eps) \mper
\]
\end{definition}
%
Note that the above definition also implies a lower bound on the distance of the code $\calC$, since for
any two distinct codewords $h_1, h_2$, we can take $g = h_1$ to get $\Delta(h_1, h_2) \geq \delta - \eps$.
%
Moreover, a code $\calC$ satisfying the above definition (even with $s=0$) must have the property that an open ball around any $g \in \Sigma^n$ of radius $(\frac{k-1}{k}) \cdot (\delta - \eps)$ contains at most $k-1$ codewords. For $k = 1/\eps$, this yields a list size of $1/\eps$ at radius $\delta - 2\eps$. 
%\tnote{should be open ball here? there can be k codewords with distance exactly $k-1/k(\delta-\eps)$ from $g$}

We show that the stronger property above of being average-radius list decodable \emph{with erasures}
interacts nicely with the AEL construction, which yields a local-to-global result for this property
\ie if the (constant-sized) inner code $\calC_{\inn} \subseteq \Sigma_{\inn}^d$ used in the AEL
construction is $(\delta, k,\eps)$ average-radius list decodable with erasures, then so is the resulting (global) code $\AELC \subseteq (\Sigma_{\inn}^d)^n$.
%

\begin{theorem}\label{thm:main_technical_avg}
%
Let $k\geq 1$ be an integer and let $\eps > 0$. Let $\AELC$ be a
code obtained using the AEL construction using  $(G, \calC_{\out}, \calC_{\inn})$, where $\calC_{\inn}$ is $(\delta_0, k_0, \eps/2)$ average-radius list decodable with erasures, and $G$ is a $(n,d,\lambda)$-expander for $\lambda \leq \frac{\delta_{\out}}{6{k_0}^{k_0}} \cdot \eps$. 
%
Then, $\AELC$ is $(\delta_{0}, k_0,\eps)$ average-radius list decodable with erasures.
\end{theorem}
%
% It follows from \cref{thm:ael_distance} that $\Delta(\AELC) \geq \delta_{\inn} - \lambda/\delta_{\out}
%   \geq \delta_{\inn} - \eps$. We will denote the quantity $\delta_{\inn} - \eps$ simply as $\Delta$ in the rest of the proof.
%
We need to prove that for any collection $\calH = \inbraces{ h_1,\cdots , h_k} \subseteq \AELC$, and any $\erase{g} \in (\Sigma_{\inn}^d \cup \{\bot\})^R$ with fraction of erasures (say) $s$, we must have 
\[
\sum_{h \in \calH} \Delta_R(\erase{g}, h) ~\geq~ (k-1) \cdot (\delta_0 - s - \eps) \mper
\]
We will prove this by induction on the size $k$ of the collection $\calH$. 
%
Note that the case $k=1$ is trivial since distances are non-negative. 
Before proceeding to the induction step, we first need to understand the ``local views" of codewords $h \in \calH$ from each vertex $\ell \in L$. We refer to them as local projections and develop some inequalities for them below.

%For the inductive step, we assume that the conclusion of the theorem is true for any $\erase{g}$ and any $\calH' \subseteq \AELC$ with $\abs{\calH'} \leq k-1$. \tnote{I feel there is little value in having this para here.}
%

\vspace{-5 pt}
\paragraph{Local projections and induced partitions.}
%
To use the inequalities for the local codes $\calC_{\inn}$, we will need to consider the ``local projections" of codewords $h \in \calH$ for each vertex $\ell \in L$, which are codewords in $\calC_{\inn} \subseteq \Sigma_{\inn}^d$. 
%
For a vertex $\ell \in L$, and $h \in \calH$, let $\hl$ be the local codeword in $\calC_{\inn}$ given by the values in $h$ for the edges incident on $\ell$.
%
We know that the codewords in $\calH$ are pairwise distinct, however, this need not be true for their local projections $\hil{1},\cdots \hil{k}$. 
%
We say that a left vertex $\ell$ \textit{induces a partition} $\tau_{\ell} = (\calH_{1,\ell}, \ldots, \calH_{p_{\ell},\ell})$ of $\calH$, wherein $h_i, h_j$ are in the same part if and only if $\hil{i} = \hil{j}$. 
%
Since the number of partitions if bounded by $k^k$, many left vertices must induce the same partition. We will need the additional fact that this must be a non-trivial partition with the number of parts $p \geq 2$.

\begin{claim}\label{lem:type_arg}
For any set of codewords $\calH = \inbraces{ h_1,\cdots , h_k}$, there exists a partition $\tau^*$
of it with at least 2 non-empty parts, and a set $L^*\subseteq L$ such that $\tau_{\ell} = \tau^*$
for all $\ell \in L^*$, and $|L^*| \geq \delta_\out \cdot n / {k^k}$.
\end{claim}
%	
\begin{proof}
%
Let $L'$ be the set of vertices in $L$ which induces a non-trivial partition, \ie for which not all local projections of the codewords in $\calH$ are identical:
\[
L' ~=~ \{\ell \in L \mid \exists\, i,j \in [k], \; \hil{i} \neq \hil{j} \}.
\]
%
Since the codewords in $\calH$ are distinct, we know that $|L'| \geq \delta_{\out} \cdot n$. 
%
The total number of partitions of $\calH$ is at most $k^k$, and thus among the vertices in $L'$, there must be at least $\nfrac{|L'|}{k^k}$ many vertices that induce a common partition of $\calH$, which by definition of $L'$ will be non-trivial.	
%
\end{proof}
	
	
Throughout the proof, we will work with one such fixed partition $\tau^* = (\calH_1, \ldots, \calH_p)$, and the corresponding set $L^*$. Fix an $\ell \in L^*$ and $j\in [p]$. 
%
By definition, the local codeword $\hl$ is the same for all $h\in \calH_j$. We denote this common codeword by $\fjl$ and the local projection of $\erase{g}$ by $\gl \subseteq (\Sigma_{\inn} \cup \{\bot\})^d$, where if $\erase{g}_r = \bot$ for a vertex $r \in R$, we take projection to be $\bot$ on all edges incident on $r$. 
%
Let $\sl$ denote the fraction of erasures ($\bot$ symbols) in $\gl$. We need the following inequality for local projections.
%
\begin{claim}\label{claim:local-bound}
Let $\tau^*$ and $L^*$ be as above, and  $\gl, \sl$ and $\fjl$ be defined as above for each $\ell \in L^*$. Then,
\[
\sum_{j\in [p]} \Delta(\gl, \fjl) ~\geq~ (p-1) \cdot (\delta_{0} - \sl - \eps/2) \mper
\]
\end{claim}
%
\begin{proof}
The claim follows from the fact that the inner code is $(\delta_0, k_0,\eps/2)$ average-radius list decodable with erasures and $p < k \leq k_0$.
\end{proof}

We will also need the following claim regarding the local erasure fractions $\sl$.
%
\begin{claim}[Sampling bound for erasures]\label{claim:sampling_erasure}
For the set $L^*$ defined as above with $\abs{L^*} \geq \delta_{\out} \cdot n / k^k$, and local erasure fractions $\sl$,
\[
		\Ex{\ell\in L^*}{\sl} ~\leq~ s + \frac{\eps}{6} \mper
\]
\end{claim}
%
\begin{proof}
Let $S = \inbraces{r \in R ~|~ \erase{g}_r = \bot}$ be the set of erasure vertices on the right with $\abs{S} = s \cdot n$. 
Then, applying the expander mixing lemma yields
\[
\sum_{\ell \in L^*} \sl \cdot d ~=~ \abs{E(L^*,S)} ~\leq~ \frac{d}{n} \cdot \abs{L^*} \cdot \abs{S} + \lambda \cdot d \cdot n 
\quad\implies\quad
\Ex{\ell\in L^*}{\sl} ~\leq~ s + \lambda \cdot \frac{n}{\abs{L^*}} \mper
\]
Using $\abs{L^*} \geq \delta_{\out} \cdot n / k^k$ and $\lambda \leq (\delta_{\out}/k^k) \cdot (\eps/6)$ then proves the claim. 
\end{proof}
%

\vspace{-5 pt}
\paragraph{Completing the induction step.} 
%
Recall that using the induction hypothesis, we can say that for any $\erase{g'}$ (with say $s'$ fraction of erasures) and any $\calH' \subseteq \AELC$ with $\abs{\calH'} \leq k-1$, we must have $\sum_{h \in \calH'} \Delta_R(\erase{g'},h) \geq (\abs{\calH'}-1) \cdot (\Delta - s' - \eps)$.
%
%
Since the partition $\tau^* = (\calH_1, \ldots, \calH_p)$ is nontrivial, the cardinality $\abs{\calH_j}$ of each part is at most $k-1$ and we can claim by induction with the given $\erase{g}$ that
\[
\forall j \in [p] \qquad \sum_{h \in \calH_j} \Delta_R(\erase{g}, h) \geq (\abs{\calH_j}-1) \cdot (\Delta - s - \eps) \mper
\]
% 
The following key lemma yields a strengthening of this bound by applying the induction hypothesis with a \emph{different} center $\gj$ for each $\calH_j$.
%


\begin{lemma}[Inductive bound on distances]\label{lemma:inductive}
Let the partition $\tau^* = (\calH_1, \ldots, \calH_p)$ and the set $L^*$ be as above, and let the local projections $\gl$ and $\fjl$ be also as defined above. If the code $\AELC$ is $(\delta_0,k-1,\eps)$ average-radius list decodable with erasures, then for every $j\in [p]$,
\[			
\sum_{h\in \calH_j} \Delta_R(\erase{g},h) ~\geq~  (|\calH_j|-1) \cdot \inparen{\Delta - s - \eps} + \Ex{\ell \in L^*}{\Delta(\gl, \fjl)} - \frac{\eps}{6} \mper
\]
\end{lemma}
%
\begin{proof}
%
By definition of $\fjl$, we have that for all $h \in \calH_j$, $\hl = \fjl$ for all $\ell \in L^*$. Thus, if $\gl$ and $\fjl$ differ on edge $(\ell,r)$ with $\erase{g} \neq \bot$, then $r$ is a \emph{common error location} for all $h \in \calH_j$. We define the set 
\[
S_j ~\defeq~ \inbraces{r \in R ~\mid~ \erase{g}_r \neq \bot ~\text{and}~ \exists \ell \in L^*, e = (\ell,r) ~\text{such that}~ (\gl)_e \neq (\fjl)_e } \mper
\]
Let $s_j = \abs{S_j}/n$ and let $\gj$ be obtained from $\erase{g}$ by replacing symbols in $S_j$ by $\bot$. 
%
The total fraction of erasures in $\gj$ is $(\abs{S}+\abs{S_j})/n = s+s_j$. Also, $\Delta_R\parens[\big]{\gj,h} = \Delta_R(\erase{g},h) - s_j$ for all $h \in \calH_j$, since all vertices in $S_j$ are known to be error locations which are erased in $\gj$. Applying the inductive hypothesis with $\gj$ now gives
\begin{align*}
&\sum_{h \in \calH_j} (\Delta_R(\erase{g},h) - s_j)
~=~ \sum_{h \in \calH_j} \Delta_R\parens[\big]{\gj,h} 
~\geq~ (\abs{\calH_j}-1) \cdot (\Delta - s - s_j - \eps) \\
\implies~~
&\sum_{h \in \calH_j} \Delta_R(\erase{g},h) ~\geq~ (\abs{\calH_j}-1) \cdot (\Delta - s - \eps) + s_j \mper
\end{align*}
%
To obtain a bound on $s_j$, we again use expander mixing lemma to deduce
\[
\sum_{\ell \in L^*} \Delta(\gl,\fjl) \cdot d ~=~ \abs{E(L^*, S_j)} ~\leq~ \frac{d}{n} \cdot \abs{L^*} \cdot \abs{S_j} + \lambda \cdot d \cdot n 
\quad \implies \quad
\Ex{\ell \in L^*}{\Delta(\gl, \fjl)} ~\leq~ s_j + \lambda \cdot \frac {n}{\abs{L^*}} 
\mper 
\]
Using $\abs{L^*} \geq \delta_{\out} \cdot n / k^k$ and $\lambda \leq (\delta_{\out}/k^k) \cdot (\eps/6)$ gives the required bound.
%
\end{proof}

%
We can now prove the induction step for the set $\calH = \inbraces{h_1, \ldots, h_k}$.
%
\begin{proof}[Proof of \cref{thm:main_technical_avg}]
%
%
The proof, as mentioned earlier, is by induction on $k$.
%Note that the case $k=1$ is trivial since distances are non-negative. For the inductive step, we assume that the conclusion of the theorem is true for any $\erase{g}$ and any $\calH' \subseteq \AELC$ with $\abs{\calH'} \leq k-1$. \tnote{Added the para from before here.}

Let $L^*$ and $\tau^* = (\calH_1, \ldots, \calH_p)$ be as above. We use the induction hypothesis to apply the bound from \cref{lemma:inductive} to each part $\calH_j$ which has size at most $k-1$ as $\tau^*$ is non-trivial. This gives,
%
\begin{align*}
\sum_{h\in \calH} \Delta_R(\erase{g},h) 
~=~ \sum_{j\in [p]} \sum_{h\in \calH_j} \Delta_R(\erase{g},h) 
&~\geq~ \sum_{j\in [p]} \inparen{(|\calH_j|-1) \cdot \inparen{\Delta - s - \eps} + \Ex{\ell \in L^*}{\Delta(\gl, \fjl)} - \frac{\eps}{6}}\\
&~=~  (k-p) \cdot \inparen{\Delta - s - \eps} + \sum_{j\in [p]} \Ex{\ell \in L^*}{\Delta(\gl, \fjl)} - \frac{p\eps}{6}.
\end{align*}
%
Using local distance inequality from \cref{claim:local-bound} and the sampling bound from \cref{claim:sampling_erasure}, we can bound the second term as
\[
\sum_{j\in [p]} \Ex{\ell \in L^*}{\Delta(\gl, \fjl)} 
~\geq~ (p-1) \cdot \parens[\Big]{\delta_{\inn} - \Ex{\ell \in L^*}{s_{\ell}} - \frac{\eps}{2}}
~\geq~ (p-1) \cdot  \parens[\Big]{\delta_{\inn} - s - \frac{2\eps}{3}} \mper
\]
Combining the above bounds and using $\delta_{\inn} \geq \Delta$ gives,
\begin{align*}
\sum_{h\in \calH} \Delta_R(\erase{g},h) 
&~\geq~ (k-p) \cdot (\Delta - s - \eps) + (p-1) \cdot  \parens[\Big]{\Delta - s - \frac{2\eps}{3}} - \frac{p\eps}{6} \\
&~=~ (k-1) \cdot (\Delta - s - \eps) + \frac{(p-1)\eps}{3} - \frac{p\eps}{6} \mcom
\end{align*}
which completes the proof since $p \geq 2$.
%
\end{proof}
%
As we will prove in \cref{sec:inner-code}, it is easy to observe using known results by Alrabiah, Guruswami and Li~\cite{AGL24} that a random
linear code satisfies \cref{def:gen_singleton} with high probability, and can thus be used as the
inner code $\calC_{\inn}$. 
%
Since the inner code is a constant-sized object, we can search over all linear codes in $(\F_q)^d$
of dimension $\rho \cdot d$ for a given rate $\rho$, and the code $\AELC$ then yields an explicit
construction of codes achieving the generalized Singleton bound.
%
Moreover, if the inner code is required to be fully explicit, it can also be obtained from folded Reed-Solomon codes, using the results by Chen and Zhang~\cite{CZ24}.
%
\begin{corollary}\label{cor:ael_instantiation}
For every $\rho, \eps \in (0,1)$ and $k \in \N$, there exist explicit inner codes $\calC_{\inn}$ and an infinite family of explicit codes $\AELC \subseteq (\F_q^d)^n$ obtained via the AEL construction that satisfy: 
\begin{enumerate}
\item $\rho(\AELC) \geq \rho$.
\item For any $g \in  (\F_q^d)^n$ and any $\calH \subseteq \AELC$ with $\abs{\calH} \leq k$ that
\[
\sum_{h \in \calH} \Delta(g,h) ~\geq~ (\abs{\calH}-1) \cdot (1 - \rho - \eps) \mper
\]
\item The alphabet size $q^d$ of the code $\AELC$ can be taken to be $2^{O(k^{3k}/\eps^9)}$.
\item $\AELC$ is characterized by parity checks of size $O(k^{2k}/\eps^{11})$ over the field $\F_q$.
\end{enumerate}
\end{corollary}
% 
\begin{proof}
Let $d = O(k^{2k}/\eps^8)$ be such that there exist explicit families of $(n,d,\lambda)$-expander graphs (for arbitrarily large $n$) with $\lambda \leq \eps^4/(2^{18} k^k)$. 
%
Let $\calC_{\inn} \subseteq \F_q^d$ be a code given by \cref{cor:random-code}, with rate $\rho_{\inn} = \rho + \eps/4$, which is $(1 - \rho, k, \eps/2)$ average-radius list decodable with erasures. Note that the alphabet size $q$ for $\calC_{\inn}$ can be taken to be $2^{O(k + 1/\eps)}$.
%
Finally, let $\calC_{\out} \subseteq (\F_q^{\rho_{\inn} \cdot d})^n$ be an outer (linear) code with rate $\rho_{\out} = 1 - \eps/4$ and distance (say) $\delta_{\out} = \eps^3/2^{15}$. Explicit families of such codes can be also be obtained (for example) via expander-based Tanner code constructions (see Theorem 11.4.6 and Corollary 11.4.8 in \cite{GRS23}). Using Tanner codes also gives that $\calC_{\out}$ has parity checks of size at most $O(1/\eps^3)$.
%

Given the above parameters, we have 
$\rho(\AELC) ~\geq~ \rho_{\out} \cdot \rho_{\inn} ~=~ (1-\eps/4) \cdot (\rho + \eps/4) ~\geq~ \rho$.
%
Since $\lambda \leq \eps \cdot \delta_{\out}/(6k^k)$ and $\calC_{\inn}$ is $(1-\rho, k, \eps/2)$ average-radius list decodable with erasures, we can use \cref{thm:main_technical_avg} to conclude that $\AELC$ is $(1-\rho, k, \eps)$ average-radius list decodable (with erasures) which yields the second condition. 
%
Since $\calC_{\out}$ has parity checks of $O(1/\eps^3)$, $\calC_{\inn} \subseteq \F_q^d$, and each symbol of $\AELC$ is a function of at most $d$ symbols from $\calC_{\out}$ (encoded via $\calC_{\inn}$), $\AELC$ can be taken to have parity checks of size at most $O(d \cdot (1/\eps^3)) = O(k^{2k}/\eps^{11})$.
%
Finally, we note that the alphabet size of the code $\AELC$ is $q^d = \exp\inparen{O((k + 1/\eps) \cdot (k^{2k}/\eps^8))} = \exp\inparen{k^{3k}/\eps^9}$, which proves the claim. 
\end{proof}

\vspace{-5 pt}
\paragraph{A weaker consequence of \cref{def:gen_singleton}.}
%
We also state the following consequence of \cref{def:gen_singleton}, which still yields a
strengthening of the average distance inequality and the generalized Singleton bound (when
instantiated with the appropriate code), and may be of independent interest. 
%
Note that this statement also yields a corollary of \cref{thm:main_technical_avg} which is simply in
terms of center $g \in \Sigma^n$ with no erasure symbols, and yields an advantage over the distance
inequality, in terms of the error locations which are \emph{common} to all the codewords $h_1, \ldots, h_k$.
%
\begin{lemma}\label{lemma:common-error-bound}
%
Let $\calC \subseteq \Sigma^n$ be $(\delta_0, k_0, \eps)$ average radius list-decodable with
erasures. Then, for any $g \in \Sigma^n$, any $k \leq k_0$ and $h_1, \ldots, h_k \in \calC$, we have
that
\[ \sum_{i \in [k]} \Delta(g,h_i) ~\geq~ (k-1) \cdot (\delta_0 - \eps) + \Ex{r \in [n]}{\prod_{i \in
    [k]}\indi{h_{i,r} \neq g_r}} \mper
\]
%
\end{lemma}
%
\begin{proof}
Define $\erase{g} \in \Sigma^n$ as 
\[
\erase{g}_r ~=~ 
\begin{cases}
\bot & \text{if} ~g_r \neq h_{i,r} ~\forall i \in [k] \\
g_r &\text{otherwise} \mper
\end{cases}
\]
Note that the fraction of erasure symbols is $s = \Ex{r \in [n]}{\prod_{i \in [k]}\indi{h_{i,r} \neq
    g_r}}$ and $\Delta(\erase{g},h_i) = \Delta(g,h) - s$ for all $i \in [k]$. Applying
\cref{def:gen_singleton} with $\erase{g}$ gives
\[
\sum_{i \in [k]} (\Delta(g,h_i) - s) 
~=~ \sum_{i \in [k]} \Delta(\erase{g},h_i) 
~\geq~ (k-1) \cdot (\delta_0 - s - \eps) \mcom
\]
and rearranging proves the claim.
\end{proof}
% 
%
\begin{remark}
A reader might notice that the definition of the $\erase{g}$ is the same as used in the proof of
\cref{lemma:inductive}. 
%
In fact, it is easy to see that the consequence \cref{lemma:common-error-bound} can directly be
proved via induction using the same proof as \cref{thm:main_technical_avg}, which avoids using a
$\erase{g}$ with erasures as part of the induction (although one still needs the list decodability
with erasures for the inner code $\calC_{\inn}$).
%
While we chose to prove the stronger local-to-global statement as \cref{thm:main_technical_avg}
above, for the algorithmic application we will only prove an algorithmic analogue of
\cref{lemma:common-error-bound} to avoid technical issues with keeping track of arbitrary erasure patterns.
%
\end{remark}
%



%!TEX root=main.tex

%%% Local Variables:
%%% mode: latex
%%% TeX-master: "main"
%%% End:


\section{Inner Codes meeting generalized Singleton Bound}\label{sec:inner-code}

In this section, we will look at two constructions of inner codes that are average-radius list decodable with erasures, the property we need to instantiate our construction. These will be a random linear code, and folded Reed-Solomon codes.

In the literature, the property of being average-radius list decodable is usually defined without reference to any erasures (\ie $s=0$ in \cref{def:gen_singleton}). Formally, we say that $\calC \subseteq \Sigma^n$ is $(\delta,k,\eps)$ {\deffont average-radius list decodable} if for all $g \in \Sigma^n$ and all $\calH \subseteq \calC$ with $\abs{\calH} \leq k$, we have
%
\[
\sum_{h \in \calH}{\Delta(g,h)} ~\geq~ \inparen{\abs{\calH}-1} \cdot (\delta - \eps) \mper
\]
%
The other way of looking at erasures is via puncturings, and we will now see that if a code and its puncturings are average-radius list decodable in the above sense, then it is average-radius list decodable with erasures.  


\vspace{-5 pt}
\paragraph{Erasures and Puncturings.}
Let $C \subseteq \F_q^n$ be a code, $S\subseteq [n]$, and denote $s:= \nfrac{|S|}{n}$. Let $\rho =
\rho(C)$ denote the rate of $C$. Define $C_S  \subseteq \F_q^{(1-s)n}$ as the punctured code obtained by removing the coordinates in $S$. 
\begin{claim}\label{claim:puncture}
If for each $S\subseteq [n]$ with  $s \leq 1-\rho$, $C_S$  is $(1-\rho,L, \frac{\varepsilon}{1-s} )$ average-radius list decodable, then $C$ is  $(1-\rho, L, \eps )$ average-radius list decodable with erasures.	
\end{claim}
\begin{proof}
	
Let $\erase{g}$ have erasures in a set $S\subseteq [n]$, and denote by $g_S$, the punctured vectored with these erasure locations removed. Similarly for any list of $L$ codewords $\calH \subseteq C$, denote the punctured list by $\calH_S$. If the code $C_S$ is $(L, \frac{\eps}{1-s} )$ average-radius list decodable (without erasures), then, 

\[
\sum_{h_S \in \calH_S} \Delta(g_S,h_S) ~\geq~ (L - 1) \cdot \parens[\Big]{1 - \rho(C_S) - \frac{\eps}{1-s} }\mper
\]
Observe that if $C$ is a rate $\rho$ code, then $\rho(C_S) \leq \frac{\rho}{1-s}$ (in fact,
$\rho(C_S) = \rho$ if the distance of $C_S$ is greater than 0, but we only need the one-sided inequality). Also $\Delta(\erase{g},h) \geq \Delta(g_S,h_S) \cdot (1-s)$. Multiplying the entire equation by $(1-s)$ and plugging this in, we get,
\[
\sum_{h_S \in \calH_S} \Delta(\erase{g},h) ~\geq~ (L - 1) \cdot \parens[\Big]{1-s - \rho - \eps }\mper \qedhere
\]% which is the requirement in \cref{def:gen_singleton}. 
%Observe that $\Delta(C_S) \geq \frac{\Delta-s}{1-s}$, and $\Delta_S(\erase{g},h) = \Delta_S(g_S,h_S) \cdot (1-s)$. Plugging this in the above equation yields the requirement in \cref{def:gen_singleton}.  
\end{proof}

%. Note that a code $C$ is average-radius list decodable with erasures (\cref{def:gen_singleton}) if $C_S$ is $(L,\eta(1-|S|/n))$\tnote{check this $\eta$ normalization thing} average-radius list decodable (without any erasures) for all subsets of size at most $\Delta n$, the set of such $S$ captures the possible set of erasures. 

\vspace{-5 pt}
\paragraph{Random Linear Codes.} Let $\cG_{n,\rho n,q}$ be the uniform distribution over $n\times \rho n$-matrices over $\F_q$, \ie where each entry of the matrix is picked uniformly at random from $\F_q$. Let $C = \im(G)$ where $G\sim \cG_{n,\rho n,q}$. Then, for any fixed $S\subseteq [n]$,  $C_S = \im(G_S) $ where $G_S\sim \cG_{n-|S|,\rho n,q}$. 

\begin{theorem}[{\cite[Thm. 1.3]{AGL24}}]
Fix an integer $L \geq 1$, $q \geq 2\cdot 2^{10L/\epsilon}$, and $\rho,\epsilon \in (0,1)$.
Then for sufficiently large $n$, a random linear code $C = \im(G)$ where $G\sim \cG_{n,\,\rho n, q}$, is $(1-\rho, L,\epsilon)$ average-radius list decodable with probablity at least $1- \kappa$ , where, 
\[\kappa ~=~  \parens[\Big]{\frac{c_{L,\epsilon}}{q}}^{\lfloor \frac{\epsilon n}{2} \rfloor}, \;\text{ for } c_{L,\epsilon} < 2\cdot 2^{10L/\epsilon}  .\] 
%\[\tau ~\leq~ 2+^{(L+2)n} \cdot {n \choose r} \cdot2^{(L+1)r}\cdot \parens[\bigg]{\frac{L}{q}}^r, \;\text{ for } r = \Big\lfloor \frac{\epsilon n}{2} \Big\rfloor .\] 
\end{theorem}
\begin{proof}
	Their definition of average-radius list decodable gives an inequality for exactly $\ell$ distinct codewords. Thus, we obtain this bound by taking a union over their bounds for $\ell = 1, \cdots, L$. 
\end{proof}


\begin{corollary}\label{cor:random-code}
Let $C$ be a random linear code as generated above for $q \geq 2^{\nfrac{(10L+2)}{\epsilon}}$. Then,
with probability at least $(1-2^{-n/3})$, $C$ is $(1-\rho,L,\eps)$ average-radius list decodable with erasures.\end{corollary}
%\begin{corollary}
%Let $C$ be a random linear code as generated above for $q \gg 2^{10L/\epsilon}\cdot 2^{\frac{3(1-\gamma)}{\epsilon}}$ where $\gamma \in (0,1)$. Then, with high probability, for all $S \subseteq [n]$ of size at most $|S| \leq \gamma n$, the punctured code $C_S$ is $ (L, \epsilon)$ average-radius list decodable. 
%\end{corollary}
\begin{proof}
We will show that with high probability, for all $S \subseteq [n]$ of size at most $|S| \leq (1-\rho)\cdot n$, the punctured code $C_S$ is $ (1-\rho(C_S), L, \epsilon)$ average-radius list decodable. 

For a fixed $S$ of fractional size  $s \leq 1-\rho$, we have that $C_S$ is not
$(1-\rho(C_S),L,\frac{\epsilon}{1-s})$ average-radius list-decodable with probablity at most, 
%
\[
\leq \parens[\Big]{\frac{c_{L,\frac{\epsilon}{1-s}}}{q}}^{\Big\lfloor \frac{ \frac{\epsilon}{1-s}
      (1- s) n}{2}\Big\rfloor} \leq \parens[\Big]{\frac{c_{L,\eps}}{q}}^{\lfloor\frac{ \eps
      n}{2}\rfloor} = \kappa .
\] 
There are at most $2^n$ choices of the subsets $S$, and thus by a union bound, the probability that
all the punctured codes are average-radius list-decodable is at least 
\[
1 - 2^{n} \kappa ~\geq~ 1 - 2^n \cdot 2^{(10L/\eps + 1) \cdot \eps n/2} \cdot q^{-\lfloor\eps
  n/2\rfloor } ~\geq~ 1 - 2^{-n/3} \mcom
\]
for $q \geq 2^{2/\epsilon}\cdot 2^{10L/\epsilon}$ and $n \geq 60L+12$. 	
\end{proof}
%	 where, 
%\[\tau ~\leq~ 2^{(L+2)n} \cdot {n \choose r} \cdot2^{(L+1)r}\cdot \parens[\bigg]{\frac{L}{q}}^r, \;\text{ for } r = \Big\lfloor \frac{\epsilon n}{2} \Big\rfloor .\] 


%Plugging in $\gamma = \Delta$, we get that 

Thus, for a large enough alphabet size, random linear codes are average-radius list-decodable with erasures.


\vspace{-5 pt}
\paragraph{Folded Reed--Solomon Codes.} A recent work of Chen and Zhang~\cite{CZ24} shows that explicit folded Reed--Solomon (RS) codes are also average-radius list-decodable. Let $\F_q[x]$ denote the set of polynomials with $\F_q$-coefficients, and $\F_q^*$ be the multiplicative cyclic group of non-zero elements. 

\begin{definition}[Folded RS Codes]
Fix $n,b >0$, $\rho \in (0,1)$, and $q \geq bn$. Let $\gamma$ be a generator of $\F_q^*$, and pick $\vec{\alpha} = (\alpha_1,\cdots, \alpha_n) \in \F_q^n$. For $f \in \F_q[x]$, let $\Gamma_i = (f(\alpha_i), f(\gamma\alpha_i), \cdots, f(\gamma^{b-1}\alpha_i))$
Then,
\[
\mathrm{FRS}^{b,\gamma}_{n,\rho}(\vec{\alpha}) = \braces{(\Gamma_1, \cdots, \Gamma_n) \mid \deg(f) < \rho b n} \subseteq (\F_q^{b})^n.
\]	
The code is called \textit{appropriate} if $\{ \gamma^i\alpha_j \mid 0\leq i \leq b-1, j \in [n] \}$ has size $bn$, \ie all values are distinct.
\end{definition}

\begin{theorem}[{\cite[Thm. 1.3]{CZ24}}]
For any integer $L \geq 1$ and $\epsilon \in (0,1)$, and $b \geq L/\epsilon$. Then, an appropriate folded Reed-Solomon code, $\mathrm{FRS}^{b,\gamma}_{n,\rho}(\vec{\alpha})$, is $(1- \rho,L,\varepsilon)$ average-radius list decodable where $\rho$ is the rate of this folded code.
 \end{theorem}
% $1- \frac{s\rho}{s-L+1}
%\tnote{Is plugging in $s = L/\epsilon$ okay?}

Note that puncturing the folded Reed--Solomon code is equivalent to the FRS code over a puncturing of $\vec{\alpha}$. Clearly, a puncturing of an appropriate folded Reed--Solomon code is also an appropriate Reed--Solomon code, and thus using \cref{claim:puncture}, one obtains:  

\begin{corollary}
	For any integer $L \geq 1$ and $\epsilon \in (0,1)$, set $b = L/\epsilon$. Then, an appropriate folded Reed-Solomon code, $\mathrm{FRS}^{s,\gamma}_{n,k}(\vec{\alpha})$, is $(1- \rho,L,\varepsilon)$ average-radius list decodable with erasures.
\end{corollary}

%\tnote{Should I change L to $k$? Not doing that as the papers we cite use $k$ for something else and it would be confusing when someone go reads the reference.}



%For a Reed--Solomon code, puncturing is the same as shrinking the evaluation set. Since the notion of \enquote{appropriate evaluation points} is closed under taking subsets, we immediately obtain that puncturings of the Folded Reed--Solomon code are also average-radius list-decodable. Therefore, by \cref{claim:puncture} 
%For a Reed--Solomon code, puncturing is the same as shrinking the evaluation set. Since the notion of \enquote{appropriate evaluation points} is closed under taking subsets, we immediately obtain that puncturings of the Folded Reed--Solomon code are also average-radius list-decodable.  

%!TEX root=main.tex

%%% Local Variables:
%%% mode: latex
%%% TeX-master: "main"
%%% End:


%
The sum-of-squares hierarchy of semidefinite programs (SDPs) provides a family of increasingly
powerful convex relaxations for several optimization problems. 
%
Each ``level" $t$ of the hierarchy is parametrized by a set of constraints corresponding to
polynomials of degree at most $t$ in the optimization variables. While the relaxations in the
hierarchy can be viewed as  semidefinite programs of size $n^{O(t)}$ \cite{BS14, FKP19}, 
it is often convenient to view the solution as a linear operator, called the ``pseudoexpectation" operator.
%
% It is well-known that such constrained pseudoexpectation operators of SoS-degree $t$ can be described as solutions to semidefinite programs of size $n^{O(t)}$ \cite{BS14, Laurent09}. This hierarchy of semidefinite programs for increasing $t$ is known as the SoS hierarchy.


%
%
%\fnote{Someone familiar with SoS will likely want to skip most of this paragraph (only use it as a reference as needed) and jump to the AEL part of the preliminaries.}
%\vspace{-5 pt}
%
\paragraph{Pseudoexpectations}
 
%
Let $t$ be a positive even integer and fix an alphabet $\Sigma$ of size $s$. Let $\zee = \{Z_{i,j}\}_{i\in[m],j\in\Sigma}$ be a collection of variables and $\R[\zee]^{\leq t}$ be the vector space of polynomials of degree at most $t$ in the variables $\zee$ (including the constants).


\begin{definition}[Constrained Pseudoexpectations]\label{def:constraints_on_sos}
%	 
Let $\calS = \inbraces{f_1 = 0, \ldots, f_m = 0, g_1 \geq 0, \ldots, g_r \geq 0}$ be a system of
polynomial constraints in $\zee$, with each polynomial in $\calS$ of degree at most $t$. We say $\tildeEx{\cdot}$ is a pseudoexpectation operator of SoS-degree $t$, over the variables $\zee$  respecting $\calS$, if it is a linear operator $ \tildeEx{\cdot}: \R[\zee]^{\leq t} \rightarrow \R$ such that:
	%
	\begin{enumerate}
	\item $\tildeEx{1} = 1$.
    \item $\tildeEx{p^2} \geq 0$ if $p$ is a polynomial in $\zee = \{Z_{i,j}\}_{i\in [m],j\in \Sigma}$ of degree $\leq t/2$.
	\item $\tildeEx{p \cdot f_i} = 0$,  $\forall\, i \in [m]$ and $\forall\, p$ such that $\deg(p \cdot f_i) \leq t$.
	\item $\tildeEx{p^2 \cdot \prod_{i \in S} g_i} \geq 0$, $\forall\, S \subseteq [r]$ and $\forall\, p$ such that $\deg(p^2\cdot \prod_{i \in S} g_i) \leq t$.
	\end{enumerate}
\end{definition}
~


%\tnote{Suggestion -- directly define the constrained version. Attempting to make the connection with assignments a bit more explicit below.}

% An SoS solution of degree $t$, or a pseudoexpectation of SoS-degree $t$, over the variables $\zee$ is represented by a linear operator $ \tildeEx{\cdot}: \R[\zee]^{\leq t} \rightarrow \R$ such that:
%%
%\vspace{-5 pt}
%%
%\begin{enumerate}[(i)]
%    \item $\tildeEx{1} = 1$.
%    \item $\tildeEx{p^2} \geq 0$ if $p$ is a polynomial in $\zee = \{Z_{i,j}\}_{i\in [m],j\in [q]}$ of degree $\leq t/2$.
%\end{enumerate}
%%
%\vspace{-5 pt}
%%
%\tnote{Is the note needed? It is reiterating that it is a linear operator}
% Note that linearity implies $\tildeEx{p_1} + \tildeEx{p_2} = \tildeEx{p_1+p_2}$ and $\tildeEx{c\cdot
%  p_1} = c \cdot \tildeEx{p_1}$ for $c\in \R$, for $p_1, p_2 \in \R[\zee]^{\leq t}$.
%%
%This also allows for a succinct representation of $\tildeEx{\cdot}$ using any basis for $\R[\zee]^{\leq t}$.
%
%\tnote{ }

% to $m$ variables in alphabet $[q]$.

Let $\mu$ be a distribution over the set of assignments, $\Sigma^m$.  Define the following collection of random variables,  \[\zee = \braces[\big]{ \, Z_{i,j}  = \indi{ i \mapsto j}\, \mid \, i\in[m],\, j\in\Sigma } .\] Then, setting $\tildeEx{p(\zee)} = \Ex{\mu}{p(\zee)} $ for any polynomial $p(\cdot)$ defines an (unconstrained) pseudoexpectation operator. However, the reverse is not true when $t < m$, and there can be degree-$t$ pseudoexpectations that do not correspond to any genuine distribution, $\mu$. Therefore, the set of all pseudoexpectations should be seen as a relaxation for the set of all possible distributions over such assignments. The upshot of this relaxation is that it is possible to optimize over the set. Under certain conditions on the bit-complexity of solutions~\cite{OD16, RW17:sos}, one can optimize over the set of degree-$t$ pseudoexpectations in time $m^{O(t)}$ via SDPs.

%   setting the true expectation operator under this distribution is also a . 
%
%
%Given an assignment $f: [m] \rightarrow [q]$, the operator $\tildeEx{Z_{i_1,j_1}\cdots Z_{i_k,j_k}} = \indi{ j_t = f(i_t) \; \forall\, t }$
%\[
%f: [m] \rightarrow [q] ~\mapsto~  \tildeEx{Z_{i_1,j_1}\cdots Z_{i_k,j_k}} = \indi{ j_t = f(i_t) \; \forall\, t }
%\]
%can be seen as a pseudoexpectation that assigns the value $1$ to a monomial consistent with $f$ and $0$ otherwise. 

%
%This can be extended via linearity to all
%polynomials, and then by convexity of the constraints to all distributions over assignments.
%


 
%
%We next define what it means for pseudoexpectations to satisfy some problem-specific constraints.
%\[
% \left\{ \substack{\text{ True expectations} \\
%\text{ \ie for an assignment } f: [m] \rightarrow [q] \\\tildeEx{Z_{} \cdots Z_{}} =  } \right\} \subset  \{ \text{ SoS Pseudoexpectations }  \}
%\]

% Indeed, any assignment $f: [m] \rightarrow [q]$ has a corresponding pseudoexpectation described below.

% Let $z^{(f)}_{i,j} = \indi{f(i) = j}$, where $\indi{\cdot}$ is the $0/1$ indicator function. Then the pseudoexpectation corresponding to assignment $f$ is given by
% \begin{align*}
% 	\PExp^{(f)}[{p(\zee)}] = p\inparen{\inbraces{z^f_{i,j}}_{i\in[m],j\in[q]}}
% \end{align*}

% for every polynomial $p$ of degree at most $t$.

% As can be verified easily, the set of pseudoexpectations is convex, and so we can also extend this correspondence to distributions over assignments in a natural way. If $\PExp^{(\calD)}[{\cdot}]$ is the pseudoexpectation corresponding to a distribution $\calD$ over assignments, it holds that
% \[
% 	\PExp^{(\calD)}[{p(\zee)}] = \Ex{f\sim \calD}{p\inparen{\inbraces{z^f_{i,j}}_{i\in[m],j\in[q]}}}
% \]
% which explains the term pseudo-expectation.

% However, pseudoexpectations only exist for low-degree polynomials, and there can be pseudoexpectation operators that do not correspond to any genuine distribution over assignments. The main reason for working with pseudoexpectations instead of genuine distributions is that the set of SoS-degree $t$ can be optimized over in time $m^{O(t)}$ via SDPs.

% As a relaxation, pseudoexpectations are used in efficient algorithm design when coupled with
% suitable rounding algorithms. 
% %
% For such applications (including ours), it is important to look at relaxations that satisfy certain problem-specific constraints. We define next what it means for pseudoexpectations to satisfy constraints.

% \paragraph{Constrained Pseudoexpectations}
% 

%
%
\paragraph{Local constraints and local functions.}
%
Any constraint that involves at most $k$ variables from $\zee$, with $k\leq t$, can be written as a degree-$k$ polynomial, and such constraints may be enforced into the SoS solution.
%
% \paragraph{Canonical usage}
%
In particular, we will always consider the following canonical constraints on the variables $\zee$.
\ifnum\confversion=1
\begin{align*}
&Z_{i,j}^2 = Z_{i,j},\ \forall i\in[m],j\in[s] \\
\text{and} \quad &\sum_j Z_{i,j} = 1,\ \forall i\in[m] \mper
\end{align*}
\else
\[
Z_{i,j}^2 = Z_{i,j},\ \forall \,i\in[m],j\in\Sigma
\quad \text{and} \quad 
\sum_j Z_{i,j} = 1,\ \forall\, i\in[m] \mper
\]
\fi
% As shown in the previous section, an assignment $f:[m]\rightarrow [q]$ is encoded using $m$ characteristic vectors, and so we wish to impose the following local constraints on the variables $\zee$:
%
% \begin{enumerate}[(i)]
% 	\item $Z_{i,j}^2 = Z_{i,j},\ \forall i\in[m],j\in[q]$.
% 	\item $\sum_j Z_{i,j} = 1,\ \forall i\in[m]$.
% \end{enumerate}
%
% These constraints are enforced as described in \cref{def:constraints_on_sos}, and we will henceforth not explicitly mention it. \snote{See Madhur's comment.}
%
We will also consider additional constraints and corresponding polynomials, defined by ``local" functions. For any $f\in \Sigma^m$ and $M\sub [m]$, we use $f_M$ to denote the restriction $f|_M$, and $f_i$ to denote $f_{\{i\}}$ for convenience.
%
\begin{definition}[$k$-local function]
	A function $\mu: \Sigma^m \rightarrow \R$ is called $k$-local if there is a set $M\subseteq [m]$ of size $k$ such that $\mu(f)$ only depends on $\inbraces{f(i)}_{i\in M}$, or equivalently, $\mu(f)$ only depends on $f|_M$.
	
	If $\mu$ is $k$-local, we abuse notation to also use $\mu: \Sigma^M \rightarrow \R$ with $\mu(\alpha) = \mu(f)$ for any $f$ such that $f|_M=\alpha$. It will be clear from the input to the function $\mu$ whether we are using $\mu$ as a function on $\Sigma^m$ or $\Sigma^M$.
\end{definition}

Let $\mu:\Sigma^m\rightarrow \R$ be a $k$-local function that depends on coordinates $M\subseteq [m]$ with $|M|=k$. Then $\mu$ can be written as a degree-$k$ polynomial $P_{\mu}$ in $\zee$:
\[
	P_{\mu}(\zee) = \sum_{\alpha \in \Sigma^M} \parens[\Big]{\mu(\alpha) \cdot\prod_{i\in M} Z_{i,\alpha_i}}
\]

% To see how $p_{\mu}$ is related to the $k$-local function $\mu$, observe that $p_{\mu}\inparen{\zee = \inbraces{z^{(f)}_{i,j}}} = \mu(f)$.

With some abuse of notation, we let $\mu(\zee)$ denote $P_{\mu}(\zee)$. We will use such $k$-local
functions inside $\tildeEx{\cdot}$ freely without worrying about their polynomial
representation. For example, $\tildeEx{ \indi{\zee_{i} \neq j}}$ denotes $\tildeEx{ 1- Z_{i,j}}$. Likewise, sometimes we will say we set $\zee_i = j$ to mean that we set $Z_{i,j} = 1$ and $Z_{i,j'} = 0$ for all $j'\in \Sigma \backslash \{j\}$.

% $\tildeEx{\cdot}$ operator applied to the polynomial corresponding to the $1$-local function $\mu: [q]^m \rightarrow \R$ which is defined as: $\mu(f)$ is $1$ if $f_i = 0$ and $\mu(f) = 0$ otherwise.
% %
The notion of $k$-local functions can also be extended from real-valued functions to vector-valued functions straightforwardly.

\begin{definition}[Vector-valued local functions]
A function $\mu: \Sigma^m \rightarrow \R^N$ is $k$-local if the $N$ real valued functions corresponding to the $N$ coordinates are all $k$-local. Note that these different coordinate functions may depend on different sets of variables, as long as the number is at most $k$ for each of the functions.
\end{definition}
%
%
\paragraph{Local distribution view of SoS}

It will be convenient to use a shorthand for the function $\indi{\zee_A = \alpha}$, and we will use $\zee_{A,\alpha}$. Likewise, we use $\zee_{i,j}$ as a shorthand for the function $\indi{\zee_i = j}$. That is, henceforth,
\ifnum\confversion=1
\begin{align*}
	&\tildeEx{\zee_{A,\alpha}} = \tildeEx{\indi{\zee_A = \alpha}} = \tildeEx{ \prod_{a\in A}
                             Z_{a,\alpha_a}}
                             \\
	\text{and } \quad & \tildeEx{\zee_{i,j}} = \tildeEx{\indi{\zee_i = j}} = \tildeEx{ Z_{i,j}}.
\end{align*}
\else
\begin{align*}
	\tildeEx{\zee_{A,\alpha}} ~=~ \tildeEx{\indi{\zee_A = \alpha}} ~=~ \tildeE\brackets[\Big]{\prod_{a\in A}
                      Z_{a,\alpha_a}
                                    }
\qquad \text{and} \qquad
	\tildeEx{\zee_{i,j}} ~=~ \tildeEx{\indi{\zee_i = j}} = \tildeEx{ Z_{i,j}}
\end{align*}
\fi

% Note that for any degree-$t$ pseudoexpectation operator $\tildeEx{\cdot}$ with $t\geq 2$,
% \[
% 	\sum_{j\in [q]} \tildeEx{\zee_{i,j}} = \sum_{j} \tildeEx{Z_{i,j}} = \tildeEx{\sum_j Z_{i,j}}= 1
% \qquad \text{and} \qquad
% 	\tildeEx{\zee_{i,j}} = \tildeEx{Z_{i,j}} = \tildeEx{Z_{i,j}^2} \geq 0
% \]
% Thus, the real values $\inbraces{\tildeEx{\zee_{i,j}}}_{j\in [q]}$ define a distribution over $[q]$
% , which we will sometimes call local distribution for $\zee_i$. 

% In fact, this argument can be extended to define local distributions for $\zee_S$ for $|S|\leq
% t/2$. Let $S \subseteq [m]$ be such that $|S|=k\leq t/2$,
Note that for any $A \subseteq [m]$ with $\abs*{A} = k \leq t/2$,
\ifnum\confversion=1
\begin{gather*}
	\sum_{ \alpha \in \Sigma^{k}} \tildeEx{\zee_{A,\alpha}} = 
% \sum_{ \alpha \in [q]^{k}} \tildeEx{ \prod_{s\in S} Z_{s,\alpha_s}} 
 \tildeEx{ \prod_{a\in A} \inparen{ \sum_{j\in \Sigma} Z_{a,j}} } = 1
\\
	\tildeEx{\zee_{A,\alpha}} = \tildeEx{ \prod_{a\in A} Z_{a,\alpha_a}} = \tildeEx{ \prod_{a\in
            A} Z^2_{a,\alpha_a}} \geq 0 \mper
\end{gather*}
\else
\[
	\sum_{ \alpha \in \Sigma^{k}} \tildeEx{\zee_{A,\alpha}} = 
% \sum_{ \alpha \in [q]^{k}} \tildeEx{ \prod_{s\in S} Z_{s,\alpha_s}} 
 \tildeE\brackets[\bigg]{ \prod_{a\in A} \parens[\bigg]{\sum_{j\in \Sigma} Z_{a,j} } } = 1
\qquad \text{and} \qquad
	\tildeEx{\zee_{A,\alpha}} = \tildeE\brackets[\bigg]{ \prod_{a\in A} Z_{a,\alpha_a}} = \tildeE\brackets[\bigg]{ \prod_{a\in
            A} Z^2_{a,\alpha_a}} \geq 0 \mper
\]
\fi

Thus, the values $\inbraces{\tildeEx{\zee_{A, \alpha}}}_{\alpha\in \Sigma^A}$ define a distribution
over $\Sigma^k$. We call this the local distribution for $\zee_A$, or simply for $A$.
% which we think of as the local distribution for $\zee_S$.
%
% Given a distribution $\calD$ over assignments in $[q]^m$, the local distribution induced by $\PExp^{(\calD)}[\cdot]$ on $\zee_S$ is precisely the marginal distribution induced by $\calD$ for the set $S$. In this view, degree-$t$ pseudoexpectations give us consistent marginal distributions over sets of size at most $t/2$ that may not correspond to any global distribution, and allow us to optimize over this set in time $m^{\calO(t)}$.
%
Let $\mu: \Sigma^m \rightarrow\R$ be a $k$-local function for $k\leq t/2$, depending on $M \subseteq
[m]$. Then,
%
\ifnum\confversion=1
\begin{align*}
	\tildeEx{\mu(\zee)} 
~=~& \tildeEx{\sum_{\alpha\in \Sigma^M} \inparen{\mu(\alpha) \cdot\prod_{i\in M} Z_{i,\alpha_i}}}\\
~=~& \sum_{\alpha\in \Sigma^M} \mu(\alpha) \cdot \tildeEx{\prod_{i\in M} Z_{i,\alpha_i}}\\
~=~& \sum_{\alpha\in \Sigma^M} \mu(\alpha) \cdot \tildeEx{\zee_{M,\alpha}}
\end{align*}
\else
\begin{align*}
	\tildeEx{\mu(\zee)} 
~=~ \tildeE\brackets[\bigg]{\sum_{\alpha\in \Sigma^M} \parens[\bigg]{\mu(\alpha) \cdot\prod_{i\in M} Z_{i,\alpha_i}}}
~=~ \sum_{\alpha\in \Sigma^M} \mu(\alpha) \cdot \tildeE\brackets[\Big]{\prod_{i\in M} Z_{i,\alpha_i}}
~=~ \sum_{\alpha\in \Sigma^M} \mu(\alpha) \cdot \tildeEx{\zee_{M,\alpha}}
\end{align*}
\fi
%

That is, $\tildeEx{\mu(\zee)}$ can be seen as the expected value of the function $\mu$ under the local distribution for $M$.

\begin{claim}\label{claim:sos_domination}
	Let $\tildeEx{\cdot}$ be a degree-$t$ pseudoexpectation. For $k \leq t/2$, let $\mu_1,\mu_2$
        be two $k$-local functions on $\Sigma^m$, depending on the same set of coordinates $M$, and
        $\mu_1(\alpha) \leq \mu_2(\alpha) ~~\forall \alpha \in \Sigma^M$. Then $\tildeEx{\mu_1(\zee)} \leq \tildeEx{\mu_2(\zee)}$.
%
 % Suppose that for any $\alpha\in [q]^M$, $\mu_1(\alpha) \leq \mu_2(\alpha)$. Then
 %        \[
 %        	\tildeEx{\mu_1(\zee)} \leq \tildeEx{\mu_2(\zee)}
 %        \]
\end{claim}

\begin{proof}
Let $\calD_M$ be the local distribution induced by $\tildeEx{\cdot}$ for $\zee_M$. Then
$\tildeEx{\mu_1(\zee)} = \Ex{\alpha \sim \calD_M}{\mu_1(\alpha)}$, and $\tildeEx{\mu_2(\zee)} =
\Ex{\alpha\sim \calD_M}{\mu_2(\alpha)}$, which implies $\tildeEx{\mu_1(\zee)} \leq \tildeEx{\mu_2(\zee)}$.
%
% Since $\mu_1(\alpha) \leq \mu_2(\alpha)$ for every $\alpha\in [q]^M$, 
% \[
% 	\Ex{\alpha \sim \calD_M}{\mu_1(\alpha)} \leq \Ex{\alpha\sim \calD_M}{\mu_2(\alpha)}
% \]
% and so,
% \[
% 	\tildeEx{\mu_1(\zee)} \leq \tildeEx{\mu_2(\zee)}
% \]
\end{proof}
%
The previous claim allows us to replace any local function inside $\tildeEx{\cdot}$ by another local function that dominates it. We will make extensive use of this fact.
%
%
\vspace{-5 pt}
%\tnote{Move covariance, conditioning stuff to appendix 1? It seems to disrupt the flow and we only need to cite it in section 6 for Lemma 6.1?}
\paragraph{Covariance for SoS solutions}
Given two sets $S,T \sub [m]$ with $|S|,|T|\leq k/4$, we can define the covariance between indicator random variables of $\zee_S$ and $\zee_T$ taking values $\alpha$ and $\beta$ respectively, according to the local distribution over $S \cup T$. This is formalized in the next definition.
\begin{definition}
Let $\tildeEx{\cdot}$ be a pseudodistribution operator of SoS-degree-$t$, and $S,T$ are two sets of
size at most $t/4$, and $\alpha\in \Sigma^S$, $\beta\in \Sigma^T$, we define the pseudo-covariance and
pseudo-variance,
%
\ifnum\confversion=1
\small
\begin{gather*}
\tildecov(\zee_{S,\alpha},\zee_{T,\beta}) 
= \tildeEx{ \zee_{S,\alpha} \cdot \zee_{T,\beta} } - \tildeEx{\zee_{S,\alpha}} \tildeEx{\zee_{T,\beta}} \\	
\tildeVar{\zee_{S,\alpha}} ~=~ \tildecov(\zee_{S,\alpha},\zee_{S,\alpha})
\end{gather*}
\normalsize
\else
\begin{align*}
\tildecov(\zee_{S,\alpha},\zee_{T,\beta}) 
~&=~ \tildeEx{ \zee_{S,\alpha} \cdot \zee_{T,\beta} } - \tildeEx{\zee_{S,\alpha}} \,\tildeEx{\zee_{T,\beta}}\\ 		
\tildeVar{\zee_{S,\alpha}} ~&=~ \tildecov(\zee_{S,\alpha},\zee_{S,\alpha})
\end{align*}
\fi
%
The above definition is extended to pseudo-covariance and pseudo-variance for pairs of sets $S,T$, 
as the sum of absolute value of pseudo-covariance for all pairs $\alpha,\beta$ :
%
\ifnum\confversion=1
\begin{gather*}
\tildecov(\zee_S,\zee_T) 
~=~ \sum_{\alpha\in \Sigma^S \atop \beta\in \Sigma^T} \abs*{ \tildecov(\zee_{S,\alpha},\zee_{T,\beta}) } \\
\tildeVar{\zee_S} ~=~ \sum_{\alpha\in \Sigma^S} \abs*{ \tildeVar{\zee_{S,\alpha} } }
\end{gather*}
\else
\begin{align*}
\tildecov(\zee_S,\zee_T) 
~&=~ \sum_{\alpha\in \Sigma^S, \beta\in \Sigma^T} \abs*{ \tildecov(\zee_{S,\alpha},\zee_{T,\beta}) }\\[3pt]
\tildeVar{\zee_S} ~&=~ \sum_{\alpha\in \Sigma^S} \abs*{ \tildeVar{\zee_{S,\alpha} } }
\end{align*}
\fi
%
\end{definition}

% These definitions can be extended to pseudocovariances and pseudo-variances for pairs of sets $S,T$ with $|S|,|T|\leq t/4$ as the sum of absolute value of pseudocovariance for all pairs $\alpha,\beta$.
% \begin{definition}
% Let $\tildeEx{\cdot}$ be a pseudodistribution operator of SoS-degree-$t$, and $S,T$ are two sets of size at most $t/4$, we define the pseudo-covariance between $\zee_S$ and $\zee_T$ as,
% 	\begin{align*}
% 		\tildecov(\zee_S,\zee_T) = \sum_{\alpha\in [q]^S,\beta\in [q]^T} \abs*{ \tildecov(\zee_{S,\alpha},\zee_{T,\beta}) }
% 	\end{align*}
% We also define the analogous pseudovariance as,
% 	\begin{align*}
% 		\tildeVar{\zee_S} = \sum_{\alpha\in [q]^S} \abs*{ \tildeVar{\zee_{S,\alpha} } }
% 	\end{align*}
% \end{definition}
%
We will need the fact that $\tildeVar{\zee_S}$ is bounded above by 1, since,
%
\ifnum\confversion=1
\begin{align*}
\tildeVar{\zee_S} 
&~=~ \sum_{\alpha} \abs*{\tildeVar{\zee_{S,\alpha}}} \\
&~=~ \sum_{\alpha}\inparen{
          \tildeEx{\zee_{S,\alpha}^2} - \tildeEx{\zee_{S,\alpha}}^2} \\
&~\leq~ \sum_{\alpha} \tildeEx{\zee_{S,\alpha}^2} \\
&~=~ \sum_{\alpha} \tildeEx{\zee_{S,\alpha}} 
~=~ 1.
\end{align*}
\else
\[
\tildeVar{\zee_S} 
~=~ \sum_{\alpha} \abs*{\tildeVar{\zee_{S,\alpha}}} 
~=~ \sum_{\alpha}\inparen{
          \tildeEx{\zee_{S,\alpha}^2} - \tildeEx{\zee_{S,\alpha}}^2} 
~\leq~ \sum_{\alpha} \tildeEx{\zee_{S,\alpha}^2} 
~=~ \sum_{\alpha} \tildeEx{\zee_{S,\alpha}} 
~=~ 1.
\]
\fi

% \begin{claim}\quad	$\tildeVar{\zee_S} \leq 1$.
% \end{claim}
% %
% \begin{proof}
% \begin{align*}
% 	\tildeVar{\zee_S} &= \sum_{\alpha} \abs*{\tildeVar{\zee_{S,\alpha}}} = \sum_{\alpha}\inparen{ \tildeEx{\zee_{S,\alpha}^2} - \tildeEx{\zee_{S,\alpha}}^2} \leq \sum_{\alpha} \tildeEx{\zee_{S,\alpha}^2} = \sum_{\alpha} \tildeEx{\zee_{S,\alpha}} = 1
% \end{align*}
% \end{proof}
%
\vspace{-5 pt}
\paragraph{Conditioning SoS solutions.}
%
We will also make use of conditioned pseudoexpectation operators, which may be defined in a way
similar to usual conditioning for true expectation operators, as long as the event we condition on
is local. 
%
The conditioned SoS solution is of a smaller degree but continues to respect the constraints that the original solution respects.

\begin{definition}[Conditioned SoS Solution] Let $F \subseteq \Sigma^m$ be subset (to be thought of as an event) such that $\one_F:\Sigma^m \rightarrow \{0,1\}$ is a $k$-local function. Then for every $t>2k$, we can condition a pseudoexpectation operator of SoS-degree $t$ on $F$ to obtain a new conditioned pseudoexpectation operator $\condPE{\cdot}{F}$ of SoS-degree $t-2k$, as long as $\tildeEx{\one^2_F(\zee)}>0$. The conditioned SoS solution is given by
\[
	\condPE{ p(\zee)}{F(\zee) } \defeq \frac{\tildeEx{p(\zee) \cdot \one^2_{F}(\zee)}}{\tildeEx{\one^2_{F}(\zee)}}
\]
where $p$ is any polynomial of degree at most $t-2k$.
\end{definition}

%\snote{Mention that constraints \emph{respected} by SoS remain true after conditioning.}

We can also define pseudocovariances and pseudo-variances for the conditioned SoS solutions.
\begin{definition}[Pseudocovariance]
	Let $F\sub \Sigma^m$ be an event such that $\one_F$ is $k$-local, and let $\tildeEx{\cdot}$ be a pseudoexpectation operator of degree $t$, with $t>2k$. Let $S,T$ be two sets of size at most $\frac{t-2k}{2}$ each. Then the pseudocovariance between $\zee_{S,\alpha}$ and $\zee_{T,\beta}$ for the solution conditioned on event $F$ is defined as,
\ifnum\confversion=1
\begin{multline*}
\tildecov(\zee_{S,\alpha},\zee_{T,\beta} \vert F) = \\
\tildeEx{ \zee_{S,\alpha} \zee_{T,\beta} \vert F} - \tildeEx{\zee_{S,\alpha} \vert F} \tildeEx{\zee_{T,\beta} \vert F}
\end{multline*}
\else
\begin{align*}
\tildecov(\zee_{S,\alpha},\zee_{T,\beta} \vert F) 
~=~ \tildeEx{ \zee_{S,\alpha} \zee_{T,\beta} \vert F} - \tildeEx{\zee_{S,\alpha} \vert F} ~ \tildeEx{\zee_{T,\beta} \vert F}
\end{align*}
\fi
\end{definition}

We also define the pseudocovariance between $\zee_{S,\alpha}$ and $\zee_{T,\beta}$ after
conditioning on a random assignment for some $\zee_V$ with $V\sub [m]$. 
%
Note that here the random assignment for $\zee_V$ is chosen according to the local distribution for
the set $V$.

\begin{definition}[Pseudocovariance for conditioned pseudoexpectation operators]
\ifnum\confversion=1
\begin{multline*}
\tildecov(\zee_{S,\alpha},\zee_{T,\beta} \vert \zee_V) 
= \\ \sum_{\gamma \in \Sigma^V}   \tildecov(\zee_{S,\alpha},\zee_{T,\beta} \vert \zee_V = \gamma) \cdot \tildeEx{\zee_{V,\gamma}}
	\end{multline*}
\else
\begin{align*}
\tildecov(\zee_{S,\alpha},\zee_{T,\beta} \vert \zee_V) 
~=~ \sum_{\gamma \in \Sigma^V}   \tildecov(\zee_{S,\alpha},\zee_{T,\beta} \vert \zee_V = \gamma) \cdot \tildeEx{\zee_{V,\gamma}}
\end{align*}
\fi
\end{definition}

And we likewise define $\tildeVar{\zee_{S,\alpha} \vert \zee_V}$, $\tildecov(\zee_S, \zee_T \vert \zee_V)$ and $\tildeVar{\zee_S \vert \zee_V}$.






%%% Local Variables:
%%% mode: latex
%%% TeX-master: "main"
%%% End:


\section{Decoding using SoS up to the generalized Singleton bound}

Until now, we have proved that when the inner and outer codes as well as the graph used in AEL amplification are suitably chosen, then the list of codewords around an arbitrary center (corrupted codeword) is of small size. 
In this section, we describe a polynomial time algorithm that takes as input the corrupted codeword, and outputs this list. 


This algorithm is based on the Sum-of-Squares (SoS) hierarchy of semidefinite programs, which gives a systematic way of tightening convex relaxations.  SoS has been used before for decoding algorithms for codes constructed using spectral expanders in \cite{AJQST20, JQST20, RR23, JST23}. 
Among these, \cite{JST23} used the SoS hierarchy to give a list decoding algorithm for AEL up to the Johnson bound, yielding rate $R$ codes efficiently decodable up to $1-\sqrt{R}-\eps$ for any $\eps>0$. 
We will heavily rely on their framework but will improve the decoding radius to $1-R-\eps$ by proving an SoS analog of the generalized Singleton bound for appropriately instantiated AEL amplification.

%
Before going into the proof, we describe additional preliminaries for the SoS hierarchy. Readers familiar with the general terms and concepts can skip ahead to \cref{sec:sos_proof} where we define a specific SoS relaxation for the AEL code, and prove an SoS analog of the generalized Singleton bound. Finally, \cref{sec:sos_algo} describes the decoding algorithm in detail.

%\snote{Just wrote some text. Some of this will go to intro, some will get scattered. Mix above with the next para to produce something reasonable.}

\subsection{Additional Preliminaries: Sum-of-Squares Hierarchy}\label{sec:sos_prelims}

The sum-of-squares hierarchy of semidefinite programs (SDPs) provides a family of increasingly
powerful convex relaxations for several optimization problems. 
%
Each ``level" $t$ of the hierarchy is parametrized by a set of constraints corresponding to
polynomials of degree at most $t$ in the optimization variables. While the relaxations in the
hierarchy can be viewed as  semidefinite programs of size $n^{O(t)}$ \cite{BS14, FKP19}, 
it is often convenient to view the solution as a linear operator, called the ``pseudoexpectation" operator.
%
% It is well-known that such constrained pseudoexpectation operators of SoS-degree $t$ can be described as solutions to semidefinite programs of size $n^{O(t)}$ \cite{BS14, Laurent09}. This hierarchy of semidefinite programs for increasing $t$ is known as the SoS hierarchy.


%
%
%\fnote{Someone familiar with SoS will likely want to skip most of this paragraph (only use it as a reference as needed) and jump to the AEL part of the preliminaries.}
%\vspace{-5 pt}
%
\paragraph{Pseudoexpectations}
 
%
Let $t$ be a positive even integer and fix an alphabet $\Sigma$ of size $s$. Let $\zee = \{Z_{i,j}\}_{i\in[m],j\in\Sigma}$ be a collection of variables and $\R[\zee]^{\leq t}$ be the vector space of polynomials of degree at most $t$ in the variables $\zee$ (including the constants).


\begin{definition}[Constrained Pseudoexpectations]\label{def:constraints_on_sos}
%	 
Let $\calS = \inbraces{f_1 = 0, \ldots, f_m = 0, g_1 \geq 0, \ldots, g_r \geq 0}$ be a system of
polynomial constraints in $\zee$, with each polynomial in $\calS$ of degree at most $t$. We say $\tildeEx{\cdot}$ is a pseudoexpectation operator of SoS-degree $t$, over the variables $\zee$  respecting $\calS$, if it is a linear operator $ \tildeEx{\cdot}: \R[\zee]^{\leq t} \rightarrow \R$ such that:
	%
	\begin{enumerate}
	\item $\tildeEx{1} = 1$.
    \item $\tildeEx{p^2} \geq 0$ if $p$ is a polynomial in $\zee = \{Z_{i,j}\}_{i\in [m],j\in \Sigma}$ of degree $\leq t/2$.
	\item $\tildeEx{p \cdot f_i} = 0$,  $\forall\, i \in [m]$ and $\forall\, p$ such that $\deg(p \cdot f_i) \leq t$.
	\item $\tildeEx{p^2 \cdot \prod_{i \in S} g_i} \geq 0$, $\forall\, S \subseteq [r]$ and $\forall\, p$ such that $\deg(p^2\cdot \prod_{i \in S} g_i) \leq t$.
	\end{enumerate}
\end{definition}
~


%\tnote{Suggestion -- directly define the constrained version. Attempting to make the connection with assignments a bit more explicit below.}

% An SoS solution of degree $t$, or a pseudoexpectation of SoS-degree $t$, over the variables $\zee$ is represented by a linear operator $ \tildeEx{\cdot}: \R[\zee]^{\leq t} \rightarrow \R$ such that:
%%
%\vspace{-5 pt}
%%
%\begin{enumerate}[(i)]
%    \item $\tildeEx{1} = 1$.
%    \item $\tildeEx{p^2} \geq 0$ if $p$ is a polynomial in $\zee = \{Z_{i,j}\}_{i\in [m],j\in [q]}$ of degree $\leq t/2$.
%\end{enumerate}
%%
%\vspace{-5 pt}
%%
%\tnote{Is the note needed? It is reiterating that it is a linear operator}
% Note that linearity implies $\tildeEx{p_1} + \tildeEx{p_2} = \tildeEx{p_1+p_2}$ and $\tildeEx{c\cdot
%  p_1} = c \cdot \tildeEx{p_1}$ for $c\in \R$, for $p_1, p_2 \in \R[\zee]^{\leq t}$.
%%
%This also allows for a succinct representation of $\tildeEx{\cdot}$ using any basis for $\R[\zee]^{\leq t}$.
%
%\tnote{ }

% to $m$ variables in alphabet $[q]$.

Let $\mu$ be a distribution over the set of assignments, $\Sigma^m$.  Define the following collection of random variables,  \[\zee = \braces[\big]{ \, Z_{i,j}  = \indi{ i \mapsto j}\, \mid \, i\in[m],\, j\in\Sigma } .\] Then, setting $\tildeEx{p(\zee)} = \Ex{\mu}{p(\zee)} $ for any polynomial $p(\cdot)$ defines an (unconstrained) pseudoexpectation operator. However, the reverse is not true when $t < m$, and there can be degree-$t$ pseudoexpectations that do not correspond to any genuine distribution, $\mu$. Therefore, the set of all pseudoexpectations should be seen as a relaxation for the set of all possible distributions over such assignments. The upshot of this relaxation is that it is possible to optimize over the set. Under certain conditions on the bit-complexity of solutions~\cite{OD16, RW17:sos}, one can optimize over the set of degree-$t$ pseudoexpectations in time $m^{O(t)}$ via SDPs.

%   setting the true expectation operator under this distribution is also a . 
%
%
%Given an assignment $f: [m] \rightarrow [q]$, the operator $\tildeEx{Z_{i_1,j_1}\cdots Z_{i_k,j_k}} = \indi{ j_t = f(i_t) \; \forall\, t }$
%\[
%f: [m] \rightarrow [q] ~\mapsto~  \tildeEx{Z_{i_1,j_1}\cdots Z_{i_k,j_k}} = \indi{ j_t = f(i_t) \; \forall\, t }
%\]
%can be seen as a pseudoexpectation that assigns the value $1$ to a monomial consistent with $f$ and $0$ otherwise. 

%
%This can be extended via linearity to all
%polynomials, and then by convexity of the constraints to all distributions over assignments.
%


 
%
%We next define what it means for pseudoexpectations to satisfy some problem-specific constraints.
%\[
% \left\{ \substack{\text{ True expectations} \\
%\text{ \ie for an assignment } f: [m] \rightarrow [q] \\\tildeEx{Z_{} \cdots Z_{}} =  } \right\} \subset  \{ \text{ SoS Pseudoexpectations }  \}
%\]

% Indeed, any assignment $f: [m] \rightarrow [q]$ has a corresponding pseudoexpectation described below.

% Let $z^{(f)}_{i,j} = \indi{f(i) = j}$, where $\indi{\cdot}$ is the $0/1$ indicator function. Then the pseudoexpectation corresponding to assignment $f$ is given by
% \begin{align*}
% 	\PExp^{(f)}[{p(\zee)}] = p\inparen{\inbraces{z^f_{i,j}}_{i\in[m],j\in[q]}}
% \end{align*}

% for every polynomial $p$ of degree at most $t$.

% As can be verified easily, the set of pseudoexpectations is convex, and so we can also extend this correspondence to distributions over assignments in a natural way. If $\PExp^{(\calD)}[{\cdot}]$ is the pseudoexpectation corresponding to a distribution $\calD$ over assignments, it holds that
% \[
% 	\PExp^{(\calD)}[{p(\zee)}] = \Ex{f\sim \calD}{p\inparen{\inbraces{z^f_{i,j}}_{i\in[m],j\in[q]}}}
% \]
% which explains the term pseudo-expectation.

% However, pseudoexpectations only exist for low-degree polynomials, and there can be pseudoexpectation operators that do not correspond to any genuine distribution over assignments. The main reason for working with pseudoexpectations instead of genuine distributions is that the set of SoS-degree $t$ can be optimized over in time $m^{O(t)}$ via SDPs.

% As a relaxation, pseudoexpectations are used in efficient algorithm design when coupled with
% suitable rounding algorithms. 
% %
% For such applications (including ours), it is important to look at relaxations that satisfy certain problem-specific constraints. We define next what it means for pseudoexpectations to satisfy constraints.

% \paragraph{Constrained Pseudoexpectations}
% 

%
%
\paragraph{Local constraints and local functions.}
%
Any constraint that involves at most $k$ variables from $\zee$, with $k\leq t$, can be written as a degree-$k$ polynomial, and such constraints may be enforced into the SoS solution.
%
% \paragraph{Canonical usage}
%
In particular, we will always consider the following canonical constraints on the variables $\zee$.
\ifnum\confversion=1
\begin{align*}
&Z_{i,j}^2 = Z_{i,j},\ \forall i\in[m],j\in[s] \\
\text{and} \quad &\sum_j Z_{i,j} = 1,\ \forall i\in[m] \mper
\end{align*}
\else
\[
Z_{i,j}^2 = Z_{i,j},\ \forall \,i\in[m],j\in\Sigma
\quad \text{and} \quad 
\sum_j Z_{i,j} = 1,\ \forall\, i\in[m] \mper
\]
\fi
% As shown in the previous section, an assignment $f:[m]\rightarrow [q]$ is encoded using $m$ characteristic vectors, and so we wish to impose the following local constraints on the variables $\zee$:
%
% \begin{enumerate}[(i)]
% 	\item $Z_{i,j}^2 = Z_{i,j},\ \forall i\in[m],j\in[q]$.
% 	\item $\sum_j Z_{i,j} = 1,\ \forall i\in[m]$.
% \end{enumerate}
%
% These constraints are enforced as described in \cref{def:constraints_on_sos}, and we will henceforth not explicitly mention it. \snote{See Madhur's comment.}
%
We will also consider additional constraints and corresponding polynomials, defined by ``local" functions. For any $f\in \Sigma^m$ and $M\sub [m]$, we use $f_M$ to denote the restriction $f|_M$, and $f_i$ to denote $f_{\{i\}}$ for convenience.
%
\begin{definition}[$k$-local function]
	A function $\mu: \Sigma^m \rightarrow \R$ is called $k$-local if there is a set $M\subseteq [m]$ of size $k$ such that $\mu(f)$ only depends on $\inbraces{f(i)}_{i\in M}$, or equivalently, $\mu(f)$ only depends on $f|_M$.
	
	If $\mu$ is $k$-local, we abuse notation to also use $\mu: \Sigma^M \rightarrow \R$ with $\mu(\alpha) = \mu(f)$ for any $f$ such that $f|_M=\alpha$. It will be clear from the input to the function $\mu$ whether we are using $\mu$ as a function on $\Sigma^m$ or $\Sigma^M$.
\end{definition}

Let $\mu:\Sigma^m\rightarrow \R$ be a $k$-local function that depends on coordinates $M\subseteq [m]$ with $|M|=k$. Then $\mu$ can be written as a degree-$k$ polynomial $P_{\mu}$ in $\zee$:
\[
	P_{\mu}(\zee) = \sum_{\alpha \in \Sigma^M} \parens[\Big]{\mu(\alpha) \cdot\prod_{i\in M} Z_{i,\alpha_i}}
\]

% To see how $p_{\mu}$ is related to the $k$-local function $\mu$, observe that $p_{\mu}\inparen{\zee = \inbraces{z^{(f)}_{i,j}}} = \mu(f)$.

With some abuse of notation, we let $\mu(\zee)$ denote $P_{\mu}(\zee)$. We will use such $k$-local
functions inside $\tildeEx{\cdot}$ freely without worrying about their polynomial
representation. For example, $\tildeEx{ \indi{\zee_{i} \neq j}}$ denotes $\tildeEx{ 1- Z_{i,j}}$. Likewise, sometimes we will say we set $\zee_i = j$ to mean that we set $Z_{i,j} = 1$ and $Z_{i,j'} = 0$ for all $j'\in \Sigma \backslash \{j\}$.

% $\tildeEx{\cdot}$ operator applied to the polynomial corresponding to the $1$-local function $\mu: [q]^m \rightarrow \R$ which is defined as: $\mu(f)$ is $1$ if $f_i = 0$ and $\mu(f) = 0$ otherwise.
% %
The notion of $k$-local functions can also be extended from real-valued functions to vector-valued functions straightforwardly.

\begin{definition}[Vector-valued local functions]
A function $\mu: \Sigma^m \rightarrow \R^N$ is $k$-local if the $N$ real valued functions corresponding to the $N$ coordinates are all $k$-local. Note that these different coordinate functions may depend on different sets of variables, as long as the number is at most $k$ for each of the functions.
\end{definition}
%
%
\paragraph{Local distribution view of SoS}

It will be convenient to use a shorthand for the function $\indi{\zee_A = \alpha}$, and we will use $\zee_{A,\alpha}$. Likewise, we use $\zee_{i,j}$ as a shorthand for the function $\indi{\zee_i = j}$. That is, henceforth,
\ifnum\confversion=1
\begin{align*}
	&\tildeEx{\zee_{A,\alpha}} = \tildeEx{\indi{\zee_A = \alpha}} = \tildeEx{ \prod_{a\in A}
                             Z_{a,\alpha_a}}
                             \\
	\text{and } \quad & \tildeEx{\zee_{i,j}} = \tildeEx{\indi{\zee_i = j}} = \tildeEx{ Z_{i,j}}.
\end{align*}
\else
\begin{align*}
	\tildeEx{\zee_{A,\alpha}} ~=~ \tildeEx{\indi{\zee_A = \alpha}} ~=~ \tildeE\brackets[\Big]{\prod_{a\in A}
                      Z_{a,\alpha_a}
                                    }
\qquad \text{and} \qquad
	\tildeEx{\zee_{i,j}} ~=~ \tildeEx{\indi{\zee_i = j}} = \tildeEx{ Z_{i,j}}
\end{align*}
\fi

% Note that for any degree-$t$ pseudoexpectation operator $\tildeEx{\cdot}$ with $t\geq 2$,
% \[
% 	\sum_{j\in [q]} \tildeEx{\zee_{i,j}} = \sum_{j} \tildeEx{Z_{i,j}} = \tildeEx{\sum_j Z_{i,j}}= 1
% \qquad \text{and} \qquad
% 	\tildeEx{\zee_{i,j}} = \tildeEx{Z_{i,j}} = \tildeEx{Z_{i,j}^2} \geq 0
% \]
% Thus, the real values $\inbraces{\tildeEx{\zee_{i,j}}}_{j\in [q]}$ define a distribution over $[q]$
% , which we will sometimes call local distribution for $\zee_i$. 

% In fact, this argument can be extended to define local distributions for $\zee_S$ for $|S|\leq
% t/2$. Let $S \subseteq [m]$ be such that $|S|=k\leq t/2$,
Note that for any $A \subseteq [m]$ with $\abs*{A} = k \leq t/2$,
\ifnum\confversion=1
\begin{gather*}
	\sum_{ \alpha \in \Sigma^{k}} \tildeEx{\zee_{A,\alpha}} = 
% \sum_{ \alpha \in [q]^{k}} \tildeEx{ \prod_{s\in S} Z_{s,\alpha_s}} 
 \tildeEx{ \prod_{a\in A} \inparen{ \sum_{j\in \Sigma} Z_{a,j}} } = 1
\\
	\tildeEx{\zee_{A,\alpha}} = \tildeEx{ \prod_{a\in A} Z_{a,\alpha_a}} = \tildeEx{ \prod_{a\in
            A} Z^2_{a,\alpha_a}} \geq 0 \mper
\end{gather*}
\else
\[
	\sum_{ \alpha \in \Sigma^{k}} \tildeEx{\zee_{A,\alpha}} = 
% \sum_{ \alpha \in [q]^{k}} \tildeEx{ \prod_{s\in S} Z_{s,\alpha_s}} 
 \tildeE\brackets[\bigg]{ \prod_{a\in A} \parens[\bigg]{\sum_{j\in \Sigma} Z_{a,j} } } = 1
\qquad \text{and} \qquad
	\tildeEx{\zee_{A,\alpha}} = \tildeE\brackets[\bigg]{ \prod_{a\in A} Z_{a,\alpha_a}} = \tildeE\brackets[\bigg]{ \prod_{a\in
            A} Z^2_{a,\alpha_a}} \geq 0 \mper
\]
\fi

Thus, the values $\inbraces{\tildeEx{\zee_{A, \alpha}}}_{\alpha\in \Sigma^A}$ define a distribution
over $\Sigma^k$. We call this the local distribution for $\zee_A$, or simply for $A$.
% which we think of as the local distribution for $\zee_S$.
%
% Given a distribution $\calD$ over assignments in $[q]^m$, the local distribution induced by $\PExp^{(\calD)}[\cdot]$ on $\zee_S$ is precisely the marginal distribution induced by $\calD$ for the set $S$. In this view, degree-$t$ pseudoexpectations give us consistent marginal distributions over sets of size at most $t/2$ that may not correspond to any global distribution, and allow us to optimize over this set in time $m^{\calO(t)}$.
%
Let $\mu: \Sigma^m \rightarrow\R$ be a $k$-local function for $k\leq t/2$, depending on $M \subseteq
[m]$. Then,
%
\ifnum\confversion=1
\begin{align*}
	\tildeEx{\mu(\zee)} 
~=~& \tildeEx{\sum_{\alpha\in \Sigma^M} \inparen{\mu(\alpha) \cdot\prod_{i\in M} Z_{i,\alpha_i}}}\\
~=~& \sum_{\alpha\in \Sigma^M} \mu(\alpha) \cdot \tildeEx{\prod_{i\in M} Z_{i,\alpha_i}}\\
~=~& \sum_{\alpha\in \Sigma^M} \mu(\alpha) \cdot \tildeEx{\zee_{M,\alpha}}
\end{align*}
\else
\begin{align*}
	\tildeEx{\mu(\zee)} 
~=~ \tildeE\brackets[\bigg]{\sum_{\alpha\in \Sigma^M} \parens[\bigg]{\mu(\alpha) \cdot\prod_{i\in M} Z_{i,\alpha_i}}}
~=~ \sum_{\alpha\in \Sigma^M} \mu(\alpha) \cdot \tildeE\brackets[\Big]{\prod_{i\in M} Z_{i,\alpha_i}}
~=~ \sum_{\alpha\in \Sigma^M} \mu(\alpha) \cdot \tildeEx{\zee_{M,\alpha}}
\end{align*}
\fi
%

That is, $\tildeEx{\mu(\zee)}$ can be seen as the expected value of the function $\mu$ under the local distribution for $M$.

\begin{claim}\label{claim:sos_domination}
	Let $\tildeEx{\cdot}$ be a degree-$t$ pseudoexpectation. For $k \leq t/2$, let $\mu_1,\mu_2$
        be two $k$-local functions on $\Sigma^m$, depending on the same set of coordinates $M$, and
        $\mu_1(\alpha) \leq \mu_2(\alpha) ~~\forall \alpha \in \Sigma^M$. Then $\tildeEx{\mu_1(\zee)} \leq \tildeEx{\mu_2(\zee)}$.
%
 % Suppose that for any $\alpha\in [q]^M$, $\mu_1(\alpha) \leq \mu_2(\alpha)$. Then
 %        \[
 %        	\tildeEx{\mu_1(\zee)} \leq \tildeEx{\mu_2(\zee)}
 %        \]
\end{claim}

\begin{proof}
Let $\calD_M$ be the local distribution induced by $\tildeEx{\cdot}$ for $\zee_M$. Then
$\tildeEx{\mu_1(\zee)} = \Ex{\alpha \sim \calD_M}{\mu_1(\alpha)}$, and $\tildeEx{\mu_2(\zee)} =
\Ex{\alpha\sim \calD_M}{\mu_2(\alpha)}$, which implies $\tildeEx{\mu_1(\zee)} \leq \tildeEx{\mu_2(\zee)}$.
%
% Since $\mu_1(\alpha) \leq \mu_2(\alpha)$ for every $\alpha\in [q]^M$, 
% \[
% 	\Ex{\alpha \sim \calD_M}{\mu_1(\alpha)} \leq \Ex{\alpha\sim \calD_M}{\mu_2(\alpha)}
% \]
% and so,
% \[
% 	\tildeEx{\mu_1(\zee)} \leq \tildeEx{\mu_2(\zee)}
% \]
\end{proof}
%
The previous claim allows us to replace any local function inside $\tildeEx{\cdot}$ by another local function that dominates it. We will make extensive use of this fact.
%
%
\vspace{-5 pt}
%\tnote{Move covariance, conditioning stuff to appendix 1? It seems to disrupt the flow and we only need to cite it in section 6 for Lemma 6.1?}
\paragraph{Covariance for SoS solutions}
Given two sets $S,T \sub [m]$ with $|S|,|T|\leq k/4$, we can define the covariance between indicator random variables of $\zee_S$ and $\zee_T$ taking values $\alpha$ and $\beta$ respectively, according to the local distribution over $S \cup T$. This is formalized in the next definition.
\begin{definition}
Let $\tildeEx{\cdot}$ be a pseudodistribution operator of SoS-degree-$t$, and $S,T$ are two sets of
size at most $t/4$, and $\alpha\in \Sigma^S$, $\beta\in \Sigma^T$, we define the pseudo-covariance and
pseudo-variance,
%
\ifnum\confversion=1
\small
\begin{gather*}
\tildecov(\zee_{S,\alpha},\zee_{T,\beta}) 
= \tildeEx{ \zee_{S,\alpha} \cdot \zee_{T,\beta} } - \tildeEx{\zee_{S,\alpha}} \tildeEx{\zee_{T,\beta}} \\	
\tildeVar{\zee_{S,\alpha}} ~=~ \tildecov(\zee_{S,\alpha},\zee_{S,\alpha})
\end{gather*}
\normalsize
\else
\begin{align*}
\tildecov(\zee_{S,\alpha},\zee_{T,\beta}) 
~&=~ \tildeEx{ \zee_{S,\alpha} \cdot \zee_{T,\beta} } - \tildeEx{\zee_{S,\alpha}} \,\tildeEx{\zee_{T,\beta}}\\ 		
\tildeVar{\zee_{S,\alpha}} ~&=~ \tildecov(\zee_{S,\alpha},\zee_{S,\alpha})
\end{align*}
\fi
%
The above definition is extended to pseudo-covariance and pseudo-variance for pairs of sets $S,T$, 
as the sum of absolute value of pseudo-covariance for all pairs $\alpha,\beta$ :
%
\ifnum\confversion=1
\begin{gather*}
\tildecov(\zee_S,\zee_T) 
~=~ \sum_{\alpha\in \Sigma^S \atop \beta\in \Sigma^T} \abs*{ \tildecov(\zee_{S,\alpha},\zee_{T,\beta}) } \\
\tildeVar{\zee_S} ~=~ \sum_{\alpha\in \Sigma^S} \abs*{ \tildeVar{\zee_{S,\alpha} } }
\end{gather*}
\else
\begin{align*}
\tildecov(\zee_S,\zee_T) 
~&=~ \sum_{\alpha\in \Sigma^S, \beta\in \Sigma^T} \abs*{ \tildecov(\zee_{S,\alpha},\zee_{T,\beta}) }\\[3pt]
\tildeVar{\zee_S} ~&=~ \sum_{\alpha\in \Sigma^S} \abs*{ \tildeVar{\zee_{S,\alpha} } }
\end{align*}
\fi
%
\end{definition}

% These definitions can be extended to pseudocovariances and pseudo-variances for pairs of sets $S,T$ with $|S|,|T|\leq t/4$ as the sum of absolute value of pseudocovariance for all pairs $\alpha,\beta$.
% \begin{definition}
% Let $\tildeEx{\cdot}$ be a pseudodistribution operator of SoS-degree-$t$, and $S,T$ are two sets of size at most $t/4$, we define the pseudo-covariance between $\zee_S$ and $\zee_T$ as,
% 	\begin{align*}
% 		\tildecov(\zee_S,\zee_T) = \sum_{\alpha\in [q]^S,\beta\in [q]^T} \abs*{ \tildecov(\zee_{S,\alpha},\zee_{T,\beta}) }
% 	\end{align*}
% We also define the analogous pseudovariance as,
% 	\begin{align*}
% 		\tildeVar{\zee_S} = \sum_{\alpha\in [q]^S} \abs*{ \tildeVar{\zee_{S,\alpha} } }
% 	\end{align*}
% \end{definition}
%
We will need the fact that $\tildeVar{\zee_S}$ is bounded above by 1, since,
%
\ifnum\confversion=1
\begin{align*}
\tildeVar{\zee_S} 
&~=~ \sum_{\alpha} \abs*{\tildeVar{\zee_{S,\alpha}}} \\
&~=~ \sum_{\alpha}\inparen{
          \tildeEx{\zee_{S,\alpha}^2} - \tildeEx{\zee_{S,\alpha}}^2} \\
&~\leq~ \sum_{\alpha} \tildeEx{\zee_{S,\alpha}^2} \\
&~=~ \sum_{\alpha} \tildeEx{\zee_{S,\alpha}} 
~=~ 1.
\end{align*}
\else
\[
\tildeVar{\zee_S} 
~=~ \sum_{\alpha} \abs*{\tildeVar{\zee_{S,\alpha}}} 
~=~ \sum_{\alpha}\inparen{
          \tildeEx{\zee_{S,\alpha}^2} - \tildeEx{\zee_{S,\alpha}}^2} 
~\leq~ \sum_{\alpha} \tildeEx{\zee_{S,\alpha}^2} 
~=~ \sum_{\alpha} \tildeEx{\zee_{S,\alpha}} 
~=~ 1.
\]
\fi

% \begin{claim}\quad	$\tildeVar{\zee_S} \leq 1$.
% \end{claim}
% %
% \begin{proof}
% \begin{align*}
% 	\tildeVar{\zee_S} &= \sum_{\alpha} \abs*{\tildeVar{\zee_{S,\alpha}}} = \sum_{\alpha}\inparen{ \tildeEx{\zee_{S,\alpha}^2} - \tildeEx{\zee_{S,\alpha}}^2} \leq \sum_{\alpha} \tildeEx{\zee_{S,\alpha}^2} = \sum_{\alpha} \tildeEx{\zee_{S,\alpha}} = 1
% \end{align*}
% \end{proof}
%
\vspace{-5 pt}
\paragraph{Conditioning SoS solutions.}
%
We will also make use of conditioned pseudoexpectation operators, which may be defined in a way
similar to usual conditioning for true expectation operators, as long as the event we condition on
is local. 
%
The conditioned SoS solution is of a smaller degree but continues to respect the constraints that the original solution respects.

\begin{definition}[Conditioned SoS Solution] Let $F \subseteq \Sigma^m$ be subset (to be thought of as an event) such that $\one_F:\Sigma^m \rightarrow \{0,1\}$ is a $k$-local function. Then for every $t>2k$, we can condition a pseudoexpectation operator of SoS-degree $t$ on $F$ to obtain a new conditioned pseudoexpectation operator $\condPE{\cdot}{F}$ of SoS-degree $t-2k$, as long as $\tildeEx{\one^2_F(\zee)}>0$. The conditioned SoS solution is given by
\[
	\condPE{ p(\zee)}{F(\zee) } \defeq \frac{\tildeEx{p(\zee) \cdot \one^2_{F}(\zee)}}{\tildeEx{\one^2_{F}(\zee)}}
\]
where $p$ is any polynomial of degree at most $t-2k$.
\end{definition}

%\snote{Mention that constraints \emph{respected} by SoS remain true after conditioning.}

We can also define pseudocovariances and pseudo-variances for the conditioned SoS solutions.
\begin{definition}[Pseudocovariance]
	Let $F\sub \Sigma^m$ be an event such that $\one_F$ is $k$-local, and let $\tildeEx{\cdot}$ be a pseudoexpectation operator of degree $t$, with $t>2k$. Let $S,T$ be two sets of size at most $\frac{t-2k}{2}$ each. Then the pseudocovariance between $\zee_{S,\alpha}$ and $\zee_{T,\beta}$ for the solution conditioned on event $F$ is defined as,
\ifnum\confversion=1
\begin{multline*}
\tildecov(\zee_{S,\alpha},\zee_{T,\beta} \vert F) = \\
\tildeEx{ \zee_{S,\alpha} \zee_{T,\beta} \vert F} - \tildeEx{\zee_{S,\alpha} \vert F} \tildeEx{\zee_{T,\beta} \vert F}
\end{multline*}
\else
\begin{align*}
\tildecov(\zee_{S,\alpha},\zee_{T,\beta} \vert F) 
~=~ \tildeEx{ \zee_{S,\alpha} \zee_{T,\beta} \vert F} - \tildeEx{\zee_{S,\alpha} \vert F} ~ \tildeEx{\zee_{T,\beta} \vert F}
\end{align*}
\fi
\end{definition}

We also define the pseudocovariance between $\zee_{S,\alpha}$ and $\zee_{T,\beta}$ after
conditioning on a random assignment for some $\zee_V$ with $V\sub [m]$. 
%
Note that here the random assignment for $\zee_V$ is chosen according to the local distribution for
the set $V$.

\begin{definition}[Pseudocovariance for conditioned pseudoexpectation operators]
\ifnum\confversion=1
\begin{multline*}
\tildecov(\zee_{S,\alpha},\zee_{T,\beta} \vert \zee_V) 
= \\ \sum_{\gamma \in \Sigma^V}   \tildecov(\zee_{S,\alpha},\zee_{T,\beta} \vert \zee_V = \gamma) \cdot \tildeEx{\zee_{V,\gamma}}
	\end{multline*}
\else
\begin{align*}
\tildecov(\zee_{S,\alpha},\zee_{T,\beta} \vert \zee_V) 
~=~ \sum_{\gamma \in \Sigma^V}   \tildecov(\zee_{S,\alpha},\zee_{T,\beta} \vert \zee_V = \gamma) \cdot \tildeEx{\zee_{V,\gamma}}
\end{align*}
\fi
\end{definition}

And we likewise define $\tildeVar{\zee_{S,\alpha} \vert \zee_V}$, $\tildecov(\zee_S, \zee_T \vert \zee_V)$ and $\tildeVar{\zee_S \vert \zee_V}$.






%%% Local Variables:
%%% mode: latex
%%% TeX-master: "main"
%%% End:


\subsection{Pseudocodewords achieve the generalized Singleton bound}\label{sec:sos_proof}

We will show that the proof of list decodability established for integral codewords can be extended to SoS relaxations of these codewords, under certain average-case low covariance conditions. 
First, we describe what an SoS relaxation of an AEL codeword looks like. This relaxation will be such that true codewords are always feasible solutions, but there may be other feasible solutions. These feasible solutions are therefore called \textit{pseudocodewords}.

The main goal will be to show that for some large constant $t$ depending on $k$ and $\eps$, but independent of $n$, the degree-$t$ SoS relaxation is tight enough to decode up to radius $\frac{k-1}{k}(1-R-\eps)$. We will actually work with $k$-tuples of pseudocodewords, and the show that this tuple---after some random conditioning---satisfies a generalized Singleton bound just like a list of true (distinct) codewords.


\paragraph{SoS relaxations for AEL codewords}
Let $G(L,R,E)$ be the bipartite $(n,d,\lambda)$-expander on which the AEL code is defined, and let $\calC_\inn\subseteq \Sigma_\inn^d$ be the inner code. 

Before going into the details of the SoS relaxation for AEL codes, let us set up some convenient notation. For sets $S\sub [k]$ and $F \sub E$, we use $\zee_{S,F}$ to index their Cartesian product $\zee_{S \times F}$. Further, since we will be often dealing with the case when $F=N(\li)$ or $F=N(\ri)$, we will use $\zee_{S,\li}$ as a shorthand for $\zee_{S,N(\li)}$, and similarly use $\zee_{S,\ri}$ as a shorthand for $\zee_{S,N(\ri)}$. For example,
\[
	\tildeVar{\zee_{[k],\li}} = \tildeVar{ \zee_{[k] \times N(\li)}}.
\]


\begin{definition}[$k$-tuple of pseudocodewords]\label{def:k_tuple}
A \emph{$k$-tuple of psueocodewords of degree-$t$} is a pseudoexpectation operator $\tildeEx{\cdot}$ of SoS-degree-$t$ defined on the variables $\{Z_{i,e}\}_{i\in [k], e\in E}$ over the alphabet $\Sigma_{\inn}$ that respects the following constraints:
\[
	\forall i\in [k], \forall \li \in L,~~~ \zee_{i,N(\li)} \in \calC_{\inn}
\] 
\end{definition}

It is easy to see that any $k$ strings $f_1,\cdots ,f_k$ in $\calC_{\inn}^L$ can be used to define a $k$-tuple of pseudocodeword, by simply setting $\zee_{i,e}$ to be $(f_i)_e$ for all $i \in [k], e\in E$. Note that these strings need not be distinct, as written. However, the set of $k$-tuples of pseudocodewords is more general that just integral strings, and in particular, is convex. However, we can still generalize notions like distance for these objects.

\begin{definition}[Distances and pseudocodewords]
	For an $i\in [k]$, we define the left and right distances between the $i^{th}$ component of a $k$-tuple of pseudocodeword $\tildeEx{\cdot}$ of SoS-degree $t\geq 2d$ and any string $g \in \Sigma_{\inn}^E$ as
	\begin{align*}
		\tildeEx{\Delta_L(g, \zee_i)} &\defeq \Ex{\li \in L}{\tildeEx{\indi{g_{\li} \neq \zee_{i, \li}}}} \\
%	\dis(\tildeEx{\cdot},h) &\defeq \Ex{e}{\tildeEx{\indi{\zee_e \neq h_e}}} \\
		\tildeEx{\Delta_R(g,\zee_i)} &\defeq \Ex{\ri\in R}{\tildeEx{\indi{g_{\ri} \neq \zee_{i, \ri}}}}.
	\end{align*}
\end{definition}

The above can be seen as counting the fraction of errors between $\zee_i$ and $g$. We will also need another piece of notation which will make it easier to track the fraction of errors that are common to an entire set $S\sub [k]$.
\begin{align*}
	\tildeEx{\Delta_R(g,\zee_S)} \defeq \Ex{\ri \in R}{\tildeEx{\Pi_{i\in S} \indi{g_{\ri} \neq \zee_{i,\ri}}}}
\end{align*}

We will also need a notion of distance between two pseudocodewords within a $k$-tuple.
\begin{definition}[Distances between pseudocodewords in a $k$-tuple]
Let $\tildeEx{\cdot}$ be a $k$-tuple of pseudocodewords. Let $i,i' \in [k]$ be indices, then the distance between $i^{th}$ and $(i')^{th}$ pseudocodewords can be defined as
\begin{align*}
		\tildeEx{\Delta_L(\zee_i, \zee_{i'})} &\defeq \Ex{\li \in L}{\tildeEx{\indi{\zee_{i, \li} \neq \zee_{i', \li}}}} \\
%	\dis(\tildeEx{\cdot},h) &\defeq \Ex{e}{\tildeEx{\indi{\zee_e \neq h_e}}} \\
		\tildeEx{\Delta_R(\zee_i,\zee_{i'})} &\defeq \Ex{\ri\in R}{\tildeEx{\indi{\zee_{i,\ri} \neq \zee_{i', \ri}}}}.
	\end{align*}
\end{definition}

\paragraph{$\eta$-goodness} Instead of working with arbitrary pseudocodewords, we will work with those  that have small pseudocovariances across a typical edge in $G$. This key property, called $\eta$-goodness, has been used in prior works and we extend its definition from \cite{JST23} to $k$-tuple of pseudocodewords as follows:

\begin{definition}[$\eta$-good pseudocodeword]
    A $k$-tuple of pseudocodewords $\tildeEx{\cdot}$ of degree at least $2kd$ is called $\eta$-good if
    \[
        \Ex{\li,\ri}{\tildeCov{\zee_{[k]\times N(\li)}}{\zee_{[k]\times N(\ri)}}} \leq \eta.
    \]
\end{definition}

The key upshot of having this property is~\cref{lem:avg_corr} which we use in our proof.  Moreover, one can obtain such $\eta$-good pseudocodewords by randomly conditioning an SoS solution. We relegate the proof of these details to \cref{sec:appendix} as they are an adaptation of earlier proofs.

%
%\begin{definition}[AEL Pseudocodewords]
%	For $t\geq 2d$, we define a degree-$t$ AEL pseuocodeword to be a degree-$t$ pseudoexpectation operator $\tildeEx{\cdot}$ on $\zee$ respecting the following constraints:
%	\begin{align*}
%		\forall \li \in L,\, i\in [k_0]\quad \zee_{i, \li} \in \calC_\inn
%	\end{align*}
%\end{definition}
%
%Next we define the distances between a pseudocodeword and a codeword of $\AELC$. 


%\tnote{Moved the definition of eta-good here. Is the notation of Z consistent?}

The main result of this section is the following generalization of~\cref{lemma:common-error-bound} to $\eta$-good pseudocodewords.

\begin{theorem}\label{thm:sos_main}
	Let $k_0\geq 1$ be an integer and let $\eps > 0$. Let $\AELC$ be a
code obtained using the AEL construction using $(G,\calC_{\out}, \calC_{\inn})$, where $\calC_{\inn}$ is $(\delta_0, k_0, \eps/2)$ average-radius list decodable with erasures, and the graph $G$ is a $(n,d,\lambda)$-expander. 
	Let $\tildeEx{\cdot}$ be an $\eta$-good $k_0$-tuple of pseudocodewords that satisfies the following pairwise distance property on the left:
	\[
		\forall i,i'\in [k_0] \text{ with } i\neq i', \qquad \tildeEx{\Delta_L(\zee_i, \zee_{i'})} \geq \beta\mper
	\]
	Further, assume that $\lambda \leq \frac{\beta}{12k_0^{k_0}}\cdot \eps$ and $\eta \leq \frac{\beta}{12k_0^{k_0}}\cdot \eps$.	Then, for any $g \in (\Sigma_{\inn}^d)^R$,
%	, and any $K \sub [k_0]$ with $|K|=k$, 
	\begin{align*}
	\sum_{i\in [k_0]} \tildeEx{\Delta_R(g,\zee_i)} ~&\geq~ (k_0-1)(\delta_0-\eps) + \Ex{\ri \in R}{\tildeEx{ \Pi_{i\in [k_0]} \indi{g_{\ri} \neq \zee_{i,\ri}} }} \\
	~&=~ (k_0-1)(\delta_0 -\eps) + \tildeEx{\Delta_R(g,\zee_{[k_0]})}.
%		\Ex{i\in [k]}{\tildeEx{\Delta_R(g,\zee_i)}} \geq \frac{k-1}{k} \inparen{ \Delta - \Ex{\ri \in R}{\tildeEx{ \Pi_{i\in[k]} \indi{g_{\ri} \neq \zee_{i,\ri}} }} }
	\end{align*}
%
\end{theorem}
%
%\begin{proof}
%By induction on $k$. The case $k=1$ is trivial.
Just like in the proof of our main theorem in \cref{sec:avg-singleton}, we need preparatory claims about the existence of a nice partition and (a variant of) $L^*$. We start by proving analytic generalizations of \cref{lem:type_arg} and \cref{lemma:inductive}. Let $\Tau$ denote the set of all partitions\footnote{Note that now we work with partitions of $[k]$ rather a list of codewords $\calH$.} of $[k]$. For an $\li \in L$ and $\tau \in \Tau$, define the local function, 
	\[
		\Lambda_{\li, \tau}(\zee) \defeq \indi{ (\zee_{1, \li}, \zee_{2, \li}, \cdots , \zee_{k, \li}) \text{ induces partition }\tau}.
	\]
	By definition, for every $\li \in L$,
	\[
		\sum_{\tau \in \Tau} \Lambda_{\li, \tau}(\zee) = 1.
	\]
	Let $\tau_1 \in \Tau$ be the trivial partition with only 1 part. In the integral proof, we said that there must be a non-trivial partition that is induced on a significant fraction of left vertices, and used it to define the set $L^*$. For tuples of pseudocodewords, each left vertex will be inducing a distribution over all possible partitions, and we will argue that there is a partition $\tau^*$ that receives a significant mass among these distributions on average over all $\li \in L$. The indicator of this partition $\Lambda_{\li,\tau^*}(\zee)$ will then play the role of the indicator of the set $L^*$ in integral proof.
	\begin{claim}[Generalization of \cref{lem:type_arg}]\label{claim:best_partition}
		There exists a $\tau^*\in \Tau \setminus \{\tau_1\}$ such that 
		\[
			\tildeEx{\Ex{\li \in L}{\Lambda_{\li, \tau^*}(\zee)}} \geq \frac{\beta}{k^k}.
		\]
	\end{claim}
	\begin{proof}
		If $(\zee_{1, \li}, \zee_{2, \li}, \cdots , \zee_{k, \li})$ induces the partition $\tau_1$, then $\zee_{1,\li} = \zee_{2,\li}$. That is,
		\[
			\Lambda_{\li,\tau_1}(\zee) \leq \indi{\zee_{1,\li} = \zee_{2,\li}}.
		\]
		We bound the contribution from the trivial partition as 
		\[
			\tildeEx{\Ex{\li}{\Lambda_{\li,\tau_1}(\zee)}} \leq \tildeEx{\Ex{\li}{\indi{\zee_{1,\li} = \zee_{2,\li}}}} = \tildeEx{1-\Delta_L(\zee_1,\zee_2)} \leq 1-\beta
		\]
%		Since $\tildeEx{\Delta_L(\zee_1,\zee_2)} \geq \beta$,
%		\begin{align*}
%			&\tildeEx{\Ex{\li}{\indi{\zee_{1, \li} \neq \zee_{2, \li}}}} ~\geq~ \beta \\
%			\implies &~\tildeEx{\Ex{\li}{\indi{\zee_{1, \li} = \zee_{2,\li}}}} ~\leq~ 1-\beta \\
%			\implies &~\tildeEx{\Ex{\li}{\Lambda_{\li,\tau_1}(\zee)}} ~\leq~ 1-\beta.
%		\end{align*}
		The above shows that the partition $\tau_1$ cannot be too common. Next, we use $\sum_{\tau\in \Tau} \Lambda_{\li,\tau}(\zee) =1$ to show that:
		\begin{align*}
			1 = \tildeEx{\Ex{\li}{\sum_{\tau \in \Tau}{\Lambda_{\li,\tau}(\zee)}}} = \tildeEx{\Ex{\li}{\Lambda_{\li,\tau_1}(\zee)}} + \sum_{\tau \in \Tau \setminus \{\tau_1\}} \tildeEx{\Ex{\li}{\Lambda_{\li,\tau}(\zee)}} \\
			\implies\sum_{\tau \in \Tau \setminus \{\tau_1\}} \tildeEx{\Ex{\li}{\Lambda_{\li,\tau}(\zee)}} = 1 - \tildeEx{\Ex{\li}{\Lambda_{\li,\tau_1}(\zee)}} \geq 1-(1-\beta) = \beta.
		\end{align*}
		By averaging, we can use $|\Tau| \leq k^k$ to conclude that there is a $\tau^* \in \Tau \setminus \{\tau_1\}$ such that,
		\[
			\tildeEx{\Ex{\li}{\Lambda_{\li,\tau^*}(\zee)}} ~\geq~ \frac{1}{|\Tau|} \sum_{\tau \in \Tau \setminus \{\tau_1\}} \tildeEx{\Ex{\li}{\Lambda_{\li,\tau}(\zee)}} ~\geq~ \frac{\beta}{k^k}.\qedhere
		\]
	\end{proof}
	
	\paragraph{Capturing common error locations}
	Suppose the partition $\tau^*$ is given by $[k] = (K_1, \cdots, K_p)$ for some $1<p<k$.  Henceforth, we will be working with this fixed partition. 	We define a function, $\dd_{S,\ri}(g,\zee)$, which is an indicator of whether $\ri \in R$ is a common error location for the pseudocodewords indexed by $S\subseteq [k]$. 
%	Let us define two shorthands for some $S\subseteq [k]$:
	\begin{align*}
		\dd_{S,\ri}(g, \zee) &= \Pi_{i\in S} \indi{g_{\ri} \neq \zee_{i,\ri}} \\
		\dd_{S,e}(g, \zee) &= \dd_{S,\ri}(g, \zee), \text{ where } e=(\li,\ri).
	\end{align*}
Of course, if a vertex $\ri $ is a common error location for all $k$ pseudocodewords, then it is also a common error location for a subset, implying
	\[
		\dd_{S,\ri}(g,\zee) ~\geq~ \dd_{[k],\ri}(g,\zee).
	\]
	We will also need to track the error locations common to $S$ but not to $[k]$, so we define two additional shorthands:
	\begin{align*}
		\fd_{S,\ri}(g, \zee) ~&=~ \dd_{S,\ri}(g,\zee) ~-~ \dd_{[k],\ri}(g,\zee) \\
		\fd_{S,e}(g, \zee) ~&=~ \fd_{S,r}(g, \zee), \text{ where } e=(\li,\ri).
	\end{align*}
		

\paragraph{Key Claims}		
		We will now prove SoS versions of the three main claims from the integral proof; ~\cref{lemma:inductive}, \cref{claim:local-bound}, and \cref{claim:sampling_erasure}. 
	%To begin, let us define the SoS generalizations of the terms in the lemma. 

%Finally, we use that the local snapshots of common errors are upper bounded by the global common errors. This will be another AEL argument.

%	The following table illustrates how the above functions capture...  
%\tnote{Suppresing the dictionary for now.}
%		\begin{table}[h]
%\begin{center}
%  \begin{tabular}{l|c}
%   Codeword & SoS Generalization \\
%    \hline\\
%   $\Delta(g_\ell, \fjl)$ & $\tildeEx{\Ex{e\in N(\li)}{\fd_{K_j,e}(g,\zee)}}$ \\
%    $\indi{\ell\in L^*}$  & $\tildeEx{\Lambda_{\li,\tau^*}(\zee)}$\\
%   	$\abs{L^*}/n$ & $\tildeEx{\Ex{\li}{\Lambda_{\li,\tau^*}(\zee)}}$  \\[2pt]
%   	$\Ex{\ell \in L^*}{\Delta(\gl, \fjl)}$ & $\frac{\Ex{\li}{\tildeEx{\Lambda_{\li,\tau^*}(\zee) \cdot \Ex{e\in N(\li)}{\fd_{K_j,e}(g,\zee)}}} }{\Ex{\li}{\tildeEx{\Lambda_{\li,\tau^*}(\zee)}}}$
%  \end{tabular}
%  \caption{Dictionary between}
%\end{center}
%\end{table}

The first of these showed that errors observed on $L^*$ serve as a lower bound for global errors on the right. The analog of "averaging over $L^*$ can be carried out by reweighing according to the indicator function $\Lambda_{\li,\tau^*}$ for the $\tau^*$ partition.
	
\begin{lemma}[Generalization of \cref{claim:sampling_erasure}]\label{lemma:local_erasures_upper_bound} 
%For any functions $\Gamma(\ell, \zee), \Psi(\zee) $ of  pseudocodewords, the following holds for $\eta$-good pseudocodeword $\zee$.
%\[
%	\tildeEx{\Ex{\li \sim \ri}{\Gamma(\zee) \cdot \Psi(\zee)}} ~\leq~ \tildeEx{\Ex{\li}{\Gamma(\zee)}} \tildeEx{\Ex{\ri}{\Psi(\zee)}} + \lambda +\eta.			
%\]
Assume that $\lambda \leq \frac{\beta}{12k^{k}}\cdot \eps$ and $\eta \leq \frac{\beta}{12k^{k}}\cdot \eps$.
For the functions $\fd, \dd$, and any set $S\subseteq [k]$ we have,
\begin{align*}
	\frac{\tildeEx{\Ex{\li \sim \ri}{\Lambda_{\li,\tau^*}(\zee) \cdot \dd_{S,\ri}(g,\zee)}}}{\tildeEx{\Ex{\li}{\Lambda_{\li,\tau^*}(\zee)}}} ~\leq~ \tildeEx{\Ex{\ri}{\dd_{S,\ri}(g,\zee)}} + \frac{\eps}{6}.\\
		\frac{\tildeEx{\Ex{\li \sim \ri}{\Lambda_{\li,\tau^*}(\zee) \cdot \fd_{S,\ri}(g,\zee)}}}{\tildeEx{\Ex{\li}{\Lambda_{\li,\tau^*}(\zee)}}} ~\leq~ \tildeEx{\Ex{\ri}{\fd_{S,\ri}(g,\zee)}} + \frac{\eps}{6}.
		\end{align*}
	\end{lemma}
\begin{proof} The proof is based on an AEL-like argument and is identical for either case. The first step uses the expander mixing lemma for pseudocodewords (\cref{lem:eml_sos}), and the second utilizes the $\eta$-good property (\cref{lem:avg_corr}),
	\begin{align*}
%		\tildeEx{\Ex{\li}{\Lambda_{\li,\tau^*}(\zee) \cdot \Ex{e\in N(\li)}{D_{[k],e}(g,\zee)}}} &= 
		\tildeEx{\Ex{\li\sim \ri}{\Lambda_{\li,\tau^*}(\zee) \cdot \dd_{S,\ri}(g,\zee)}}	~&\leq~ \tildeEx{\Ex{\li , \ri}{\Lambda_{\li,\tau^*}(\zee) \cdot \dd_{S,\ri}(g,\zee)}} + \lambda \\
		~&\leq~ \tildeEx{\Ex{\li}{\Lambda_{\li,\tau^*}(\zee)}} \cdot \tildeEx{\Ex{\ri}{\dd_{S,\ri}(g,\zee)}} + \lambda + \eta.
	\end{align*}
	To obtain the final consequence we divide by $\tildeEx{\Ex{\li}{\Lambda_{\li,\tau^*}(\zee)}}$ and use~\cref{claim:best_partition}.
		\begin{align*}
		\frac{\tildeEx{\Ex{\li\sim \ri}{\Lambda_{\li,\tau^*}(\zee) \cdot \dd_{S,\ri}(g,\zee)}}}{\tildeEx{\Ex{\li}{\Lambda_{\li,\tau^*}(\zee)}}} ~&\leq~ \tildeEx{\Ex{\ri}{\dd_{S,\ri}(g,\zee)}} + \frac{\lambda + \eta}{\tildeEx{\Ex{\li}{\Lambda_{\li,\tau^*}(\zee)}}} \\
%		~&\leq  \tildeEx{\Ex{\ri}{D_{[k],\ri}(g,\zee)}} + \frac{\lambda + \eta}{(\beta/k^k)} \\
		~&\leq~  \tildeEx{\Ex{\ri}{\dd_{S,\ri}(g,\zee)}} + \frac{\eps}{6}. \qedhere
%		~&=~  \tildeEx{\Delta_R(g,\zee_{[k]})} + \frac{\eps}{6}.
	\end{align*}
\end{proof}

	
%	Finally, define $\lstar := \tildeEx{\Ex{\li \in L}{\Lambda_{\li, \tau^*}(\zee)}}$
%	\snote{Move these definitions outside theorem? $\fd$ is a macro, feel free to change.}
	
	
	
%	\begin{align*}
%	\Delta(\gl, \fjl) ~&\mapsto~~   \tildeEx{\Ex{e\in N(\li)}{\fd_{K_j,e}(g,\zee)}}\;,	\\
%	\indi{\ell\in L^*}  ~&\mapsto~~ \tildeEx{\Lambda_{\li,\tau^*}(\zee)}\;, \\
%	\abs{L^*}/n ~&\mapsto~~ \tildeEx{\Ex{\li}{\Lambda_{\li,\tau^*}(\zee)}}.
%	\end{align*}
	
%	It is easy to see that when $Z$ is an integral codeword, the two definitions coincide. Combining these, 
%	\begin{align*}
%	\Ex{\ell \in L^*}{\Delta(\gl, \fjl)} = \frac{\Ex{\ell}{\indi{\ell\in L^*} \cdot\Delta(\gl, \fjl)}}{|L^*|} ~&\mapsto~~   \frac{\Ex{\li}{\tildeEx{\Lambda_{\li,\tau^*}(\zee) \cdot \Ex{e\in N(\li)}{\fd_{K_j,e}(g,\zee)}}} }{\Ex{\li}{\tildeEx{\Lambda_{\li,\tau^*}(\zee)}}} .
%	\end{align*}
%	\snote{Maybe above is not a great idea since we want to discourage $\erase{g}$ in the SoS section.}
	
%	$\tildeEx{\Lambda_{\li,\tau^*}(\zee) \cdot \Ex{e\in N(\li)}{\indi{g_e \neq \zee_{i,e}} \cdot (1-D_{[k],e}(g,\zee))}} $
%	
%	$\Ex{\ell \in L^*}{\Delta(\gl, \fl_j)}$

The second lemma showed that the number of common error locations increases when only considering a subset of indices $K_j \sub [k]$, and this increase can be lower bounded in terms of the distance between $g_{\li}$ (actually, $\gl$) and the common local projections of $K_j$, averaged over $L^*$. The second term on the RHS in the lemma below is the analog of these errors between $\gl$ and common local projections $f_j$, averaged over $L^*$.
	\begin{lemma}[Generalization of  \cref{lemma:inductive}]\label{lemma:more_common_errors}
	For any part $K_j$ in the partition $\tau^*$, we have,
	%, and an arbitrary $i\in H_j$,
	\begin{align*}
		\tildeEx{\Delta_R(g,\zee_{K_j})}
%		&= \Ex{\ri}{\tildeEx{D_{K_j,\ri}(g,\zee)}}
%		&\geq \Ex{\ri}{\tildeEx{D_{[k],\ri}(g,\zee)}} + \frac{\Ex{\li}{\tildeEx{\Lambda_{\li,\tau^*}(\zee) \cdot \Ex{e\in N(\li)}{\fd_{K_j,e}(g,\zee)} }}}{\Ex{\li}{\tildeEx{\Lambda_{\li,\tau^*}(\zee)}}} - \frac{\eps}{6}\\
		~\geq~ \tildeEx{\Delta_R(g,\zee_{[k]})}  + \frac{\tildeEx{\Ex{\li \sim \ri}{\Lambda_{\li,\tau^*}(\zee) \cdot \fd_{K_j,\ri}(g,\zee)}}}{\tildeEx{\Ex{\li}{\Lambda_{\li,\tau^*}(\zee)}}} - \frac{\eps}{6} .
	\end{align*}
	\end{lemma}
	
	\begin{proof} By using the definition of the functions $\dd, \fd$ and some rearragement, we get,
	\begin{align*}
		\tildeEx{\Delta_R(g,\zee_{K_j})} ~&=~ \Ex{\ri}{\tildeEx{\dd_{K_j,\ri}(g,\zee)}} \\
		~&=~ \Ex{\ri}{\tildeEx{\dd_{[k],\ri}(g,\zee) \cdot \dd_{K_j,\ri}(g,\zee) + (1-\dd_{[k],\ri}(g,\zee)) \cdot \dd_{K_j,\ri}(g,\zee)}} \\
		~&=~ \Ex{\ri}{\tildeEx{\dd_{[k],\ri}(g,\zee)}} + \Ex{\ri}{\tildeEx{\fd_{K_j,\ri}(g,\zee)}}.
	\end{align*}
	
The proof finishes by replacing the second term by the bound from~\cref{lemma:local_erasures_upper_bound}.
%a typical AEL argument. The first step uses the expander mixing lemma (\cref{lem:eml_sos}), and the second utilizes the $\eta$-good property,
%	\begin{align*}
%		\tildeEx{\Ex{\li \sim \ri}{\Lambda_{\li, \tau^*}(\zee) \cdot \fd_{K_j,\ri}(g,\zee)}} \leq~&~ \tildeEx{\Ex{\li, \ri}{\Lambda_{\li, \tau^*}(\zee) \cdot \fd_{K_j,\ri}(g,\zee)}} + \lambda \\
%		\leq~&~ \tildeEx{\Ex{\li}{\Lambda_{\li, \tau^*}(\zee)}} \cdot \tildeEx{\Ex{\ri}{\fd_{K_j,\ri}(g,\zee)}} + \lambda + \eta\\
%		=~&~ \lstar \cdot \tildeEx{\Ex{\ri}{\fd_{K_j,\ri}(g,\zee)}} + \lambda + \eta.
%	\end{align*}
%	
%Dividing both sides by $\lstar$, we get 
%%\cref{eq:more_common_errors_lower_bound} and \cref{eq:more_common_errors_upper_bound}, we get
%	\begin{align*}
%		\tildeEx{\Ex{\ri}{\fd_{K_j,\ri}(g,\zee)}} &~\geq~ \frac{1}{\lstar}\cdot {\tildeEx{\Ex{\li \sim \ri}{\Lambda_{\li, \tau^*}(\zee) \cdot \fd_{K_j,\ri}(g,\zee)}}}- \frac{\lambda+\eta}{\lstar} \\
%		&~\geq~ \frac{1}{\lstar}\cdot {\tildeEx{\Ex{\li \sim \ri}{\Lambda_{\li, \tau^*}(\zee) \cdot \fd_{K_j,\ri}(g,\zee)}}}- \frac{\eps}{6}.\qedhere
%	\end{align*}
\end{proof}


	Finally, we state the local inequality needed from the inner code. Since this inequality is only valid for vertices in $L^*$, we multiply the required equation by indicator $\Lambda_{\li,\tau^*}(\zee)$ so that it is trivial when the indicator is 0. Subject to this indicator being 1, the inequality says that the generalized Singleton bound with erasures holds for the inner code. Note that this bound holds for each vertex in $L^*$ unlike previous sampling lemmas which only work in an average sense over $L^*$.
%		\tnote{Shashank:Check}
	\begin{lemma}[Generalization of \cref{claim:local-bound}]\label{lemma:local_avg_singleton}
		For every $\li \in L$ and for every $j\in [p]$,
		\[
			\Lambda_{\li, \tau^*}(\zee) \parens[\bigg]{ \sum_{j = 1}^p \Ex{e\in N(\li)}{\fd_{K_j, e}(g,\zee)}} ~\geq~ \Lambda_{\li, \tau^*}(\zee) \cdot (p-1) \parens[\Big]{ \delta_0 - \Ex{e\in N(\li)}{\dd_{[k],e}(g,\zee)}  - \frac{\eps}{2}}.
		\]
	\end{lemma}
	\begin{proof}
	Let $M_{\li} \sub E$ be the union of edge neighborhoods over vertices in $R$ that are adjacent to $\li$. That is, 
	\[
		M_{\li} = \bigcup_{\ri 
		\sim \li}  N(\ri)
	\]
	Let us consider the following two local functions that depend on $[k] \times M_{\li}$.
	\begin{align*}
		\mu_1(\cdot) ~&=~ \Lambda_{\li, \tau^*}(\cdot) \;\parens[\bigg]{\; \sum_{j\in [p]} \Ex{e\in N(\li)}{\fd_{K_j, e}(g,\cdot)}\,} \\
		\mu_2(\cdot) ~&=~ \Lambda_{\li, \tau^*}(\cdot) (p-1) \inparen{ \delta_0 - \Ex{e\in N(\li)}{\dd_{[k],e}(g,\cdot)}  - \frac{\eps}{2}}
	\end{align*}
	We will prove this inequality pointwise by showing that for any $\alpha \in \Sigma_{\inn}^{[k] \times M_{\li}}$, $\mu_1(\alpha) \geq \mu_2(\alpha)$.
	
	If $\alpha$ is such that $\Lambda_{\li,\tau^*}(\alpha) = 0$, then the inequality is trivially true. Henceforth, we assume that $\Lambda_{\li,\tau^*}(\alpha) = 1$. This means that $(\alpha_{1,\li},\alpha_{2,\li},\cdots ,\alpha_{k,\li})$ induces the partition $\tau^*$. 
	By definition of $K_j$, for any $i,i'\in K_j$, we get that $\alpha_{i,\li} = \alpha_{i',\li}$. Let us denote this common codeword in $\calC_{\inn}$ as $\fjl$.
	
	Let $g_{\li} \in \Sigma_{\inn}^{d}$ be the local projection of $g$ to the edge neighborhood of $\li$. For every $e\in N(\li)$ that satisfies $\dd_{[k],e}(g,\alpha) = 1$, we replace the corresponding coordinate in $g_{\li}$ by an erasure to obtain $\gl\in \inparen{ \Sigma_{\inn} \cup \{\bot\} }^{N(\li)}$. 
	The fraction of erasures in $g_{\li}$ is
	\[
		s_{\li} = \Ex{e\in N(\li)}{\dd_{[k],e}(g,\alpha)}.
	\]
	Next, we calculate,
	\begin{align*}
		\Ex{e\in N(\li)}{\fd_{K_j, e}(g,\alpha)} ~&=~ \Ex{e\in N(\li)}{\dd_{K_j, e}(g,\alpha) \cdot \inparen{1-\dd_{[k],e}(g,\alpha)}} \\
		~&=~ \Ex{e\in N(\li)}{\dd_{K_j, e}(g,\alpha)} - \Ex{e\in N(\li)}{\dd_{[k],e}(g,\alpha)} \\
		~&=~ \Ex{e\in N(\li)}{\indi{g_{e} \neq f_{j,e}}} - s_{\li} \\
		~&=~ \Delta(g_{\li},\fjl) - s_{\li} \\
		~&=~ \Delta(\gl, \fjl).
	\end{align*}
	Applying the $(\delta_0,k,\eps/2)$-average radius list decodability with erasures of inner code to $\gl$ and the set of codewords $\{\fjl\}_{j\in [p]}$, we get,
	\[
		\sum_{j=1}^p \Delta(\gl,\fjl) ~\geq~ (p-1) \parens[\Big]{\delta_0 - s_{\li} - \frac{\eps}{2}}.
	\]
	Substituting $\Delta(\gl, \fjl) = \Ex{e\in N(\li)}{\fd_{K_j, e}(g,\alpha)}$ and $s_{\li} = \Ex{e\in N(\li)}{\dd_{[k],e}(g,\alpha)}$ gives
	\begin{align*}
		\sum_{j=1}^p \Ex{e\in N(\li)}{\fd_{K_j, e}(g,\alpha)} ~\geq~ (p-1) \cdot\parens[\bigg]{\delta_0 -  \Ex{e\in N(\li)}{\dd_{[k],e}(g,\alpha)} - \frac{\eps}{2}}.\qedhere
	\end{align*}
\end{proof}



	
\subsubsection{Proof of Main Theorem}	

We restate the main theorem and finish the proof using the above lemmas.

\begin{theorem}[Restatement of \cref{thm:sos_main}]
	Let $k_0\geq 1$ be an integer and let $\eps > 0$. Let $\AELC$ be a
code obtained using the AEL construction using $(G,\calC_{\out}, \calC_{\inn})$, where $\calC_{\inn}$ is $(\delta_0, k_0, \eps/2)$ average-radius list decodable with erasures, and the graph $G$ is a $(n,d,\lambda)$-expander. 
	Let $\tildeEx{\cdot}$ be an $\eta$-good $k_0$-tuple of pseudocodewords that satisfies the following pairwise distance property on the left:
	\[
		\forall i,i'\in [k_0] \text{ with } i\neq i', \qquad \tildeEx{\Delta_L(\zee_i, \zee_{i'})} \geq \beta\mper
	\]
	Further, assume that $\lambda \leq \frac{\beta}{12k_0^{k_0}}\cdot \eps$ and $\eta \leq \frac{\beta}{12k_0^{k_0}}\cdot \eps$.	Then, for any $g \in (\F_q^d)^R$,
	\begin{align*}
	\sum_{i\in [k_0]} \tildeEx{\Delta_R(g,\zee_i)} ~&\geq~ (k_0-1)(\delta_0-\eps) + \Ex{\ri \in R}{\tildeEx{ \Pi_{i\in[k_0]} \indi{g_{\ri} \neq \zee_{i,\ri}} }} \\
	~&=~ (k_0-1)(\delta_0 -\eps) + \tildeEx{\Delta_R(g,\zee_{[k_0]})}.
%		\Ex{i\in [k]}{\tildeEx{\Delta_R(g,\zee_i)}} \geq \frac{k-1}{k} \inparen{ \Delta - \Ex{\ri \in R}{\tildeEx{ \Pi_{i\in[k]} \indi{g_{\ri} \neq \zee_{i,\ri}} }} }
	\end{align*}
%
\end{theorem}

\begin{proof}
The proof, as before, is by induction on $k_0$. The base case, $k_0 =1$ is trivial. So we assume the statement holds upto $k-1$, and the goal is to prove it for $k_0=k$. Fix a partition $\tau^* = (K_1,\cdots, K_p)$ as guaranteed by~\cref{claim:best_partition}.  For each part $K_j$, we may apply the induction hypothesis to the sub-tuple defined by it as the pairwise distance property is already assumed. Using this we get, 
	\begin{align*}
\sum_{i=1}^k \tildeEx{\Delta_R(g,\zee_i)} 
		~&=~ \sum_{j=1}^p \sum_{i\in K_j} \tildeEx{\Delta_R(g,\zee_i)} \\
		~&\geq~ \sum_{j=1}^p \inparen{(|K_j|-1)(\delta_0 -\eps) + \tildeEx{\Delta_R(g,\zee_{K_j})}}.
		\end{align*}

The first term is simply,   
$
\sum_{j=1}^p (|K_j|-1)(\delta_0 -\eps) ~=~ (k-p)(\delta_0 -\eps) . 
$ The goal is now to show that, 
\[\sum_{j=1}^p\tildeEx{\Delta_R(g,\zee_{K_j})}	 ~\geq~ (p-1)(\delta_0 -\eps) + \tildeEx{\Delta_R(g,\zee_{[k]})}.\]
 
 For a fixed $j \in [p]$, we apply \cref{lemma:more_common_errors} to obtain, 

		\[
		\tildeEx{\Delta_R(g,\zee_{K_j})}	~\geq~  \tildeEx{\Delta_R(g,\zee_{[k]})}  + \frac{\Ex{\li \sim \ri}{\;\tildeEx{\Lambda_{\li,\tau^*}(\zee) \cdot \fd_{K_j,\ri}(g,\zee)}\,} }{\Ex{\li}{\tildeEx{\Lambda_{\li,\tau^*}(\zee)}}} - \frac{\eps}{6}.
		\]

The term $c := \tildeEx{\Delta_R(g,\zee_{[k]})}  - \frac{\eps}{6}$ is independent of $j$.
Summing the RHS over $j \in [p]$, 

\begin{align}
\sum_{j=1}^p	\tildeEx{\Delta_R(g,\zee_{K_j})} - p\cdot c ~&\geq~ \sum_{j=1}^p \frac{\Ex{\li \sim \ri}{\tildeEx{\Lambda_{\li,\tau^*}(\zee) \cdot \fd_{K_j,\ri}(g,\zee)}} }{\Ex{\li}{\tildeEx{\Lambda_{\li,\tau^*}(\zee)}}}\\
%~&=~ \sum_{j=1}^p \frac{\Ex{\li}{\tildeEx{\Lambda_{\li,\tau^*}(\zee) \cdot \inparen{ \Ex{e\in N(\li)}{ \fd_{K_j,e}(g,\zee)}}}} }{\Ex{\li}{\tildeEx{\Lambda_{\li,\tau^*}(\zee)}}}\\
~&=~   { \frac{\Ex{\li}{\tildeEx{\Lambda_{\li,\tau^*}(\zee) \cdot \sum_{j=1}^p \inparen{\Ex{e\in N(\li)}{\fd_{K_j,e}(g,\zee)}} }}}{ \Ex{\li}{\tildeEx{\Lambda_{\li,\tau^*}(\zee)}} } }\label{eq:main}
		\end{align}

We can now apply \cref{lemma:local_avg_singleton} to the RHS
\begin{align}		
	~&\geq~ { \frac{\Ex{\li}{\tildeEx{\Lambda_{\li, \tau^*}(\zee) \cdot (p-1) \inparen{ \delta_0 - \Ex{e\in N(\li)}{\dd_{[k],e}(g,\zee)}  - \frac{\eps}{2}} }}}{\Ex{\li}{\tildeEx{\Lambda_{\li,\tau^*}(\zee)}}}} \\
	~&=~	(p-1)(\delta_0 - \frac{\eps}{2}) -(p-1) \frac{\Ex{\li}{\tildeEx{\Lambda_{\li, \tau^*}(\zee) \cdot \inparen{\Ex{e\in N(\li)}{\dd_{[k],e}(g,\zee)} } }}}{\Ex{\li}{\tildeEx{\Lambda_{\li,\tau^*}(\zee)}}}.\label{eq:second}
		 \end{align}

		
To bound the term on the right, we use~\cref{lemma:local_erasures_upper_bound},			\begin{align*}
				\frac{\Ex{\li}{\tildeEx{\Lambda_{\li, \tau^*}(\zee) \cdot \inparen{\Ex{e\in N(\li)}{\dd_{[k],e}(g,\zee)} } }}}{\Ex{\li}{\tildeEx{\Lambda_{\li,\tau^*}(\zee)}}} ~&=~ \frac{\Ex{\li\sim \ri}{\tildeEx{\Lambda_{\li, \tau^*}(\zee) \cdot \dd_{[k],\ri}(g,\zee)  }}}{\Ex{\li}{\tildeEx{\Lambda_{\li,\tau^*}(\zee)}}}\\[1em]
\text{(\cref{lemma:local_erasures_upper_bound})}\;\;				~&\leq~  \tildeEx{\Ex{\ri}{\dd_{S,\ri}(g,\zee)}} + \frac{\eps}{6}\\
				~&=~	\tildeEx{\Delta_R(g,\zee_{[k]})} + \frac{\eps}{6}.  
				\end{align*}

Plugging this bound in~\cref{eq:second} and then back in~\cref{eq:main},  we get,		
\begin{align*}	
		\tildeEx{\Delta_R(g,\zee_{K_j})}  ~&\geq~ p\cdot c + (p-1)\parens[\Big]{\delta_0 - \frac{\eps}{2}} -  (p-1) \inparen{\tildeEx{\Delta_R(g,\zee_{[k]})} + \frac{\eps}{6}}\\
		~&=~ \tildeEx{\Delta_R(g,\zee_{[k]})}  +(p-1)\parens[\Big]{\delta_0 - \frac{\eps}{2}} - (2p-1)\cdot \frac{\eps}{6} \\
		~&\geq~ \tildeEx{\Delta_R(g,\zee_{[k]})}  + (p-1)(\delta_0 - \eps) \;. \qedhere
		\end{align*}
%	\begin{align*}
%		&= (k-1)(\Delta - \eps) + \Delta_R(g,\zee_{[k]}) +(p-1)\frac{\eps}{2} - (2p-1)\cdot k^k\inparen{\frac{\lambda+\eta}{\beta}} \\
%		&\geq (k-1)(\Delta - \eps) + \Delta_R(g,\zee_{[k]})
%	\end{align*}
\end{proof}

%\paragraph{Proof of the three lemmas}



\subsection{The final algorithm}\label{sec:sos_algo}

In this subsection, we describe the final decoding algorithm. We will mainly rely on the main result of the previous subsection: that for appropriately instantiated AEL codes and for $\eta$ small enough, $\eta$-good $k$-tuple of pseudocodewords satisfy the generalized Singleton bound.

\subsubsection{Decoding from Distributions}
Before describing our main algorithm, we argue that a simple idea based on randomized rounding can be used to extend the unique decoder of $\calC_{\out}$ to decode not only from integral strings close to a codeword, but an ensemble of distributions - one for each coordinate - that is close to a codeword in average sense. It is also standard to derandomize this process via threshold rounding, which is what we describe next. We will be needing this strengthening of the unique decoder of $\calC_{\out}$ as a subroutine in the main algorithm.

It will be helpful to index the codewords in $\calC_{\inn}$ by integers. Let $\abs*{\calC_{\inn}} = M$, and let its codewords be $\alpha_1,\alpha_2,\cdots ,\alpha_M$. Recall that for any $k$-tuple of pseudocodewords $\tildeEx{\cdot}$, for any $j\in [k]$ and $\li \in L$, the set of values
\[
	\inbraces{ \tildeEx{\indi{\zee_{j,\li} = \alpha_i}} }_{i\in [M]}
\]
correspond to a probability distribution over $\calC_{\inn}$. Therefore, the set of intervals
\[
	\inbraces{ \left[\sum_{i=1}^{m-1} \tildeEx{\indi{\zee_{j,\li} = \alpha_i}}, \sum_{i=1}^{m} \tildeEx{\indi{\zee_{j,\li} = \alpha_i}} \right) }_{m\in [M]}
\]
partitions the interval $[0,1)$ into at most $M$ parts.

\begin{lemma}\label{lem:decode_from_distrib}
	Suppose the code $\calC_{\out}$ is unique decodable up to radius $\delta_{\out}^{\dec}$, and let $h$ be a codeword in $\AELC$. Given a collection of distributions $\{\calD_{\li}\}_{\li \in L}$, with each distribution over $\calC_{\inn}$ such that
	\[
		\Ex{\li}{\Ex{f\sim \calD_{\li}}{{\indi{f\neq h_{\li}}}}} \leq \delta_{\out}^{\dec},
	\]
	the \cref{algo:unique-decoding} finds $h$.
\end{lemma}

\begin{figure}[!ht]
\begin{algorithm}{\DECODE}{$\{\calD_{\li}\}_{\li\in L}$}{Codeword $h\in \AELC$ such that $\Ex{\li}{\Ex{f\sim \calD_{\li}}{\indi{f\neq h_{\li}}}} \leq \delta_{\out}^{\dec}$}
\label{algo:unique-decoding}
%
\begin{itemize}
%
%	\item For $p=1$ to $k$:\label{step2}
%	\begin{enumerate}[(i)]
%%		\item Find a $p$-tuple of pseudocodewords $\pcod{p}{\cdot}$ of degree $t = \cdots$ that respects the following constraints:
%%		\begin{itemize}
%%			\item For every $i,i' \in [p]$ with $i\neq i'$, $~\pcod{p}{\Delta_L(\zee_i,\zee_{i'})} > \delta_{\out}^{\dec}$.
%%			\item For every $i\in [p]$, $\pcod{p}{\Delta_R(g,\zee_i)} \leq \frac{k-1}{k}(1-\rho-\eps)$.
%%		\end{itemize}
%%		\item If no such $\pcod{p}{\cdot}$ exists:
%%		\begin{itemize}
%%		\item $p^* = p-1$
%%		\item Exit loop.
%%		\end{itemize}
%	\end{enumerate}
	\item Let $w_{\li,j}$ be the weight on codeword $\alpha_i$ according to distribution $\calD_{\li}$.
	\item For every threshold $\theta \in [0,1]$:\footnote{As written, this involves trying uncountably many thresholds. However, we only need to try at most $M \cdot |L|$ thresholds since the algorithm only depends on which intervals $\theta$ belongs to. Since the number of endpoints of these intervals is at most $M\cdot |L|$ in total, it suffices to try only $M\cdot |L| = M\cdot n$ many distinct thresholds. This is a standard method called threshold rounding.}
			\begin{enumerate}[(i)]
				\item Construct an $f_{\theta} \in \calC_{\inn}^L$ by assigning 
				\[(f_{\theta})_{\li} = \alpha_m  \iff \theta \in \left[ \sum_{i=1}^{m-1} w_{\li, i}, \sum_{i=1}^{m} w_{\li,i} \right)\]
				for every $\li \in L$.
				\item Let $f_\theta^* \in \Sigma_{\out}^L$ defined as $(f_{\theta}^*)_{\li} = \phi^{-1}((f_{\theta})_{\li})$. 
				\item Use the unique decoder of $\calC_{\out}$ to find an $h^* \in \calC_{\out}$ whose distance from $f_{\theta}^*$ is at most $\delta_{\out}^{\dec}$, if such an $h^*$ exists. That is, $h^* \leftarrow \mathrm{Dec}_{\calC_{\out}}(f_{\theta}^*,\delta_{\out}^{\dec})$.
				\item Let $h\in \AELC$ be the codeword corresponding to $h^* \in \calC_{\out}$. Return $h$.
				\end{enumerate}
		\end{itemize}
%		\item For every threshold $\theta \in [0,1]$:\footnote{As written, this involves trying uncountably many thresholds. However, we only need to try at most $M \cdot |L|$ thresholds since the algorithm only depends on which intervals $\theta$ belongs to. Since the number of thresholds is at most $M\cdot |L|$, it suffices to try only $M\cdot |L|$ many distinct thresholds. This is a standard method called threshold rounding.}
%			\begin{enumerate}[(i)]
%				\item Construct an $f_{\theta} \in \calC_{\inn}^L$ by assigning 
%				\[f_{\li} = \alpha_m  \iff \theta \in \left[ \sum_{i=1}^{m-1} \pcod{p^*}{\indi{\zee_{j,\li} = \alpha_i}}, \sum_{i=1}^{m} \pcod{p^*}{\indi{\zee_{j,\li} = \alpha_i}} \right)\]
%				for every $\li \in L$.
%				\item Let $f_\theta^* \in \Sigma_{\out}^L$ defined as $(f_{\theta}^*)_{\li} = \phi^{-1}((f_{\theta})_{\li})$. 
%				\item Use the unique decoder of $\calC_{\out}$ to find an $h^* \in \calC_{\out}$ whose distance from $f_{\theta}^*$ is at most $\delta_{\out}^{\dec}$, if such an $h^*$ exists. That is, $h^* \leftarrow \mathrm{Dec}(f_{\theta}^*,\delta_{\out}^{\dec})$.
%				\item Let $h$ be the codeword of $\AELC$ corresponding to $h^* \in \calC_{\out}$. If $\Delta_R(g,h) < \frac{k-1}{k} (1-\rho-\eps)$, add $h$ to $\calL$.
%			\end{enumerate}
%
\vspace{5pt}
%
\end{algorithm}
\end{figure}

\begin{proof}
	It suffices to show that there exists a threshold $\theta \in [0,1)$ for which the distance between $f_{\theta}^*$ constructed by \cref{algo:unique-decoding} and $h^*$ is at most $\delta_{\out}^{\dec}$. In fact, we will show that this is true for an average $\theta$.
	\begin{align*}
		\Ex{\theta \in [0,1)}{ \Delta(f_{\theta}^*,h^*)} &= \Ex{\theta \in [0,1)}{ \Delta_L(f_{\theta},h)} = \Ex{\theta \in [0,1)}{ \Ex{\li}{\indi{(f_{\theta})_{\li} \neq h_{\li}}}}  = \Ex{\theta \in [0,1)}{ \Ex{\li}{\sum_{i\in [M] : \alpha_i \neq h_{\li}}\indi{(f_{\theta})_{\li} = \alpha_i}}} 
	\end{align*}
	We can move the expectation over $\theta$ inside and use the fact that $(f_{\theta})_{\li}$ is $\alpha_i$ with probability exactly $w_{\li,i}$ to get
	\begin{align*}
		\Ex{\li}{\sum_{i\in [M] : \alpha_i \neq h_{\li}} \Ex{\theta \in [0,1)}{\indi{(f_{\theta})_{\li} = \alpha_i}}} =  \Ex{\li}{\sum_{i\in [M] : \alpha_i \neq h_{\li}} w_{\li,i}} = \Ex{\li}{\Ex{f\sim \calD_{\li}}{{\indi{f\neq h_{\li}}}}} \leq \delta_{\out}^{\dec}. \qedhere
	\end{align*}
\end{proof}

\subsubsection{Decoding Algorithm}
\begin{table}[h]
\hrule
\vline
\begin{minipage}[t]{0.99\linewidth}
\vspace{-5 pt}
{\small
\begin{align*}
    &\mbox{find}\quad ~~ \tildeEx{\cdot} ~\text{on}~ \zee_{[p],E} \text{ and alphabet }\Sigma_{\inn}%\tag{List Decoding Program}\label{sos:list_dec}
    \\
&\mbox{subject to}\quad \quad ~\\
	&\qquad \text{(i)}~~ \tildeEx{\cdot} \text{is a }p\text{-tuple of pseudocodewords of SoS-degree } t \\
    &\qquad \text{(ii)}~~ \forall i \in [p],~ \text{the constraint }\Delta_R(g,\zee_i) < \frac{k-1}{k}(1-\rho-\eps) \text{ is respected by }\tildeEx{\cdot}\label{cons:agreement-ld}    \\
&\qquad \text{(iii)}~ \forall F \sub E \text{ such that } |F|\leq \frac{t-2pd}{2}, \sigma \in \Sigma_{\inn}^{[p]\times F} \text{ and }\forall~i,i' \in [p] \text{ with } i\neq i', \\
&\qquad \qquad \qquad \qquad \tildeEx{\inparen{\Delta_L(\zee_i,\zee_{i'}) - \delta_{\out}^{\dec}}\cdot  \indi{\zee_{[p],F} = \sigma}^2 } \geq 0
\end{align*}}
\vspace{-10 pt}
\end{minipage}
\hfill\vline
\hrule
\caption{$\mathrm{SDP}(p, t)$}
\label{tab:SDP_for_feasibility}
\end{table}

\begin{theorem}\label{thm:sos_technical}
%
Let $k\geq 1$ be an integer and let $\eps > 0$. Let $\AELC$ be a
code obtained using the AEL construction using  $(G, \calC_{\out}, \calC_{\inn})$, where $\calC_{\inn}$ is $(\delta_0, k, \eps/2)$ average-radius list decodable with erasures, and $G$ is a $(n,d,\lambda)$-expander. 
%
Suppose that $\calC_{\out}$ is unique decodable from radius $\delta_{\out}^{\dec}$ in time $\calT(n)$.

If $\lambda \leq \frac{\delta_{\out}^{\dec}}{12{k}^{k}} \cdot \eps$, then there is a deterministic algorithm that takes as input $g\in (\Sigma_{\inn}^d)^R$, runs in time $\calT(n) + n^{O\inparen{ \frac{d \cdot k^{3k}|\Sigma_{\inn}|^{3kd}}{(\delta_{\out}^{\dec})^2\cdot \eps^2}}}$, and outputs the list $\calL(g,\frac{k-1}{k}(\delta_0-\eps))$.
\end{theorem}

\begin{figure}[!ht]
\begin{algorithm}{List Decoding algorithm up to $\frac{k-1}{k}(\delta_0-\eps)$}{$k$, $g \in (\Sigma_{\inn}^d)^R$}{List of codewords $\calL\inparen{g, \frac{k-1}{k}\inparen{\delta_0-\eps}}$}\label{algo:sos-decoding}
%
\begin{enumerate}
%
	\item Initialize $\calL = \{ \}$, $t = 2kd\cdot \inparen{1+\frac{144k^{2k}|\Sigma_{\inn}|^{3kd}}{(\delta_{\out}^{\dec})^2\cdot \eps^2}}$.
	\item Let $p^*$ be the largest $p\in [k]$ such that $\mathrm{SDP}(p,t)$ is feasible.
%	\item For $p=1$ to $k$:\label{step2}
%	\begin{enumerate}[(i)]
%%		\item Find a $p$-tuple of pseudocodewords $\pcod{p}{\cdot}$ of degree $t = \cdots$ that respects the following constraints:
%%		\begin{itemize}
%%			\item For every $i,i' \in [p]$ with $i\neq i'$, $~\pcod{p}{\Delta_L(\zee_i,\zee_{i'})} > \delta_{\out}^{\dec}$.
%%			\item For every $i\in [p]$, $\pcod{p}{\Delta_R(g,\zee_i)} \leq \frac{k-1}{k}(1-\rho-\eps)$.
%%		\end{itemize}
%%		\item If no such $\pcod{p}{\cdot}$ exists:
%%		\begin{itemize}
%%		\item $p^* = p-1$
%%		\item Exit loop.
%%		\end{itemize}
%	\end{enumerate}
	\item For every $F \sub E$ with $|F|\leq \frac{t-2kd}{2}$ and $\sigma\in \Sigma_{\inn}^F$ with $\pcod{p^*}{\indi{\zee_{[p^*],F} = \sigma}} > 0$:\label{step3}
		\begin{itemize}
		\item For $j=1$ to $p^*$:
		\begin{itemize}
			\item Construct an ensemble of distributions $\calD = \{\calD_{\li}\}_{\li\in L}$, with each $\calD_{\li}$ over $\calC_{\inn}$ by using the local distribution induced by the conditioned $\pcod{p^*}{ ~\cdot \given \zee_{[p^*],F} = \sigma}$ over the set $\{j\} \times N(\li)$, or equivalently, over variables $\zee_{j,\li}$.
			\item $h\leftarrow \text{\DECODE} \inparen{\calD}$.
			\item If $\Delta_R(g,h) < \frac{k-1}{k} (\delta_0-\eps)$, add $h$ to $\calL$.
		\end{itemize}
%		\item For every threshold $\theta \in [0,1]$:\footnote{As written, this involves trying uncountably many thresholds. However, we only need to try at most $M \cdot |L|$ thresholds since the algorithm only depends on which intervals $\theta$ belongs to. Since the number of thresholds is at most $M\cdot |L|$, it suffices to try only $M\cdot |L|$ many distinct thresholds. This is a standard method called threshold rounding.}
%			\begin{enumerate}[(i)]
%				\item Construct an $f_{\theta} \in \calC_{\inn}^L$ by assigning 
%				\[f_{\li} = \alpha_m  \iff \theta \in \left[ \sum_{i=1}^{m-1} \pcod{p^*}{\indi{\zee_{j,\li} = \alpha_i}}, \sum_{i=1}^{m} \pcod{p^*}{\indi{\zee_{j,\li} = \alpha_i}} \right)\]
%				for every $\li \in L$.
%				\item Let $f_\theta^* \in \Sigma_{\out}^L$ defined as $(f_{\theta}^*)_{\li} = \phi^{-1}((f_{\theta})_{\li})$. 
%				\item Use the unique decoder of $\calC_{\out}$ to find an $h^* \in \calC_{\out}$ whose distance from $f_{\theta}^*$ is at most $\delta_{\out}^{\dec}$, if such an $h^*$ exists. That is, $h^* \leftarrow \mathrm{Dec}(f_{\theta}^*,\delta_{\out}^{\dec})$.
%				\item Let $h$ be the codeword of $\AELC$ corresponding to $h^* \in \calC_{\out}$. If $\Delta_R(g,h) < \frac{k-1}{k} (1-\rho-\eps)$, add $h$ to $\calL$.
%			\end{enumerate}
		\end{itemize}
	\item Return $\calL$.
\end{enumerate}
%
\vspace{5pt}
%
\end{algorithm}
\end{figure}

\begin{proof}
	\cref{algo:sos-decoding} describes this algorithm. In the rest of the proof, we argue the correctness of this algorithm.
	
	We first start with a lemma that readily follows from \cref{thm:sos_main}.
	
	\begin{lemma}\label{lem:pseudocodeword_list_size}
	If $\lambda \leq  \frac{\delta_{\out}^{\dec}}{12k^k} \cdot \eps$, and $t \geq 2kd\cdot \inparen{1+\frac{144k^{2k}|\Sigma_{\inn}|^{3kd}}{(\delta_{\out}^{\dec})^2\cdot \eps^2}}$, then the $\mathrm{SDP}(k,t)$ is infeasible.
	\end{lemma}

	In other words, the lemma says that $p^* < k$, and that $\mathrm{SDP}(p^*+1,t)$ is infeasible. Let $\pcod{p^*}{\cdot}$ be the $p^*$-tuple of pseudocodewords found by $\mathrm{SDP}(p^*,t)$.
%	
	\begin{lemma}\label{lem:covering}
	For any $h \in \calL\inparen{g,\frac{k-1}{k}(1-\rho-\eps)}$,
%	 and its corresponding $h^* \in \calC_{\out}$, 
there exists an $i\in [p^*]$, a set $F\sub E$ with $|F|\leq \frac{t-2kd}{2}$, and a $\sigma \in \Sigma_{\inn}^{[p^*]\times F}$ with $\pcod{p^*}{\indi{\zee_{[p^*],F} = \sigma}} > 0$, such that 
	\[
	\pcod{p^*}{\Delta_L(h, \zee_i) \given \zee_{[p^*],F} = \sigma} \leq \delta_{\out}^{\dec}.
	\] 
	\end{lemma}
	This lemma proves the correctness of the algorithm, since then $h$ will be added to $\calL$ in \hyperref[step3]{Step~\ref*{step3}}, by \cref{lem:decode_from_distrib}.
\end{proof}

\begin{proof}[Proof of \cref{lem:covering}]
	To show this, we will construct a $(p^*+1)$-tuple of pseudocodewords, say $\pcod{p^*+1}{\cdot}$, and use its infeasibility for $\mathrm{SDP}(p^*+1,t)$.
	
	Recall that $\pcod{p^*}{\cdot}$ is an SoS relaxation over the variables $\zee_{[p^*] \times E }$,
% = \{ Z_{i,e,s} \}_{i\in [p^*], e\in E,s \in \Sigma_{\inn}}$
while $\pcod{p^*+1}{\cdot}$ will be an SoS relaxation over the variables $\zee_{[p^*+1]\times E }$.
% = \{ Z_{i,e,s} \}_{i\in [p^*+1], e\in E,s\in \Sigma_{\inn}}$. 
	
	To describe the new $(p^*+1)$-tuple of pseudocodewords, we explicitly specify the corresponding pseudoexpectation operator $\pcod{p^*+1}{\cdot}$.
	Let $P$ be an arbitrary polynomial of degree $t$ over $\zee_{[p^*+1], E}$. We define a new polynomial $P_h$ over $\zee_{[p^*], E}$ by assigning the variables $\zee_{p^*+1,e}$ to be $h_e$ for every $e\in E$.
%	\begin{align*}
%		Z_{p^*+1,e,s} = \begin{cases}
%		1 \quad & h_e = s \\
%		0 \quad & h_e \neq s
%		\end{cases}
%	\end{align*}	
	Then, we define $\pcod{p^*+1}{\cdot}$ using the following
	\[
		\pcod{p^*+1}{P(\zee_{[p^*+1],E})} = \pcod{p^*}{P_h(\zee_{[p^*],E})}
	\]
	This is well-defined since the degree of $P_h$ cannot be more than $t$.
	
	By optimality of $p^*$, $\pcod{p^*+1}{\cdot}$ must be infeasible for $\mathrm{SDP}(p^*+1,t)$. It can be verified that $\pcod{p^*+1}{\cdot}$ is a valid $(p^*+1)$-tuple of pseudocodewords. Further, the constraints 
	\[
		\Delta_R(g,\zee_i) < \frac{k-1}{k}(\delta_0 - \eps)
	\]
	are respected for all $i\in [p^*]$. This is because 
	\[
		\pcod{p^*+1}{P(\zee)^2 \cdot \inparen{\Delta_R(g,\zee_i) - \frac{k-1}{k}(\delta_0 - \eps)}} = \pcod{p^*}{P_h(\zee)^2 \cdot \inparen{\Delta_R(g,\zee_i) - \frac{k-1}{k}(\delta_0 - \eps)}} < 0
	\]
	Further, for $i=p^*+1$,
	\begin{align*}
		\pcod{p^*+1}{P(\zee)^2 \cdot \inparen{\Delta_R(g,\zee_i) - \frac{k-1}{k}(\delta_0 - \eps)}} &= \pcod{p^*}{P_h(\zee)^2 \cdot \inparen{\Delta_R(g,h) - \frac{k-1}{k}(\delta_0 - \eps)}} \\
		&= \inparen{\Delta_R(g,h) - \frac{k-1}{k}(\delta_0 - \eps)} \cdot \pcod{p^*}{P_h(\zee)^2} \\
		&< 0
	\end{align*}
	Therefore, the infeasibility must be due to constraints of type (iii) in the SDP. In other words, there exist
	\begin{enumerate}
	\item a set $F\sub E$ with $|F|\leq \frac{t-2(p^*+1)d}{2}$,
	\item a $\sigma \in \Sigma_{\inn}^{[p^*+1]\times F}$ 
	%with $\pcod{p^*+1}{\zee_{[p^*+1],F,\sigma}} >0$
	, and 
	\item a pair $i,i'\in [p^*+1]$ with $i\neq i'$
	\end{enumerate}
	such that
	\[
		\pcod{p^*+1}{ \inparen{\Delta_L(\zee_i,\zee_{i'}) - \delta_{\out}^{\dec}} \cdot \indi{\zee_{[p^*+1],F} = \sigma}^2} < 0
	\]
	Case I. $i,i' \in [p^*]$:
	
	In this case,
	\begin{align*}
		0~>&~ \pcod{p^*+1}{ \inparen{\Delta_L(\zee_i,\zee_{i'}) - \delta_{\out}^{\dec}} \cdot \indi{\zee_{[p^*+1],F} = \sigma}} \\
		=~~ & \pcod{p^*+1}{ \inparen{\Delta_L(\zee_i,\zee_{i'}) - \delta_{\out}^{\dec}} \cdot \indi{\zee_{[p^*],F} = \sigma_1} \cdot \indi{\zee_{p^*+1,F} = \sigma_2}} \\
		=~~ & \pcod{p^*}{ \inparen{\Delta_L(\zee_i,\zee_{i'}) - \delta_{\out}^{\dec}} \cdot \indi{\zee_{[p^*],F} = \sigma_1} \cdot \indi{h_F = \sigma_2}} \\
		=~~ & \indi{h_F = \sigma_2} \cdot \pcod{p^*}{ \inparen{\Delta_L(\zee_i,\zee_{i'}) - \delta_{\out}^{\dec}} \cdot \indi{\zee_{[p^*],F} = \sigma_1} }
	\end{align*}
Because of the strict inequality with $0$, it must be the case that $\indi{h_F = \sigma_2} =1$, and the resulting equation contradicts the feasibility of $\pcod{p^*}{\cdot}$. So this case cannot happen.

\medskip
%	
\noindent Case II. Without loss of generality, $i'=p^*+1$.
	
%	We may assume $\pcod{p^*+1}{\indi{\zee_{[p^*+1],F} = \sigma}^2} > 0$. If not, 
%	\begin{align*}
%		&\pcod{p^*+1}{\indi{\zee_{[p^*+1],F} = \sigma}^2} = 0 \\
%		\implies & \pcod{p^*+1}{ \Delta_L(\zee_i,\zee_{i'}) \cdot \indi{\zee_{[p^*+1],F} = \sigma}^2} < 0
%	\end{align*}
%	
%	Further, it can be checked that both $i$ and $i'$ cannot be $\leq p^*$, since $\pcod{p^*}{\cdot}$ is feasible for $\mathrm{SDP}(p^*,t)$. Without loss of generality, let $i' = p^*+1$. Then,
In this case,
	\begin{align*}
		0 &~>~  \pcod{p^*+1}{ \inparen{\Delta_L(\zee_i,\zee_{p^*+1}) - \delta_{\out}^{\dec}} \cdot \indi{\zee_{[p^*+1],F} = \sigma}} \\
		&=  \pcod{p^*+1}{ \inparen{\Delta_L(\zee_i,\zee_{p^*+1}) - \delta_{\out}^{\dec}} \cdot \indi{\zee_{[p^*],F} = \sigma_1} \cdot \indi{\zee_{p^*+1,F} = \sigma_2}} \\
		&=\pcod{p^*}{ \inparen{\Delta_L(\zee_i,h) - \delta_{\out}^{\dec}} \cdot \indi{\zee_{[p^*],F} = \sigma_1} \cdot \indi{h_F = \sigma_2}}\\
		&=\indi{h_F = \sigma_2} \cdot \pcod{p^*}{ \inparen{\Delta_L(\zee_i,h) - \delta_{\out}^{\dec}} \cdot \indi{\zee_{[p^*],F} = \sigma_1}}\\
%		&=\frac{\pcod{p^*}{ \inparen{\Delta_L(\zee_i,h) - \delta_{\out}^{\dec}} \cdot \indi{\zee_{[p^*],F} = \sigma_1}}}{\pcod{p^*}{\indi{\zee_{[p^*],F} = \sigma_1}}} \\
%		&=\pcod{p^*}{ \inparen{\Delta_L(\zee_i,h) - \delta_{\out}^{\dec}} \given \zee_{[p^*],F} = \sigma_1}
	\end{align*}
	Again, as before we must have $\indi{h_F = \sigma_2} =1$, so that
	\begin{align}\label{eq:without_h}
		\pcod{p^*}{ \inparen{\Delta_L(\zee_i,h) - \delta_{\out}^{\dec}} \cdot \indi{\zee_{[p^*],F} = \sigma_1}} < 0
	\end{align}
	Finally, we may also assume $\pcod{p^*}{\indi{\zee_{[p^*+1],F} =\sigma}} >0$. If not, then
	\[
		\pcod{p^*}{ \Delta_L(\zee_i,h) \cdot \indi{\zee_{[p^*],F} = \sigma_1}} < 0
	\]
	which is impossible since $\pcod{p^*}{P(\zee)} \geq 0$ for all polynomials that are sum of squares of polynomials. It is easy to verify that $\Delta_L(\zee_i,h) \cdot \indi{\zee_{[p^*],F} = \sigma_1}$ is a sum of squares of polynomials.
	
	Therefore, dividing \cref{eq:without_h} by $\pcod{p^*}{\indi{\zee_{[p^*+1],F} =\sigma}}$, we get that the set $F$ and the assignment $\sigma_1 \in \Sigma_{\inn}^{[p^*] \times F}$ have the property that
	\[
		\pcod{p^*}{\Delta_L(\zee_i,h) \given \zee_{[p^*],F} = \sigma_1} \leq \delta_{\out}^{\dec}
	\]
	as needed.
\end{proof}

\begin{proof}[Proof of \cref{lem:pseudocodeword_list_size}]
	Suppose $\mathrm{SDP}(k,t)$ is feasible with $t = 2kd\cdot \inparen{1+\frac{144k^{2k}|\Sigma_{\inn}|^{3kd}}{(\delta_{\out}^{\dec})^2\cdot \eps^2}}$, so that there exists a $k$-tuple of pseudocodewords $\tildeEx{\cdot}$ that satisfies all the constraints in the SDP. 
	Let $\eta = \frac{\delta_{\out}^{\dec}}{12k^k} \cdot \eps$.
	
	 Then $t\geq 2kd\cdot \inparen{1+\frac{|\Sigma_{\inn}|^{3kd}}{\eta^2}}$, so that \cref{lem:condition_for_eta_good} says that there exists a set $S \sub E$ of size at most $d\cdot \frac{|\Sigma_{\inn}|^{3kd}}{\eta^2}$ such that \[\pcod{S}{\cdot} \defeq \tildeEx{~\cdot \given \zee_{[k],S}} \] is $\eta$-good.
	
	Further, since the constraint $\Delta_R(g,\zee_i) < \frac{k-1}{k}(1-\rho-\eps)$ is \emph{respected} by $\tildeEx{\cdot}$, we can also conclude that for all $i\in [k]$,
	\begin{align}\label{eqn:zee_i_in_ball}
		\pcod{S}{\Delta_R(g,\zee_i)} < \frac{k-1}{k}(1-\rho-\eps)
	\end{align}
	Moreover, for any $i,i'\in [k]$ with $i\neq i'$,
	\[
		\pcod{S}{\inparen{\Delta_L(\zee_i,\zee_{i'}) - \delta_{\out}^{\dec}}} = \Ex{\sigma \sim \calD_{[k]\times S}}{\frac{\tildeEx{ \inparen{\Delta_L(\zee_i,\zee_{i'}) - \delta_{\out}^{\dec}} \cdot \indi{\zee_{[k], S} = \sigma}^2}}{\tildeEx{\indi{\zee_{[k], S}}^2}}} \geq 0
	\]
	where $\calD_{[k]\times S}$ is the local distribution on $\zee_{[k]\times S}$ according to $\tildeEx{\cdot}$. This means that
	\[
		\pcod{S}{\Delta_L(\zee_i,\zee_{i'})} \geq \delta_{\out}^{\dec}
	\]
	Since $\lambda \leq \frac{\delta_{\out}^{\dec}}{12k^k} \cdot \eps$ and $\eta \leq \frac{\delta_{\out}^{\dec}}{12k^k} \cdot \eps$, we can apply \cref{thm:sos_main} to $\pcod{S}{\cdot}$ with $\beta = \delta_{\out}^{\dec}$ and get
	\[
		\sum_{i\in [k]} \pcod{S}{\Delta_R(g,\zee_i)} \geq \frac{k-1}{k}(1-\rho-\eps) + \pcod{S}{\Delta_R(g,\zee_{[k]})} \geq \frac{k-1}{k}(1-\rho-\eps)
	\]
	which is contradicted by \cref{eqn:zee_i_in_ball}.
\end{proof}

Finally, we instantiate \cref{thm:sos_technical} with unique decodable outer codes to obtain codes that can be efficiently decoded up to the list decoding capacity.
\begin{corollary}\label{cor:algo-main}
For every $\rho, \eps \in (0,1)$ and $k \in \N$, there exist explicit inner codes $\calC_{\inn}$ and an infinite family explicit codes $\AELC \subseteq (\F_q^d)^n$ obtained via the AEL construction that satisfy: 
\begin{enumerate}
\item $\rho(\AELC) \geq \rho$
\item For any $g \in  (\F_q^d)^n$ and any $\calH \subseteq \AELC$ with $\abs{\calH} \leq k$ that
\[
\sum_{h \in \calH} \Delta(g,h) ~\geq~ (\abs{\calH}-1) \cdot (1 - \rho - \eps) \mper
\]
\item The alphabet size $q^d$ of the code $\AELC$ can be taken to be $2^{O(k^{3k}/\eps^9)}$.
\item $\AELC$ can be decoded from radius $\frac{k-1}{k}(1-\rho-\eps)$ in time $n^{2^{O(k^{4k}/\eps^{10})}}$ with a list of size at most $k-1$.
\end{enumerate}
\end{corollary}

\begin{proof}
	The first three properties can be ensured by instantiating the AEL amplification procedure as in \cref{cor:ael_instantiation}. We briefly recall the choices in that instantiation.
	
	\begin{itemize}
		\item We picked a graph $G$ with $d = O(k^{2k}/\eps^8)$ and $\lambda \leq \eps^4/(2^{21} k^k)$. 
%
		\item We picked $\calC_{\inn} \subseteq \Sigma_{\inn}^d$ to be a code with rate $\rho_{\inn} = \rho + \eps/4$, which is $(1 - \rho, k, \eps/2)$ average-radius list decodable with erasures. The alphabet size $|\Sigma_{\inn}|$ for $\calC_{\inn}$ was $2^{O(k + 1/\eps)}$.
%
		\item Finally, we picked $\calC_{\out} \subseteq (\Sigma_{\inn}^{\rho_{\inn} \cdot d})^n$ be an outer code with rate $\rho_{\out} = 1 - \eps/4$ and distance (say) $\delta_{\out} = \eps^3/2^{15}$. We note that this code can also be decoded from a radius $\delta_{\out}^{\dec} = \frac{\eps^3}{2^{17}}$ \cite{Zemor01, GRS23} in linear time.
	\end{itemize}
%

The choice of $\lambda$ above is sufficient to ensure $\lambda \leq \frac{\delta_{\out}^{\dec}}{12k^k}\cdot \eps$, and so we can use \cref{thm:sos_technical} to conclude that there is a deterministic algorithm, that takes as input $g\in (\Sigma_{\inn}^d)^R$, runs in time $n^{O\inparen{ \frac{d \cdot k^{3k}|\Sigma_{\inn}|^{3kd}}{(\delta_{\out}^{\dec})^2\cdot \eps^2}}}$, and outputs the list $\calL\inparen{g,\frac{k-1}{k}\inparen{1-\rho-\eps}}$. With the above choice of parameters, the exponent of $n$ in the runtime becomes $2^{O(\frac{k^{4k}}{\eps^{10}})}$.
%\begin{align*}
%	\frac{d \cdot k^{3k}|\Sigma_{\inn}|^{3kd}}{(\delta_{\out}^{\dec})^2\cdot \eps^2} &= \frac{k^{2k}}{6} \cdot k^{3k} \cdot \frac{1}{\eps^2} \frac{1024}{\eps^4} 2^{O(kd(k+\frac{1}{\eps}))} \\
%	&= \frac{k^{2k}}{6} \cdot k^{3k} \cdot \frac{1}{\eps^2} \frac{1024}{\eps^4} 2^{O(\frac{k^{3k}}{\eps^7})} \\
%	 &= 2^{O(\frac{k^{4k}}{\eps^8})}
%\end{align*}
%
%Given the above parameters, we have 
%$\rho(\AELC) ~\geq~ \rho_{\out} \cdot \rho_{\inn} ~=~ (1-\eps/4) \cdot (\rho + \eps/4) ~\geq~ \rho$.
%%
%Since $\lambda \leq \eps \cdot \delta_{\out}/(6k^k)$ and $\calC_{\inn}$ is $(1-\rho, k, \eps/2)$ average-radius list decodable with erasures, we can use \cref{thm:main_technical_avg} to conclude that $\AELC$ is $(1-\rho, k, \eps)$ average-radius list decodable (with erasures) which yields the second condition. Finally, we note that the alphabet size of the code $\AELC$ is $q^d = \exp\inparen{O((k + 1/\eps) \cdot (k^{2k}/\eps^6))} = \exp\inparen{k^{3k}/\eps^7}$, which proves the claim. 
%
%We only need to argue that the codes constructed there can be decoded efficiently.
\end{proof}


\section*{Acknowledgements}
%
We are grateful to the STOC 2025 reviewers, for a careful reading of the paper, and
for helpful comments to improve the presentation.

%\section{Todo}
%%--------------- TODO Commands --------------
\newcommand{\todoc}[2]{{\textcolor{#1}{\textbf{#2}}}}

\newcommand{\todored}[1]{{\todoc{red}{\textbf{[[#1]]}}}}
\newcommand{\todogreen}[1]{\todoc{green}{\textbf{[[#1]]}}}
\newcommand{\todoblue}[1]{\todoc{blue}{\textbf{[[#1]]}}}
\newcommand{\todoorange}[1]{\todoc{orange}{\textbf{[[#1]]}}}
\newcommand{\todobrown}[1]{\todoc{brown}{\textbf{[[#1]]}}}
\newcommand{\todogray}[1]{\todoc{gray}{\textbf{[[#1]]}}}
\newcommand{\todopink}[1]{\todoc{pink}{\textbf{[[#1]]}}}
\newcommand{\todopurple}[1]{\todoc{purple}{\textbf{[[#1]]}}}
\newcommand{\todo}[1]{\todored{TODO: #1}}
\newcommand{\TODO}[1]{\todored{TODO: #1}}
\newcommand{\orange}[1]{{\todoc{orange}{[[} #1 \todoc{orange}{]]}}}
%% To disable colored comments, just uncomment this line: 
%\renewcommand{\todoc}[2]{\relax}

%----------- Comment command for each person ------------
\newcommand{\ma}[1]{\todored{Ma: #1}}
\newcommand{\cmt}[1]{\todoorange{Comment: #1}}
\newcommand{\Update}[1]{{TODO: #1}}
\newcommand{\jwp}[1]{\todored{JWP: #1}}

\newif\ifrevise
\newif\ifrevisenew
\newif\iffinal

\revisetrue
\revisenewtrue
\finaltrue

\revisefalse
\revisenewfalse
\finalfalse

\newcommand{\revisionC}[2]{\ifrevise \todoorange{#1} : \orange{#2} \else#2\fi}
% \newcommand{\revisionB}[1]{\ifrevisenew \textcolor{red}{\DIFdelbegin \DIFdel{#1} \DIFdelend} \else \fi}
 \newcommand{\revisionB}[1]{\ifrevisenew\textcolor{red}{\sout{#1}}\else\fi}
\newcommand{\revisionA}[1]{\ifrevisenew\textcolor{blue}{#1}\else#1\fi}
\newcommand{\revisionD}[1]{\iffinal\textcolor{blue}{#1}\else#1\fi}

\bibliographystyle{alphaurl}
\bibliography{macros,madhur}

\appendix
\section{Auxiliary SoS claims}\label{sec:appendix}
%\snote{We prove $\eta$-goodness and EML for pseudocodewords here. May move these to appendix eventually. .}

This appendix is adapted from \cite{JST23} to prove properties about $\eta$-good pseudocodewords and how one can obtain them via random conditioning. The key change is that we need these notions for $k$-tuples of pseudocodewords. Finally, we prove a generalization of the expander mixing lemma for such $k$-tuples of pseudowords.


%\tnote{does the proof use any of this cartesian notation? or is it only for the appendix?}



\subsection{Low average correlation}


\begin{lemma}\label{lem:avg_corr}
    Let $\tildeEx{\cdot}$ be an $\eta$-good $k$-tuple of pseudocodewords.
    
    Let $\{X_{\li}\}_{\li \in L}$ be a collection of $kd$-local functions on $\Sigma_{\inn}^{[k]\times E}$, such that $X_{\li}(f)$ only depends $f_{[k],\li}$. Likewise, let $\{Y_{\ri}\}_{\ri \in R}$ be a collection of $kd$-local functions on $\Sigma_{\inn}^{[k]\times E}$, such that $Y_{\ri}(f)$ only depends $f_{[k],\ri}$.
    
    Then,
    \begin{align*}
         \tildeEx{\Ex{\li,\ri}{X_{\li}(\zee)\cdot Y_{\ri}(\zee) }} - \tildeEx{\Ex{\li}{X_{\li}(\zee)}} \cdot \tildeEx{\Ex{\ri}{Y_{\ri}(\zee)}} \leq \eta \cdot \inparen{ \max_{\li} \norm{X_{\li}}_{\infty}} \cdot \inparen{\max_{\ri} \norm{Y_{\ri}}}_{\infty}
    \end{align*}
\end{lemma}
\begin{proof}
    \begin{align*}
    	& \tildeEx{\Ex{\li,\ri}{X_{\li}(\zee)\cdot Y_{\ri}(\zee) }} - \tildeEx{\Ex{\li}{X_{\li}(\zee)}} \cdot \tildeEx{\Ex{\ri}{Y_{\ri}(\zee)}}  \\
    	=~ & \Ex{\li,\ri}{\tildeEx{X_{\li}(\zee)\cdot Y_{\ri}(\zee) } - \tildeEx{X_{\li}(\zee)} \cdot \tildeEx{Y_{\ri}(\zee)} } \\
    	=~ & \Ex{\li,\ri}{ \sum_{\substack{\alpha \in \Sigma_{\inn}^{[k]\times N(\li)} \\ \beta \in \Sigma_{\inn}^{[k]\times N(\ri)}}} X_{\li}(\alpha) \cdot Y_{\ri}(\beta) \tildeEx{Z_{[k],N(\li),\alpha} \cdot Z_{[k],N(\ri),\beta} } - \sum_{\substack{\alpha \in \Sigma_{\inn}^{[k]\times N(\li)} \\ \beta \in \Sigma_{\inn}^{[k]\times N(\ri)}}} X_{\li}(\alpha) \cdot Y_{\ri}(\beta) \tildeEx{Z_{[k],N(\li),\alpha}} \cdot \tildeEx{Z_{[k],N(\ri),\beta} } } \\
    	\leq~ & \Ex{\li,\ri}{ \norm{X_{\li}}_{\infty} \cdot \norm{Y_{\ri}}_{\infty} \sum_{\substack{\alpha \in \Sigma_{\inn}^{[k]\times N(\li)} \\ \beta \in \Sigma_{\inn}^{[k]\times N(\ri)}}} \abs*{ \tildeEx{Z_{[k],N(\li),\alpha} \cdot Z_{[k],N(\ri),\beta} } -  \tildeEx{Z_{[k],N(\li),\alpha}} \cdot \tildeEx{Z_{[k],N(\ri),\beta} } } } \\
    	=~ & \Ex{\li,\ri}{ \norm{X_{\li}}_{\infty} \cdot \norm{Y_{\ri}}_{\infty} \cdot  \tildeCov{\zee_{[k],N(\li)}}{\zee_{[k],N(\ri)}} } \\
    	\leq~ & \inparen{\max_{\li} \norm{X_{\li}}_{\infty}} \cdot \inparen{\max_{\ri} \norm{Y_{\ri}}_{\infty}} \cdot \Ex{\li,\ri}{\tildeCov{\zee_{[k],N(\li)}}{\zee_{[k],N(\ri)}}} \\
    	\leq~ & \eta \cdot \inparen{\max_{\li} \norm{X_{\li}}_{\infty}} \cdot \inparen{\max_{\ri} \norm{Y_{\ri}}_{\infty}}.\qedhere
    \end{align*}
\end{proof}

\subsection{Conditioning reduces variance}

We next show how to obtain the $\eta$-good property by randomly conditioning the $k$-tuple of pseudocodewords. This is a straightforward generalization of the argument in \cite{JST23}, which can be seen as the $k=1$ case.

Just as in \cite{JST23}, we start with a lemma from \cite{BRS11} that quantifies the decrease in variance of the local distribution corresponding to a set $S$ when conditioning on another small set $T$, in terms of covariances between the two sets $S$ and $T$.

\begin{lemma}[{\cite[Lemma 5.2]{BRS11}}]
	Let $\tildeEx{\cdot}$ be a $k$-tuple of pseudocodewords of degree $t$. Let $S,T$ be subsets of $[k]\times E$ of size at most $t/2$ each. Then,
	\[
		\tildeVar{\zee_S \given \zee_T} \leq \tildeVar{\zee_S} - \frac{1}{|\Sigma_{\inn}|^{|T|}} \cdot \sum_{\substack{\alpha \in \Sigma_{\inn}^S \\ \beta \in \Sigma_{\inn}^T}} \frac{\inparen{\tildeCov{\zee_{S,\alpha}}{\zee_{T,\beta}}}^2}{\tildeVar{\zee_{T,\beta}}}
	\]
\end{lemma}

Now we use this lemma to track the decrease in average (pseudo-)variance $\Ex{\li \in L}{\tildeVar{\zee_{[k],N(\li)}}}$ when conditioning on $\zee_{[k],N(\ri)}$ for a random $\ri\in R$.

\begin{lemma}\label{lem:avg_conditioning}
	\begin{align*}
		\Ex{\ri\in R}{\Ex{\li\in L}{\tildeVar{\zee_{[k],N(\li)} \given \zee_{[k],N(\ri)}}}} \leq \Ex{\li\in L}{\tildeVar{\zee_{[k],N(\li)}}} - \frac{1}{|\Sigma_{\inn}|^{3kd}} \cdot \inparen{\Ex{\li,\ri}{\tildeCov{\zee_{[k],N(\li)}}{\zee_{[k],N(\ri)}}}}^2
	\end{align*}
\end{lemma}

\begin{proof}
	\begin{align*}
		&~~~\Ex{\ri\in R}{\Ex{\li\in L}{\tildeVar{\zee_{[k],N(\li)} \given \zee_{[k],N(\ri)}}}} \\
		~&\leq~ \Ex{\ri\in R}{\Ex{\li\in L}{\tildeVar{\zee_{[k],N(\li)}} - \frac{1}{|\Sigma_{\inn}|^{kd}} \sum_{\substack{\alpha \in \Sigma_{\inn}^{[k]\times N(\li)} \\ \beta \in \Sigma_{\inn}^{[k]\times N(\ri)}}} \frac{ \inparen{ \tildeCov{\zee_{[k],N(\li),\alpha}}{\zee_{[k],N(\ri),\beta}}}^2}{\tildeVar{\zee_{[k],N(\ri),\beta} }}}} \\
		~&\leq~ \Ex{\li\in L}{\tildeVar{\zee_{[k],N(\li)}}} - \frac{1}{|\Sigma_{\inn}|^{kd}} \cdot \Ex{\substack{\li\in L \\ \ri\in R}}{ \sum_{\substack{\alpha \in \Sigma_{\inn}^{[k]\times N(\li)} \\ \beta \in \Sigma_{\inn}^{[k]\times N(\ri)}}} \inparen{ \tildeCov{\zee_{[k],N(\li),\alpha}}{\zee_{[k],N(\ri),\beta}}}^2} \\
		~&\leq~ \Ex{\li\in L}{\tildeVar{\zee_{[k],N(\li)}}} - \frac{1}{|\Sigma_{\inn}|^{3kd}} \cdot \Ex{\substack{\li\in L \\ \ri\in R}}{ \inparen{ \sum_{\substack{\alpha \in \Sigma_{\inn}^{[k]\times N(\li)} \\ \beta \in \Sigma_{\inn}^{[k]\times N(\ri)}}}  \abs*{\tildeCov{\zee_{[k],N(\li),\alpha}}{\zee_{[k],N(\ri),\beta}}}}^2} \\
		~&=~ \Ex{\li\in L}{\tildeVar{\zee_{[k],N(\li)}}} - \frac{1}{|\Sigma_{\inn}|^{3kd}} \cdot \Ex{\substack{\li\in L \\ \ri\in R}}{ \inparen{ \tildeCov{\zee_{[k],N(\li)}}{\zee_{[k],N(\ri)}}}^2} \\
		~&=~ \Ex{\li\in L}{\tildeVar{\zee_{[k],N(\li)}}} - \frac{1}{|\Sigma_{\inn}|^{3kd}} \cdot \inparen{\Ex{\substack{\li\in L \\ \ri\in R}}{  \tildeCov{\zee_{[k],N(\li)}}{\zee_{[k],N(\ri)}}}}^2
	\end{align*}
\end{proof}

Now that we can use average correlation to quantify the decrease in variance, we show that for a $k$-tuple of pseudocodewords with high-enough degree, there always exists a constant-sized subset of $R$ such that randomly conditioning on this set gives an $\eta$-good $k$-tuple of pseudocodewords. 
%If not, we can keep decreasing variance which must remain non-negative.

\begin{lemma}\label{lem:condition_for_eta_good}
	Let $\tildeEx{\cdot}$ be a $k$-tuple of pseudocodewords of degree $t \geq 2kd\inparen{1+\frac{|\Sigma_{\inn}|^{3kd}}{\eta^2}}$. Then there exists a set $S \subseteq E$ of size at most $d\cdot \frac{|\Sigma_{\inn}|^{3kd}}{\eta^2}$ such that the conditioned pseudocodeword $\tildeEx{\cdot \given \zee_{[k],S}}$ is $\eta$-good.
\end{lemma}

\begin{proof}
%	Let $\pcod{0}{\cdot} = \tildeEx{\cdot}$.
	For an integer $C>0$ to be chosen later, consider the following sequence of pseudocodewords, obtained by sequentially conditioning on $\zee_{[k],\ri}$ for a random $\ri$. For $c\in [C+1]$, define
	\[
		\Psi(c) = \Ex{\ri_1,\ri_2, \cdots \ri_c}{ \Ex{\li}{\tildeVar{ \zee_{[k],\li} \given \zee_{[k],\ri_1},\zee_{[k],\ri_2},\cdots ,\zee_{[k],r_c}}}}
	\]
	We have that
	\[
		1 \geq \Psi(0) \geq \Psi(1) \geq \cdots \Psi(C+1) \geq 0
	\]
	Therefore, there must exist a $c^*$, with $0\leq c^* \leq C$, such that
	\[
		\Psi(c^*) - \Psi(c^*+1) \leq \frac{1}{C+1} < \frac{1}{C} = \frac{\eta^2}{|\Sigma_{\inn}|^{3kd}}
	\]
	For an $R'\sub R$, let us denote $\pcod{R'}{\cdot} = \tildeEx{~\cdot~ \given \zee_{[k],N(R')}}$, and its associated variance and covariance operators be $\widetilde{\operatorname{Var}}^{(R')}$ and $\widetilde{\operatorname{Cov}}^{(R')}$. Then,
	\[
		\Psi(c^*) - \Psi(c^*+1) = \Ex{R' \sim R^{c^*}}{\Ex{\li}{\widetilde{\operatorname{Var}}^{(R')}\left[\zee_{[k],\li}\right]}} - \Ex{R'\sim R^{c^*}}{ \Ex{\ri_{c^*+1}}{\Ex{\li}{\widetilde{\operatorname{Var}}^{(R')}\left[\zee_{[k],\li} \given \zee_{[k], \ri_{c^*+1} }\right]}}}
	\]
	The set $R'$ above is sampled by sampling $c^*$ times from $R$ uniformly at random and with replacement. Therefore, there exists a set $R'$ of size at most $c^*$ such that
	\[
		\Ex{\li}{\widetilde{\operatorname{Var}}^{(R')}\left[\zee_{[k],\li}\right]} -  \Ex{\ri_{c^*+1}}{\Ex{\li}{\widetilde{\operatorname{Var}}^{(R')}\left[\zee_{[k],\li} \given \zee_{[k], \ri_{c^*+1} }\right]}} < \frac{\eta^2}{|\Sigma_{\inn}|^{3kd}}
	\]
	Applying the contrapositive of \cref{lem:avg_conditioning} to $\pcod{R'}{\cdot}$, we get that
		\begin{align*}
		&\Ex{\li,\ri}{\widetilde{ \operatorname{Cov}}^{(R')} \left[ \zee_{[k],\li}, \zee_{[k],\ri} \right]}\leq \eta \\
		\implies &\Ex{\li,\ri}{\widetilde{\operatorname{Cov}}\left[\zee_{[k],\li}, \zee_{[k],\ri} \given \zee_{[k],N(R')} \right]} \leq \eta
	\end{align*}
	Therefore, the set $S = N(R')$ proves the lemma.
%	Let us denote $\pcod{S}{\cdot} = \Ex{\ri_1,\cdots ,\ri_{c^*}}{\tildeEx{~\cdot~ \given \zee_{[k],S},\zee_{[k],\ri_2},\cdots ,\zee_{[k],\ri_{c^*}}}}$, and its associated variance and covariance operators be $\widetilde{\operatorname{Var}}^{(c^*)}$ and $\widetilde{\operatorname{Cov}}^{(c^*)}$. Then 
%	\[
%		\Psi(c^*+1) - \Psi(c^*) = \Ex{\li}{\widetilde{\operatorname{Var}}^{(c^*)}\left[\zee_{[k],\li}\right]} - \Ex{\ri_{c^*+1}}{\Ex{\li}{\widetilde{\operatorname{Var}}^{(c^*)}\left[\zee_{[k],\li} \given \zee_{[k],\ri_{c^*+1}}\right]}}
%	\]
%	Applying the contrapositive of \cref{lem:avg_conditioning} to $\pcod{c^*}{\cdot}$, we get that
%	\begin{align*}
%		&\Ex{\li,\ri}{\widetilde{ \operatorname{Cov}}^{(c^*)} \left[ \zee_{[k],\li}, \zee_{[k],\ri} \right]}\leq \eta \\
%		\implies &\Ex{\ri_1,\cdots ,\ri_{c^*}}{\widetilde{\operatorname{Cov}}\left[\zee_{[k],\li}, \zee_{[k],\ri} \given \zee_{[k],\ri_1}, \cdots ,\zee_{[k],\ri_{c^*}} \right]} \leq \eta
%	\end{align*}
%	Therefore, there exists a set of $c^*$ right vertices $\ri_1,\ri_2,\cdots ,\ri_{c^*}$ such that conditioning on $\zee_{[k], \{\ri_1,\ri_2,\cdots ,\ri_{c^*}\}}$ makes the pseudocodeword $\eta$-good.
\end{proof}

\subsection{Expander Mixing Lemma}

\begin{lemma}[EML for pseudoexpectations]\label{lem:eml_sos}
	Let $\tildeEx{\cdot}$ be a $k$-tuple of pseudocodewords of degree at least $2k d$. Let $G = (L,R,E)$ be a bipartite $\lambda$-expander. Recall that $\li \sim \ri$ if and only if $(\li,\ri)$ is an edge in $G$.
	
	Let $\{X_{\li}\}_{\li \in L}$ be a collection of $kd$-local functions on $\Sigma_{\inn}^{[k]\times E}$, such that $X_{\li}(f)$ only depends $f_{[k],\li}$. Likewise, let $\{Y_{\ri}\}_{\ri \in R}$ be a collection of $kd$-local functions on $\Sigma_{\inn}^{[k]\times E}$, such that $Y_{\ri}(f)$ only depends $f_{[k],\ri}$. 
	Then,
	\begin{align*}
		\abs*{\tildeEx{\Ex{\li\sim \ri}{X_{\li}(\zee) \cdot Y_{\ri}(\zee)} - \Ex{\li,\ri}{X_{\li}(\zee) \cdot Y_{\ri}(\zee)}}} \leq \lambda \cdot \sqrt{\tildeEx{\Ex{\li}{X_{\li}(\zee)^2}}} \cdot \sqrt{\tildeEx{\Ex{\ri}{Y_{\ri}(\zee)^2}}}
	\end{align*}
\end{lemma}
\begin{proof}
    Let $A_G$ be the $L \times R$ normalized biadjacency matrix of $G$, so that
    \[
    	A_G(\li,\ri) = \begin{cases}
    		\frac{1}{d}, \qquad (\li,\ri) \in E \\
    		0, \qquad (\li,\ri)\not\in E
    	\end{cases}
    \]
    The singular value decomposition for $A_G$ can be written as
    \[
    	A_G = \sum_{i=1}^n \sigma_i u_i v_i^T
    \]
    with $\sigma_1 \geq \cdots \geq \sigma_n \geq 0$, and $\{u_i\}_{i\in [n]}$, $\{v_i\}_{i \in [n]}$ being two orthonormal bases of $\R^n$. Moreover, since the graph is regular, $\sigma_1 =1$ and $u_1 = v_1 = \frac{1}{\sqrt{n}} \one$.
    
    Let $X(\zee)$ be the vector-valued local function defined as the $n$-dimensional vector with coordinates corresponding to $\li \in L$, and the respective entry being $X_{\li}(\zee)$. Likewise, let $Y(\zee)$ be the vector-valued local function so that coordinates correspond to $\ri\in R$, and the entries being $\{Y_{\ri}(\zee)\}_{\ri\in R}$. Then,
%
    \begin{align*}
    	&\abs*{\tildeEx{\Ex{\li\sim \ri}{X_{\li}(\zee) \cdot Y_{\ri}(\zee)} - \Ex{\li,\ri}{X_{\li}(\zee) \cdot Y_{\ri}(\zee)}}} \\
    	=~ &\abs*{ \tildeEx{\frac{1}{n} X(\zee)^T A_G Y(\zee)} - \tildeEx{\frac{1}{n} X(\zee)^T (\frac{1}{\sqrt{n}} \one) (\frac{1}{\sqrt{n}} \one)^T Y(\zee)}} \\
    	=~ &\frac{1}{n} \abs*{\tildeEx{X(\zee)^T \inparen{A_G - (\frac{1}{\sqrt{n}} \one) (\frac{1}{\sqrt{n}} \one)^T}Y(\zee)}} \\
    	=~ &\frac{1}{n} \abs*{\tildeEx{X(\zee)^T \inparen{\sum_
    	{i=2}^n \sigma_i u_iv_i^T}Y(\zee)}} \\
    	\leq~ &\frac{1}{n} \abs*{\tildeEx{\sum_{i=2}^n \sigma_i ~ \inparen{X(\zee)^T u_i}\cdot \inparen{Y(\zee)^T v_i}}} \\
    	=~ &\frac{1}{n} \abs*{\tildeEx{\sum_{i=2}^n \inparen{\sqrt{\sigma_i} \cdot X(\zee)^T u_i}\cdot \inparen{\sqrt{\sigma_i}\cdot Y(\zee)^T v_i}}} \\
    	\leq~ &\frac{1}{n} \inparen{\tildeEx{\sum_{i=2}^n \inparen{\sqrt{\sigma_i} \cdot X(\zee)^T u_i}^2}}^{1/2} \cdot \inparen{\tildeEx{\sum_{i=2}^n \inparen{\sqrt{\sigma_i}\cdot Y(\zee)^T v_i}^2}}^{1/2} && (\text{{SoS Cauchy--Schwarz}})\\
    	\leq~ &\frac{1}{n}\cdot \lambda \cdot \inparen{\tildeEx{\sum_{i=2}^n \inparen{ X(\zee)^T u_i}^2}}^{1/2} \cdot \inparen{\tildeEx{\sum_{i=2}^n \inparen{ Y(\zee)^T v_i}^2}}^{1/2} \\
%    	\leq~ &\frac{1}{n} \lambda \cdot \sum_{i=2}^n \abs*{ \tildeEx{\inparen{X(\zee)^T u_i}\cdot \inparen{Y(\zee)^T v_i}}} \\
%    	\leq~ &\frac{1}{n} \lambda \cdot \parens*{ \tildeEx{ \sum_{i=2}^n \inparen{X(\zee)^T u_i}^2}}^{1/2} \cdot \parens*{ \tildeEx{\sum_{i=2}^n \inparen{Y(\zee)^T v_i}^2}}^{1/2} && [\text{\snote{SoS Cauchy Schwarz}}]\\
    	=~ &\frac{1}{n} \lambda \cdot \inparen{ \tildeEx{ X(\zee)^T X(\zee) - (X(\zee)^T u_1)^2}}^{1/2} \cdot \inparen{ \tildeEx{ Y(\zee)^T Y(\zee) - (Y(\zee)^T v_1)^2}}^{1/2} && (\text{{Using orthonormality}})\\
    	\leq~ &\frac{1}{n} \lambda \cdot \inparen{ \tildeEx{ X(\zee)^T X(\zee)}}^{1/2} \cdot \inparen{ \tildeEx{ Y(\zee)^T Y(\zee)}}^{1/2} \\
    	 =~ & \lambda \cdot \inparen{ \tildeEx{\Ex{\li}{ X_{\li}(\zee)^2}}}^{1/2} \cdot \inparen{ \tildeEx{ \Ex{\ri}{Y_{\ri}(\zee)^2}}}^{1/2}. \qedhere
    \end{align*}
\end{proof}



\end{document}

%
%
%
%
%
%
%
%
%
%
%
%
%
%
%
%
