\section{Decoding using SoS up to the generalized Singleton bound}

Until now, we have proved that when the inner and outer codes as well as the graph used in AEL amplification are suitably chosen, then the list of codewords around an arbitrary center (corrupted codeword) is of small size. 
In this section, we describe a polynomial time algorithm that takes as input the corrupted codeword, and outputs this list. 


This algorithm is based on the Sum-of-Squares (SoS) hierarchy of semidefinite programs, which gives a systematic way of tightening convex relaxations.  SoS has been used before for decoding algorithms for codes constructed using spectral expanders in \cite{AJQST20, JQST20, RR23, JST23}. 
Among these, \cite{JST23} used the SoS hierarchy to give a list decoding algorithm for AEL up to the Johnson bound, yielding rate $R$ codes efficiently decodable up to $1-\sqrt{R}-\eps$ for any $\eps>0$. 
We will heavily rely on their framework but will improve the decoding radius to $1-R-\eps$ by proving an SoS analog of the generalized Singleton bound for appropriately instantiated AEL amplification.

%
Before going into the proof, we describe additional preliminaries for the SoS hierarchy. Readers familiar with the general terms and concepts can skip ahead to \cref{sec:sos_proof} where we define a specific SoS relaxation for the AEL code, and prove an SoS analog of the generalized Singleton bound. Finally, \cref{sec:sos_algo} describes the decoding algorithm in detail.

%\snote{Just wrote some text. Some of this will go to intro, some will get scattered. Mix above with the next para to produce something reasonable.}

\subsection{Additional Preliminaries: Sum-of-Squares Hierarchy}\label{sec:sos_prelims}

The sum-of-squares hierarchy of semidefinite programs (SDPs) provides a family of increasingly
powerful convex relaxations for several optimization problems. 
%
Each ``level" $t$ of the hierarchy is parametrized by a set of constraints corresponding to
polynomials of degree at most $t$ in the optimization variables. While the relaxations in the
hierarchy can be viewed as  semidefinite programs of size $n^{O(t)}$ \cite{BS14, FKP19}, 
it is often convenient to view the solution as a linear operator, called the ``pseudoexpectation" operator.
%
% It is well-known that such constrained pseudoexpectation operators of SoS-degree $t$ can be described as solutions to semidefinite programs of size $n^{O(t)}$ \cite{BS14, Laurent09}. This hierarchy of semidefinite programs for increasing $t$ is known as the SoS hierarchy.


%
%
%\fnote{Someone familiar with SoS will likely want to skip most of this paragraph (only use it as a reference as needed) and jump to the AEL part of the preliminaries.}
%\vspace{-5 pt}
%
\paragraph{Pseudoexpectations}
 
%
Let $t$ be a positive even integer and fix an alphabet $\Sigma$ of size $s$. Let $\zee = \{Z_{i,j}\}_{i\in[m],j\in\Sigma}$ be a collection of variables and $\R[\zee]^{\leq t}$ be the vector space of polynomials of degree at most $t$ in the variables $\zee$ (including the constants).


\begin{definition}[Constrained Pseudoexpectations]\label{def:constraints_on_sos}
%	 
Let $\calS = \inbraces{f_1 = 0, \ldots, f_m = 0, g_1 \geq 0, \ldots, g_r \geq 0}$ be a system of
polynomial constraints in $\zee$, with each polynomial in $\calS$ of degree at most $t$. We say $\tildeEx{\cdot}$ is a pseudoexpectation operator of SoS-degree $t$, over the variables $\zee$  respecting $\calS$, if it is a linear operator $ \tildeEx{\cdot}: \R[\zee]^{\leq t} \rightarrow \R$ such that:
	%
	\begin{enumerate}
	\item $\tildeEx{1} = 1$.
    \item $\tildeEx{p^2} \geq 0$ if $p$ is a polynomial in $\zee = \{Z_{i,j}\}_{i\in [m],j\in \Sigma}$ of degree $\leq t/2$.
	\item $\tildeEx{p \cdot f_i} = 0$,  $\forall\, i \in [m]$ and $\forall\, p$ such that $\deg(p \cdot f_i) \leq t$.
	\item $\tildeEx{p^2 \cdot \prod_{i \in S} g_i} \geq 0$, $\forall\, S \subseteq [r]$ and $\forall\, p$ such that $\deg(p^2\cdot \prod_{i \in S} g_i) \leq t$.
	\end{enumerate}
\end{definition}
~


%\tnote{Suggestion -- directly define the constrained version. Attempting to make the connection with assignments a bit more explicit below.}

% An SoS solution of degree $t$, or a pseudoexpectation of SoS-degree $t$, over the variables $\zee$ is represented by a linear operator $ \tildeEx{\cdot}: \R[\zee]^{\leq t} \rightarrow \R$ such that:
%%
%\vspace{-5 pt}
%%
%\begin{enumerate}[(i)]
%    \item $\tildeEx{1} = 1$.
%    \item $\tildeEx{p^2} \geq 0$ if $p$ is a polynomial in $\zee = \{Z_{i,j}\}_{i\in [m],j\in [q]}$ of degree $\leq t/2$.
%\end{enumerate}
%%
%\vspace{-5 pt}
%%
%\tnote{Is the note needed? It is reiterating that it is a linear operator}
% Note that linearity implies $\tildeEx{p_1} + \tildeEx{p_2} = \tildeEx{p_1+p_2}$ and $\tildeEx{c\cdot
%  p_1} = c \cdot \tildeEx{p_1}$ for $c\in \R$, for $p_1, p_2 \in \R[\zee]^{\leq t}$.
%%
%This also allows for a succinct representation of $\tildeEx{\cdot}$ using any basis for $\R[\zee]^{\leq t}$.
%
%\tnote{ }

% to $m$ variables in alphabet $[q]$.

Let $\mu$ be a distribution over the set of assignments, $\Sigma^m$.  Define the following collection of random variables,  \[\zee = \braces[\big]{ \, Z_{i,j}  = \indi{ i \mapsto j}\, \mid \, i\in[m],\, j\in\Sigma } .\] Then, setting $\tildeEx{p(\zee)} = \Ex{\mu}{p(\zee)} $ for any polynomial $p(\cdot)$ defines an (unconstrained) pseudoexpectation operator. However, the reverse is not true when $t < m$, and there can be degree-$t$ pseudoexpectations that do not correspond to any genuine distribution, $\mu$. Therefore, the set of all pseudoexpectations should be seen as a relaxation for the set of all possible distributions over such assignments. The upshot of this relaxation is that it is possible to optimize over the set. Under certain conditions on the bit-complexity of solutions~\cite{OD16, RW17:sos}, one can optimize over the set of degree-$t$ pseudoexpectations in time $m^{O(t)}$ via SDPs.

%   setting the true expectation operator under this distribution is also a . 
%
%
%Given an assignment $f: [m] \rightarrow [q]$, the operator $\tildeEx{Z_{i_1,j_1}\cdots Z_{i_k,j_k}} = \indi{ j_t = f(i_t) \; \forall\, t }$
%\[
%f: [m] \rightarrow [q] ~\mapsto~  \tildeEx{Z_{i_1,j_1}\cdots Z_{i_k,j_k}} = \indi{ j_t = f(i_t) \; \forall\, t }
%\]
%can be seen as a pseudoexpectation that assigns the value $1$ to a monomial consistent with $f$ and $0$ otherwise. 

%
%This can be extended via linearity to all
%polynomials, and then by convexity of the constraints to all distributions over assignments.
%


 
%
%We next define what it means for pseudoexpectations to satisfy some problem-specific constraints.
%\[
% \left\{ \substack{\text{ True expectations} \\
%\text{ \ie for an assignment } f: [m] \rightarrow [q] \\\tildeEx{Z_{} \cdots Z_{}} =  } \right\} \subset  \{ \text{ SoS Pseudoexpectations }  \}
%\]

% Indeed, any assignment $f: [m] \rightarrow [q]$ has a corresponding pseudoexpectation described below.

% Let $z^{(f)}_{i,j} = \indi{f(i) = j}$, where $\indi{\cdot}$ is the $0/1$ indicator function. Then the pseudoexpectation corresponding to assignment $f$ is given by
% \begin{align*}
% 	\PExp^{(f)}[{p(\zee)}] = p\inparen{\inbraces{z^f_{i,j}}_{i\in[m],j\in[q]}}
% \end{align*}

% for every polynomial $p$ of degree at most $t$.

% As can be verified easily, the set of pseudoexpectations is convex, and so we can also extend this correspondence to distributions over assignments in a natural way. If $\PExp^{(\calD)}[{\cdot}]$ is the pseudoexpectation corresponding to a distribution $\calD$ over assignments, it holds that
% \[
% 	\PExp^{(\calD)}[{p(\zee)}] = \Ex{f\sim \calD}{p\inparen{\inbraces{z^f_{i,j}}_{i\in[m],j\in[q]}}}
% \]
% which explains the term pseudo-expectation.

% However, pseudoexpectations only exist for low-degree polynomials, and there can be pseudoexpectation operators that do not correspond to any genuine distribution over assignments. The main reason for working with pseudoexpectations instead of genuine distributions is that the set of SoS-degree $t$ can be optimized over in time $m^{O(t)}$ via SDPs.

% As a relaxation, pseudoexpectations are used in efficient algorithm design when coupled with
% suitable rounding algorithms. 
% %
% For such applications (including ours), it is important to look at relaxations that satisfy certain problem-specific constraints. We define next what it means for pseudoexpectations to satisfy constraints.

% \paragraph{Constrained Pseudoexpectations}
% 

%
%
\paragraph{Local constraints and local functions.}
%
Any constraint that involves at most $k$ variables from $\zee$, with $k\leq t$, can be written as a degree-$k$ polynomial, and such constraints may be enforced into the SoS solution.
%
% \paragraph{Canonical usage}
%
In particular, we will always consider the following canonical constraints on the variables $\zee$.
\ifnum\confversion=1
\begin{align*}
&Z_{i,j}^2 = Z_{i,j},\ \forall i\in[m],j\in[s] \\
\text{and} \quad &\sum_j Z_{i,j} = 1,\ \forall i\in[m] \mper
\end{align*}
\else
\[
Z_{i,j}^2 = Z_{i,j},\ \forall \,i\in[m],j\in\Sigma
\quad \text{and} \quad 
\sum_j Z_{i,j} = 1,\ \forall\, i\in[m] \mper
\]
\fi
% As shown in the previous section, an assignment $f:[m]\rightarrow [q]$ is encoded using $m$ characteristic vectors, and so we wish to impose the following local constraints on the variables $\zee$:
%
% \begin{enumerate}[(i)]
% 	\item $Z_{i,j}^2 = Z_{i,j},\ \forall i\in[m],j\in[q]$.
% 	\item $\sum_j Z_{i,j} = 1,\ \forall i\in[m]$.
% \end{enumerate}
%
% These constraints are enforced as described in \cref{def:constraints_on_sos}, and we will henceforth not explicitly mention it. \snote{See Madhur's comment.}
%
We will also consider additional constraints and corresponding polynomials, defined by ``local" functions. For any $f\in \Sigma^m$ and $M\sub [m]$, we use $f_M$ to denote the restriction $f|_M$, and $f_i$ to denote $f_{\{i\}}$ for convenience.
%
\begin{definition}[$k$-local function]
	A function $\mu: \Sigma^m \rightarrow \R$ is called $k$-local if there is a set $M\subseteq [m]$ of size $k$ such that $\mu(f)$ only depends on $\inbraces{f(i)}_{i\in M}$, or equivalently, $\mu(f)$ only depends on $f|_M$.
	
	If $\mu$ is $k$-local, we abuse notation to also use $\mu: \Sigma^M \rightarrow \R$ with $\mu(\alpha) = \mu(f)$ for any $f$ such that $f|_M=\alpha$. It will be clear from the input to the function $\mu$ whether we are using $\mu$ as a function on $\Sigma^m$ or $\Sigma^M$.
\end{definition}

Let $\mu:\Sigma^m\rightarrow \R$ be a $k$-local function that depends on coordinates $M\subseteq [m]$ with $|M|=k$. Then $\mu$ can be written as a degree-$k$ polynomial $P_{\mu}$ in $\zee$:
\[
	P_{\mu}(\zee) = \sum_{\alpha \in \Sigma^M} \parens[\Big]{\mu(\alpha) \cdot\prod_{i\in M} Z_{i,\alpha_i}}
\]

% To see how $p_{\mu}$ is related to the $k$-local function $\mu$, observe that $p_{\mu}\inparen{\zee = \inbraces{z^{(f)}_{i,j}}} = \mu(f)$.

With some abuse of notation, we let $\mu(\zee)$ denote $P_{\mu}(\zee)$. We will use such $k$-local
functions inside $\tildeEx{\cdot}$ freely without worrying about their polynomial
representation. For example, $\tildeEx{ \indi{\zee_{i} \neq j}}$ denotes $\tildeEx{ 1- Z_{i,j}}$. Likewise, sometimes we will say we set $\zee_i = j$ to mean that we set $Z_{i,j} = 1$ and $Z_{i,j'} = 0$ for all $j'\in \Sigma \backslash \{j\}$.

% $\tildeEx{\cdot}$ operator applied to the polynomial corresponding to the $1$-local function $\mu: [q]^m \rightarrow \R$ which is defined as: $\mu(f)$ is $1$ if $f_i = 0$ and $\mu(f) = 0$ otherwise.
% %
The notion of $k$-local functions can also be extended from real-valued functions to vector-valued functions straightforwardly.

\begin{definition}[Vector-valued local functions]
A function $\mu: \Sigma^m \rightarrow \R^N$ is $k$-local if the $N$ real valued functions corresponding to the $N$ coordinates are all $k$-local. Note that these different coordinate functions may depend on different sets of variables, as long as the number is at most $k$ for each of the functions.
\end{definition}
%
%
\paragraph{Local distribution view of SoS}

It will be convenient to use a shorthand for the function $\indi{\zee_A = \alpha}$, and we will use $\zee_{A,\alpha}$. Likewise, we use $\zee_{i,j}$ as a shorthand for the function $\indi{\zee_i = j}$. That is, henceforth,
\ifnum\confversion=1
\begin{align*}
	&\tildeEx{\zee_{A,\alpha}} = \tildeEx{\indi{\zee_A = \alpha}} = \tildeEx{ \prod_{a\in A}
                             Z_{a,\alpha_a}}
                             \\
	\text{and } \quad & \tildeEx{\zee_{i,j}} = \tildeEx{\indi{\zee_i = j}} = \tildeEx{ Z_{i,j}}.
\end{align*}
\else
\begin{align*}
	\tildeEx{\zee_{A,\alpha}} ~=~ \tildeEx{\indi{\zee_A = \alpha}} ~=~ \tildeE\brackets[\Big]{\prod_{a\in A}
                      Z_{a,\alpha_a}
                                    }
\qquad \text{and} \qquad
	\tildeEx{\zee_{i,j}} ~=~ \tildeEx{\indi{\zee_i = j}} = \tildeEx{ Z_{i,j}}
\end{align*}
\fi

% Note that for any degree-$t$ pseudoexpectation operator $\tildeEx{\cdot}$ with $t\geq 2$,
% \[
% 	\sum_{j\in [q]} \tildeEx{\zee_{i,j}} = \sum_{j} \tildeEx{Z_{i,j}} = \tildeEx{\sum_j Z_{i,j}}= 1
% \qquad \text{and} \qquad
% 	\tildeEx{\zee_{i,j}} = \tildeEx{Z_{i,j}} = \tildeEx{Z_{i,j}^2} \geq 0
% \]
% Thus, the real values $\inbraces{\tildeEx{\zee_{i,j}}}_{j\in [q]}$ define a distribution over $[q]$
% , which we will sometimes call local distribution for $\zee_i$. 

% In fact, this argument can be extended to define local distributions for $\zee_S$ for $|S|\leq
% t/2$. Let $S \subseteq [m]$ be such that $|S|=k\leq t/2$,
Note that for any $A \subseteq [m]$ with $\abs*{A} = k \leq t/2$,
\ifnum\confversion=1
\begin{gather*}
	\sum_{ \alpha \in \Sigma^{k}} \tildeEx{\zee_{A,\alpha}} = 
% \sum_{ \alpha \in [q]^{k}} \tildeEx{ \prod_{s\in S} Z_{s,\alpha_s}} 
 \tildeEx{ \prod_{a\in A} \inparen{ \sum_{j\in \Sigma} Z_{a,j}} } = 1
\\
	\tildeEx{\zee_{A,\alpha}} = \tildeEx{ \prod_{a\in A} Z_{a,\alpha_a}} = \tildeEx{ \prod_{a\in
            A} Z^2_{a,\alpha_a}} \geq 0 \mper
\end{gather*}
\else
\[
	\sum_{ \alpha \in \Sigma^{k}} \tildeEx{\zee_{A,\alpha}} = 
% \sum_{ \alpha \in [q]^{k}} \tildeEx{ \prod_{s\in S} Z_{s,\alpha_s}} 
 \tildeE\brackets[\bigg]{ \prod_{a\in A} \parens[\bigg]{\sum_{j\in \Sigma} Z_{a,j} } } = 1
\qquad \text{and} \qquad
	\tildeEx{\zee_{A,\alpha}} = \tildeE\brackets[\bigg]{ \prod_{a\in A} Z_{a,\alpha_a}} = \tildeE\brackets[\bigg]{ \prod_{a\in
            A} Z^2_{a,\alpha_a}} \geq 0 \mper
\]
\fi

Thus, the values $\inbraces{\tildeEx{\zee_{A, \alpha}}}_{\alpha\in \Sigma^A}$ define a distribution
over $\Sigma^k$. We call this the local distribution for $\zee_A$, or simply for $A$.
% which we think of as the local distribution for $\zee_S$.
%
% Given a distribution $\calD$ over assignments in $[q]^m$, the local distribution induced by $\PExp^{(\calD)}[\cdot]$ on $\zee_S$ is precisely the marginal distribution induced by $\calD$ for the set $S$. In this view, degree-$t$ pseudoexpectations give us consistent marginal distributions over sets of size at most $t/2$ that may not correspond to any global distribution, and allow us to optimize over this set in time $m^{\calO(t)}$.
%
Let $\mu: \Sigma^m \rightarrow\R$ be a $k$-local function for $k\leq t/2$, depending on $M \subseteq
[m]$. Then,
%
\ifnum\confversion=1
\begin{align*}
	\tildeEx{\mu(\zee)} 
~=~& \tildeEx{\sum_{\alpha\in \Sigma^M} \inparen{\mu(\alpha) \cdot\prod_{i\in M} Z_{i,\alpha_i}}}\\
~=~& \sum_{\alpha\in \Sigma^M} \mu(\alpha) \cdot \tildeEx{\prod_{i\in M} Z_{i,\alpha_i}}\\
~=~& \sum_{\alpha\in \Sigma^M} \mu(\alpha) \cdot \tildeEx{\zee_{M,\alpha}}
\end{align*}
\else
\begin{align*}
	\tildeEx{\mu(\zee)} 
~=~ \tildeE\brackets[\bigg]{\sum_{\alpha\in \Sigma^M} \parens[\bigg]{\mu(\alpha) \cdot\prod_{i\in M} Z_{i,\alpha_i}}}
~=~ \sum_{\alpha\in \Sigma^M} \mu(\alpha) \cdot \tildeE\brackets[\Big]{\prod_{i\in M} Z_{i,\alpha_i}}
~=~ \sum_{\alpha\in \Sigma^M} \mu(\alpha) \cdot \tildeEx{\zee_{M,\alpha}}
\end{align*}
\fi
%

That is, $\tildeEx{\mu(\zee)}$ can be seen as the expected value of the function $\mu$ under the local distribution for $M$.

\begin{claim}\label{claim:sos_domination}
	Let $\tildeEx{\cdot}$ be a degree-$t$ pseudoexpectation. For $k \leq t/2$, let $\mu_1,\mu_2$
        be two $k$-local functions on $\Sigma^m$, depending on the same set of coordinates $M$, and
        $\mu_1(\alpha) \leq \mu_2(\alpha) ~~\forall \alpha \in \Sigma^M$. Then $\tildeEx{\mu_1(\zee)} \leq \tildeEx{\mu_2(\zee)}$.
%
 % Suppose that for any $\alpha\in [q]^M$, $\mu_1(\alpha) \leq \mu_2(\alpha)$. Then
 %        \[
 %        	\tildeEx{\mu_1(\zee)} \leq \tildeEx{\mu_2(\zee)}
 %        \]
\end{claim}

\begin{proof}
Let $\calD_M$ be the local distribution induced by $\tildeEx{\cdot}$ for $\zee_M$. Then
$\tildeEx{\mu_1(\zee)} = \Ex{\alpha \sim \calD_M}{\mu_1(\alpha)}$, and $\tildeEx{\mu_2(\zee)} =
\Ex{\alpha\sim \calD_M}{\mu_2(\alpha)}$, which implies $\tildeEx{\mu_1(\zee)} \leq \tildeEx{\mu_2(\zee)}$.
%
% Since $\mu_1(\alpha) \leq \mu_2(\alpha)$ for every $\alpha\in [q]^M$, 
% \[
% 	\Ex{\alpha \sim \calD_M}{\mu_1(\alpha)} \leq \Ex{\alpha\sim \calD_M}{\mu_2(\alpha)}
% \]
% and so,
% \[
% 	\tildeEx{\mu_1(\zee)} \leq \tildeEx{\mu_2(\zee)}
% \]
\end{proof}
%
The previous claim allows us to replace any local function inside $\tildeEx{\cdot}$ by another local function that dominates it. We will make extensive use of this fact.
%
%
\vspace{-5 pt}
%\tnote{Move covariance, conditioning stuff to appendix 1? It seems to disrupt the flow and we only need to cite it in section 6 for Lemma 6.1?}
\paragraph{Covariance for SoS solutions}
Given two sets $S,T \sub [m]$ with $|S|,|T|\leq k/4$, we can define the covariance between indicator random variables of $\zee_S$ and $\zee_T$ taking values $\alpha$ and $\beta$ respectively, according to the local distribution over $S \cup T$. This is formalized in the next definition.
\begin{definition}
Let $\tildeEx{\cdot}$ be a pseudodistribution operator of SoS-degree-$t$, and $S,T$ are two sets of
size at most $t/4$, and $\alpha\in \Sigma^S$, $\beta\in \Sigma^T$, we define the pseudo-covariance and
pseudo-variance,
%
\ifnum\confversion=1
\small
\begin{gather*}
\tildecov(\zee_{S,\alpha},\zee_{T,\beta}) 
= \tildeEx{ \zee_{S,\alpha} \cdot \zee_{T,\beta} } - \tildeEx{\zee_{S,\alpha}} \tildeEx{\zee_{T,\beta}} \\	
\tildeVar{\zee_{S,\alpha}} ~=~ \tildecov(\zee_{S,\alpha},\zee_{S,\alpha})
\end{gather*}
\normalsize
\else
\begin{align*}
\tildecov(\zee_{S,\alpha},\zee_{T,\beta}) 
~&=~ \tildeEx{ \zee_{S,\alpha} \cdot \zee_{T,\beta} } - \tildeEx{\zee_{S,\alpha}} \,\tildeEx{\zee_{T,\beta}}\\ 		
\tildeVar{\zee_{S,\alpha}} ~&=~ \tildecov(\zee_{S,\alpha},\zee_{S,\alpha})
\end{align*}
\fi
%
The above definition is extended to pseudo-covariance and pseudo-variance for pairs of sets $S,T$, 
as the sum of absolute value of pseudo-covariance for all pairs $\alpha,\beta$ :
%
\ifnum\confversion=1
\begin{gather*}
\tildecov(\zee_S,\zee_T) 
~=~ \sum_{\alpha\in \Sigma^S \atop \beta\in \Sigma^T} \abs*{ \tildecov(\zee_{S,\alpha},\zee_{T,\beta}) } \\
\tildeVar{\zee_S} ~=~ \sum_{\alpha\in \Sigma^S} \abs*{ \tildeVar{\zee_{S,\alpha} } }
\end{gather*}
\else
\begin{align*}
\tildecov(\zee_S,\zee_T) 
~&=~ \sum_{\alpha\in \Sigma^S, \beta\in \Sigma^T} \abs*{ \tildecov(\zee_{S,\alpha},\zee_{T,\beta}) }\\[3pt]
\tildeVar{\zee_S} ~&=~ \sum_{\alpha\in \Sigma^S} \abs*{ \tildeVar{\zee_{S,\alpha} } }
\end{align*}
\fi
%
\end{definition}

% These definitions can be extended to pseudocovariances and pseudo-variances for pairs of sets $S,T$ with $|S|,|T|\leq t/4$ as the sum of absolute value of pseudocovariance for all pairs $\alpha,\beta$.
% \begin{definition}
% Let $\tildeEx{\cdot}$ be a pseudodistribution operator of SoS-degree-$t$, and $S,T$ are two sets of size at most $t/4$, we define the pseudo-covariance between $\zee_S$ and $\zee_T$ as,
% 	\begin{align*}
% 		\tildecov(\zee_S,\zee_T) = \sum_{\alpha\in [q]^S,\beta\in [q]^T} \abs*{ \tildecov(\zee_{S,\alpha},\zee_{T,\beta}) }
% 	\end{align*}
% We also define the analogous pseudovariance as,
% 	\begin{align*}
% 		\tildeVar{\zee_S} = \sum_{\alpha\in [q]^S} \abs*{ \tildeVar{\zee_{S,\alpha} } }
% 	\end{align*}
% \end{definition}
%
We will need the fact that $\tildeVar{\zee_S}$ is bounded above by 1, since,
%
\ifnum\confversion=1
\begin{align*}
\tildeVar{\zee_S} 
&~=~ \sum_{\alpha} \abs*{\tildeVar{\zee_{S,\alpha}}} \\
&~=~ \sum_{\alpha}\inparen{
          \tildeEx{\zee_{S,\alpha}^2} - \tildeEx{\zee_{S,\alpha}}^2} \\
&~\leq~ \sum_{\alpha} \tildeEx{\zee_{S,\alpha}^2} \\
&~=~ \sum_{\alpha} \tildeEx{\zee_{S,\alpha}} 
~=~ 1.
\end{align*}
\else
\[
\tildeVar{\zee_S} 
~=~ \sum_{\alpha} \abs*{\tildeVar{\zee_{S,\alpha}}} 
~=~ \sum_{\alpha}\inparen{
          \tildeEx{\zee_{S,\alpha}^2} - \tildeEx{\zee_{S,\alpha}}^2} 
~\leq~ \sum_{\alpha} \tildeEx{\zee_{S,\alpha}^2} 
~=~ \sum_{\alpha} \tildeEx{\zee_{S,\alpha}} 
~=~ 1.
\]
\fi

% \begin{claim}\quad	$\tildeVar{\zee_S} \leq 1$.
% \end{claim}
% %
% \begin{proof}
% \begin{align*}
% 	\tildeVar{\zee_S} &= \sum_{\alpha} \abs*{\tildeVar{\zee_{S,\alpha}}} = \sum_{\alpha}\inparen{ \tildeEx{\zee_{S,\alpha}^2} - \tildeEx{\zee_{S,\alpha}}^2} \leq \sum_{\alpha} \tildeEx{\zee_{S,\alpha}^2} = \sum_{\alpha} \tildeEx{\zee_{S,\alpha}} = 1
% \end{align*}
% \end{proof}
%
\vspace{-5 pt}
\paragraph{Conditioning SoS solutions.}
%
We will also make use of conditioned pseudoexpectation operators, which may be defined in a way
similar to usual conditioning for true expectation operators, as long as the event we condition on
is local. 
%
The conditioned SoS solution is of a smaller degree but continues to respect the constraints that the original solution respects.

\begin{definition}[Conditioned SoS Solution] Let $F \subseteq \Sigma^m$ be subset (to be thought of as an event) such that $\one_F:\Sigma^m \rightarrow \{0,1\}$ is a $k$-local function. Then for every $t>2k$, we can condition a pseudoexpectation operator of SoS-degree $t$ on $F$ to obtain a new conditioned pseudoexpectation operator $\condPE{\cdot}{F}$ of SoS-degree $t-2k$, as long as $\tildeEx{\one^2_F(\zee)}>0$. The conditioned SoS solution is given by
\[
	\condPE{ p(\zee)}{F(\zee) } \defeq \frac{\tildeEx{p(\zee) \cdot \one^2_{F}(\zee)}}{\tildeEx{\one^2_{F}(\zee)}}
\]
where $p$ is any polynomial of degree at most $t-2k$.
\end{definition}

%\snote{Mention that constraints \emph{respected} by SoS remain true after conditioning.}

We can also define pseudocovariances and pseudo-variances for the conditioned SoS solutions.
\begin{definition}[Pseudocovariance]
	Let $F\sub \Sigma^m$ be an event such that $\one_F$ is $k$-local, and let $\tildeEx{\cdot}$ be a pseudoexpectation operator of degree $t$, with $t>2k$. Let $S,T$ be two sets of size at most $\frac{t-2k}{2}$ each. Then the pseudocovariance between $\zee_{S,\alpha}$ and $\zee_{T,\beta}$ for the solution conditioned on event $F$ is defined as,
\ifnum\confversion=1
\begin{multline*}
\tildecov(\zee_{S,\alpha},\zee_{T,\beta} \vert F) = \\
\tildeEx{ \zee_{S,\alpha} \zee_{T,\beta} \vert F} - \tildeEx{\zee_{S,\alpha} \vert F} \tildeEx{\zee_{T,\beta} \vert F}
\end{multline*}
\else
\begin{align*}
\tildecov(\zee_{S,\alpha},\zee_{T,\beta} \vert F) 
~=~ \tildeEx{ \zee_{S,\alpha} \zee_{T,\beta} \vert F} - \tildeEx{\zee_{S,\alpha} \vert F} ~ \tildeEx{\zee_{T,\beta} \vert F}
\end{align*}
\fi
\end{definition}

We also define the pseudocovariance between $\zee_{S,\alpha}$ and $\zee_{T,\beta}$ after
conditioning on a random assignment for some $\zee_V$ with $V\sub [m]$. 
%
Note that here the random assignment for $\zee_V$ is chosen according to the local distribution for
the set $V$.

\begin{definition}[Pseudocovariance for conditioned pseudoexpectation operators]
\ifnum\confversion=1
\begin{multline*}
\tildecov(\zee_{S,\alpha},\zee_{T,\beta} \vert \zee_V) 
= \\ \sum_{\gamma \in \Sigma^V}   \tildecov(\zee_{S,\alpha},\zee_{T,\beta} \vert \zee_V = \gamma) \cdot \tildeEx{\zee_{V,\gamma}}
	\end{multline*}
\else
\begin{align*}
\tildecov(\zee_{S,\alpha},\zee_{T,\beta} \vert \zee_V) 
~=~ \sum_{\gamma \in \Sigma^V}   \tildecov(\zee_{S,\alpha},\zee_{T,\beta} \vert \zee_V = \gamma) \cdot \tildeEx{\zee_{V,\gamma}}
\end{align*}
\fi
\end{definition}

And we likewise define $\tildeVar{\zee_{S,\alpha} \vert \zee_V}$, $\tildecov(\zee_S, \zee_T \vert \zee_V)$ and $\tildeVar{\zee_S \vert \zee_V}$.






%%% Local Variables:
%%% mode: latex
%%% TeX-master: "main"
%%% End:


\subsection{Pseudocodewords achieve the generalized Singleton bound}\label{sec:sos_proof}

We will show that the proof of list decodability established for integral codewords can be extended to SoS relaxations of these codewords, under certain average-case low covariance conditions. 
First, we describe what an SoS relaxation of an AEL codeword looks like. This relaxation will be such that true codewords are always feasible solutions, but there may be other feasible solutions. These feasible solutions are therefore called \textit{pseudocodewords}.

The main goal will be to show that for some large constant $t$ depending on $k$ and $\eps$, but independent of $n$, the degree-$t$ SoS relaxation is tight enough to decode up to radius $\frac{k-1}{k}(1-R-\eps)$. We will actually work with $k$-tuples of pseudocodewords, and the show that this tuple---after some random conditioning---satisfies a generalized Singleton bound just like a list of true (distinct) codewords.


\paragraph{SoS relaxations for AEL codewords}
Let $G(L,R,E)$ be the bipartite $(n,d,\lambda)$-expander on which the AEL code is defined, and let $\calC_\inn\subseteq \Sigma_\inn^d$ be the inner code. 

Before going into the details of the SoS relaxation for AEL codes, let us set up some convenient notation. For sets $S\sub [k]$ and $F \sub E$, we use $\zee_{S,F}$ to index their Cartesian product $\zee_{S \times F}$. Further, since we will be often dealing with the case when $F=N(\li)$ or $F=N(\ri)$, we will use $\zee_{S,\li}$ as a shorthand for $\zee_{S,N(\li)}$, and similarly use $\zee_{S,\ri}$ as a shorthand for $\zee_{S,N(\ri)}$. For example,
\[
	\tildeVar{\zee_{[k],\li}} = \tildeVar{ \zee_{[k] \times N(\li)}}.
\]


\begin{definition}[$k$-tuple of pseudocodewords]\label{def:k_tuple}
A \emph{$k$-tuple of psueocodewords of degree-$t$} is a pseudoexpectation operator $\tildeEx{\cdot}$ of SoS-degree-$t$ defined on the variables $\{Z_{i,e}\}_{i\in [k], e\in E}$ over the alphabet $\Sigma_{\inn}$ that respects the following constraints:
\[
	\forall i\in [k], \forall \li \in L,~~~ \zee_{i,N(\li)} \in \calC_{\inn}
\] 
\end{definition}

It is easy to see that any $k$ strings $f_1,\cdots ,f_k$ in $\calC_{\inn}^L$ can be used to define a $k$-tuple of pseudocodeword, by simply setting $\zee_{i,e}$ to be $(f_i)_e$ for all $i \in [k], e\in E$. Note that these strings need not be distinct, as written. However, the set of $k$-tuples of pseudocodewords is more general that just integral strings, and in particular, is convex. However, we can still generalize notions like distance for these objects.

\begin{definition}[Distances and pseudocodewords]
	For an $i\in [k]$, we define the left and right distances between the $i^{th}$ component of a $k$-tuple of pseudocodeword $\tildeEx{\cdot}$ of SoS-degree $t\geq 2d$ and any string $g \in \Sigma_{\inn}^E$ as
	\begin{align*}
		\tildeEx{\Delta_L(g, \zee_i)} &\defeq \Ex{\li \in L}{\tildeEx{\indi{g_{\li} \neq \zee_{i, \li}}}} \\
%	\dis(\tildeEx{\cdot},h) &\defeq \Ex{e}{\tildeEx{\indi{\zee_e \neq h_e}}} \\
		\tildeEx{\Delta_R(g,\zee_i)} &\defeq \Ex{\ri\in R}{\tildeEx{\indi{g_{\ri} \neq \zee_{i, \ri}}}}.
	\end{align*}
\end{definition}

The above can be seen as counting the fraction of errors between $\zee_i$ and $g$. We will also need another piece of notation which will make it easier to track the fraction of errors that are common to an entire set $S\sub [k]$.
\begin{align*}
	\tildeEx{\Delta_R(g,\zee_S)} \defeq \Ex{\ri \in R}{\tildeEx{\Pi_{i\in S} \indi{g_{\ri} \neq \zee_{i,\ri}}}}
\end{align*}

We will also need a notion of distance between two pseudocodewords within a $k$-tuple.
\begin{definition}[Distances between pseudocodewords in a $k$-tuple]
Let $\tildeEx{\cdot}$ be a $k$-tuple of pseudocodewords. Let $i,i' \in [k]$ be indices, then the distance between $i^{th}$ and $(i')^{th}$ pseudocodewords can be defined as
\begin{align*}
		\tildeEx{\Delta_L(\zee_i, \zee_{i'})} &\defeq \Ex{\li \in L}{\tildeEx{\indi{\zee_{i, \li} \neq \zee_{i', \li}}}} \\
%	\dis(\tildeEx{\cdot},h) &\defeq \Ex{e}{\tildeEx{\indi{\zee_e \neq h_e}}} \\
		\tildeEx{\Delta_R(\zee_i,\zee_{i'})} &\defeq \Ex{\ri\in R}{\tildeEx{\indi{\zee_{i,\ri} \neq \zee_{i', \ri}}}}.
	\end{align*}
\end{definition}

\paragraph{$\eta$-goodness} Instead of working with arbitrary pseudocodewords, we will work with those  that have small pseudocovariances across a typical edge in $G$. This key property, called $\eta$-goodness, has been used in prior works and we extend its definition from \cite{JST23} to $k$-tuple of pseudocodewords as follows:

\begin{definition}[$\eta$-good pseudocodeword]
    A $k$-tuple of pseudocodewords $\tildeEx{\cdot}$ of degree at least $2kd$ is called $\eta$-good if
    \[
        \Ex{\li,\ri}{\tildeCov{\zee_{[k]\times N(\li)}}{\zee_{[k]\times N(\ri)}}} \leq \eta.
    \]
\end{definition}

The key upshot of having this property is~\cref{lem:avg_corr} which we use in our proof.  Moreover, one can obtain such $\eta$-good pseudocodewords by randomly conditioning an SoS solution. We relegate the proof of these details to \cref{sec:appendix} as they are an adaptation of earlier proofs.

%
%\begin{definition}[AEL Pseudocodewords]
%	For $t\geq 2d$, we define a degree-$t$ AEL pseuocodeword to be a degree-$t$ pseudoexpectation operator $\tildeEx{\cdot}$ on $\zee$ respecting the following constraints:
%	\begin{align*}
%		\forall \li \in L,\, i\in [k_0]\quad \zee_{i, \li} \in \calC_\inn
%	\end{align*}
%\end{definition}
%
%Next we define the distances between a pseudocodeword and a codeword of $\AELC$. 


%\tnote{Moved the definition of eta-good here. Is the notation of Z consistent?}

The main result of this section is the following generalization of~\cref{lemma:common-error-bound} to $\eta$-good pseudocodewords.

\begin{theorem}\label{thm:sos_main}
	Let $k_0\geq 1$ be an integer and let $\eps > 0$. Let $\AELC$ be a
code obtained using the AEL construction using $(G,\calC_{\out}, \calC_{\inn})$, where $\calC_{\inn}$ is $(\delta_0, k_0, \eps/2)$ average-radius list decodable with erasures, and the graph $G$ is a $(n,d,\lambda)$-expander. 
	Let $\tildeEx{\cdot}$ be an $\eta$-good $k_0$-tuple of pseudocodewords that satisfies the following pairwise distance property on the left:
	\[
		\forall i,i'\in [k_0] \text{ with } i\neq i', \qquad \tildeEx{\Delta_L(\zee_i, \zee_{i'})} \geq \beta\mper
	\]
	Further, assume that $\lambda \leq \frac{\beta}{12k_0^{k_0}}\cdot \eps$ and $\eta \leq \frac{\beta}{12k_0^{k_0}}\cdot \eps$.	Then, for any $g \in (\Sigma_{\inn}^d)^R$,
%	, and any $K \sub [k_0]$ with $|K|=k$, 
	\begin{align*}
	\sum_{i\in [k_0]} \tildeEx{\Delta_R(g,\zee_i)} ~&\geq~ (k_0-1)(\delta_0-\eps) + \Ex{\ri \in R}{\tildeEx{ \Pi_{i\in [k_0]} \indi{g_{\ri} \neq \zee_{i,\ri}} }} \\
	~&=~ (k_0-1)(\delta_0 -\eps) + \tildeEx{\Delta_R(g,\zee_{[k_0]})}.
%		\Ex{i\in [k]}{\tildeEx{\Delta_R(g,\zee_i)}} \geq \frac{k-1}{k} \inparen{ \Delta - \Ex{\ri \in R}{\tildeEx{ \Pi_{i\in[k]} \indi{g_{\ri} \neq \zee_{i,\ri}} }} }
	\end{align*}
%
\end{theorem}
%
%\begin{proof}
%By induction on $k$. The case $k=1$ is trivial.
Just like in the proof of our main theorem in \cref{sec:avg-singleton}, we need preparatory claims about the existence of a nice partition and (a variant of) $L^*$. We start by proving analytic generalizations of \cref{lem:type_arg} and \cref{lemma:inductive}. Let $\Tau$ denote the set of all partitions\footnote{Note that now we work with partitions of $[k]$ rather a list of codewords $\calH$.} of $[k]$. For an $\li \in L$ and $\tau \in \Tau$, define the local function, 
	\[
		\Lambda_{\li, \tau}(\zee) \defeq \indi{ (\zee_{1, \li}, \zee_{2, \li}, \cdots , \zee_{k, \li}) \text{ induces partition }\tau}.
	\]
	By definition, for every $\li \in L$,
	\[
		\sum_{\tau \in \Tau} \Lambda_{\li, \tau}(\zee) = 1.
	\]
	Let $\tau_1 \in \Tau$ be the trivial partition with only 1 part. In the integral proof, we said that there must be a non-trivial partition that is induced on a significant fraction of left vertices, and used it to define the set $L^*$. For tuples of pseudocodewords, each left vertex will be inducing a distribution over all possible partitions, and we will argue that there is a partition $\tau^*$ that receives a significant mass among these distributions on average over all $\li \in L$. The indicator of this partition $\Lambda_{\li,\tau^*}(\zee)$ will then play the role of the indicator of the set $L^*$ in integral proof.
	\begin{claim}[Generalization of \cref{lem:type_arg}]\label{claim:best_partition}
		There exists a $\tau^*\in \Tau \setminus \{\tau_1\}$ such that 
		\[
			\tildeEx{\Ex{\li \in L}{\Lambda_{\li, \tau^*}(\zee)}} \geq \frac{\beta}{k^k}.
		\]
	\end{claim}
	\begin{proof}
		If $(\zee_{1, \li}, \zee_{2, \li}, \cdots , \zee_{k, \li})$ induces the partition $\tau_1$, then $\zee_{1,\li} = \zee_{2,\li}$. That is,
		\[
			\Lambda_{\li,\tau_1}(\zee) \leq \indi{\zee_{1,\li} = \zee_{2,\li}}.
		\]
		We bound the contribution from the trivial partition as 
		\[
			\tildeEx{\Ex{\li}{\Lambda_{\li,\tau_1}(\zee)}} \leq \tildeEx{\Ex{\li}{\indi{\zee_{1,\li} = \zee_{2,\li}}}} = \tildeEx{1-\Delta_L(\zee_1,\zee_2)} \leq 1-\beta
		\]
%		Since $\tildeEx{\Delta_L(\zee_1,\zee_2)} \geq \beta$,
%		\begin{align*}
%			&\tildeEx{\Ex{\li}{\indi{\zee_{1, \li} \neq \zee_{2, \li}}}} ~\geq~ \beta \\
%			\implies &~\tildeEx{\Ex{\li}{\indi{\zee_{1, \li} = \zee_{2,\li}}}} ~\leq~ 1-\beta \\
%			\implies &~\tildeEx{\Ex{\li}{\Lambda_{\li,\tau_1}(\zee)}} ~\leq~ 1-\beta.
%		\end{align*}
		The above shows that the partition $\tau_1$ cannot be too common. Next, we use $\sum_{\tau\in \Tau} \Lambda_{\li,\tau}(\zee) =1$ to show that:
		\begin{align*}
			1 = \tildeEx{\Ex{\li}{\sum_{\tau \in \Tau}{\Lambda_{\li,\tau}(\zee)}}} = \tildeEx{\Ex{\li}{\Lambda_{\li,\tau_1}(\zee)}} + \sum_{\tau \in \Tau \setminus \{\tau_1\}} \tildeEx{\Ex{\li}{\Lambda_{\li,\tau}(\zee)}} \\
			\implies\sum_{\tau \in \Tau \setminus \{\tau_1\}} \tildeEx{\Ex{\li}{\Lambda_{\li,\tau}(\zee)}} = 1 - \tildeEx{\Ex{\li}{\Lambda_{\li,\tau_1}(\zee)}} \geq 1-(1-\beta) = \beta.
		\end{align*}
		By averaging, we can use $|\Tau| \leq k^k$ to conclude that there is a $\tau^* \in \Tau \setminus \{\tau_1\}$ such that,
		\[
			\tildeEx{\Ex{\li}{\Lambda_{\li,\tau^*}(\zee)}} ~\geq~ \frac{1}{|\Tau|} \sum_{\tau \in \Tau \setminus \{\tau_1\}} \tildeEx{\Ex{\li}{\Lambda_{\li,\tau}(\zee)}} ~\geq~ \frac{\beta}{k^k}.\qedhere
		\]
	\end{proof}
	
	\paragraph{Capturing common error locations}
	Suppose the partition $\tau^*$ is given by $[k] = (K_1, \cdots, K_p)$ for some $1<p<k$.  Henceforth, we will be working with this fixed partition. 	We define a function, $\dd_{S,\ri}(g,\zee)$, which is an indicator of whether $\ri \in R$ is a common error location for the pseudocodewords indexed by $S\subseteq [k]$. 
%	Let us define two shorthands for some $S\subseteq [k]$:
	\begin{align*}
		\dd_{S,\ri}(g, \zee) &= \Pi_{i\in S} \indi{g_{\ri} \neq \zee_{i,\ri}} \\
		\dd_{S,e}(g, \zee) &= \dd_{S,\ri}(g, \zee), \text{ where } e=(\li,\ri).
	\end{align*}
Of course, if a vertex $\ri $ is a common error location for all $k$ pseudocodewords, then it is also a common error location for a subset, implying
	\[
		\dd_{S,\ri}(g,\zee) ~\geq~ \dd_{[k],\ri}(g,\zee).
	\]
	We will also need to track the error locations common to $S$ but not to $[k]$, so we define two additional shorthands:
	\begin{align*}
		\fd_{S,\ri}(g, \zee) ~&=~ \dd_{S,\ri}(g,\zee) ~-~ \dd_{[k],\ri}(g,\zee) \\
		\fd_{S,e}(g, \zee) ~&=~ \fd_{S,r}(g, \zee), \text{ where } e=(\li,\ri).
	\end{align*}
		

\paragraph{Key Claims}		
		We will now prove SoS versions of the three main claims from the integral proof; ~\cref{lemma:inductive}, \cref{claim:local-bound}, and \cref{claim:sampling_erasure}. 
	%To begin, let us define the SoS generalizations of the terms in the lemma. 

%Finally, we use that the local snapshots of common errors are upper bounded by the global common errors. This will be another AEL argument.

%	The following table illustrates how the above functions capture...  
%\tnote{Suppresing the dictionary for now.}
%		\begin{table}[h]
%\begin{center}
%  \begin{tabular}{l|c}
%   Codeword & SoS Generalization \\
%    \hline\\
%   $\Delta(g_\ell, \fjl)$ & $\tildeEx{\Ex{e\in N(\li)}{\fd_{K_j,e}(g,\zee)}}$ \\
%    $\indi{\ell\in L^*}$  & $\tildeEx{\Lambda_{\li,\tau^*}(\zee)}$\\
%   	$\abs{L^*}/n$ & $\tildeEx{\Ex{\li}{\Lambda_{\li,\tau^*}(\zee)}}$  \\[2pt]
%   	$\Ex{\ell \in L^*}{\Delta(\gl, \fjl)}$ & $\frac{\Ex{\li}{\tildeEx{\Lambda_{\li,\tau^*}(\zee) \cdot \Ex{e\in N(\li)}{\fd_{K_j,e}(g,\zee)}}} }{\Ex{\li}{\tildeEx{\Lambda_{\li,\tau^*}(\zee)}}}$
%  \end{tabular}
%  \caption{Dictionary between}
%\end{center}
%\end{table}

The first of these showed that errors observed on $L^*$ serve as a lower bound for global errors on the right. The analog of "averaging over $L^*$ can be carried out by reweighing according to the indicator function $\Lambda_{\li,\tau^*}$ for the $\tau^*$ partition.
	
\begin{lemma}[Generalization of \cref{claim:sampling_erasure}]\label{lemma:local_erasures_upper_bound} 
%For any functions $\Gamma(\ell, \zee), \Psi(\zee) $ of  pseudocodewords, the following holds for $\eta$-good pseudocodeword $\zee$.
%\[
%	\tildeEx{\Ex{\li \sim \ri}{\Gamma(\zee) \cdot \Psi(\zee)}} ~\leq~ \tildeEx{\Ex{\li}{\Gamma(\zee)}} \tildeEx{\Ex{\ri}{\Psi(\zee)}} + \lambda +\eta.			
%\]
Assume that $\lambda \leq \frac{\beta}{12k^{k}}\cdot \eps$ and $\eta \leq \frac{\beta}{12k^{k}}\cdot \eps$.
For the functions $\fd, \dd$, and any set $S\subseteq [k]$ we have,
\begin{align*}
	\frac{\tildeEx{\Ex{\li \sim \ri}{\Lambda_{\li,\tau^*}(\zee) \cdot \dd_{S,\ri}(g,\zee)}}}{\tildeEx{\Ex{\li}{\Lambda_{\li,\tau^*}(\zee)}}} ~\leq~ \tildeEx{\Ex{\ri}{\dd_{S,\ri}(g,\zee)}} + \frac{\eps}{6}.\\
		\frac{\tildeEx{\Ex{\li \sim \ri}{\Lambda_{\li,\tau^*}(\zee) \cdot \fd_{S,\ri}(g,\zee)}}}{\tildeEx{\Ex{\li}{\Lambda_{\li,\tau^*}(\zee)}}} ~\leq~ \tildeEx{\Ex{\ri}{\fd_{S,\ri}(g,\zee)}} + \frac{\eps}{6}.
		\end{align*}
	\end{lemma}
\begin{proof} The proof is based on an AEL-like argument and is identical for either case. The first step uses the expander mixing lemma for pseudocodewords (\cref{lem:eml_sos}), and the second utilizes the $\eta$-good property (\cref{lem:avg_corr}),
	\begin{align*}
%		\tildeEx{\Ex{\li}{\Lambda_{\li,\tau^*}(\zee) \cdot \Ex{e\in N(\li)}{D_{[k],e}(g,\zee)}}} &= 
		\tildeEx{\Ex{\li\sim \ri}{\Lambda_{\li,\tau^*}(\zee) \cdot \dd_{S,\ri}(g,\zee)}}	~&\leq~ \tildeEx{\Ex{\li , \ri}{\Lambda_{\li,\tau^*}(\zee) \cdot \dd_{S,\ri}(g,\zee)}} + \lambda \\
		~&\leq~ \tildeEx{\Ex{\li}{\Lambda_{\li,\tau^*}(\zee)}} \cdot \tildeEx{\Ex{\ri}{\dd_{S,\ri}(g,\zee)}} + \lambda + \eta.
	\end{align*}
	To obtain the final consequence we divide by $\tildeEx{\Ex{\li}{\Lambda_{\li,\tau^*}(\zee)}}$ and use~\cref{claim:best_partition}.
		\begin{align*}
		\frac{\tildeEx{\Ex{\li\sim \ri}{\Lambda_{\li,\tau^*}(\zee) \cdot \dd_{S,\ri}(g,\zee)}}}{\tildeEx{\Ex{\li}{\Lambda_{\li,\tau^*}(\zee)}}} ~&\leq~ \tildeEx{\Ex{\ri}{\dd_{S,\ri}(g,\zee)}} + \frac{\lambda + \eta}{\tildeEx{\Ex{\li}{\Lambda_{\li,\tau^*}(\zee)}}} \\
%		~&\leq  \tildeEx{\Ex{\ri}{D_{[k],\ri}(g,\zee)}} + \frac{\lambda + \eta}{(\beta/k^k)} \\
		~&\leq~  \tildeEx{\Ex{\ri}{\dd_{S,\ri}(g,\zee)}} + \frac{\eps}{6}. \qedhere
%		~&=~  \tildeEx{\Delta_R(g,\zee_{[k]})} + \frac{\eps}{6}.
	\end{align*}
\end{proof}

	
%	Finally, define $\lstar := \tildeEx{\Ex{\li \in L}{\Lambda_{\li, \tau^*}(\zee)}}$
%	\snote{Move these definitions outside theorem? $\fd$ is a macro, feel free to change.}
	
	
	
%	\begin{align*}
%	\Delta(\gl, \fjl) ~&\mapsto~~   \tildeEx{\Ex{e\in N(\li)}{\fd_{K_j,e}(g,\zee)}}\;,	\\
%	\indi{\ell\in L^*}  ~&\mapsto~~ \tildeEx{\Lambda_{\li,\tau^*}(\zee)}\;, \\
%	\abs{L^*}/n ~&\mapsto~~ \tildeEx{\Ex{\li}{\Lambda_{\li,\tau^*}(\zee)}}.
%	\end{align*}
	
%	It is easy to see that when $Z$ is an integral codeword, the two definitions coincide. Combining these, 
%	\begin{align*}
%	\Ex{\ell \in L^*}{\Delta(\gl, \fjl)} = \frac{\Ex{\ell}{\indi{\ell\in L^*} \cdot\Delta(\gl, \fjl)}}{|L^*|} ~&\mapsto~~   \frac{\Ex{\li}{\tildeEx{\Lambda_{\li,\tau^*}(\zee) \cdot \Ex{e\in N(\li)}{\fd_{K_j,e}(g,\zee)}}} }{\Ex{\li}{\tildeEx{\Lambda_{\li,\tau^*}(\zee)}}} .
%	\end{align*}
%	\snote{Maybe above is not a great idea since we want to discourage $\erase{g}$ in the SoS section.}
	
%	$\tildeEx{\Lambda_{\li,\tau^*}(\zee) \cdot \Ex{e\in N(\li)}{\indi{g_e \neq \zee_{i,e}} \cdot (1-D_{[k],e}(g,\zee))}} $
%	
%	$\Ex{\ell \in L^*}{\Delta(\gl, \fl_j)}$

The second lemma showed that the number of common error locations increases when only considering a subset of indices $K_j \sub [k]$, and this increase can be lower bounded in terms of the distance between $g_{\li}$ (actually, $\gl$) and the common local projections of $K_j$, averaged over $L^*$. The second term on the RHS in the lemma below is the analog of these errors between $\gl$ and common local projections $f_j$, averaged over $L^*$.
	\begin{lemma}[Generalization of  \cref{lemma:inductive}]\label{lemma:more_common_errors}
	For any part $K_j$ in the partition $\tau^*$, we have,
	%, and an arbitrary $i\in H_j$,
	\begin{align*}
		\tildeEx{\Delta_R(g,\zee_{K_j})}
%		&= \Ex{\ri}{\tildeEx{D_{K_j,\ri}(g,\zee)}}
%		&\geq \Ex{\ri}{\tildeEx{D_{[k],\ri}(g,\zee)}} + \frac{\Ex{\li}{\tildeEx{\Lambda_{\li,\tau^*}(\zee) \cdot \Ex{e\in N(\li)}{\fd_{K_j,e}(g,\zee)} }}}{\Ex{\li}{\tildeEx{\Lambda_{\li,\tau^*}(\zee)}}} - \frac{\eps}{6}\\
		~\geq~ \tildeEx{\Delta_R(g,\zee_{[k]})}  + \frac{\tildeEx{\Ex{\li \sim \ri}{\Lambda_{\li,\tau^*}(\zee) \cdot \fd_{K_j,\ri}(g,\zee)}}}{\tildeEx{\Ex{\li}{\Lambda_{\li,\tau^*}(\zee)}}} - \frac{\eps}{6} .
	\end{align*}
	\end{lemma}
	
	\begin{proof} By using the definition of the functions $\dd, \fd$ and some rearragement, we get,
	\begin{align*}
		\tildeEx{\Delta_R(g,\zee_{K_j})} ~&=~ \Ex{\ri}{\tildeEx{\dd_{K_j,\ri}(g,\zee)}} \\
		~&=~ \Ex{\ri}{\tildeEx{\dd_{[k],\ri}(g,\zee) \cdot \dd_{K_j,\ri}(g,\zee) + (1-\dd_{[k],\ri}(g,\zee)) \cdot \dd_{K_j,\ri}(g,\zee)}} \\
		~&=~ \Ex{\ri}{\tildeEx{\dd_{[k],\ri}(g,\zee)}} + \Ex{\ri}{\tildeEx{\fd_{K_j,\ri}(g,\zee)}}.
	\end{align*}
	
The proof finishes by replacing the second term by the bound from~\cref{lemma:local_erasures_upper_bound}.
%a typical AEL argument. The first step uses the expander mixing lemma (\cref{lem:eml_sos}), and the second utilizes the $\eta$-good property,
%	\begin{align*}
%		\tildeEx{\Ex{\li \sim \ri}{\Lambda_{\li, \tau^*}(\zee) \cdot \fd_{K_j,\ri}(g,\zee)}} \leq~&~ \tildeEx{\Ex{\li, \ri}{\Lambda_{\li, \tau^*}(\zee) \cdot \fd_{K_j,\ri}(g,\zee)}} + \lambda \\
%		\leq~&~ \tildeEx{\Ex{\li}{\Lambda_{\li, \tau^*}(\zee)}} \cdot \tildeEx{\Ex{\ri}{\fd_{K_j,\ri}(g,\zee)}} + \lambda + \eta\\
%		=~&~ \lstar \cdot \tildeEx{\Ex{\ri}{\fd_{K_j,\ri}(g,\zee)}} + \lambda + \eta.
%	\end{align*}
%	
%Dividing both sides by $\lstar$, we get 
%%\cref{eq:more_common_errors_lower_bound} and \cref{eq:more_common_errors_upper_bound}, we get
%	\begin{align*}
%		\tildeEx{\Ex{\ri}{\fd_{K_j,\ri}(g,\zee)}} &~\geq~ \frac{1}{\lstar}\cdot {\tildeEx{\Ex{\li \sim \ri}{\Lambda_{\li, \tau^*}(\zee) \cdot \fd_{K_j,\ri}(g,\zee)}}}- \frac{\lambda+\eta}{\lstar} \\
%		&~\geq~ \frac{1}{\lstar}\cdot {\tildeEx{\Ex{\li \sim \ri}{\Lambda_{\li, \tau^*}(\zee) \cdot \fd_{K_j,\ri}(g,\zee)}}}- \frac{\eps}{6}.\qedhere
%	\end{align*}
\end{proof}


	Finally, we state the local inequality needed from the inner code. Since this inequality is only valid for vertices in $L^*$, we multiply the required equation by indicator $\Lambda_{\li,\tau^*}(\zee)$ so that it is trivial when the indicator is 0. Subject to this indicator being 1, the inequality says that the generalized Singleton bound with erasures holds for the inner code. Note that this bound holds for each vertex in $L^*$ unlike previous sampling lemmas which only work in an average sense over $L^*$.
%		\tnote{Shashank:Check}
	\begin{lemma}[Generalization of \cref{claim:local-bound}]\label{lemma:local_avg_singleton}
		For every $\li \in L$ and for every $j\in [p]$,
		\[
			\Lambda_{\li, \tau^*}(\zee) \parens[\bigg]{ \sum_{j = 1}^p \Ex{e\in N(\li)}{\fd_{K_j, e}(g,\zee)}} ~\geq~ \Lambda_{\li, \tau^*}(\zee) \cdot (p-1) \parens[\Big]{ \delta_0 - \Ex{e\in N(\li)}{\dd_{[k],e}(g,\zee)}  - \frac{\eps}{2}}.
		\]
	\end{lemma}
	\begin{proof}
	Let $M_{\li} \sub E$ be the union of edge neighborhoods over vertices in $R$ that are adjacent to $\li$. That is, 
	\[
		M_{\li} = \bigcup_{\ri 
		\sim \li}  N(\ri)
	\]
	Let us consider the following two local functions that depend on $[k] \times M_{\li}$.
	\begin{align*}
		\mu_1(\cdot) ~&=~ \Lambda_{\li, \tau^*}(\cdot) \;\parens[\bigg]{\; \sum_{j\in [p]} \Ex{e\in N(\li)}{\fd_{K_j, e}(g,\cdot)}\,} \\
		\mu_2(\cdot) ~&=~ \Lambda_{\li, \tau^*}(\cdot) (p-1) \inparen{ \delta_0 - \Ex{e\in N(\li)}{\dd_{[k],e}(g,\cdot)}  - \frac{\eps}{2}}
	\end{align*}
	We will prove this inequality pointwise by showing that for any $\alpha \in \Sigma_{\inn}^{[k] \times M_{\li}}$, $\mu_1(\alpha) \geq \mu_2(\alpha)$.
	
	If $\alpha$ is such that $\Lambda_{\li,\tau^*}(\alpha) = 0$, then the inequality is trivially true. Henceforth, we assume that $\Lambda_{\li,\tau^*}(\alpha) = 1$. This means that $(\alpha_{1,\li},\alpha_{2,\li},\cdots ,\alpha_{k,\li})$ induces the partition $\tau^*$. 
	By definition of $K_j$, for any $i,i'\in K_j$, we get that $\alpha_{i,\li} = \alpha_{i',\li}$. Let us denote this common codeword in $\calC_{\inn}$ as $\fjl$.
	
	Let $g_{\li} \in \Sigma_{\inn}^{d}$ be the local projection of $g$ to the edge neighborhood of $\li$. For every $e\in N(\li)$ that satisfies $\dd_{[k],e}(g,\alpha) = 1$, we replace the corresponding coordinate in $g_{\li}$ by an erasure to obtain $\gl\in \inparen{ \Sigma_{\inn} \cup \{\bot\} }^{N(\li)}$. 
	The fraction of erasures in $g_{\li}$ is
	\[
		s_{\li} = \Ex{e\in N(\li)}{\dd_{[k],e}(g,\alpha)}.
	\]
	Next, we calculate,
	\begin{align*}
		\Ex{e\in N(\li)}{\fd_{K_j, e}(g,\alpha)} ~&=~ \Ex{e\in N(\li)}{\dd_{K_j, e}(g,\alpha) \cdot \inparen{1-\dd_{[k],e}(g,\alpha)}} \\
		~&=~ \Ex{e\in N(\li)}{\dd_{K_j, e}(g,\alpha)} - \Ex{e\in N(\li)}{\dd_{[k],e}(g,\alpha)} \\
		~&=~ \Ex{e\in N(\li)}{\indi{g_{e} \neq f_{j,e}}} - s_{\li} \\
		~&=~ \Delta(g_{\li},\fjl) - s_{\li} \\
		~&=~ \Delta(\gl, \fjl).
	\end{align*}
	Applying the $(\delta_0,k,\eps/2)$-average radius list decodability with erasures of inner code to $\gl$ and the set of codewords $\{\fjl\}_{j\in [p]}$, we get,
	\[
		\sum_{j=1}^p \Delta(\gl,\fjl) ~\geq~ (p-1) \parens[\Big]{\delta_0 - s_{\li} - \frac{\eps}{2}}.
	\]
	Substituting $\Delta(\gl, \fjl) = \Ex{e\in N(\li)}{\fd_{K_j, e}(g,\alpha)}$ and $s_{\li} = \Ex{e\in N(\li)}{\dd_{[k],e}(g,\alpha)}$ gives
	\begin{align*}
		\sum_{j=1}^p \Ex{e\in N(\li)}{\fd_{K_j, e}(g,\alpha)} ~\geq~ (p-1) \cdot\parens[\bigg]{\delta_0 -  \Ex{e\in N(\li)}{\dd_{[k],e}(g,\alpha)} - \frac{\eps}{2}}.\qedhere
	\end{align*}
\end{proof}



	
\subsubsection{Proof of Main Theorem}	

We restate the main theorem and finish the proof using the above lemmas.

\begin{theorem}[Restatement of \cref{thm:sos_main}]
	Let $k_0\geq 1$ be an integer and let $\eps > 0$. Let $\AELC$ be a
code obtained using the AEL construction using $(G,\calC_{\out}, \calC_{\inn})$, where $\calC_{\inn}$ is $(\delta_0, k_0, \eps/2)$ average-radius list decodable with erasures, and the graph $G$ is a $(n,d,\lambda)$-expander. 
	Let $\tildeEx{\cdot}$ be an $\eta$-good $k_0$-tuple of pseudocodewords that satisfies the following pairwise distance property on the left:
	\[
		\forall i,i'\in [k_0] \text{ with } i\neq i', \qquad \tildeEx{\Delta_L(\zee_i, \zee_{i'})} \geq \beta\mper
	\]
	Further, assume that $\lambda \leq \frac{\beta}{12k_0^{k_0}}\cdot \eps$ and $\eta \leq \frac{\beta}{12k_0^{k_0}}\cdot \eps$.	Then, for any $g \in (\F_q^d)^R$,
	\begin{align*}
	\sum_{i\in [k_0]} \tildeEx{\Delta_R(g,\zee_i)} ~&\geq~ (k_0-1)(\delta_0-\eps) + \Ex{\ri \in R}{\tildeEx{ \Pi_{i\in[k_0]} \indi{g_{\ri} \neq \zee_{i,\ri}} }} \\
	~&=~ (k_0-1)(\delta_0 -\eps) + \tildeEx{\Delta_R(g,\zee_{[k_0]})}.
%		\Ex{i\in [k]}{\tildeEx{\Delta_R(g,\zee_i)}} \geq \frac{k-1}{k} \inparen{ \Delta - \Ex{\ri \in R}{\tildeEx{ \Pi_{i\in[k]} \indi{g_{\ri} \neq \zee_{i,\ri}} }} }
	\end{align*}
%
\end{theorem}

\begin{proof}
The proof, as before, is by induction on $k_0$. The base case, $k_0 =1$ is trivial. So we assume the statement holds upto $k-1$, and the goal is to prove it for $k_0=k$. Fix a partition $\tau^* = (K_1,\cdots, K_p)$ as guaranteed by~\cref{claim:best_partition}.  For each part $K_j$, we may apply the induction hypothesis to the sub-tuple defined by it as the pairwise distance property is already assumed. Using this we get, 
	\begin{align*}
\sum_{i=1}^k \tildeEx{\Delta_R(g,\zee_i)} 
		~&=~ \sum_{j=1}^p \sum_{i\in K_j} \tildeEx{\Delta_R(g,\zee_i)} \\
		~&\geq~ \sum_{j=1}^p \inparen{(|K_j|-1)(\delta_0 -\eps) + \tildeEx{\Delta_R(g,\zee_{K_j})}}.
		\end{align*}

The first term is simply,   
$
\sum_{j=1}^p (|K_j|-1)(\delta_0 -\eps) ~=~ (k-p)(\delta_0 -\eps) . 
$ The goal is now to show that, 
\[\sum_{j=1}^p\tildeEx{\Delta_R(g,\zee_{K_j})}	 ~\geq~ (p-1)(\delta_0 -\eps) + \tildeEx{\Delta_R(g,\zee_{[k]})}.\]
 
 For a fixed $j \in [p]$, we apply \cref{lemma:more_common_errors} to obtain, 

		\[
		\tildeEx{\Delta_R(g,\zee_{K_j})}	~\geq~  \tildeEx{\Delta_R(g,\zee_{[k]})}  + \frac{\Ex{\li \sim \ri}{\;\tildeEx{\Lambda_{\li,\tau^*}(\zee) \cdot \fd_{K_j,\ri}(g,\zee)}\,} }{\Ex{\li}{\tildeEx{\Lambda_{\li,\tau^*}(\zee)}}} - \frac{\eps}{6}.
		\]

The term $c := \tildeEx{\Delta_R(g,\zee_{[k]})}  - \frac{\eps}{6}$ is independent of $j$.
Summing the RHS over $j \in [p]$, 

\begin{align}
\sum_{j=1}^p	\tildeEx{\Delta_R(g,\zee_{K_j})} - p\cdot c ~&\geq~ \sum_{j=1}^p \frac{\Ex{\li \sim \ri}{\tildeEx{\Lambda_{\li,\tau^*}(\zee) \cdot \fd_{K_j,\ri}(g,\zee)}} }{\Ex{\li}{\tildeEx{\Lambda_{\li,\tau^*}(\zee)}}}\\
%~&=~ \sum_{j=1}^p \frac{\Ex{\li}{\tildeEx{\Lambda_{\li,\tau^*}(\zee) \cdot \inparen{ \Ex{e\in N(\li)}{ \fd_{K_j,e}(g,\zee)}}}} }{\Ex{\li}{\tildeEx{\Lambda_{\li,\tau^*}(\zee)}}}\\
~&=~   { \frac{\Ex{\li}{\tildeEx{\Lambda_{\li,\tau^*}(\zee) \cdot \sum_{j=1}^p \inparen{\Ex{e\in N(\li)}{\fd_{K_j,e}(g,\zee)}} }}}{ \Ex{\li}{\tildeEx{\Lambda_{\li,\tau^*}(\zee)}} } }\label{eq:main}
		\end{align}

We can now apply \cref{lemma:local_avg_singleton} to the RHS
\begin{align}		
	~&\geq~ { \frac{\Ex{\li}{\tildeEx{\Lambda_{\li, \tau^*}(\zee) \cdot (p-1) \inparen{ \delta_0 - \Ex{e\in N(\li)}{\dd_{[k],e}(g,\zee)}  - \frac{\eps}{2}} }}}{\Ex{\li}{\tildeEx{\Lambda_{\li,\tau^*}(\zee)}}}} \\
	~&=~	(p-1)(\delta_0 - \frac{\eps}{2}) -(p-1) \frac{\Ex{\li}{\tildeEx{\Lambda_{\li, \tau^*}(\zee) \cdot \inparen{\Ex{e\in N(\li)}{\dd_{[k],e}(g,\zee)} } }}}{\Ex{\li}{\tildeEx{\Lambda_{\li,\tau^*}(\zee)}}}.\label{eq:second}
		 \end{align}

		
To bound the term on the right, we use~\cref{lemma:local_erasures_upper_bound},			\begin{align*}
				\frac{\Ex{\li}{\tildeEx{\Lambda_{\li, \tau^*}(\zee) \cdot \inparen{\Ex{e\in N(\li)}{\dd_{[k],e}(g,\zee)} } }}}{\Ex{\li}{\tildeEx{\Lambda_{\li,\tau^*}(\zee)}}} ~&=~ \frac{\Ex{\li\sim \ri}{\tildeEx{\Lambda_{\li, \tau^*}(\zee) \cdot \dd_{[k],\ri}(g,\zee)  }}}{\Ex{\li}{\tildeEx{\Lambda_{\li,\tau^*}(\zee)}}}\\[1em]
\text{(\cref{lemma:local_erasures_upper_bound})}\;\;				~&\leq~  \tildeEx{\Ex{\ri}{\dd_{S,\ri}(g,\zee)}} + \frac{\eps}{6}\\
				~&=~	\tildeEx{\Delta_R(g,\zee_{[k]})} + \frac{\eps}{6}.  
				\end{align*}

Plugging this bound in~\cref{eq:second} and then back in~\cref{eq:main},  we get,		
\begin{align*}	
		\tildeEx{\Delta_R(g,\zee_{K_j})}  ~&\geq~ p\cdot c + (p-1)\parens[\Big]{\delta_0 - \frac{\eps}{2}} -  (p-1) \inparen{\tildeEx{\Delta_R(g,\zee_{[k]})} + \frac{\eps}{6}}\\
		~&=~ \tildeEx{\Delta_R(g,\zee_{[k]})}  +(p-1)\parens[\Big]{\delta_0 - \frac{\eps}{2}} - (2p-1)\cdot \frac{\eps}{6} \\
		~&\geq~ \tildeEx{\Delta_R(g,\zee_{[k]})}  + (p-1)(\delta_0 - \eps) \;. \qedhere
		\end{align*}
%	\begin{align*}
%		&= (k-1)(\Delta - \eps) + \Delta_R(g,\zee_{[k]}) +(p-1)\frac{\eps}{2} - (2p-1)\cdot k^k\inparen{\frac{\lambda+\eta}{\beta}} \\
%		&\geq (k-1)(\Delta - \eps) + \Delta_R(g,\zee_{[k]})
%	\end{align*}
\end{proof}

%\paragraph{Proof of the three lemmas}



\subsection{The final algorithm}\label{sec:sos_algo}

In this subsection, we describe the final decoding algorithm. We will mainly rely on the main result of the previous subsection: that for appropriately instantiated AEL codes and for $\eta$ small enough, $\eta$-good $k$-tuple of pseudocodewords satisfy the generalized Singleton bound.

\subsubsection{Decoding from Distributions}
Before describing our main algorithm, we argue that a simple idea based on randomized rounding can be used to extend the unique decoder of $\calC_{\out}$ to decode not only from integral strings close to a codeword, but an ensemble of distributions - one for each coordinate - that is close to a codeword in average sense. It is also standard to derandomize this process via threshold rounding, which is what we describe next. We will be needing this strengthening of the unique decoder of $\calC_{\out}$ as a subroutine in the main algorithm.

It will be helpful to index the codewords in $\calC_{\inn}$ by integers. Let $\abs*{\calC_{\inn}} = M$, and let its codewords be $\alpha_1,\alpha_2,\cdots ,\alpha_M$. Recall that for any $k$-tuple of pseudocodewords $\tildeEx{\cdot}$, for any $j\in [k]$ and $\li \in L$, the set of values
\[
	\inbraces{ \tildeEx{\indi{\zee_{j,\li} = \alpha_i}} }_{i\in [M]}
\]
correspond to a probability distribution over $\calC_{\inn}$. Therefore, the set of intervals
\[
	\inbraces{ \left[\sum_{i=1}^{m-1} \tildeEx{\indi{\zee_{j,\li} = \alpha_i}}, \sum_{i=1}^{m} \tildeEx{\indi{\zee_{j,\li} = \alpha_i}} \right) }_{m\in [M]}
\]
partitions the interval $[0,1)$ into at most $M$ parts.

\begin{lemma}\label{lem:decode_from_distrib}
	Suppose the code $\calC_{\out}$ is unique decodable up to radius $\delta_{\out}^{\dec}$, and let $h$ be a codeword in $\AELC$. Given a collection of distributions $\{\calD_{\li}\}_{\li \in L}$, with each distribution over $\calC_{\inn}$ such that
	\[
		\Ex{\li}{\Ex{f\sim \calD_{\li}}{{\indi{f\neq h_{\li}}}}} \leq \delta_{\out}^{\dec},
	\]
	the \cref{algo:unique-decoding} finds $h$.
\end{lemma}

\begin{figure}[!ht]
\begin{algorithm}{\DECODE}{$\{\calD_{\li}\}_{\li\in L}$}{Codeword $h\in \AELC$ such that $\Ex{\li}{\Ex{f\sim \calD_{\li}}{\indi{f\neq h_{\li}}}} \leq \delta_{\out}^{\dec}$}
\label{algo:unique-decoding}
%
\begin{itemize}
%
%	\item For $p=1$ to $k$:\label{step2}
%	\begin{enumerate}[(i)]
%%		\item Find a $p$-tuple of pseudocodewords $\pcod{p}{\cdot}$ of degree $t = \cdots$ that respects the following constraints:
%%		\begin{itemize}
%%			\item For every $i,i' \in [p]$ with $i\neq i'$, $~\pcod{p}{\Delta_L(\zee_i,\zee_{i'})} > \delta_{\out}^{\dec}$.
%%			\item For every $i\in [p]$, $\pcod{p}{\Delta_R(g,\zee_i)} \leq \frac{k-1}{k}(1-\rho-\eps)$.
%%		\end{itemize}
%%		\item If no such $\pcod{p}{\cdot}$ exists:
%%		\begin{itemize}
%%		\item $p^* = p-1$
%%		\item Exit loop.
%%		\end{itemize}
%	\end{enumerate}
	\item Let $w_{\li,j}$ be the weight on codeword $\alpha_i$ according to distribution $\calD_{\li}$.
	\item For every threshold $\theta \in [0,1]$:\footnote{As written, this involves trying uncountably many thresholds. However, we only need to try at most $M \cdot |L|$ thresholds since the algorithm only depends on which intervals $\theta$ belongs to. Since the number of endpoints of these intervals is at most $M\cdot |L|$ in total, it suffices to try only $M\cdot |L| = M\cdot n$ many distinct thresholds. This is a standard method called threshold rounding.}
			\begin{enumerate}[(i)]
				\item Construct an $f_{\theta} \in \calC_{\inn}^L$ by assigning 
				\[(f_{\theta})_{\li} = \alpha_m  \iff \theta \in \left[ \sum_{i=1}^{m-1} w_{\li, i}, \sum_{i=1}^{m} w_{\li,i} \right)\]
				for every $\li \in L$.
				\item Let $f_\theta^* \in \Sigma_{\out}^L$ defined as $(f_{\theta}^*)_{\li} = \phi^{-1}((f_{\theta})_{\li})$. 
				\item Use the unique decoder of $\calC_{\out}$ to find an $h^* \in \calC_{\out}$ whose distance from $f_{\theta}^*$ is at most $\delta_{\out}^{\dec}$, if such an $h^*$ exists. That is, $h^* \leftarrow \mathrm{Dec}_{\calC_{\out}}(f_{\theta}^*,\delta_{\out}^{\dec})$.
				\item Let $h\in \AELC$ be the codeword corresponding to $h^* \in \calC_{\out}$. Return $h$.
				\end{enumerate}
		\end{itemize}
%		\item For every threshold $\theta \in [0,1]$:\footnote{As written, this involves trying uncountably many thresholds. However, we only need to try at most $M \cdot |L|$ thresholds since the algorithm only depends on which intervals $\theta$ belongs to. Since the number of thresholds is at most $M\cdot |L|$, it suffices to try only $M\cdot |L|$ many distinct thresholds. This is a standard method called threshold rounding.}
%			\begin{enumerate}[(i)]
%				\item Construct an $f_{\theta} \in \calC_{\inn}^L$ by assigning 
%				\[f_{\li} = \alpha_m  \iff \theta \in \left[ \sum_{i=1}^{m-1} \pcod{p^*}{\indi{\zee_{j,\li} = \alpha_i}}, \sum_{i=1}^{m} \pcod{p^*}{\indi{\zee_{j,\li} = \alpha_i}} \right)\]
%				for every $\li \in L$.
%				\item Let $f_\theta^* \in \Sigma_{\out}^L$ defined as $(f_{\theta}^*)_{\li} = \phi^{-1}((f_{\theta})_{\li})$. 
%				\item Use the unique decoder of $\calC_{\out}$ to find an $h^* \in \calC_{\out}$ whose distance from $f_{\theta}^*$ is at most $\delta_{\out}^{\dec}$, if such an $h^*$ exists. That is, $h^* \leftarrow \mathrm{Dec}(f_{\theta}^*,\delta_{\out}^{\dec})$.
%				\item Let $h$ be the codeword of $\AELC$ corresponding to $h^* \in \calC_{\out}$. If $\Delta_R(g,h) < \frac{k-1}{k} (1-\rho-\eps)$, add $h$ to $\calL$.
%			\end{enumerate}
%
\vspace{5pt}
%
\end{algorithm}
\end{figure}

\begin{proof}
	It suffices to show that there exists a threshold $\theta \in [0,1)$ for which the distance between $f_{\theta}^*$ constructed by \cref{algo:unique-decoding} and $h^*$ is at most $\delta_{\out}^{\dec}$. In fact, we will show that this is true for an average $\theta$.
	\begin{align*}
		\Ex{\theta \in [0,1)}{ \Delta(f_{\theta}^*,h^*)} &= \Ex{\theta \in [0,1)}{ \Delta_L(f_{\theta},h)} = \Ex{\theta \in [0,1)}{ \Ex{\li}{\indi{(f_{\theta})_{\li} \neq h_{\li}}}}  = \Ex{\theta \in [0,1)}{ \Ex{\li}{\sum_{i\in [M] : \alpha_i \neq h_{\li}}\indi{(f_{\theta})_{\li} = \alpha_i}}} 
	\end{align*}
	We can move the expectation over $\theta$ inside and use the fact that $(f_{\theta})_{\li}$ is $\alpha_i$ with probability exactly $w_{\li,i}$ to get
	\begin{align*}
		\Ex{\li}{\sum_{i\in [M] : \alpha_i \neq h_{\li}} \Ex{\theta \in [0,1)}{\indi{(f_{\theta})_{\li} = \alpha_i}}} =  \Ex{\li}{\sum_{i\in [M] : \alpha_i \neq h_{\li}} w_{\li,i}} = \Ex{\li}{\Ex{f\sim \calD_{\li}}{{\indi{f\neq h_{\li}}}}} \leq \delta_{\out}^{\dec}. \qedhere
	\end{align*}
\end{proof}

\subsubsection{Decoding Algorithm}
\begin{table}[h]
\hrule
\vline
\begin{minipage}[t]{0.99\linewidth}
\vspace{-5 pt}
{\small
\begin{align*}
    &\mbox{find}\quad ~~ \tildeEx{\cdot} ~\text{on}~ \zee_{[p],E} \text{ and alphabet }\Sigma_{\inn}%\tag{List Decoding Program}\label{sos:list_dec}
    \\
&\mbox{subject to}\quad \quad ~\\
	&\qquad \text{(i)}~~ \tildeEx{\cdot} \text{is a }p\text{-tuple of pseudocodewords of SoS-degree } t \\
    &\qquad \text{(ii)}~~ \forall i \in [p],~ \text{the constraint }\Delta_R(g,\zee_i) < \frac{k-1}{k}(1-\rho-\eps) \text{ is respected by }\tildeEx{\cdot}\label{cons:agreement-ld}    \\
&\qquad \text{(iii)}~ \forall F \sub E \text{ such that } |F|\leq \frac{t-2pd}{2}, \sigma \in \Sigma_{\inn}^{[p]\times F} \text{ and }\forall~i,i' \in [p] \text{ with } i\neq i', \\
&\qquad \qquad \qquad \qquad \tildeEx{\inparen{\Delta_L(\zee_i,\zee_{i'}) - \delta_{\out}^{\dec}}\cdot  \indi{\zee_{[p],F} = \sigma}^2 } \geq 0
\end{align*}}
\vspace{-10 pt}
\end{minipage}
\hfill\vline
\hrule
\caption{$\mathrm{SDP}(p, t)$}
\label{tab:SDP_for_feasibility}
\end{table}

\begin{theorem}\label{thm:sos_technical}
%
Let $k\geq 1$ be an integer and let $\eps > 0$. Let $\AELC$ be a
code obtained using the AEL construction using  $(G, \calC_{\out}, \calC_{\inn})$, where $\calC_{\inn}$ is $(\delta_0, k, \eps/2)$ average-radius list decodable with erasures, and $G$ is a $(n,d,\lambda)$-expander. 
%
Suppose that $\calC_{\out}$ is unique decodable from radius $\delta_{\out}^{\dec}$ in time $\calT(n)$.

If $\lambda \leq \frac{\delta_{\out}^{\dec}}{12{k}^{k}} \cdot \eps$, then there is a deterministic algorithm that takes as input $g\in (\Sigma_{\inn}^d)^R$, runs in time $\calT(n) + n^{O\inparen{ \frac{d \cdot k^{3k}|\Sigma_{\inn}|^{3kd}}{(\delta_{\out}^{\dec})^2\cdot \eps^2}}}$, and outputs the list $\calL(g,\frac{k-1}{k}(\delta_0-\eps))$.
\end{theorem}

\begin{figure}[!ht]
\begin{algorithm}{List Decoding algorithm up to $\frac{k-1}{k}(\delta_0-\eps)$}{$k$, $g \in (\Sigma_{\inn}^d)^R$}{List of codewords $\calL\inparen{g, \frac{k-1}{k}\inparen{\delta_0-\eps}}$}\label{algo:sos-decoding}
%
\begin{enumerate}
%
	\item Initialize $\calL = \{ \}$, $t = 2kd\cdot \inparen{1+\frac{144k^{2k}|\Sigma_{\inn}|^{3kd}}{(\delta_{\out}^{\dec})^2\cdot \eps^2}}$.
	\item Let $p^*$ be the largest $p\in [k]$ such that $\mathrm{SDP}(p,t)$ is feasible.
%	\item For $p=1$ to $k$:\label{step2}
%	\begin{enumerate}[(i)]
%%		\item Find a $p$-tuple of pseudocodewords $\pcod{p}{\cdot}$ of degree $t = \cdots$ that respects the following constraints:
%%		\begin{itemize}
%%			\item For every $i,i' \in [p]$ with $i\neq i'$, $~\pcod{p}{\Delta_L(\zee_i,\zee_{i'})} > \delta_{\out}^{\dec}$.
%%			\item For every $i\in [p]$, $\pcod{p}{\Delta_R(g,\zee_i)} \leq \frac{k-1}{k}(1-\rho-\eps)$.
%%		\end{itemize}
%%		\item If no such $\pcod{p}{\cdot}$ exists:
%%		\begin{itemize}
%%		\item $p^* = p-1$
%%		\item Exit loop.
%%		\end{itemize}
%	\end{enumerate}
	\item For every $F \sub E$ with $|F|\leq \frac{t-2kd}{2}$ and $\sigma\in \Sigma_{\inn}^F$ with $\pcod{p^*}{\indi{\zee_{[p^*],F} = \sigma}} > 0$:\label{step3}
		\begin{itemize}
		\item For $j=1$ to $p^*$:
		\begin{itemize}
			\item Construct an ensemble of distributions $\calD = \{\calD_{\li}\}_{\li\in L}$, with each $\calD_{\li}$ over $\calC_{\inn}$ by using the local distribution induced by the conditioned $\pcod{p^*}{ ~\cdot \given \zee_{[p^*],F} = \sigma}$ over the set $\{j\} \times N(\li)$, or equivalently, over variables $\zee_{j,\li}$.
			\item $h\leftarrow \text{\DECODE} \inparen{\calD}$.
			\item If $\Delta_R(g,h) < \frac{k-1}{k} (\delta_0-\eps)$, add $h$ to $\calL$.
		\end{itemize}
%		\item For every threshold $\theta \in [0,1]$:\footnote{As written, this involves trying uncountably many thresholds. However, we only need to try at most $M \cdot |L|$ thresholds since the algorithm only depends on which intervals $\theta$ belongs to. Since the number of thresholds is at most $M\cdot |L|$, it suffices to try only $M\cdot |L|$ many distinct thresholds. This is a standard method called threshold rounding.}
%			\begin{enumerate}[(i)]
%				\item Construct an $f_{\theta} \in \calC_{\inn}^L$ by assigning 
%				\[f_{\li} = \alpha_m  \iff \theta \in \left[ \sum_{i=1}^{m-1} \pcod{p^*}{\indi{\zee_{j,\li} = \alpha_i}}, \sum_{i=1}^{m} \pcod{p^*}{\indi{\zee_{j,\li} = \alpha_i}} \right)\]
%				for every $\li \in L$.
%				\item Let $f_\theta^* \in \Sigma_{\out}^L$ defined as $(f_{\theta}^*)_{\li} = \phi^{-1}((f_{\theta})_{\li})$. 
%				\item Use the unique decoder of $\calC_{\out}$ to find an $h^* \in \calC_{\out}$ whose distance from $f_{\theta}^*$ is at most $\delta_{\out}^{\dec}$, if such an $h^*$ exists. That is, $h^* \leftarrow \mathrm{Dec}(f_{\theta}^*,\delta_{\out}^{\dec})$.
%				\item Let $h$ be the codeword of $\AELC$ corresponding to $h^* \in \calC_{\out}$. If $\Delta_R(g,h) < \frac{k-1}{k} (1-\rho-\eps)$, add $h$ to $\calL$.
%			\end{enumerate}
		\end{itemize}
	\item Return $\calL$.
\end{enumerate}
%
\vspace{5pt}
%
\end{algorithm}
\end{figure}

\begin{proof}
	\cref{algo:sos-decoding} describes this algorithm. In the rest of the proof, we argue the correctness of this algorithm.
	
	We first start with a lemma that readily follows from \cref{thm:sos_main}.
	
	\begin{lemma}\label{lem:pseudocodeword_list_size}
	If $\lambda \leq  \frac{\delta_{\out}^{\dec}}{12k^k} \cdot \eps$, and $t \geq 2kd\cdot \inparen{1+\frac{144k^{2k}|\Sigma_{\inn}|^{3kd}}{(\delta_{\out}^{\dec})^2\cdot \eps^2}}$, then the $\mathrm{SDP}(k,t)$ is infeasible.
	\end{lemma}

	In other words, the lemma says that $p^* < k$, and that $\mathrm{SDP}(p^*+1,t)$ is infeasible. Let $\pcod{p^*}{\cdot}$ be the $p^*$-tuple of pseudocodewords found by $\mathrm{SDP}(p^*,t)$.
%	
	\begin{lemma}\label{lem:covering}
	For any $h \in \calL\inparen{g,\frac{k-1}{k}(1-\rho-\eps)}$,
%	 and its corresponding $h^* \in \calC_{\out}$, 
there exists an $i\in [p^*]$, a set $F\sub E$ with $|F|\leq \frac{t-2kd}{2}$, and a $\sigma \in \Sigma_{\inn}^{[p^*]\times F}$ with $\pcod{p^*}{\indi{\zee_{[p^*],F} = \sigma}} > 0$, such that 
	\[
	\pcod{p^*}{\Delta_L(h, \zee_i) \given \zee_{[p^*],F} = \sigma} \leq \delta_{\out}^{\dec}.
	\] 
	\end{lemma}
	This lemma proves the correctness of the algorithm, since then $h$ will be added to $\calL$ in \hyperref[step3]{Step~\ref*{step3}}, by \cref{lem:decode_from_distrib}.
\end{proof}

\begin{proof}[Proof of \cref{lem:covering}]
	To show this, we will construct a $(p^*+1)$-tuple of pseudocodewords, say $\pcod{p^*+1}{\cdot}$, and use its infeasibility for $\mathrm{SDP}(p^*+1,t)$.
	
	Recall that $\pcod{p^*}{\cdot}$ is an SoS relaxation over the variables $\zee_{[p^*] \times E }$,
% = \{ Z_{i,e,s} \}_{i\in [p^*], e\in E,s \in \Sigma_{\inn}}$
while $\pcod{p^*+1}{\cdot}$ will be an SoS relaxation over the variables $\zee_{[p^*+1]\times E }$.
% = \{ Z_{i,e,s} \}_{i\in [p^*+1], e\in E,s\in \Sigma_{\inn}}$. 
	
	To describe the new $(p^*+1)$-tuple of pseudocodewords, we explicitly specify the corresponding pseudoexpectation operator $\pcod{p^*+1}{\cdot}$.
	Let $P$ be an arbitrary polynomial of degree $t$ over $\zee_{[p^*+1], E}$. We define a new polynomial $P_h$ over $\zee_{[p^*], E}$ by assigning the variables $\zee_{p^*+1,e}$ to be $h_e$ for every $e\in E$.
%	\begin{align*}
%		Z_{p^*+1,e,s} = \begin{cases}
%		1 \quad & h_e = s \\
%		0 \quad & h_e \neq s
%		\end{cases}
%	\end{align*}	
	Then, we define $\pcod{p^*+1}{\cdot}$ using the following
	\[
		\pcod{p^*+1}{P(\zee_{[p^*+1],E})} = \pcod{p^*}{P_h(\zee_{[p^*],E})}
	\]
	This is well-defined since the degree of $P_h$ cannot be more than $t$.
	
	By optimality of $p^*$, $\pcod{p^*+1}{\cdot}$ must be infeasible for $\mathrm{SDP}(p^*+1,t)$. It can be verified that $\pcod{p^*+1}{\cdot}$ is a valid $(p^*+1)$-tuple of pseudocodewords. Further, the constraints 
	\[
		\Delta_R(g,\zee_i) < \frac{k-1}{k}(\delta_0 - \eps)
	\]
	are respected for all $i\in [p^*]$. This is because 
	\[
		\pcod{p^*+1}{P(\zee)^2 \cdot \inparen{\Delta_R(g,\zee_i) - \frac{k-1}{k}(\delta_0 - \eps)}} = \pcod{p^*}{P_h(\zee)^2 \cdot \inparen{\Delta_R(g,\zee_i) - \frac{k-1}{k}(\delta_0 - \eps)}} < 0
	\]
	Further, for $i=p^*+1$,
	\begin{align*}
		\pcod{p^*+1}{P(\zee)^2 \cdot \inparen{\Delta_R(g,\zee_i) - \frac{k-1}{k}(\delta_0 - \eps)}} &= \pcod{p^*}{P_h(\zee)^2 \cdot \inparen{\Delta_R(g,h) - \frac{k-1}{k}(\delta_0 - \eps)}} \\
		&= \inparen{\Delta_R(g,h) - \frac{k-1}{k}(\delta_0 - \eps)} \cdot \pcod{p^*}{P_h(\zee)^2} \\
		&< 0
	\end{align*}
	Therefore, the infeasibility must be due to constraints of type (iii) in the SDP. In other words, there exist
	\begin{enumerate}
	\item a set $F\sub E$ with $|F|\leq \frac{t-2(p^*+1)d}{2}$,
	\item a $\sigma \in \Sigma_{\inn}^{[p^*+1]\times F}$ 
	%with $\pcod{p^*+1}{\zee_{[p^*+1],F,\sigma}} >0$
	, and 
	\item a pair $i,i'\in [p^*+1]$ with $i\neq i'$
	\end{enumerate}
	such that
	\[
		\pcod{p^*+1}{ \inparen{\Delta_L(\zee_i,\zee_{i'}) - \delta_{\out}^{\dec}} \cdot \indi{\zee_{[p^*+1],F} = \sigma}^2} < 0
	\]
	Case I. $i,i' \in [p^*]$:
	
	In this case,
	\begin{align*}
		0~>&~ \pcod{p^*+1}{ \inparen{\Delta_L(\zee_i,\zee_{i'}) - \delta_{\out}^{\dec}} \cdot \indi{\zee_{[p^*+1],F} = \sigma}} \\
		=~~ & \pcod{p^*+1}{ \inparen{\Delta_L(\zee_i,\zee_{i'}) - \delta_{\out}^{\dec}} \cdot \indi{\zee_{[p^*],F} = \sigma_1} \cdot \indi{\zee_{p^*+1,F} = \sigma_2}} \\
		=~~ & \pcod{p^*}{ \inparen{\Delta_L(\zee_i,\zee_{i'}) - \delta_{\out}^{\dec}} \cdot \indi{\zee_{[p^*],F} = \sigma_1} \cdot \indi{h_F = \sigma_2}} \\
		=~~ & \indi{h_F = \sigma_2} \cdot \pcod{p^*}{ \inparen{\Delta_L(\zee_i,\zee_{i'}) - \delta_{\out}^{\dec}} \cdot \indi{\zee_{[p^*],F} = \sigma_1} }
	\end{align*}
Because of the strict inequality with $0$, it must be the case that $\indi{h_F = \sigma_2} =1$, and the resulting equation contradicts the feasibility of $\pcod{p^*}{\cdot}$. So this case cannot happen.

\medskip
%	
\noindent Case II. Without loss of generality, $i'=p^*+1$.
	
%	We may assume $\pcod{p^*+1}{\indi{\zee_{[p^*+1],F} = \sigma}^2} > 0$. If not, 
%	\begin{align*}
%		&\pcod{p^*+1}{\indi{\zee_{[p^*+1],F} = \sigma}^2} = 0 \\
%		\implies & \pcod{p^*+1}{ \Delta_L(\zee_i,\zee_{i'}) \cdot \indi{\zee_{[p^*+1],F} = \sigma}^2} < 0
%	\end{align*}
%	
%	Further, it can be checked that both $i$ and $i'$ cannot be $\leq p^*$, since $\pcod{p^*}{\cdot}$ is feasible for $\mathrm{SDP}(p^*,t)$. Without loss of generality, let $i' = p^*+1$. Then,
In this case,
	\begin{align*}
		0 &~>~  \pcod{p^*+1}{ \inparen{\Delta_L(\zee_i,\zee_{p^*+1}) - \delta_{\out}^{\dec}} \cdot \indi{\zee_{[p^*+1],F} = \sigma}} \\
		&=  \pcod{p^*+1}{ \inparen{\Delta_L(\zee_i,\zee_{p^*+1}) - \delta_{\out}^{\dec}} \cdot \indi{\zee_{[p^*],F} = \sigma_1} \cdot \indi{\zee_{p^*+1,F} = \sigma_2}} \\
		&=\pcod{p^*}{ \inparen{\Delta_L(\zee_i,h) - \delta_{\out}^{\dec}} \cdot \indi{\zee_{[p^*],F} = \sigma_1} \cdot \indi{h_F = \sigma_2}}\\
		&=\indi{h_F = \sigma_2} \cdot \pcod{p^*}{ \inparen{\Delta_L(\zee_i,h) - \delta_{\out}^{\dec}} \cdot \indi{\zee_{[p^*],F} = \sigma_1}}\\
%		&=\frac{\pcod{p^*}{ \inparen{\Delta_L(\zee_i,h) - \delta_{\out}^{\dec}} \cdot \indi{\zee_{[p^*],F} = \sigma_1}}}{\pcod{p^*}{\indi{\zee_{[p^*],F} = \sigma_1}}} \\
%		&=\pcod{p^*}{ \inparen{\Delta_L(\zee_i,h) - \delta_{\out}^{\dec}} \given \zee_{[p^*],F} = \sigma_1}
	\end{align*}
	Again, as before we must have $\indi{h_F = \sigma_2} =1$, so that
	\begin{align}\label{eq:without_h}
		\pcod{p^*}{ \inparen{\Delta_L(\zee_i,h) - \delta_{\out}^{\dec}} \cdot \indi{\zee_{[p^*],F} = \sigma_1}} < 0
	\end{align}
	Finally, we may also assume $\pcod{p^*}{\indi{\zee_{[p^*+1],F} =\sigma}} >0$. If not, then
	\[
		\pcod{p^*}{ \Delta_L(\zee_i,h) \cdot \indi{\zee_{[p^*],F} = \sigma_1}} < 0
	\]
	which is impossible since $\pcod{p^*}{P(\zee)} \geq 0$ for all polynomials that are sum of squares of polynomials. It is easy to verify that $\Delta_L(\zee_i,h) \cdot \indi{\zee_{[p^*],F} = \sigma_1}$ is a sum of squares of polynomials.
	
	Therefore, dividing \cref{eq:without_h} by $\pcod{p^*}{\indi{\zee_{[p^*+1],F} =\sigma}}$, we get that the set $F$ and the assignment $\sigma_1 \in \Sigma_{\inn}^{[p^*] \times F}$ have the property that
	\[
		\pcod{p^*}{\Delta_L(\zee_i,h) \given \zee_{[p^*],F} = \sigma_1} \leq \delta_{\out}^{\dec}
	\]
	as needed.
\end{proof}

\begin{proof}[Proof of \cref{lem:pseudocodeword_list_size}]
	Suppose $\mathrm{SDP}(k,t)$ is feasible with $t = 2kd\cdot \inparen{1+\frac{144k^{2k}|\Sigma_{\inn}|^{3kd}}{(\delta_{\out}^{\dec})^2\cdot \eps^2}}$, so that there exists a $k$-tuple of pseudocodewords $\tildeEx{\cdot}$ that satisfies all the constraints in the SDP. 
	Let $\eta = \frac{\delta_{\out}^{\dec}}{12k^k} \cdot \eps$.
	
	 Then $t\geq 2kd\cdot \inparen{1+\frac{|\Sigma_{\inn}|^{3kd}}{\eta^2}}$, so that \cref{lem:condition_for_eta_good} says that there exists a set $S \sub E$ of size at most $d\cdot \frac{|\Sigma_{\inn}|^{3kd}}{\eta^2}$ such that \[\pcod{S}{\cdot} \defeq \tildeEx{~\cdot \given \zee_{[k],S}} \] is $\eta$-good.
	
	Further, since the constraint $\Delta_R(g,\zee_i) < \frac{k-1}{k}(1-\rho-\eps)$ is \emph{respected} by $\tildeEx{\cdot}$, we can also conclude that for all $i\in [k]$,
	\begin{align}\label{eqn:zee_i_in_ball}
		\pcod{S}{\Delta_R(g,\zee_i)} < \frac{k-1}{k}(1-\rho-\eps)
	\end{align}
	Moreover, for any $i,i'\in [k]$ with $i\neq i'$,
	\[
		\pcod{S}{\inparen{\Delta_L(\zee_i,\zee_{i'}) - \delta_{\out}^{\dec}}} = \Ex{\sigma \sim \calD_{[k]\times S}}{\frac{\tildeEx{ \inparen{\Delta_L(\zee_i,\zee_{i'}) - \delta_{\out}^{\dec}} \cdot \indi{\zee_{[k], S} = \sigma}^2}}{\tildeEx{\indi{\zee_{[k], S}}^2}}} \geq 0
	\]
	where $\calD_{[k]\times S}$ is the local distribution on $\zee_{[k]\times S}$ according to $\tildeEx{\cdot}$. This means that
	\[
		\pcod{S}{\Delta_L(\zee_i,\zee_{i'})} \geq \delta_{\out}^{\dec}
	\]
	Since $\lambda \leq \frac{\delta_{\out}^{\dec}}{12k^k} \cdot \eps$ and $\eta \leq \frac{\delta_{\out}^{\dec}}{12k^k} \cdot \eps$, we can apply \cref{thm:sos_main} to $\pcod{S}{\cdot}$ with $\beta = \delta_{\out}^{\dec}$ and get
	\[
		\sum_{i\in [k]} \pcod{S}{\Delta_R(g,\zee_i)} \geq \frac{k-1}{k}(1-\rho-\eps) + \pcod{S}{\Delta_R(g,\zee_{[k]})} \geq \frac{k-1}{k}(1-\rho-\eps)
	\]
	which is contradicted by \cref{eqn:zee_i_in_ball}.
\end{proof}

Finally, we instantiate \cref{thm:sos_technical} with unique decodable outer codes to obtain codes that can be efficiently decoded up to the list decoding capacity.
\begin{corollary}\label{cor:algo-main}
For every $\rho, \eps \in (0,1)$ and $k \in \N$, there exist explicit inner codes $\calC_{\inn}$ and an infinite family explicit codes $\AELC \subseteq (\F_q^d)^n$ obtained via the AEL construction that satisfy: 
\begin{enumerate}
\item $\rho(\AELC) \geq \rho$
\item For any $g \in  (\F_q^d)^n$ and any $\calH \subseteq \AELC$ with $\abs{\calH} \leq k$ that
\[
\sum_{h \in \calH} \Delta(g,h) ~\geq~ (\abs{\calH}-1) \cdot (1 - \rho - \eps) \mper
\]
\item The alphabet size $q^d$ of the code $\AELC$ can be taken to be $2^{O(k^{3k}/\eps^9)}$.
\item $\AELC$ can be decoded from radius $\frac{k-1}{k}(1-\rho-\eps)$ in time $n^{2^{O(k^{4k}/\eps^{10})}}$ with a list of size at most $k-1$.
\end{enumerate}
\end{corollary}

\begin{proof}
	The first three properties can be ensured by instantiating the AEL amplification procedure as in \cref{cor:ael_instantiation}. We briefly recall the choices in that instantiation.
	
	\begin{itemize}
		\item We picked a graph $G$ with $d = O(k^{2k}/\eps^8)$ and $\lambda \leq \eps^4/(2^{21} k^k)$. 
%
		\item We picked $\calC_{\inn} \subseteq \Sigma_{\inn}^d$ to be a code with rate $\rho_{\inn} = \rho + \eps/4$, which is $(1 - \rho, k, \eps/2)$ average-radius list decodable with erasures. The alphabet size $|\Sigma_{\inn}|$ for $\calC_{\inn}$ was $2^{O(k + 1/\eps)}$.
%
		\item Finally, we picked $\calC_{\out} \subseteq (\Sigma_{\inn}^{\rho_{\inn} \cdot d})^n$ be an outer code with rate $\rho_{\out} = 1 - \eps/4$ and distance (say) $\delta_{\out} = \eps^3/2^{15}$. We note that this code can also be decoded from a radius $\delta_{\out}^{\dec} = \frac{\eps^3}{2^{17}}$ \cite{Zemor01, GRS23} in linear time.
	\end{itemize}
%

The choice of $\lambda$ above is sufficient to ensure $\lambda \leq \frac{\delta_{\out}^{\dec}}{12k^k}\cdot \eps$, and so we can use \cref{thm:sos_technical} to conclude that there is a deterministic algorithm, that takes as input $g\in (\Sigma_{\inn}^d)^R$, runs in time $n^{O\inparen{ \frac{d \cdot k^{3k}|\Sigma_{\inn}|^{3kd}}{(\delta_{\out}^{\dec})^2\cdot \eps^2}}}$, and outputs the list $\calL\inparen{g,\frac{k-1}{k}\inparen{1-\rho-\eps}}$. With the above choice of parameters, the exponent of $n$ in the runtime becomes $2^{O(\frac{k^{4k}}{\eps^{10}})}$.
%\begin{align*}
%	\frac{d \cdot k^{3k}|\Sigma_{\inn}|^{3kd}}{(\delta_{\out}^{\dec})^2\cdot \eps^2} &= \frac{k^{2k}}{6} \cdot k^{3k} \cdot \frac{1}{\eps^2} \frac{1024}{\eps^4} 2^{O(kd(k+\frac{1}{\eps}))} \\
%	&= \frac{k^{2k}}{6} \cdot k^{3k} \cdot \frac{1}{\eps^2} \frac{1024}{\eps^4} 2^{O(\frac{k^{3k}}{\eps^7})} \\
%	 &= 2^{O(\frac{k^{4k}}{\eps^8})}
%\end{align*}
%
%Given the above parameters, we have 
%$\rho(\AELC) ~\geq~ \rho_{\out} \cdot \rho_{\inn} ~=~ (1-\eps/4) \cdot (\rho + \eps/4) ~\geq~ \rho$.
%%
%Since $\lambda \leq \eps \cdot \delta_{\out}/(6k^k)$ and $\calC_{\inn}$ is $(1-\rho, k, \eps/2)$ average-radius list decodable with erasures, we can use \cref{thm:main_technical_avg} to conclude that $\AELC$ is $(1-\rho, k, \eps)$ average-radius list decodable (with erasures) which yields the second condition. Finally, we note that the alphabet size of the code $\AELC$ is $q^d = \exp\inparen{O((k + 1/\eps) \cdot (k^{2k}/\eps^6))} = \exp\inparen{k^{3k}/\eps^7}$, which proves the claim. 
%
%We only need to argue that the codes constructed there can be decoded efficiently.
\end{proof}
