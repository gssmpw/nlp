\pdfoutput=1

\documentclass[11pt]{article}

\usepackage[final]{acl}

\usepackage{times}
\usepackage{latexsym}

\usepackage[T1]{fontenc}

\usepackage[utf8]{inputenc}

\usepackage{microtype}

\usepackage{inconsolata}

\usepackage{graphicx}
\usepackage{enumitem}
\usepackage{listings}       
\usepackage{svg}
\usepackage{makecell}
\usepackage{tabularx}
\usepackage{longtable}
\usepackage{xcolor}         
\usepackage{pifont}
\usepackage{caption}        
\usepackage{tcolorbox}      
\usepackage{booktabs} 
\usepackage{multirow} 
\usepackage{colortbl}
\usepackage{amsmath}
\usepackage{float}
\usepackage{balance}
\definecolor{myyellow}{RGB}{255,171,64}
\definecolor{mylightyellow}{RGB}{252,229,205}

\lstdefinestyle{mystyle}{
    backgroundcolor=\color{mylightyellow},
    basicstyle=\ttfamily\footnotesize,
    breaklines=true,                 
    captionpos=b,                    
    numberstyle=\tiny\color{gray},
    stepnumber=1,
    numbersep=5pt,
    tabsize=2,
    frame=single,
    rulecolor=\color{myyellow},
    keywordstyle=\color{blue},
    stringstyle=\color{black},
    showstringspaces=false           %
}

\definecolor{myblue}{rgb}{0.2, 0.2, 1.0}

\lstset{style=mystyle}

\tcbset{colback=gray!5, colframe=gray!80, fonttitle=\bfseries}

\usepackage[english, status=draft, margin=false, inline=true]{fixme}
\fxusetheme{color}
\FXRegisterAuthor{jf}{ajf}{\color{red}JF}
\FXRegisterAuthor{gs}{ags}{\color{blue}GS}

\lstdefinelanguage{JSON}{
    basicstyle=\ttfamily\footnotesize,
    keywordstyle=\bfseries\color{blue},
    stringstyle=\color{brown},
    commentstyle=\color{gray},
    showstringspaces=false,
    morestring=[b]",
    moredelim=[s][\color{blue}]{:}{\ },
    moredelim=[l][\color{magenta}]{,}
}

\title{Problem Solved? Information Extraction Design Space for Layout-Rich Documents using LLMs}

% \author{
% Gaye Colakoglu$^{1, 2}$, 
% Gürkan Solmaz$^{2}$,
% Jonathan Fürst$^{1}$
% \\
% $^{1}$Zurich University of Applied Sciences, Switzerland \\
% $^{2}$NEC Laboratories Europe, Heidelberg, Germany\\
% colgay01@students.zhaw.ch, gurkan.solmaz@neclab.eu, jonathan.fuerst@zhaw.ch\\
% }

\author{
Gaye Colakoglu$^{1,2}$ \quad Gürkan Solmaz$^{2}$ \quad Jonathan Fürst$^{1}$ \\
$^{1}$Zurich University of Applied Sciences, Switzerland \\
$^{2}$NEC Laboratories Europe, Heidelberg, Germany \\
\texttt{colgay01@students.zhaw.ch, gurkan.solmaz@neclab.eu, jonathan.fuerst@zhaw.ch}
}

\newcommand{\fix}{\marginpar{FIX}}
\newcommand{\new}{\marginpar{NEW}}

\newcommand{\rmspace}{\vspace{-2ex}}

\begin{document}
\maketitle
\begin{abstract}

This paper defines and explores the design space for information extraction (IE) from \textit{layout-rich documents} using large language models (LLMs).
The three core challenges of layout-aware IE with LLMs are \textit{1) data structuring, 2) model engagement, and 3) output refinement}. Our study delves into the sub-problems within these core challenges, such as input representation, chunking, prompting, and selection of LLMs and multimodal models. It examines the outcomes of different design choices through a new \textit{layout-aware IE test suite}, benchmarking against the state-of-art (SoA) model LayoutLMv3. The results show that the configuration from one-factor-at-a-time (OFAT) trial achieves near-optimal results with $14.1$ points F1-score gain from the baseline model, while full factorial exploration yields only a slightly higher $15.1$ points gain at \(\sim36\times \)  greater token usage. We demonstrate that well-configured general-purpose LLMs can match the performance of specialized models, providing a cost-effective alternative. Our test-suite is freely available at \url{https://github.com/gayecolakoglu/LayIE-LLM}.

\end{abstract}

\section{Introduction}

% State of the world (robots for creative activites)
The term ``robot,'' originally signifying `forced labor,' has long been associated with labor and work. Robots have demonstrated their utility in various automated productive and social contexts, where the primary goals are improving productivity, safety, and fostering social interactions with humans~\cite{simoes2022designing, weidemann2021role, honig2018understanding}. However, an increasing number of cases feature using of robots in creative settings. Unlike productive contexts, where the focus is on efficiency and task completion~\cite{arents2022smart}, or social contexts, where communication and trust are prioritized~\cite{nam2020trust, saunderson2019robots}, creative environments prioritize artistic innovation and expression~\cite{hsueh2024counts}. This shift fundamentally alters the dynamics of human-robot interaction, redefining the roles and expectations for both humans and robots.

For instance, robots’ social behaviors are leveraged to support the generation and expression of creative ideas~\cite{hu2021exploring, sandoval2022human, alves2020creativity}, and programmable robotic movements and trajectories are employed to inspire artistic activities such as sketching~\cite{lin2020your}. These studies often engage participants from creative fields who possess limited prior experience with robotics, and are typically conducted in short-term, experimental settings. Consequently, the findings from these studies remain constrained since much can be learned from professional practitioners' experiences to inform system design such as digital fabrication~\cite{hirsch2023nothing}. There is a notable gap in research examining the long-term, active, and practical experience of integrating robotic systems into the creative processes. As a result, the deeper insights into how robots facilitate and shape creative processes, beyond simply augmenting human creativity, remain underexplored. In this study, we aim to better understand the impacts of robots on creative processes and outcomes.

As early as Leonardo da Vinci's 16th century ``Automaton,'' artists have explored the creative affordances of robotic systems~\cite{shanken2002cybernetics, pagliarini2009development, jeon2017robotic}. The artistic creation process typically encompasses various stages, including the exploration of materials and techniques, ongoing experimentation and iteration, and the continual refinement of the artists' insights into their creative subjects~\cite{lewis2023art, sturdee2022state}. Therefore, investigating the artistic process involving robots offers an opportunity to gain deeper insights into robots' creative potential. Robotic art, in particular, provides a compelling case for this exploration.

We define robotic art as artworks that utilize robotic or automated machines to create artistic experiences and tangible artifacts. One example is robotic installation art, in which robots are programmed to follow specific rules that embody the artist’s expression (\autoref{fig:teaser} (a)). Another example is responsive art, in which robots react to their environment, with behaviors that change over time or in response to spectators (\autoref{fig:teaser} (b)). Additionally, there are robotic creators, which possess a degree of agency, allowing them to collaborate with human artists and produce works that extend beyond mere replication of human-created art (\autoref{fig:teaser} (c) and (d)). As such, robotic art becomes a rich case for exploring human-machine interactions in creative contexts. Gaining a deeper understanding of how robots facilitate artistic expression can provide insights for designing computing systems to support creative activities~\cite{gomez2021robot}.

% Therefore, we did...
We draw on semi-structured, in-depth interviews with renowned professional robotic artists to investigate the use of robots in artistic practice. Specifically, our goal is to understand how artistic exploration of robotic systems challenges conventional assumptions about the functions of robots, such as their roles in automating repetitive tasks or serving human needs. We also explore the implications of robots in the artistic process and examine how creativity may emerge within robotic art. To address these interrelated inquiries, our study focuses on the practice of robotic art, posing the research question: \textit{How do robotic artists utilize robots in their artistic practice?} We approach this inquiry through the perspectives and experiences of robotic artists, who creatively design, modify, and repurpose robotic systems for artistic expression and exploration.

% The key findings are...
Our findings highlight the social, material, and temporal dimensions of artists' practices that shape their creativity and artistic outcomes. The creation of robotic art is largely a social process, as artists receive both explicit and implicit feedback through the audience's reactions and reception of their work. Simultaneously, the embodiment and malfunctions inherent to robotic systems drive artistic experimentation. The temporal processes of creation and exhibition, beyond just the final product, further enhance the creative value. Our empirical analysis presents how creativity emerges through the interplay of social, material, and temporal interactions among artists, robots, audiences, and the environment.

% The contributions of this work are...
We make two main contributions to HCI in this study. 
First, we elucidate the interactive mechanisms among key actors---human creators, machines, audiences, and environments---within the practice of robotic art, a topic that remains underexplored in HCI. Our findings reveal the significance of sociality (e.g., interactions between artists and audiences), materiality (e.g., the embodiment and malfunctions of robots), and temporality (e.g., the processes of creation and exhibition) in shaping creative values. We propose that these three facets are central to the creative process and facilitate the emergence of creativity in robotic art.
Second, drawing from the findings, we offer implications for \textit{socially informed}, \textit{material-attentive}, and \textit{process-oriented} creation with computing systems. We suggest leveraging these three aspects to enhance creativity and the creative experience. Specifically, we discuss the value of incorporating implicit audience feedback, designing with technical malfunctions, and focusing on the post-creation process to foster alternative creative experiences with machines~\cite{alter2010designing, juarez2022glitch}.



\section{Design Space of IE from Layout-rich Documents with LLMs}
\label{sec:design_space_IE}

\paragraph{Task Definition.}IE from LRDs involves identifying and extracting information from documents where textual content is intertwined with complex visual layouts and mapping them into structured information instances such that

\begin{equation}
\text{IE}: (D, S) \rightarrow E
\end{equation}
,where
\begin{itemize}[leftmargin=*,noitemsep,topsep=0pt]
    \item \textbf{D} represents the set of LRDs, each with content and layout information.
    \item \textbf{S} is the target schema that defines the set of slots to be filled.
    Each slot is defined by an attribute (key) $a_i$ and its corresponding data type (domain) $T_i$, such that $S = \{(a_1, T_1), (a_2, T_2), ..., (a_k, T_k)\}$.
    
    \item Finally, \textbf{E} represents the set of extracted information instances, where each instance is a set of slot-value pairs derived from a document in \textbf{D}, leveraging both content and layout to determine the correct values for the slots in \textbf{S}. Each value in an instance must conform to the data type $T$, specified in the schema for that attribute.
    
\end{itemize}

\subsection{Using LLMs for Information Extraction}
IE systems that utilize LLMs have to tackle the three main challenge areas that we consider as part of the Design Space: Data Structuring, Model Engagement, and Output Refinement. Each stage, in their respective order, plays an important role for having an IE system with satisfactory accuracy and robustness.

\noindent \textbf{Data Structuring.} For multimodal LLMs, LRDs can be directly given as input. On the other hand, for purely text-based LLM, the input documents must be transformed into textual representations. This involves converting documents into machine-readable formats using OCR systems to extract features such as text, bounding boxes, and visual elements~\cite{mieskes-schmunk-2019-ocr, smith2007overview}. Alternatively, a formatting language such as Markdown can be employed to represent the document's layout, allowing the LLM to understand the structural context of the text better.
The impact of OCR quality on IE performance has been documented~\cite{bhadauria-etal-2024-effects}, and structured formats tend to yield better results~\cite{bai-etal-2024-schema}.
To process larger documents efficiently, they are often divided into smaller, manageable chunks based on page boundaries, sections, or semantic units~\cite{liu2024lost}.
\colorbox{blue!10}{\parbox{\columnwidth}{\emph{Markdown as an input format compared to raw OCR outputs remains underexplored, representing a potential research gap in IE system development.}}}

\noindent \textbf{Model Engagement.} Once preprocessed, the document is fed to an LLM for IE.  Ensuring alignment between the extracted text and layout information is crucial for accurate representation~\cite{xu2020layoutlmv2, appalaraju2021docformer}. Prompt-driven extraction leverages general-purpose models, using tailored prompts to guide the extraction process~\cite{brown2020language, radford2021learning, zhou2022least}. 
As such, the model needs to be instructed to extract information, and usually, the schema is provided. This step can involve more advanced IT and ICL techniques (few-shot, CoT).
\colorbox{blue!10}{\parbox{\columnwidth}{\emph{The influence of prompting techniques in interaction with various stages of the IE pipeline to enhance performance and robustness remains a research gap that requires further investigation.}}}

\noindent \textbf{Output Refinement.} After inference, the extracted information undergoes post-processing to ensure accuracy and conformance to Schema S. This step involves refining and validating the outputs generated by the LLM through tasks such as mapping extracted entities E to their original document positions, merging overlapping or fragmented predictions, and resolving ambiguities in the results~\cite{xu2020layoutlm}. 
Refining entity extraction through post-processing has been explored in various studies~\cite{j-wang-etal-2022-globalpointer, tamayo-etal-2022-nlp}. Rule-based entity alignment has notable accuracy improvements~\cite{luo2024asgeaexploitinglogicrules}.
\colorbox{blue!10}{\parbox{\columnwidth}{\emph{There exists no analysis of post-processing techniques tailored to LRDs in conjunction with LLMs.
}}}









\section{Test-Suite for IE from Layout-rich Documents with LLMs}
\label{sec:test_suite_IE_layout}

We implement a comprehensive test suite to assess IE tasks from LRDs. The pipeline, depicted in Figure~\ref{fig:pipeline}, systematically transforms raw input into structured output across multiple stages for evaluating the efficacy of design decisions.

\begin{figure*}[h!]
    \centering
    \includegraphics[width=\linewidth]{figs/pipeline_v2.pdf}
    \vspace{-40mm}
    \caption{Overview of our two-stage training pipeline {\ours}.}
    \label{fig:pipeline}
\end{figure*}


\subsection{Data Structuring}
We convert raw documents into machine-readable formats by extracting text and layout features. The process involves two conversions: %
1) Extracting textual content and layout information using OCR data, and 2) creating a markdown representation of PDFs.

\noindent \textbf{Chunking.}
We employ three chunk sizes: (1) max: 4096 tokens, (2) medium: 2048 tokens, and (3) small: 1024 tokens. Documents are segmented into $N$ chunks based on document length and chunk size, preparing them for the prompting phase. Each chunk is formed by accumulating whole words until the token limit. This ensures intact word boundaries and maintaining a sequential, non-overlapping structure. Layout information is preserved by associating each text segment with
normalized and quantized spatial coordinates, retaining the structural context of the document.
\subsection{Model Engagement}
Model engagement consists of constructing input to the LLM comprising at least three components: (1) a task instruction outlining the IE task, (2) the target schema $S$, and (3) the document chunk.
We adhere to best practices from NLP for prompt structure and IE task instruction. The schema $S$ is implemented as a dictionary of key-value pairs, where values specify the format of the corresponding attribute using regex expressions in Listing~\ref{lst:schema}.

\begin{lstlisting}[language=Python, caption={Schema Definition}, label={lst:schema}]
"file_date": r"\d{4}-\d{2}-\d{2}",
"foreign_princ_name": r"[\w\s.,'&-]+",
"registrant_name": r"[\w\s.,'&-]+",
"registration_num": r"\d+",
"signer_name": r"[\w\s'.-]+",
"signer_title": r"[\w\s.,'&-]+",
\end{lstlisting}
We also implement two ICL strategies: (1) Few-Shot and (2) Chain-of-Thought (CoT)
(see Appendix~\ref{appendix:prompt_generation_details} for more details).
For $N$ prompts, the LLM performs \emph{N completions}, with one completion per prompt. The outputs are collected and stored as raw predictions, ready for Output Refinement.

\subsection{Output Refinement}
\label{subsec:post_processing}
We refine the raw outputs from the LLM to ensure alignment with the target schema, addressing challenges related to prediction variability, schema definition differences, and data formatting inconsistencies (e.g., varying date formats). We implement three techniques inspired by related work in data integration~\cite{dong2013big}: \textit{Decoding}, \textit{Schema Mapping}, and \textit{Data Cleaning}. This process results in three sets of predictions: initial predictions, mapped predictions, and cleaned predictions.

\noindent \textbf{Decoding.} The decoding step parses each LLM completion as a JSON object, discarding any that fail to parse. The process then consolidates predictions for each document by reconciling outputs generated across individual pages and chunks. With $N$ completions for $N$ prompts, corresponding to $N$ chunks, the model generates multiple predictions for a single document. Reconciliation ensures a unified document-level output by deduplicating nested predictions and aggregating unique values. If multiple unique values exist for a single entity, they are stored together to preserve variability. The outcome of this step is referred to as the \emph{initial predictions}.

\noindent \textbf{Schema Mapping.} LLMs are expected to return only keys $\{a_1, a_2, ..., a_k\}$ specified in the target schema $S$. 
However, they may occasionally fail to return the keys as expected. E.g., `file date'' is returned instead of ``file\_date''. Such LLM ``overcorrection'' can hinder strict schema conformance.
As a countermeasure, we implement a post-processing step that maps the predicted keys to align with the target schema.
Our mapping step integrates multiple weak-supervision signals, such as \textit{exact matching, partial matching,} and \textit{synonym-based logic}, inspired by the recent techniques for ontology alignment~\cite{furst2023versamatch}.
The outcome of this step is the \emph{mapped predictions}, where entity keys $\{a_1, a_2, ..., a_k\}$ are standardized and fully aligned with the schema $S$.

\noindent \textbf{Data Cleaning.} A common issue concerns the format of the values $T_k$ of predicted key-value pairs $\{(a_1, T_1), (a_2, T_2), ..., (a_k, T_k)\}$. We must standardize formats, such as dates and names, to align with the target schema $S$.
One source of error is LLM hallucinations~\cite{ji2023survey, huang2023survey, xu2024hallucination}, while another problem is that information is often not aligned to a common format inside the source data. For instance, two documents might use two different formats for dates (``April 1992'' vs ``1992-04-01''). Additional issues include capitalization, redundant whitespace, or special characters. We utilize the regex-defined data types in our schema to automatically apply data cleaning functions. 
The outcome of this step is the \emph{cleaned predictions}, representing the final fully normalized outputs.

\noindent \textbf{Evaluation Techniques}.
\label{subsec:evaluation_techniques} 
Evaluating IE for LRDs requires comparing the extracted data against an annotated test dataset. We implement three metrics for this evaluation: \textit{exact match, substring match}, and \textit{fuzzy match}.








\begin{itemize}[leftmargin=*,noitemsep,topsep=0pt]
\item \textbf{Exact Match} searches for perfect alignment between predicted and ground truth values. A match is valid only when the values are identical. This strict approach is ideal for extracting specific, unambiguous entities like dates or numerical identifiers.

\item \textbf{Substring Match} checks whether ground truth values are fully contained within the predicted values as complete substrings, without being split or partially matched. It ensures all ground truth values appear in their entirety within predictions, making it effective for tasks such as extracting full names or addresses, where additional contextual details (e.g., titles like Mr. and Mrs.) may be included in the predictions without making the extraction incorrect.

\item  \textbf{Fuzzy Match} uses similarity metrics for approximate matches. A match is valid if the highest similarity ratio exceeds a predefined threshold (default: 0.8). This method is well-suited for scenarios with minor variations caused by OCR errors or formatting discrepancies.
\end{itemize}





















\section{Experimental Evaluation}
\label{sec:experimental_evaluation}

\subsection{Experimental Design}
\paragraph{Methodology.} The goal of our experimental setup is to study how different parameters in the pipeline, shown in Figure~\ref{fig:pipeline}, affect the overall performance of IE from LRDs using LLMs. To investigate the design dimensions, we start with a baseline configuration and and systematically alter factors at a single dimension at a time, following a one-factor-at-a-time (OFAT) methodology. This approach allows us to isolate and understand the impact of each parameter change on the IE performance.

Our intuition is that aggregating the knowledge gained for each dimension independently, we can achieve a deeper understanding of the design space and possibly identify an effective overall configuration for IE, without the need for a comprehensive factorial exploration. However, we also validate the findings by comparing the results of the OFAT method with those obtained from a brute-force approach, which is based on conducting 432 experiments.

\paragraph{Dataset and LLMs.}
We utilize the Visually Rich Document Understanding (VRDU) dataset~\cite{wang2023vrdu}, which includes two benchmarks. Each benchmark includes training samples of 10, 50, 100, and 200 documents with high-quality OCR for assessing data efficiency, as well as generalization tasks: \textit{Single Template Learning (STL), Unseen Template Learning (UTL)}, and \textit{Mixed Template Learning (MTL)}. For further details on how this dataset is tailored for our experiments and diverse models, please refer to Appendix~\ref{appendix:dataset_details}.

We evaluate GPT-3.5, GPT-4o, and LLaMA3-70B for text-only structured data extraction from LRDs.
Additionally, we compare their results with GPT-4 Vision and LayoutLMv3 to assess the performance gap between multimodal LLMs and domain-specific, fine-tuned models, respectively.

\paragraph{The Baseline Configuration.}
The baseline configuration is outlined in Table~\ref{tab:all_conf},%
where the configuration is selected based on best practices such as in \cite{perot-etal-2024-lmdx}
for the following reasons: (1) \textbf{OCR} reflects real-world scenarios for digitized LRDs. (2) \textbf{Medium chunk size} balances efficiency and context preservation, addressing token limits in LLMs. (3) \textbf{Few-shot prompting} combines pre-trained knowledge with minimal task-specific guidance. (4) Using \textbf{zero examples} provides a clear benchmark for assessing the model's raw performance. (5) \textbf{Initial predictions} are retained to evaluate models' raw output without modifications, ensuring a direct assessment of their capabilities. (6) Finally, \textbf{exact match} provides a stringent measure of correctness, offering a reliable baseline for comparison across configurations.

\begin{table}[h!]
\centering
\footnotesize
\caption{Overall configuration parameters. Baseline configuration is highlighted with \colorbox{blue!10}{light blue}.}
\label{tab:all_conf}
\begin{tabular}{p{3.2cm}p{3.6cm}}
\toprule  
\textbf{Parameter}       & \textbf{Values}                         \\ \midrule
Input Type              & \colorbox{blue!10}{OCR}, Markdown          \\ 
Chunk Size Category     & Small, \colorbox{blue!10}{Medium}, Max     \\ 
Prompt Type             & \colorbox{blue!10}{Few-Shot}, CoT         \\ 
Example Number          & \colorbox{blue!10}{0}, 1, 3, 5             \\ 
Post-processing Strategy & \shortstack[l]{\colorbox{blue!10}{Initial}, Mapped, Cleaned} \\
Evaluation Technique    & \shortstack[l]{\colorbox{blue!10}{Exact}, Substring, Fuzzy} \\
 \bottomrule
\end{tabular}
\end{table}


\subsection{The Input Dimension}

We substitute OCR input with Markdown and as outcomes in both STL and UTL scenarios. The differences in performance between OCR and Markdown are model- and context-dependent, exhibiting no consistent trend favoring one input type over the other, as shown in Table~\ref{tab:input_type}.

\begin{table}[h]
    \centering %
    \scriptsize %
    \caption{Performance results for different LLMs across \textbf{STL} and \textbf{UTL} levels with different input types. Baseline configuration in \colorbox{blue!10}{light blue}.} 
    \label{tab:input_type}  
    \setlength{\tabcolsep}{4pt}
    \begin{tabular}{l l l l}
        \toprule  
        \multirow{2}{*}{\textbf{Models}} & \multirow{2}{*}{\textbf{Level}} & \multicolumn{2}{c}{\textbf{Exact Match (F1)}} \\  
        \cmidrule(lr){3-4}
        & & \textbf{OCR} & \textbf{Markdown} \\
        \midrule
        \multirow{2}{*}{GPT-3.5} & STL  & \colorbox{blue!10}{0.650} & 0.647 \textbf{{\color{gray}\fontsize{5.5}{8.4}\selectfont(-0.003) }} \\  
                                 & UTL  & \colorbox{blue!10}{0.645} & 0.657 \textbf{{\color{gray}\fontsize{5.5}{8.4}\selectfont(+0.012) }} \\ 
        \cmidrule(lr){2-4}
        \multirow{2}{*}{GPT-4o} & STL  & \colorbox{blue!10}{0.670} & 0.633 \textbf{{\color{gray}\fontsize{5.5}{8.4}\selectfont(-0.037) }} \\  
                                 & UTL  & \colorbox{blue!10}{0.659} & 0.633 \textbf{{\color{gray}\fontsize{5.5}{8.4}\selectfont(-0.026) }} \\  
        \cmidrule(lr){2-4}
        \multirow{2}{*}{LLaMA3} & STL  & \colorbox{blue!10}{0.640} & 0.657 \textbf{{\color{gray}\fontsize{5.5}{8.4}\selectfont(+0.017) }} \\ 
                                    & UTL  & \colorbox{blue!10}{0.640} & 0.662 \textbf{{\color{gray}\fontsize{5.5}{8.4}\selectfont(+0.022) }} \\ 
    \bottomrule  
    Avg \textbf{{\color{gray}\fontsize{5.5}{8.4}\selectfont(\textpm stdev.) }} & & \colorbox{blue!10}{0.650 \textbf{{\color{gray}\fontsize{5.5}{8.4}\selectfont(\textpm 0.011) }}} & 0.648 \textbf{{\color{gray}\fontsize{5.5}{8.4}\selectfont(\textpm 0.012) }}
    \end{tabular}
\end{table}

OCR input serves as a stable baseline for IE tasks, delivering consistent performance across models. GPT-4o has noticeable performance drops with Markdown input, indicating its reliance on OCR for optimal results. In contrast, Markdown marginally improves performance for LLaMA3-70B at both STL and UTL scenarios, suggesting its potential benefits from the additional structure or semantic cues. GPT-3.5 demonstrates robustness to changes in input type, with only slight fluctuations in performance. On average, OCR marginally outperforms Markdown (0.650 vs. 0.648), but the differences are minor, with standard deviations indicating similar stability. 

\subsection{The Chunk Dimension}

To evaluate the impact of chunk size, we varied it from medium to max and small while keeping all other parameters constant.
Table~\ref{tab:chunk_size} demonstrates how chunk size affects performance across STL and UTL levels.
\begin{table}[h]
    \small
    \caption{Performance results for different LLMs across \textbf{STL} and \textbf{UTL} levels with different chunk size categories. Baseline configuration in \colorbox{blue!10}{light blue}.} 
    \label{tab:chunk_size}
    \resizebox{\linewidth}{!}{
    \setlength{\tabcolsep}{4pt}
    \begin{tabular}{l c c c c}
    \toprule
    \multirow{2}{*}{\textbf{Models}} & \multirow{2}{*}{\textbf{Level}} & \multicolumn{3}{c}{\textbf{Exact Match (F1)}} \\
    \cmidrule(lr){3-5}
    & & \textbf{Small\textbf{{\color{gray}\fontsize{7}{8.4}\selectfont($\leq 1024$)}}} & \textbf{Medium\textbf{{\color{gray}\fontsize{7}{8.4}\selectfont($\leq 2048$)}}} & \textbf{Max\textbf{{\color{gray}\fontsize{7}{8.4}\selectfont($\leq 4096$)}}} \\
    \midrule
    \multirow{2}{*}{GPT-3.5} & STL & 0.562\textbf{{\color{gray}\fontsize{5.5}{8.4}\selectfont(-0.088) }} & \colorbox{blue!10}{0.650} & 0.645\textbf{{\color{gray}\fontsize{5.5}{8.4}\selectfont(-0.005) }} \\
             & UTL & 0.561\textbf{{\color{gray}\fontsize{5.5}{8.4}\selectfont(-0.084) }} & \colorbox{blue!10}{0.645} & 0.644\textbf{{\color{gray}\fontsize{5.5}{8.4}\selectfont(-0.001) }} \\
    \cmidrule(lr){2-5}
    \multirow{2}{*}{GPT-4o} & STL & 0.602\textbf{{\color{gray}\fontsize{5.5}{8.4}\selectfont(-0.068) }} & \colorbox{blue!10}{0.670} & 0.674\textbf{{\color{gray}\fontsize{5.5}{8.4}\selectfont(+0.004) }} \\
             & UTL & 0.600\textbf{{\color{gray}\fontsize{5.5}{8.4}\selectfont(-0.059) }} & \colorbox{blue!10}{0.659} & 0.657\textbf{{\color{gray}\fontsize{5.5}{8.4}\selectfont(-0.002) }} \\
    \cmidrule(lr){2-5}
    \multirow{2}{*}{LLaMA3} & STL & 0.615\textbf{{\color{gray}\fontsize{5.5}{8.4}\selectfont(-0.025) }} & \colorbox{blue!10}{0.640} & 0.647\textbf{{\color{gray}\fontsize{5.5}{8.4}\selectfont(+0.007) }} \\
               & UTL & 0.608\textbf{{\color{gray}\fontsize{5.5}{8.4}\selectfont(-0.032) }} & \colorbox{blue!10}{0.640} & 0.644\textbf{{\color{gray}\fontsize{5.5}{8.4}\selectfont(+0.004) }} \\
    \bottomrule
    Avg\textbf{{\color{gray}\fontsize{5.5}{8.4}\selectfont(\textpm stdev.) }} & & 0.591\textbf{{\color{gray}\fontsize{5.5}{8.4}\selectfont(\textpm 0.023) }} & \colorbox{blue!10}{0.650\textbf{{\color{gray}\fontsize{5.5}{8.4}\selectfont(\textpm 0.011) }}} & 0.651\textbf{{\color{gray}\fontsize{5.5}{8.4}\selectfont(\textpm 0.011) }}
    \end{tabular}}
\end{table}

Medium and max chunk sizes provide the most consistent and stable results across models, with an average F1 score of 0.650 (±0.011) and 0.651 (±0.011), respectively.
Due to insufficient context, small chunk sizes result in significant performance drops, particularly for GPT-3.5 and GPT-4o.
\emph{These findings suggest that max chunk size is optimal, but medium can be a good option for LLMs with limited context lengths.}.

\subsection{The Prompt Dimension}

Table~\ref{tab:prompt_type_with_examples} presents the impact of prompt type and the number of examples on model performance at STL and UTL levels. Surprisingly, in-context demonstrations do not enhance performance for either few-shot or CoT experiments. For both experiments, the setting with zero examples achieves the highest average performance: few-shot 0.650 (±0.011) and CoT 0.649 (±0.008). Performance consistently declines as the number of examples increase, likely due to noise that impairs generalization. \textit{Overall, there is no significant difference between few-shot and CoT.}

\begin{table}[h]
    \centering %
    \scriptsize %
    \caption{Different LLMs across \textbf{STL} and \textbf{UTL} levels with different prompt types and example numbers. Baseline Configuration is highlighted in \colorbox{blue!10}{light blue}.}
\label{tab:prompt_type_with_examples}  
    \setlength{\tabcolsep}{1pt}
    \begin{tabular}{l c c c c c}
    \toprule  
    \multirow{2}{*}{\textbf{Models}} & \multirow{2}{*}{\textbf{Level}} & 
    \multicolumn{4}{c}{\textbf{Exact Match (F1)}} \\  
    \cmidrule(lr){3-6}
    & & \textbf{0} & \textbf{1} & \textbf{3} & \textbf{5} \\  
    \midrule
    \rowcolor{black!10!} \multicolumn{6}{c}{\textbf{\textit{few-shot}}} \\
    \multirow{2}{*}{GPT-3.5} & STL & \colorbox{blue!10}{0.650} & 0.586\textbf{{\color{gray}\fontsize{5.5}{8.4}\selectfont(-0.064) }} & 0.593\textbf{{\color{gray}\fontsize{5.5}{8.4}\selectfont(-0.057) }} & 0.548\textbf{{\color{gray}\fontsize{5.5}{8.4}\selectfont(-0.102) }} \\  
                              & UTL & \colorbox{blue!10}{0.645} & 0.566\textbf{{\color{gray}\fontsize{5.5}{8.4}\selectfont(-0.079) }} & 0.564\textbf{{\color{gray}\fontsize{5.5}{8.4}\selectfont(-0.081) }} & 0.541\textbf{{\color{gray}\fontsize{5.5}{8.4}\selectfont(-0.104) }} \\  
    \cmidrule(lr){2-6}
    \multirow{2}{*}{GPT-4o} & STL & \colorbox{blue!10}{0.670} & 0.608\textbf{{\color{gray}\fontsize{5.5}{8.4}\selectfont(-0.062) }} & 0.602\textbf{{\color{gray}\fontsize{5.5}{8.4}\selectfont(-0.068) }} & 0.595\textbf{{\color{gray}\fontsize{5.5}{8.4}\selectfont(-0.075) }} \\  
                              & UTL & \colorbox{blue!10}{0.659} & 0.597\textbf{{\color{gray}\fontsize{5.5}{8.4}\selectfont(-0.062) }} & 0.607\textbf{{\color{gray}\fontsize{5.5}{8.4}\selectfont(-0.052) }} & 0.601\textbf{{\color{gray}\fontsize{5.5}{8.4}\selectfont(-0.058) }} \\  
    \cmidrule(lr){2-6} 
    \multirow{2}{*}{LLaMA3} & STL & \colorbox{blue!10}{0.640} & 0.599\textbf{{\color{gray}\fontsize{5.5}{8.4}\selectfont(-0.041) }} & 0.606\textbf{{\color{gray}\fontsize{5.5}{8.4}\selectfont(-0.034) }} & 0.603\textbf{{\color{gray}\fontsize{5.5}{8.4}\selectfont(-0.037) }} \\  
                                & UTL & \colorbox{blue!10}{0.640} & 0.582\textbf{{\color{gray}\fontsize{5.5}{8.4}\selectfont(-0.058) }} & 0.601\textbf{{\color{gray}\fontsize{5.5}{8.4}\selectfont(-0.039) }} & 0.597\textbf{{\color{gray}\fontsize{5.5}{8.4}\selectfont(-0.043) }} \\
    \midrule
  
    Avg\textbf{{\color{gray}\fontsize{5.5}{8.4}\selectfont(\textpm stdev.) }} &  & \colorbox{blue!10}{0.650\textbf{{\color{gray}\fontsize{5.5}{8.4}\selectfont(\textpm 0.011) }}} & 0.589\textbf{{\color{gray}\fontsize{5.5}{8.4}\selectfont(\textpm 0.014) }}& 0.595\textbf{{\color{gray}\fontsize{5.5}{8.4}\selectfont(\textpm 0.016) }}& 0.580\textbf{{\color{gray}\fontsize{5.5}{8.4}\selectfont(\textpm 0.028) }} \\

    \rowcolor{black!10!} \multicolumn{6}{c}{\textbf{\textit{CoT}}} \\
    \addlinespace[0.7mm]
    \multirow{2}{*}{GPT-3.5} & STL & 0.653\textbf{{\color{gray}\fontsize{5.5}{8.4}\selectfont(+0.003) }} & 0.602\textbf{{\color{gray}\fontsize{5.5}{8.4}\selectfont(-0.048) }} & 0.544\textbf{{\color{gray}\fontsize{5.5}{8.4}\selectfont(-0.106) }} & 0.533\textbf{{\color{gray}\fontsize{5.5}{8.4}\selectfont(-0.117) }} \\  
                              & UTL & 0.650\textbf{{\color{gray}\fontsize{5.5}{8.4}\selectfont(+0.005) }} & 0.575\textbf{{\color{gray}\fontsize{5.5}{8.4}\selectfont(-0.007) }} & 0.548\textbf{{\color{gray}\fontsize{5.5}{8.4}\selectfont(-0.097) }} & 0.516\textbf{{\color{gray}\fontsize{5.5}{8.4}\selectfont(-0.129) }} \\  
    \cmidrule(lr){2-6}
    \multirow{2}{*}{GPT-4o} & STL & 0.655\textbf{{\color{gray}\fontsize{5.5}{8.4}\selectfont(-0.015) }} & 0.615\textbf{{\color{gray}\fontsize{5.5}{8.4}\selectfont(-0.055) }} & 0.612\textbf{{\color{gray}\fontsize{5.5}{8.4}\selectfont(-0.058) }} & 0.605\textbf{{\color{gray}\fontsize{5.5}{8.4}\selectfont(-0.065) }} \\  
                              & UTL & 0.659\textbf{{\color{gray}\fontsize{5.5}{8.4}\selectfont(0) }} & 0.614\textbf{{\color{gray}\fontsize{5.5}{8.4}\selectfont(-0.045) }} & 0.611\textbf{{\color{gray}\fontsize{5.5}{8.4}\selectfont(-0.048) }} & 0.607\textbf{{\color{gray}\fontsize{5.5}{8.4}\selectfont(-0.052) }} \\  
    \cmidrule(lr){2-6} 
    \multirow{2}{*}{LLaMA3} & STL & 0.635\textbf{{\color{gray}\fontsize{5.5}{8.4}\selectfont(-0.005) }} & 0.603\textbf{{\color{gray}\fontsize{5.5}{8.4}\selectfont(-0.037) }} & 0.613\textbf{{\color{gray}\fontsize{5.5}{8.4}\selectfont(-0.027) }} & 0.610\textbf{{\color{gray}\fontsize{5.5}{8.4}\selectfont(-0.003) }} \\  
                                & UTL & 0.644\textbf{{\color{gray}\fontsize{5.5}{8.4}\selectfont(+0.004) }} & 0.586\textbf{{\color{gray}\fontsize{5.5}{8.4}\selectfont(-0.054) }} & 0.601\textbf{{\color{gray}\fontsize{5.5}{8.4}\selectfont(-0.039) }} & 0.598\textbf{{\color{gray}\fontsize{5.5}{8.4}\selectfont(-0.042) }} \\  
    \midrule

    
    Avg\textbf{{\color{gray}\fontsize{5.5}{8.4}\selectfont(\textpm stdev.) }} &  & 0.649\textbf{{\color{gray}\fontsize{5.5}{8.4}\selectfont(\textpm 0.008) }} & 0.599\textbf{{\color{gray}\fontsize{5.5}{8.4}\selectfont(\textpm 0.015) }}& 0.588\textbf{{\color{gray}\fontsize{5.5}{8.4}\selectfont(\textpm 0.032) }}& 0.578\textbf{{\color{gray}\fontsize{5.5}{8.4}\selectfont(\textpm 0.042) }}\\
        \bottomrule

    \end{tabular}
\end{table}

\subsection{Output Refinement}

We examine two output refinement strategies, Schema Mapping and Data Cleaning (see Sec.~\ref{subsec:post_processing}), to evaluate their impact shown in Table~\ref{tab:post_processing}.
\begin{table}[h]
    \small
    \caption{Different LLMs across \textbf{STL} and \textbf{UTL} levels with different post-processing strategies. Baseline configuration is highlighted in \colorbox{blue!10}{light blue}.} 
    \label{tab:post_processing}
    \resizebox{\linewidth}{!}{
    \setlength{\tabcolsep}{4pt}
    \begin{tabular}{l c c c c}
    \toprule
    \multirow{2}{*}{\textbf{Models}} & \multirow{2}{*}{\textbf{Level}} & \multicolumn{3}{c}{\textbf{Exact Match (F1)}} \\
    \cmidrule(lr){3-5}
    & & \textbf{Initial Pred.} & \textbf{Mapped Pred.} & \textbf{Cleaned Pred.} \\
    \midrule
    \multirow{2}{*}{GPT-3.5} & STL & \colorbox{blue!10}{0.650} & 0.650\textbf{{\color{gray}\fontsize{5.5}{8.4}\selectfont(0) }} & 0.737\textbf{{\color{gray}\fontsize{5.5}{8.4}\selectfont(+0.087) }} \\
             & UTL & \colorbox{blue!10}{0.645} & 0.645\textbf{{\color{gray}\fontsize{5.5}{8.4}\selectfont(0) }} & 0.733\textbf{{\color{gray}\fontsize{5.5}{8.4}\selectfont(+0.088) }} \\
    \cmidrule(lr){2-5}
    \multirow{2}{*}{GPT-4o} & STL & \colorbox{blue!10}{0.670} & 0.670\textbf{{\color{gray}\fontsize{5.5}{8.4}\selectfont(0) }} & 0.749\textbf{{\color{gray}\fontsize{5.5}{8.4}\selectfont(+0.079) }} \\
             & UTL & \colorbox{blue!10}{0.659} & 0.659\textbf{{\color{gray}\fontsize{5.5}{8.4}\selectfont(0) }} & 0.741\textbf{{\color{gray}\fontsize{5.5}{8.4}\selectfont(+0.082) }} \\
    \cmidrule(lr){2-5}
    \multirow{2}{*}{LLaMA3} & STL & \colorbox{blue!10}{0.640} & 0.640\textbf{{\color{gray}\fontsize{5.5}{8.4}\selectfont(0) }} & 0.724\textbf{{\color{gray}\fontsize{5.5}{8.4}\selectfont(+0.084) }} \\
               & UTL & \colorbox{blue!10}{0.640} & 0.640\textbf{{\color{gray}\fontsize{5.5}{8.4}\selectfont(0) }} & 0.725\textbf{{\color{gray}\fontsize{5.5}{8.4}\selectfont(+0.085) }} \\
    \bottomrule
    Avg\textbf{{\color{gray}\fontsize{5.5}{8.4}\selectfont(\textpm stdev.) }} & & \colorbox{blue!10}{0.650\textbf{{\color{gray}\fontsize{5.5}{8.4}\selectfont(\textpm 0.011) }}} & 0.650\textbf{{\color{gray}\fontsize{5.5}{8.4}\selectfont(\textpm 0.011) }} & 0.734\textbf{{\color{gray}\fontsize{5.5}{8.4}\selectfont(\textpm 0.009) }}
    \end{tabular}}
\end{table}

\paragraph{Schema Mapping} involves mapping the predicted schema keys to the target schema keys. Our results show no change in F1 scores compared to the initial predictions. This suggests that the models already effectively return the correct attributes, making the mapping step unnecessary.

\paragraph{Data Cleaning} uses the defined data types to perform automatic value cleaning, consistently achieving the highest F1 scores across all models. \emph{This underscores the need for post-processing steps for IE with LLMs to align the extracted data with the target format to handle LLM hallucinations and inconsistent source data formats (see Sec.~\ref{subsec:post_processing}).}

\subsection{Evaluation Techniques}

We explore three evaluation techniques to assess their impact on model performance (see Sec.~\ref{subsec:evaluation_techniques}).
On average, Fuzzy Match achieved the highest F1 score (0.733), outperforming Substring Match (0.676) and Exact Match, as shown in Table~\ref{tab:evaluation_techniques}. We provide a detailed error analysis of fuzzy and substring match accuracy in Appendix~\ref{appendix:eval_tech}, showing that they provide a near-perfect precision when manually checked for semantic equivalence with precision scores of 0.98 and 1.00, respectively.
\textit{This shows Fuzzy Match's ability to balance flexibility and precision. 
}
\begin{table}[h]
    \small
    \caption{ Different LLMs across \textbf{STL} and \textbf{UTL} levels with different evaluation techniques. Baseline configuration in \colorbox{blue!10}{light blue}.}
    \label{tab:evaluation_techniques}
    \resizebox{\linewidth}{!}{
    \setlength{\tabcolsep}{4pt}
    \begin{tabular}{l c c c c}
    \toprule
    \multirow{2}{*}{\textbf{Models}} & \multirow{2}{*}{\textbf{Level}} & \textbf{Exact Match} & \textbf{Substring Match} & \textbf{Fuzzy Match} \\
    & & \textbf{(F1)} & \textbf{(F1)} & \textbf{(F1)} \\
    \midrule
    \multirow{2}{*}{GPT-3.5} & STL & \colorbox{blue!10}{0.650} & 0.683\textbf{{\color{gray}\fontsize{5.5}{8.4}\selectfont(+0.033) }} & 0.730\textbf{{\color{gray}\fontsize{5.5}{8.4}\selectfont(+0.080) }} \\
             & UTL & \colorbox{blue!10}{0.645} & 0.682\textbf{{\color{gray}\fontsize{5.5}{8.4}\selectfont(+0.037) }} & 0.726\textbf{{\color{gray}\fontsize{5.5}{8.4}\selectfont(+0.081) }} \\
    \cmidrule(lr){2-5}
    \multirow{2}{*}{GPT-4o} & STL & \colorbox{blue!10}{0.670} & 0.690\textbf{{\color{gray}\fontsize{5.5}{8.4}\selectfont(+0.020) }} & 0.750\textbf{{\color{gray}\fontsize{5.5}{8.4}\selectfont(+0.080) }} \\
             & UTL & \colorbox{blue!10}{0.659} & 0.678\textbf{{\color{gray}\fontsize{5.5}{8.4}\selectfont(+0.019) }} & 0.744\textbf{{\color{gray}\fontsize{5.5}{8.4}\selectfont(+0.085) }} \\
    \cmidrule(lr){2-5}
    \multirow{2}{*}{LLaMA3} & STL & \colorbox{blue!10}{0.640} & 0.661\textbf{{\color{gray}\fontsize{5.5}{8.4}\selectfont(+0.021) }} & 0.727\textbf{{\color{gray}\fontsize{5.5}{8.4}\selectfont(+0.087) }} \\
               & UTL & \colorbox{blue!10}{0.640} & 0.662\textbf{{\color{gray}\fontsize{5.5}{8.4}\selectfont(+0.022) }} & 0.723\textbf{{\color{gray}\fontsize{5.5}{8.4}\selectfont(+0.083) }} \\
    \bottomrule
    Avg\textbf{{\color{gray}\fontsize{5.5}{8.4}\selectfont(\textpm stdev.) }} & & \colorbox{blue!10}{0.650\textbf{{\color{gray}\fontsize{5.5}{8.4}\selectfont(\textpm 0.011) }}} & 0.676\textbf{{\color{gray}\fontsize{5.5}{8.4}\selectfont(\textpm 0.011) }} & 0.733\textbf{{\color{gray}\fontsize{5.5}{8.4}\selectfont(\textpm 0.010) }}
    \end{tabular}}
\end{table}

\subsection{Putting it All Together}

In the preceding sections, we investigated the influence of various parameters on model performance along the IE extraction pipeline, analyzing one factor at a time. Drawing from the underlying 12 experiments, we identified the optimal parameter for each step and each model based on the experimental outcomes (Table~\ref{tab:OFAT_configurations_per_model}). In addition, we conducted an exhaustive full factorial exploration with 432 configurations ($2*3*2*4*3*3=432$, see Table~\ref{tab:all_conf}) to find the best parametrization per LLM (Table~\ref{tab:Brute_force_configurations_per_model}). Lastly, we find the worst configuration on a per LLM and per model basis (Table~\ref{tab:worst_configurations_per_model}).
The performance of these different configurations is depicted in Figure~\ref{fig:configuration_comparison}.
We gain several insights:

\begin{itemize}[leftmargin=*,noitemsep,topsep=0pt]

\item \emph{OFAT approximates well the Brute-Force configuration with a fraction ($\sim2.8\%$) of the required computation.} We see in Table~\ref{tab:OFAT_configurations_per_model} and Table~\ref{tab:Brute_force_configurations_per_model} that they match except for 4 parameter choices (\texttt{prompt} and \texttt{example No.} parameters for GPT-4o and LlaMa3). Likewise, their F1 scores are close to each other (Figure~\ref{fig:configuration_comparison}), with OFAT achieving 0.791 and Brute-Force achieving 0.801 overall.


\item \emph{Adapting the IE pipeline to the LLM is necessary to achieve a competitive performance.} 
Overall, the OFAT configuration improves from the baseline F1 of 0.650 to 0.791, a $22\%$ improvement. For Brute-Force, the improvement is $23\%$. In comparison, the worst configuration achieves only 0.45 on average, almost half of the best configuration.

\item \emph{Besides the need for pipeline customization, common patterns exist across LLMs.} First, output refinement and evaluation techniques boost performance across all models. Second, there is a tendency to larger context sizes. Lastly, larger models (GPT-4o, LLaMa3) benefit more from examples, while the CoT pattern generally aids IE.

\end{itemize}

\begin{table}[h]
\centering
\footnotesize
\caption{OFAT configurations on a per-model basis and corresponding performance results.}
\begin{tabular}{lcccc}
\toprule
\textbf{Parameter} & \textbf{GPT-3.5} & \textbf{GPT-4o} & \textbf{LLaMA3} \\
\midrule
Input Type & Markdown & OCR & Markdown  \\
Chunk Size & Medium & Max & Max \\
Prompt & CoT & Few-Shot & Few-Shot &\\
Example No. & 0 & 0 & 0 \\
Output Refin. & Cleaned & Cleaned & Cleaned \\
Evaluation & Fuzzy & Fuzzy & Fuzzy \\
\bottomrule
\end{tabular}
\label{tab:OFAT_configurations_per_model}
\end{table}

\begin{table}[h]
\centering
\footnotesize
\caption{Brute-Force configurations on a per-model basis and corresponding performance results.}
\begin{tabular}{lcccc}
\toprule
\textbf{Parameter} & \textbf{GPT-3.5} & \textbf{GPT-4o} & \textbf{LLaMA3} \\
\midrule
Input Type & Markdown & OCR & Markdown  \\
Chunk Size & Medium & Max & Max \\
Prompt & CoT & CoT & CoT &\\
Example No. & 0 & 5 & 5 \\
Output Refin. & Cleaned & Cleaned & Cleaned \\
Evaluation & Fuzzy & Fuzzy & Fuzzy \\
\bottomrule
\end{tabular}
\label{tab:Brute_force_configurations_per_model}
\end{table}

\begin{table}[h]
\centering
\footnotesize
\caption{Worst performance configurations on a per-model basis and corresponding performance results.}
\begin{tabular}{lcccc}
\toprule
\textbf{Parameter} & \textbf{GPT-3.5} & \textbf{GPT-4o} & \textbf{LLaMA3} \\
\midrule
Input Type & OCR & Markdown & OCR  \\
Chunk Size & Small & Small & Small \\
Prompt & Few-shot & Few-shot & Few-shot \\
Example No. & 5 & 5 & 5 \\
Output Refin. & Initial & Initial & Initial \\
Evaluation & Exact & Exact & Exact \\
\bottomrule
\end{tabular}
\label{tab:worst_configurations_per_model}
\end{table}

\begin{figure}[ht]
    \centering
    \includegraphics[width=\columnwidth]{images/configuration_comparison.pdf}
    \caption{F1 scores of different LLMs across three configurations (Baseline, OFAT, and Brute-Force). Each bar represents the mean F1 score of a model for the corresponding configuration.}
    \label{fig:configuration_comparison}
\end{figure}



Finally, we compare our purely text-based approach to IE to (1) GPT4-vision, a multimodal LLM, and (2) LayoutLMv3, a leading layout-aware model that we fine-tune to our dataset (Table~\ref{tab:model_comparison}).
Despite fine-tuning, LayoutLMv3 does not perform as well as our evaluated LLMs, even on documents that contain the same structure as the fine-tuning data (STL). We also observe that LLMs remain stable across template types. The best-performing model is GPT4-vision, a multi-modal LLM, where we directly provide an image of the PDF for IE. The model benefits from less context loss due to chunking and its ability to jointly use textual and visual features. However, when considering token usage and cost, GPT4-vision requires $\sim2$ times the tokens compared to text-only approaches and $>10$ times the cost under the current OpenAI pricing scheme (Nov. 2024). Therefore, text-only approaches constitute a good trade-off between performance and cost for practical applications.

\begin{table}[h]
    \centering %
    \footnotesize %
    \caption{Cost (per experiment) and performance of LLMs across \textbf{STL} and \textbf{UTL} levels for the OFAT configuration. The tokens for LayoutLMv3 include the fine-tuning. $^*$We run LauoutLMv3 and LlaMa3 locally, without API costs. } 
    \label{tab:model_comparison}
    \setlength{\tabcolsep}{3pt}
    \begin{tabular}{l r l l l}
    \toprule
    \textbf{Models} & \textbf{Tokens} & \textbf{API Cost} & \textbf{STL (F1)} & \textbf{UTL (F1)} \\
    \midrule
    GPT-3.5                          & 322K  & \$0.18  & 0.778  & 0.754  \\ 
    GPT-4o                           & 278K  & \$0.40  & 0.797  & 0.790  \\
    LLaMA3                           & 239K  & \$0$^*$      & 0.843  & 0.820  \\
    GPT-4-vision                     & 585K  & \$4.76   & 0.902  & 0.897  \\
    LayoutLMv3                       & 176.1M  & \$0$^*$      & 0.603  & 0.194  \\
    \bottomrule
    \end{tabular}
    \rmspace
\end{table}



    









\balance
\section{Related Works}
We review two core directions relevant to our study. Connections with other areas including asynchronous stochastic gradient descent and continual learning are discussed in Appendix~\ref{sec:connections-with-other-areas}.

\smallsection{Federated Learning and Heterogeneity Problem}
Federated learning~\cite{mcmahan2017communication} is a distributed learning paradigm that allows multiple parties to jointly train machine learning models without data sharing, preserving data privacy. Despite the potential, it faces significant challenges due to heterogeneity among participating clients, which is typically classified into two main categories: data heterogeneity and system heterogeneity. Data heterogeneity appears as clients own non-IID (independent and identically distributed) data~\cite{li2020federated,karimireddy2020scaffold,wang2020tackling,zhang2023navigating}. The difference in data distribution causes the local updates to deviate from the global objective, making the aggregation of these models drift from the global optimum and deteriorating convergence. System heterogeneity refers to variations in client device capabilities, such as computational power, network bandwidth, and availability~\cite{wang2020tackling,zhang2021parameterized,li2021fedmask,fang2022robust,alam2022fedrolex,zhang2024few}. These disparities lead to uneven progress among clients, and the overall training process is delayed by slow devices. Traditional federated learning approaches rely on synchronization for weight aggregation~\cite{mcmahan2017communication,li2020federated,reddi2020adaptive}, where the server waits for all clients selected in a round to complete and return model updates before proceeding with aggregation. This synchronization leads to inefficient resource utilization and extended training times, particularly in large-scale deployments involving hundreds or thousands of clients. Addressing the heterogeneity issues is a critical problem for improving the scalability and efficiency of federated learning systems in real-world deployment.

\smallsection{Asynchronous Federated Learning}
Much of the asynchronous federated learning literature focuses on staleness management by assigning weights for aggregating updates according to factors including delay in updates~\cite{xie2019asynchronous}, divergence from the global model~\cite{su2022asynchronous,zang2024efficient} and local losses~\cite{liu2024fedasmu}. For example, \cite{xie2019asynchronous} lets the server aggregate client updates into the global model with a weight determined by staleness. Another line of research caches client updates at the server and reuses them to calibrate global updates~\cite{gu2021fast,wang2024tackling}. For example, \cite{wang2024tackling} maintains the latest update for every client to estimate their contribution to the current aggregation and calibrate global updates. Furthermore, semi-asynchronous methods~\cite{nguyen2022federated,zang2024efficient} balance between efficiency and training stability. For example, \cite{nguyen2022federated} buffers a fixed number of client updates before aggregation. We select representative methods from each category for our comparisons. Besides, some works improve efficiency from a different perspective---through enhanced parallelization. Methods include decoupling local computation and communication~\cite{avdiukhin2021federated} and parallelizing server-client computation~\cite{zhang2023no}. In addition, asynchronous architectures have been explored in other paradigms such as vertical~\cite{zhang2024asynchronous} and clustered~\cite{liu2024casa} federated learning. While these directions complement our work, they fall outside the scope of this study.
\section{Conclusion}
We introduced \methodname, an effective training framework defending against MIAs for LLMs. The extensive experiments demonstrate its robustness in protecting privacy while maintaining strong language modeling performance across various datasets and architectures. Although our study focuses on fine-tuning due to computational constraints, \methodname can be seamlessly applied to large-scale pretraining, as done in prior selective pretraining work~\cite{lin2024not}. By categorizing tokens and treating them appropriately, \methodname opens a novel pathway for MIA defense. Future work can explore improved token selection strategies and multi-objective training approaches.
\clearpage
\section*{Limitations}
\label{sec:limitation}
This work has the following limitations:
\textbf{(i)} Although the proportion of high/medium/low resource languages in \dcad has greatly increased compare to existing multilingual datasets, a significant portion of the languages are still very low resource languages. Future work will explore to collect data for extremely low-resource languages through other modalities (e.g., images) through technologies like OCR.
% \textbf{(ii)} We evaluate the new data cleaning framework only on four classical anomaly detection algorithms; however, since the framework is algorithm-independent, it should also be effective with the latest anomaly detection algorithms~\cite{batzner2024efficientad}.
\textbf{(ii)} Due to computational limitations, we evaluate our framework and dataset only on the LLaMA-3.2-1B model. However, we expect similar evaluation results on larger models, such as 7B, 13B and even 70B models.

\balance
\bibliography{bibliography.bib}

\clearpage
\appendix %

\section{Appendix}

\subsection{Prompt Generation Details}
\label{appendix:prompt_generation_details}
\textit{Document representation} in the example section of the prompt is generated by the LLM, condensing the original OCR data while retaining information relevant to the target schema to reduce token usage and cost. This version integrates only text features. However, in the new example section, both text and layout features are preserved without modification, embedding spatial structure into text sequences to enhance model understanding. This is the only part that differs between the example and new document sections of the prompt. \textit{Task descriptions} provide clear extraction guidelines, specifying what information to retrieve, while \textit{schema representation} defines the expected JSON format to ensure consistency in extracted data. Additionally, CoT prompting includes a \textit{reasoning} component, guiding the model through logical steps to improve accuracy on complex tasks (see Figures~\ref{fig:few-shot-prompt},~\ref{fig:chain-of-thought-prompt}).

\begin{figure}[!htb]
\centering
\begin{tcolorbox}[colframe=myyellow, colback=mylightyellow, sharp corners=south, title=Prompt Template: Few-shot,
boxsep=0.5mm,
left=1mm,
right=1mm,
top=1mm,
bottom=1mm]
\footnotesize

\begin{lstlisting}[frame=none, 
                  basicstyle=\ttfamily, 
                  tabsize=4, 
                  morekeywords={Document, Task, Extraction},
                  stringstyle=\color{myblue}]
        ### Examples ###
(<Document>) 
{DOCUMENT_REPRESENTATION} 
(</Document>) 
(<Task>) 
{TASK_DESCRIPTION} 
{SCHEMA_REPRESENTATION} 
(</Task>) 
(<Extraction>) 
{EXTRACTION} 
(</Extraction>)
        ### New Documents ###
(<Document>) 
{DOCUMENT_REPRESENTATION} 
(</Document>) 
(<Task>) 
{TASK_DESCRIPTION} 
{SCHEMA_REPRESENTATION} 
(</Task>) 
(<Extraction>)
\end{lstlisting}
\end{tcolorbox}
\caption{Few-Shot Prompt Structure with 1-shot Example }
\label{fig:few-shot-prompt}
\end{figure}

\begin{figure}[ht]
\centering
\begin{tcolorbox}[colframe=myyellow, colback=mylightyellow, sharp corners=south, title=Prompt Template: CoT,
boxsep=0.5mm,
left=1mm,
right=1mm,
top=1mm,
bottom=1mm]
\footnotesize
\begin{lstlisting}[frame=none, 
                  basicstyle=\ttfamily, 
                  tabsize=4, 
                  escapeinside={  }, 
                  morekeywords={Document, Task, Extraction, Reasoning},  
                  stringstyle=\color{myblue}]
        ### Examples ###
(<Document>) 
{DOCUMENT_REPRESENTATION} 
(</Document>) 
(<Task>) 
{TASK_DESCRIPTION} 
{SCHEMA_REPRESENTATION} 
(</Task>)
(<Reasoning>) 
{REASONING} 
(</Reasoning>)
(<Extraction>) 
{EXTRACTION} 
(</Extraction>)
        ### New Documents ###
(<Document>) 
{DOCUMENT_REPRESENTATION} 
(</Document>) 
(<Task>) 
{TASK_DESCRIPTION} 
{SCHEMA_REPRESENTATION} 
(</Task>) 
(<Reasoning>) 
{REASONING} 
(</Reasoning>)
(<Extraction>)
\end{lstlisting}
\end{tcolorbox}
\caption{Chain of Thought Prompt Structure with 1-shot Example }
\label{fig:chain-of-thought-prompt}
\end{figure}

\subsection{Tailoring Dataset for our Test-Suite}
\label{appendix:dataset_details}
We used the Registration Form with six entity types for this project, as shown in Figure~\ref{fig:vrdu_registration_entities}.
\begin{figure}[!htb]
\footnotesize
\begin{lstlisting}[frame=single, basicstyle=\ttfamily, tabsize=4]
{
  "file_date": "",
  "foreign_principle_name": "",
  "registrant_name": "",
  "registration_num": "",
  "signer_name": "",
  "signer_title": ""
}
\end{lstlisting}
\caption{VRDU Registration Form Entities.} 
\label{fig:vrdu_registration_entities}
\end{figure}

The VRDU dataset includes predefined few-shot splits consisting of training, testing, and validation sets. These splits contain 10, 50, 100, and 200 training samples, each with three subsets, as shown in Figure~\ref{fig:dataset_splits}. The dataset also includes different levels (Lv1: Single, Lv2: Mixed, and Lv3: Unseen Type) and various template types (Amendment, Dissemination, and Short-Form). 


\begin{figure}[ht]
    \centering    
\includegraphics[width=0.98\linewidth]{images/dataset_splits.png}

\caption{Representation of few\_shot\_splits from the VRDU dataset.}
\label{fig:dataset_splits}
\end{figure}



For each template-level combination, we selected the first JSON file ending in 0 with 10 training samples. Since this training data will be utilized for few-shot and Chain-of-Thought (CoT) prompting, only the first five documents were chosen from the training samples of the selected JSON files. Each level type (STL, UTL) includes template types (Amendment, Dissemination, Short-Form), each with 0, 1, 3, or 5 examples. In STL, these categories use the first document for one-example prompts, the first three for three-example prompts, and all five for five-example prompts. The same structure applies to UTL, with examples specific to its categories. This ensures consistency across template-level combinations while varying the number of examples in the prompt. Figure~\ref{fig:train_examples} shows an example of few-shot and CoT examples. This process was repeated for every level, template type, and example count. The example texts were generated using a Large Language Model (LLM), which was instructed to summarize the provided OCR text for the given document while ensuring the inclusion of target schemas and entities.

\begin{figure}[ht]
\footnotesize
\begin{lstlisting}[frame=single, basicstyle=\ttfamily, tabsize=4]
few_shot_examples = {
  "STL": {
    "Amendment": {
      0: [],
      1: [
        {
          "text": "This document is an amendment to the regis...",
          "entities": {
            "file_date": "1982-10-31",
            "foreign_principle_name": "Japan Trade Center...",
            "registrant_name": "PressAid Center",
            "registration_num": "1833",
            "signer_name": "Akira Tsutsumi",
            "signer_title": "Director General"
          }
        }
      ],
      3: [
        {...},
        {...},
        {...}
      ],
      5: [
        {...},
        {...},
        {...},
        {...},
        {...},
      ]
    },
    "Dissemination": {...
\end{lstlisting}
\caption{VRDU Registration Form Entities.} 
\label{fig:train_examples}
\end{figure}

For the test files, to ensure a fair comparison, we selected the first 40 common files from the chosen JSON files within each template type at each level. This means that for Lv1 Amendment and Lv3 Amendment test files, the first common 40 files were selected as test files, and the same strategy was applied for other template types as well. Due to the mixed nature of test files in Lv2, the mixed template type was excluded from this project. 

For the models GPT-3.5, GPT-4o, and LLaMA3-70B, we created these few-shot and CoT examples. However, no few-shot examples were used for GPT-4 Vision; we used the same test dataset as for the other three models with basic instructions. For LayoutLMv3, we used the same training and test datasets as the other models but included the entire validation set (300 samples) instead of selecting only five. The training and test datasets remained unchanged across all models to ensure a fair evaluation and consistency.

\subsection{Success of Evaluation Techniques}
\label{appendix:eval_tech}

We manually reviewed the results of the baseline experiment from GPT-3.5, GPT-4o, and LLaMA3-70B to assess the success of the substring and fuzzy match metrics. The analysis focused on cases where the exact match score was 0 but substring/fuzzy was 1, highlighting predictions that failed strict matching but were successfully handled in other techniques. This created a dataset to test how well substring and fuzzy matching handle difficult cases. In total, we examined 91 key-value pairs for fuzzy match and 37 for substring match. 


\balance

\begin{figure}[ht]
    \centering
    \includegraphics[width=0.5\textwidth]{images/error_rates.png}
    \caption{Comparison of error rates in Substring vs. Fuzzy Matching }
    \label{fig:error_rates}
\end{figure}

Based on our intubation, we came up with seven different error categories see Figure~\ref{fig:error_rates}. From left to right in Error Types 5 OCR error means errors occur because of OCR; for example, handwritten names' or signatures are sometimes extracted as different names incorrectly (e.g Jim Slattery handwriting extracted as Jim Slatters) GT error represents the case where GT information is not completely true sometimes its missing some words and sometimes it has additional words for example gt value is 'Om Saudi Arabia 1' but it is Kingdom' Saudi Arabia in the document. LLM Halucination represents where LLM come up a prediction that is not exist data in the OCR for example gt value is for file date '1992-04-21' but prediction is '1992-04-24' even though OCR data and document itself does not contain any number 24. Additional info represents that prediction is correct, but there is some additional info' for example, while the signer's name is 'Daniel Manatt'' LLM predicts it as 'Daniel Manatt Todd' where Todd exists in OCR and the document has a notary name as a handwritten signature. Wrong info means gt is exist in the OCR and the document but LLM choose another date as prediction for example while file date '2016-10-08' exist in OCR LLM return 2016-10-31 as file date which belongs registration date in the document. 
\begin{table}
    \centering %
    \scriptsize %
    \caption{Performance results of substring and fuzzy evaluation techniques over exact match based on manually annotated data, considering categorized error types.} 
    \label{tab:eval_performances}
    \setlength{\tabcolsep}{4pt}
    \begin{tabular}{l c c c c}
    \toprule
    \textbf{Data} & \textbf{Evaluation} & \multirow{2}{*}{\textbf{Precision}} & \multirow{2}{*}{\textbf{Recall}} & \multirow{2}{*}{\textbf{F1}} \\
    \textbf{Points} & \textbf{Techniques} & & & \\
    \midrule
    \multirow{2}{*}{37} & Exact Match & 0.000 & 0.000 & 0.000 \\
             & Substring Match & 1.000 &  1.000 &  1.000  \\
    \cmidrule(lr){2-5}
    \multirow{2}{*}{91} & Exact Match & 0.000 & 0.000 & 0.000  \\
             & Fuzzy Match & 0.984 & 1.000 & 0.992  \\
    \bottomrule
    \end{tabular}
\end{table}

Human error is for the errors occurred physically by humans failures for example sometimes documents ar scanned partly/crooked by humans. Finally incomplete predictions represent prediction is include gt but not completely for example sometime gt is 'Japan External Trade Organization Tokyo, Japan' but prediction is missing only last part which is Tokyo, Japan.  Figure~\ref{fig:error_rates} represent all these error types and their rates in both Substring and Fuzzy match categories where exact match is 0 but they labeled as 1.

For our evaluation the success of Fuzzy and Substring we decided to label data points as 0 only in the case where wrong info is 1. For other error categories we accept them and label gt as 1 for those. Table~\ref{tab:eval_performances} represent performance metrics for fuzzy and substring calculated with gt labels explained above with the data points (exact match 0 substring/fuzzy 1) 

\end{document}
