\section{Design Space of IE from Layout-rich Documents with LLMs}
\label{sec:design_space_IE}

\paragraph{Task Definition.}IE from LRDs involves identifying and extracting information from documents where textual content is intertwined with complex visual layouts and mapping them into structured information instances such that

\begin{equation}
\text{IE}: (D, S) \rightarrow E
\end{equation}
,where
\begin{itemize}[leftmargin=*,noitemsep,topsep=0pt]
    \item \textbf{D} represents the set of LRDs, each with content and layout information.
    \item \textbf{S} is the target schema that defines the set of slots to be filled.
    Each slot is defined by an attribute (key) $a_i$ and its corresponding data type (domain) $T_i$, such that $S = \{(a_1, T_1), (a_2, T_2), ..., (a_k, T_k)\}$.
    
    \item Finally, \textbf{E} represents the set of extracted information instances, where each instance is a set of slot-value pairs derived from a document in \textbf{D}, leveraging both content and layout to determine the correct values for the slots in \textbf{S}. Each value in an instance must conform to the data type $T$, specified in the schema for that attribute.
    
\end{itemize}

\subsection{Using LLMs for Information Extraction}
IE systems that utilize LLMs have to tackle the three main challenge areas that we consider as part of the Design Space: Data Structuring, Model Engagement, and Output Refinement. Each stage, in their respective order, plays an important role for having an IE system with satisfactory accuracy and robustness.

\noindent \textbf{Data Structuring.} For multimodal LLMs, LRDs can be directly given as input. On the other hand, for purely text-based LLM, the input documents must be transformed into textual representations. This involves converting documents into machine-readable formats using OCR systems to extract features such as text, bounding boxes, and visual elements~\cite{mieskes-schmunk-2019-ocr, smith2007overview}. Alternatively, a formatting language such as Markdown can be employed to represent the document's layout, allowing the LLM to understand the structural context of the text better.
The impact of OCR quality on IE performance has been documented~\cite{bhadauria-etal-2024-effects}, and structured formats tend to yield better results~\cite{bai-etal-2024-schema}.
To process larger documents efficiently, they are often divided into smaller, manageable chunks based on page boundaries, sections, or semantic units~\cite{liu2024lost}.
\colorbox{blue!10}{\parbox{\columnwidth}{\emph{Markdown as an input format compared to raw OCR outputs remains underexplored, representing a potential research gap in IE system development.}}}

\noindent \textbf{Model Engagement.} Once preprocessed, the document is fed to an LLM for IE.  Ensuring alignment between the extracted text and layout information is crucial for accurate representation~\cite{xu2020layoutlmv2, appalaraju2021docformer}. Prompt-driven extraction leverages general-purpose models, using tailored prompts to guide the extraction process~\cite{brown2020language, radford2021learning, zhou2022least}. 
As such, the model needs to be instructed to extract information, and usually, the schema is provided. This step can involve more advanced IT and ICL techniques (few-shot, CoT).
\colorbox{blue!10}{\parbox{\columnwidth}{\emph{The influence of prompting techniques in interaction with various stages of the IE pipeline to enhance performance and robustness remains a research gap that requires further investigation.}}}

\noindent \textbf{Output Refinement.} After inference, the extracted information undergoes post-processing to ensure accuracy and conformance to Schema S. This step involves refining and validating the outputs generated by the LLM through tasks such as mapping extracted entities E to their original document positions, merging overlapping or fragmented predictions, and resolving ambiguities in the results~\cite{xu2020layoutlm}. 
Refining entity extraction through post-processing has been explored in various studies~\cite{j-wang-etal-2022-globalpointer, tamayo-etal-2022-nlp}. Rule-based entity alignment has notable accuracy improvements~\cite{luo2024asgeaexploitinglogicrules}.
\colorbox{blue!10}{\parbox{\columnwidth}{\emph{There exists no analysis of post-processing techniques tailored to LRDs in conjunction with LLMs.
}}}








