\section{Introduction}
\label{sec:introduction}
\begin{figure*}[t]
    \centering
    \includegraphics[width=\linewidth]{imgs/intro1.pdf}
    % \vspace{-3mm}
    \caption{This advancement provides a scalable and precise solution for diverse educational needs. (b). Previous retrieval models focus on text-query retrieval data. (c) Our style-diversified retrieval setting accommodates the various query styles preferred by real educational content.}%(a). Teachers need to collect data from four categories of STEM teaching contents, to accurately convey the teaching content. (b). Previous Retrieval Models focus on text-query retrieval data, neglecting the retrieval ability for educational query styles. However, teachers want to identify image data in different situation. (c). Our style-diversified retrieval setting considers the various query styles that real educational content may prefer, including sketches, art, low-resolution images, text, audio, and their combinations, such as sketch+text, art+text, etc.}
    \label{fig:motivation}
    \vspace{-5mm}
\end{figure*}

% (AIED)
Artificial Intelligence for Education (AI4EDU) is an emerging interdisciplinary field that leverages AI techniques to transform and enhance instructional design, learning processes, and assessment for education \cite{hwang2020vision}.
With the growing global emphasis on the importance of Science, Technology, Engineering, and Mathematics (STEM) education, retrieving accurate resources from massive interdisciplinary knowledge databases has become a critical challenge. 

Traditional retrieval systems are typically designed for natural text-image contents, which limits their utility in multi-modal STEM education contexts \cite{li2024alleviating,wang2023improving}. Research indicates that these systems often fail to capture the complexity of materials such as images, diagrams, or interactive content, which are vital in STEM disciplines \cite{intro3}. Effective retrieval in STEM education should be able to handle different representations (i.e., different styles of text, image, audio, etc.) to accommodate the diverse learning and teaching needs within STEM subjects. 

%Advancements in multi-modal algorithms have enhanced retrieval systems' capability to effectively manage various input types, providing substantial opportunities to improve content retrieval for teachers in STEM education.
%Some studies successfully integrate multi-modal inputs, allowing educators to access resources tailored to specific educational contexts and subjects, thereby increasing the relevance of materials retrieved from cross-subjects domains\cite{intro11}.
%Moreover, the adoption of diverse retrieval models not only facilitates the retrieval of educational resources but also supports personalized learning experiences\cite{intro12}. These models can be adjusted to meet the unique needs and preferences of individual students, ensuring that educational content aligns more closely with each learner's requirements\cite{intro13}.

Despite advancements in text-image matching techniques \cite{intro12,intro11}, current retrieval systems still encounter challenges when implemented in STEM education \cite{li2025freestyleret}.
These models are primarily optimized for matching text and images, neglecting the variety of query types essential in educational scenarios, including voice, sketches, and low-resolution images \cite{intro6}.  
%Only a few models can effectively accommodate various query from huge databases, which restricts their applicability in educational environments where query expressions can be highly diverse\cite{intro7}. 
The constraints of current frameworks within educational contexts frequently result in imprecise, biased, or inefficient retrieval outcomes, such deficiencies can impede teachers' access to suitable instructional resources \cite{intro9}. 

%However, as the scale of current retrieval models continues to grow, direct fine-tuning or training from scratch demands vast amounts of education-related data and extensive training time. Additionally, poorly performing retrieval models can lead to significant time and resource waste during usage, which is unacceptable in educational scenarios where resources are often limited\cite{guo2020vcbir}. Addressing these challenges is crucial for developing efficient and practical retrieval solutions tailored to educational needs.

To address above challenges, we propose a multi-style and multi-modal retrieval task tailored for STEM education scenarios, as illustrated in Fig.\ref{fig:motivation}.
The input types for this task include text, audio, and various styles of images, designed to meet the diverse needs of educational contexts.
To adapt to this task, we introduce the STEM Education Multi-Style Retrieval Dataset (SER), curated by 20 graduate and doctoral students from disciplines such as computer science, physics, energy, engineering, and mathematics. The dataset comprises 6,000 natural images with queries in various styles, including sketches, art, low-resolution images, text, and corresponding audio from different STEM fields.
%For modeling, we propose a novel plug-and-play feature representation structure called Prompt Bank.
%By matching the abstract features of data with the hidden information in Prompt Bank, this structure provides a more universal representation of information across different disciplines and semantics, achieving better generalization in educational scenarios.
For modeling, we propose a novel plug-and-play feature representation structure called Prompt Bank, which provides a more universal representation of information across different disciplines and semantics by matching the abstract features of data.
Building on Prompt Bank, we develop a lightweight retrieval model named {\bf Uni-Retrieval}.
Uni-Retrieval integrates Prompt Bank with various basic retrieval models, enabling efficient fine-tuning for educational scenarios.
With only a small increase in parameters and inference time, Uni-Retrieval delivers significant performance improvements and fine-tuning capabilities, making it an ideal solution for STEM education retrieval tasks.
Furthermore, it demonstrates the ability to perform effective retrieval across multiple unknown datasets, showcasing its scalability and adaptability in diverse scenarios.
The main contributions of this work can be summarized as follows:
\vspace{-0.5\intextsep}
\begin{itemize}
    %\item To bridge the content gap between teacher and student or some abstract expression in STEM education field, we propose the multi-style multi-modal retrieval task for the STEM education, a sub-condition of Query-based retrieval task, and can also expand to text and audio modalities. For this task, we collect the STEM Education Retrieval dataset~(SER), which contains several different subjects in STEM Education, and can expand to other subjects even not belong to STEM.
    \item To bridge the content gap between teacher and student or some abstract expression in STEM education field, we propose the multi-style multi-modal retrieval task for the STEM education. To adapt to this task, we construct the STEM Education Retrieval Dataset (SER), which contains several different subjects in STEM education.
\vspace{-0.5\intextsep}
    \item To efficiently and effectively train a model tailored to our task and dataset, we devise the novel Uni-Retrieval algorithm, which incorporates a sustainably updatable Prompt Bank. Leveraging this bank, the Uni-Retrieval can represent different subjects at a high level and express the features of any other subjects by combining several prompts. 
\vspace{-0.5\intextsep}
    \item Our Uni-Retrieval shows a more strong performance than any other previous method, not only on our SER dataset, but also on traditional retrieval dataset. We believe Uni-Retrieval can bring an infinite potential to STEM education community.%our community.
\end{itemize}
\vspace{-0.5\intextsep}

  
% xxx\cite{freestyleret}