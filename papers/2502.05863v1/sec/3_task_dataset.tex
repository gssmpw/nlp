\section{Preliminary}
\subsection{Task Formulation}
We provide a formal problem formulation for query-based retrieval.
Specifically, given an image $I_i$ or a text prompt $P_t$ from the style-specific query set $Q_s$, the retrieval model needs to compute the score between input and target queries and rank the corresponding answers $A$ as high as possible. In the task settings for STEM education retrieval, which share a similar goal, the objective is to rank all answers correctly in response to input queries across various style-specific query sets ${Q}_{s=1}^n$.
If the dataset does not contain the corresponding different style queries, the model should list the same category queries as the suggestions.

\subsection{Dataset Construction}
\begin{figure}[!tbp]
  \centering
   \includegraphics[width=\linewidth]{imgs/data_collection.pdf}
   \vspace{-7mm}
   \caption{\textbf{Data construction pipeline.} 1. STEM education knowledge base. 2. Data sources: from online resources and dataset researches. 3. Data processing: extracting essential information from collected data, using AIGC algorithms to generate diverse modalities. 4. Retrieval dataset: construct total 24,000 images and multi-modal STEM educational dataset.}
   \label{fig:data_collection}
   \vspace{-5mm}
\end{figure}
SER is a multi-style benchmark dataset we construct to facilitate accurate retrieval for teachers in STEM education. 
It contains a total of 24,000 text captions, audio clips and different style queries to accommodate different educational scenarios.
As illustrated in Fig.\ref{fig:data_sample}, SER contains:
{\bf Text and Natural Image:} the most common query type, allowing teachers to describe problems using natural language or images for retrieval.
{\bf Audio:} a communication medium in education, enabling teachers to articulate complex queries, which can be further enhanced with LLMs or audio encoders.
{\bf Sketch:} hand-drawn sketches, whether created by users or written on blackboards, provide structural cues such as shape, pose, line, and edges to describe the problem.
{\bf Art:} art-style images as queries help bridge the gap between stylistically different images and original images, improving retrieval consistency across styles. 
{\bf Low-Resolution:} queries involving lower-resolution images, such as those captured from a distance, ensure usability in scenarios where high-quality images are unavailable.

The details of the dataset construction pipeline are shown in Fig.\ref{fig:data_collection}, we use the original STEM education image in the source dataset, and extensively collected datasets from the following sources: 1. online resources such as \href{https://www.kaggle.com/datasets}{Kaggle}, \href{https://github.com/carbon-app/carbon}{GitHub}, \href{https://jr.brainpop.com/subject/science/}{BrainPOP}, \href{https://keypoint.keystonesymposia.org/}{Frontiers}, \href{https://learning.dk.com/us}{DKlearning}, \href{https://www.inaturalist.org/}{iNaturalist}, \href{https://www.asiastem.org/steam-free-resources}{AAES}, etc. 2. relevant education dataset research, such as GAN \cite{jin2023gan} and PromptAloud \cite{lee2024prompt}. To ensure high-quality data, more than 20 Ph.D. students from disciplines such as mathematics, physics, chemistry, biology, electronics, computer science, and education conducted a secondary screening of the raw images. They also generate multi-modal combinations (image/text/audio) by leveraging AIGC models.
Based on the natural images, the following steps were undertaken:
{\bf Text Generation:} we manually proofread the natural text descriptions.
{\bf Audio Recording:} the corresponding audio parts are recorded to match the text captions.
{\bf Sketch Images:} Canny algorithms are used to produce sketch images, Pidinet \cite{pdc} is employed to optimize and enhance low-quality sketches, and manual refinement is performed to achieve the final results.
{\bf Art-Style Images:} Flux model \cite{flux} is utilized to create art-style images.
{\bf Low-Resolution Images:} Gaussian Blur algorithms are applied to generate low-resolution images.
According to the National Science Foundation (NSF)’s classification of STEM education, we collect 6,000 original samples spanning six styles, three modalities, and over 22 subjects. This comprehensive dataset, as illustrated in Appendix Fig.\ref{fig:distribution}, ensures diversity and quality across multiple educational domains.



%SER is a multi-style benchmark dataset we constructed to facilitate accurate retrieval for teachers in STEM education. 
% It contains a total of 24,000 text captions, audio clips and different style queries to accommodate different educational scenarios.
%As illustrated in Fig.\ref{fig:data_sample}, SER contains:
% the SER dataset includes natural images paired with corresponding queries in six styles across three modalities: text, audio, and visual (sketch, art, low-resolution).
%(1-2). Text and Natural Image: The most common query type, allowing teachers to describe problems using natural language or images for retrieval. (3). Audio: A direct communication medium in education, enabling teachers to articulate complex queries, which can be further enhanced with large language models (LLMs) or audio encoders. (4). Sketch: Hand-drawn sketches, whether created by users or written on blackboards, provide structural cues such as shape, pose, line, and edges to describe the problem. (5). Art: Art-style images as queries help bridge the gap between stylistically different images and original images, improving retrieval consistency across styles. (6). Low-Resolution: Queries involving lower-resolution images, such as those captured from a distance, ensure usability in scenarios where high-quality images are unavailable.