\begin{abstract}

% In AI-facilitated teaching, leveraging various query styles to interpret abstract text descriptions is crucial for obtaining optimal results and ensuring high-quality teaching. However, current retrieval models mainly focus on text-image retrieval, and cutting-edge multi-style retrieval models cannot precisely adapt to educational scenarios, which may introduce potential ambiguities or biases. Besides, retrieving target from huge database also need efficient and precise method.
% In this paper, we propose a diverse expression retrieval task based on educational scenarios, which supports retrieval based on different styles and expressions. In order to adapt to educational scenarios, we introduce a \textbf{S}TEM \textbf{E}ducation \textbf{R}etrieval dataset~(SER dataset) includes multiple query types and styles.
% In addition, we propose Uni-Retrieval, an efficient style-diversified retrieval model that leverages prompt learning to effectively drive vision-language contrastive models.
% Specifically, we extract the query style feature as the prototype and establish a continuously updated prompt bank containing prompt tokens for various queries, which can be regularly updated to represent the corresponding knowledge for different subject retrieval scenarios.
% Our Uni-Retrieval framework exemplifies scalability and robustness by retrieving prompt tokens based on prototype similarity to facilitate learning when confronted with unknown queries.
% Experimental results show that our model outperforms existing retrieval models and larger-scale multimodal models in the stylistically diverse educational knowledge retrieval task. This has significant implications for teacher information retrieval in higher education.

In AI-facilitated teaching, leveraging various query styles to interpret abstract text descriptions is crucial for ensuring high-quality teaching.
However, current retrieval models primarily focus on natural text-image retrieval, making them insufficiently tailored to educational scenarios due to the ambiguities in the retrieval process.
In this paper, we propose a diverse expression retrieval task tailored to educational scenarios, supporting retrieval based on multiple query styles and expressions.
We introduce the \textbf{S}TEM \textbf{E}ducation \textbf{R}etrieval Dataset~(SER), which contains over 24,000 query pairs of different styles, and the Uni-Retrieval, an efficient and style-diversified retrieval vision-language model based on prompt tuning.
Uni-Retrieval extracts query style features as prototypes and builds a continuously updated Prompt Bank containing prompt tokens for diverse queries.
This bank can updated during test time to represent domain-specific knowledge for different subject retrieval scenarios.
Our framework demonstrates scalability and robustness by dynamically retrieving prompt tokens based on prototype similarity, effectively facilitating learning for unknown queries.
Experimental results indicate that Uni-Retrieval outperforms existing retrieval models in most retrieval tasks.
%This advancement holds significant implications for teacher information retrieval in higher education, providing a scalable and precise solution for diverse educational needs.
This advancement provides a scalable and precise solution for diverse educational needs.
\end{abstract}
