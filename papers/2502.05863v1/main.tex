% This must be in the first 5 lines to tell arXiv to use pdfLaTeX, which is strongly recommended.
\pdfoutput=1
% In particular, the hyperref package requires pdfLaTeX in order to break URLs across lines.

\documentclass[11pt]{article}

% Change "review" to "final" to generate the final (sometimes called camera-ready) version.
% Change to "preprint" to generate a non-anonymous version with page numbers.
\usepackage[final]{acl}

% Standard package includes
\usepackage{times}
\usepackage{latexsym}

% For proper rendering and hyphenation of words containing Latin characters (including in bib files)
\usepackage[T1]{fontenc}
% For Vietnamese characters
% \usepackage[T5]{fontenc}
% See https://www.latex-project.org/help/documentation/encguide.pdf for other character sets

% This assumes your files are encoded as UTF8
\usepackage[utf8]{inputenc}

% This is not strictly necessary, and may be commented out,
% but it will improve the layout of the manuscript,
% and will typically save some space.
\usepackage{microtype}

% This is also not strictly necessary, and may be commented out.
% However, it will improve the aesthetics of text in
% the typewriter font.
\usepackage{inconsolata}

%Including images in your LaTeX document requires adding
%additional package(s)
\usepackage{graphicx}

% custom package
\usepackage{booktabs}
\usepackage{amsmath}
\usepackage{array}
\usepackage{color}
\usepackage{bbding} % \Checkmark对应的包
\usepackage{amsfonts} % 防止\mathbb报错
\usepackage{algorithm}
\usepackage{algpseudocode}
\usepackage{multirow}
\usepackage{stfloats}
\usepackage[none]{hyphenat}

\definecolor{cred}{HTML}{FF6B6B}
\definecolor{cyellow}{HTML}{FEC260}
%\definecolor{cgreen}{HTML}{6BCB77}
\definecolor{cgreen}{HTML}{70AD47}
\definecolor{cblue}{HTML}{4D96FF}
\definecolor{cpurple}{HTML}{2A0944}
\definecolor{ggray}{RGB}{127,127,127}
\definecolor{aliceblue}{rgb}{0.94, 0.97, 1.0}
\definecolor{cvprblue}{rgb}{0.21,0.49,0.74}

% If the title and author information does not fit in the area allocated, uncomment the following
%
%\setlength\titlebox{<dim>}
%
% and set <dim> to something 5cm or larger.

\title{Uni-Retrieval: A Multi-Style Retrieval Framework for STEM's Education}

% Author information can be set in various styles:
% For several authors from the same institution:
% \author{Author 1 \and ... \and Author n \\
%         Address line \\ ... \\ Address line}
% if the names do not fit well on one line use
%         Author 1 \\ {\bf Author 2} \\ ... \\ {\bf Author n} \\
% For authors from different institutions:
% \author{Author 1 \\ Address line \\  ... \\ Address line
%         \And  ... \And
%         Author n \\ Address line \\ ... \\ Address line}
% To start a separate ``row'' of authors use \AND, as in
% \author{Author 1 \\ Address line \\  ... \\ Address line
%         \AND
%         Author 2 \\ Address line \\ ... \\ Address line \And
%         Author 3 \\ Address line \\ ... \\ Address line}



\author{Yanhao Jia \textsuperscript{1${*}$},
        Xinyi Wu \textsuperscript{2}\thanks{ Equal Contribution, $^{\dagger}$ Corresponding Authors.}, 
        Hao Li \textsuperscript{3},
        Qinglin Zhang \textsuperscript{2}, \vspace{0.2mm}  \\
        {\bf Yuxiao Hu \textsuperscript{4},}
       {\bf Shuai Zhao \textsuperscript{1 ${\dagger}$},
       {\bf Wenqi Fan \textsuperscript{4 ${\dagger}$}}}\\
{ 
\textsuperscript{1} Nanyang Technological University, Singapore;
}\vspace{-0.1mm} \\
{ 
\textsuperscript{2} Shanghai Jiao Tong University, China;
}\vspace{-0.1mm} 
{
\textsuperscript{3} Peking University, China;
}\vspace{-0.1mm} \\
{
\textsuperscript{4} Hong Kong Polytechnic University, Hong Kong, China;
}\vspace{-0.1mm} \\}

% 
  
\newcommand{\yh}[1]{{\color{red}#1}}


\begin{document}
\maketitle


% \twocolumn[{%
% \renewcommand\twocolumn[1][]{#1}%
% \maketitle
% \begin{center}
%     % \centering
%     \vspace{-5mm}
%     \includegraphics[width=\linewidth]{imgs/intro1.pdf}
%     \captionof{figure}{(a). Teachers need to collect data from four categories of STEM teaching contents, to accurately convey the teaching content. (b). Previous Retrieval Models focus on text-query retrieval data, neglecting the retrieval ability for educational query styles. However, teachers want to identify image data in different situation. (c). Our style-diversified retrieval setting considers the various query styles that real educational content may prefer, including sketches, art, low-resolution images, text, audio, and their combinations, such as sketch+text, art+text, etc. Our model performs fine-grained retrieval based on the shape, color, and pose features extracted from these style-diversified query inputs.}
%     \label{fig:motivation}
% \end{center}
% }]

\begin{abstract}

% Recent works to jointly reconstruct 3D human and object from a single RGB image, are mostly model-based, that fail to capture the fine details of the clothed human body and object surface. In this paper, we introduce ReCHOR, a novel, model-free, first-method to produce realistic clothed human-object reconstructions from a monocular view. This is extremely challenging due to human-object occlusions, diverse interactions and depth ambiguity, as it needs to infer both 3D spatial awareness and high resolution details. Our core idea is based on estimating neural implicit representations for human and object respectively by an attention-based neural implicit model that attends to pixel-aligned features from both the global human-object image for spatial awareness and  the local separate view of human and object images for high quality details. Additionally, the network is conditioned on semantic features from an initial estimated human-object pose prior and a generative diffusion model that inpaints occluded regions, thus enabling the retrieval of details from them.
% We also propose a synthetic dataset with rendered scenes of diverse, inter-occluded 3D human and object scans, to train our network. We evaluate our method on the synthetic and real world BEHAVE dataset. Our experiments show that our method outperforms the SOTA in achieving realistic clothed human-object reconstructions.
Recent approaches to jointly reconstruct 3D humans and objects from a single RGB image represent 3D shapes with template-based or coarse models, which fail to capture details of loose clothing on human bodies. In this paper, we introduce a novel implicit approach for jointly reconstructing realistic 3D clothed humans and objects from a monocular view. For the first time, we model both the human and the object with an implicit representation, allowing to capture more realistic details such as clothing. This task is extremely challenging due to human-object occlusions and the lack of 3D information in 2D images, often leading to poor detail reconstruction and depth ambiguity. To address these problems, we propose a novel attention-based neural implicit model that leverages image pixel alignment from both the input human-object image for a global understanding of the human-object scene and from local separate views of the human and object images to improve realism with, for example, clothing details. Additionally, the network is conditioned on semantic features derived from an estimated human-object pose prior, which provides 3D spatial information about the shared space of humans and objects. To handle human occlusion caused by objects, we use a generative diffusion model that inpaints the occluded regions, recovering otherwise lost details. For training and evaluation, we introduce a synthetic dataset featuring rendered scenes of inter-occluded 3D human scans and diverse objects. Extensive evaluation on both synthetic and real-world datasets demonstrates the superior quality of the proposed human-object reconstructions over competitive methods.
\end{abstract}
\section{Introduction}\label{sec:intro}

In computational finance, Monte Carlo simulations are used extensively to estimate the expected value of financial payoffs based on the solution of stochastic differential equations (SDEs) which model the evolution of stock prices, interest rates, exchange rates and other quantities \cite{glasserman04}.  Monte Carlo methods are very general and flexible, but for high accuracy it requires generating a large number of costly SDE path approximations, which has motivated research into a number of variance reduction or, equivalently, cost reduction techniques. One such method is
Multilevel Monte Carlo (MLMC), which was proposed in \cite{GILES2008} and was adapted for various applications that are summarised in \cite{Giles_overview17} and successfully combined with other methods such as quasi-Monte Carlo methods. The main idea of MLMC is to approximate the payoff using different time stepping resolutions when numerically solving the underlying SDE and to generate an optimal number of samples on each level, such that the overall computational cost is minimised subject to the desired bound on the variance. %, such that the total computational cost is minimised. 
The computational savings come from the fact that most samples are computed on the coarser levels and hence are less expensive while only a few samples from the finest levels are required \cite{GILES2008}.


Among the directions in which the computational cost 
of MLMC methods could further be reduced, an important avenue is the use of lower precision calculations, especially for the first Monte Carlo levels where the targeted accuracy is relatively low. 
 An overview of the research on mixed precision for the standard Monte Carlo (MC) framework is provided in \cite{ChowMixedPrecisionStandardMC} but only a few references study the potential of low precision computation in the MLMC framework \cite{Rounding_error_oliver}. To the best of our knowledge, the only MLMC framework with customised precision in the literature is \cite{brugger2014mixed}, but they use a uniform precision for all operations on each Monte Carlo level instead of optimising 
 the precision of each intermediary variable to reduce as much as possible the cost of path generation.
 
An important motivation for an MLMC framework with variable precision would be performing the low precision computations on reconfigurable hardware devices such as Field Programmable Gate Arrays (FPGAs). FPGAs contain customizable logic blocks and connectors that make it easy to adapt the digital circuit architecture for a specific application, leading to a highly parallel and optimised implementation. Therefore they are successfully exploited in applications that require high speed and have high computational workload, such as signal processing \cite{woods2008fpga}, and real time applications like high frequency trading \cite{HFT1,HFT2}. That is why a number of previous works in hardware architecture design implemented the MLMC algorithm to price financial options using FPGAs as accelerators, which resulted in improved speed and power efficiency compared to full CPU architectures \cite{Schryver2013AMM}. The paper \cite{lindsey2016domain} also proposed 
a Domain Specific Language to automate the configuration of FPGAs for this specific application. However, only \cite{brugger2014mixed} proposed a heuristic to reduce the precision in calculations.

In addition, all aforementioned works considered that the random number generation (RNG) is performed in single or double precision. Yet in most cases an important portion of the workload in the overall MLMC simulation comes from the RNG and in \cite{brugger2014mixed} this limited the total computational savings.
To reduce the cost of MLMC simulations in particular those based on the Geometric Brownian Motion (GBM), \cite{approximateICDF_Oliver, NestedOliver} have proposed to use approximate random numbers that are generated by applying an approximation of the inverse CDF to uniform random numbers. In \cite{NestedOliver}, the authors proposed a way to integrate these lower precision random variables into a \textit{nested} MLMC framework and completed a numerical analysis to bound the resulting error at each MC level by a product of the time step and the error in the random number approximation. The same authors show in \cite{approximateICDF_Oliver} that using approximate random variables reduces the cost of path generation by a factor 7.


In this paper we propose a nested MLMC framework that combines the use of approximate random normal variables and lower precision calculations to reduce the computational cost of MLMC even further than \cite{brugger2014mixed,NestedOliver}. We illustrate the efficiency of our framework in Matlab, after making several assumptions on the cost of operations and size of the errors that we carefully justify. We focus on the case of GBM and use the approximate RNG methods presented in \cite{approximateICDF_Oliver} as well as a new slightly modified method that combines CDF inversion and the central limit theorem. To choose the precision of the variables in the low precision path generation, we introduce a novel method to optimise the bit-widths. This optimisation is performed before the main path generation loop is executed and is based on a linear model of the payoff error  
due to rounding when computing in low precision. The error model relies on algorithmic differentiation in a similar manner to \cite{unifying-bwoptim,bitwidth-AD,ADAPT}. The bit-width optimisation procedure can be performed off-line, so this stage can be excluded from the on-line time complexity of our framework. The user specified desired accuracy is then enforced by calculating on-line the number of samples that need to be generated.

In terms of hardware design, we suggest implementing the low precision path generation on FPGAs and the full-precision ones on a CPU or GPU. 
The FPGA offers enough flexibility to define a separate bit-width for every variable in the low precision path generation, and can be reconfigured periodically to update the bit-widths when the market parameters have changed considerably. 


The paper is organized as follows : \Cref{sec:MLMC} introduces MLMC and nested MLMC to make clear the estimator that is implemented in our framework. Then in \Cref{sec:RNG} we detail the methods that could be used to obtain approximate random normally distributed numbers very cheaply for the low precision path generation. In \Cref{sec:error_model} and \Cref{sec:costModel} we propose an error model and a cost model (resp.) that we then use to formulate the optimisation problem that is solved to obtain the optimal bit-widths of fixed point variables in \Cref{sec:optimisation}. Finally we summarise our results and future directions in \Cref{sec:conclusion}.



\section{Preliminary}
\subsection{Task Formulation}
We provide a formal problem formulation for query-based retrieval.
Specifically, given an image $I_i$ or a text prompt $P_t$ from the style-specific query set $Q_s$, the retrieval model needs to compute the score between input and target queries and rank the corresponding answers $A$ as high as possible. In the task settings for STEM education retrieval, which share a similar goal, the objective is to rank all answers correctly in response to input queries across various style-specific query sets ${Q}_{s=1}^n$.
If the dataset does not contain the corresponding different style queries, the model should list the same category queries as the suggestions.

\subsection{Dataset Construction}
\begin{figure}[!tbp]
  \centering
   \includegraphics[width=\linewidth]{imgs/data_collection.pdf}
   \vspace{-7mm}
   \caption{\textbf{Data construction pipeline.} 1. STEM education knowledge base. 2. Data sources: from online resources and dataset researches. 3. Data processing: extracting essential information from collected data, using AIGC algorithms to generate diverse modalities. 4. Retrieval dataset: construct total 24,000 images and multi-modal STEM educational dataset.}
   \label{fig:data_collection}
   \vspace{-5mm}
\end{figure}
SER is a multi-style benchmark dataset we construct to facilitate accurate retrieval for teachers in STEM education. 
It contains a total of 24,000 text captions, audio clips and different style queries to accommodate different educational scenarios.
As illustrated in Fig.\ref{fig:data_sample}, SER contains:
{\bf Text and Natural Image:} the most common query type, allowing teachers to describe problems using natural language or images for retrieval.
{\bf Audio:} a communication medium in education, enabling teachers to articulate complex queries, which can be further enhanced with LLMs or audio encoders.
{\bf Sketch:} hand-drawn sketches, whether created by users or written on blackboards, provide structural cues such as shape, pose, line, and edges to describe the problem.
{\bf Art:} art-style images as queries help bridge the gap between stylistically different images and original images, improving retrieval consistency across styles. 
{\bf Low-Resolution:} queries involving lower-resolution images, such as those captured from a distance, ensure usability in scenarios where high-quality images are unavailable.

The details of the dataset construction pipeline are shown in Fig.\ref{fig:data_collection}, we use the original STEM education image in the source dataset, and extensively collected datasets from the following sources: 1. online resources such as \href{https://www.kaggle.com/datasets}{Kaggle}, \href{https://github.com/carbon-app/carbon}{GitHub}, \href{https://jr.brainpop.com/subject/science/}{BrainPOP}, \href{https://keypoint.keystonesymposia.org/}{Frontiers}, \href{https://learning.dk.com/us}{DKlearning}, \href{https://www.inaturalist.org/}{iNaturalist}, \href{https://www.asiastem.org/steam-free-resources}{AAES}, etc. 2. relevant education dataset research, such as GAN \cite{jin2023gan} and PromptAloud \cite{lee2024prompt}. To ensure high-quality data, more than 20 Ph.D. students from disciplines such as mathematics, physics, chemistry, biology, electronics, computer science, and education conducted a secondary screening of the raw images. They also generate multi-modal combinations (image/text/audio) by leveraging AIGC models.
Based on the natural images, the following steps were undertaken:
{\bf Text Generation:} we manually proofread the natural text descriptions.
{\bf Audio Recording:} the corresponding audio parts are recorded to match the text captions.
{\bf Sketch Images:} Canny algorithms are used to produce sketch images, Pidinet \cite{pdc} is employed to optimize and enhance low-quality sketches, and manual refinement is performed to achieve the final results.
{\bf Art-Style Images:} Flux model \cite{flux} is utilized to create art-style images.
{\bf Low-Resolution Images:} Gaussian Blur algorithms are applied to generate low-resolution images.
According to the National Science Foundation (NSF)’s classification of STEM education, we collect 6,000 original samples spanning six styles, three modalities, and over 22 subjects. This comprehensive dataset, as illustrated in Appendix Fig.\ref{fig:distribution}, ensures diversity and quality across multiple educational domains.



%SER is a multi-style benchmark dataset we constructed to facilitate accurate retrieval for teachers in STEM education. 
% It contains a total of 24,000 text captions, audio clips and different style queries to accommodate different educational scenarios.
%As illustrated in Fig.\ref{fig:data_sample}, SER contains:
% the SER dataset includes natural images paired with corresponding queries in six styles across three modalities: text, audio, and visual (sketch, art, low-resolution).
%(1-2). Text and Natural Image: The most common query type, allowing teachers to describe problems using natural language or images for retrieval. (3). Audio: A direct communication medium in education, enabling teachers to articulate complex queries, which can be further enhanced with large language models (LLMs) or audio encoders. (4). Sketch: Hand-drawn sketches, whether created by users or written on blackboards, provide structural cues such as shape, pose, line, and edges to describe the problem. (5). Art: Art-style images as queries help bridge the gap between stylistically different images and original images, improving retrieval consistency across styles. (6). Low-Resolution: Queries involving lower-resolution images, such as those captured from a distance, ensure usability in scenarios where high-quality images are unavailable.
\section{Causal IL as CMRs}\label{sec:method}

In this section, we demonstrate that performing causal IL in our framework is possible using trajectory histories as instruments. In the next step, we show that the problem can be described as CMRs and propose an effective algorithm to solve it.

The typical target for IL would be the expert policy $\pi_E$ itself. However, since the expert has access to information, namely $u^o_t$, which the imitator does not, the best thing an imitator can do is to learn a history-dependent policy $\pi_h$ that is the closest to the expert. A natural choice is the conditional expectation of $\pi_E(s_t,u^o_t)$ on the history $h_t$:
\begin{align}
\pi_h(h_t)\coloneqq \expectE_{\probP(u^o_t\mid h_t)}[\pi_E(s_t,u^o_t)]=\expectE[\pi_E(s_t,u^o_t)\mid h_t],\nonumber
\end{align}
% where $p(u^o_t\mid h_t)$ is a distribution over expert-observable confounders and captures the information about $u^o_t$ can be inferred from the trajectory history. 
because the conditional expectation minimizes the least squares criterion~\citep{hastie01statisticallearning} and $\pi_h$ is the best predictor of $\pi_E$ given $h_t$. In $\pi_h$, the distribution $\probP(u^o_t\mid h_t)$ captures the information about $u^o_t$ that can be inferred from trajectory histories.
\begin{remark}
\emph{Learning $\pi_h$ is not trivial. Policies learnt naively using behaviour cloning (i.e., $\expectE[a_t\mid h_t]$) fail to match $\pi_E$. In view of~\cref{eq:action}, we have that
\begin{align} 
\expectE[a_t\mid h_t]&=\expectE[\pi_E(s_t,u^o_t) \mid h_{t}]+\expectE[u^\epsilon_t\mid h_{t}]\nonumber\\
&=\pi_h(h_t)+\expectE[u^\epsilon_t\mid h_{t}],\label{eq:history_policy}
\end{align}
where $\expectE[u^\epsilon_t\mid h_{t}]\neq 0$ due to the spurious correlation between $u^\epsilon_t$ and the trajectory history $h_t$. As a result, $\expectE[a_t\mid h_t]$ becomes biased, which can lead to arbitrarily worse performance compared to $\pi_E$.   }
\end{remark}

\vspace{-5pt}
\paragraph{Derivation of CMRs.} 
Leveraging the confounding horizon from Assumption~\ref{assump:horizon}, it becomes possible to break the spurious correlation using the independence of $u^\epsilon_t$ and $u^\epsilon_{t-k}$. We propose to use the $k$-step trajectory history $h_{t-k}=(s_{1},a_{1},...,s_{t-k})$ as an instrument for the current state $s_t$. Taking the expectation conditional on $h_{t-k}$ in~\cref{eq:history_policy} yields
\begin{align*}
    \expectE[a_t\mid h_{t-k}] & = \expectE\left[\expectE[a_t\mid h_{t}]\mid h_{t-k}\right] \\ & = \expectE[\pi_h(h_t)\mid h_{t-k}]+\expectE[\expectE[u^\epsilon_t\mid h_{t}]\mid h_{t-k}] \\
    & = \expectE[\pi_h(h_t) \mid h_{t-k}]+\expectE[u^\epsilon_t\mid h_{t-k}]
\end{align*}
where we use the fact that $h_{t-k}$ is $\sigma(h_t)$-measurable because $h_{t-k}\subseteq h_t$. Next, recall that $u^\epsilon_t\indep u^\epsilon_{t-k}$ by Assumption~\ref{assump:horizon}, which implies $u^\epsilon_t\indep h_{t-k}$, so that % Hence, since $\expectE[u^\epsilon_t] = 0$, we obtain
\begin{align}
    \expectE[a_t\mid h_{t-k}] &= \expectE[\pi_h(h_t) \mid h_{t-k}]+\expectE[u^\epsilon_t]\nonumber\\
    &=\expectE[\pi_h(h_t) \mid h_{t-k}].
\end{align}

As a result, the problem of learning $\pi_h$ reduces to solving for $\pi_h$ that satisfies the following identity
\begin{align}
    \expectE[a_t-\pi_h(h_t)\mid h_{t-k}]=0,\label{eq:CMR}
\end{align}
which is a CMR problem as defined in~\cref{sec:cmr}. In this case, both $a_t$ and $h_t$ are observed in the confounded expert demonstrations, and $h_{t-k}$ acts as the instrument. 

To make sure the instrument $h_{t-k}$ is valid, we check that it satisfies the conditions of~\cref{assump:iv}. Firstly, we have checked that $u^\epsilon_t\indep h_{t-k}$. Secondly, the environment and the expert policy are non-trivial, which means $\probP(h_t\mid h_{t-k})$ is not constant in $h_{t-k}$. Finally, $h_{t-k}$ indeed only affects $a_t$ through $s_t$ by the Markovian property. However, the strength of the instrument, which informally represents the correlation between the instrument $h_{t-k}$ and $h_t$, plays an important role in how well we can identify $\pi_h(h_t)$ by solving the CMRs in~\cref{eq:CMR}. In particular, we see that, as the confounding horizon $k$ increases, the correlation between $h_{t-k}$ and $h_t$ weakens and $h_{t-k}$ becomes a weaker instrument. This means that it is less able to identify $\pi_h$ via the CMR in~\cref{eq:CMR} and the final learnt imitator will have poorer performance. This is confirmed theoretically in Proposition~\ref{prop:ill-posed} and experimentally in~\cref{sec:exps}, and we will formalise this notion of instrument strength in~\cref{sec:theory}.


% Note this problem is equivalent to solving an IV regression on~\cref{eq:history_policy}, where $Y=\expectE[a_t\lvert h_t]$, $f(x)=\pi_h(h_t)$, $\epsilon=\expectE[u^\epsilon_t$ and the instrument $Z=h_{t-k}$.




\subsection{Practical Algorithms for Solving the CMRs}

\begin{algorithm}[tb]
   \caption{DML-IL}
   \label{alg:DML-IL}
\begin{algorithmic}[1]
   \STATE {\bfseries input} Dataset $\dataset_E$ of expert demonstrations, Confounding noise horizon $k$
   \STATE Initialize the roll-out model $\hat{M}$ as a Gaussian mixture model\label{algo:roll_out_1}
    \REPEAT
   \STATE Sample $(h_{t},a_t)$ from data $\dataset_E$
   \STATE Fit the roll-out model $(h_t,a_t)\sim\hat{M}(h_{t-k})$ to maximize the log likelihood 
\UNTIL{convergence}\label{algo:roll_out_2}
   \STATE Initialize the expert model $\hat \pi_h$ as a neural network
   \REPEAT
   % \FOR{$k=1$ {\bfseries to} $K$}
   \STATE Sample $h_{t-k}$ from $\dataset_E$
   \STATE Generate $\hat{h}_t$ and $\hat{a}_t$ using the roll-out model $\hat{M}$
   \STATE Update $\hat \pi_h$ to minimise the loss $\ell:= \norm{\hat{a}_t - \hat{\pi}_h (\hat h_t)}_2$
   % \ENDFOR
    \UNTIL{convergence}
    \STATE {\bfseries return} A history-dependent imitator policy $\hat{\pi}_h$
\end{algorithmic}
\end{algorithm}

There are various techniques~\citep{Shao2024,Bennett2019,Xu2020,Dikkala2020} for solving the CMRs $\expectE[a_t\lvert h_{t-k}]=\expectE[\pi_h(h_t) \lvert h_{t-k}]$. Here, the \textit{CMR error} that we aim to minimise is given by 
\begin{align*}
\sqrt{\expectE\big[\expectE[a_t-\hat{\pi}_h(h_t)\lvert h_{t-k}]^2\big]}=\norm{\expectE[a_t-\hat{\pi}_h(h_t)\lvert h_{t-k}]}_{2}.    
\end{align*}
In~\cref{alg:DML-IL}, we introduce DML-IL, an algorithm adapted from the IV regression algorithm DML-IV~\citep{Shao2024}\footnote{DML stands for double machine learning~\citep{Chernozhukov2018Double}, which is a statistical technique to ensure fast convergence rate for two-step regression, as is the case in~\cref{alg:DML-IL}.}, which solves our CMRs by minimising the CMR error. The first part of the algorithm (line 3-7) learns a roll-out model $\hat{M}$ that generates a trajectory $k$ steps ahead given $h_{t-k}$. Then, the roll-out model $\hat{M}$ is used to train the policy model $\hat{\pi}_h$ (line 8-13). $\hat{\pi}_h$ takes the generated trajectory $\hat{h}_t$ from $\hat{M}(h_{t-k})$ as inputs, and minimises the mean squared error to the next action. Using generated trajectories is crucial in breaking the spurious correlation caused by $u^\epsilon_t$ between past states and actions, and using the trajectory history before $h_{t-k}$ allows the imitator to infer information about $u^o_t$.

DML-IL can also be implemented with $K$-fold cross-fitting, where the dataset is partitioned into $K$ folds, with each fold alternately used to train $\hat{\pi}_h$ and the remaining folds to train $\hat{M}$. This ensures unbiased estimation and improves the stability of training. The base IV algorithm DML-IV with $K$-fold cross-fitting is theoretically shown to converge at the rate of $O(N^{-1/2})$~\citep{Shao2024}, where $N$ is the sample size, under regularity conditions. DML-IL with $K$-fold cross-fitting (see~\cref{appendix:dmlil} for details) will thus inherit this convergence rate guarantee. 

Note that~\cref{alg:DML-IL} requires the confounding noise horizon $k$ as input. While the exact value of $k$ can be difficult to obtain in reality, any upper bound $\bar{k}$ of $k$ is sufficient to guarantee the correctness of ~\cref{alg:DML-IL}, since $h_{t-\bar{k}}$ is also a valid instrument. Ideally, we would like a data-driven approach to determine $k$. Unfortunately, it is generally intractable to empirically verify whether $h_{t-k}$ is a valid instrument from a static dataset, especially the unconfounded instrument condition (i.e., $h_{t-k}\indep u^\epsilon_t$). Therefore, we rely on the user to provide a sensible choice of $\bar{k}$ based on the environment that does not substantially overestimate $k$.


\subsection{Theoretical Analysis}\label{sec:theory}

% \begin{align}
% p(u_t\lvert do(a_{t-k+1}),...,do(a_{t-1}),s_{t-k+1},...,s_{t-1})&\propto p(h_t)p_{\mu_0}(s_{t-k+1})\prod_{i=t-k+1}^{t-1} \transitions(s_{i+1}\lvert s_i,a_i,u_i)
% \end{align}

% since $$(u_t\indep a_{(t-k+1)...(t-1)} \lvert s_{(t-k+1)...(t_1)})_{\mathcal{G}_{\underline{a{(t-k+1)...(t-1)}}}}$$
% on the causal graph $\mathcal{G}_{\underline{a{(t-k+1)...(t-1)}}}$ where the arrows going into $a_{(t-k+1)...(t-1)}$ are removed.



In this section, we derive theoretical guarantees for our algorithm, focusing on the imitation gap and its relationship with existing work.


On a high level, in order to bound the imitation gap of the learnt policy $\hat{\pi}_h$, i.e., $J(\pi_E)-J(\hat{\pi}_h)$, we need to control:
\begin{enumerate}
    \item[($i$)] The amount of information about the hidden confounders that can be inferred from trajectory histories;
    \item[($ii$)] The ill-posedness (or identifiability) of the set of CMRs, which intuitively measures the strength of the instrument $h_{t-k}$;
    \item[($iii$)] The disturbance of the confounding noise to the states and actions at test time.
\end{enumerate}
These factors are all determined by the environment and the expert policy. To control ($i$), we measure how much information about $u^o_t$ is captured by the trajectory history $h_t$ by analysing the Total Variation (TV) distance between the distribution of $u^o_t$ and $\expectE[u^o_t\lvert h_t]$ along the trajectories of $\pi_E$. To control ($ii$) and ($iii$), we need to introduce the following two key concepts.

\begin{definition}[The ill-posedness of CMRs~\citep{Dikkala2020,Chen2012}]

Given the derived CMRs in~\cref{eq:CMR}, for a policy $\pi\in\Pi$, $\norm{\pi_E-\pi}_2$ is the root mean squared error to the expert and $\norm{\expectE[a_t-\pi(s_t)\lvert s_{t-k}]}_2$ is the CMR error we aim to minimise. Then, the \emph{ill-posedness} $\ill(\Pi,k)$ of the policy space with confounding noise horizon $k$ is given by
\begin{align*}
    \ill(\Pi,k)=\sup_{\pi\in\Pi} \frac{\norm{\pi_E-\pi}_{2}}{\norm{\expectE[a_t-\pi(h_t)\lvert h_{t-k}]}_{2}}.
\end{align*}
\end{definition}
The ill-posedness $\ill(\Pi,k)$ measures the strength of the instrument where a higher $\ill(\Pi,k)$ indicates a weaker instrument. It bounds the ratio between the learning error of the imitator following our CMR objective and its $L_2$ error to the expert policy. 

As discussed previously, intuitively, the strength of the instrument would decrease as the confounding horizon $k$ increases. This is in fact true and is confirmed by the following proposition. The proof is deferred to~\cref{appendix:prop}. 
\begin{proposition}\label{prop:ill-posed}
The ill-posedness $\ill(\Pi,k)$ is monotonically increasing as the confounded horizon $k$ increases.
\end{proposition}

Next, we introduce the notion of c-TV stability.
\begin{definition}[c-total variation stability~\citep{Bassily2021,Swamy2022_temporal}]
Let $P(X)$ be the distribution of a random variable $X:\Omega\rightarrow \mathcal{X}$. $P(X)$ is c-TV stable if for $a_1,a_2\in \mathcal{X}$ and $\Delta>0$,
\begin{align*}
\norm{a_1-a_2}\leq\Delta \implies \delta_{TV}(a_1+X,a_2+X)\leq c\Delta.
\end{align*}
where $\norm{\cdot}$ is some norm defined on $\mathcal{X}$ and $\delta_{TV}$ is the total variation distance.
\end{definition}
A wide range of distributions are c-TV stable. For example, standard normal distributions are $\frac{1}{2}$-TV stable. We apply this notion to the distribution over $u^\epsilon_t$ to bound the disturbance it induces in the trajectory and the expected return.

With the notion of ill-posedness and c-TV stability, we can now analyse and upper bound the imitation gap $J(\pi_E)-J(\hat{\pi}_h)$ by controlling the three components $(i)-(iii)$ discussed above. 
% We present the main result for this paper, where t
The full proof is deferred to~\cref{appendix:gap}.

\begin{theorem}[Imitation Gap Bound]\label{thm:gap}
Let $\hat{\pi}_h$ be the learnt policy with CMR error $\epsilon$ and let $\ill(\Pi,k)$ be the ill-posedness of the problem. Assume that $\delta_{TV}(u^o_t,\expectE_{\pi_E}[u^o_t\lvert h_t])\leq\delta$ for $\delta\in\realNumber^+$, $P(u^\epsilon_t)$ is c-TV stable and $\pi_E$ is deterministic. Then, the imitation gap is upper bounded by 
\begin{align*}
    J(\pi_E)-J(\hat{\pi}_h)\leq T^2\big(c\epsilon\ill(\Pi,k)+2\delta\big)=\mathcal{O}\big(T^2(\delta+\epsilon)\big).
\end{align*}
\end{theorem}
This upper bound scales at the rate of $T^2$, which aligns with the expected behaviour of imitation learning without an interactive expert~\citep{Ross2010}.
Next, we show that the upper bounds on the imitation gap from prior work~\citep{Swamy2022_temporal, Swamy2022} are special cases of
% of  subsumed by the unifying causal IL framework introduced in Section~\ref{sec:setting} are special cases of 
Theorem~\ref{thm:gap}. The proofs are deferred to~\cref{appendix:corollaries}.
\begin{corollary}\label{corollary:noUo}
In the special case that $u^o_t = 0$, i.e., there are no expert-observable confounders, or $u^o_t=\expectE_{\pi_E}[u^o_t\lvert h_t]$, i.e., $u^o_t$ is $\sigma(h_t)$ measurable (all information about $u^o_t$ is contained in the history), the imitation gap is upper bounded by
\begin{align*}
    J(\pi_E)-J(\hat{\pi}_h)\leq T^2\big(c\epsilon\ill(\Pi,k)\big)=\mathcal{O}\big(T^2\epsilon\big),
\end{align*}
which coincides with Theorem 5.1 of~\citet{Swamy2022_temporal}.
\end{corollary}

When there are no hidden confounders, i.e, $u^\epsilon_t=0$, our framework is reduced to that of~\citet{Swamy2022}. However, \citet{Swamy2022} provided an abstract bound that directly uses the supremum of key components in the imitation gap over all possible Q functions to bound the imitation gap. We further extend and concretise the bound using the learning error $\epsilon$ and the TV distance bound $\delta$ instead of relying on the suprema.


\begin{corollary}\label{corollary:unconfounded}
In the special case that $u^\epsilon_t=0$, if the learnt policy has optimisation error $\epsilon$,  the imitation gap is upper bounded by
\begin{align*}
    J(\pi_E)-J(\hat{\pi}_h)\leq T^2\left(\frac{2}{\sqrt{\dim(A)}}\epsilon+2\delta \right),
\end{align*}
which is a concrete bound that extends the abstract bound in Theorem 5.4 of~\cite{Swamy2022}.
\end{corollary}

\begin{remark}
\emph{If both $u^\epsilon_t$ and $u^o_t$ are zero, we then recover the classic setting of IL without confounders~\citep{Ross2010}, and the imitation gap bound is $T^2\epsilon$, where $\epsilon$ is the optimisation error of the algorithm.}
\end{remark}
\section{Experiment}
In this section, we conduct extensive experiments to evaluate the performance of various LLMs on our Hellaswag-Pro benchmark. Our study is guided by three key research questions:
\textbf{RQ1}: How do different LLMs perform across all variants?
\textbf{RQ2}: What is the relative difficulty of different variants?
\textbf{RQ3}: How robust are LLMs to diverse prompts during evaluation?

\subsection{Experiment Setup} 
\subsubsection{Model Selection and Implementation Details}
We select 41 representative commercial and open-source models, including English LLMs, such as GPT-4o, Claude-3.5-Sonnet, Gemini-1.5-Pro,Mistral series, Llama3 series and Chinese LLMs, like Qwen-Max,  Qwen2.5 series, InternLM-2.5 series, Yi-1.5 series, Baichuan-2 series and DeepSeek series.

We integrate both Chinese HellaSwag and HellaSwagPro into the lm-evaluation-harness platform. For the open-source models, we use the default settings of lm-evaluation-harness: do\_sample is set to false and the temperature is set to the default value of the hugging-face library. For the closed-source models, we set the temperature to 0.7. In addition, we set the maximum output length to 1024.

\subsubsection{Prompt Strategy}
Taking into account the influence of language and shot, we design 9 prompting strategies, including Direct, CN-CoT, EN-CoT, CN-XLT and EN-XLT. The last four setups include both zero-shot and few-shot variants.\footnote{
For open-source models, Direct adopts an approach similar to the official implementation of HellaSwag, computing the log-likelihood for each option and selecting the one with the highest log-likelihood. And we report normalized accuracy that accounts for the impact of option length. Other prompting strategies use a generation setup and report accuracy based on exact match.}
\textbf {(1)Direct}: LLMs makes the selection directly without any CoT process.
\textbf{(2)CN-CoT}: LLMs performs CoT in Chinese, regardless of dataset language.
\textbf{(3)EN-CoT}: Similar to CN-CoT, but CoT is conducted in English. 
\textbf{(4)CN-XLT}: LLMs are instructed to first translate English questions and options to Chinese, and then reason in Chinese.
\textbf{(5)EN-XLT}: Similar to CN-XLT, but translates from Chinese dataset to English and reasons in English. 

%\textbf {CN-CoT}: LLMs perform Chinese reasoning and then output the answer and 3 shots are provided.
%\textbf {CN-CoT}: Similar as CNCoTFewShot without any shots.
%\textbf {EN-CoT}: The reasoning process in English is executed and then the answer is output and 3 shots are provided.
%\textbf {CN-XLT}: Inspired by this, we instruct LLMs to translate questions in Chinese and then output the answer after performing reasoning in Chinese too. And 3 shots are provided.
%\textbf {EN-XLT}: Inspired by this, we instruct LLMs to translate questions in Englsih and then output the answer after performing reasoning in Englsih too. Three shots are provided.

\subsubsection{Evaluation metric}

To comprehensively evaluate the robustness of each LLM, we consider four metrics: 
% Original Accuracy (\textbf{OA}), Average Robust Accuracy (\textbf{ARA}), Robust Loss Accuracy (\textbf{RLA}), and  Consistent Robust Accuracy (\textbf{CRA}).
\noindent %
\textbf{- Original Accuracy (OA)} measures accuracy on original problems.
\begin{equation}\label{eq1}
OA=\frac{\sum_{(x, y) \in D} \mathds{1}[L M(x), y]}{|D|}.
\end{equation}
\noindent %
\textbf{- Average Robust Accuracy  (ARA)} represents average accuracy across all variants, gauging overall performance on the robustness tasks.
\begin{equation}\label{eq2}
ARA=\frac{\sum_{\left(x^{\prime}, y^{\prime}\right) \in D_{R}} \mathds{1}\left(L M\left(x^{\prime}, y^{\prime}\right)\right.}{\left|D_{R}\right|}.
\end{equation}

\noindent %
\textbf{- Robust Loss Accuracy (RLA)} is the difference between ARA and OA, indicating performance degradation on robustness data versus original data.
%\begin{tiny}
%\begin{equation}\label{eq3}
%RLA=\frac{\sum_{\left(x^{\prime}, y^{\prime}\right) \in D_{R}} %\mathds{1}\left(L M\left(x^{\prime}, y^{\prime}\right)\right.}{\left|D_{R}\right|}-\frac{\sum_{(x, y) \in D}\mathds{1}[L M(x), y]}{|D|}
%\end{equation}
%\end{tiny}
\begin{equation}\label{eq3}
RLA= OA - ARA.
\end{equation}
\noindent %
\textbf{- Consistent Robust Accuracy (CRA)} shows accuracy when the model correctly answers both original and variant data, reflecting the model do understand the problem.
% consistency in problem-solving.
\begin{equation}\label{eq4}
CRA=\frac{\sum_{x, y, x^{\prime}, y^{\prime}}\mathds{1}[L M(x), y] \cdot \mathds{1}[L M(x^{\prime}), y^{\prime}]}{\left|D_{R}\right|}.
\end{equation}
For all equation above, $D$ denotes the original dataset, where $x$ represents the input question and options, and $y$ represents the correct label, while $D_{R}$ is the robust dataset with $x^{\prime}$ and $y^{\prime}$ representing similar to $x$ and $y$.


\begin{table*}[ht]
\centering
\setlength{\tabcolsep}{5pt}
% \footnotesize
\scalebox{0.6}{
% Please add the following required packages to your document preamble:
% \usepackage{multirow}
% \usepackage[table,xcdraw]{xcolor}
% Beamer presentation requires \usepackage{colortbl} instead of \usepackage[table,xcdraw]{xcolor}
% Please add the following required packages to your document preamble:
% \usepackage{multirow}
% \usepackage[table,xcdraw]{xcolor}
% Beamer presentation requires \usepackage{colortbl} instead of \usepackage[table,xcdraw]{xcolor}
\begin{tabular}{ccccccccccccc}
\hline
\multicolumn{1}{c|}{{ }}& \multicolumn{4}{c|}{Chinese}& \multicolumn{4}{c|}{English}& \multicolumn{4}{c}{AVG}\\ \cline{2-13} 
\multicolumn{1}{c|}{\multirow{-2}{*}{{ Model}}} & { OA(\%)$\uparrow$}& { ARA(\%)$\uparrow$} & {RLA(\%)$\downarrow$}& \multicolumn{1}{l|}{{CRA(\%)$\uparrow$}} & { OA(\%)$\uparrow$}& { ARA(\%)$\uparrow$} & { RLA(\%)$\downarrow$}& \multicolumn{1}{l|}{{CRA(\%)$\uparrow$}} & {OA(\%)$\uparrow$}& { ARA(\%)$\uparrow$} & {RLA(\%)$\downarrow$}& { CRA(\%)$\uparrow$} \\ \hline
\multicolumn{1}{c|}{{ Human}} & 96.41& 97.79& -1.38 & \multicolumn{1}{l|}{92.03}& 95.56& 96.04& -0.48 & \multicolumn{1}{l|}{90.02}& 95.99 & 96.92 & -0.93& 91.03 \\ \hline
\multicolumn{13}{c}{\textit{Close-source LLMs}}\\ 
\multicolumn{1}{c|}{{ GPT-4o}}& { 91.37} & { 81.97} & { 9.40}& \multicolumn{1}{l|}{{ 75.55}} & { \textbf{88.63}} & { \textbf{70.17}} & { \textbf{18.46}} & \multicolumn{1}{l|}{{ \textbf{63.06}}} & { 90.00} & { \textbf{76.07}} & { \textbf{13.93}} & { \textbf{69.31}} \\
\multicolumn{1}{c|}{{ Claude3.5}}& { \textbf{95.37}} & { 80.15} & { 15.22} & \multicolumn{1}{l|}{{ 75.04}} & { 85.11} & { 66.02} & { 19.08} & \multicolumn{1}{l|}{{ 57.20}} & { 90.24} & { 73.09} & { 17.15} & { 66.12} \\
\multicolumn{1}{c|}{{ Gemini-1.5-Pro}}& { 90.62} & { 78.36} & { 12.26} & \multicolumn{1}{l|}{{ 70.48}} & { 87.75} & { 60.74} & { 27.01} & \multicolumn{1}{l|}{{ 58.27}} & { 89.19} & { 69.55} & { 19.63} & { 64.38} \\
\multicolumn{1}{c|}{{ Qwen-Max}}& { 93.50} & { \textbf{84.82}} & { \textbf{8.68}}& \multicolumn{1}{l|}{{ \textbf{78.91}}} & { 87.60} & { 62.61} & { 24.99} & \multicolumn{1}{l|}{{ 59.65}} & { \textbf{90.55}} & { 73.72} & { 16.83} & { 69.28} \\ \hline
\multicolumn{13}{c}{\textit{Chinese open-source LLMs}} \\ 
\multicolumn{1}{c|}{{ Qwen2.5-0.5B}}& { 60.75} & { 45.18} & { \textbf{15.57}} & \multicolumn{1}{l|}{{ 28.70}} & { 49.50} & { 38.21} & { \textbf{11.29}} & \multicolumn{1}{l|}{{ 20.57}} & { 55.13} & { 41.70} & { \textbf{13.43}} & { 24.64} \\
\multicolumn{1}{c|}{{ Qwen2.5-1.5B}}& { 63.25} & { 46.16} & { 17.09} & \multicolumn{1}{l|}{{ 29.89}} & { 56.88} & { 39.57} & { 17.30} & \multicolumn{1}{l|}{{ 23.48}} & { 60.06} & { 42.87} & { 17.20} & { 26.69} \\
\multicolumn{1}{c|}{{ Qwen2.5-3B}}& { 67.50} & { 48.75} & { 18.75} & \multicolumn{1}{l|}{{ 33.79}} & { 61.75} & { 39.98} & { 21.77} & \multicolumn{1}{l|}{{ 25.75}} & { 64.63} & { 44.37} & { 20.26} & { 29.77} \\
\multicolumn{1}{c|}{{ Qwen2.5-7B}}& { 67.63} & { 50.59} & { 17.04} & \multicolumn{1}{l|}{{ 35.62}} & { 65.63} & { 43.93} & { 21.70} & \multicolumn{1}{l|}{{ 30.77}} & { 66.63} & { 47.26} & { 19.37} & { 33.20} \\
\multicolumn{1}{c|}{{ Qwen2.5-14B}} & { 69.00} & { 51.41} & { 17.59} & \multicolumn{1}{l|}{{ 35.84}} & { 68.50} & { 45.20} & { 23.30} & \multicolumn{1}{l|}{{ 32.12}} & { 68.75} & { 48.30} & { 20.45} & { 33.98} \\
\multicolumn{1}{c|}{{ Qwen2.5-32B}} & { 69.75} & { 53.11} & { 16.64} & \multicolumn{1}{l|}{{ 37.54}} & { 70.00} & { 46.10} & { 23.90} & \multicolumn{1}{l|}{{ 32.68}} & { 69.88} & { 49.61} & { 20.27} & { 35.11} \\
\multicolumn{1}{c|}{{ Qwen2.5-72B}} & { \textbf{70.87}} & { \textbf{54.75}} & { 16.12} & \multicolumn{1}{l|}{{ \textbf{39.64}}} & { \textbf{72.00}} & { \textbf{47.75}} & { 24.25} & \multicolumn{1}{l|}{{\textbf{ 35.12}}} & { \textbf{71.44}} & { \textbf{51.25}} & {20.19} & { \textbf{37.38}} \\ \hdashline[0.5pt/5pt]
\multicolumn{1}{c|}{{ Baichuan2-7B}}& { 67.00} & { 46.16} & { 20.84} & \multicolumn{1}{l|}{{ 31.50}} & { 60.62} & { 39.04} & { 21.58} & \multicolumn{1}{l|}{{ 25.21}} & { 63.81} & { 42.60} & { 21.21} & { 28.36} \\
\multicolumn{1}{c|}{{ Baichua2-13B}}& { 69.13} & { 46.98} & { 22.15} & \multicolumn{1}{l|}{{ 33.45}} & { 64.62} & { 38.82} & { 25.80} & \multicolumn{1}{l|}{{ 26.07}} & { 66.88} & { 42.90} & { 23.97} & { 29.76} \\ \hdashline[0.5pt/5pt]
\multicolumn{1}{c|}{{ DeepSeek-7B}} & { 68.13} & { 47.96} & { 20.17} & \multicolumn{1}{l|}{{ 33.30}} & { 63.38} & { 40.39} & { 22.99} & \multicolumn{1}{l|}{{ 26.70}} & { 65.76} & { 44.18} & { 21.58} & { 30.00} \\
\multicolumn{1}{c|}{{ DeepSeek-67B}}& { 71.50} & { 49.21} & { 22.29} & \multicolumn{1}{l|}{{ 35.89}} & { 71.37} & { 40.63} & { 30.75} & \multicolumn{1}{l|}{{ 29.71}} & { 71.44} & { 44.92} & { 26.52} & { 32.80} \\ \hdashline[0.5pt/5pt]
\multicolumn{1}{c|}{{ InternLM2.5-1.8B}}& { 61.62} & { 42.07} & { 19.55} & \multicolumn{1}{l|}{{ 26.99}} & { 55.37} & { 38.46} & { 16.91} & \multicolumn{1}{l|}{{ 22.61}} & { 58.50} & { 40.27} & { 18.23} & { 24.80} \\
\multicolumn{1}{c|}{{ InternLM2.5-7B}}& { 67.25} & { 49.77} & { 17.48} & \multicolumn{1}{l|}{{ 34.57}} & { 69.50} & { 40.89} & { 28.61} & \multicolumn{1}{l|}{{ 29.75}} & { 68.38} & { 45.33} & { 23.04} & { 32.16} \\
\multicolumn{1}{c|}{{ InternLM2.5-20B}} & { 67.37} & { 48.08} & { 19.29} & \multicolumn{1}{l|}{{ 33.21}} & { 73.62} & { 41.11} & { 32.51} & \multicolumn{1}{l|}{{ 31.23}} & { 70.50} & { 44.60} & { 25.90} & { 32.22} \\ \hdashline[0.5pt/5pt]
\multicolumn{1}{c|}{{ Yi-1.5-6B}} & { 67.00} & { 49.59} & { 17.41} & \multicolumn{1}{l|}{{ 34.27}} & { 64.38} & { 39.37} & { 25.01} & \multicolumn{1}{l|}{{ 26.62}} & { 65.69} & { 44.48} & { 21.21} & { 30.45} \\
\multicolumn{1}{c|}{{ Yi-1.5-9B}} & { 68.50} & { 50.18} & { 18.32} & \multicolumn{1}{l|}{{ 35.55}} & { 66.37} & { 39.58} & { 26.79} & \multicolumn{1}{l|}{{ 27.48}} & { 67.44} & { 44.88} & { 22.56} & { 31.52} \\
\multicolumn{1}{c|}{{ Yi-1.5-34B}}& { 71.00} & { 52.23} & { 18.77} & \multicolumn{1}{l|}{{ 38.09}} & { 71.00} & { 40.75} & { 30.25} & \multicolumn{1}{l|}{{ 29.91}} & { 71.00} & { 46.49} & { 24.51} & { 34.00} \\ \hline
\multicolumn{13}{c}{\textit{English open-source LLMs}} \\ 
\multicolumn{1}{c|}{{ Llama3-8B}} & { 59.13} & { 46.62} & { 12.51} & \multicolumn{1}{l|}{{ 28.23}} & { 66.25} & { 40.21} & { 26.04} & \multicolumn{1}{l|}{{ 27.34}} & { 62.69} & { 43.42} & { 19.27} & { 27.79} \\
\multicolumn{1}{c|}{{ Llama3-70B}}& { 65.75} & { 48.63} & { 17.12} & \multicolumn{1}{l|}{{ 32.70}} & { \textbf{72.50}} & { 41.27} & { 31.23} & \multicolumn{1}{l|}{{\textbf{ 30.63}}} & {\textbf{ 69.13}} & { 44.95} & { 24.18} & { 31.67} \\ \hdashline[0.5pt/5pt]
\multicolumn{1}{c|}{{ Mistral-7B-v0.2}} & { 57.75} & { 46.25} & { \textbf{11.50}} & \multicolumn{1}{l|}{{ 27.57}} & { 67.50} & { \textbf{41.52}} & { 25.98} & \multicolumn{1}{l|}{{ 28.93}} & { 62.63} & { 43.88} & { 18.74} & { 28.25} \\
\multicolumn{1}{c|}{{ Mixtral-8x7B-v0.1}} & { 63.62} & { 46.80} & { 16.82} & \multicolumn{1}{l|}{{ 30.82}} & { 69.75} & { 41.21} & { 28.54} & \multicolumn{1}{l|}{{ 29.39}} & { 66.69} & { 44.01} & { 22.68} & { 30.11} \\
\multicolumn{1}{c|}{{ Mixtral-8x22B-v0.1}}& { 66.00} & {\textbf{ 50.73}} & { 15.27} & \multicolumn{1}{l|}{{ \textbf{34.32}}} & { 72.12} & { 41.25} & { 30.87} & \multicolumn{1}{l|}{{ 30.61}} & { 69.06} & { \textbf{45.99}} & { 23.07} & { \textbf{32.47}} \\ \hdashline[0.5pt/5pt]
\multicolumn{1}{c|}{{ Gemma-2-2B}}& { 61.88} & { 45.38} & { 16.51} & \multicolumn{1}{l|}{{ 29.02}} & { 59.62} & { 39.13} & { \textbf{20.50}} & \multicolumn{1}{l|}{{ 24.88}} & { 60.75} & { 42.25} & {\textbf{ 18.50}} & { 26.95} \\
\multicolumn{1}{c|}{{ Gemma-2-9B}}& { \textbf{69.13}} & { 46.75} & { 22.38} & \multicolumn{1}{l|}{{ 33.29}} & { 64.88} & { 39.80} & { 25.08} & \multicolumn{1}{l|}{{ 26.91}} & { 67.01} & { 43.28} & { 23.73} & { 30.10} \\
\multicolumn{1}{c|}{{ Gemma-2-27B}} & { 63.38} & { 48.52} & { 14.86} & \multicolumn{1}{l|}{{ 31.96}} & { 71.88} & { 40.91} & { 30.97} & \multicolumn{1}{l|}{{ 30.25}} & { 67.63} & { 44.71} & { 22.92} & { 31.11} \\ \hline
\end{tabular}
}
\caption{TODO: bolded is not result. Results of existing LLMs on our HellaSwag-Pro dataset using \textbf{Direct} prompt. ``AVG'' indicates the average performance of each model on Chinese and English parts of the dataset.
The best results for each metric in each model category are \textbf{bolded}. }
\label{tab:main experiment.}
\end{table*}

\subsection{Model Performance (RQ1)}
\paragraph{Overall Performance}
Table \ref{tab:main experiment.} provides a comprehensive evaluation of various LLMs across four performance metrics\footnote{The results of instruct/chat models of Qwen2.5, Llama3 and Mixtral latest series are shown in Appendix.}. The main observations are as follow:
\begin{itemize}[leftmargin=*,topsep=0pt]
% \setlength{}{0}
    \item Upon evaluating all available models, we found that all performed well in overall accuracy (e.g., GPT-4 scored 90.00 in AVG OA, Claude 3.5 scored 90.24 in AVG OA). However, all models struggled with variations of the questions, as evidenced by a positive RLA value for each model. In contrast, humans received a negative RLA value, suggesting that the question variants were not more challenging than the originals. This disparity further illustrates that current LLMs lack a true understanding of the reasoning process and can easily be misled by question variants.
    \item When comparing open-source and close-source models, the close-source models demonstrate stronger capabilities in both OA and ARA scores, similar to most existing benchmarks. Overall, the RLA values for close-source models are also smaller, indicating that they are more robust in commonsense reasoning tasks compared to open-source models.
    \item When we compare models within the same series (e.g., Qwen, Llama), we observe that larger models often achieve higher scores on OA, ARA, and CRA. However, they are also more susceptible to variations, i.e., they have higher RLA values, a phenomenon particularly evident in English datasets. We attribute this phenomenon to the fact that larger models, compared to smaller ones, may have memorized more data, allowing them to rely on memorization to solve some problems more easily and making them more prone to the influence of variations~\cite{}.
\end{itemize}
% 1. When evaluating all available models, We find although 
% 2. When comparing the opensource LLMs and close source LLMs, 
% 3. When looking into each serious details
% \noindent
% \textbf{Overall Model Performance.}
% 1. close-source > open-source 2. the large the better 3. all have a performance decline when meeting varients.

% To evaluate the performance of various models, we observed patterns consistent with current mainstream trends: closed-source models generally outperform open-source models across metrics. 
% For instance, the closed-source model GPT-4o achieved scores of 90.00 in OA, 76.07 in ARA, and 69.31 in CRA, whereas the open-source model Qwen2.5-72B scored 71.44, 51.25, and 37.38, respectively. 
% Furthermore, within each model series, performance tends to improve with larger model sizes. 
% Nevertheless, even the strongest closed-source models struggle with variations in questions, as indicated by positive values in RLA for all models. In contrast, human performance yields a negative RLA value, highlighting that current LLMs do not genuinely grasp the reasoning process and are prone to falling into traps set by question variants. 
% This suggests that there is still significant room for improvement in developing models that can robustly understand and reason through complex linguistic challenges.
% It reveals a consistent pattern across Chinese, English, and average scores, with close-sourced LLMs generally outperforming open-sourced models. 
% However, all models exhibit a significant drop in performance when faced with robust variants, as indicated by RLA and CRA. Among closed-source models, GPT-4o demonstrates the highest ARA of 76.07\% in average scores, demonstrating its overwhelming superiority. Among open-sourced models, larger models tend to perform better, with Qwen2.5-72B achieving the highest OA (71.44\%) and ARA (51.25\%) in the average scores. However, even these top performers still struggle with robustness, as evidenced by the substantial RLA of 13.93\% for GPT-4o and 20.19\% for Qwen2.5-72B. Interestingly, some English open-sourced models, such as Llama3-70B and Mixtral-8x22B-v0.1, show competitive performance in English tasks but lag in Chinese tasks, highlighting the importance of language-specific training.

% \noindent
% \textbf{Chinese Models vs English Models.}
% Chinese models generally demonstrate higher OA in Chinese tasks compared to English tasks, with Qwen-Max achieving 93.50\% OA in Chinese versus 87.60\% in English. Conversely, English models tend to perform better in English tasks, exemplified by Llama3-70B's 72.50\% OA in English compared to 65.75\% in Chinese. 
% However, both Chinese and English models exhibit important drops in ARA across languages, indicating challenges in maintaining performance when faced with variations. This trend suggests that while models may excel in their primary language, they struggle with robustness across linguistic boundaries. 
% Notably, larger models tend to achieve higher ARA scores but also experience more substantial RLA, as seen with Qwen2.5-0.5B (41.70\% ARA, 13.43\% RLA in total) and Qwen2.5-72B (51.25\% ARA, 20.19\% RLA in total). 
% This pattern indicates that while increased model size enhances overall performance, it doesn't necessarily improve robustness proportionally. 
% The discrepancy between OA and ARA across languages underscores the need for improved cross-lingual robustness in language models, particularly as they scale in size and capability.


% \noindent
% \textbf{Comparison between Chinese and English datasets.}
% Generally, models demonstrate higher accuracy on the Chinese dataset compared to the English one, as evidenced by the consistently higher OA, ARA and CRA scores. For instance, GPT-4o achieves an OA of 91.37\%, an ARA of 81.97\% , an CRA of 75.55\% on the Chinese dataset, compared to 88.63\% and 70.17\% respectively on the English dataset. This trend is observed across most models, suggesting that the Chinese dataset is easier than English one. Moreover, the RLA values are typically lower for Chinese, indicating smaller performance drops when dealing with robust variants of Chinese questions. For example, Qwen-Max shows an RLA of 8.68\% for Chinese versus 24.99\% for English, highlighting a more consistent performance in Chinese. The CRA scores further reinforce this observation, with models generally maintaining higher consistency in correct answers for both original and variant Chinese questions.
% We attribute this phenomenon to the fact that blablabla

\noindent
\textbf{Reasoning Transferable Capability.}
% 为了进一步
To further analyze whether the model can transfer reasoning ability from the original question to its variant, Figure \ref{consis} presents the distribution of model performance on the original question and variant pairs. For all models, the pairs of (HellaSwag \ding{51} HellaSwag-Pro \ding{55}) occupy a significant proportion, indicating a challenge in transferring reasoning capabilities for current LLMs to more complex scenarios. Looking deeply, closed-source models like GPT-4 and Qwen-Max achieve around a 69\% portion of (HellaSwag \ding{51} HellaSwag-Pro \ding{51}) and a 3\% portion of (HellaSwag \ding{55} HellaSwag-Pro \ding{55}), while in contrast, open-source models struggle with around a 30\% portion of (HellaSwag \ding{51} HellaSwag-Pro \ding{51}) and a 20\% portion of (HellaSwag \ding{55} HellaSwag-Pro \ding{55}), further indicating the robustness of reasoning abilities in closed-source models.
% If a model can get both the original question and the variant right, we consider it to have transferable reasoning ability. Table \ref{consis} presents the distribution of model performance on the original question and variant pairs. Among all models, the pairs of (HellaSwag \ding{51}HellaSwag-Pro \ding{55}) account for a considerable proportion, i 
% The closed-source models like GPT-4o and Qwen-Max achieve around 69\% portion of (HellaSwag \ding{51}HellaSwag-Pro \ding{51}) and 3\% portion of (HellaSwag \ding{55} HellaSwag-Pro \ding{55}), indicating stronger reasoning transfer ability than other models. In contrast, open-source models struggle more, with around 30\% portion of (HellaSwag \ding{51}HellaSwag-Pro \ding{51}) and 20\% portion of (HellaSwag \ding{55} HellaSwag-Pro \ding{55}). 
% A notable trend is observed among the Qwen2.5 series, where increasing model size from 7B to 72B parameters correlates with improved performance on correct answers for both datasets (33.20\% to 37.38\%) and decreased failure rates (17.69\% to 14.7\%). It underscores the importance of model size in commonsense reasoning tasks.

\begin{figure}[t]
\centering
\setlength{\abovecaptionskip}{0.1cm}
\setlength{\belowcaptionskip}{0cm}
\includegraphics[width=\linewidth,scale=1.00]{images/consis.pdf}
\caption{Analysis of the transferable ability of model reasoning based on question pair performance. The green part, where both the original and the variant data are right, represents the transferable performance of model reasoning.}
\label{consis}
\vspace{-15pt}
\end{figure}

\begin{figure*}[ht]
\centering
\setlength{\abovecaptionskip}{0.1cm}
\setlength{\belowcaptionskip}{0cm}
\includegraphics[width=\linewidth,scale=1.00]{images/xing.pdf}
\caption{The impact of different few-shot prompts on model performance. With - as the separator, the first two parts of the legend represent the prompt name, and the third part represents the language of the dataset.}
\label{xing}
\vspace{-15pt}
\end{figure*}

\begin{figure}[ht]
\centering
\setlength{\abovecaptionskip}{0.1cm}
\setlength{\belowcaptionskip}{0cm}
\includegraphics[width=1.05\linewidth,scale=1.05]{images/zhu.pdf}
\caption{The RLA Distribution for 7 variants of commonsense reasoning. Parts below the 0 axis indicate that the model’s performance on the variant is improved compared to the original problem.}
\label{fig:zhu}
\vspace{-15pt}
\end{figure}


\subsection{Variant Analysis (RQ2)}
To further analyze the impact of different variants, we assessed the contribution of each variant to the RLA score. A higher contribution indicates that the model is more likely to make errors in that type. Figure~\ref{fig:zhu} presents the overall results, and the key observations are as follows:
\begin{itemize}[leftmargin=*]
    \item For problem restatement, causal inference, and sentence ordering, these three categories are the least challenging. Almost all models, particularly the close-source and Qwen series models, perform well on these variants, indicating that current LLMs can effectively handle these forms and we do not pay more attention on this kind of varients.
    \item For reverse conversion and critical testing, these two varients each contribute about 10\% to the RLA score. This indicates that current LLMs struggle to fully generalize to these simple scenarios, possibly because these types of questions are not commonly encountered, and reaserchers should pay some attention to this type of varients.
    \item For negative transformation and scenario refinement, this are the two most difficult tasks, with negative transformation being particularly challenging. For almost all models, these two varients accounts for more than 50\% of the RLA score. This may be due to intuitively counterintuitive questions—such as the use of "will not"  or counterfactual scenarios in scenario refinement. These setups are less common in LLM training data and cannot be easily tackled through memory alone. Only those LLMs which truely understand the question could answer the varient correctly, wihch better reflect the true performance of the model.. In the future, researchers should focus more on enhancing LLM's capability to address such types of questions.
\end{itemize}

% 1. Problem restCausal Inference 
% To further analysis the impact of different varients, we further 
% Figure \ref{fig: zhu} presents a comprehensive analysis of various LLMs' performance across different variant types. Negative transformation emerges as the most challenging task for all models, with scores consistently above 50.00\% and peaking at 78.38\% for Gemini-1.5-Pro. Conversely, problem restatement appears to be the least challenging, with most models scoring in the negative range. Intriguingly, smaller models like Qwen2.5-0.5B demonstrate unexpected strengths in certain areas, such as sentence sorting (7.75\%), outperforming some larger counterparts. A detailed analysis of each variant type follows.

% \noindent
% \textbf{Causal inference.} In this category, scores vary widely from -4.73\% for Qwen-Max to 12.25\% for Baichuan2-13B, illustrating differing degrees of sensitivity to causal reasoning among the models. Smaller models, such as Qwen2.5-0.5B and Qwen2.5-1.5B, achieve better scores, indicating relatively stronger robustness in causal reasoning. Conversely, larger models, like Baichuan2-13B, have higher scores, suggesting greater sensitivity to the challenges of inferring causality.

% \noindent
% \textbf{Critical testing.} Larger models, including Qwen2.5-72B and DeepSeek-67B, exhibit higher RLA scores of 30.50\% and 31.37\%, respectively, suggesting increased sensitivity when dealing with incomplete key information. In contrast, GPT-4o achieves the lowest score, highlighting its superior robustness in critical reasoning. This trend indicates that more complex models might struggle to handle incomplete contexts, underscoring potential areas for improvement in sophisticated architectures.

% \noindent
% \textbf{Negative transformation.} This aspect remains consistently challenging for all models, with scores ranging from 48.88\% to 78.38\%. Advanced commercial models like Gemini-1.5-Pro and Claude-3.5 also score higher (78.38\% and 76.43\%, respectively), indicating a prevalent sensitivity issue in reasoning processes when handling negations, irrespective of model size or architecture.

% \noindent
% \textbf{Problem restatement.} The negative values in this category for nearly all models suggest it is not particularly challenging. This is surprising, given that previous models were quite sensitive to sentence representation.

% \noindent
% \textbf{Reverse conversion.} This variation, which involves swapping the roles of the question and answer, seems to specifically impact larger models. For example, Qwen2.5-72B and DeepSeek-67B exhibit higher RLA scores of 24.38\% and 27.43\%, respectively, indicating heightened sensitivity to reverse reasoning compared to their performance on original questions.

% \noindent
% \textbf{Scenario refinement.} The scores range from 16.06\% for Gemma-2-2B to 32.56\% for Qwen2.5-72B, with larger models displaying more sensitivity in adapting to counterfactual predictions. This suggests that larger models may rely more heavily on general commonsense rather than flexibly adapting to specific contexts. Consequently, increased model complexity might adversely affect adaptability to scenario changes, underscoring the need for enhanced flexibility in advanced models.

% \noindent
% \textbf{Sentence sorting.} This category exhibits the most varied results across models. Some larger models like DeepSeek-67B and InternLM2.5-20B display higher scores (26.69\% and 26.68\%), indicating sensitivity, while others like Qwen2.5-72B and Gemini-1.5-Pro excel with lower scores (-9.88\% and -1.07\%, respectively). This suggests that sentence sorting ability may depend more on specific training approaches rather than being solely contingent on model size.


\subsection{Prompt Robustness (RQ3)}
% To investigate how prompt  influence our benchmark, we apply sereral prompt strategy on our datasets and showcase the average performance of all models on different kind of prompt strategies.
% Table~\ref{prompt} illustrates the final results. For both Chinese and English datasets, CN LLMs achieve the highest performance using CN-CoT-Few-Shot, followed closely by EN-CoT-Few-Shot, with overall performance scores of 67.36\% and 67.03\%, respectively. In contrast, English LLMs perform best with the EN-CoT-Few-Shot, reaching 67.55\% on the Chinese dataset and 60.36\% on the English dataset.
% Contrary to previous findings, translating the dataset to the model's advantage language before performing reasoning does not enhance performance. Moreover, Figure~\ref{xing} also shows the similar phenomenon. Conducting CoT reasoning in the model’s advantage language generally leads to better outcomes compared to Direct. Additionally, increasing the number of shots consistently improves performance across most configurations, highlighting the benefits of exposing models to multiple examples. 
To explore the impact of various prompt strategies on our benchmarks, we evaluated several approaches across our datasets and present the average performance of all models using different prompting techniques. Table~\ref{prompt} summarizes the results. For both Chinese and English datasets, Chinese LLMs performed best with the CN-CoT-Few-Shot strategy, followed closely by EN-CoT-Few-Shot, achieving overall scores of 67.36\% and 67.03\%, respectively. Conversely, English LLMs showed optimal performance with the EN-CoT-Few-Shot approach, attaining 67.55\% on the Chinese dataset and 60.36\% on the English dataset.
Besides, translating datasets into the model's native language before reasoning did not enhance performance. This phenomenon is further illustrated in Figure~\ref{xing}. Conducting CoT reasoning in the model's native language generally yields better results compared to direct reasoning. Furthermore, increasing the number of examples (shots) consistently boosts performance across most configurations, emphasizing the advantages of exposing models to multiple examples.
% Overall, the interaction between question language, prompt language, and the number of shots underscores the importance of aligning these factors to optimize task performance and robustness in LLMs.



% Please add the following required packages to your document preamble:
% \usepackage{multirow}
% Please add the following required packages to your document preamble:
% \usepackage{multirow}
\begin{table}[t]
\setlength{\tabcolsep}{8pt}
% \footnotesize
\scalebox{0.65}{
\begin{tabular}{c|l|lll}
\hline
\multicolumn{1}{l|}{Dataset}  & Prompt  & CN LLMs & EN LLMs &  LLMs \\ \hline
\multirow{7}{*}{\begin{tabular}[c]{@{}c@{}}Chinese\\ HellaSwag-Pro\end{tabular}} & Direct  & 48.95& 41.16& 45.06  \\
& CN-CoT-Few  & \textbf{71.04}& 51.90& 61.47  \\
& EN-CoT-Few  & 70.95& \textbf{67.55}& \textbf{69.25}  \\
& EN-XLT-Few  & 41.48& 28.69& 35.09  \\
& CN-CoT-Zero & 44.82& 23.89& 34.36  \\
& EN-CoT-Zero & 45.38& 31.39& 38.39  \\
& EN-XLT-Zero & 28.57& 12.93& 20.75  \\ \hline
\multirow{7}{*}{\begin{tabular}[c]{@{}c@{}}English\\ HellaSwag-Pro\end{tabular}} & Direct  & 47.46& 40.66& 44.06  \\
& CN-CoT-Few  & \textbf{63.67}& 47.24& 55.46  \\
& EN-CoT-Few  & 63.12& \textbf{60.36}& \textbf{61.74}  \\
& CN-XLT-Few  & 48.77& 16.61& 32.69  \\
& CN-CoT-Zero & 34.89& 18.25& 26.57  \\
& EN-CoT-Zero & 42.41& 31.03& 36.72  \\
& CN-XLT-Zero & 16.36& 11.22& 13.79  \\ \hline
\multirow{9}{*}{HellaSwag-Pro}& Direct  & 48.21& 40.91& 44.83  \\
& CN-CoT-Few  & \textbf{67.36}& 49.57& 58.46  \\
& EN-CoT-Few  & 67.03& \textbf{63.95}& \textbf{65.49}  \\
& CN-XLT-Few  & 59.91& 34.26& 47.08  \\
& EN-XLT-Few  & 52.30& 44.52& 48.41  \\
& CN-CoT-Zero & 39.86& 21.07& 30.46  \\
& EN-CoT-Zero & 43.90& 31.21& 37.55  \\
& CN-XLT-Zero & 30.59& 17.55& 24.07  \\
& EN-XLT-Zero & 35.49& 21.98& 28.74  \\ \hline
\end{tabular}
}
\caption{Average ARA of all open-source models on different prompts. CN-LLMs contains 17 LLMs, and EN-LLMs contains 7 LLMs. The bast results for each dataset are \textbf{bolded}.}
\label{prompt}
\end{table}




\section{Concluding Remarks}
In this paper, we proposed a novel approach utilizing multimodal LLMs to generate gesture-aware speech recognition transcripts for patients with language disorders. Our framework integrates verbal speech and iconic gestures, enabling the generation of enriched transcripts that capture the latent meaning conveyed through both modalities. Through extensive experimentation, we demonstrated that the proposed method effectively contextualizes incomplete or disfluent speech by incorporating gesture information, leading to more accurate and meaningful representations of the speaker's intent. These findings highlight the potential of our approach to significantly contribute to the field of speech and language therapy, offering innovative tools that can enhance the quality of life for individuals with language disorders by facilitating better communication and assessment methods.

\subsection{Ethical Statement} 
Our dataset was obtained from AphasiaBank with the approval of the Institutional Review Board (IRB) and adheres to the data sharing guidelines set by TalkBank\footnote{https://talkbank.org/share/ethics.html}. This includes complying with the Ground Rules for all TalkBank databases, which are based on the American Psychological Association Code of Ethics~\cite{american2002ethical}.

\subsection{Limitation \& Future Work} 
%This study represents a preliminary investigation into using multimodal LLMs to generate gesture-aware speech recognition transcripts. 
While the results are promising, we recognize several limitations and outline our plans to extend this work further.

One primary limitation is the absence of a definitive ground truth for quantitative evaluation. Since our model generates transcripts by synthesizing speech and gesture data from scratch, traditional benchmarks, such as comparisons with standard speech recognition outputs, are insufficient. Moreover, existing original transcripts lack gesture annotations, making direct comparisons challenging. In future work, we aim to address this gap by collaborating with certified pathologists to conduct qualitative assessments, such as A-B preference tests, to evaluate the effectiveness of gesture-enriched transcripts in accurately conveying the speaker's intentions.

To support quantitative evaluations, we plan to develop novel metrics that assess transcript quality, including grammar accuracy, semantic consistency, and the integration of multimodal information. Such metrics will provide a more objective basis for assessing our model's performance and facilitate comparisons with other multimodal and unimodal approaches.

Another limitation of this study is its focus on structured gestures from a specific task, the Peanut Butter Sandwich Task. While this task offers a controlled context for testing our approach, it does not encompass the diversity of gestures and communication patterns seen in everyday scenarios. As part of our future work, we plan to expand the scope of our model to include tasks such as the Cinderella Story Recall Task~\cite{bird1996cinderella}, which involves unstructured and complex narrative gestures. This expansion will allow us to evaluate the adaptability and robustness of our model in handling varied linguistic and gestural contexts.

In summary, while this study establishes a strong foundation for gesture-aware speech recognition, we aim to refine and extend our methods through collaborative qualitative evaluations, the development of robust quantitative metrics, and broader task applications. These efforts will ensure that our approach continues to evolve, ultimately contributing to more effective communication tools and interventions for individuals with language disorders.





% Bibliography entries for the entire Anthology, followed by custom entries
%\bibliography{anthology,custom}
% Custom bibliography entries only
\bibliography{custom}

\clearpage

\appendix


\section{Related Works}
\label{sec:related_work}

\subsection{Dataset Adaptation in Education}
Within the realm of education, image retrieval in education has distinct characteristics, as images often reflect the teaching intentions of educators. This facilitates the rapid and accurate alignment of visual content with teaching materials, thereby reducing educators' preparation workload and enhancing the precision of learning data. While existing researches has focused on classifying educational data \cite{choi2020ednet}, they often encounter constraints. Due to the complexity of incorporating an expansive range of teaching scenarios in STEM education and the scarcity of data, numerous studies often narrow the scope to a limited set of subject applications \cite{hendrycks2021measuring,pal2022medmcqa} or to a limited set of teaching strategy retrievals \cite{kwon2024biped,welbl2017crowdsourcing}.

\begin{figure*}[!t]
    \centering
    \includegraphics[width=\linewidth]{imgs/data.pdf}
    \vspace{-7mm}
    \caption{\textbf{The SER Dataset} contains 24,000+ text captions and their corresponding queries with various styles, including Natural, Sketch, Art, Low-Resolution~(Low-Res) images and audio clips from different STEM subjects.}
    \vspace{-5mm}
    \label{fig:data_sample}
\end{figure*}


There is a considerable variation across existing STEM education datasets regarding its specific composition. Many datasets are cluttered with irrelevant or invalid data, lack comprehensive coverage of specialised content, and suffer from quality assurance issues \cite{patrinos2018global}. 
Although STEM education datasets are assembled from interactions between learners and large language models \cite{hou2024eval, wang2024large}, they are generally not well-suited for use by educators and learners across multiple domains. Furthermore, creating a precise and professional data retrieval repository for the educational domain requires efficient retrieval algorithms as support \cite{alzoubi2024enhancing}. To ensure efficient retrieval and usability in STEM education scenarios, we construct the SER dataset, which includes multiple query styles to enhance retrieval diversity.
%There are datasets assembled from interactions between learners and large models\cite{}\cite{}. 
%However, creating a precise and professional data retrieval repository for the educational domain requires grounding it in authentic teaching scenarios.
%Additionally, most datasets and retrieval methods are not well-suited for use by educators and learners across multiple domains. 
%To ensure efficient retrieval and usability, image-text alignment and voice input play pivotal roles.

\subsection{Multi-task Learning}

In STEM education, the multi-style retrieval model needs to leverage multi-task learning to align features and learning across different modal samples. Multi-task learning refers to the simultaneous training and optimization of multiple related tasks within a single model \cite{9392366}. By sharing parameters and representations across functions, it improves overall performance. Compared to other transfer learning methods, including domain adaptation \cite{farahani2021brief} and domain generalization \cite{zhou2022domain}, multi-task learning annotates data and achieves CLIP-level model fine-tuning and convergence, the data of each task in multi-task learning is well-labeled. 

Overall, multi-task learning introduces a new tool for STEM education practitioners that may help meet requirements, especially if speed and efficiency are preferred over performance. While many recent multi-task learning employ two clusters of contemporary techniques, hard parameter sharing and soft parameter sharing \cite{ruder2017overview}. In hard parameter sharing, most or all of the parameters in the network are shared among all tasks \cite{kokkinos2017ubernet}. In soft parameter sharing,the models are tied together either by information sharing or by requiring parameters to be similar \cite{yang2016trace}. Consequently, our Uni-Retrieval adopts a blended multi-task learning paradigm, adopt both hard and soft  parameter in different styles of tasks. Building upon successul multi-task learning method for CLIP, such as CoCoOP \cite{zhou2022conditional}, MaPLe \cite{khattak2023maple}, and FreestyleRet \cite{li2025freestyleret}, our study leverages these techniques to strengthen domain adaptation and multi-task learning. 

\begin{figure*}[!t]
  \centering
   \includegraphics[width=\linewidth]{imgs/categories.pdf}
   \vspace{-6mm}
   \caption{\textbf{Concept distribution of our SER dataset}. Our dataset exhibits a diverse distribution on different concept domains.}
   \label{fig:distribution}
   \vspace{-3mm}
\end{figure*}

\subsection{Query-based Retrieval}
Existing work in Query-based Image Retrieval (QBIR) primarily includes content-based image retrieval \cite{chen2022deep}, text-based image retrieval \cite{li2011text}, and multi-modal retrieval \cite{neculai2022probabilistic}. In content-based image retrieval, the visual features of images are directly utilized for retrieval. However, its reliance on fixed content and location makes it relatively inflexible in capturing diverse user intents \cite{lee-etal-2024-interactive}. Alternative methods like sketching \cite{chowdhury2022fs, chowdhury2023scenetrilogy} and scene graph construction \cite{johnson2015image} enable the retrieval of abstract images that are hard to describe verbally, though they lack the intuitive ease of natural language-based retrieval. In text-based image retrieval, enhancements to text queries often involve indicating content structure. However, these approaches are either restricted by closed vocabularies \cite{mai2017spatial, kilickaya2021structured} or face substantial challenges \cite{li2017generating} in deriving structures from natural language descriptions. Recent multi-modal approaches, such as cross-modal scene graph-based image-text retrieval \cite{wang2020cross} and joint visual-scene graph embedding for image retrieval \cite{belilovsky2017joint}, still depend on word embeddings and image features.

Despite advancements in QBIR, challenges including the semantic gap that can lead to inaccurate retrieval results, high computational complexity and resource costs for large-scale image databases, and the high cost of obtaining quality data annotations \cite{li-etal-2024-generative}. The application of QBIR to educational resource retrieval is promising but has been hindered by the complexity of educational discourse, the limitations of educational databases, and the associated costs \cite{zhou-etal-2024-vista}.  Our query model effectively combines multi-modal retrieval methods, integrating audio and natural language with multi-style image inputs. The former enables natural and rapid expression of content, while the latter facilitates accurate and intuitive image localization, enhancing educational data retrieval.

\subsection{Prompt Tuning}

Prompt tuning \cite{brown2020language} was first proposed in natural language processing~(NLP) and has been an efficient approach that bridges the gap between pre-trained language models and downstream tasks \cite{li2023weakly}. Prompt tuning leverages natural language prompts to optimize the language model’s ability to understand tasks, which demonstrates exceptional performance in few-shot and zero-shot learning. 
Recent studies have focused on optimizing various components of prompt tuning, such as  prompt generation, continuous prompt optimization \cite{prompt3}, and adapting to large-scale models through methods like in-context learning \cite{dong2024survey}, instruction-tuning \cite{wang2024pandalm}, and chain-of-thought \cite{prompt4}. For example, \citet{lester2021power} leverage soft prompts to condition frozen language models to enhance the performance of specific downstream tasks. \citet{long2024prompt} propose an adversarial in-context learning algorithm, which leverages adversarial learning to optimize task-related prompts.

Furthermore, prompt tuning has gradually become a pivotal technique in computer vision \cite{shen2024multitask}, enabling efficient adaptation of pre-trained models to diverse tasks. Notable methods include visual prompt tuning for classification \cite{jia2022visual}, learning to prompt for continual learning \cite{wang2022learning}, context optimization and conditional prompt learning for multi-modal models \cite{zhou2022conditional}, and prompt-based domain adaptation strategies \cite{ge2023domain}.
For example, \citet{nie2023pro} introduce the pro-tuning algorithm for learning task-specific vision prompts, applied to downstream task input images with the pre-trained model remaining frozen.
\citet{shen2024multitask} leverage cross-task knowledge to optimize prompts, thereby enhancing the performance of vision-language models and avoiding the need to independently learn prompt vectors for each task from scratch.
\citet{cho2023distribution} introduce distribution-aware prompt tuning for vision-language models, optimizing prompts by balancing inter-class dispersion and intra-class similarity.
MaPLe \cite{khattak2023maple} further transfers text features to the visual encoder during prompt tuning to avoid overfitting. These approaches leverage learnable prompts to enhance model performance across various applications. Despite significant advancements in previous research, challenges remain in extracting semantic features from style-diversified images and optimizing templates within cont.inuous prompt tuning. 
In this study, we employ both NLP and visual prompt tuning to optimize STEM educational content retrieval, enhancing retrieval accuracy and efficiency by adjusting prompt tokens.
%VPT vision prompt tuning

\begin{figure}[!tbp]
  \centering
   \includegraphics[width=\linewidth]{imgs/pipeline.pdf}
   \vspace{-5mm}
   \caption{\textbf{The pipeline of Uni-Retrieval.} The image-text/audio pairs are input into their respective modality encoders. During the training procedure, contrastive learning is applied between the modality features of the positive samples~(image-text/audio pairs) and the negative samples. During the inference procedure, the model calculates the similarity between the modality features of the query and the embeddings stored in the database. The retrieved results are ranked, and the performance is evaluated using R@1/R@5 as metrics.}
   \label{fig:pipeline}
   \vspace{-5mm}
\end{figure}

\section{Motivation and Scenarios}
In practical teaching scenarios, teachers often encounter the need for precise image retrieval, such as searching for hand-drawn sketches, student-created artistic images, blurry blackboard drawings captured from a distance, classroom photographs of physical objects, or images from textbooks. However, current retrieval models predominantly focus on text-natural image queries, overlooking the diverse query styles common in educational contexts. This limitation makes it challenging for teachers to efficiently identify and retrieve educational images or texts tailored to diverse teaching scenarios, such as accurately setting learning contexts, articulating key teaching points, presenting instructional materials, and quickly locating supplementary resources.

Our proposed method enables teachers to query various styles' answers with a range of retrieval approaches, including text, image, audio, or combinations of these modalities. This approach ensures fast and convenient retrieval, significantly reducing preparation time for teaching. Once teachers input their queries, our Uni-Retrieval system employs contrastive learning to compare images and text, calculating similarities based on attributes like objects, shapes, quantities, and orientations. The system ranks all database entries by similarity, outputting the top-1 or top-5 results to identify the most relevant and accurate teaching resource images, as illustrated in Fig.~\ref{fig:pipeline}. This approach empowers teachers to manage complex and dynamic teaching scenarios effortlessly, enhancing the clarity and effectiveness of STEM education.



\section{Experiments}
\label{supsubsec:Experiment Settings}

% \subsection{Dataset Construction Details}
% \label{supsec:dataset construction details}

In the database, the texts, their corresponding four images and audio structures for each dataset can share a single index, significantly reducing query time. All images and text are preprocessed using pretrained models to extract features, which are stored as embeddings. This approach eliminates the need for repeated feature extraction during use, saving time and reducing computational overhead, improving the efficiency of the retrieval system.

For the dataset selection, we choose four another datasets except our SER dataset, including the DSR dataset \cite{li2025freestyleret}, the ImageNet-X dataset, the SketchCOCO dataset \cite{gao2020sketchycoco} and the DomainNet dataset \cite{peng2019moment}. We use internVL-1.5 \cite{Chen_2024_CVPR} to annotate the paint/sketch caption for the SketchCOCO and the DomainNet dataset. For the model in the prototype learning module, we choose the VGG \cite{vgg} as the feature extractor. For the baseline selection, we apply two cross-modality pre-trained models (CLIP \cite{clip}, BLIP \cite{blip}), two multi-modality pre-trained models~(LanguageBind \cite{languagebind}, Unified-IO2 \cite{uio2}), two style retrieval models (SceneTrilogy \cite{scenetrilogy}, FashionNTM \cite{fashionntm}), four most recent cross-modality prompt learning models (VPT \cite{jia2022visual}, CoCoOP \cite{zhou2022conditional}, MaPLe \cite{khattak2023maple}, FreestyleRet \cite{li2025freestyleret}), and two database-driven retrieval models (GASKN \cite{GASKN}, MKG \cite{mkg}) for the fair comparison. Specifically, we fine-tune the cross-modality models (CLIP, BLIP) on SER for convergence. We also train the prompt learning models on SER dataset based on VPT's settings to adapt STEM style-diversified inputs. As for the multi-modality models, we evaluate the zero-shot performance on the style-diversified STEM education retrieval task due to multi-modality models' comprehensionability on multi-style image inputs.

For the experiments on the SER dataset, Uni-Retrieval is initialized with OpenCLIP's weights and trained on 8 A100 GPUs with batch size 24 per GPU and 20 training epochs. We use AdamW as the optimizer, set the learning rate to 1e-5 with a linearly warmed up operation in the first epochs and then decayed by the cosine learning rate schedule. The seed is set as 42. What's more, all input images are resized into $224\times224$ resolution and then augmented by normalized operation. All text are padding zero to the max length of 20.

For the fine-tuning CLIP and BLIP models, all experiment settings are the same as Uni-Retrieval except the learning rate is set as 1e-6. For prompt tuning models, we both use 4 prompt tokens to expand the token sequence. For all transformer-based models, we use the ViT-Large and 24-layers text transformer as the foundation models to keep balance between performance and efficiency.







\end{document}
