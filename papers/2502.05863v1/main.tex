% This must be in the first 5 lines to tell arXiv to use pdfLaTeX, which is strongly recommended.
\pdfoutput=1
% In particular, the hyperref package requires pdfLaTeX in order to break URLs across lines.

\documentclass[11pt]{article}

% Change "review" to "final" to generate the final (sometimes called camera-ready) version.
% Change to "preprint" to generate a non-anonymous version with page numbers.
\usepackage[final]{acl}

% Standard package includes
\usepackage{times}
\usepackage{latexsym}

% For proper rendering and hyphenation of words containing Latin characters (including in bib files)
\usepackage[T1]{fontenc}
% For Vietnamese characters
% \usepackage[T5]{fontenc}
% See https://www.latex-project.org/help/documentation/encguide.pdf for other character sets

% This assumes your files are encoded as UTF8
\usepackage[utf8]{inputenc}

% This is not strictly necessary, and may be commented out,
% but it will improve the layout of the manuscript,
% and will typically save some space.
\usepackage{microtype}

% This is also not strictly necessary, and may be commented out.
% However, it will improve the aesthetics of text in
% the typewriter font.
\usepackage{inconsolata}

%Including images in your LaTeX document requires adding
%additional package(s)
\usepackage{graphicx}

% custom package
\usepackage{booktabs}
\usepackage{amsmath}
\usepackage{array}
\usepackage{color}
\usepackage{bbding} % \Checkmark对应的包
\usepackage{amsfonts} % 防止\mathbb报错
\usepackage{algorithm}
\usepackage{algpseudocode}
\usepackage{multirow}
\usepackage{stfloats}
\usepackage[none]{hyphenat}

\definecolor{cred}{HTML}{FF6B6B}
\definecolor{cyellow}{HTML}{FEC260}
%\definecolor{cgreen}{HTML}{6BCB77}
\definecolor{cgreen}{HTML}{70AD47}
\definecolor{cblue}{HTML}{4D96FF}
\definecolor{cpurple}{HTML}{2A0944}
\definecolor{ggray}{RGB}{127,127,127}
\definecolor{aliceblue}{rgb}{0.94, 0.97, 1.0}
\definecolor{cvprblue}{rgb}{0.21,0.49,0.74}

% If the title and author information does not fit in the area allocated, uncomment the following
%
%\setlength\titlebox{<dim>}
%
% and set <dim> to something 5cm or larger.

\title{Uni-Retrieval: A Multi-Style Retrieval Framework for STEM's Education}

% Author information can be set in various styles:
% For several authors from the same institution:
% \author{Author 1 \and ... \and Author n \\
%         Address line \\ ... \\ Address line}
% if the names do not fit well on one line use
%         Author 1 \\ {\bf Author 2} \\ ... \\ {\bf Author n} \\
% For authors from different institutions:
% \author{Author 1 \\ Address line \\  ... \\ Address line
%         \And  ... \And
%         Author n \\ Address line \\ ... \\ Address line}
% To start a separate ``row'' of authors use \AND, as in
% \author{Author 1 \\ Address line \\  ... \\ Address line
%         \AND
%         Author 2 \\ Address line \\ ... \\ Address line \And
%         Author 3 \\ Address line \\ ... \\ Address line}



\author{Yanhao Jia \textsuperscript{1${*}$},
        Xinyi Wu \textsuperscript{2}\thanks{ Equal Contribution, $^{\dagger}$ Corresponding Authors.}, 
        Hao Li \textsuperscript{3},
        Qinglin Zhang \textsuperscript{2}, \vspace{0.2mm}  \\
        {\bf Yuxiao Hu \textsuperscript{4},}
       {\bf Shuai Zhao \textsuperscript{1 ${\dagger}$},
       {\bf Wenqi Fan \textsuperscript{4 ${\dagger}$}}}\\
{ 
\textsuperscript{1} Nanyang Technological University, Singapore;
}\vspace{-0.1mm} \\
{ 
\textsuperscript{2} Shanghai Jiao Tong University, China;
}\vspace{-0.1mm} 
{
\textsuperscript{3} Peking University, China;
}\vspace{-0.1mm} \\
{
\textsuperscript{4} Hong Kong Polytechnic University, Hong Kong, China;
}\vspace{-0.1mm} \\}

% 
  
\newcommand{\yh}[1]{{\color{red}#1}}


\begin{document}
\maketitle


% \twocolumn[{%
% \renewcommand\twocolumn[1][]{#1}%
% \maketitle
% \begin{center}
%     % \centering
%     \vspace{-5mm}
%     \includegraphics[width=\linewidth]{imgs/intro1.pdf}
%     \captionof{figure}{(a). Teachers need to collect data from four categories of STEM teaching contents, to accurately convey the teaching content. (b). Previous Retrieval Models focus on text-query retrieval data, neglecting the retrieval ability for educational query styles. However, teachers want to identify image data in different situation. (c). Our style-diversified retrieval setting considers the various query styles that real educational content may prefer, including sketches, art, low-resolution images, text, audio, and their combinations, such as sketch+text, art+text, etc. Our model performs fine-grained retrieval based on the shape, color, and pose features extracted from these style-diversified query inputs.}
%     \label{fig:motivation}
% \end{center}
% }]

\begin{abstract}


The choice of representation for geographic location significantly impacts the accuracy of models for a broad range of geospatial tasks, including fine-grained species classification, population density estimation, and biome classification. Recent works like SatCLIP and GeoCLIP learn such representations by contrastively aligning geolocation with co-located images. While these methods work exceptionally well, in this paper, we posit that the current training strategies fail to fully capture the important visual features. We provide an information theoretic perspective on why the resulting embeddings from these methods discard crucial visual information that is important for many downstream tasks. To solve this problem, we propose a novel retrieval-augmented strategy called RANGE. We build our method on the intuition that the visual features of a location can be estimated by combining the visual features from multiple similar-looking locations. We evaluate our method across a wide variety of tasks. Our results show that RANGE outperforms the existing state-of-the-art models with significant margins in most tasks. We show gains of up to 13.1\% on classification tasks and 0.145 $R^2$ on regression tasks. All our code and models will be made available at: \href{https://github.com/mvrl/RANGE}{https://github.com/mvrl/RANGE}.

\end{abstract}


\section{Introduction}

Video generation has garnered significant attention owing to its transformative potential across a wide range of applications, such media content creation~\citep{polyak2024movie}, advertising~\citep{zhang2024virbo,bacher2021advert}, video games~\citep{yang2024playable,valevski2024diffusion, oasis2024}, and world model simulators~\citep{ha2018world, videoworldsimulators2024, agarwal2025cosmos}. Benefiting from advanced generative algorithms~\citep{goodfellow2014generative, ho2020denoising, liu2023flow, lipman2023flow}, scalable model architectures~\citep{vaswani2017attention, peebles2023scalable}, vast amounts of internet-sourced data~\citep{chen2024panda, nan2024openvid, ju2024miradata}, and ongoing expansion of computing capabilities~\citep{nvidia2022h100, nvidia2023dgxgh200, nvidia2024h200nvl}, remarkable advancements have been achieved in the field of video generation~\citep{ho2022video, ho2022imagen, singer2023makeavideo, blattmann2023align, videoworldsimulators2024, kuaishou2024klingai, yang2024cogvideox, jin2024pyramidal, polyak2024movie, kong2024hunyuanvideo, ji2024prompt}.


In this work, we present \textbf{\ours}, a family of rectified flow~\citep{lipman2023flow, liu2023flow} transformer models designed for joint image and video generation, establishing a pathway toward industry-grade performance. This report centers on four key components: data curation, model architecture design, flow formulation, and training infrastructure optimization—each rigorously refined to meet the demands of high-quality, large-scale video generation.


\begin{figure}[ht]
    \centering
    \begin{subfigure}[b]{0.82\linewidth}
        \centering
        \includegraphics[width=\linewidth]{figures/t2i_1024.pdf}
        \caption{Text-to-Image Samples}\label{fig:main-demo-t2i}
    \end{subfigure}
    \vfill
    \begin{subfigure}[b]{0.82\linewidth}
        \centering
        \includegraphics[width=\linewidth]{figures/t2v_samples.pdf}
        \caption{Text-to-Video Samples}\label{fig:main-demo-t2v}
    \end{subfigure}
\caption{\textbf{Generated samples from \ours.} Key components are highlighted in \textcolor{red}{\textbf{RED}}.}\label{fig:main-demo}
\end{figure}


First, we present a comprehensive data processing pipeline designed to construct large-scale, high-quality image and video-text datasets. The pipeline integrates multiple advanced techniques, including video and image filtering based on aesthetic scores, OCR-driven content analysis, and subjective evaluations, to ensure exceptional visual and contextual quality. Furthermore, we employ multimodal large language models~(MLLMs)~\citep{yuan2025tarsier2} to generate dense and contextually aligned captions, which are subsequently refined using an additional large language model~(LLM)~\citep{yang2024qwen2} to enhance their accuracy, fluency, and descriptive richness. As a result, we have curated a robust training dataset comprising approximately 36M video-text pairs and 160M image-text pairs, which are proven sufficient for training industry-level generative models.

Secondly, we take a pioneering step by applying rectified flow formulation~\citep{lipman2023flow} for joint image and video generation, implemented through the \ours model family, which comprises Transformer architectures with 2B and 8B parameters. At its core, the \ours framework employs a 3D joint image-video variational autoencoder (VAE) to compress image and video inputs into a shared latent space, facilitating unified representation. This shared latent space is coupled with a full-attention~\citep{vaswani2017attention} mechanism, enabling seamless joint training of image and video. This architecture delivers high-quality, coherent outputs across both images and videos, establishing a unified framework for visual generation tasks.


Furthermore, to support the training of \ours at scale, we have developed a robust infrastructure tailored for large-scale model training. Our approach incorporates advanced parallelism strategies~\citep{jacobs2023deepspeed, pytorch_fsdp} to manage memory efficiently during long-context training. Additionally, we employ ByteCheckpoint~\citep{wan2024bytecheckpoint} for high-performance checkpointing and integrate fault-tolerant mechanisms from MegaScale~\citep{jiang2024megascale} to ensure stability and scalability across large GPU clusters. These optimizations enable \ours to handle the computational and data challenges of generative modeling with exceptional efficiency and reliability.


We evaluate \ours on both text-to-image and text-to-video benchmarks to highlight its competitive advantages. For text-to-image generation, \ours-T2I demonstrates strong performance across multiple benchmarks, including T2I-CompBench~\citep{huang2023t2i-compbench}, GenEval~\citep{ghosh2024geneval}, and DPG-Bench~\citep{hu2024ella_dbgbench}, excelling in both visual quality and text-image alignment. In text-to-video benchmarks, \ours-T2V achieves state-of-the-art performance on the UCF-101~\citep{ucf101} zero-shot generation task. Additionally, \ours-T2V attains an impressive score of \textbf{84.85} on VBench~\citep{huang2024vbench}, securing the top position on the leaderboard (as of 2025-01-25) and surpassing several leading commercial text-to-video models. Qualitative results, illustrated in \Cref{fig:main-demo}, further demonstrate the superior quality of the generated media samples. These findings underscore \ours's effectiveness in multi-modal generation and its potential as a high-performing solution for both research and commercial applications.
\section{Preliminary}
\subsection{Task Formulation}
We provide a formal problem formulation for query-based retrieval.
Specifically, given an image $I_i$ or a text prompt $P_t$ from the style-specific query set $Q_s$, the retrieval model needs to compute the score between input and target queries and rank the corresponding answers $A$ as high as possible. In the task settings for STEM education retrieval, which share a similar goal, the objective is to rank all answers correctly in response to input queries across various style-specific query sets ${Q}_{s=1}^n$.
If the dataset does not contain the corresponding different style queries, the model should list the same category queries as the suggestions.

\subsection{Dataset Construction}
\begin{figure}[!tbp]
  \centering
   \includegraphics[width=\linewidth]{imgs/data_collection.pdf}
   \vspace{-7mm}
   \caption{\textbf{Data construction pipeline.} 1. STEM education knowledge base. 2. Data sources: from online resources and dataset researches. 3. Data processing: extracting essential information from collected data, using AIGC algorithms to generate diverse modalities. 4. Retrieval dataset: construct total 24,000 images and multi-modal STEM educational dataset.}
   \label{fig:data_collection}
   \vspace{-5mm}
\end{figure}
SER is a multi-style benchmark dataset we construct to facilitate accurate retrieval for teachers in STEM education. 
It contains a total of 24,000 text captions, audio clips and different style queries to accommodate different educational scenarios.
As illustrated in Fig.\ref{fig:data_sample}, SER contains:
{\bf Text and Natural Image:} the most common query type, allowing teachers to describe problems using natural language or images for retrieval.
{\bf Audio:} a communication medium in education, enabling teachers to articulate complex queries, which can be further enhanced with LLMs or audio encoders.
{\bf Sketch:} hand-drawn sketches, whether created by users or written on blackboards, provide structural cues such as shape, pose, line, and edges to describe the problem.
{\bf Art:} art-style images as queries help bridge the gap between stylistically different images and original images, improving retrieval consistency across styles. 
{\bf Low-Resolution:} queries involving lower-resolution images, such as those captured from a distance, ensure usability in scenarios where high-quality images are unavailable.

The details of the dataset construction pipeline are shown in Fig.\ref{fig:data_collection}, we use the original STEM education image in the source dataset, and extensively collected datasets from the following sources: 1. online resources such as \href{https://www.kaggle.com/datasets}{Kaggle}, \href{https://github.com/carbon-app/carbon}{GitHub}, \href{https://jr.brainpop.com/subject/science/}{BrainPOP}, \href{https://keypoint.keystonesymposia.org/}{Frontiers}, \href{https://learning.dk.com/us}{DKlearning}, \href{https://www.inaturalist.org/}{iNaturalist}, \href{https://www.asiastem.org/steam-free-resources}{AAES}, etc. 2. relevant education dataset research, such as GAN \cite{jin2023gan} and PromptAloud \cite{lee2024prompt}. To ensure high-quality data, more than 20 Ph.D. students from disciplines such as mathematics, physics, chemistry, biology, electronics, computer science, and education conducted a secondary screening of the raw images. They also generate multi-modal combinations (image/text/audio) by leveraging AIGC models.
Based on the natural images, the following steps were undertaken:
{\bf Text Generation:} we manually proofread the natural text descriptions.
{\bf Audio Recording:} the corresponding audio parts are recorded to match the text captions.
{\bf Sketch Images:} Canny algorithms are used to produce sketch images, Pidinet \cite{pdc} is employed to optimize and enhance low-quality sketches, and manual refinement is performed to achieve the final results.
{\bf Art-Style Images:} Flux model \cite{flux} is utilized to create art-style images.
{\bf Low-Resolution Images:} Gaussian Blur algorithms are applied to generate low-resolution images.
According to the National Science Foundation (NSF)’s classification of STEM education, we collect 6,000 original samples spanning six styles, three modalities, and over 22 subjects. This comprehensive dataset, as illustrated in Appendix Fig.\ref{fig:distribution}, ensures diversity and quality across multiple educational domains.



%SER is a multi-style benchmark dataset we constructed to facilitate accurate retrieval for teachers in STEM education. 
% It contains a total of 24,000 text captions, audio clips and different style queries to accommodate different educational scenarios.
%As illustrated in Fig.\ref{fig:data_sample}, SER contains:
% the SER dataset includes natural images paired with corresponding queries in six styles across three modalities: text, audio, and visual (sketch, art, low-resolution).
%(1-2). Text and Natural Image: The most common query type, allowing teachers to describe problems using natural language or images for retrieval. (3). Audio: A direct communication medium in education, enabling teachers to articulate complex queries, which can be further enhanced with large language models (LLMs) or audio encoders. (4). Sketch: Hand-drawn sketches, whether created by users or written on blackboards, provide structural cues such as shape, pose, line, and edges to describe the problem. (5). Art: Art-style images as queries help bridge the gap between stylistically different images and original images, improving retrieval consistency across styles. (6). Low-Resolution: Queries involving lower-resolution images, such as those captured from a distance, ensure usability in scenarios where high-quality images are unavailable.
Effective human-robot cooperation in CoNav-Maze hinges on efficient communication. Maximizing the human’s information gain enables more precise guidance, which in turn accelerates task completion. Yet for the robot, the challenge is not only \emph{what} to communicate but also \emph{when}, as it must balance gathering information for the human with pursuing immediate goals when confident in its navigation.

To achieve this, we introduce \emph{Information Gain Monte Carlo Tree Search} (IG-MCTS), which optimizes both task-relevant objectives and the transmission of the most informative communication. IG-MCTS comprises three key components:
\textbf{(1)} A data-driven human perception model that tracks how implicit (movement) and explicit (image) information updates the human’s understanding of the maze layout.
\textbf{(2)} Reward augmentation to integrate multiple objectives effectively leveraging on the learned perception model.
\textbf{(3)} An uncertainty-aware MCTS that accounts for unobserved maze regions and human perception stochasticity.
% \begin{enumerate}[leftmargin=*]
%     \item A data-driven human perception model that tracks how implicit (movement) and explicit (image transmission) information updates the human’s understanding of the maze layout.
%     \item Reward augmentation to integrate multiple objectives effectively leveraging on the learned perception model.
%     \item An uncertainty-aware MCTS that accounts for unobserved maze regions and human perception stochasticity.
% \end{enumerate}

\subsection{Human Perception Dynamics}
% IG-MCTS seeks to optimize the expected novel information gained by the human through the robot’s actions, including both movement and communication. Achieving this requires a model of how the human acquires task-relevant information from the robot.

% \subsubsection{Perception MDP}
\label{sec:perception_mdp}
As the robot navigates the maze and transmits images, humans update their understanding of the environment. Based on the robot's path, they may infer that previously assumed blocked locations are traversable or detect discrepancies between the transmitted image and their map.  

To formally capture this process, we model the evolution of human perception as another Markov Decision Process, referred to as the \emph{Perception MDP}. The state space $\mathcal{X}$ represents all possible maze maps. The action space $\mathcal{S}^+ \times \mathcal{O}$ consists of the robot's trajectory between two image transmissions $\tau \in \mathcal{S}^+$ and an image $o \in \mathcal{O}$. The unknown transition function $F: (x, (\tau, o)) \rightarrow x'$ defines the human perception dynamics, which we aim to learn.

\subsubsection{Crowd-Sourced Transition Dataset}
To collect data, we designed a mapping task in the CoNav-Maze environment. Participants were tasked to edit their maps to match the true environment. A button triggers the robot's autonomous movements, after which it captures an image from a random angle.
In this mapping task, the robot, aware of both the true environment and the human’s map, visits predefined target locations and prioritizes areas with mislabeled grid cells on the human’s map.
% We assume that the robot has full knowledge of both the actual environment and the human’s current map. Leveraging this knowledge, the robot autonomously navigates to all predefined target locations. It then randomly selects subsequent goals to reach, prioritizing grid locations that remain mislabeled on the human’s map. This ensures that the robot’s actions are strategically focused on providing useful information to improve map accuracy.

We then recruited over $50$ annotators through Prolific~\cite{palan2018prolific} for the mapping task. Each annotator labeled three randomly generated mazes. They were allowed to proceed to the next maze once the robot had reached all four goal locations. However, they could spend additional time refining their map before moving on. To incentivize accuracy, annotators receive a performance-based bonus based on the final accuracy of their annotated map.


\subsubsection{Fully-Convolutional Dynamics Model}
\label{sec:nhpm}

We propose a Neural Human Perception Model (NHPM), a fully convolutional neural network (FCNN), to predict the human perception transition probabilities modeled in \Cref{sec:perception_mdp}. We denote the model as $F_\theta$ where $\theta$ represents the trainable weights. Such design echoes recent studies of model-based reinforcement learning~\cite{hansen2022temporal}, where the agent first learns the environment dynamics, potentially from image observations~\cite{hafner2019learning,watter2015embed}.

\begin{figure}[t]
    \centering
    \includegraphics[width=0.9\linewidth]{figures/ICML_25_CNN.pdf}
    \caption{Neural Human Perception Model (NHPM). \textbf{Left:} The human's current perception, the robot's trajectory since the last transmission, and the captured environment grids are individually processed into 2D masks. \textbf{Right:} A fully convolutional neural network predicts two masks: one for the probability of the human adding a wall to their map and another for removing a wall.}
    \label{fig:nhpm}
    \vskip -0.1in
\end{figure}

As illustrated in \Cref{fig:nhpm}, our model takes as input the human’s current perception, the robot’s path, and the image captured by the robot, all of which are transformed into a unified 2D representation. These inputs are concatenated along the channel dimension and fed into the CNN, which outputs a two-channel image: one predicting the probability of human adding a new wall and the other predicting the probability of removing a wall.

% Our approach builds on world model learning, where neural networks predict state transitions or environmental updates based on agent actions and observations. By leveraging the local feature extraction capabilities of CNNs, our model effectively captures spatial relationships and interprets local changes within the grid maze environment. Similar to prior work in localization and mapping, the CNN architecture is well-suited for processing spatially structured data and aligning the robot’s observations with human map updates.

To enhance robustness and generalization, we apply data augmentation techniques, including random rotation and flipping of the 2D inputs during training. These transformations are particularly beneficial in the grid maze environment, which is invariant to orientation changes.

\subsection{Perception-Aware Reward Augmentation}
The robot optimizes its actions over a planning horizon \( H \) by solving the following optimization problem:
\begin{subequations}
    \begin{align}
        \max_{a_{0:H-1}} \;
        & \mathop{\mathbb{E}}_{T, F} \left[ \sum_{t=0}^{H-1} \gamma^t \left(\underbrace{R_{\mathrm{task}}(\tau_{t+1}, \zeta)}_{\text{(1) Task reward}} + \underbrace{\|x_{t+1}-x_t\|_1}_{\text{(2) Info reward}}\right)\right] \label{obj}\\ 
        \subjectto \quad
        &x_{t+1} = F(x_t, (\tau_t, a_t)), \quad a_t\in\Ocal \label{const:perception_update}\\ 
        &\tau_{t+1} = \tau_t \oplus T(s_t, a_t), \quad a_t\in \Ucal\label{const:history_update}
    \end{align}
\end{subequations} 

The objective in~\eqref{obj} maximizes the expected cumulative reward over \( T \) and \( F \), reflecting the uncertainty in both physical transitions and human perception dynamics. The reward function consists of two components: 
(1) The \emph{task reward} incentivizes efficient navigation. The specific formulation for the task in this work is outlined in \Cref{appendix:task_reward}.
(2) The \emph{information reward} quantifies the change in the human’s perception due to robot actions, computed as the \( L_1 \)-norm distance between consecutive perception states.  

The constraint in~\eqref{const:history_update} ensures that for movement actions, the trajectory history \( \tau_t \) expands with new states based on the robot’s chosen actions, where \( s_t \) is the most recent state in \( \tau_t \), and \( \oplus \) represents sequence concatenation. 
In constraint~\eqref{const:perception_update}, the robot leverages the learned human perception dynamics \( F \) to estimate the evolution of the human’s understanding of the environment from perception state $x_t$ to $x_{t+1}$ based on the observed trajectory \( \tau_t \) and transmitted image \( a_t\in\Ocal \). 
% justify from a cognitive science perspective
% Cognitive science research has shown that humans read in a way to maximize the information gained from each word, aligning with the efficient coding principle, which prioritizes minimizing perceptual errors and extracting relevant features under limited processing capacity~\cite{kangassalo2020information}. Drawing on this principle, we hypothesize that humans similarly prioritize task-relevant information in multimodal settings. To accommodate this cognitive pattern, our robot policy selects and communicates high information-gain observations to human operators, akin to summarizing key insights from a lengthy article.
% % While the brain naturally seeks to gain information, the brain employs various strategies to manage information overload, including filtering~\cite{quiroga2004reducing}, limiting/working memory, and prioritizing information~\cite{arnold2023dealing}.
% In this context of our setup, we optimize the selection of camera angles to maximize the human operator's information gain about the environment. 

\subsection{Information Gain Monte Carlo Tree Search (IG-MCTS)}
IG-MCTS follows the four stages of Monte Carlo tree search: \emph{selection}, \emph{expansion}, \emph{rollout}, and \emph{backpropagation}, but extends it by incorporating uncertainty in both environment dynamics and human perception. We introduce uncertainty-aware simulations in the \emph{expansion} and \emph{rollout} phases and adjust \emph{backpropagation} with a value update rule that accounts for transition feasibility.

\subsubsection{Uncertainty-Aware Simulation}
As detailed in \Cref{algo:IG_MCTS}, both the \emph{expansion} and \emph{rollout} phases involve forward simulation of robot actions. Each tree node $v$ contains the state $(\tau, x)$, representing the robot's state history and current human perception. We handle the two action types differently as follows:
\begin{itemize}
    \item A movement action $u$ follows the environment dynamics $T$ as defined in \Cref{sec:problem}. Notably, the maze layout is observable up to distance $r$ from the robot's visited grids, while unexplored areas assume a $50\%$ chance of walls. In \emph{expansion}, the resulting search node $v'$ of this uncertain transition is assigned a feasibility value $\delta = 0.5$. In \emph{rollout}, the transition could fail and the robot remains in the same grid.
    
    \item The state transition for a communication step $o$ is governed by the learned stochastic human perception model $F_\theta$ as defined in \Cref{sec:nhpm}. Since transition probabilities are known, we compute the expected information reward $\bar{R_\mathrm{info}}$ directly:
    \begin{align*}
        \bar{R_\mathrm{info}}(\tau_t, x_t, o_t) &= \mathbb{E}_{x_{t+1}}\|x_{t+1}-x_t\|_1 \\
        &= \|p_\mathrm{add}\|_1 + \|p_\mathrm{remove}\|_1,
    \end{align*}
    where $(p_\mathrm{add}, p_\mathrm{remove}) \gets F_\theta(\tau_t, x_t, o_t)$ are the estimated probabilities of adding or removing walls from the map. 
    Directly computing the expected return at a node avoids the high number of visitations required to obtain an accurate value estimate.
\end{itemize}

% We denote a node in the search tree as $v$, where $s(v)$, $r(v)$, and $\delta(v)$ represent the state, reward, and transition feasibility at $v$, respectively. The visit count of $v$ is denoted as $N(v)$, while $Q(v)$ represents its total accumulated return. The set of child nodes of $v$ is denoted by $\mathbb{C}(v)$.

% The goal of each search is to plan a sequence for the robot until it reaches a goal or transmits a new image to the human. We initialize the search tree with the current human guidance $\zeta$, and the robot's approximation of human perception $x_0$. Each search node consists consists of the state information required by our reward augmentation: $(\tau, x)$. A node is terminal if it is the resulting state of a communication step, or if the robot reaches a goal location. 

% A rollout from the expanded node simulates future transitions until reaching a terminal state or a predefined depth $H$. Actions are selected randomly from the available action set $\mathcal{A}(s)$. If an action's feasibility is uncertain due to the environment's unknown structure, the transition occurs with probability $\delta(s, a)$. When a random number draw deems the transition infeasible, the state remains unchanged. On the other hand, for communication steps, we don't resolve the uncertainty but instead compute the expected information gain reward: \philip{TODO: adjust notation}
% \begin{equation}
%     \mathbb{E}\left[R_\mathrm{info}(\tau, x')\right] = \sum \mathrm{NPM(\tau, o)}.
% \end{equation}

\subsubsection{Feasibility-Adjusted Backpropagation}
During backpropagation, the rewards obtained from the simulation phase are propagated back through the tree, updating the total value $Q(v)$ and the visitation count $N(v)$ for all nodes along the path to the root. Due to uncertainty in unexplored environment dynamics, the rollout return depends on the feasibility of the transition from the child node. Given a sample return \(q'_{\mathrm{sample}}\) at child node \(v'\), the parent node's return is:
\begin{equation}
    q_{\mathrm{sample}} = r + \gamma \left[ \delta' q'_{\mathrm{sample}} + (1 - \delta') \frac{Q(v)}{N(v)} \right],
\end{equation}
where $\delta'$ represents the probability of a successful transition. The term \((1 - \delta')\) accounts for failed transitions, relying instead on the current value estimate.

% By incorporating uncertainty-aware rollouts and backpropagation, our approach enables more robust decision-making in scenarios where the environment dynamics is unknown and avoids simulation of the stochastic human perception dynamics.

\section{Experiments}
\label{section5}

In this section, we conduct extensive experiments to show that \ourmethod~can significantly speed up the sampling of existing MR Diffusion. To rigorously validate the effectiveness of our method, we follow the settings and checkpoints from \cite{luo2024daclip} and only modify the sampling part. Our experiment is divided into three parts. Section \ref{mainresult} compares the sampling results for different NFE cases. Section \ref{effects} studies the effects of different parameter settings on our algorithm, including network parameterizations and solver types. In Section \ref{analysis}, we visualize the sampling trajectories to show the speedup achieved by \ourmethod~and analyze why noise prediction gets obviously worse when NFE is less than 20.


\subsection{Main results}\label{mainresult}

Following \cite{luo2024daclip}, we conduct experiments with ten different types of image degradation: blurry, hazy, JPEG-compression, low-light, noisy, raindrop, rainy, shadowed, snowy, and inpainting (see Appendix \ref{appd1} for details). We adopt LPIPS \citep{zhang2018lpips} and FID \citep{heusel2017fid} as main metrics for perceptual evaluation, and also report PSNR and SSIM \citep{wang2004ssim} for reference. We compare \ourmethod~with other sampling methods, including posterior sampling \citep{luo2024posterior} and Euler-Maruyama discretization \citep{kloeden1992sde}. We take two tasks as examples and the metrics are shown in Figure \ref{fig:main}. Unless explicitly mentioned, we always use \ourmethod~based on SDE solver, with data prediction and uniform $\lambda$. The complete experimental results can be found in Appendix \ref{appd3}. The results demonstrate that \ourmethod~converges in a few (5 or 10) steps and produces samples with stable quality. Our algorithm significantly reduces the time cost without compromising sampling performance, which is of great practical value for MR Diffusion.


\begin{figure}[!ht]
    \centering
    \begin{minipage}[b]{0.45\textwidth}
        \centering
        \includegraphics[width=1\textwidth, trim=0 20 0 0]{figs/main_result/7_lowlight_fid.pdf}
        \subcaption{FID on \textit{low-light} dataset}
        \label{fig:main(a)}
    \end{minipage}
    \begin{minipage}[b]{0.45\textwidth}
        \centering
        \includegraphics[width=1\textwidth, trim=0 20 0 0]{figs/main_result/7_lowlight_lpips.pdf}
        \subcaption{LPIPS on \textit{low-light} dataset}
        \label{fig:main(b)}
    \end{minipage}
    \begin{minipage}[b]{0.45\textwidth}
        \centering
        \includegraphics[width=1\textwidth, trim=0 20 0 0]{figs/main_result/10_motion_fid.pdf}
        \subcaption{FID on \textit{motion-blurry} dataset}
        \label{fig:main(c)}
    \end{minipage}
    \begin{minipage}[b]{0.45\textwidth}
        \centering
        \includegraphics[width=1\textwidth, trim=0 20 0 0]{figs/main_result/10_motion_lpips.pdf}
        \subcaption{LPIPS on \textit{motion-blurry} dataset}
        \label{fig:main(d)}
    \end{minipage}
    \caption{\textbf{Perceptual evaluations on \textit{low-light} and \textit{motion-blurry} datasets.}}
    \label{fig:main}
\end{figure}

\subsection{Effects of parameter choice}\label{effects}

In Table \ref{tab:ablat_param}, we compare the results of two network parameterizations. The data prediction shows stable performance across different NFEs. The noise prediction performs similarly to data prediction with large NFEs, but its performance deteriorates significantly with smaller NFEs. The detailed analysis can be found in Section \ref{section5.3}. In Table \ref{tab:ablat_solver}, we compare \ourmethod-ODE-d-2 and \ourmethod-SDE-d-2 on the \textit{inpainting} task, which are derived from PF-ODE and reverse-time SDE respectively. SDE-based solver works better with a large NFE, whereas ODE-based solver is more effective with a small NFE. In general, neither solver type is inherently better.


% In Table \ref{tab:hazy}, we study the impact of two step size schedules on the results. On the whole, uniform $\lambda$ performs slightly better than uniform $t$. Our algorithm follows the method of \cite{lu2022dpmsolverplus} to estimate the integral part of the solution, while the analytical part does not affect the error.  Consequently, our algorithm has the same global truncation error, that is $\mathcal{O}\left(h_{max}^{k}\right)$. Note that the initial and final values of $\lambda$ depend on noise schedule and are fixed. Therefore, uniform $\lambda$ scheduling leads to the smallest $h_{max}$ and works better.

\begin{table}[ht]
    \centering
    \begin{minipage}{0.5\textwidth}
    \small
    \renewcommand{\arraystretch}{1}
    \centering
    \caption{Ablation study of network parameterizations on the Rain100H dataset.}
    % \vspace{8pt}
    \resizebox{1\textwidth}{!}{
        \begin{tabular}{cccccc}
			\toprule[1.5pt]
            % \multicolumn{6}{c}{Rainy} \\
            % \cmidrule(lr){1-6}
             NFE & Parameterization      & LPIPS\textdownarrow & FID\textdownarrow &  PSNR\textuparrow & SSIM\textuparrow  \\
            \midrule[1pt]
            \multirow{2}{*}{50}
             & Noise Prediction & \textbf{0.0606}     & \textbf{27.28}   & \textbf{28.89}     & \textbf{0.8615}    \\
             & Data Prediction & 0.0620     & 27.65   & 28.85     & 0.8602    \\
            \cmidrule(lr){1-6}
            \multirow{2}{*}{20}
              & Noise Prediction & 0.1429     & 47.31   & 27.68     & 0.7954    \\
              & Data Prediction & \textbf{0.0635}     & \textbf{27.79}   & \textbf{28.60}     & \textbf{0.8559}    \\
            \cmidrule(lr){1-6}
            \multirow{2}{*}{10}
              & Noise Prediction & 1.376     & 402.3   & 6.623     & 0.0114    \\
              & Data Prediction & \textbf{0.0678}     & \textbf{29.54}   & \textbf{28.09}     & \textbf{0.8483}    \\
            \cmidrule(lr){1-6}
            \multirow{2}{*}{5}
              & Noise Prediction & 1.416     & 447.0   & 5.755     & 0.0051    \\
              & Data Prediction & \textbf{0.0637}     & \textbf{26.92}   & \textbf{28.82}     & \textbf{0.8685}    \\       
            \bottomrule[1.5pt]
        \end{tabular}}
        \label{tab:ablat_param}
    \end{minipage}
    \hspace{0.01\textwidth}
    \begin{minipage}{0.46\textwidth}
    \small
    \renewcommand{\arraystretch}{1}
    \centering
    \caption{Ablation study of solver types on the CelebA-HQ dataset.}
    % \vspace{8pt}
        \resizebox{1\textwidth}{!}{
        \begin{tabular}{cccccc}
			\toprule[1.5pt]
            % \multicolumn{6}{c}{Raindrop} \\     
            % \cmidrule(lr){1-6}
             NFE & Solver Type     & LPIPS\textdownarrow & FID\textdownarrow &  PSNR\textuparrow & SSIM\textuparrow  \\
            \midrule[1pt]
            \multirow{2}{*}{50}
             & ODE & 0.0499     & 22.91   & 28.49     & 0.8921    \\
             & SDE & \textbf{0.0402}     & \textbf{19.09}   & \textbf{29.15}     & \textbf{0.9046}    \\
            \cmidrule(lr){1-6}
            \multirow{2}{*}{20}
              & ODE & 0.0475    & 21.35   & 28.51     & 0.8940    \\
              & SDE & \textbf{0.0408}     & \textbf{19.13}   & \textbf{28.98}    & \textbf{0.9032}    \\
            \cmidrule(lr){1-6}
            \multirow{2}{*}{10}
              & ODE & \textbf{0.0417}    & 19.44   & \textbf{28.94}     & \textbf{0.9048}    \\
              & SDE & 0.0437     & \textbf{19.29}   & 28.48     & 0.8996    \\
            \cmidrule(lr){1-6}
            \multirow{2}{*}{5}
              & ODE & \textbf{0.0526}     & 27.44   & \textbf{31.02}     & \textbf{0.9335}    \\
              & SDE & 0.0529    & \textbf{24.02}   & 28.35     & 0.8930    \\
            \bottomrule[1.5pt]
        \end{tabular}}
        \label{tab:ablat_solver}
    \end{minipage}
\end{table}


% \renewcommand{\arraystretch}{1}
%     \centering
%     \caption{Ablation study of step size schedule on the RESIDE-6k dataset.}
%     % \vspace{8pt}
%         \resizebox{1\textwidth}{!}{
%         \begin{tabular}{cccccc}
% 			\toprule[1.5pt]
%             % \multicolumn{6}{c}{Raindrop} \\     
%             % \cmidrule(lr){1-6}
%              NFE & Schedule      & LPIPS\textdownarrow & FID\textdownarrow &  PSNR\textuparrow & SSIM\textuparrow  \\
%             \midrule[1pt]
%             \multirow{2}{*}{50}
%              & uniform $t$ & 0.0271     & 5.539   & 30.00     & 0.9351    \\
%              & uniform $\lambda$ & \textbf{0.0233}     & \textbf{4.993}   & \textbf{30.19}     & \textbf{0.9427}    \\
%             \cmidrule(lr){1-6}
%             \multirow{2}{*}{20}
%               & uniform $t$ & 0.0313     & 6.000   & 29.73     & 0.9270    \\
%               & uniform $\lambda$ & \textbf{0.0240}     & \textbf{5.077}   & \textbf{30.06}    & \textbf{0.9409}    \\
%             \cmidrule(lr){1-6}
%             \multirow{2}{*}{10}
%               & uniform $t$ & 0.0309     & 6.094   & 29.42     & 0.9274    \\
%               & uniform $\lambda$ & \textbf{0.0246}     & \textbf{5.228}   & \textbf{29.65}     & \textbf{0.9372}    \\
%             \cmidrule(lr){1-6}
%             \multirow{2}{*}{5}
%               & uniform $t$ & 0.0256     & 5.477   & \textbf{29.91}     & 0.9342    \\
%               & uniform $\lambda$ & \textbf{0.0228}     & \textbf{5.174}   & 29.65     & \textbf{0.9416}    \\
%             \bottomrule[1.5pt]
%         \end{tabular}}
%         \label{tab:ablat_schedule}



\subsection{Analysis}\label{analysis}
\label{section5.3}

\begin{figure}[ht!]
    \centering
    \begin{minipage}[t]{0.6\linewidth}
        \centering
        \includegraphics[width=\linewidth, trim=0 20 10 0]{figs/trajectory_a.pdf} %trim左下右上
        \subcaption{Sampling results.}
        \label{fig:traj(a)}
    \end{minipage}
    \begin{minipage}[t]{0.35\linewidth}
        \centering
        \includegraphics[width=\linewidth, trim=0 0 0 0]{figs/trajectory_b.pdf} %trim左下右上
        \subcaption{Trajectory.}
        \label{fig:traj(b)}
    \end{minipage}
    \caption{\textbf{Sampling trajectories.} In (a), we compare our method (with order 1 and order 2) and previous sampling methods (i.e., posterior sampling and Euler discretization) on a motion blurry image. The numbers in parentheses indicate the NFE. In (b), we illustrate trajectories of each sampling method. Previous methods need to take many unnecessary paths to converge. With few NFEs, they fail to reach the ground truth (i.e., the location of $\boldsymbol{x}_0$). Our methods follow a more direct trajectory.}
    \label{fig:traj}
\end{figure}

\textbf{Sampling trajectory.}~ Inspired by the design idea of NCSN \citep{song2019ncsn}, we provide a new perspective of diffusion sampling process. \cite{song2019ncsn} consider each data point (e.g., an image) as a point in high-dimensional space. During the diffusion process, noise is added to each point $\boldsymbol{x}_0$, causing it to spread throughout the space, while the score function (a neural network) \textit{remembers} the direction towards $\boldsymbol{x}_0$. In the sampling process, we start from a random point by sampling a Gaussian distribution and follow the guidance of the reverse-time SDE (or PF-ODE) and the score function to locate $\boldsymbol{x}_0$. By connecting each intermediate state $\boldsymbol{x}_t$, we obtain a sampling trajectory. However, this trajectory exists in a high-dimensional space, making it difficult to visualize. Therefore, we use Principal Component Analysis (PCA) to reduce $\boldsymbol{x}_t$ to two dimensions, obtaining the projection of the sampling trajectory in 2D space. As shown in Figure \ref{fig:traj}, we present an example. Previous sampling methods \citep{luo2024posterior} often require a long path to find $\boldsymbol{x}_0$, and reducing NFE can lead to cumulative errors, making it impossible to locate $\boldsymbol{x}_0$. In contrast, our algorithm produces more direct trajectories, allowing us to find $\boldsymbol{x}_0$ with fewer NFEs.

\begin{figure*}[ht]
    \centering
    \begin{minipage}[t]{0.45\linewidth}
        \centering
        \includegraphics[width=\linewidth, trim=0 0 0 0]{figs/convergence_a.pdf} %trim左下右上
        \subcaption{Sampling results.}
        \label{fig:convergence(a)}
    \end{minipage}
    \begin{minipage}[t]{0.43\linewidth}
        \centering
        \includegraphics[width=\linewidth, trim=0 20 0 0]{figs/convergence_b.pdf} %trim左下右上
        \subcaption{Ratio of convergence.}
        \label{fig:convergence(b)}
    \end{minipage}
    \caption{\textbf{Convergence of noise prediction and data prediction.} In (a), we choose a low-light image for example. The numbers in parentheses indicate the NFE. In (b), we illustrate the ratio of components of neural network output that satisfy the Taylor expansion convergence requirement.}
    \label{fig:converge}
\end{figure*}

\textbf{Numerical stability of parameterizations.}~ From Table 1, we observe poor sampling results for noise prediction in the case of few NFEs. The reason may be that the neural network parameterized by noise prediction is numerically unstable. Recall that we used Taylor expansion in Eq.(\ref{14}), and the condition for the equality to hold is $|\lambda-\lambda_s|<\boldsymbol{R}(s)$. And the radius of convergence $\boldsymbol{R}(t)$ can be calculated by
\begin{equation}
\frac{1}{\boldsymbol{R}(t)}=\lim_{n\rightarrow\infty}\left|\frac{\boldsymbol{c}_{n+1}(t)}{\boldsymbol{c}_n(t)}\right|,
\end{equation}
where $\boldsymbol{c}_n(t)$ is the coefficient of the $n$-th term in Taylor expansion. We are unable to compute this limit and can only compute the $n=0$ case as an approximation. The output of the neural network can be viewed as a vector, with each component corresponding to a radius of convergence. At each time step, we count the ratio of components that satisfy $\boldsymbol{R}_i(s)>|\lambda-\lambda_s|$ as a criterion for judging the convergence, where $i$ denotes the $i$-th component. As shown in Figure \ref{fig:converge}, the neural network parameterized by data prediction meets the convergence criteria at almost every step. However, the neural network parameterized by noise prediction always has components that cannot converge, which will lead to large errors and failed sampling. Therefore, data prediction has better numerical stability and is a more recommended choice.


\paragraph{Summary}
Our findings provide significant insights into the influence of correctness, explanations, and refinement on evaluation accuracy and user trust in AI-based planners. 
In particular, the findings are three-fold: 
(1) The \textbf{correctness} of the generated plans is the most significant factor that impacts the evaluation accuracy and user trust in the planners. As the PDDL solver is more capable of generating correct plans, it achieves the highest evaluation accuracy and trust. 
(2) The \textbf{explanation} component of the LLM planner improves evaluation accuracy, as LLM+Expl achieves higher accuracy than LLM alone. Despite this improvement, LLM+Expl minimally impacts user trust. However, alternative explanation methods may influence user trust differently from the manually generated explanations used in our approach.
% On the other hand, explanations may help refine the trust of the planner to a more appropriate level by indicating planner shortcomings.
(3) The \textbf{refinement} procedure in the LLM planner does not lead to a significant improvement in evaluation accuracy; however, it exhibits a positive influence on user trust that may indicate an overtrust in some situations.
% This finding is aligned with prior works showing that iterative refinements based on user feedback would increase user trust~\cite{kunkel2019let, sebo2019don}.
Finally, the propensity-to-trust analysis identifies correctness as the primary determinant of user trust, whereas explanations provided limited improvement in scenarios where the planner's accuracy is diminished.

% In conclusion, our results indicate that the planner's correctness is the dominant factor for both evaluation accuracy and user trust. Therefore, selecting high-quality training data and optimizing the training procedure of AI-based planners to improve planning correctness is the top priority. Once the AI planner achieves a similar correctness level to traditional graph-search planners, strengthening its capability to explain and refine plans will further improve user trust compared to traditional planners.

\paragraph{Future Research} Future steps in this research include expanding user studies with larger sample sizes to improve generalizability and including additional planning problems per session for a more comprehensive evaluation. Next, we will explore alternative methods for generating plan explanations beyond manual creation to identify approaches that more effectively enhance user trust. 
Additionally, we will examine user trust by employing multiple LLM-based planners with varying levels of planning accuracy to better understand the interplay between planning correctness and user trust. 
Furthermore, we aim to enable real-time user-planner interaction, allowing users to provide feedback and refine plans collaboratively, thereby fostering a more dynamic and user-centric planning process.


% Bibliography entries for the entire Anthology, followed by custom entries
%\bibliography{anthology,custom}
% Custom bibliography entries only
\bibliography{custom}

\clearpage

\appendix


\section{Related Works}
\label{sec:related_work}

\subsection{Dataset Adaptation in Education}
Within the realm of education, image retrieval in education has distinct characteristics, as images often reflect the teaching intentions of educators. This facilitates the rapid and accurate alignment of visual content with teaching materials, thereby reducing educators' preparation workload and enhancing the precision of learning data. While existing researches has focused on classifying educational data \cite{choi2020ednet}, they often encounter constraints. Due to the complexity of incorporating an expansive range of teaching scenarios in STEM education and the scarcity of data, numerous studies often narrow the scope to a limited set of subject applications \cite{hendrycks2021measuring,pal2022medmcqa} or to a limited set of teaching strategy retrievals \cite{kwon2024biped,welbl2017crowdsourcing}.

\begin{figure*}[!t]
    \centering
    \includegraphics[width=\linewidth]{imgs/data.pdf}
    \vspace{-7mm}
    \caption{\textbf{The SER Dataset} contains 24,000+ text captions and their corresponding queries with various styles, including Natural, Sketch, Art, Low-Resolution~(Low-Res) images and audio clips from different STEM subjects.}
    \vspace{-5mm}
    \label{fig:data_sample}
\end{figure*}


There is a considerable variation across existing STEM education datasets regarding its specific composition. Many datasets are cluttered with irrelevant or invalid data, lack comprehensive coverage of specialised content, and suffer from quality assurance issues \cite{patrinos2018global}. 
Although STEM education datasets are assembled from interactions between learners and large language models \cite{hou2024eval, wang2024large}, they are generally not well-suited for use by educators and learners across multiple domains. Furthermore, creating a precise and professional data retrieval repository for the educational domain requires efficient retrieval algorithms as support \cite{alzoubi2024enhancing}. To ensure efficient retrieval and usability in STEM education scenarios, we construct the SER dataset, which includes multiple query styles to enhance retrieval diversity.
%There are datasets assembled from interactions between learners and large models\cite{}\cite{}. 
%However, creating a precise and professional data retrieval repository for the educational domain requires grounding it in authentic teaching scenarios.
%Additionally, most datasets and retrieval methods are not well-suited for use by educators and learners across multiple domains. 
%To ensure efficient retrieval and usability, image-text alignment and voice input play pivotal roles.

\subsection{Multi-task Learning}

In STEM education, the multi-style retrieval model needs to leverage multi-task learning to align features and learning across different modal samples. Multi-task learning refers to the simultaneous training and optimization of multiple related tasks within a single model \cite{9392366}. By sharing parameters and representations across functions, it improves overall performance. Compared to other transfer learning methods, including domain adaptation \cite{farahani2021brief} and domain generalization \cite{zhou2022domain}, multi-task learning annotates data and achieves CLIP-level model fine-tuning and convergence, the data of each task in multi-task learning is well-labeled. 

Overall, multi-task learning introduces a new tool for STEM education practitioners that may help meet requirements, especially if speed and efficiency are preferred over performance. While many recent multi-task learning employ two clusters of contemporary techniques, hard parameter sharing and soft parameter sharing \cite{ruder2017overview}. In hard parameter sharing, most or all of the parameters in the network are shared among all tasks \cite{kokkinos2017ubernet}. In soft parameter sharing,the models are tied together either by information sharing or by requiring parameters to be similar \cite{yang2016trace}. Consequently, our Uni-Retrieval adopts a blended multi-task learning paradigm, adopt both hard and soft  parameter in different styles of tasks. Building upon successul multi-task learning method for CLIP, such as CoCoOP \cite{zhou2022conditional}, MaPLe \cite{khattak2023maple}, and FreestyleRet \cite{li2025freestyleret}, our study leverages these techniques to strengthen domain adaptation and multi-task learning. 

\begin{figure*}[!t]
  \centering
   \includegraphics[width=\linewidth]{imgs/categories.pdf}
   \vspace{-6mm}
   \caption{\textbf{Concept distribution of our SER dataset}. Our dataset exhibits a diverse distribution on different concept domains.}
   \label{fig:distribution}
   \vspace{-3mm}
\end{figure*}

\subsection{Query-based Retrieval}
Existing work in Query-based Image Retrieval (QBIR) primarily includes content-based image retrieval \cite{chen2022deep}, text-based image retrieval \cite{li2011text}, and multi-modal retrieval \cite{neculai2022probabilistic}. In content-based image retrieval, the visual features of images are directly utilized for retrieval. However, its reliance on fixed content and location makes it relatively inflexible in capturing diverse user intents \cite{lee-etal-2024-interactive}. Alternative methods like sketching \cite{chowdhury2022fs, chowdhury2023scenetrilogy} and scene graph construction \cite{johnson2015image} enable the retrieval of abstract images that are hard to describe verbally, though they lack the intuitive ease of natural language-based retrieval. In text-based image retrieval, enhancements to text queries often involve indicating content structure. However, these approaches are either restricted by closed vocabularies \cite{mai2017spatial, kilickaya2021structured} or face substantial challenges \cite{li2017generating} in deriving structures from natural language descriptions. Recent multi-modal approaches, such as cross-modal scene graph-based image-text retrieval \cite{wang2020cross} and joint visual-scene graph embedding for image retrieval \cite{belilovsky2017joint}, still depend on word embeddings and image features.

Despite advancements in QBIR, challenges including the semantic gap that can lead to inaccurate retrieval results, high computational complexity and resource costs for large-scale image databases, and the high cost of obtaining quality data annotations \cite{li-etal-2024-generative}. The application of QBIR to educational resource retrieval is promising but has been hindered by the complexity of educational discourse, the limitations of educational databases, and the associated costs \cite{zhou-etal-2024-vista}.  Our query model effectively combines multi-modal retrieval methods, integrating audio and natural language with multi-style image inputs. The former enables natural and rapid expression of content, while the latter facilitates accurate and intuitive image localization, enhancing educational data retrieval.

\subsection{Prompt Tuning}

Prompt tuning \cite{brown2020language} was first proposed in natural language processing~(NLP) and has been an efficient approach that bridges the gap between pre-trained language models and downstream tasks \cite{li2023weakly}. Prompt tuning leverages natural language prompts to optimize the language model’s ability to understand tasks, which demonstrates exceptional performance in few-shot and zero-shot learning. 
Recent studies have focused on optimizing various components of prompt tuning, such as  prompt generation, continuous prompt optimization \cite{prompt3}, and adapting to large-scale models through methods like in-context learning \cite{dong2024survey}, instruction-tuning \cite{wang2024pandalm}, and chain-of-thought \cite{prompt4}. For example, \citet{lester2021power} leverage soft prompts to condition frozen language models to enhance the performance of specific downstream tasks. \citet{long2024prompt} propose an adversarial in-context learning algorithm, which leverages adversarial learning to optimize task-related prompts.

Furthermore, prompt tuning has gradually become a pivotal technique in computer vision \cite{shen2024multitask}, enabling efficient adaptation of pre-trained models to diverse tasks. Notable methods include visual prompt tuning for classification \cite{jia2022visual}, learning to prompt for continual learning \cite{wang2022learning}, context optimization and conditional prompt learning for multi-modal models \cite{zhou2022conditional}, and prompt-based domain adaptation strategies \cite{ge2023domain}.
For example, \citet{nie2023pro} introduce the pro-tuning algorithm for learning task-specific vision prompts, applied to downstream task input images with the pre-trained model remaining frozen.
\citet{shen2024multitask} leverage cross-task knowledge to optimize prompts, thereby enhancing the performance of vision-language models and avoiding the need to independently learn prompt vectors for each task from scratch.
\citet{cho2023distribution} introduce distribution-aware prompt tuning for vision-language models, optimizing prompts by balancing inter-class dispersion and intra-class similarity.
MaPLe \cite{khattak2023maple} further transfers text features to the visual encoder during prompt tuning to avoid overfitting. These approaches leverage learnable prompts to enhance model performance across various applications. Despite significant advancements in previous research, challenges remain in extracting semantic features from style-diversified images and optimizing templates within cont.inuous prompt tuning. 
In this study, we employ both NLP and visual prompt tuning to optimize STEM educational content retrieval, enhancing retrieval accuracy and efficiency by adjusting prompt tokens.
%VPT vision prompt tuning

\begin{figure}[!tbp]
  \centering
   \includegraphics[width=\linewidth]{imgs/pipeline.pdf}
   \vspace{-5mm}
   \caption{\textbf{The pipeline of Uni-Retrieval.} The image-text/audio pairs are input into their respective modality encoders. During the training procedure, contrastive learning is applied between the modality features of the positive samples~(image-text/audio pairs) and the negative samples. During the inference procedure, the model calculates the similarity between the modality features of the query and the embeddings stored in the database. The retrieved results are ranked, and the performance is evaluated using R@1/R@5 as metrics.}
   \label{fig:pipeline}
   \vspace{-5mm}
\end{figure}

\section{Motivation and Scenarios}
In practical teaching scenarios, teachers often encounter the need for precise image retrieval, such as searching for hand-drawn sketches, student-created artistic images, blurry blackboard drawings captured from a distance, classroom photographs of physical objects, or images from textbooks. However, current retrieval models predominantly focus on text-natural image queries, overlooking the diverse query styles common in educational contexts. This limitation makes it challenging for teachers to efficiently identify and retrieve educational images or texts tailored to diverse teaching scenarios, such as accurately setting learning contexts, articulating key teaching points, presenting instructional materials, and quickly locating supplementary resources.

Our proposed method enables teachers to query various styles' answers with a range of retrieval approaches, including text, image, audio, or combinations of these modalities. This approach ensures fast and convenient retrieval, significantly reducing preparation time for teaching. Once teachers input their queries, our Uni-Retrieval system employs contrastive learning to compare images and text, calculating similarities based on attributes like objects, shapes, quantities, and orientations. The system ranks all database entries by similarity, outputting the top-1 or top-5 results to identify the most relevant and accurate teaching resource images, as illustrated in Fig.~\ref{fig:pipeline}. This approach empowers teachers to manage complex and dynamic teaching scenarios effortlessly, enhancing the clarity and effectiveness of STEM education.



\section{Experiments}
\label{supsubsec:Experiment Settings}

% \subsection{Dataset Construction Details}
% \label{supsec:dataset construction details}

In the database, the texts, their corresponding four images and audio structures for each dataset can share a single index, significantly reducing query time. All images and text are preprocessed using pretrained models to extract features, which are stored as embeddings. This approach eliminates the need for repeated feature extraction during use, saving time and reducing computational overhead, improving the efficiency of the retrieval system.

For the dataset selection, we choose four another datasets except our SER dataset, including the DSR dataset \cite{li2025freestyleret}, the ImageNet-X dataset, the SketchCOCO dataset \cite{gao2020sketchycoco} and the DomainNet dataset \cite{peng2019moment}. We use internVL-1.5 \cite{Chen_2024_CVPR} to annotate the paint/sketch caption for the SketchCOCO and the DomainNet dataset. For the model in the prototype learning module, we choose the VGG \cite{vgg} as the feature extractor. For the baseline selection, we apply two cross-modality pre-trained models (CLIP \cite{clip}, BLIP \cite{blip}), two multi-modality pre-trained models~(LanguageBind \cite{languagebind}, Unified-IO2 \cite{uio2}), two style retrieval models (SceneTrilogy \cite{scenetrilogy}, FashionNTM \cite{fashionntm}), four most recent cross-modality prompt learning models (VPT \cite{jia2022visual}, CoCoOP \cite{zhou2022conditional}, MaPLe \cite{khattak2023maple}, FreestyleRet \cite{li2025freestyleret}), and two database-driven retrieval models (GASKN \cite{GASKN}, MKG \cite{mkg}) for the fair comparison. Specifically, we fine-tune the cross-modality models (CLIP, BLIP) on SER for convergence. We also train the prompt learning models on SER dataset based on VPT's settings to adapt STEM style-diversified inputs. As for the multi-modality models, we evaluate the zero-shot performance on the style-diversified STEM education retrieval task due to multi-modality models' comprehensionability on multi-style image inputs.

For the experiments on the SER dataset, Uni-Retrieval is initialized with OpenCLIP's weights and trained on 8 A100 GPUs with batch size 24 per GPU and 20 training epochs. We use AdamW as the optimizer, set the learning rate to 1e-5 with a linearly warmed up operation in the first epochs and then decayed by the cosine learning rate schedule. The seed is set as 42. What's more, all input images are resized into $224\times224$ resolution and then augmented by normalized operation. All text are padding zero to the max length of 20.

For the fine-tuning CLIP and BLIP models, all experiment settings are the same as Uni-Retrieval except the learning rate is set as 1e-6. For prompt tuning models, we both use 4 prompt tokens to expand the token sequence. For all transformer-based models, we use the ViT-Large and 24-layers text transformer as the foundation models to keep balance between performance and efficiency.







\end{document}
