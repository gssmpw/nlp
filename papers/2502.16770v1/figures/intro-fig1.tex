% \begin{figure*}[ht]
% \setlength{\belowcaptionskip}{-4.5mm}
% \setlength{\abovecaptionskip}{-0.6mm}
%     \begin{center}
%         \subfigure[]{
%         \includegraphics[width=0.39\textwidth]{figures/figs/intro1-left.pdf}
%         \label{fig:intro-left}
%         }
%         \subfigure[]{
%         \includegraphics[width=0.56\textwidth]{figures/figs/intro1-right.pdf}
%          \label{fig:intro-right}
%         }
%         \caption{(a): Comparison results between utilities (math, code) and safety, with reporting accuracy on GSM8K \cite{gsm8k} and Pass@1 rates on HumanEvalPack \cite{humanevalpack} against safety scores on SORRY-Bench \cite{xie2024sorrybench}. Methods bounded by a purple box represent the single-score methods, while method bounded by a green box represent our method.
%          (b): The top depicts the single score misidentification and cross-tasks interference issues, in which the single score (e.g. magnitude) fails to differentiate safety neurons and the gradient induced by neurons across task vectors may cause update collision. The bottom illustrates disjointed important neurons to avoid conflicts during LED merging. }
%     \end{center}
% \end{figure*}



% \begin{figure*}[t]
%     \begin{center}
%         \begin{subfigure}[b]{\textwidth}
%             \centering
%             \includegraphics[width=\textwidth]{figures/figs/chat.pdf}
%             \caption{}
%             \label{fig:intro-chat}
%         \end{subfigure}
        
%         \begin{subfigure}[b]{0.39\textwidth}
%             \centering
%             \includegraphics[width=\textwidth]{figures/figs/intro1-left.pdf}
%             \caption{}
%             \label{fig:intro-left}
%         \end{subfigure}
%         \hfill
%         \begin{subfigure}[b]{0.56\textwidth}
%             \centering
%             \includegraphics[width=\textwidth]{figures/figs/intro1-right.pdf}
%             \caption{}
%             \label{fig:intro-right}
%         \end{subfigure}
%         \vspace{-10pt}
%         \caption{(a): Merging different models often occurs safety-utility conflicts, LLMs may be good at math while tending to output harmful sentences. The current model merging methods are struggling to obtain such an safe AI expert, as shwon in the right conversation. (a): Comparison results between utilities (math, code) and safety, with reporting accuracy on GSM8K \cite{gsm8k} and Pass@1 rates on HumanEvalPack \cite{humanevalpack} against safety scores on SORRY-Bench \cite{xie2024sorrybench}. Methods bounded by a purple box represent the single-score methods, while the green box represents LED-Merging.
%          (b): The top depicts the neuron misidentification and cross-task interference issues, where the single score (\emph{e.g.,} magnitude) fails to differentiate safety neurons and the gradient induced by neurons across task vectors may cause update collision. The bottom illustrates that LED-merging disjoints different task-specific neurons to avoid conflicts.}
%     \end{center}
%     \vspace{-20pt}
% \end{figure*}


\begin{figure*}[t]
    \centering
    \includegraphics[width=\textwidth]{figures/figs/final_intro.pdf}
    \vspace{-15pt}
    \caption{(a): Merging different models suffer from safety-utility conflicts, LLMs may be good at math while tending to output harmful sentences. %The current model merging methods are struggling to obtain such an safe AI expert, as shwon in the right conversation. 
    (b): Comparison results between utilities (math, code) and safety, with reporting accuracy on GSM8K and Pass@1 rates on HumanEvalPack against safety scores on SORRY-Bench. Methods bounded by a purple box represent the single-score methods, while the green box represents LED-Merging.
     (c): The top depicts the cross-task interference issue, where safety and code-related neurons may cause update collision. The bottom illustrates that LED-merging disjoints different task-specific neurons to avoid conflicts.}
     \label{fig:intro-final}
    \vspace{-13pt}
\end{figure*}