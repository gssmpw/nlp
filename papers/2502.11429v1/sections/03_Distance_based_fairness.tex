\section{Amortized Fair Ranking}
\label{ref:section_ranking_def}
Given a query $q$ at time $t$ ($q_t$), we consider a fair ranking task where the goal is to order individuals corresponding to the query optimally~\cite{robertson1977probability}, given relevance score $r_i^t \in \mathbb{R}$ for each individual $i$. The task typically consists of three components: (i) the ``query", (ii) the set of individuals to be ranked, and (iii) the relevance scores. Due to position bias, individuals gain exposure based on their position in the ranking, which directly influences the attention they receive~\cite{biega2018equity,singh2018fairness}. Under the normative principle of equal opportunity, the objective of exposure-based fair ranking is to assign rankings such that the attention allocated to each individual is proportional to their merit~\cite{singh2019policy,zehlike2021fairness,zehlike2022fairness,balagopalan2023role}. In practical terms, merit is operationalized as a value proportional to relevance.

The concept of amortized fair ranking in existing literature seeks to find a sequence of ranking assignments that minimize the discrepancy between the average cumulative attention and the average cumulative relevance of individuals (or groups) over time. Said differently, relevance and attention are averaged over a sequence of queries, and the goal is to ensure fairness over this horizon. In this section, we introduce the notations, definitions, and limitations associated with amortized fair ranking for fair attention allocation. Furthermore, we introduce our distribution and polarity-aware generalization of amortized fair ranking, which results in a more robust solution to the normative goal of equal opportunity.








\subsection{Notation}
Consider a dataset $\mathcal{D}$ of $n$ individuals. Note that we use the term ``individual" here interchangeably with any item or entity being ranked throughout the paper. Each individual $i$ belongs to a group $g \in \Gcal$, where $\Gcal$ represents the set of $G$ possible groups. Let $g_k$ denote the $k^{th}$ group in $\Gcal$ where $k \le G$ and denote group membership as $i \in g_k$ where $g_k \subset \Dcal$. We assume that each individual belongs to exactly one group. Denote $q$ to be a sequence of queries, where queries $q_t$ are submitted at discrete time steps $t \in \Tcal$ and $\mathcal{T} = \{1, 2, \ldots, T\}$. A ranking system accepts each of these $T$ queries independently and returns a distinct ranking of $n$ individuals for each query $q_t \in \Qcal$, where $\Qcal$ denotes the space of all queries.

For each individual $i$ at time $t$, let the binary random variables $X^t_i$ and $Y^t_i$ denote the attention and relevance, respectively. Specifically, $X^t_i \sim \text{Bernoulli}(a^t_i)$~\cite{wang2018position} denotes whether individual $i$ receives attention at time $t$, and $Y^t_i \sim \text{Bernoulli}(r^t_i)$ denotes targets for the attention-distribution based on the relevance of individual $i$ to $q_t$, the query at time $t$. We assume that $X^t_i$ and $X^t_j$ are independent $\forall t$ when $i \neq j$. That is, under the attention models we study, the likelihood of attention is independent of other individuals being ranked, similar to prior work in fair ranking~\cite{biega2018equity,singh2018fairness}. We also assume that queries are independent. Crucially, for each time step $t$, the total attention and relevance are constrained such that \[\sum_{i \in n} a_i^t = 1 \quad \quad \text{and} \quad \quad \sum_{i \in n} r_i^t = 1,\]
such that attention and relevance for individuals are normalized~\cite{biega2018equity} with respect to $n$ individuals at each time step.

Furthermore, denote the cumulative attention and relevance distributions for individual $i$ over the full sequence of queries (all time steps) \[X_i = \sum_{t \in \mathcal{T}} X^t_i \quad \quad \text{and} \quad \quad Y_i = \sum_{t \in \mathcal{T}} Y^t_i,\]
respectively. A glossary of key terms is in Appendix Table~\ref{tab:glossary}.




\begin{theorem} \label{theorem:chernoff}
Let $X_i^t \sim \text{Bernoulli}(p_i^t)$ and
\[
    X_i = \sum_{t\in \Tcal} X_i^t.
\]
Then, for any $\delta > 0$, we have the following:
\[
    P\left(|X_i - \EE[X_i]| \geq \delta \EE[X_i]\right) \leq 2\exp\left(-\frac{\delta^2 \EE[X_i]}{2 + \delta}\right).
\]
\end{theorem}

\begin{remark}
    Note that Theorem \ref{theorem:chernoff}'s bound depends solely on the expected value of the cumulative attention (and relevance), not the number of queries observed.
\end{remark}

Theorem \ref{theorem:chernoff} bounds the likelihood of observing a deviation based on delta $\delta$ from the true cumulative attention for an individual over time $t$. We can apply the same exact bound for cumulative relevance. 


In our setup, the ranking quality at each timestep $t$ is evaluated using the Discounted Cumulative Gain (DCG) at rank $K$, denoted as DCG@K~\cite{jarvelin2002cumulated}. The DCG@K score measures the quality of the top-$K$ ranked individuals based on their relevance, adjusting for the rank position using a logarithmic discount factor.

\subsection{Motivation For Amortized Fairness Across Different Queries}

This work focuses on a class of attention weights where user attention only depends on their ranking position. We assume that attention is proportional to position and follows a distribution informed by domain knowledge. For example, one such distribution used in several prior works is the log-decaying attention distribution~\cite{singh2018fairness,castillo2019fairness}. Under this distribution, at time $t$, if an individual $i$ is at position $j$, their attention score $a^t_i \propto \frac{1}{log(j+1)}$. 

Individual $i$'s attention $X^t_i$ is distributed as $A^t_i=\text{Bernoulli}(a^t_i)$, where $a^t_i$ is normalized.  Ideally, the probability of an individual receiving attention is proportional to their relevance score $r^t_i$, where relevance scores are $0-1$ normalized across all individuals for a given query.  Thus, under a fair ranking, individual $i$'s attention should be similarly distributed as their relevance, i.e., $a^t_i \approx r^t_i$. However, as mentioned above, the rate at which attention decays across positions in a ranking is usually very different from the variation in relevance across individuals. This makes it difficult to match the attention distribution to that of relevance within a single ranking.  Thus, it may be impossible to achieve the targeted fair attention distribution within a single deterministic ranking~\cite{biega2018equity,diaz2020evaluating}.


Alternatively, we compare \emph{cumulative} --- \emph{amortized}  --- attention and relevance over time. We also assume a more realistic multi-query setup since search systems typically process many queries over time. That is, we consider \emph{online ranking} where a sequence of queries (with corresponding relevance score per individual) arrive over time. We post-process the ranking corresponding to each query (without knowledge of the future queries) to improve fairness.

\subsection{Current Limitations}
Current amortized fairness metrics have two primary limitations: 

\emph{Insufficient measures of distributional differences between cumulated attention and relevance.} Current definitions compare \emph{expected (average) attention} ($\sum_{t \in \mathcal{T}} a_i^t$) to \emph{expected (average) relevance} ($\sum_{t \in \mathcal{T}} r_i^t$) across rankings, which leads to less reliably fair solutions for attention and relevance distributions where first moments (means) are not sufficient statistics (see Appendix Figure~\ref{fig:distribution_reliability}). %

\emph{Failure to capture the impact of query polarity:} Fairness definitions in the literature currently assume that all attention is good. However, increased attention in the context of queries with negative connotations relative to other similarly relevant individuals can lead to unfairness (see Appendix Figure~\ref{fig:fairwashing_amortized_ranking}). %
Hence, incorporating query polarity is necessary to model the real-world impact of unfair rankings. %

\subsection{Problem Statement}
We consider (un)fairness to be a function:
\begin{equation}
  f \colon \Pcal(\Xcal) \times \Pcal(\Ycal) \times \RR^T \longmapsto \mathbb{R}
\end{equation}
where $A \in \Pcal(\Xcal)$ denotes a distribution of cumulative attention and $R \in \Pcal(\Ycal)$ denotes a distribution of cumulative relevance for an individual. $Polarity$ is a vector containing the polarity of each of $T$ queries over which attention and relevance are cumulated. A lower value is desired.

Our task is to find a class of such functions such that:
\begin{equation}
    f(A, R, \text{polarity}) = 0 \implies \text{set of rankings is fair}
\end{equation}
We define $f$ to take the form of a scoring function for distribution-aware fairness in ranking (DistFaiR) and identify compatible measures for cumulative attention and relevance distributions in Section~\ref{sec:defining_amortized_distribution_fairness}. We then show that these measures can be modified to depend on query polarity in measurement (Section~\ref{sec:accounting_query_polarity}). Lastly, we test the sensitivity of current fair ranking metrics to polarity. This is an important step to assess \emph{fairwashing} effects in rankings~\cite{aivodji2019fairwashing}.  That is, whether fairness measured using query polarity is higher than that measured without using query polarity.




    









