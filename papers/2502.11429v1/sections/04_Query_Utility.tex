\section{Accounting for Query Polarity}
\newcommand{\polarity}{\eta}
\label{sec:accounting_query_polarity}
Prior work in fair ranking assumes that all attention is positive~\cite{biega2018equity,singh2018fairness} and query independent, implying that achieving a higher rank is universally desirable. However, individuals should not be given higher attention for queries with negative connotations than those with similar relevance. Consequently, we extend our fairness definition to account for query properties such as \emph{sentiment polarity} by introducing a {\em context function} associated with each query. 

In this work, we focus on the scalar sentiment polarity associated with each query. Alternative properties may include the clarity of the query, the perceived economic value associated with being highly ranked for the query, etc. This variable will be influenced --- at least partially --- by the information contained in the query, and may be positive or negative, determining if a higher or lower ranking is more favorable. We also show how this context function can be extended beyond scalars to include a vector of query properties.

Let $\Tilde{X}_i^t$ represent a random variable denoting attention allocated to individual $i$ at time t that incorporates query polarity. Assuming that polarity is searcher-independent (no personalization), it can be decomposed into: (1) the real-world value associated with the attention allocated in response to a query at time $t$ and (2) individual attention. Similar to previous works on fair ranking, we may assume that searcher attention can be modeled well with models like position bias~\cite{chuklin2015click,craswell2008experimental}. %

We denote the context function $\polarity(q_t)$ as the polarity associated with query at time $t$. Query polarity-aware attention and relevance allocated to individual $i$ at time $t$ is then defined as:
\begin{equation}
    \Tilde{X}_i^t = X_i^t \cdot \polarity(q_t) \quad \quad \text{ and } \quad \quad \Tilde{Y}_i^t = Y_i^t \times \polarity(q_t),
\end{equation}
where each corresponds to a cumulative distribution $\Tilde{A}_i$ and $\Tilde{R}_i$, respectively. This formulation is free of two assumptions inherent to the exposure-based fairness metrics: (1) the contribution of exposure to amortized ranking is now dependent on properties of the query and (2) exposure can be any real-valued number. Notably, $\eta(q_t) \in \RR$, including \emph{negative} values and \emph{zero}, unlike previous work. Then, amortized fairness under DistFaiR can be computed over time, with all notations following from the previous section. %
We refer to fairness measures defined in the prior section as \emph{query polarity-agnostic}, and those relying on $\eta(q_t)$ as \emph{query polarity-aware}.

\begin{theorem} \label{theorem:hoeffdings}
Let $X_i^t \sim \text{Bernoulli}(p_i^t)$ and $\eta(q_t) \in [a_t, b_t]$; $a_t, b_t \in \RR$. With a slight abuse of notation, let $\Tilde{X}_i^t = X_i^t\cdot \eta(q_t) \in [a_t, b_t]$ and
\[
    \Tilde{X}_i = \sum_{t\in \Tcal} \Tilde{X}_i^t,
\]
Then, for any $\delta > 0$, we have the following:
\[
    P\left(|\Tilde{X}_i - \EE[\Tilde{X}_i]| \geq \delta\right) \leq 2\exp\left(-\frac{2\delta^2}{\sum_{t \in \Tcal} (b_t - a_t)^2}\right).
\]
\end{theorem}

\begin{remark}
    Unlike in Theorem \ref{theorem:chernoff}, the bounds in Theorem \ref{theorem:hoeffdings} now depend on both the expected value of the cumulative attention (and relevance) and the number of queries observed.
\end{remark}

Thereom \ref{theorem:chernoff} bounds the likelihood of observing a given deviation $\delta$ from the true polarity-aware cumulative attention for an individual over time $t$. We can apply the same exact bound for polarity-aware cumulative relevance.








