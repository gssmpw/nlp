\section{Literature Review}
Addressing grid connection constraints requires a deeper understanding of the geographical factors that influence solar PV deployment. Previous studies have highlighted a range of key factors that impact deployment, including social, economic, climatic, and land use variables. Population is positively correlated with solar PV deployment at the country level globally \cite{alghanem_global_2024}. This is expected, as countries with larger populations generally require more electricity, which drives higher solar PV installation rates. However, the relationship between solar PV deployment and population becomes more complex when examined at regional and subregional scales. For example, studies have found a negative correlation between solar PV deployment and population at the regional level in both Germany \cite{mayer_deepsolar_2020} and the UK \cite{balta-ozkan_regional_2015} and at the subregional level in Australia \cite{fuentes_solar_2024}. In the United States, \citet{yu_deepsolar_2018} observed that residential solar PV deployment peaks at a population density of 1,000 people per square mile. Areas with very high population densities tend to have lower levels of small-scale residential solar PV deployment, as urban environments often lack suitable rooftops for installations. Conversely, regions with medium population densities are more likely to have higher solar PV capacities due to the prevalence of detached houses with rooftops that are ideal for solar PV systems. 

Education is positively correlated with solar PV deployment at the country level globally \cite{alghanem_global_2024}, at the regional level in the UK \cite{balta-ozkan_regional_2015}, and at the subregional level in Connecticut, USA \cite{yu_deepsolar_2018}, and England \cite{laura_williams_identifying_2012}. Countries with higher education levels tend to have stronger economies, which often translates into higher electricity consumption and greater investment in solar PV installations. Furthermore, higher levels of education are associated with increased environmental awareness and pro-environmental behaviour \cite{meyer_does_2015,wang_green_2022,ozbay_exploring_2022}, which can further drive solar PV adoption.

The influence of neighbours positively impacts solar PV adoption at the subregional level in both the United States and Australia \cite{graziano_spatial_2015,fuentes_solar_2024}.

Average household size has a mixed impact on solar PV deployment. In the UK, it is positively correlated with deployment at the subregional level \cite{collier_distributed_2023}. This positive relationship could be influenced by larger energy bills or increased daytime electricity usage. Conversely, a negative relationship is observed at the regional and subregional levels in the UK and at the subregional level in Australia \cite{balta-ozkan_regional_2015,alderete_peralta_spatio-temporal_2022,fuentes_solar_2024}. This negative relationship may be explained by larger households having reduced cash flow or prioritizing aesthetics over cost savings from solar PV systems \cite{balta-ozkan_regional_2015}.

Household size may also be linked to age demographics, which play a significant role in solar PV adoption. Individuals above age 40 are more likely to have greater access to cash for solar PV investments, while those aged 25–40 may face financial constraints that limit their ability to install solar PV systems. In England, a higher share of the population above age 40 is associated with greater PV installation at the subregional level \cite{laura_williams_identifying_2012}. Conversely, in Australia, a higher share of the population aged 25–40 has a negative impact on PV adoption at the subregional level \cite{fuentes_solar_2024}.

Income shows a mixed relationship with solar PV deployment. In the UK, income is positively correlated with PV adoption at the subregional and regional levels, and similarly, a positive correlation is observed at the subregional level in the United States \cite{alderete_peralta_spatio-temporal_2022,balta-ozkan_regional_2015,yu_deepsolar_2018}. However, a negative correlation between income and solar PV adoption has been found at the subregional level in Australia \cite{fuentes_solar_2024}. This mixed relationship may be explained by differences in motivation. While individuals with higher incomes can afford to install solar PV systems, they may choose not to if they are not concerned about cost savings or if they dislike the aesthetics of solar panels.

GDP is strongly correlated with solar PV capacity at the country level globally and in China \cite{alghanem_global_2024,liu_forecasting_2022}. However, in other studies, GDP does not show a proportional relationship to a country's solar PV capacity \cite{celik_review_2020}. At the regional level in Germany, GDP is negatively correlated with PV deployment \cite{mayer_deepsolar_2020}.

Industrial added value is positively correlated with solar PV deployment at the country level \cite{alghanem_global_2024,liu_forecasting_2022}. This relationship may reflect the fact that countries with higher industrial output tend to have more resources and investments available for renewable energy technologies, including solar PV. Additionally, industrial sectors often benefit from renewable energy adoption both through direct use in manufacturing processes and through the promotion of green technologies, which may further drive solar PV deployment.

Electricity consumption is positively correlated with solar PV deployment at the country level globally \cite{alghanem_global_2024}. This relationship extends to the regional and subregional levels in the UK, where higher electricity consumption is also associated with greater PV deployment \cite{alderete_peralta_spatio-temporal_2022,laura_williams_identifying_2012,balta-ozkan_regional_2015,collier_distributed_2023}.

Solar radiation has been found to have a positive correlation with solar PV deployment at the regional and subregional levels \cite{westacott_novel_2016,yu_deepsolar_2018,aklin_geography_2018,balta-ozkan_regional_2015}. This is logical, as regions with abundant sunlight provide more favourable conditions for solar power generation, making solar PV an attractive energy option. However, other studies suggest that a country's solar radiation potential is not always proportional to solar PV deployment \cite{celik_review_2020,alghanem_global_2024}. This discrepancy may be linked to the geography of a country or region. In countries with abundant agricultural land receiving high levels of solar radiation, there is often more solar PV capacity, as large open spaces are ideal for solar farms. In contrast, countries such as Austria, where two-thirds of the land is covered by the Alpine mountains \cite{kanonier_arthur_spatial_2018}, and where solar radiation is concentrated in urban areas, may face challenges in deploying solar PV, as urban environments typically offer limited space for large-scale installations. 

% In contrast, countries such as Japan and the Netherlands, where solar radiation is concentrated in urban areas may face challenges in deploying solar PV, as urban environments typically have limited space for large-scale installations.



Rural areas are positively correlated with solar PV deployment \cite{westacott_novel_2016,laura_williams_identifying_2012,aklin_geography_2018}. This is likely due to the availability of space for solar installations and the higher likelihood that rural areas have houses with suitable rooftops for PV systems. 

Solar PV is correlated with agricultural areas globally \cite{kruitwagen_global_2021,alghanem_global_2024,van_de_ven_potential_2021}, as well as with the gross value added by agriculture in Germany \cite{mayer_deepsolar_2020}. This is expected, as large-scale PV installations are often sited on agricultural land.

These studies show that the relationship between solar PV deployment and geographical factors is complex and varies depending on the geographical region and analysis resolution (national, regional, subregional). These factors are crucial in determining where and how solar PV systems are installed. By incorporating them into a comprehensive model, we can better predict regional solar PV deployment patterns, identify areas with high potential for future installations, and uncover barriers to deployment in underserved regions.

None of the existing studies comprehensively account for all types of solar PV capacity (residential, commercial, and utility-scale) installed in a country. A key challenge in modelling regional and subregional PV capacity is the limited availability of relevant social, economic, climatic, and land use data. This study introduces the first comprehensive regional PV capacity model that disaggregates national capacity across 168 NUTS 3 regions. The model will be used to allocate unknown capacity to specific geographic regions, as a benchmarking tool for assessing regional performance, and to serve as a forecasting tool. 






% None of the existing studies on the UK comprehensively account for all types of solar PV capacity (residential, commercial, and utility-scale) across England, Wales, and Scotland. A key challenge in modelling regional and subregional capacity is the limited availability of high-resolution data for these regions. To address this limitation, this study uses land cover data as a proxy for geographical factors. It is the first to model regional PV capacity at the NUTS 3 level across Great Britain using publicly available datasets. The models developed will be used to disaggregate national solar PV capacity into regional levels, function as benchmarking tools for assessing regional performance, and serve as forecasting tools.