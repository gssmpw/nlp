\pdfoutput=1
\documentclass[11pt]{article}
\usepackage{amsmath}
\usepackage[final]{acl}
\usepackage{times}
\usepackage{latexsym}
\usepackage[T1]{fontenc}
\usepackage[utf8]{inputenc}
\usepackage{microtype}
\usepackage{inconsolata}
\usepackage{graphicx}
\usepackage{booktabs}
\usepackage{multirow}
\usepackage{subfigure}
\usepackage{amsmath}
\usepackage{amssymb}
\usepackage{hyperref}
\usepackage{float}
\usepackage{multirow}
\usepackage{booktabs}
\usepackage{amsmath}
\hyphenpenalty=5000
\tolerance=1000

\newcommand{\MODELNAME}{\textsc{VerifyScore}}
\newcommand{\jc}[1]{\textcolor{red}{[JC: #1]}}
\newcommand{\hy}[1]{\textcolor{blue}{[HY: #1]}}

\newcommand{\jy}[1]{\textcolor{purple}{[JY: #1]}}

\title{Constraining Sequential Model Editing with Editing Anchor Compression}


\author{
Hao-Xiang Xu$^1$\thanks{Equal contribution.}, Jun-Yu Ma$^1$\footnotemark[1], Zhen-Hua Ling$^1$\thanks{Corresponding author.}, Ningyu Zhang$^2$, Jia-Chen Gu$^3$  \\
  $^1$National Engineering Research Center of Speech and Language Information Processing, \\
      University of Science and Technology of China, Hefei, China \\
  $^2$Zhejiang University \\
  $^3$University of California, Los Angeles \\
{\tt \{nh2001620,mjy1999\}@mail.ustc.edu.cn}, {\tt zhling@ustc.edu.cn}, \\ 
{\tt zhangningyu@zju.edu.cn}, {\tt gujc@ucla.edu} 
}


\begin{document}
\maketitle
\begin{abstract}
Large language models (LLMs) struggle with hallucinations due to false or outdated knowledge. Given the high resource demands of retraining these models, there is an increasing focus on developing \emph{model editing}. 
However, the general abilities of LLMs across downstream tasks are prone to significant degradation during sequential editing.
This paper statistically observes that the parameter matrix after editing exhibits a significant deviation compared to its previous state as the number of edits increases.
This serious deviation affects the original knowledge associations within LLMs and leads to the degradation of their general abilities.
To this end, a framework termed \textbf{E}diting \textbf{A}nchor \textbf{C}ompression (EAC) is proposed to constrain the deviation of the parameter matrix during sequential editing. 
It compresses the editing information by selecting editing anchors that are important in encoding new relations without deviating too much from the original matrix, thereby preserving the general abilities. 
Experiments of applying EAC to two popular editing methods on three LLMs across four tasks are conducted. 
Evaluation results show that EAC effectively minimizes unreasonable deviations caused by model editing, preserving over 70\% of the general abilities while better retaining the editing knowledge compared to the original counterpart methods\footnote{Code is available: \href{https://github.com/famoustourist/EAC}{https://github.com/famoustourist/EAC}}.
\end{abstract}

\section{Introduction}

% State of the world (robots for creative activites)
The term ``robot,'' originally signifying `forced labor,' has long been associated with labor and work. Robots have demonstrated their utility in various automated productive and social contexts, where the primary goals are improving productivity, safety, and fostering social interactions with humans~\cite{simoes2022designing, weidemann2021role, honig2018understanding}. However, an increasing number of cases feature using of robots in creative settings. Unlike productive contexts, where the focus is on efficiency and task completion~\cite{arents2022smart}, or social contexts, where communication and trust are prioritized~\cite{nam2020trust, saunderson2019robots}, creative environments prioritize artistic innovation and expression~\cite{hsueh2024counts}. This shift fundamentally alters the dynamics of human-robot interaction, redefining the roles and expectations for both humans and robots.

For instance, robots’ social behaviors are leveraged to support the generation and expression of creative ideas~\cite{hu2021exploring, sandoval2022human, alves2020creativity}, and programmable robotic movements and trajectories are employed to inspire artistic activities such as sketching~\cite{lin2020your}. These studies often engage participants from creative fields who possess limited prior experience with robotics, and are typically conducted in short-term, experimental settings. Consequently, the findings from these studies remain constrained since much can be learned from professional practitioners' experiences to inform system design such as digital fabrication~\cite{hirsch2023nothing}. There is a notable gap in research examining the long-term, active, and practical experience of integrating robotic systems into the creative processes. As a result, the deeper insights into how robots facilitate and shape creative processes, beyond simply augmenting human creativity, remain underexplored. In this study, we aim to better understand the impacts of robots on creative processes and outcomes.

As early as Leonardo da Vinci's 16th century ``Automaton,'' artists have explored the creative affordances of robotic systems~\cite{shanken2002cybernetics, pagliarini2009development, jeon2017robotic}. The artistic creation process typically encompasses various stages, including the exploration of materials and techniques, ongoing experimentation and iteration, and the continual refinement of the artists' insights into their creative subjects~\cite{lewis2023art, sturdee2022state}. Therefore, investigating the artistic process involving robots offers an opportunity to gain deeper insights into robots' creative potential. Robotic art, in particular, provides a compelling case for this exploration.

We define robotic art as artworks that utilize robotic or automated machines to create artistic experiences and tangible artifacts. One example is robotic installation art, in which robots are programmed to follow specific rules that embody the artist’s expression (\autoref{fig:teaser} (a)). Another example is responsive art, in which robots react to their environment, with behaviors that change over time or in response to spectators (\autoref{fig:teaser} (b)). Additionally, there are robotic creators, which possess a degree of agency, allowing them to collaborate with human artists and produce works that extend beyond mere replication of human-created art (\autoref{fig:teaser} (c) and (d)). As such, robotic art becomes a rich case for exploring human-machine interactions in creative contexts. Gaining a deeper understanding of how robots facilitate artistic expression can provide insights for designing computing systems to support creative activities~\cite{gomez2021robot}.

% Therefore, we did...
We draw on semi-structured, in-depth interviews with renowned professional robotic artists to investigate the use of robots in artistic practice. Specifically, our goal is to understand how artistic exploration of robotic systems challenges conventional assumptions about the functions of robots, such as their roles in automating repetitive tasks or serving human needs. We also explore the implications of robots in the artistic process and examine how creativity may emerge within robotic art. To address these interrelated inquiries, our study focuses on the practice of robotic art, posing the research question: \textit{How do robotic artists utilize robots in their artistic practice?} We approach this inquiry through the perspectives and experiences of robotic artists, who creatively design, modify, and repurpose robotic systems for artistic expression and exploration.

% The key findings are...
Our findings highlight the social, material, and temporal dimensions of artists' practices that shape their creativity and artistic outcomes. The creation of robotic art is largely a social process, as artists receive both explicit and implicit feedback through the audience's reactions and reception of their work. Simultaneously, the embodiment and malfunctions inherent to robotic systems drive artistic experimentation. The temporal processes of creation and exhibition, beyond just the final product, further enhance the creative value. Our empirical analysis presents how creativity emerges through the interplay of social, material, and temporal interactions among artists, robots, audiences, and the environment.

% The contributions of this work are...
We make two main contributions to HCI in this study. 
First, we elucidate the interactive mechanisms among key actors---human creators, machines, audiences, and environments---within the practice of robotic art, a topic that remains underexplored in HCI. Our findings reveal the significance of sociality (e.g., interactions between artists and audiences), materiality (e.g., the embodiment and malfunctions of robots), and temporality (e.g., the processes of creation and exhibition) in shaping creative values. We propose that these three facets are central to the creative process and facilitate the emergence of creativity in robotic art.
Second, drawing from the findings, we offer implications for \textit{socially informed}, \textit{material-attentive}, and \textit{process-oriented} creation with computing systems. We suggest leveraging these three aspects to enhance creativity and the creative experience. Specifically, we discuss the value of incorporating implicit audience feedback, designing with technical malfunctions, and focusing on the post-creation process to foster alternative creative experiences with machines~\cite{alter2010designing, juarez2022glitch}.



\section{Related Work}
\label{sec:related}
\subsection{Collaborative Systems}
In the era of rapid growth in medical foundational models~\cite{huang2023visual,wang2022medclip, zhang2024data}, the top-down model development paradigm limits model capabilities by heavily relying on the resources available to the model builders. 
Such paradigm often restricts the potential of these models, as they cannot effectively utilize the diverse, private, and decentralized resources that exist within the broader medical community.
In contrast, collaborative systems present a promising alternative, offering a more flexible approach to model development.

Collaborative systems enable institutions to share knowledge by allowing distributed collaborators to contribute to a common goal~\cite{boulemtafes2020review}. 
To further protect patient privacy, federated learning (FL)~\cite{mcmahan2017communication} was proposed to alleviate such privacy concerns as server aggregating parameter updates from multiple clients without sharing their local data. 
While subsequent optimizations, such as aggregation algorithms~\cite{mcmahan2017communication, zhao2018federated, li2020federated}, secure learning~\cite{hardy2017private, xie2021crfl}, fairness improvements~\cite{sharma2022federated, zhao2022dynamic} and its application in medicine~\cite{kumar2024privacy}, have enhanced the capacity and applicability of FL, its real-world flexibility remains limited. This is primarily due to the need for synchronous updates, which require the server and clients to stay in sync, or model updates will be blocked.
This synchrony issue can be mitigated by open-source software platforms (e.g., GitHub~\cite{github}), allowing independent contributions from individual developers asynchronously. Such an asynchronous scheme enables faster iteration and the integration of specialized expertise, thus offering a more flexible and scalable approach.

Unlike synchronous collaboration, asynchronous collaboration does not require collaborators to work simultaneously and collaborators can individually complete their updates.
While the concept of asynchronous collaboration has been widely used in software development, its machine-learning applications remain limited~\cite{kandpal2023git, raffel2023building}. 
With the rise of global collaboration, large models~\cite{sahajBERT, le2023bloom} are usually co-developed by collaborators given various levels of data availability. However, this collaborative scheme requires the aggregation of local data and online synchronous cooperation of developers.
Software-like model update system~\cite{raffel2023building} alleviates the synchronous problem, where models are updated incrementally, similar to software development, by introducing merging and version control to model development.
However, the existing collaborative version control system~\cite{kandpal2023git} fails to address the complexities of medical scenarios because of the heavy dependency on plain parameter averaging across the full model without accounting for the varying requirements of different tasks.
To respond, we propose MedForge, which enables an asynchronous collaborative system and ensures strong robustness toward a continuous, community-driven enhancement of medical models while overcoming potential data leakage.

\begin{figure*}[t]
\begin{center}
\includegraphics[width=.85\linewidth]{fig_overview_v3.pdf}
\end{center}
\caption{
FastAtlas Overview: In each frame, we compute charts spanning fully or partially visible triangles (a), determine texture space bounding boxes for the visible portions of the view-space projections of each chart, and tightly pack these boxes into atlases (b, here $2K \times 2K$). We simultaneously bijectively parameterize and shade the charts into their atlas boxes, obtaining high quality texture space shading (c), and use this shading to render the shaded frames (d).}
\label{fig:overview}
\label{fig:alg_overview}
\end{figure*}

\section{Overview}
\label{sec:overview}
Our work has two core contributions: a real-time, GPU-based algorithm for tight packing of general parameterized charts into compact atlases; and a real-time TSS method that
utilizes this packing.  

\paragraph*{FastAtlas Packing.}
FastAtlas runs entirely on the GPU as a series of compute shaders. It takes the bounding boxes of parameterized charts as input, and packs them into an atlas (Fig~\ref{fig:overview}b, Sec.~\ref{sec:pack}). As such, the only input it requires are the dimensions of the bounding boxes.
Its outputs are deterministic; identical input charts are packed into identical atlases. This is critical for TSS and similar applications, as it ensures that consecutive frames taken from the same camera view have the same shading. Even minute shading differences across such frames can cause sampling jitter, leading to undesirable flicker \cite{baker2012rock}. 
While prior methods such as \cite{mueller2018shading,hladky2019tessellated,hladky2021snakebinning,Neff2022MSA} cap the dimensions of the charts that can be packed as-is for a given atlas size, and scale down all charts that exceed these dimensions, we scale all charts by the same factor, and do so only when strictly necessary to achieve packing success (Figs~\ref{fig:atlas},~\ref{fig:sas_issues}). 

\paragraph*{TSS using FastAtlas.}
Our end-to-end TSS atlas generation method combines the packing method above with a novel approach for computing seamless per-frame charts. 
We define our charts as the connected components of the visible surfaces in each frame (Fig.~\ref{fig:overview}a), and efficiently compute them using a parallel union-find algorithm (Sec.~\ref{sec:visible}). Since the boundaries of these charts coincide with the contours of the rendered surface, they are {\em invisible} to the viewer. This approach 
eliminates the artifacts caused by shading discontinuities along visible seams (Fig.~\ref{fig:seams}). 

\begin{parWithWrapFigure}
\begin{wrapfigure}{l}{.27\columnwidth}%
\includegraphics[width=\linewidth]{fig_inset_view_plane.pdf}%
\end{wrapfigure}
We bijectively parametrize the {\em visible portions} of our charts by projecting them to view space (inset). This maps a constant number of texels to each pixel in the final rendered output, evenly distributing residual undersampling error across all image pixels. While conceptually straightforward, efficiently parameterizing charts containing partially visible triangles using viewspace projection is non-trivial, as the visible portions may no longer be triangular (e.g. green triangle in the inset); applying naive projection to triangles with vertices behind the camera may produce ill-posed results. Clipping triangles before projection is both computationally expensive and significantly complicates downstream operations. We avoid explicit clipping by observing that all that is required for atlas packing is the dimensions of, potentially conservative, bounding boxes of these projected visible portions. We compute such bounding boxes without explicit chart clipping by adapting a conservative screen coverage estimator \shortcite{Blinn:CalculatingScreenCoverage} (Sec.~\ref{sec:box}). We then pack the computed boxes using FastAtlas. 
\end{parWithWrapFigure}

Finally, we shade the visible portion of each chart into its corresponding atlas bounding box (Fig~\ref{fig:overview}c). 
The resulting texture is then used during rasterization as a standard texture map (Fig. ~\ref{fig:overview}d). 
Our framework is compatible with all existing approaches for texture space shading, including forward shading, raytraced illumination, or deferred shading in texture space \cite{baker:2016}. In the examples shown, we use the standard forward shading based rendering pipeline included in the G3D Innovation Engine \cite{G3D17}, a commercial grade renderer.


\subsection{Model Merging}
In collaborative systems, proper model merging becomes increasingly vital for improving model knowledge integration from multiple sources in a resource-limited environment~\cite{li2023deep, yang2024model, goddard2024arcee}. Conceptually, model merging strategies can be categorized into entire model merging and partial model merging.

Entire model merging involves combining multiple model parameters to participate in the merging process by several means. Entire model merging can be viewed as an optimization problem~\cite{Matena_Raffel_2021, jin2022dataless, mavromatis2024packllm} or an alignment problem~\cite{ainsworth2022git, jordan2022repair, xu2024training, ainsworth2022git}, each offering unique advantages depending on the task at hand.
In the optimization-based approach, the goal is to find the best combination of multiple models to enhance performance and efficiency. For instance, using Fisher information approximation~\cite{Matena_Raffel_2021}, the optimization-based model merging can be interpreted as selecting parameters that maximize the joint likelihood of the models' posterior distributions. The optimization of model merging can also be guided by minimizing the prediction differences between the merged model and individual models~\cite{jin2022dataless}. 
With the development of large language models (LLM), optimization-based method is used to fuse multiple LLMs at test-time by minimizing perplexity over the input prompt~\cite{mavromatis2024packllm}.
To highlight, optimization-based methods are beneficial for scenarios requiring enhanced model performance and efficiency to integrate model parameters, while alignment-based methods~\cite{ainsworth2022git, jordan2022repair} are better suited for maintaining consistency and interpretability, facilitating critical information sharing across models.
For example, a training-free model merging strategy aligns relevant models by using a similarity matrix of their representations in both activation and weight spaces~\cite{xu2024training}.
Further, the alignment between the independently trained model and a reference model not only works for models with the same architecture but also for arbitrary model architectures~\cite{ainsworth2022git}.
In summary, the entire model merging methods can effectively integrate existing models into a merged model with enhanced functionality. However, they could lead to increased computational complexity and reduced flexibility, making them less scalable and harder to implement across diverse tasks.

Partial model merging refers to combining only specific components or layers of models to improve model merging efficiency and decrease the computational cost. 
Such specific components can come from the same network~\cite{kingetsu2021neural}, where the original network is divided into subnetworks for different purposes, and these subnetworks can then be recombined for new tasks.
Additionally, modules can originate from different functional networks and be merged using various strategies. For instance, arithmetic operations are applied in \cite{zhang2023composing} to fuse parameter-efficient modules.
While merging modules from different networks provides flexibility, the process also requires a selection strategy to ensure the resulting model aligns with the specific needs of the inference stage. 
The selection strategies are commonly designed based on the similarity of task~\cite{lv2023parameter} and domain clustering performance~\cite{chronopoulou2023adaptersoup}. Alternatively, the mixture-of-experts methods use a routing strategy to select appropriate component modules~\cite{ponti2023combining}. However, these strategies often require significant time and computational resources to filter through a large model pool. 
In contrast, LoRAHub~\cite{huang2023lorahub} offers a more lightweight approach, combining various LoRA modules for different tasks with minimal model training. Nevertheless, LoRAHub lacks flexibility for incremental updates, especially when handling unseen tasks.

Although the existing model merging approaches effectively combine the capabilities of individual models, these approaches often rely on raw data, leading to potential privacy risks. Our proposed MedForge emphasizes the prevention of raw data usage, which is particularly crucial in medical scenarios. Additionally, MedForge offers an extensible capability for incremental learning, enabling continuous model improvement.



\section{Preliminaries}
\noindent \textbf{Variational autoencoder.}
A variational autoencoder (VAE) ~\citep{kingma2013auto} is a generative model that represents high-dimensional data distributions in a lower-dimensional latent space. 
The encoder maps the input data $\mathbf{x}$ to a latent variable $\mathbf{z}$ by estimating the parameters of a posterior distribution $q_{\phi}(\mathbf{z}|\mathbf{x})$. The posterior is typically assumed to follow the Gaussian distribution, parameterized by a mean $\mu_{\phi}(\mathbf{x})$ and a variance $\sigma_{\phi}(\mathbf{x})$. The latent variable $\mathbf{z}$ is sampled from this posterior distribution, i.e., $\mathbf{z} \sim q_{\phi}(\mathbf{z}|\mathbf{x}) = \mathcal{N}(\mathbf{z}; \mu_{\phi}(\mathbf{x}), \sigma_{\phi}(\mathbf{x})^2)$. The decoder reconstructs the input $\mathbf{x}$ by mapping $\mathbf{z}$ back to the data space through the likelihood $p_{\theta}(\mathbf{x}|\mathbf{z})$. The learning objective of is:
\begin{equation}
\mathcal{L}_{\text{VAE}}(\theta, \phi; \mathbf{X}) = \mathbb{E}_{q_{\phi}(\mathbf{Z}|\mathbf{X})}[\log p_{\theta}(\mathbf{X}|\mathbf{Z})] - \text{KL}(q_{\phi}(\mathbf{Z}|\mathbf{X}) \| p(\mathbf{Z})).
\label{eq:vae}
\end{equation}



\noindent \textbf{3D Gaussian Splatting.}
3DGS~\citep{kerbl3Dgaussians} is an efficient NVS framework that uses a set of 3D Gaussian primitives to represent a scene explicitly. Each Gaussian primitive has a position vector $\boldsymbol{\mu} \in \mathbb{R}^3$, a 3D covariance matrix $\boldsymbol{\Sigma} \in \mathbb{R}^{3\times 3}$, an opacity $\alpha \in \mathbb{R}$, and a spherical harmonics (SH) coefficient  $\boldsymbol{c} \in \mathbb{R}^k$ \citep{ramamoorthi2001efficient} representing the view dependent colors.
\begin{equation}
G(x)=e^{-\frac{1}{2}(x-\boldsymbol{\mu})^T \boldsymbol{\Sigma}^{-1}(x-\boldsymbol{\mu)}},
\end{equation}
where ${\Sigma}={R}{S}{S}^T{R}^T$, ${S}$ denotes the scaling matrix and ${R}$ is the rotation matrix. Then, rasterization~\citep{zwicker2001surface} can transform the 3D Gaussian spheres to the 2D camera plane to calculate the 2D covariance matrix in the camera space as 
\begin{equation}
{\Sigma}^{'} = {J}{W}{\Sigma} {W}^T{J}^T,
\end{equation}
where ${W}$ is the perspective transformation matrix and ${J}$ is Jacobin of the projection matrix.
For every pixel, the Gaussians are traversed in depth order from the image plane, and their pixel colors $c_i$  are combined through alpha compositing, forming pixel color ${C}$ as
\begin{equation}
{C}=\sum_{i \in N} c_i \alpha_i \prod_{j=1}^{i-1}\left(1-\alpha_j\right).
\end{equation}


\section{Analysis of Ability Degradation}
\label{analysis}

ROME~\cite{DBLP:conf/nips/MengBAB22} and MEMIT~\cite{DBLP:conf/iclr/MengSABB23} are 
 currently popular model editing methods. Given that MEMIT builds upon the foundations of ROME by implementing residual distribution across multiple layers,  our analysis in the main text focuses primarily on ROME. 
This section presents a detailed analysis of how the model is affected during sequential editing using ROME. 
Statistical and visual analyses reveal that the degradation of general abilities is related to the unintentional introduction of the non-trivial noise that can make the parameter matrix after editing deviate from its original semantics space.

\subsection{Comparison with Fine-tuning Approach}
First, a statistical analysis is conducted by editing GPT2-XL~\cite{radford2019language} using the ZsRE~\cite{DBLP:conf/conll/LevySCZ17} dataset. 
Considering the L1 norm effectively quantifies the absolute changes in parameter values pre- and post-editing, while providing insights into feature weight distributions within the matrix, it is used to represent the degree of change in the parameter matrix.
As illustrated in Figure~\ref{fig-edit}, when using editing-based methods such as ROME and MEMIT, the L1 norm of the matrix at the edited layer increases significantly with the number of edits.
It can be seen that the norm increases by 317\% (ROME) and 61\% (MEMIT), respectively by the end of sequential editing.
This result highlights a significant deviation from the unedited model, emphasizing the impact of sequential edits on stability.

\begin{figure}[t]
  \subfigure[Editing-based methods]{
  \includegraphics[width=0.22\textwidth]{figures/L1-Norm-GPT2XL.pdf}
  \label{fig-edit}}
  \subfigure[Fine-tuning approach]{
  \includegraphics[width=0.22\textwidth]{figures/finetune-GPT2XL.pdf}
  \label{fig-finetune}}
\vspace{-2mm}
\caption{Illustration of the change of L1 norm 
(a) in sequential editing at the edited layer using editing-based methods and 
(b) in fine-tuning different batch steps for selected layers.
Here we uniformly selected the layers of GPT2-XL for clarity when fine-tuning.} 
\vspace{-2mm}
\end{figure}

A gradient-based fine-tuning approach can markedly enhance the performance of the model on specific tasks while preserving its general abilities across other downstream tasks~\cite{DBLP:journals/corr/abs-2312-12740, DBLP:journals/corr/abs-2310-10047}. 
As depicted in Figure~\ref{fig-finetune}, there are no significant changes in the norm of the parameter matrix for the given layers, with a maximum change of only 0.27\%, even as the amount of fine-tuning knowledge increases. This stability in the parameter matrix norms suggests that the fine-tuning approach does not introduce significant non-trivial noise during the editing process. Thus, fine-tuning maintains the integrity and stability of the model's parameters, which is crucial for preserving its general abilities and preventing unintended non-trivial noise.

This comparison indicates that each editing method not only updates the intended fact as expected but also unintentionally introduces non-trivial noise into the model. This noise manifests as deviations in the parameter matrix during the sequential editing process. With each additional edit, the noise accumulates, progressively increasing the deviation in the parameter matrix. Consequently, as the number of edits grows, there is a significant deviation in the parameter matrix observed before and after the editing. This accumulated noise highlights the challenge of maintaining the stability and integrity of the parameter matrix through multiple edits, which can ultimately impact the general abilities of the model.

\begin{figure}[t]%[!htb]
  \subfigure{
  \includegraphics[width=0.45\textwidth]{figures/legend_visible.pdf}}
  \subfigure{
  \includegraphics[width=0.22\textwidth]{figures/pca-gpt2-right.pdf}}
  \subfigure{
  \includegraphics[width=0.22\textwidth]{figures/pca-llama3-right.pdf}}
\vspace{-2mm}
\caption{Visialization of six sets of facts recalled by LLMs using 2-dimensional PCA. Note that this hidden state is also projected by a language modeling head (linear mapping) for next-token prediction, implying the linear structure in the corresponding representation space (the PCA assumption).} 
\vspace{-3mm}
\label{fig-pca}
\end{figure}

\subsection{Visualization Analysis} \label{sec_visual}
Following the ROME, the second layer of MLP \( W^{(l)}_{\text{proj}} \) is viewed as a linear associative memory~\cite{anderson1972simple, DBLP:journals/tc/Kohonen72}. This perspective observes that any linear operation \( W \) can operate as a key-value store for a set of vector keys \( K = [\mathbf{k}_1 | \mathbf{k}_2 | \dots] \) and corresponding vector values \( V = [\mathbf{v}_1 | \mathbf{v}_2 | \dots] \), by solving \( WK \approx V \)~\cite{DBLP:conf/nips/MengBAB22}.
The key-value pair \( (\mathbf{k}_i, \mathbf{v}_i) \) represents the representation of the input prompt, where $\mathbf{k}_i$ identifies patterns of the input and $\mathbf{v}_i$ is the fact recalled by the model, which is considered to gather all the information about how the model understands the prompt and how it will respond. By stacking $\mathbf{k}_i$ and $\mathbf{v}_i$ separately for each prompt, matrix \( K \) and \( V \) are obtained.
Based on this, 200 prompts of the same downstream task are collected to compute \( K \) and \( V \). 

On GPT2-XL and LLaMA-3 (8B), Principal Component Analysis is employed to visualize the hidden state of the facts of the downstream task recalled by the model. 
The first two principal components of six sets of facts, representing most features, are computed~\cite{zheng2024prompt}. Two of these are derived from recalling the model before editing and the model after editing without any constraint, respectively.
To explore the relationship between the deviation of the parameter matrix after editing and the resulting degradation of general abilities, four additional settings were tested by setting different percentages of the columns in the update matrix to zero, evenly distributed according to an arithmetic progression.

As illustrated in Figure~\ref{fig-pca}, the principal
components of facts recalled by the original model and the edited model without any constraint can be largely distinguished, whose boundaries (black dashed lines) can be easily fitted using logistic regression.
This indicates a significant semantic discrepancy between the facts recalled by the unconstrained edited model and the original model, explaining the decline in general abilities is related to matrix deviation.
Furthermore, when the deviation of the parameter matrix is constrained by reducing the norm of the update matrix, as shown by the black arrows, the principal components of the recalled facts by the edited model gradually align with those of the original model.
This shows that by reducing the norm of the update matrix, the deviation of the parameter matrix after editing can be constrained, making the semantic distribution of the model before and after editing similar, thereby preserving the general abilities of the edited model.
\section{EAC: \underline{E}diting \underline{A}nchor \underline{C}ompression}
\label{method}

\begin{figure}[t]
  \centering
  \includegraphics[width=0.48\textwidth]{figures/EAC.pdf}
  \vspace{-4mm}
  \caption{Proposed method: EAC. We first identify the key dimensions of the editing anchors using a weighted-gradient saliency map, followed by retraining on these dimensions to achieve the final optimization.}
  \vspace{-3mm}
  \label{EAC}
\end{figure}

In Section~\ref{analysis}, an in-depth analysis is provided on the factors that lead to the decrease in the general abilities of the model. 
Besides, it is found that the deviation of the edited parameter matrix could be constrained
by reducing the norm of the update matrix at each edit.
This helps maintain the semantic similarity of the facts recalled by the model before and after editing, ultimately preserving the general abilities of the edited model.
As depicted in Figure \ref{EAC}, a framework called EAC is proposed to compress information only in certain dimensions, thereby reducing the norm of the edited matrix and further constraining its deviation.

\subsection{Definition of Editing Anchors}
ROME uses an update matrix to insert a new knowledge triple \( t = (subject, relation, object) \). 
As mentioned in Section~\ref{sec_visual}, ROME calculates the update matrix by multiplying the pair \( (\mathbf{k}_*, \mathbf{v}_*) \), where \( \mathbf{k}_* \) identifies patterns of the input at the specified layer\footnotemark{} and \( \mathbf{v}_* \) is the fact recalled by the model.
Readers can refer to Appendix~\ref{b} for the details of ROME.
When injecting a new knowledge triple \( t = (subject, relation, object^*) \) to replace an old one \( t = (subject, relation, object) \), the specific part we aim to modify is the new relation \((relation, object^*) \), which is a property of the subject. It is believed that \( \mathbf{v}_* \) gathers all the information about how the model understands the subject and how it will respond thus we think that the new relation is primarily encoded in \( \mathbf{v}_* \).
Thus, EAC chooses to reduce the norm of \( \mathbf{v}_* \) for compression. Using a weighted gradient saliency map, EAC identifies high-scoring editing anchors crucial for encoding new relations. A scored elastic net is then applied to retrain and compress the editing information in key dimensions.
\footnotetext{Found by causal tracing methods~\cite{DBLP:conf/nips/MengBAB22}.}

\begin{figure*}[ht]
  \centering
  \includegraphics[width=0.5\textwidth]{figures/legend_edit.pdf}
  \vspace{-4mm}
\end{figure*}

\begin{figure*}[t]
  \centering
  \subfigure{
  \includegraphics[width=0.23\textwidth]{figures/ROME-GPT2XL-ZsRE-edit.pdf}}
  \subfigure{
  \includegraphics[width=0.23\textwidth]{figures/ROME-LLaMA3-8B-ZsRE-edit.pdf}}
  \subfigure{
  \includegraphics[width=0.23\textwidth]{figures/MEMIT-GPT2XL-ZsRE-edit.pdf}}
  \subfigure{
  \includegraphics[width=0.23\textwidth]{figures/MEMIT-LLaMA3-8B-ZsRE-edit.pdf}}
  \vspace{-2mm}
  \caption{Edited on the ZsRE dataset, the sequential editing performance of ROME and MEMIT with GPT2-XL and LLaMA-3 (8B) before and after the introduction of EAC, as the number of edits increases.}
  \vspace{-3mm}
  \label{edit-performance}
\end{figure*}

\subsection{Weighted-gradient Saliency Map}\label{a}
To reduce the norm of \( \mathbf{v}_* \) while preserving as much editing information as possible about the new relation, the goal is to compress this information over the smallest possible dimension. Drawing inspiration from gradient-based input saliency maps~\cite{DBLP:journals/corr/SmilkovTKVW17, DBLP:conf/nips/AdebayoGMGHK18}, a question is posed that whether a \textit{weight saliency map} can be constructed to aid compression. 
In previous work, ROME set \( \mathbf{v}_* = \arg\min_z \mathcal{L}(\mathbf{z}) \)~\cite{DBLP:conf/nips/MengBAB22}. Similar to the input saliency map, the gradient of this loss function with respect to each feature is utilized, and the magnitude of the values of \( \mathbf{v}_* \) is weighted accordingly to serve as the score for each feature: 
\begin{equation}
\text{score} = \mathbf{v}_* \odot \nabla \mathcal{L}(\mathbf{z}),
\label{score}
\end{equation}
where \(\odot\) is element-wise product. For GPT2-XL, the vectors $\mathbf{v}_*$ and $\mathbf{z}$ have dimensions of $1600 \times $1, whereas for LLaMA-3 (8B), the dimensions of these vectors are $4096 \times $1.
Based on Eq.~(\ref{score}), the desired weighted-gradient saliency map is obtained by applying a hard threshold:
\begin{equation}
\mathbf{m}_S = \mathbf{1} \left( \left| \text{score} \right| \geq \gamma \right),
\label{map}
\end{equation}
where \(\mathbf{1}(\mathbf{g} \geq \gamma)\) is an element-wise indicator function, which yields a value of 1 for the \(i\)-th element if \(g_i \geq \gamma\) and 0 otherwise.
\(| \cdot |\) is an element-wise absolute value operation, and \(\gamma > 0\) is a hard threshold. In practice, the value of \(\gamma\) is chosen according to different models and methods.
Specifically, the value of \(\gamma\) is chosen to retain the editing performance of the model in single editing. For more details refer to Appendix \ref{threshhold}.

With the introduction of the weighted gradient saliency map, the dimension of \( \mathbf{v}_* \) is split into two parts: one part represents the dimensions where the important editing anchors are located and it will be retrained to encode new relation, while the other part is set to 0, thereby reducing the norm of \( \mathbf{v}_* \). Based on Eq.~(\ref{map}), the \( \mathbf{v'}_* \), can be expressed as:
\begin{equation}
\mathbf{v'}_* \leftarrow \mathbf{m}_S \odot (\Delta \mathbf{v}_* + \mathbf{v}_*) + (\mathbf{1} - \mathbf{m}_S) \odot \mathbf{0},
\label{v}
\end{equation}
where \(\mathbf{1}\) denotes an all-one vector and \(\mathbf{0}\) denotes an all-zero vector. \(\Delta  \mathbf{v}_* \) is the part of \( \mathbf{v}_* \) that requires updating during retraining. Eq.~(\ref{v}) demonstrates that during retraining, only the dimensions where these important anchors are located need to be retrained to compress the editing information.

\subsection{Retraining Based on Scored Elastic Net}
After choosing important anchors, retraining for $\mathbf{v}_*$ is performed in this section.
To further compress the editing information, inspired by~\citet{zou2005regularization}, a score-based elastic net is also introduced during retraining:
\begin{equation}
\mathcal{L}_{0}(\mathbf{z}) = \lambda \|\mathbf{z}\|_{1, \alpha} + \mu \|\mathbf{z}\|_2^2,
\label{net}
\end{equation}
where \(\lambda \) and \(\mu \) are the hyper-parameters that control the strength of regularization and \( \mathbf{z} \) is the vector that causes the network to predict the target object in response to the factual prompt. Detailed hyper-parameters can be referred to in Appendix~\ref{hy}. Considering that the score computed in Eq.~(\ref{score}) represents the importance of the anchors for encoding the new relation, a weighted L1 norm is utilized when computing the L1 norm:
\begin{equation}
\|\mathbf{z}\|_{1, \alpha} = \sum_{i=1}^n \alpha_i |z_i|.
\end{equation}

In practice, we set \(\mathbf{\alpha} = \frac{1}{\text{score} + \epsilon}\), a small positive number \(\epsilon \) is introduced to prevent the score from being zero. 
Applying an elastic network, we ultimately derived the loss function during the retraining process to get the $\mathbf{v'}_*$ in Eq.~(\ref{v}):
\begin{equation}
\mathcal{L}_{r}(\mathbf{z}) =  \mathcal{L}(\mathbf{z}) + \mathcal{L}_{0}(\mathbf{z}).
\end{equation}

It is worth noting that when we make optimization here, only the dimensions where the editing anchors identified in section~\ref{a} are modified.
By introducing the elastic net, L1 regularization enables refining the selection of the editing anchors identified through the weighted-gradient saliency map during the retraining process. Meanwhile, L2 regularization effectively prevents model overfitting and improves the model's stability.
Finally, we complete the optimization. For specific optimization details, we recommend interested readers to refer to Appendix~\ref{b}.
Furthermore, the scored elastic net can also be applied to the FT. Readers can refer to Appendix \ref{FT} for more details.

\section{Experiments}

% In this section, we first introduce the experimental setup, including benchmarks, comparisons, and implementation details.
% Then we present the main results on downstream tasks in terms of model performance, attention sparsity ratios, and auxiliary overhead.
% Besides, we conduct ablation study to validate our proposed WRoPE and query-aware vector quantization methods.
% Finally, we demonstrate end-to-end inference speedup to highlight {\name}'s potential in improving the efficiency for long context LLM serving.

\subsection{Experimental Setup}

\noindent \textbf{Tasks.} 
We utilize RULER~\citep{ruler} as our benchmark for downstream tasks evaluation.
This synthetic benchmark 
% supports configurable context lengths and 
contains thirteen subtasks organized into four categories: information retrieval, multi-hop tracing, information aggregation, and question answering. 
It evaluates long-context comprehension and reasoning capabilities of LLMs, while effectively revealing the accuracy drop caused by KV cache reduction methods.

\noindent \textbf{Models.}
We conduct our main experiments on  
% state-of-the-art LLMs including 
Llama-3.1-8B-Instruct~\citep{llama-3} and MegaBeam-Mistral-7B-512k~\citep{megabeam-mistral}.
These models feature long-context processing capabilities with context windows of up to 128K and 512K tokens, respectively. 
As for the ablation study and end-to-end throughput evaluation, we apply the Llama-3.1-8B-Instruct model.

\noindent \textbf{Methods.}
For main experiments, we compare the proposed {\name} with the following four KV cache reduction methods, along with the full attention baseline: H2O~\citep{h2o}, SnapKV~\citep{snapkv}, Quest~\citep{quest}, MagicPIG~\citep{magicpig}.
For a fair comparison, the sparsity ratios of all KV cache reduction methods are controlled around 0.06.
% , which means approximately 6\% of KV cache is accessed at each inference (see Appendix~\ref{appendix:baselines} 
Detailed discussions are presented in Appendix~\ref{appendix:baselines}.

For ablation studies, we evaluate the following configurations on Llama-3.1-8B-Instruct:
\begin{itemize}[nosep]
    % \item \textbf{Baseline}: Substitutes both WRoPE and query-aware vector quantization with conventional RoPE and vector quantization.
    \item \textbf{Baseline}: It utilizes standard RoPE and conventional vector quantization.
    %\item \textbf{Baseline w/ WRoPE}: 
    \item \textbf{WRoPE}: 
    It utilizes WRoPE and conventional vector quantization.
    %The baseline configuration with standard RoPE replaced by WRoPE;
  %  \item \textbf{Baseline w/ Query-Aware Vector Quantization}: 
    \item \textbf{QAVQ}:
    It utilizes standard RoPE and query-aware vector quantization.
    %The baseline configuration with conventional vector quantization replaced by query-aware vector quantization;
    \item \textbf{{\name}}: The proposed method with WRoPE and query-aware vector quantization.
\end{itemize}

\noindent \textbf{Implementation Details.}
The hyper-parameters \(w\) and \(b\) of WRoPE are set to \(64\) and \(2048\), respectively.
The codebooks of query-aware vector quantization, with a size of 4096 each, are constructed from a set of sample inputs consisting of approximately 64K tokens of text randomly sampled from FineWeb~\citep{fineweb} and 16K tokens of randomly generated uuid strings.
% The prompts of RULER benchmark used in our experiments are identical to those used in MagicPIG~\citep{magicpig}.
The end-to-end speedup experiments are conducted on a server equipped with an NVIDIA H800 GPU with 80GB memory, and an Intel Xeon Platinum 8469C CPU.

\subsection{Main Results on Downstream Tasks}

\begin{table*}[htb]
    \centering
    \resizebox{0.8\textwidth}{!}{
        \begin{tabular}{l|cc|cccc|c}
            \toprule   
    		\multirow{2}*{\textbf{Models}} &
            \multirow{2}*{\textbf{Sparsity$\downarrow$}} & 
            \multirow{2}*{\textbf{Aux Mem$\downarrow$}}& 
            \multicolumn{5}{c}{\textbf{Accuracy$\uparrow$}} \\
                
            \cmidrule{4-8}
        
            ~ & ~ & ~ &
            \textbf{16K} & 
            \textbf{32K} & 
            \textbf{64K}  & 
            \textbf{96K}  & 
            \textbf{Average} \\
            
            \midrule

            % 94.37692	91.86692	85.92538	83.11000	88.81981
            \textit{Llama-3.1-8B-Instruct} &
            1.000 &
            0.000 &
            94.4 &
            91.9 &
            85.9 &
            83.1 &
            88.8 \\

            % 27.59231	30.62308	24.86385	25.02846	27.02692
            H2O &
            0.060 &
            0.008 &
            27.6 &
            30.6 &
            24.9 &
            25.0 &
            27.0 \\

            % 72.73615	75.06154	72.16923	70.68231	72.66231
            SnapKV &
            0.060 &
            0.008 &
            72.7 &
            75.1 &
            72.2 &
            70.7 &
            72.7 \\

            % 84.30000	84.00000	80.10000	74.40000	80.70000
            Quest &
            0.060 &
            0.031 &
            84.3 &
            84.0 &
            80.0 &
            74.4 &
            80.7 \\

            % 92.28692	87.58231	83.91308	79.08231	85.71615
            MagicPIG &
            0.068 &
            2.344 &
            \textbf{92.3} &
            87.6 &
            83.9 &
            79.1 &
            85.7 \\

            % 92.18462	90.43615	84.28462	79.57462	86.62000
            {\name} &
            0.060 &
            0.008 &
            92.2 &
            \textbf{90.4} &
            \textbf{84.3} &
            \textbf{79.6} &
            \textbf{86.6} \\

            \midrule

            % 91.81538	88.17462	83.34385	83.42077	86.68865
            \textit{MegaBeam-Mistral-7B-512K} &
            1.000 &
            0.000 &
            91.8 &
            88.2 &
            83.3 &
            83.4 &
            86.7 \\

            % 22.49000	23.43308	20.74385	22.63615	22.32577
            H2O & 
            0.060 & 
            0.008 &
            22.5 &
            23.4 &
            20.7 &
            22.6 &
            22.3 \\

            % 68.34846	68.46385	68.45923	65.19231	67.61596
            SnapKV &
            0.060 &
            0.008 &
            69.3 &
            68.5 &
            69.5 &
            65.2 &
            67.6 \\

            % 81.50000	80.80000	76.70000	74.40000	78.35000
            Quest &
            0.060 &
            0.031 &
            81.5 &
            80.8 &
            76.7 &
            74.4 &
            78.4 \\

            % 88.67923	85.24077	82.56923	81.76923	84.56462
            MagicPIG &
            0.064 & 
            2.344 &
            88.7 &
            85.2 &
            82.6 &
            81.8 &
            84.6 \\

            % 91.62846	88.14846	83.36154	82.15385	86.32308
            {\name} &
            0.062 &
            0.008 &
            \textbf{91.6} &
            \textbf{88.1} &
            \textbf{83.4} &
            \textbf{82.2} &
            \textbf{86.3} \\
            
            \bottomrule
        \end{tabular}
    }
    \caption{Comparison of sparsity ratio, auxiliary memory usage and accuracy on RULER benchmark. `Aux Mem' refers to `Auxialiary Memory Usage', which denotes the extra memory usage caused by KV cache reduction methods compared to the original key cache. `16K', `32K', `64K' and `96K' denote the input context length.}
    \label{tab:main_experiments}
\end{table*}

Table~\ref{tab:main_experiments} compares the accuracies of different methods on RULER, along with attention sparsity ratios and auxiliary memory overhead.
Experimental results draw the following conclusions:

\noindent \textbf{{\name} minimizes accuracy degradation under comparable sparsity ratios.}
For Llama models,
% {\name} achieves an average accuracy of 86.6, outperforming H2O (27.0) by +59.6, SnapKV (72.7) by +13.9, Quest (80.7) by +5.9, and MagicPIG (85.7) by +0.9.
{\name} achieves an average accuracy of 86.6, outperforming H2O (27.0), SnapKV (72.7), Quest (80.7), and MagicPIG (85.7).
For Mistral models,
% {\name} achieves an average accuracy of 86.3, surpassing H2O (22.3) by +64.0, SnapKV (67.6) by +18.7, Quest (78.4) by +7.9, and MagicPIG (84.6) by +1.7.
{\name} achieves an average accuracy of 86.3, surpassing H2O (22.3), SnapKV (67.6), Quest (78.4), and MagicPIG (84.6).
Notably, {\name} achieves these accuracies with comparable sparsity ratios (0.060 for Llama, 0.062 for Mistral) to H2O, SnapKV and Quest, lower than those of MagicPIG (0.068 for Llama, 0.064 for Mistral). 
% It is important to note that the sparsity ratio shown in Table~\ref{tab:main_experiments} differs in definition from the cost2 in MagicPIG~\citep{magicpig}. 
% Specifically, cost2 measures the ratio of computation overhead (FLOPs) compared to full attention, whereas our sparsity ratio measures the ratio of memory access overhead (MOPs) relative to full attention. 
% Since attention modules are typically considered memory-bound~\citep{transformer-survey}, we argue that the latter metric provides more meaningful insights on potential overhead reduction.
These results demonstrate the superior accuracy preservation of {\name}.

\noindent \textbf{{\name} causes comparable or lower auxiliary memory overhead compared to existing methods.}
{\name} causes auxiliary memory usage (0.008) identical to eviction-based methods, i.e., H2O and SnapKV, while being significantly more memory efficient than retrieval-based approaches, i.e., Quest (0.031) and MagicPIG (2.344).
This efficiency comes from the inherent nature of vector quantization, requiring only one codeword index for each token in each attention head, eliminating the need for storing either per-page metadata (Quest) or large LSH tables (MagicPIG).

\subsection{Ablation Study}

\begin{table}[htb]
    \centering
    \resizebox{1.0\columnwidth}{!}{
        \begin{tabular}{l|cccc|c}
        
            \toprule
            
            \textbf{Config} & 
            \textbf{16K$\uparrow$} & 
            \textbf{32K$\uparrow$} & 
            \textbf{64K$\uparrow$}  & 
            \textbf{96K$\uparrow$}  & 
            \textbf{Average$\uparrow$} \\
            
            \midrule

            % 86.42308	86.30769	81.49231	71.30538	81.38212
            Baseline &
            86.4 &
            86.3 &
            81.5 &
            71.3 &
            81.4 \\

            % 92.27923	90.04615	82.82077	78.39	85.88404
            WRoPE &
            \textbf{92.3} &
            90.0 &
            82.8 &
            78.4 &
            85.9 \\

            % 91.72308	86.88462	76.27462	69.44846	81.08270
            QAVQ &
            91.7 &
            86.9 &
            76.3 &
            69.4 &
            81.1 \\
            
            % 92.18462	90.43615	84.28462	79.57462	86.62000
            {\name} &
            92.2 &
            \textbf{90.4} &
            \textbf{84.3} &
            \textbf{79.6} &
            \textbf{86.6} \\
            
            \bottomrule
        \end{tabular}
    }
    \caption{Ablation study on the importance of WRoPE and query-aware vector quantization for accuracy. 
    %`WRoPE' and `QAVQ' refer to `Baseline w/ WRoPE' and `Baseline w/ Query-Aware Vector Quantization', respectively.
    }
    \label{tab:ablation}
\end{table}

Table~\ref{tab:ablation} validates the effectiveness of WRoPE and query-aware vector quantization
% \note{, mainly for accuracy}.
on improving model accuracy.
Experimental results draw the following conclusions: 
(1) WRoPE is fundamental to attention score approximation using shared codebooks.
(2) Query-aware vector quantization provides a further improvement in model accuracy by aligning the objectives of vector quantization and attention score approximation.
Detailed discussions are presented in Appendix~\ref{appendix:ablation}.

% \noindent \textbf{WRoPE is fundamental to attention score approximation using shared codebooks.}
% WRoPE achieves an average improvement of +4.5 over the baseline, with consistent gains across all context lengths.
% This result confirms WRoPE's critical role in preventing representation divergence of key states caused by positional embedding.

% \noindent \textbf{Query-aware vector quantization provides a further improvement in model accuracy by aligning the objectives of vector quantization and attention score approximation}.
% Our full method, incorporating query-aware vector quantization, demonstrates further improvements, particularly at longer context lengths (+1.5 at 64K, +1.2 at 96K, respectively).
% However, query-aware vector quantization alone underperforms the baseline,
% % (-0.3 on average)
% exhibiting a more significant drop at longer context lengths.
% This performance degradation suggests that representation divergence is not solely limited to key states but also affects query states, hindering the effectiveness of query-aware vector quantization.
% With WRoPE mitigating positional dependencies, query-aware vector quantization further optimizes the attention score approximation, achieving state-of-the-art performance.

\subsection{End-to-End Inference Speedup}

\begin{figure}[t]
    \centering
    \begin{tabular}{cc}
        \begin{subfigure}[b]{0.475\columnwidth}
            \centering
            \includegraphics[width=1.0\columnwidth]{images/throughput_16k.pdf}
            \caption{Inference throughput with a context length of 16K.}
        \end{subfigure}
        & 
        \begin{subfigure}[b]{0.475\columnwidth}
            \centering
            \includegraphics[width=1.0\columnwidth]{images/throughput_64k.pdf}
            \caption{Inference throughput with a context length of 64K.}
        \end{subfigure}
    \end{tabular}

    \caption{End-to-end inference throughput of Llama-3.1-8B-Instruct across varying context lengths and batch sizes.}
    \label{fig:throughput}
\end{figure}

Figure~\ref{fig:throughput} compares the inference throughput of the proposed {\name} against full attention on Llama-3.1-8B-Instruct.
At a context length of 16K, {\name} initially exhibits marginally lower throughput than full attention for small batch sizes \((\leq 5)\), primarily due to CPU-GPU data transfer overhead.
As batch sizes increase, the throughput of {\name} grows linearly to exceed 100 tokens/s, while full attention plateaus below 80 tokens/s due to memory bandwidth bottleneck and ends up out of memory at a batch size of 22.
{\name} achieves a peak throughput of over 160 tokens/s, a 2.1× speedup over full attention, with a maximum batch size of over 64.
This trend becomes more pronounced at 64K context lengths, where full attention struggles with batches over 5, while {\name} serving a batch size of up to 16 with a throughput of up to 45 tokens/s, delivering a \(2.7 \times\) performance advantage.
These results highlight {\name}'s potential in mitigating the memory bottleneck in long context LLM serving.
\section{Conclusion}\label{sec:conclusion}

In this paper, we proposed a prototype ASL generation system aimed at improving the naturalness, comprehensiveness, and overall quality of generated signs, addressing key limitations in existing approaches. Our technical evaluations indicate that our proposed approaches improve these aspects, enhancing the quality of generated ASL content. Feedback from DHH participants was mixed; while there was general interest in the system, concerns regarding visual quality and naturalness were noted. Reflecting on our design process and study findings, we discuss key insights and identify key areas for future improvement. While further work is needed, our study takes an initial step toward developing sign language generation systems that better meet the needs of the DHH and signing communities, offering real-world value.
\addition{
\section{Discussion}
\label{sec:discussion}
\DAF provides a systematic method to classify and analyze \MR deception attacks. %, addressing their effects on information channels and cognitive processes. 
While we focus on \MR headsets, \DAF is applicable to other forms of \MR and even other areas of human-computer interaction (HCI).
Kopp et al.'s information-theoretic framework ~\cite{kopp:2018} applied the Borden-Kopp model of deception to news media.
We have broadened its use to \MR deception attacks.
Future work should extend the scope to other areas of HCI that involve information processing and decision-making.
Our information-theoretic model and decision-making model are not tied to specific technologies or attacks, but rather provide generalizable models for studying the effects of deception in computing.
To enhance \DAF, future work should validate it empirically, expand its applicability to diverse contexts, incorporate individual cognitive factors, and refine models for processing attacks.

%By focusing on how \MR deception attacks impact information channels and decision-making processes, 
Researchers and practitioners can use \DAF to assess the security threat of \MR deception attacks.
For example, we can assign values of 1 to 3 for Low to High ratings, respectively.
Then, we can sum the values to identify which attacks pose the highest threat to perception and attention.
Further, \DAF can help develop deception detection and prevention approaches. % by using information theory to model \MR communication channels.
For example, we can compare differences between rendered frames to see how the signal is changing.
%For example, we can diff displayed frames with previous ones or an expected frame to identify changes in visual information presented to a \MR user.
%These diffs can reveal changes in channel capacity as information is either hidden or injected possibly along with noise.
High volatility in changes may indicate overt degradation attacks, particularly if we can identify noise based on differences between expected and actual frames.
More subtle changes that are spatial located in unexpected areas may indicate covert degradation attacks.
Using eye-tracking sensors on these headsets, we can derive models of attention that can help identify when different types of attention are being employed or disrupted.

% For example, we could use display-capture or eye-tracking data to detect a momentary misdirection attack.
% A misdirection attack 
%It offers foundational understanding for both technical and psychological dimensions of deception, with significant implications for future \MR research. 
%Our framework is adaptable for diverse \MR platforms and can guide empirical research into attack impacts and countermeasures. 

}

\addition{
%\subsection{Limitations}
%\label{sec:limitations}
\textbf{Limitations:} This SoK synthesizes existing knowledge towards developing a field of study around \MR deception. % by establishing a generalizable framework.
%It is essential to acknowledge the limitations of this research, which highlight potential areas for further exploration.
%One significant limitation is the lack of empirical validation. 
It is theoretical in nature and would benefit from further empirical validation.
%While it is rooted in empirical evidence from prior work, it lacks empirical validation.
%DAF’s models and  require real-world testing to confirm their accuracy and practical effectiveness. 
Controlled experiments involving \MR deception attacks are essential for refining the framework and assessing its relevance to diverse scenarios. 
Furthermore, \DAF does not fully account for cognitive diversity among users. 
Individual differences in cognitive capacity, attention, and susceptibility to deception are critical factors that could influence the effectiveness of both attacks and countermeasures. 
%Incorporating these factors would improve the framework’s precision and personalization.
% As \MR technology advances, the sophistication of attacks will also increase, highlighting the importance of further study in this area.
}
\input{10-Acknowledgements}

\bibliography{custom}


\clearpage
\newpage
\appendix
\onecolumn
\newpage
\appendix
\onecolumn
\section{Full Results on Longbench}
\label{appendix}
% \renewcommand{\arraystretch}{1.2} % 设置行高
\begin{table*}[ht]
\setlength{\tabcolsep}{2.5pt} % 设置列间距
\caption{\textbf{Result on Longbench.} The highest score in each task is marked in bold (except for "Full"). We also note the relative error of Twilight when integrated with the corresponding base algorithm. Green indicates an increase in score, while red indicates a decrease.}
\label{table:longbench}
    \centering
    \scalebox{0.69}{
    \begin{tabular}{lcccccccccccccc}
        \toprule
        \multirow{2}*{\textbf{Methods}} &
        \multirow{2}*{\textbf{Budget }} &
        \multicolumn{2}{c}{\textbf{Single-Doc. QA}} & \multicolumn{3}{c}{\textbf{Multi-Doc. QA}} & \multicolumn{3}{c}{\textbf{Summarization}} & \multicolumn{1}{c}{\textbf{Few-shot}} & \multicolumn{2}{c}{\textbf{Code}} & \multicolumn{1}{c}{\textbf{Synthetic}} & \multirow{2}*{\textbf{Avg. Score}}  \\
        \cmidrule(lr){3-4}\cmidrule(lr){5-7}\cmidrule(lr){8-10} \cmidrule(lr){11-11} \cmidrule(lr){12-13} \cmidrule(lr){14-14} 
        & & \textit{Qasper} & \textit{MF-en} & \textit{HotpotQA} & \textit{2WikiMQA} &  \textit{Musique} & \textit{GovReport} & \textit{QMSum} & \textit{MultiNews} & \textit{TriviaQA} &  \textit{LCC} & \textit{Repobench-P} & \textit{PR-en} \\
        \midrule
        \multicolumn{15}{c}{\textsc{Longchat-7B-32k}} \\
        \midrule
        \multirow{2}*{Full} & 32k & 29.48 & 42.11 & 30.97 & 23.74 & 13.11 & 31.03 & 22.77 & 26.09 & 83.25 & 30.50 & 52.70 & 55.62 & 36.78 \\
         & \textbf{Twilight (Avg. 146)} & 31.74 & \textbf{43.91} & 33.59 & \textbf{25.65} & \textbf{13.93} & 32.19 & \textbf{23.15} & 26.30 & 85.14 & 34.50 & 54.98 & 57.12 & 38.52\textcolor{teal}{(+4.7\%)}\\
        \midrule
        \multirow{5}*{Quest}
         & 256 & 26.00 & 32.83 & 23.23 & 22.14 & 7.45 & 22.64 & 20.98 & 25.05 & 67.40 & 33.60 & 48.70 & 45.07 & 31.26 \\
      & 1024 & 31.63 & 42.36 & 30.47 & 24.42 & 10.11 & 29.94 & 22.70 & 26.39 & 84.21 & 34.5 & 51.52 & 53.95 & 36.85 \\
       & 4096 & 29.77 & 42.71 & 32.94 & 23.94 & 13.24 & 31.54 & 22.86 & 26.45 & 84.37 & 31.50 & 53.17 & 55.52 & 37.33 \\
        & 8192 & 29.34 & 41.70 & 33.27 & 23.46 & 13.51 & 31.18 & 23.02 & 26.48 & 84.70 & 30.00 & 53.02 & 55.57 & 37.10 \\
             & \textbf{Twilight (Avg. 131)} & 31.95 & 43.28 & 31.62 & 24.87 & 13.48 & \textbf{32.21} & 22.79 & 26.33 & 84.93 & 33.50 & 54.86 & 56.70 & 38.04\textcolor{teal}{(+2.5\%)} \\
        \midrule
    \multirow{5}*{DS}
         & 256 & 28.28 & 39.78 & 27.10 & 20.75 & 9.34 & 29.68 & 21.79 & 25.69 & 83.97 & 32.00 & 52.01 & 53.44 & 35.32 \\
      & 1024 & 30.55 & 41.27 & 30.85 & 21.87 & 7.27 & 26.82 & 22.95 & 26.51 & 83.22 & 31.50 & 53.23 & 55.50 & 35.96 \\
       & 4096 & 28.95 & 41.90 & 32.52 & 23.65 & 8.07 & 29.68 & 22.75 & \textbf{26.55} & 83.34 & 30.00 & 52.77 & 55.48 & 36.31 \\
        & 8192 & 29.05 & 41.42 & 31.79 & 22.95 & 12.50 & 30.44 & 22.50 & 26.43 & 83.63 & 30.50 & 52.87 & 55.33 & 36.62 \\
             & \textbf{Twilight (Avg. 126)} & \textbf{32.34} & 43.89 & \textbf{34.67} & 25.43 & 13.84 & 31.88 & 23.01 & 26.32 & \textbf{85.29} & \textbf{35.50} & \textbf{55.03} & \textbf{57.27} & \textbf{38.71}\textcolor{teal}{(+5.7\%)} \\
        \midrule
        \multicolumn{15}{c}{\textsc{Llama-3.1-8B-Instruct}} \\
        \midrule
        \multirow{2}*{Full} & 128k & 46.17 & 53.33 & 55.36 & 43.95 & 27.08 & 35.01 & 25.24 & 27.37 & 91.18 & 99.50 & 62.17 & 57.76 & 52.01 \\
         & \textbf{Twilight (Avg. 478)} & 43.08 & 52.99 & 52.22 & 44.83 & 25.79 & 34.21 & \textbf{25.47} & 26.98 & 91.85 & \textbf{100.00} & \textbf{64.06} & 58.22 & 51.64\textcolor{red}{(-0.7\%)} \\
        \midrule
        \multirow{5}*{Quest}
         & 256 & 24.65 & 37.50 & 30.12 & 23.60 & 12.93 & 27.53 & 20.11 & 26.59 & 65.34 & 95.00 & 49.70 & 45.27 & 38.20 \\
      & 1024 & 38.47 & 49.32 & 47.43 & 38.48 & 20.59 & 33.71 & 23.67 & 26.60 & 81.94 & 99.50 & 60.78 & 52.96 & 47.79 \\
       & 4096 & 43.97 & 53.64 & 51.94 & 42.54 & 24.00 & 34.34 & 24.36 & 26.75 & 90.96 & 99.50 & 62.03 & 55.49 & 50.79 \\
        & 8192 &\textbf{44.34} & 53.25 & 54.72 & 44.84 & \textbf{25.98} & 34.62 & 24.98 & 26.70 & 91.61 & \textbf{100.00} & 62.02 & 54.20 & 51.44 \\
         & \textbf{Twilight (Avg. 427)} & 43.44 & 53.2 & 53.77 & 43.56 & 25.42 & 34.39 & 25.23 & 26.99 & 91.25 & 100.0 & 63.55 & 58.06 & 51.57\textcolor{teal}{(+0.3\%)} \\
        \midrule
    \multirow{5}*{DS}
         & 256 & 38.24 & 49.58 & 43.38 & 31.98 & 15.52 & 33.40 & 24.06 & 26.86 & 84.41 & 99.50 & 53.28 & 48.64 & 45.74 \\
      & 1024 & 42.97 & \textbf{54.65} & 51.75 & 33.92 & 20.39 & 34.50 & 24.92 & 26.71 & \textbf{92.81} & 99.50 & 62.66 & 48.37 & 49.43 \\
       & 4096 & 43.50 & 53.17 & 54.21 & 44.70 & 23.14 & \textbf{34.73} & 25.40 & 26.71 & 92.78 & 99.50 & 62.59 & 51.31 & 50.98 \\
        & 8192 & 43.82 & 53.71 & 54.19 & \textbf{45.13} & 23.72 & 34.27 & 24.98 & 26.69 & 91.61 & \textbf{100.00} & 62.40 & 52.87 & 51.14 \\
             & \textbf{Twilight (Avg. 446)} & 43.08 & 52.89 & \textbf{54.68} & 44.86 & 24.88 & 34.09 & 25.20 & \textbf{27.00} & 91.20 & \textbf{100.00} & 63.95 & \textbf{58.93} & \textbf{51.73}\textcolor{teal}{(+1.2\%)} \\
\bottomrule
\end{tabular}
}
\end{table*}


\end{document}
