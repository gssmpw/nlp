\pdfoutput=1
\documentclass[11pt]{article}
\usepackage{amsmath}
\usepackage[final]{acl}
\usepackage{times}
\usepackage{latexsym}
\usepackage[T1]{fontenc}
\usepackage[utf8]{inputenc}
\usepackage{microtype}
\usepackage{inconsolata}
\usepackage{graphicx}
\usepackage{booktabs}
\usepackage{multirow}
\usepackage{subfigure}
\usepackage{amsmath}
\usepackage{amssymb}
\usepackage{hyperref}
\usepackage{float}
\usepackage{multirow}
\usepackage{booktabs}
\usepackage{amsmath}
\hyphenpenalty=5000
\tolerance=1000

\newcommand{\MODELNAME}{\textsc{VerifyScore}}
\newcommand{\jc}[1]{\textcolor{red}{[JC: #1]}}
\newcommand{\hy}[1]{\textcolor{blue}{[HY: #1]}}

\newcommand{\jy}[1]{\textcolor{purple}{[JY: #1]}}

\title{Constraining Sequential Model Editing with Editing Anchor Compression}


\author{
Hao-Xiang Xu$^1$\thanks{Equal contribution.}, Jun-Yu Ma$^1$\footnotemark[1], Zhen-Hua Ling$^1$\thanks{Corresponding author.}, Ningyu Zhang$^2$, Jia-Chen Gu$^3$  \\
  $^1$National Engineering Research Center of Speech and Language Information Processing, \\
      University of Science and Technology of China, Hefei, China \\
  $^2$Zhejiang University \\
  $^3$University of California, Los Angeles \\
{\tt \{nh2001620,mjy1999\}@mail.ustc.edu.cn}, {\tt zhling@ustc.edu.cn}, \\ 
{\tt zhangningyu@zju.edu.cn}, {\tt gujc@ucla.edu} 
}


\begin{document}
\maketitle
\begin{abstract}
Large language models (LLMs) struggle with hallucinations due to false or outdated knowledge. Given the high resource demands of retraining these models, there is an increasing focus on developing \emph{model editing}. 
However, the general abilities of LLMs across downstream tasks are prone to significant degradation during sequential editing.
This paper statistically observes that the parameter matrix after editing exhibits a significant deviation compared to its previous state as the number of edits increases.
This serious deviation affects the original knowledge associations within LLMs and leads to the degradation of their general abilities.
To this end, a framework termed \textbf{E}diting \textbf{A}nchor \textbf{C}ompression (EAC) is proposed to constrain the deviation of the parameter matrix during sequential editing. 
It compresses the editing information by selecting editing anchors that are important in encoding new relations without deviating too much from the original matrix, thereby preserving the general abilities. 
Experiments of applying EAC to two popular editing methods on three LLMs across four tasks are conducted. 
Evaluation results show that EAC effectively minimizes unreasonable deviations caused by model editing, preserving over 70\% of the general abilities while better retaining the editing knowledge compared to the original counterpart methods\footnote{Code is available: \href{https://github.com/famoustourist/EAC}{https://github.com/famoustourist/EAC}}.
\end{abstract}

\section{Introduction}

Despite the remarkable capabilities of large language models (LLMs)~\cite{DBLP:conf/emnlp/QinZ0CYY23,DBLP:journals/corr/abs-2307-09288}, they often inevitably exhibit hallucinations due to incorrect or outdated knowledge embedded in their parameters~\cite{DBLP:journals/corr/abs-2309-01219, DBLP:journals/corr/abs-2302-12813, DBLP:journals/csur/JiLFYSXIBMF23}.
Given the significant time and expense required to retrain LLMs, there has been growing interest in \emph{model editing} (a.k.a., \emph{knowledge editing})~\cite{DBLP:conf/iclr/SinitsinPPPB20, DBLP:journals/corr/abs-2012-00363, DBLP:conf/acl/DaiDHSCW22, DBLP:conf/icml/MitchellLBMF22, DBLP:conf/nips/MengBAB22, DBLP:conf/iclr/MengSABB23, DBLP:conf/emnlp/YaoWT0LDC023, DBLP:conf/emnlp/ZhongWMPC23, DBLP:conf/icml/MaL0G24, DBLP:journals/corr/abs-2401-04700}, 
which aims to update the knowledge of LLMs cost-effectively.
Some existing methods of model editing achieve this by modifying model parameters, which can be generally divided into two categories~\cite{DBLP:journals/corr/abs-2308-07269, DBLP:conf/emnlp/YaoWT0LDC023}.
Specifically, one type is based on \emph{Meta-Learning}~\cite{DBLP:conf/emnlp/CaoAT21, DBLP:conf/acl/DaiDHSCW22}, while the other is based on \emph{Locate-then-Edit}~\cite{DBLP:conf/acl/DaiDHSCW22, DBLP:conf/nips/MengBAB22, DBLP:conf/iclr/MengSABB23}. This paper primarily focuses on the latter.

\begin{figure}[t]
  \centering
  \includegraphics[width=0.48\textwidth]{figures/demonstration.pdf}
  \vspace{-4mm}
  \caption{(a) Comparison of regular model editing and EAC. EAC compresses the editing information into the dimensions where the editing anchors are located. Here, we utilize the gradients generated during training and the magnitude of the updated knowledge vector to identify anchors. (b) Comparison of general downstream task performance before editing, after regular editing, and after constrained editing by EAC.}
  \vspace{-3mm}
  \label{demo}
\end{figure}

\emph{Sequential} model editing~\cite{DBLP:conf/emnlp/YaoWT0LDC023} can expedite the continual learning of LLMs where a series of consecutive edits are conducted.
This is very important in real-world scenarios because new knowledge continually appears, requiring the model to retain previous knowledge while conducting new edits. 
Some studies have experimentally revealed that in sequential editing, existing methods lead to a decrease in the general abilities of the model across downstream tasks~\cite{DBLP:journals/corr/abs-2401-04700, DBLP:conf/acl/GuptaRA24, DBLP:conf/acl/Yang0MLYC24, DBLP:conf/acl/HuC00024}. 
Besides, \citet{ma2024perturbation} have performed a theoretical analysis to elucidate the bottleneck of the general abilities during sequential editing.
However, previous work has not introduced an effective method that maintains editing performance while preserving general abilities in sequential editing.
This impacts model scalability and presents major challenges for continuous learning in LLMs.

In this paper, a statistical analysis is first conducted to help understand how the model is affected during sequential editing using two popular editing methods, including ROME~\cite{DBLP:conf/nips/MengBAB22} and MEMIT~\cite{DBLP:conf/iclr/MengSABB23}.
Matrix norms, particularly the L1 norm, have been shown to be effective indicators of matrix properties such as sparsity, stability, and conditioning, as evidenced by several theoretical works~\cite{kahan2013tutorial}. In our analysis of matrix norms, we observe significant deviations in the parameter matrix after sequential editing.
Besides, the semantic differences between the facts before and after editing are also visualized, and we find that the differences become larger as the deviation of the parameter matrix after editing increases.
Therefore, we assume that each edit during sequential editing not only updates the editing fact as expected but also unintentionally introduces non-trivial noise that can cause the edited model to deviate from its original semantics space.
Furthermore, the accumulation of non-trivial noise can amplify the negative impact on the general abilities of LLMs.

Inspired by these findings, a framework termed \textbf{E}diting \textbf{A}nchor \textbf{C}ompression (EAC) is proposed to constrain the deviation of the parameter matrix during sequential editing by reducing the norm of the update matrix at each step. 
As shown in Figure~\ref{demo}, EAC first selects a subset of dimension with a high product of gradient and magnitude values, namely editing anchors, that are considered crucial for encoding the new relation through a weighted gradient saliency map.
Retraining is then performed on the dimensions where these important editing anchors are located, effectively compressing the editing information.
By compressing information only in certain dimensions and leaving other dimensions unmodified, the deviation of the parameter matrix after editing is constrained. 
To further regulate changes in the L1 norm of the edited matrix to constrain the deviation, we incorporate a scored elastic net ~\cite{zou2005regularization} into the retraining process, optimizing the previously selected editing anchors.

To validate the effectiveness of the proposed EAC, experiments of applying EAC to \textbf{two popular editing methods} including ROME and MEMIT are conducted.
In addition, \textbf{three LLMs of varying sizes} including GPT2-XL~\cite{radford2019language}, LLaMA-3 (8B)~\cite{llama3} and LLaMA-2 (13B)~\cite{DBLP:journals/corr/abs-2307-09288} and \textbf{four representative tasks} including 
natural language inference~\cite{DBLP:conf/mlcw/DaganGM05}, 
summarization~\cite{gliwa-etal-2019-samsum},
open-domain question-answering~\cite{DBLP:journals/tacl/KwiatkowskiPRCP19},  
and sentiment analysis~\cite{DBLP:conf/emnlp/SocherPWCMNP13} are selected to extensively demonstrate the impact of model editing on the general abilities of LLMs. 
Experimental results demonstrate that in sequential editing, EAC can effectively preserve over 70\% of the general abilities of the model across downstream tasks and better retain the edited knowledge.

In summary, our contributions to this paper are three-fold:
(1) This paper statistically elucidates how deviations in the parameter matrix after editing are responsible for the decreased general abilities of the model across downstream tasks after sequential editing.
(2) A framework termed EAC is proposed, which ultimately aims to constrain the deviation of the parameter matrix after editing by compressing the editing information into editing anchors. 
(3) It is discovered that on models like GPT2-XL and LLaMA-3 (8B), EAC significantly preserves over 70\% of the general abilities across downstream tasks and retains the edited knowledge better.
\section{Related Works}

\textbf{Enhancing LLMs' Theory of Mind.} There has been systematic evaluation that revealed LLMs' limitations in achieving robust Theory of Mind inference \citep{ullman2023large, shapira2023clever}. To enhance LLMs' Theory of Mind capacity, recent works have proposed various prompting techniques. For instance, SimToM \citep{wilf2023think} encourages LLMs to adopt perspective-taking, PercepToM \citep{jung2024perceptions} improves perception-to-belief inference by extracting relevant contextual details, and \citet{huang2024notion} utilize an LLM as a world model to track environmental changes and refine prompts. Explicit symbolic modules also seem to improve LLM's accuracy through dynamic updates based on inputs. Specifically, TimeToM \citep{hou2024timetom} constructs a temporal reasoning framework to support inference, while SymbolicToM \citep{sclar2023minding} uses graphical representations to track characters' beliefs. Additionally, \citet{wagner2024mind} investigates ToM's necessity and the level of recursion required for specific tasks. However, these approaches continue to exhibit systematic errors in long contexts, complex behaviors, and recursive reasoning due to inherent limitations in inference and modeling \citep{jin2024mmtom,shi2024muma}. Most of them rely on domain-specific designs, lacking open-endedness.


\textbf{Model-based Theory of Mind inference.} Model-based Theory of Mind inference, in particular, Bayesian inverse planning (BIP) \citep{baker2009action,ullman2009help,baker2017rational,zhi2020online}, explicitly constructs representations of agents' mental states and how mental states guide agents' behavior via Bayesian Theory of Mind (BToM) models. These methods can reverse engineer human ToM inference in simple domains \citep[e.g.,][]{baker2017rational,netanyahu2021phase,shu2021agent}. Recent works have proposed to combine BIP with LLMs to achieve robust ToM inference in more realistic settings \citep{ying2023neuro, jin2024mmtom, shi2024muma}. However, these methods require manual specification of the BToM models as well as rigid, domain-specific implementations of Bayesian inference, limiting their adaptability to open-ended scenarios. To overcome this limitation, we propose \ours, a method capable of automatically modeling mental variables across diverse conditions and conducting automated BIP without domain-specific knowledge or implementations.


\begin{figure*}[ht]
  \centering
  \includegraphics[width=\linewidth]{figures/benchmarks_and_models.pdf}
    \vspace{-15pt}
  \caption{Examples questions (top panels) and the necessary Bayesian Theory of Mind (BToM) model for Bayesian inverse planning (bottom panels) in diverse Theory of Mind benchmarks. \ours aims to answer any Theory of Mind question in a variety of benchmarks, encompassing different mental variables, observable contexts, numbers of agents, the presence or absence of utterances, wording styles, and modalities. It proposes and iteratively adjusts an appropriate BToM and conducts automated Bayesian inverse planning based on the model.
  There can be more types of questions/models in each benchmark beyond the examples shown in this figure.}
  \label{fig:benchmarks_and_models}
  %\vspace{-0.75em}
  \vspace{-10pt}
\end{figure*}



\textbf{Automated Modeling with LLMs.} There has been an increasing interest in integrating LLMs with inductive reasoning and probabilistic inference for automated modeling. \citet{piriyakulkij2024doing} combine LLMs with Sequential Monte Carlo to perform probabilistic inference about underlying rules. Iterative hypothesis refinement techniques \citep{qiu2023phenomenal} further enhance LLM-based inductive reasoning by iteratively proposing, selecting, and refining textual hypotheses of rules. Beyond rule-based hypotheses, \citet{wang2023hypothesis} prompt LLMs to generate natural language hypotheses that are then implemented as verifiable programs, while \citet{li2024automated} propose a method in which LLMs construct, critique, and refine statistical models represented as probabilistic programs for data modeling. \citet{cross2024hypothetical} leverage LLMs to propose and evaluate agent strategies for multi-agent planning but do not specifically infer individual mental variables. Our method also aims to achieve automated modeling with LLMs. Unlike prior works, we propose a novel automated model discovery approach for Bayesian inverse planning, where the objective is to confidently infer any mental variable given any context via constructing a suitable Bayesian Theory of Mind model.
\section{Preliminary Study}
\label{sec:preliminary-study}


\noindent
\shepherd{ \textbf{Study 1: WPT Depths and Spectrogram Resolution.}
As discussed in Section~\ref{sec:sound-recognition}, the Wavelet Packet Transform (WPT) decomposes signals into finer sub-frequency bands at each level, with spectrogram resolution depending on WPT depth. Greater depth improves classification but increases computational cost. We conduct a preliminary study on WPT depth in environmental sound classification using ESC10~\cite{piczak2015esc} and US8K~\cite{salamon2017us8k}. On an MSP430 microcontroller~\cite{texas2021msp430}, we implemented a simple CNN classifier using WPT spectrograms at varying resolutions, measuring accuracy and energy consumption. Figure~\ref{fig:resolution-accuracy-energy} shows that higher resolution improves accuracy but greatly increases energy consumption, highlighting the need for cost-efficient approaches to balance performance and efficiency. This experiment also implies that to achieve good classification in on-cloud inference, high-resolution spectrogram will need to be transmitted. This results in even larger energy and communication overhead for edge devices, hence motivating keeping the inference pipeline local.}

\noindent
\shepherd{\textbf{Study 2: Effects of Frequency Bands.} WPT also allows us to selectively upsample frequency-domain resolutions on certain frequency bands. We argue that the discriminative information for different sound classes is distributed differently across different frequency bands.} To verify that, in the second preliminary experiment, we classify spectrograms of the same resolution but with either high-frequency bands only or low-frequency bands only. The results, shown in Figure~\ref{fig:high-low-frequency}, indicate that, for sounds of helicopters, waves, and drilling, high-frequency bands are more important for making the correct classification, whereas low-frequency bands are more important for some other classes.

\shepherd{These observations motivate the use of frequency-domain attention to guide the wavelet transform in generating multi-resolution spectrograms, achieving high accuracy while minimizing WPT and classification costs. This insight informs the design of our novel neural architecture, detailed in the following section.}



%\Eric{Exp2: Discrimnative information to distinguish different sound classes is distributed non-uniformly across spectral bands. Expected result: For some class high-freq bands are more important for classification, others low-freq bands. ExpPlan: Find appropriate classes mask out certain bands and do classification.}

%%%%%%%%%%%%%%%%%%%%%%%%%%%%%%%%%%%%%%%%%%%%%
\begin{figure}[tp]
    \centering
    \includegraphics[width=\linewidth]{figures/resolution-accuracy-energy.png}
    \vspace{-0.8cm}
    \caption{\shepherd{Accuracy (left) and energy consumption (right) at various spectrogram resolutions.}}
    \label{fig:resolution-accuracy-energy}
    \vspace{-0.3cm}
\end{figure}
%%%%%%%%%%%%%%%%%%%%%%%%%%%%%%%%%%%%%%%%%%%%%
\begin{figure}[tp]
    \centering
    \includegraphics[width=\linewidth]{figures/high-low-frequency.png}
    \vspace{-0.8cm}
    \caption{\shepherd{Accuracy of using high- and low-frequency band for ESC10 (left) and US8k (right).}}
    \vspace{-0.5cm}
    \label{fig:high-low-frequency}
\end{figure}
%%%%%%%%%%%%%%%%%%%%%%%%%%%%%%%%%%%%%%%%%%%%%
\section{Analysis}

Our analysis hand-annotated all LLM-generated code for the presence/absence of dark patterns and used those counts to calculate statistical measures of difference. The original response for each prompt pair was a single file using HTML and CSS to create a single component of an ecommerce website. Rather than evaluate the code directly, we developed an automated pipeline to compile the code and screenshot the design. While these visual representations can include minor issues (the most common being that LLMs were prompted to use placeholder image URLs which do not compile), they were generally much easier to assess for the presence of dark patterns than the original code.%is built it is . It is this visual representation of the design 

Three independent, trained designers labeled each output for the presence of dark patterns. In addition, drawing on a taxonomy developed in prior work~\cite{a:44}, we labeled six attributes defined in Table~\ref{tab:darkpattern-definitions} for each LLM-generated component design: asymmetric, covert, deceptive, information hiding, restrictive, and disparate treatment. After an initial 30 components were labeled, the designers met to review any points of disagreement or uncertainty. On the basis of this, the schema was slightly updated, and the designers were able to produce labels more consistently. Nevertheless, there continued to be some opportunities for disagreement. For example, in one instance, there was a debate about whether disparate treatment dark patterns could occur in the LLM-generated designs. One designer initially believed that disparate treatment was unlikely because the LLMs generate only one component at a time, lacking distinct groups of users for comparison. However, another designer pointed out examples like discounts offered only to canceling users or first-time customers, which inherently treat different user groups unequally. This example reflected that the interpretation of these attributes can sometimes depend on personal understanding and tolerance. However, we made every effort to maintain a consistent schema %by keeping 
through real-time communication about controversial attributes, %when a component attribute seemed controversial and d
discussing each collectively as a group. The final label for each component was assigned by majority vote. 

Our analysis included both the presence/absence of dark patterns as well as the mechanisms of those dark patterns. To compare the frequency of producing dark patterns across different models and across different stakeholder interests, we used Chi-squared tests for statistical significance. 
\section{Baseline Method: Wearable Multi-Camera Body Pose Estimation}\label{sec:method}

\label{sec:pose_estimation}
To demonstrate the benefits of \dataset{}, we trained a neural network to estimate 3D ego body poses using multiple body-worn cameras. The input to the network consists of the aligned video sequences \( \bm{X} \in [0,1]^{C \times F \times 3 \times H \times W} \), with \( F \) frames from \( C \) body-attached cameras. Based on these inputs, the network predicts a pose \( \bm{\hat{p}}_i \) for each input frame $i$.

\subsection{Network architecture}
Our network is a Vision Transformer Model based on Sparse Video Tube ViTs~\cite{piergiovanni2023rethinking}.
We extract feature vectors from each input video using a sparse view tokenizer SVT with a shared interpolated kernel.
The extracted feature vectors from the sparse tube tokenizers are then added to their fixed spatio-temporal position encoding \( \bm{\kappa}_{p} \) and their learnable view encoding \( \bm{\kappa}_{v,c} \) per camera $c$.

\begin{equation}
    \bm{v}_{c} = \mathrm{SVT}(\bm{X}_c, \bm{W}) + \bm{\kappa}_{p} + \bm{\kappa}_{v,c}, \quad \text{ where } \bm{W} \text{ are the shared weights of the kernel.}
\end{equation}
The resulting feature vectors for the different cameras $\bm{v}^c$ are concatenated with the pose token $\bm{\phi}_{j} = \bm{\tau}(j) + \bm{\psi}, j \in[0, F-1]$, where $\bm{\psi}$ is a trainable pose token and $\bm{\tau}$ is a sinusoidal positional encoding.
The resulting token sequence is then processed using a Vision Transformer Encoder.
\begin{equation}
    \{\bm{z}_{0}, \ldots, \bm{z}_{F-1}\} = \mathrm{ViT}(\text{concat}(\bm{\phi}_0, ..., \bm{\phi}_{F-1}, \bm{v}_{0}, ..., \bm{v}_{c-1}))
\end{equation}
Based on each embedded pose token $\bm{z}$, we obtain the 6D representation \cite{zhouContinuityRotationRepresentations2019jun} of the SMPL pose parameters $\bm{\theta}$, the 6D relative rotation $\bm{R}_{r}$, and 3D relative translation of the root $\bm{t}{r}$ with respect the previous frame.
\begin{equation}
\hat{\bm{\theta}} = W_{\theta}\bm{z}, \quad \hat{\bm{R}}_r = W_{R}\bm{z}, \quad  \hat{\bm{t}}_{r} = W_{t}\bm{z}
\end{equation}

To improve generalization, the network is trained to predict the pose difference, i.e., the relative root pose with respect to the previous pose, instead of directly predicting global root poses.

Using Forward Kinematics, we obtain the global body pose $\bm{p}$ with respect to the starting pose.
\begin{equation}
    \{\bm{\hat{p}}_{0}, \ldots, \bm{\hat{p}}_{F-1}\} = \mathrm{FK}_{\theta}(\bm{\theta}, \hat{\bm{R}}_{g}, \hat{\bm{t}}_{g}, \beta), \text{where}  \quad\hat{\bm{R}}_{g}, \hat{\bm{t}}_{g} = \mathrm{FK}_{g}(\hat{\bm{R}}_{r}, \hat{\bm{t}}_{r}) 
\end{equation}
Where $\beta$ are the shape parameters of the SMPL-X model \cite{SMPL-X:2019} for a given person.

We use 4 tubes with the following configurations: \(16 \times 16 \times 16\) with stride \((12, 48, 48)\) and offset \((0, 0, 0)\), \(24 \times 6 \times 6\) with stride \((12, 32, 32)\) and offset \((8, 12, 12)\), \(12 \times 24 \times 24\) with stride \((24, 48, 48)\) and offset \((0, 28, 28)\), and \(1 \times 32 \times 32\) with stride \((12, 64, 64)\) and offset \((0, 0, 0)\). The pose embedding parameter is initialized using the Kaiming uniform distribution~\cite{heDelvingDeepRectifiers2015feb}, and the pose token is initialized using the Normal distribution.

\subsection{Loss function}
We supervise the network with the following loss function:
\begin{equation}
    \mathcal{L} = \lambda_{\theta}\mathcal{L}_{\theta} +  \lambda_{p}\mathcal{L}_{p} +  \lambda_{v}\mathcal{L}_{v} +  \lambda_{t_{r}}\mathcal{L}_{t_{r}} +  \lambda_{R_{r}}\mathcal{L}_{R_{r}} +  \lambda_{t_{g}}\mathcal{L}_{t_{g}} + \lambda_{R_{g}}\mathcal{L}_{R_{g}}  + \lambda_{z}\mathcal{L}_{z}
\end{equation}

The angle loss $\mathcal{L}_{\theta}$ encourages the model to learn the SMPL angles $\bm{\theta}$, while the joint position loss $\mathcal{L}_{p}$ forces the predicted joint positions through forward kinematics to be close to the ground-truth joint positions.
This way, both the local and the accumulated errors are considered.
\begin{equation}
    \mathcal{L}_{\theta} = |\bm{\theta}_{\text{6D}}- \bm{\hat{\theta}}_{\text{6D}}|_{1}\quad \textrm{and} \quad \mathcal{L}_{p} = |\bm{p} - \hat{\bm{p}}|_{1},
\end{equation}
where \textit{6D} indicates the six-dimensional representation of the rotation matrices~\cite{zhouContinuityRotationRepresentations2019jun}. 
For the root pose, we penalize both the relative and absolute translation and orientation error accumulated through the kinematic chain, 
\begin{equation}
\begin{split}
    \mathcal{L}_{R_{r}} = |\bm{R}_{r, \text{6D}} - \bm{\hat{R}}_{r, \text{6D}}|_{1}\quad \textrm{and} \quad \mathcal{L}_{t_{r}} = |\bm{t}_{r} - \bm{\hat{t}}_{r}|_{1} \\
    \mathcal{L}_{R_{g}} = \lVert \hat{\bm{R}}_g\bm{R}^{-1}_{g} - I \rVert_{2} \quad \textrm{and} \quad \mathcal{L}_{t_{g}} = |\bm{t}_{g} - \hat{\bm{t}}_{g}|_{1}
\end{split}
\end{equation}

To encourage the model to estimate more expressive motions accurately, we add a velocity loss $\mathcal{L}_{v}$.
We also regularize the embedded pose token $\bm{z}$ using an $l_2$-regularization term $\mathcal{L}_{z}$.
\begin{equation}
    \mathcal{L}_v = | (\bm{p}_i - \bm{p}_{i-1}) - (\hat{\bm{p}}_i - \hat{\bm{p}}_{i-1})|_1 \quad \text{and} \quad \mathcal{L}_{z} = \lVert \bm{z}\rVert_{2}
\end{equation}

We set $\lambda_{\theta}=10$,  $\lambda_{p}=25$,  $\lambda_{v}=40$,  $\lambda_{t_r}=25$, $\lambda_{R_r}=15$, $\lambda_{t_g}=1$, $\lambda_{R_g}=0.025$, and $\lambda_{z}=0.0005$.
\section{Experiments}
\label{sec:sec-Experiment}

In this section, we present a comprehensive evaluation of our proposed algorithms against three different baseline algorithms on both synthetic and real-world datasets. 



\subsection{Baseline Algorithms}


\begin{figure*}[t]
	\centering
		\begin{tabular}{cccc}
	 			\multicolumn{4}{c}{\hspace{-8mm} \includegraphics[height=2.8mm]{figures/socod_legend.eps}}   \\[-1mm]
                 \includegraphics[height=26mm]{figures/uniform-max-error.eps} &
			 \includegraphics[height=26mm]{figures/normal-max-error.eps} &
			\ \includegraphics[height=26mm]{figures/multi-modal-max-error.eps} &
			 \includegraphics[height=26mm]{figures/hpmax-max-error.eps}
			\\[-3mm]
             (a) Uniform Random &
                 (b) Random Noisy &
			 (c) Multimodal Data &
			 (d) HPMaX  \\[-1mm]
		\end{tabular}
		\vspace{-3mm}
		\caption{Maximum Sketch Size vs. Maximum Error.} \label{fig:max-error}
		\vspace{-1mm}
\end{figure*}

\begin{figure*}[t]
	\centering
 \vspace{-2mm}
		\begin{tabular}{cccc}
	 	%		\multicolumn{4}{c}{\hspace{-8mm} \includegraphics[height=2.8mm]{figures/socod_legend.pdf}}   \\
			\includegraphics[height=26mm]{figures/uniform-avg-error.eps} &
             \includegraphics[height=26mm]{figures/normal-avg-error.eps} &
			 \includegraphics[height=26mm]{figures/multi-modal-avg-error.eps} &
			 \includegraphics[height=26mm]{figures/hpmax-avg-error.eps}
			\\[-3mm]
                 (a) Uniform Random &
                (b) Random Noisy &
			 (c) Multimodal Data &
			 (d) HPMaX  \\[-1mm]
		\end{tabular}
		\vspace{-3mm}
		\caption{Maximum Sketch Size vs. Average Error.} \label{fig:avg-error}
		% \vspace{-1mm}
\end{figure*}

\begin{figure*}[t]
	\centering
 \vspace{-2mm}
		\begin{tabular}{cccc}
	 	%		\multicolumn{4}{c}{\hspace{-8mm} \includegraphics[height=2.8mm]{figures/socod_legend.pdf}}   \\
			 \includegraphics[height=26mm]{figures/uniform-sketch-size.eps} &
            		 \includegraphics[height=26mm]{figures/normal-sketch-size.eps} &
			 \includegraphics[height=26mm]{figures/multi-modal-sketch-size.eps} &
			 \includegraphics[height=26mm]{figures/hpmax-sketch-size.eps}
			\\[-3mm]
             (a) Uniform Random &
                (b) Random Noisy &
			 (c) Multimodal Data &
			(d) HPMaX  \\[-1mm]
		\end{tabular}
		\vspace{-2mm}
		\caption{$\log_{10}(1/\epsilon)$ vs. Maximum Sketch Size ($\log_{10}(\frac{1}{0.25}) \approx 0.6$, \text{and} $\log_{10}(\frac{1}{0.016}) \approx 1.8$).}\label{fig:sketch-size}
		\vspace{-1mm}
\end{figure*}

\htitle{Sampling method.} For the AMM problem, the algorithm samples a small proportion of the matrices. Specifically, each pair of columns $(x_i,y_i)$ is assigned to a priority $\rho=u^{1/(\xinorm\yinorm)}$, where $u$ is uniformly sampled from the interval $(0,1)$ \cite{efraimidis2006weighted}. This priority-based sampling strategy ensures that columns with larger norms are more likely to be selected, thereby preserving the most significant contributions to the matrix product. To achieve an $\epsilon$-approximation guarantee, the algorithm requires $O(\frac{1}{\epsilon^2})$ independent samples selected based on the highest priorities \cite{drineas2006fast, efraimidis2006weighted}. To extend the priority sampling on the sliding window, we use the technique from \cite{babcock2001sampling}, leading to a space complexity of $O(\frac{d_x+d_y}{\epsilon^2}\log{N})$ for the normalized model and $O(\frac{d_x+d_y}{\epsilon^2}\log{NR})$ for general unnormalized model.

\htitle{DI-COD.} DI-COD applied the Dyadic Interval approach \cite{arasu2004approximate} to Co-Occurring Directions, maintaining a hierarchical structure with $L=\log{\frac{R}{\epsilon}}$ parallel levels, each of which contains a dynamic number of blocks. For $i$-th level, the window is segmented into at most $2^{L-i+1}$ block, and each block maintains a COD sketch. The space cost for DI-COD is $O(\frac{(d_x+d_y)R}{\epsilon}\log^2{\frac{R}{\epsilon}})$. 

\htitle{EH-COD.} Exponential Histogram Co-occurring Directions (EH-COD) combines the Exponential Histograms technique \cite{DatarGIM02} and incorporates the COD algorithm for efficiently approximating matrix multiplication within the sliding window model. The space cost for EH-COD is $O(\frac{d_x+d_y}{\epsilon^2}\log{\epsilon NR})$.



\subsection{Experiments Setup}
\htitle{Datasets.} Experiments are conducted on both synthetic and real-world datasets widely used in matrix multiplication \cite{YaoLCWC24,YeLZ16,MrouehMG17,GhashamiDP14,KangKK20}. All datasets are \emph{unnormalized}.
The details are listed below:
\begin{itemize}[leftmargin=10pt]
\item \textbf{Uniform Random \cite{YaoLCWC24,YeLZ16}.} We generate two random matrices: one of size $2000 \times 10000$ and another of size $1000 \times 10000$. The entries of both matrices are drawn uniformly at random from the interval $[0, 1)$. The window size for this dataset is $N = 4000$.
    
\item \textbf{Random Noisy \cite{MrouehMG17,GhashamiDP14}.} 
We generate the input matrix $\boldsymbol{X}^T = \boldsymbol{SDU} + \boldsymbol{F} / \zeta \in \mathbb{R}^{n \times d_x}$. Here, the term $\boldsymbol{SDU}$ represents an $m$-dimensional signal, while the other part $\boldsymbol{F} / \zeta$ is a Gaussian noise matrix, with scalar parameter $\zeta$ controlling the noise-to-signal ratio.
Specifically, $\boldsymbol{S}\in\mathbb{R}^{n\times m}$ is a random matrix where each entry is drawn from a standard normal distribution. $\boldsymbol{D}\in\mathbb{R}^{m\times m}$ is a diagonal matrix with entries $\boldsymbol{D}_{i,i}=1-(i-1)/m$, and $\boldsymbol{U}\in\mathbb{R}^{m\times d_x}$ is a random rotation which represents the row space of the signal and satisfies that $\boldsymbol{U}^T\boldsymbol{U}=I_m$. $\boldsymbol{F}$ is again a Gaussian matrix with each entries generated i.i.d. from a normal distribution $N(0,1)$. Matrix $\boldsymbol{Y}$ is generated in the same manner as $\boldsymbol{X}$. We set $d_x = 2000$, $d_y=1000$, $m = 400$, $\zeta = 100$, and the window size $N = 4000$.

 \item \textbf{Multimodal Data \cite{MrouehMG17}.} We study the empirical performance of the algorithms in approximating correlation between images and captions. Following \cite{MrouehMG17}, we consider Microsoft COCO dataset \cite{LinMBHPRDZ14}. For visual features we use the residual CNN Resnet101 \cite{HeZRS16} to generate a feature vector of dimension $d_x = 2048$ for each picture. For text we use the Hierarchical Kernel Sentence Embedding \cite{mroueh2015asymmetrically}, resulting in a feature vector of dimensions $d_y = 3000$. We construct the matrices $\boldsymbol{X}$ and $\boldsymbol{Y}$ with sizes $2048 \times 123287$ and $3000 \times 123287$, respectively, where each column represents a feature vector. The window size is set to $N = 10000$.

 \item \textbf{HPMaX \cite{KangKK20}:} We also include the dataset HPMaX, which is used to test the performance of heterogenous parallel algorithms for matrix multiplication. In this dataset, both of $\boldsymbol{X}$ and $\boldsymbol{Y}$  have size of $16384\times 32768$. The window size $N$ is $10000$.
\end{itemize}


\htitle{Evaluation Metrics.} Recall that our \oursolution achieves the optimal space complexity while providing an $\epsilon$-approximation guarantee. Therefore, we design the experiments to explicitly demonstrate the trade-off between space consumption and empirical accuracy across different datasets. Specifically, we tune the parameters of each algorithm and report both the maximum sketch size and the empirical relative correlation error.  
\begin{itemize}[topsep=0.5mm, partopsep=0pt, itemsep=0pt, leftmargin=10pt] 
    \item \textbf{Maximum sketch size}. This metric is measured by the \textit{maximum} number of column vectors maintained by a matrix sketching algorithm. The maximum sketch size metric represents the peak space cost of a matrix sketching algorithm. 
    \item \textbf{Relative correlation error}. This metric is used to assess the approximation quality of the output matrices. It is defined as $\left\| \boldsymbol{X}_W \boldsymbol{Y}_W^T - \boldsymbol{A}_W \boldsymbol{B}_W^T\right\|_2 /\left\|\boldsymbol{X}_W\right\|_F\left\|\boldsymbol{Y}_W\right\|_F$, where $\boldsymbol{X}_W$ and $\boldsymbol{X}_W$ denotes the exact matrices covered by the current window, and $\boldsymbol{A}_W$ and $\boldsymbol{B}_W$ denotes sketch matrices for $\boldsymbol{X}_W\boldsymbol{Y}_W^T$. 
\end{itemize} 





\subsection{Experimental Results}
We first adjust the error parameter $\epsilon$ for each algorithm to analyze the trade-off between space efficiency and empirical accuracy. Generally, when the error parameter $\epsilon$ decreases, the maximum sketch size increases. As shown in Figures~\ref{fig:max-error}--\ref{fig:avg-error}, we report the maximum sketch size, as well as the maximum and average relative correlation errors, for each algorithm. Both the x-axis and y-axis are displayed on a logarithmic scale to encompass the wide range of values. 

First, we observe that the curve representing our solution \oursolution consistently resides in the lower-left corner compared to other baselines, in terms of both maximum and average errors.  This implies that for a given space cost (i.e., maximum sketch size), our \oursolution consistently produces matrices with much lower correlation errors. Therefore, our solution demonstrates a superior space-error trade-off, aligning with its optimal space complexity as discussed in Section~\ref{sec:unnormalized-setting}. Second, on certain datasets (e.g., Multimodal Data and HPMax), the second-best algorithm, EH-COD, produces results comparable to our solution when the maximum sketch size is small (i.e., the error parameter $\epsilon$ is large). However, the gap between the two curves widens as the maximum sketch size increases (i.e., the error parameter $\epsilon$ decreases). This is also aligned with the theoretical result that the suboptimal space complexity $O(\frac{d_x + d_y}{\epsilon^2}\log{\epsilon NR})$ of EH-COD is outperformed by our optimal complexity $O(\frac{d_x + d_y}{\epsilon}\log{R})$. Finally, we observe that the EH-COD baseline performs better than the DI-COD baseline in almost all cases, which aligns with the observations in \cite{YaoLCWC24}.

Then, we examine the impact of the error parameter on the space cost of each algorithm. We vary the parameter $\epsilon$ and report the maximum value of sketch size. The results are shown in Figure~\ref{fig:sketch-size}.  The curve of our solution \oursolution consistently remains the lowest. This indicates that, for a given error parameter, \oursolution requires the least space, thereby confirming the conclusion of space optimality. One may note that as $\log_{10}(1/\epsilon)$ increases, the space growth of the EH-COD algorithm gradually slows down. This occurs because, as $\epsilon$ decreases, the storage capacity of EH-COD increases, and the entire sketch becomes sufficient to store the entire window without significant COD compression operations. Consequently, the maximum sketch size approaches the window size.


In summary, when space is the primary concern, our \oursolution is the preferred choice, delivering the best accuracy under space constraints compared to all competitors.

\section{Conclusion}

In this paper, we introduce STeCa, a novel agent learning framework designed to enhance the performance of LLM agents in long-horizon tasks. 
STeCa identifies deviated actions through step-level reward comparisons and constructs calibration trajectories via reflection. 
These trajectories serve as critical data for reinforced training. Extensive experiments demonstrate that STeCa significantly outperforms baseline methods, with additional analyses underscoring its robust calibration capabilities.
\addition{
\section{Discussion}
\label{sec:discussion}
\DAF provides a systematic method to classify and analyze \MR deception attacks. %, addressing their effects on information channels and cognitive processes. 
While we focus on \MR headsets, \DAF is applicable to other forms of \MR and even other areas of human-computer interaction (HCI).
Kopp et al.'s information-theoretic framework ~\cite{kopp:2018} applied the Borden-Kopp model of deception to news media.
We have broadened its use to \MR deception attacks.
Future work should extend the scope to other areas of HCI that involve information processing and decision-making.
Our information-theoretic model and decision-making model are not tied to specific technologies or attacks, but rather provide generalizable models for studying the effects of deception in computing.
To enhance \DAF, future work should validate it empirically, expand its applicability to diverse contexts, incorporate individual cognitive factors, and refine models for processing attacks.

%By focusing on how \MR deception attacks impact information channels and decision-making processes, 
Researchers and practitioners can use \DAF to assess the security threat of \MR deception attacks.
For example, we can assign values of 1 to 3 for Low to High ratings, respectively.
Then, we can sum the values to identify which attacks pose the highest threat to perception and attention.
Further, \DAF can help develop deception detection and prevention approaches. % by using information theory to model \MR communication channels.
For example, we can compare differences between rendered frames to see how the signal is changing.
%For example, we can diff displayed frames with previous ones or an expected frame to identify changes in visual information presented to a \MR user.
%These diffs can reveal changes in channel capacity as information is either hidden or injected possibly along with noise.
High volatility in changes may indicate overt degradation attacks, particularly if we can identify noise based on differences between expected and actual frames.
More subtle changes that are spatial located in unexpected areas may indicate covert degradation attacks.
Using eye-tracking sensors on these headsets, we can derive models of attention that can help identify when different types of attention are being employed or disrupted.

% For example, we could use display-capture or eye-tracking data to detect a momentary misdirection attack.
% A misdirection attack 
%It offers foundational understanding for both technical and psychological dimensions of deception, with significant implications for future \MR research. 
%Our framework is adaptable for diverse \MR platforms and can guide empirical research into attack impacts and countermeasures. 

}

\addition{
%\subsection{Limitations}
%\label{sec:limitations}
\textbf{Limitations:} This SoK synthesizes existing knowledge towards developing a field of study around \MR deception. % by establishing a generalizable framework.
%It is essential to acknowledge the limitations of this research, which highlight potential areas for further exploration.
%One significant limitation is the lack of empirical validation. 
It is theoretical in nature and would benefit from further empirical validation.
%While it is rooted in empirical evidence from prior work, it lacks empirical validation.
%DAF’s models and  require real-world testing to confirm their accuracy and practical effectiveness. 
Controlled experiments involving \MR deception attacks are essential for refining the framework and assessing its relevance to diverse scenarios. 
Furthermore, \DAF does not fully account for cognitive diversity among users. 
Individual differences in cognitive capacity, attention, and susceptibility to deception are critical factors that could influence the effectiveness of both attacks and countermeasures. 
%Incorporating these factors would improve the framework’s precision and personalization.
% As \MR technology advances, the sophistication of attacks will also increase, highlighting the importance of further study in this area.
}
\input{10-Acknowledgements}

\bibliography{custom}


\clearpage
\newpage
\appendix
\onecolumn
\section{Hard Threshold of EAC}\label{threshhold}
In constructing a weighted-gradient saliency map, the value of \(\gamma\) determines the number of the dimensions we select where important feature anchors are located. As the value of \(\gamma\) increases, the number of selected dimensions decreases, requiring the editing information to be compressed into a smaller space during the compression process. 
During compression, it is desired for the compression space to be as small as possible to preserve the general abilities of the model. However, reducing the compression space inevitably increases the loss of editing information, which reduces the editing performance of the model.
Therefore, to ensure editing performance in a single editing scenario, different values of \(\gamma\) are determined for various models, methods, and datasets. Fifty pieces of knowledge were randomly selected from the dataset, and reliability, generalization, and locality were measured after editing. The averages of these metrics were then taken as a measure of the editing performance of the model.
Table~\ref{value} presents the details of \(\gamma\), while Table~\ref{s} illustrates the corresponding editing performance before and after the introduction of EAC. $P_{x}$ denotes the value below which x\% of the values in the dataset.


\begin{table}[!htb]
\caption{The value of $\gamma$.}
\centering
\resizebox{0.45\textwidth}{!}{
\begin{tabular}{lcccc}
\toprule
\textbf{Datasets} & \textbf{Model} & \textbf{ROME} & \textbf{MEMIT} \\
\midrule
\multirow{2}{*}{\textbf{ZSRE}} & GPT-2 XL & $P_{80}$ & $P_{80}$ \\
 & LLaMA-3 (8B) & $P_{90}$ & $P_{95}$ \\
\midrule
\multirow{2}{*}{\textbf{COUNTERFACT}} & GPT-2 XL & $P_{85}$ & $P_{85}$ \\
 & LLaMA-3 (8B) & $P_{95}$ & $P_{95}$ \\
\bottomrule
\end{tabular}}
\label{value}
\end{table}


\begin{table}[!htb]
\caption{The value of $\gamma$.}
\centering
\resizebox{\textwidth}{!}{%
\begin{tabular}{lccccccccccccc}
\toprule
\multirow{1}{*}{Dataset} & \multirow{1}{*}{Method} & \multicolumn{3}{c}{\textbf{GPT-2 XL}} & \multicolumn{3}{c}{\textbf{LLaMA-3 (8B)}} \\
\cmidrule(lr){3-5} \cmidrule(lr){6-8}
& & \multicolumn{1}{c}{Reliability} & \multicolumn{1}{c}{Generalization} & \multicolumn{1}{c}{Locality} & \multicolumn{1}{c}{Reliability} & \multicolumn{1}{c}{Generalization} & \multicolumn{1}{c}{Locality} \\
\midrule
\multirow{1}{*}{ZsRE} & ROME & 1.0000 & 0.9112 & 0.9661 & 1.0000 & 0.9883 & 0.9600  \\
& ROME-EAC & 1.0000 & 0.8923 & 0.9560 & 0.9933 & 0.9733 & 0.9742  \\
\cmidrule(lr){2-8}
& MEMIT & 0.6928 & 0.5208 & 1.0000 & 0.9507 & 0.9333 & 0.9688  \\
& MEMIT-EAC & 0.6614 & 0.4968 & 0.9971 & 0.9503 & 0.9390 & 0.9767  \\
\midrule
\multirow{1}{*}{CounterFact} & ROME & 1.0000 & 0.4200 & 0.9600 & 1.0000 & 0.3600 & 0.7800  \\
& ROME-EAC & 0.9800 & 0.3800 & 0.9600 & 1.0000 & 0.3200 & 0.8800  \\
\cmidrule(lr){2-8}
& MEMIT & 0.9000 & 0.2200 & 1.0000 & 1.0000 & 0.3800 & 0.9500  \\
& MEMIT-EAC & 0.8000 & 0.1800 & 1.0000 & 1.0000 & 0.3200 & 0.9800  \\
\bottomrule
\end{tabular}%
}
\label{s}
\end{table}

\section{Optimization Details}\label{b}
ROME derives a closed-form solution to achieve the optimization:
\begin{equation}
\text{minimize} \ \| \widehat{W}K - V \| \ \text{such that} \ \widehat{W}\mathbf{k}_* = \mathbf{v}_* \ \text{by setting} \ \widehat{W} = W + \Lambda (C^{-1}\mathbf{k}_*)^T.
\end{equation}

Here \( W \) is the original matrix, \( C = KK^T \) is a constant that is pre-cached by estimating the uncentered covariance of \( \mathbf{k} \) from a sample of Wikipedia text, and \( \Lambda = (\mathbf{v}_* - W\mathbf{k}_*) / ( (C^{-1}\mathbf{k}_*)^T \mathbf{k}_*) \) is a vector proportional to the residual error of the new key-value pair on the original memory matrix.

In ROME, \(\mathbf{k}_*\) is derived from the following equation:
\begin{equation}
\mathbf{k}_* = \frac{1}{N} \sum_{j=1}^{N} \mathbf{k}(x_j + s), \quad \text{where} \quad \mathbf{k}(x) = \sigma \left( W_{f_c}^{(l^*)} \gamma \left( a_{[x],i}^{(l^*)} + h_{[x],i}^{(l^*-1)} \right) \right).
\end{equation}

ROME set $\mathbf{v}_* = \arg\min_z \mathcal{L}(z)$, where the objective $\mathcal{L}(z)$ is:
\begin{equation}
\frac{1}{N} \sum_{j=1}^{N} -\log \mathbb{P}_{G(m_{i}^{l^*}:=z))} \left[ o^* \mid x_j + p \right] + D_{KL} \left( \mathbb{P}_{G(m_{i}^{l^*}:=z)} \left[ x \mid p' \right] \parallel \mathbb{P}_{G} \left[ x \mid p' \right] \right).
\end{equation}

\section{Experimental Setup} \label{detail}

\subsection{Editing Methods}\label{EM}

In our experiments, Two popular editing methods including ROME and MEMIT were selected as baselines.

\textbf{ROME} \cite{DBLP:conf/nips/MengBAB22}: it first localized the factual knowledge at a specific layer in the transformer MLP modules, and then updated the knowledge by directly writing new key-value pairs in the MLP module.

\textbf{MEMIT} \cite{DBLP:conf/iclr/MengSABB23}: it extended ROME to edit a large set of facts and updated a set of MLP layers to update knowledge.

The ability of these methods was assessed based on EasyEdit~\cite{DBLP:journals/corr/abs-2308-07269}, an easy-to-use knowledge editing framework which integrates the released codes and hyperparameters from previous methods.

\subsection{Editing Datasets}\label{dat}
In our experiment, two popular model editing datasets \textsc{ZsRE}~\cite{DBLP:conf/conll/LevySCZ17} and \textsc{CounterFact}~\cite{DBLP:conf/nips/MengBAB22} were adopted.

\textbf{\textsc{ZsRE}} is a QA dataset using question rephrasings generated by back-translation as the equivalence neighborhood.
Each input is a question about an entity, and plausible alternative edit labels are sampled from the top-ranked predictions of a BART-base model trained on \textsc{ZsRE}.

\textbf{\textsc{CounterFact}} accounts for counterfacts that start with low scores in comparison to correct facts. It constructs out-of-scope data by substituting the subject entity for a proximate subject entity sharing a predicate. This alteration enables us to differentiate between superficial wording changes and more significant modifications that correspond to a meaningful shift in a fact. 

\subsection{Metrics for Evaluating Editing Performance}\label{Mediting performance}
\paragraph{Reliability} means that given an editing factual knowledge, the edited model should produce the expected predictions. The reliability is measured as the average accuracy on the edit case:
\begin{equation}
\mathbb{E}_{(x'_{ei}, y'_{ei}) \sim \{(x_{ei}, y_{ei})\}} \mathbf{1} \left\{ \arg\max_y f_{\theta_{i}} \left( y \mid x'_{ei} \right) = y'_{ei} \right\}.
\label{rel}
\end{equation}

\paragraph{Generalization} means that edited models should be able to recall the updated knowledge when prompted within the editing scope. The generalization is assessed by the average accuracy of the model on examples uniformly sampled from the equivalence neighborhood:
\begin{equation}
\mathbb{E}_{(x'_{ei}, y'_{ei}) \sim N(x_{ei}, y_{ei})} \mathbf{1} \left\{ \arg\max_y f_{\theta_{i}} \left( y \mid x'_{ei} \right) = y'_{ei} \right\}.
\label{gen}
\end{equation}

\paragraph{Locality} means that the edited model should remain unchanged in response to prompts that are irrelevant or the out-of-scope. The locality is evaluated by the rate at which the edited model's predictions remain unchanged compared to the pre-edit model.
\begin{equation}
\mathbb{E}_{(x'_{ei}, y'_{ei}) \sim O(x_{ei}, y_{ei})} \mathbf{1} \left\{ f_{\theta_{i}} \left( y \mid x'_{ei} \right) = f_{\theta_{i-1}} \left( y \mid x'_{ei} \right) \right\}.
\label{loc}
\end{equation}

\subsection{Downstream Tasks}\label{pro}

Four downstream tasks were selected to measure the general abilities of models before and after editing:
\textbf{Natural language inference (NLI)} on the RTE~\cite{DBLP:conf/mlcw/DaganGM05}, and the results were measured by accuracy of two-way classification.
\textbf{Open-domain QA} on the Natural Question~\cite{DBLP:journals/tacl/KwiatkowskiPRCP19}, and the results were measured by exact match (EM) with the reference answer after minor normalization as in \citet{DBLP:conf/acl/ChenFWB17} and \citet{DBLP:conf/acl/LeeCT19}.
\textbf{Summarization} on the SAMSum~\cite{gliwa-etal-2019-samsum}, and the results were measured by the average of ROUGE-1, ROUGE-2 and ROUGE-L as in \citet{lin-2004-rouge}.
\textbf{Sentiment analysis} on the SST2~\cite{DBLP:conf/emnlp/SocherPWCMNP13}, and the results were measured by accuracy of two-way classification.

The prompts for each task were illustrated in Table~\ref{tab-prompt}.

\begin{table*}[!htb]
% \small
\centering
\begin{tabular}{p{0.95\linewidth}}
\toprule

NLI:\\
\{\texttt{SENTENCE1}\} entails the \{\texttt{SENTENCE2}\}. True or False? answer:\\

\midrule

Open-domain QA:\\
Refer to the passage below and answer the following question. Passage: \{\texttt{DOCUMENT}\} Question: \{\texttt{QUESTION}\}\\

\midrule

Summarization:\\
\{\texttt{DIALOGUE}\} TL;DR:\\

\midrule


Sentiment analysis:\\
For each snippet of text, label the sentiment of the text as positive or negative. The answer should be exact 'positive' or 'negative'. text: \{\texttt{TEXT}\} answer:\\

\bottomrule
\end{tabular}
\caption{The prompts to LLMs for evaluating their zero-shot performance on these general tasks.}
\label{tab-prompt}
\end{table*}

\subsection{Hyper-parameters for Elastic Net}\label{hy}

In our experiment, we set \(\lambda = 5 \times 10^{-7} \), \(\mu = 5 \times 10^{-1} \) for GPT2-XL\cite{radford2019language} and \(\lambda = 5 \times 10^{-7} \), \(\mu = 1 \times 10^{-3} \) for LLaMA-3 (8B)\cite{llama3}.

\begin{figure*}[!hbt]
  \centering
  \includegraphics[width=0.5\textwidth]{figures/legend_edit.pdf}
  \vspace{-4mm}
\end{figure*}

\begin{figure*}[!hbt]
  \centering
  \subfigure{
  \includegraphics[width=0.23\textwidth]{figures/ROME-GPT2XL-CF-edit.pdf}}
  \subfigure{
  \includegraphics[width=0.23\textwidth]{figures/ROME-LLaMA3-8B-CF-edit.pdf}}
  \subfigure{
  \includegraphics[width=0.23\textwidth]{figures/MEMIT-GPT2XL-CF-edit.pdf}}
  \subfigure{
  \includegraphics[width=0.23\textwidth]{figures/MEMIT-LLaMA3-8B-CF-edit.pdf}}
  \caption{Edited on CounterFact, editing performance of edited models using the ROME~\cite{DBLP:conf/nips/MengBAB22} and MEMIT~\cite{DBLP:conf/iclr/MengSABB23} on GPT2-XL~\cite{radford2019language} and LLaMA-3 (8B)~\cite{llama3}, as the number of edits increases before and after the introduction of EAC.}
  \vspace{-4mm}
  \label{edit-performance-cf}
\end{figure*}

\begin{figure*}[!hbt]
  \centering
  \includegraphics[width=0.75\textwidth]{figures/legend.pdf}
  \vspace{-4mm}
\end{figure*}

\begin{figure*}[!htb]
  \centering
  \subfigure{
  \includegraphics[width=0.23\textwidth]{figures/ROME-GPT2XL-CounterFact.pdf}}
  \subfigure{
  \includegraphics[width=0.23\textwidth]{figures/ROME-LLaMA3-8B-CounterFact.pdf}}
  \subfigure{
  \includegraphics[width=0.23\textwidth]{figures/MEMIT-GPT2XL-CounterFact.pdf}}
  \subfigure{
  \includegraphics[width=0.23\textwidth]{figures/MEMIT-LLaMA3-8B-CounterFact.pdf}}
  \caption{Edited on CounterFact, performance on general tasks using the ROME~\cite{DBLP:conf/nips/MengBAB22} and MEMIT~\cite{DBLP:conf/iclr/MengSABB23} on GPT2-XL~\cite{radford2019language} and LLaMA-3 (8B)~\cite{llama3}, as the number of edits increases before and after the introduction of EAC.}
  \vspace{-4mm}
  \label{task-performance-cf}
\end{figure*}

\section{Experimental Results}\label{app}

\subsection{Results of Editing Performance}\label{cf-performance}
Applying CounterFact as the editing dataset, Figure~\ref{edit-performance-cf} presents the editing performance of the ROME~\cite{DBLP:conf/nips/MengBAB22} and MEMIT~\cite{DBLP:conf/iclr/MengSABB23} methods on GPT2-XL~\cite{radford2019language} and LLaMA-3 (8B)~\cite{llama3}, respectively, as the number of edits increases before and after the introduction of EAC.
The dashed line represents the ROME or MEMIT, while the solid line represents the ROME or MEMIT applying the EAC.


\subsection{Results of General Abilities}\label{cf-ability}
Applying CounterFact as the editing dataset, Figure~\ref{task-performance-cf} presents the performance on general tasks of edited models using the ROME~\cite{DBLP:conf/nips/MengBAB22} and MEMIT~\cite{DBLP:conf/iclr/MengSABB23} methods on GPT2-XL~\cite{radford2019language} and LLaMA-3 (8B)~\cite{llama3}, respectively, as the number of edits increases before and after the introduction of EAC. 
The dashed line represents the ROME or MEMIT, while the solid line represents the ROME or MEMIT applying the EAC.

\subsection{Results of Larger Model}\label{13 B}
To better demonstrate the scalability and efficiency of our approach, we conducted experiments using the LLaMA-2 (13B)~\cite{DBLP:journals/corr/abs-2307-09288}.
Figure~\ref{13B-edit} presents the editing performance of the ROME~\cite{DBLP:conf/nips/MengBAB22} and MEMIT~\cite{DBLP:conf/iclr/MengSABB23} methods on LLaMA-2 (13B)
~\cite{DBLP:journals/corr/abs-2307-09288}, as the number of edits increases before and after the introduction of EAC.
Figure~\ref{13B} presents the performance on general tasks of edited models using the ROME and MEMIT methods on LLaMA-2 (13B), as the number of edits increases before and after the introduction of EAC.
The dashed line represents the ROME or MEMIT, while the solid line represents the ROME or MEMIT applying the EAC.

\begin{figure*}[!hbt]
  \centering
  \includegraphics[width=0.5\textwidth]{figures/legend_edit.pdf}
  \vspace{-4mm}
\end{figure*}

\begin{figure*}[!hbt]
  \centering
  \subfigure{
  \includegraphics[width=0.23\textwidth]{figures/ROME-LLaMA2-13B-ZsRE-edit.pdf}}
  \subfigure{
  \includegraphics[width=0.23\textwidth]{figures/MEMIT-LLaMA2-13B-ZsRE-edit.pdf}}
  \subfigure{
  \includegraphics[width=0.23\textwidth]{figures/ROME-LLaMA2-13B-CF-edit.pdf}}
  \subfigure{
  \includegraphics[width=0.23\textwidth]{figures/MEMIT-LLaMA2-13B-CF-edit.pdf}}
  \caption{Editing performance of edited models using the ROME~\cite{DBLP:conf/nips/MengBAB22} and MEMIT~\cite{DBLP:conf/iclr/MengSABB23} on LLaMA-2 (13B)~\cite{DBLP:journals/corr/abs-2307-09288}, as the number of edits increases before and after the introduction of EAC.}
  \vspace{-4mm}
  \label{13B-edit}
\end{figure*}

\begin{figure*}[!hbt]
  \centering
  \includegraphics[width=0.75\textwidth]{figures/legend.pdf}
  \vspace{-4mm}
\end{figure*}

\begin{figure*}[!htb]
  \centering
  \subfigure{
  \includegraphics[width=0.23\textwidth]{figures/ROME-LLaMA2-13B-ZsRE.pdf}}
  \subfigure{
  \includegraphics[width=0.23\textwidth]{figures/MEMIT-LLaMA2-13B-ZsRE.pdf}}
  \subfigure{
  \includegraphics[width=0.23\textwidth]{figures/ROME-LLaMA2-13B-CounterFact.pdf}}
  \subfigure{
  \includegraphics[width=0.23\textwidth]{figures/MEMIT-LLaMA2-13B-CounterFact.pdf}}
  \caption{Performance on general tasks using the ROME~\cite{DBLP:conf/nips/MengBAB22} and MEMIT~\cite{DBLP:conf/iclr/MengSABB23} on LLaMA-2 (13B)~\cite{DBLP:journals/corr/abs-2307-09288}, as the number of edits increases before and after the introduction of EAC.}
  \vspace{-4mm}
  \label{13B}
\end{figure*}

\section{Analysis of Elastic Net}
\label{FT}
It is worth noting that the elastic net introduced in EAC can be applied to methods beyond ROME and MEMIT, such as FT~\cite{DBLP:conf/emnlp/CaoAT21}, to preserve the general abilities of the model.
Unlike the previously mentioned fine-tuning, FT is a model editing approach. It utilized the gradient to gather information about the knowledge to be updated and applied this information directly to the model parameters for updates.
Similar to the approaches of ROME and MEMIT, which involve locating parameters and modifying them, the FT method utilizes gradient information to directly update the model parameters for editing. Therefore, we incorporate an elastic net during the training process to constrain the deviation of the edited matrix.
Figure~\ref{ft} shows the sequential editing performance of FT on GPT2-XL and LLaMA-3 (8B) before and after the introduction of elastic net.
The dashed line represents the FT, while the solid line represents the FT applying the elastic net.
The experimental results indicate that when using the FT method to edit the model, the direct use of gradient information to modify the parameters destroys the general ability of the model. By constraining the deviation of the edited matrix, the incorporation of the elastic net effectively preserves the general abilities of the model.

\begin{figure*}[t]
  \centering
  \subfigure{
  \includegraphics[width=0.43\textwidth]{figures/legend_FT.pdf}}
\end{figure*}

\begin{figure*}[t]%[!ht]
  \centering
  \subfigure{
  \includegraphics[width=0.22\textwidth]{figures/FT-GPT2XL-ZsRE.pdf}}
  \subfigure{
  \includegraphics[width=0.22\textwidth]{figures/FT-GPT2XL-CounterFact.pdf}}
  \vspace{-2mm}
  \caption{Edited on the ZsRE or CounterFact datasets, the sequential editing performance of FT~\cite{DBLP:conf/emnlp/CaoAT21} and FT with elastic net on GPT2-XL before and after the introduction of elastic net.}
  \vspace{-2mm}
  \label{ft}
\end{figure*}



\end{document}
