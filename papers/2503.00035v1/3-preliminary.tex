\section{Preliminary}
Model editing involves modifying the knowledge stored within LMs without requiring retraining, to better meet specific tasks or requirements. This process aims to refine various complex learned beliefs, including logical, spatial, and numerical knowledge.
In this paper, we study editing factual knowledge in the form of $(x_e, y_e)$\footnotemark{}.
\footnotetext{Can be also represented as knowledge triple \( t = (subject, relation, object) \).}
The language model $f_\theta \in \mathcal{F}$ can be defined as a function $f_\theta : \mathcal{X} \rightarrow \mathcal{Y}$, mapping input $x \in \mathcal{X}$ to its prediction $y \in \mathcal{Y}$. For an editing factual knowledge $(x_e, y_e)$, where $f_\theta(x_e) \neq y_e$, the goal of the model editing is to edit the parameters $\theta \in \Theta$ of the model $f_\theta$ to obtain an edited model $f_{\theta'}$, such that $f_{\theta'}(x_e) = y_e$.
In sequential editing, this process continues iteratively. Given a set of editing facts $\mathcal{E} = \left\{ (x_{ei}, y_{ei}) \mid i = 1, \ldots, n \right\}$ and an initial model $f_{\theta_{0}}$, each model editing step involves learning a function $K$ that produces an edited language model $f_{\theta_i}$ such that $K(f_{\theta_{i-1}}, (x_{ei}, y_{ei})) = f_{\theta_i}$.

The model editing process typically affects the predictions for a broad set of inputs closely related to the edited factual knowledge. This collection of inputs is referred to as the \textit{editing scope}. A successful edit should modify the model's behavior within the target scope while preserving performance on out-of-scope examples:
\[ 
f_{\theta_i}(x_{ei}) = 
\begin{cases} 
y_{ei} & \text{if} \,\, x_{ei} \in I(x_{ei}, y_{ei}), \\
f_{\theta_{i-1}}(x_{ei}) & \text{if} \,\, x_{ei} \in O(x_{ei}, y_{ei}). 
\end{cases} 
\]
The \textit{in-scope} \(I(x_{ei}, y_{ei})\) typically includes \(x_{ei}\) and its equivalence neighborhood \(N(x_{ei}, y_{ei})\), which encompasses related input/output pairs. In contrast, the \textit{out-of-scope} \(O(x_{ei}, y_{ei})\) comprises inputs unrelated to the edit example.
To evaluate the effectiveness of various model editing methods, previous works focus on evaluation along three dimensions: \emph{reliability}, \emph{generalization} and \emph{locality}~\cite{DBLP:conf/emnlp/CaoAT21,DBLP:conf/iclr/MitchellLBFM22,DBLP:conf/nips/MengBAB22,DBLP:conf/iclr/MengSABB23,DBLP:conf/emnlp/YaoWT0LDC023}. 
