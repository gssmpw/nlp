%%%%%%%% ICML 2025 EXAMPLE LATEX SUBMISSION FILE %%%%%%%%%%%%%%%%%

\documentclass{article}

% Recommended, but optional, packages for figures and better typesetting:
\usepackage{microtype}
\usepackage{graphicx}
\usepackage{subfigure}
\usepackage{booktabs} % for professional tables

% hyperref makes hyperlinks in the resulting PDF.
% If your build breaks (sometimes temporarily if a hyperlink spans a page)
% please comment out the following usepackage line and replace
% \usepackage{icml2025} with \usepackage[nohyperref]{icml2025} above.
\usepackage{hyperref}


% Attempt to make hyperref and algorithmic work together better:
\newcommand{\theHalgorithm}{\arabic{algorithm}}

% Use the following line for the initial blind version submitted for review:
\usepackage[accepted]{arxiv}

% If accepted, instead use the following line for the camera-ready submission:
%\usepackage[accepted]{icml2025}

% For theorems and such
\usepackage{amsmath}
\usepackage{amssymb}
\usepackage{mathtools}
\usepackage{amsthm}

% if you use cleveref..
\usepackage[capitalize,noabbrev]{cleveref}

\usepackage{xurl}

% For grey boxes in each section
\usepackage{tcolorbox}

% itemize noindent
\usepackage{enumitem}

\usepackage[T1]{fontenc}
\usepackage[scaled=0.85]{roboto-mono}         % Better monospace font

%%%%%%%%%%%%%%%%%%%%%%%%%%%%%%%%
% THEOREMS
%%%%%%%%%%%%%%%%%%%%%%%%%%%%%%%%
\theoremstyle{plain}
\newtheorem{theorem}{Theorem}[section]
\newtheorem{proposition}[theorem]{Proposition}
\newtheorem{lemma}[theorem]{Lemma}
\newtheorem{corollary}[theorem]{Corollary}
\theoremstyle{definition}
\newtheorem{definition}[theorem]{Definition}
\newtheorem{assumption}[theorem]{Assumption}
\theoremstyle{remark}
\newtheorem{remark}[theorem]{Remark}

\newcommand{\action}{{\boldsymbol a}}
\newcommand{\setaction}{\boldsymbol{\mathcal{A}}}

% Todonotes is useful during development; simply uncomment the next line
%    and comment out the line below the next line to turn off comments
%\usepackage[disable,textsize=tiny]{todonotes}
\usepackage[textsize=tiny]{todonotes}
\newcommand{\response}[1]{\vspace{3pt}\hrule\vspace{3pt}\textbf{#1:} }


% The \icmltitle you define below is probably too long as a header.
% Therefore, a short form for the running title is supplied here:
\icmltitlerunning{Societal Alignment Frameworks Can Improve LLM Alignment}

\begin{document}

\twocolumn[
%\icmltitle{Position: Societal Frameworks Address Limitations of LLM Alignment}
%\icmltitle{Social theories help addressess the limitations of technical approaches to LLM alignment}
\icmltitle{Societal Alignment Frameworks Can Improve LLM Alignment}
%\icmltitle{Position: Societal Alignment Frameworks Help in Addressing \\ the Limitations of LLM Alignment}
%\icmltitle{Position: Contract Incompleteness in LLM Alignment Can be Mitigated by Societal Frameworks}

% It is OKAY to include author information, even for blind
% submissions: the style file will automatically remove it for you
% unless you've provided the [accepted] option to the icml2025
% package.

% List of affiliations: The first argument should be a (short)
% identifier you will use later to specify author affiliations
% Academic affiliations should list Department, University, City, Region, Country
% Industry affiliations should list Company, City, Region, Country

% You can specify symbols, otherwise they are numbered in order.
% Ideally, you should not use this facility. Affiliations will be numbered
% in order of appearance and this is the preferred way.
\icmlsetsymbol{equal}{*}
\icmlsetsymbol{prior}{\tiny{$\dagger$}}

\begin{icmlauthorlist}
\icmlauthor{Karolina Sta\'nczak}{mila,mcgill}
\icmlauthor{Nicholas Meade}{mila,mcgill}
\icmlauthor{Mehar Bhatia}{mila,mcgill}
\icmlauthor{Hattie Zhou}{mila,mont,anth,prior}
\icmlauthor{Konstantin Böttinger}{frau}
\icmlauthor{Jeremy Barnes}{serv}
\icmlauthor{Jason Stanley}{serv}
\icmlauthor{Jessica Montgomery}{cam}
\icmlauthor{Richard Zemel}{colu}
\icmlauthor{Nicolas Papernot}{toro,deep}
\icmlauthor{Nicolas Chapados}{mila,serv}
\icmlauthor{Denis Therien}{mcgill,serv}
\icmlauthor{Timothy P Lillicrap}{deep}
\icmlauthor{Ana Marasovi\'c}{utah} \\
\icmlauthor{Sylvie Delacroix}{king} 
\icmlauthor{Gillian K Hadfield}{jhs}
\icmlauthor{Siva Reddy}{mila,mcgill,serv}
\end{icmlauthorlist}

\icmlaffiliation{mila}{Mila – Quebec AI Institute}
\icmlaffiliation{mcgill}{McGill University}
\icmlaffiliation{mont}{Universit\'e de Montr\'eal}
\icmlaffiliation{anth}{Anthropic}
\icmlaffiliation{frau}{Fraunhofer AISEC}
\icmlaffiliation{serv}{ServiceNow}
\icmlaffiliation{cam}{University of Cambridge}
\icmlaffiliation{colu}{Columbia University}
\icmlaffiliation{toro}{University of Toronto} 
\icmlaffiliation{deep}{Google DeepMind}
\icmlaffiliation{utah}{University of Utah}
\icmlaffiliation{king}{King's College London} 
\icmlaffiliation{jhs}{Johns Hopkins University} 

\icmlcorrespondingauthor{Karolina Sta\'nczak}{karolina.stanczak@mila.quebec}

% You may provide any keywords that you
% find helpful for describing your paper; these are used to populate
% the "keywords" metadata in the PDF but will not be shown in the document
\icmlkeywords{Alignment, LLM, ICML}

\vskip 0.3in
]

% this must go after the closing bracket ] following \twocolumn[ ...

% This command actually creates the footnote in the first column
% listing the affiliations and the copyright notice.
% The command takes one argument, which is text to display at the start of the footnote.
% The \icmlEqualContribution command is standard text for equal contribution.
% Remove it (just {}) if you do not need this facility.

%\printAffiliationsAndNotice{}  % leave blank if no need to mention equal contribution
\printAffiliationsAndNotice{\textsuperscript{{\tiny{$\dagger$}}}Work done by HZ prior to joining Anthropic.} % otherwise use the standard text.

\begin{abstract}
Recent progress in large language models (LLMs) has focused on producing responses that meet human expectations and align with shared values\,---\,a process coined \textit{alignment}. 
However, aligning LLMs remains challenging due to the inherent disconnect between the complexity of human values and the narrow nature of the technological approaches designed to address them. 
Current alignment methods often lead to misspecified objectives, reflecting the broader issue of \textit{incomplete contracts}, the impracticality of specifying a contract between a model developer, and the model that accounts for every scenario in LLM alignment. 
In this paper, we argue that improving LLM alignment requires incorporating insights from societal alignment frameworks, including social, economic, and contractual alignment, and discuss potential solutions drawn from these domains. 
Given the role of uncertainty within societal alignment frameworks, we then investigate how it manifests in LLM alignment.
We end our discussion by offering an alternative view on LLM alignment, framing the under-specified nature of its objectives as an opportunity rather than perfect their specification. Beyond technical improvements in LLM alignment, we discuss the need for participatory alignment interface designs.


\end{abstract}

\section{Introduction}\label{sec:Intro} 


Novel view synthesis offers a fundamental approach to visualizing complex scenes by generating new perspectives from existing imagery. 
This has many potential applications, including virtual reality, movie production and architectural visualization \cite{Tewari2022NeuRendSTAR}. 
An emerging alternative to the common RGB sensors are event cameras, which are  
 bio-inspired visual sensors recording events, i.e.~asynchronous per-pixel signals of changes in brightness or color intensity. 

Event streams have very high temporal resolution and are inherently sparse, as they only happen when changes in the scene are observed. 
Due to their working principle, event cameras bring several advantages, especially in challenging cases: they excel at handling high-speed motions 
and have a substantially higher dynamic range of the supported signal measurements than conventional RGB cameras. 
Moreover, they have lower power consumption and require varied storage volumes for captured data that are often smaller than those required for synchronous RGB cameras \cite{Millerdurai_3DV2024, Gallego2022}. 

The ability to handle high-speed motions is crucial in static scenes as well,  particularly with handheld moving cameras, as it helps avoid the common problem of motion blur. It is, therefore, not surprising that event-based novel view synthesis has gained attention, although color values are not directly observed.
Notably, because of the substantial difference between the formats, RGB- and event-based approaches require fundamentally different design choices. %

The first solutions to event-based novel view synthesis introduced in the literature demonstrate promising results \cite{eventnerf, enerf} and outperform non-event-based alternatives for novel view synthesis in many challenging scenarios. 
Among them, EventNeRF \cite{eventnerf} enables novel-view synthesis in the RGB space by assuming events associated with three color channels as inputs. 
Due to its NeRF-based architecture \cite{nerf}, it can handle single objects with complete observations from roughly equal distances to the camera. 
It furthermore has limitations in training and rendering speed: 
the MLP used to represent the scene requires long training time and can only handle very limited scene extents or otherwise rendering quality will deteriorate. 
Hence, the quality of synthesized novel views will degrade for larger scenes. %

We present Event-3DGS (E-3DGS), i.e.,~a new method for novel-view synthesis from event streams using 3D Gaussians~\cite{3dgs} 
demonstrating fast reconstruction and rendering as well as handling of unbounded scenes. 
The technical contributions of this paper are as follows: 
\begin{itemize}
\item With E-3DGS, we introduce the first approach for novel view synthesis from a color event camera that combines 3D Gaussians with event-based supervision. 
\item We present frustum-based initialization, adaptive event windows, isotropic 3D Gaussian regularization and 3D camera pose refinement, and demonstrate that high-quality results can be obtained. %

\item Finally, we introduce new synthetic and real event datasets for large scenes to the community to study novel view synthesis in this new problem setting. 
\end{itemize}
Our experiments demonstrate systematically superior results compared to EventNeRF \cite{eventnerf} and other baselines. 
The source code and dataset of E-3DGS are released\footnote{\url{https://4dqv.mpi-inf.mpg.de/E3DGS/}}. 






\section{Contemporary Approach to LLM Alignment}
\label{sec:alignment}

Aligning LLMs with human values is commonly understood as training them to act in accordance with user intentions \citep{leike2018scalableagentalignmentreward}.
The objective of LLM alignment is often conceptualized as fulfilling three core qualities, often referred to as the ``3H'' framework: honesty (regarding their capabilities, internal states, and knowledge), helpfulness (in performing requested tasks or answering questions within safe bounds), and harmlessness (encompassing both the refusal to fulfill harmful requests and the avoidance of generating harmful content) \citep{askell2021generallanguageassistantlaboratory,Bai2022TrainingAH}.\looseness=-1 

A prominent approach to achieve this alignment is through a preference-based approach like RLHF. The RLHF pipeline usually includes three stages: supervised fine-tuning (SFT), preference sampling and reward model training \citep{NIPS2017_d5e2c0ad,steinnon2020learning}, and reinforcement learning fine-tuning either using proximal policy optimization (PPO; \citealt{schulman2017proximalpolicyoptimizationalgorithms}), or directly through policy optimization (DPO; \citealt{rafailov2023direct}). The process usually starts with a generic pre-trained language model, which undergoes supervised learning on a high-quality dataset for specific downstream tasks. In this paper, we focus on the implications of the reward modeling stage due to its connection to an incomplete contract, which we will lay out in \Cref{sec:contract}.\looseness=-1

\subsection{Reward modeling from human preference.}
In the reward modeling stage, for a given input prompt $x$, the SFT model generates paired outputs, ${y_0, y_1} \in \mathcal{Y} \times \mathcal{Y}$, where $\mathcal{Y}$ denotes the set of all possible outputs that the model can generate in response to a given input. Human evaluators then select their preferred response, $y \in {y_0, y_1}$, providing data that guides the alignment process \citep{NIPS2017_d5e2c0ad,steinnon2020learning}. 
Human preferences are modeled probabilistically using frameworks like the Bradley-Terry model \citep{19ff28b9-64f9-3656-ba40-08326a05748e}. The preference probability for one response over another is expressed as%\looseness=-1
\begin{equation}
p(y_1 \succ y_2 \mid x) = \frac{\exp(r(x, y_1))}{\exp(r(x, y_1)) + \exp(r(x, y_2))},
\end{equation}
where $r(x, y)$ is a latent reward function approximated by a parametric reward model, $r_\phi(x, y)$. Using a dataset of comparisons $\mathcal{D}$, the reward model is trained by minimizing the negative log-likelihood
\begin{align}
\mathcal{L}_R(r_\phi, \mathcal{D}) &= \\ \nonumber
&-\mathbb{E}_{(x, y_w, y_l) \sim \mathcal{D}} \Big[\log \sigma \big( r_\phi(x, y_w) - r_\phi(x, y_l) \big)\Big],
\end{align}
where $\sigma$ is the logistic function and $y_w$ and $y_l$ denote the preferred and dispreferred completions among $(y_1, y_2)$.


\section{LLM Alignment as a Contract}
\label{sec:contract}

In the following, we formalize LLM alignment through the lens of contract theory \citep{RePEc:mtp:titles:0262025760,echenique2023online}, a subfield of economics that studies how agreements are designed under conditions of incomplete information.
%In addition to the technological approach, the challenge of aligning artificial intelligence with human values can be effectively analyzed through the lens of contract theory \citep{hadfieldmanell2019}. 
We describe human-LLM interactions as a \textit{principal-agent} relationship, where a \textit{principal} (e.g., the user, system designer, or a company) seeks to incentivize an \textit{agent} (an LLM) to act in a desired manner \citep{garen1994} (see \Cref{fig:intro}). 
This framework provides a way to conceptualize how the principal tries to align the agent's behavior with their objectives, using the agent's action and its reward function as a \textit{contract}. In this section, we explore the contract formalization (\Cref{sec:contract-formal}) and how the incompleteness of this contract (\Cref{sec:contract-incomplete}) directly leads to misalignment (\Cref{sec:contract-misalignment}) in the context of LLM alignment.\looseness=-1

\subsection{Contract Formalization}
\label{sec:contract-formal}
Following \citet{echenique2023online}, we define a contract as a pair $(a, r)$, where $a \in \mathcal{A}$  represents an action of an agent and $r:(\mathcal{X} \times \mathcal{Y}) \to \mathbb{R}$ is a reward function.\footnote{Here we loosely refer $a$ to mean one action or a series of actions that lead to an LLM output.}
The function $r$ determines the agent's reward based on the observed input-output pair $(x, y)$. In the context of a user-LLM interaction, an input $x \in \mathcal{X}$ corresponds to a user prompt, and output $y \in \mathcal{Y}$ is the LLM-generated response. A contract might be, for instance, a positive reward if the model avoids hate speech in the output. Here, the reward function would be trained on prompt-response pairs, awarding higher scores to responses that do not contain hate speech.\looseness=-1 

The framework is initiated, for instance, when a user, acting as the principal, initiates the interaction by prompting an LLM, thus implicitly proposing a contract. 
The LLM, acting as an agent, then implicitly either accepts or rejects this contract. Rejection of the contract manifests in the LLM not converging towards the desired output, which is a generated response without hate speech.
Upon implicitly accepting the contract, the LLM conducts an action $a$, which can be viewed as a probability distribution over all possible model outputs that satisfy the contract. We note that the user does not directly observe the LLM's internal decision of its action but only the output $y$. Consequently, the agent is rewarded according to the agreed-upon reward function, $r(x,y)$, implemented as a reward signal during the training phase. The principal experiences the utility derived from the output $y$; that is, the user benefits from the generated response but also suffers if the model behaves adversarially. This is illustrated when, despite a contract penalizing hate speech, the LLM generates responses that subtly convey harmful biases.\looseness=-1

\subsection{The Challenge of Incomplete Contracting in AI}
\label{sec:contract-incomplete}

Although the specific implications of incomplete contracting for LLM alignment remain underexplored, the concept has been studied in the broader context of AI alignment \citep{hadfieldmanell2019}.
In theory, alignment between the principal and the agent theoretically requires a \textit{complete contract} \citep{williamson1975markets,hadfieldmanell2019}. A complete contract would perfectly align the principal's objectives with the agent's behavior in all possible states of the world. This requires that action $a$ and reward function $r(x,y)$ be optimally defined for all input-output pairs. However, achieving complete contracts is practically infeasible for AI systems, rendering incomplete contracting unavoidable \citep{hadfieldmanell2019}. This is primarily due to the fact that machine learning systems inherently operate with underspecified objectives \citep{10.5555/3586589.3586815}, which stems from the practical difficulty in defining a reward function $r(x,y)$ that fully captures the complexities of the desired behavior.


The difficulty in specifying such a complete reward function arises from several issues.
First, a key challenge for AI alignment generally, real-world applications are too complex to generate all possibilities, hindering the specification of every possible $(a,r)$ pair \citep{openai_faulty_reward_functions}. The space of possible outcomes, denoted by $\mathcal{Y}$ in the formalization is not tractable. This mirrors the challenge of LLM in generating outputs for new input it might receive during inference. The challenge extends beyond the practical limitations of fully specifying objectives. 
Second, and particularly relevant for LLMs, even beyond these practical limitations, the challenge of translating complex human values into reward functions remains. Ambiguities and gaps in defining the desired action contribute to unintended and often undesirable outcomes.\looseness=-1 


\subsection{Misalignment due to an Incomplete Contract} 
\label{sec:contract-misalignment}

We frame LLM alignment as a challenge of incomplete contracting, which leads to misalignment. In the context of LLMs, this misalignment occurs when the reward function, $r(x,y)$ is underspecified, and thus might incentivize outputs that diverge from the users's true objectives.\looseness=-1

A common outcome of reward misspecification is \textit{reward hacking}, where an agent optimizes for the reward itself rather than the intended behavior. 
For example, LLMs may exploit gaps in the specifications, such as in the ``jailbreaking'' phenomenon. Here, carefully crafted prompts elicit harmful responses by bypassing weak guardrails because their reward function is not specific enough, allowing the model to optimize without complying with safety requirements \citep{chao2024jailbreakingblackboxlarge,zou2023universaltransferableadversarialattacks}. 
Another example of reward hacking is an LLM trained to generate ``helpful'' responses might learn to produce lengthy and verbose answers to prompts, as this might result in a higher score from the reward function even if it is not actually helpful to the user \citep{saito2023verbosity}. 
A related issue, ``fake alignment,'' occurs where the agents superficially comply with the training objective without adopting the intended internal goals \citep{greenblatt2024alignmentfakinglargelanguage}.
Another challenge is the \textit{inherent context dependence} of reward functions, which need to adapt appropriately to evolving contexts. A contract might specify desired behavior in a narrow scenario, but leave ambiguities for broader applications \citep{NIPS2017_32fdab65}. For example, a contract that stipulates ``no harmful bias'' in a model is inherently underspecified since the definitions of ``harmful'' and ``bias'' are context-dependent. 

\section{Societal Alignment Frameworks}
\label{sec:societal-frameworks}

We present societal alignment frameworks that can provide guidelines for LLM alignment in an incomplete contracting environment. 
In the following, we discuss the alignment mechanisms of social theory (\Cref{sec:social}), economic theory (\Cref{sec:economic}), and contractual theory (\Cref{sec:legal}), and explore potential solutions for improving current LLM alignment approaches.\looseness=-1

\subsection{Social Alignment}
\label{sec:social}

Human communication relies on a complex, largely implicit set of norms, values, and cues that guide individuals in interpreting each other's intentions and the world around them \citep{10.1093/acprof:oso/9780190622046.001.0001}. 
However, this process is inherently ambiguous, as much of the meaning is conveyed implicitly rather than explicitly stated. Nonetheless, humans possess a unique ability called \textit{normative competence}, which allows them to understand and judge whether certain behaviors are appropriate or inappropriate in a given context \citep{Schutz1976}. This capability is often ingrained in cultures across the world \citep{hershcovich-etal-2022-challenges, doi:10.1177/0956797615586188,10.3389/fpsyg.2018.00849}, shaping shared understanding that facilitates communication and fosters mutual understanding \citep{mercier2018enigma}. 
A similar challenge arises in user-LLM interactions, where the absence of shared norms and values can result in misaligned outputs.
For example, an LLM providing evening activity recommendations without accounting for cultural context might suggest visiting a bar or consuming alcohol in a region where such activities are prohibited or socially unacceptable, leading to responses that fail to align with local norms. 
Incorporating societal norms and values into LLMs could equip them with mechanisms to interpret and dynamically adapt to human normative systems \citep{10.5555/2447556.2447672}, much like these aid alignment within human interactions \citep{10.1093/acprof:oso/9780190622046.001.0001}.

% Normative systems
\subsubsection{Instilling norms and values} 

While norms constitute context-dependent behavioral rules that individuals follow, values represent broader ideals representing overarching goals and aspirations, shaping what individuals strive for \citep{matsumoto}. As a fundamental tool of cooperative intelligence, language plays a crucial role in expressing and reinforcing both norms and values. 
These can be instilled during LLM alignment in several ways.\looseness=-1 

LLMs, trained on vast datasets, absorb a multitude of signals about norms and values during training. However, while some attention has been given to broad ethical principles like helpfulness and harmlessness, an important aspect remains underexplored: ``contextual rules''\,---\,human norms related to cultural conventions. 
These contextual rules, while not directly influencing primary optimization objectives, are often followed due to tradition, or social norms. 
Despite their indirect nature, such rules can provide valuable signals about broader societal dynamics, thereby guiding the alignment of LLMs, as discussed by \citet{hadfieldmenell2019silly} and~\citet{koster2020silly} within the broader context of AI alignment. Although efforts such as \citet{ziems-etal-2022-moral}, \citet{zhan-etal-2024-renovi}, and \citet{chiu2024culturalteaming} introduce datasets with collections of social norms, the influence of the collected norms on improving alignment in LLMs remains underexplored \citep{aakanksha-etal-2024-multilingual}.
Contextual rules could guide the style of language to align with cultural expectations. For instance, when interacting with users from diverse cultural backgrounds, LLM could account for cultural preferences by avoiding humor that might not translate well across cultures. 
However, existing models have been shown to predominantly reflect Western values, as they have been primarily trained on Western-centric data, which limits their ability to represent multi-cultural values \citep{durmus2024towards,nayak-etal-2024-benchmarking}.\looseness=-1

Human social norms and values are continuously shaped and evaluated through daily interactions with others. These interactions involve the exchange of multimodal signals, such as language, facial expressions, and gestures \citep{doi:10.1098/rstb.2013.0302}. However, when interacting with LLMs, these cues are inherently absent, creating a normative gap in communication.
Exploring multimodality for alignment\,---\,integrating non-verbal forms of communication such as visual, or auditory signals\,---\,can serve as a promising line of research to address this normative void. By incorporating multimodal interactions, models could better align with the implicit social expectations typically conveyed through non-verbal cues.\looseness=-1

% Dynamic environment
\subsubsection{Allowing for dynamic norms and values} 
Norms and values are not static objects but dynamic equilibria that evolve through ongoing social interactions \citep{doi:10.1073/pnas.1817095116}. They are continuously re-articulated and negotiated within social contexts, evolving to address new challenges and cultural shifts \citep{annurev-psych-033020-013319}. Stereotypes, as a form of social norm, are accordingly fluid, emerging and transforming over time. An example is the shifting perception of remote work. Once seen as unprofessional or less productive, it is now widely accepted in many industries. If an LLM were trained primarily on pre-COVID data, it could reinforce outdated assumptions.\looseness=-1

While model editing and continual learning have been extensively explored for updating factual knowledge in LLMs \citep{pmlr-v162-mitchell22a,pmlr-v199-prado22a}, their application for adapting to evolving societal values and norms remains underexplored. Developing approaches to enable LLMs to dynamically identify, adapt to, and mitigate emerging biases dynamically is a crucial area for future research. Notably, even factual updates pose significant challenges, as highlighted by recent work on knowledge editing \citep{cooper2024machine, hase2024fundamental}.

\subsection{Economic Alignment}
\label{sec:economic}

Economic systems rely on specialization and the division of labor, requiring coordination among groups of people to ensure efficient allocation of resources \citep{arrow1951extension}. A central challenge in modern economic theory is aligning individual actors' interests with collective objectives \citep{hadfieldmanell2019}. Welfare economics provides a complementary perspective by formalizing optimization functions for resource allocation to maximize overall system objectives under given constraints. 
Similarly, aligning LLMs with diverse human values involves navigating trade-offs between individual and collective goals. Additionally, a coherent social welfare objective function for LLMs cannot rely solely on subjective values. Instead, real-world implementations demand collective decisions about which values to prioritize \citep{arrow1951extension,18108}. Building from this, we explore strategies for integrating economic alignment frameworks to coordinate individual preferences to achieve collective, fair objectives, and facilitating group-level aggregation, offering an alternative view to imposing monolithic objective functions across diverse user groups.\looseness=-1

\subsubsection{Economic Mechanisms for Fair Alignment} 


% Pareto-efficiency
In theoretical economics, perfect markets are often posited as achieving a Pareto-efficient distribution of welfare under a utilitarian framework \citep{arrow1951extension}. Pareto efficiency refers to a state, where no individual can be made better off without making someone worse off, and is a benchmark for efficient resource allocation \citep{black2017dictionary}. As shown in \citet{boldi2024paretooptimallearningpreferenceshidden}, Pareto efficiency offers a valuable lens for balancing competing human preferences and can serve as a foundation for techniques optimizing specific notions of group fairness, ensuring inclusive and equitable LLM alignment. Achieving such efficiency would mean tailoring the model's behavior to address diverse needs equitably, ensuring no group is disproportionately advantaged or disadvantaged without justification. 
This problem has been investigated in the field of social welfare economics, where the aggregation of diverse preferences must be balanced to ensure the collective well-being of multiple groups \citep{DASPREMONT2002459}.
For LLM alignment, these objective functions can guide the development of reward systems. As shown in general RLHF, developing welfare-centric objectives can improve fairness objectives \citep{pardeshi2024learning,cousins2024welfare}.

\subsubsection{Economic Mechanisms for Pluralistic Alignment} 

% Pluralistic alignment
Decision-making often involves multiple actors with diverse and sometimes conflicting preferences. In the context of LLMs, this necessitates approaches that account for a broad range of values. Pluralistic alignment addresses this challenge by designing models that can represent and respect diverse perspectives \citep{SorensenMFGMRYJ24,tanmay2023probingmoraldevelopmentlarge}. Unlike monolithic approaches, which attempt to impose a singular objective function, pluralistic alignment embraces the complexity of modern societies.

% Social welfare functions (ordinal and cardinal)
A critical aspect of LLM alignment involves determining how to elicit and aggregate preferences when multiple humans are affected by the behavior of an artificial agent \citep{rossi2011preferences,rao-etal-2023-ethical,pmlr-v235-conitzer24a}. This challenge extends beyond individual alignment to group alignment, where many societal issues arise from collective behavior rather than isolated actions and can be addressed by incorporating multiple objectives into the alignment process, leveraging methods such as few-shot learning to capture diverse perspectives \citep{zhou-etal-2024-beyond,zhao2024group}. 

Another critical issue in enabling pluralistic values is the trade-off between developing general-purpose models and specialized models. While specialized models tailored to specific domains, such as healthcare or justice, can better align with local norms and regulatory frameworks, they risk fragmenting values. Conversely, general-purpose models may provide broader applicability but struggle to adapt to ethically complex, domain-specific requirements. Cooperative game theory offers a framework to navigate these tensions by promoting fair resource allocation, fostering collaboration among stakeholders, and ensuring equitable outcomes \citep{10.5555/2132771}.



\subsection{Contractual Alignment}
\label{sec:legal}

% Add external and internal distinction

Law-making and legal interpretation serve as mechanisms to translate opaque human goals and values into explicit, actionable directives. Legal scholars have long recognized the inherent impossibility of drafting complete contracts \citep{macneil1977contracts,williamson1975markets,shavell1980damage,maskin1999unforseen,tirole1999incomplete,aghion2011incomplete}. This limitation stems from several key challenges. First, certain states of the world are either unobservable or unverifiable, e.g., hiding assets in complex financial arrangements can be difficult for tax authorities to identify \citep{69c1e19e-b0f8-307a-abda-571627b432cb}. Second, the limited rationality of humans restricts their ability to anticipate and optimize across the entire, combinatorially large space of potential scenarios \citep{williamson1975markets}. Consequently, precisely computing optimal outcomes becomes intractable. Furthermore, the very description of all possible contingencies is often beyond human foresight, leading to loopholes in the design of rules \citep{183a5147-673b-36d2-8ddd-a7835149772a}. Even if feasible, the costs associated with drafting and enforcing fully specified contracts would likely be prohibitive. 
Given that these challenges are analogous to those encountered in aligning LLMs, where developers aim to ensure that models produce safe and correct outputs even for inputs not directly represented in training or alignment data, we investigate insights from contract theory as potential solutions for improving LLM alignment.


% Formalization of contracts for clarification of the anticipated behaviour
\subsubsection{External Contractual Alignment}
The formalization of contracts offers a framework for anticipating and specifying desired behaviors in human-LLM interactions \citep{jacovi2021formalizing}. 
In this context, standardized documentation plays a crucial role in defining and communicating the LLMs' performance characteristics. Initiatives such as datasheets \citep{10.1145/3458723}, data statements \citep{10.1162/tacl_a_00041}, model cards \citep{10.1145/3287560.3287596}, reproducibility checklists \citep{pineau2020checklist}, fairness checklists \citep{10.1145/3313831.3376445}, and factsheets \citep{8843893} exemplify efforts to create clear, standardized guidelines that could inform the development of future regulations and legal frameworks for LLM alignment and data governance.

The rules that guide LLM alignment are currently largely constructed in consultation with domain and legal experts, by adapting documents such as the UN Declaration of Human Rights \citep{anthropic2023claude}, through public input \citep{anthropic2023collective}, or in some cases, relying on designer instincts \citep{anthropic2023claude,10.5555/3540261.3540709}. 
Importantly, the European Commission has developed detailed guidelines for trustworthy AI, which provide a structured approach to ensuring that AI systems, including LLMs, adhere to ethical principles and societal norms.\footnote{The guidelines are available at \url{https://ec.europa.eu/digital-single-market/en/news/ethics-guidelines-trustworthy-ai/}.} These documents serve as critical tools for defining the terms of human-LLM contracts and offer a principled way to ensure that the view not only reflects the developer's personal views.


\subsubsection{Internal Contractual Alignment}
While the above discussion focused on aligning LLMs through external rules, another approach takes inspiration from how parties in a contract, laws, and democratic institutions enforce principles. Instead of relying solely on external oversight, this approach embeds normative principles directly within the model's internal mechanisms. Known as \emph{constitutional AI}, this method enables LLMs to develop an internalized set of ``principles'' that guide the model to self-critique and re-write the response to ensure alignment with predefined norms. By integrating desired rules into the training objectives, constitutional AI aims to instill structural governance within models, much like how legal frameworks encode societal values into enforceable policies.
These methods provide scalable oversight precisely because they move beyond the need for direct, case-by-case human intervention. Traditional preference-based training methods, such as collecting annotations on preferred and rejected outputs, aggregate multiple annotators' judgments into a shared standard, but they still require extensive human effort at scale 
\citep{shen2023largelanguagemodelalignment,amodei2016concreteproblemsaisafety}. 
In contrast, scalable oversight techniques generalize beyond individual preferences by structuring decision-making mechanisms, similar to how democratic systems use institutionalized processes to apply laws across diverse contexts \citep{shen2023largelanguagemodelalignment}. \looseness=-1

One such method, debate \citep{irving2018aisafetydebate,irving2019ai}, mirrors adversarial legal reasoning: agents (i.e., LLMs) propose answers, engage in structured argumentation, and refine their positions, with a human judge selecting the best-supported response \citep{HAFNER20018675}. 
Similarly, constitutional AI guides LLMs using a concise constitution of high-level principles (e.g., promoting fairness or avoiding harm) \citep{bai2022constitutionalaiharmlessnessai,10.5555/3666122.3666237}. This constitution provides the basis for generating synthetic comparison examples, which are then used to fine-tune the LLM's policy. While primarily developed for integrating human values, these methods have the potential to enforce norms and regulations in a structured manner, drawing parallels to how societal governance mechanisms uphold laws and ethical standards.\looseness=-1






%

%I have left the references below + small para below in case they can be re-integrated?
% \citep{jacovi2021formalizing}. 

%Moreover,  relying on language alone to convey nuanced human expectations to AI systems is insufficient \citep{Bisconti2021}. Humans must be able to express these expectations unambiguously, which itself is a major challenge, and even if humans were able to, the AI system needs to interpret them correctly. This underscores the need for more robust and adaptable mechanisms for understanding and internalizing human intentions that move beyond linguistic specifications.


% Structure of Arugments:
% 1. AI (LLM) alignment is an incomplete contracting problem.
% 2. Three other fields have dealt with incomplete contracting problems: (i) Legal Theory; (ii) Economic Theory; and (iii) Social Theory.
% 3. For each field:
%    - Show how the incomplete contracting problem arises in {Legal, Economic, Social} theory.
%    - Describe how {Legal, Economic, Social} theory has dealt with the incomplete contracting problem.
%    - Describe what findings AI (LLM) alignment researchers should use from how {Legal, Economic, Social} theory dealt with incomplete contracting problems.
% 4. [Maybe not Applicable] Describe how, in general, the incomplete contracting problem has been dealt with in {Legal, Economic, Social} theory using a *dynamic* approach.

\section{Societal Alignment Frameworks and their View on Uncertainty}
\label{sec:uncertainty}


% Uncertainty coming from incomplete contracting (already above)
%The incomplete contracts in LLM alignment present a source of uncertainty in these models. 
%However, u
By framing LLM alignment as a problem of contractual incompleteness and analyzing it through the lens of societal alignment frameworks, we observe that these frameworks recognize establishing contracts, much like alignment, as inherently uncertain \citep{seita1984uncertainty}. 
In the following, we examine uncertainty in the specific case of LLM alignment through the lens of societal alignment frameworks. First, we analyze the sources of unwanted uncertainty in LLM alignment (\Cref{sec:uncertainty-alignment}). Next, we explore types of uncertainty that are essential for ethical alignment (\Cref{sec:uncertainty-values}). Finally, we highlight the need for reliable uncertainty communication in LLM alignment (\Cref{sec:uncertainty-communication}).


% Uncertainty of LLMs 
\subsection{Unwanted Uncertainty in LLM Alignment} 
\label{sec:uncertainty-alignment}

Prior research has identified epistemic uncertainty as one of the main challenges in LLM development \citep{shorinwa2024surveyuncertaintyquantificationlarge}. This form of uncertainty arises from gaps in the model's knowledge, leading to uncertainty about factual information \citep{shorinwa2024surveyuncertaintyquantificationlarge,jiang-etal-2021-know,yadkori2024believebelievellm}. Even aligned models remain susceptible to epistemic uncertainty, often failing to recognize their own knowledge limitations. For example, \citet{shorinwa2024surveyuncertaintyquantificationlarge} illustrate how an LLM confidently responded to the question, ``What is the lowest-ever temperature recorded in Antarctica?'' with incorrect information ($-\text{128.6}^\circ$F/$-\text{89.2}^\circ$C instead of the actual record of $-\text{135.8}^\circ$F/$-\text{94.7}^\circ$C), claiming 100\% confidence despite factual inaccuracy. This inability to calibrate confidence scores to actual knowledge reflects a fundamental limitation of current LLM architectures.\looseness=-1 
 

However, uncertainty in aligned LLMs presents additional complexity. The conversational nature of LLMs often creates an illusion of omniscience, making it difficult for users to discern the model's uncertainty \citep{delacroix2024lost}. 
Furthermore, human interaction with models, combined with their in-context learning capabilities \citep{NEURIPS2020_1457c0d6}, allows users to provide task-specific context that can inadvertently bypass safety guardrails and mitigations implemented during training. As highlighted by \citet{glukhov2024breachthousandleaksunsafe}, this can lead to models leaking unsafe information or performing harmful actions despite their intended safeguards.


% Uncertainty and Bias

\subsection{Uncertainty Needed in LLM Alignment} 
\label{sec:uncertainty-values}

While the unwanted epistemic uncertainty can undermine the reliability of language models, certain types of uncertainty are not only unavoidable but essential for their ethical deployment \citep{delacroix2024lost}. In the context of LLMs, this essential uncertainty can arise from evolving human values, conflicting societal norms, and the difficulty of translating abstract principles into model behavior. \looseness=-1

Aligning models to navigate trade-offs, such as between helpfulness and harmlessness or accuracy and fairness, requires addressing conflicting and often underspecified priorities, which introduces another source of uncertainty \citep{zollo2024prompt, yaghini2023learning}. For instance, when deploying an LLM, we often want to maximize performance subject to some constraints or guardrails on behavior, e.g., a chatbot should give users their desired output, as long as it is not too toxic. The effectiveness of balancing these conflicting priorities and the unintended consequences are often difficult to predict. However, this balancing act is also essential because it allows models to operate within complex, context-dependent environments where rigid adherence to a single objective could lead to harmful outcomes. 


% Communicating Uncertainty
\subsection{Uncertainty Communication} 
\label{sec:uncertainty-communication}

Building on the above, the inherent uncertainty in LLM alignment is not a weakness but often a valuable feature that enables models to handle complex scenarios ethically \citep{delacroix2024lost}. In fact, as highlighted by \citet{10.1145/3461702.3462571}, uncertainty communication can be useful for obtaining fairer models by revealing data biases, improving decision-making by guiding reliance on predictions, and building trust in automated systems.
Therefore, it is essential to develop methods for communicating uncertainty to users. Unlike humans, however, LLMs lack the non-verbal and contextual cues that naturally support communication \citep{Bisconti2021}. Existing research has shown that LLMs struggle to convey their uncertainty to users, both implicitly (e.g., hedging language) and explicitly (e.g., confidence scores), a skill that humans possess intuitively \citep{Alkaissi2023ArtificialHI, liu2024trustworthyllmssurveyguideline,shorinwa2024surveyuncertaintyquantificationlarge}. 
On the other hand, humans themselves have varying levels of understanding regarding probability and statistics, which are needed to interpret model uncertainty estimates \citep{10.1145/3461702.3462571,10.1001/archinternmed.2009.481}. Furthermore, human cognition is subject to biases that can impede accurate interpretation of uncertainty \citep{Kahneman,REYNA200889}. These challenges can be partially addressed by choosing the appropriate communication methods, a key consideration for the design of effective user interfaces \citep{8457476}, and by designing collaborative interaction environments, as discussed by \citet{Montemayor2021}.



\section{Alternative View: The Democratic Opportunity Inherent in the Under-specified Nature of LLMs' Objectives}
\label{sec:alternative}

The challenge of aligning LLMs is often framed as a technical problem, one that can be solved through better reward modeling, training objectives, or oversight mechanisms. However, alignment is not merely a technological issue. It is fundamentally a societal one.

To understand the significance of this alternative view, one needs to take a step back and start from the following: we humans are constantly in the process of finding our way around the world. Part of that process involves imagining better ways of living together. We may find some of our practices to be inadequate, for instance, but may not always be able to articulate why. In such cases, we often resort to conversations, to refine our intuitions and distill their underlying structures. These evolving dialogues shape and refine our moral and social expectations, which, in turn, influence the values that guide our decision-making. The fact that these values change and often clash, is a good sign---a sign of ongoing critical engagement and willingness to question existing norms. 

Now consider a team of engineers considering how to design AI tools that will be deployed within contexts such as education, healthcare, or justice practices. Some of these tools, like LLMs, can be used as conversational partners. The feedback given as a context can be leveraged to refine LLMs' behavior. Given the inherently dynamic nature of the values that inform education, healthcare, or justice practices, as we previously discussed, the key problem is to establish how to structure this feedback process. Different groups of users will evolve different values over time. Are there ways of incentivizing collective, critical engagement with LLMs? Can bottom-up, iterative refinements be configured to support users' in defining the very values that preside over their practices \citep{delacroix2024lost}?

The contrast between the above and a characterization of the AI alignment problem as `fundamentally a challenge of incomplete contracting' is significant. The contract metaphor, as discussed by \citet{goldoni2018}, oversimplifies complex systems by framing alignment as a straightforward agreement between stakeholders, neglecting the broader socio-political forces, conflicting norms, and inherent tensions that shape such systems. This technology-centric framing risks oversimplifying the dynamic and pluralistic nature of alignment challenges. While societal alignment frameworks aim to address these issues, they too often rely on oversimplified assumptions. Beyond the issues with the contract metaphor, the focus on incompleteness (i.e., information asymmetries between the principal/agent) frames alignment as an epistemic designer-centric issue, rather than recognizing it first and foremost as a political question \citep{terzis2024}.
%Aside from the problematic contract metaphor \citep{goldoni2018}, the focus on information asymmetries gives a designer-centric, epistemic spin to what is first and foremost a political question \citep{terzis2024}. 
Given LLMs' unavoidable, normative effect on the practices within which they are deployed, the under-specified nature of LLMs' objectives presents an opportunity\,---\,not to perfect our specification methods, but to democratize the very process of determining what LLMs should optimize for.

The implications of this reframing extend to both research and practice. It suggests that alongside technical work such as reward modeling, we need equally sophisticated work on participatory interface designs. This dual focus acknowledges that effective participation requires not just theoretical frameworks for inclusion, but also concrete mechanisms through which diverse stakeholders can meaningfully shape LLM development \citep{kirk2024}. This might include developing new methodologies for collective value articulation \citep{Bergman2024}, creating institutional structures for meaningful public participation in LLM development, and establishing mechanisms for ongoing societal oversight and input into LLMs' objectives and constraints.

%The view of incomplete contracting as an inherent challenge of LLM alignment that we have so far presented in this paper takes inspiration from economic theory. In this perspective, the model developer is the responsible agent navigating the constraints of an incomplete contract and uncertainty. The developer's task is to anticipate diverse moral contexts and enable systems to adapt to them while demonstrating accountability for the outputs generated by the model. 




\section{Conclusion}
We present live monitoring and mid-run interventions for multi-agent systems. We demonstrate that monitors based on simple statistical measures can effectively predict future agent failures, and these failures can be prevented by restarting the communication channel. Experiments across multiple environments and models show consistent gains of up to 17.4\%-20\% in system performance, with an addition in inference-time compute.
Our work also introduces \ourenv{}, a new environment for studying multi-agent cooperation.


\section*{Impact Statement}

This paper highlights the importance of further research and collaboration among experts in law, economics, social sciences, and LLM developers.
We believe the approach proposed in this paper would result in systems that better align with societal values by incorporating diverse perspectives into their design and oversight. In this sense, it holds positive societal impacts. 
To underscore our position that LLM alignment decisions should not be made exclusively by system designers, we discuss in \Cref{sec:alternative} the necessity of examining who is responsible for these decisions and exploring approaches to create more participatory alignment frameworks.

% Talk about the impossibility of adopting all the recommendations from societal frameworks, say that they do not 

% Acknowledgements should only appear in the accepted version.
\section*{Acknowledgements}
This paper originated from the Bellairs Invitational Workshop on Contemporary, Foreseeable, and Catastrophic Risks of Large Language Models in April 2024. We thank all workshop participants for their valuable discussions and contributions.

% In the unusual situation where you want a paper to appear in the
% references without citing it in the main text, use \nocite

\bibliography{custom}
\bibliographystyle{icml2025}


%%%%%%%%%%%%%%%%%%%%%%%%%%%%%%%%%%%%%%%%%%%%%%%%%%%%%%%%%%%%%%%%%%%%%%%%%%%%%%%
%%%%%%%%%%%%%%%%%%%%%%%%%%%%%%%%%%%%%%%%%%%%%%%%%%%%%%%%%%%%%%%%%%%%%%%%%%%%%%%
% APPENDIX
%%%%%%%%%%%%%%%%%%%%%%%%%%%%%%%%%%%%%%%%%%%%%%%%%%%%%%%%%%%%%%%%%%%%%%%%%%%%%%%
%%%%%%%%%%%%%%%%%%%%%%%%%%%%%%%%%%%%%%%%%%%%%%%%%%%%%%%%%%%%%%%%%%%%%%%%%%%%%%%
\newpage
\appendix
\onecolumn
%%%%%%%%%%%%%%%%%%%%%%%%%%%%%%%%%%%%%%%%%%%%%%%%%%%%%%%%%%%%%%%%%%%%%%%%%%%%%%%
%%%%%%%%%%%%%%%%%%%%%%%%%%%%%%%%%%%%%%%%%%%%%%%%%%%%%%%%%%%%%%%%%%%%%%%%%%%%%%%


\end{document}


% This document was modified from the file originally made available by
% Pat Langley and Andrea Danyluk for ICML-2K. This version was created
% by Iain Murray in 2018, and modified by Alexandre Bouchard in
% 2019 and 2021 and by Csaba Szepesvari, Gang Niu and Sivan Sabato in 2022.
% Modified again in 2023 and 2024 by Sivan Sabato and Jonathan Scarlett.
% Previous contributors include Dan Roy, Lise Getoor and Tobias
% Scheffer, which was slightly modified from the 2010 version by
% Thorsten Joachims & Johannes Fuernkranz, slightly modified from the
% 2009 version by Kiri Wagstaff and Sam Roweis's 2008 version, which is
% slightly modified from Prasad Tadepalli's 2007 version which is a
% lightly changed version of the previous year's version by Andrew
% Moore, which was in turn edited from those of Kristian Kersting and
% Codrina Lauth. Alex Smola contributed to the algorithmic style files.
