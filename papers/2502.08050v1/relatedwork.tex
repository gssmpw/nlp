\section{Background and Related Work}
% \label{sec:relatedwork}

% %SLR update definition
% An SLR update is a more recent (updated) version of an SLR that includes new evidence (primary studies) \cite{Mendes2020}. For the inclusion of new and relevant evidence, one of the initial steps is to conduct the study selection activity which consists of analyzing the retrieved studies from the search process to evaluate the need and to perform the SLR update. 

% %related work
% As mentioned before, there are several initiatives in SE towards improvements for SLR update (e.g. \cite{felizardo16, Garces17, Mendes2020, Wohlin2020}). However, considering the focus of our study on automation to select studies for SLR updates, we highlighted three main related works \cite{Watanabe20, Felizardo14, Napoleao2021} described in the following.

% The work of Watanabe \textit{et al.} \cite{Watanabe20} also evaluated the use of text classification (text mining combined with ML Models) to support the study selection activity for SLR updates in SE. They performed an evaluation with 8 SLRs from different research domains in a cross-validation procedure using Decision Tree (DT) and SVM as ML classification algorithms. The results achieved on average a \textit{F-score} of 0.92, \textit{Recall} of 0.93 and \textit{Precision} of 0.92. Unlike the approach proposed in \cite{Watanabe20}, our study evaluates the ML Models SVM and RF using a detailed database of a solid ongoing SLR update conducted by renowned researchers in the field of EBSE. Furthermore, we compare the agreement level of the adopted ML Models with the expert reviewers through \textit{Kappa} analysis \cite{Cohen10, Kitchenham15}.

% %While the contributions of Watanabe et al. \cite{Watanabe20} laid a valuable foundation by showing the potential of supervised ML Models in assisting the study selection process for SLR updates, our study takes a different approach to enhance the applicability of such methodologies. Our study used a detailed database of a solid ongoing SLR update conducted by renowned researchers in the field of EBSE, we  also used a distinct approach to retrieve the non-selected studies from the original SLR, and we explored different Feature Selection (FS) strategies, ML Models and techniques to train our classifiers. Furthermore, our study's contribution goes beyond evaluating how much effort could be reduced by the ML classifiers during the selection of studies process, we also  compared in detail the agreement level of the adopted ML classifiers with the expert reviewers through \textit{kappa} analysis \cite{Kitchenham15}.

% Felizardo \textit{et al.} \cite{Felizardo14} also proposes an automated alternative to support the selection of studies for SLR updates. The authors proposes a tool called Revis which links new evidence with the original SLR's evidence using the K-Nearest Neighbor (KNN) Edges Connection technique. The tool output is presented in two distinct visualizations, a content map and an Edge Bundles diagram. The results showed an increase in the number of studies correctly included compared to the traditional manual approach.

% Napoleão et al. \cite{Napoleao2021} performed a cross-domain Systematic Mapping (SM) on existing automated support for searching and selecting studies for SLRs and SMs in SE and Medicine. The authors indicated potential ML Models that can be adopted to support the study selection activities. They also indicated the most adopted methods (cross-validation and experiment) and metrics (\textit{Recall}, \textit{Precision} and \textit{F-measure}) to assess text classification approaches. The choice of the ML Models and the assessment metrics and methods for this study are considered finds of this work.


%