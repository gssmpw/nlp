
%% bare_conf.tex
%% V1.4b
%% 2015/08/26
%% by Michael Shell
%% See:
%% http://www.michaelshell.org/
%% for current contact information.
%%
%% This is a skeleton file demonstrating the use of IEEEtran.cls
%% (requires IEEEtran.cls version 1.8b or later) with an IEEE
%% conference paper.
%%
%% Support sites:
%% http://www.michaelshell.org/tex/ieeetran/
%% http://www.ctan.org/pkg/ieeetran
%% and
%% http://www.ieee.org/

%%*************************************************************************
%% Legal Notice:
%% This code is offered as-is without any warranty either expressed or
%% implied; without even the implied warranty of MERCHANTABILITY or
%% FITNESS FOR A PARTICULAR PURPOSE! 
%% User assumes all risk.
%% In no event shall the IEEE or any contributor to this code be liable for
%% any damages or losses, including, but not limited to, incidental,
%% consequential, or any other damages, resulting from the use or misuse
%% of any information contained here.
%%
%% All comments are the opinions of their respective authors and are not
%% necessarily endorsed by the IEEE.
%%
%% This work is distributed under the LaTeX Project Public License (LPPL)
%% ( http://www.latex-project.org/ ) version 1.3, and may be freely used,
%% distributed and modified. A copy of the LPPL, version 1.3, is included
%% in the base LaTeX documentation of all distributions of LaTeX released
%% 2003/12/01 or later.
%% Retain all contribution notices and credits.
%% ** Modified files should be clearly indicated as such, including  **
%% ** renaming them and changing author support contact information. **
%%*************************************************************************


% *** Authors should verify (and, if needed, correct) their LaTeX system  ***
% *** with the testflow diagnostic prior to trusting their LaTeX platform ***
% *** with production work. The IEEE's font choices and paper sizes can   ***
% *** trigger bugs that do not appear when using other class files.       ***                          ***
% The testflow support page is at:
% http://www.michaelshell.org/tex/testflow/



\documentclass[conference]{IEEEtran}
\usepackage{array}
\usepackage{booktabs} % For formal tables
\usepackage{multirow}
\usepackage{lipsum}
\usepackage{algorithm}
\usepackage{color, soul}
\usepackage{amsmath}
\usepackage{graphicx}
\usepackage{mdframed}
\usepackage{url}
\usepackage{cite}
%\usepackage{appendix}
\usepackage[noend]{algpseudocode}
\usepackage[utf8]{inputenc}
\usepackage{comment}
\usepackage{hyperref}

% Some Computer Society conferences also require the compsoc mode option,
% but others use the standard conference format.
%
% If IEEEtran.cls has not been installed into the LaTeX system files,
% manually specify the path to it like:
% \documentclass[conference]{../sty/IEEEtran}
\newcolumntype{P}[1]{>{\centering\arraybackslash}m{#1}}
\newcommand{\specialcell}[2][c]{%
  \begin{tabular}[#1]{@{}c@{}}#2\end{tabular}}

% Some Computer Society conferences also require the compsoc mode option,
% but others use the standard conference format.
%
% If IEEEtran.cls has not been installed into the LaTeX system files,
% manually specify the path to it like:
% \documentclass[conference]{../sty/IEEEtran}





% Some very useful LaTeX packages include:
% (uncomment the ones you want to load)


% *** MISC UTILITY PACKAGES ***
%
%\usepackage{ifpdf}
% Heiko Oberdiek's ifpdf.sty is very useful if you need conditional
% compilation based on whether the output is pdf or dvi.
% usage:
% \ifpdf
%   % pdf code
% \else
%   % dvi code
% \fi
% The latest version of ifpdf.sty can be obtained from:
% http://www.ctan.org/pkg/ifpdf
% Also, note that IEEEtran.cls V1.7 and later provides a builtin
% \ifCLASSINFOpdf conditional that works the same way.
% When switching from latex to pdflatex and vice-versa, the compiler may
% have to be run twice to clear warning/error messages.






% *** CITATION PACKAGES ***
%
%\usepackage{cite}
% cite.sty was written by Donald Arseneau
% V1.6 and later of IEEEtran pre-defines the format of the cite.sty package
% \cite{} output to follow that of the IEEE. Loading the cite package will
% result in citation numbers being automatically sorted and properly
% "compressed/ranged". e.g., [1], [9], [2], [7], [5], [6] without using
% cite.sty will become [1], [2], [5]--[7], [9] using cite.sty. cite.sty's
% \cite will automatically add leading space, if needed. Use cite.sty's
% noadjust option (cite.sty V3.8 and later) if you want to turn this off
% such as if a citation ever needs to be enclosed in parenthesis.
% cite.sty is already installed on most LaTeX systems. Be sure and use
% version 5.0 (2009-03-20) and later if using hyperref.sty.
% The latest version can be obtained at:
% http://www.ctan.org/pkg/cite
% The documentation is contained in the cite.sty file itself.






% *** GRAPHICS RELATED PACKAGES ***
%
\ifCLASSINFOpdf
  % \usepackage[pdftex]{graphicx}
  % declare the path(s) where your graphic files are
  % \graphicspath{{../pdf/}{../jpeg/}}
  % and their extensions so you won't have to specify these with
  % every instance of \includegraphics
  % \DeclareGraphicsExtensions{.pdf,.jpeg,.png}
\else
  % or other class option (dvipsone, dvipdf, if not using dvips). graphicx
  % will default to the driver specified in the system graphics.cfg if no
  % driver is specified.
  % \usepackage[dvips]{graphicx}
  % declare the path(s) where your graphic files are
  % \graphicspath{{../eps/}}
  % and their extensions so you won't have to specify these with
  % every instance of \includegraphics
  % \DeclareGraphicsExtensions{.eps}
\fi
% graphicx was written by David Carlisle and Sebastian Rahtz. It is
% required if you want graphics, photos, etc. graphicx.sty is already
% installed on most LaTeX systems. The latest version and documentation
% can be obtained at: 
% http://www.ctan.org/pkg/graphicx
% Another good source of documentation is "Using Imported Graphics in
% LaTeX2e" by Keith Reckdahl which can be found at:
% http://www.ctan.org/pkg/epslatex
%
% latex, and pdflatex in dvi mode, support graphics in encapsulated
% postscript (.eps) format. pdflatex in pdf mode supports graphics
% in .pdf, .jpeg, .png and .mps (metapost) formats. Users should ensure
% that all non-photo figures use a vector format (.eps, .pdf, .mps) and
% not a bitmapped formats (.jpeg, .png). The IEEE frowns on bitmapped formats
% which can result in "jaggedy"/blurry rendering of lines and letters as
% well as large increases in file sizes.
%
% You can find documentation about the pdfTeX application at:
% http://www.tug.org/applications/pdftex





% *** MATH PACKAGES ***
%
%\usepackage{amsmath}
% A popular package from the American Mathematical Society that provides
% many useful and powerful commands for dealing with mathematics.
%
% Note that the amsmath package sets \interdisplaylinepenalty to 10000
% thus preventing page breaks from occurring within multiline equations. Use:
%\interdisplaylinepenalty=2500
% after loading amsmath to restore such page breaks as IEEEtran.cls normally
% does. amsmath.sty is already installed on most LaTeX systems. The latest
% version and documentation can be obtained at:
% http://www.ctan.org/pkg/amsmath





% *** SPECIALIZED LIST PACKAGES ***
%
%\usepackage{algorithmic}
% algorithmic.sty was written by Peter Williams and Rogerio Brito.
% This package provides an algorithmic environment fo describing algorithms.
% You can use the algorithmic environment in-text or within a figure
% environment to provide for a floating algorithm. Do NOT use the algorithm
% floating environment provided by algorithm.sty (by the same authors) or
% algorithm2e.sty (by Christophe Fiorio) as the IEEE does not use dedicated
% algorithm float types and packages that provide these will not provide
% correct IEEE style captions. The latest version and documentation of
% algorithmic.sty can be obtained at:
% http://www.ctan.org/pkg/algorithms
% Also of interest may be the (relatively newer and more customizable)
% algorithmicx.sty package by Szasz Janos:
% http://www.ctan.org/pkg/algorithmicx




% *** ALIGNMENT PACKAGES ***
%
%\usepackage{array}
% Frank Mittelbach's and David Carlisle's array.sty patches and improves
% the standard LaTeX2e array and tabular environments to provide better
% appearance and additional user controls. As the default LaTeX2e table
% generation code is lacking to the point of almost being broken with
% respect to the quality of the end results, all users are strongly
% advised to use an enhanced (at the very least that provided by array.sty)
% set of table tools. array.sty is already installed on most systems. The
% latest version and documentation can be obtained at:
% http://www.ctan.org/pkg/array


% IEEEtran contains the IEEEeqnarray family of commands that can be used to
% generate multiline equations as well as matrices, tables, etc., of high
% quality.




% *** SUBFIGURE PACKAGES ***
%\ifCLASSOPTIONcompsoc
%  \usepackage[caption=false,font=normalsize,labelfont=sf,textfont=sf]{subfig}
%\else
%  \usepackage[caption=false,font=footnotesize]{subfig}
%\fi
% subfig.sty, written by Steven Douglas Cochran, is the modern replacement
% for subfigure.sty, the latter of which is no longer maintained and is
% incompatible with some LaTeX packages including fixltx2e. However,
% subfig.sty requires and automatically loads Axel Sommerfeldt's caption.sty
% which will override IEEEtran.cls' handling of captions and this will result
% in non-IEEE style figure/table captions. To prevent this problem, be sure
% and invoke subfig.sty's "caption=false" package option (available since
% subfig.sty version 1.3, 2005/06/28) as this is will preserve IEEEtran.cls
% handling of captions.
% Note that the Computer Society format requires a larger sans serif font
% than the serif footnote size font used in traditional IEEE formatting
% and thus the need to invoke different subfig.sty package options depending
% on whether compsoc mode has been enabled.
%
% The latest version and documentation of subfig.sty can be obtained at:
% http://www.ctan.org/pkg/subfig




% *** FLOAT PACKAGES ***
%
%\usepackage{fixltx2e}
% fixltx2e, the successor to the earlier fix2col.sty, was written by
% Frank Mittelbach and David Carlisle. This package corrects a few problems
% in the LaTeX2e kernel, the most notable of which is that in current
% LaTeX2e releases, the ordering of single and double column floats is not
% guaranteed to be preserved. Thus, an unpatched LaTeX2e can allow a
% single column figure to be placed prior to an earlier double column
% figure.
% Be aware that LaTeX2e kernels dated 2015 and later have fixltx2e.sty's
% corrections already built into the system in which case a warning will
% be issued if an attempt is made to load fixltx2e.sty as it is no longer
% needed.
% The latest version and documentation can be found at:
% http://www.ctan.org/pkg/fixltx2e


%\usepackage{stfloats}
% stfloats.sty was written by Sigitas Tolusis. This package gives LaTeX2e
% the ability to do double column floats at the bottom of the page as well
% as the top. (e.g., "\begin{figure*}[!b]" is not normally possible in
% LaTeX2e). It also provides a command:
%\fnbelowfloat
% to enable the placement of footnotes below bottom floats (the standard
% LaTeX2e kernel puts them above bottom floats). This is an invasive package
% which rewrites many portions of the LaTeX2e float routines. It may not work
% with other packages that modify the LaTeX2e float routines. The latest
% version and documentation can be obtained at:
% http://www.ctan.org/pkg/stfloats
% Do not use the stfloats baselinefloat ability as the IEEE does not allow
% \baselineskip to stretch. Authors submitting work to the IEEE should note
% that the IEEE rarely uses double column equations and that authors should try
% to avoid such use. Do not be tempted to use the cuted.sty or midfloat.sty
% packages (also by Sigitas Tolusis) as the IEEE does not format its papers in
% such ways.
% Do not attempt to use stfloats with fixltx2e as they are incompatible.
% Instead, use Morten Hogholm'a dblfloatfix which combines the features
% of both fixltx2e and stfloats:
%
% \usepackage{dblfloatfix}
% The latest version can be found at:
% http://www.ctan.org/pkg/dblfloatfix




% *** PDF, URL AND HYPERLINK PACKAGES ***
%
%\usepackage{url}
% url.sty was written by Donald Arseneau. It provides better support for
% handling and breaking URLs. url.sty is already installed on most LaTeX
% systems. The latest version and documentation can be obtained at:
% http://www.ctan.org/pkg/url
% Basically, \url{my_url_here}.




% *** Do not adjust lengths that control margins, column widths, etc. ***
% *** Do not use packages that alter fonts (such as pslatex).         ***
% There should be no need to do such things with IEEEtran.cls V1.6 and later.
% (Unless specifically asked to do so by the journal or conference you plan
% to submit to, of course. )


% correct bad hyphenation here
\hyphenation{op-tical net-works semi-conduc-tor}


\begin{document}
% %
% % paper title
% % Titles are generally capitalized except for words such as a, an, and, as,
% % at, but, by, for, in, nor, of, on, or, the, to and up, which are usually
% % not capitalized unless they are the first or last word of the title.
% % Linebreaks \\ can be used within to get better formatting as desired.
% % Do not put math or special symbols in the title.
\title{Can Machine Learning Support the Selection of Studies for Systematic Literature Review Updates?}

\author{

\IEEEauthorblockN{Marcelo Costalonga}
\IEEEauthorblockA{\textit{PUC-Rio} \\
Rio de Janeiro, Brazil \\
mcardoso@inf.puc-rio.br}
\and
\IEEEauthorblockN{Bianca Minetto Napole\~ao}
\IEEEauthorblockA{\textit{Université du Québec à Chicoutimi} \\
Chicoutimi, Canada \\
bianca.minetto-napoleao1@uqac.ca}
\and
\IEEEauthorblockN{Maria Teresa Baldassarre}
\IEEEauthorblockA{\textit{University of Bari} \\
Bari, Italy \\
mariateresa.baldassarre@uniba.it}
\and
\IEEEauthorblockN{Katia Romero Felizardo}
\IEEEauthorblockA{\textit{Universidade Tecnológica Federal do Paraná} \\
Cornélio Procópio, Brazil \\
katiascannavino@utfpr.edu.br}
\and
\IEEEauthorblockN{Igor Steinmacher}
\IEEEauthorblockA{\textit{Northern Arizona University} \\
Flagstaff, USA \\
igor.steinmacher@nau.edu}
\and
\IEEEauthorblockN{Marcos Kalinowski}
\IEEEauthorblockA{\textit{PUC-Rio} \\
Rio de Janeiro, Brazil \\
kalinowski@inf.puc-rio.br}
}
% // TODO: ajeitar emails
%\and
%\IEEEauthorblockN{Daniel Méndez Fernández}
%\IEEEauthorblockA{Software \& Systems Engineering\\
%Technical University of Munich\\
%Garching, Germany\\
%Email: daniel.mendez@tum.de}










% % conference papers do not typically use \thanks and this command
% % is locked out in conference mode. If really needed, such as for
% % the acknowledgment of grants, issue a \IEEEoverridecommandlockouts
% % after \documentclass

% % for over three affiliations, or if they all won't fit within the width
% % of the page, use this alternative format:
% % 
% %\author{\IEEEauthorblockN{Michael Shell\IEEEauthorrefmark{1},
% %Homer Simpson\IEEEauthorrefmark{2},
% %James Kirk\IEEEauthorrefmark{3}, 
% %Montgomery Scott\IEEEauthorrefmark{3} and
% %Eldon Tyrell\IEEEauthorrefmark{4}}
% %\IEEEauthorblockA{\IEEEauthorrefmark{1}School of Electrical and Computer Engineering\\
% %Georgia Institute of Technology,
% %Atlanta, Georgia 30332--0250\\ Email: see http://www.michaelshell.org/contact.html}
% %\IEEEauthorblockA{\IEEEauthorrefmark{2}Twentieth Century Fox, Springfield, USA\\
% %Email: homer@thesimpsons.com}
% %\IEEEauthorblockA{\IEEEauthorrefmark{3}Starfleet Academy, San Francisco, California 96678-2391\\
% %Telephone: (800) 555--1212, Fax: (888) 555--1212}
% %\IEEEauthorblockA{\IEEEauthorrefmark{4}Tyrell Inc., 123 Replicant Street, Los Angeles, California 90210--4321}}




% % use for special paper notices
% %\IEEEspecialpapernotice{(Invited Paper)}


% make the title area
\maketitle

% % no keywords
% \begin{IEEEkeywords}
% requirements engineering, machine learning, systematic mapping study
% \end{IEEEkeywords}



% % For peer review papers, you can put extra information on the cover
% % page as needed:
% % \ifCLASSOPTIONpeerreview
% % \begin{center} \bfseries EDICS Category: 3-BBND \end{center}
% % \fi
% %
% % For peerreview papers, this IEEEtran command inserts a page break and
% % creates the second title. It will be ignored for other modes.
% % \IEEEpeerreviewmaketitle

% 
The increasing reliance on LLMs for multimodal tasks across far-reaching sectors such as healthcare, finance, and manufacturing underscores the need to assess the accuracy and reliability of the information they generate. Vision-Language Models (VLM) have achieved state-of-the-art (SoTA) performance on Visual Question-Answering (VQA) benchmarks, and these models often utilize Retrieval-Augmented Generation (RAG) to maintain factual accuracy and relevance in a dynamic information environment. However, this has led to uncertainty in the information the LLM bases its answer on, as it may choose between parametric memory and retrieved sources. When models rely on memorized information instead of dynamically retrieving information, they may inadvertently propagate outdated or incorrect information, causing serious legal and ethical risks and undermining trust and reliability in AI systems \citep{huang2023survey}.
% The ability to strike a balance between generalization and specialization in AI systems is therefore crucial for ensuring the safe, reliable use of these technologies in real-world applications.

Despite these concerns, the way that Vision-Language models (VLMs) memorize and retrieve information, particularly in complex multimodal tasks, remains under-explored. Current research often focuses on either the general capabilities of large language models (LLMs) or the specialized retrieval mechanisms in retrieval augmented generation systems (RAG) \citep{incontext_rag,chen_murag_2022,liu_universal_2023}. Particularly in the context of multimodal retrieval and multihop reasoning, few studies analyze the tradeoff between finetuning for specialized tasks and zero-shot prompting for general-purpose vision-language capabilities. A lack of consensus on how to approach this tradeoff motivates the development of measures to quantify reliance on parametric memory, as well as metrics for quantifying the potential performance impact of extending LLMs with RAG systems.

To address this gap, we investigate how multimodal QA models balance accuracy with memorization on the WebQA benchmark. We compare finetuned multimodal systems against zero-shot VLMs, analyzing how retrieval performance influences QA accuracy. In particular, we focus on cases where retrieval fails, allowing us to measure reliance on parametric memory through two proposed metrics---the \ppr (\PPR) which quantifies how much model accuracy is influenced by retrieval quality, contrasting performance in best-case versus worst-case retrieval scenarios, and the \ucr (\UCR) which measures how often correct QA responses are generated when the retriever fails, providing a proxy for memorization.

To enable this analysis, we make several methodological contributions. For the finetuned QA models, we investigate Vision-Transformer (ViT) architectures, which allow for multihop reasoning over multiple sources. To investigate the impact of retrieval performance on trained LMs, we propose a variable-input Fusion-in-Decoder (FiD) model \cite{tanaka_slidevqa_2023, nlvr2}, building upon the VoLTA architecture \citep{pramanick_volta_2023}. For the zero-shot case, we build upon previous research on In-Context Retrieval \citep{incontext_rag} by demonstrating that LLMs such as GPT-4o are capable of performing the final ranking step of the retrieval process. In doing so, we find that GPT-4o, a general-purpose LLM, achieves SoTA performance on the WebQA task, outperforming existing finetuned RAG models by a significant margin (7\% higher accuracy). 

Crucially, our results reveal that while retrieval-augmented models reduce memorization, the training paradigm plays an important role. Finetuned models exhibit higher reliance on parametric memory, whereas zero-shot RAG approaches have lower memorization scores at the cost of accuracy. This suggests that while retrieval modules may mitigate the risks associated with outdated or incorrect information, SoTA performance requires that they be coupled with specialized QA models. Our memorization measures contribute to the development of transparent and reliable AI systems, particularly in applications where the sourcing of up-to-date, factual information is critical.



% We investigate the impact of question complexity on the ability of these models to integrate multiple data sources—such as images, text, and external retrievers—and produce coherent and accurate answers. We also explore whether in-context retrieval can be a viable alternative to traditional retrieval-augmented systems, offering a more streamlined approach to multimodal QA.

% To achieve this, we first compare zero-shot prompting multimodal LLMs with finetuned multimodal systems. We evaluate both types of models on the WebQA benchmark, a dataset designed for complex question answering that requires reasoning across both image and text sources. For the finetuned models, we use a Fusion-in-Decoder (FiD) architecture, which allows for multihop reasoning over multiple sources. Additionally, we introduce the concept of In-Context Retrieval Language Modeling (RLM), where the LLM itself performs retrieval tasks without the need for external retrievers. This method builds upon existing research in in-context learning  and aims to explore the viability of LLMs retrieving relevant sources and generating accurate answers directly from their context window.

% In order to investigate source utilization in finetuned multimodal models and LLMs, three lines of inquiry are established; 
% \begin{itemize}
%     \item Study 1: retrieval vs QA performance on webQA (motivating example, does QA answer correctly even with incorrect sources?)
%     \item Study 2: performance on adversarial examples where parametric knowledge would be incorrect by design
%     \item Study 3: improving performance on adversarial examples by fine-tuning (i.e model robustness)
% \end{itemize}

% Note, there is one weakness in this plan which is tying in the work we've already done. 
% If we added something from adversarial generation to the retrieval experiment (like a combination of study 1 + 3) it would be complete. So for instance we could try fine-tuning the retriever with adversarial examples (and not just the QA model)

% \begin{figure}
%     \centering
%     \includegraphics[width=0.95\linewidth]{figures/segmentation/webqa_segment_infill.png}
%     \caption{Example of the segmentation substitution pipeline from the WebQA task.}
%     % d5c76d760dba11ecb1e81171463288e9
%     \label{fig:seg_sub_pipeline}
% \end{figure}



% Retrieval augmented generation (RAG) with zero-shot prompting and fine-tuning Large Language Models (LLMs) have become the go-to methods for tasks relying on information retrieval and text generation. In many cases the LLMs parametric memory can sufficiently generalize to answer questions without being provided with retrieval mechanisms for out-of-domain knowledge. However, LLMs often hallucinate and provide wrong information in certain scenarios. This problem is amplified even further on open-domain Question Answering (QA) tasks involving multiple modalities. Grounded text generation using retrieved sources \citep{lewis2021retrievalaugmented} has been extensively studied for text-to-text QA tasks, but its application in multimodal settings has not been studied as much.


% Multimodal reasoning and question answering have gained prominence in recent research endeavors, with an increasing emphasis on handling various forms of data, particularly text and images. In this study, we address a specific gap in the existing literature by focusing on the development of a versatile multihop model capable of accommodating varying numbers of input images.

% Our motivation for this research lies in the growing complexity of answering questions using information on the web, where the challenge of navigating the open-domain setting is further complicated by the presence of multiple modalities and sometimes requires reasoning over multiple sources. WebQA is an ideal dataset on which to compare performance of finetuned RAG systems against general purpose LLMs; it is multimodal, with correct answers requiring reasoning over image and text sources. It is multihop, requiring a complex reasoning process over multiple sources. Finally, WebQA questions from different categories can be broken down into subdomains to analyze performance over domains of varying cardinality.

% Motivated by the real-world challenges of building retrieval and question answering (QA) systems, we design and finetune a closed domain, multimodal, multihop QA model, that is capable of reasoning over a varying number of sources taken as input from an external retriever module. This research contributes to the relatively underexplored domain of multihop reasoning across various input sources and modalities. Our goal is to explore the challenges posed by these scenarios and develop strategies that enable QA models to retrieve relevant information, conduct logical or numerical reasoning across diverse modalities, and generate coherent responses in natural language. To our knowledge, this is the first application of the Fusion-in-Decoder (FiD) architecture \cite{tanaka_slidevqa_2023, nlvr2} that is shown to work with a variable number of inputs, enabling multi-hop reasoning over sources.

% In-Context Learning refers to the ability of LLMs to perform any task by simply providing examples in the input prompt \citep{dong2022survey,min2022rethinking}. Inspired by this research, we propose a method to use the LLM itself as a multimodal retriever, potentially eschewing the requirement of a distinct retrieval module, thereby allowing the design of simpler retrieval-augmented QA systems. We dub this method In-Context Retrieval Language Modeling (RLM). To the best of the authors knowledge, In-Content RLM is disparate from other retrieval augmented approaches which utilize external retrieval modules \citep{incontext_rag,chen_murag_2022,liu_universal_2023}. Despite being a natural extension of In-Context learning, In-Context RLM has not yet been studied empirically.

% To expand on our contribution of In-Context Retrieval, this stems from the well-researched in-context learning of LLMs. In-context learning is the ability of a model to perform any task given a sufficient context window \citep{dong2022survey,min2022rethinking}. Such tasks could include retrieval and ranking, but typically, the go-to solution for tasks requiring retrieval has been RAG. To the best of the authors knowledge, In-Context Retrieval is distinct from In-Context Retrieval Augmented Language Modelling (RALM), and despite being a natural extension of In-Context learning, In-Context Retrieval has not yet been shown empirically.

% Finally, we explore the tradeoff between using zero-shot prompting LLMs and the fine-tuning approach. While we find that, overall, GPT-4o obtains SoTA performance on the WebQA task, outperforming the accuracy of existing finetuned RAG approaches by 7\%, finetuned approaches still perform better on more restricted subdomains\footnote{``In-Context RLM" @ \url{https://eval.ai/web/challenges/challenge-page/1255/leaderboard/3168}}. Finally, we validate that GPT-4o is relying on retrieval abilities to solve the task; we find that GPT-4o is capable of retrieving relevant sources in the presence of distractors and furthermore, when GPT-4o fails to retrieve correct sources, it answers incorrectly 75\% of the time, meaning that it is not relying on parametric memory for this task.

% \paragraph{Contributions}
% Based on our experimentation and analysis on the WebQA benchmark, we make the following contributions:
% \begin{itemize}
%     \item Propose a new architecture for multimodal multihop QA that takes variable number of input sources inspired by the Fusion-in-Decoder method.
%     \item Comparison of general purpose LLMs vs specialized models on the WebQA benchmark.
%     \item Observation of In-Context Multimodal Retrieval abilities of GPT-4o and that it does not rely on parametric memory for multimodal QA.
%     \item Analysis of relationship between retrieval and QA task performance.
%     \item Analysis of task and query complexity on the performance of retrieval and QA tasks.
% \end{itemize}
















% Throughout this paper, we will present our methodology, experiments, and findings, emphasizing our approach to multihop reasoning over varying numbers of input images. We believe that our work contributes to a deeper understanding of multimodal reasoning and has the potential to enhance the capabilities of question-answering systems in the intricate, multimodal landscape of web-based information.
% \section{Background and Motivation}
\label{sec:background}

We introduce the background on serverless workload serving and motivate the use of runtime resource adaptation to address resource inefficiency in existing serverless platforms.

\subsection{Resource Inefficiency with Early Binding}
% In current serverless platforms, developers are required to specify immutable sizes for their deployed functions.
% Then, providers consider functions' runtime workloads  (e.g., concurrency)  and resource usage to scale out/in their instances.
% Moreover, due to high runtime variability, functions must size their functions for worst-case scenarios.
% This, however, incurs considerable resource inefficiency.
Current serverless workflow platforms (e.g., AWS Step Functions~\cite{aws-step-function} and Azure Durable Functions~\cite{azure-durable-function}) offer the opportunity for developers to build various applications with advanced logic like chaining, branching, and parallel execution.
These applications can be defined by JSON-based structured languages (e.g., Amazon States Language) or other programming languages.
Meanwhile, developers require to specify resource configurations, including memory size, CPU cores, and scaling options, for individual functions---an early-binding approach.
The serverless platform is responsible for monitoring the workload intensity and resource usage at runtime and scaling out/in function instances accordingly.
To account for potential runtime variability, developers must size the functions in their application workflow accounting for the worst case in order to provide SLO guarantees over the end-to-end delay of request processing, e.g., the 99th percentile (P99) of the end-to-end delay must be within a given target. 
After deployment, the function sizes become immutable. The worst case is not representative and over-shoots most of the time, leading to resource inefficiency. 


To verify this claim, we conduct a data-driven analysis with a dataset from Microsoft Azure Functions~\cite{azure-dataset} to explicitly demonstrate the resource inefficiency issue. % , deriving from the worst-case based early bind.
To quantify the inefficiency, we define a metric called \emph{slack}---the margin between the actual execution time and the SLO, which is calculated as $1-l/T$ with $l$ and $T$ representing end-to-end latency and SLO, respectively.
Under certain SLO defined with P99 latency as done by existing works (e.g., \cite{osdi22-orion,mac22-wisefuse}),  we can see from Figure \ref{fig:bg:slack} that more than 60\% function invocations have slacks over 60\%.
Particularly, we analyze slacks of the top 100 most popular functions, whose invocations account for 81.6\% of the total function invocations. % (depicted in Figure~\ref{fig:bg:popular_func}) of overall invocations.
The result shows that only 20\% of the invocations of the popular functions (blue line in Figure~\ref{fig:bg:slack}) have slacks less than 40\%.
This means the majority of requests are processed faster than necessary.
Notably, in DAG-based workloads (i.e., Azure Durable Functions), the resource inefficiency further deteriorates wherein the ratio between the 95th percentile and 50th percentile is by up to three times \cite{mac22-wisefuse}.

% \begin{figure}[t!]
% \centering
% \includegraphics[width=0.25\textwidth]{./figure/motivation/Average_P99_cdf_top=100.pdf}
% \vspace{-0.3cm}
% \caption{Sufficient function slacks in production traces.}
% \label{fig:bg:slack}
% \end{figure}

\subsection{Runtime Dynamics}
\label{sec:bg:worst-case}

The resource inefficiency caused by the large slack can be mainly attributed to the over-provisioning of resources by the developer. This is to ensure that the SLO is guaranteed even in the worst case (i.e., P99). However, normal cases deviate from the worst case significantly due to runtime dynamics. 
In particular, we observe that functions face two major dynamic factors at runtime: varying working sets and inevitable performance interference. These two factors contribute significantly to the variance of the function execution time. 
% Functions face two remarkably dynamic factors at runtime: working sets and performance interference, which lead to considerable variance of execution latency.

\begin{figure*}[!t]
	\centering
	\subfloat[]{
		\includegraphics[width=0.24\textwidth]{./figure/motivation/Average_P99_cdf_top=100.pdf}
		\label{fig:bg:slack}
	}
	\hspace{8mm}
	\subfloat[]{
		\includegraphics[width=0.25\textwidth]{./figure/motivation/function-latency-ml-analyze-varying-worksets.pdf}
		\label{fig:bg:ml-func-latency}
	}
	\hspace{8mm}
	\subfloat[]{
	\includegraphics[width=0.28\textwidth]{./figure/motivation/coresident-perf.pdf}   
	\label{fig:bg:perf-inteference}
	}
	%\vspace{-0.1cm}
	\caption{(a) slacks of function invocations in production traces, (b) function latency variance caused by varying input worksets for functions object detection (OD), question answering (QA), and and text-to-speech (TS), respectively,
 (c) performance interference attributed to co-location of homogeneous function with different dominant resource demands.}
 %\vspace{-0.4cm}
\end{figure*}

%'ml-analyze':{'text-to-speech': 'text-to-speech', 'question-answer': 'question answer',
%                      'object-detection': 'object detection'
\textbf{\textit{Varying working sets.}} 
The working set, i.e., input data like videos, audios, and texts, can have varying sizes.
Taking Microsoft Azure Function Blobs (storage service) as an example, their data size difference can be as high as nine orders of magnitude~\cite{azure-function-blob}.
Such a large difference results in substantial variance of the execution time even for the same function~\cite{socc21-faast,eurosys21-ofc}.
Specifically, we measure the execution time of three functions under different working sets (detailed in \S\ref{exp:setup}).
Figure~\ref{fig:bg:ml-func-latency} illustrates the results, where we can observe a variance of up to 3.8 times in function execution caused by varying working set sizes.

% \begin{figure}[t!]
% \centering
% \includegraphics[width=0.25\textwidth]{././figure/motivation/function-latency-ml-analyze-varying-worksets.pdf}
% \vspace{-0.3cm}
% \caption{Function latency variance caused by varying input worksets}
% \label{fig:bg:ml-func-latency}
% \end{figure}	

\textbf{\textit{Performance interference.}}
% On the other hand, function deployment, which decides when and where to deploy functions, is completely undertaken by providers.
For simplicity and security, commercial serverless platforms, such as Alibaba Function Compute, Microsoft Azure, and AWS Lambda, exclusively deploy function instances belonging to the same tenant, or even belonging to the same function, in the same virtual machine~\cite{socc22-owl,atc18-peek-bench}.
For example, the empirical study in~\cite{socc22-owl} shows that in Alibaba Function Compute 65\% of the virtual machines exclusively deploy instances of the same function.
This co-location of homogeneous function instances, however, can incur severe resource contention on the same resource dimensions, particularly for network bandwidth and memory bandwidth of virtual machines~\cite{sc21-gsight,micro19-faaSprofiler,socc22-owl,atc18-peek-bench}.
To verify this observation, we use a virtual machine to run a function increasing the number of co-located instances from one to six while measuring the execution time of four different functions with resource dominance on different dimensions namely computing, I/O, network, and memory, respectively (detailed in \S\ref{exp:setup}). 
As shown in Figure~\ref{fig:bg:perf-inteference}, the co-location of homogeneous functions leads to substantial resource contention and performance interference, prolonging the function execution time up to 8.1 times. The performance interference is often hard to model and predict.

% this co-residency results in substantial increase of execution latency by up to 8.1 times,leading to considerable variance in function execution time.
% when compared with that with concurrency as one.

%for CPU-, IO-, network- and memory-intensive functions as the concurrency rises from one to six.
%Figure shows that significant performance interference can be observed, . 
%compared with the inclusive deployment (concurrency as one), 
% this exclusive deployment (gray bar) results in substantial increase of execution latency by up to 8.1$\times$ for CPU-, IO-, network- and memory-intensive functions as the concurrency rises from one to six.

% this exclusive deployment (gray bar) results in substantial increase of execution latency by up to 8.1$\times$ for CPU-, IO-, network- and memory-intensive functions as the concurrency rises from one to six.
% As depicted in Figure~\ref{fig:bg:concurrent_latency}, with the concurrency rising  from one to six,  the exclusive deployment results in substantial increase of execution latency by up to 8.1$\times$.
% This significantly magnifies execution latency variance.

% \begin{figure}[t!]
% \centering
% \includegraphics[width=0.25\textwidth]{./figure/motivation/coresident-perf.pdf}
% \vspace{-0.3cm}
% \caption{Performance interference attributed to co-residency of homogeneous function.}
% \label{fig:bg:perf-inteference}
% \end{figure}




\subsection{Runtime Resource Adaptation}
\label{sec:bg:adaptive-allocation}
To tackle the aforementioned resource inefficiency issue, we can adopt a late-binding approach through \emph{runtime resource adaptation}, which resizes functions on the fly based on runtime information (e.g., function slacks), achieving higher resource efficiency without violating SLO. For example, given a workflow as a chain of functions, the resource allocation of the downstream functions can be adjusted when the first function finishes execution. This way, the slack from the first function can be exploited to optimize resource efficiency. 

The idea sounds straightforward and has been considered in some existing works \cite{infocom22-stepconf,middleware20-fifer,socc21-llama,socc21-kraken,middleware20-xanadu}.
However, most of these works make an unrealistic assumption that either the developer performs the adaptation decision with access to runtime information or the serverless platform provider performs the adaptation with domain knowledge of the application workflow. These assumptions render these solutions impractical to deploy in real-world serverless systems. The information barrier between the developer and the provider calls for a new solution. 

We identify the following challenges and opportunities for a full-fledged design for runtime resource adaptation. 

\textbf{\textit{Skewed function execution time distribution.}} 
Resource allocation for a serverless workflow is typically done by leveraging performance profiles of all the functions in the workflow. 
During the offline profiling, the execution time distribution for each function is first obtained by running the function with a variety of sample inputs under different resource conditions. Then, given a time budget, existing approaches typically use P99 of the function execution time as a target and calculate the corresponding resource demands. However, due to the high runtime variability, the distribution of the function execution time is highly skewed where the difference between P50 and P99 can be as high as 100 times~\cite{socc23-huawei-cloud}. This means that if only the function execution time at a single percentile (P50 or P99) is used for resource allocation, there will be significant resource under-provisioning and over-provisioning for most requests at runtime. To address this issue, our idea is to allow for the exploration of the function execution time at diverse percentiles during resource allocation. 


% It is a prerequisite to profile execution latency for adaptive resource allocation.  
% As aforementioned, owing to a variety of unexpected runtime dynamics,  execution latency demonstrates skewed distributions, by up to 100$\times$ between 99\% percentile and 50\% percentile on Huawei cloud serverless~\cite{socc23-huawei-cloud} .
% This makes the current a single statistic (e.g., mean) or 99\% percentile distribution based profiling suffer significant under- and over-estimation.
% To fix this issue, our insight is to \textit{introduce more diverse percentiles to profile execution latency}. 

\textbf{\textit{Dependencies of adaptation decisions.}}
As the function execution progresses, a sub-workflow will be generated by removing the finished function(s) from the workflow. Within each sub-workflow, the resource adaptation decisions for remaining functions are dependent on each other due to the constraint imposed by the end-to-end latency SLO. For example, under-provisioning a function will result in a reduction of the time budget for executing its downstream functions, thus calling for more resources for these downstream functions to avoid SLO violations. Meanwhile, the selection of the percentile for the execution time of each function dictates resource-latency tradeoff for that function. For example, a higher percentile means that more resources will be allocated to ensure that more requests processed by the function will finish within the given time budget. On the contrary, a lower percentile means that more requests will risk SLO violation, but at the benefits of reduced resource consumption. To address such complex dependencies, we propose the following ideas: (1) We introduce two metrics (i.e., the timeout metric and the resilience metric detailed in \S\ref{sec:profilier}) to balance the resource adaptation decisions of the head function of the current sub-workflow and those of the remaining downstream functions. These metrics help us connect the decision making across sub-workflows and avoids sub-optimal adaptation decisions in each sub-workflow. 
(2) We explore lower percentiles for the head function and a high percentile (i.e., P99) for other functions in each sub-workflow. Using lower percentiles maximizes the opportunity for resource optimization since any over-time execution of the head function can later be compensated by resource adaptation in the next round. The high percentile ensures that the resource adaptation is not too radical to cause SLO violations. 

% Each workflow generates multiple sub-workflows as the execution moves forwards. 
% Within sub-workflows, the provisioning is inter-corrected.
% For instance, under-provisioning upstream functions may directly shrink the time budget for downstream functions, which dictates more resources required by the latter against (sub-) SLO violation. 
% This makes sub-workflows generally adopted as the basic unit to make adaptation decisions~\cite{socc21-llama,rtas22-fa2}. 
%  Moreover,  due to the high variance of execution performance, runtime adaptation requires to carry out function by function, i.e.,  discrete adaptation.
%  This, however, can easily lead to a sub-optimal (analyzed in~\S~\ref{sec:synthesizer:generate}).
% Our insight is to \emph{introduce a metric (i.e., resilience detailed in \S~\ref{sec:profilier}) to quantify the inter-correlation as well as a heuristic design (i.e., heavier head explained in \S~\ref{sec:synthesizer:generate})  to calibrate the sub-optimal,  such that resource adaptation can explore higher resource efficiency without SLO guarantee}.

% In particular, latency percentiles (introduced by the profiling)  involves resource adaptation as a new knob.
% Specifically, higher percentile earns  stronger guarantees in SLOs but may be highly prone to resource over-allocation because of its latency over-estimation, impairing resource efficiency.
% In contrast, decreasing percentiles offers the opportunity to explore higher resource efficiency, but suffers the risk of timeout, i.e., execution latency beyond specified time budget, and  may thus incur  SLO violations.
% Here, our insight is to \emph{moderately explore percentiles (detailed in~\S~\ref{sec:synthesizer:generate}), where head functions of  (sub-)workflows can explore lower percentiles because this creates the opportunity to reap higher resource efficiency while possible timeout can be recovered by subsequent functions' re-adaptive allocation.
% On the other head, non head functions maintain percentiles as 99\%}.
% This can well keep the trade-off between opportunities of exploring higher resource efficiency and risks of SLO violations. 
% Additionally, it effectively shrinks the searching space, benefiting the adaptation with higher time-efficiency.


\textbf{\textit{Tight resource adaptation window.}}
Runtime resource adaptation requires to calculate a new resource allocation decision for the remaining sub-workflow immediately when a function finishes execution. Since serverless functions are typically short-lived (less than 1s on average)~\cite{atc18-peek-bench,socc22-owl,atc20-serverless-in-the-wild,socc23-huawei-cloud}, the window for resource adaptation is quite tight. Assuming the serverless platform will perform the runtime adaptation on behalf of the developer since the platform has access to full runtime information, the resource adaptation decision making should be fast without involving complex calculations and logic or exploring a large space. As discussed before, the serverless platform provider does not have domain knowledge of the serverless workflow. Hence, the developer must pass the necessary information to the serverless platform for runtime adaptation decision making. Our idea is to let the developer synthesize critical hints containing resource allocation rules and options, which the serverless platform provider utilizes to perform runtime resource adaptation. The hints should be highly condensed so the serverless platform can make adaptation decisions quickly enough. 


% Apart from highly varying execution performance, serverless functions are also short-living (less than 1s on average)~\cite{atc18-peek-bench,socc22-owl,atc20-serverless-in-the-wild,socc23-huawei-cloud}, so is the window for adaptive allocation. 
% This variance and volatility calls for a well-preparation of hints for all possible runtime situations while promising them compact and straightforward enough for providers to easily take action.

% Here, our insight is to \emph{holistically synthesize hints in an offline manner, and then utilize the discreteness of adaptive allocation in both decision-making and decision-executing (detailed in~\S~\ref{sec:synthesizer:condense}) to fully condense the hints.
% Finally, hints are warped into a simple and compact table.
% Base on that, providers can accomplish the runtime adaption promptly and properly}.

To demonstrate the potential of runtime resource adaptation incorporating all the above ideas, we take a real-world serverless workflow (explained in \S\ref{exp:setup}) as an example, and evaluate its end-to-end latency (denoted by E2E) and resource consumption (CPU cores).
As illustrated in Figure~\ref{fig:bg:size}, the late-binding (blue triangle) reduces the resource consumption by up to 42.2\% compared with existing early-binding solutions (orange circle), while ensuring SLO guarantees. This highlights the significant gains from runtime resource adaptation. 


\begin{figure}[t!]
\centering
\includegraphics[width=0.45\textwidth]{./figure/motivation/size_early_bind_vs_ours.pdf}
%\vspace{-0.1cm}
\caption{Performance comparison between early-binding (left)~\cite{eurosys19-grandslam} and late-binding (runtime resource adaptation), where the CPU consumption (right) is normalized by the optimal obtained with exhaustive search.} 
%\vspace{-0.3cm}
\label{fig:bg:size}
\end{figure}

   
	







% \input{sections/03-protocol.tex}
% \section{Accuracy in Distinguishing AI-generated Images from Real Photographs}\label{sec:results}

In the main phase of the experiment, we collected 539,749 responses on 599 images from 37,568 participants from February 5, 2024 to June 22, 2024. Sections~\ref{sec:acc-general} through \ref{sec:acc-model} focus on data from the main phase of the experiment. The second phase of the experiment started on June 22 and ended on August 30, with 83,577 responses on 482 images from 3,787 participants. Sections~\ref{sec:imagestimuli} and~\ref{sec:human-curation} describe the influence of human curation of the stimuli on how accurately participants identify the stimuli as AI-generated or real. 

The design of our experiment involves several important design choices. First, we selected the three models
of Midjourney, Firefly, and Stable Diffusion as the diffusion models. Second, we crafted prompts to produce realistic
outputs across various pose categories and content types. Third, we curated 450 images from over 3000 images generated
to use as image stimuli in the experiment. These images were selected to maximize realism while also representing
different visual artifacts and implausibilities. Inevitably, these design choices on models, prompts, and stimuli introduce some selection bias  into the experiment.

Additionally, we implemented two exclusion criteria that should be considered when interpreting our results. First, for all the analyses in Section ~\ref{sec:results}, we excluded observations where participants checked the box on the website ``I have seen this before''. These observations, which account for 2\% of the total observations, were excluded because of the strong possibility that participants who had previously seen the images were already aware of whether they were fake or real.
%the experiment aimed to evaluate whether participants could identify if an unfamiliar image was AI-generated.. 
For these observations marked as having been seen before, 38\% of these observations were on AI-generated stimuli and 62\% were on real images. The image most frequently reported as 'seen before' is a real portrait of Martin Luther King Jr, which was one of the few real images of a well-known celebrity included in the experiment. 

Second, in line with our goals of studying detection ability on images for which there was some ambiguity, we excluded all images where participants' accuracy suggested very little ambiguity. We operationalized this as accuracy above 90\%. 

These exclusion criteria remove all observations on 68 fake images and 4 real images, which represent 14\% of observations from the entire experiment. 

In the human-coded analysis of artifacts discussed in Section~\ref{sec:acc-presence-artifacts}, we apply an additional exclusion criterion to make the coding tractable. Specifically, we exclude all images accurately identified in more than 80\% of observations. This exclusion criterion focuses the analysis on the most challenging images by excluding the most egregious distortions that lead to low photorealism (i.e., high participant accuracy).
%and offers a lower bound on the full extent of the differences between image categories because .

\subsection{Overall Accuracy} \label{sec:acc-general}

In the main study, participants correctly identified AI-generated images and authentic photographs in 76\% and 74\% of observations, respectively. Accuracy varied substantially across images. Prior to implementing our accuracy-based exclusion described above, we found that for AI-generated images, accuracy ranged from 32\% to 99\%. Similarly, accuracy on real photographs ranged from 28\% to 92\%. Figure~\ref{fig:accuracy_real_fake} shows the distribution of accuracy in both AI-generated and real images with example images selected from the top, bottom, and middle deciles of each distribution. At the image level, the mean accuracy for identifying AI-generated and real images was 76\% (95\% CI:[74,77]) and 74\% (95\% CI:[72,76]), respectively. 

Despite our efforts to minimize obvious artifacts, some images - particularly non-portraits - were challenging to generate without noticeable artifacts. As a result, participants achieved nearly 100\% accuracy on a few AI-generated images with obvious features. We present examples of these images in Figure~\ref{fig:three-fake-images}.  %that further motivate the exclusion criteria that we apply to the rest of the results section. 
In contrast to AI-generated images, real photographs rarely contain definitive artifacts and visual cues often seen in AI-generated images, which limits participants from achieving near-perfect accuracy on real photographs.
\begin{figure}[H]
    \centering
    \includegraphics[width=\linewidth]{sections/images/general_accuracy.pdf}
    \caption{Distribution of accuracy scores for real and AI-generated images with example images representing different accuracy levels.}
    \label{fig:accuracy_real_fake}
    \Description{Histograms showing the distribution of accuracy scores for real and AI-generated images, accompanied by example images representing various accuracy levels.}
\end{figure}
\begin{figure}[H]
\centering
\captionsetup{justification=raggedright, singlelinecheck=false, skip=2pt, font=small}
\begin{subfigure}[t]{0.3\linewidth}
    \subcaption{}\vtop{\vskip0pt\hbox{\includegraphics[width=\linewidth]{sections/images/sd_portrait3_040.jpg}}}
\end{subfigure}
\hfill
\begin{subfigure}[t]{0.3\linewidth}
    \subcaption{}\vtop{\vskip0pt\hbox{\includegraphics[width=\linewidth]{sections/images/ff_pg3_010.jpeg}}}
\end{subfigure}
\hfill
\begin{subfigure}[t]{0.3\linewidth}
    \subcaption{}\vtop{\vskip0pt\hbox{\includegraphics[width=\linewidth]{sections/images/ff_fullbody3_007.jpeg}}}
\end{subfigure}
\caption{\mybold{Examples of obviously AI-generated images and their corresponding accuracy.} \normalfont{\textbf{A.} AI-generated portrait with 92\% accuracy. \textbf{B.} AI-generated posed group image with 95\% accuracy. \textbf{C.} AI-generated full-body image with 99\% accuracy.}}
\label{fig:three-fake-images}
\Description{Three examples of obviously AI-generated images with corresponding accuracy scores: A. Portrait with 92\% accuracy, B. Posed group image with 95\% accuracy, C. Full-body image with 99\% accuracy. }
\end{figure}

\subsection{Participant Level Accuracy}\label{sec:indiv-acc}

\begin{figure*}[h]
    \centering
    \captionsetup{justification=raggedright, singlelinecheck=false, skip=2pt, font=small}

    % Top Row - A and B
    \begin{subfigure}[t]{0.4\textwidth}  
        \centering
        \subcaption[]{}  
        \vspace{-3pt}  
        \includegraphics[width=\linewidth]{sections/images/scatter_plot_fake_images.jpg} 
        % \subcaption{}
    \end{subfigure}
    \hspace{0.05\textwidth} 
    \begin{subfigure}[t]{0.38\textwidth}  
        \centering
        \subcaption[]{}  
        \vspace{-3pt}
        \includegraphics[width=\linewidth]{sections/images/scatter_plot_real_images.jpg}
        % \subcaption{}
    \end{subfigure}

    % Bottom Row - C and D
    \begin{subfigure}[t]{0.36\textwidth}  
        \centering
        \subcaption[]{}  
        \vspace{-3pt}
        \includegraphics[width=\linewidth]{sections/images/accuracy_distribution.jpg}
        % \subcaption{}
    \end{subfigure}
    \hspace{0.05\textwidth}  
    \begin{subfigure}[t]{0.36\textwidth}  
        \centering
        \subcaption[]{}  
        \vspace{-3pt}
        \includegraphics[width=\linewidth]{sections/images/learning_curve_with_fake_real_bias.png}
        % \subcaption{}
    \end{subfigure}

    \caption{\textbf{Participant-level accuracy and learning trends.}  
    \normalfont{ \textbf{A.} Scatterplot of participant-level accuracy for AI-generated images. \textbf{B.} Scatterplot of participant-level accuracy for real images. \textbf{C.} Histogram showing the distribution of accuracy across the first ten images seen by participants who viewed at least 10 images. \textbf{D.} Learning curve illustrating accuracy trends and classification biases when detecting AI-generated and real images.}}
    
    \label{fig:combined-participant-accuracy}

    \Description{A composite figure showing participant accuracy trends:  
    A. Scatterplot displaying accuracy levels for detecting AI-generated images, with points representing individual participants.  
    B. Scatterplot for real images, structured similarly to A.  
    C. Histogram showing the distribution of accuracy for participants' first 10 images.  
    D. Learning curve tracking accuracy trends over time, highlighting biases in AI-generated and real image classification.}
\end{figure*}

Given the organic nature of participants' engagement with this experiment, we did not impose restrictions on the number of images a participant saw. Most participants in this study provided responses to at least seven images, but some participants only provided a single response, and one participant provided 502 responses. 

The vast majority of participants (75\%) saw 16 or fewer images. Figure~\ref{fig:combined-participant-accuracy}A and B present the distribution of participant--level accuracy by number of viewed images. 

In order to compare participant performance and avoid issues that arise with differential attrition, we focus on the first ten images seen by participants who saw at least 10 images, which includes 152,050 observations from 15,205 participants. First, we note that 34\% of these participants achieved 90\% accuracy or higher on the first ten images seen. If the AI-generated images were perfectly photorealistic such that the human ability to distinguish is no higher than random guessing, then we would have expected only 1\% of participants to achieve this threshold of accuracy (assuming random guessing at 50\% accuracy, with participants evaluating 10 images each, achieving at least 9 out of 10 correct responses would occur with a probability of approximately 1.07\%, based on the binomial probability distribution). Figure~\ref{fig:combined-participant-accuracy}C shows the distribution of accuracy across the first ten images seen by participants who saw at least 10 images.

In Figure~\ref{fig:combined-participant-accuracy}D, we present accuracy rates by the number of images seen. We find that on average, participants begin the experiment by disproportionally identifying images as fake in 63\% of observations. Notably, this bias is reduced after only a few images.


\subsection{Accuracy by Scene Complexity} \label{sec:acc-scene-complexity}

We find that on average, participants' accuracy increases as scene complexity increases. For example, we find that 16\% of portraits appear in the bottom decile of accuracy scores (representing the highest level of photorealism), whereas only 3\% of AI-generated posed group images appear in the bottom decile. Figure~\ref{fig:pose-complexity} presents the distribution of accuracy for each category, separately for real and AI-generated images. For AI-generated images, the mean accuracy was 72.7\% (95\% CI: [72.4, 72.9]) for portraits, 77.2\% (95\% CI: [76.8, 78.6]) for full body, 76.2\% (95\% CI: [75.8, 76.7]) for posed groups, and 73.4\% (95\% CI: [73, 73.8]) for candid groups. For real images, the accuracy was 71.1\% (95\% CI: [70, 71.4]) for portraits, 75.5\% (95\% CI: [75.1, 75.8]) for full body, 76.7\% (95\% CI: [76.3, 77]) for posed groups, and 74.8\% (95\% CI: [74.4, 75.1]) for candid groups. 

As exemplified in Figure~\ref{fig:pose-complexity}C, we note that portraits, relative to the other levels of scene complexity, typically have less detail, simpler and more standardized poses, more blurred backgrounds, and fewer available cues than full-body or group images. 
\begin{figure}[h]
    \captionsetup{justification=raggedright, singlelinecheck=false, skip=2pt, font=small}
    \centering

    % First subfigure - Reduce space
    \begin{subfigure}[t]{\linewidth}
        \subcaption{}
        \vspace{-12pt}  % Reduce vertical space
        \includegraphics[width=\linewidth]{sections/images/scene_complexity_filtered90_Real.png}
    \end{subfigure}
    
    % Second subfigure - Reduce space
    \begin{subfigure}[t]{\linewidth}
        \subcaption{}
        \vspace{-12pt}  % Reduce vertical space
        \includegraphics[width=\linewidth]{sections/images/scene_complexity_filtered90_AI-generated.png}
    \end{subfigure}
    
    % Third subfigure - Reduce space
    \begin{subfigure}[t]{\linewidth}
    \centering
        \subcaption{}
        \vspace{-12pt}  % Reduce vertical space
        \includegraphics[width=0.9\linewidth]{sections/images/scene_complexity_B.pdf}
    \end{subfigure}
    
    \caption{\textbf{Scene complexity}: Accuracy of real and AI-generated images by scene complexity levels. 
    \normalfont{Beeswarm plots of image-level accuracy for each dimension of scene complexity with bootstrapped 95\% confidence intervals. We exclude images identified with above 90\% accuracy in this analysis. \textbf{A.} Real images \textbf{B.} AI-generated images \textbf{C.} AI-generated images across scene complexities.}}
    
    \label{fig:pose-complexity}
    
    \Description{Beeswarm plots showing the accuracy of real and AI-generated images by pose complexity. Each dot represents an individual image, with error bars indicating the bootstrapped 95\% confidence interval around the mean.}
\end{figure}


\subsection{Accuracy by Presence of Artifacts}\label{sec:acc-presence-artifacts}
\begin{figure*}[h]
\centering
\captionsetup{justification=raggedright, singlelinecheck=false, skip=2pt, font=small}

% Full PDF Image
\includegraphics[width=\linewidth]{sections/images/combined_artifact_trends.pdf}

\caption{\textbf{Accuracy by artifact types and display times} \normalfont{\textbf{A. Mean accuracy for different artifact types.} Distribution of accuracy scores by artifact type
for images with at least one artifact. \textbf{B. Mean accuracy over display time.} Change in mean accuracy across different display time assignments (1 second, 5 seconds, 10 seconds, 20 seconds, and unlimited) with 95\% confidence intervals and bee swarm plots of image accuracy for AI-generated and real images. \textbf{C. Mean accuracy over time for different artifact types.} Change in mean accuracy across different time assignments (1 second, 5 seconds, 10 seconds, 20 seconds, and unlimited) with 95\% confidence intervals and bee swarm plots of image accuracy for
images with anatomical (pink), functional (green), and stylistic
(blue) artifacts. The x–axis shows the display time intervals, and the
y–axis shows accuracy.}}

\label{fig:combined-artifact-trends}

\Description{A composite figure showing accuracy-related analyses:  
(A) A beeswarm plot displaying the distribution of accuracy scores for images containing at least one artifact, categorized by artifact type.  
(B) A line plot illustrating mean accuracy across different display time conditions (1 second, 5 seconds, 10 seconds, 20 seconds, and unlimited), with 95\% confidence intervals. Overlaid bee swarm plots represent individual accuracy scores for AI-generated and real images.  
(C) A line plot showing mean accuracy over time for different artifact types (Anatomical, Functional, and Stylistic). Each artifact type is color-coded (pink for Anatomical, green for Functional, and blue for Stylistic). Bee swarm plots depict individual accuracy scores for images within each artifact category. The x-axis represents display time intervals, and the y-axis represents accuracy.}

\end{figure*}


In order to analyze accuracy by artifact type, we annotated images with diffusion model artifact categories from the taxonomy based on a three-step process. First, four co-authors independently annotated all 218 images with accuracy below 80\%, identifying artifacts and providing detailed explanations for their annotations. Second, each of these annotations was reviewed and edited by two additional co-authors. Third, a fifth co-author reviewed all annotations for consistency. Figures~\ref{fig:combined-varying-artifacts-visibility}A--C and \ref{fig:combined-varying-artifacts-visibility}D--F provide examples of how we annotated images, displaying the identified artifact categories, the reasoning behind their identification, and the associated detection accuracy for each image. During this process, we observed that the three main artifact types---anatomical implausibilities, stylistic artifacts, and functional artifacts---each appeared in nearly a third of the images we annotated. In contrast, violations of physics and sociocultural implausibilities were less common, appearing in only 20 and 12 images, respectively. In light of this distribution of artifacts, Figure~\ref{fig:combined-artifact-trends}A presents the distribution of accuracy scores across images containing at least the three listed artifact types.

Based on our annotations of artifacts in images, we find participants are less accurate on images with functional implausibilities than images with anatomical implausibilities or stylistic artifacts. The mean accuracy on images with at least one functional implausibility, one anatomical implausibility, and one stylistic artifact is 64.1\% (95\% CI: [63.8, 64.5]), 65\% (95\% CI: [64.6, 65.4]), and 64.9\% (95\% CI: [64.5, 65.3]), respectively. While the accuracy on images with functional implausibilities is lower than on images with other implausibilities and artifacts, the mean accuracy scores are similar. However, this similarity in means masks the differences in the distribution of accuracy scores, as shown in Figure~\ref{fig:combined-artifact-trends}A. We find that images with participant accuracy scores in the 40--60\% range (which represent images approaching indistinguishability between real and AI-generated) make up 32.8\% of images annotated with functional implausibilities compared to 21.4\% and 22.4\% of images annotated with anatomical implausibilities and stylistic artifacts, respectively. 


We find that images that we annotated as containing multiple artifacts can still appear photorealistic enough to make detection difficult for most people. Artifacts vary in levels of visibility, as shown in Figure~\ref{fig:combined-varying-artifacts-visibility}A--C. While Figure~\ref{fig:combined-varying-artifacts-visibility}A and C contain stylistic artifacts, they are far more apparent in Figure~\ref{fig:combined-varying-artifacts-visibility}B, which is reflected in its higher detection accuracy. Despite Figure~\ref{fig:combined-varying-artifacts-visibility}A and C containing multiple artifact categories, they had low detection accuracy, suggesting that the presence of multiple artifacts does not necessarily make images easier to identify and that artifact visibility is also a contributing factor. 


The visibility of artifacts is highly variable, and Figure~\ref{fig:combined-varying-artifacts-visibility}D--F present examples highlighting this variability. The anatomical implausibility in the fingers in image Figure~\ref{fig:combined-varying-artifacts-visibility}D is very noticeable, whereas the functional implausibilities in the tennis racket and shirt design of Figure~\ref{fig:combined-varying-artifacts-visibility}F are more subtle. The corresponding accuracy scores for these images--- 62\% for Figure~\ref{fig:combined-varying-artifacts-visibility}E and 54\% for Figure~\ref{fig:combined-varying-artifacts-visibility}F —reinforce the observation that anatomical artifacts tend to be more easily detected, while functional implausibilities often require closer attention and familiarity with depicted objects. The stylistic artifacts in the cinematization of Figure~\ref{fig:combined-varying-artifacts-visibility}E and plastic-like skin texture fall in between, further showing the spectrum of detectability across different artifact categories and visibility. 


\begin{figure*}[h!]
\centering
\captionsetup{justification=raggedright, singlelinecheck=false, skip=2pt, font=small}

% First Row of Images
\begin{subfigure}[t]{0.27\linewidth}
\centering
    \subcaption{}
    \includegraphics[width=\linewidth]{sections/images/8059c316907c586bdf33ad3cb9ca3f95.jpeg}
\end{subfigure}
\hspace{1cm}
\begin{subfigure}[t]{0.27\linewidth}
\centering
    \subcaption{}
    \includegraphics[width=\linewidth]{sections/images/616ba73f50088eb13244a807076248f7.jpeg}
\end{subfigure}
\hspace{1cm}
\begin{subfigure}[t]{0.27\linewidth}
\centering
    \subcaption{}
    \includegraphics[width=\linewidth]{sections/images/2c4c0b171577884f5c0991cacb5c5ebc.jpeg}
\end{subfigure}

\vskip 5mm % Adds vertical spacing between rows

% Second Row of Images
\begin{subfigure}[t]{0.27\linewidth}
\centering
    \subcaption{}
    \includegraphics[width=\linewidth]{sections/images/ff_pg3_009.jpeg}
\end{subfigure}
\hspace{1cm}
\begin{subfigure}[t]{0.27\linewidth}
\centering
    \subcaption{}
    \includegraphics[width=\linewidth]{sections/images/mj_portrait3_010.jpeg}
\end{subfigure}
\hspace{1cm}
\begin{subfigure}[t]{0.27\linewidth}
\centering
    \subcaption{}
    \includegraphics[width=\linewidth]{sections/images/ff_portrait3_004.jpeg}
\end{subfigure}

\caption{\textbf{Examples of images with varying artifact visibility.}  
\normalfont{\textbf{Top row (A--C):} Example images showcasing stylistic and functional artifacts with varying visibility.  
\textbf{A.} A subtle stylistic artifact in the soft and wispy textures of the woman's hair and a minor functional implausibility in the atypical design of her shirt collar (Accuracy: 47\%). \textbf{B.} An obvious stylistic artifact due to the overall cinematization of the image (Accuracy: 73\%). \textbf{C.} A combination of multiple artifacts, including anatomical implausibilities in the woman's hand, functional implausibilities in the table shape and wall panels, and a stylistic artifact in the soft texture of the woman's face (Accuracy: 38\%). \textbf{Bottom row (D--F):} Images with anatomical, stylistic, and functional artifacts of varying visibility. \textbf{D.} Anatomical implausibilities in the fingers of the three students (Accuracy: 84\%). \textbf{E.} A stylistic artifact in the cinematized look and plastic-like texture of the woman's skin (Accuracy: 62\%). \textbf{F.} No obvious anatomical or stylistic artifacts, but closer inspection reveals functional implausibilities: the tennis racket is asymmetrical, its strings are not taut, and the shirt has irregularly shaped designs with glitch-like inconsistencies (Accuracy: 54\%).}}

\label{fig:combined-varying-artifacts-visibility}

\Description{A composite figure showing six images with varying visibility of AI-generated artifacts.  
(A--C) The first row highlights stylistic and functional artifacts, including wispy hair, cinematized lighting, and a distorted table.  
(D--F) The second row focuses on anatomical, stylistic, and functional artifacts, including distorted fingers, plastic-like textures, and inconsistencies in objects like a tennis racket.}
\end{figure*}

\subsection{Accuracy by Randomized Display Time}\label{sec:acc-time}
\begin{figure*}[h]
\centering
\captionsetup{justification=raggedright, singlelinecheck=false, skip=2pt, font=small}
\begin{subfigure}[t]{0.27\linewidth}
\centering
    \subcaption{}
    \includegraphics[width=\linewidth]{sections/images/bbbfb2a12cd66783ce7e4015ec0084b9.jpg}
\end{subfigure}
\hspace{1cm}
\begin{subfigure}[t]{0.27\linewidth}
\centering
  \subcaption{}
    \includegraphics[width=\linewidth]{sections/images/5204545de13342cbefdc0e9022d821d2.jpg}
\end{subfigure}
\hspace{1cm}
\begin{subfigure}[t]{0.27\linewidth}
\centering
  \subcaption{}
    \includegraphics[width=\linewidth]{sections/images/ccc04b661d52c055a44fc01718c6a2bc.jpg}
\end{subfigure}
\caption{\textbf{Exemplar AI-generated images for which a closer look improves accuracy.} \normalfont{\textbf{A.} Accuracy: 38\% at 1 second display time to 65\% at 20 second display time. \textbf{B.} Accuracy: 44\% at 1 second display time to 82\% at 20 second display time. \textbf{C.} Accuracy: 27\% at 1 second display time to 70\% at 20 second display time.}}
\label{fig:displaytime}
\Description{Three images showing AI-generated images for which a closer look improves accuracy. (A) Image of woman generated by Stable Diffusion: Accuracy is 38\% at 1 second display time and it improves to 65\% at 20 second display time with  (B) Image of people generated by Stable Diffusion: Accuracy is 44\% at 1 second display time and it improves to 82\% at 20 second display time with (C) Image of people generated by Stable Diffusion: Accuracy is 27\% at 1 second display time and it improves to 70\% at 20 second display time with.}
\end{figure*}

By randomizing the display time of images in this experiment, our results support evaluating how viewing duration influences participants' accuracy. We find that longer viewing times improve performance. With just 1 second of display time, participants are  72\% accurate (95\% CI=[71.6, 72.5], 95\% CI=[71.3, 72.2]) on AI-generated and real images, respectively. With 5 seconds of display time, accuracy increases to 77\% (95\% CI=[77.0, 77.8], 95\% CI=[76.6, 77.4]) for both AI-generated and real images, respectively. While accuracy on real images appears to plateau by 5 seconds of display time, accuracy on AI-generated images increases up to 80\% (95\% CI=[79.6, 80.4]) at 10 seconds and 82\% (95\% CI=[81.2, 81.9]) at 20 seconds. Figure~\ref{fig:combined-artifact-trends}B presents the distribution of accuracy scores across display time conditions. Across the observations where display time was randomized, we find that the proportion of AI-generated images that are identified below random chance decreases from 43\% when participants only have 1 second to view the image to 30\%, 25\%, 17\%, and 17\% when participants have 5, 10, 20 seconds, and unlimited time to view the image.

In some images, AI artifacts can be noticed with a quick glance, but for others, careful attention to detail is necessary to spot the artifact. Figure~\ref{fig:displaytime} presents three images that require careful attention, as evidenced by the fact that most participants mark as real when they are limited to seeing the image for a second but
fake once they take into account the details of the scene.

Accuracy across all artifact types improved with increased display time. As shown in Figure~\ref{fig:combined-artifact-trends}C, participants showed higher accuracy when images were displayed for longer time (anatomical artifacts: 63\% at 5 seconds vs. 59\% at 1 second; stylistic artifacts: 63\% at 5 seconds vs. 60\% at 1 second; functional artifacts: 60\% at 5 seconds vs. 55\% at 1 second). For all artifacts, there is a significant improvement in detection accuracy when increasing display time from 5 seconds to unlimited. 

In Figure~\ref{fig:combined-artifact-trends}C, we observe that participants improved the most in identifying functional artifacts, with an 18\% improvement from 1 second to unlimited viewing time. In comparison, anatomical and stylistic artifacts showed smaller improvements of 11\% each over the same time interval. Unlike anatomical and stylistic implausibilities that can be identified at first glance, functional artifacts often require a closer look and familiarity with the elements in the image as they often appear in parts of the image that are not the main subject. 


\subsection{Qualitative Analysis of Participant Comments}\label{sec:qualitative-analysis}

We collected 34,675 comments from participants who filled out the optional text input box asking participants: ``If you think this is AI-generated, please explain why.'' In order to identify themes from these 34,675 comments, we prompted GPT-3.5 Turbo to identify 10 main themes across these comments. GPT-3.5 Turbo responded with the following ten themes, which we manually reviewed and refined to mitigate the ambiguities and generalization typical of large language models \cite{stephan2024rlvflearningverbalfeedback}: (1) Image quality focusing on the overall appearance, smoothness, and sometimes unrealistic perfection of image elements; (2) Facial and anatomical inconsistencies where participants pointed to irregularities in eyes, mouths, noses, skin texture, expressions, and general human anatomy; (3) Anatomical and functional anomalies such as deformities, misplaced body parts, and irregularities in objects or environments; (4) Lighting and environmental inconsistencies including unnatural lighting, inconsistent shadows, and reflections; (5) Digital manipulation indicators suggesting suspicions of AI-generation or digital alteration; (6) Biometric discrepancies particularly unnatural or imperfect body parts like hands and fingers; (7) Uncanny valley perceptions where images almost looked human but had subtle unnatural features that caused discomfort; (8) Contextual incongruities such as unrealistic scenarios and mismatched social elements; (9) Physical anomalies highlighting illogical physical interactions within the images; and (10) holistic authenticity assessment making overall judgments based on a combination of multiple cues and inconsistencies. Based on these ten main themes, we prompted GPT-3.5 to label each comment with one of the ten themes. Figure~\ref{fig:comments-all} illustrates examples of participant comments for four images and how they were categorized into themes. Figure~\ref{fig:themes} displays the distribution of themes across the comments and the related concept from our taxonomy in parentheses.

Based on GPT-3.5 Turbo, we find that 61\% of participants' comments mentioned relying on anatomical implausibilities. The next most common concept referred to is stylistic artifacts, which is mentioned in 30\% of comments. Participants mentioned functional implausibilities in 21\% of comments, violations of physics in 15\% of comments, and sociocultural implausibilities in only 4\% of comments. 

Based on the authors' annotations of artifacts, we find functional implausibilities to be the most prevalent, appearing in 58.7\% of images, followed by anatomical implausibilities in 51.4\% and stylistic artifacts in 39.0\% of images. We identify violations of physics and sociocultural implausibilities in only 9.17\% and 5.50\% of images, respectively. 
\begin{figure}[H]
\centering
\captionsetup{justification=raggedright, singlelinecheck=false, skip=2pt}
\begin{subfigure}[t]{0.9\linewidth}
    % \subcaption{}
    \includegraphics[width=\linewidth]{sections/images/theme_distribution_single_column.jpg}
\end{subfigure}
\caption{\mybold{Distribution of themes identified in participant comments.}}
\label{fig:themes}
\Description{A horizontal bar chart showing the distribution of themes identified in participant comments explaining their reasoning for AI image detection. The themes include Image Quality, Facial and Anatomical Inconsistencies, Anatomical and Functional Anomalies, Lighting and Environmental Inconsistencies, Digital Manipulation Indicators, Biometric Discrepancies, Uncanny Valley Perceptions, Contextual Incongruities, Physical Anomalies, and Holistic Authenticity Assessment.}
\end{figure}
While functional artifacts were the most prevalent in human researcher annotated images, they were less frequently mentioned in participant comments annotated by GPT--3.5. Conversely, anatomical artifacts were emphasized more in participant comments than in their prevalence in annotated images. 

\begin{figure}[htb]
\centering
\captionsetup{justification=raggedright, singlelinecheck=false, skip=2pt, font=small}
\begin{subfigure}[t]{0.22\textwidth}
    \subcaption{}\vtop{\vskip0pt\hbox{\includegraphics[width=\linewidth]{sections/images/mj_ng3_007.jpg}}}
\end{subfigure}
\hfill
\begin{subfigure}[t]{0.22\textwidth}
    \subcaption{}\vtop{\vskip0pt\hbox{\includegraphics[width=\linewidth]{sections/images/mj_portrait3_001.jpg}}}
\end{subfigure}
\hfill
\begin{subfigure}[t]{0.22\textwidth}
    \subcaption{}\vtop{\vskip0pt\hbox{\includegraphics[width=\linewidth]{sections/images/mj_fullbody3_003.jpg}}}
\end{subfigure}
% \hfill
% \begin{subfigure}[t]{0.19\textwidth}\subcaption{}\vtop{\vskip0pt\hbox{\includegraphics[width=\linewidth]{sections/images/fullbody3_023.jpeg}}}
% \end{subfigure}
\hfill
\begin{subfigure}[t]{0.22\textwidth}
    \subcaption{}\vtop{\vskip0pt\hbox{\includegraphics[width=\linewidth]{sections/images/mj_pg3_002.jpg}}}
\end{subfigure}
\caption{\mybold{Examples of participant comments mapped to themes.} \normalfont{\textbf{A.} ``Cosmetic style out of character with vintage setting": Contextual Incongruities. \textbf{B.} ``Skin too smooth, depth of field shallow.": Image Quality, Lighting Inconsistencies. \textbf{C.} ``If this is not AI then it is a staged photograph like a movie set because of the lighting and he is an actor.": Lighting inconsistencies, Contextual Incongruities.
\textbf{D.} ``Group looks pasted onto background.": Digital Manipulation Indicators.}}
\label{fig:comments-all}
\Description{Four images with participant comments mapped to themes.(A) AI-generated candid image with a comment on cosmetic style being out of character with a vintage setting.(B) AI image with smooth skin, with a comment on skin being too smooth and shallow depth of field.(C) AI-generated full body shot of a man, with a comment suggesting it resembles a staged photograph due to lighting.
(D) AI image of a group, with a comment on the group looking pasted onto the background.}
\end{figure}

\subsection{Accuracy by Models} \label{sec:acc-model}

In the process of generating the images for this experiment's stimuli set, we noticed that Midjourney, Firefly, and Stable Diffusion have different capabilities and limitations. For example, we noticed that Midjourney often produced images with persistent stylistic artifacts that were challenging to eliminate. Firefly, on the other hand, frequently exhibited a tendency toward synthetic emotional expressions, with subjects often appearing unnaturally and overly cheerful, necessitating multiple iterations to produce more realistic results. Stable Diffusion struggled significantly with generating group images, often introducing artifacts such as anatomical inconsistencies. In light of the limitations to generate non-portrait images with Stable Diffusion, 75\% of the Stable Diffusion-generated stimuli in this experiment were portraits. On the other hand, 30\% of Midjourney and Firefly-generated images in this experiment depict portraits. In order to compare the three models fairly, we focus our comparison on portrait images. Figure~\ref{fig:models} presents accuracy shown on portraits by each of the three models and reveals that participants' mean accuracy on Midjourney, Stable Diffusion, and Firefly were  76\% (95\% CI: [75.2, 75.8]), 74\% (95\% CI: [73.9, 74.8]), and 73\% (95\% CI: [72.7, 73.3]), respectively. 


\begin{figure}[H]
    \centering
    \includegraphics[width=0.8\linewidth]{sections/images/model_accuracy_combined.jpg} 
    \caption{\textbf{Accuracy across generative AI models} \normalfont{Each point represents an image. The black dots and error bars show the mean accuracy and 95\% bootstrapped confidence intervals for each model}}
    \label{fig:models}
    \Description{Bee swarm chart showing accuracy across different generative AI models. The chart compares the accuracy rates for identifying AI-generated content among various models along with bootstrapped 95\% confidence intervals}
\end{figure}


\subsection{Accuracy on Human Curated Images vs. Uncurated Images} \label{sec:human-curation}
\begin{figure*}[h]
\centering
\captionsetup{justification=raggedright, singlelinecheck=false, skip=2pt, font=small}
\begin{subfigure}[t]{0.24\linewidth}
    \subcaption{}\vtop{\vskip0pt\hbox{\includegraphics[width=\linewidth]{sections/images/sd_portrait3_003.jpg}}}
\end{subfigure}
\hfill
\begin{subfigure}[t]{0.24\linewidth}
    \subcaption{}\vtop{\vskip0pt\hbox{\includegraphics[width=\linewidth]{sections/images/0bce6c35c24ec5ce8ae8ad5bb4f67d59_r4.jpg}}}
\end{subfigure}
\hfill
\begin{subfigure}[t]{0.24\linewidth}
    \subcaption{}\vtop{\vskip0pt\hbox{\includegraphics[width=\linewidth]{sections/images/0bce6c35c24ec5ce8ae8ad5bb4f67d59_r11.jpg}}}
\end{subfigure}
\hfill
\begin{subfigure}[t]{0.24\linewidth}
    \subcaption{}\vtop{\vskip0pt\hbox{\includegraphics[width=\linewidth]{sections/images/0bce6c35c24ec5ce8ae8ad5bb4f67d59_r2.jpg}}}
\end{subfigure}
\caption{\mybold{Re-generated images from the same prompt.} \normalfont{ \textbf{A.} Stage 1 image generated by Stable Diffusion and curated by our team (37\% accuracy) \textbf{B.} Most photorealistic of 12 prompt-matched image generations by Stable Diffusion (42\% accuracy) \textbf{C.} Median photorealistic of 12 prompt-matched image generations by Stable Diffusion (59\% accuracy) \textbf{D.} Least photorealistic of 12 prompt-matched image generations by Stable Diffusion(83\% accuracy)}}
\label{fig:regeneration}
\Description{Four images showing re-generated outputs from the same prompt.\textbf{A.} Stage 1 image generated by Stable Diffusion and curated by our team (37\% accuracy) \textbf{B.} Most photorealistic of 12 prompt-matched image generations by Stable Diffusion (42\% accuracy) \textbf{C.} Median photorealistic of 12 prompt-matched image generations by Stable Diffusion (59\% accuracy) \textbf{D.} Least photorealistic of 12 prompt-matched image generations by Stable Diffusion(83\% accuracy)}
\end{figure*}

\begin{figure*}[h]
\centering
\captionsetup{justification=raggedright, singlelinecheck=false, skip=2pt}
\begin{subfigure}[t]{0.48\textwidth}  
\subcaption{}
\vtop{\vskip0pt\hbox{\includegraphics[width=\linewidth]{sections/images/curation_value_add_min.jpg}}}
\end{subfigure}
\hfill
\begin{subfigure}[t]{0.48\textwidth}
\subcaption{}
\vtop{\vskip0pt\hbox{\includegraphics[width=\linewidth]{sections/images/curation_value_add_mean.jpg}}}
\end{subfigure}
\caption{\mybold{Comparing accuracy scores on curated images and uncurated prompt-matched images.} \normalfont{\textbf{A.} Scatterplot showing human detection accuracy of the original curated image compared to human detection accuracy of its most photorealistic regeneration out of 11 to 24 prompt-matched images labeled as re-generations. \textbf{B.} Scatterplot showing human detection accuracy of the original curated image compared to human detection accuracy of its mean photorealistic regeneration out of 11 to 24 re-generations.}}
\label{fig:curation-value}
\Description{Two scatterplots showing curation value analysis.(A) Minimum curation value added.(B) Mean curation value added. Each chart illustrates the impact of curation on the overall value added to the dataset.}
\end{figure*}

Generating photorealistic AI-generated images involves three key ingredients: the diffusion model, the prompt, and human curation. In this section, we examine how human curation of diffusion model-generated images influences the aggregate accuracy scores of human participants. In order to show this influence, we compare diffusion model images from the main experiment, which were curated by our research team, with multiple diffusion model images generated from the same prompt as the curated images. This comparison reveals the increase in photorealism (as measured by the decrease in participants' accuracy) on the curated images relative to the prompt-matched images.

In this second phase of the experiment, we randomly sampled 39 AI-generated images from the main stimuli set, where the sample was stratified on 10 percentage point wide bins on human detection accuracy. For each of these 39 images, we generated at least 11 prompt-matched images using Midjourney, Firefly, and the same pipeline in Stable Diffusion. Figure~\ref{fig:regeneration} displays a Stable Diffusion-generated image from our original stimuli set and three of the twelve generations using the same prompt. We generated 482 total additional images, with at least 11 per prompt. These 482 images were included alongside the 149 real images on the experiment website.

In Figure~\ref{fig:curation-value}, we present scatterplots comparing human detection accuracy on the initial curated images and the best prompt-matched images in panel A, and mean prompt-matched images in panel B. We find the human-curated images have lower human detection accuracy than the best regenerated image in 18 of 39 instances and the mean re-generated image in 35 of 39 instances. In total, the human-curated images were perceived to be more photorealistic than 408 of the 482 (84\%) uncurated prompt-matched images. Specifically, we find the marginal value added by human curation for images that were initially detected in the range of 30\% to 50\% is 31 percentage points, 50 to 60\% is 23 percentage points, 60-70\% is 11 percentage points, 70-80\% is 8 percentage points, and 80+\% is 4 percentage points. Across the stimuli selected from Midjourney, Firefly, and Stable Diffusion, the marginal value of human curation is 7.8, 19.0, and 16.9 percentage points, respectively. 


The two panels in Figure~\ref{fig:curation-value} illustrate the positive correlation between accuracy on the human-curated image and accuracy on the regeneration. This reveals how the prompt influences photorealism. The Pearson Correlation Coefficient between accuracy on curated images and their best, mean, and worst re-generations are .58, .53, and .32, respectively. This positive correlation suggests the choice of a prompt plays a significant role in the photorealism of an image. Figure~\ref{fig:goodandbadprompt} displays two original curated images where A is generated by a prompt in which re-generations achieved low human detection accuracy (a `good' prompt), and B is generated by a prompt in which re-generations achieved a high human detection accuracy (a `bad' prompt). Prompts that consistently generate easily detectable images often have elements that are difficult to generate and result in artifacts. The prompt ``Persian woman astronaut in astronaut clothes, family photo with husband and two toddlers, high resolution, realistic" for Figure~\ref{fig:goodandbadprompt}B  generates a posed group image that tends to be easy to detect. On the other hand, the prompt ``American woman faculty portrait, not a close-up, blond" for Figure~\ref{fig:goodandbadprompt}A generates a portrait image that tends to be perceived as more photorealistic.
% \section{Conclusions and Future Work}
\label{sec:conclusions}
%\nabanita{Nabanita: ADD Percentage increase and other comparisons as necessary!}
Embodied agents assisting humans frequently have to complete previously unseen tasks or operate in new scenario. This paper describes a framework that leverages the complementary strengths of Large Language Models (LLMs), Knowledge Graphs (KGs), and Human-in-the-Loop (HITL) feedback to satisfy this requirement. Specifically, the generic task decomposition ability of LLMs is used to predict a sequence of abstract actions to complete any given task. This sequence is adapted to the specific scenario(s) and the task-, agent-, or domain-specific constraints using a KG that encodes prior knowledge of some objects, object attributes, and action capabilities. Any unresolved mismatch between the KG and the LLM output, and any unexpected action outcomes, are addressed by soliciting and using human input. This HITL feedback corrects errors and refines the existing knowledge (in the KG) for subsequent operation. Experimental evaluation in two simulated domains demonstrates substantial performance improvement compared with baselines, and illustrates incremental acquisition of knowledge to adapt to new classes of tasks.

%Our framework successfully merges the unique capabilities of Large Language Models (LLMs), Knowledge Graphs (KGs), and human feedback to improve how embodied agents respond to unfamiliar situations. LLMs provide a foundation by generating high-level action plans, which are then refined with detailed, domain-specific knowledge from KGs. This allows agents to adjust on the fly, even when encountering new objects or tasks. Additionally, incorporating human-in-the-loop (HITL) feedback offers real-time updates and fine-tuning, ensuring the agents continually evolve and expand their understanding. This strategy is particularly useful in everyday activities like cooking or cleaning, where agents can quickly understand and execute tasks without requiring extensive retraining. By combining LLMs, KGs, and human insights, the system ensures that agents remain adaptable and efficient, even in unpredictable environments.

% paves the way for exciting future developments, such as expanding the framework to handle a wider variety of tasks and environments, fine-tuning the balance between automation and human input, and exploring ways to make knowledge refinement more autonomous. Ultimately, our approach marks a significant step toward building smarter, more adaptive, and human-centered assistive agents.
\vspace{-0.75em}
This research opens up multiple avenues for further research. First, we will explore the use of this framework in many more classes of tasks, building on (and reinforcing) the promising results obtained so far.  Second, we will investigate the trade-off between automating the generation of an action sequence for any given task, and soliciting and incorporating human feedback as needed. Furthermore, we will explore the use of this framework on a physical robot platform assisting humans. The long-term objective is to create assistive agents and robots that can interact and collaborate with humans in different application domains.
% \section{Threats to Validity}
\label{Section:Threats}

In this section, we describe potential threats to the validity of our research method and the actions we took to mitigate them.

\textbf{Internal Validity.} The accuracy of our analysis is primarily dependent on the precision of the refactoring mining tools, as these tools may miss the detection of some refactorings. However, previous studies \citep{silva2016we,tsantalis2018accurate,silva2017refdiff} report that \texttt{RefactoringMiner} and \texttt{RefDiff} have high precision and recall scores compared to other state-of-the-art refactoring detection tools, giving us confidence in using the tools. Another potential threat to validity is related to commit messages. \textcolor{black}{This study does not exclude commits containing tangle code changes \citep{herzig2016impact,kirinuki2014hey}, where developers made changes related to different tasks and one of these tasks could be related to quality improvement. If these changes were committed at once, there is a possibility that the individual changes merge and that the original task cannot be traced back. Similarly to the previous study \cite{pantiuchina2018improving}, we did not consider filtering out such changes in this study}. Moreover, our manual analysis is time-consuming and error-prone, which we tried to mitigate by focusing mainly on commits known to contain refactorings. 

Another potential threat to validity is sample bias, where the choice of the data can directly impact the results. Therefore, we explored a large sample of projects from the SmartSHARK dataset \citep{trautsch2021msr}, to ensure the quality of the findings and diversify the sources to reduce the bias of the data belonging to the same entity. The qualitative analysis was conducted by a single author, which could introduce bias into the process. However, commits that were debatable were discarded. We also provide our dataset online for further refinement and analysis. %During our qualitative analysis, we consideblack only commits where a consensus between authors was reached on whether a message clearly states the removal of duplicate code. Commits that were debatable were discarded. We also provide our dataset online for further refinement and analysis.

\textbf{Construct Validity.} A potential threat to construct validity relates to the set of metrics, as it may miss some properties of the selected internal quality attributes. To address this potential threat, we mitigate it by choosing well-known metrics that encompass various properties of each attribute, as reported in the literature \citep{chidamber1994metrics}.

\textbf{External Validity.} Our analysis was limited to only open-source Java projects. However, we were able to examine 128 projects, which were well-commented and exhibited diversity in terms of size, contributors, number of commits, and refactorings. \textcolor{black}{Still, we believe that the results found in this study are largely language-agnostic. However, certain language-specific characteristics, such as syntax complexity and tooling support, can influence duplication patterns. Although we expect similar trends across languages with similar paradigms, a comprehensive analysis encompassing various languages is recommended to confirm this generalization.}

%Still, we believe that the removal of duplicates is largely language-agnostic. However, certain language-specific characteristics, such as syntax complexity and tooling support, can influence duplication patterns. Although we expect similar trends across languages with similar paradigms, a comprehensive analysis encompassing various languages is suggested to is recommended to confirm this generalization.
% \input{sections/07-remarks.tex}

% \bibliographystyle{IEEEtranS}
% \bibliography{bibTex/sigproc} 

% \onecolumn
\appendix
\section{Relevant proofs}
\subsection{Proof of Theorem 2}
\label{proofth1}
\begin{proof}
\label{th1proof}   
 The proof is based on Borkar's Theorem for
 general stochastic approximation recursions with two time scales \cite{borkar1997stochastic}. 
 A new one-step linear TD solution is defined as: 
\begin{equation*}
0=\mathbb{E}[(\delta-\mathbb{E}[\delta]) \phi]=-A\theta+b.
\end{equation*}
Thus, the VMTD's solution is $\theta_{\text{VMTD}}=A^{-1}b$. First, note that recursion (\ref{theta}) can be rewritten as
\begin{equation*}
\theta_{k+1}\leftarrow \theta_k+\beta_k\xi(k),
\end{equation*}
where
\begin{equation*}
\xi(k)=\frac{\alpha_k}{\beta_k}(\delta_k-\omega_k)\phi_k
\end{equation*}
Due to the settings of step-size schedule $\alpha_k = o(\beta_k)$,
$\xi(k)\rightarrow 0$ almost surely as $k\rightarrow\infty$. 
 That is the increments in iteration (\ref{omega}) are uniformly larger than
 those in (\ref{theta}), thus (\ref{omega}) is the faster recursion.
 Along the faster time scale, iterations of (\ref{omega}) and (\ref{theta})
 are associated with the ODEs system as follows:
\begin{equation}
 \dot{\theta}(t) = 0,
\label{thetaFast}
\end{equation}
\begin{equation}
 \dot{\omega}(t)=\mathbb{E}[\delta_t|\theta(t)]-\omega(t).
\label{omegaFast}
\end{equation}
 Based on the ODE (\ref{thetaFast}), $\theta(t)\equiv \theta$ when
 viewed from the faster timescale. 
 By the Hirsch lemma \cite{hirsch1989convergent}, it follows that
$||\theta_k-\theta||\rightarrow 0$ a.s. as $k\rightarrow \infty$ for some
$\theta$ that depends on the initial condition $\theta_0$ of recursion
 (\ref{theta}).
 Thus, the ODE pair (\ref{thetaFast})-(\ref{omegaFast}) can be written as
\begin{equation}
 \dot{\omega}(t)=\mathbb{E}[\delta_t|\theta]-\omega(t).
\label{omegaFastFinal}
\end{equation}
 Consider the function $h(\omega)=\mathbb{E}[\delta|\theta]-\omega$,
 i.e., the driving vector field of the ODE (\ref{omegaFastFinal}).
 It is easy to find that the function $h$ is Lipschitz with coefficient
$-1$.
 Let $h_{\infty}(\cdot)$ be the function defined by
$h_{\infty}(\omega)=\lim_{x\rightarrow \infty}\frac{h(x\omega)}{x}$.
 Then  $h_{\infty}(\omega)= -\omega$,  is well-defined. 
 For (\ref{omegaFastFinal}), $\omega^*=\mathbb{E}[\delta|\theta]$
 is the unique globally asymptotically stable equilibrium.
 For the ODE
  \begin{equation}
 \dot{\omega}(t) = h_{\infty}(\omega(t))= -\omega(t),
 \label{omegaInfty}
 \end{equation}
 apply $\vec{V}(\omega)=(-\omega)^{\top}(-\omega)/2$ as its
 associated strict Liapunov function. Then,
 the origin of (\ref{omegaInfty}) is a globally asymptotically stable
 equilibrium. Consider now the recursion (\ref{omega}).
 Let $M_{k+1}=(\delta_k-\omega_k)
 -\mathbb{E}[(\delta_k-\omega_k)|\mathcal{F}(k)]$,
 where $\mathcal{F}(k)=\sigma(\omega_l,\theta_l,l\leq k;\phi_s,\phi_s',r_s,s<k)$, $k\geq 1$ are the sigma fields
 generated by $\omega_0,\theta_0,\omega_{l+1},\theta_{l+1},\phi_l,\phi_l'$, $0\leq l<k$.
 It is easy to verify that $M_{k+1},k\geq0$ are integrable random variables that 
 satisfy $\mathbb{E}[M_{k+1}|\mathcal{F}(k)]=0$, $\forall k\geq0$.
 Because $\phi_k$, $r_k$, and $\phi_k'$ have
 uniformly bounded second moments, it can be seen that for some constant $c_1>0$, $\forall k\geq0$,
\begin{equation*}
 \mathbb{E}[||M_{k+1}||^2|\mathcal{F}(k)]\leq
 c_1(1+||\omega_k||^2+||\theta_k||^2).
\end{equation*}
Now Assumptions (A1) and (A2) of \cite{borkar2000ode} are verified.
 Furthermore, Assumptions (TS) of \cite{borkar2000ode} are satisfied by our
 conditions on the step-size sequences $\alpha_k$, $\beta_k$. Thus,
 by Theorem 2.2 of \cite{borkar2000ode} we obtain that $||\omega_k-\omega^*||\rightarrow 0$ almost surely as $k\rightarrow \infty$.
Consider now the slower time scale recursion (\ref{theta}).
 Based on the above analysis, (\ref{theta}) can be rewritten as 
\begin{equation*}
\theta_{k+1}\leftarrow
\theta_{k}+\alpha_k(\delta_k-\mathbb{E}[\delta_k|\theta_k])\phi_k.
\end{equation*}
Let $\mathcal{G}(k)=\sigma(\theta_l,l\leq k;\phi_s,\phi_s',r_s,s<k)$, 
$k\geq 1$ be the sigma fields
 generated by $\theta_0,\theta_{l+1},\phi_l,\phi_l'$,
$0\leq l<k$.
 Let $Z_{k+1} = Y_{t}-\mathbb{E}[Y_{t}|\mathcal{G}(k)]$,
 where
\begin{equation*}
 Y_{k}=(\delta_k-\mathbb{E}[\delta_k|\theta_k])\phi_k.
\end{equation*}
 Consequently,
\begin{equation*}
\begin{array}{ccl}
 \mathbb{E}[Y_t|\mathcal{G}(k)]&=&\mathbb{E}[(\delta_k-\mathbb{E}[\delta_k|\theta_k])\phi_k|\mathcal{G}(k)]\\
&=&\mathbb{E}[\delta_k\phi_k|\theta_k]
 -\mathbb{E}[\mathbb{E}[\delta_k|\theta_k]\phi_k]\\
&=&\mathbb{E}[\delta_k\phi_k|\theta_k]
 -\mathbb{E}[\delta_k|\theta_k]\mathbb{E}[\phi_k]\\
&=&\mathrm{Cov}(\delta_k|\theta_k,\phi_k),
\end{array}
\end{equation*}
 where $\mathrm{Cov}(\cdot,\cdot)$ is a covariance operator.
Thus,
 \begin{equation*}
\begin{array}{ccl}
 Z_{k+1}&=&(\delta_k-\mathbb{E}[\delta_k|\theta_k])\phi_k-\mathrm{Cov}(\delta_k|\theta_k,\phi_k).
\end{array}
\end{equation*}
 It is easy to verify that $Z_{k+1},k\geq0$ are integrable random variables that 
 satisfy $\mathbb{E}[Z_{k+1}|\mathcal{G}(k)]=0$, $\forall k\geq0$.
 Also, because $\phi_k$, $r_k$, and $\phi_k'$ have
 uniformly bounded second moments, it can be seen that for some constant
$c_2>0$, $\forall k\geq0$,
\begin{equation*}
 \mathbb{E}[||Z_{k+1}||^2|\mathcal{G}(k)]\leq
 c_2(1+||\theta_k||^2).
\end{equation*}

 Consider now the following ODE associated with (\ref{theta}):
\begin{equation}
\begin{array}{ccl}
 \dot{\theta}(t)&=&\mathrm{Cov}(\delta|\theta(t),\phi)\\
&=&\mathrm{Cov}(r+(\gamma\phi'-\phi)^{\top}\theta(t),\phi)\\
&=&\mathrm{Cov}(r,\phi)-\mathrm{Cov}(\theta(t)^{\top}(\phi-\gamma\phi'),\phi)\\
&=&\mathrm{Cov}(r,\phi)-\theta(t)^{\top}\mathrm{Cov}(\phi-\gamma\phi',\phi)\\
&=&\mathrm{Cov}(r,\phi)-\mathrm{Cov}(\phi-\gamma\phi',\phi)^{\top}\theta(t)\\
&=&\mathrm{Cov}(r,\phi)-\mathrm{Cov}(\phi,\phi-\gamma\phi')\theta(t)\\
&=&-A\theta(t)+b.
\end{array}
\label{odetheta}
\end{equation}
 Let $\vec{h}(\theta(t))$ be the driving vector field of the ODE
 (\ref{odetheta}).
\begin{equation*}
 \vec{h}(\theta(t))=-A\theta(t)+b.
\end{equation*}
 Consider the cross-covariance matrix,
\begin{equation}
\begin{array}{ccl}
 A &=& \mathrm{Cov}(\phi,\phi-\gamma\phi')\\
  &=&\frac{\mathrm{Cov}(\phi,\phi)+\mathrm{Cov}(\phi-\gamma\phi',\phi-\gamma\phi')-\mathrm{Cov}(\gamma\phi',\gamma\phi')}{2}\\
  &=&\frac{\mathrm{Cov}(\phi,\phi)+\mathrm{Cov}(\phi-\gamma\phi',\phi-\gamma\phi')-\gamma^2\mathrm{Cov}(\phi',\phi')}{2}\\
  &=&\frac{(1-\gamma^2)\mathrm{Cov}(\phi,\phi)+\mathrm{Cov}(\phi-\gamma\phi',\phi-\gamma\phi')}{2},\\
\end{array}
\label{covariance}
\end{equation}
 where we eventually used $\mathrm{Cov}(\phi',\phi')=\mathrm{Cov}(\phi,\phi)$
\footnote{The covariance matrix $\mathrm{Cov}(\phi',\phi')$ is equal to
 the covariance matrix $\mathrm{Cov}(\phi,\phi)$ if the initial state is re-reachable or
 initialized randomly in a Markov chain for on-policy update.}.
 Note that the covariance matrix $\mathrm{Cov}(\phi,\phi)$ and
$\mathrm{Cov}(\phi-\gamma\phi',\phi-\gamma\phi')$ are semi-positive
 definite. Then, the matrix $A$ is semi-positive definite because  $A$ is
 linearly combined  by  two positive-weighted semi-positive definite matrice
 (\ref{covariance}).
 Furthermore, $A$ is nonsingular due to the assumption.
 Hence, the cross-covariance matrix $A$ is positive definite.

 Therefore,
$\theta^*=A^{-1}b$ can be seen to be the unique globally asymptotically
 stable equilibrium for ODE (\ref{odetheta}).
 Let $\vec{h}_{\infty}(\theta)=\lim_{r\rightarrow
\infty}\frac{\vec{h}(r\theta)}{r}$. Then
$\vec{h}_{\infty}(\theta)=-A\theta$ is well-defined. 
 Consider now the ODE
\begin{equation}
 \dot{\theta}(t)=-A\theta(t).
\label{odethetafinal}
\end{equation}
 The ODE (\ref{odethetafinal}) has the origin of its unique globally asymptotically stable equilibrium.
 Thus, the assumption (A1) and (A2) are verified.
    \end{proof}

\subsection{Proof of Theorem 3}
\label{proofth2}
\begin{proof}
The proof is similar to that given by \cite{sutton2009fast} for TDC, but it is based on multi-time-scale stochastic approximation.

For the VMTDC algorithm, a new one-step linear TD solution is defined as:
\begin{equation*}
    0=\mathbb{E}[({\phi} - \gamma {\phi}' - \mathbb{E}[{\phi} - \gamma {\phi}']){\phi}^\top]\mathbb{E}[{\phi} {\phi}^{\top}]^{-1}\mathbb{E}[(\delta -\mathbb{E}[\delta]){\phi}]=\textbf{A}^{\top}\textbf{C}^{-1}(-\textbf{A}{\theta}+{b}).
\end{equation*}
The matrix $\textbf{A}^{\top}\textbf{C}^{-1}\textbf{A}$ is positive definite. Thus, the  VMTD's solution is
${\theta}_{\text{VMTDC}}=\textbf{A}^{-1}{b}$.

First, note that recursion (\ref{thetavmtdc}) and (\ref{uvmtdc}) can be rewritten as, respectively, 
\begin{equation*}
 {\theta}_{k+1}\leftarrow {\theta}_k+\zeta_k {x}(k),
\end{equation*}
\begin{equation*}
 {u}_{k+1}\leftarrow {u}_k+\beta_k {y}(k),
\end{equation*}
where 
\begin{equation*}
 {x}(k)=\frac{\alpha_k}{\zeta_k}[(\delta_{k}- \omega_k) {\phi}_k - \gamma{\phi}'_{k}({\phi}^{\top}_k {u}_k)],
\end{equation*}
\begin{equation*}
 {y}(k)=\frac{\zeta_k}{\beta_k}[\delta_{k}-\omega_k - {\phi}^{\top}_k {u}_k]{\phi}_k.
\end{equation*}

Recursion (\ref{thetavmtdc}) can also be rewritten as
\begin{equation*}
 {\theta}_{k+1}\leftarrow {\theta}_k+\beta_k z(k),
\end{equation*}
where
\begin{equation*}
 z(k)=\frac{\alpha_k}{\beta_k}[(\delta_{k}- \omega_k) {\phi}_k - \gamma{\phi}'_{k}({\phi}^{\top}_k {u}_k)],
\end{equation*}

Due to the settings of the step-size schedule 
$\alpha_k = o(\zeta_k)$, $\zeta_k = o(\beta_k)$, ${x}(k)\rightarrow 0$, ${y}(k)\rightarrow 0$, $z(k)\rightarrow 0$ almost surely as $k\rightarrow 0$.
That is the increments in iteration (\ref{omegavmtdc}) are uniformly larger than
those in (\ref{uvmtdc}) and  the increments in iteration (\ref{uvmtdc}) are uniformly larger than
those in (\ref{thetavmtdc}), thus (\ref{omegavmtdc}) is the fastest recursion, (\ref{uvmtdc}) is the second fast recursion, and (\ref{thetavmtdc}) is the slower recursion.
Along the fastest time scale, iterations of (\ref{thetavmtdc}), (\ref{uvmtdc}) and (\ref{omegavmtdc})
are associated with the ODEs system as follows:
\begin{equation}
 \dot{{\theta}}(t) = 0,
    \label{thetavmtdcFastest}
\end{equation}
\begin{equation}
 \dot{{u}}(t) = 0,
    \label{uvmtdcFastest}
\end{equation}
\begin{equation}
 \dot{\omega}(t)=\mathbb{E}[\delta_t|{u}(t),{\theta}(t)]-\omega(t).
    \label{omegavmtdcFastest}
\end{equation}

Based on the ODE (\ref{thetavmtdcFastest}) and (\ref{uvmtdcFastest}), both ${\theta}(t)\equiv {\theta}$
and ${u}(t)\equiv {u}$ when viewed from the fastest timescale.
By the Hirsch lemma \cite{hirsch1989convergent}, it follows that
$||{\theta}_k-{\theta}||\rightarrow 0$ a.s. as $k\rightarrow \infty$ for some
${\theta}$ that depends on the initial condition ${\theta}_0$ of recursion
(\ref{thetavmtdc}) and $||{u}_k-{u}||\rightarrow 0$ a.s. as $k\rightarrow \infty$ for some
$u$ that depends on the initial condition $u_0$ of recursion
(\ref{uvmtdc}). Thus, the ODE pair (\ref{thetavmtdcFastest})-(ref{omegavmtdcFastest})
can be written as 
\begin{equation}
 \dot{\omega}(t)=\mathbb{E}[\delta_t|{u},{\theta}]-\omega(t).
    \label{omegavmtdcFastestFinal}
\end{equation}

Consider the function $h(\omega)=\mathbb{E}[\delta|{\theta},{u}]-\omega$,
i.e., the driving vector field of the ODE (\ref{omegavmtdcFastestFinal}).
It is easy to find that the function $h$ is Lipschitz with coefficient
$-1$.
Let $h_{\infty}(\cdot)$ be the function defined by
 $h_{\infty}(\omega)=\lim_{r\rightarrow \infty}\frac{h(r\omega)}{r}$.
 Then  $h_{\infty}(\omega)= -\omega$,  is well-defined. 
 For (\ref{omegavmtdcFastestFinal}), $\omega^*=\mathbb{E}[\delta|{\theta},{u}]$
is the unique globally asymptotically stable equilibrium.
For the ODE
\begin{equation}
 \dot{\omega}(t) = h_{\infty}(\omega(t))= -\omega(t),
 \label{omegavmtdcInfty}
\end{equation}
apply $\vec{V}(\omega)=(-\omega)^{\top}(-\omega)/2$ as its
associated strict Liapunov function. Then,
the origin of (\ref{omegavmtdcInfty}) is a globally asymptotically stable
equilibrium.

Consider now the recursion (\ref{omegavmtdc}).
Let
$M_{k+1}=(\delta_k-\omega_k)
-\mathbb{E}[(\delta_k-\omega_k)|\mathcal{F}(k)]$,
where $\mathcal{F}(k)=\sigma(\omega_l,{u}_l,{\theta}_l,l\leq k;{\phi}_s,{\phi}_s',r_s,s<k)$, 
$k\geq 1$ are the sigma fields
generated by $\omega_0,u_0,{\theta}_0,\omega_{l+1},{u}_{l+1},{\theta}_{l+1},{\phi}_l,{\phi}_l'$,
$0\leq l<k$.
It is easy to verify that $M_{k+1},k\geq0$ are integrable random variables that 
satisfy $\mathbb{E}[M_{k+1}|\mathcal{F}(k)]=0$, $\forall k\geq0$.
Because ${\phi}_k$, $r_k$, and ${\phi}_k'$ have
uniformly bounded second moments, it can be seen that for some constant
$c_1>0$, $\forall k\geq0$,
\begin{equation*}
\mathbb{E}[||M_{k+1}||^2|\mathcal{F}(k)]\leq
c_1(1+||\omega_k||^2+||{u}_k||^2+||{\theta}_k||^2).
\end{equation*}


Now Assumptions (A1) and (A2) of \cite{borkar2000ode} are verified.
Furthermore, Assumptions (TS) of \cite{borkar2000ode} is satisfied by our
conditions on the step-size sequences $\alpha_k$,$\zeta_k$, $\beta_k$. Thus,
by Theorem 2.2 of \cite{borkar2000ode} we obtain that
$||\omega_k-\omega^*||\rightarrow 0$ almost surely as $k\rightarrow \infty$.

Consider now the second time scale recursion (\ref{uvmtdc}).
Based on the above analysis, (\ref{uvmtdc}) can be rewritten as
% \begin{equation*}
%     {u}_{k+1}\leftarrow u_{k}+\zeta_{k}[\delta_{k}-\mathbb{E}[\delta_k|{u}_k,{\theta}_k] - {\phi}^{\top} (s_k) {u}_k]{\phi}(s_k).
% \end{equation*}
\begin{equation}
 \dot{{\theta}}(t) = 0,
    \label{thetavmtdcFaster}
\end{equation}
\begin{equation}
 \dot{u}(t) = \mathbb{E}[(\delta_t-\mathbb{E}[\delta_t|{u}(t),{\theta}(t)]){\phi}_t|{\theta}(t)] - \textbf{C}{u}(t).
    \label{uvmtdcFaster}
\end{equation}
The ODE (\ref{thetavmtdcFaster}) suggests that ${\theta}(t)\equiv {\theta}$ (i.e., a time-invariant parameter)
when viewed from the second fast timescale.
By the Hirsch lemma \cite{hirsch1989convergent}, it follows that
$||{\theta}_k-{\theta}||\rightarrow 0$ a.s. as $k\rightarrow \infty$ for some
${\theta}$ that depends on the initial condition ${\theta}_0$ of recursion
(\ref{thetavmtdc}). 

Consider now the recursion (\ref{uvmtdc}).
Let
$N_{k+1}=((\delta_k-\mathbb{E}[\delta_k]) - {\phi}_k {\phi}^{\top}_k {u}_k) -\mathbb{E}[((\delta_k-\mathbb{E}[\delta_k]) - {\phi}_k {\phi}^{\top}_k {u}_k)|\mathcal{I} (k)]$,
where $\mathcal{I}(k)=\sigma({u}_l,{\theta}_l,l\leq k;{\phi}_s,{\phi}_s',r_s,s<k)$, 
$k\geq 1$ are the sigma fields
generated by ${u}_0,{\theta}_0,{u}_{l+1},{\theta}_{l+1},{\phi}_l,{\phi}_l'$,
$0\leq l<k$.
It is easy to verify that $N_{k+1},k\geq0$ are integrable random variables that 
satisfy $\mathbb{E}[N_{k+1}|\mathcal{I}(k)]=0$, $\forall k\geq0$.
Because ${\phi}_k$, $r_k$, and ${\phi}_k'$ have
uniformly bounded second moments, it can be seen that for some constant
$c_2>0$, $\forall k\geq0$,
\begin{equation*}
\mathbb{E}[||N_{k+1}||^2|\mathcal{I}(k)]\leq
c_2(1+||{u}_k||^2+||{\theta}_k||^2).
\end{equation*}

Because ${\theta}(t)\equiv {\theta}$ from (\ref{thetavmtdcFaster}), the ODE pair (\ref{thetavmtdcFaster})-(\ref{uvmtdcFaster})
can be written as 
\begin{equation}
 \dot{{u}}(t) = \mathbb{E}[(\delta_t-\mathbb{E}[\delta_t|{\theta}]){\phi}_t|{\theta}] - \textbf{C}{u}(t).
    \label{uvmtdcFasterFinal}
\end{equation}
Now consider the function $h({u})=\mathbb{E}[\delta_t-\mathbb{E}[\delta_t|{\theta}]|{\theta}] -\textbf{C}{u}$, i.e., the
driving vector field of the ODE (\ref{uvmtdcFasterFinal}). For (\ref{uvmtdcFasterFinal}),
${u}^* = \textbf{C}^{-1}\mathbb{E}[(\delta-\mathbb{E}[\delta|{\theta}]){\phi}|{\theta}]$ is the unique globally asymptotically
stable equilibrium. Let $h_{\infty}({u})=-\textbf{C}{u}$.
For the ODE
\begin{equation}
 \dot{{u}}(t) = h_{\infty}({u}(t))= -\textbf{C}{u}(t),
    \label{uvmtdcInfty}
\end{equation}
the origin of (\ref{uvmtdcInfty}) is a globally asymptotically stable
equilibrium because $\textbf{C}$ is a positive definite matrix (because it is nonnegative definite and nonsingular).
Now Assumptions (A1) and (A2) of \cite{borkar2000ode} are verified.
Furthermore, Assumptions (TS) of \cite{borkar2000ode} is satisfied by our
conditions on the step-size sequences $\alpha_k$,$\zeta_k$, $\beta_k$. Thus,
by Theorem 2.2 of \cite{borkar2000ode} we obtain that
$||{u}_k-{u}^*||\rightarrow 0$ almost surely as $k\rightarrow \infty$.

Consider now the slower timescale recursion (\ref{thetavmtdc}). In the light of the above,
(\ref{thetavmtdc}) can be rewritten as 
\begin{equation}
 {\theta}_{k+1} \leftarrow {\theta}_{k} + \alpha_k (\delta_k -\mathbb{E}[\delta_k|{\theta}_k]) {\phi}_k\\
 - \alpha_k \gamma{\phi}'_{k}({\phi}^{\top}_k \textbf{C}^{-1}\mathbb{E}[(\delta_k -\mathbb{E}[\delta_k|{\theta}_k]){\phi}|{\theta}_k]).
\end{equation}
Let $\mathcal{G}(k)=\sigma({\theta}_l,l\leq k;{\phi}_s,{\phi}_s',r_s,s<k)$, 
$k\geq 1$ be the sigma fields
generated by ${\theta}_0,{\theta}_{l+1},{\phi}_l,{\phi}_l'$,
$0\leq l<k$. Let
\begin{equation*}
    \begin{array}{ccl}
 Z_{k+1}&=&((\delta_k -\mathbb{E}[\delta_k|{\theta}_k]) {\phi}_k - \gamma {\phi}'_{k}{\phi}^{\top}_k \textbf{C}^{-1}\mathbb{E}[(\delta_k -\mathbb{E}[\delta_k|{\theta}_k]){\phi}|{\theta}_k])\\ 
     & &-\mathbb{E}[((\delta_k -\mathbb{E}[\delta_k|{\theta}_k]) {\phi}_k - \gamma {\phi}'_{k}{\phi}^{\top}_k \textbf{C}^{-1}\mathbb{E}[(\delta_k -\mathbb{E}[\delta_k|{\theta}_k]){\phi}|{\theta}_k])|\mathcal{G}(k)]\\
    &=&((\delta_k -\mathbb{E}[\delta_k|{\theta}_k]) {\phi}_k - \gamma {\phi}'_{k}{\phi}^{\top}_k \textbf{C}^{-1}\mathbb{E}[(\delta_k -\mathbb{E}[\delta_k|{\theta}_k]){\phi}|{\theta}_k])\\
    & &-\mathbb{E}[(\delta_k -\mathbb{E}[\delta_k|{\theta}_k]) {\phi}_k|{\theta}_k] - \gamma\mathbb{E}[{\phi}' {\phi}^{\top}]\textbf{C}^{-1}\mathbb{E}[(\delta_k -\mathbb{E}[\delta_k|{\theta}_k]) {\phi}_k|{\theta}_k].
    \end{array}
\end{equation*}
It is easy to see that $Z_k$, $k\geq 0$ are integrable random variables and $\mathbb{E}[Z_{k+1}|\mathcal{G}(k)]=0$, $\forall k\geq0$. Further,
\begin{equation*}
\mathbb{E}[||Z_{k+1}||^2|\mathcal{G}(k)]\leq
c_3(1+||{\theta}_k||^2), k\geq 0
\end{equation*}
for some constant $c_3 \geq 0$, again because ${\phi}_k$, $r_k$, and ${\phi}_k'$ have
uniformly bounded second moments, it can be seen that for some constant.

Consider now the following ODE associated with (\ref{thetavmtdc}):
\begin{equation}
 \dot{{\theta}}(t) = (\textbf{I} - \mathbb{E}[\gamma {\phi}' {\phi}^{\top}]\textbf{C}^{-1})\mathbb{E}[(\delta -\mathbb{E}[\delta|{\theta}(t)]) {\phi}|{\theta}(t)].
    \label{thetavmtdcSlowerFinal}
\end{equation}
Let 
\begin{equation*}
\begin{array}{ccl}
 \vec{h}({\theta}(t))&=&(\textbf{I} - \mathbb{E}[\gamma {\phi}' {\phi}^{\top}]\textbf{C}^{-1})\mathbb{E}[(\delta -\mathbb{E}[\delta|{\theta}(t)]) {\phi}|{\theta}(t)]\\
    &=&(\textbf{C} - \mathbb{E}[\gamma {\phi}' {\phi}^{\top}])\textbf{C}^{-1}\mathbb{E}[(\delta -\mathbb{E}[\delta|{\theta}(t)]) {\phi}|{\theta}(t)]\\
    &=& (\mathbb{E}[{\phi} {\phi}^{\top}] - \mathbb{E}[\gamma {\phi}' {\phi}^{\top}])\textbf{C}^{-1}\mathbb{E}[(\delta -\mathbb{E}[\delta|{\theta}(t)]) {\phi}|{\theta}(t)]\\
    &=& \textbf{A}^{\top}\textbf{C}^{-1}(-\textbf{A}{\theta}(t)+{b}),
\end{array}
\end{equation*}
because $\mathbb{E}[(\delta -\mathbb{E}[\delta|{\theta}(t)]) {\phi}|{\theta}(t)]=-\textbf{A}{\theta}(t)+{b}$, where 
$\textbf{A} = \mathrm{Cov}({\phi},{\phi}-\gamma{\phi}')$, ${b}=\mathrm{Cov}(r,{\phi})$, and $\textbf{C}=\mathbb{E}[{\phi}{\phi}^{\top}]$

Therefore,
${\theta}^*=\textbf{A}^{-1}{b}$ can be seen to be the unique globally asymptotically
stable equilibrium for ODE (\ref{thetavmtdcSlowerFinal}).
Let $\vec{h}_{\infty}({\theta})=\lim_{r\rightarrow
\infty}\frac{\vec{h}(r{\theta})}{r}$. Then
$\vec{h}_{\infty}({\theta})=-\textbf{A}^{\top}\textbf{C}^{-1}\textbf{A}{\theta}$ is well-defined. 
Consider now the ODE
\begin{equation}
\dot{{\theta}}(t)=-\textbf{A}^{\top}\textbf{C}^{-1}\textbf{A}{\theta}(t).
\label{odethetavmtdcfinal}
\end{equation}

Because $\textbf{C}^{-1}$ is positive definite and $\textbf{A}$ has full rank (as it
is nonsingular by assumption), the matrix $\textbf{A}^{\top} \textbf{C}^{-1}\textbf{A}$ is also
positive definite. 
The ODE (\ref{odethetavmtdcfinal}) has the origin of its unique globally asymptotically stable equilibrium.
Thus, the assumption (A1) and (A2) are verified.

The proof is given above.
In the fastest time scale, the parameter $w$ converges to
$\mathbb{E}[\delta|{u}_k,{\theta}_k]$.
In the second fast time scale,
the parameter $u$ converges to $\textbf{C}^{-1}\mathbb{E}[(\delta-\mathbb{E}[\delta|{\theta}_k]){\phi}|{\theta}_k]$.
In the slower time scale,
the parameter ${\theta}$ converges to $\textbf{A}^{-1}{b}$.
\end{proof}

\subsection{Proof of Theorem 4}
\label{proofVMETD}
\begin{proof}
\label{th4proof}   
The proof of VMETD's convergence is also based on Borkar's Theorem   for
general stochastic approximation recursions with two time scales
\cite{borkar1997stochastic}. 

The  VMTD's solution is
${\theta}_{\text{VMETD}}=\textbf{A}_{\text{VMETD}}^{-1}{b}_{\text{VMETD}}$.
First, note that recursion (\ref{thetavmetd}) can be rewritten as
\begin{equation*}
 {\theta}_{k+1}\leftarrow {\theta}_k+\beta_k{\xi}(k),
\end{equation*}
 where
\begin{equation*}
 {\xi}(k)=\frac{\alpha_k}{\beta_k} (F_k \rho_k\delta_k - \omega_{k+1}){\phi}_k
\end{equation*}
 Due to the settings of step-size schedule $\alpha_k = o(\beta_k)$,
${\xi}(k)\rightarrow 0$ almost surely as $k\rightarrow\infty$. 
 That is the increments in iteration (\ref{omegavmetd}) are uniformly larger than
 those in (\ref{thetavmetd}), thus (\ref{omegavmetd}) is the faster recursion.
 Along the faster time scale, iterations of (\ref{thetavmetd}) and (\ref{omegavmetd})
 are associated with the ODEs system as follows:
\begin{equation}
 \dot{{\theta}}(t) = 0,
\label{vmetdthetaFast}
\end{equation}
\begin{equation}
 \dot{\omega}(t)=\mathbb{E}_{\mu}[F_t\rho_t\delta_t|{\theta}(t)]-\omega(t).
\label{vmetdomegaFast}
\end{equation}
 Based on the ODE (\ref{vmetdthetaFast}), ${\theta}(t)\equiv {\theta}$ when
 viewed from the faster timescale. 
 By the Hirsch lemma \cite{hirsch1989convergent}, it follows that
$||{\theta}_k-{\theta}||\rightarrow 0$ a.s. as $k\rightarrow \infty$ for some
${\theta}$ that depends on the initial condition ${\theta}_0$ of recursion
(\ref{thetavmetd}).
 Thus, the ODE pair (\ref{vmetdthetaFast})-(\ref{vmetdomegaFast}) can be written as
\begin{equation}
 \dot{\omega}(t)=\mathbb{E}_{\mu}[F_t\rho_t\delta_t|{\theta}]-\omega(t).
\label{vmetdomegaFastFinal}
\end{equation}
 Consider the function $h(\omega)=\mathbb{E}_{\mu}[F\rho\delta|{\theta}]-\omega$,
 i.e., the driving vector field of the ODE (\ref{vmetdomegaFastFinal}).
 It is easy to find that the function $h$ is Lipschitz with coefficient
$-1$.
 Let $h_{\infty}(\cdot)$ be the function defined by
 $h_{\infty}(\omega)=\lim_{x\rightarrow \infty}\frac{h(x\omega)}{x}$.
 Then  $h_{\infty}(\omega)= -\omega$,  is well-defined. 
 For (\ref{vmetdomegaFastFinal}), $\omega^*=\mathbb{E}_{\mu}[F\rho\delta|{\theta}]$
 is the unique globally asymptotically stable equilibrium.
 For the ODE
  \begin{equation}
 \dot{\omega}(t) = h_{\infty}(\omega(t))= -\omega(t),
 \label{vmetdomegaInfty}
 \end{equation}
 apply $\vec{V}(\omega)=(-\omega)^{\top}(-\omega)/2$ as its
 associated strict Liapunov function. Then,
 the origin of (\ref{vmetdomegaInfty}) is a globally asymptotically stable
 equilibrium.


 Consider now the recursion (\ref{omegavmetd}).
 Let
$M_{k+1}=(F_k\rho_k\delta_k-\omega_k)
 -\mathbb{E}_{\mu}[(F_k\rho_k\delta_k-\omega_k)|\mathcal{F}(k)]$,
 where $\mathcal{F}(k)=\sigma(\omega_l,{\theta}_l,l\leq k;{\phi}_s,{\phi}_s',r_s,s<k)$, 
$k\geq 1$ are the sigma fields
 generated by $\omega_0,{\theta}_0,\omega_{l+1},{\theta}_{l+1},{\phi}_l,{\phi}_l'$,
$0\leq l<k$.
 It is easy to verify that $M_{k+1},k\geq0$ are integrable random variables that 
 satisfy $\mathbb{E}[M_{k+1}|\mathcal{F}(k)]=0$, $\forall k\geq0$.
 Because ${\phi}_k$, $r_k$, and ${\phi}_k'$ have
 uniformly bounded second moments, it can be seen that for some constant
$c_1>0$, $\forall k\geq0$,
\begin{equation*}
 \mathbb{E}[||M_{k+1}||^2|\mathcal{F}(k)]\leq
 c_1(1+||\omega_k||^2+||{\theta}_k||^2).
\end{equation*}


 Now Assumptions (A1) and (A2) of \cite{borkar2000ode} are verified.
 Furthermore, Assumptions (TS) of \cite{borkar2000ode} are satisfied by our
 conditions on the step-size sequences $\alpha_k$, $\beta_k$. Thus,
 by Theorem 2.2 of \cite{borkar2000ode} we obtain that
$||\omega_k-\omega^*||\rightarrow 0$ almost surely as $k\rightarrow \infty$.

 Consider now the slower time scale recursion (\ref{thetavmetd}).
 Based on the above analysis, (\ref{thetavmetd}) can be rewritten as 

\begin{equation*}
    \begin{split}
 {\theta}_{k+1}&\leftarrow {\theta}_k+\alpha_k (F_k \rho_k\delta_k - \omega_k){\phi}_k -\alpha_k \omega_{k+1}{\phi}_k\\
&={\theta}_{k}+\alpha_k(F_k\rho_k\delta_k-\mathbb{E}_{\mu}[F_k\rho_k\delta_k|{\theta}_k]){\phi}_k\\
    &={\theta}_k+\alpha_k F_k \rho_k (R_{k+1}+\gamma {\theta}_k^{\top}{\phi}_{k+1}-{\theta}_k^{\top}{\phi}_k){\phi}_k -\alpha_k \mathbb{E}_{\mu}[F_k \rho_k \delta_k]{\phi}_k\\
    &= {\theta}_k+\alpha_k \{\underbrace{(F_k\rho_kR_{k+1}-\mathbb{E}_{\mu}[F_k\rho_k R_{k+1}]){\phi}_k}_{{b}_{\text{VMETD},k}}
 -\underbrace{(F_k\rho_k{\phi}_k({\phi}_k-\gamma{\phi}_{k+1})^{\top}-{\phi}_k\mathbb{E}_{\mu}[F_k\rho_k ({\phi}_k-\gamma{\phi}_{k+1})]^{\top})}_{\textbf{A}_{\text{VMETD},k}}{\theta}_k\}
\end{split}
\end{equation*}

 Let $\mathcal{G}(k)=\sigma({\theta}_l,l\leq k;{\phi}_s,{\phi}_s',r_s,s<k)$, 
$k\geq 1$ be the sigma fields
 generated by ${\theta}_0,{\theta}_{l+1},{\phi}_l,{\phi}_l'$,
$0\leq l<k$.
 Let 
$
 Z_{k+1} = Y_{k}-\mathbb{E}[Y_{k}|\mathcal{G}(k)],
$
 where
\begin{equation*}
 Y_{k}=(F_k\rho_k\delta_k-\mathbb{E}_{\mu}[F_k\rho_k\delta_k|{\theta}_k]){\phi}_k.
\end{equation*}
 Consequently,
\begin{equation*}
\begin{array}{ccl}
 \mathbb{E}_{\mu}[Y_k|\mathcal{G}(k)]&=&\mathbb{E}_{\mu}[(F_k\rho_k\delta_k-\mathbb{E}_{\mu}[F_k\rho_k\delta_k|{\theta}_k]){\phi}_k|\mathcal{G}(k)]\\
&=&\mathbb{E}_{\mu}[F_k\rho_k\delta_k{\phi}_k|{\theta}_k]
 -\mathbb{E}_{\mu}[\mathbb{E}_{\mu}[F_k\rho_k\delta_k|{\theta}_k]{\phi}_k]\\
&=&\mathbb{E}_{\mu}[F_k\rho_k\delta_k{\phi}_k|{\theta}_k]
 -\mathbb{E}_{\mu}[F_k\rho_k\delta_k|{\theta}_k]\mathbb{E}_{\mu}[{\phi}_k]\\
&=&\mathrm{Cov}(F_k\rho_k\delta_k|{\theta}_k,{\phi}_k),
\end{array}
\end{equation*}
 where $\mathrm{Cov}(\cdot,\cdot)$ is a covariance operator.

 Thus,
 \begin{equation*}
\begin{array}{ccl}
 Z_{k+1}&=&(F_k\rho_k\delta_k-\mathbb{E}[\delta_k|{\theta}_k]){\phi}_k-\mathrm{Cov}(F_k\rho_k\delta_k|{\theta}_k,{\phi}_k).
\end{array}
\end{equation*}
 It is easy to verify that $Z_{k+1},k\geq0$ are integrable random variables that 
 satisfy $\mathbb{E}[Z_{k+1}|\mathcal{G}(k)]=0$, $\forall k\geq0$.
 Also, because ${\phi}_k$, $r_k$, and ${\phi}_k'$ have
 uniformly bounded second moments, it can be seen that for some constant
$c_2>0$, $\forall k\geq0$,
\begin{equation*}
 \mathbb{E}[||Z_{k+1}||^2|\mathcal{G}(k)]\leq
 c_2(1+||{\theta}_k||^2).
\end{equation*}

 Consider now the following ODE associated with (\ref{thetavmetd}):
\begin{equation}
\begin{array}{ccl}
 \dot{{\theta}}(t)&=&-\textbf{A}_{\text{VMETD}}{\theta}(t)+{b}_{\text{VMETD}}.
\end{array}
\label{odethetavmetd}
\end{equation}
\begin{equation}
    \begin{split}
 \textbf{A}_{\text{VMETD}}&=\lim_{k \rightarrow \infty} \mathbb{E}[\textbf{A}_{\text{VMETD},k}]\\
&= \lim_{k \rightarrow \infty} \mathbb{E}_{\mu}[F_k \rho_k {\phi}_k ({\phi}_k - \gamma {\phi}_{k+1})^{\top}]- \lim_{k\rightarrow \infty} \mathbb{E}_{\mu}[  {\phi}_k]\mathbb{E}_{\mu}[F_k \rho_k ({\phi}_k - \gamma {\phi}_{k+1})]^{\top}\\  
% &= \lim_{k \rightarrow \infty} \mathbb{E}_{\mu}[\underbrace{{\phi}_k}_{X}\underbrace{F_k \rho_k  ({\phi}_k - \gamma {\phi}_{k+1})^{\top}}_{Y}]- \lim_{k\rightarrow \infty} \mathbb{E}_{\mu}[  {\phi}_k]\mathbb{E}_{\mu}[F_k \rho_k ({\phi}_k - \gamma {\phi}_{k+1})]^{\top}\\  
&= \lim_{k \rightarrow \infty} \mathbb{E}_{\mu}[{\phi}_kF_k \rho_k  ({\phi}_k - \gamma {\phi}_{k+1})^{\top}]- \lim_{k\rightarrow \infty} \mathbb{E}_{\mu}[  {\phi}_k]\mathbb{E}_{\mu}[F_k \rho_k ({\phi}_k - \gamma {\phi}_{k+1})]^{\top}\\ 
&= \lim_{k \rightarrow \infty} \mathbb{E}_{\mu}[{\phi}_kF_k \rho_k ({\phi}_k - \gamma {\phi}_{k+1})^{\top}]- \lim_{k \rightarrow \infty} \mathbb{E}_{\mu}[ {\phi}_k]\lim_{k \rightarrow \infty}\mathbb{E}_{\mu}[F_k \rho_k ({\phi}_k - \gamma {\phi}_{k+1})]^{\top}\\   
&=\sum_{s} f(s) {\phi}(s)({\phi}(s) - \gamma \sum_{s'}[\textbf{P}_{\pi}]_{ss'}{\phi}(s'))^{\top} - \sum_{s} d_{\mu}(s) {\phi}(s) * \sum_{s} f(s)({\phi}(s) - \gamma \sum_{s'}[\textbf{P}_{\pi}]_{ss'}{\phi}(s'))^{\top}  \\
&={{\Phi}}^{\top} \textbf{F} (\textbf{I} - \gamma \textbf{P}_{\pi}) {\Phi} - {{\Phi}}^{\top} \textbf{d}_{\mu} \textbf{f}^{\top} (\textbf{I} - \gamma \textbf{P}_{\mu}) {\Phi}  \\
&={{\Phi}}^{\top} (\textbf{F} - \textbf{d}_{\mu} \textbf{f}^{\top}) (\textbf{I} - \gamma \textbf{P}_{\pi}){{\Phi}} \\
&={{\Phi}}^{\top} (\textbf{F} (\textbf{I} - \gamma \textbf{P}_{\pi})-\textbf{d}_{\mu} \textbf{f}^{\top} (\textbf{I} - \gamma \textbf{P}_{\pi})){{\Phi}} \\
&={{\Phi}}^{\top} (\textbf{F} (\textbf{I} - \gamma \textbf{P}_{\pi})-\textbf{d}_{\mu} \textbf{d}_{\mu}^{\top} ){{\Phi}} \\
    \end{split}
\end{equation}
\begin{equation}
    \begin{split}
 {b}_{\text{VMETD}}&=\lim_{k \rightarrow \infty} \mathbb{E}[{b}_{\text{VMETD},k}]\\
&= \lim_{k \rightarrow \infty} \mathbb{E}_{\mu}[F_k\rho_kR_{k+1}{\phi}_k]- \lim_{k\rightarrow \infty} \mathbb{E}_{\mu}[{\phi}_k]\mathbb{E}_{\mu}[F_k\rho_kR_{k+1}]\\  
&= \lim_{k \rightarrow \infty} \mathbb{E}_{\mu}[{\phi}_kF_k\rho_kR_{k+1}]- \lim_{k\rightarrow \infty} \mathbb{E}_{\mu}[  {\phi}_k]\mathbb{E}_{\mu}[{\phi}_k]\mathbb{E}_{\mu}[F_k\rho_kR_{k+1}]\\ 
&= \lim_{k \rightarrow \infty} \mathbb{E}_{\mu}[{\phi}_kF_k\rho_kR_{k+1}]- \lim_{k \rightarrow \infty} \mathbb{E}_{\mu}[ {\phi}_k]\lim_{k \rightarrow \infty}\mathbb{E}_{\mu}[F_k\rho_kR_{k+1}]\\  
&=\sum_{s} f(s) {\phi}(s)r_{\pi} - \sum_{s} d_{\mu}(s) {\phi}(s) * \sum_{s} f(s)r_{\pi}  \\
&={{\Phi}}^{\top}(\textbf{F}-\textbf{d}_{\mu} \textbf{f}^{\top})\textbf{r}_{\pi} \\
    \end{split}
\end{equation}
 Let $\vec{h}({\theta}(t))$ be the driving vector field of the ODE
 (\ref{odethetavmetd}).
\begin{equation*}
 \vec{h}({\theta}(t))=-\textbf{A}_{\text{VMETD}}{\theta}(t)+{b}_{\text{VMETD}}.
\end{equation*}

 An ${\Phi}^{\top}{\text{X}}{\Phi}$ matrix of this
 form will be positive definite whenever the matrix ${\text{X}}$ is positive definite.
 Any matrix ${\text{X}}$ is positive definite if and only if
 the symmetric matrix ${\text{S}}={\text{X}}+{\text{X}}^{\top}$ is positive definite. 
 Any symmetric real matrix ${\text{S}}$ is positive definite if the absolute values of
 its diagonal entries are greater than the sum of the absolute values of the corresponding
 off-diagonal entries\cite{sutton2016emphatic}. 

\begin{equation}
    \label{rowsum}
    \begin{split}
 (\textbf{F} (\textbf{I} - \gamma \textbf{P}_{\pi})-\textbf{d}_{\mu} \textbf{d}_{\mu}^{\top} )\textbf{1}
    &=\textbf{F} (\textbf{I} - \gamma \textbf{P}_{\pi})\textbf{1}-\textbf{d}_{\mu} \textbf{d}_{\mu}^{\top} \textbf{1}\\
    &=\textbf{F}(\textbf{1}-\gamma \textbf{P}_{\pi} \textbf{1})-\textbf{d}_{\mu} \textbf{d}_{\mu}^{\top} \textbf{1}\\
    &=(1-\gamma)\textbf{F}\textbf{1}-\textbf{d}_{\mu} \textbf{d}_{\mu}^{\top} \textbf{1}\\
    &=(1-\gamma)\textbf{f}-\textbf{d}_{\mu} \textbf{d}_{\mu}^{\top} \textbf{1}\\
    &=(1-\gamma)\textbf{f}-\textbf{d}_{\mu} \\
    &=(1-\gamma)(\textbf{I}-\gamma\textbf{P}_{\pi}^{\top})^{-1}\textbf{d}_{\mu}-\textbf{d}_{\mu} \\
    &=(1-\gamma)[(\textbf{I}-\gamma\textbf{P}_{\pi}^{\top})^{-1}-\textbf{I}]\textbf{d}_{\mu} \\
    &=(1-\gamma)[\sum_{t=0}^{\infty}(\gamma\textbf{P}_{\pi}^{\top})^{t}-\textbf{I}]\textbf{d}_{\mu} \\
    &=(1-\gamma)[\sum_{t=1}^{\infty}(\gamma\textbf{P}_{\pi}^{\top})^{t}]\textbf{d}_{\mu} > 0 \\
    \end{split}
    \end{equation}
\begin{equation}
    \label{columnsum}
    \begin{split}
 \textbf{1}^{\top}(\textbf{F} (\textbf{I} - \gamma \textbf{P}_{\pi})-\textbf{d}_{\mu} \textbf{d}_{\mu}^{\top} )
    &=\textbf{1}^{\top}\textbf{F} (\textbf{I} - \gamma \textbf{P}_{\pi})-\textbf{1}^{\top}\textbf{d}_{\mu} \textbf{d}_{\mu}^{\top} \\
    &=\textbf{d}_{\mu}^{\top}-\textbf{1}^{\top}\textbf{d}_{\mu} \textbf{d}_{\mu}^{\top} \\
    &=\textbf{d}_{\mu}^{\top}- \textbf{d}_{\mu}^{\top} \\
    &=0
    \end{split}
\end{equation}
 (\ref{rowsum}) and (\ref{columnsum}) show that the matrix $\textbf{F} (\textbf{I} - \gamma \textbf{P}_{\pi})-\textbf{d}_{\mu} \textbf{d}_{\mu}^{\top}$ of
 diagonal entries are positive and its off-diagonal entries are negative. So each row sum plus the corresponding column sum is positive. 
 So $\textbf{A}_{\text{VMETD}}$ is positive definite.



 Therefore,
${\theta}^*=\textbf{A}_{\text{VMETD}}^{-1}{b}_{\text{VMETD}}$ can be seen to be the unique globally asymptotically
 stable equilibrium for ODE (\ref{odethetavmetd}).
 Let $\vec{h}_{\infty}({\theta})=\lim_{r\rightarrow
\infty}\frac{\vec{h}(r{\theta})}{r}$. Then
$\vec{h}_{\infty}({\theta})=-\textbf{A}_{\text{VMETD}}{\theta}$ is well-defined. 
 Consider now the ODE
\begin{equation}
 \dot{{\theta}}(t)=-\textbf{A}_{\text{VMETD}}{\theta}(t).
\label{odethetavmetdfinal}
\end{equation}
 The ODE (\ref{odethetavmetdfinal}) has the origin of its unique globally asymptotically stable equilibrium.
 Thus, the assumption (A1) and (A2) are verified.
    \end{proof}



\section{Experimental details}
\label{experimentaldetails}
The 2-state version of Baird's off-policy counterexample: All learning rates follow linear learning rate decay.
For TD algorithm, $\frac{\alpha_k}{\omega_k}=4$ and $\alpha_0 = 0.1$.
For TDC algorithm, $\frac{\alpha_k}{\zeta_k}=5$ and $\alpha_0 = 0.1$.
For VMTDC algorithm, $\frac{\alpha_k}{\zeta_k}=5$, $\frac{\alpha_k}{\omega_k}=4$,and $\alpha_0 = 0.1$.
For VMTD algorithm, $\frac{\alpha_k}{\omega_k}=4$ and $\alpha_0 = 0.1$.

The 2-state version of Baird's off-policy counterexample: All learning rates follow linear learning rate decay.
For TD algorithm, $\frac{\alpha_k}{\omega_k}=4$ and $\alpha_0 = 0.1$.
For TDC algorithm, $\frac{\alpha_k}{\zeta_k}=5$ and $\alpha_0 = 0.1$.For ETD algorithm, $\alpha_0 = 0.1$.
For VMTDC algorithm, $\frac{\alpha_k}{\zeta_k}=5$, $\frac{\alpha_k}{\omega_k}=4$,and $\alpha_0 = 0.1$.For VMETD algorithm, $\frac{\alpha_k}{\omega_k}=4$ and $\alpha_0 = 0.1$.
For VMTD algorithm, $\frac{\alpha_k}{\omega_k}=4$ and $\alpha_0 = 0.1$.

For all policy evaluation experiments, each experiment 
is independently run 100 times.

For the four control experiments: The learning rates for each 
algorithm in all experiments are shown in Table \ref{lrofways}.
For all control experiments, each experiment is independently run 50 times.

\begin{table*}[htb]
    \centering
    \caption{Learning rates ($lr$) of four control experiments.}
    \label{lrofways}
    \begin{tabular}{ccccc}
      \toprule
      \multicolumn{1}{c}{Algorithms ($lr$)} & Maze & Cliff walking & Mountain Car & Acrobot \\
      \midrule
      Sarsa($\alpha$) & 0.1 & 0.1 & 0.1 & 0.1 \\
      GQ($\alpha,\zeta$) & 0.1, 0.003 & 0.1, 0.004 & 0.1, 0.01 & 0.1, 0.01 \\
      EQ($\alpha$) & 0.006 & 0.005 & 0.001 & 0.0005 \\
      VMSarsa($\alpha,\beta$) & 0.1, 0.001 & 0.1, 1e-4 & 0.1, 1e-4 & 0.1, 1e-4 \\
      VMGQ($\alpha,\zeta,\beta$) & 0.1, 0.001, 0.001 & 0.1, 0.005, 1e-4 & 0.1, 5e-4, 1e-4 & 0.1, 5e-4, 1e-4 \\
      VMEQ($\alpha,\beta$) & 0.001, 0.0005 & 0.005, 0.0001 & 0.001, 0.0001 & 0.0005, 0.0001 \\
      Q-learning($\alpha$) & 0.1 & 0.1 & 0.1 & 0.1 \\
      VMQ($\alpha,\beta$) & 0.1, 0.001 & 0.1, 1e-4 & 0.1, 1e-4 & 0.1, 1e-4 \\
      \bottomrule
    \end{tabular}
  \end{table*}
% \end{document}


% % As a general rule, do not put math, special symbols or citations
% % in the abstract
\begin{abstract}
[Background] Systematic literature reviews (SLRs) are essential for synthesizing evidence in Software Engineering (SE), but keeping them up-to-date requires substantial effort. Study selection, one of the most labor-intensive steps, involves reviewing numerous studies and requires multiple reviewers to minimize bias and avoid loss of evidence.
[Objective] This study aims to evaluate if Machine Learning (ML) text classification models can support reviewers in the study selection for SLR updates.
[Method] We reproduce the study selection of an SLR update performed by three SE researchers. We trained two supervised ML models (Random Forest and Support Vector Machines) with different configurations using data from the original SLR. We calculated the study selection effectiveness of the ML models for the SLR update in terms of precision, recall, and F-measure. We also compared the performance of human-ML pairs with human-only pairs when selecting studies.
[Results] The ML models achieved a modest F-score of 0.33, which is insufficient for reliable automation. However, we found that such models can reduce the study selection effort by 33.9\% without loss of evidence (keeping a 100\% recall). Our analysis also showed that the initial screening by pairs of human reviewers produces results that are much better aligned with the final SLR update result. [Conclusion] Based on our results, we conclude that although ML models can help reduce the effort involved in SLR updates, achieving rigorous and reliable outcomes still requires the expertise of experienced human reviewers for the initial screening phase.


\end{abstract}

% no keywords
\begin{IEEEkeywords}
Systematic Review Automation, Selection of Studies, Machine Learning, Systematic Literature Review Update
\end{IEEEkeywords}
% \keywords{Systematic Review Automation, Selection of Studies, Machine Learning}

% \received{20 February 2007}
% \received[revised]{12 March 2009}
% \received[accepted]{5 June 2009}


The increasing reliance on LLMs for multimodal tasks across far-reaching sectors such as healthcare, finance, and manufacturing underscores the need to assess the accuracy and reliability of the information they generate. Vision-Language Models (VLM) have achieved state-of-the-art (SoTA) performance on Visual Question-Answering (VQA) benchmarks, and these models often utilize Retrieval-Augmented Generation (RAG) to maintain factual accuracy and relevance in a dynamic information environment. However, this has led to uncertainty in the information the LLM bases its answer on, as it may choose between parametric memory and retrieved sources. When models rely on memorized information instead of dynamically retrieving information, they may inadvertently propagate outdated or incorrect information, causing serious legal and ethical risks and undermining trust and reliability in AI systems \citep{huang2023survey}.
% The ability to strike a balance between generalization and specialization in AI systems is therefore crucial for ensuring the safe, reliable use of these technologies in real-world applications.

Despite these concerns, the way that Vision-Language models (VLMs) memorize and retrieve information, particularly in complex multimodal tasks, remains under-explored. Current research often focuses on either the general capabilities of large language models (LLMs) or the specialized retrieval mechanisms in retrieval augmented generation systems (RAG) \citep{incontext_rag,chen_murag_2022,liu_universal_2023}. Particularly in the context of multimodal retrieval and multihop reasoning, few studies analyze the tradeoff between finetuning for specialized tasks and zero-shot prompting for general-purpose vision-language capabilities. A lack of consensus on how to approach this tradeoff motivates the development of measures to quantify reliance on parametric memory, as well as metrics for quantifying the potential performance impact of extending LLMs with RAG systems.

To address this gap, we investigate how multimodal QA models balance accuracy with memorization on the WebQA benchmark. We compare finetuned multimodal systems against zero-shot VLMs, analyzing how retrieval performance influences QA accuracy. In particular, we focus on cases where retrieval fails, allowing us to measure reliance on parametric memory through two proposed metrics---the \ppr (\PPR) which quantifies how much model accuracy is influenced by retrieval quality, contrasting performance in best-case versus worst-case retrieval scenarios, and the \ucr (\UCR) which measures how often correct QA responses are generated when the retriever fails, providing a proxy for memorization.

To enable this analysis, we make several methodological contributions. For the finetuned QA models, we investigate Vision-Transformer (ViT) architectures, which allow for multihop reasoning over multiple sources. To investigate the impact of retrieval performance on trained LMs, we propose a variable-input Fusion-in-Decoder (FiD) model \cite{tanaka_slidevqa_2023, nlvr2}, building upon the VoLTA architecture \citep{pramanick_volta_2023}. For the zero-shot case, we build upon previous research on In-Context Retrieval \citep{incontext_rag} by demonstrating that LLMs such as GPT-4o are capable of performing the final ranking step of the retrieval process. In doing so, we find that GPT-4o, a general-purpose LLM, achieves SoTA performance on the WebQA task, outperforming existing finetuned RAG models by a significant margin (7\% higher accuracy). 

Crucially, our results reveal that while retrieval-augmented models reduce memorization, the training paradigm plays an important role. Finetuned models exhibit higher reliance on parametric memory, whereas zero-shot RAG approaches have lower memorization scores at the cost of accuracy. This suggests that while retrieval modules may mitigate the risks associated with outdated or incorrect information, SoTA performance requires that they be coupled with specialized QA models. Our memorization measures contribute to the development of transparent and reliable AI systems, particularly in applications where the sourcing of up-to-date, factual information is critical.



% We investigate the impact of question complexity on the ability of these models to integrate multiple data sources—such as images, text, and external retrievers—and produce coherent and accurate answers. We also explore whether in-context retrieval can be a viable alternative to traditional retrieval-augmented systems, offering a more streamlined approach to multimodal QA.

% To achieve this, we first compare zero-shot prompting multimodal LLMs with finetuned multimodal systems. We evaluate both types of models on the WebQA benchmark, a dataset designed for complex question answering that requires reasoning across both image and text sources. For the finetuned models, we use a Fusion-in-Decoder (FiD) architecture, which allows for multihop reasoning over multiple sources. Additionally, we introduce the concept of In-Context Retrieval Language Modeling (RLM), where the LLM itself performs retrieval tasks without the need for external retrievers. This method builds upon existing research in in-context learning  and aims to explore the viability of LLMs retrieving relevant sources and generating accurate answers directly from their context window.

% In order to investigate source utilization in finetuned multimodal models and LLMs, three lines of inquiry are established; 
% \begin{itemize}
%     \item Study 1: retrieval vs QA performance on webQA (motivating example, does QA answer correctly even with incorrect sources?)
%     \item Study 2: performance on adversarial examples where parametric knowledge would be incorrect by design
%     \item Study 3: improving performance on adversarial examples by fine-tuning (i.e model robustness)
% \end{itemize}

% Note, there is one weakness in this plan which is tying in the work we've already done. 
% If we added something from adversarial generation to the retrieval experiment (like a combination of study 1 + 3) it would be complete. So for instance we could try fine-tuning the retriever with adversarial examples (and not just the QA model)

% \begin{figure}
%     \centering
%     \includegraphics[width=0.95\linewidth]{figures/segmentation/webqa_segment_infill.png}
%     \caption{Example of the segmentation substitution pipeline from the WebQA task.}
%     % d5c76d760dba11ecb1e81171463288e9
%     \label{fig:seg_sub_pipeline}
% \end{figure}



% Retrieval augmented generation (RAG) with zero-shot prompting and fine-tuning Large Language Models (LLMs) have become the go-to methods for tasks relying on information retrieval and text generation. In many cases the LLMs parametric memory can sufficiently generalize to answer questions without being provided with retrieval mechanisms for out-of-domain knowledge. However, LLMs often hallucinate and provide wrong information in certain scenarios. This problem is amplified even further on open-domain Question Answering (QA) tasks involving multiple modalities. Grounded text generation using retrieved sources \citep{lewis2021retrievalaugmented} has been extensively studied for text-to-text QA tasks, but its application in multimodal settings has not been studied as much.


% Multimodal reasoning and question answering have gained prominence in recent research endeavors, with an increasing emphasis on handling various forms of data, particularly text and images. In this study, we address a specific gap in the existing literature by focusing on the development of a versatile multihop model capable of accommodating varying numbers of input images.

% Our motivation for this research lies in the growing complexity of answering questions using information on the web, where the challenge of navigating the open-domain setting is further complicated by the presence of multiple modalities and sometimes requires reasoning over multiple sources. WebQA is an ideal dataset on which to compare performance of finetuned RAG systems against general purpose LLMs; it is multimodal, with correct answers requiring reasoning over image and text sources. It is multihop, requiring a complex reasoning process over multiple sources. Finally, WebQA questions from different categories can be broken down into subdomains to analyze performance over domains of varying cardinality.

% Motivated by the real-world challenges of building retrieval and question answering (QA) systems, we design and finetune a closed domain, multimodal, multihop QA model, that is capable of reasoning over a varying number of sources taken as input from an external retriever module. This research contributes to the relatively underexplored domain of multihop reasoning across various input sources and modalities. Our goal is to explore the challenges posed by these scenarios and develop strategies that enable QA models to retrieve relevant information, conduct logical or numerical reasoning across diverse modalities, and generate coherent responses in natural language. To our knowledge, this is the first application of the Fusion-in-Decoder (FiD) architecture \cite{tanaka_slidevqa_2023, nlvr2} that is shown to work with a variable number of inputs, enabling multi-hop reasoning over sources.

% In-Context Learning refers to the ability of LLMs to perform any task by simply providing examples in the input prompt \citep{dong2022survey,min2022rethinking}. Inspired by this research, we propose a method to use the LLM itself as a multimodal retriever, potentially eschewing the requirement of a distinct retrieval module, thereby allowing the design of simpler retrieval-augmented QA systems. We dub this method In-Context Retrieval Language Modeling (RLM). To the best of the authors knowledge, In-Content RLM is disparate from other retrieval augmented approaches which utilize external retrieval modules \citep{incontext_rag,chen_murag_2022,liu_universal_2023}. Despite being a natural extension of In-Context learning, In-Context RLM has not yet been studied empirically.

% To expand on our contribution of In-Context Retrieval, this stems from the well-researched in-context learning of LLMs. In-context learning is the ability of a model to perform any task given a sufficient context window \citep{dong2022survey,min2022rethinking}. Such tasks could include retrieval and ranking, but typically, the go-to solution for tasks requiring retrieval has been RAG. To the best of the authors knowledge, In-Context Retrieval is distinct from In-Context Retrieval Augmented Language Modelling (RALM), and despite being a natural extension of In-Context learning, In-Context Retrieval has not yet been shown empirically.

% Finally, we explore the tradeoff between using zero-shot prompting LLMs and the fine-tuning approach. While we find that, overall, GPT-4o obtains SoTA performance on the WebQA task, outperforming the accuracy of existing finetuned RAG approaches by 7\%, finetuned approaches still perform better on more restricted subdomains\footnote{``In-Context RLM" @ \url{https://eval.ai/web/challenges/challenge-page/1255/leaderboard/3168}}. Finally, we validate that GPT-4o is relying on retrieval abilities to solve the task; we find that GPT-4o is capable of retrieving relevant sources in the presence of distractors and furthermore, when GPT-4o fails to retrieve correct sources, it answers incorrectly 75\% of the time, meaning that it is not relying on parametric memory for this task.

% \paragraph{Contributions}
% Based on our experimentation and analysis on the WebQA benchmark, we make the following contributions:
% \begin{itemize}
%     \item Propose a new architecture for multimodal multihop QA that takes variable number of input sources inspired by the Fusion-in-Decoder method.
%     \item Comparison of general purpose LLMs vs specialized models on the WebQA benchmark.
%     \item Observation of In-Context Multimodal Retrieval abilities of GPT-4o and that it does not rely on parametric memory for multimodal QA.
%     \item Analysis of relationship between retrieval and QA task performance.
%     \item Analysis of task and query complexity on the performance of retrieval and QA tasks.
% \end{itemize}
















% Throughout this paper, we will present our methodology, experiments, and findings, emphasizing our approach to multihop reasoning over varying numbers of input images. We believe that our work contributes to a deeper understanding of multimodal reasoning and has the potential to enhance the capabilities of question-answering systems in the intricate, multimodal landscape of web-based information.
\section{Background and Motivation}
\label{sec:background}

We introduce the background on serverless workload serving and motivate the use of runtime resource adaptation to address resource inefficiency in existing serverless platforms.

\subsection{Resource Inefficiency with Early Binding}
% In current serverless platforms, developers are required to specify immutable sizes for their deployed functions.
% Then, providers consider functions' runtime workloads  (e.g., concurrency)  and resource usage to scale out/in their instances.
% Moreover, due to high runtime variability, functions must size their functions for worst-case scenarios.
% This, however, incurs considerable resource inefficiency.
Current serverless workflow platforms (e.g., AWS Step Functions~\cite{aws-step-function} and Azure Durable Functions~\cite{azure-durable-function}) offer the opportunity for developers to build various applications with advanced logic like chaining, branching, and parallel execution.
These applications can be defined by JSON-based structured languages (e.g., Amazon States Language) or other programming languages.
Meanwhile, developers require to specify resource configurations, including memory size, CPU cores, and scaling options, for individual functions---an early-binding approach.
The serverless platform is responsible for monitoring the workload intensity and resource usage at runtime and scaling out/in function instances accordingly.
To account for potential runtime variability, developers must size the functions in their application workflow accounting for the worst case in order to provide SLO guarantees over the end-to-end delay of request processing, e.g., the 99th percentile (P99) of the end-to-end delay must be within a given target. 
After deployment, the function sizes become immutable. The worst case is not representative and over-shoots most of the time, leading to resource inefficiency. 


To verify this claim, we conduct a data-driven analysis with a dataset from Microsoft Azure Functions~\cite{azure-dataset} to explicitly demonstrate the resource inefficiency issue. % , deriving from the worst-case based early bind.
To quantify the inefficiency, we define a metric called \emph{slack}---the margin between the actual execution time and the SLO, which is calculated as $1-l/T$ with $l$ and $T$ representing end-to-end latency and SLO, respectively.
Under certain SLO defined with P99 latency as done by existing works (e.g., \cite{osdi22-orion,mac22-wisefuse}),  we can see from Figure \ref{fig:bg:slack} that more than 60\% function invocations have slacks over 60\%.
Particularly, we analyze slacks of the top 100 most popular functions, whose invocations account for 81.6\% of the total function invocations. % (depicted in Figure~\ref{fig:bg:popular_func}) of overall invocations.
The result shows that only 20\% of the invocations of the popular functions (blue line in Figure~\ref{fig:bg:slack}) have slacks less than 40\%.
This means the majority of requests are processed faster than necessary.
Notably, in DAG-based workloads (i.e., Azure Durable Functions), the resource inefficiency further deteriorates wherein the ratio between the 95th percentile and 50th percentile is by up to three times \cite{mac22-wisefuse}.

% \begin{figure}[t!]
% \centering
% \includegraphics[width=0.25\textwidth]{./figure/motivation/Average_P99_cdf_top=100.pdf}
% \vspace{-0.3cm}
% \caption{Sufficient function slacks in production traces.}
% \label{fig:bg:slack}
% \end{figure}

\subsection{Runtime Dynamics}
\label{sec:bg:worst-case}

The resource inefficiency caused by the large slack can be mainly attributed to the over-provisioning of resources by the developer. This is to ensure that the SLO is guaranteed even in the worst case (i.e., P99). However, normal cases deviate from the worst case significantly due to runtime dynamics. 
In particular, we observe that functions face two major dynamic factors at runtime: varying working sets and inevitable performance interference. These two factors contribute significantly to the variance of the function execution time. 
% Functions face two remarkably dynamic factors at runtime: working sets and performance interference, which lead to considerable variance of execution latency.

\begin{figure*}[!t]
	\centering
	\subfloat[]{
		\includegraphics[width=0.24\textwidth]{./figure/motivation/Average_P99_cdf_top=100.pdf}
		\label{fig:bg:slack}
	}
	\hspace{8mm}
	\subfloat[]{
		\includegraphics[width=0.25\textwidth]{./figure/motivation/function-latency-ml-analyze-varying-worksets.pdf}
		\label{fig:bg:ml-func-latency}
	}
	\hspace{8mm}
	\subfloat[]{
	\includegraphics[width=0.28\textwidth]{./figure/motivation/coresident-perf.pdf}   
	\label{fig:bg:perf-inteference}
	}
	%\vspace{-0.1cm}
	\caption{(a) slacks of function invocations in production traces, (b) function latency variance caused by varying input worksets for functions object detection (OD), question answering (QA), and and text-to-speech (TS), respectively,
 (c) performance interference attributed to co-location of homogeneous function with different dominant resource demands.}
 %\vspace{-0.4cm}
\end{figure*}

%'ml-analyze':{'text-to-speech': 'text-to-speech', 'question-answer': 'question answer',
%                      'object-detection': 'object detection'
\textbf{\textit{Varying working sets.}} 
The working set, i.e., input data like videos, audios, and texts, can have varying sizes.
Taking Microsoft Azure Function Blobs (storage service) as an example, their data size difference can be as high as nine orders of magnitude~\cite{azure-function-blob}.
Such a large difference results in substantial variance of the execution time even for the same function~\cite{socc21-faast,eurosys21-ofc}.
Specifically, we measure the execution time of three functions under different working sets (detailed in \S\ref{exp:setup}).
Figure~\ref{fig:bg:ml-func-latency} illustrates the results, where we can observe a variance of up to 3.8 times in function execution caused by varying working set sizes.

% \begin{figure}[t!]
% \centering
% \includegraphics[width=0.25\textwidth]{././figure/motivation/function-latency-ml-analyze-varying-worksets.pdf}
% \vspace{-0.3cm}
% \caption{Function latency variance caused by varying input worksets}
% \label{fig:bg:ml-func-latency}
% \end{figure}	

\textbf{\textit{Performance interference.}}
% On the other hand, function deployment, which decides when and where to deploy functions, is completely undertaken by providers.
For simplicity and security, commercial serverless platforms, such as Alibaba Function Compute, Microsoft Azure, and AWS Lambda, exclusively deploy function instances belonging to the same tenant, or even belonging to the same function, in the same virtual machine~\cite{socc22-owl,atc18-peek-bench}.
For example, the empirical study in~\cite{socc22-owl} shows that in Alibaba Function Compute 65\% of the virtual machines exclusively deploy instances of the same function.
This co-location of homogeneous function instances, however, can incur severe resource contention on the same resource dimensions, particularly for network bandwidth and memory bandwidth of virtual machines~\cite{sc21-gsight,micro19-faaSprofiler,socc22-owl,atc18-peek-bench}.
To verify this observation, we use a virtual machine to run a function increasing the number of co-located instances from one to six while measuring the execution time of four different functions with resource dominance on different dimensions namely computing, I/O, network, and memory, respectively (detailed in \S\ref{exp:setup}). 
As shown in Figure~\ref{fig:bg:perf-inteference}, the co-location of homogeneous functions leads to substantial resource contention and performance interference, prolonging the function execution time up to 8.1 times. The performance interference is often hard to model and predict.

% this co-residency results in substantial increase of execution latency by up to 8.1 times,leading to considerable variance in function execution time.
% when compared with that with concurrency as one.

%for CPU-, IO-, network- and memory-intensive functions as the concurrency rises from one to six.
%Figure shows that significant performance interference can be observed, . 
%compared with the inclusive deployment (concurrency as one), 
% this exclusive deployment (gray bar) results in substantial increase of execution latency by up to 8.1$\times$ for CPU-, IO-, network- and memory-intensive functions as the concurrency rises from one to six.

% this exclusive deployment (gray bar) results in substantial increase of execution latency by up to 8.1$\times$ for CPU-, IO-, network- and memory-intensive functions as the concurrency rises from one to six.
% As depicted in Figure~\ref{fig:bg:concurrent_latency}, with the concurrency rising  from one to six,  the exclusive deployment results in substantial increase of execution latency by up to 8.1$\times$.
% This significantly magnifies execution latency variance.

% \begin{figure}[t!]
% \centering
% \includegraphics[width=0.25\textwidth]{./figure/motivation/coresident-perf.pdf}
% \vspace{-0.3cm}
% \caption{Performance interference attributed to co-residency of homogeneous function.}
% \label{fig:bg:perf-inteference}
% \end{figure}




\subsection{Runtime Resource Adaptation}
\label{sec:bg:adaptive-allocation}
To tackle the aforementioned resource inefficiency issue, we can adopt a late-binding approach through \emph{runtime resource adaptation}, which resizes functions on the fly based on runtime information (e.g., function slacks), achieving higher resource efficiency without violating SLO. For example, given a workflow as a chain of functions, the resource allocation of the downstream functions can be adjusted when the first function finishes execution. This way, the slack from the first function can be exploited to optimize resource efficiency. 

The idea sounds straightforward and has been considered in some existing works \cite{infocom22-stepconf,middleware20-fifer,socc21-llama,socc21-kraken,middleware20-xanadu}.
However, most of these works make an unrealistic assumption that either the developer performs the adaptation decision with access to runtime information or the serverless platform provider performs the adaptation with domain knowledge of the application workflow. These assumptions render these solutions impractical to deploy in real-world serverless systems. The information barrier between the developer and the provider calls for a new solution. 

We identify the following challenges and opportunities for a full-fledged design for runtime resource adaptation. 

\textbf{\textit{Skewed function execution time distribution.}} 
Resource allocation for a serverless workflow is typically done by leveraging performance profiles of all the functions in the workflow. 
During the offline profiling, the execution time distribution for each function is first obtained by running the function with a variety of sample inputs under different resource conditions. Then, given a time budget, existing approaches typically use P99 of the function execution time as a target and calculate the corresponding resource demands. However, due to the high runtime variability, the distribution of the function execution time is highly skewed where the difference between P50 and P99 can be as high as 100 times~\cite{socc23-huawei-cloud}. This means that if only the function execution time at a single percentile (P50 or P99) is used for resource allocation, there will be significant resource under-provisioning and over-provisioning for most requests at runtime. To address this issue, our idea is to allow for the exploration of the function execution time at diverse percentiles during resource allocation. 


% It is a prerequisite to profile execution latency for adaptive resource allocation.  
% As aforementioned, owing to a variety of unexpected runtime dynamics,  execution latency demonstrates skewed distributions, by up to 100$\times$ between 99\% percentile and 50\% percentile on Huawei cloud serverless~\cite{socc23-huawei-cloud} .
% This makes the current a single statistic (e.g., mean) or 99\% percentile distribution based profiling suffer significant under- and over-estimation.
% To fix this issue, our insight is to \textit{introduce more diverse percentiles to profile execution latency}. 

\textbf{\textit{Dependencies of adaptation decisions.}}
As the function execution progresses, a sub-workflow will be generated by removing the finished function(s) from the workflow. Within each sub-workflow, the resource adaptation decisions for remaining functions are dependent on each other due to the constraint imposed by the end-to-end latency SLO. For example, under-provisioning a function will result in a reduction of the time budget for executing its downstream functions, thus calling for more resources for these downstream functions to avoid SLO violations. Meanwhile, the selection of the percentile for the execution time of each function dictates resource-latency tradeoff for that function. For example, a higher percentile means that more resources will be allocated to ensure that more requests processed by the function will finish within the given time budget. On the contrary, a lower percentile means that more requests will risk SLO violation, but at the benefits of reduced resource consumption. To address such complex dependencies, we propose the following ideas: (1) We introduce two metrics (i.e., the timeout metric and the resilience metric detailed in \S\ref{sec:profilier}) to balance the resource adaptation decisions of the head function of the current sub-workflow and those of the remaining downstream functions. These metrics help us connect the decision making across sub-workflows and avoids sub-optimal adaptation decisions in each sub-workflow. 
(2) We explore lower percentiles for the head function and a high percentile (i.e., P99) for other functions in each sub-workflow. Using lower percentiles maximizes the opportunity for resource optimization since any over-time execution of the head function can later be compensated by resource adaptation in the next round. The high percentile ensures that the resource adaptation is not too radical to cause SLO violations. 

% Each workflow generates multiple sub-workflows as the execution moves forwards. 
% Within sub-workflows, the provisioning is inter-corrected.
% For instance, under-provisioning upstream functions may directly shrink the time budget for downstream functions, which dictates more resources required by the latter against (sub-) SLO violation. 
% This makes sub-workflows generally adopted as the basic unit to make adaptation decisions~\cite{socc21-llama,rtas22-fa2}. 
%  Moreover,  due to the high variance of execution performance, runtime adaptation requires to carry out function by function, i.e.,  discrete adaptation.
%  This, however, can easily lead to a sub-optimal (analyzed in~\S~\ref{sec:synthesizer:generate}).
% Our insight is to \emph{introduce a metric (i.e., resilience detailed in \S~\ref{sec:profilier}) to quantify the inter-correlation as well as a heuristic design (i.e., heavier head explained in \S~\ref{sec:synthesizer:generate})  to calibrate the sub-optimal,  such that resource adaptation can explore higher resource efficiency without SLO guarantee}.

% In particular, latency percentiles (introduced by the profiling)  involves resource adaptation as a new knob.
% Specifically, higher percentile earns  stronger guarantees in SLOs but may be highly prone to resource over-allocation because of its latency over-estimation, impairing resource efficiency.
% In contrast, decreasing percentiles offers the opportunity to explore higher resource efficiency, but suffers the risk of timeout, i.e., execution latency beyond specified time budget, and  may thus incur  SLO violations.
% Here, our insight is to \emph{moderately explore percentiles (detailed in~\S~\ref{sec:synthesizer:generate}), where head functions of  (sub-)workflows can explore lower percentiles because this creates the opportunity to reap higher resource efficiency while possible timeout can be recovered by subsequent functions' re-adaptive allocation.
% On the other head, non head functions maintain percentiles as 99\%}.
% This can well keep the trade-off between opportunities of exploring higher resource efficiency and risks of SLO violations. 
% Additionally, it effectively shrinks the searching space, benefiting the adaptation with higher time-efficiency.


\textbf{\textit{Tight resource adaptation window.}}
Runtime resource adaptation requires to calculate a new resource allocation decision for the remaining sub-workflow immediately when a function finishes execution. Since serverless functions are typically short-lived (less than 1s on average)~\cite{atc18-peek-bench,socc22-owl,atc20-serverless-in-the-wild,socc23-huawei-cloud}, the window for resource adaptation is quite tight. Assuming the serverless platform will perform the runtime adaptation on behalf of the developer since the platform has access to full runtime information, the resource adaptation decision making should be fast without involving complex calculations and logic or exploring a large space. As discussed before, the serverless platform provider does not have domain knowledge of the serverless workflow. Hence, the developer must pass the necessary information to the serverless platform for runtime adaptation decision making. Our idea is to let the developer synthesize critical hints containing resource allocation rules and options, which the serverless platform provider utilizes to perform runtime resource adaptation. The hints should be highly condensed so the serverless platform can make adaptation decisions quickly enough. 


% Apart from highly varying execution performance, serverless functions are also short-living (less than 1s on average)~\cite{atc18-peek-bench,socc22-owl,atc20-serverless-in-the-wild,socc23-huawei-cloud}, so is the window for adaptive allocation. 
% This variance and volatility calls for a well-preparation of hints for all possible runtime situations while promising them compact and straightforward enough for providers to easily take action.

% Here, our insight is to \emph{holistically synthesize hints in an offline manner, and then utilize the discreteness of adaptive allocation in both decision-making and decision-executing (detailed in~\S~\ref{sec:synthesizer:condense}) to fully condense the hints.
% Finally, hints are warped into a simple and compact table.
% Base on that, providers can accomplish the runtime adaption promptly and properly}.

To demonstrate the potential of runtime resource adaptation incorporating all the above ideas, we take a real-world serverless workflow (explained in \S\ref{exp:setup}) as an example, and evaluate its end-to-end latency (denoted by E2E) and resource consumption (CPU cores).
As illustrated in Figure~\ref{fig:bg:size}, the late-binding (blue triangle) reduces the resource consumption by up to 42.2\% compared with existing early-binding solutions (orange circle), while ensuring SLO guarantees. This highlights the significant gains from runtime resource adaptation. 


\begin{figure}[t!]
\centering
\includegraphics[width=0.45\textwidth]{./figure/motivation/size_early_bind_vs_ours.pdf}
%\vspace{-0.1cm}
\caption{Performance comparison between early-binding (left)~\cite{eurosys19-grandslam} and late-binding (runtime resource adaptation), where the CPU consumption (right) is normalized by the optimal obtained with exhaustive search.} 
%\vspace{-0.3cm}
\label{fig:bg:size}
\end{figure}

   
	







\section{Goal and Research Questions}
\label{sec:researchissues}

The goal of this study is to evaluate the adoption of ML models to support the selection of studies for SLR updates. We translated our goal into three different Research Questions (RQs).

    \textbf{RQ1:} \textit{How effective are ML models in selecting studies for SLR updates?}

    To answer this research question, we represent the effectiveness of the ML models in supporting study selection using metrics such as \textit{Recall}, \textit{Precision} and \textit{F-measure} \cite{Napoleao2021, Watanabe20}. Our ML automated analysis considers only the title, abstract, and keywords of the studies. Our ML models were trained with data from the original SLR and asked to select studies for the SLR update. The results were compared with the final results of the included and excluded studies (according to the consensus discussion of the three experienced SE researchers) for the SLR update.
    
    %Our ML automated analysis considers only title and abstract of the studies and the metrics are calculated at first considering the results from the expert reviewers analysis only on the title, abstract and keywords and next, considering also their results from the full-text analysis.

    \textbf{RQ2:} \textit{How much effort can ML models reduce during the study selection activity of SLR updates?}

    For this research question, we calculate the effort reduction by the relation of the number of studies that need to have their title, abstract, and keywords manually analyzed without the support of ML models versus the number of studies to be analyzed after discarding studies that would have a low probability of being included according to the ML model. \textit{I.e.}, we analyzed the percentage of studies that could be safely discarded based on their inclusion probability while keeping a 100\% recall of the included studies. 
   
    \textbf{RQ3:} \textit{How does the support of ML in the selection of studies compare to the support of an additional human reviewer?}
    % We compared the agreement level of the ML Model with the highest \textit{F-score} value, supporting a single reviewer with the agreement level of each pair of reviewers, by calculating their Cohen's \textit{Kappa} coefficient \cite{Cohen10, Kitchenham15}.

    In this research question, we assessed how a pair of a human and an ``ML model reviewer" would compare against pairs of human reviewers in determining the list of studies to be included. Therefore, to improve the chances of providing good support, we used the ML model with the highest \textit{F-score}. In the initial screening, each of the three SE researchers had assessed each paper on a scale from zero to two (0 - exclude, 1 - unsure, 2 - include). We adjusted the outcome of the ML model (probabilities for inclusion) to that same scale of integers. Thereafter, we calculated the aggregated outcome for the list of papers for each possible pair of reviewers using the average score of the pair members. By using the average, we fairly hypothesize that, when working in pairs, each member of the pair would equally influence inclusion or exclusion. Finally, we compared the aggregated outcome of each pair with the final results using the Euclidean distance to understand how far each pair was from the oracle.

    % \end{itemize}

    %---- Here I did not defined Kappa. I think we can defined it in the methodology section.

    %Kappa analysis ->  Euclidean Distance
    %agremeent level do algoritmo com os revisores - titulo, abstract and keywords
    % future assessement + Hipótese :  Machine learning can replace a reviewer in a SLR update?
\section {Study Design}
\label{sec:methodology}

%Bianca: Aqui estava faltando definir qual o research metodo que a gente utilizou para testar o que foi desenvolvido. Eu utilizei a ideia do small-scale evaluation, visto que segundo o trabalho do Wohlin o que fizemos nao é um estudo de caso, nem um experimento. 

% 5 steps Runeson (referencia/exemplo)
% We follow the five main steps for conducting case studies
% proposed by [21]: Design, preparation, collecting data, analysis
% and reporting.

% TODO: Explicar o Precison/Recall nessa seção e não nos Resultados -- %Bianca: Eu coloquei uma versao de definicao, veja o que acha. 

In this section, we present the study design. To evaluate our research questions, we performed a small-scale evaluation \cite{Wohlin2022cs}, following two main steps: (i) Data Collection and (ii) Design \& Execution. We describe the data collection to train and test our ML models in Section \ref{subsec:data}. In Section \ref{subsec:studydesing}, we detail how we designed and executed our strategy to train and configure the ML models. 

\subsection{Data Collection}
\label{subsec:data}

\begin{figure*} [ht]
    \centering
    \includegraphics[width=380pt]{figures/fig03-data-acquisition-v2.pdf}
    \caption{Data collection process}
    \label{fig:fig-data-selection}
\end{figure*}

We used an ongoing SLR update conducted by the same first three authors of this replication \cite{Wohlin2022} as the instrument of our study. We chose this ongoing SLR update since the inclusion and exclusion of new studies were rigorously conducted based on individual assessments and the consensus of three experienced SLR researchers. First, each researcher screened all the papers, analyzing the title, abstract, and keywords and registering his individual assessment (\textit{i.e.}, we had three assessments for each paper). Then, the full texts of the studies were analyzed, and discussions were held to reach a final consensus on the list of included and excluded papers. Hence, we had the results of the initial screening by each researcher and also the final list of papers to be included and excluded.

We had access to all the studies the team analyzed during the SLR update (.bib files): a total of 591 papers were analyzed for the SLR update, of which 39 were included, and 552 were excluded. We filtered the studies to consider only the studies in English and containing an abstract. In the end, we used 551 studies in our testing set, of which 38 were included by the team assessment and 513 were excluded. We used these studies to form the testing set for our ML models. 

To train our ML models, we used a training set with 128 studies, of which 45 were included and 83 were excluded. The 45 studies used to train our models with what should be included were the same studies included in the original SLR \cite{Wohlin2022}. Since we did not have access to the list of excluded studies during the study selection phase of the original SLR, we performed a backward snowballing \cite{Wohlin14} on the original references to obtain the 83 studies used to train our models with what should be excluded. Figure \ref{fig:fig-data-selection} summarizes this process. The bib files for the included and excluded studies of the training and testing sets are available in our open science repository~\cite{zenodoOpenScience}.

\subsection{Design \& Execution}
\label{subsec:studydesing}

We developed a pipeline with the following steps to automate the study selection process of an SLR update by using ML and answering our research questions. Our pipeline is illustrated in Figure \ref{fig:fig-study-design}. 

\begin{figure*} [ht]
    \centering
    \includegraphics[width=380pt]{figures/fig-pipeline-details-v4.pdf}
    \caption{Study design pipeline}
    \label{fig:fig-study-design}
\end{figure*}

In summary, our pipeline processes a set of .bib files containing the list of studies to train the ML models and the list of studies to be analyzed. After its execution, it returns a report file in .xlsx format with the ML model predictions, informing which studies should be included and excluded, as well as metrics about the ML model predictions and the configuration used.

The pipeline receives four different .bib files as input, one containing the list of studies that should be excluded and one containing the list of studies that should be included for each set (training and testing). In case there are any errors in the input files, the pipeline will stop its execution and will inform which entry was associated with each error as well as the type of error. 

As shown in Figure~\ref{fig:fig-study-design}, we first validated the .bib files of our testing and training sets to ensure the completeness of the set, avoiding duplicated entries or missing keys. Each study entry must have a title, the year of publication, an abstract text, and an author list. 
Secondly, we applied text filtering techniques with Natural Language Processing (NLP) \cite{NLTK}, such as Lemmatization and Tokenization, to remove irrelevant characters. Thirdly, we applied Text Vectorization on the filtered texts using  Term-frequency/Inverse-Document-Frequency (TF/IDF), a technique that transforms text data into a numerical matrix of features. Fourthly, we used statistical methods to compute and select the most relevant features. In the fifth step, we trained and tuned our ML models using our training set. Finally, in the last step, we used our ML models to predict which studies of our testing set should be included and excluded and compared the results with the final list of included and excluded studies.

Additionally, an optional .env file can be passed as input to our pipeline; this file allows some steps in our pipeline to use a specific configuration, such as choosing the configuration of the Feature Selection (FS) method to compute the features, as well as the number of features to be selected in step four, and choosing the configuration for the ML models regarding which algorithm to be used, or which metric should be targeted when tuning the model as well as the type of cross-validation to be performed, in step five. All parameters that can be configured are also shown in Figure~\ref{fig:fig-study-design}.

Building on the work by Napoleão \textit{et al.} \cite{Napoleao2021}, which highlighted Support Vector Machines (SVM) as one of the most commonly used machine learning classifiers for assisting study selection in systematic literature reviews (SLRs), we chose to evaluate SVM in our study. Additionally, inspired by the findings of Pintas \textit{et al.} \cite{pintas2021feature}, who analyzed the most widely adopted ML classifiers and feature selection techniques for text classification, identifying SVM, Naive Bayes, k-Nearest Neighbors, Decision Trees, and Random Forest (RF) as the top five classifiers, we conducted initial tests with these classifiers. Our preliminary results showed that SVM and RF outperformed the other classifiers. Consequently, we focused our evaluation on SVM and RF.

% We experimented multiple configurations of our pipeline and evaluated different configurations for Feature Selection and for training and tuning of our ML classifiers. To select the best features, we tested different statistical methods such as Chi-squared (Chi2) \cite{Chi2}, Pearson Correlation \cite{pearson_r} and Analysis of Variance (Anova-F) \cite{ANOVA}. We tested different techniques to tune our ML classifiers such as K-fold cross-validxation, Times-Series cross-validations and hyperparameter tuning with GirdSearch \cite{GridSearch}.

We experimented with multiple pipeline configurations and evaluated different configurations for FS and training and tuning of our ML classifiers. During step four, to compute the best features, we tested different statistical methods such as Chi-squared (Chi2) \cite{Chi2}, Pearson Correlation \cite{pearson_r} and Analysis of Variance (Anova-F) \cite{ANOVA} as well as different ranges of features. After applying Text Filtering and Text Vectorization techniques, presented in steps three and four of our pipeline, our training set comprised 23,630 features. We identified the range with the most relevant features in our training set as 900 to 1,500 features. Notably, the best results, both in terms of F-score and Recall, were consistently achieved with experiments that selected the 1200 best features.

For each evaluation, we executed the pipeline from start to finish in a clean environment using one statistical method at a time. To avoid data leakage and bias, feature selection was conducted solely based on the training set. Furthermore, we used GridSearch for parameter tuning when creating the model, which inherently includes k-fold cross-validation for measuring the most efficient parameter configuration when developing the model based on the training set. Finally, the trained model was applied to predict the inclusion or exclusion of the unseen (holdout) papers of the testing set. Hence, the hereafter reported results refer not to cross-validations conducted during model creation but to evaluating the tuned model based on the holdout testing set for which three experienced SE researchers had manually and rigorously crafted the information on inclusion and exclusion.

The complete Python code that automates our pipeline is available in our open science repository~\cite{zenodoOpenScience}.
\section{Results}
\label{sec:results}
% In this section, we answer the research questions formulated in Section \ref{experiments}.

\subsection{Effectiveness of indexing succinct facts to improve information retrieval efficiency} To answer \textbf{RQ1}, we measure the memory and computational costs of fact-checking using full-Wikipedia compared to the pruned version proposed in this work. We first measure the index size on disk, measuring the raw JSON file size containing the article titles and texts, for each experiment setting. In \autoref{fig:disk_size}, we observe a significant reduction in disk space usage with HoVer having a reduction ranging from \textbf{44-55\%}, and WiCE from \textbf{44-57\%}. Additionally, the number of sentences stored in the index also decreases, with HoVer showing a reduction from 52-61\% and WiCE 52-59\%, indicating that at least half of the sentences are not helpful in claim verification.

\begin{table}[htb!]
\small
\centering
\footnotesize
\begin{tabular}{c c c c c}
\toprule
\small
\textbf{Method} & \textbf{Disk Size} & \textbf{Size reduction} & \textbf{\#Sentences} & \textbf{\% decrease}\\
\hline \hline
\multicolumn{1}{l}{\colorg\textbf{HoVer}} & \colorg& \colorg & \colorg & \colorg  \\
Full-Wiki  &  11.28 GiB & - & 94,914,378 & -   \\
Fact Extraction & 6.19 GiB & \down{45}& 45,894,704 & \down{52} \\
Citation Extraction & \textbf{5.07 GiB} & \down{\textbf{55} } & \textbf{36,886,889} & \down{\textbf{61}} \\
Fusion & 5.45 GiB & \down{52} & 39,842,574 & \down{{58}} \\

\hline
\multicolumn{1}{l}{\colorg \textbf{WiCE}} & \colorg &  \colorg & \colorg & \colorg \\
Full-Wiki & 15.28 GiB & - &  126,533,841 & -  \\
Fact Extraction & 8.56 GiB & \down{44} & 61,040,380 & \down{52}\\
Citation Extraction & \textbf{6.56 GiB} & \down{\textbf{57}} & \textbf{51,735,961} & \down{\textbf{59}}  \\
Fusion & 6.85 GiB & \down{55}& 54,070,295 & \down{57} \\

\hline
\end{tabular}
\caption{Comparison sizes for the corpora per experiment setting, consisting of English Wikipedia articles 2017 (HoVer) and 2024 (WiCE). Reduction is measured compared to the  Full-Wiki data setting. \down{} denotes a reduction in corpus size and number of sentence compared to Full-Wiki setting.}
\label{fig:disk_size}
\end{table}
\vspace{-2em}

 Following the reduction in disk size, a notable improvement in retrieval latency is evident, as demonstrated in \autoref{tab:bm25_latency}.
Regarding document retrieval latency (which encompasses both column values), there's an observed speedup ranging from approximately 1.5x (334 ms) to 1.6x (316 ms) compared to the original experimental setting for HoVer (495 ms). Similarly, in WiCE experiments, we witness a comparable speedup rate ranging from 1.4x (446 ms) to 1.6x (399 ms) compared to the original experimental setting (636 ms). This observation suggests that while the reduced text size contributes to efficient retrieval, it could further be improved.

\begin{table}[htb!]
\centering
\small
\footnotesize
% \vspace{-1cm}
\begin{tabular}{l c c c c}
\multirow{2}{*}{\makecell{\textbf{Methods}}} & \multirow{2}{*}{\textbf{Retrieval}} & \multirow{2}{*}{\makecell{\textbf{Total Latency}}} & \multirow{2}{*}{\makecell{\textbf{Speedup}}} \\
& \\
\hline
\multicolumn{1}{l}{\colorg\textit{HoVer}} & \colorg & \colorg & \colorg \\
 Full-Wiki & 426  ms & 659 ms & - \\
Fact Extraction & 257 ms  & 338  ms & 1.9x \\
Citation Extraction & 246 ms  &  327  ms & \speedup{2.0x}  \\
Fusion & 265 ms  & 345  ms & 1.9x  \\

\multicolumn{1}{l}{\colorg\textit{WiCE}} & \colorg & \colorg & \colorg \\
  Full-Wiki &  559 ms  &  831  ms & - \\
Fact Extraction & 372 ms  & 468   ms & 1.8x \\
Citation Extraction & 330 ms   & 419  ms & \speedup{2.0x} \\
Fusion & 347 ms & 436  ms & 1.9x \\
\hline
\end{tabular}
\caption{Retrieval and total latency for Sparse retrieval with Re-ranking. Speedup is compared with respect to the total latency of the Full-Wiki setup.}
\label{tab:bm25_latency}
\vspace{-2em}
\end{table}

% \begin{table}[htpb!]
\centering
\footnotesize
\begin{tabular}{l c c c c c c c c}
\multirow{2}{*}{} & \multicolumn{2}{c}{\makecell{Document retrieval}} & \multirow{2}{*}{\makecell{Sentence \\ Retrieval}} & \multirow{2}{*}{\makecell{Claim \\ Verification}} & \multicolumn{2}{c}{Total Latency} & \multicolumn{2}{c}{Speedup} \\
\cline{2-3}\cline{6-7}\cline{8-9}
& CPU & GPU &  & &  CPU & GPU & CPU & GPU \\ 
\hline
\multicolumn{1}{l}{\textit{HoVer}} &  \\
 \textbf{Full-Wiki (S+R) } &  \multicolumn{2}{c}{\textbf{491  ms}} & \textbf{157  ms} & \textbf{7 ms} &  \multicolumn{2}{c}{\textbf{659 ms}} & - & - \\

Full-Wiki & 515 ms & 31 ms & 153 ms & 8 ms & 676 ms & 192  ms & 1.0x & 3.4x \\
% Original  &  523 ms & 30 ms  & - & 9 ms & 532 ms & 39 ms & 1.2x & 16.9x \\
Claim detection & 513 ms & 23 ms & - & 8 ms & 521 ms & 31 ms & 1.3x & 21.3x  \\
Citation Extraction & 479 ms & 23 ms & - & 9 ms &  488 ms & 32 ms & 1.4x & 20.6x \\
Fusion & 500 ms  & 23 ms & - & 9 ms & 509 ms & 32  ms & 1.3x & 20.6x \\

\multicolumn{1}{l}{\textit{WiCE}} & \\
 \textbf{Full-Wiki (S+R)} & \multicolumn{2}{c}{\textbf{636 ms}} & \textbf{186 ms} & \textbf{9  ms}  & \multicolumn{2}{c}{\textbf{831  ms}} & - & - \\
Full-Wiki & 685 ms & 34 ms & 184 ms & 9 ms & 878 ms & 227  ms & 1.0x & 3.7x \\
% Original  & 610  ms & 34 ms  & - & 9 ms & 619  ms & 43 ms & 1.3x & 19.3x \\
Claim detection &  622 ms & 31 ms  & - & 9 ms & 631 ms & 40 ms & 1.3x & 20.8x \\
Citation Extraction &  610 ms & 31 ms  & - & 9 ms & 619 ms & 40 ms & 1.3x & 20.8x  \\
Fusion & 619  ms & 31 ms  & - & 9 ms & 628 ms & 40 ms & 1.3x & 20.8x  \\[5mm]
\hline
\end{tabular}
\caption{Retrieval and inference latency for Dense retrieval setup on data settings. Speedup is compared with respect to the total latency of the Sparse Retrieval setup with original data setting (bold font).}
\label{tab:faiss_latency}
\end{table}




% However, sparse retrieval does not capture semantic information and requires a costly re-ranking stage. While transitioning from Sparse to Dense retrieval may help improve the performance dense retrieval introduces additional computational costs as seen in Table \ref{tab:faiss_latency}. This is due to our use of FAISS utilising fixed-dimensionality vectors where despite varying article text lengths, the constant number of text embeddings minimizes impact on retrieval speed. However, dense retrieval libraries offer GPU support, which can yield substantial speedups compared to the CPU-based retrieval of BM25 and FAISS. GPU retrieval shows substantial speedups: 16.6-22.3x for HoVer and 17.9-20.2x for WiCE compared to CPU retrieval. Compared to BM25, GPU retrieval offers 16.0-21.5x speedup for HoVer and 18.7-20.5x speedup for WiCE. Thus, making Dense Retrieval a highly efficient and viable option over standard Sparse Retrieval with re-ranking.
%\vspace{-0.5em}

\noindent \textbf{Insight 1}: \textit{
Extraction of succinct facts reduces storage requirements and improves latency for Sparse retrieval while only leading to a minor loss in task performance.}
\vspace{-2em}
% %%%%%%%%%%%%%%%%%%%%%%%%%%%%%%%%%%%%%%%%%%%%%%%%%%%%%%%%%%%%%%%%%%%%%%%%%%%%%%%%%%%

\subsection{Effectiveness of pruned knowledge sources on overall pipeline efficiency and downstream fact-checking performance?}
To answer \textbf{RQ2}, we now shift focus to analyzing the inference time throughout the entire pipeline instead of solely the retrieval part. Extraction of just supporting facts not only has a improvement in the retrieval stage but also on the overall inference latency across the pipeline. For HoVer this being a 1.9-2.0x speedup and 1.8-2.0x for WiCE experiments. This improvement can be attributed to not only faster retrieval times but also the elimination of the Sentence Retrieval stage, which previously imposed significant latency overhead. 

%\begin{table}[htb!]
\small
\begin{tabular}{c c c c c c c c}
\multirow{2}{*}{Experiment setting} & \multirow{2}{*}{Accuracy} & \multicolumn{2}{c}{F1} & \multicolumn{2}{c}{Precision} & \multicolumn{2}{c}{Recall}  \\ 
\cline{3-8}
  & &  Weighted  & Macro & Weighted & Macro & Weighted & Macro      \\
\hline
\multicolumn{1}{l}{\textit{Sparse + Re-ranking}} & & & & \\
Full-Wiki & \textbf{67.79} & \textbf{67.59} & \textbf{67.63} & \textbf{68.45} & \textbf{68.39} & \textbf{67.79}  & \textbf{67.93}\\
Claim detection & \underline{62.33} & 62.02 & 62.08 & \underline{62.98} & \underline{62.92} & \underline{62.33} & \underline{62.50} \\
Citation Extraction & 60.91 & 60.61 & 60.66 & 61.47 & 61.42 & 60.91 & 61.07 \\
Fusion & 62.28 & \underline{62.15} & \underline{62.18} & 62.60 & 62.56 & 62.28 & 62.39  \\[5mm]

\hline
\multicolumn{1}{l}{\textit{Dense Retrieval}} & & & & \\
Full-Wiki & 64.60 & 64.45 & 64.45 & 64.86 & 64.86 & 64.60 & 64.60 \\
% Original & 62.90 & 62.72 & 62.76 & 63.33 & 63.28 & 62.90 & 63.02 \\
Claim detection & \underline{61.50} & \underline{60.94} & \underline{60.94} & \underline{62.20} & \underline{62.20} & \underline{61.50} & \underline{61.50} \\
Citation Extraction & 59.67 & 59.40 & 59.46 & 60.13 & 60.09 & 59.67 & 59.82 \\
Fusion & 59.51 & 59.32 & 59.37 & 59.85 & 59.81 & 59.51 & 59.64  \\[5mm]

\hline
\multicolumn{1}{l}{\textit{Index Compression}} & & & & & & &  \\
Full-Wiki & 63.30 & 62.54 & 62.54 & 64.48 & 64.48 & 63.30 & 63.30 \\
% Original & 63.02 & 62.08 & 62.08 & 64.46 & 64.46 & 63.02 & 63.02  \\
Claim detection & \underline{61.92} & \underline{61.71} & \underline{61.71} & \underline{62.19} & \underline{62.19} & \underline{61.92} & \underline{61.93}  \\
Citation Extraction & 59.98 & 59.12 & 59.12 & 60.89 & 60.89 & 59.98 & 59.98   \\
Fusion & 61.58 & 61.43 & 61.43 & 61.75 & 61.75 & 61.58 & 61.58   \\[5mm]

\hline
\end{tabular}
\caption{Performance experiments on HoVer data and adjustments using full document text of English Wikipedia. The underlined-styled values represent the second best  within each retrieval setup.}
\label{tab:hover_performance_metrics}
\end{table}
% Full document text 
% 
% \begin{figure}
%     \centering
%     \includegraphics[width=0.82\linewidth]{figs/graphs/hover_perf.png}
%     \caption{HoVer performance comparison}
%     \label{fig:enter-label}
% \end{figure}
% %\begin{table}[htb!]
\centering
\footnotesize
\begin{tabular}{c c c c c c c c}
\multirow{2}{*}{Experiment setting} & \multirow{2}{*}{Accuracy} & \multicolumn{2}{c}{F1} & \multicolumn{2}{c}{Precision} & \multicolumn{2}{c}{Recall}  \\ 
\cline{3-8}
  & &  Weighted  & Macro & Weighted & Macro & Weighted & Macro      \\
\hline
\multicolumn{1}{l}{\textit{Sparse + Re-ranking}} & & & & \\
Full-Wiki & \textbf{63.69} &  \textbf{61.84} &  \textbf{55.24} &  \textbf{61.12} &  \textbf{56.54} &  \textbf{63.69} &  \textbf{55.32 } \\
Claim detection & 61.90 & 60.12 & \underline{53.33} & 59.27 & 54.26 & 61.90 & \underline{53.53} \\
Citation Extraction & 61.01 & 59.56 & 52.96 & 58.75 & 53.59 & 61.01 & 53.09 \\
Fusion & \underline{63.39} & \underline{60.21} & 52.48 & \underline{59.46} & \underline{54.69} & \underline{63.39} & 53.27  \\[5mm]

\hline
\multicolumn{1}{l}{\textit{Dense Retrieval}} & & & & \\
Full-Wiki &  61.61 & 60.95 & 55.21 & 60.47 & 55.49 & 61.61 & 55.13 \\
% Full-Wiki  & 60.42 & 58.80 & 51.96 & 57.90 & 52.58 & 60.42 & 52.19 \\
Claim detection & 61.01 & 58.94 & 51.78 & 57.96 & 52.70 & 61.01 & 52.17 \\
Citation Extraction & 58.63 & 58.48 & \underline{52.92} & 58.35 & 52.95 & 58.63 & \underline{52.91} \\
Fusion & \underline{61.31} & \underline{59.34} & 52.30 & 58.40 & \underline{53.23} & \underline{61.31} & 52.62  \\[5mm]

\hline
\multicolumn{1}{l}{\textit{Index Compression}} & & & & & & &  \\
Full-Wiki & 62.46 & 61.38 & 55.27 & 60.74 & 55.84 & 62.46 & 55.20  \\
% Original  & 60.46 & 60.63 & 55.64 & 60.81 & 55.60 & 60.46 & 55.70  \\
Claim detection & 59.31 & 59.02 & \underline{53.32} & 58.77 & 53.39 & 59.31 & \underline{53.30}  \\
Citation Extraction & 60.74 & 59.21 & 52.42 & 58.34 & 53.04 & 60.74 & 52.60  \\
Fusion & \underline{63.04} & \underline{59.79} & 51.89 & 58.94 & \underline{53.97} & \underline{63.04} & 52.76 \\[5mm]

\hline
\end{tabular}
\caption{Performance experiments on WiCE data and adjustments using full document text of English Wikipedia. The bold-styled values represent the baseline while the underlined-styled values represent the highest scores of the re-ranked data within a retrieval setup category.}
\label{tab:wice_performance_metrics}
\end{table}
% Full document text 
% 

% \begin{figure}[hbt!]
%     \centering
%     \includegraphics[width=0.85\linewidth]{figs/graphs/wice_perf.png}
%     \caption{WiCe performance comparison}
%     \label{fig:enter-label}
% \end{figure}

\begin{figure*}[hbt!]
    \begin{subfigure}{.5\textwidth}
        

\begin{tikzpicture}
\edef\mylst{"67.59","64.45","62.54"}
\edef\explora{"62.15","59.32","61.43"}

    \begin{axis}[
            ybar=1.5pt,
            width=6.7cm,
            bar width=0.35,
            every axis plot/.append style={fill},
            grid=major,
            xtick={1, 4, 8,9,11},
            xticklabels={Sparse + re-rank, Dense, IC},
            ylabel style = {font=\tiny},
        yticklabel style = {font=\boldmath \tiny,xshift=0.05ex},
        xticklabel style ={font=\tiny,yshift=0.5ex},
            ylabel={Performance (F1 weighted)},
            enlarge x limits=0.15,
            ymin=0,
            ymax=86,
            legend style ={font=\tiny,yshift=0.05ex},
            area legend,
            nodes near coords style={font=\tiny,align=center,text width=1em},
            legend entries={FW, FE, CE, Fu},
            legend cell align={left},
            legend pos=north west,
            legend columns=-1,
            legend style={/tikz/every even column/.append style={column sep=0.06cm}},
        ]
        \addplot+[
            ybar,
            plotColor1*,
            postaction={
                    pattern=north east lines
                },
                    nodes near coords=\pgfmathsetmacro{\mystring}{{\mylst}[\coordindex]}\textbf{\mystring},
        ] plot coordinates {
                (1,67.59)
                (4,64.45)
                (8,62.54)
            };
        \addplot+[
            ybar,
            plotColor2*,
        ] plot coordinates {
                (1,62.02)
                (4,60.94)
                (8,61.71)
                (9,0)
            };

                    \addplot+[
            ybar,
            plotColor3*,
            draw=black,
    nodes near coords align={vertical},
            postaction={
                    pattern=north west lines
                },
        ] plot coordinates {
                (1,60.61)
                (4,59.40)
                (8,59.12)
                (9,0)
            };
             \addplot+[
            ybar,
            plotColor4*,
            draw=black,
            postaction={
                    pattern=north east lines
                },
            nodes near coords=\pgfmathsetmacro{\mystring}{{\explora}[\coordindex]}\textbf{\mystring},
        ] plot coordinates {
                (1,62.15)
                (4,59.32)
                (8,61.43)
            };
    \end{axis}
\end{tikzpicture}

\subcaption{HoVer}
    \end{subfigure}
        \begin{subfigure}{.5\textwidth}
    

\begin{tikzpicture}
\edef\mylst{"61.84","60.95","61.38"}
\edef\explora{"60.21","59.34","59.79"}

    \begin{axis}[
            ybar=1.5pt,
            width=6.7cm,
            bar width=0.35,
            every axis plot/.append style={fill},
            grid=major,
            xtick={1, 4, 8,9,11},
            xticklabels={Sparse + re-rank, Dense, IC},
            ylabel style = {font=\tiny},
        yticklabel style = {font=\boldmath \tiny,xshift=0.05ex},
        xticklabel style ={font=\tiny,yshift=0.5ex},
            ylabel={Performance (F1 weighted)},
            enlarge x limits=0.15,
            ymin=0,
            ymax=86,
            legend style ={font=\tiny,yshift=0.05ex},
            area legend,
            nodes near coords style={font=\tiny,align=center,text width=1em},
            legend entries={FW, FE, CE, Fu},
            legend cell align={left},
            legend pos=north west,
            legend columns=-1,
            legend style={/tikz/every even column/.append style={column sep=0.06cm}},
        ]
        \addplot+[
            ybar,
            plotColor1*,
            postaction={
                    pattern=north east lines
                },
                    nodes near coords=\pgfmathsetmacro{\mystring}{{\mylst}[\coordindex]}\textbf{\mystring},
        ] plot coordinates {
                (1,61.84)
                (4,60.95)
                (8,61.38)
            };
        \addplot+[
            ybar,
            plotColor2*,
        ] plot coordinates {
                (1,60.12)
                (4,58.94)
                (8,59.02)
                (9,0)
            };

                    \addplot+[
            ybar,
            plotColor3*,
            draw=black,
    nodes near coords align={vertical},
            postaction={
                    pattern=north west lines
                },
        ] plot coordinates {
                (1,59.56)
                (4,58.48)
                (8,59.21)
                (9,0)
            };
             \addplot+[
            ybar,
            plotColor4*,
            draw=black,
            postaction={
                    pattern=north east lines
                },
            nodes near coords=\pgfmathsetmacro{\mystring}{{\explora}[\coordindex]}\textbf{\mystring},
        ] plot coordinates {
                (1,60.21)
                (4,59.34)
                (8,59.79)
            };
    \end{axis}
\end{tikzpicture}

    \subcaption{Wice}

    \end{subfigure}
    \caption{HoVer and WiCe task performance (FW- Full-Wiki, FE - Fact Extraction, IC- Index Compression, CE - Citation Extraction, Fu - Fusion)}
    \label{fig:performance_plot}
    \end{figure*}

\begin{figure*}[hbt!]
    \begin{subfigure}{.5\textwidth}
        

\begin{tikzpicture}
\edef\mylst{"67.59","64.45","62.54"}
\edef\explora{"62.15","59.32","61.43"}

    \begin{axis}[
            ybar=1.5pt,
            width=6.2cm,
            bar width=0.35,
            every axis plot/.append style={fill},
            grid=major,
            xtick={1, 4, 8,9,11},
            xticklabels={Sparse + re-rank, Dense, IC},
            ylabel style = {font=\tiny},
        yticklabel style = {font=\boldmath \tiny,xshift=0.05ex},
        xticklabel style ={font=\tiny,yshift=0.5ex},
            ylabel={Recall@10},
            enlarge x limits=0.15,
            ymin=0,
            ymax=0.5,
            legend style ={font=\tiny,yshift=0.05ex},
            area legend,
            nodes near coords style={font=\tiny,align=center,text width=1em},
            legend entries={FW, FE, CE, Fu},
            legend cell align={left},
            legend pos=north west,
            legend columns=-1,
            legend style={/tikz/every even column/.append style={column sep=0.06cm}},
        ]
        \addplot+[
            ybar,
            plotColor1*,
            postaction={
                    pattern=north east lines
                },
        ] plot coordinates {
                (1,0.136)
                (4,0.123)
                (8,0.098)
            };
        \addplot+[
            ybar,
            plotColor2*,
        ] plot coordinates {
                (1,0.105)
                (4,0.094)
                (8,0.098)
                (9,0)
            };

                    \addplot+[
            ybar,
            plotColor3*,
            draw=black,
    nodes near coords align={vertical},
            postaction={
                    pattern=north west lines
                },
        ] plot coordinates {
                (1,0.126)
                (4,0.143)
                (8,0.096)
                (9,0)
            };
             \addplot+[
            ybar,
            plotColor4*,
            draw=black,
        ] plot coordinates {
                (1,0.124)
                (4,0.141)
                (8,0.097)
            };
    \end{axis}
\end{tikzpicture}

\subcaption{WiCE (nDCG@10)}
    \end{subfigure}
        \begin{subfigure}{.5\textwidth}
    

\begin{tikzpicture}
\edef\mylst{"67.59","64.45","62.54"}
\edef\explora{"62.15","59.32","61.43"}

    \begin{axis}[
            ybar=1.5pt,
            width=6.4cm,
            bar width=0.35,
            every axis plot/.append style={fill},
            grid=major,
            xtick={1, 4, 8,9,11},
            xticklabels={Sparse + re-rank, Dense, IC},
            ylabel style = {font=\tiny},
        yticklabel style = {font=\boldmath \tiny,xshift=0.05ex},
        xticklabel style ={font=\tiny,yshift=0.5ex},
            ylabel={Recall@10},
            enlarge x limits=0.15,
            ymin=0,
            ymax=0.5,
            legend style ={font=\tiny,yshift=0.05ex},
            area legend,
            nodes near coords style={font=\tiny,align=center,text width=1em},
            legend entries={FW, FE, CE, Fu},
            legend cell align={left},
            legend pos=north west,
            legend columns=-1,
            legend style={/tikz/every even column/.append style={column sep=0.06cm}},
        ]
        \addplot+[
            ybar,
            plotColor1*,
        ] plot coordinates {
                (1,0.309)
                (4,0.195)
                (8,0.160)
            };
        \addplot+[
            ybar,
            plotColor2*,
        ] plot coordinates {
                (1,0.241)
                (4,0.166)
                (8,0.163)
                (9,0)
            };

                    \addplot+[
            ybar,
            plotColor3*,
            draw=black,
    nodes near coords align={vertical},
            postaction={
                    pattern=north west lines
                },
        ] plot coordinates {
                (1,0.295)
                (4,0.226)
                (8,0.174)
                (9,0)
            };
             \addplot+[
            ybar,
            plotColor4*,
            draw=black,
        ] plot coordinates {
                (1,0.286)
                (4,0.223)
                (8,0.160)
            };
    \end{axis}
\end{tikzpicture}

    \subcaption{WiCE (Recall@10)}

    \end{subfigure}
    \caption{Retrieval performance comparison}
    \label{fig:retrieval_perf}
    \end{figure*}
The evaluation of Sparse and Dense Retrieval setups in HoVer and WiCE experiments reveals that Sparse Retrieval, particularly fact extraction (FE) and Fusion approaches, maintains performance closest to the Full-Wiki setup as measured by weighted F1 in Figure \ref{fig:performance_plot}, while citation extraction has a larger drop in performance. Most notably, the Fusion method compared to the other methods has relatively high scores, underscoring the importance of combining supporting facts extraction methods for optimal results. We also report retrieval performance for WiCE Figure \ref{fig:retrieval_perf} using measures nDCG@10 and Recall@10 using annotated documents provided for WiCE. We observe trends similar to overall task performance demonstrating that efficient retrieval approaches explored do not negatively impact task performance to a significant extent.

\noindent\textbf{Insight 2}: \textit{We find that extracting supporting facts improves efficiency across the entire pipeline, with Sparse setups achieving up to 2.0x speedups with only a minimal performance decline.}
\vspace{-1em}

\begin{table}[htb!]
\centering
\footnotesize
\begin{tabular}{l  c c c c }
\hline
\multirow{2}{*}{Method}  & \multicolumn{2}{c}{Total Latency}  & \multicolumn{2}{c}{Speedup} \\
\cline{2-5}
& CPU & GPU  &  CPU & GPU  \\ 
\hline \hline
\multicolumn{1}{l}{\colorg \textit{HoVer}} & \colorg & \colorg & \colorg & \colorg \\
 Full-Wiki (S+R) &  \multicolumn{2}{c}{659 ms} & - & - \\
Full-Wiki & 214  ms & 174 ms & 3.1x &  3.8x \\
% Original  &  55 ms & 13 ms  & - & 12 ms & 67 ms & 25 ms & 9.8x &  26.4x \\
Fact Extraction  & 60 ms & 21 ms & 11.0x & 31.4x  \\
Citation Extraction  & \textbf{51 ms} & \textbf{20 ms} & \textbf{\speedup{12.9x}} & \textbf{\speedup{33.0x}} \\
Fusion  & 63 ms & 24 ms &  10.5x & 27.5x \\
\hline
\multicolumn{1}{l}{\colorg\textit{WiCE}} & \colorg & \colorg & \colorg & \colorg   \\
 Full-Wiki (S+R) & \multicolumn{2}{c}{831  ms} & - & - \\
Full-Wiki &  292 ms & 238 ms & 2.8x  &  3.5x \\
% Original  &  95 ms & 43  ms  & - & 11 ms & 106 ms & 54 ms & 7.8x &  15.4x \\
Fact Extraction  & 103 ms & 48 ms & 8.1x & 17.3x \\
Citation Extraction  & 98 ms & 46 ms & 8.5x & 18.1x \\
Fusion  &\textbf{98 ms} & \textbf{46  ms} & \speedup{8.5x} &  \speedup{18.1x} \\
\hline
\end{tabular}
\caption{Latency and speedup measurements for Index compression setup. Speedup is compared with respect to the total latency of Sparse-retrieval + Re-rank (S+R) pipeline with the Full-Wiki setup. (S+R) runs on both CPU and GPU, sparse retrieval running on CPU and rest of components running on GPU}
%\vspace{-1cm}
\label{tab:jpq_latency}
\end{table}

%\begin{table}[htb!]
\centering
\footnotesize
\begin{tabular}{l c c c c c c c c}
\hline
\multirow{2}{*}{Method} & \multicolumn{2}{c}{\makecell{Term-based \\ document retrieval}} & \multirow{2}{*}{\makecell{Sentence \\ Retrieval}} & \multirow{2}{*}{\makecell{Claim \\ Verification}} & \multicolumn{2}{c}{Total Latency}  & \multicolumn{2}{c}{Speedup} \\
\cline{2-3}\cline{6-7}\cline{8-9}
& CPU & GPU &  & &  CPU & GPU & CPU & GPU \\ 
\hline \hline
 \multicolumn{1}{l}{\colorg\textit{HoVer}} & \colorg & \colorg & \colorg & \colorg & \colorg & \colorg & \colorg & \colorg \\
 \textbf{Full-Wiki (S+R)} &  \multicolumn{2}{c}{\textbf{491  ms}} & \textbf{157  ms} & \textbf{7 ms} &  \multicolumn{2}{c}{\textbf{659 ms}} & - & - \\
Full-Wiki &  53 ms & 13 ms & 153 ms & 8 ms & 214  ms & 174 ms & 3.1x &  3.8x \\
% Original  &  55 ms & 13 ms  & - & 12 ms & 67 ms & 25 ms & 9.8x &  26.4x \\
Claim detection  &  51 ms & 12  ms  & - &  9 ms & 60 ms & 21 ms & 11.0x & 31.4x  \\
Citation Extraction  & 46  ms & 11 ms  & - & 9 ms & 51 ms & 20 ms & 12.9x & 33.0x \\
Fusion  & 51  ms & 12 ms  & - & 12 ms & 63 ms & 24 ms &  10.5x & 27.5x \\
\hline
\multicolumn{1}{l}{\colorg\textit{WiCE}} & \colorg & \colorg & \colorg & \colorg & \colorg & \colorg & \colorg & \colorg  \\
 \textbf{Full-Wiki (S+R)} & \multicolumn{2}{c}{\textbf{636 ms}} & \textbf{186 ms} & \textbf{9  ms}  & \multicolumn{2}{c}{\textbf{831  ms}} & - & - \\
Full-Wiki & 97 ms & 43 ms & 186 ms & 9 ms & 292 ms & 238 ms & 2.8x  &  3.5x \\
% Original  &  95 ms & 43  ms  & - & 11 ms & 106 ms & 54 ms & 7.8x &  15.4x \\
Claim detection  &  92 ms & 37 ms  & - & 11 ms & 103 ms & 48 ms & 8.1x & 17.3x \\
Citation Extraction  & 89  ms & 37 ms  & - & 9 ms & 98 ms & 46 ms & 8.5x & 18.1x \\
Fusion  & 89  ms & 37  ms  & - &  9 ms & 98 ms & 46  ms & 8.5x &  18.1x \\
\hline
\end{tabular}
\caption{Retrieval and inference latency for Index compression setup. Speedup is compared to the total latency of (S+R) pipeline with Full-Wiki setup.}
\label{tab:jpq_latency}
\vspace{-2em}
\end{table}

% %%%%%%%%%%%%%%%%%%%%%%%%%%%%%%%%%%%%%%%%%%%%%%%%%%%%%%%%%%%%%%%%%%%%%%%%%%%%%%%%%%%
\vspace{-2em}
\subsection{Effectiveness of index compression on enhancing the efficiency of dense retrieval and fact-checking systems?}

To answer \textbf{RQ3}, we make use of index compression to further improve upon Dense Retrieval setups, not only with respect to memory requirements but also enhancing total inference latency compared to the sparse retrieve + re-rank setups in classical pipelines.  The index sizes of Wikipedia collection for standard dense retrieval are 7.51 GiB for HoVer and 9.70 GiB for WiCE. Using the JPQ index compression model with M=96 subvectors, we significantly reduced the storage space for vector embeddings from 1.5 KiB to 104.12 B. This reduced the HoVer index size to 544.89 MiB and the WiCE index to \textbf{672.95 MiB}, achieving a \textbf{93\% reduction (14.4:1 compression ratio)}. Further reducing subvectors could decrease the index size but may impact performance.


The utilization of JPQ index compression leads to significant reductions in retrieval latency compared to dense Retrieval and sparse retrieval, as demonstrated in \autoref{tab:jpq_latency}. CPU retrieval gains notable speedups of approximately 10.0x for HoVer experiments and 7.0x for WiCE experiments, while GPU retrieval shows about 2.0x and 0.8x speedups, respectively. These improvements are attributed to learned compression in JPQ, enhancing computational efficiency. 
Significant improvements are also observed when examining the inference latency of the whole pipeline. The CPU-based approaches shows impressive speedups (upto \textbf{12.9x} for HoVer and \textbf{8.5x} for WiCE), and GPU-based approaches achieve even higher gains (\textbf{33.0x} for HoVer and \textbf{18.1x} for WiCE). 

Surprisingly, in our experiments we observe that JPQ yields better results than standard Dense Retrieval as shown in Figure \ref{fig:performance_plot}. This is particularly due to joint training of the query encoder and index compression. In addition, JPQ employs end-end negative sampling, which further improves retrieval performance despite significant compression of embeddings.

\mpara{Insight 3}: \textit{We find that index compression reduces index size by \textbf{93\%} resulting in speedups for CPU-based setups up to 10x and GPU-based setups more than 20x compared to classical fact-checking pipeline.}

\subsection{Discussion of Live Fact-checking results}
We employ the pruned index (2024 Wiki collection) using our Fusion approach followed by compression of the index for live Fact-checking of 2024 presidential debate. The pipeline comprises a dense retrieval using compressed index followed by claim verification. We use the query encoder and NLI models trained on HoVer for this application. We compare this approach to also the classical sparse-retrieval+re-rank fact-checking pipeline over the Full-Wiki collection (without pruning). The task performance is shown in Figure \ref{fig:livefc} and the corresponding pipeline efficiency is shown in Table \ref{tab:livefc}. We observe that the pruned collection coupled with retrieval using index compression leads to impressive speedups (\textbf{3.4x}) over classical pipeline over the full collection without significant drop in task performance (Figure \ref{fig:livefc}). The results demonstrate that efficient retrieval is critical for live fact-checking. Our experiments demonstrate that our approach for efficient retrieval provides significant speedups on CPUs further making the technology accessible even in low-resource scenarios which has significant implications in aiding detection of misinformation and disinformation at scale.
\begin{figure}
\centering
 \hspace{6em}     \begin{subfigure}{.8\textwidth}
        

\begin{tikzpicture}
\edef\mylst{"56.95","56.66","56.94",""}
\edef\explora{"55.92","57.82","52.93",""}

    \begin{axis}[
            ybar=14pt,
            width=6cm,
            bar width=0.35,
            every axis plot/.append style={fill},
            grid=major,
            xtick={1, 4, 8,9,11},
            xticklabels={Sparse + re-rank, Dense, IC},
            ylabel style = {font=\small},
        yticklabel style = {font=\boldmath \tiny,xshift=0.05ex},
        xticklabel style ={font=\tiny,yshift=0.5ex},
            ylabel={Performance (F1 weighted)},
            enlarge x limits=0.15,
            ymin=0,
            ymax=86,
            legend style ={font=\tiny,yshift=0.05ex},
            area legend,
            nodes near coords style={font=\tiny,align=center,text width=1em},
            legend entries={Full-Wiki, Fusion},
            legend cell align={left},
            legend pos=north west,
            legend columns=-1,
            legend style={/tikz/every even column/.append style={column sep=0.06cm}},
        ]
        \addplot+[
            ybar,
            plotColor1*,
            postaction={
                    pattern=north east lines
                },
                    nodes near coords=\pgfmathsetmacro{\mystring}{{\mylst}[\coordindex]}\textbf{\mystring},
        ] plot coordinates {
                (1,56.95)
                (4,56.66)
                (8,56.94)
            };
        \addplot+[
            ybar,
            plotColor2*,
            postaction={
                    pattern=north east lines
                },
            nodes near coords=\pgfmathsetmacro{\mystring}{{\explora}[\coordindex]}\textbf{\mystring},
        ] plot coordinates {
                (1,55.92)
                (4,57.82)
                (8,52.83)
                (9,0)
            };

    \end{axis}
\end{tikzpicture}


    \end{subfigure}
    \caption{Live fact-checking performance across different corpus setups}
    \label{fig:livefc}
\end{figure}
\vspace{-2em}
\begin{table}[htb!]
\centering
\footnotesize % Reduced font size
\setlength{\tabcolsep}{3pt} % Reduce space between columns
\renewcommand{\arraystretch}{0.9} % Reduce space between rows
\begin{tabular}{l c c c c c c}
\multirow{2}{*} & \multicolumn{2}{c}{\makecell{ Retrieval}}   & \multicolumn{2}{c}{Total Latency} & \multicolumn{2}{c}{Speedup} \\
\cline{2-3}\cline{4-5}\cline{5-7} \\[-1mm]
& CPU & GPU & CPU& GPU& CPU & GPU \\
\hline \\

% Term-based Document Retrieval
\colorg\textit{Sparse + Re-ranking} & \colorg & \colorg & \colorg  & \colorg & \colorg & \colorg\\ 
Full-Wiki & 463  & -   & 695  & - & \multicolumn{2}{c}{-} \\
Fusion & 274  & -   & 479   & - & \multicolumn{2}{c}{1.5x} \\
\hline \\
% Dense Retrieval setup
\colorg\textit{Dense Retrieval} & \colorg & \colorg & \colorg & \colorg & \colorg  & \colorg\\ 
Full-Wiki &  433   & 32  & 553   &  152  & 1.3x & 4.6x \\
Fusion & 412   & 32 & 511  & 131  & 1.4x & 5.3x \\
\hline \\
% Index Compression setup
\colorg\textit{Index Compression} & \colorg & \colorg & \colorg & \colorg & \colorg & \colorg\\  
Full-Wiki & 100  & 50   & 228  & 178  & 3.0x & 3.9x \\
\textbf{Fusion (ours)} & \textbf{89 }   & \textbf{43}  & \textbf{203}  & 157  & \speedup{3.4x} & 4.4x \\

\hline
\end{tabular}
\caption{Latency Comparisons for Live Fact-checking (in milliseconds (ms))}

\label{tab:livefc}
\end{table}


% \section{RQ 1: How does indexing supporting facts improve information retrieval efficiency?}
% In this section, we investigate the impact of indexing supporting facts on information retrieval efficiency by comparing the disk space utilization and retrieval latency across different experiment settings. Here we aim to discern the benefits of storing only supporting facts in the index as opposed to the entire corpus. 

% \subsection{Corpus Size}
% \begin{table}[htb!]
\small
\centering
\footnotesize
\begin{tabular}{c c c c c}
\toprule
\small
\textbf{Method} & \textbf{Disk Size} & \textbf{Size reduction} & \textbf{\#Sentences} & \textbf{\% decrease}\\
\hline \hline
\multicolumn{1}{l}{\colorg\textbf{HoVer}} & \colorg& \colorg & \colorg & \colorg  \\
Full-Wiki  &  11.28 GiB & - & 94,914,378 & -   \\
Fact Extraction & 6.19 GiB & \down{45}& 45,894,704 & \down{52} \\
Citation Extraction & \textbf{5.07 GiB} & \down{\textbf{55} } & \textbf{36,886,889} & \down{\textbf{61}} \\
Fusion & 5.45 GiB & \down{52} & 39,842,574 & \down{{58}} \\

\hline
\multicolumn{1}{l}{\colorg \textbf{WiCE}} & \colorg &  \colorg & \colorg & \colorg \\
Full-Wiki & 15.28 GiB & - &  126,533,841 & -  \\
Fact Extraction & 8.56 GiB & \down{44} & 61,040,380 & \down{52}\\
Citation Extraction & \textbf{6.56 GiB} & \down{\textbf{57}} & \textbf{51,735,961} & \down{\textbf{59}}  \\
Fusion & 6.85 GiB & \down{55}& 54,070,295 & \down{57} \\

\hline
\end{tabular}
\caption{Comparison sizes for the corpora per experiment setting, consisting of English Wikipedia articles 2017 (HoVer) and 2024 (WiCE). Reduction is measured compared to the  Full-Wiki data setting. \down{} denotes a reduction in corpus size and number of sentence compared to Full-Wiki setting.}
\label{fig:disk_size}
\end{table}
\vspace{-2em}
% To get an idea of how storing just the supporting facts data in the index improves efficiency compared to storing the entire corpus, a comparison can be made on how much these different settings occupy disk space. As mentioned in \autoref{sec:metrics}, to get an accurate estimate, only the dictionaries containing the article's title and document text are saved to raw JSON files. Across all experiment settings as seen in \autoref{fig:disk_size},  a notable reduction in disk space usage is observed compared to the original Wikipedia document corpus. This reduction ranges from approximately 45\% (claim detection) to 55\% depending on the setting for the HoVer corpus data. Likewise, for the WiCE corpus data, we can observe approximately 44\% to 57\% reduction. Moreover, in correlation with the reduced disk size, it is evident that the number of sentences stored in the index also decreases across each experiment setting compared to the original corpus data. For HoVer this ranges from 52\%  (claim detection) to 61\% (citation extraction) and WiCE ranges from 52\% to 59\%. This indicates that at least half of the sentences are considered as not claim-worthy across the different re-ranking methods.

% \subsection{Retrieval Latency}\label{ssec:retrieval_latency}
% \paragraph{Sparse retrieval} Following the reduction in disk size, a notable enhancement in retrieval latency is evident, as demonstrated in both the Term-based and Neural-based document retrieval columns of \autoref{tab:bm25_latency}. To avoid any ambiguity, it's crucial to clarify that the speedup listed in the table pertains to the total latency, which is relevant for addressing RQ2, rather than solely focusing on document retrieval.
% Regarding document retrieval latency (which encompasses both column values), there's an observed speedup ranging from approximately 1.5x (334 ms) to 1.6x (316 ms) compared to the original experimental setting for HoVer (495 ms). Similarly, in WiCE experiments, we witness a comparable speedup rate ranging from 1.4x (446 ms) to 1.6x (399 ms) compared to the original experimental setting (636 ms). This observation suggests that while the reduced text size contributes to expedited retrieval, the enhancement is only somewhat proportional.

% \begin{table}[htb!]
\centering
\small
\footnotesize
% \vspace{-1cm}
\begin{tabular}{l c c c c}
\multirow{2}{*}{\makecell{\textbf{Methods}}} & \multirow{2}{*}{\textbf{Retrieval}} & \multirow{2}{*}{\makecell{\textbf{Total Latency}}} & \multirow{2}{*}{\makecell{\textbf{Speedup}}} \\
& \\
\hline
\multicolumn{1}{l}{\colorg\textit{HoVer}} & \colorg & \colorg & \colorg \\
 Full-Wiki & 426  ms & 659 ms & - \\
Fact Extraction & 257 ms  & 338  ms & 1.9x \\
Citation Extraction & 246 ms  &  327  ms & \speedup{2.0x}  \\
Fusion & 265 ms  & 345  ms & 1.9x  \\

\multicolumn{1}{l}{\colorg\textit{WiCE}} & \colorg & \colorg & \colorg \\
  Full-Wiki &  559 ms  &  831  ms & - \\
Fact Extraction & 372 ms  & 468   ms & 1.8x \\
Citation Extraction & 330 ms   & 419  ms & \speedup{2.0x} \\
Fusion & 347 ms & 436  ms & 1.9x \\
\hline
\end{tabular}
\caption{Retrieval and total latency for Sparse retrieval with Re-ranking. Speedup is compared with respect to the total latency of the Full-Wiki setup.}
\label{tab:bm25_latency}
\vspace{-2em}
\end{table}
% \paragraph{CPU-based Dense Retrieval} One might typically anticipate a more pronounced disparity between the original data and the reranked data in the document retrieval phase. However when transitioning from the Sparse retrieval setup to the Dense retrieval setup, as depicted in the first column of \autoref{tab:faiss_latency}, only negligible differences between the different settings are observed. This is attributed to FAISS utilizing vectors instead of computing the relevance ranking of documents to the query, as is the case with BM25. Despite variations in the length of each article across settings, the number of text embeddings (with fixed dimensionality size) created remains constant, corresponding to the number of encoded text spans, which is consistent across settings. Thus minimizing the impact of extracting supporting facts on document retrieval latency when using Dense Retrieval. Comparing the Dense document retrieval (CPU) column in \autoref{tab:faiss_latency} to the baselines listed in \autoref{tab:bm25_latency}, it is observed to be of a similar latency or even slightly slower. For HoVer, we can observe a 0.9x (523 ms) to 1.0x (479 ms) compared to the baseline (495 ms). Likewise, for WiCE we can observe a similar latency speedup of 0.9x (685 ms) to 1.0x (610 ms) speedup compared to its baseline (636 ms). This suggests that the indexing of supporting facts would not significantly impact information retrieval efficiency in such scenarios. 

% \begin{table}[htpb!]
\centering
\footnotesize
\begin{tabular}{l c c c c c c c c}
\multirow{2}{*}{} & \multicolumn{2}{c}{\makecell{Document retrieval}} & \multirow{2}{*}{\makecell{Sentence \\ Retrieval}} & \multirow{2}{*}{\makecell{Claim \\ Verification}} & \multicolumn{2}{c}{Total Latency} & \multicolumn{2}{c}{Speedup} \\
\cline{2-3}\cline{6-7}\cline{8-9}
& CPU & GPU &  & &  CPU & GPU & CPU & GPU \\ 
\hline
\multicolumn{1}{l}{\textit{HoVer}} &  \\
 \textbf{Full-Wiki (S+R) } &  \multicolumn{2}{c}{\textbf{491  ms}} & \textbf{157  ms} & \textbf{7 ms} &  \multicolumn{2}{c}{\textbf{659 ms}} & - & - \\

Full-Wiki & 515 ms & 31 ms & 153 ms & 8 ms & 676 ms & 192  ms & 1.0x & 3.4x \\
% Original  &  523 ms & 30 ms  & - & 9 ms & 532 ms & 39 ms & 1.2x & 16.9x \\
Claim detection & 513 ms & 23 ms & - & 8 ms & 521 ms & 31 ms & 1.3x & 21.3x  \\
Citation Extraction & 479 ms & 23 ms & - & 9 ms &  488 ms & 32 ms & 1.4x & 20.6x \\
Fusion & 500 ms  & 23 ms & - & 9 ms & 509 ms & 32  ms & 1.3x & 20.6x \\

\multicolumn{1}{l}{\textit{WiCE}} & \\
 \textbf{Full-Wiki (S+R)} & \multicolumn{2}{c}{\textbf{636 ms}} & \textbf{186 ms} & \textbf{9  ms}  & \multicolumn{2}{c}{\textbf{831  ms}} & - & - \\
Full-Wiki & 685 ms & 34 ms & 184 ms & 9 ms & 878 ms & 227  ms & 1.0x & 3.7x \\
% Original  & 610  ms & 34 ms  & - & 9 ms & 619  ms & 43 ms & 1.3x & 19.3x \\
Claim detection &  622 ms & 31 ms  & - & 9 ms & 631 ms & 40 ms & 1.3x & 20.8x \\
Citation Extraction &  610 ms & 31 ms  & - & 9 ms & 619 ms & 40 ms & 1.3x & 20.8x  \\
Fusion & 619  ms & 31 ms  & - & 9 ms & 628 ms & 40 ms & 1.3x & 20.8x  \\[5mm]
\hline
\end{tabular}
\caption{Retrieval and inference latency for Dense retrieval setup on data settings. Speedup is compared with respect to the total latency of the Sparse Retrieval setup with original data setting (bold font).}
\label{tab:faiss_latency}
\end{table}




% \paragraph{GPU-based Dense Retrieval} However, it is worth noting that Dense retrieval can still be faster, particularly with dense retrieval libraries such as FAISS offering GPU support, which can yield substantial speedups compared to both CPU retrieval of BM25 and FAISS. This advantage is evident in the data, showcasing notable speedups ranging from 16.6x to 22.3x speedup for HoVer GPU retrieval over CPU retrieval, and 17.9x to 20.2x speedup for WiCE. Furthermore, when comparing FAISS GPU retrieval to the BM25 retrieval, we can see an approximate 16.0x (31 ms) to 21.5x (23 ms) speedup for HoVer and 18.7x (34 ms) to 20.5x (31 ms) speedup for WiCE. Therefore the GPU-based approach makes Dense Retrieval a viable option, unlike the CPU-based variant. 

% \subsection{Key Takeaways} 
% Extracting supporting facts from the data corpus can lead to only requiring to store at least half of the data. Although this has a positive effect on the latency for Sparse retrieval, with Dense document retrieval this is not the case due to how the vector embeddings are constructed (being per article rather than per sentence). Furthermore, while CPU-based Dense retrieval may not necessarily outperform Sparse retrieval methods in terms of latency, thereby presenting less immediate appeal, the incorporation of GPU support leads to significant speed enhancements. Thus, although extracting supporting facts does not help much in Dense document retrieval unlike Sparse retrieval in terms of retrieval latency, the incorporation of the GPU-based Dense retrieval renders it a much more compelling option for achieving efficiency. 

% %%%%%%%%%%%%%%%%%%%%%%%%%%%%%%%%%%%%%%%%%%%%%%%%%%%%%%%%%%%%%%%%%%%%%%%%%%%%%%%%%%%

% \section{RQ 2: How does indexing supporting facts affect overall pipeline efficiency and downstream fact-checking performance?}
% In continuation of the previous research inquiry concerning retrieval latency and disk size, this section delves into an analysis of the overall inference time across the entire pipeline. Additionally, recognizing that faster processing times do not necessarily equate to better performance a further analysis will be done on the performance metrics.

% \subsection{Inference Latency}
% \paragraph{Sparse Retrieval Setup:} The enhancement in retrieval latency, as evidenced in \autoref{tab:bm25_latency}, mirrors a noticeable improvement in the overall inference latency across the pipeline. This improvement spans approximately 1.9x to 2.0x for the HoVer experiments and 1.8x to 2.0x for WiCE experiments. However, the reduction in total latency cannot be solely ascribed to faster retrieval times. It also arises from the elimination of the Sentence Retrieval stage, which previously imposed significant latency overhead. Upon closer inspection of \autoref{tab:bm25_latency}, it becomes apparent that the absence of the Sentence Retrieval stage impacts the Claim Verification stage. Notably, experiments conducted on the original corpus data exhibit much lower inference latency compared to supporting facts data. Nevertheless, the variance between these experiment settings is minimal, and the impact on total latency results is insignificant. This overall trend indicates that indexing supporting facts for the BM25 retrieval setup predominantly benefits inference times for the Rule-based document retrieval and Sentence Retrieval stages. Furthermore, it reveals that the Claim Verification stage is slightly, yet negligibly, affected when considering the entire pipeline inference.

% \paragraph{Dense Retrieval Setup:} In a similar vein as the document retrieval comparisons of RQ1 (see \autoref{ssec:retrieval_latency}), the total inference of the Dense retrieval setup presents notable differences in results between CPU- and GPU-based Dense retrieval compared to Sparse retrieval. This divergence is evident in \autoref{tab:faiss_latency}, where for HoVer experiments, the CPU-based approach exhibits a 1.2x to 1.4x speedup, while the GPU-based approach demonstrates a 16.9x to 21.3x speedup compared to the baseline. Similarly, WiCE experiments show approximately a 1.3x speedup for the CPU-based approach and 19.3x to 20.8x speedup for the GPU-based approach. The key distinction lies in the influence of omitting the Sentence Retrieval stage for the original corpus data. Its omission introduces significant overhead to the total latency. For the CPU-based approach, this translates to a 1.3x speedup for HoVer (676 ms vs. 532 ms) and a 1.4x speedup for WiCE (878 ms vs. 619 ms). Conversely, the GPU-based approach experiences a 4.9x speedup for HoVer (192 ms vs. 39 ms) and a 5.3x speedup for WiCE (227 ms vs. 43 ms). Overall, this underscores that including Sentence Retrieval adds substantial overhead, especially for GPU-based approaches operating with lower latency magnitudes. Therefore, the supporting facts data for Dense Retrieval, while not significantly impacting document retrieval, offers significant speedup for total inference latency, allowing for the effective omission of the Sentence Retrieval stage and its associated latency overhead.

% \subsection{Performance Metrics Evaluation}

% \begin{table}[htb!]
\small
\begin{tabular}{c c c c c c c c}
\multirow{2}{*}{Experiment setting} & \multirow{2}{*}{Accuracy} & \multicolumn{2}{c}{F1} & \multicolumn{2}{c}{Precision} & \multicolumn{2}{c}{Recall}  \\ 
\cline{3-8}
  & &  Weighted  & Macro & Weighted & Macro & Weighted & Macro      \\
\hline
\multicolumn{1}{l}{\textit{Sparse + Re-ranking}} & & & & \\
Full-Wiki & \textbf{67.79} & \textbf{67.59} & \textbf{67.63} & \textbf{68.45} & \textbf{68.39} & \textbf{67.79}  & \textbf{67.93}\\
Claim detection & \underline{62.33} & 62.02 & 62.08 & \underline{62.98} & \underline{62.92} & \underline{62.33} & \underline{62.50} \\
Citation Extraction & 60.91 & 60.61 & 60.66 & 61.47 & 61.42 & 60.91 & 61.07 \\
Fusion & 62.28 & \underline{62.15} & \underline{62.18} & 62.60 & 62.56 & 62.28 & 62.39  \\[5mm]

\hline
\multicolumn{1}{l}{\textit{Dense Retrieval}} & & & & \\
Full-Wiki & 64.60 & 64.45 & 64.45 & 64.86 & 64.86 & 64.60 & 64.60 \\
% Original & 62.90 & 62.72 & 62.76 & 63.33 & 63.28 & 62.90 & 63.02 \\
Claim detection & \underline{61.50} & \underline{60.94} & \underline{60.94} & \underline{62.20} & \underline{62.20} & \underline{61.50} & \underline{61.50} \\
Citation Extraction & 59.67 & 59.40 & 59.46 & 60.13 & 60.09 & 59.67 & 59.82 \\
Fusion & 59.51 & 59.32 & 59.37 & 59.85 & 59.81 & 59.51 & 59.64  \\[5mm]

\hline
\multicolumn{1}{l}{\textit{Index Compression}} & & & & & & &  \\
Full-Wiki & 63.30 & 62.54 & 62.54 & 64.48 & 64.48 & 63.30 & 63.30 \\
% Original & 63.02 & 62.08 & 62.08 & 64.46 & 64.46 & 63.02 & 63.02  \\
Claim detection & \underline{61.92} & \underline{61.71} & \underline{61.71} & \underline{62.19} & \underline{62.19} & \underline{61.92} & \underline{61.93}  \\
Citation Extraction & 59.98 & 59.12 & 59.12 & 60.89 & 60.89 & 59.98 & 59.98   \\
Fusion & 61.58 & 61.43 & 61.43 & 61.75 & 61.75 & 61.58 & 61.58   \\[5mm]

\hline
\end{tabular}
\caption{Performance experiments on HoVer data and adjustments using full document text of English Wikipedia. The underlined-styled values represent the second best  within each retrieval setup.}
\label{tab:hover_performance_metrics}
\end{table}
% Full document text 
% 

% \begin{table}[htb!]
\centering
\footnotesize
\begin{tabular}{c c c c c c c c}
\multirow{2}{*}{Experiment setting} & \multirow{2}{*}{Accuracy} & \multicolumn{2}{c}{F1} & \multicolumn{2}{c}{Precision} & \multicolumn{2}{c}{Recall}  \\ 
\cline{3-8}
  & &  Weighted  & Macro & Weighted & Macro & Weighted & Macro      \\
\hline
\multicolumn{1}{l}{\textit{Sparse + Re-ranking}} & & & & \\
Full-Wiki & \textbf{63.69} &  \textbf{61.84} &  \textbf{55.24} &  \textbf{61.12} &  \textbf{56.54} &  \textbf{63.69} &  \textbf{55.32 } \\
Claim detection & 61.90 & 60.12 & \underline{53.33} & 59.27 & 54.26 & 61.90 & \underline{53.53} \\
Citation Extraction & 61.01 & 59.56 & 52.96 & 58.75 & 53.59 & 61.01 & 53.09 \\
Fusion & \underline{63.39} & \underline{60.21} & 52.48 & \underline{59.46} & \underline{54.69} & \underline{63.39} & 53.27  \\[5mm]

\hline
\multicolumn{1}{l}{\textit{Dense Retrieval}} & & & & \\
Full-Wiki &  61.61 & 60.95 & 55.21 & 60.47 & 55.49 & 61.61 & 55.13 \\
% Full-Wiki  & 60.42 & 58.80 & 51.96 & 57.90 & 52.58 & 60.42 & 52.19 \\
Claim detection & 61.01 & 58.94 & 51.78 & 57.96 & 52.70 & 61.01 & 52.17 \\
Citation Extraction & 58.63 & 58.48 & \underline{52.92} & 58.35 & 52.95 & 58.63 & \underline{52.91} \\
Fusion & \underline{61.31} & \underline{59.34} & 52.30 & 58.40 & \underline{53.23} & \underline{61.31} & 52.62  \\[5mm]

\hline
\multicolumn{1}{l}{\textit{Index Compression}} & & & & & & &  \\
Full-Wiki & 62.46 & 61.38 & 55.27 & 60.74 & 55.84 & 62.46 & 55.20  \\
% Original  & 60.46 & 60.63 & 55.64 & 60.81 & 55.60 & 60.46 & 55.70  \\
Claim detection & 59.31 & 59.02 & \underline{53.32} & 58.77 & 53.39 & 59.31 & \underline{53.30}  \\
Citation Extraction & 60.74 & 59.21 & 52.42 & 58.34 & 53.04 & 60.74 & 52.60  \\
Fusion & \underline{63.04} & \underline{59.79} & 51.89 & 58.94 & \underline{53.97} & \underline{63.04} & 52.76 \\[5mm]

\hline
\end{tabular}
\caption{Performance experiments on WiCE data and adjustments using full document text of English Wikipedia. The bold-styled values represent the baseline while the underlined-styled values represent the highest scores of the re-ranked data within a retrieval setup category.}
\label{tab:wice_performance_metrics}
\end{table}
% Full document text 
% 

% \paragraph{Sparse Retrieval performance} Utilising the metrics laid out in \autoref{sec:metrics}, the pipeline results have been evaluated for the different settings and laid out in \autoref{tab:hover_performance_metrics} for the HoVer experiments and \autoref{tab:wice_performance_metrics} for WiCE experiments. When comparing the different HoVer experiment settings within the Sparse Retrieval setup, Claim detection comes the closest to the baseline with close to 5.5 points difference across the metrics for the HoVer experiments. Important to note is that Fusion follows close with less than a point difference. For the WiCE Sparse retrieval setup, the opposite occurred with the Fusion data being the closest with a marginal 0.3 point difference followed by Claim detection with a 1.5 points difference. In both datasets, the Citation extraction takes the biggest loss in accuracy that being 6.9 points for HoVer and 2.7 points difference for WiCE. We can reason the fact that citation extraction takes the biggest performance degradation to the fact that not all claim-worthy sentences contain citations, therefore missing out on crucial evidence sentences. Unlike the other settings which consider the complete text instead of only the cited sentences and determine claim-worthiness on what the claim-detection model selects. Overall, relating to the inference time, we can see that for HoVer with a speedup of at least 1.5x to 1.6x, we only lose 6.9 to 5.5 points in performance across various metrics for the best re-ranking setup. Likewise, for WiCE, with a speedup of 1.4x to 1.6x we only lose 2.7 to 0.3 points. This positively demonstrates that indexing just the supporting facts does show meaningful results in terms of overall pipeline efficiency, while maintaining roughly the same performance. Additionally, this also indicates we can achieve good results by using a combination of citation extraction together with another supporting facts extraction method such as Claim detection.

% \paragraph{Dense Retrieval performance} When examining the performance results of Dense retrieval compared to Sparse Retrieval, it becomes evident that there is a slight decline across all experiments. For HoVer, this decline ranges from a modest 0.8 point difference in Claim detection to a more substantial 2.9 points in Fusion data. Similarly, WiCE experiences a loss ranging from a 0.9 difference in accuracy between Claim detection settings to approximately 2.4 points in Citation extraction. Crucially, it is to assess how these performances compare against the baselines. In HoVer, the accuracy loss ranges from 8.3 points for Fusion data to 6.3 points for Claim detection. WiCE experiences a loss ranging from 5.1 points in Citation extraction to 2.4 points in Fusion. These findings suggest that while transitioning from Sparse retrieval to just a Dense retrieval component incurs some loss, it's not substantial across various experiments involving supporting facts data. Moreover, the performance is notably strong in claim detection, while citation extraction lags behind by only a few points. Interestingly, while Fusion performs as well as Citation extraction in HoVer experiments, Fusion data outperforms Claim detection in WiCE. This highlights the significance of combining citation extraction with another supporting facts extraction method to achieve optimal results, similar to the Sparse retrieval setup.

% \paragraph{Sentence Retrieval stage ablation} Comparing experiments on the original data between the two retrieval methods reveals a more significant decline for HoVer, with a loss of 3.2 points with Sentence Selection and 4.9 points without it. For WiCE, the difference is 2.1 points with Sentence Selection and 3.3 points without it. When assessing these losses against the baselines, it becomes evident that both methods generally outperform the supporting facts data experiments by a few points. This suggests that the contribution of the Sentence Retrieval stage in the pipeline to performance improvement is marginal. With the supporting facts extraction thus becomes quite effective in achieving nearly the same performance. Consequently, to enhance efficiency, eliminating this Sentence Retrieval stage would result in only a loss of less than a few points.

% \subsection{Key Takeaways} 
% Incorporating supporting facts into both Sparse and Dense retrieval setups yields noteworthy enhancements in overall pipeline efficiency. Sparse retrieval setups demonstrate speedups ranging from up to around 1.5x, while Dense retrieval setups exhibit even more substantial improvements, achieving up to approximately 20.0x with GPU-based approaches. These notable speedups are primarily attributed to the removal of the Sentence Retrieval stage, which incurs considerable latency overhead. Further evaluation indicates a minor decline in performance when transitioning from Sparse to Dense retrieval, though the loss is not substantial. Specifically, claim detection remains robust, while citation extraction may lag behind by a few points. However, Fusion data yields promising results, often comparable to or outperforming other extraction methods, emphasizing the significance of amalgamating various extraction techniques for supporting facts. Moreover, ablation experiments on the Sentence Retrieval stage reveal its marginal contribution to performance improvement. Comparisons between original data and supporting facts data show only a slight decline in performance, showcasing that utilising only the supporting facts only incurs a modest loss in performance (around 6 points for HoVer and 3 points for WiCE). This suggests that although supporting facts do not affect document retrieval latency in the Dense Retrieval setup, it does help with overall pipeline latency due to avoiding the latency overhead of Sentence Selection. In conclusion, these results underscore the meaningful impact of indexing supporting facts on the overall pipeline efficiency, with only minimal losses in downstream fact-checking performance.

% %%%%%%%%%%%%%%%%%%%%%%%%%%%%%%%%%%%%%%%%%%%%%%%%%%%%%%%%%%%%%%%%%%%%%%%%%%%%%%%%%%%

% \section{RQ 3: In what ways does index compression enhance the efficiency of dense retrieval and fact-checking systems?}

% In this final research inquiry concerning the addition of index compression, this section explores how index compression improves upon Dense Retrieval in not only the constructed index size, but also document retrieval and total inference latency. Additionally, a final comparison will be made on the overall performance against Sparse retrieval and standard Dense Retrieval.

% \subsection{Compressed Index Size}
% In our FAISS experiments, we consistently observe an index size of approximately 7.51 GiB across all HoVer settings and 9.70 GiB across all WiCE settings. While one might anticipate that re-ranking would influence the amount of text utilized for generating vector embeddings, it's crucial to note that the index size remains unchanged. This is due to the fact that we generate vector embeddings on a per-article basis with only the text itself being altered. To address this issue, we employed JPQ, an index compression model. Despite using a relatively high number of subvectors for the JPQ model (M=96), we observed a significant reduction in the total index size. Specifically, the individual vector embeddings now occupy only 104.12 B in storage space, down from 1.5 KiB previously. This reduction is remarkable. For the HoVer experiments, the index size decreased from 7.51 GiB to 544.89 MiB, and for the WiCE experiments, we observed a decrease from 9.70 GiB to 672.95 MiB. Overall, this constitutes an impressive reduction of nearly 93\% or a compression ratio of 14.4:1 in index size for both experiment setups. It's worth noting that employing fewer sub-vectors could potentially lead to an even more substantial reduction in index size; however, this would come at the cost of decreased performance.

% \begin{table}[htb!]
\centering
\footnotesize
\begin{tabular}{l c c c c c c c c}
\hline
\multirow{2}{*}{Method} & \multicolumn{2}{c}{\makecell{Term-based \\ document retrieval}} & \multirow{2}{*}{\makecell{Sentence \\ Retrieval}} & \multirow{2}{*}{\makecell{Claim \\ Verification}} & \multicolumn{2}{c}{Total Latency}  & \multicolumn{2}{c}{Speedup} \\
\cline{2-3}\cline{6-7}\cline{8-9}
& CPU & GPU &  & &  CPU & GPU & CPU & GPU \\ 
\hline \hline
 \multicolumn{1}{l}{\colorg\textit{HoVer}} & \colorg & \colorg & \colorg & \colorg & \colorg & \colorg & \colorg & \colorg \\
 \textbf{Full-Wiki (S+R)} &  \multicolumn{2}{c}{\textbf{491  ms}} & \textbf{157  ms} & \textbf{7 ms} &  \multicolumn{2}{c}{\textbf{659 ms}} & - & - \\
Full-Wiki &  53 ms & 13 ms & 153 ms & 8 ms & 214  ms & 174 ms & 3.1x &  3.8x \\
% Original  &  55 ms & 13 ms  & - & 12 ms & 67 ms & 25 ms & 9.8x &  26.4x \\
Claim detection  &  51 ms & 12  ms  & - &  9 ms & 60 ms & 21 ms & 11.0x & 31.4x  \\
Citation Extraction  & 46  ms & 11 ms  & - & 9 ms & 51 ms & 20 ms & 12.9x & 33.0x \\
Fusion  & 51  ms & 12 ms  & - & 12 ms & 63 ms & 24 ms &  10.5x & 27.5x \\
\hline
\multicolumn{1}{l}{\colorg\textit{WiCE}} & \colorg & \colorg & \colorg & \colorg & \colorg & \colorg & \colorg & \colorg  \\
 \textbf{Full-Wiki (S+R)} & \multicolumn{2}{c}{\textbf{636 ms}} & \textbf{186 ms} & \textbf{9  ms}  & \multicolumn{2}{c}{\textbf{831  ms}} & - & - \\
Full-Wiki & 97 ms & 43 ms & 186 ms & 9 ms & 292 ms & 238 ms & 2.8x  &  3.5x \\
% Original  &  95 ms & 43  ms  & - & 11 ms & 106 ms & 54 ms & 7.8x &  15.4x \\
Claim detection  &  92 ms & 37 ms  & - & 11 ms & 103 ms & 48 ms & 8.1x & 17.3x \\
Citation Extraction  & 89  ms & 37 ms  & - & 9 ms & 98 ms & 46 ms & 8.5x & 18.1x \\
Fusion  & 89  ms & 37  ms  & - &  9 ms & 98 ms & 46  ms & 8.5x &  18.1x \\
\hline
\end{tabular}
\caption{Retrieval and inference latency for Index compression setup. Speedup is compared to the total latency of (S+R) pipeline with Full-Wiki setup.}
\label{tab:jpq_latency}
\vspace{-2em}
\end{table}



% \subsection{Pipeline Efficiency}
%  \paragraph{Document Retrieval Latency} When examining the retrieval latency outlined in \autoref{tab:jpq_latency}, a notable observation can be made towards the Dense document retrieval compared to the Dense Retrieval results outlined in \autoref{tab:faiss_latency}. This significant enhancement can be primarily attributed to the utilization of the index compression model, which effectively reduces the index size. As a result, retrieval latency experiences a considerable improvement due to the smaller vector embeddings, facilitating faster similarity computation. Here one can observe a substantial speedup achieved in CPU retrieval of approximately 10.0x across the HoVer experiment settings and 7.0x for WiCE experiments. Similarly, GPU retrieval exhibits a speedup of approximately 2.0x for HoVer experiments and 0.8x for WiCE experiments. This is generally in line with the reported results in the original JPQ paper \cite{zhan2021jointly}. Although the measurements for HoVer fall in line with these reported results, one may notice that the WiCE retrieval speedup is lower than that of HoVer. This is even worse for the GPU-based retrieval latency instead of being better than the standard GPU-based Dense retrieval. We reason this to the fact that the WiCE claim dataset is a lot more complex. In the original WiCE paper, the results that were reported already indicate a not so particularly high performance being achieved. This coupled with the use of a different model for creating the embeddings results in marginally worse performance instead of a speedup such as the case with HoVer. 
 
%  \paragraph{Pipeline Inference Latency} In examining the total inference latency, as further detailed in \autoref{tab:jpq_latency}, the utilization of compressed indexing and the ensuing document retrieval speed enhancements result in a notable boost across the board. The advancements brought about by JPQ, which further build upon the foundations of Dense Retrieval, are particularly significant. Notably, CPU latency has seen a substantial improvement compared to previous benchmarks on the supporting fact data, exhibiting a noteworthy speedup ranging from 10.5x to 12.9x for HoVer experiments, and 8.1x to 8.5x for WiCE experiments relative to their respective baselines. Meanwhile, the GPU-based approach, especially in the case of HoVer experiments, has yielded even more impressive results, achieving speedups ranging from 27.5x to 33.0x. While WiCE experiments on the GPU may not experience such dramatic speedups, they still showcase marked enhancements over their original baselines that range from 17.3x to 18.1x speedups. When assessing the impact of the Sentence Selection stage on the original corpus data settings, the findings reinforce the observations made in the standard Dense Retrieval setup. Furthermore, the disparity in the reported speedups between the tables underscores the significance of incorporating index compression. 
  
% \subsection{Performance Metrics Evaluation}
% When comparing the performance of JPQ in the HoVer experiments (as shown in \autoref{tab:hover_performance_metrics}) as well as the performance of the WiCE experiments (presented in \autoref{tab:wice_performance_metrics}), a notable trend emerges. The index compression brought by JPQ generally yields higher scores compared to the standard Dense retrieval experiments. This improvement is particularly striking as the gap between the JPQ experiments and the baseline performances is further narrowed. In the HoVer experiments, this enhancement ranges from marginal increases of less than a point in Claim detection and Citation extraction to a significant 2-point boost in the Fusion data. Conversely, in the WiCE experiments, while Claim detection experiences a slight decline of almost 2 points, Citation extraction and Fusion demonstrate the opposite trend.
% Typically, one might expect index compression techniques to yield inferior results compared to the standard Dense retrieval setup due to the lossy nature of compressing embeddings. However, a straightforward explanation for this unexpected improvement lies in the utilization of different pre-trained models for generating the embeddings. In the standard Dense retrieval, we rely on the all-MiniLM-L6-v2 model, which maps sentences and paragraphs to a 384-dimensional dense vector space. In contrast, the JPQ model employed for index compression initially generates embeddings of size 768 and subsequently reduces the embedding size using PQ centroids to achieve smaller vector sizes. Furthermore, it's worth noting that JPQ learns the index for the query vectors, unlike the approach in standard Dense retrieval where the index is kept separate. The latter essentially operates in a zero-shot inference manner, as we do not fine-tune the encoders on specific datasets but instead store and retrieve the created embeddings directly in our FAISS setup.


% \subsection{Key Takeaways} 
% Enhancing Dense retrieval through the use of index compression via the JPQ model has remarkably reduced the index size for Dense retrieval by a substantial 93\%. Further analysis indicates significant speedups of up to 10.0x for the CPU-based approach, while the GPU-based approach achieves a modest speedup of up to 2.0x in the HoVer experiments. However, it experiences a slight slowdown in the WiCE experiments. A huge emphasis on achieving efficiency is particularly pertinent in the context of CPU-based Dense Retrieval with index compression. Here the latency times of the CPU-based approach come in close to the GPU-based approach. These findings not only signify efficiency gains concerning resource utilization for index storage, but also pave the way for experiments on lower-end machines especially ones lacking GPU capabilities. Thereby maximizing the benefits of CPU-based methodologies. Regarding performance, experiments involving index compression generally outperform standard Dense retrieval. This superiority can be attributed to the utilization of different pre-trained models and learned index techniques, resulting in slightly enhanced outcomes.

\section{Conclusion and Discussion}
\label{sec:discussion}
We present an analysis of hypothetical and disjunctive syllogisms on propositional and modal logic and systematically analyze the LLM performance on the dataset.
Our analysis provides novel insights on explaining and predicting LLM performance: in addition to the perplexity or probability of the input text, the underlying logic forms play an important role in determining the performance of LLMs.
In addition, we compare the behaviors of LLMs and humans using the same data through human behavioral experiments.
We discuss the implications of our results as follows.

\vspace{2pt}
\noindent\textbf{Probability in language models.}
Probability and perplexity are often used as intrinsic evaluation metrics for language models.
While \citet{gonen-etal-2023-demystifying} and \citet{mccoyEmbersAutoregressionShow2024} show that probability and perplexity correlate well with LLM performance, literature in program synthesis with LLMs shows little correlation between probability and execution-based evaluation results \citep{li2022competition,shi-etal-2022-natural}.
This work does not necessarily contradict either line but rather provides complementary factors for analyzing LLM performance.

We argue that probability may have become an overloaded term in analyzing LLMs.
Low probability may be due to one or more of the following non-exhaustive reasons: (1) out-of-context content, (2) ungrammatical language, or (3) grammatical but semantically awkward content (cf. the mirror dataset in \cref{sec:perplexity}), (4) reasonable but rare content.
We hypothesize that the probability of language models may not be essentially able to capture all these nuanced differences, and call for encoding and decoding algorithms---such as \citet{meister-etal-2023-locally}---that can better decompose the probability into finer-grained and explainable components.

\vspace{2pt}
\noindent\textbf{Comparing humans and LLMs.}
What is our goal for building LLMs?
To achieve better performance on practical tasks or to build a more human-like model?
Our results, together with \citet{eisape-etal-2024-systematic}, suggest that these two goals may not be perfectly aligned by revealing a mixture of similarity and discrepancy between LLMs and humans---for example, while LLMs exhibit higher benchmark performance than humans on our dataset and show the same argument form preferences with humans (\cref{fig:emmeans-lm,fig:emmeans-human}), they also show systematic biases that we do not find significant in human reasoning (e.g., disfavoring the necessity modality, \cref{subsec:affirmation-bias}).
While there has been positive evidence of using LLMs as human models in psycholinguistic studies \interalia{misra-kim-2024-generating}, our results suggest executing such approaches cautiously.

\vspace{2pt}
\noindent\textbf{On the relation between modality and performance.}
Our results show that there is a significant difference in performance between necessity and possibility modalities, with the former much lower than the latter (\cref{tab:softacc-base}).
Part of the reason for this is that LLMs have a significant tendency to say ``No'' to the necessity modality (\cref{fig:affirmation-rejection}).

On the one hand, our results extend the conclusion of \citet{dentella-etal-2023-systematic} that LLMs generally respond positively---LLM behaviors may be significantly affected by finer-grained factors, including but not necessarily limited to the modality involved in the input.
On the other hand, while LLMs systematically tend to answer ``No'' to questions in necessity modality, we do not find related evidence in human experiments, which leads us to hypothesize that such rejection bias comes from either the model architecture or the training strategies, such as the reinforcement learning with human feedback \citep[RLHF;][]{ouyang-etal-2022-training} protocol.
We leave this as an open question for future research.

\vspace{2pt}
\noindent\textbf{Modal logic and theory of mind.}
Modality, in principle, encodes mental states and beliefs.
The reasoning of beliefs also resonates with the theory of mind \interalia{premackDoesChimpanzeeHave1978,baron-cohenDoesAutisticChild1985} and machine theory of mind \interalia{rabinowitzMachineTheoryMind2018, maHolisticLandscapeSituated2023}.
Following the effort by \citet{sileo-lernould-2023-mindgames} that uses epistemic modal logic to model the machine theory of mind, our work assesses the behaviors of LLMs on alethic modal logic, distantly revealing the future potential of LLMs in achieving the theory of mind.

\section{Threats to Validity}
\label{sec:threats}

In the following, we enumerate the main threats to the validity of our study, using the categories suggested by~\cite{Runeson12}.

\textbf{Construct Validity.}
Our evaluation results might have been affected by the choice of the SLR update and of the ML algorithms. Regarding the chosen SLR update, it is very difficult to get access to details such as individual assessments by reviewers during the initial screening process, which we needed for our analyses. Our SLR update dataset had such detailed information for 551 studies and is available online~\cite{zenodoOpenScience}. Regarding the algorithms, we analyzed the most used ones for text classification~\cite{pintas2021feature} and dug deeper into the two that showed the most prominent initial evaluation results on our dataset. 

\textbf{Internal Validity.}
Our training dataset comprised only studies included in the SLR replication \cite{Wohlin2022} (training included) and those obtained through backward snowballing (training excluded). We deliberately excluded studies not in English or those categorized as Ph.D. dissertations or book chapters from our testing set, the same criteria adopted by the SLR. A potential threat to internal validity that could have favored human reviewers is that, during the manual initial screening process, while this was not part of the procedure, the human reviewers could have ended up reading other sections of the studies besides the title, abstracts, and keywords.

\textbf{External Validity.} The dataset used in our analysis might not represent the diversity of SLR updates in SE. However, we did not find other SLRs with available data on the individual assessments applied during the initial screening. Replicating the investigations on other SLR updates to strengthen external validity would require significant effort for which we would have to involve the wider community. While not claiming external validity, we believe that sharing our initial evaluation results can already provide some valuable insights.

\textbf{Reliability.} The data used in our evaluation, including the individual initial screening assessments and the final list of papers to be included in the SLR update, was generated by the same (first three) authors who performed the SLR replication~\cite{Wohlin2022}. In addition, to improve the reliability of our results, our ML models and the evaluation datasets are openly available and auditable. 
\section{Conclusion}
\label{sec:conclusion}

This study investigated the application of supervised ML models as a supporting tool for researchers during study selection in SLR updates. Therefore, we developed a supervised ML pipeline for the study selection process. The focus was on investigating the effectiveness of ML models, the potential to reduce human effort, and the ability to provide support to individual human reviewers. We employed two ML models, Random Forest (RF) and Support Vector Machine (SVM), and assessed them on a dataset derived from a carefully manually curated SLR update. During this investigation process, our work also highlighted different configurations used for our ML models that correlate to their recall and F-score, providing results that can be useful for further exploration in this area.

Our results indicate that while ML can assist in preliminary study selection by reducing the volume of studies requiring manual review, it is not yet effective enough to automate this process or to directly assist a single human reviewer to produce more accurate selection results. Specifically, RF, our best model for study selection effectiveness, achieved a modest F-score of 0.33, with limited precision and recall, which is clearly insufficient for study selection. Meanwhile, SVM demonstrated potential in reducing effort by excluding up to 33.9\% of irrelevant studies without sacrificing recall. The comparison between human-only reviewer pairs and human-ML reviewer pairs for the initial screening showed that pairs of human reviewers produce results that are much better aligned with the final curated result of the SLR update. 

Considering our findings, we put forward that serious SLR update efforts should still rely on (at least two) experienced human researchers for the initial screening of papers to be included. Hence, this study contributes to understanding the practical limitations of ML in study selection and highlights the need for careful human involvement in this process to ensure the quality and rigor of SLR outcomes. 

Future research could focus on refining ML configurations, investigating adaptive thresholds to improve model performance in SLR update contexts, and exploring hybrid approaches (\textit{e.g.}, humans assisted by ML to reduce the overall screening effort by discarding studies with low probability of being included). We also recommend further investigating large language models (LLMs) within the SLR update context. 



\section*{Acknowledgment}

We express our gratitude to CNPq (Grant 312275/2023-4), FAPERJ (Grant E-26/204.256/2024), and Stone Co. for their generous support.

\bibliographystyle{IEEEtranS}
\bibliography{bibTex/sigproc} 

\end{document}

% \maketitle



% %--------------------------------------------------------
% \section{Background and Related Work}
% \label{sec:relatedwork}

% %SLR update definition
% An SLR update is a more recent (updated) version of an SLR that includes new evidence (primary studies) \cite{Mendes2020}. For the inclusion of new and relevant evidence, one of the initial steps is to conduct the study selection activity which consists of analyzing the retrieved studies from the search process to evaluate the need and to perform the SLR update. 

% %related work
% As mentioned before, there are several initiatives in SE towards improvements for SLR update (e.g. \cite{felizardo16, Garces17, Mendes2020, Wohlin2020}). However, considering the focus of our study on automation to select studies for SLR updates, we highlighted three main related works \cite{Watanabe20, Felizardo14, Napoleao2021} described in the following.

% The work of Watanabe \textit{et al.} \cite{Watanabe20} also evaluated the use of text classification (text mining combined with ML Models) to support the study selection activity for SLR updates in SE. They performed an evaluation with 8 SLRs from different research domains in a cross-validation procedure using Decision Tree (DT) and SVM as ML classification algorithms. The results achieved on average a \textit{F-score} of 0.92, \textit{Recall} of 0.93 and \textit{Precision} of 0.92. Unlike the approach proposed in \cite{Watanabe20}, our study evaluates the ML Models SVM and RF using a detailed database of a solid ongoing SLR update conducted by renowned researchers in the field of EBSE. Furthermore, we compare the agreement level of the adopted ML Models with the expert reviewers through \textit{Kappa} analysis \cite{Cohen10, Kitchenham15}.

% %While the contributions of Watanabe et al. \cite{Watanabe20} laid a valuable foundation by showing the potential of supervised ML Models in assisting the study selection process for SLR updates, our study takes a different approach to enhance the applicability of such methodologies. Our study used a detailed database of a solid ongoing SLR update conducted by renowned researchers in the field of EBSE, we  also used a distinct approach to retrieve the non-selected studies from the original SLR, and we explored different Feature Selection (FS) strategies, ML Models and techniques to train our classifiers. Furthermore, our study's contribution goes beyond evaluating how much effort could be reduced by the ML classifiers during the selection of studies process, we also  compared in detail the agreement level of the adopted ML classifiers with the expert reviewers through \textit{kappa} analysis \cite{Kitchenham15}.

% Felizardo \textit{et al.} \cite{Felizardo14} also proposes an automated alternative to support the selection of studies for SLR updates. The authors proposes a tool called Revis which links new evidence with the original SLR's evidence using the K-Nearest Neighbor (KNN) Edges Connection technique. The tool output is presented in two distinct visualizations, a content map and an Edge Bundles diagram. The results showed an increase in the number of studies correctly included compared to the traditional manual approach.

% Napoleão et al. \cite{Napoleao2021} performed a cross-domain Systematic Mapping (SM) on existing automated support for searching and selecting studies for SLRs and SMs in SE and Medicine. The authors indicated potential ML Models that can be adopted to support the study selection activities. They also indicated the most adopted methods (cross-validation and experiment) and metrics (\textit{Recall}, \textit{Precision} and \textit{F-measure}) to assess text classification approaches. The choice of the ML Models and the assessment metrics and methods for this study are considered finds of this work.


% \section{Goal and Research Questions}
% \label{sec:researchissues}

% Our goal is to evaluate the adoption of ML Models to support the selection of studies for SLR updates. We translated our goal into three different research questions (RQs).

%     \textbf{RQ1:} \textit{How effective are ML Models in selecting studies for SLR updates?}

%     We represent the effectiveness of the ML models in supporting the selection of studies activity using metrics such as \textit{Recall}, \textit{Precision} and \textit{F-measure} \cite{Napoleao2021, Watanabe20}. Our ML automated analysis considers only title and abstract of the studies. Our ML models results are compared with the included studies selected manually for the SLR update under evaluation. 
    
%     %Our ML automated analysis considers only title and abstract of the studies and the metrics are calculated at first considering the results from the expert reviewers analysis only on the title, abstract and keywords and next, considering also their results from the full-text analysis.

%     \textbf{RQ2:} \textit{How much effort can ML Models reduce during the study selection activity for SLR updates?}

%     We calculate the effort reduction by the relation of the number of studies that will need to have their title, abstract and keywords manually analyzed without the support of ML Models versus the number of studies to be analyzed after the use of the ML solution.
   
%     % Justificar o recall diferente de 100 (relatar o caso 100 e os casos 99)
%     % Mostrar um gráfico exibindo as variações de Falso Negativo / Positivo variando o Recall

%     \textbf{RQ3:} \textit{Can Machine Learning replace a reviewer in the selection of studies for SLRs?}
%     % We compared the agreement level of the ML Model with the highest \textit{F-score} value, supporting a single reviewer with the agreement level of each pair of reviewers, by calculating their Cohen's \textit{Kappa} coefficient \cite{Cohen10, Kitchenham15}.

%     We compared the levels of agreement and similarity of the ML Model with the highest \textit{F-score} value by performing two different analysis. For the agreement analysis, we used the Cohen's Kappa coefficient to measure the level of concordance between the ML Model and reviewers \cite{Cohen10, Kitchenham15}. For the similarity analysis, we used the Euclidean Distance to measure the distance between the ML Model and reviewers from different perspectives \cite{SERRA2014305}, to verify if the ML model had a higher similarity with any reviewer in comparison to the similarity within the assessment team, if the ML model had a higher similarity with the final than any reviewers, if the ML model decreased the distance from the final results when combining its answers with a single reviewer in comparison to other pairs of reviewers and if the ML model decreased the distance from the final results when combining its answers with two reviewers in comparison to the assessment team. The last evaluations intended to simulate the ML model working together with other reviewers during the agreement criteria from the selection of studies.  

%     % \end{itemize}

%     %---- Here I did not defined Kappa. I think we can defined it in the methodology section.

%     %Kappa analysis ->  Euclidean Distance
%     %agremeent level do algoritmo com os revisores - titulo, abstract and keywords
%     % future assessement + Hipótese :  Machine learning can replace a reviewer in a SLR update?

% %--------------------------------------------------------
% \section {Study Design}
% \label{sec:methodology}

% %Bianca: Aqui estava faltando definir qual o research metodo que a gente utilizou para testar o que foi desenvolvido. Eu utilizei a ideia do small-scale evaluation, visto que segundo o trabalho do Wohlin o que fizemos nao é um estudo de caso, nem um experimento. 

% % 5 steps Runeson (referencia/exemplo)
% % We follow the five main steps for conducting case studies
% % proposed by [21]: Design, preparation, collecting data, analysis
% % and reporting.

% % TODO: Explicar o Precison/Recall nessa seção e não nos Resultados -- %Bianca: Eu coloquei uma versao de definicao, veja o que acha. 

% In this Section, we present the key aspects of the study design. In order to evaluate our proposition, we performed a small-scale evaluation \cite{Wohlin2022cs}. According to the smell indicator proposed by Wohlin \& Rainer \cite{Wohlin2022cs}, the correct label for our evaluation is small-scale evaluation instead of a case study. In order to guide and report our study design, we divided our study design into two main parts: (i) Data Collection and (ii) Design \& Execution. 

% In Section \ref{subsec:data} we describe the  data collection process used in our small-scale evaluation to train and test our ML models. In Section \ref{subsec:studydesing} we detail our proposed solution developed to train and configure the investigated ML models. 

% \subsection{Data Collection}
% \label{subsec:data}

% \begin{figure*} [ht]
%     \centering
%     \includegraphics[width=400pt]{pictures/latest/fig03-data-acquisition-v2.pdf}
%     \caption{Data collection process}
%     \label{fig:fig-data-selection}
% \end{figure*}

% We used as instrument of our small-scale evaluation an ongoing SLR update of \cite{Wohlin2022} conducted by the same authors of this replication (team assessment). We chose this ongoing SLR update since the inclusion and exclusion of new studies were conducted based on individual assessments and the consensus of three experienced SLR researchers by analysing title, abstract, keywords and then full-text of the studies manually, allowing us to have confidence in this data for building reliable training and testing sets.

% The team assessment provided us all the studies they analyzed during the SLR update (.bib files), a total of 591 references, of which 39 were included and 552 were excluded for the update. We used these studies to form our testing set for our ML models, we filtered the studies to consider only first studies in English with a valid abstract. At the end, we used 551 studies in our testing set, of which 38 were included by the team assessment and 513 were excluded for the update.

% To train our ML models, we used a training set with 128 studies, of which 45 studies were included and 83 were excluded. The 45 studies used to train our models with what should be included were the same studies included in the original SLR. Since the team assessment did not list the studies that were excluded during the study selection phase of the original SLR, we performed a backward snowballing on the original references to obtain the 83 studies used to train our models with what should be excluded. Figure \ref{fig:fig-data-selection} summarizes this process.

% \subsection{Design \& Execution}
% \label{subsec:studydesing}

% We developed a pipeline with the following steps to automate the study selection process of an SLR update by using ML and answer our research questions. Our pipeline is illustrated in Figure \ref{fig:fig-study-design}. 

% \begin{figure*} [hb]
%     \centering
%     \includegraphics[width=1.0\linewidth]{pictures/latest/fig-pipeline-details-v2.pdf}
%     \caption{Study design pipeline}
%     \label{fig:fig-study-design}
% \end{figure*}

% In summary, our pipeline process a set of .bib files containing the list of studies to train the ML models and the list of studies to be analyzed. After completing its execution, it returns a report file in .xlsx format informing which studies should be included and excluded, as well as metrics about the predictions made by the ML model and the configuration that was used to run its execution.
% The pipeline must receive four different .bib files as input, one file containing the list of studies that should be excluded and one file containing the list of studies that should be included for each set (training and testing). In case there are any errors in the input files, the pipeline will stop its execution and will inform which entry was associated to each error as well as the type of error. Currently, our pipeline doesn't support automatic error resolution for invalid .bib files, they need to be manually fixed by the user.

% As shown in Figure~\ref{fig:fig-study-design} we firstly validated the .bib files of our testing and training sets to ensure completeness of the set avoiding duplicated entries or missing keys. Each study entry must have a title, the year of publication, an abstract text and a list of authors. 
% Secondly, we to applied text filtering techniques with Natural Language Processing (NLP) \cite{NLTK}, such as Lemmatization and Tokenization, to remove irrelevant characters from the texts. Thirdly, we applied Text Vectorization on the filtered texts using  Term-frequency/Inverse-Document-Frequency (TF/IDF), a technique that transforms text data into a numerical matrix of features. Fourthly, we used statistical methods to compute and select the most relevant features. In the fifth step, we trained and tuned our ML Models using our training set. Finally, in the last step, we used our ML Models to predict which studies of our testing set should be included and excluded and compared the results of each one and the agreement level in comparison with the team assessment.

% Additionally, an optional .env file can be passed as input to our pipeline, this file allows some steps in our pipeline to use a specific configuration, such as choosing the configuration of the Feature Selection (FS) method to compute the features, as well as the number of features to be selected in step four, and choosing the configuration for the ML models regarding which algorithm to be used, or which metric should be targeted when tuning the model as well as the type of cross validation to be performed, in step five. All parameters that can be configured are also illustrated in Figure~\ref{fig:fig-study-design}.

% % Based on the work of Napoleão \textit{et al.} \cite{Napoleao2021} which indicates the most used ML classifiers for assisting the selection of studies of SLRs and on the contributions of Pintas \textit{et al.} \cite{pintas2021feature} that evaluated the most adopted ML classifiers and Feature Selection (FS) techniques for text classification, we chose to evaluate two of them: Support Vector Machines (SVM) and Random Forest (RF).

% Based on the promising results achieved using SVM in the work of Watanabe \textit{et al.} \cite{Watanabe20} and the work of Napoleão \textit{et al.} \cite{Napoleao2021}, which highlighted SVM as one of the most used ML classifiers for assisting the selection of studies in SLRs, we decided to evaluate SVM in our work. Additionally, considering the work of Pintas \textit{et al.} \cite{pintas2021feature}, which evaluated the most adopted ML classifiers and Feature Selection (FS) techniques for text classification and concluded that the five most used classifiers are SVM, NB, KNN, DT, and RF, we performed initial tests using these classifiers. Our first tests showed that SVM and RF were achieving better results than the others. Therefore, we decided to focus our evaluation on these two classifiers: Support Vector Machines (SVM) and Random Forest (RF).

% % We experimented multiple configurations of our pipeline and evaluated different configurations for Feature Selection and for training and tuning of our ML classifiers. To select the best features, we tested different statistical methods such as Chi-squared (Chi2) \cite{Chi2}, Pearson Correlation \cite{pearson_r} and Analysis of Variance (Anova-F) \cite{ANOVA}. We tested different techniques to tune our ML classifiers such as K-fold cross-validation, Times-Series cross-validations and hyperparameter tuning with GirdSearch \cite{GridSearch}.

% We experimented multiple configurations of our pipeline and evaluated different configurations for FS and for training and tuning of our ML classifiers. During step four to compute the best features, we tested different statistical methods such as Chi-squared (Chi2) \cite{Chi2}, Pearson Correlation \cite{pearson_r} and Analysis of Variance (Anova-F) \cite{ANOVA} as well as a different range of features. We also tested different techniques to tune our ML classifiers such as K-fold cross-validation, Time-Series cross-validations and hyperparameter tuning with GirdSearch \cite{GridSearch}.

% For each evaluation, we executed the pipeline from start to finish in a clean environment using one statistical method at a time. To avoid introducing bias, the feature selection step was conducted solely based on the training set texts. Once the best features were identified in the training set, the same feature set was applied to the testing set to ensure consistency.
% To prevent overfitting, our machine learning classifiers were trained using a single type of cross-validation in each evaluation. Specifically, when utilizing GridSearch for parameter tuning, we didn't perform any other cross-validation technique, as GridSearch inherently includes cross-validation for measuring the most efficient parameter configuration. We chose this approach to maintain evaluation coherence and rigor.


% %--------------------------------------------------------
% \section{Results}
% \label{sec:results}

% In this section, we present the results of our study, following the research questions presented earlier in Section \ref{sec:researchissues}. To answer each question, we analyzed the results of the ML Models and then compared them with the results obtained manually by the SLR update authors under evaluation. 

% To answer questions RQ1 and RQ2, we reproduced the same evaluation performed by SLR update authors during the selection of studies for the SLR update by using our classifiers to predict  which studies should be included and excluded. For RQ1 we configured our model maximizing F-score and for RQ2 we configured our model maximizing Recall. Then we performed the agreement and similarity analysis using the predictions made by our model used in RQ1.

% Precision indicates how accurate the positive predictions made by the model are, while Recall indicates how well the model captures all the actual positive instances. For Precision, values closer to 1 indicate a lower number of False Positives (FP) results and values closer to 0 indicate a higher number of FP. And for Recall, values closer to 1 indicate a lower number of False Negatives (FN) and values closer to 0 indicate a higher number of FN.
% % Jogar para dentro da secao das RQs se for util

% We conducted our evaluation by varying the number of best features to be considered in each execution. After applying Text Filtering and Text Vectorization techniques, presented in steps three and four of our pipeline, our training set comprised a total of 23630 features, in contrast to our testing set, which comprised 119560 features. Given that the number of features in our testing set was more than five times greater than our training set, maximizing the number of best features in our training set was crucial to the performance of our ML models.

% We identified the range with the most relevant features in our training set as 900 to 1500 features, which was the range used in most of our evaluations. Notably, the best results, both in terms of F-score (RQ1) and Recall (RQ2), were consistently achieved with experiments that selected the 1200 best features.
% % Mover para RQ3

% % In the original table provided by the team assessment, besides what studies were included and excluded for the update, it also had the opinion of each reviewer about each study during the study selection process they performed. The reviewers could express their opinions about each study in three different level of certainty: 0 -- certain that the study should be excluded, 1 -- uncertain if the study should be excluded or included and 2 -- certain that the study should be included. We used this information to answer RQ3 and perform the \textit{Kappa} analysis.
% The document provided by the team assessment contained the final result of each study (if they were included or excluded), and also had the opinion of each reviewer about each study during the study selection process they performed. The reviewers could express their opinions in three different levels of certainty: 0 – certain that the study should be excluded, 1 – uncertain if the study should be excluded or included and 2 – certain that the study should be included. We used this information to answer RQ3 and perform the agreement and similarity analysis. Table~\ref{tab:slr-update-results-sample} illustrates the format of this document. 
% % TODO: Add table

% \begin{table}[!h]
% \caption{Example of the document format provided by the assessment team.}
% \tiny
% \begin{center}
% \begin{tabular}{ m{1.8cm} m{0.95cm} m{0.95cm} m{0.95cm} m{0.95cm} } 
%  \hline 
%   \textbf{Study} & \textbf{Final Result} & \textbf{R1} & \textbf{R2} & \textbf{R3} \\
%  First Study & 1 & 2 & 1 & 2 \\ 
%  Second Study & 0 & 2 & 0 & 0 \\ 
%  Third Study & 1 & 2 & 1 & 0 \\ 
%  Fourth Study & 0 & 0 & 2 & 1 \\ 
%  Fifth Study & 0 & 0 & 0 & 0 \\
%  \end{tabular}
% \end{center}
%  \label{tab:slr-update-results-sample}
% \end{table}


% We displayed the complete information of our results for the best configurations we found for RQ1 and RQ2, as well as all the other tests executions in an \textit{Appendix document}\footnote{https://zenodo.org/records/11021614} available online.

% \subsection{RQ1: \textit{How effective are ML Models in selecting studies for SLR updates?}}
% \label{results:RQ1}

% % TODO: Descrever os algoritmos de ML usados previamente na seção IV (Methodology)
% % To answer this question, during the ML classifiers tuning step, we trained our classifiers with GridSearch focusing on maximizing the F-score. Our best result was obtained by RF with a Precision of 0.22, Recall of 0.63 and F-score of 0.33 using the Anova-F statistical method. Figure \ref{fig:fig-rf-distribution} illustrates the distribution of the predictions' probabilities made by RF with this configuration. As we can see, its distribution is closer to the behavior of the selection studies task of SLRs and SLR updates, where most of the studies are excluded and not included.
% % % Colocar tabela com todos os resultados (é possível ver que o melhor resultado foi obtido com RF usando Anova-F...)
% To answer this question, during the ML models tuning step, we trained our classifiers with GridSearch focusing on maximizing the F-score. Our best result was obtained by RF with a Precision of 0.22, Recall of 0.63 and F-score of 0.33 using the Anova-F statistical method, with 1200 features. We used a default threshold of 0.5 to consider which studies should be included and excluded by our ML models. The parameters tested and selected by GridSearch for this configuration can be found in this document\footnote{https://zenodo.org/api/records/11021614/draft/files/RQ1-RQ3-best-configuration-RF.csv/content}. 
% And the table containing all the predictions made by our ML model for each study for this question can be seen in this document\footnote{https://zenodo.org/api/records/11019279/draft/files/RQ1-RF-predictions.csv/content}.

% % TODO: Add tables with config for rq1 and rq2

% Figure \ref{fig:fig-rf-distribution} illustrates the distribution of the predictions' scores made by RF with this configuration. In order to calculate each of these metrics, we compared our ML models' predictions with the final results only, obtained after the agreement criteria was applied by the team assessment, which is illustrated by the first column of Table~\ref{tab:slr-update-results-sample}.

% \begin{figure} [!h]
%     \centering
%     \includegraphics[width=1\linewidth]{pictures/latest/RF_test_v4-gridsearch_pearson_fs-k1200_bins10_v1.pdf}
%     \caption{RF Predictions Distribution}
%     \label{fig:fig-rf-distribution}
% \end{figure}


% % < Aumentar fontes das legendas e valores dos eixos X e Y de todos os graficos >


% \subsection{RQ2: \textit{How much effort can ML Models reduce during
% the study selection activity for SLR updates?}}
% \label{results:RQ2}

% % To answer this question, we tuned the ML Models with  the intention of maximizing the Recall. Since the purpose of this question was to evaluate how much human effort could be reduced by the use of ML Models during the selection of studies, we wanted to mitigate the chances of a false negative (FN) result, so the reviewers could simply ignore the studies excluded by the ML Model without worrying about losing a relevant study.

% % As demonstrated in Figure \ref{fig:fig-svm-distribution}, our best result was obtained by the SVM algorithm with a Precision of 0.10, Recall of 1.0 and F-score of 0.19 using the Pearson Correlation statistical method, with 1200 features. We used a default threshold of 0.5 to consider which studies should be included and excluded by our algorithms. 

% % According to Table \ref{tab:effort_reduction}, by maximizing the Recall, SVM was able to exclude a total of 187 studies, which represents 33.9\% of the total amount of studies in our testing set. By increasing the threshold, we are able to see that we can reduce the human effort even more at the risk of having more FN results. For a threshold range greater than 0.5 until 0.75, only one false negative was found. Notably, this FN result was one of the few cases where the team assessment had a lot of disparity. Initially, considering only the analysis of the reviewers individually (before they discussed it with each other), the reviewer R1 voted 2, R2 voted 1, and R3 voted 0.

% To answer this question, we tuned the ML models with the intention of maximizing the Recall. Since the purpose of this question was to evaluate how much human effort could be reduced by the use of ML Models during the selection of studies, we wanted to mitigate the chances of a false negative (FN) result, so the reviewers could simply ignore the studies excluded by the ML model without worrying about losing a relevant study.

% Our best result was obtained by using the SVM algorithm with a Precision of 0.10, Recall of 1.0 and F-score of 0.19 using the Pearson Correlation statistical method, with 1200 features. We used a default threshold of 0.5 to consider which studies should be included and excluded by our ML models. The parameters tested and selected by GridSearch for this configuration can be found in this document\footnote{https://zenodo.org/api/records/11021614/draft/files/RQ2-best-configuration-SVM.csv/content}. The table containing all the predictions made by our ML model for each study for this question can be seen in this document\footnote{https://zenodo.org/api/records/11019279/draft/files/RQ2-SVM-predictions.csv/content}.  

% \begin{figure} [!h]
%     \centering
%     \includegraphics[width=1\linewidth]{pictures/latest/fig-rq2-SVM_test_v4-gridsearch_pearson_fs-k1200_bins10.pdf}
%     \caption{SVM Predictions Distribution}
%     \label{fig:fig-svm-distribution}
% \end{figure}


% According to Table \ref{tab:effort_reduction}, by maximizing the Recall, SVM was able to exclude a total of 187 studies, which represents 33.9\% of the total amount of studies in our testing set. By increasing the threshold, we are able to see that we can reduce the human effort even more at the risk of having more FN results. For a threshold range greater than 0.5 until 0.75, only one false negative was found, while the number of TN increased by 87. Notably, this FN result was one of the few cases where the team assessment had a lot of disparity. Considering only the initial analysis of the reviewers individually (before they discussed it with each other), the reviewer R1 voted 2, R2 voted 1, and R3 voted 0. 
% For a threshold range greater than 0.75 until 0.80, when compared to the previous threshold range, the number of FN results increased by 1, while the number of TN increased by 18. 
% Finally, for a threshold range greater than 0.80 until 0.85, when compared to the previous threshold range, the number of FN results increased by 2, while the number of TN increased by 24. 

% As well as RQ1, to answer this question, we compared our ML models' predictions with the final results only, obtained after the agreement criteria was applied by the team assessment, which is illustrated by the first column of Table~\ref{tab:slr-update-results-sample}.


% \begin{table}
% \centering
% \begin{tabular}{|c|c|c|c|c|c|c|}
% \hline
% \textbf{Threshold(\%)} & \textbf{RECALL (\%)} & \textbf{TN} & \textbf{TP} & \textbf{FN} & \textbf{FP} & \textbf{Reduced (\%)} \\
% \hline
% 0.50\% & 100.00\% & 187 & 38 & 0 & 326 & 33.9\% \\ 
% 0.75\% & 97.37\% & 265 & 37 & 1 & 248 & 48.3\% \\  
% 0.80\% & 94.74\% & 283 & 36 & 2 & 267 & 51.7\% \\  
% 0.85\% & 89.49\% & 307 & 34 & 4 & 206 & 56.4\% \\ 
% \hline
% \end{tabular}
% \caption{Tradeoff between effort reduction and number of FN.}
% \label{tab:effort_reduction}
% \end{table}

% \subsection{RQ3: \textit{Can Machine Learning replace a reviewer in the selection of studies for SLRs?}}
% \label{results:RQ3}

% To answer this question, we conducted a two-fold analysis to evaluate both aspects of agreement and similarity, considering not only the comparison between our ML model and single reviewer results but also the final results and the average answer between multiple reviewers.

% The agreement analysis indicates how two or more raters make the same classifications, it measures the concordance of results, we used the Cohen's Kappa coefficient for this. While, the similarity analysis indicates the resemblance between the classifications of two or more raters, it measures how close the results are, even if they are not exactly the same, we used the Euclidean Distance for this. 

% Firstly, we looked at the agreement and similarity levels between the reviewers among the team assessment, and analyzed the information provided by the team assessment regarding each reviewer's vote before applying the agreement criteria. Table~\ref{tab:reviewers-agreement-table} shows the Kappa values between reviewers, and Table~\ref{tab:reviewers-similarity-table} shows the Euclidean Distance between reviewers.

% \begin{table}
% \centering
% \begin{tabular}{|c|c|c|c|}
% \hline
% \textbf{} & \textbf{R1} & \textbf{R2} & \textbf{R3} \\
% \hline
% \textbf{R1} & 1 & 0.47 & 0.35 \\
% \textbf{R2} & 0.47 & 1 & 0.43 \\
% \textbf{R3} & 0.35 & 0.43 & 1 \\
% \hline
% \end{tabular}
% \caption{Agreement Among the Assessment Team.}
% \label{tab:reviewers-agreement-table}
% \end{table}

% \begin{table}
% \centering
% \begin{tabular}{|c|c|c|c|}
% \hline
% \textbf{} & \textbf{R1} & \textbf{R2} & \textbf{R3} \\
% \hline
% \textbf{R1} & 0 & 13.0 & 14.04 \\
% \textbf{R2} & 13.0 & 0 & 11.22 \\
% \textbf{R3} & 14.04 & 11.22 & 0 \\
% \hline
% \end{tabular}
% \caption{Similarity Among the Assessment Team.}
% \label{tab:reviewers-similarity-table}
% \end{table}

% We considered the following rage of Cohen's Kappa coefficient values as the following agreement levels \cite{carletta1996assessing}:
% \begin{itemize}
%     \item from 0.00 to 0.20: Poor Agreement
%     \item from 0.21 to 0.40: Fair Agreement
%     \item from 0.41 to 0.60: Moderate Agreement
%     \item from 0.61 to 0.80: Substantial Agreement
%     \item from 0.81 to 1.00: Almost Perfect Agreement
% \end{itemize}

% The highest agreement was achieved between reviewer 1 (R1) and reviewer 2 (R2) with 0.47 of agreement, which can be considered a Moderate Agreement. Whilst, the lowest agreement was between the reviewer 3 (R3) and R1 with 0.35 of agreement, which can be considered a Fair Agreement. Since the similarity is inversely proportional to the Euclidean Distance, we can notice that R2 and R3 had the most similar opinions with the smallest distance of 11.22 in comparison to R1 and R3 with the highest distance of 14.04.

% To compare our ML models with the reviewers, we used the same RF Model used to answer RQ1, since it had the better f-score, it made more realistic predictions in contrast to our SVM model used to answer RQ2, which was configured to maximize recall. 
% We normalized our RF results in three ranges to represent the same three categories used by the reviewers during the step to apply the inclusion and exclusion criteria. Figure~\ref{fig:fig-reviewers-votes-distribution} shows the distribution of votes by each reviewer in each category. We decided the threshold for the exclusion of studies should be greater than the rest, since the frequency of occurrence of votes "0" was clearly the highest for the assessment team. And we decided the range of "uncertainty" should be the smallest, since it was the vote less frequent. Considering that, we decided to use the following range of predictions score to normalize our RF model. 
% \begin{itemize}
%     \item from 0.00 to 0.50: should be excluded
%     \item from 0.51 to 0.60: uncertain
%     \item from 0.61 to 1.00: should be included
% \end{itemize}

% Figure~\ref{fig:fig-rf-normalized-distribution} illustrates the normalized distribution of our ML model used in RQ1.

% \begin{figure} [!h]
%     \centering
%     \includegraphics[width=1\linewidth]{pictures/latest/fig-rq3-reviewers-votes-distribution.pdf}
%     \caption{Reviewers votes distribution.}
%     \label{fig:fig-reviewers-votes-distribution}
% \end{figure}

% \begin{figure} [!h]
%     \centering
%     \includegraphics[width=1\linewidth]{pictures/latest/rf_test_v4_normlized_agreement.pdf}
%     \caption{RQ3: RF Predictions Distribution considering "uncertain" range.}
%     \label{fig:fig-rf-normalized-distribution}
% \end{figure}



% We used the normalized results of our RF model to evaluate its agreement and similarity with the reviewers. Table~\ref{tab:rf-kappa-and-euclidean-table} illustrates the Cohen's Kappa coefficient and the Euclidean Distance between our RF model and each reviewer.

% % After we converted the probabilities given by our RF model used in RQ1 into the three categories, we compared the normalized results of the RF model with the results of each reviewer and calculated the \textit{Cohen's Kappa Coefficient} and the Euclidean Distance between them. The agreement level between the RF model and each reviewer was: RF | R1 = 0.27, RF | R2 = 0.37, RF | R3 = 0.30. The Euclidean Distance between the RF model and each reviewer was: RF | R1 = 17.89, RF | R2 = 16.76, RF | R3 = 17.52. Table~\ref{tab:rf-kappa-and-euclidean-table} illustrates both results.

% \begin{table}[h!]
% \centering
% \begin{tabular}{|c|c|c|}
% \hline
% \textbf{Comparison} & \textbf{Kappa} & \textbf{Euclidean Distance} \\ \hline
% R1 vs RF & 0.27 & 17.89 \\ \hline
% R2 vs RF & 0.37 & 16.76 \\ \hline
% R3 vs RF & 0.30 & 17.52 \\ \hline
% \end{tabular}
% \caption{Cohen's Kappa coefficient and Euclidean Distance between RF model and Reviewers}
% \label{tab:rf-kappa-and-euclidean-table}
% \end{table}

% Then, we used the Euclidean Distance to evaluate the similarity between the Final Results (FR) and our RF model and reviewers, when compared individually and collectively. In order to evaluate the answers from the FR accurately, we normalize the binary answers in FR to considered that the included studies a value of 2 and excluded studies had a value 0 in FR. So if a reviewer voted to include a study with certainty and the study was indeed included, the distance between these two points would be zero.

% We evaluated the Euclidean Distance (ED) by three different perspectives, as follows:
% \begin{itemize}
%     \item Similarity between single answers and FR: 
%     \[
%     \textit{ED}(i, FR) \text{ where } i \in \{\text{R1}, \text{R2}, \text{R3}, \text{RF}\}
%     \]
%     \item Similarity between pairs and FR:
%     \[
%     \textit{ED}\left(\frac{i + j}{2}, \text{FR}\right) \text{ where } i \neq j \text{ and } i, j \in \{\text{R1}, \text{R2}, \text{R3}, \text{RF}\}
%     \]
%     \item Similarity between groups and FR:
%     \[
%     \textit{ED}\left(\frac{i + j + k}{3}, \text{FR}\right) \text{ where } i \neq j \neq k \text{ and } i, j, k \in \{\text{R1}, \text{R2}, \text{R3}, \text{RF}\}
%     \]
% \end{itemize}

% Table~\ref{tab:similarity-distance-FR-table} shows the ED measured in each case. As we can see, the smallest distance in comparison to the FR in each case was given by: R2 with ED = 9.95, pair(R2,R3) with ED = 8.86 and team assessment with ED = 8.17.  

% \begin{table}[h!]
% \centering
% \begin{tabular}{|c|c|}
% \hline
% \textbf{Comparison} & \textbf{Euclidean Distance} \\ \hline
% R1 vs FR & 12.00 \\ \hline
% R2 vs FR & \textbf{9.95} \\ \hline
% R3 vs FR & 11.00 \\ \hline
% RF vs FR & 17.94 \\ \hline
% \hline
% avg(R1,R2) vs FR & 8.90 \\ \hline
% avg(R1,R3) vs FR & 9.12 \\ \hline
% avg(R2,R3) vs FR & \textbf{8.86} \\ \hline
% avg(R1,RF) vs FR & 12.37 \\ \hline
% avg(R2,RF) vs FR & 11.84 \\ \hline
% avg(R3,RF) vs FR & 12.03 \\ \hline
% \hline
% avg(R1,R2,R3) vs FR & \textbf{8.17 }\\ \hline
% avg(RF,R2,R3) vs FR & 10.06 \\ \hline
% avg(R1,RF,R3) vs FR & 10.20 \\ \hline
% avg(R1,R2,RF) vs FR & 10.15 \\ \hline
% \end{tabular}
% \caption{Euclidean Distance Analysis considering FR}
% \label{tab:similarity-distance-FR-table}
% \end{table}

% %--------------------------------------------------------
% \section{Discussion}
% \label{sec:discussion}

% \subsection{\textit{Research Question 1}}
% \label{discussion:RQ1}
% Regarding RQ1 - “How effective are ML models in selecting studies for SLR updates?”, based on our results, we concluded that the best configuration to maximize the F-score of our ML models was using RF with Anova-F. Not only that, but in almost all of our tests configurations, RF outperformed SVM in terms of F-score with exception for one test, where we selected only 900 best features and used the Pearson Correlation method to compute the best features. We also noticed that for most cases, the Anova-F improved the F-score rating at the cost of lowering its Recall. 
% However, even noticing that the RF was more successful in this task than SVM and considering its best result, its F-score was still not good enough to be considered to automate the process of selecting studies for the SLR update.


% % TODO: O watanabe coloca tabelas com o resultados dos dois algoritmos DT e SVM variando o numero de features e na discussion compara o seus desempenhos...

% \subsection{\textit{Research Question 2}}
% \label{discussion:RQ2}
% Regarding RQ2 - “How much effort can ML models reduce during the study selection activity for SLR updates?”, our results showed that the best configuration to maximize the Recall of our ML models was using SVM with Pearson Correlation. On one hand, it's possible to see that in all of our tests, SVM had higher Recall marks than RF. Particularly, when used with Pearson Correlation to select the best features, it had the highest marks for Recall in most cases (although even in executions using Anova-F it still reached a high Recall of 0.97 for some tests). On the other hand, its F-score and Precision marks were very low. However, even if a big number of FP studies remained after that, we believe that our SVM model showed potential for reducing human effort during the study selection task of SLR updates, by automatically excluding part of the irrelevant studies.

% \subsection{\textit{Research Question 3}}
% \label{discussion:RQ3}
% Regarding RQ3 - “Can Machine Learning Replace a Reviewer in the Selection of Studies for Systematic Literature Reviews?” our evaluation showed that the best result was achieved by the RF classifier with the same configuration used for RQ1. As we can see in Table~\ref{tab:reviewers-agreement-table}, the strongest level of agreement was between R1 and R2 with a score of 0.47 for Cohen's Kappa Coefficient, followed by an agreement of 0.43 between R2 and R3, and followed by the weakest agreement between R1 and R3 with a score of 0.35. In this case, two pairs of reviewers had a moderate level of agreement and one pair had only a fair level of agreement. 

% On the other hand, when compared with our RF classifier, it had its strongest agreement of 0.37 with R2, followed by an agreement of 0.30 with R3 and an agreement of 0.27 with R1, all considered as fair level of agreement. Even though the pair RF classifier and R2 had a stronger agreement than the pair R1 and R3, both still had the same agreement level (fair agreement), also the R2 had a Moderate Agreement with both reviewers, while it had only a fair agreement with the RF classifier. Also, when comparing the agreement of RF with R3 and its respective pairs (R3|R1 and R3|R2), it had a weaker agreement than both, the same was true when comparing RF with R2 and its respective pairs. 

% Looking at Table~\ref{tab:reviewers-similarity-table}, we can see that R2 and R3 had the most similar answers, with an ED of 11.22 between them. As showed in Table~\ref{tab:similarity-distance-FR-table} R2 also had the smallest distance from FR when compared individually (with an ED of 9.95), as expected, the pair R2 and R3 had the strongest similarity with FR (with an ED of 8.86) in contrast to the other pairs. Finally, the closest distance was given by the team assessment with an ED of 8.17, which is expected since the FR was generated from their answers. It's possible to see that our RF model had the highest distances in all comparisons and even though looking at the distance to FR obtained when working together with R1 didn't cause much negative impact (ED(R1,RF) = 12.37 vs ED(R1) = 12.00) as the rest, it shows that our ML model didn't help any of the reviewers to get closer to the FR. 

% Therefore, we concluded that our supervised ML Models are not ready to replace a reviewer during the selection of studies for SLR updates.

% It is worth mentioning that the similarity and agreement levels between our classifier and the team assessment could increase or decrease depending on how we configure the thresholds to normalize the probabilities given by our ML classifier into the three categories used by the team assessment. We noticed that reducing the range to consider a vote as uncertain would increase the level of agreement with all members of the assessment team. But in most cases, our classifier still had only a fair level of agreement, so our conclusion was the same.


% \section{Threats to Validity}
% \label{sec:threats}

% In the following, we enumerate the main threats to the validity of our study.

% \textbf{Construct Validity.}
% Our evaluation results might have been affected by the choice of ML algorithms. Other algorithms could have been explored in our study and can be considered as part of future work.

% \textbf{Internal Validity.}
% Our training dataset comprised only studies including in the SLR replication \cite{Wohlin2022} (training included) and those obtained through backward snowballing (training excluded). We deliberately excluded studies not in English or those categorized as Ph.D. dissertations or book chapters from our testing set, same approach adopted by the SLR replication. This focused approach aimed to provide our models with only relevant and essential data. Another potential threat is that during manual process performed by the team assessment, the authors could end up reading other sections of the studies besides the title and abstracts when they are not completely sure if the study should be included or excluded just by reading its abstract. This is a possible advantage manual process could have over our models that consider only content from title and abstract. 

% \textbf{External Validity.} The dataset used in our analysis might not represent the diversity of SLR Updates in SE. Similar analyses could have been conducted based on other SLRs to improve the generalizability of our results. However, replicating our results on other SLRs to strengthen external validity would require significant effort. Moreover, it is challenging to acquire a reliable and detailed SLR dataset for SLRs updates that could be considered in our evaluation. 

% \textbf{Reliability.} One limitation of our study is associated with the dataset used in our evaluation and the possibility of sample bias. The data used in our evaluation was acquired from the same authors who performed the SLR replication, also through a rigorous analysis process. In addition, to improve the reliability of our results, our ML models and the small-scale evaluation dataset are openly available. 

% %--------------------------------------------------------
% \section{Conclusion}
% \label{sec:conclusion}
% % In this paper we have presented an evaluation using our ML models to replicate the process of an ongoing SLR update performed by three experienced researchers, to evaluate it as a supporting tool for the studies’ selection task.  We compared the results of our ML models with the members of the team assessment. 

% % We concluded that our ML models are not ready to automatically select studies for the study selection task, and may also not be used to replace an additional reviewer. However, there is potential for reducing human effort during the study selection task of SLR updates, by automatically excluding part of the irrelevant studies.

% % Our next steps is to investigate even further the use of ML models with different configurations and text classification techniques (variar algoritmos, métodos estatísticos de FS, técnicas de TF-IDF) with the same goals. In addition, we intend to explore ML models with different datasets to evaluate their performance. We also believe we could achieve promising results by adapting our pipeline to have multiple iterations, such as: first we could only exclude irrelevant studies using a ML model with a configuration maximized by the recall, than we could try to predict the most relevant in the remaining results using another ML model maximized by the f-score.

% This study advances the application of supervised ML models as a supporting tool for researchers during SLRs updates, by developing and testing a comprehensive supervised ML-based pipeline to automate the study selection process. We have demonstrated through our tests, using realistic data to build our datasets, that while our ML models, have shown promise in reducing human effort to perform the study selection activity by pre-filtering irrelevant studies, they currently lack the precision required to completely automate the selection of studies for SLR updates. We also concluded that they did not achieve a level of similarity and agreement strong enough to be considered sufficient to replace a human reviewer during an SLR update in the study selection phase. Our work also highlights different configurations used for our ML models that correlates to their recall and F-score, providing results that can be useful for further exploration in this area.

% Our next step is to investigate even further the use of ML models with different configurations and text classification techniques with the same goals. In addition, we intend to explore our pipeline using datasets with a different structure to evaluate its performance. Additionally, we intend to perform new tests with the objective of improving our pipeline's automation, for instance, by experimenting to use the ML model configured to maximize the recall to perform an initial filter on the studies and then use another ML model configured to maximize the f-score to make predicts based on the results previously filtered. Also, we believe we could leverage from the work of Pintas \textit{et al.} \cite{pintas2021feature} by performing a deeper analysis of our dataset structure and use the contributions of their work to decide which FS strategies should work the best. Lastly, considering the increase number of studies using LLMs to support SLRs and SLR updates \cite{bolanos2024artificial}, we intend to evaluate the use of LLMs in this context as well and compare with our models.

% \balance
% \bibliographystyle{ACM-Reference-Format}
% \bibliography{acmart}

% % \bibliographystyle{ACM-Reference-Format}
% %\bibliography{bib/base}
% %\printbibliography

% \end{document}
% \endinput

