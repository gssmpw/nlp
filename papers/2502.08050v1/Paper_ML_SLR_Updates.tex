
%% bare_conf.tex
%% V1.4b
%% 2015/08/26
%% by Michael Shell
%% See:
%% http://www.michaelshell.org/
%% for current contact information.
%%
%% This is a skeleton file demonstrating the use of IEEEtran.cls
%% (requires IEEEtran.cls version 1.8b or later) with an IEEE
%% conference paper.
%%
%% Support sites:
%% http://www.michaelshell.org/tex/ieeetran/
%% http://www.ctan.org/pkg/ieeetran
%% and
%% http://www.ieee.org/

%%*************************************************************************
%% Legal Notice:
%% This code is offered as-is without any warranty either expressed or
%% implied; without even the implied warranty of MERCHANTABILITY or
%% FITNESS FOR A PARTICULAR PURPOSE! 
%% User assumes all risk.
%% In no event shall the IEEE or any contributor to this code be liable for
%% any damages or losses, including, but not limited to, incidental,
%% consequential, or any other damages, resulting from the use or misuse
%% of any information contained here.
%%
%% All comments are the opinions of their respective authors and are not
%% necessarily endorsed by the IEEE.
%%
%% This work is distributed under the LaTeX Project Public License (LPPL)
%% ( http://www.latex-project.org/ ) version 1.3, and may be freely used,
%% distributed and modified. A copy of the LPPL, version 1.3, is included
%% in the base LaTeX documentation of all distributions of LaTeX released
%% 2003/12/01 or later.
%% Retain all contribution notices and credits.
%% ** Modified files should be clearly indicated as such, including  **
%% ** renaming them and changing author support contact information. **
%%*************************************************************************


% *** Authors should verify (and, if needed, correct) their LaTeX system  ***
% *** with the testflow diagnostic prior to trusting their LaTeX platform ***
% *** with production work. The IEEE's font choices and paper sizes can   ***
% *** trigger bugs that do not appear when using other class files.       ***                          ***
% The testflow support page is at:
% http://www.michaelshell.org/tex/testflow/



\documentclass[conference]{IEEEtran}
\usepackage{array}
\usepackage{booktabs} % For formal tables
\usepackage{multirow}
\usepackage{lipsum}
\usepackage{algorithm}
\usepackage{color, soul}
\usepackage{amsmath}
\usepackage{graphicx}
\usepackage{mdframed}
\usepackage{url}
\usepackage{cite}
%\usepackage{appendix}
\usepackage[noend]{algpseudocode}
\usepackage[utf8]{inputenc}
\usepackage{comment}
\usepackage{hyperref}

% Some Computer Society conferences also require the compsoc mode option,
% but others use the standard conference format.
%
% If IEEEtran.cls has not been installed into the LaTeX system files,
% manually specify the path to it like:
% \documentclass[conference]{../sty/IEEEtran}
\newcolumntype{P}[1]{>{\centering\arraybackslash}m{#1}}
\newcommand{\specialcell}[2][c]{%
  \begin{tabular}[#1]{@{}c@{}}#2\end{tabular}}

% Some Computer Society conferences also require the compsoc mode option,
% but others use the standard conference format.
%
% If IEEEtran.cls has not been installed into the LaTeX system files,
% manually specify the path to it like:
% \documentclass[conference]{../sty/IEEEtran}





% Some very useful LaTeX packages include:
% (uncomment the ones you want to load)


% *** MISC UTILITY PACKAGES ***
%
%\usepackage{ifpdf}
% Heiko Oberdiek's ifpdf.sty is very useful if you need conditional
% compilation based on whether the output is pdf or dvi.
% usage:
% \ifpdf
%   % pdf code
% \else
%   % dvi code
% \fi
% The latest version of ifpdf.sty can be obtained from:
% http://www.ctan.org/pkg/ifpdf
% Also, note that IEEEtran.cls V1.7 and later provides a builtin
% \ifCLASSINFOpdf conditional that works the same way.
% When switching from latex to pdflatex and vice-versa, the compiler may
% have to be run twice to clear warning/error messages.






% *** CITATION PACKAGES ***
%
%\usepackage{cite}
% cite.sty was written by Donald Arseneau
% V1.6 and later of IEEEtran pre-defines the format of the cite.sty package
% \cite{} output to follow that of the IEEE. Loading the cite package will
% result in citation numbers being automatically sorted and properly
% "compressed/ranged". e.g., [1], [9], [2], [7], [5], [6] without using
% cite.sty will become [1], [2], [5]--[7], [9] using cite.sty. cite.sty's
% \cite will automatically add leading space, if needed. Use cite.sty's
% noadjust option (cite.sty V3.8 and later) if you want to turn this off
% such as if a citation ever needs to be enclosed in parenthesis.
% cite.sty is already installed on most LaTeX systems. Be sure and use
% version 5.0 (2009-03-20) and later if using hyperref.sty.
% The latest version can be obtained at:
% http://www.ctan.org/pkg/cite
% The documentation is contained in the cite.sty file itself.






% *** GRAPHICS RELATED PACKAGES ***
%
\ifCLASSINFOpdf
  % \usepackage[pdftex]{graphicx}
  % declare the path(s) where your graphic files are
  % \graphicspath{{../pdf/}{../jpeg/}}
  % and their extensions so you won't have to specify these with
  % every instance of \includegraphics
  % \DeclareGraphicsExtensions{.pdf,.jpeg,.png}
\else
  % or other class option (dvipsone, dvipdf, if not using dvips). graphicx
  % will default to the driver specified in the system graphics.cfg if no
  % driver is specified.
  % \usepackage[dvips]{graphicx}
  % declare the path(s) where your graphic files are
  % \graphicspath{{../eps/}}
  % and their extensions so you won't have to specify these with
  % every instance of \includegraphics
  % \DeclareGraphicsExtensions{.eps}
\fi
% graphicx was written by David Carlisle and Sebastian Rahtz. It is
% required if you want graphics, photos, etc. graphicx.sty is already
% installed on most LaTeX systems. The latest version and documentation
% can be obtained at: 
% http://www.ctan.org/pkg/graphicx
% Another good source of documentation is "Using Imported Graphics in
% LaTeX2e" by Keith Reckdahl which can be found at:
% http://www.ctan.org/pkg/epslatex
%
% latex, and pdflatex in dvi mode, support graphics in encapsulated
% postscript (.eps) format. pdflatex in pdf mode supports graphics
% in .pdf, .jpeg, .png and .mps (metapost) formats. Users should ensure
% that all non-photo figures use a vector format (.eps, .pdf, .mps) and
% not a bitmapped formats (.jpeg, .png). The IEEE frowns on bitmapped formats
% which can result in "jaggedy"/blurry rendering of lines and letters as
% well as large increases in file sizes.
%
% You can find documentation about the pdfTeX application at:
% http://www.tug.org/applications/pdftex





% *** MATH PACKAGES ***
%
%\usepackage{amsmath}
% A popular package from the American Mathematical Society that provides
% many useful and powerful commands for dealing with mathematics.
%
% Note that the amsmath package sets \interdisplaylinepenalty to 10000
% thus preventing page breaks from occurring within multiline equations. Use:
%\interdisplaylinepenalty=2500
% after loading amsmath to restore such page breaks as IEEEtran.cls normally
% does. amsmath.sty is already installed on most LaTeX systems. The latest
% version and documentation can be obtained at:
% http://www.ctan.org/pkg/amsmath





% *** SPECIALIZED LIST PACKAGES ***
%
%\usepackage{algorithmic}
% algorithmic.sty was written by Peter Williams and Rogerio Brito.
% This package provides an algorithmic environment fo describing algorithms.
% You can use the algorithmic environment in-text or within a figure
% environment to provide for a floating algorithm. Do NOT use the algorithm
% floating environment provided by algorithm.sty (by the same authors) or
% algorithm2e.sty (by Christophe Fiorio) as the IEEE does not use dedicated
% algorithm float types and packages that provide these will not provide
% correct IEEE style captions. The latest version and documentation of
% algorithmic.sty can be obtained at:
% http://www.ctan.org/pkg/algorithms
% Also of interest may be the (relatively newer and more customizable)
% algorithmicx.sty package by Szasz Janos:
% http://www.ctan.org/pkg/algorithmicx




% *** ALIGNMENT PACKAGES ***
%
%\usepackage{array}
% Frank Mittelbach's and David Carlisle's array.sty patches and improves
% the standard LaTeX2e array and tabular environments to provide better
% appearance and additional user controls. As the default LaTeX2e table
% generation code is lacking to the point of almost being broken with
% respect to the quality of the end results, all users are strongly
% advised to use an enhanced (at the very least that provided by array.sty)
% set of table tools. array.sty is already installed on most systems. The
% latest version and documentation can be obtained at:
% http://www.ctan.org/pkg/array


% IEEEtran contains the IEEEeqnarray family of commands that can be used to
% generate multiline equations as well as matrices, tables, etc., of high
% quality.




% *** SUBFIGURE PACKAGES ***
%\ifCLASSOPTIONcompsoc
%  \usepackage[caption=false,font=normalsize,labelfont=sf,textfont=sf]{subfig}
%\else
%  \usepackage[caption=false,font=footnotesize]{subfig}
%\fi
% subfig.sty, written by Steven Douglas Cochran, is the modern replacement
% for subfigure.sty, the latter of which is no longer maintained and is
% incompatible with some LaTeX packages including fixltx2e. However,
% subfig.sty requires and automatically loads Axel Sommerfeldt's caption.sty
% which will override IEEEtran.cls' handling of captions and this will result
% in non-IEEE style figure/table captions. To prevent this problem, be sure
% and invoke subfig.sty's "caption=false" package option (available since
% subfig.sty version 1.3, 2005/06/28) as this is will preserve IEEEtran.cls
% handling of captions.
% Note that the Computer Society format requires a larger sans serif font
% than the serif footnote size font used in traditional IEEE formatting
% and thus the need to invoke different subfig.sty package options depending
% on whether compsoc mode has been enabled.
%
% The latest version and documentation of subfig.sty can be obtained at:
% http://www.ctan.org/pkg/subfig




% *** FLOAT PACKAGES ***
%
%\usepackage{fixltx2e}
% fixltx2e, the successor to the earlier fix2col.sty, was written by
% Frank Mittelbach and David Carlisle. This package corrects a few problems
% in the LaTeX2e kernel, the most notable of which is that in current
% LaTeX2e releases, the ordering of single and double column floats is not
% guaranteed to be preserved. Thus, an unpatched LaTeX2e can allow a
% single column figure to be placed prior to an earlier double column
% figure.
% Be aware that LaTeX2e kernels dated 2015 and later have fixltx2e.sty's
% corrections already built into the system in which case a warning will
% be issued if an attempt is made to load fixltx2e.sty as it is no longer
% needed.
% The latest version and documentation can be found at:
% http://www.ctan.org/pkg/fixltx2e


%\usepackage{stfloats}
% stfloats.sty was written by Sigitas Tolusis. This package gives LaTeX2e
% the ability to do double column floats at the bottom of the page as well
% as the top. (e.g., "\begin{figure*}[!b]" is not normally possible in
% LaTeX2e). It also provides a command:
%\fnbelowfloat
% to enable the placement of footnotes below bottom floats (the standard
% LaTeX2e kernel puts them above bottom floats). This is an invasive package
% which rewrites many portions of the LaTeX2e float routines. It may not work
% with other packages that modify the LaTeX2e float routines. The latest
% version and documentation can be obtained at:
% http://www.ctan.org/pkg/stfloats
% Do not use the stfloats baselinefloat ability as the IEEE does not allow
% \baselineskip to stretch. Authors submitting work to the IEEE should note
% that the IEEE rarely uses double column equations and that authors should try
% to avoid such use. Do not be tempted to use the cuted.sty or midfloat.sty
% packages (also by Sigitas Tolusis) as the IEEE does not format its papers in
% such ways.
% Do not attempt to use stfloats with fixltx2e as they are incompatible.
% Instead, use Morten Hogholm'a dblfloatfix which combines the features
% of both fixltx2e and stfloats:
%
% \usepackage{dblfloatfix}
% The latest version can be found at:
% http://www.ctan.org/pkg/dblfloatfix




% *** PDF, URL AND HYPERLINK PACKAGES ***
%
%\usepackage{url}
% url.sty was written by Donald Arseneau. It provides better support for
% handling and breaking URLs. url.sty is already installed on most LaTeX
% systems. The latest version and documentation can be obtained at:
% http://www.ctan.org/pkg/url
% Basically, \url{my_url_here}.




% *** Do not adjust lengths that control margins, column widths, etc. ***
% *** Do not use packages that alter fonts (such as pslatex).         ***
% There should be no need to do such things with IEEEtran.cls V1.6 and later.
% (Unless specifically asked to do so by the journal or conference you plan
% to submit to, of course. )


% correct bad hyphenation here
\hyphenation{op-tical net-works semi-conduc-tor}


\begin{document}
% %
% % paper title
% % Titles are generally capitalized except for words such as a, an, and, as,
% % at, but, by, for, in, nor, of, on, or, the, to and up, which are usually
% % not capitalized unless they are the first or last word of the title.
% % Linebreaks \\ can be used within to get better formatting as desired.
% % Do not put math or special symbols in the title.
\title{Can Machine Learning Support the Selection of Studies for Systematic Literature Review Updates?}

\author{

\IEEEauthorblockN{Marcelo Costalonga}
\IEEEauthorblockA{\textit{PUC-Rio} \\
Rio de Janeiro, Brazil \\
mcardoso@inf.puc-rio.br}
\and
\IEEEauthorblockN{Bianca Minetto Napole\~ao}
\IEEEauthorblockA{\textit{Université du Québec à Chicoutimi} \\
Chicoutimi, Canada \\
bianca.minetto-napoleao1@uqac.ca}
\and
\IEEEauthorblockN{Maria Teresa Baldassarre}
\IEEEauthorblockA{\textit{University of Bari} \\
Bari, Italy \\
mariateresa.baldassarre@uniba.it}
\and
\IEEEauthorblockN{Katia Romero Felizardo}
\IEEEauthorblockA{\textit{Universidade Tecnológica Federal do Paraná} \\
Cornélio Procópio, Brazil \\
katiascannavino@utfpr.edu.br}
\and
\IEEEauthorblockN{Igor Steinmacher}
\IEEEauthorblockA{\textit{Northern Arizona University} \\
Flagstaff, USA \\
igor.steinmacher@nau.edu}
\and
\IEEEauthorblockN{Marcos Kalinowski}
\IEEEauthorblockA{\textit{PUC-Rio} \\
Rio de Janeiro, Brazil \\
kalinowski@inf.puc-rio.br}
}
% // TODO: ajeitar emails
%\and
%\IEEEauthorblockN{Daniel Méndez Fernández}
%\IEEEauthorblockA{Software \& Systems Engineering\\
%Technical University of Munich\\
%Garching, Germany\\
%Email: daniel.mendez@tum.de}










% % conference papers do not typically use \thanks and this command
% % is locked out in conference mode. If really needed, such as for
% % the acknowledgment of grants, issue a \IEEEoverridecommandlockouts
% % after \documentclass

% % for over three affiliations, or if they all won't fit within the width
% % of the page, use this alternative format:
% % 
% %\author{\IEEEauthorblockN{Michael Shell\IEEEauthorrefmark{1},
% %Homer Simpson\IEEEauthorrefmark{2},
% %James Kirk\IEEEauthorrefmark{3}, 
% %Montgomery Scott\IEEEauthorrefmark{3} and
% %Eldon Tyrell\IEEEauthorrefmark{4}}
% %\IEEEauthorblockA{\IEEEauthorrefmark{1}School of Electrical and Computer Engineering\\
% %Georgia Institute of Technology,
% %Atlanta, Georgia 30332--0250\\ Email: see http://www.michaelshell.org/contact.html}
% %\IEEEauthorblockA{\IEEEauthorrefmark{2}Twentieth Century Fox, Springfield, USA\\
% %Email: homer@thesimpsons.com}
% %\IEEEauthorblockA{\IEEEauthorrefmark{3}Starfleet Academy, San Francisco, California 96678-2391\\
% %Telephone: (800) 555--1212, Fax: (888) 555--1212}
% %\IEEEauthorblockA{\IEEEauthorrefmark{4}Tyrell Inc., 123 Replicant Street, Los Angeles, California 90210--4321}}




% % use for special paper notices
% %\IEEEspecialpapernotice{(Invited Paper)}


% make the title area
\maketitle

% % no keywords
% \begin{IEEEkeywords}
% requirements engineering, machine learning, systematic mapping study
% \end{IEEEkeywords}



% % For peer review papers, you can put extra information on the cover
% % page as needed:
% % \ifCLASSOPTIONpeerreview
% % \begin{center} \bfseries EDICS Category: 3-BBND \end{center}
% % \fi
% %
% % For peerreview papers, this IEEEtran command inserts a page break and
% % creates the second title. It will be ignored for other modes.
% % \IEEEpeerreviewmaketitle

% 
% humans are sensitive to the way information is presented.

% introduce framing as the way we address framing. say something about political views and how information is represented.

% in this paper we explore if models show similar sensitivity.

% why is it important/interesting.



% thought - it would be interesting to test it on real world data, but it would be hard to test humans because they come already biased about real world stuff, so we tested artificial.


% LLMs have recently been shown to mimic cognitive biases, typically associated with human behavior~\citep{ malberg2024comprehensive, itzhak-etal-2024-instructed}. This resemblance has significant implications for how we perceive these models and what we can expect from them in real-world interactions and decisionmaking~\citep{eigner2024determinants, echterhoff-etal-2024-cognitive}.

The \textit{framing effect} is a well-known cognitive phenomenon, where different presentations of the same underlying facts affect human perception towards them~\citep{tversky1981framing}.
For example, presenting an economic policy as only creating 50,000 new jobs, versus also reporting that it would cost 2B USD, can dramatically shift public opinion~\cite{sniderman2004structure}. 
%%%%%%%% 图1:  %%%%%%%%%%%%%%%%
\begin{figure}[t]
    \centering
    \includegraphics[width=\columnwidth]{Figs/01.pdf}
    \caption{Performance comparison (Top-1 Acc (\%)) under various open-vocabulary evaluation settings where the video learners except for CLIP are tuned on Kinetics-400~\cite{k400} with frozen text encoders. The satisfying in-context generalizability on UCF101~\cite{UCF101} (a) can be severely affected by static bias when evaluating on out-of-context SCUBA-UCF101~\cite{li2023mitigating} (b) by replacing the video background with other images.}
    \label{fig:teaser}
\end{figure}


Previous research has shown that LLMs exhibit various cognitive biases, including the framing effect~\cite{lore2024strategic,shaikh2024cbeval,malberg2024comprehensive,echterhoff-etal-2024-cognitive}. However, these either rely on synthetic datasets or evaluate LLMs on different data from what humans were tested on. In addition, comparisons between models and humans typically treat human performance as a baseline rather than comparing patterns in human behavior. 
% \gabis{looks good! what do we mean by ``most studies'' or ``rarely'' can we remove those? or we want to say that we don't know of previous work doing both at the same time?}\gili{yeah the main point is that some work has done each separated, but not all of it together. how about now?}

In this work, we evaluate LLMs on real-world data. Rather than measuring model performance in terms of accuracy, we analyze how closely their responses align with human annotations. Furthermore, while previous studies have examined the effect of framing on decision making, we extend this analysis to sentiment analysis, as sentiment perception plays a key explanatory role in decision-making \cite{lerner2015emotion}. 
%Based on this, we argue that examining sentiment shifts in response to reframing can provide deeper insights into the framing effect. \gabis{I don't understand this last claim. Maybe remove and just say we extend to sentiment analysis?}

% Understanding how LLMs respond to framing is crucial, as they are increasingly integrated into real-world applications~\citep{gan2024application, hurlin2024fairness}.
% In some applications, e.g., in virtual companions, framing can be harnessed to produce human-like behavior leading to better engagement.
% In contrast, in other applications, such as financial or legal advice, mitigating the effect of framing can lead to less biased decisions.
% In both cases, a better understanding of the framing effect on LLMs can help develop strategies to mitigate its negative impacts,
% while utilizing its positive aspects. \gabis{$\leftarrow$ reading this again, maybe this isn't the right place for this paragraph. Consider putting in the conclusion? I think that after we said that people have worked on it, we don't necessarily need this here and will shorten the long intro}


% If framing can influence their outputs, this could have significant societal effects,
% from spreading biases in automated decision-making~\citep{ghasemaghaei2024understanding} to reducing public trust in AI-generated content~\citep{afroogh2024trust}. 
% However, framing is not inherently negative -- understanding how it affects LLM outputs can offer valuable insights into both human and machine cognition.
% By systematically investigating the framing effect,


%It is therefore crucial to systematically investigate the framing effect, to better understand and mitigate its impact. \gabis{This paragraph is important - I think that right now it's saying that we don't want models to be influenced by framing (since we want to mitigate its impact, right?) When we talked I think we had a more nuanced position?}




To better understand the framing effect in LLMs in comparison to human behavior,
we introduce the \name{} dataset (Section~\ref{sec:data}), comprising 1,000 statements, constructed through a three-step process, as shown in Figure~\ref{fig:fig1}.
First, we collect a set of real-world statements that express a clear negative or positive sentiment (e.g., ``I won the highest prize'').
%as exemplified in Figure~\ref{fig:fig1} -- ``I won the highest prize'' positive base statement. (2) next,
Second, we \emph{reframe} the text by adding a prefix or suffix with an opposite sentiment (e.g., ``I won the highest prize, \emph{although I lost all my friends on the way}'').
Finally, we collect human annotations by asking different participants
if they consider the reframed statement to be overall positive or negative.
% \gabist{This allows us to quantify the extent of \textit{sentiment shifts}, which is defined as labeling the sentiment aligning with the opposite framing, rather then the base sentiment -- e.g., voting ``negative'' for the statement ``I won the highest prize, although I lost all my friends on the way'', as it aligns with the opposite framing sentiment.}
We choose to annotate Amazon reviews, where sentiment is more robust, compared to e.g., the news domain which introduces confounding variables such as prior political leaning~\cite{druckman2004political}.


%While the implications of framing on sensitive and controversial topics like politics or economics are highly relevant to real-world applications, testing these subjects in a controlled setting is challenging. Such topics can introduce confounding variables, as annotators might rely on their personal beliefs or emotions rather than focusing solely on the framing, particularly when the content is emotionally charged~\cite{druckman2004political}. To balance real-world relevance with experimental reliability, we chose to focus on statements derived from Amazon reviews. These are naturally occurring, sentiment-rich texts that are less likely to trigger strong preexisting biases or emotional reactions. For instance, a review like ``The book was engaging'' can be framed negatively without invoking specific cultural or political associations. 

 In Section~\ref{sec:results}, we evaluate eight state-of-the-art LLMs
 % including \gpt{}~\cite{openai2024gpt4osystemcard}, \llama{}~\cite{dubey2024llama}, \mistral{}~\cite{jiang2023mistral}, \mixtral{}~\cite{mistral2023mixtral}, and \gemma{}~\cite{team2024gemma}, 
on the \name{} dataset and compare them against human annotations. We find  that LLMs are influenced by framing, somewhat similar to human behavior. All models show a \emph{strong} correlation ($r>0.57$) with human behavior.
%All models show a correlation with human responses of more than $0.55$ in Pearson's $r$ \gabis{@Gili check how people report this?}.
Moreover, we find that both humans and LLMs are more influenced by positive reframing rather than negative reframing. We also find that larger models tend to be more correlated with human behavior. Interestingly, \gpt{} shows the lowest correlation with human behavior. This raises questions about how architectural or training differences might influence susceptibility to framing. 
%\gabis{this last finding about \gpt{} stands in opposition to the start of the statement, right? Even though it's probably one of the largest models, it doesn't correlate with humans? If so, better to state this explicitly}

This work contributes to understanding the parallels between LLM and human cognition, offering insights into how cognitive mechanisms such as the framing effect emerge in LLMs.\footnote{\name{} data available at \url{https://huggingface.co/datasets/gililior/WildFrame}\\Code: ~\url{https://github.com/SLAB-NLP/WildFrame-Eval}}

%\gabist{It also raises fundamental philosophical and practical questions -- should LLMs aim to emulate human-like behavior, even when such behavior is susceptible to harmful cognitive biases? or should they strive to deviate from human tendencies to avoid reproducing these pitfalls?}\gabis{$\leftarrow$ also following Itay's comment, maybe this is better in the dicsussion, since we don't address these questions in the paper.} %\gabis{This last statement brings the nuance back, so I think it contradicts the previous parapgraph where we talked about ``mitigating'' the effect of framing. Also, I think it would be nice to discuss this a bit more in depth, maybe in the discussion section.}






% \section{Background on Causal Inference}
\label{sec:background-causal} 



 \newtextold{In this section, we 
 %formalize the notion of {\em Average Treatment Effect and understand the 
 review the basic concepts and key assumptions for inferring the effects of an intervention on the outcome on collected datasets without performing randomized controlled experiments. 
We use {\em Pearl's graphical causal model} for {\em observational causal analysis} \cite{pearl2009causal} to define these concepts.}


\par
\paratitle{Causal Inference and Causal DAGs} The primary goal of causal inference is to model causal dependencies between attributes and evaluate how changing one variable (referred to as intervention) would affect the other.
Pearl's Probabilistic Graphical Causal Model \cite{pearl2009causal} can be written as a tuple $(\exo, \edvar, Pr_{\exo}, \psi)$, where $\exo$ is a set of {\em exogenous} variables, $\Pr_{\exo}$ is the joint distribution of \exo, and $\edvar$ is a set of observed {\em endogenous variables}.
Here $\psi$ is a set of structural equations that encode dependencies among variables. The equation for $A \in \edvar$ takes the following form:
%that encode the dependencies among the variables.  These equations are of the form 
$$\psi_{A}: 
\dom(Pa_{\exo}(A)) {\times} \dom(Pa_{\edvar}(A)) \to \dom(A)$$
Here $Pa_{\exo}(A) {\subseteq} {\exo}$ and $Pa_{\edvar}(A) {\subseteq} \edvar \setminus \{A\}$ respectively denote the exogenous and endogenous parents of $A$. A causal relational model is associated with a directed acyclic graph ({\em causal DAG}) $G$, whose nodes are the endogenous variables $\edvar$ and there is a directed edge from $X$ to $O$ if  $X {\in} Pa_{\edvar}(O)$. The causal DAG obfuscates exogenous variables as they are unobserved. %Any given set of values for the exogenous variables completely determines the values of the endogenous variables by the structural equations (we do not need any known closed-form expressions of the structural equations in this work). 
The probability distribution $\Pr_{\exo}$ on exogenous variables $\exo$ induces a probability distribution  
on the endogenous variables $\edvar$ by the structural equations $\psi$.  A causal DAG can be constructed by a domain expert as in the above example, or using existing {\em causal discovery} algorithms~\cite{glymour2019review}. 



\begin{figure}
    \centering
    \small
    \begin{tikzpicture}[node distance=0.6cm and 1cm, every node/.style={minimum size=0.5cm}]
        \tikzset{vertex/.style = {draw, circle, align=center}}

        \node[vertex] (Ethnicity) {\bf\scriptsize{{Ethnicity}}};
        \node[vertex, right=0.3cm of Ethnicity] (Gender) {\bf{\scriptsize{Gender}}};
        \node[vertex, right=0.3cm of Gender] (Age) {\bf{\scriptsize{Age}}};
        \node[vertex, below=0.3cm of Gender] (Role) {\bf{\scriptsize{Role}}};
        \node[vertex, right=0.3cm of Role] (Education) {\bf{\small{\scriptsize{Education}}}};
        \node[vertex, below=0.3cm of Role] (Salary) {\bf{\scriptsize{Salary}}};

        \draw[->] (Ethnicity) -- (Salary);
        \draw[->] (Gender) -- (Role);
        \draw[->] (Age) -- (Role);
         \draw[->] (Education) -- (Role);
           \draw[->] (Education) -- (Salary);
             \draw[->] (Ethnicity) -- (Education);
                \draw[->] (Ethnicity) -- (Role);
             \draw[->] (Gender) -- (Education);
               \draw[->] (Age) -- (Education);
                 \draw[->] (Role) -- (Salary);
        \draw[->] (Gender) to[bend right] (Salary);
        \draw[->] (Age) -- (Salary);
    \end{tikzpicture}
    \caption{Partial causal DAG for the Stack Overflow dataset.}
    \label{fig:causal_DAG}
\end{figure}



 \begin{example}
Figure \ref{fig:causal_DAG} depicts a partial causal DAG for the SO dataset over the attributes in Table \ref{tab:data} as endogenous variables (we use a larger causal DAG with all 20 attributes in our experiments). 
  Given this causal DAG, we can observe that the role that a coder has in their company depends on their education, age gender and ethnicity.
\end{example}
\par


\par
\paratitle{Intervention} In Pearl's model, a treatment $T = t$ (on one or more variables) is considered as an {\em intervention} to a causal DAG by mechanically changing the DAG such that the values of node(s) of $T$ in $G$ are set to the value(s) in $t$, which is denoted by $\doop(T = t)$. Following this operation, the probability distribution of the nodes in the graph changes as the treatment nodes no longer depend on the values of their parents. Pearl's model gives an approach to estimate the new probability distribution by identifying the confounding factors $Z$ described earlier using conditions such as {\em d-separation} and {\em backdoor criteria} \cite{pearl2009causal}, which we do not discuss in this paper.


\par
\paratitle{Average Treatment Effect} The effects of an intervention are often measured by evaluating
% \par
% \paratitle{Causal inference, Treatment, ATE, and CATE}
% \newtextold{One of the primary goals  of {\em causal inference} is to estimate the effect of making a change in terms of a {\em treatment} $T$ (often referred to as an intervention)
% on the outcome $O$. 
% %A variable that is modified is often referred to as the treatment variable $T$ and the metric used to captures 
% The effect of treatment $T$ on outcome $O$ is measured by 
% %is known as 
{\em Conditional Average treatment effect (CATE)}, 
%a {\em treatment variable} $T$ on an outcome variable $O$ (e.g., what is the effect of higher \verb|Education| on \verb|Salary|). 
measuring the effect of an intervention on a subset of records~\cite{rubin1971use,holland1986statistics} by calculating the difference in average outcomes between the group that receives the treatment and the group that does not (called the {\em control} group), providing an estimate of how the intervention by $T$ influences an outcome $O$ for a given subpopulation. 
% Mathematically,
% \begin{equation}
%     %{\small ATE(T,O) = \mathbb{E}[O \mid \doop(T=1)] -      \mathbb{E}[O \mid \doop(T=0)]}
%     {\small ATE(T, O) = \mathbb{E}[O \mid \doop(T=1)] -  
%     \mathbb{E}[O \mid \doop(T=0)]}
% \label{eq:ate}
% \end{equation}
% In our work, where the treatment with maximum effect may vary among different subpopulations, we are interested in computing the \emph{Conditional Average Treatment Effect} (CATE), which measures the effect of a treatment on an outcome on \emph{a subset of input units}~\cite{rubin1971use,holland1986statistics}. 
Given a subset of the records defined by (a vector of) attributes $B$ and their values $b$, 
%g {\in} \Qagg(\db)$ defined by a predicate $G {=} g$ 
we can compute $CATE(T,O \mid B = b)$ as:
{
\begin{eqnarray}    
    %CATE(T,O \mid G=g) = \mathbb{E}[O \mid \doop(T=1)&, G=g] -  \mathbb{E}[O \mid \doop(T=0), G=g] 
   % CATE(T,O \mid B = b) = 
    \mathbb{E}[O \mid \doop(T=1), B = b] -  
    \mathbb{E}[O \mid \doop(T=0), B = b]\label{eq:cate}
\end{eqnarray}
}
Setting $B=\phi$ is equivalent to the ATE estimate.
The above definitions assumes that the treatment assigned to one unit does not affect the outcome of another unit (called the {Stable Unit Treatment Value Assumption (SUTVA)) \cite{rubin2005causal}}\footnote{This assumption does not hold for causal inference on multiple tables and even on a single table where tuples depend on each other.}. 


The ideal way of estimating the ATE and CATE is through {\em randomized controlled experiments}, 
where the population is randomly divided into two groups (treated and control, for binary treatments): 
%treated group that receives the treatment and control group that does not (denoted by 
%{the \em treated} group 
denoted by 
$\doop(T = 1)$ 
%for a binary treatment)  (the {\em control} group, 
and $\doop(T = 0)$ resp.)~\cite{pearl2009causal}.
%\sr{edited up to here, going to read the rest first, this section should not look like causumx}
%\par
%\par
However, randomized experiments cannot always be performed due to ethical or feasibility issues. In these scenarios, observational data is used to estimate the treatment effect, which requires the following additional assumptions. 
% {\em Observational Causal Analysis} still allows sound causal inference under additional assumptions. Randomization in controlled trials mitigates the effect of {\em confounding factors}, i.e., attributes that can affect the treatment assignment and outcome. Suppose we want to understand the causal effect of \verb|Education| on \verb|Salary| from the SO dataset.  %in Example~\ref{ex:running_example}. 
% We no longer apply Eq. (\ref{eq:ate}) since the values of \verb|Education| were not assigned at random in this data, and obtaining higher education largely depends on other attributes like \verb|Gender|, \verb|Age|, and \verb|Country|. 
% Pearl's model provides ways to account for these confounding attributes $Z$ to get an unbiased causal estimate from observational data under the following assumptions ($\independent$ denotes independence):
% \vspace{-2mm}
\newtextold{
The first assumption is called {\em unconfoundedness} or {\em strong ignorability}  \cite{rosenbaum1983central} says that the independence of outcome $O$ and treatment $T$ conditioning on a set of confounder variables  (covariates) $Z$, i.e.,
%\begin{eqnarray}
 $    O \independent T | Z {=} z$.
 %\label{eq:unconfoundedness}
%\end{eqnarray}
The second assumption called {\em overlap or positivity} says that there is a chance of observing individuals in both the treatment and control groups for every combination of covariate values, i.e., 
%\begin{eqnarray}
   $ 0 < Pr(T {=} 1 ~~|~~Z {=} z)< 1 $.
   %\label{eq:overlap}
%\end{eqnarray}
}
%\sg{Is this overlap or positivity? maybe both are the same?} \sr{yeah - same - from Google AI - The overlap assumption, also known as the positivity assumption, is a key assumption in causal inference that states that there is a chance of observing individuals in both the treatment and control groups for every combination of covariate values.}
% The above conditions are known as {\em Strong Ignorability} in Rubin's model \cite{rubin2005causal}.
The unconfoundedness assumption requires that the treatment $T$ and the outcome $O$ be independent when conditioned on a set of variables $Z$. In SO, assuming that only $Z$ =\{\verb|Gender|, \verb|Age|, \verb|Country|\} affects $T = $ \verb|Education|, if we condition on a fixed set of values of $Z$, i.e., consider people of a given gender, from a given country, and at a given age, then $T = $ \verb|Education| and $O = $ \verb|Salary| are independent. For such confounding factors $Z$,  Eq. (\ref{eq:cate}) reduces to the following form 
(positivity 
gives the feasibility of the expectation difference): 
 \vspace{-1mm}
{\small
\begin{flalign}    
% \begin{eqnarray}
   % % & ATE(T,O) = \mathbb{E}_Z \left[\mathbb{E}[O \mid T=1, Z = z] -  
   %  \mathbb{E}[O \mid T=0, Z = z] \right] \label{eq:conf-ate}\\
 & CATE(T,O {\mid} B {=} b) {=} \nonumber
    \mathbb{E}_Z \left[\mathbb{E}[O {\mid} T{=}1, B {=} b, Z {=} z] {-}  
    \mathbb{E}[O {\mid} T{=}0, B {=} b, Z {=} z]\right]\label{eq:conf-cate}
\end{flalign}
% \end{eqnarray}
}
% \vspace{-4mm}
This equation contains conditional probabilities and not $\doop(T = b)$, which can be estimated from an observed data. 
Pearl's model gives a systematic way to find such a $Z$ when a causal DAG is available. 




% \input{sections/03-protocol.tex}
% \label{evaluation-results}
% \setlength{\tabcolsep}{4.6pt}
% \begin{table*}[t]
% \centering
% \footnotesize
% \begin{tabular}{rcccccc}
% \toprule
%                                & \multicolumn{2}{c}{\textbf{DDxPlus}} & \multicolumn{2}{c}{\textbf{iCraft-MD}} & \multicolumn{2}{c}{\textbf{RareBench}} \\ \cmidrule(lr){2-3} \cmidrule(lr){4-5} \cmidrule(lr){6-7}
%                                & \textbf{GTPA@1 $\uparrow$}          & \textbf{Avg Rank $\downarrow$}   & \textbf{GTPA@1 $\uparrow$}       & \textbf{Avg Rank $\downarrow$}       & \textbf{GTPA@1 $\uparrow$}        & \textbf{Avg Rank $\downarrow$}       \\\midrule
%                                & \multicolumn{6}{c}{\textbf{GPT-4o}}                                                                 \\\midrule
% \textcolor{cyan}{Zero-shot}                      &                &            &             &                &              &                \\
% \textcolor{cyan}{Few-shot (Standard, Dyn\_BAII)} &                &            &             &                &              &                \\
% \textcolor{cyan}{Few-shot (CoT, Dyn\_BAII)}      &                &            &             &                &              &                \\
% History Taking (\textit{n}=5)         & 0.45           & 4.13       & 0.40        & 5.58           & 0.11         & 7.84           \\
% %History Taking (\textit{n}=10)        & 0.59           & 3.16       & 0.45        & 5.35           & 0.24         & 6.67           \\
% History Taking (\textit{n}=15)        & 0.69           & 2.47       & 0.46        & 5.23           & 0.36         & 5.49           \\
% Retrieval (PubMed) \textcolor{red}{rerun/ignore?}                   & 0.69           & 2.27       & 0.68        & 3.23           & 0.45         & 3.92           \\
% MEDDxAgent (\textbf{Ours})         &                &            &             &                &              &                \\
% \textit{iter} =  1                       & 0.74           & 1.91       & 0.52        & 4.93           & 0.51         & 4.37           \\
% \textit{iter} =  2                       & 0.78           & 1.56       & \textbf{0.54}        & \textbf{4.71}           & \textbf{0.56}         & 4.10           \\
% \textit{iter} =  3                       & \textbf{0.86}           & \textbf{1.29}       & \textbf{0.54}        & 4.80           & 0.50         & \textbf{4.09}           \\\midrule
%                                & \multicolumn{6}{c}{\textbf{Llama3.1-70B}}                                                           \\ \midrule
% \textcolor{cyan}{Zero-shot}                      &                &            &             &                &              &                \\
% \textcolor{cyan}{Few-shot (Standard, Dyn\_BAII)} &                &            &             &                &              &                \\
% \textcolor{cyan}{Few-shot (CoT, Dyn\_BAII)}      &                &            &             &                &              &                \\
% History Taking (\textit{n}=5)         & 0.45           & 4.15       & 0.29        & 6.48           & 0.30         & 6.04           \\
% %History Taking (\textit{n}=10)        & 0.58           & 3.12       & 0.33        & 5.82           & 0.36         & 4.51           \\
% History Taking (\textit{n}=15)        & 0.56           & 3.50       & 0.36        & 5.36           & 0.31         & 4.80           \\
% Retrieval (PubMed)  \textcolor{red}{rerun/ignore?}                 & 0.56           & 3.42       & 0.44        & 4.72           & 0.38         & 3.96           \\
% MEDDxAgent (\textbf{Ours})         &                &            &             &                &              &                \\
% \textit{iter} =  1                       & 0.61           & 2.91       & 0.29        & 7.05           & 0.39         & 5.05           \\
% \textit{iter} =  2                       & \textbf{0.71}   & \textbf{2.20}       & 0.37        & \textbf{6.26}           & \textbf{0.48}         & 4.48           \\
% \textit{iter} =  3                       & 0.68   & 2.30       & \textbf{0.42}        & 6.31           & \textbf{0.48}         & \textbf{4.30}           \\\midrule
%                                & \multicolumn{6}{c}{\textbf{Llama3.1-8B}}                                                            \\\midrule
% \textcolor{cyan}{Zero-shot}                      &                &            &             &                &              &                \\
% \textcolor{cyan}{Few-shot (Standard, Dyn\_BAII)} &                &            &             &                &              &                \\
% \textcolor{cyan}{Few-shot (CoT, Dyn\_BAII)}     &                &            &             &                &              &                \\
% History Taking (\textit{n}=5)         & 0.23           & 6.85       & 0.10        & 8.78           & 0.05         & 8.38           \\
% %History Taking (\textit{n}=10)        & 0.35           & 5.46       & 0.12        & 8.39           & \textbf{0.13}         & 8.25           \\
% History Taking (\textit{n}=15)        & 0.40           & 5.44       & 0.11        & 8.30           & \textbf{0.11}        & 8.95           \\
% Retrieval (PubMed)  \textcolor{red}{rerun/ignore?}                 & 0.42           & 4.50       & 0.29        & 6.93           & 0.35         & 5.33           \\
% MEDDxAgent (\textbf{Ours})         &                &            &             &                &              &                \\
% \textit{iter} =  1                       & 0.34           & 5.25       & 0.11        & 9.38           & 0.08         & 8.47           \\
% \textit{iter} =  2                       & 0.56           & 3.59       & \textbf{0.14}        & 9.22           & 0.09         & \textbf{8.11}           \\
% \textit{iter} =  3                       & \textbf{0.58}           & \textbf{3.10}       & 0.12        & \textbf{9.07}           & 0.07         & 8.56        \\  
% \bottomrule
%     \end{tabular}
%     \caption{Iterative experiment performance across 3 datasets. \textcolor{red}{The \textbf{best results} are based on ignoring the Pubmed retrieval results!}}
%     \label{tab:iterative_overall}
% \end{table*}

\setlength{\tabcolsep}{3.8pt}
\begin{table*}[ht]
\centering
\scriptsize
\begin{tabular}{rccccccccc}
\toprule
                               & \multicolumn{3}{c}{\textbf{DDxPlus}} & \multicolumn{3}{c}{\textbf{iCraft-MD}} & \multicolumn{3}{c}{\textbf{RareBench}} \\ \cmidrule(lr){2-4} \cmidrule(lr){5-7} \cmidrule(lr){8-10}
                               & \textbf{GTPA@1 $\uparrow$}          & \textbf{Avg Rank $\downarrow$}   & \textbf{$\Delta$ Progress} & \textbf{GTPA@1 $\uparrow$}       & \textbf{Avg Rank $\downarrow$}     & \textbf{$\Delta$ Progress}   & \textbf{GTPA@1 $\uparrow$}        & \textbf{Avg Rank $\downarrow$}   & \textbf{$\Delta$ Progress}     \\\midrule
                               & \multicolumn{9}{c}{\textbf{GPT-4o}}                                                                 \\\midrule
%\textcolor{cyan}{Zero-shot}                      &     0.69           &    2.21        &      -      &       0.68         &     3.37         &         -       &       0.46       & 3.99            &   -              \\
%\textcolor{cyan}{Zero-shot (CoT)}                      &     0.71          &    2.10        &      -      &       0.68         &     3.35         &         -       &       0.47       & 4.02            &   -              \\
%\textcolor{cyan}{Few-shot (CoT, Dyn\_BAII)} &                &            &     -        &                &              &          -      &              &             &            -     \\
%\textcolor{cyan}{Few-shot (CoT, Dyn\_BERT/Dyn\_BAII)}      &                &            &       -      &                &              &          -      &              &             &           -      \\
%\textit{Single-Turn}      &                &            &       -      &                &              &          -      &              &             &           -      \\
KR (\textit{n}=0)                &      0.18      & 7.33  &  -  &   0.15      &    8.27     & -  &       0.07   &  9.07  &    -   \\
DS (\textit{n}=0)     &  0.27    &    6.01        &       -      &      0.18          &      7.87        &          -      &       0.11       &     8.38        &           -      \\
%SDS (\textit{n}=5)         & 0.45           & 4.13     & - & 0.40        & 5.58     &    -  & 0.11         & 7.84     &  -    \\
KR (\textit{n}=5)  &      0.52      & 3.32   &  -  &  0.49  &  5.36       & -  &     0.40   &   5.27 &    -   \\
DS (\textit{n}=5)  &    0.72       &  2.14 &  -  &  0.40 &    5.55   & -  &   0.50    &   4.94 &    -   \\\cmidrule(lr){2-10}
%History Taking (\textit{n}=10)        & 0.59           & 3.16    & -  & 0.45        & 5.35        & -  & 0.24         & 6.67      &  -   \\
%SDS (\textit{n}=15)        & 0.69           & 2.47     & - & 0.46        & 5.23      &  -   & 0.36         & 5.49      &    - \\\cmidrule(lr){2-10}

%Retrieval (Wiki) \textcolor{red}{rerun}                   &            &    &  -  &         &         & -  &          &    &    -   \\
%MEDDxAgent         &                &            &             &                &              &               &              &             &                  \\
 MEDDx (\textit{iter}=1, \textit{n}=5)                       & 0.74           & 1.91     & ~~0.00 & 0.52        & 4.93      &  ~~0.00   & 0.51         & 4.37        &   ~~0.00\\
MEDDx (\textit{iter}=2, \textit{n}=10)                       & 0.78           & 1.56    & +0.32  & \textbf{0.54}        & \textbf{4.71}    &    +0.26   & \textbf{0.56}         & 4.10   &     +0.13   \\
MEDDx (\textit{iter}=3, \textit{n}=15)                       & \textbf{0.86}           & \textbf{1.29}    & +0.32  & \textbf{0.54}        & 4.80      & +0.17    & 0.50         & \textbf{4.09}       &   +0.16 \\\midrule
                               & \multicolumn{9}{c}{\textbf{Llama3.1-70B}}                                                           \\ \midrule
%\textcolor{cyan}{Zero-shot}                      &      0.54          &     3.53       &       -      &     0.40           &       4.87       &         -      &      0.39        &    4.05         &         -         \\
%\textcolor{cyan}{Zero-shot (CoT)}                      &     0.45          &    3.69       &      -      &       0.48         &     4.50         &         -       &       0.49       & 3.91            &   -              \\
%\textcolor{cyan}{Few-shot (Standard, Dyn\_BAII)} &                &            &      -       &                &              &          -      &              &             &        -         \\
%\textcolor{cyan}{Few-shot (CoT, Dyn\_BERT/Dyn\_BAII)}      &                &            &      -       &                &              &          -    &              &             &                -   \\
KR (\textit{n}=0)           &   0.19         &  7.58  &  -  &      0.13   &   8.19      & -  &    0.09      &  9.13  &    -   \\
DS (\textit{n}=0)    &        0.17        &       7.28     &       -      &      0.11          &      8.74        &          -      &       0.20       &      6.81       &           -      \\
%History Taking (\textit{n}=5)         & 0.45           & 4.15    &  - & 0.29        & 6.48   &     -   & 0.30         & 6.04      &   -  \\
KR (\textit{n}=5)  &      0.39      & 5.03   &  -  &  0.34  &  6.86       & -  &     0.29   &   5.86 &    -   \\
DS (\textit{n}=5)  &     0.50      &  2.89 &  -  & 0.24 &    7.33   & -  &   0.23    &  5.77  &    -   \\\cmidrule(lr){2-10}
%History Taking (\textit{n}=10)        & 0.58           & 3.12     & - & 0.33        & 5.82    &    -   & 0.36         & 4.51    &    -   \\
%History Taking (\textit{n}=15)        & 0.56           & 3.50      &- & 0.36        & \textbf{5.36}       &  -  & 0.31         & 4.80    &    -   \\\cmidrule(lr){2-10}
%Retrieval (Wiki) \textcolor{red}{rerun}                   &            &    &  -  &         &         & -  &          &    &    -   \\
%MEDDxAgent         &                &            &             &                &              &            &                &              &        \\
MEDDx (\textit{iter}=1, \textit{n}=5)                       & 0.61           & 2.91    & ~~0.00  & 0.29        & 7.05       & ~~0.00   & 0.39         & 5.05   &     ~~0.00   \\
MEDDx (\textit{iter}=2, \textit{n}=10)                       & \textbf{0.71}   & \textbf{2.20}      & +0.41 & 0.37        & \textbf{6.26}      & +0.07   & \textbf{0.48}         & 4.48  &    +0.75     \\
MEDDx (\textit{iter}=3, \textit{n}=15)                       & 0.68   & 2.30    & +0.17  & \textbf{0.42}        & 6.31     &   +0.26   & \textbf{0.48}         & \textbf{4.30}      &   +0.44  \\\midrule
                               & \multicolumn{9}{c}{\textbf{Llama3.1-8B}}                                                            \\\midrule
%\textcolor{cyan}{Zero-shot}                      &    0.45            &   9.00         &   -          &    0.27             &     7.02         &      -         &    0.33           &  5.45           &             -     \\
%\textcolor{cyan}{Zero-shot (CoT)}                      &    0.45            &   4.51         &   -          &    0.27             &     7.25         &      -         &    0.24           &  5.65           &             -     \\
%\textcolor{cyan}{Few-shot (Standard, Dyn\_BAII)} &                &            &     -        &                &              &        -      &              &             &             -      \\
%\textcolor{cyan}{Few-shot (CoT, Dyn\_BERT/Dyn\_BAII)}     &                &            &       -      &                &              &         -     &              &             &           -        \\
KR (\textit{n}=0)     &     0.20       &  7.49  &  -  &   0.11      &  \textbf{8.86}       & -  &     \textbf{0.11}     &  8.58  &    -   \\
DS (\textit{n}=0)  &       0.16         &       8.45     &       -      &      0.03          &      10.37        &          -      &         0.04     &       8.52      &           -      \\
%History Taking (\textit{n}=5)         & 0.23           & 6.85    & -  & 0.10        & 8.78        &  - & 0.05         & 8.38       &   - \\
KR (\textit{n}=5)  &      0.21      & 7.42   &  -  &  0.09  &  9.48       & -  &     0.04   &   9.69 &    -   \\
DS (\textit{n}=5)  &     0.23      &  5.77  &  -  &  0.03 &   10.08    & -  &   0.06    &  8.64  &    -   \\\cmidrule(lr){2-10}
%History Taking (\textit{n}=10)        & 0.35           & 5.46    &  - & 0.12        & 8.39     &   -   & \textbf{0.13}         & 8.25   &     -   \\
%History Taking (\textit{n}=15)        & 0.40           & 5.44   &  -  & 0.11        & \textbf{8.30}       &  -  & \textbf{0.11}        & 8.95       &  -  \\\cmidrule(lr){2-10}
%Retrieval (Wiki) \textcolor{red}{rerun}                   &            &    &  -  &         &         & -  &          &    &    -   \\
%MEDDxAgent         &                &            &             &                &              &               &                &              &     \\
MEDDx (\textit{iter}=1, \textit{n}=5)                       & 0.34           & 5.25   &   ~~0.00 & 0.11        & 9.38       &  ~~0.00  & 0.08         & 8.47    &    ~~0.00   \\
MEDDx (\textit{iter}=2, \textit{n}=10)                       & 0.56           & 3.59    & +1.73  & \textbf{0.14}        & 9.22       &  +0.22  & 0.09         & \textbf{8.11}      &  +0.44   \\
MEDDx (\textit{iter}=3, \textit{n}=15)                       & \textbf{0.58}           & \textbf{3.10}    &  +1.23 & 0.12        & 9.07     & +0.17     & 0.07         & 8.56    &  +0.38  \\  
\bottomrule
    \end{tabular}
    \vspace{-0.8em}
    \caption{Interactive experiment performance across 3 datasets without \textit{full} patient profile, with KR: knowledge retrieval agent; DS: diagnosis strategy agent; $n$ is the number of turns of the simulator; MEDDx uses KR+DS.
    %We compare the single-turn (\textit{upper}) with the proposed iterative setup for MEDDxAgent (\textit{bottom}). The selection of the agents and simulator are optimized (\autoref{subsec:optimize-agents}), unless controlled by the number of questions ($n$) asked from history taking simulator.\cc{May be we just need 3 entries of single turn for best agents and simulator compared to MEDDxAgent? For the others we leave it for ablation study?}} %\cl{why we don't have the baseline with diagnosis strategy only without full patient profile (e.g., few-shot CoT, Dyn\_BAII?)}
    }
    \label{tab:interactive_overall}
    \vspace{-1.8em}
\end{table*}

We experiment on two configurations: (1) optimizing individual agents (\autoref{subsec:optimize-agents}), by determining the best settings for knowledge retrieval and diagnosis strategy agents; and (2) interactive differential diagnosis (\autoref{subsec:iterative_learning}), where the optimized agents are used to assess MEDDxAgent's performance in the interactive DDx setup.

\subsection{Optimizing Individual Agents}
\label{subsec:optimize-agents}

We first explore the optimal single-turn configuration for the knowledge retrieval and diagnosis strategy agents, before integrating them into iterative setup. For this, we provide the full patient profile as in previous work~\cite{wu2024streambench,chen2024rarebench}, and present the results in~\autoref{tab:with_patient_profile}. For the knowledge retrieval agent, PubMed performs slightly better overall than Wikipedia, especially for Rarebench, which demands more complex disease information. For the diagnosis strategy agent, the best setting varies by dataset. 
Namely, dynamic few-shot with BAII embeddings performs the best on DDxPlus and RareBench, where relevant patient examples offer reliable contextual cues to likely diseases. 
In contrast, iCraft-MD benefits more from zero-shot CoT, which enables structured reasoning through complex clinical vignettes. Few-shot learning often decreases performance for iCraft-MD because each patient vignette is distinct, so additional examples can introduce noise.
Based on the above findings, we select the following configurations for the iterative scenario:\footnote{We do not run all possible settings in the interactive environment due to cost reasons.} PubMed for knowledge retrieval agent; few-shot (dynamic BAII) for DDxPlus and RareBench, and zero-shot (CoT) for iCraft-MD for diagnosis strategy agent.

\begin{figure*}[t]
    \centering
    \begin{subfigure}{0.48\textwidth}
    \includegraphics[trim={0.2cm 0cm 0cm 0cm },clip, width=\textwidth]{img/ddxplus_history.pdf}
    \vspace{-1.8em}
    \caption{}
    \end{subfigure}
    \begin{subfigure}{0.48\textwidth}
    \includegraphics[trim={0.2cm 0cm 0cm 0cm}, clip, width=\textwidth]{img/agent_iterations_plot_ddxplus.pdf}
    \vspace{-1.8em}
    \caption{}
    \end{subfigure}
    \vspace{-0.5em}
    \caption{Results of DDxPlus compared between (a) history taking simulator, and (b) MEDDxAgent, over the number of questions and iterations. For brevity, the results of iCraft-MD and RareBench are in~\autoref{subsec:comparison_history_taking_iterative}.}
    \label{fig:ddxplus_comparison}
    \vspace{-1.8em}
\end{figure*}

\subsection{Interactive Differential Diagnosis}
\label{subsec:interactive_differential_diagnosis}
We now evaluate the more challenging task of interactive DDx, where we begin with limited patient information and the history taking simulator enables the interactive environment~(\autoref{tab:interactive_overall}).
At $n=0$, the simulator has not yet learned any patient information, and performance drops significantly from observing the full patient profile (\autoref{tab:with_patient_profile}). 
For GPT-4o in RareBench, the knowledge retrieval agent (KR)'s GTPA@1 drops from 0.45  to 0.07. Similarly, the diagnosis strategy agent (DS) drops from 0.46 (zero-shot) to 0.11. This simple baseline showcases that previous evaluations do not hold well in the interactive setup with initially limited patient information. 
Already for $n=5$, we find a large boost in performance for both KR and DS. These findings reinforce the importance of history taking for diagnostic precision. 
We illustrate the trend for changing $n$ in~\autoref{fig:ddxplus_comparison} and find that gains also plateau around \textit{n}=10-15 questions, reinforcing the optimal balance between information gathering and diagnostic efficiency \cite{ely1999analysis}.

Finally, we run MEDDxAgent, which calls KR+DS in the \textit{fixed iteration} pipeline (\autoref{subsec:iterative_learning}). MEDDxAgent exhibits clear improvements over the KR and DS baselines for $n=5$, supporting our hypothesis that all three modules are important for interactive DDx. It also improves significantly over the history taking baselines, as we illustrate in \autoref{fig:ddxplus_comparison}. MEDDxAgent is also capable of improving upon the zero-shot setting with the full patient profile (\autoref{tab:with_patient_profile}). For DDxPlus, GTPA@1 for GPT-4o and Llama3.1-70B rise from 0.56 to 0.86 and from 0.46 to 0.71, respectively. For Llama3.1-8B, the trend continues for DDxPlus but inconsistently for iCraft-MD and RareBench, highlighting the importance of model scale. Notably, MEDDxAgent improves over successive iterations, though the optimal number of iterations (2, 3) depends on the dataset and LLM. The values of $\Delta$ are consistently positive, indicating that MEDDxAgent iteratively increases the rank of the ground-truth diagnosis over time. $\Delta$ Progress also varies by dataset and model, offering explainable insight to the diagnosistic improvement of MEDDxAgent. The overall results show that MEDDxAgent can operate well in the challenging, realistic setup of interactive DDx. Additionally, MEDDxAgent logs all intermediate reasoning, action, and observations, providing critical insight into its DDx process (\autoref{fig:Example}).
\vspace{-0.5em}
% \section{Conclusions and Future Work}
\label{sec:conclusions}
%\nabanita{Nabanita: ADD Percentage increase and other comparisons as necessary!}
Embodied agents assisting humans frequently have to complete previously unseen tasks or operate in new scenario. This paper describes a framework that leverages the complementary strengths of Large Language Models (LLMs), Knowledge Graphs (KGs), and Human-in-the-Loop (HITL) feedback to satisfy this requirement. Specifically, the generic task decomposition ability of LLMs is used to predict a sequence of abstract actions to complete any given task. This sequence is adapted to the specific scenario(s) and the task-, agent-, or domain-specific constraints using a KG that encodes prior knowledge of some objects, object attributes, and action capabilities. Any unresolved mismatch between the KG and the LLM output, and any unexpected action outcomes, are addressed by soliciting and using human input. This HITL feedback corrects errors and refines the existing knowledge (in the KG) for subsequent operation. Experimental evaluation in two simulated domains demonstrates substantial performance improvement compared with baselines, and illustrates incremental acquisition of knowledge to adapt to new classes of tasks.

%Our framework successfully merges the unique capabilities of Large Language Models (LLMs), Knowledge Graphs (KGs), and human feedback to improve how embodied agents respond to unfamiliar situations. LLMs provide a foundation by generating high-level action plans, which are then refined with detailed, domain-specific knowledge from KGs. This allows agents to adjust on the fly, even when encountering new objects or tasks. Additionally, incorporating human-in-the-loop (HITL) feedback offers real-time updates and fine-tuning, ensuring the agents continually evolve and expand their understanding. This strategy is particularly useful in everyday activities like cooking or cleaning, where agents can quickly understand and execute tasks without requiring extensive retraining. By combining LLMs, KGs, and human insights, the system ensures that agents remain adaptable and efficient, even in unpredictable environments.

% paves the way for exciting future developments, such as expanding the framework to handle a wider variety of tasks and environments, fine-tuning the balance between automation and human input, and exploring ways to make knowledge refinement more autonomous. Ultimately, our approach marks a significant step toward building smarter, more adaptive, and human-centered assistive agents.
\vspace{-0.75em}
This research opens up multiple avenues for further research. First, we will explore the use of this framework in many more classes of tasks, building on (and reinforcing) the promising results obtained so far.  Second, we will investigate the trade-off between automating the generation of an action sequence for any given task, and soliciting and incorporating human feedback as needed. Furthermore, we will explore the use of this framework on a physical robot platform assisting humans. The long-term objective is to create assistive agents and robots that can interact and collaborate with humans in different application domains.
% \section{Threats to Validity}
\label{Section:Threats}

In this section, we describe potential threats to the validity of our research method and the actions we took to mitigate them.

\textbf{Internal Validity.} The accuracy of our analysis is primarily dependent on the precision of the refactoring mining tools, as these tools may miss the detection of some refactorings. However, previous studies \citep{silva2016we,tsantalis2018accurate,silva2017refdiff} report that \texttt{RefactoringMiner} and \texttt{RefDiff} have high precision and recall scores compared to other state-of-the-art refactoring detection tools, giving us confidence in using the tools. Another potential threat to validity is related to commit messages. \textcolor{black}{This study does not exclude commits containing tangle code changes \citep{herzig2016impact,kirinuki2014hey}, where developers made changes related to different tasks and one of these tasks could be related to quality improvement. If these changes were committed at once, there is a possibility that the individual changes merge and that the original task cannot be traced back. Similarly to the previous study \cite{pantiuchina2018improving}, we did not consider filtering out such changes in this study}. Moreover, our manual analysis is time-consuming and error-prone, which we tried to mitigate by focusing mainly on commits known to contain refactorings. 

Another potential threat to validity is sample bias, where the choice of the data can directly impact the results. Therefore, we explored a large sample of projects from the SmartSHARK dataset \citep{trautsch2021msr}, to ensure the quality of the findings and diversify the sources to reduce the bias of the data belonging to the same entity. The qualitative analysis was conducted by a single author, which could introduce bias into the process. However, commits that were debatable were discarded. We also provide our dataset online for further refinement and analysis. %During our qualitative analysis, we consideblack only commits where a consensus between authors was reached on whether a message clearly states the removal of duplicate code. Commits that were debatable were discarded. We also provide our dataset online for further refinement and analysis.

\textbf{Construct Validity.} A potential threat to construct validity relates to the set of metrics, as it may miss some properties of the selected internal quality attributes. To address this potential threat, we mitigate it by choosing well-known metrics that encompass various properties of each attribute, as reported in the literature \citep{chidamber1994metrics}.

\textbf{External Validity.} Our analysis was limited to only open-source Java projects. However, we were able to examine 128 projects, which were well-commented and exhibited diversity in terms of size, contributors, number of commits, and refactorings. \textcolor{black}{Still, we believe that the results found in this study are largely language-agnostic. However, certain language-specific characteristics, such as syntax complexity and tooling support, can influence duplication patterns. Although we expect similar trends across languages with similar paradigms, a comprehensive analysis encompassing various languages is recommended to confirm this generalization.}

%Still, we believe that the removal of duplicates is largely language-agnostic. However, certain language-specific characteristics, such as syntax complexity and tooling support, can influence duplication patterns. Although we expect similar trends across languages with similar paradigms, a comprehensive analysis encompassing various languages is suggested to is recommended to confirm this generalization.
% \input{sections/07-remarks.tex}

% \bibliographystyle{IEEEtranS}
% \bibliography{bibTex/sigproc} 

% \onecolumn
\appendix
\section{Relevant proofs}
\subsection{Proof of Theorem 2}
\label{proofth1}
\begin{proof}
\label{th1proof}   
 The proof is based on Borkar's Theorem for
 general stochastic approximation recursions with two time scales \cite{borkar1997stochastic}. 
 A new one-step linear TD solution is defined as: 
\begin{equation*}
0=\mathbb{E}[(\delta-\mathbb{E}[\delta]) \phi]=-A\theta+b.
\end{equation*}
Thus, the VMTD's solution is $\theta_{\text{VMTD}}=A^{-1}b$. First, note that recursion (\ref{theta}) can be rewritten as
\begin{equation*}
\theta_{k+1}\leftarrow \theta_k+\beta_k\xi(k),
\end{equation*}
where
\begin{equation*}
\xi(k)=\frac{\alpha_k}{\beta_k}(\delta_k-\omega_k)\phi_k
\end{equation*}
Due to the settings of step-size schedule $\alpha_k = o(\beta_k)$,
$\xi(k)\rightarrow 0$ almost surely as $k\rightarrow\infty$. 
 That is the increments in iteration (\ref{omega}) are uniformly larger than
 those in (\ref{theta}), thus (\ref{omega}) is the faster recursion.
 Along the faster time scale, iterations of (\ref{omega}) and (\ref{theta})
 are associated with the ODEs system as follows:
\begin{equation}
 \dot{\theta}(t) = 0,
\label{thetaFast}
\end{equation}
\begin{equation}
 \dot{\omega}(t)=\mathbb{E}[\delta_t|\theta(t)]-\omega(t).
\label{omegaFast}
\end{equation}
 Based on the ODE (\ref{thetaFast}), $\theta(t)\equiv \theta$ when
 viewed from the faster timescale. 
 By the Hirsch lemma \cite{hirsch1989convergent}, it follows that
$||\theta_k-\theta||\rightarrow 0$ a.s. as $k\rightarrow \infty$ for some
$\theta$ that depends on the initial condition $\theta_0$ of recursion
 (\ref{theta}).
 Thus, the ODE pair (\ref{thetaFast})-(\ref{omegaFast}) can be written as
\begin{equation}
 \dot{\omega}(t)=\mathbb{E}[\delta_t|\theta]-\omega(t).
\label{omegaFastFinal}
\end{equation}
 Consider the function $h(\omega)=\mathbb{E}[\delta|\theta]-\omega$,
 i.e., the driving vector field of the ODE (\ref{omegaFastFinal}).
 It is easy to find that the function $h$ is Lipschitz with coefficient
$-1$.
 Let $h_{\infty}(\cdot)$ be the function defined by
$h_{\infty}(\omega)=\lim_{x\rightarrow \infty}\frac{h(x\omega)}{x}$.
 Then  $h_{\infty}(\omega)= -\omega$,  is well-defined. 
 For (\ref{omegaFastFinal}), $\omega^*=\mathbb{E}[\delta|\theta]$
 is the unique globally asymptotically stable equilibrium.
 For the ODE
  \begin{equation}
 \dot{\omega}(t) = h_{\infty}(\omega(t))= -\omega(t),
 \label{omegaInfty}
 \end{equation}
 apply $\vec{V}(\omega)=(-\omega)^{\top}(-\omega)/2$ as its
 associated strict Liapunov function. Then,
 the origin of (\ref{omegaInfty}) is a globally asymptotically stable
 equilibrium. Consider now the recursion (\ref{omega}).
 Let $M_{k+1}=(\delta_k-\omega_k)
 -\mathbb{E}[(\delta_k-\omega_k)|\mathcal{F}(k)]$,
 where $\mathcal{F}(k)=\sigma(\omega_l,\theta_l,l\leq k;\phi_s,\phi_s',r_s,s<k)$, $k\geq 1$ are the sigma fields
 generated by $\omega_0,\theta_0,\omega_{l+1},\theta_{l+1},\phi_l,\phi_l'$, $0\leq l<k$.
 It is easy to verify that $M_{k+1},k\geq0$ are integrable random variables that 
 satisfy $\mathbb{E}[M_{k+1}|\mathcal{F}(k)]=0$, $\forall k\geq0$.
 Because $\phi_k$, $r_k$, and $\phi_k'$ have
 uniformly bounded second moments, it can be seen that for some constant $c_1>0$, $\forall k\geq0$,
\begin{equation*}
 \mathbb{E}[||M_{k+1}||^2|\mathcal{F}(k)]\leq
 c_1(1+||\omega_k||^2+||\theta_k||^2).
\end{equation*}
Now Assumptions (A1) and (A2) of \cite{borkar2000ode} are verified.
 Furthermore, Assumptions (TS) of \cite{borkar2000ode} are satisfied by our
 conditions on the step-size sequences $\alpha_k$, $\beta_k$. Thus,
 by Theorem 2.2 of \cite{borkar2000ode} we obtain that $||\omega_k-\omega^*||\rightarrow 0$ almost surely as $k\rightarrow \infty$.
Consider now the slower time scale recursion (\ref{theta}).
 Based on the above analysis, (\ref{theta}) can be rewritten as 
\begin{equation*}
\theta_{k+1}\leftarrow
\theta_{k}+\alpha_k(\delta_k-\mathbb{E}[\delta_k|\theta_k])\phi_k.
\end{equation*}
Let $\mathcal{G}(k)=\sigma(\theta_l,l\leq k;\phi_s,\phi_s',r_s,s<k)$, 
$k\geq 1$ be the sigma fields
 generated by $\theta_0,\theta_{l+1},\phi_l,\phi_l'$,
$0\leq l<k$.
 Let $Z_{k+1} = Y_{t}-\mathbb{E}[Y_{t}|\mathcal{G}(k)]$,
 where
\begin{equation*}
 Y_{k}=(\delta_k-\mathbb{E}[\delta_k|\theta_k])\phi_k.
\end{equation*}
 Consequently,
\begin{equation*}
\begin{array}{ccl}
 \mathbb{E}[Y_t|\mathcal{G}(k)]&=&\mathbb{E}[(\delta_k-\mathbb{E}[\delta_k|\theta_k])\phi_k|\mathcal{G}(k)]\\
&=&\mathbb{E}[\delta_k\phi_k|\theta_k]
 -\mathbb{E}[\mathbb{E}[\delta_k|\theta_k]\phi_k]\\
&=&\mathbb{E}[\delta_k\phi_k|\theta_k]
 -\mathbb{E}[\delta_k|\theta_k]\mathbb{E}[\phi_k]\\
&=&\mathrm{Cov}(\delta_k|\theta_k,\phi_k),
\end{array}
\end{equation*}
 where $\mathrm{Cov}(\cdot,\cdot)$ is a covariance operator.
Thus,
 \begin{equation*}
\begin{array}{ccl}
 Z_{k+1}&=&(\delta_k-\mathbb{E}[\delta_k|\theta_k])\phi_k-\mathrm{Cov}(\delta_k|\theta_k,\phi_k).
\end{array}
\end{equation*}
 It is easy to verify that $Z_{k+1},k\geq0$ are integrable random variables that 
 satisfy $\mathbb{E}[Z_{k+1}|\mathcal{G}(k)]=0$, $\forall k\geq0$.
 Also, because $\phi_k$, $r_k$, and $\phi_k'$ have
 uniformly bounded second moments, it can be seen that for some constant
$c_2>0$, $\forall k\geq0$,
\begin{equation*}
 \mathbb{E}[||Z_{k+1}||^2|\mathcal{G}(k)]\leq
 c_2(1+||\theta_k||^2).
\end{equation*}

 Consider now the following ODE associated with (\ref{theta}):
\begin{equation}
\begin{array}{ccl}
 \dot{\theta}(t)&=&\mathrm{Cov}(\delta|\theta(t),\phi)\\
&=&\mathrm{Cov}(r+(\gamma\phi'-\phi)^{\top}\theta(t),\phi)\\
&=&\mathrm{Cov}(r,\phi)-\mathrm{Cov}(\theta(t)^{\top}(\phi-\gamma\phi'),\phi)\\
&=&\mathrm{Cov}(r,\phi)-\theta(t)^{\top}\mathrm{Cov}(\phi-\gamma\phi',\phi)\\
&=&\mathrm{Cov}(r,\phi)-\mathrm{Cov}(\phi-\gamma\phi',\phi)^{\top}\theta(t)\\
&=&\mathrm{Cov}(r,\phi)-\mathrm{Cov}(\phi,\phi-\gamma\phi')\theta(t)\\
&=&-A\theta(t)+b.
\end{array}
\label{odetheta}
\end{equation}
 Let $\vec{h}(\theta(t))$ be the driving vector field of the ODE
 (\ref{odetheta}).
\begin{equation*}
 \vec{h}(\theta(t))=-A\theta(t)+b.
\end{equation*}
 Consider the cross-covariance matrix,
\begin{equation}
\begin{array}{ccl}
 A &=& \mathrm{Cov}(\phi,\phi-\gamma\phi')\\
  &=&\frac{\mathrm{Cov}(\phi,\phi)+\mathrm{Cov}(\phi-\gamma\phi',\phi-\gamma\phi')-\mathrm{Cov}(\gamma\phi',\gamma\phi')}{2}\\
  &=&\frac{\mathrm{Cov}(\phi,\phi)+\mathrm{Cov}(\phi-\gamma\phi',\phi-\gamma\phi')-\gamma^2\mathrm{Cov}(\phi',\phi')}{2}\\
  &=&\frac{(1-\gamma^2)\mathrm{Cov}(\phi,\phi)+\mathrm{Cov}(\phi-\gamma\phi',\phi-\gamma\phi')}{2},\\
\end{array}
\label{covariance}
\end{equation}
 where we eventually used $\mathrm{Cov}(\phi',\phi')=\mathrm{Cov}(\phi,\phi)$
\footnote{The covariance matrix $\mathrm{Cov}(\phi',\phi')$ is equal to
 the covariance matrix $\mathrm{Cov}(\phi,\phi)$ if the initial state is re-reachable or
 initialized randomly in a Markov chain for on-policy update.}.
 Note that the covariance matrix $\mathrm{Cov}(\phi,\phi)$ and
$\mathrm{Cov}(\phi-\gamma\phi',\phi-\gamma\phi')$ are semi-positive
 definite. Then, the matrix $A$ is semi-positive definite because  $A$ is
 linearly combined  by  two positive-weighted semi-positive definite matrice
 (\ref{covariance}).
 Furthermore, $A$ is nonsingular due to the assumption.
 Hence, the cross-covariance matrix $A$ is positive definite.

 Therefore,
$\theta^*=A^{-1}b$ can be seen to be the unique globally asymptotically
 stable equilibrium for ODE (\ref{odetheta}).
 Let $\vec{h}_{\infty}(\theta)=\lim_{r\rightarrow
\infty}\frac{\vec{h}(r\theta)}{r}$. Then
$\vec{h}_{\infty}(\theta)=-A\theta$ is well-defined. 
 Consider now the ODE
\begin{equation}
 \dot{\theta}(t)=-A\theta(t).
\label{odethetafinal}
\end{equation}
 The ODE (\ref{odethetafinal}) has the origin of its unique globally asymptotically stable equilibrium.
 Thus, the assumption (A1) and (A2) are verified.
    \end{proof}

\subsection{Proof of Theorem 3}
\label{proofth2}
\begin{proof}
The proof is similar to that given by \cite{sutton2009fast} for TDC, but it is based on multi-time-scale stochastic approximation.

For the VMTDC algorithm, a new one-step linear TD solution is defined as:
\begin{equation*}
    0=\mathbb{E}[({\phi} - \gamma {\phi}' - \mathbb{E}[{\phi} - \gamma {\phi}']){\phi}^\top]\mathbb{E}[{\phi} {\phi}^{\top}]^{-1}\mathbb{E}[(\delta -\mathbb{E}[\delta]){\phi}]=\textbf{A}^{\top}\textbf{C}^{-1}(-\textbf{A}{\theta}+{b}).
\end{equation*}
The matrix $\textbf{A}^{\top}\textbf{C}^{-1}\textbf{A}$ is positive definite. Thus, the  VMTD's solution is
${\theta}_{\text{VMTDC}}=\textbf{A}^{-1}{b}$.

First, note that recursion (\ref{thetavmtdc}) and (\ref{uvmtdc}) can be rewritten as, respectively, 
\begin{equation*}
 {\theta}_{k+1}\leftarrow {\theta}_k+\zeta_k {x}(k),
\end{equation*}
\begin{equation*}
 {u}_{k+1}\leftarrow {u}_k+\beta_k {y}(k),
\end{equation*}
where 
\begin{equation*}
 {x}(k)=\frac{\alpha_k}{\zeta_k}[(\delta_{k}- \omega_k) {\phi}_k - \gamma{\phi}'_{k}({\phi}^{\top}_k {u}_k)],
\end{equation*}
\begin{equation*}
 {y}(k)=\frac{\zeta_k}{\beta_k}[\delta_{k}-\omega_k - {\phi}^{\top}_k {u}_k]{\phi}_k.
\end{equation*}

Recursion (\ref{thetavmtdc}) can also be rewritten as
\begin{equation*}
 {\theta}_{k+1}\leftarrow {\theta}_k+\beta_k z(k),
\end{equation*}
where
\begin{equation*}
 z(k)=\frac{\alpha_k}{\beta_k}[(\delta_{k}- \omega_k) {\phi}_k - \gamma{\phi}'_{k}({\phi}^{\top}_k {u}_k)],
\end{equation*}

Due to the settings of the step-size schedule 
$\alpha_k = o(\zeta_k)$, $\zeta_k = o(\beta_k)$, ${x}(k)\rightarrow 0$, ${y}(k)\rightarrow 0$, $z(k)\rightarrow 0$ almost surely as $k\rightarrow 0$.
That is the increments in iteration (\ref{omegavmtdc}) are uniformly larger than
those in (\ref{uvmtdc}) and  the increments in iteration (\ref{uvmtdc}) are uniformly larger than
those in (\ref{thetavmtdc}), thus (\ref{omegavmtdc}) is the fastest recursion, (\ref{uvmtdc}) is the second fast recursion, and (\ref{thetavmtdc}) is the slower recursion.
Along the fastest time scale, iterations of (\ref{thetavmtdc}), (\ref{uvmtdc}) and (\ref{omegavmtdc})
are associated with the ODEs system as follows:
\begin{equation}
 \dot{{\theta}}(t) = 0,
    \label{thetavmtdcFastest}
\end{equation}
\begin{equation}
 \dot{{u}}(t) = 0,
    \label{uvmtdcFastest}
\end{equation}
\begin{equation}
 \dot{\omega}(t)=\mathbb{E}[\delta_t|{u}(t),{\theta}(t)]-\omega(t).
    \label{omegavmtdcFastest}
\end{equation}

Based on the ODE (\ref{thetavmtdcFastest}) and (\ref{uvmtdcFastest}), both ${\theta}(t)\equiv {\theta}$
and ${u}(t)\equiv {u}$ when viewed from the fastest timescale.
By the Hirsch lemma \cite{hirsch1989convergent}, it follows that
$||{\theta}_k-{\theta}||\rightarrow 0$ a.s. as $k\rightarrow \infty$ for some
${\theta}$ that depends on the initial condition ${\theta}_0$ of recursion
(\ref{thetavmtdc}) and $||{u}_k-{u}||\rightarrow 0$ a.s. as $k\rightarrow \infty$ for some
$u$ that depends on the initial condition $u_0$ of recursion
(\ref{uvmtdc}). Thus, the ODE pair (\ref{thetavmtdcFastest})-(ref{omegavmtdcFastest})
can be written as 
\begin{equation}
 \dot{\omega}(t)=\mathbb{E}[\delta_t|{u},{\theta}]-\omega(t).
    \label{omegavmtdcFastestFinal}
\end{equation}

Consider the function $h(\omega)=\mathbb{E}[\delta|{\theta},{u}]-\omega$,
i.e., the driving vector field of the ODE (\ref{omegavmtdcFastestFinal}).
It is easy to find that the function $h$ is Lipschitz with coefficient
$-1$.
Let $h_{\infty}(\cdot)$ be the function defined by
 $h_{\infty}(\omega)=\lim_{r\rightarrow \infty}\frac{h(r\omega)}{r}$.
 Then  $h_{\infty}(\omega)= -\omega$,  is well-defined. 
 For (\ref{omegavmtdcFastestFinal}), $\omega^*=\mathbb{E}[\delta|{\theta},{u}]$
is the unique globally asymptotically stable equilibrium.
For the ODE
\begin{equation}
 \dot{\omega}(t) = h_{\infty}(\omega(t))= -\omega(t),
 \label{omegavmtdcInfty}
\end{equation}
apply $\vec{V}(\omega)=(-\omega)^{\top}(-\omega)/2$ as its
associated strict Liapunov function. Then,
the origin of (\ref{omegavmtdcInfty}) is a globally asymptotically stable
equilibrium.

Consider now the recursion (\ref{omegavmtdc}).
Let
$M_{k+1}=(\delta_k-\omega_k)
-\mathbb{E}[(\delta_k-\omega_k)|\mathcal{F}(k)]$,
where $\mathcal{F}(k)=\sigma(\omega_l,{u}_l,{\theta}_l,l\leq k;{\phi}_s,{\phi}_s',r_s,s<k)$, 
$k\geq 1$ are the sigma fields
generated by $\omega_0,u_0,{\theta}_0,\omega_{l+1},{u}_{l+1},{\theta}_{l+1},{\phi}_l,{\phi}_l'$,
$0\leq l<k$.
It is easy to verify that $M_{k+1},k\geq0$ are integrable random variables that 
satisfy $\mathbb{E}[M_{k+1}|\mathcal{F}(k)]=0$, $\forall k\geq0$.
Because ${\phi}_k$, $r_k$, and ${\phi}_k'$ have
uniformly bounded second moments, it can be seen that for some constant
$c_1>0$, $\forall k\geq0$,
\begin{equation*}
\mathbb{E}[||M_{k+1}||^2|\mathcal{F}(k)]\leq
c_1(1+||\omega_k||^2+||{u}_k||^2+||{\theta}_k||^2).
\end{equation*}


Now Assumptions (A1) and (A2) of \cite{borkar2000ode} are verified.
Furthermore, Assumptions (TS) of \cite{borkar2000ode} is satisfied by our
conditions on the step-size sequences $\alpha_k$,$\zeta_k$, $\beta_k$. Thus,
by Theorem 2.2 of \cite{borkar2000ode} we obtain that
$||\omega_k-\omega^*||\rightarrow 0$ almost surely as $k\rightarrow \infty$.

Consider now the second time scale recursion (\ref{uvmtdc}).
Based on the above analysis, (\ref{uvmtdc}) can be rewritten as
% \begin{equation*}
%     {u}_{k+1}\leftarrow u_{k}+\zeta_{k}[\delta_{k}-\mathbb{E}[\delta_k|{u}_k,{\theta}_k] - {\phi}^{\top} (s_k) {u}_k]{\phi}(s_k).
% \end{equation*}
\begin{equation}
 \dot{{\theta}}(t) = 0,
    \label{thetavmtdcFaster}
\end{equation}
\begin{equation}
 \dot{u}(t) = \mathbb{E}[(\delta_t-\mathbb{E}[\delta_t|{u}(t),{\theta}(t)]){\phi}_t|{\theta}(t)] - \textbf{C}{u}(t).
    \label{uvmtdcFaster}
\end{equation}
The ODE (\ref{thetavmtdcFaster}) suggests that ${\theta}(t)\equiv {\theta}$ (i.e., a time-invariant parameter)
when viewed from the second fast timescale.
By the Hirsch lemma \cite{hirsch1989convergent}, it follows that
$||{\theta}_k-{\theta}||\rightarrow 0$ a.s. as $k\rightarrow \infty$ for some
${\theta}$ that depends on the initial condition ${\theta}_0$ of recursion
(\ref{thetavmtdc}). 

Consider now the recursion (\ref{uvmtdc}).
Let
$N_{k+1}=((\delta_k-\mathbb{E}[\delta_k]) - {\phi}_k {\phi}^{\top}_k {u}_k) -\mathbb{E}[((\delta_k-\mathbb{E}[\delta_k]) - {\phi}_k {\phi}^{\top}_k {u}_k)|\mathcal{I} (k)]$,
where $\mathcal{I}(k)=\sigma({u}_l,{\theta}_l,l\leq k;{\phi}_s,{\phi}_s',r_s,s<k)$, 
$k\geq 1$ are the sigma fields
generated by ${u}_0,{\theta}_0,{u}_{l+1},{\theta}_{l+1},{\phi}_l,{\phi}_l'$,
$0\leq l<k$.
It is easy to verify that $N_{k+1},k\geq0$ are integrable random variables that 
satisfy $\mathbb{E}[N_{k+1}|\mathcal{I}(k)]=0$, $\forall k\geq0$.
Because ${\phi}_k$, $r_k$, and ${\phi}_k'$ have
uniformly bounded second moments, it can be seen that for some constant
$c_2>0$, $\forall k\geq0$,
\begin{equation*}
\mathbb{E}[||N_{k+1}||^2|\mathcal{I}(k)]\leq
c_2(1+||{u}_k||^2+||{\theta}_k||^2).
\end{equation*}

Because ${\theta}(t)\equiv {\theta}$ from (\ref{thetavmtdcFaster}), the ODE pair (\ref{thetavmtdcFaster})-(\ref{uvmtdcFaster})
can be written as 
\begin{equation}
 \dot{{u}}(t) = \mathbb{E}[(\delta_t-\mathbb{E}[\delta_t|{\theta}]){\phi}_t|{\theta}] - \textbf{C}{u}(t).
    \label{uvmtdcFasterFinal}
\end{equation}
Now consider the function $h({u})=\mathbb{E}[\delta_t-\mathbb{E}[\delta_t|{\theta}]|{\theta}] -\textbf{C}{u}$, i.e., the
driving vector field of the ODE (\ref{uvmtdcFasterFinal}). For (\ref{uvmtdcFasterFinal}),
${u}^* = \textbf{C}^{-1}\mathbb{E}[(\delta-\mathbb{E}[\delta|{\theta}]){\phi}|{\theta}]$ is the unique globally asymptotically
stable equilibrium. Let $h_{\infty}({u})=-\textbf{C}{u}$.
For the ODE
\begin{equation}
 \dot{{u}}(t) = h_{\infty}({u}(t))= -\textbf{C}{u}(t),
    \label{uvmtdcInfty}
\end{equation}
the origin of (\ref{uvmtdcInfty}) is a globally asymptotically stable
equilibrium because $\textbf{C}$ is a positive definite matrix (because it is nonnegative definite and nonsingular).
Now Assumptions (A1) and (A2) of \cite{borkar2000ode} are verified.
Furthermore, Assumptions (TS) of \cite{borkar2000ode} is satisfied by our
conditions on the step-size sequences $\alpha_k$,$\zeta_k$, $\beta_k$. Thus,
by Theorem 2.2 of \cite{borkar2000ode} we obtain that
$||{u}_k-{u}^*||\rightarrow 0$ almost surely as $k\rightarrow \infty$.

Consider now the slower timescale recursion (\ref{thetavmtdc}). In the light of the above,
(\ref{thetavmtdc}) can be rewritten as 
\begin{equation}
 {\theta}_{k+1} \leftarrow {\theta}_{k} + \alpha_k (\delta_k -\mathbb{E}[\delta_k|{\theta}_k]) {\phi}_k\\
 - \alpha_k \gamma{\phi}'_{k}({\phi}^{\top}_k \textbf{C}^{-1}\mathbb{E}[(\delta_k -\mathbb{E}[\delta_k|{\theta}_k]){\phi}|{\theta}_k]).
\end{equation}
Let $\mathcal{G}(k)=\sigma({\theta}_l,l\leq k;{\phi}_s,{\phi}_s',r_s,s<k)$, 
$k\geq 1$ be the sigma fields
generated by ${\theta}_0,{\theta}_{l+1},{\phi}_l,{\phi}_l'$,
$0\leq l<k$. Let
\begin{equation*}
    \begin{array}{ccl}
 Z_{k+1}&=&((\delta_k -\mathbb{E}[\delta_k|{\theta}_k]) {\phi}_k - \gamma {\phi}'_{k}{\phi}^{\top}_k \textbf{C}^{-1}\mathbb{E}[(\delta_k -\mathbb{E}[\delta_k|{\theta}_k]){\phi}|{\theta}_k])\\ 
     & &-\mathbb{E}[((\delta_k -\mathbb{E}[\delta_k|{\theta}_k]) {\phi}_k - \gamma {\phi}'_{k}{\phi}^{\top}_k \textbf{C}^{-1}\mathbb{E}[(\delta_k -\mathbb{E}[\delta_k|{\theta}_k]){\phi}|{\theta}_k])|\mathcal{G}(k)]\\
    &=&((\delta_k -\mathbb{E}[\delta_k|{\theta}_k]) {\phi}_k - \gamma {\phi}'_{k}{\phi}^{\top}_k \textbf{C}^{-1}\mathbb{E}[(\delta_k -\mathbb{E}[\delta_k|{\theta}_k]){\phi}|{\theta}_k])\\
    & &-\mathbb{E}[(\delta_k -\mathbb{E}[\delta_k|{\theta}_k]) {\phi}_k|{\theta}_k] - \gamma\mathbb{E}[{\phi}' {\phi}^{\top}]\textbf{C}^{-1}\mathbb{E}[(\delta_k -\mathbb{E}[\delta_k|{\theta}_k]) {\phi}_k|{\theta}_k].
    \end{array}
\end{equation*}
It is easy to see that $Z_k$, $k\geq 0$ are integrable random variables and $\mathbb{E}[Z_{k+1}|\mathcal{G}(k)]=0$, $\forall k\geq0$. Further,
\begin{equation*}
\mathbb{E}[||Z_{k+1}||^2|\mathcal{G}(k)]\leq
c_3(1+||{\theta}_k||^2), k\geq 0
\end{equation*}
for some constant $c_3 \geq 0$, again because ${\phi}_k$, $r_k$, and ${\phi}_k'$ have
uniformly bounded second moments, it can be seen that for some constant.

Consider now the following ODE associated with (\ref{thetavmtdc}):
\begin{equation}
 \dot{{\theta}}(t) = (\textbf{I} - \mathbb{E}[\gamma {\phi}' {\phi}^{\top}]\textbf{C}^{-1})\mathbb{E}[(\delta -\mathbb{E}[\delta|{\theta}(t)]) {\phi}|{\theta}(t)].
    \label{thetavmtdcSlowerFinal}
\end{equation}
Let 
\begin{equation*}
\begin{array}{ccl}
 \vec{h}({\theta}(t))&=&(\textbf{I} - \mathbb{E}[\gamma {\phi}' {\phi}^{\top}]\textbf{C}^{-1})\mathbb{E}[(\delta -\mathbb{E}[\delta|{\theta}(t)]) {\phi}|{\theta}(t)]\\
    &=&(\textbf{C} - \mathbb{E}[\gamma {\phi}' {\phi}^{\top}])\textbf{C}^{-1}\mathbb{E}[(\delta -\mathbb{E}[\delta|{\theta}(t)]) {\phi}|{\theta}(t)]\\
    &=& (\mathbb{E}[{\phi} {\phi}^{\top}] - \mathbb{E}[\gamma {\phi}' {\phi}^{\top}])\textbf{C}^{-1}\mathbb{E}[(\delta -\mathbb{E}[\delta|{\theta}(t)]) {\phi}|{\theta}(t)]\\
    &=& \textbf{A}^{\top}\textbf{C}^{-1}(-\textbf{A}{\theta}(t)+{b}),
\end{array}
\end{equation*}
because $\mathbb{E}[(\delta -\mathbb{E}[\delta|{\theta}(t)]) {\phi}|{\theta}(t)]=-\textbf{A}{\theta}(t)+{b}$, where 
$\textbf{A} = \mathrm{Cov}({\phi},{\phi}-\gamma{\phi}')$, ${b}=\mathrm{Cov}(r,{\phi})$, and $\textbf{C}=\mathbb{E}[{\phi}{\phi}^{\top}]$

Therefore,
${\theta}^*=\textbf{A}^{-1}{b}$ can be seen to be the unique globally asymptotically
stable equilibrium for ODE (\ref{thetavmtdcSlowerFinal}).
Let $\vec{h}_{\infty}({\theta})=\lim_{r\rightarrow
\infty}\frac{\vec{h}(r{\theta})}{r}$. Then
$\vec{h}_{\infty}({\theta})=-\textbf{A}^{\top}\textbf{C}^{-1}\textbf{A}{\theta}$ is well-defined. 
Consider now the ODE
\begin{equation}
\dot{{\theta}}(t)=-\textbf{A}^{\top}\textbf{C}^{-1}\textbf{A}{\theta}(t).
\label{odethetavmtdcfinal}
\end{equation}

Because $\textbf{C}^{-1}$ is positive definite and $\textbf{A}$ has full rank (as it
is nonsingular by assumption), the matrix $\textbf{A}^{\top} \textbf{C}^{-1}\textbf{A}$ is also
positive definite. 
The ODE (\ref{odethetavmtdcfinal}) has the origin of its unique globally asymptotically stable equilibrium.
Thus, the assumption (A1) and (A2) are verified.

The proof is given above.
In the fastest time scale, the parameter $w$ converges to
$\mathbb{E}[\delta|{u}_k,{\theta}_k]$.
In the second fast time scale,
the parameter $u$ converges to $\textbf{C}^{-1}\mathbb{E}[(\delta-\mathbb{E}[\delta|{\theta}_k]){\phi}|{\theta}_k]$.
In the slower time scale,
the parameter ${\theta}$ converges to $\textbf{A}^{-1}{b}$.
\end{proof}

\subsection{Proof of Theorem 4}
\label{proofVMETD}
\begin{proof}
\label{th4proof}   
The proof of VMETD's convergence is also based on Borkar's Theorem   for
general stochastic approximation recursions with two time scales
\cite{borkar1997stochastic}. 

The  VMTD's solution is
${\theta}_{\text{VMETD}}=\textbf{A}_{\text{VMETD}}^{-1}{b}_{\text{VMETD}}$.
First, note that recursion (\ref{thetavmetd}) can be rewritten as
\begin{equation*}
 {\theta}_{k+1}\leftarrow {\theta}_k+\beta_k{\xi}(k),
\end{equation*}
 where
\begin{equation*}
 {\xi}(k)=\frac{\alpha_k}{\beta_k} (F_k \rho_k\delta_k - \omega_{k+1}){\phi}_k
\end{equation*}
 Due to the settings of step-size schedule $\alpha_k = o(\beta_k)$,
${\xi}(k)\rightarrow 0$ almost surely as $k\rightarrow\infty$. 
 That is the increments in iteration (\ref{omegavmetd}) are uniformly larger than
 those in (\ref{thetavmetd}), thus (\ref{omegavmetd}) is the faster recursion.
 Along the faster time scale, iterations of (\ref{thetavmetd}) and (\ref{omegavmetd})
 are associated with the ODEs system as follows:
\begin{equation}
 \dot{{\theta}}(t) = 0,
\label{vmetdthetaFast}
\end{equation}
\begin{equation}
 \dot{\omega}(t)=\mathbb{E}_{\mu}[F_t\rho_t\delta_t|{\theta}(t)]-\omega(t).
\label{vmetdomegaFast}
\end{equation}
 Based on the ODE (\ref{vmetdthetaFast}), ${\theta}(t)\equiv {\theta}$ when
 viewed from the faster timescale. 
 By the Hirsch lemma \cite{hirsch1989convergent}, it follows that
$||{\theta}_k-{\theta}||\rightarrow 0$ a.s. as $k\rightarrow \infty$ for some
${\theta}$ that depends on the initial condition ${\theta}_0$ of recursion
(\ref{thetavmetd}).
 Thus, the ODE pair (\ref{vmetdthetaFast})-(\ref{vmetdomegaFast}) can be written as
\begin{equation}
 \dot{\omega}(t)=\mathbb{E}_{\mu}[F_t\rho_t\delta_t|{\theta}]-\omega(t).
\label{vmetdomegaFastFinal}
\end{equation}
 Consider the function $h(\omega)=\mathbb{E}_{\mu}[F\rho\delta|{\theta}]-\omega$,
 i.e., the driving vector field of the ODE (\ref{vmetdomegaFastFinal}).
 It is easy to find that the function $h$ is Lipschitz with coefficient
$-1$.
 Let $h_{\infty}(\cdot)$ be the function defined by
 $h_{\infty}(\omega)=\lim_{x\rightarrow \infty}\frac{h(x\omega)}{x}$.
 Then  $h_{\infty}(\omega)= -\omega$,  is well-defined. 
 For (\ref{vmetdomegaFastFinal}), $\omega^*=\mathbb{E}_{\mu}[F\rho\delta|{\theta}]$
 is the unique globally asymptotically stable equilibrium.
 For the ODE
  \begin{equation}
 \dot{\omega}(t) = h_{\infty}(\omega(t))= -\omega(t),
 \label{vmetdomegaInfty}
 \end{equation}
 apply $\vec{V}(\omega)=(-\omega)^{\top}(-\omega)/2$ as its
 associated strict Liapunov function. Then,
 the origin of (\ref{vmetdomegaInfty}) is a globally asymptotically stable
 equilibrium.


 Consider now the recursion (\ref{omegavmetd}).
 Let
$M_{k+1}=(F_k\rho_k\delta_k-\omega_k)
 -\mathbb{E}_{\mu}[(F_k\rho_k\delta_k-\omega_k)|\mathcal{F}(k)]$,
 where $\mathcal{F}(k)=\sigma(\omega_l,{\theta}_l,l\leq k;{\phi}_s,{\phi}_s',r_s,s<k)$, 
$k\geq 1$ are the sigma fields
 generated by $\omega_0,{\theta}_0,\omega_{l+1},{\theta}_{l+1},{\phi}_l,{\phi}_l'$,
$0\leq l<k$.
 It is easy to verify that $M_{k+1},k\geq0$ are integrable random variables that 
 satisfy $\mathbb{E}[M_{k+1}|\mathcal{F}(k)]=0$, $\forall k\geq0$.
 Because ${\phi}_k$, $r_k$, and ${\phi}_k'$ have
 uniformly bounded second moments, it can be seen that for some constant
$c_1>0$, $\forall k\geq0$,
\begin{equation*}
 \mathbb{E}[||M_{k+1}||^2|\mathcal{F}(k)]\leq
 c_1(1+||\omega_k||^2+||{\theta}_k||^2).
\end{equation*}


 Now Assumptions (A1) and (A2) of \cite{borkar2000ode} are verified.
 Furthermore, Assumptions (TS) of \cite{borkar2000ode} are satisfied by our
 conditions on the step-size sequences $\alpha_k$, $\beta_k$. Thus,
 by Theorem 2.2 of \cite{borkar2000ode} we obtain that
$||\omega_k-\omega^*||\rightarrow 0$ almost surely as $k\rightarrow \infty$.

 Consider now the slower time scale recursion (\ref{thetavmetd}).
 Based on the above analysis, (\ref{thetavmetd}) can be rewritten as 

\begin{equation*}
    \begin{split}
 {\theta}_{k+1}&\leftarrow {\theta}_k+\alpha_k (F_k \rho_k\delta_k - \omega_k){\phi}_k -\alpha_k \omega_{k+1}{\phi}_k\\
&={\theta}_{k}+\alpha_k(F_k\rho_k\delta_k-\mathbb{E}_{\mu}[F_k\rho_k\delta_k|{\theta}_k]){\phi}_k\\
    &={\theta}_k+\alpha_k F_k \rho_k (R_{k+1}+\gamma {\theta}_k^{\top}{\phi}_{k+1}-{\theta}_k^{\top}{\phi}_k){\phi}_k -\alpha_k \mathbb{E}_{\mu}[F_k \rho_k \delta_k]{\phi}_k\\
    &= {\theta}_k+\alpha_k \{\underbrace{(F_k\rho_kR_{k+1}-\mathbb{E}_{\mu}[F_k\rho_k R_{k+1}]){\phi}_k}_{{b}_{\text{VMETD},k}}
 -\underbrace{(F_k\rho_k{\phi}_k({\phi}_k-\gamma{\phi}_{k+1})^{\top}-{\phi}_k\mathbb{E}_{\mu}[F_k\rho_k ({\phi}_k-\gamma{\phi}_{k+1})]^{\top})}_{\textbf{A}_{\text{VMETD},k}}{\theta}_k\}
\end{split}
\end{equation*}

 Let $\mathcal{G}(k)=\sigma({\theta}_l,l\leq k;{\phi}_s,{\phi}_s',r_s,s<k)$, 
$k\geq 1$ be the sigma fields
 generated by ${\theta}_0,{\theta}_{l+1},{\phi}_l,{\phi}_l'$,
$0\leq l<k$.
 Let 
$
 Z_{k+1} = Y_{k}-\mathbb{E}[Y_{k}|\mathcal{G}(k)],
$
 where
\begin{equation*}
 Y_{k}=(F_k\rho_k\delta_k-\mathbb{E}_{\mu}[F_k\rho_k\delta_k|{\theta}_k]){\phi}_k.
\end{equation*}
 Consequently,
\begin{equation*}
\begin{array}{ccl}
 \mathbb{E}_{\mu}[Y_k|\mathcal{G}(k)]&=&\mathbb{E}_{\mu}[(F_k\rho_k\delta_k-\mathbb{E}_{\mu}[F_k\rho_k\delta_k|{\theta}_k]){\phi}_k|\mathcal{G}(k)]\\
&=&\mathbb{E}_{\mu}[F_k\rho_k\delta_k{\phi}_k|{\theta}_k]
 -\mathbb{E}_{\mu}[\mathbb{E}_{\mu}[F_k\rho_k\delta_k|{\theta}_k]{\phi}_k]\\
&=&\mathbb{E}_{\mu}[F_k\rho_k\delta_k{\phi}_k|{\theta}_k]
 -\mathbb{E}_{\mu}[F_k\rho_k\delta_k|{\theta}_k]\mathbb{E}_{\mu}[{\phi}_k]\\
&=&\mathrm{Cov}(F_k\rho_k\delta_k|{\theta}_k,{\phi}_k),
\end{array}
\end{equation*}
 where $\mathrm{Cov}(\cdot,\cdot)$ is a covariance operator.

 Thus,
 \begin{equation*}
\begin{array}{ccl}
 Z_{k+1}&=&(F_k\rho_k\delta_k-\mathbb{E}[\delta_k|{\theta}_k]){\phi}_k-\mathrm{Cov}(F_k\rho_k\delta_k|{\theta}_k,{\phi}_k).
\end{array}
\end{equation*}
 It is easy to verify that $Z_{k+1},k\geq0$ are integrable random variables that 
 satisfy $\mathbb{E}[Z_{k+1}|\mathcal{G}(k)]=0$, $\forall k\geq0$.
 Also, because ${\phi}_k$, $r_k$, and ${\phi}_k'$ have
 uniformly bounded second moments, it can be seen that for some constant
$c_2>0$, $\forall k\geq0$,
\begin{equation*}
 \mathbb{E}[||Z_{k+1}||^2|\mathcal{G}(k)]\leq
 c_2(1+||{\theta}_k||^2).
\end{equation*}

 Consider now the following ODE associated with (\ref{thetavmetd}):
\begin{equation}
\begin{array}{ccl}
 \dot{{\theta}}(t)&=&-\textbf{A}_{\text{VMETD}}{\theta}(t)+{b}_{\text{VMETD}}.
\end{array}
\label{odethetavmetd}
\end{equation}
\begin{equation}
    \begin{split}
 \textbf{A}_{\text{VMETD}}&=\lim_{k \rightarrow \infty} \mathbb{E}[\textbf{A}_{\text{VMETD},k}]\\
&= \lim_{k \rightarrow \infty} \mathbb{E}_{\mu}[F_k \rho_k {\phi}_k ({\phi}_k - \gamma {\phi}_{k+1})^{\top}]- \lim_{k\rightarrow \infty} \mathbb{E}_{\mu}[  {\phi}_k]\mathbb{E}_{\mu}[F_k \rho_k ({\phi}_k - \gamma {\phi}_{k+1})]^{\top}\\  
% &= \lim_{k \rightarrow \infty} \mathbb{E}_{\mu}[\underbrace{{\phi}_k}_{X}\underbrace{F_k \rho_k  ({\phi}_k - \gamma {\phi}_{k+1})^{\top}}_{Y}]- \lim_{k\rightarrow \infty} \mathbb{E}_{\mu}[  {\phi}_k]\mathbb{E}_{\mu}[F_k \rho_k ({\phi}_k - \gamma {\phi}_{k+1})]^{\top}\\  
&= \lim_{k \rightarrow \infty} \mathbb{E}_{\mu}[{\phi}_kF_k \rho_k  ({\phi}_k - \gamma {\phi}_{k+1})^{\top}]- \lim_{k\rightarrow \infty} \mathbb{E}_{\mu}[  {\phi}_k]\mathbb{E}_{\mu}[F_k \rho_k ({\phi}_k - \gamma {\phi}_{k+1})]^{\top}\\ 
&= \lim_{k \rightarrow \infty} \mathbb{E}_{\mu}[{\phi}_kF_k \rho_k ({\phi}_k - \gamma {\phi}_{k+1})^{\top}]- \lim_{k \rightarrow \infty} \mathbb{E}_{\mu}[ {\phi}_k]\lim_{k \rightarrow \infty}\mathbb{E}_{\mu}[F_k \rho_k ({\phi}_k - \gamma {\phi}_{k+1})]^{\top}\\   
&=\sum_{s} f(s) {\phi}(s)({\phi}(s) - \gamma \sum_{s'}[\textbf{P}_{\pi}]_{ss'}{\phi}(s'))^{\top} - \sum_{s} d_{\mu}(s) {\phi}(s) * \sum_{s} f(s)({\phi}(s) - \gamma \sum_{s'}[\textbf{P}_{\pi}]_{ss'}{\phi}(s'))^{\top}  \\
&={{\Phi}}^{\top} \textbf{F} (\textbf{I} - \gamma \textbf{P}_{\pi}) {\Phi} - {{\Phi}}^{\top} \textbf{d}_{\mu} \textbf{f}^{\top} (\textbf{I} - \gamma \textbf{P}_{\mu}) {\Phi}  \\
&={{\Phi}}^{\top} (\textbf{F} - \textbf{d}_{\mu} \textbf{f}^{\top}) (\textbf{I} - \gamma \textbf{P}_{\pi}){{\Phi}} \\
&={{\Phi}}^{\top} (\textbf{F} (\textbf{I} - \gamma \textbf{P}_{\pi})-\textbf{d}_{\mu} \textbf{f}^{\top} (\textbf{I} - \gamma \textbf{P}_{\pi})){{\Phi}} \\
&={{\Phi}}^{\top} (\textbf{F} (\textbf{I} - \gamma \textbf{P}_{\pi})-\textbf{d}_{\mu} \textbf{d}_{\mu}^{\top} ){{\Phi}} \\
    \end{split}
\end{equation}
\begin{equation}
    \begin{split}
 {b}_{\text{VMETD}}&=\lim_{k \rightarrow \infty} \mathbb{E}[{b}_{\text{VMETD},k}]\\
&= \lim_{k \rightarrow \infty} \mathbb{E}_{\mu}[F_k\rho_kR_{k+1}{\phi}_k]- \lim_{k\rightarrow \infty} \mathbb{E}_{\mu}[{\phi}_k]\mathbb{E}_{\mu}[F_k\rho_kR_{k+1}]\\  
&= \lim_{k \rightarrow \infty} \mathbb{E}_{\mu}[{\phi}_kF_k\rho_kR_{k+1}]- \lim_{k\rightarrow \infty} \mathbb{E}_{\mu}[  {\phi}_k]\mathbb{E}_{\mu}[{\phi}_k]\mathbb{E}_{\mu}[F_k\rho_kR_{k+1}]\\ 
&= \lim_{k \rightarrow \infty} \mathbb{E}_{\mu}[{\phi}_kF_k\rho_kR_{k+1}]- \lim_{k \rightarrow \infty} \mathbb{E}_{\mu}[ {\phi}_k]\lim_{k \rightarrow \infty}\mathbb{E}_{\mu}[F_k\rho_kR_{k+1}]\\  
&=\sum_{s} f(s) {\phi}(s)r_{\pi} - \sum_{s} d_{\mu}(s) {\phi}(s) * \sum_{s} f(s)r_{\pi}  \\
&={{\Phi}}^{\top}(\textbf{F}-\textbf{d}_{\mu} \textbf{f}^{\top})\textbf{r}_{\pi} \\
    \end{split}
\end{equation}
 Let $\vec{h}({\theta}(t))$ be the driving vector field of the ODE
 (\ref{odethetavmetd}).
\begin{equation*}
 \vec{h}({\theta}(t))=-\textbf{A}_{\text{VMETD}}{\theta}(t)+{b}_{\text{VMETD}}.
\end{equation*}

 An ${\Phi}^{\top}{\text{X}}{\Phi}$ matrix of this
 form will be positive definite whenever the matrix ${\text{X}}$ is positive definite.
 Any matrix ${\text{X}}$ is positive definite if and only if
 the symmetric matrix ${\text{S}}={\text{X}}+{\text{X}}^{\top}$ is positive definite. 
 Any symmetric real matrix ${\text{S}}$ is positive definite if the absolute values of
 its diagonal entries are greater than the sum of the absolute values of the corresponding
 off-diagonal entries\cite{sutton2016emphatic}. 

\begin{equation}
    \label{rowsum}
    \begin{split}
 (\textbf{F} (\textbf{I} - \gamma \textbf{P}_{\pi})-\textbf{d}_{\mu} \textbf{d}_{\mu}^{\top} )\textbf{1}
    &=\textbf{F} (\textbf{I} - \gamma \textbf{P}_{\pi})\textbf{1}-\textbf{d}_{\mu} \textbf{d}_{\mu}^{\top} \textbf{1}\\
    &=\textbf{F}(\textbf{1}-\gamma \textbf{P}_{\pi} \textbf{1})-\textbf{d}_{\mu} \textbf{d}_{\mu}^{\top} \textbf{1}\\
    &=(1-\gamma)\textbf{F}\textbf{1}-\textbf{d}_{\mu} \textbf{d}_{\mu}^{\top} \textbf{1}\\
    &=(1-\gamma)\textbf{f}-\textbf{d}_{\mu} \textbf{d}_{\mu}^{\top} \textbf{1}\\
    &=(1-\gamma)\textbf{f}-\textbf{d}_{\mu} \\
    &=(1-\gamma)(\textbf{I}-\gamma\textbf{P}_{\pi}^{\top})^{-1}\textbf{d}_{\mu}-\textbf{d}_{\mu} \\
    &=(1-\gamma)[(\textbf{I}-\gamma\textbf{P}_{\pi}^{\top})^{-1}-\textbf{I}]\textbf{d}_{\mu} \\
    &=(1-\gamma)[\sum_{t=0}^{\infty}(\gamma\textbf{P}_{\pi}^{\top})^{t}-\textbf{I}]\textbf{d}_{\mu} \\
    &=(1-\gamma)[\sum_{t=1}^{\infty}(\gamma\textbf{P}_{\pi}^{\top})^{t}]\textbf{d}_{\mu} > 0 \\
    \end{split}
    \end{equation}
\begin{equation}
    \label{columnsum}
    \begin{split}
 \textbf{1}^{\top}(\textbf{F} (\textbf{I} - \gamma \textbf{P}_{\pi})-\textbf{d}_{\mu} \textbf{d}_{\mu}^{\top} )
    &=\textbf{1}^{\top}\textbf{F} (\textbf{I} - \gamma \textbf{P}_{\pi})-\textbf{1}^{\top}\textbf{d}_{\mu} \textbf{d}_{\mu}^{\top} \\
    &=\textbf{d}_{\mu}^{\top}-\textbf{1}^{\top}\textbf{d}_{\mu} \textbf{d}_{\mu}^{\top} \\
    &=\textbf{d}_{\mu}^{\top}- \textbf{d}_{\mu}^{\top} \\
    &=0
    \end{split}
\end{equation}
 (\ref{rowsum}) and (\ref{columnsum}) show that the matrix $\textbf{F} (\textbf{I} - \gamma \textbf{P}_{\pi})-\textbf{d}_{\mu} \textbf{d}_{\mu}^{\top}$ of
 diagonal entries are positive and its off-diagonal entries are negative. So each row sum plus the corresponding column sum is positive. 
 So $\textbf{A}_{\text{VMETD}}$ is positive definite.



 Therefore,
${\theta}^*=\textbf{A}_{\text{VMETD}}^{-1}{b}_{\text{VMETD}}$ can be seen to be the unique globally asymptotically
 stable equilibrium for ODE (\ref{odethetavmetd}).
 Let $\vec{h}_{\infty}({\theta})=\lim_{r\rightarrow
\infty}\frac{\vec{h}(r{\theta})}{r}$. Then
$\vec{h}_{\infty}({\theta})=-\textbf{A}_{\text{VMETD}}{\theta}$ is well-defined. 
 Consider now the ODE
\begin{equation}
 \dot{{\theta}}(t)=-\textbf{A}_{\text{VMETD}}{\theta}(t).
\label{odethetavmetdfinal}
\end{equation}
 The ODE (\ref{odethetavmetdfinal}) has the origin of its unique globally asymptotically stable equilibrium.
 Thus, the assumption (A1) and (A2) are verified.
    \end{proof}



\section{Experimental details}
\label{experimentaldetails}
The 2-state version of Baird's off-policy counterexample: All learning rates follow linear learning rate decay.
For TD algorithm, $\frac{\alpha_k}{\omega_k}=4$ and $\alpha_0 = 0.1$.
For TDC algorithm, $\frac{\alpha_k}{\zeta_k}=5$ and $\alpha_0 = 0.1$.
For VMTDC algorithm, $\frac{\alpha_k}{\zeta_k}=5$, $\frac{\alpha_k}{\omega_k}=4$,and $\alpha_0 = 0.1$.
For VMTD algorithm, $\frac{\alpha_k}{\omega_k}=4$ and $\alpha_0 = 0.1$.

The 2-state version of Baird's off-policy counterexample: All learning rates follow linear learning rate decay.
For TD algorithm, $\frac{\alpha_k}{\omega_k}=4$ and $\alpha_0 = 0.1$.
For TDC algorithm, $\frac{\alpha_k}{\zeta_k}=5$ and $\alpha_0 = 0.1$.For ETD algorithm, $\alpha_0 = 0.1$.
For VMTDC algorithm, $\frac{\alpha_k}{\zeta_k}=5$, $\frac{\alpha_k}{\omega_k}=4$,and $\alpha_0 = 0.1$.For VMETD algorithm, $\frac{\alpha_k}{\omega_k}=4$ and $\alpha_0 = 0.1$.
For VMTD algorithm, $\frac{\alpha_k}{\omega_k}=4$ and $\alpha_0 = 0.1$.

For all policy evaluation experiments, each experiment 
is independently run 100 times.

For the four control experiments: The learning rates for each 
algorithm in all experiments are shown in Table \ref{lrofways}.
For all control experiments, each experiment is independently run 50 times.

\begin{table*}[htb]
    \centering
    \caption{Learning rates ($lr$) of four control experiments.}
    \label{lrofways}
    \begin{tabular}{ccccc}
      \toprule
      \multicolumn{1}{c}{Algorithms ($lr$)} & Maze & Cliff walking & Mountain Car & Acrobot \\
      \midrule
      Sarsa($\alpha$) & 0.1 & 0.1 & 0.1 & 0.1 \\
      GQ($\alpha,\zeta$) & 0.1, 0.003 & 0.1, 0.004 & 0.1, 0.01 & 0.1, 0.01 \\
      EQ($\alpha$) & 0.006 & 0.005 & 0.001 & 0.0005 \\
      VMSarsa($\alpha,\beta$) & 0.1, 0.001 & 0.1, 1e-4 & 0.1, 1e-4 & 0.1, 1e-4 \\
      VMGQ($\alpha,\zeta,\beta$) & 0.1, 0.001, 0.001 & 0.1, 0.005, 1e-4 & 0.1, 5e-4, 1e-4 & 0.1, 5e-4, 1e-4 \\
      VMEQ($\alpha,\beta$) & 0.001, 0.0005 & 0.005, 0.0001 & 0.001, 0.0001 & 0.0005, 0.0001 \\
      Q-learning($\alpha$) & 0.1 & 0.1 & 0.1 & 0.1 \\
      VMQ($\alpha,\beta$) & 0.1, 0.001 & 0.1, 1e-4 & 0.1, 1e-4 & 0.1, 1e-4 \\
      \bottomrule
    \end{tabular}
  \end{table*}
% \end{document}


% % As a general rule, do not put math, special symbols or citations
% % in the abstract
\begin{abstract}
[Background] Systematic literature reviews (SLRs) are essential for synthesizing evidence in Software Engineering (SE), but keeping them up-to-date requires substantial effort. Study selection, one of the most labor-intensive steps, involves reviewing numerous studies and requires multiple reviewers to minimize bias and avoid loss of evidence.
[Objective] This study aims to evaluate if Machine Learning (ML) text classification models can support reviewers in the study selection for SLR updates.
[Method] We reproduce the study selection of an SLR update performed by three SE researchers. We trained two supervised ML models (Random Forest and Support Vector Machines) with different configurations using data from the original SLR. We calculated the study selection effectiveness of the ML models for the SLR update in terms of precision, recall, and F-measure. We also compared the performance of human-ML pairs with human-only pairs when selecting studies.
[Results] The ML models achieved a modest F-score of 0.33, which is insufficient for reliable automation. However, we found that such models can reduce the study selection effort by 33.9\% without loss of evidence (keeping a 100\% recall). Our analysis also showed that the initial screening by pairs of human reviewers produces results that are much better aligned with the final SLR update result. [Conclusion] Based on our results, we conclude that although ML models can help reduce the effort involved in SLR updates, achieving rigorous and reliable outcomes still requires the expertise of experienced human reviewers for the initial screening phase.


\end{abstract}

% no keywords
\begin{IEEEkeywords}
Systematic Review Automation, Selection of Studies, Machine Learning, Systematic Literature Review Update
\end{IEEEkeywords}
% \keywords{Systematic Review Automation, Selection of Studies, Machine Learning}

% \received{20 February 2007}
% \received[revised]{12 March 2009}
% \received[accepted]{5 June 2009}


% humans are sensitive to the way information is presented.

% introduce framing as the way we address framing. say something about political views and how information is represented.

% in this paper we explore if models show similar sensitivity.

% why is it important/interesting.



% thought - it would be interesting to test it on real world data, but it would be hard to test humans because they come already biased about real world stuff, so we tested artificial.


% LLMs have recently been shown to mimic cognitive biases, typically associated with human behavior~\citep{ malberg2024comprehensive, itzhak-etal-2024-instructed}. This resemblance has significant implications for how we perceive these models and what we can expect from them in real-world interactions and decisionmaking~\citep{eigner2024determinants, echterhoff-etal-2024-cognitive}.

The \textit{framing effect} is a well-known cognitive phenomenon, where different presentations of the same underlying facts affect human perception towards them~\citep{tversky1981framing}.
For example, presenting an economic policy as only creating 50,000 new jobs, versus also reporting that it would cost 2B USD, can dramatically shift public opinion~\cite{sniderman2004structure}. 
%%%%%%%% 图1:  %%%%%%%%%%%%%%%%
\begin{figure}[t]
    \centering
    \includegraphics[width=\columnwidth]{Figs/01.pdf}
    \caption{Performance comparison (Top-1 Acc (\%)) under various open-vocabulary evaluation settings where the video learners except for CLIP are tuned on Kinetics-400~\cite{k400} with frozen text encoders. The satisfying in-context generalizability on UCF101~\cite{UCF101} (a) can be severely affected by static bias when evaluating on out-of-context SCUBA-UCF101~\cite{li2023mitigating} (b) by replacing the video background with other images.}
    \label{fig:teaser}
\end{figure}


Previous research has shown that LLMs exhibit various cognitive biases, including the framing effect~\cite{lore2024strategic,shaikh2024cbeval,malberg2024comprehensive,echterhoff-etal-2024-cognitive}. However, these either rely on synthetic datasets or evaluate LLMs on different data from what humans were tested on. In addition, comparisons between models and humans typically treat human performance as a baseline rather than comparing patterns in human behavior. 
% \gabis{looks good! what do we mean by ``most studies'' or ``rarely'' can we remove those? or we want to say that we don't know of previous work doing both at the same time?}\gili{yeah the main point is that some work has done each separated, but not all of it together. how about now?}

In this work, we evaluate LLMs on real-world data. Rather than measuring model performance in terms of accuracy, we analyze how closely their responses align with human annotations. Furthermore, while previous studies have examined the effect of framing on decision making, we extend this analysis to sentiment analysis, as sentiment perception plays a key explanatory role in decision-making \cite{lerner2015emotion}. 
%Based on this, we argue that examining sentiment shifts in response to reframing can provide deeper insights into the framing effect. \gabis{I don't understand this last claim. Maybe remove and just say we extend to sentiment analysis?}

% Understanding how LLMs respond to framing is crucial, as they are increasingly integrated into real-world applications~\citep{gan2024application, hurlin2024fairness}.
% In some applications, e.g., in virtual companions, framing can be harnessed to produce human-like behavior leading to better engagement.
% In contrast, in other applications, such as financial or legal advice, mitigating the effect of framing can lead to less biased decisions.
% In both cases, a better understanding of the framing effect on LLMs can help develop strategies to mitigate its negative impacts,
% while utilizing its positive aspects. \gabis{$\leftarrow$ reading this again, maybe this isn't the right place for this paragraph. Consider putting in the conclusion? I think that after we said that people have worked on it, we don't necessarily need this here and will shorten the long intro}


% If framing can influence their outputs, this could have significant societal effects,
% from spreading biases in automated decision-making~\citep{ghasemaghaei2024understanding} to reducing public trust in AI-generated content~\citep{afroogh2024trust}. 
% However, framing is not inherently negative -- understanding how it affects LLM outputs can offer valuable insights into both human and machine cognition.
% By systematically investigating the framing effect,


%It is therefore crucial to systematically investigate the framing effect, to better understand and mitigate its impact. \gabis{This paragraph is important - I think that right now it's saying that we don't want models to be influenced by framing (since we want to mitigate its impact, right?) When we talked I think we had a more nuanced position?}




To better understand the framing effect in LLMs in comparison to human behavior,
we introduce the \name{} dataset (Section~\ref{sec:data}), comprising 1,000 statements, constructed through a three-step process, as shown in Figure~\ref{fig:fig1}.
First, we collect a set of real-world statements that express a clear negative or positive sentiment (e.g., ``I won the highest prize'').
%as exemplified in Figure~\ref{fig:fig1} -- ``I won the highest prize'' positive base statement. (2) next,
Second, we \emph{reframe} the text by adding a prefix or suffix with an opposite sentiment (e.g., ``I won the highest prize, \emph{although I lost all my friends on the way}'').
Finally, we collect human annotations by asking different participants
if they consider the reframed statement to be overall positive or negative.
% \gabist{This allows us to quantify the extent of \textit{sentiment shifts}, which is defined as labeling the sentiment aligning with the opposite framing, rather then the base sentiment -- e.g., voting ``negative'' for the statement ``I won the highest prize, although I lost all my friends on the way'', as it aligns with the opposite framing sentiment.}
We choose to annotate Amazon reviews, where sentiment is more robust, compared to e.g., the news domain which introduces confounding variables such as prior political leaning~\cite{druckman2004political}.


%While the implications of framing on sensitive and controversial topics like politics or economics are highly relevant to real-world applications, testing these subjects in a controlled setting is challenging. Such topics can introduce confounding variables, as annotators might rely on their personal beliefs or emotions rather than focusing solely on the framing, particularly when the content is emotionally charged~\cite{druckman2004political}. To balance real-world relevance with experimental reliability, we chose to focus on statements derived from Amazon reviews. These are naturally occurring, sentiment-rich texts that are less likely to trigger strong preexisting biases or emotional reactions. For instance, a review like ``The book was engaging'' can be framed negatively without invoking specific cultural or political associations. 

 In Section~\ref{sec:results}, we evaluate eight state-of-the-art LLMs
 % including \gpt{}~\cite{openai2024gpt4osystemcard}, \llama{}~\cite{dubey2024llama}, \mistral{}~\cite{jiang2023mistral}, \mixtral{}~\cite{mistral2023mixtral}, and \gemma{}~\cite{team2024gemma}, 
on the \name{} dataset and compare them against human annotations. We find  that LLMs are influenced by framing, somewhat similar to human behavior. All models show a \emph{strong} correlation ($r>0.57$) with human behavior.
%All models show a correlation with human responses of more than $0.55$ in Pearson's $r$ \gabis{@Gili check how people report this?}.
Moreover, we find that both humans and LLMs are more influenced by positive reframing rather than negative reframing. We also find that larger models tend to be more correlated with human behavior. Interestingly, \gpt{} shows the lowest correlation with human behavior. This raises questions about how architectural or training differences might influence susceptibility to framing. 
%\gabis{this last finding about \gpt{} stands in opposition to the start of the statement, right? Even though it's probably one of the largest models, it doesn't correlate with humans? If so, better to state this explicitly}

This work contributes to understanding the parallels between LLM and human cognition, offering insights into how cognitive mechanisms such as the framing effect emerge in LLMs.\footnote{\name{} data available at \url{https://huggingface.co/datasets/gililior/WildFrame}\\Code: ~\url{https://github.com/SLAB-NLP/WildFrame-Eval}}

%\gabist{It also raises fundamental philosophical and practical questions -- should LLMs aim to emulate human-like behavior, even when such behavior is susceptible to harmful cognitive biases? or should they strive to deviate from human tendencies to avoid reproducing these pitfalls?}\gabis{$\leftarrow$ also following Itay's comment, maybe this is better in the dicsussion, since we don't address these questions in the paper.} %\gabis{This last statement brings the nuance back, so I think it contradicts the previous parapgraph where we talked about ``mitigating'' the effect of framing. Also, I think it would be nice to discuss this a bit more in depth, maybe in the discussion section.}






\section{Background on Causal Inference}
\label{sec:background-causal} 



 \newtextold{In this section, we 
 %formalize the notion of {\em Average Treatment Effect and understand the 
 review the basic concepts and key assumptions for inferring the effects of an intervention on the outcome on collected datasets without performing randomized controlled experiments. 
We use {\em Pearl's graphical causal model} for {\em observational causal analysis} \cite{pearl2009causal} to define these concepts.}


\par
\paratitle{Causal Inference and Causal DAGs} The primary goal of causal inference is to model causal dependencies between attributes and evaluate how changing one variable (referred to as intervention) would affect the other.
Pearl's Probabilistic Graphical Causal Model \cite{pearl2009causal} can be written as a tuple $(\exo, \edvar, Pr_{\exo}, \psi)$, where $\exo$ is a set of {\em exogenous} variables, $\Pr_{\exo}$ is the joint distribution of \exo, and $\edvar$ is a set of observed {\em endogenous variables}.
Here $\psi$ is a set of structural equations that encode dependencies among variables. The equation for $A \in \edvar$ takes the following form:
%that encode the dependencies among the variables.  These equations are of the form 
$$\psi_{A}: 
\dom(Pa_{\exo}(A)) {\times} \dom(Pa_{\edvar}(A)) \to \dom(A)$$
Here $Pa_{\exo}(A) {\subseteq} {\exo}$ and $Pa_{\edvar}(A) {\subseteq} \edvar \setminus \{A\}$ respectively denote the exogenous and endogenous parents of $A$. A causal relational model is associated with a directed acyclic graph ({\em causal DAG}) $G$, whose nodes are the endogenous variables $\edvar$ and there is a directed edge from $X$ to $O$ if  $X {\in} Pa_{\edvar}(O)$. The causal DAG obfuscates exogenous variables as they are unobserved. %Any given set of values for the exogenous variables completely determines the values of the endogenous variables by the structural equations (we do not need any known closed-form expressions of the structural equations in this work). 
The probability distribution $\Pr_{\exo}$ on exogenous variables $\exo$ induces a probability distribution  
on the endogenous variables $\edvar$ by the structural equations $\psi$.  A causal DAG can be constructed by a domain expert as in the above example, or using existing {\em causal discovery} algorithms~\cite{glymour2019review}. 



\begin{figure}
    \centering
    \small
    \begin{tikzpicture}[node distance=0.6cm and 1cm, every node/.style={minimum size=0.5cm}]
        \tikzset{vertex/.style = {draw, circle, align=center}}

        \node[vertex] (Ethnicity) {\bf\scriptsize{{Ethnicity}}};
        \node[vertex, right=0.3cm of Ethnicity] (Gender) {\bf{\scriptsize{Gender}}};
        \node[vertex, right=0.3cm of Gender] (Age) {\bf{\scriptsize{Age}}};
        \node[vertex, below=0.3cm of Gender] (Role) {\bf{\scriptsize{Role}}};
        \node[vertex, right=0.3cm of Role] (Education) {\bf{\small{\scriptsize{Education}}}};
        \node[vertex, below=0.3cm of Role] (Salary) {\bf{\scriptsize{Salary}}};

        \draw[->] (Ethnicity) -- (Salary);
        \draw[->] (Gender) -- (Role);
        \draw[->] (Age) -- (Role);
         \draw[->] (Education) -- (Role);
           \draw[->] (Education) -- (Salary);
             \draw[->] (Ethnicity) -- (Education);
                \draw[->] (Ethnicity) -- (Role);
             \draw[->] (Gender) -- (Education);
               \draw[->] (Age) -- (Education);
                 \draw[->] (Role) -- (Salary);
        \draw[->] (Gender) to[bend right] (Salary);
        \draw[->] (Age) -- (Salary);
    \end{tikzpicture}
    \caption{Partial causal DAG for the Stack Overflow dataset.}
    \label{fig:causal_DAG}
\end{figure}



 \begin{example}
Figure \ref{fig:causal_DAG} depicts a partial causal DAG for the SO dataset over the attributes in Table \ref{tab:data} as endogenous variables (we use a larger causal DAG with all 20 attributes in our experiments). 
  Given this causal DAG, we can observe that the role that a coder has in their company depends on their education, age gender and ethnicity.
\end{example}
\par


\par
\paratitle{Intervention} In Pearl's model, a treatment $T = t$ (on one or more variables) is considered as an {\em intervention} to a causal DAG by mechanically changing the DAG such that the values of node(s) of $T$ in $G$ are set to the value(s) in $t$, which is denoted by $\doop(T = t)$. Following this operation, the probability distribution of the nodes in the graph changes as the treatment nodes no longer depend on the values of their parents. Pearl's model gives an approach to estimate the new probability distribution by identifying the confounding factors $Z$ described earlier using conditions such as {\em d-separation} and {\em backdoor criteria} \cite{pearl2009causal}, which we do not discuss in this paper.


\par
\paratitle{Average Treatment Effect} The effects of an intervention are often measured by evaluating
% \par
% \paratitle{Causal inference, Treatment, ATE, and CATE}
% \newtextold{One of the primary goals  of {\em causal inference} is to estimate the effect of making a change in terms of a {\em treatment} $T$ (often referred to as an intervention)
% on the outcome $O$. 
% %A variable that is modified is often referred to as the treatment variable $T$ and the metric used to captures 
% The effect of treatment $T$ on outcome $O$ is measured by 
% %is known as 
{\em Conditional Average treatment effect (CATE)}, 
%a {\em treatment variable} $T$ on an outcome variable $O$ (e.g., what is the effect of higher \verb|Education| on \verb|Salary|). 
measuring the effect of an intervention on a subset of records~\cite{rubin1971use,holland1986statistics} by calculating the difference in average outcomes between the group that receives the treatment and the group that does not (called the {\em control} group), providing an estimate of how the intervention by $T$ influences an outcome $O$ for a given subpopulation. 
% Mathematically,
% \begin{equation}
%     %{\small ATE(T,O) = \mathbb{E}[O \mid \doop(T=1)] -      \mathbb{E}[O \mid \doop(T=0)]}
%     {\small ATE(T, O) = \mathbb{E}[O \mid \doop(T=1)] -  
%     \mathbb{E}[O \mid \doop(T=0)]}
% \label{eq:ate}
% \end{equation}
% In our work, where the treatment with maximum effect may vary among different subpopulations, we are interested in computing the \emph{Conditional Average Treatment Effect} (CATE), which measures the effect of a treatment on an outcome on \emph{a subset of input units}~\cite{rubin1971use,holland1986statistics}. 
Given a subset of the records defined by (a vector of) attributes $B$ and their values $b$, 
%g {\in} \Qagg(\db)$ defined by a predicate $G {=} g$ 
we can compute $CATE(T,O \mid B = b)$ as:
{
\begin{eqnarray}    
    %CATE(T,O \mid G=g) = \mathbb{E}[O \mid \doop(T=1)&, G=g] -  \mathbb{E}[O \mid \doop(T=0), G=g] 
   % CATE(T,O \mid B = b) = 
    \mathbb{E}[O \mid \doop(T=1), B = b] -  
    \mathbb{E}[O \mid \doop(T=0), B = b]\label{eq:cate}
\end{eqnarray}
}
Setting $B=\phi$ is equivalent to the ATE estimate.
The above definitions assumes that the treatment assigned to one unit does not affect the outcome of another unit (called the {Stable Unit Treatment Value Assumption (SUTVA)) \cite{rubin2005causal}}\footnote{This assumption does not hold for causal inference on multiple tables and even on a single table where tuples depend on each other.}. 


The ideal way of estimating the ATE and CATE is through {\em randomized controlled experiments}, 
where the population is randomly divided into two groups (treated and control, for binary treatments): 
%treated group that receives the treatment and control group that does not (denoted by 
%{the \em treated} group 
denoted by 
$\doop(T = 1)$ 
%for a binary treatment)  (the {\em control} group, 
and $\doop(T = 0)$ resp.)~\cite{pearl2009causal}.
%\sr{edited up to here, going to read the rest first, this section should not look like causumx}
%\par
%\par
However, randomized experiments cannot always be performed due to ethical or feasibility issues. In these scenarios, observational data is used to estimate the treatment effect, which requires the following additional assumptions. 
% {\em Observational Causal Analysis} still allows sound causal inference under additional assumptions. Randomization in controlled trials mitigates the effect of {\em confounding factors}, i.e., attributes that can affect the treatment assignment and outcome. Suppose we want to understand the causal effect of \verb|Education| on \verb|Salary| from the SO dataset.  %in Example~\ref{ex:running_example}. 
% We no longer apply Eq. (\ref{eq:ate}) since the values of \verb|Education| were not assigned at random in this data, and obtaining higher education largely depends on other attributes like \verb|Gender|, \verb|Age|, and \verb|Country|. 
% Pearl's model provides ways to account for these confounding attributes $Z$ to get an unbiased causal estimate from observational data under the following assumptions ($\independent$ denotes independence):
% \vspace{-2mm}
\newtextold{
The first assumption is called {\em unconfoundedness} or {\em strong ignorability}  \cite{rosenbaum1983central} says that the independence of outcome $O$ and treatment $T$ conditioning on a set of confounder variables  (covariates) $Z$, i.e.,
%\begin{eqnarray}
 $    O \independent T | Z {=} z$.
 %\label{eq:unconfoundedness}
%\end{eqnarray}
The second assumption called {\em overlap or positivity} says that there is a chance of observing individuals in both the treatment and control groups for every combination of covariate values, i.e., 
%\begin{eqnarray}
   $ 0 < Pr(T {=} 1 ~~|~~Z {=} z)< 1 $.
   %\label{eq:overlap}
%\end{eqnarray}
}
%\sg{Is this overlap or positivity? maybe both are the same?} \sr{yeah - same - from Google AI - The overlap assumption, also known as the positivity assumption, is a key assumption in causal inference that states that there is a chance of observing individuals in both the treatment and control groups for every combination of covariate values.}
% The above conditions are known as {\em Strong Ignorability} in Rubin's model \cite{rubin2005causal}.
The unconfoundedness assumption requires that the treatment $T$ and the outcome $O$ be independent when conditioned on a set of variables $Z$. In SO, assuming that only $Z$ =\{\verb|Gender|, \verb|Age|, \verb|Country|\} affects $T = $ \verb|Education|, if we condition on a fixed set of values of $Z$, i.e., consider people of a given gender, from a given country, and at a given age, then $T = $ \verb|Education| and $O = $ \verb|Salary| are independent. For such confounding factors $Z$,  Eq. (\ref{eq:cate}) reduces to the following form 
(positivity 
gives the feasibility of the expectation difference): 
 \vspace{-1mm}
{\small
\begin{flalign}    
% \begin{eqnarray}
   % % & ATE(T,O) = \mathbb{E}_Z \left[\mathbb{E}[O \mid T=1, Z = z] -  
   %  \mathbb{E}[O \mid T=0, Z = z] \right] \label{eq:conf-ate}\\
 & CATE(T,O {\mid} B {=} b) {=} \nonumber
    \mathbb{E}_Z \left[\mathbb{E}[O {\mid} T{=}1, B {=} b, Z {=} z] {-}  
    \mathbb{E}[O {\mid} T{=}0, B {=} b, Z {=} z]\right]\label{eq:conf-cate}
\end{flalign}
% \end{eqnarray}
}
% \vspace{-4mm}
This equation contains conditional probabilities and not $\doop(T = b)$, which can be estimated from an observed data. 
Pearl's model gives a systematic way to find such a $Z$ when a causal DAG is available. 




\section{Goal and Research Questions}
\label{sec:researchissues}

The goal of this study is to evaluate the adoption of ML models to support the selection of studies for SLR updates. We translated our goal into three different Research Questions (RQs).

    \textbf{RQ1:} \textit{How effective are ML models in selecting studies for SLR updates?}

    To answer this research question, we represent the effectiveness of the ML models in supporting study selection using metrics such as \textit{Recall}, \textit{Precision} and \textit{F-measure} \cite{Napoleao2021, Watanabe20}. Our ML automated analysis considers only the title, abstract, and keywords of the studies. Our ML models were trained with data from the original SLR and asked to select studies for the SLR update. The results were compared with the final results of the included and excluded studies (according to the consensus discussion of the three experienced SE researchers) for the SLR update.
    
    %Our ML automated analysis considers only title and abstract of the studies and the metrics are calculated at first considering the results from the expert reviewers analysis only on the title, abstract and keywords and next, considering also their results from the full-text analysis.

    \textbf{RQ2:} \textit{How much effort can ML models reduce during the study selection activity of SLR updates?}

    For this research question, we calculate the effort reduction by the relation of the number of studies that need to have their title, abstract, and keywords manually analyzed without the support of ML models versus the number of studies to be analyzed after discarding studies that would have a low probability of being included according to the ML model. \textit{I.e.}, we analyzed the percentage of studies that could be safely discarded based on their inclusion probability while keeping a 100\% recall of the included studies. 
   
    \textbf{RQ3:} \textit{How does the support of ML in the selection of studies compare to the support of an additional human reviewer?}
    % We compared the agreement level of the ML Model with the highest \textit{F-score} value, supporting a single reviewer with the agreement level of each pair of reviewers, by calculating their Cohen's \textit{Kappa} coefficient \cite{Cohen10, Kitchenham15}.

    In this research question, we assessed how a pair of a human and an ``ML model reviewer" would compare against pairs of human reviewers in determining the list of studies to be included. Therefore, to improve the chances of providing good support, we used the ML model with the highest \textit{F-score}. In the initial screening, each of the three SE researchers had assessed each paper on a scale from zero to two (0 - exclude, 1 - unsure, 2 - include). We adjusted the outcome of the ML model (probabilities for inclusion) to that same scale of integers. Thereafter, we calculated the aggregated outcome for the list of papers for each possible pair of reviewers using the average score of the pair members. By using the average, we fairly hypothesize that, when working in pairs, each member of the pair would equally influence inclusion or exclusion. Finally, we compared the aggregated outcome of each pair with the final results using the Euclidean distance to understand how far each pair was from the oracle.

    % \end{itemize}

    %---- Here I did not defined Kappa. I think we can defined it in the methodology section.

    %Kappa analysis ->  Euclidean Distance
    %agremeent level do algoritmo com os revisores - titulo, abstract and keywords
    % future assessement + Hipótese :  Machine learning can replace a reviewer in a SLR update?
\section{Methodology}
\label{sec:methodology}

\subsection{Overview}

In this section, we present the design of \alias.
The system architecture is illustrated in Figure~\ref{fig:tracezip_system}, in which we add a \textit{compression module} (\ding{198}) to the service side and a corresponding \textit{decompression module} (\ding{199}) to the entry of the backend trace collectors.
In the compression module, we maintain two data structures, namely, a \name (\sname) and a dictionary, to constantly capture the redundancy across the spans.
Upon the generation of a span at the tracepoint, it undergoes compression utilizing the above data structures.
If the span carries a new redundancy pattern, it will be seamlessly integrated into the \sname and dictionary.
This integration is crucial as it enriches the structures, thereby facilitating the compression for subsequent spans.
We accelerate the above process by employing a combination of mapping and hashing techniques.
At the decompression module, the spans are restored to their original form by referring to the SRT and the dictionary.
To ensure a consistent and reliable data transmission,  it is imperative that these data structures are accurately synchronized between the service and backend sides.
To achieve this, we develop a differential update mechanism (\ding{200}).
This mechanism is designed to precisely pinpoint and propagate only the incremental changes in the data structures, ensuring an efficient synchronization process that minimizes overhead while maximizing data consistency.

\begin{figure}[t]
    \centering
    \includegraphics[width=0.62\linewidth]{figures/tracezip_system.pdf}
    \caption{System Architecture of \alias}
    \label{fig:tracezip_system}
\end{figure}


% \usetikzlibrary{trees}

% In microservice tracing technology, a microservice sends current trace data to the tracing backend. The trace data, known as Spans, are defined as individual units of work within a trace. As we mentioned above, in high-concurrency scenarios, it is likely that many key-value pairs in the spans array sent by a microservice to the tracing backend are repetitive. To better capture the redundancy in the Spans data sent by microservices, we have designed an algorithm named the Merging Algorithm. The Merging Algorithm transforms the Spans array data sent by microservices into a data structure similar to a Prefix Tree, which we will refer to as the Merging Tree in subsequent text. In the Merging Tree, all non-leaf nodes, except the root, are key-value pairs, while the leaf nodes are arrays of key-value pairs. By merging all the key-value pairs along the path from the root to any leaf node, one can reconstruct a span data entry. Figure \ref{fig:Spans_Array_Example} and Figure \ref{fig:Merging_Tree_Example} show an example of spans array and its corresponding Merging Tree. Additionally, we can capture uncommon Spans attributes and report anomalies to the tracing backend using the Spans attribute frequency information generated by the Merging Algorithm. Moreover, by using Dictionary Compression Algorithm, we can assign shorter codes to frequently occurring attribute values based on the frequency information produced by the Merging Algorithm.
% At the same time, when we use the Merging Algorithm and Dictionary Compression Algorithm to compress data, attribute value frequency data is generated. We can utilize this attribute value frequency data to identify potential anomalous link data. 
% The compression algorithm, while compressing the Spans data, will also mark potential anomalous link data and report it back to the Spans collection service backend.


% \begin{table}[]
%     \centering
%     \normalsize
%     \caption{Caption}
%     \begin{tabular}{c|c|c|c}
%     \toprule
%         name&db server&url&start\_time\\
%         \midrule
%         \midrule
%         interface\_1&GET&/interface1&... \\
%         interface\_1&GET&/interface1&... \\
%         interface\_2&PATCH&/interface2&... \\
%         interface\_2&PATCH&/interface2&... \\
%     \bottomrule
%     \end{tabular}
%     \label{tab:my_label}
% \end{table}

% \begin{table}[]
% \captionsetup{justification=centering}
% \centering
% \caption{Ablation study of components in}
% \label{tab: RQ2-ablation}
%  % \resizebox{0.48\textwidth}{!}{%\
% \begin{NiceTabular}{c|c|c|c|c}
% \toprule
% \rowcolor{grey}\textbf{name} & \textbf{operation} & \textbf{db server} & \textbf{results.count} & \textbf{other attrib.}\\
% \midrule
% \midrule
% Access DB & SELECT & MySQL & 10 & ...\\
% Access DB & INSERT & MySQL & 2 & ...\\
% Access DB & UPDATE & MongoDB & 5 & ...\\
% Access DB & DELETE & MongoDB & 1 & ...\\
% \bottomrule
% \end{NiceTabular}
% % }
% \end{table}

% \begin{figure}[H]
%     \centering
%     \begin{tabular}{c c c c}
%         \toprule
%         \multicolumn4c{Spans Array}\\
%         \cmidrule(lr){1-4}
%         Span Name&HTTP Method&URL&Details\\
%         \midrule
%         interface\_1&GET&/interface1&... \\
%         interface\_1&GET&/interface1&... \\
%         interface\_2&PATCH&/interface2&... \\
%         interface\_2&PATCH&/interface2&... \\
%         \bottomrule
%     \end{tabular}
%     \caption{An Example of Spans Array}
%     \label{fig:Spans_Array_Example}
% \end{figure}


% \tikzstyle{every node}=[draw=black,thick,anchor=west]
% \tikzstyle{selected}=[draw=red,fill=red!30]
% \tikzstyle{optional}=[dashed,fill=gray!50]
% \begin{figure*}
% \centering
% \begin{tikzpicture}[%
%   level distance=1.5cm,
%   sibling distance=2.5cm,
%   grow via three points={one child at (0.5,-0.7) and
%   two children at (0.5,-0.7) and (0.5,-1.4)},
%   edge from parent path={(\tikzparentnode.south) |- (\tikzchildnode.west)}]
%   \node {Merging Trees: global\_time\_base}
%     child { node {span\_name: interface\_1}
%         child {
%             node {http.method: GET}
%             child {
%                 node {url: /interface1}
%                 child [missing] {}
%                 child {
%                     node {\makecell{
%                             start\_time: start\_time\_offset \\
%                             end\_time: end\_time\_offset \\
%                             parent\_id: ... \\
%                             context: \{ ... \}
%                         }
%                     }
%                 }
%                 child [missing] {}
%                 child [missing] {}
%                 child {
%                     node {\makecell{
%                             start\_time: start\_time\_offset \\
%                             end\_time: end\_time\_offset \\
%                             parent\_id: ... \\
%                             context: \{ ... \}
%                         }
%                     }
%                 }
%                 child [missing] {}
%                 child [missing] {}
%             }
%         }
%         child [missing] {}
%     }		
%     child [missing] {}				
%     child [missing] {}				
%     child [missing] {}
%     child [missing] {}
%     child [missing] {}
%     child [missing] {}				
%     child [missing] {}	
%     child [missing] {}
%     child { node {span\_name: interface\_2}
%         child { node {http.method: PATCH}
%             child {
%                 node {url: /interface2}
%                 child [missing] {}
%                 child {
%                     node {\makecell{
%                             start\_time: start\_time\_offset \\
%                             end\_time: end\_time\_offset \\
%                             parent\_id: ... \\
%                             context: \{ ... \}
%                         }
%                     }
%                 }
%                 child [missing] {}
%                 child [missing] {}
%                 child {
%                     node {\makecell{
%                             start\_time: start\_time\_offset \\
%                             end\_time: end\_time\_offset \\
%                             parent\_id: ... \\
%                             context: \{ ... \}
%                         }
%                     }
%                 }
%                 child [missing] {}
%                 child [missing] {}
%             }
%         }
%     };
% \end{tikzpicture}
% \caption{Merging Tree Structure of Spans Array} \label{fig:Merging_Tree_Example}
% \end{figure*}


% The Merging Algorithm performs two operations on the microservice side: preprocessing and compressing the Spans Array, resulting in the creation of the Merging Tree. On the trace backend, a decompression algorithm is executed. During the preprocessing phase, the Merging Algorithm identifies which attribute in the Spans Array should be allocated to the non-leaf nodes of the Merging Tree by analyzing the repetition of these attribute values. This is done in linear time complexity. It also determines an order of attributes to minimize the number of non-leaf nodes in the Merging Tree. In the compression phase, the Merging Algorithm inserts each span into the Merging Tree according to the attribute order established during preprocessing. During the decompression phase, the Merging Algorithm reconstructs the Spans Array from the Merging Tree. The Merging Tree can be combined with other general compression algorithms, such as LZ77 and Huffman coding, to achieve higher compression ratios. Notably, the Merging Algorithm has superior time complexity compared to most existing general-purpose data compression algorithms, a topic that will be discussed in a subsequent subsection.

\subsection{Span Format Conventions}

To compress spans by leveraging their recurring patterns, we first stipulate the format of a span.
For simplicity and readibility, we assume that a span adheres to the standard JSON data format that defines it as a structured set of key-value pairs.
The key is a string, while the value can be either primitive types (strings, numbers, booleans, and null) or two structured types (nested key-value pairs and arrays).
This aligns with the format specifications used in many tracing frameworks and tools, e.g., OpenTelemetry~\cite{opentelemetry_traces}, Jaeger~\cite{jaeger}, Zipkin~\cite{zipkin}.
Typical fields (keys) of a span include: \textit{Name} (a human-readable string representing the operation done), \textit{Parent Span ID} (the span that caused the creation of this span, empty for root spans), \textit{Start and End Timestamps} (the start and end time of the span), \textit{Span Context} (the context of the span including the trace ID, the span ID, etc.), \textit{Attributes} (key-value pairs representing additional information about the span), \textit{Span Events} (structured log messages/annotations on a span), etc.
It is important to note that our proposed algorithm is not restricted to JSON or any particular serialization format.
For example, \alias can work effectively when Protobuf (Protocol Buffers)~\cite{protobuf} is used for trace data serialization.
With Protobuf's powerful deserialization capabilities, we can leverage its reflection-like APIs or direct-access methods to dynamically access the fields and values of spans.
Additionally, Protobuf is designed to be backward and forward compatible, allowing us to modify the message definition by adding or removing fields while maintaining compatibility with older data.
Such operations are essential for reducing trace redundancy, e.g., removing span elements that are deemed repetitive.

We also assume that spans possess \textit{structural locality}, meaning that during the continuous execution of a service or component, all spans sharing a common span Name will exhibit an identical structure.
In other words, spans with the same Name will consistently retain the same set of keys (e.g., attributes, tags, and metadata), differing only in the specific values associated with them.
This assumption arises naturally from the way distributed tracing systems operate, where spans typically represent predefined operations or events within the service workflow.
These operations are implemented as part of the service's codebase, which enforces a fixed schema or structure for spans generated by specific instrumentation points.
% For example, a span representing a database query will always include fields such as \texttt{db.statement}, \texttt{db.type}, or \texttt{db.instance}, while their values—such as the specific SQL query or database name—may vary.
This structural consistency allows for reliable trace analysis, optimization, and redundancy reduction, as the predictability of span structures minimizes the need for per-span schema discovery during processing.

% We also assume that spans possess \textit{structural locality}.
% That is, throughout the continuous execution of a service/component, all generated spans sharing a common span Name will have an identical structure.
% This implies they will retain the exact same keys, differing only in the values.
% Nevertheless, if it is violated\zb{what cases}, we xxx
% Such a property allow us to chain the values \zb{not finished}

% \newtheorem{definition}{Definition}

% A Spans Array is defined as follows:
% \begin{definition}\label{def_spans_array}
% A Spans Array is a set of key-value pair sets. Any key-value pair set in the Spans Array is referred to as a Span. Each Span has a key named "Span Name," and for all Spans with the same "Span Name," the set of keys for the remaining key-value pairs is identical. In other words, all Spans with the same Span Name have the same structure, meaning they have exactly the same keys, differing only in the values of those keys.
% \end{definition}


\subsection{Span Retrieval Compression and Uncompression}
\label{sec:compression_uncompression}

% A straightforward approach to compress spans by exploiting their repetitiveness involves the use of a dictionary.
A straightforward approach to compressing spans involves the use of a dictionary.
This method creates a dictionary where every unique key and value is assigned a unique identifier.
During the compression process, the keys and values of each span are substituted by the corresponding identifiers.
The size of the span can then be reduced as the identifiers are much smaller than the original data.
However, as revealed by our empirical study, there can still be redundant information among the identifiers.
The pure dictionary approach compresses data on a one-to-one basis, i.e., one identifier corresponds to one KV pair.
If multiple spans share a collection of common key-value pairs, it is possible to utilize a single identifier to represent this entire set of shared pairs, thereby amplifying the compression efficiency.
Thus, we propose to leverage the correlations among the values of spans to further eliminate repetitive information.

% \begin{algorithm}[t]
% \caption{Performance Anomaly Detection}
% \label{algo:anomaly_detection}
% \normalsize
% \SetAlgoLined
% \KwIn{$t$, $\mathcal{P}_a$, and $\mu_C$}
% \KwOut{Anomaly detection result for $t$}

% $\mathcal{D}_t\gets {\rm PairWiseDistance}(t, \mu_C)$

% $idx\gets {\rm MinIndex}(\mathcal{D}_t)$

% \eIf{$idx\in \mathcal{P}_a$}{
%     return True
% }{
%     return False
% }
% \end{algorithm}

% If there are more KV pairs shared across multiple spans, it is possible to represent this set of KV pairs with one identifier.
% \zb{Examples of events may include uncaught exceptions, button clicks, user logouts, network disconnections, etc. Its structure is also similar to that of Spans.}

\begin{algorithm}
    \caption{Span Retrieval Tree (SRT) Reconstruction and Span Compression}
    \label{algo:srt}
    \begin{algorithmic}[1]
    \State \textbf{Input:} a stream of continuously generated \textit{spans}, a threshold $\psi$
    \State \textbf{Output:} a constructed SRT $\mathcal{T}$, \textit{compressed spans}
    \State Initialize an empty SRT $\mathcal{T}$
    \For{each \textit{span} in \textit{spans}}
        \If{\textit{span Name} not in $\mathcal{T}$}
            \State Chain all key-value pairs of \textit{span} and add the path to the root of $\mathcal{T}$
            % \State Each key of $\mathcal{T}$ has a unique value number of 1
            \State Assign an identifier to this new path
        \Else
        \For{each \textit{key} at every depth of $\mathcal{T}$} \Comment{traverse $\mathcal{T}$ from the root to the leaf}
            \State Get the corresponding \{\textit{key}: \textit{value}\} from \textit{span}
            \If{\{\textit{key}: \textit{value}\} \textbf{does not exist} at the current depth of $\mathcal{T}$}
                \State Chain the remaining key-value pairs of \textit{span} and extend a new branch from the direct parent node of \textit{key}
                % \State Increase the unique value number of the corresponding keys by 1
                \State Calculate the number of unique nodes at each depth of $\mathcal{T}$
                \State Move the keys to the leaf whose unique value number exceeds $\psi$, i.e., \textit{local fields}
                \State Reorder the keys of $\mathcal{T}$ based on the ascending number of their unique values
                \State Reassign path identifiers
                \State \textbf{break}
            \EndIf
        \EndFor
        \EndIf

    \State Compress \textit{span} based on the corresponding path identifier and the values of local fields
    \EndFor
    \end{algorithmic}
\end{algorithm}

\begin{figure}[t]
    \centering
    \includegraphics[width=0.74\linewidth]{figures/span_retrieval_tree.pdf}
    \caption{An Example of \name}
    \label{fig:span_retrieval_tree}
\end{figure}

Our idea is that for spans generated in each service instance, we organize their key-value pairs as a prefix-tree-like data structure, i.e., \sname.
The \sname functions as a multi-way tree, with all non-leaf nodes (except for the root) associated with a key-value pair.
For each type of span (which is identified by a unique span Name), there is only one leaf node connected to all the last non-leaf nodes stemming from it.
This leaf node holds a collection of keys without values.
Figure~\ref{fig:span_retrieval_tree} illustrates an example of SRT, where the gray node and the yellow nodes represent the root and the leaves, respectively, while the remaining are the non-leaf nodes.
Each span can be ``spelled out'' by tracing a path from the root down to the leaf.
The path of \sname represents the set of KV pairs shared across multiple spans.
% The tracing process is done by searching the node that corresponds to the next key-value pair in the span.
The non-leaf nodes contain the fields that are more repetitive, which we refer to as \textit{universal fields}.
Although spans may exhibit commonality, they will still have some unique KV pairs, such as those related to ID and timestamps.
We refer to such pairs as \textit{local fields} and only store their keys at the leaf.
The rationale is that such unique fields are incompressible, i.e., not shared with other spans, so we discard their values.
\textit{Based on \sname, a span can be represented as a unique path identifier plus its exclusive values that are extracted according to the keys in the leaf node.}
Each path identifier collectively represents the KV pairs shared among spans, instead of one identifier for each key and value.
Since these common KV pairs constitute a significant portion, the trace size can be substantially reduced, enhancing the overall efficiency.

% Specifically, the \name is a multi-way tree where the root node is associated with a key-value pair having the key "Span Name."
% Except for the root node, all non-leaf nodes are associated with a key-value pair, and all leaf nodes are associated with a key-value pair set. We define the root node's depth as 1, and the keys of all nodes at depth 2 as "Span Name."
% The Merging Tree satisfies the following property: For a Merging Tree, except for the leaf nodes, all nodes at the same depth have key-value pairs with the same keys. We will refer to this property as the Keep-Order property.

\begin{table}
    \centering
    \caption{Span Examples of a Data-processing Service}
    \label{tab:span_examples}
    \begin{subtable}{1\textwidth}
    \small
        \centering
        \begin{NiceTabular}{c|c|c|c|c|c}
            % \toprule
            \specialrule{0.35mm}{0em}{0em}
            \rowcolor{grey}\textbf{name} & \textbf{operation} & \textbf{address} & \textbf{data\_size} & \textbf{span\_id} & \textbf{others}\\
            % \midrule
            % \midrule
            \specialrule{0.15mm}{0em}{0em}
            \specialrule{0.15mm}{.1em}{0em}
            Access Mem & WRITE & address1 & 64 bytes & id1 & ...\\
            Access Mem & READ & address2 & 128 bytes & id2 & ...\\
            Access Mem & READ & address2 & 64 bytes & id3 & ...\\
            Access Mem & WRITE & address1 & 64 bytes & id4 & ...\\
            Access Mem & READ & address2 & 256 bytes & id5 & ...\\
            % \bottomrule
            \specialrule{0.35mm}{0em}{0em}
        \end{NiceTabular}
        % \vspace{6pt}
        \caption{Span examples of ``Access Mem''}
    \end{subtable}
    \vfill
    \begin{subtable}{1\textwidth}
    \small
        \centering
        \begin{NiceTabular}{c|c|c|c|c|c}
            % \toprule
            \specialrule{0.35mm}{0em}{0em}
            \rowcolor{grey}\textbf{name} & \textbf{type} & \textbf{DB system} & \textbf{status} & \textbf{row.num} & \textbf{others}\\
            % \midrule
            % \midrule
            \specialrule{0.15mm}{0em}{0em}
            \specialrule{0.15mm}{.1em}{0em}
            Access DB & INSERT & MySQL & SUCCESS & 1 & ...\\
            Access DB & SELECT & MySQL & SUCCESS & 1 & ...\\
            Access DB & DELETE & MySQL & SUCCESS & 1 & ...\\
            % \bottomrule
            \specialrule{0.35mm}{0em}{0em}
        \end{NiceTabular}
        % \vspace{6pt}
        \caption{Span examples of ``Access DB''}
    \end{subtable}
\end{table}


% \begin{table}[h]
%     \centering
%     \caption{Span Examples of a Data-processing Service}
%     \label{tab:span_examples}
%     \centering
%     \begin{NiceTabular}{c|c|c|c|c|c}
%         % \toprule
%         \specialrule{0.35mm}{0em}{0em}
%         \rowcolor{grey}\textbf{name} & \textbf{operation} & \textbf{address} & \textbf{data.size} & \textbf{span.id} & \textbf{...}\\
%         % \midrule
%         % \midrule
%         \specialrule{0.15mm}{0em}{0em}
%         \specialrule{0.15mm}{.1em}{0em}
%         Access Mem & WRITE & address1 & 64 bytes & id1 & ...\\
%         Access Mem & READ & address2 & 128 bytes & id2 & ...\\
%         Access Mem & READ & address2 & 64 bytes & id3 & ...\\
%         Access Mem & WRITE & address1 & 64 bytes & id4 & ...\\
%         Access Mem & READ & address2 & 256 bytes & id5 & ...\\
%         % \bottomrule
%         \specialrule{0.35mm}{0em}{0em}
%     \end{NiceTabular}
%     \vspace{6pt}
%     \caption{Span examples of ``Access Mem''}
% \end{table}

% \begin{table}[h]
%     \centering
%     \caption{Span examples of ``Access DB''}
%     \begin{NiceTabular}{c|c|c|c|c}
%         % \toprule
%         \specialrule{0.35mm}{0em}{0em}
%         \rowcolor{grey}\textbf{name} & \textbf{type} & \textbf{DB system} & \textbf{status} & \textbf{row.num}\\
%         % \midrule
%         % \midrule
%         \specialrule{0.15mm}{0em}{0em}
%         \specialrule{0.15mm}{.1em}{0em}
%         Access DB & INSERT & MySQL & SUCCESS & 1\\
%         Access DB & SELECT & MySQL & SUCCESS & 1\\
%         Access DB & DELETE & MySQL & SUCCESS & 1\\
%         % \bottomrule
%         \specialrule{0.35mm}{0em}{0em}
%     \end{NiceTabular}
%     \vspace{6pt}
% \end{table}

We present our algorithm for \sname construction and span compression (i.e., Algorithm~\ref{algo:srt}) and explain it using span examples in Table~\ref{tab:span_examples}.
Suppose these spans are continuously generated by different tracepoints of a data-accessing service, including memory and database.
Each tracepoint produces a specific type of span with varying attributes.
The algorithm takes the stream of spans as input, and the resulting \sname is shown in Figure~\ref{fig:span_retrieval_tree}.
For each new type of span with a previously unseen span Name, we simply chain all fields of the span (line 6) and add the resulting path to the \sname root.
For example, the first row of Table~\ref{tab:span_examples}-(a) will be structured as \textit{Access Mem}$\hookrightarrow$ \textit{WRITE}$\hookrightarrow$\textit{address1}$\hookrightarrow$\textit{64~bytes}$\hookrightarrow$\textit{id1} (we omit the keys of the nodes and the other attributes), shown as the pink dashed rectangles.
For spans with a known Name, we traverse the \sname from the root to the leaf, and use the key at each depth to retrieve the corresponding key-value pair from the span (line 10).
If a retrieved pair does not exist in the SRT, the remaining key-value pairs are chained to construct a sub-path, which is then added as a new branch to the direct parent node (line 12).
% For the next spans with new values, they will be added to the corresponding branches of the \sname.
For example, the second row adds a new path, \textit{READ}$\hookrightarrow$\textit{address2}$\hookrightarrow$\textit{128~bytes}$\hookrightarrow$\textit{id2}, to node \textit{Access~Mem}.
Each path of \sname will be assigned a unique identifier, as described in Section~\ref{sec:hashing_acceleration}.
% In particular, as spans follow the JSON format, they can have nested JSON object.
% We represent this relationship by prefixing the keys of the child JSON with the parent's key, which facilitates the restoration of the nested structure at the backend.
% To reduce the complexity, the default nesting depth is set as two.
% Deeper nested JSON object (not recommended by OTel) will be converted to string.
% \red{We flatten the span as a list of key-value pairs by extracting all elements in it. use a separation symbol to record the span structure (but the other still cannot be kept?). Also there will be an event \sname. Otel recommends primitives kv pairs?}
% As spans with the same name share an identical structure, their original structure can be easily retrieved\zb{how?}.

% \zb{talk about the root node and time base}\zb{existing log compression can be used to further process values}

In \sname, once a new path emerges, we calculate the number of distinct nodes at the same depth (line 13), which represents the number of different values of a key, e.g., the key \textit{span.id} has five distinct values \textit{id1}$\backsim$\textit{id5}.
We set a threshold $\psi$ for the size of values a key can have.
A key with too many values will be regarded as a local field and moved to the leaf (line 14).
For example, \textit{span.id} will be in the leaf if $\psi=3$.
With the constructed SRT, the fourth row can be compactly represented as the identifier of the first path, i.e., the pink path, coupled with its unique value, i.e., \textit{id4} (line 21).
% Note the leaf will not store the values of local fields.
% This is to avoid the \sname growing to large and consumes excessive memory.
Time-related fields such as span start/end time is also a typical local field.
Since spans generated in a short time period will have close temporal fields, we set a \textit{time\_base} at the root node, which allows the leaves to store only the offset relative to the time base.
This is a common way to compress temporal data.
% \zb{more detail (64-bit unsigned integer), how the time base is updated}
Another special local field is the nested JSON object, such as \textit{\{``attributes'': \{``ip'': ``172.17.0.1'', ``port'': 26040\}\}}.
We represent the nested structure by prefixing the keys of the child JSON object with the parent's key (e.g., \textit{``attributes-ip'': ``172.17.0.1''} and \textit{``attributes-port'': 26040}), which allows the backend to easily restore the original hierarchy.
Technically, spans can extend to any depth as required by the tracing needs.
For the consideration of \sname's size, we set a depth limit, which defaults to two, and convert the values of the overly deep keys into pure string type.
Similar to other string fields, they will be moved to the leaves if exhibiting too much diversity.
% \red{Based on our study, most fields have a depth two.}
% In Figure~\ref{fig:span_retrieval_tree}, the pink and dashed rectangles presents a compete path for the fourth row in Table~\ref{tab:span_examples}-(a).

After compression, the span data that needs to be transmitted to the backend trace collector become significantly smaller in size, i.e., only the path identifier and the values of local fields specified by the leaf.
At the backend side, the uncompression process to restore the original span is straightforward and efficient.
This involves reconstructing the local fields based on the corresponding values received and combining them with the universal fields based on the path identifier.
In this process, the backend side should keep the latest copy of the \sname and the value of \textit{time\_base}.
We introduce an efficient synchronization mechanism later in Section~\ref{sec:differential_sync}.
For \textit{time\_base}, we periodically reset it, e.g., every second, ensuring that the time offset remains consistently small.

% For each span, we traverse the \sname and compare the KV pairs stored in the non-leaf nodes with the corresponding fields of the span.
% If all universal fields of the span \red{match}, the span can be compressed as a path identifier with the values of its local fields.

% In this section, we formally define the Spans Array, Merging Tree, and the execution steps of the Merging Algorithm.

% The Merging Tree is defined as follows:
% \begin{definition}\label{def_Merging_tree}

% \end{definition}

% We define the following operation as restoring a Span:
% \begin{definition}\label{def_restoring_span}
% Select a leaf node of the Merging Tree, and combine all the key-value pairs corresponding to the nodes on the unique path from this leaf node to the root node to form a new key-value pair set.
% \end{definition}

% For a Merging Tree, restoring all Spans means performing the span restoration operation for all leaf nodes of the Merging Tree, resulting in a new set of key-value pair sets. If all the Spans restored from a Merging Tree have the same "Span Name" and are identical to those in a Spans Array, then we say the Merging Tree and the Spans Array are equivalent.

% The goal of the Merging Algorithm is to convert the Spans Array into its equivalent Merging Tree to achieve data compression.
% As implied by the name Merging Algorithm, this algorithm merges identical key-value pairs together using a data structure similar to a Prefix Tree, thereby reducing the frequency of repeated key-value pairs.
% Specifically, Merging Tree reduces the occurrence frequency of

% \begin{equation*}\label{nt_reduce}
% f=\sum_{\mathrm{node} \in U} \mathrm{Son}(\mathrm{node}) -1
% \end{equation*}

% key-value pairs, where $U$ means a set of all non-leaf nodes of an Merging Tree, $\mathrm{Son}(x)$ equals the number of leaf nodes in the subtree formed by node $x$.

% In the above definition, we stipulate that the Merging Tree must satisfy the Keep-Order property. This means that for a Spans Array with a specific Span Name S, we can fix the shape of the Merging Tree by fixing an order of the spans' attribute keys. Specifically, for spans with Span Name equals S, if the attribute key A appears in the $i$th position in the permutation $P$ of their attribute keys, we place nodes corresponding to attribute key A at the $(i + 1)$th level in the subtree rooted at Span Name equals S in the Merging Tree. Therefore, given the permutation $P$, we can generate the corresponding Merging Tree according to this order. It is not difficult to see that the possible shapes of the Merging Tree correspond one-to-one with all possible permutations $P$.

% Here, it is easy to see that for different permutations $P$ corresponding to different Merging Trees, the value of $f$ may vary. This means that the order of attribute keys will ultimately affect the compression efficiency of the Merging Tree. Next, we will present the Merging Algorithm, which is the process of generating the Merging Tree, without further explanation. The impact of permutation $P$ on the value of $f$, as well as the optimization of the value of $f$, will be discussed in detail in the next subsection.

% \begin{algorithm}
% \caption{Preprocessing Stage}
% \begin{algorithmic}[1]
% \State \textbf{Input\:} Spans Array \texttt{spansArray}
% \State \textbf{Initialize:}
% \State \quad \texttt{O} $\leftarrow$ empty 2D dictionary
% \State \quad \texttt{E} $\leftarrow$ empty 3D dictionary
% \For{\texttt{s} \textbf{in} \texttt{spansArray}}
%     \For{\texttt{(k, v)} \textbf{in} \texttt{s}}
%         \If{\texttt{E[s.Name][k][v]} = \textbf{false}}
%             \State \texttt{O[s.Name][k]} $\leftarrow$ \texttt{O[s.Name][k]} $+ 1$
%             \State \texttt{E[s.Name][k][v]} $\leftarrow$ \textbf{true}
%         \EndIf
%     \EndFor
% \EndFor
% \For{\texttt{Name} \textbf{in} \texttt{O}}
%     \State \texttt{keys} $\leftarrow$ \texttt{O[Name].keys()}
%     \State \texttt{sortedKeys} $\leftarrow$ \texttt{sort(keys, key=lambda k: O[Name][k])}
%     \State \texttt{P[Name]} $\leftarrow$ \texttt{sortedKeys}
% \EndFor
% \State \textbf{Output\:} Permutations of keys \texttt{P}
% \end{algorithmic}
% \end{algorithm}

% Algorithm 1 describes the tasks accomplished during the preprocessing stage of the algorithm. In brief, for all spans with the same span name, we count the number of unique values for each attribute key. Then, we sort the attribute keys in ascending order based on the number of unique values, forming a permutation.

% \begin{algorithm}
% \caption{Compression Stage}
% \begin{algorithmic}[1]
% \State \textbf{Initialize:} Create node \texttt{Root}
% \For{\texttt{span} \textbf{in} \texttt{spansArray}}
%     \State \texttt{P\_span} $\leftarrow$ \texttt{P[span.Name]}
%     \State \texttt{now} $\leftarrow$ \texttt{Root\_span\_name}
%     \For{\texttt{i} \textbf{in} \texttt{range(len(P\_span))}}
%         \State \texttt{k} $\leftarrow$ \texttt{P\_span[i]}
%         \If{\texttt{now} has a child associated with \texttt{<k,span[k]>}}
%             \State \texttt{now} $\leftarrow$ \texttt{child}
%         \Else
%             \State Create node \texttt{newNode}
%             \State Associate \texttt{<k,span[k]>} with \texttt{newNode}
%             \State Connect \texttt{newNode} to \texttt{now}
%             \State \texttt{now} $\leftarrow$ \texttt{newNode}
%         \EndIf
%     \EndFor
%     \State Create leaf node \texttt{leafNode}
%     \State Associate rest span info. with \texttt{leafNode}
%     \State Connect \texttt{leafNode} to \texttt{now}
% \EndFor
% \end{algorithmic}
% \end{algorithm}

% Algorithm 2 describes the tasks accomplished during the compression stage of the algorithm. In brief, during the compression stage, we insert the span's attributes into the Merging Tree in the order determined during the preprocessing stage.

% \begin{algorithm}
% \caption{Decompression Stage}
% \begin{algorithmic}[1]
% \Function{D}{now}
%     \If{\texttt{now} is a leaf node}
%         \State \Return \texttt{rest of span key-value pairs}
%     \Else
%         \State \texttt{results} $\leftarrow$ empty list
%         \For{\texttt{child} \textbf{in} \texttt{now.children}}
%             \State \texttt{childResults} $\leftarrow$ \Call{D}{child}
%             \For{\texttt{span} \textbf{in} \texttt{childResults}}
%                 \State \texttt{append span to results}
%             \EndFor
%         \EndFor
%         \For{\texttt{span} \textbf{in} \texttt{results}}
%             \State \texttt{add key-value pairs associated with now to span}
%         \EndFor
%         \State \Return \texttt{results}
%     \EndIf
% \EndFunction
% \State \textbf{Output\:} \Call{D}{Root}
% \end{algorithmic}
% \end{algorithm}

% Algorithm 3 describes the tasks accomplished during the decompression stage of the algorithm. Simply put, we designed a recursive function $D(x)$ that restores the Spans Array from the bottom up.

% \begin{figure*}[htp]
%     \centering
%     \includegraphics[width=18cm]{figures/compress.pdf}
%     \caption{Process of compression}
% \end{figure*}


\subsection{Optimizations for \name}

So far, we have introduced the algorithms for span compression and uncompression.
It can be seen that \alias has a small computational complexity.
This is because for each span, these processes involve only a single path traversal of the \sname from the root to a leaf.
However, the issue of space complexity presents a more significant challenge.
The \sname can potentially grow too large and consume an excessive amount of memory.
Besides setting a hard constraint on the memory, we have also identified some opportunities to optimize its size.

\subsubsection{\name Restructuring}

During the construction of \sname in Figure~\ref{fig:span_retrieval_tree}, we simply follow the left-to-right order of keys in Table~\ref{tab:span_examples} to form the parent-child relations among nodes.
For example, key \textit{address} is the child of \textit{name} and also the parent of \textit{data.size}.
We observe that this may result in a sub-optimal \sname structure.
Specifically, for the \sname in Figure~\ref{fig:sft_restructuring} which is built based on the spans in Table~\ref{tab:span_examples}-(b), we can see that the three paths differ only in the \textit{type} field.
A better structure can be obtained by moving \textit{type} down to the bottom, which avoids the recurrence of the other three fields.
Based on this finding, we propose the following way to restructure the \sname.
In Section~\ref{sec:compression_uncompression}, we have calculated the number of possible values associated with each key once a new path emerges.
If a parent field has more values than its child, we swap their positions in the \sname.
That is, we reorder the keys of \sname based on the ascending number of their unique values (line 15).
% This process continues until all fields find their appropriate places.
After reordering, the identical nodes at the same depth will be merged, e.g., \textit{MySQL}, \textit{SUCCESS}, and \textit{1} in Figure~\ref{fig:sft_restructuring}.
Finally, the path identifiers of the restructured \sname will be adjusted (line 16).

% In the previous section, we mentioned that the efficiency of the Merging Tree compression is determined by the formula:
% \begin{equation*}
% f=\sum_{\mathrm{node} \in U} \mathrm{Son}(\mathrm{node}) -1
% \end{equation*}
% In other words, since the number of leaf nodes in the Merging Tree is fixed (which equals the number of Spans in the Spans Array), the fewer non-leaf nodes the Merging Tree has, the better the compression effect of the Merging Tree. Through our research on the dataset, we discovered the following pattern: attributes with more possible values often imply attributes with fewer possible values. The diagram below illustrates this principle. \texttt{func1} and \texttt{func2} are HTTP handlers for the GET method, and \texttt{func3} and \texttt{func4} are HTTP handlers for the POST method. When the \texttt{method} attribute is placed above the \texttt{handler} attribute, the number of non-leaf nodes is less than when the \texttt{handler} attribute is placed above the \texttt{method} attribute. Based on this principle, we construct the Merging Tree by sorting the attributes in ascending order of the number of possible values.

% \begin{figure}[htp]
%     \centering
%     \includegraphics[width=9cm]{figures/order.pdf}
%     \caption{How order affects compression}
% \end{figure}

\begin{figure}
    \centering
    \includegraphics[width=0.66\linewidth]{figures/sft_restructuring.pdf}
    \caption{Span Retrieval Tree Restructuring}
    \label{fig:sft_restructuring}
\end{figure}

% \begin{figure}[t]
%     \centering
%     \includegraphics[width=0.74\linewidth]{figures/span_retrieval_tree.pdf}
%     \caption{An Example of \name}
%     \label{fig:span_retrieval_tree}
% \end{figure}

\subsubsection{Mapping-based Tree Compression}
\label{sec:mapping-based tree compression}

Although we have restructured the \sname to eliminate redundant nodes, there could still be repeated keys and values in it.
For example, in Figure~\ref{fig:span_retrieval_tree}, key \textit{data.size} appears in all \textit{data.size} nodes, e.g., \textit{\{``data.size'': ``64 bytes''\}} and two of them also share value \textit{64 bytes.}
Thus, to further compress the size of \sname, we employ a dictionary to map keys/values that occur multiple times to shorter identifiers.
We construct the identifiers using the standard alphanumeric set, i.e., [0-9a-zA-Z].
Initially, the hashed output consists of a single character, from '0' to '9,' followed by 'a' through 'z,' and finally 'A' through 'Z.'
Upon exhausting the single character possibilities, the function increases the length of the hash output to two characters, starting from '00,' '01,' and so forth.
Since each universal field has limited distinct values, i.e., smaller than $\psi$, the dictionary will also be small in size.
Note we do not encode the values of local fields (not in the \sname), which may inevitably make the dictionary too big given their diversity.
% \zb{how to do the encoding: should include separation by space, comma, etc.}
% \zb{will the dictionary map the local values?}
Similar to the \sname synchronization process between services and the tracing backend, the dictionary will be sent to the backend every time it undergoes an update.
% \zb{enough details?}
% As discussed above, we use the Merging Algorithm to compress a large amount of repetitive attribute value data in the Spans data. However, even with the Merging Algorithm, due to different choices of the order of attributes, it is still possible for a certain attribute value to repeatedly appear on the compressed Merging Tree. Therefore, we count the number of occurrences of the same attribute value on the Merging Tree and then sort them in descending order of occurrences, assigning shorter codes to more frequently occurring attribute values.
% After Dictionary Compression, the Spans Collectors send the compressed data and the dictionary to the tracing backend.
% When compressing using a character stream format like JSON, the assigned codes are allocated from a character set (e.g., [0-9a-zA-Z]), and the corresponding codes are determined based on the character set and the ranking of frequency. For gRPC, since gRPC uses variable-length encoding for numbers, we can directly encode them as numbers.

Based on our empirical study (Section~\ref{sec:redundancy_study}), there exists structural redundancy among the attributes of a span.
% We propose the idea of \textit{recursive dictionary} to further compress both \red{\sname} and the dictionary.
We address this issue by examining the ingredients of the attributes.
Specifically, when constructing the dictionary, we encode the common sub-fields shared among spans (instead of the entire fields) as identifiers.
These sub-field identifiers are then used to compose the complete attributes.
% The idea is that instead of encoding the entire field as a single identifier, we represent the common sub-fields shared among spans as identifiers and use them to compose the complete attributes.
% if an attribute is (partially) composed by a combination of other fields, then the sub-fields will be replaced by the code of the fields.
% When restoring the attribute, such .
Take a span example from OpenTelemetry~\cite{opentelemetry_traces}, which contains the following fields:

\begin{figure}[h]
    \centering
    \includegraphics[width=0.6\linewidth]{figures/structural_correlation.pdf}
    \label{fig:structural_correlation}
\end{figure}

% \begin{lstlisting}
% {
%     "net.transport": "IP.TCP",
%     "net.peer.ip": "172.17.0.1",
%     "net.peer.port": "51820",
%     "net.host.ip": "10.177.2.152",
%     "net.host.port": "26040"
%     ...
% }
% \end{lstlisting}

\noindent We can see that in the keys, there are some words that appear multiple times, e.g., \texttt{net}, \texttt{host}, \texttt{port}.
Without considering such correlations, we could potentially introduce too much lengthy keys to the dictionary.
To remove such redundancy, we first separate each key into a list of tokens based on delimiters dot (``.'') and underline (``\_''), which are configurable.
When encoding the key, each of its tokens will be mapped to the corresponding identifier.
As the value part exhibits more diversity, we only apply this technique to the keys to avoid too much computational overhead.

% \subsection{Edge-case Span Detection}
% \zb{need to talk more about why we need to do this}
% Based on the idea that failure symptoms are locally observable~\cite{DBLP:conf/nsdi/ZhangXAVM23}, we incorporate a detection mechanism for edge-case spans in the compression process.\zb{why need it at the collection side (efficiency? we have already gather related information), cant we just do it at the backend?}
% Specifically, when performing \sname construction and dictionary-based tree compression, we collect information about the occurrence numbers of different values in the spans.
% This allows us to pinpoint the values that rarely occur, which often signify anomalous events or outliers.
% In particular, certain fields such as \textit{span ID} and \textit{start/end time} are intended to hold unique values.
% Thus, we allow users to configure the set of fields whose rare occurrences should be closely monitored\zb{examples?}.
% Upon the detection of such rare values, we send the corresponding \textit{trace ID}\zb{then we dont know which span is the anomaly} to the tracing backend.
% The backend can immediately trigger an alert and notify system operators to investigate potential issues or anomalies.
% \zb{One advantage is interpretability, how can we achieve it? add a flag along with the trace ID?}

% % When generating the Merging Tree and performing Dictionary Compression, we collect information about the frequency of attribute values in the Spans. Based on this, we can calculate the frequency of occurrence for a particular category of attribute values. By considering the frequency and the number of unique attribute values for a specific attribute, we can determine if an attribute value is rare or uncommon. Generally, a rare attribute value signifies an anomalous behavior pattern. We believe that while performing the compression algorithm at the Spans Collector, it is possible to identify potential exceptional Spans and send the Trace ID of those potential exceptional Spans to the tracing backend.

% We use the following two criteria to pinpoint rare spans\zb{which are commonly used?} based on the type of attributes under monitoring:

% \begin{itemize}
%     \item \textit{Categorical attributes}: We employ a frequency-based approach to manage categorical attributes such as HTTP status codes and IP addresses.
%     Each node preserving one possible value of the target attribute will maintain a frequency ratio, meaning the proportion of spans carrying that specific value relative to the total span count.
%     Upon the arrival of each span, if the frequency of its carried value is below a pre-defined threshold (the default setting is 0.01, which is configurable), that span will be regarded as a rare span.
%     Subsequently, the frequency ratio across all nodes will be updated.
%     This approach helps in pinpointing spans that may indicate \red{unusual system events}.

%     \item \textit{Numerical attributes}: For numerical attributes such as duration, we apply a statistical outlier detection method, i.e., checking whether the value falls within the three-sigma range.
%     To this end, each node will maintain the current average ($\mu$) and standard deviation $\sigma$ of the attribute values, along with a cumulative count.
%     These parameters are updated dynamically as new spans are processed.
%     If a particular value significantly deviates from the norm, it will be flagged as a rare span.
%     This allows us to spot any abnormal variations in the data.
% \end{itemize}

% If the ratio of the total number of spans with the same \texttt{span\_name} to the total number of sampled spans is lower than \texttt{abnormal\_span\_name\_rate}, the span is considered rare.
% When iterating through the attributes of a specific span:
% \begin{itemize}
%   \item If an attribute is present in the Merging Tree (indicating a low ratio of spans to possible attribute values, i.e., lower than \texttt{threshold\_rate}), and the ratio of the total number of spans with the same \texttt{span\_name} to the frequency of occurrence of that attribute is lower than \texttt{abnormal\_attributes\_frequency\_rate}, the span is considered rare.
%   \item If an attribute is not present in the Merging Tree (indicating a high ratio of spans to possible attribute values, i.e., higher than \texttt{threshold\_rate}), and the attribute's data type is integer or floating-point, the attribute is treated as continuous. The average $\mu$ and standard deviation $\sigma$ of this attribute are calculated. If the attribute value is not within the interval $[\mu - 3\sigma, \mu + 3\sigma]$, the span is considered rare.
% \end{itemize}
\section{Experiments}
\subsection{Implementation Details}
\label{sec:implemenet}
\paragraph{Datasets} We use the Multiview Rendering Dataset \cite{qiu2023richdreamer,zuo2024sparse3d} based on Objaverse \cite{objaverse} for training. The dataset includes 260K objects, with 38 views rendered for each object, with a resolution of $512\times512$. To obtain the surface point clouds, we transform the 3D models according to the rendering settings, filter out those that are not aligned with the rendered images, and use Poisson sampling method\cite{poisson} to sample the surface.  We randomly split the final processed data into training and testing sets, with the training dataset consisting of 200K objects.
% We take our in-domain evaluation using the test set from Objaverse, including 2000 objects. To evaluate our model's cross-domain 能力, We 在Google Scanned Objects(GSO) dataset进行评估,which 包含1030个真实的扫描3D model, and we take 32 views renderd for each model in 球面. 
We conduct our in-domain evaluation using the test set from Objaverse, which includes 2,000 objects. To assess our model's cross-domain capabilities, we evaluate it on the Google Scanned Objects (GSO) \cite{downs2022google}dataset, which contains 1,030 real-world scanned 3D models, with 32 views rendered for each model on a spherical surface.

% 我们使用单图作为输入,以所有available的views作为noval views 来评测我们和所有比较方法的单图生成质量. And take the peak-signal-to-noise ratio (PSNR),
% perceptual quality measure LPIPS, structural similarity index (SSIM) 作为evaluation metrics, which is same to previous work\cite{zou2024triplane, chen2025lara}


% 在我们的实现中,anchor latents的fix length是2048,维度是8. the model dim of Anchor-GS VAE是512, with 两个transformer block in encoder and eight transformer block in decoder. For training the Anchor-GS VAE, we set the weights of the losses with $\lambda_s=1, \lambda_l=1, \lambda_c=1, \lambda_e=1 $ and $\lambda_{KL}=0.001$.
%
% For the Seed-Anchor Mapping Module, we use 24 transformer bolck to implement with a model dim 512, 其中 4 bolcks for downsample and 4 blocks for upsample. For the Seed Points Generation, we use 24 transfomrmer blocks to implement with model dim 512. And thanks to the seed points 的sparse 特性, we can directly learn the distribution of seed points 而不需要去train 一个 VAE. 
%
% We train the Anchor-GS VAE use only a subset of our collected datasets, with a batchsize of 128 on 8 40G A100 with 24K steps. For trainging the Seed-Anchor Mapping Module, we use our full collected datasets, with a batchsize 0f 1280 on 64 32G V100 with 20K steps. For training the Seed Points Generation Module, We training on 48 32G V100 with 54K steps.
% In our implementation, the anchor latents have a fixed length of 2048 and a dimension of 8, and the model dimension in our transformer blocks is 512, each transformer block has two attention layers and a feed-forward layer, similar to \cite{zou2024triplane}. The Anchor-GS VAE consists of two transformer blocks in the encoder and eight transformer blocks in the decoder. For training the Anchor-GS VAE, we random select 8 views,  one view as input and all 8 views as the ground truth images for supervision, and we set the loss weights as follows: \(\lambda_s = 1\), \(\lambda_l = 1\), \(\lambda_c = 1\), \(\lambda_e = 1\), and \(\lambda_{KL} = 0.001\).  


\paragraph{Network}
In our implementation, the anchor latents have a fixed length of 2048 and a dimension of 8. The model dimension in our transformer blocks is 512, with each transformer block comprising two attention layers and a feed-forward layer, following the design in \cite{zou2024triplane}. The Anchor-GS VAE consists of two transformer blocks in the encoder and eight transformer blocks in the decoder. 
%
The Seed-Anchor Mapping Module is implemented using 24 transformer blocks, with 4 blocks for downsampling and 4 blocks for upsampling. Similarly, the Seed Points Generation Module is implemented with 24 transformer blocks. Leveraging the sparsity of seed points, we directly learn their distribution without requiring a VAE. The image conditioning in our model is extracted using DINOv2\cite{oquab2023dinov2}.


\paragraph{Training Details}
For training the Anchor-GS VAE, we randomly select 8 views per object, using one view as the input and all 8 views as ground truth images for supervision. The loss weights are set as \(\lambda_s = 1\), \(\lambda_l = 1\), \(\lambda_c = 1\), \(\lambda_e = 1\), and \(\lambda_{KL} = 0.001\). We train the Anchor-GS VAE on a subset of our collected dataset containing approximately 40K objects, using a batch size of 128 on 8 A100 GPUs (40GB) for 24K steps. The Seed-Anchor Mapping Module is trained on the full dataset with a batch size of 1280 on 64 V100 GPUs(32GB) for 20K steps. The Seed Points Generation Module is trained on 48 V100 GPUs (32GB) for 54K steps.  We use the AdamW optimizer with an initial learning rate of \(4 \times 10^{-4}\), which is gradually reduced to zero using cosine annealing during training. The sampling steps for both the Seed-Anchor Mapping Module and the Seed Points Generation Module are set to 50 during inference.
% For training the Anchor-GS VAE, we randomly select 8 views per object, using one view as the input and all 8 views as ground truth images for supervision. The loss weights are set as follows: \(\lambda_s = 1\), \(\lambda_l = 1\), \(\lambda_c = 1\), \(\lambda_e = 1\), and \(\lambda_{KL} = 0.001\).
% We train the Anchor-GS VAE using a subset of our collected dataset around 40K objects with a batch size of 128 on 8 A100 GPUs (40GB) for 24K steps. The Seed-Anchor Mapping Module is trained on the full dataset with a batch size of 1280 on 64 V100 GPUs (32GB) for 20K steps. For the Seed Points Generation Module, we train on 48 V100 GPUs (32GB) for 54K steps. We use the AdamW optimizer with a learning rate of 4e-4, and the learning rate is cosine anneled to zero during training.


\paragraph{Baseline}
% We compared our methods with 之前的SOTA的3D生成模型 in one image input setting. LGM and LaRa use one image as input, then use multi-view diffusion models to get four views of the object, then get corresponding 3DGS from the multiview images in a feed-forard mamner. TriplaneGS first get dense point clouds from the single input image, then use a triplane to 聚合特征 then get the corresponding attributes of 3DGS, achieving SOTA performances.
We compared our method with previous SOTA 3DGS generation models in the single-image input setting. LGM and LaRA rely on 2D multi-view diffusion priors to obtain multi-view images, which are then used to generate the output 3DGS in a feed-forward manner, as described in ~\ref{sec:related-2d-diffusion}. TriplaneGS~\cite{zou2024triplane} does not require a 2D diffusion prior, directly generating 3DGS from a single input image, as outlined in ~\ref{sec:related_3d}. Both of them achieving SOTA performance. For each compared method, we use the official models and provided weights and ensure careful alignment of the camera parameters.

% Both  LGM~\cite{tang2025lgm} and LaRa~\cite{chen2025lara} take one image as input and then use multi-view diffusion models\cite{shi2023mvdream} to generate four views of the object. These multi-view images are subsequently converted into corresponding 3D Gaussian Splatting (3DGS) representations in a feed-forward manner. TriplaneGS~\cite{zou2024triplane}, on the other hand, first generates dense point clouds from the single input image and then aggregates features using a triplane representation to infer the corresponding attributes of 3DGS, also achieving SOTA performance. For each compared method, we use the official models and provided weights and ensure careful alignment of the camera parameters.\todo{remove duplicate description, mention LaRA is designed for four views input}
% compared methods






\subsection{Results of VAE Reconstruction}
In Fig. \ref{fig:vae}, we present the results of our Anchor-GS VAE. Given point clouds and a single image, our Anchor-GS VAE achieves high-quality reconstructions with detailed geometry and textures.



\subsection{Results of 3D Generation }
\label{sec:comparison}
% 1. 首先讲在哪些数据上进行评估。然后逐个分析结果的值,最后说我们的效果达到了SOTA的效果
% 2. 展示可视化的结果,再逐个分析。表明我们的方法相比于没有用diffusion的方法能更好的学习三维物体的分布。
% Table 1展示了我们的方法和previous SOTA methods在Objaverse和GSO上的评测结果。As described in \ref{sec:implemenet},评测在一个dense viewpoint settings下进行,the results are average use all available objects and viewpoints in the testing datasets. The 多视角不一致 in the multiview diffusion model used by LGM 和 LARA 会导致生成几何不一致的3DGS,特别是导致在新视角下的伪影,So 他们会产生相对较低的值在我们的dense viewpoint评估settings下。相比于他们,我们的方法不借助于2D diffusion先验,可以直接从单张图像中得到理想的3D表示。TriplaneGS使用通过tansformer block一次forward facing得到point clouds,但是这种方式往往不能准确学习到3D点云的分布,always failed in 输入图像中没有的区域。与之相比,我们的方法使用diffusion-based methods, 首先学习一个sparse 的seed points, which is easy to learn and can学习到3D的distribution,then mapping from seed points to anchor latents.SO get more 鲁邦的结果。
\paragraph{Metrics} 
Following previous works \cite{zou2024triplane, chen2025lara}, we use peak signal-to-noise ratio (PSNR), perceptual quality measure LPIPS, and structural similarity index (SSIM) as evaluation metrics to assess different aspects of image similarity between the predicted and ground truth. Additionally, we report the time required to infer a single 3DGS. We use a single image as input and evaluate the 3D generation quality using all available views as testing views to compare our method with others, all renderings are performed at a resolution of 512.

Tab. \ref{tab:quantitative comparison} presents the quantitative evaluation results of our method compared to previous SOTA methods on the Objaverse and GSO datasets, along with qualitative results shown in Fig. \ref{fig:image-3d}. The multi-view diffusion model used in LGM tend to produce more diverse but uncontrollable results, and lacks precise camera pose control. As a result, it fails in our dense viewpoints evaluation, achieving PSNR scores of 12.76 and 13.81 on the Objaverse and GSO test sets, respectively.

As shown in Tab. \ref{tab:quantitative comparison}, LGM and LaRa, influenced by the multi-view inconsistency of 2D diffusion models, achieve relatively lower scores in our dense viewpoint evaluation. In contrast, our method achieves the best results across both datasets, with only a slight overhead in inference time.

Fig. \ref{fig:image-3d} presents the first six rows from the Objaverse dataset and the last three rows from the GSO dataset. All methods are compared using the same camera viewpoints. For the Objaverse dataset, the rendering viewpoints are the left and rear views relative to the input viewpoint, while for the GSO dataset, the views are selected to showcase the object as completely as possible. Compared to methods using 2D diffusion priors, such as LGM and LaRa, our method demonstrates better multi-view geometric consistency, while the former tends to generate artifacts or unrealistic results in our displayed views. Compared to TGS, our method learns the 3D object distribution more effectively, resulting in more geometrically consistent multi-view results, such as the sharp feature in the left view in the first knife case.
% Compared to these methods, 我们的方法能取得更multiview geometry consistent的结果. 例如所有的方法都在第一个case中不能正确的表示输入图像中的sharp feature,导致在所有视图中都是近似的结果.

% As described in Sec. \ref{sec:implemenet}, the evaluations are conducted under a dense viewpoint setting, with the results averaged over all available objects and viewpoints in the testing datasets. The multiview diffusion models used by LGM \todo{add lgm fail reason: Image dream can't control viewpoint}and LaRa exhibit inconsistencies across viewpoints, resulting in geometrically inconsistent 3D Gaussian Splatting (3DGS) representations. This inconsistency particularly manifests as artifacts when rendering from novel viewpoints, leading to relatively lower performance under our dense viewpoint evaluation setting. In contrast, our method does not rely on 2D diffusion priors and directly generates a high-quality 3D representation from a single input image. 

% TriplaneGS employs a Transformer-based approach to predict dense point clouds in a single forward pass. However, this approach can face challenges in accurately capturing the 3D distribution of points, particularly in regions not visible in the input image, which may lead to less optimal performance in some cases. In comparison, our method adopts a diffusion-based strategy, first learning a sparse set of seed points. This approach simplifies the learning process, allowing the model to better capture the underlying 3D distribution. The seed points are then mapped to anchor latents, resulting in more robust and consistent outcomes.

\subsection{Editing Results Based on Drag}
% The drag results are presented in Fig. \ref{fig:edit}.
As shown in Fig. \ref{fig:edit}, our method enables Seed-Points-Driven Deformation. Starting with generated seed points from the input image, the sparse nature of the seed points allows for easy editing using 3D tools (e.g., Blender\cite{blender}) with a few drag operations. The edited 3DGS can then be obtained within 2 seconds.
% 如Fig. \ref{fig:edit}.所示,我们的方法可以进行Seed-Points-Driven Deformation. 对于一个generated seed points from input image,
% 由于seed points稀疏的特性,我们可以很方便借助3D编辑工具(blender)使用有限的几次Drag操作对seed points进行编辑,and以2s时间得到编辑后的3DGS.
\subsection{Ablation Study}
% \paragraph{coarse2fine in vae}
% \paragraph{two-stage generation}
% \todo{not finish here}
\paragraph{Seed Points Generation}
% We use the Recitified flow model to learn the generation of seed points with the conditon of single input image. Due to the sparse of seed points, the flow model is easy to learn and 可以很好的学到seed points' distribution. We also 实现this module using transformer block 使用一次feed forward的方法从learnable embeddings 中得到point cloud,like \cite{zou2024triplane}. As shown in Fig. \ref{fig:ablation-seed-gen}, the Feed-forward method failed to learn the distribution of the seed points, 在图片上不可见的区域无法生成理想的结果。
We employ a Rectified Flow model to generate seed points conditioned on a single input image. Owing to the sparsity of the seed points, the flow model is easier to train and effectively learns the distribution of the seed points. However, we also explored an alternative implementation using a transformer-based feed-forward approach, where point clouds are generated from learnable embeddings in a single pass, as in \cite{zou2024triplane}. As demonstrated in Fig. \ref{fig:ablation-seed-gen}, the feed-forward approach struggles to capture the true distribution of seed points and fails to produce satisfactory results in regions not visible in the input image.


\paragraph{Dimension Alignment}
% 为了让Seed-Anchor Mapping Module 的起点和target具有相同的维度,我们将Seed points 通过VAE的encoder based on Eq. \ref{eq:encode_seed_latents}. 这可以保证Mapping的起点的分布和终点的分布更接近,从而降低了Seed-Anchor Mapping Module的学习难度并且避免了Mapping时对image condition的过度依赖. And the alignment bewteen points and image achieved by projection in encoder is vivtal in the Seed-Points-Driven Deformation, as we can change the position of draged seed points while preserve it's correponding projed feats.
To match the dimension of the starting and target points in the Seed-Anchor Mapping Module, we encode the seed points using the Anchor-GS VAE encoder (Eq. \ref{eq:encode_seed_latents}). This process brings their distributions closer, reducing learning difficulty and reliance on image conditions. 
% Additionally, the projection-based alignment between points and the image in the encoder is critical for Seed-Points-Driven Deformation, enabling position adjustments of dragged seed points while preserving their projected features, as shown in Eq.\ref{eq:edit_seed_encode}. 
To validate this method, we conducted experiments by replacing the encoding approach with positional encoding .  When using positional encoding, the Seed-Anchor Mapping overly relied on the image condition, neglecting the contribution of the seed points and failing to enable seed-driven 3DGS deformation, as shown in Fig. \ref{fig:ablation-seed-enc}. 

% When positional encoding is used, the Seed-Anchor Mapping overly relies on the image condition, neglecting the true geometric state of the seed points and failing to achieve seed-driven 3DGS deformation.
% When using positional encoding, the Seed-Anchor Mapping overly relied on the image condition, 忽略了seed points真实的几何状态 and failing to enable seed-driven 3DGS deformation. 
% To validate this method, we conducted experiments by independently testing two variations: replacing the encoding approach with positional encoding and removing the projection of seed points onto the input image during encoding (Fig. \ref{fig:ablation-seed-enc}).  When using positional encoding, the Seed-Anchor Mapping overly relied on the image condition, neglecting the contribution of the seed points and failing to enable seed-driven 3DGS deformation. Separately, without projection-based alignment, the Mapping Module failed when the seed points and the input image were misaligned under the given viewpoint.
% 为了验证这个方法的有效性,我们测试了将基于encoder of anchor-GS VAE的方法换成positional encoding 和 在encode时不采用seed points与输入图像的投影。结果如图所示。在采用positional encoding时,the Seed-Anchor Mapping 将会过度依赖于image condition,导致忽略了从start points本身出发,从而无法实现基于seed points 的3DGS deformation. And lack of the projection-based alignment, the Mapping Module 会在seed points与input image 在给定视角下不一致的情况下fail.


\paragraph{Token Alignment}
We ensure token alignment in Flow Matching by organizing tokens around seed points, followed by  cluster-based partition and repetition. To evaluate its effectiveness, we conducted two ablation experiments, as shown in Tab. \ref{tab:ablation-tokenalign}. In the \textit{No-cluster+No-repetition} setting, we omitted the clustering step, aligning only the corresponding seed and anchor latents while filling unmatched portions with noise. This also prevented cluster-based downsampling in the Flow Model, resulting in doubled memory consumption. In the \textit{No-cluster} setting, we repeated the seed latents to match the number of anchor latents but left them unordered, leading to disorganized token matching. As shown in Tab. \ref{tab:ablation-tokenalign}, on a 40K subset with the same number of training steps, the absence of token alignment significantly degraded Flow Matching performance, resulting in inaccurate correspondences.
% We ensure token alignment in Flow Matching by organizing tokens around seed points, followed by repetition and cluster-based rearrangement. To validate its effectiveness, we conducted two ablation experiments. In the first, we repeated seed latents to match the number of anchor latents but left them unordered, leading to disordered token matching. In the second, we aligned only the corresponding seed and anchor latents, filling the unmatched portions with noise. Without cluster-based rearrangement, downsampling in the Rectified Flow Model became impossible, doubling memory consumption. Tab. \ref{tab:ablation-tokenalign} shows that on a 40K subset, with the same number of training steps, flow matching performance is significantly degraded without token alignment, failing to produce accurate correspondences.

% We ensure token alignment in Flow Matching by organizing tokens around seed points, followed by repetition and cluster-based rearrangement. To validate its effectiveness, we conducted two ablation experiments, as shown in Tab. \ref{tab:ablation-tokenalign}. In the No-cluster, 我们不再进行分cluster,aligned only the corresponding seed and anchor latents, filling the unmatched portions with noise. And we can't do cluster-based downsample in the Flow Model, 这会导致 memory consumption double. In the No-rearrange, we repeated seed latents to match the number of anchor latents but left them unordered, leading to disordered token matching. Tab. \ref{tab:ablation-tokenalign} shows that on a 40K subset, with the same number of training steps, flow matching performance is significantly degraded without token alignment, failing to produce accurate correspondences.


% we repeated seed latents to match the number of anchor latents but left them unordered, leading to disordered token matching. In the second, we aligned only the corresponding seed and anchor latents, filling the unmatched portions with noise. Without cluster-based rearrangement, downsampling in the Rectified Flow Model became impossible, doubling memory consumption. Tab. \ref{tab:ablation-tokenalign} shows that on a 40K subset, with the same number of training steps, flow matching performance is significantly degraded without token alignment, failing to produce accurate correspondences.
% Token alignment 保证了Flow matching 中start point和end point中token数目和对应位置上的语义是匹配的,使用以Seed points为center的方式来组织token,并进行repeat和rearrange。我们设计了消融实验来验证这个模块的必要性和有效性,第一个对照实验是在对seed latets repeat到和anchor latents相同的个数后,我们不对anchor latents进行重排序, so the token之间的匹配关系是杂乱的。第二个对照实验是我们仅仅将对应的seed latents和anchor latents进行对齐,其余无法对齐的部分使用noise进行填充。并且由于没有进行cluster based rearrange,我们无法在Rectified flow Modle中进行downsaple, which 增加了两倍的计算时的显存消耗。在一个40K subset上 经过同样的training steps后,结果如表所示。Without the token alignment, the Flow matching的效果大打折扣,无法得到理想的对应的结果。
% We ensure token alignment in Flow Matching by organizing the tokens around seed points and performing repeating and rearranging operations. This guarantees that the number of tokens and their semantic correspondence between the start and end points are aligned. To validate the necessity and effectiveness of this module, we designed ablation experiments. The first experiment involved repeating the seed latents to match the number of anchor latents but without reordering the anchor latents, resulting in a disordered matching relationship between tokens. The second experiment aligned only the corresponding seed latents and anchor latents, filling the unmatched portions with noise. Without the cluster-based rearrangement, we were unable to downsample in the Rectified Flow Model, which increased the memory consumption during computation by a factor of two. After training on a 40K subset for the same number of steps, the results, shown in the table, indicate that without token alignment, the performance of flow matching is significantly degraded, failing to achieve the desired correspondence.


% Table
\begin{table}%
\caption{ Quantitative evaluation of our method compared to previous work. $\dagger$ achieves relatively lower PSNR values in the evaluation, so we display the results in Sec. \ref{sec:comparison}.}
\label{tab:quantitative comparison}
% \begin{minipage}{\columnwidth}
\resizebox{0.5\textwidth}{!}{
% \begin{center}
\begin{tabular}{llllllll}
  \toprule
  \multirow{2}{*}{Method}  & \multicolumn{3}{c}{Objaverse\cite{objaverse}}   & \multicolumn{3}{c}{GSO\cite{downs2022google}}& \multirow{2}{*}{Time(s)}\\
% \cline{2-4}   \cline{5-7} \cline{8-10} \cline{11-13}
\cmidrule(r){2-4}  \cmidrule(r){5-7} 
   & PSNR$\uparrow$& SSIM$\uparrow$& LPIPS$\downarrow$ & PSNR$\uparrow$& SSIM$\uparrow$& LPIPS $\downarrow$
   \\ \midrule
  LGM$\dagger$\cite{tang2025lgm}     & -&0.836&0.211&-&0.833&0.21&4.82\\
  LaRa\cite{chen2025lara}  & 16.57&0.860&0.174&15.98&9.869&0.162&9.50\\
  TriplaneGS\cite{zou2024triplane}  &18.80 &0.883&0.143&19.84&0.900&0.120&0.70\\
  Ours &20.92&0.896&0.120&20.52&0.904&0.1122&4.71\\
  \bottomrule
\end{tabular}
% \end{center}
}
\end{table}%


\begin{table}%
\caption{Ablation about token alignment}
\label{tab:ablation-tokenalign}
\begin{minipage}{\columnwidth}
\begin{center}
\begin{tabular}{llll}
  \toprule
   & PSNR$\uparrow$& SSIM$\uparrow$& LPIPS$\downarrow$ 
   \\ \midrule
  No-cluster+No-repetition  & 18.84&0.877&0.141\\
  No-cluster     & 19.20 &0.876&0.142\\
  ours-full  &19.94 &0.881&0.134\\
  \bottomrule
\end{tabular}
\end{center}
\bigskip\centering


\end{minipage}
\end{table}%

%figure
\begin{figure}
  \includegraphics[width=\linewidth]{figs/ablation_seed.pdf}
  \caption{Ablation study about different seed points geneartion methods: (a) using our method, (b) using Transformer.}
  \label{fig:ablation-seed-gen}
\end{figure}

\begin{figure}
  \includegraphics[width=\linewidth]{figs/ablation_enc.pdf}
  \caption{Without Dimension Alignment, seed-points-driven deformation fails}
  \label{fig:ablation-seed-enc}
\end{figure}



\section{Discussion}
\label{sec:discussion}

The results provide valuable insights into the limitations of machine learning (ML) models to support systematic literature review (SLR) updates. In this discussion, we interpret these results in light of the research questions, contextualize their implications, and outline the trade-offs associated with applying ML models in this domain.

\subsection{Effectiveness of ML Models for SLR Study Selection (RQ1)}

The results for RQ1 indicate that our best-performing model, Random Forest (RF), achieved a modest balance between precision and recall with an F-score of 0.33 at the default threshold of 0.5. This result suggests that while the ML model was able to identify some relevant studies, its overall ability to precisely distinguish between relevant and irrelevant studies was limited. Adjusting the threshold improved the F-score to 0.41, highlighting the sensitivity of the model’s performance to the chosen threshold. However, this improvement came at the cost of increasing false negatives (FNs), potentially missing valuable studies. We interpret the RF model’s performance as indicating that ML may assist in informally identifying a subset of relevant studies but is not yet reliable for the selection of studies for SLR updates.

\subsection{Effort Reduction through ML Models (RQ2)}

In answering RQ2, we focused on maximizing recall to avoid FNs. In our investigations, the SVM model was more suitable for focusing on achieving a high recall and demonstrating some potential for reducing human screening efforts. Results demonstrated that with a recall of 100\%, the SVM model could exclude 33.9\% of studies from the review process without missing any relevant studies. This reduction represents a significant decrease in the manual workload, suggesting ML’s potential to assist researchers with the initial screening stage. However, to achieve this high recall, the model produced a high rate of false positives (FPs), still requiring significant human review effort to discard many non-relevant studies.

As shown in Table \ref{tab:effort_reduction}, gradually increasing the inclusion probability threshold reduced the number of FPs at the cost of a minor drop in recall. For instance, at a threshold of 0.75, the model achieved a recall of 97.37\%, with a reduction of 48.3\% in the number of studies needing review. We interpret this result as indicating that, while ML can reduce screening efforts, care must be taken when applying thresholds to avoid introducing a risk of overlooking critical studies.

\subsection{Supporting Human Reviewers (RQ3)}

For RQ3, we evaluated the support ML could provide compared to that of an additional human reviewer. When we treated the RF model as an additional reviewer and calculated Euclidean Distance (ED) to assess alignment with the final inclusion decision, individual human reviewers outperformed the RF model. Furthermore, pairs of human reviewers clearly outperformed human-ML pairs, suggesting that human-only review teams achieve more accurate results.

This finding reinforces the challenges ML models face in fully replicating the nuanced judgment of human reviewers. Hence, ML can not replace additional human reviewers, and ML assistance is not a valid argument for quality in the selection process. Pairs of human reviewers are still highly recommended for selecting studies in SLR updates.
\section{Threats to Validity}
\label{sec:threats}

In the following, we enumerate the main threats to the validity of our study, using the categories suggested by~\cite{Runeson12}.

\textbf{Construct Validity.}
Our evaluation results might have been affected by the choice of the SLR update and of the ML algorithms. Regarding the chosen SLR update, it is very difficult to get access to details such as individual assessments by reviewers during the initial screening process, which we needed for our analyses. Our SLR update dataset had such detailed information for 551 studies and is available online~\cite{zenodoOpenScience}. Regarding the algorithms, we analyzed the most used ones for text classification~\cite{pintas2021feature} and dug deeper into the two that showed the most prominent initial evaluation results on our dataset. 

\textbf{Internal Validity.}
Our training dataset comprised only studies included in the SLR replication \cite{Wohlin2022} (training included) and those obtained through backward snowballing (training excluded). We deliberately excluded studies not in English or those categorized as Ph.D. dissertations or book chapters from our testing set, the same criteria adopted by the SLR. A potential threat to internal validity that could have favored human reviewers is that, during the manual initial screening process, while this was not part of the procedure, the human reviewers could have ended up reading other sections of the studies besides the title, abstracts, and keywords.

\textbf{External Validity.} The dataset used in our analysis might not represent the diversity of SLR updates in SE. However, we did not find other SLRs with available data on the individual assessments applied during the initial screening. Replicating the investigations on other SLR updates to strengthen external validity would require significant effort for which we would have to involve the wider community. While not claiming external validity, we believe that sharing our initial evaluation results can already provide some valuable insights.

\textbf{Reliability.} The data used in our evaluation, including the individual initial screening assessments and the final list of papers to be included in the SLR update, was generated by the same (first three) authors who performed the SLR replication~\cite{Wohlin2022}. In addition, to improve the reliability of our results, our ML models and the evaluation datasets are openly available and auditable. 
In this work, we introduce \ours, a skill chaining model designed for long horizon dexterous manipulation. 
Given a task decomposed into $N$ skills, \ours trains $N+1$ heads: $N$ heads to learn individual skills and an additional head for skill progress estimation. 
Based on the estimated progress values, a skill progress guided skill selector \progss chooses the proper skill to execute at each time step. 
Qualitative results demonstrate that \progss effectively adapts to unexpected disturbance.
Comprehensive experiments in both simulation and real world settings reveal the performance advantages of \ours over the single-head Octo baseline, as well as its capability to handle various skill sequences and diverse object sets.



\section*{Acknowledgment}

We express our gratitude to CNPq (Grant 312275/2023-4), FAPERJ (Grant E-26/204.256/2024), and Stone Co. for their generous support.

\bibliographystyle{IEEEtranS}
\bibliography{bibTex/sigproc} 

\end{document}

% \maketitle



% %--------------------------------------------------------
% \section{Background and Related Work}
% \label{sec:relatedwork}

% %SLR update definition
% An SLR update is a more recent (updated) version of an SLR that includes new evidence (primary studies) \cite{Mendes2020}. For the inclusion of new and relevant evidence, one of the initial steps is to conduct the study selection activity which consists of analyzing the retrieved studies from the search process to evaluate the need and to perform the SLR update. 

% %related work
% As mentioned before, there are several initiatives in SE towards improvements for SLR update (e.g. \cite{felizardo16, Garces17, Mendes2020, Wohlin2020}). However, considering the focus of our study on automation to select studies for SLR updates, we highlighted three main related works \cite{Watanabe20, Felizardo14, Napoleao2021} described in the following.

% The work of Watanabe \textit{et al.} \cite{Watanabe20} also evaluated the use of text classification (text mining combined with ML Models) to support the study selection activity for SLR updates in SE. They performed an evaluation with 8 SLRs from different research domains in a cross-validation procedure using Decision Tree (DT) and SVM as ML classification algorithms. The results achieved on average a \textit{F-score} of 0.92, \textit{Recall} of 0.93 and \textit{Precision} of 0.92. Unlike the approach proposed in \cite{Watanabe20}, our study evaluates the ML Models SVM and RF using a detailed database of a solid ongoing SLR update conducted by renowned researchers in the field of EBSE. Furthermore, we compare the agreement level of the adopted ML Models with the expert reviewers through \textit{Kappa} analysis \cite{Cohen10, Kitchenham15}.

% %While the contributions of Watanabe et al. \cite{Watanabe20} laid a valuable foundation by showing the potential of supervised ML Models in assisting the study selection process for SLR updates, our study takes a different approach to enhance the applicability of such methodologies. Our study used a detailed database of a solid ongoing SLR update conducted by renowned researchers in the field of EBSE, we  also used a distinct approach to retrieve the non-selected studies from the original SLR, and we explored different Feature Selection (FS) strategies, ML Models and techniques to train our classifiers. Furthermore, our study's contribution goes beyond evaluating how much effort could be reduced by the ML classifiers during the selection of studies process, we also  compared in detail the agreement level of the adopted ML classifiers with the expert reviewers through \textit{kappa} analysis \cite{Kitchenham15}.

% Felizardo \textit{et al.} \cite{Felizardo14} also proposes an automated alternative to support the selection of studies for SLR updates. The authors proposes a tool called Revis which links new evidence with the original SLR's evidence using the K-Nearest Neighbor (KNN) Edges Connection technique. The tool output is presented in two distinct visualizations, a content map and an Edge Bundles diagram. The results showed an increase in the number of studies correctly included compared to the traditional manual approach.

% Napoleão et al. \cite{Napoleao2021} performed a cross-domain Systematic Mapping (SM) on existing automated support for searching and selecting studies for SLRs and SMs in SE and Medicine. The authors indicated potential ML Models that can be adopted to support the study selection activities. They also indicated the most adopted methods (cross-validation and experiment) and metrics (\textit{Recall}, \textit{Precision} and \textit{F-measure}) to assess text classification approaches. The choice of the ML Models and the assessment metrics and methods for this study are considered finds of this work.


% \section{Goal and Research Questions}
% \label{sec:researchissues}

% Our goal is to evaluate the adoption of ML Models to support the selection of studies for SLR updates. We translated our goal into three different research questions (RQs).

%     \textbf{RQ1:} \textit{How effective are ML Models in selecting studies for SLR updates?}

%     We represent the effectiveness of the ML models in supporting the selection of studies activity using metrics such as \textit{Recall}, \textit{Precision} and \textit{F-measure} \cite{Napoleao2021, Watanabe20}. Our ML automated analysis considers only title and abstract of the studies. Our ML models results are compared with the included studies selected manually for the SLR update under evaluation. 
    
%     %Our ML automated analysis considers only title and abstract of the studies and the metrics are calculated at first considering the results from the expert reviewers analysis only on the title, abstract and keywords and next, considering also their results from the full-text analysis.

%     \textbf{RQ2:} \textit{How much effort can ML Models reduce during the study selection activity for SLR updates?}

%     We calculate the effort reduction by the relation of the number of studies that will need to have their title, abstract and keywords manually analyzed without the support of ML Models versus the number of studies to be analyzed after the use of the ML solution.
   
%     % Justificar o recall diferente de 100 (relatar o caso 100 e os casos 99)
%     % Mostrar um gráfico exibindo as variações de Falso Negativo / Positivo variando o Recall

%     \textbf{RQ3:} \textit{Can Machine Learning replace a reviewer in the selection of studies for SLRs?}
%     % We compared the agreement level of the ML Model with the highest \textit{F-score} value, supporting a single reviewer with the agreement level of each pair of reviewers, by calculating their Cohen's \textit{Kappa} coefficient \cite{Cohen10, Kitchenham15}.

%     We compared the levels of agreement and similarity of the ML Model with the highest \textit{F-score} value by performing two different analysis. For the agreement analysis, we used the Cohen's Kappa coefficient to measure the level of concordance between the ML Model and reviewers \cite{Cohen10, Kitchenham15}. For the similarity analysis, we used the Euclidean Distance to measure the distance between the ML Model and reviewers from different perspectives \cite{SERRA2014305}, to verify if the ML model had a higher similarity with any reviewer in comparison to the similarity within the assessment team, if the ML model had a higher similarity with the final than any reviewers, if the ML model decreased the distance from the final results when combining its answers with a single reviewer in comparison to other pairs of reviewers and if the ML model decreased the distance from the final results when combining its answers with two reviewers in comparison to the assessment team. The last evaluations intended to simulate the ML model working together with other reviewers during the agreement criteria from the selection of studies.  

%     % \end{itemize}

%     %---- Here I did not defined Kappa. I think we can defined it in the methodology section.

%     %Kappa analysis ->  Euclidean Distance
%     %agremeent level do algoritmo com os revisores - titulo, abstract and keywords
%     % future assessement + Hipótese :  Machine learning can replace a reviewer in a SLR update?

% %--------------------------------------------------------
% \section {Study Design}
% \label{sec:methodology}

% %Bianca: Aqui estava faltando definir qual o research metodo que a gente utilizou para testar o que foi desenvolvido. Eu utilizei a ideia do small-scale evaluation, visto que segundo o trabalho do Wohlin o que fizemos nao é um estudo de caso, nem um experimento. 

% % 5 steps Runeson (referencia/exemplo)
% % We follow the five main steps for conducting case studies
% % proposed by [21]: Design, preparation, collecting data, analysis
% % and reporting.

% % TODO: Explicar o Precison/Recall nessa seção e não nos Resultados -- %Bianca: Eu coloquei uma versao de definicao, veja o que acha. 

% In this Section, we present the key aspects of the study design. In order to evaluate our proposition, we performed a small-scale evaluation \cite{Wohlin2022cs}. According to the smell indicator proposed by Wohlin \& Rainer \cite{Wohlin2022cs}, the correct label for our evaluation is small-scale evaluation instead of a case study. In order to guide and report our study design, we divided our study design into two main parts: (i) Data Collection and (ii) Design \& Execution. 

% In Section \ref{subsec:data} we describe the  data collection process used in our small-scale evaluation to train and test our ML models. In Section \ref{subsec:studydesing} we detail our proposed solution developed to train and configure the investigated ML models. 

% \subsection{Data Collection}
% \label{subsec:data}

% \begin{figure*} [ht]
%     \centering
%     \includegraphics[width=400pt]{pictures/latest/fig03-data-acquisition-v2.pdf}
%     \caption{Data collection process}
%     \label{fig:fig-data-selection}
% \end{figure*}

% We used as instrument of our small-scale evaluation an ongoing SLR update of \cite{Wohlin2022} conducted by the same authors of this replication (team assessment). We chose this ongoing SLR update since the inclusion and exclusion of new studies were conducted based on individual assessments and the consensus of three experienced SLR researchers by analysing title, abstract, keywords and then full-text of the studies manually, allowing us to have confidence in this data for building reliable training and testing sets.

% The team assessment provided us all the studies they analyzed during the SLR update (.bib files), a total of 591 references, of which 39 were included and 552 were excluded for the update. We used these studies to form our testing set for our ML models, we filtered the studies to consider only first studies in English with a valid abstract. At the end, we used 551 studies in our testing set, of which 38 were included by the team assessment and 513 were excluded for the update.

% To train our ML models, we used a training set with 128 studies, of which 45 studies were included and 83 were excluded. The 45 studies used to train our models with what should be included were the same studies included in the original SLR. Since the team assessment did not list the studies that were excluded during the study selection phase of the original SLR, we performed a backward snowballing on the original references to obtain the 83 studies used to train our models with what should be excluded. Figure \ref{fig:fig-data-selection} summarizes this process.

% \subsection{Design \& Execution}
% \label{subsec:studydesing}

% We developed a pipeline with the following steps to automate the study selection process of an SLR update by using ML and answer our research questions. Our pipeline is illustrated in Figure \ref{fig:fig-study-design}. 

% \begin{figure*} [hb]
%     \centering
%     \includegraphics[width=1.0\linewidth]{pictures/latest/fig-pipeline-details-v2.pdf}
%     \caption{Study design pipeline}
%     \label{fig:fig-study-design}
% \end{figure*}

% In summary, our pipeline process a set of .bib files containing the list of studies to train the ML models and the list of studies to be analyzed. After completing its execution, it returns a report file in .xlsx format informing which studies should be included and excluded, as well as metrics about the predictions made by the ML model and the configuration that was used to run its execution.
% The pipeline must receive four different .bib files as input, one file containing the list of studies that should be excluded and one file containing the list of studies that should be included for each set (training and testing). In case there are any errors in the input files, the pipeline will stop its execution and will inform which entry was associated to each error as well as the type of error. Currently, our pipeline doesn't support automatic error resolution for invalid .bib files, they need to be manually fixed by the user.

% As shown in Figure~\ref{fig:fig-study-design} we firstly validated the .bib files of our testing and training sets to ensure completeness of the set avoiding duplicated entries or missing keys. Each study entry must have a title, the year of publication, an abstract text and a list of authors. 
% Secondly, we to applied text filtering techniques with Natural Language Processing (NLP) \cite{NLTK}, such as Lemmatization and Tokenization, to remove irrelevant characters from the texts. Thirdly, we applied Text Vectorization on the filtered texts using  Term-frequency/Inverse-Document-Frequency (TF/IDF), a technique that transforms text data into a numerical matrix of features. Fourthly, we used statistical methods to compute and select the most relevant features. In the fifth step, we trained and tuned our ML Models using our training set. Finally, in the last step, we used our ML Models to predict which studies of our testing set should be included and excluded and compared the results of each one and the agreement level in comparison with the team assessment.

% Additionally, an optional .env file can be passed as input to our pipeline, this file allows some steps in our pipeline to use a specific configuration, such as choosing the configuration of the Feature Selection (FS) method to compute the features, as well as the number of features to be selected in step four, and choosing the configuration for the ML models regarding which algorithm to be used, or which metric should be targeted when tuning the model as well as the type of cross validation to be performed, in step five. All parameters that can be configured are also illustrated in Figure~\ref{fig:fig-study-design}.

% % Based on the work of Napoleão \textit{et al.} \cite{Napoleao2021} which indicates the most used ML classifiers for assisting the selection of studies of SLRs and on the contributions of Pintas \textit{et al.} \cite{pintas2021feature} that evaluated the most adopted ML classifiers and Feature Selection (FS) techniques for text classification, we chose to evaluate two of them: Support Vector Machines (SVM) and Random Forest (RF).

% Based on the promising results achieved using SVM in the work of Watanabe \textit{et al.} \cite{Watanabe20} and the work of Napoleão \textit{et al.} \cite{Napoleao2021}, which highlighted SVM as one of the most used ML classifiers for assisting the selection of studies in SLRs, we decided to evaluate SVM in our work. Additionally, considering the work of Pintas \textit{et al.} \cite{pintas2021feature}, which evaluated the most adopted ML classifiers and Feature Selection (FS) techniques for text classification and concluded that the five most used classifiers are SVM, NB, KNN, DT, and RF, we performed initial tests using these classifiers. Our first tests showed that SVM and RF were achieving better results than the others. Therefore, we decided to focus our evaluation on these two classifiers: Support Vector Machines (SVM) and Random Forest (RF).

% % We experimented multiple configurations of our pipeline and evaluated different configurations for Feature Selection and for training and tuning of our ML classifiers. To select the best features, we tested different statistical methods such as Chi-squared (Chi2) \cite{Chi2}, Pearson Correlation \cite{pearson_r} and Analysis of Variance (Anova-F) \cite{ANOVA}. We tested different techniques to tune our ML classifiers such as K-fold cross-validation, Times-Series cross-validations and hyperparameter tuning with GirdSearch \cite{GridSearch}.

% We experimented multiple configurations of our pipeline and evaluated different configurations for FS and for training and tuning of our ML classifiers. During step four to compute the best features, we tested different statistical methods such as Chi-squared (Chi2) \cite{Chi2}, Pearson Correlation \cite{pearson_r} and Analysis of Variance (Anova-F) \cite{ANOVA} as well as a different range of features. We also tested different techniques to tune our ML classifiers such as K-fold cross-validation, Time-Series cross-validations and hyperparameter tuning with GirdSearch \cite{GridSearch}.

% For each evaluation, we executed the pipeline from start to finish in a clean environment using one statistical method at a time. To avoid introducing bias, the feature selection step was conducted solely based on the training set texts. Once the best features were identified in the training set, the same feature set was applied to the testing set to ensure consistency.
% To prevent overfitting, our machine learning classifiers were trained using a single type of cross-validation in each evaluation. Specifically, when utilizing GridSearch for parameter tuning, we didn't perform any other cross-validation technique, as GridSearch inherently includes cross-validation for measuring the most efficient parameter configuration. We chose this approach to maintain evaluation coherence and rigor.


% %--------------------------------------------------------
% \section{Results}
% \label{sec:results}

% In this section, we present the results of our study, following the research questions presented earlier in Section \ref{sec:researchissues}. To answer each question, we analyzed the results of the ML Models and then compared them with the results obtained manually by the SLR update authors under evaluation. 

% To answer questions RQ1 and RQ2, we reproduced the same evaluation performed by SLR update authors during the selection of studies for the SLR update by using our classifiers to predict  which studies should be included and excluded. For RQ1 we configured our model maximizing F-score and for RQ2 we configured our model maximizing Recall. Then we performed the agreement and similarity analysis using the predictions made by our model used in RQ1.

% Precision indicates how accurate the positive predictions made by the model are, while Recall indicates how well the model captures all the actual positive instances. For Precision, values closer to 1 indicate a lower number of False Positives (FP) results and values closer to 0 indicate a higher number of FP. And for Recall, values closer to 1 indicate a lower number of False Negatives (FN) and values closer to 0 indicate a higher number of FN.
% % Jogar para dentro da secao das RQs se for util

% We conducted our evaluation by varying the number of best features to be considered in each execution. After applying Text Filtering and Text Vectorization techniques, presented in steps three and four of our pipeline, our training set comprised a total of 23630 features, in contrast to our testing set, which comprised 119560 features. Given that the number of features in our testing set was more than five times greater than our training set, maximizing the number of best features in our training set was crucial to the performance of our ML models.

% We identified the range with the most relevant features in our training set as 900 to 1500 features, which was the range used in most of our evaluations. Notably, the best results, both in terms of F-score (RQ1) and Recall (RQ2), were consistently achieved with experiments that selected the 1200 best features.
% % Mover para RQ3

% % In the original table provided by the team assessment, besides what studies were included and excluded for the update, it also had the opinion of each reviewer about each study during the study selection process they performed. The reviewers could express their opinions about each study in three different level of certainty: 0 -- certain that the study should be excluded, 1 -- uncertain if the study should be excluded or included and 2 -- certain that the study should be included. We used this information to answer RQ3 and perform the \textit{Kappa} analysis.
% The document provided by the team assessment contained the final result of each study (if they were included or excluded), and also had the opinion of each reviewer about each study during the study selection process they performed. The reviewers could express their opinions in three different levels of certainty: 0 – certain that the study should be excluded, 1 – uncertain if the study should be excluded or included and 2 – certain that the study should be included. We used this information to answer RQ3 and perform the agreement and similarity analysis. Table~\ref{tab:slr-update-results-sample} illustrates the format of this document. 
% % TODO: Add table

% \begin{table}[!h]
% \caption{Example of the document format provided by the assessment team.}
% \tiny
% \begin{center}
% \begin{tabular}{ m{1.8cm} m{0.95cm} m{0.95cm} m{0.95cm} m{0.95cm} } 
%  \hline 
%   \textbf{Study} & \textbf{Final Result} & \textbf{R1} & \textbf{R2} & \textbf{R3} \\
%  First Study & 1 & 2 & 1 & 2 \\ 
%  Second Study & 0 & 2 & 0 & 0 \\ 
%  Third Study & 1 & 2 & 1 & 0 \\ 
%  Fourth Study & 0 & 0 & 2 & 1 \\ 
%  Fifth Study & 0 & 0 & 0 & 0 \\
%  \end{tabular}
% \end{center}
%  \label{tab:slr-update-results-sample}
% \end{table}


% We displayed the complete information of our results for the best configurations we found for RQ1 and RQ2, as well as all the other tests executions in an \textit{Appendix document}\footnote{https://zenodo.org/records/11021614} available online.

% \subsection{RQ1: \textit{How effective are ML Models in selecting studies for SLR updates?}}
% \label{results:RQ1}

% % TODO: Descrever os algoritmos de ML usados previamente na seção IV (Methodology)
% % To answer this question, during the ML classifiers tuning step, we trained our classifiers with GridSearch focusing on maximizing the F-score. Our best result was obtained by RF with a Precision of 0.22, Recall of 0.63 and F-score of 0.33 using the Anova-F statistical method. Figure \ref{fig:fig-rf-distribution} illustrates the distribution of the predictions' probabilities made by RF with this configuration. As we can see, its distribution is closer to the behavior of the selection studies task of SLRs and SLR updates, where most of the studies are excluded and not included.
% % % Colocar tabela com todos os resultados (é possível ver que o melhor resultado foi obtido com RF usando Anova-F...)
% To answer this question, during the ML models tuning step, we trained our classifiers with GridSearch focusing on maximizing the F-score. Our best result was obtained by RF with a Precision of 0.22, Recall of 0.63 and F-score of 0.33 using the Anova-F statistical method, with 1200 features. We used a default threshold of 0.5 to consider which studies should be included and excluded by our ML models. The parameters tested and selected by GridSearch for this configuration can be found in this document\footnote{https://zenodo.org/api/records/11021614/draft/files/RQ1-RQ3-best-configuration-RF.csv/content}. 
% And the table containing all the predictions made by our ML model for each study for this question can be seen in this document\footnote{https://zenodo.org/api/records/11019279/draft/files/RQ1-RF-predictions.csv/content}.

% % TODO: Add tables with config for rq1 and rq2

% Figure \ref{fig:fig-rf-distribution} illustrates the distribution of the predictions' scores made by RF with this configuration. In order to calculate each of these metrics, we compared our ML models' predictions with the final results only, obtained after the agreement criteria was applied by the team assessment, which is illustrated by the first column of Table~\ref{tab:slr-update-results-sample}.

% \begin{figure} [!h]
%     \centering
%     \includegraphics[width=1\linewidth]{pictures/latest/RF_test_v4-gridsearch_pearson_fs-k1200_bins10_v1.pdf}
%     \caption{RF Predictions Distribution}
%     \label{fig:fig-rf-distribution}
% \end{figure}


% % < Aumentar fontes das legendas e valores dos eixos X e Y de todos os graficos >


% \subsection{RQ2: \textit{How much effort can ML Models reduce during
% the study selection activity for SLR updates?}}
% \label{results:RQ2}

% % To answer this question, we tuned the ML Models with  the intention of maximizing the Recall. Since the purpose of this question was to evaluate how much human effort could be reduced by the use of ML Models during the selection of studies, we wanted to mitigate the chances of a false negative (FN) result, so the reviewers could simply ignore the studies excluded by the ML Model without worrying about losing a relevant study.

% % As demonstrated in Figure \ref{fig:fig-svm-distribution}, our best result was obtained by the SVM algorithm with a Precision of 0.10, Recall of 1.0 and F-score of 0.19 using the Pearson Correlation statistical method, with 1200 features. We used a default threshold of 0.5 to consider which studies should be included and excluded by our algorithms. 

% % According to Table \ref{tab:effort_reduction}, by maximizing the Recall, SVM was able to exclude a total of 187 studies, which represents 33.9\% of the total amount of studies in our testing set. By increasing the threshold, we are able to see that we can reduce the human effort even more at the risk of having more FN results. For a threshold range greater than 0.5 until 0.75, only one false negative was found. Notably, this FN result was one of the few cases where the team assessment had a lot of disparity. Initially, considering only the analysis of the reviewers individually (before they discussed it with each other), the reviewer R1 voted 2, R2 voted 1, and R3 voted 0.

% To answer this question, we tuned the ML models with the intention of maximizing the Recall. Since the purpose of this question was to evaluate how much human effort could be reduced by the use of ML Models during the selection of studies, we wanted to mitigate the chances of a false negative (FN) result, so the reviewers could simply ignore the studies excluded by the ML model without worrying about losing a relevant study.

% Our best result was obtained by using the SVM algorithm with a Precision of 0.10, Recall of 1.0 and F-score of 0.19 using the Pearson Correlation statistical method, with 1200 features. We used a default threshold of 0.5 to consider which studies should be included and excluded by our ML models. The parameters tested and selected by GridSearch for this configuration can be found in this document\footnote{https://zenodo.org/api/records/11021614/draft/files/RQ2-best-configuration-SVM.csv/content}. The table containing all the predictions made by our ML model for each study for this question can be seen in this document\footnote{https://zenodo.org/api/records/11019279/draft/files/RQ2-SVM-predictions.csv/content}.  

% \begin{figure} [!h]
%     \centering
%     \includegraphics[width=1\linewidth]{pictures/latest/fig-rq2-SVM_test_v4-gridsearch_pearson_fs-k1200_bins10.pdf}
%     \caption{SVM Predictions Distribution}
%     \label{fig:fig-svm-distribution}
% \end{figure}


% According to Table \ref{tab:effort_reduction}, by maximizing the Recall, SVM was able to exclude a total of 187 studies, which represents 33.9\% of the total amount of studies in our testing set. By increasing the threshold, we are able to see that we can reduce the human effort even more at the risk of having more FN results. For a threshold range greater than 0.5 until 0.75, only one false negative was found, while the number of TN increased by 87. Notably, this FN result was one of the few cases where the team assessment had a lot of disparity. Considering only the initial analysis of the reviewers individually (before they discussed it with each other), the reviewer R1 voted 2, R2 voted 1, and R3 voted 0. 
% For a threshold range greater than 0.75 until 0.80, when compared to the previous threshold range, the number of FN results increased by 1, while the number of TN increased by 18. 
% Finally, for a threshold range greater than 0.80 until 0.85, when compared to the previous threshold range, the number of FN results increased by 2, while the number of TN increased by 24. 

% As well as RQ1, to answer this question, we compared our ML models' predictions with the final results only, obtained after the agreement criteria was applied by the team assessment, which is illustrated by the first column of Table~\ref{tab:slr-update-results-sample}.


% \begin{table}
% \centering
% \begin{tabular}{|c|c|c|c|c|c|c|}
% \hline
% \textbf{Threshold(\%)} & \textbf{RECALL (\%)} & \textbf{TN} & \textbf{TP} & \textbf{FN} & \textbf{FP} & \textbf{Reduced (\%)} \\
% \hline
% 0.50\% & 100.00\% & 187 & 38 & 0 & 326 & 33.9\% \\ 
% 0.75\% & 97.37\% & 265 & 37 & 1 & 248 & 48.3\% \\  
% 0.80\% & 94.74\% & 283 & 36 & 2 & 267 & 51.7\% \\  
% 0.85\% & 89.49\% & 307 & 34 & 4 & 206 & 56.4\% \\ 
% \hline
% \end{tabular}
% \caption{Tradeoff between effort reduction and number of FN.}
% \label{tab:effort_reduction}
% \end{table}

% \subsection{RQ3: \textit{Can Machine Learning replace a reviewer in the selection of studies for SLRs?}}
% \label{results:RQ3}

% To answer this question, we conducted a two-fold analysis to evaluate both aspects of agreement and similarity, considering not only the comparison between our ML model and single reviewer results but also the final results and the average answer between multiple reviewers.

% The agreement analysis indicates how two or more raters make the same classifications, it measures the concordance of results, we used the Cohen's Kappa coefficient for this. While, the similarity analysis indicates the resemblance between the classifications of two or more raters, it measures how close the results are, even if they are not exactly the same, we used the Euclidean Distance for this. 

% Firstly, we looked at the agreement and similarity levels between the reviewers among the team assessment, and analyzed the information provided by the team assessment regarding each reviewer's vote before applying the agreement criteria. Table~\ref{tab:reviewers-agreement-table} shows the Kappa values between reviewers, and Table~\ref{tab:reviewers-similarity-table} shows the Euclidean Distance between reviewers.

% \begin{table}
% \centering
% \begin{tabular}{|c|c|c|c|}
% \hline
% \textbf{} & \textbf{R1} & \textbf{R2} & \textbf{R3} \\
% \hline
% \textbf{R1} & 1 & 0.47 & 0.35 \\
% \textbf{R2} & 0.47 & 1 & 0.43 \\
% \textbf{R3} & 0.35 & 0.43 & 1 \\
% \hline
% \end{tabular}
% \caption{Agreement Among the Assessment Team.}
% \label{tab:reviewers-agreement-table}
% \end{table}

% \begin{table}
% \centering
% \begin{tabular}{|c|c|c|c|}
% \hline
% \textbf{} & \textbf{R1} & \textbf{R2} & \textbf{R3} \\
% \hline
% \textbf{R1} & 0 & 13.0 & 14.04 \\
% \textbf{R2} & 13.0 & 0 & 11.22 \\
% \textbf{R3} & 14.04 & 11.22 & 0 \\
% \hline
% \end{tabular}
% \caption{Similarity Among the Assessment Team.}
% \label{tab:reviewers-similarity-table}
% \end{table}

% We considered the following rage of Cohen's Kappa coefficient values as the following agreement levels \cite{carletta1996assessing}:
% \begin{itemize}
%     \item from 0.00 to 0.20: Poor Agreement
%     \item from 0.21 to 0.40: Fair Agreement
%     \item from 0.41 to 0.60: Moderate Agreement
%     \item from 0.61 to 0.80: Substantial Agreement
%     \item from 0.81 to 1.00: Almost Perfect Agreement
% \end{itemize}

% The highest agreement was achieved between reviewer 1 (R1) and reviewer 2 (R2) with 0.47 of agreement, which can be considered a Moderate Agreement. Whilst, the lowest agreement was between the reviewer 3 (R3) and R1 with 0.35 of agreement, which can be considered a Fair Agreement. Since the similarity is inversely proportional to the Euclidean Distance, we can notice that R2 and R3 had the most similar opinions with the smallest distance of 11.22 in comparison to R1 and R3 with the highest distance of 14.04.

% To compare our ML models with the reviewers, we used the same RF Model used to answer RQ1, since it had the better f-score, it made more realistic predictions in contrast to our SVM model used to answer RQ2, which was configured to maximize recall. 
% We normalized our RF results in three ranges to represent the same three categories used by the reviewers during the step to apply the inclusion and exclusion criteria. Figure~\ref{fig:fig-reviewers-votes-distribution} shows the distribution of votes by each reviewer in each category. We decided the threshold for the exclusion of studies should be greater than the rest, since the frequency of occurrence of votes "0" was clearly the highest for the assessment team. And we decided the range of "uncertainty" should be the smallest, since it was the vote less frequent. Considering that, we decided to use the following range of predictions score to normalize our RF model. 
% \begin{itemize}
%     \item from 0.00 to 0.50: should be excluded
%     \item from 0.51 to 0.60: uncertain
%     \item from 0.61 to 1.00: should be included
% \end{itemize}

% Figure~\ref{fig:fig-rf-normalized-distribution} illustrates the normalized distribution of our ML model used in RQ1.

% \begin{figure} [!h]
%     \centering
%     \includegraphics[width=1\linewidth]{pictures/latest/fig-rq3-reviewers-votes-distribution.pdf}
%     \caption{Reviewers votes distribution.}
%     \label{fig:fig-reviewers-votes-distribution}
% \end{figure}

% \begin{figure} [!h]
%     \centering
%     \includegraphics[width=1\linewidth]{pictures/latest/rf_test_v4_normlized_agreement.pdf}
%     \caption{RQ3: RF Predictions Distribution considering "uncertain" range.}
%     \label{fig:fig-rf-normalized-distribution}
% \end{figure}



% We used the normalized results of our RF model to evaluate its agreement and similarity with the reviewers. Table~\ref{tab:rf-kappa-and-euclidean-table} illustrates the Cohen's Kappa coefficient and the Euclidean Distance between our RF model and each reviewer.

% % After we converted the probabilities given by our RF model used in RQ1 into the three categories, we compared the normalized results of the RF model with the results of each reviewer and calculated the \textit{Cohen's Kappa Coefficient} and the Euclidean Distance between them. The agreement level between the RF model and each reviewer was: RF | R1 = 0.27, RF | R2 = 0.37, RF | R3 = 0.30. The Euclidean Distance between the RF model and each reviewer was: RF | R1 = 17.89, RF | R2 = 16.76, RF | R3 = 17.52. Table~\ref{tab:rf-kappa-and-euclidean-table} illustrates both results.

% \begin{table}[h!]
% \centering
% \begin{tabular}{|c|c|c|}
% \hline
% \textbf{Comparison} & \textbf{Kappa} & \textbf{Euclidean Distance} \\ \hline
% R1 vs RF & 0.27 & 17.89 \\ \hline
% R2 vs RF & 0.37 & 16.76 \\ \hline
% R3 vs RF & 0.30 & 17.52 \\ \hline
% \end{tabular}
% \caption{Cohen's Kappa coefficient and Euclidean Distance between RF model and Reviewers}
% \label{tab:rf-kappa-and-euclidean-table}
% \end{table}

% Then, we used the Euclidean Distance to evaluate the similarity between the Final Results (FR) and our RF model and reviewers, when compared individually and collectively. In order to evaluate the answers from the FR accurately, we normalize the binary answers in FR to considered that the included studies a value of 2 and excluded studies had a value 0 in FR. So if a reviewer voted to include a study with certainty and the study was indeed included, the distance between these two points would be zero.

% We evaluated the Euclidean Distance (ED) by three different perspectives, as follows:
% \begin{itemize}
%     \item Similarity between single answers and FR: 
%     \[
%     \textit{ED}(i, FR) \text{ where } i \in \{\text{R1}, \text{R2}, \text{R3}, \text{RF}\}
%     \]
%     \item Similarity between pairs and FR:
%     \[
%     \textit{ED}\left(\frac{i + j}{2}, \text{FR}\right) \text{ where } i \neq j \text{ and } i, j \in \{\text{R1}, \text{R2}, \text{R3}, \text{RF}\}
%     \]
%     \item Similarity between groups and FR:
%     \[
%     \textit{ED}\left(\frac{i + j + k}{3}, \text{FR}\right) \text{ where } i \neq j \neq k \text{ and } i, j, k \in \{\text{R1}, \text{R2}, \text{R3}, \text{RF}\}
%     \]
% \end{itemize}

% Table~\ref{tab:similarity-distance-FR-table} shows the ED measured in each case. As we can see, the smallest distance in comparison to the FR in each case was given by: R2 with ED = 9.95, pair(R2,R3) with ED = 8.86 and team assessment with ED = 8.17.  

% \begin{table}[h!]
% \centering
% \begin{tabular}{|c|c|}
% \hline
% \textbf{Comparison} & \textbf{Euclidean Distance} \\ \hline
% R1 vs FR & 12.00 \\ \hline
% R2 vs FR & \textbf{9.95} \\ \hline
% R3 vs FR & 11.00 \\ \hline
% RF vs FR & 17.94 \\ \hline
% \hline
% avg(R1,R2) vs FR & 8.90 \\ \hline
% avg(R1,R3) vs FR & 9.12 \\ \hline
% avg(R2,R3) vs FR & \textbf{8.86} \\ \hline
% avg(R1,RF) vs FR & 12.37 \\ \hline
% avg(R2,RF) vs FR & 11.84 \\ \hline
% avg(R3,RF) vs FR & 12.03 \\ \hline
% \hline
% avg(R1,R2,R3) vs FR & \textbf{8.17 }\\ \hline
% avg(RF,R2,R3) vs FR & 10.06 \\ \hline
% avg(R1,RF,R3) vs FR & 10.20 \\ \hline
% avg(R1,R2,RF) vs FR & 10.15 \\ \hline
% \end{tabular}
% \caption{Euclidean Distance Analysis considering FR}
% \label{tab:similarity-distance-FR-table}
% \end{table}

% %--------------------------------------------------------
% \section{Discussion}
% \label{sec:discussion}

% \subsection{\textit{Research Question 1}}
% \label{discussion:RQ1}
% Regarding RQ1 - “How effective are ML models in selecting studies for SLR updates?”, based on our results, we concluded that the best configuration to maximize the F-score of our ML models was using RF with Anova-F. Not only that, but in almost all of our tests configurations, RF outperformed SVM in terms of F-score with exception for one test, where we selected only 900 best features and used the Pearson Correlation method to compute the best features. We also noticed that for most cases, the Anova-F improved the F-score rating at the cost of lowering its Recall. 
% However, even noticing that the RF was more successful in this task than SVM and considering its best result, its F-score was still not good enough to be considered to automate the process of selecting studies for the SLR update.


% % TODO: O watanabe coloca tabelas com o resultados dos dois algoritmos DT e SVM variando o numero de features e na discussion compara o seus desempenhos...

% \subsection{\textit{Research Question 2}}
% \label{discussion:RQ2}
% Regarding RQ2 - “How much effort can ML models reduce during the study selection activity for SLR updates?”, our results showed that the best configuration to maximize the Recall of our ML models was using SVM with Pearson Correlation. On one hand, it's possible to see that in all of our tests, SVM had higher Recall marks than RF. Particularly, when used with Pearson Correlation to select the best features, it had the highest marks for Recall in most cases (although even in executions using Anova-F it still reached a high Recall of 0.97 for some tests). On the other hand, its F-score and Precision marks were very low. However, even if a big number of FP studies remained after that, we believe that our SVM model showed potential for reducing human effort during the study selection task of SLR updates, by automatically excluding part of the irrelevant studies.

% \subsection{\textit{Research Question 3}}
% \label{discussion:RQ3}
% Regarding RQ3 - “Can Machine Learning Replace a Reviewer in the Selection of Studies for Systematic Literature Reviews?” our evaluation showed that the best result was achieved by the RF classifier with the same configuration used for RQ1. As we can see in Table~\ref{tab:reviewers-agreement-table}, the strongest level of agreement was between R1 and R2 with a score of 0.47 for Cohen's Kappa Coefficient, followed by an agreement of 0.43 between R2 and R3, and followed by the weakest agreement between R1 and R3 with a score of 0.35. In this case, two pairs of reviewers had a moderate level of agreement and one pair had only a fair level of agreement. 

% On the other hand, when compared with our RF classifier, it had its strongest agreement of 0.37 with R2, followed by an agreement of 0.30 with R3 and an agreement of 0.27 with R1, all considered as fair level of agreement. Even though the pair RF classifier and R2 had a stronger agreement than the pair R1 and R3, both still had the same agreement level (fair agreement), also the R2 had a Moderate Agreement with both reviewers, while it had only a fair agreement with the RF classifier. Also, when comparing the agreement of RF with R3 and its respective pairs (R3|R1 and R3|R2), it had a weaker agreement than both, the same was true when comparing RF with R2 and its respective pairs. 

% Looking at Table~\ref{tab:reviewers-similarity-table}, we can see that R2 and R3 had the most similar answers, with an ED of 11.22 between them. As showed in Table~\ref{tab:similarity-distance-FR-table} R2 also had the smallest distance from FR when compared individually (with an ED of 9.95), as expected, the pair R2 and R3 had the strongest similarity with FR (with an ED of 8.86) in contrast to the other pairs. Finally, the closest distance was given by the team assessment with an ED of 8.17, which is expected since the FR was generated from their answers. It's possible to see that our RF model had the highest distances in all comparisons and even though looking at the distance to FR obtained when working together with R1 didn't cause much negative impact (ED(R1,RF) = 12.37 vs ED(R1) = 12.00) as the rest, it shows that our ML model didn't help any of the reviewers to get closer to the FR. 

% Therefore, we concluded that our supervised ML Models are not ready to replace a reviewer during the selection of studies for SLR updates.

% It is worth mentioning that the similarity and agreement levels between our classifier and the team assessment could increase or decrease depending on how we configure the thresholds to normalize the probabilities given by our ML classifier into the three categories used by the team assessment. We noticed that reducing the range to consider a vote as uncertain would increase the level of agreement with all members of the assessment team. But in most cases, our classifier still had only a fair level of agreement, so our conclusion was the same.


% \section{Threats to Validity}
% \label{sec:threats}

% In the following, we enumerate the main threats to the validity of our study.

% \textbf{Construct Validity.}
% Our evaluation results might have been affected by the choice of ML algorithms. Other algorithms could have been explored in our study and can be considered as part of future work.

% \textbf{Internal Validity.}
% Our training dataset comprised only studies including in the SLR replication \cite{Wohlin2022} (training included) and those obtained through backward snowballing (training excluded). We deliberately excluded studies not in English or those categorized as Ph.D. dissertations or book chapters from our testing set, same approach adopted by the SLR replication. This focused approach aimed to provide our models with only relevant and essential data. Another potential threat is that during manual process performed by the team assessment, the authors could end up reading other sections of the studies besides the title and abstracts when they are not completely sure if the study should be included or excluded just by reading its abstract. This is a possible advantage manual process could have over our models that consider only content from title and abstract. 

% \textbf{External Validity.} The dataset used in our analysis might not represent the diversity of SLR Updates in SE. Similar analyses could have been conducted based on other SLRs to improve the generalizability of our results. However, replicating our results on other SLRs to strengthen external validity would require significant effort. Moreover, it is challenging to acquire a reliable and detailed SLR dataset for SLRs updates that could be considered in our evaluation. 

% \textbf{Reliability.} One limitation of our study is associated with the dataset used in our evaluation and the possibility of sample bias. The data used in our evaluation was acquired from the same authors who performed the SLR replication, also through a rigorous analysis process. In addition, to improve the reliability of our results, our ML models and the small-scale evaluation dataset are openly available. 

% %--------------------------------------------------------
% \section{Conclusion}
% \label{sec:conclusion}
% % In this paper we have presented an evaluation using our ML models to replicate the process of an ongoing SLR update performed by three experienced researchers, to evaluate it as a supporting tool for the studies’ selection task.  We compared the results of our ML models with the members of the team assessment. 

% % We concluded that our ML models are not ready to automatically select studies for the study selection task, and may also not be used to replace an additional reviewer. However, there is potential for reducing human effort during the study selection task of SLR updates, by automatically excluding part of the irrelevant studies.

% % Our next steps is to investigate even further the use of ML models with different configurations and text classification techniques (variar algoritmos, métodos estatísticos de FS, técnicas de TF-IDF) with the same goals. In addition, we intend to explore ML models with different datasets to evaluate their performance. We also believe we could achieve promising results by adapting our pipeline to have multiple iterations, such as: first we could only exclude irrelevant studies using a ML model with a configuration maximized by the recall, than we could try to predict the most relevant in the remaining results using another ML model maximized by the f-score.

% This study advances the application of supervised ML models as a supporting tool for researchers during SLRs updates, by developing and testing a comprehensive supervised ML-based pipeline to automate the study selection process. We have demonstrated through our tests, using realistic data to build our datasets, that while our ML models, have shown promise in reducing human effort to perform the study selection activity by pre-filtering irrelevant studies, they currently lack the precision required to completely automate the selection of studies for SLR updates. We also concluded that they did not achieve a level of similarity and agreement strong enough to be considered sufficient to replace a human reviewer during an SLR update in the study selection phase. Our work also highlights different configurations used for our ML models that correlates to their recall and F-score, providing results that can be useful for further exploration in this area.

% Our next step is to investigate even further the use of ML models with different configurations and text classification techniques with the same goals. In addition, we intend to explore our pipeline using datasets with a different structure to evaluate its performance. Additionally, we intend to perform new tests with the objective of improving our pipeline's automation, for instance, by experimenting to use the ML model configured to maximize the recall to perform an initial filter on the studies and then use another ML model configured to maximize the f-score to make predicts based on the results previously filtered. Also, we believe we could leverage from the work of Pintas \textit{et al.} \cite{pintas2021feature} by performing a deeper analysis of our dataset structure and use the contributions of their work to decide which FS strategies should work the best. Lastly, considering the increase number of studies using LLMs to support SLRs and SLR updates \cite{bolanos2024artificial}, we intend to evaluate the use of LLMs in this context as well and compare with our models.

% \balance
% \bibliographystyle{ACM-Reference-Format}
% \bibliography{acmart}

% % \bibliographystyle{ACM-Reference-Format}
% %\bibliography{bib/base}
% %\printbibliography

% \end{document}
% \endinput

