Visual Question Answering (VQA) is an AI task designed to reason about images.  Commonly, the image is transformed into a ``scene graph'' that enables the deployment of more formal reasoning tools.  For example, in recent work, both the scene graph and associated query were represented as an ASP Program~\cite{eiter2022neuro, kinjal2020aqua}; however,  notably the scene graph itself only contains information about the scene, but lacks commonsense knowledge -- in particular, knowledge about the domains of attributes identified by the scene. 
Existing work to address this shortcoming relies on leveraging large commonsense knowledge graphs for obtaining domain knowledge~\cite{marino2019ok, schwenk2022okvqa, wang2017fvqa}.  However, such approaches require the ability to accurately align the language of the knowledge graph with the language of the scene graph.  Further, for some applications, this does not guarantee that the aligned knowledge graph will necessarily improve VQA performance (e.g., if domain knowledge relevant to the queries is not possessed in the knowledge graph).  In this paper, we provide an orthogonal and complementary approach that leverages logical representations of the scene graph and query to abduce domain relationships that can improve query answering performance.  We frame the abduction problem and provide a simple algorithm that provides a valid solution. 
We also provide an implementation and show on a standard dataset that we can improve question answering accuracy from $59.98\%$ to $81.01\%$, and provide comparable results with few historical examples.

\medskip
\noindent\textbf{Motivating Example.} Consider the simple scene graph depicted in Figure \ref{fig:sgaug} and the query ``\textit{What is the color of the fruit to the right of the juice?}''.
Without the shaded nodes (which indicate domain information external to the image) there is no attribute of any constant associated with \textit{banana} that is associated with the domain \textit{color} or the domain \textit{fruit}.  Hence, the only answer would be to assume that there is no fruit or the color information is not given, or randomly guess \textit{large} 
(while not a color, it is an attribute) or \textit{yellow}.  
In this paper, we will look to abduce these domain relationships from a limited number of examples.






\begin{figure}[t]
    \centering
    \begin{subfigure}[t]{0.45\textwidth}
        \centering
        \includegraphics[width=\textwidth]{images/2317538Tagged.png}
        \label{fig:im}
    \end{subfigure}
    \hfill
    \begin{subfigure}[t]{0.45\textwidth}
        \centering
        \begin{tikzpicture}[node distance=1cm, auto, font=\itshape]
        % Initial Styles 
        \tikzstyle{rect} = [rectangle, draw, minimum width=1cm, minimum height=0.75cm]
        \tikzstyle{oval} = [ellipse, draw, minimum width=1cm, minimum height=0.75cm]
        \tikzstyle{dom} = [oval, fill=gray!50, text=black]
        \tikzstyle{arrow} = [-{Triangle[scale=1.5]}, thick]
        \tikzstyle{dashedarrow} = [arrow, dashed]
        
        % Objects 
        \node[rect] (n57) {57};
        \node[rect, above=1.5cm of n57] (n55) {55};
        
        % Attributes for n57
        \node[oval, right=1.25cm and 1.5cm of n57] (juice) {juice};
        \node[oval, left=1.5cm and 1.5cm of n57] (yellow) {yellow};
        \node[dom, left=0.5cm and 1.0cm of n55] (color) {color};
        \node[dom, above left=1.75cm and -0.5cm of juice] (drink) {drink};
        
        % Attributes for n55
        \node[oval, above left=1.25cm and 1.25cm of n55] (large) {large};
        \node[oval, above right=1.25cm and 1.25cm of n55] (banana) {banana};
        \node[dom, above left=1.25cm and -2.25cm of large] (size) {size};
        \node[dom, above left=1.25cm and 0.25cm of banana] (fruit) {fruit};
    
        % Edges from n57 to its children
        \draw[arrow] (n57) -- (yellow) node[midway, above, sloped] {attr};
        \draw[arrow] (n57) -- (juice) node[midway, above, sloped] {name};
        \draw[dashedarrow] (color) -- (yellow) node[midway, above, sloped] {assign};
        \draw[dashedarrow] (drink) -- (juice) node[midway, above, sloped] {assign};
        
        % Edges from n55 to its children
        \draw[arrow] (n55) -- (yellow) node[midway, above, sloped] {attr};
        \draw[arrow] (n55) -- (banana) node[midway, above, sloped] {name};
        \draw[arrow] (n55) -- (large) node[midway, above, sloped] {attr};
        \draw[dashedarrow] (fruit) -- (banana) node[midway, above, sloped] {assign};
        \draw[dashedarrow] (size) -- (large) node[midway, above, sloped] {assign};
    
        % Relationships
        \draw[arrow] (n55) -- (n57) node[midway, above, sloped] {right};
        \draw[arrow] (n57) -- (n55) node[midway, above, sloped] {left};
        \end{tikzpicture}
        \label{fig:sg}
    \end{subfigure}
    \caption{An image (left) and a section of its corresponding scene graph (right). In the scene graph, square nodes represent objects, oval nodes represent attributes, and solid edges connect objects to attributes. Shaded nodes represent domain knowledge, connected to attributes by dashed edges.}
    \label{fig:sgaug}
\end{figure}







