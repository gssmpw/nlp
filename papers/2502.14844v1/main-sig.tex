


\documentclass[acmtog]{acmart}

%% \BibTeX command to typeset BibTeX logo in the docs
\AtBeginDocument{%
  \providecommand\BibTeX{{%
    Bib\TeX}}}

%% Rights management information.  This information is sent to you
%% when you complete the rights form.  These commands have SAMPLE
%% values in them; it is your responsibility as an author to replace
%% the commands and values with those provided to you when you
%% complete the rights form.
\setcopyright{none}
\settopmatter{printacmref=false}
\acmDOI{XXXXXXX.XXXXXXX}


\hypersetup{
    colorlinks=true,       % Enable colored links
    urlcolor=pink,         % Set URL link color to pink
    linkcolor=blue,        % Optional: Set link color to blue (e.g., for internal links)
    citecolor=red          % Optional: Set citation link color to red
}


%%
%% Submission ID.
%% Use this when submitting an article to a sponsored event. You'll
%% receive a unique submission ID from the organizers
%% of the event, and this ID should be used as the parameter to this command.
%%\acmSubmissionID{123-A56-BU3}

%%
%% For managing citations, it is recommended to use bibliography
%% files in BibTeX format.
%%
%% You can then either use BibTeX with the ACM-Reference-Format style,
%% or BibLaTeX with the acmnumeric or acmauthoryear sytles, that include
%% support for advanced citation of software artefact from the
%% biblatex-software package, also separately available on CTAN.
%%
%% Look at the sample-*-biblatex.tex files for templates showcasing
%% the biblatex styles.
%%

%%
%% The majority of ACM publications use numbered citations and
%% references.  The command \citestyle{authoryear} switches to the
%% "author year" style.
%%
%% If you are preparing content for an event
%% sponsored by ACM SIGGRAPH, you must use the "author year" style of
%% citations and references.
% \citestyle{acmauthoryear}
\settopmatter{printacmref=false} % Suppress ACM reference format

\newcommand{\thought}[1]{{\color[rgb]{0.2,0.39,0.66}(#1)}}
\newcommand{\todo}[1]{{\color[rgb]{1.0,0.0,0.0}(#1)}}
\newcommand{\hsh}[1]{{\color{green!50!black} Henrik: #1}}
\newcommand{\st}[1]{{\color{red!50!black} Sebastian: #1}}

\newcommand{\ulm}[1]{_{\scaleto{\mathrm{#1}}{3pt}}}
\newcommand\at[2]{\left.#1\right|_{#2}}











\newtheorem{assumption}{Assumption}

\DeclareMathOperator*{\argmax}{arg\,max}
\DeclareMathOperator*{\argmin}{arg\,min}

\newcommand{\swname}[1]{\texttt{#1}}
\newcommand{\ie}{i\/.\/e\/.,\/~}
\newcommand{\eg}{e\/.\/g\/.,\/~}
\newcommand{\cf}{cf\/.\/~}

\newcommand{\fig}{Fig\/.\/~}
\newcommand{\defn}{Def\/.\/~}
\newcommand{\sect}{Sec\/.\/~}
\newcommand{\tabl}{Tab\/.\/~}
\newcommand{\algo}{Algorithm~}
\newcommand{\theo}{Theorem~}

\newcommand{\bnnl}{3 hidden layers}
\newcommand{\bnnn}{50 neurons}
\newcommand{\bnna}{tanh activations}

\newcommand{\capt}[1]{\mdseries{\emph{#1}}}

\newcommand{\videolink}{at \url{https://youtu.be/_d7AqTRjz6g}}
\newcommand{\codelink}{\url{https://github.com/wheelbot/mini-wheelbot}}

\newcommand{\fakepar}[1]{\vspace{0mm}\noindent\textbf{#1.}}

\newcommand{\needref}{\textcolor{red}{[REF]}}

\newcommand{\plotfontsize}{9pt}

\usepackage{amsmath}
\usepackage{booktabs}
\usepackage{multicol}
\usepackage{multirow}
\usepackage{siunitx}
\usepackage{hyperref}
\usepackage{xcolor}


\definecolor{pink}{rgb}{1.0, 0.6, 0.9} % Light pink color

% Hyperref setup for clickable pink URLs
\hypersetup{
    colorlinks=true,
    urlcolor=pink, % Sets the URL color to pink
}
%%


%% end of the preamble, start of the body of the document source.
\begin{document}


%%
%% The "title" command has an optional parameter,
%% allowing the author to define a "short title" to be used in page headers.
\title{\hspace{1.7cm}  Dynamic Concepts Personalization from Single Videos}





\author{ \vspace{-0.3cm} \hspace*{0.3cm} Rameen Abdal \hspace{1.0cm} Or Patashnik \hspace{1.0cm} Ivan Skorokhodov \hspace{1.0cm} Willi Menapace \\
    \\  Aliaksandr Siarohin \hspace{0.7cm} Sergey Tulyakov \hspace{0.7cm} Daniel Cohen-Or \hspace{0.7cm} Kfir Aberman
    \\ \\  \centerline{\textit{ Snap Research}} }  % Snap Research centered and smaller
     % Centered clickable pink URL
%  % Centered pink URL





%%
%% By default, the full list of authors will be used in the page
%% headers. Often, this list is too long, and will overlap
%% other information printed in the page headers. This command allows
%% the author to define a more concise list
%% of authors' names for this purpose.
\renewcommand{\shortauthors}{Abdal et al.}

%%
%% The abstract is a short summary of the work to be presented in the
%% article.
\begin{abstract}

\begin{abstract}

% Recent works to jointly reconstruct 3D human and object from a single RGB image, are mostly model-based, that fail to capture the fine details of the clothed human body and object surface. In this paper, we introduce ReCHOR, a novel, model-free, first-method to produce realistic clothed human-object reconstructions from a monocular view. This is extremely challenging due to human-object occlusions, diverse interactions and depth ambiguity, as it needs to infer both 3D spatial awareness and high resolution details. Our core idea is based on estimating neural implicit representations for human and object respectively by an attention-based neural implicit model that attends to pixel-aligned features from both the global human-object image for spatial awareness and  the local separate view of human and object images for high quality details. Additionally, the network is conditioned on semantic features from an initial estimated human-object pose prior and a generative diffusion model that inpaints occluded regions, thus enabling the retrieval of details from them.
% We also propose a synthetic dataset with rendered scenes of diverse, inter-occluded 3D human and object scans, to train our network. We evaluate our method on the synthetic and real world BEHAVE dataset. Our experiments show that our method outperforms the SOTA in achieving realistic clothed human-object reconstructions.
Recent approaches to jointly reconstruct 3D humans and objects from a single RGB image represent 3D shapes with template-based or coarse models, which fail to capture details of loose clothing on human bodies. In this paper, we introduce a novel implicit approach for jointly reconstructing realistic 3D clothed humans and objects from a monocular view. For the first time, we model both the human and the object with an implicit representation, allowing to capture more realistic details such as clothing. This task is extremely challenging due to human-object occlusions and the lack of 3D information in 2D images, often leading to poor detail reconstruction and depth ambiguity. To address these problems, we propose a novel attention-based neural implicit model that leverages image pixel alignment from both the input human-object image for a global understanding of the human-object scene and from local separate views of the human and object images to improve realism with, for example, clothing details. Additionally, the network is conditioned on semantic features derived from an estimated human-object pose prior, which provides 3D spatial information about the shared space of humans and objects. To handle human occlusion caused by objects, we use a generative diffusion model that inpaints the occluded regions, recovering otherwise lost details. For training and evaluation, we introduce a synthetic dataset featuring rendered scenes of inter-occluded 3D human scans and diverse objects. Extensive evaluation on both synthetic and real-world datasets demonstrates the superior quality of the proposed human-object reconstructions over competitive methods.
\end{abstract}
\end{abstract}

\begin{teaserfigure}
\centering
\vspace{-0.3cm}
\centerline{\href{https://snap-research.github.io/dynamic_concepts}}{\large \textcolor{pink}{\texttt{https://snap-research.github.io/dynamic$\_$concepts}}}
\includegraphics[width=\linewidth]
			{images/teaser.jpg}
\caption{We personalize a video model to capture \textbf{dynamic concepts} -- entities defined not only by their appearance but also by their unique motion patterns, such as the fluid motion of ocean waves or the flickering dynamics of a bonfire (left). This enables high-fidelity generation, editing, and the composition of these dynamic elements into a single video, where they interact naturally (right).}
\label{fig:teaser}
\end{teaserfigure}

\settopmatter{printacmref=false} % Removes ACM reference format
\renewcommand\footnotetextcopyrightpermission[1]{} % Removes copyright footnote


\maketitle
% \mytitle


% \section{Introduction}
\label{sec:intro}
% Image editing methods in diffusion models depend on user-defined control directions - users can unlock their creativity using these methods by specifying the desired manipulation through prompts~\cite{gandikota2023concept}, reference images~\cite{ruiz2022dreambooth, kumari2022customdiffusion, gal2022image, chen2024trainingfreeregionalpromptingdiffusion}, or attribute vectors~\cite{parmar2023zero,hertz2022prompt}. In this work, we ask a fundamentally different question: \emph{Can we automatically discover the underlying visual structure of a concept within diffusion model's knowledge?} %Rather than requiring user-specified controls, we aim to decompose the model's internal knowledge into meaningful directions.

% This question touches on a fundamental limitation in how we interact with diffusion models. Current control methods ~\cite{zhang2023addingconditionalcontroltexttoimage, gandikota2023concept, ye2023ipadaptertextcompatibleimage,ye2023ipadaptertextcompatibleimage, hertz2024stylealignedimagegeneration, li2023photomaker, shi2024instantbooth, chen2024trainingfreeregionalpromptingdiffusion} require users to specify their desired manipulations in advance, limiting interactive creativity. This contrasts with natural human artistic workflows, where creators dynamically explore creative ideas while jointly refining them toward meaningful artistic outcomes~\cite{hoffmann2016modeling}. This synergy between specification and exploration is not new to generative models. Early GAN architectures naturally developed disentangled latent spaces that enabled continuous\cite{harkonen2020ganspace,radford2015unsupervised, wu2021stylespace, shen2020interfacegan}, compositional control over generated images. Users could explore these spaces to discover interesting variations that would be difficult to describe in words~\cite{wu2021stylespace}, then combine them to achieve their creative goals~\cite{grabe2022towards}. 


% While diffusion models have largely superseded GANs in conditional image synthesis~\cite{dhariwal2021diffusion},  their underlying structure remains less understood. Diffusion models achieve remarkable diversity through high-dimensional latents, unlike GANs' compact latent spaces.  With a single prompt, diffusion models can generate radically different variations through different random initializations of input noise. We ask - Is it possible to discover interpretable structure within this vast space of variations?

Text-to-image diffusion models are capable of generating remarkable visual variations from a single prompt through different random initializations. However, this vast creative potential remains largely opaque to users---while we can generate diverse images, we lack understanding of the underlying structure of these variations. This presents a fundamental challenge: how can we discover and expose the latent visual capabilities encoded within these models?

\let\thefootnote\relax \footnote{$^{*}$Correspondence to \texttt{gandikota.ro@northeastern.edu}}

The challenge touches on a key limitation in how we interact with diffusion models today. Current control methods require users to explicitly specify their desired edits in advance through prompts~\cite{gandikota2023concept}, reference images~\cite{zhang2023addingconditionalcontroltexttoimage, chen2024trainingfreeregionalpromptingdiffusion, ruiz2022dreambooth,kumari2022customdiffusion, Ryu_lora, hu2021lora}, or attribute vectors~\cite{ye2023ipadaptertextcompatibleimage, hertz2024stylealignedimagegeneration, li2023photomaker, shi2024instantbooth,parmar2023zero,hertz2022prompt}. That contrasts sharply with natural human creative workflows, where artists dynamically explore creative ideas and jointly refine them toward meaningful artistic outcomes~\cite{hoffmann2016modeling}. The need for pre-specified controls creates a barrier between users and the full creative potential of these models.

Interestingly, earlier generative models like GANs~\cite{gans,karras2019style,brock2018large} naturally developed more interpretable internal structures. Their compact latent spaces often exhibited emergent disentanglement~\cite{harkonen2020ganspace,radford2015unsupervised, wu2021stylespace, shen2020interfacegan}, enabling continuous and compositional control over generated images. Users could explore these spaces to discover interesting variations that would be difficult to describe in words~\cite{wu2021stylespace}, then combine them to achieve their creative goals~\cite{grabe2022towards}.

Diffusion models have largely superseded GANs in conditional image synthesis~\cite{dhariwal2021diffusion}, achieving greater diversity through much higher-dimensional latents. And yet an understanding of the underlying structure of these larger latent spaces has remained elusive. In this work, we ask a fundamental question: \emph{Can we automatically discover the visual structure within a diffusion model's knowledge of a concept?} Rather than requiring user-specified controls, we aim to decompose the model's internal representations into expressive directions that users can explore and combine.

To address these needs, we present \textbf{SliderSpace}, a framework that brings systematic explorability to diffusion models. Given just a text prompt, SliderSpace discovers a canonical set of meaningful, diverse, and controllable directions within the model's knowledge of that concept. Each direction is implemented as a low-rank adapter~\cite{hu2021lora} that can be scaled and composed with others, allowing users to explore and smoothly combine different aspects of variation, as shown in Figure~\ref{fig:intro}.

We ground SliderSpace discovery in three key requirements for meaningful decomposition of a diffusion model's visual manifold: 
\begin{enumerate}
    \item \textbf{Unsupervised Discovery:} The decomposition process should emerge from the intrinsic structure of the model's learned representation, rather than being guided by predefined attributes. This ensures we capture the true topology of the model's knowledge space rather than projecting our assumptions onto it.
    
    \item \textbf{Semantic Orthogonality:} Each discovered control must represent a distinct semantic direction. This is enforced in a semantic feature space, like CLIP, where every slider has an orthogonal effect in embeddings. This prevents discovering multiple controls that create similar semantic effects, making the system more efficient and easier.
    
    \item \textbf{Distribution Consistency:} Directions must induce consistent transformations across both random seeds and prompt variations. 
\end{enumerate}

These requirements naturally lead to our proposed framework, which we formalize in Section~\ref{sec:method}. As we show in our experiments, SliderSpace is architecture-agnostic, working with both conventional U-Net based models like Stable Diffusion~\cite{rombach2022high, rombach2022sd20, podell2023sdxl, turbo, dmd} and recent transformer-based architectures like Flux~\cite{flux}.

We demonstrate the expressiveness of SliderSpace through three applications: First, we show how SliderSpace can decompose high-level concepts into diverse and expressive components, revealing the natural axes of variation in the model's understanding. Second, we explore artistic style variation, where SliderSpace discovers directions that match or exceed the diversity of manually curated artist lists while being judged more useful by human evaluators. Finally, we show how SliderSpace can help reverse the mode collapse commonly observed in distilled diffusion models, restoring diversity while maintaining generation speed.

Beyond providing practical creative control, SliderSpace opens new avenues for understanding and utilizing the latent capabilities of diffusion models. By mapping these models' visual potential into intuitive, composable directions, we take a step toward making their creative possibilities more accessible and interpretable to users.

% Image editing methods in diffusion models unlock the creativity of users. In this work we ask an alternate question: \emph{Can we organize and expose what of the diffusion model is already capable of?}.
% Existing methods for controlling image generation typically require users to manually specify edit directions for desired changes. This process is time-consuming, requires technical expertise, and limits the spontaneity of the creative process. For instance, if a user wants to adjust the smile of a generated person, they must explicitly request this edit, often through imprecise prompt engineering or model fine-tuning. This approach of predefined controls or manual specifications restricts users from fully exploring the latent capabilities of the model. There may be interesting stylistic variations or attributes that the model can generate, but users have no easy way to discover or utilize these.

% Natural visual disentanglement was an emergent property in the latent space of Generative Adversarial Models (GANs) \cite{harkonen2020ganspace,radford2015unsupervised, wu2021stylespace, shen2020interfacegan}. In particular, it has been observed that StyleGAN~\cite{karras2019style} stylespace neurons offer detailed control over many meaningful aspects of images that would be difficult to describe in words~\cite{wu2021stylespace}. However, diffusion models do not share such a compact latent space~\cite{park2023unsupervised}; and efforts to uncover such a space in the semantic embeddings of the text conditioning have met with limited success \nik{Nick - is there a specific citation you were thinking about?}.

% In this work we introduce \textbf{SliderSpace}, which takes a step towards uncovering an analogous low dimensional representation of diffusion models' visual breadth; in essence treating the diffusion model as many generators sharing parameters, where a particular generator is defined by a specific prompt. For a given prompt we sample many random seeds (and optionally prompt expansions using an LLM), generate the corresponding images, and apply an off the shelf feature extractor (in this work CLIP, but our method can be applied to any differentiable feature extractor). We use PCA to analyze these features, and for each of the leading $k$ principal components we train a LoRA \cite{} which causes the diffusion model to produces images which increase the feature magnitude along that component when passed back through the same feature extractor. This leads to a 'Slider' for each principal component, because each LoRA can be scaled and applied to the original diffusion model, continuously varying those visual features in the generated results (as measured, in our case, by CLIP).

% There are many other works that enhance the controllability of diffusion models. One common approach is enabling users to add spatial constraints to a generation either manually, or via a reference image \cite{zhang2023addingconditionalcontroltexttoimage, chen2024trainingfreeregionalpromptingdiffusion}, a second is leveraging more abstract embeddings (e.g. identity, style) extracted from a reference image \cite{ye2023ipadaptertextcompatibleimage, hertz2024stylealignedimagegeneration, li2023photomaker, shi2024instantbooth}, a third is finetuning a foundation model to better generate a concept important to the user \cite{ruiz2022dreambooth, kumari2022customdiffusion, Ryu_lora, hu2021lora}, and a fourth (most relevant to this work) is finding low-rank adaptors of the model based on a prompt or small training set which can be scaled to provide continous control over one aspect of generated image (e.g. night vs day, basic vs luxury, etc.) \cite{gandikota2023concept}. SliderSpace is complementary to all of these methods and offers something distinct. All of the other methods we are aware require the user (and / or model designer) to know in advance what type of control they want. In contrast SliderSpace assists users in discovering and controlling hidden capabilities present in the diffusion model's distribution of possible generations.

%We propose that truly intuitive creative control in a text-to-image model should meet three key criteria: \emph{discoverability}, \emph{intuitiveness}, and \emph{specificity}. The model should reveal controllable attributes that may not be immediately obvious, offer controls that are easy to understand and manipulate, and ensure each control affects a distinct attribute of the generated image.

% We demonstrate the utility and power of SliderSpace using three applications built on top of SDXL-DMD \cite{dmd}, because its fast generation speed lends itself well to the continuous control offered by SliderSpace.

% First, we study concept decomposition (Section \ref{sec:concept_exp}), where we learn sliders for a specific concept (e.g. 'monster', 'waterfall', 'car'). Through quantitative metrics of diversity and text alignment we demonstrate that the learned sliders dramatically boost the diversity of generations when randomly applied without harming text alignment; we also ask humans to qualitatively judge these results in a user study where they find the SliderSpace results to be more 'Diverse', 'Useful', and 'Creative' than our baselines.

% Second, we attempt to compare the automatic discoveries of SliderSpace to a large scale manual study of artistic styles (Section \ref{sec:art_exp}), open-sourced by ParrotZone \cite{parrotzone}. In this study SDXL was prompted with over 4300 artist names,  and based on visual inspection the cases of successful stylistic mimicry recorded. Quantitatively SliderSpace more closely matches the distribution of artistic variation discovered by ParrotZone than other baselines, and in our user studies was judged to be significantly more 'Diverse' and 'Useful' than the baselines. To our surprise humans even judged SliderSpace results to be slightly more 'Diverse' than the results generated by the manually discovered artist names of \cite{parrotzone}.

% Third, we attempt to use SliderSpace to reverse the mode collapse commonly observed in distilled few-step diffusion models relative to the original teacher model (Section \ref{sec:diverse_exp}). We quantitatively demonstrate that applying SliderSpace to SDXL-DMD leads to more closely matching the distribution of images by the original teacher, SDXL.

%Through extensive experiments on various state-of-the-art text-to-image models, we demonstrate that SliderSpace significantly enhances user control and creative expression in AI-assisted image generation tasks. Our method enables a range of applications, including concept decomposition and control, diversity improvement in generated images, customization dissection and edits, and the exploration of artistic styles inherent in the model.

% SliderSpace goes beyond providing a practical tool for enhanced creative control. By mapping the visual potential of diffusion models it can open new avenues for generative creativity and deepens our understanding of each model's hidden potential.
\section{Introduction}\label{sec:intro}

In computational finance, Monte Carlo simulations are used extensively to estimate the expected value of financial payoffs based on the solution of stochastic differential equations (SDEs) which model the evolution of stock prices, interest rates, exchange rates and other quantities \cite{glasserman04}.  Monte Carlo methods are very general and flexible, but for high accuracy it requires generating a large number of costly SDE path approximations, which has motivated research into a number of variance reduction or, equivalently, cost reduction techniques. One such method is
Multilevel Monte Carlo (MLMC), which was proposed in \cite{GILES2008} and was adapted for various applications that are summarised in \cite{Giles_overview17} and successfully combined with other methods such as quasi-Monte Carlo methods. The main idea of MLMC is to approximate the payoff using different time stepping resolutions when numerically solving the underlying SDE and to generate an optimal number of samples on each level, such that the overall computational cost is minimised subject to the desired bound on the variance. %, such that the total computational cost is minimised. 
The computational savings come from the fact that most samples are computed on the coarser levels and hence are less expensive while only a few samples from the finest levels are required \cite{GILES2008}.


Among the directions in which the computational cost 
of MLMC methods could further be reduced, an important avenue is the use of lower precision calculations, especially for the first Monte Carlo levels where the targeted accuracy is relatively low. 
 An overview of the research on mixed precision for the standard Monte Carlo (MC) framework is provided in \cite{ChowMixedPrecisionStandardMC} but only a few references study the potential of low precision computation in the MLMC framework \cite{Rounding_error_oliver}. To the best of our knowledge, the only MLMC framework with customised precision in the literature is \cite{brugger2014mixed}, but they use a uniform precision for all operations on each Monte Carlo level instead of optimising 
 the precision of each intermediary variable to reduce as much as possible the cost of path generation.
 
An important motivation for an MLMC framework with variable precision would be performing the low precision computations on reconfigurable hardware devices such as Field Programmable Gate Arrays (FPGAs). FPGAs contain customizable logic blocks and connectors that make it easy to adapt the digital circuit architecture for a specific application, leading to a highly parallel and optimised implementation. Therefore they are successfully exploited in applications that require high speed and have high computational workload, such as signal processing \cite{woods2008fpga}, and real time applications like high frequency trading \cite{HFT1,HFT2}. That is why a number of previous works in hardware architecture design implemented the MLMC algorithm to price financial options using FPGAs as accelerators, which resulted in improved speed and power efficiency compared to full CPU architectures \cite{Schryver2013AMM}. The paper \cite{lindsey2016domain} also proposed 
a Domain Specific Language to automate the configuration of FPGAs for this specific application. However, only \cite{brugger2014mixed} proposed a heuristic to reduce the precision in calculations.

In addition, all aforementioned works considered that the random number generation (RNG) is performed in single or double precision. Yet in most cases an important portion of the workload in the overall MLMC simulation comes from the RNG and in \cite{brugger2014mixed} this limited the total computational savings.
To reduce the cost of MLMC simulations in particular those based on the Geometric Brownian Motion (GBM), \cite{approximateICDF_Oliver, NestedOliver} have proposed to use approximate random numbers that are generated by applying an approximation of the inverse CDF to uniform random numbers. In \cite{NestedOliver}, the authors proposed a way to integrate these lower precision random variables into a \textit{nested} MLMC framework and completed a numerical analysis to bound the resulting error at each MC level by a product of the time step and the error in the random number approximation. The same authors show in \cite{approximateICDF_Oliver} that using approximate random variables reduces the cost of path generation by a factor 7.


In this paper we propose a nested MLMC framework that combines the use of approximate random normal variables and lower precision calculations to reduce the computational cost of MLMC even further than \cite{brugger2014mixed,NestedOliver}. We illustrate the efficiency of our framework in Matlab, after making several assumptions on the cost of operations and size of the errors that we carefully justify. We focus on the case of GBM and use the approximate RNG methods presented in \cite{approximateICDF_Oliver} as well as a new slightly modified method that combines CDF inversion and the central limit theorem. To choose the precision of the variables in the low precision path generation, we introduce a novel method to optimise the bit-widths. This optimisation is performed before the main path generation loop is executed and is based on a linear model of the payoff error  
due to rounding when computing in low precision. The error model relies on algorithmic differentiation in a similar manner to \cite{unifying-bwoptim,bitwidth-AD,ADAPT}. The bit-width optimisation procedure can be performed off-line, so this stage can be excluded from the on-line time complexity of our framework. The user specified desired accuracy is then enforced by calculating on-line the number of samples that need to be generated.

In terms of hardware design, we suggest implementing the low precision path generation on FPGAs and the full-precision ones on a CPU or GPU. 
The FPGA offers enough flexibility to define a separate bit-width for every variable in the low precision path generation, and can be reconfigured periodically to update the bit-widths when the market parameters have changed considerably. 


The paper is organized as follows : \Cref{sec:MLMC} introduces MLMC and nested MLMC to make clear the estimator that is implemented in our framework. Then in \Cref{sec:RNG} we detail the methods that could be used to obtain approximate random normally distributed numbers very cheaply for the low precision path generation. In \Cref{sec:error_model} and \Cref{sec:costModel} we propose an error model and a cost model (resp.) that we then use to formulate the optimisation problem that is solved to obtain the optimal bit-widths of fixed point variables in \Cref{sec:optimisation}. Finally we summarise our results and future directions in \Cref{sec:conclusion}.



\section{Related Work}
\label{sec:related_work}

The original investigation \cite{gibson1979ecological} on the relationship between visual perception and human action defines \emph{affordance} as the opportunities for interaction with the surrounding environment. Behavioral studies on regular and cognitively impaired persons have shown evidence that perception results in both visual and motor signals in the human brain. An extended study \cite{anderson2002attentional} shows that visual attention to the spatial characteristics of the perceived objects initiates automatic motor signals for different actions. In computer vision, human affordance learning involves novel pose prediction such that the estimated pose represents a valid human action within the scene context. The task is fundamental to many problems requiring robust semantic reasoning about the environment, such as human motion synthesis \cite{wang2021scene} and scene-aware human pose generation \cite{wang2017binge, roy2016multi, zhang2022inpaint, yao2023scene}.

Earlier methods of affordance learning have explored knowledge mining \cite{zhu2014reasoning} and multimodal feature cues \cite{roy2016multi} to address the problem. In \cite{zhu2014reasoning}, the authors use a Markov Logic Network for constructing a knowledge base by extracting several object attributes from different image and metadata sources, which can perform various downstream visual inference tasks without any additional classifier, including zero-shot affordance prediction. In \cite{roy2016multi}, the authors use depth map, surface normals, and segmentation map as multimodal cues to train a multi-scale convolutional neural network (CNN) for scene-level semantic label assignment associated with specific human actions. In \cite{do2018affordancenet}, the authors design a multi-branch end-to-end CNN with two separate pathways for object detection and affordance label assignment to achieve high real-time inference throughput. Researchers \cite{chuang2018learning} have also explored socially imposed constraints for affordance learning. In \cite{chuang2018learning}, the authors propose a graph neural network (GNN) to propagate contextual scene information from egocentric views for action-object affordance reasoning.

Probabilistic modeling of scene-aware human motion generation also involves semantic reasoning of human interaction with the environment. Initial works on human motion synthesis have taken different architectural approaches, such as sequence-to-sequence models \cite{barsoum2018hp}, generative adversarial networks (GAN) \cite{barsoum2018hp, cai2018deep, yang2018pose}, graph convolutional networks (GCN) \cite{yan2019convolutional}, and variational autoencoders (VAE) \cite{guo2020action2motion}. However, these methods have mostly ignored the role of environmental semantics. Due to potential uncertainty in human motion, in a recent approach \cite{wang2021scene}, the authors address such motion synthesis with a GAN conditioned on scene attributes and motion trajectory to predict probable body pose dynamics.

One key challenge of human affordance generation in 2D scenes is the lack of large-scale datasets with rich pose annotations. In \cite{wang2017binge}, the authors compile the only public dataset of annotated human body poses in complex 2D indoor scenes by extracting frames from sitcom videos. Aiming to generate a contextually valid human affordance at a user-defined location, the authors propose sampling the scale and deformation parameters for an existing human pose template using a VAE conditioned on the localized image patches as scene context. In \cite{zhang2022inpaint}, the authors introduce a two-stage GAN architecture for achieving a similar goal by estimating the affine bounding box parameters to localize a probable human in the scene and then generating a potential body pose at that location. The method uses the input scene, corresponding depth, and segmentation maps as semantic guidance. In \cite{yao2023scene}, the authors propose a transformer-based approach with knowledge distillation for generating human affordances in 2D indoor scenes.




\section{Methodology}
\paragraph{Preliminaries.}
We primarily focus on the homologous model merging, in which $\boldsymbol{\theta}_i$ all come from the same base model $\boldsymbol{\theta}_{\rm{base}}$. Given $K$ tasks $\{T_1,T_2,\cdots,T_K\}$ and $K$ corresponding fine-tuned models with parameters $\{\boldsymbol{\theta}_1,\boldsymbol{\theta}_2,\cdots,\boldsymbol{\theta}_K\}$, model merging aims to combine $K$ fine-tuned models into one single model simultaneously performing on $\{T_1,T_2,\cdots,T_K\}$ without post-training~\cite{method_p1_1,method_p1_2}.
Task vector~\cite{ilharco2023editing,yang2024adamerging} is a key element in merging method which could enhances the base model‘s ability or enable the model to handle other tasks. Specifically, for task $T_i$, the task vector $\boldsymbol\tau_i\in \mathbb{R}^D$ is defined as the vector obtained by subtracting the SFT weights $\boldsymbol{\theta}_i$ from the base model weight
$\boldsymbol{\theta}_{\rm{base}}$, \emph{i.e.}, $\boldsymbol\tau_i=\boldsymbol{\theta}_i-\boldsymbol{\theta}_{\rm{base}}$. The merged model could be denoted as $\boldsymbol{\theta}_m=\boldsymbol{\theta}_{\rm{base}}+\sum_i \lambda_i\boldsymbol{\tau}_i$, which $\lambda_i$ is the scaling factor measuring the importance of task vector. For clarification, we also denote the neuron set in $\boldsymbol{\theta}_i$ as $\mathcal{N}_i$, the neuron set in $\boldsymbol{\tau}_i$ as $\mathcal{T}_i$.



\begin{algorithm}[!ht]
    \caption{LED-Merging}
    \label{alg1}
    \begin{algorithmic}[1]
        \REQUIRE  base model $\boldsymbol{\theta}_{\rm{base}}$, SFT models $\{\boldsymbol{\theta}_{i}\mid i\in [K]\}$, mask ratios \{$r_{i} \mid i\in [K]\}$, scaling factors $\{\lambda_i\mid i\in[K]\}$, location datasets $\{\mathcal{X}_{i}\mid i\in[K]\}$
        \ENSURE merged parameter $\boldsymbol{\theta}_{m}$
        \STATE $\mathcal{M}\leftarrow\phi$
        \STATE $\boldsymbol{\theta}_{m}\leftarrow \boldsymbol{\theta}_{\rm{base}}$
        \FOR{$i\in [K]$}
        \STATE $I(\boldsymbol{\theta}_i)=\mathbb{E}_{x\sim \mathcal{X}_i}|\boldsymbol{\theta}_{i}\odot \nabla_{\boldsymbol{\theta}_i}\mathcal{L}(x)|$
        \STATE $I(\boldsymbol{\theta}_{\rm{base}})=\mathbb{E}_{x\sim \mathcal{X}_i}|\boldsymbol{\theta}_{\rm{base}}\odot \nabla_{\boldsymbol{\theta}_{\rm{base}}}\mathcal{L}(x)|$
        
        \STATE calculate $\mathcal{T}^{r_i}_{i}$ following Equation \ref{vote}
        \STATE  $\mathcal{M}\leftarrow \mathcal{M}\cup\{\mathcal{T}^{r_i}_i\}$
       
        
   
        
        
        \ENDFOR  
        \FOR{$i\in [K]$}
        
        \STATE calculate $\text{Disjoint}(\mathcal{T}_i^{r_i})$ use Equation~\ref{disjoint_safety}
        \STATE $\boldsymbol{m}_i \leftarrow \boldsymbol{0}$
        \FOR{$d\in \mathcal{T}_i^{r_i}$}
        \STATE $\boldsymbol{m}_{i,d}=1$
        \ENDFOR
        \STATE $\boldsymbol{\theta}_{m}\leftarrow \boldsymbol{\theta}_{m}+\lambda_i \boldsymbol{\tau}_i\odot \boldsymbol{m}_{i}$
        \ENDFOR
    \end{algorithmic}
\end{algorithm}
    %\vspace{-5pt}
\begin{figure*}[h!]
    \centering
    \includegraphics[width=\linewidth]{figs/pipeline_v2.pdf}
    \vspace{-40mm}
    \caption{Overview of our two-stage training pipeline {\ours}.}
    \label{fig:pipeline}
\end{figure*}


\paragraph{LED-Merging: Location, Election, and Disjoint Merging}
To address the neuron misidentification and interference issues in existing model merging methods, we propose LED-Merging (Location, Election, and Disjoint Merging). Specifically, previous studies \cite{modelstock, ilharco2023editing, tiesmerging} fail to accurately identify safety-related neurons in task vectors with a single magnitude score, namely \textit{neuron misidentification}. Meanwhile, there exists an interference between safety-related and utility-related task vector neurons during the merging process, namely \textit{neuron interference}. To address neuron misidentification, we first locate important neurons both in the base and fine-tuned models and then elect neurons from the task vector considering these two scores together. Subsequently, to mitigate the interference, we introduce a disjoint step, isolating these important neurons so that they influence different base neurons. The whole process is illustrated in Figure~\ref{fig:method}. 




In the location and election step, we consider the importance score from base and fine-tuned models simultaneously to locate task-specific neurons. In this way, it is more accurate than relying on the magnitude score alone because task-specific neurons with high importance score in the fine-tuned model may not necessarily score high in the base model, and vice versa.

{\textbf{Location}}.  We first calculate importance scores for each neuron in a base/fine-tuned model. Given a location dataset $\mathcal{X}_i=\{(x,y)_k\}$, where $x$ is the question and $y$ is the answer, we calculate the importance scores for the weight $\boldsymbol{\theta}_i\in\mathbb{R}^D$ in any  layer as follows~\cite{snip,spareseGPT,sun2024a}:
\begin{equation}
    I(\boldsymbol{\theta}_i)=\mathbb{E}_{x\sim \mathcal{X}_i}[\boldsymbol{\theta}_i\odot \nabla _{\boldsymbol{\theta}_i}\mathcal{L}(x)],
    \label{location}
\end{equation}
which $\mathcal{L}(x)=-\log p(y\mid x)$ is the conditional negative log-likelihood loss. We choose the SNIP score~\cite{snip} because it balances computational efficiency and performance~\cite{cq}. Please refer to Sec.~\ref{sec:ablation} for the comparison between different location methods. After computing importance scores, we choose top-$r_i$ neurons as the important neuron subset $\mathcal{N}_{i}^{r_i}$ from $I(\boldsymbol{\theta}_i)$.
 
 % After computing locating scores, we select the neurons scoring both high in base and fine-tuned models as important neurons in task vectors. Then in the disjoint step,  with preventing  polysemantic neurons  from receiving gradient updates towards different directions,
 % we use set difference to isolate the safety   and utility-related neurons  and construct corresponding masks for merging process,

{\textbf{Election}}. A natural question is how to select important neurons in the task vector $\boldsymbol{\tau}_i$ based on $I(\boldsymbol{\theta}_{\rm{base}})$ and $I(\boldsymbol{\theta}_{i})$. The important neurons in the base model may be different from neurons in the fine-tuned model. Therefore, we introduce the following election strategy to select neurons with high scores in both base and fine-tuned models:
\begin{equation}
    \mathcal{T}_i^{r_i}=\mathcal{N}_i^{r_i}\cap \mathcal{N}_{\rm{base}}^{r_i}.
    \label{vote}
\end{equation}
\emph{Remark}. We compare different choosing methods, including scoring low or high in base or fine-tuned model in Section~\ref{sec:ablation} and find that Equation \ref{vote} achieves the best performance.





{\textbf{Disjoint}}. As important neurons from different task vectors may conflict with each other at the same position, we use the set difference to disjoint the neurons from others to prevent interference:
\begin{equation}
    \text{Disjoint}(\mathcal{T}^{r_i}_{i})=\mathcal{T}^{r_i}_{i}-\mathop{\cup}\limits_{{J}\subsetneqq [K],|J|\geq 2}\mathop{\cap}\limits_{j\in {J}}\mathcal{T}^{r_j}_{j}.
    \label{disjoint_safety}
\end{equation}

Next, we construct a mask $\boldsymbol{m}_i\in\mathbb{R}^D$ to implement disjoint in the merging process. Specifically, this mask $\boldsymbol{m}_i$ is used to select neurons from $\mathcal{T}_i$. The mask ratio is $r_i$, where $r\in(0,1]$. The mask $\boldsymbol{m}_i$ can be derived from:
\begin{equation}
    \boldsymbol{m}_{i,d}=\begin{aligned} &\left\{ \begin{array}{ll} 1, & \text{if } d\in \text{Disjoint}(\mathcal{T}_{i}^{r_i}), \\ 0, & \text{otherwise}. \end{array} \right. \end{aligned}
    \label{mask_safety}
\end{equation}


% \subsection{Merging Models with Masks}
{\textbf{Merging}}. The final
merged task vector $\boldsymbol{\tau}_m$ is as follows:
\begin{equation}
    \boldsymbol{\tau}_m= \sum_i \lambda_i\boldsymbol{\tau}_{i}\odot\boldsymbol{m}_i.
    \label{merged_task_vector}
\end{equation}
We summarize the workflow in Algorithm \ref{alg1}.



\section{Experiments}
\label{sec:experiments}

\begin{figure*}[t]
\vspace{-6mm}
    \centering
    \includegraphics[width=0.8\linewidth]{figs/compare.pdf}
    \vspace{-4mm}
    \caption{\textbf{Qualitative comparison} with the baseline for generating a sequence of novel view images.  
    The results demonstrate that our method synthesizes more consistent multi-view images compared to our baseline model (Zero123). In addition, compared to SyncDreamer, our method visually maintains better similarity to the conditioned image and appears more natural.}
    \label{fig:sota_compare}
\vspace{-5mm}
\end{figure*}

\subsection{Experimental Setups}
\textbf{Dataset.}
Following previous work~\cite{zero123, SyncDreamer}, we evaluate our work on the Google Scanned Object (GSO)~\cite{GSO} dataset to verify the zero-shot novel view image synthesis capability. 
We also provide results for additional datasets in the Supplementary Material.
Specifically, we randomly select 30 objects from the GSO dataset with various object categories. 
Unlike recent approaches~\cite{mvdream, SyncDreamer} that aim to enhance the consistency of novel view synthesis models by generating multiple fixed-view images, our method can generate images from any camera pose and any number of views. Therefore, we conduct experiments under different camera pose settings to validate our approach:
specifically, 
1) \textit{16-views with free camera pose}: for each object, we circularly render 16 views with the elevation angles ranging in $[-10\degree, 40\degree]$ and the azimuth angles are evenly distributed in $[0\degree, 360\degree]$. 
2) \textit{16-views with fixed camera pose}: We maintain a constant elevation angle of $30\degree$ and uniformly sample azimuth angles (same as SyncDreamer~\cite{SyncDreamer}).
3) \textit{32-views with free camera pose}: Similar to the first setting, but we sample 32 views.
It's important to note that our method does not require additional training or fine-tuning on any datasets.

\noindent\textbf{Metrics.}
To validate the effectiveness of our method, we mainly evaluate it based on three criteria:
1) \textit{Quality Score}. We evaluate the image quality of synthesized multi-view images by measuring their similarity with ground truth images. Following prior research~\cite{zero123, sparsefusion}, we report the similarity between the synthesized images and the ground truth images with standard metrics: PSNR, SSIM~\cite{ssim}, and LPIPS~\cite{lpips}.
2) \textit{Multi-view Consistency Score}. As the primary goal of our work is to improve the consistency of generated images, we also employ the 3D consistency score~\cite{3dim} to verify the consistency among the synthesized images. Specifically, we train an Instant-NGP~\cite{instant_ngp} with the input image and part of the synthesized novel view images of our model and evaluate the similarity between the remaining synthesized images and the rendered images of Instant-NGP. For the synthesized multi-view images of each object, we allocate $3/4$ for training and reserve the remaining $1/4$ for validation.
Intuitively, if the consistency of synthesized images is improved, the NeRF-like model will train a better object representation, and the re-rendered images will agree more with the validation images.
3) \textit{Input Consistency Score}. To assess the faithfulness of synthesized images in preserving the identity of the input condition image, we introduce the input consistency score. This score calculates the similarity of each synthesized image with the input condition image, utilizing the LPIPS metric.

In addition, we use synthesized multi-view images to train a neural 3D reconstruction model (NeuS~\cite{neus}) and report commonly used Chamfer Distances (CD) and Volume IoUs between the trained 3D model and the ground truth.

\noindent\textbf{Baselines.}
Given that our main goal is to improve the consistency of the trained baseline model without further fine-tuning, we mainly compare our approach with the used baseline model Zero123~\cite{zero123}. Additionally, we compare our method to the SOTA approaches such as PGD~\cite{tseng2023consistent} and SyncDreamer~\cite{SyncDreamer} using the same Zero123 base model.

\noindent\textbf{Implementation Details.}
We use the official checkpoint provided by Zero123~\cite{zero123}, which is trained on objaverse~\cite{objaverse} for 165,000 steps. We inject our epipolar attention layer after step $T=4$ and layer $L=10$ by default. We find that feature fusion weight $\alpha=0.5$, and the number of context views $M=2$ work better.

\begin{table}[t]
\centering
\caption{Comparison of multi-view consistency, image quality, and input consistency of synthesized multi-view images at the 16-view setting with free camera pose.}
\label{tab:view16_free_compare}
\vspace{-2mm}
\scalebox{0.6}{
\begin{tabular}{c ccc ccc c}
\toprule
              & \multicolumn{3}{c}{Multi-view Consistency} & \multicolumn{3}{c}{Quality Score} & \multicolumn{1}{c}{Input Consis.} \\
              \cmidrule(lr){2-4} \cmidrule(lr){5-7} \cmidrule(lr){8-8}
              & PSNR$\uparrow$  & SSIM$\uparrow$ & LPIPS$\downarrow$ 
              & PSNR$\uparrow$  & SSIM$\uparrow$ & LPIPS$\downarrow$ 
              & LPIPS$\downarrow$ 
              \\ \midrule

Zero123
& 15.225        & 0.645       & 0.408
& 14.255        & 0.747       &	0.208
& 0.303         
\\
SyncDreamer
& 14.830        & 0.626       & 0.434
& 12.650        & 0.713       &	0.254
& 0.317         
\\
Ours 
& \best{18.300}	& \best{0.734}	& \best{0.355}
& \best{14.947}	& \best{0.763}	& \best{0.191}
& \best{0.282}
\\

\bottomrule
\end{tabular}
}
\end{table}

\begin{table}[t]
\vspace{-1mm}
\centering
\caption{Comparison of multi-view consistency, image quality, and input consistency at the 16-view setting with fixed camera pose as SyncDreamer~\cite{SyncDreamer}.}
\label{tab:view16_fxied_compare}
\vspace{-3mm}
\scalebox{0.6}{
\begin{tabular}{c ccc ccc c}
\toprule
              & \multicolumn{3}{c}{Multi-view Consistency} & \multicolumn{3}{c}{Quality Score} & \multicolumn{1}{c}{Input Consis.} \\
              \cmidrule(lr){2-4} \cmidrule(lr){5-7} \cmidrule(lr){8-8}
              & PSNR$\uparrow$  & SSIM$\uparrow$ & LPIPS$\downarrow$ 
              & PSNR$\uparrow$  & SSIM$\uparrow$ & LPIPS$\downarrow$ 
              & LPIPS$\downarrow$ 
              \\ \midrule

Zero123
& 16.556        & 0.682       & 0.378
& 14.592        & 0.750       &	0.207
& 0.305         
\\
SyncDreamer
& \best{22.424}        & \best{0.812}       & \best{0.268}
& 15.269        & 0.749       &	0.196
& 0.300         
\\
Ours 
& 21.151	& 0.780	& 0.302
& \best{15.293}	& \best{0.764}	& \best{0.184}
& \best{0.287}
\\

\bottomrule
\end{tabular}
}
\vspace{-4mm}
\end{table}


\subsection{Comparison With Baseline Models}
The quantitative comparison on three settings are shown in Tab.~\ref{tab:view16_free_compare}, Tab.~\ref{tab:view16_fxied_compare}, and Tab.~\ref{tab:view32_free_compare}. The qualitative comparison is shown in Fig.~\ref{fig:sota_compare}.

\begin{table}[t]
\centering
\caption{Comparison of multi-view consistency and image quality scores of synthesized multi-view images at the 32-view setting with free camera pose.}
\vspace{-3mm}
\label{tab:view32_free_compare}
\scalebox{0.7}{
\begin{tabular}{c ccc ccc}
\toprule
              & \multicolumn{3}{c}{Multi-view Consistency} & \multicolumn{3}{c}{Quality Score} \\
              \cmidrule(lr){2-4} \cmidrule(lr){5-7}
              & PSNR$\uparrow$  & SSIM$\uparrow$ & LPIPS$\downarrow$ 
              & PSNR$\uparrow$  & SSIM$\uparrow$ & LPIPS$\downarrow$ 
              \\ \midrule

Zero123
& 16.515        & 0.694       & 0.378
& 15.142        & 0.733       &	0.211
\\
PGD~\cite{tseng2023consistent}
& 18.481        & 0.720       & 0.343
& 15.281        & 0.739       &	0.205
\\
Ours 
& \best{20.655}	& \best{0.792}	& \best{0.305}
& \best{15.268}	& \best{0.742}	& \best{0.203}
\\

\bottomrule
\end{tabular}
}
\vspace{-3mm}
\end{table}

\begin{table*}
  [t]
  \centering
  \resizebox{\textwidth}{!}{%
  \begin{tabular}{cccccccccccc}
    \toprule \multicolumn{2}{c}{Components}                                                             & \multicolumn{5}{c}{Re-executability Rate (\%)} & \multicolumn{5}{c}{Readability (\#)} \\
    \cmidrule(lr){1-2} \cmidrule(lr){3-7} \cmidrule(lr){8-12}        \hspace{8pt}\labelemoji\hspace{8pt}                                                                & \hspace{8pt}\toolemoji\hspace{8pt}                                      & O0                                 & O1             & O2             & O3             & AVG            & O0             & O1             & O2             & O3             & AVG            \\
    \hline
    \rowcolor[rgb]{0.93,0.93,0.93}\multicolumn{12}{c}{\textbf{Initialize with LLM4Decompile-End-6.7B~\citep{llm4decompile}}}   \\
    \xmark                                                                                              & \xmark                                    & 69.51                              & 46.95          & 50.61          & 46.34          & 53.35          & 3.98 & 3.41 & 3.44 & 3.38 & 3.55 \\
    \cmark                                                                                              & \xmark                                    & 75.61                              & 50.61          & 50.00          & 50.00          & 56.55          & 4.01 & 3.44 & 3.39 & \textbf{3.49} & 3.58 \\
    \xmark                                                                                              & \cmark                                    & 83.54                     & \textbf{56.10}          & 51.22          & 50.61 & 60.37 & 4.05 & 3.51 & 3.51 & 3.42 & 3.62 \\
    \cmark                                                                                              & \cmark                                    & \textbf{85.37}                            & \textbf{56.10}                     & \textbf{51.83} & \textbf{52.43}          & \textbf{61.43} & \textbf{4.13} & \textbf{3.60} & \textbf{3.54} & \textbf{3.49} & \textbf{3.69} \\

    \rowcolor[rgb]{0.93,0.93,0.93}\multicolumn{12}{c}{\textbf{Initialize with Deepseek-Coder-6.7B-base~\citep{deepseekcoder}}} \\
    \xmark                                                                                              & \xmark                                    & 59.15                              & 35.98          & 39.02          & 37.80          & 42.99          & 3.71 & 3.05 & 3.16 & 3.05 & 3.24 \\
    \cmark                                                                                              & \xmark                                    & 66.46                              & 41.46          & 38.41          & 36.59          & 45.73          & 3.76 & 3.17 & \textbf{3.21} & 3.08 & 3.31 \\
    \xmark                                                                                              & \cmark                                    & 70.73                              & 39.63          & 39.02          & 40.24          & 47.41          & 3.90 & 3.17 & 3.08 & 3.11 & 3.31 \\
    \cmark                                                                                              & \cmark                                    & \textbf{79.88}                     & \textbf{45.73} & \textbf{43.90} & \textbf{42.68} & \textbf{53.05} & \textbf{3.96} & \textbf{3.21} & 3.18 & \textbf{3.19} & \textbf{3.38} \\
    \bottomrule
  \end{tabular}%
  }
  \caption{The ablation study of different methods across four optimization levels
  (O0, O1, O2, O3), as well as their average scores (AVG). The results in bold represent the optimal performance. The ~\labelemoji~ and ~\toolemoji~ means Relabedling and Function Call. \textbf{Bold} denotes the best performance.}
  \label{tab:ablation}
\end{table*}



\begin{figure*}[ht]
    \centering
    \begin{minipage}{0.65\textwidth}
        \centering
        \includegraphics[width=0.95\linewidth]{figs/ablation.pdf}
        \vspace{-2mm}
        \captionof{figure}{Qualitative Comparison for different design choices. Our method, employing multi-view epipolar attention, demonstrates the best consistency.}
        \label{fig:ablation}
    \end{minipage}\hfill
    \begin{minipage}{0.33\textwidth}
        \centering
        \includegraphics[width=0.8\linewidth]{figs/neus_ver.pdf}
        \vspace{-3mm}
        \caption{Our method shows better direct 3D reconstruction~\cite{neus}.}
        \label{fig:neus}
    \end{minipage}
    \vspace{-5mm}
\end{figure*}

\noindent\textbf{Multi-view Consistency.}
Tab.~\ref{tab:view16_fxied_compare} presents the 3D consistency scores compared to our baseline model (Zero123) and SyncDreamer. The results indicate a significant improvement across all three metrics achieved by our method when compared with Zero123.
While our method exhibits a marginally lower numerical consistency score compared to SyncDreamer, it enables the synthesis of images with arbitrary camera poses.	
This capability is illustrated in Tab.~\ref{tab:view16_free_compare}, where our method consistently enhances consistency with changes in camera pose settings, whereas SyncDreamer fails to do so and exhibits inferior results compared to Zero123.
Furthermore, our method facilitates the synthesis of multi-view images with any number of camera views. This versatility is demonstrated in Tab.~\ref{tab:view32_free_compare}, where our method continues to achieve significant improvements in consistency scores, while SyncDreamer is unable to operate under such conditions.	

Meanwhile, Fig.~\ref{fig:sota_compare} provides a qualitative comparison with the baseline. While both our method and SyncDreamer enhance consistency, our method visually preserves better similarity to the input image, including color and texture details. The input consistency score further corroborates this.

\noindent\textbf{Image Quality.}
While our primary goal centers around enhancing the consistency of synthesized multi-view images, we also evaluate the image quality by comparing the similarity with the ground truth images. The results shown in Tab.~\ref{tab:view16_free_compare}, Tab.~\ref{tab:view16_fxied_compare}, and Tab.~\ref{tab:view32_free_compare} indicate that our method also enhances the image quality under different settings besides improving the consistency.
Moreover, our method shows better image quality compared with SyncDreamer even in the 16-view setting with fixed camera pose.

\noindent\textbf{Input Consistency.}
Input consistency terms whether the results align with the input image.
Fig.~\ref{fig:sota_compare} illustrates that both our method and SyncDreamer enhance multi-view consistency. However, the color and texture details of SyncDreamer's results diverge from the input image and appear visually unnatural.
This discrepancy is evident in the input consistency score presented in Tab.~\ref{tab:view16_fxied_compare}, indicating lower similarity with the condition image in the SyncDreamer results.	

\subsection{Ablation Study}
The overall quantitative results are shown in Tab.~\ref{tab:ablation}, and the qualitative comparisons are shown in Fig.~\ref{fig:ablation}.

\noindent \textbf{Full Attention \vs Epipolar Attention.}
The results presented in Tab.\ref{tab:ablation} and Fig.\ref{fig:ablation} demonstrate that our epipolar attention mechanism can synthesize more consistent multi-view images compared with full attention. Furthermore, our epipolar attention achieves a greater performance improvement compared to full attention when using multiple reference images. This could be attributed to the fact that our epipolar attention more effectively localizes target information, as depicted in Fig.~\ref{fig:full_attn_compare}, thereby reducing noise from the reference images. In the multi-view setting, where multiple reference images are utilized, this noise reduction becomes particularly crucial.
Moreover, it is noteworthy that the epipolar attention mechanism consumes less GPU memory compared to our baseline, as discussed in Sec.~\ref{sec:attn_analysis}.

\noindent \textbf{Attending Single-View \vs Multi-View.}
Applying the epipolar attention significantly improves the consistency between the input and target views. However, the consistency between different views in the unobserved regions of the input view is not well preserved.
After implementing our epipolar attention in the multi-view setting, the consistency across the generated multi-view images is further improved. The last row in Tab.~\ref{tab:ablation} shows that after applying our multi-view epipolar attention, the consistency score is further improved compared with the single-view setting. Besides, the qualitative result in Fig.~\ref{fig:ablation} also shows better consistency among different target views.



\begin{table}[t]
\centering
\vspace{-1mm}
\caption{Comparison of 3D reconstruction results. Our method significantly improves the reconstruction quality.}
\vspace{-3mm}
\label{tab:neus}
\scalebox{0.7}{
\begin{tabular}{c cc}
\toprule
              &  Chamfer Dist.$\downarrow$  & Volume IoU$\uparrow$
\\ \midrule

            Zero123         & 0.017         & 0.819    \\
            SyncDreamer     & \best{0.013}         & \best{0.847}    \\
            Ours            & 0.014	& 0.842 \\

\bottomrule
\end{tabular}
}
\vspace{-5mm}
\end{table}


\vspace{-2mm}
\subsection{Downstream Application}
\vspace{-2mm}
To demonstrate the effectiveness of our method, we also applied it to the downstream 3D reconstruction task. Specifically, we trained the NeuS model~\cite{neus} directly using images synthesized by our method, Zero123, and SyncDreamer, respectively.
The quantitative results in Tab.~\ref{tab:neus} show that the consistent multi-view images synthesized by our method can significantly improve the 3D reconstruction quality.
Additionally, our method exhibits similar performance to SyncDreamer which requires time-consuming re-training.
The qualitative results in Fig.~\ref{fig:neus} show that it is challenging to train the NeuS model directly due to the lack of consistency in the images generated by Zero123. In contrast, our method generates more consistent multi-view images and, therefore, better reconstructs the geometry and texture details.
We show improvements on other downstream applications such as image-to-3D in the Supplementary Material.


\section{Conclusion}

%In this paper, w
We propose a new PEFT method called DiffoRA, which enables efficient and adaptive LLM fine-tuning based on LoRA. 
Instead of adjusting every interior rank, 
%of the decomposition matrices 
%of all modules, 
we argue that adopting LoRA module-wisely is sufficient. 
To achieve this, we construct a DAM to select the modules that are most suitable and essential to fine-tune. We theoretically analyze how the DAM impacts the convergence rate and generalization capability.
%of the pre-trained model. 
Furthermore, we adopt continuous relaxation and discretization to establish DAM.
%for each task. 
To alleviate the issue of discretization discrepancy, we utilize the weight-sharing strategy for optimization. 
%We fully implement our method and t
The experimental results demonstrate that our DiffoRA works consistently better than the baselines across all benchmarks. 
\begin{acks}
We thank Gordon Guocheng Qian and Kuan-Chieh (Jackson) Wang for their feedback and support.
\end{acks}
\section*{Supplementary Materials}
\label{sec:supp}

\subsection*{1. S2I+I2V \textbf{\textit{vs}} S2V}
\label{sec:supp:s2i2v}
\begin{figure}[h]
	\centering
	\includegraphics[width=0.98\columnwidth]{figures/s2i2v.pdf} 
	\caption{Comparison of subject-to-image-to-video \cite{dreamina} and subject-to-video (ours).}
	\label{fig:s2i2v}
\end{figure}

As mentioned in the main text, combining subject-to-image (S2I) and image-to-video (I2V) can achieve similar effects to subject-to-video (S2V), but there are some difficult limitations. Firstly, existing methods \cite{guo2024pulid, huang2024realcustom, dreamina} for generating subject-consistent images or ID-consistent images still exhibit noticeable artificial artifacts, and there is significant room for improvement in the dimension of subject consistency. Equally important, I2V cannot ensure consistency of the subject during motion. As illustrated in Figure \ref{fig:s2i2v}, when inputting a reference portrait, S2I first generates a reference image for the initial frame of I2V. If the initial frame includes a back view or occlusions, I2V may "imagine" a false ID during the process of removing the occlusion, leading to a failure in maintaining consistency.


\subsection*{2. Copy-paste problem}
\label{sec:supp:copypaste}

\begin{figure}[h]
	\centering
	\includegraphics[width=0.98\columnwidth]{figures/copypaste.pdf} 
	\caption{Intuitive cases of copy-paste problems. The red font in the text prompt does not function as intended.}
	\label{fig:copypaste}
\end{figure}

In the field of video generation, the copy-paste issue is particularly prominent, manifesting as the leakage of image content into the generated video. Some methods sample keyframes from a video and use them as image conditions to reconstruct the video. However, this approach allows the model to employ shortcut learning strategies, simplifying the content understanding process. Figure \ref{fig:copypaste} shows examples of the copy-paste issue, sampling from the initial, middle, and final frames: In the first row, the girl's expression remains unchanged, ignoring the text prompt. In the second row, the cartoon character's movements remain stiff and identical to the reference. The third row illustrates a common case where the generated video is too similar to I2V, diminishing the effectiveness of scene-related text and reducing content diversity. To address this, we focus on constructing cross-video multi-subject pairings, ensuring subjects match in content while allowing for non-rigid deformations and changes in color distribution, thereby avoiding the copy-paste problem.

\begin{figure*}[t]
	\centering
	\includegraphics[width=\textwidth]{figures/data_dist.pdf} 
	\caption{Distribution of object frequencies and class.}
	\label{fig:data_dist}
\end{figure*}


\subsection*{3. Ablation study supplement}
\begin{figure}[t]
	\centering
	\includegraphics[width=0.98\columnwidth]{figures/confused_case.pdf} 
	\caption{Examples of multi-subject confusion: On the left are the multi-subject reference images, while the right columns present the cases of confusion and the successful cases after improvement.}
	\label{fig:confused}
\end{figure}

\begin{table}[b]
	\centering
	\resizebox{\columnwidth}{!}{
	\begin{tabular}{lcc}
		\toprule
		& \textbf{\textit{w/o}} text-image alignment & \textbf{\textit{w/}} text-image alignment \\
		\midrule
		Success rate & 65\% & 95\% \\
		\bottomrule
	\end{tabular}
	}
	\caption{Success rate of multi-subject generation with and without text-image alignment.}
	\label{tab:success_rate}
\end{table}

\textbf{Multi-subject confusion issue.} When multiple reference subjects are input simultaneously, appearance confusion may occur. Our solution aligns text descriptions with video subjects during training, ensuring distinct descriptions for each subject. During inference, a rephraser adjusts the input text prompts to align with the training data format. For example, in the first row of Figure \ref{fig:confused}, the original prompt "A family of three is having a meal at the table" caused confusion. The rephrased prompt "a woman in black, a young girl in white, and an elderly man in a suit eating together at the table" resolved this issue. In the second row of Figure \ref{fig:confused}, the original prompt "a girl in casual clothes walking by the beach" failed to match the reference. The rephrased prompt "a girl in a white T-shirt and jeans walking by the beach" successfully matched the reference.
Quantitative analysis, shown in Table \ref{tab:success_rate}, indicates a significant increase in the success rate of subject-consistent generation with this method.
Aligning image and text is crucial for multi-subject generation tasks. This approach, which requires no additional complex data structures or model designs, significantly optimizes the multi-subject confusion problem.


\subsection*{4. Data pipeline for face ID }
\label{sec:supp:facedata}

\begin{figure}[h]
	\centering
	\includegraphics[width=0.98\columnwidth]{figures/face_pipe.pdf} 
	\caption{Facial data processing pipeline for constructing ID cross-pair}
	\label{fig:facepipe}
\end{figure}

To enhance facial ID consistency, we developed an additional data pipeline for processing facial data. As shown in Figure \ref{fig:facepipe}, the facial data pipeline reuses the scene segmentation, video filtering, and annotation steps from the general subject pipeline. During the detection stage, we use an internal facial detection tool to identify each face in the video reference frames and calibrate it with the VLM \cite{Qwen2.5-VL} results from the captions using IOU (Intersection Over Union). In the matching stage, we calculate facial similarity using Arcface \cite{deng2019arcface} features and add a deduplication operator \cite{idealods2019imagededup} to further calibrate the recognition results.


\subsection*{5. Data distribution}
\textbf{Distribution of video object quantities.} We sample three frames at [0.05, 0.5, 0.95] of the video timeline and perform object detection on these frames. We filter out objects that meet the following criteria: (1) objects that are small in size or occupy a small proportion of the frame; (2) objects with a high degree of overlap with other objects; and (3) incomplete objects judged by the VLM \cite{Qwen2.5-VL}. The final distribution of the number of objects per video is shown in the table on the left side of Figure \ref{fig:data_dist}.

\noindent \textbf{Distribution of video object types.} We use LLM \cite{GPT4} to classify the noun fields in all captions into the following categories: human, animal, clothes, product, landmark, IP character, and others. The distribution is shown in the accompanying Figure \ref{fig:data_dist}, with human, clothes, and product categories accounting for the majority.

\subsection*{6. Model architecture}
\label{sec:supp:model}

\begin{figure}[t]
	\centering
	\includegraphics[width=\columnwidth]{figures/architecture.pdf} 
	\caption{The supplementary diagram of the Phantom framework.}
	\label{fig:architecture}
\end{figure}
The architecture of the \textit{Phantom} model is shown in Figure \ref{fig:architecture}, which supplements the missing details in the main text. As illustrated, it integrates the VAE and CLIP encoders to process reference images, while the text encoder handles captions. The encoded features are combined with added noise and processed through multiple MMDiT blocks, resulting in the final output. This design ensures a balance between detailed reconstruction and high-level information alignment while also guaranteeing a unified training paradigm for single and multiple subject inputs.


\subsection*{7. Qualitative analysis}
\label{sec:supp:qualitative}

Qualitative comparison results of single-subject consistency generation are shown in Figure \ref{fig:supp_sip}. Firstly, Vidu \cite{Vidu} performs well in both image consistency and text following for the first two cases but fails in the third shoe case with two different seeds. The effectiveness of Pika \cite{Pika} is evident, as the first two cases show significant disadvantages in maintaining subject consistency, tending towards a cartoonish appearance. The major issue with Kling \cite{Keling} is that most cases resemble the I2V mode, where the initial frame directly replicates the reference image (as indicated by the red box in Figure \ref{fig:supp_sip}), followed by subject motion generated based on text, thereby limiting the effectiveness of textual descriptions.

Figure \ref{fig:supp_mip} displays some qualitative comparisons of multi-subject consistency generation. Firstly, Kling still reflects the I2V pattern, appearing unnatural transitions in the first few frames of the video. Additionally, in the second example with three reference images of persons, confusion issues are evident in all methods except ours. Vidu shows the first man's clothes and the second man's face, and includes a person unrelated to the reference images. Pika misses one person, and Kling also lacks one person and shows the same issue as Vidu. The final case demonstrates that Vidu and Pika appear more realistic, indicating that their text responsiveness is stronger than their subject consistency.

\begin{figure*}[t]
	\centering
	\includegraphics[width=0.78\textwidth]{figures/supp_sip.pdf} 
	\caption{Comparative results of single reference subject-to-video generation.}
	\label{fig:supp_sip}
\end{figure*}


\begin{figure*}[t]
	\centering
	\includegraphics[width=0.78\textwidth]{figures/supp_mip.pdf} 
	\caption{Comparative results of multi-reference subject-to-video generation.}
	\label{fig:supp_mip}
\end{figure*}


\subsection*{8. Limitations and future work}
\label{sec:supp:limitations}
\textbf{Limitations.} While \textit{Phantom} demonstrates strong performance in subject-consistent video generation, several challenges persist. First, handling uncommon subjects (e.g., rare animals or niche objects) remains difficult due to biases in training data coverage. Second, complex multi-subject interactions (e.g., overlapping movements or fine-grained spatial relationships) often lead to partial confusion or inconsistent relative subject sizes. Third, generating videos that strictly adhere to intricate text responses (e.g., precise spatial layouts or nuanced temporal dynamics) is limited by the current cross-modal alignment mechanism. These issues stem from three core factors: (1) gaps in dataset diversity, particularly for non-human-centric scenarios; (2) the inherent rigidity of the reference image injection strategy, which struggles to disentangle entangled features from multiple subjects; and (3) biases inherited from pre-trained base models and visual encoders, such as CLIP’s semantic oversimplification and VAE’s over-referenced details. 

\noindent \textbf{Future work.} Addressing these limitations will require multi-faceted innovations, and we propose the following directions: 

\begin{itemize}
    \item Enhanced Cross-Modal Alignment: Develop adaptive injection mechanisms that dynamically prioritize text or image conditions based on task requirements, reducing content leakage and improving text responsiveness. 
    
    \item Spatiotemporal Disentanglement: Integrate spatial-aware attention modules and physics-inspired motion priors to better model multi-subject interactions and enforce consistent relative scales. 
    
    \item Bias-Aware Training: Mitigate dataset and model biases through adversarial debiasing techniques and synthetic data augmentation for underrepresented subjects. 
    
    \item Granular Control: Explore auxiliary control signals (e.g., depth maps, segmentation masks) to complement text prompts, enabling precise alignment with complex instructions. 
    
    \item Foundation Model Adaptation: Fine-tune pre-trained encoders on domain-specific data (e.g., medical imaging, animation) to broaden \textit{Phantom}’s applicability while preserving generalization. 
\end{itemize}

By advancing these areas, \textit{Phantom} could evolve into a versatile tool for industrial applications such as virtual try-ons, interactive storytelling, and educational content creation, ultimately narrowing the gap between academic research and real-world demands.






%%
%% The next two lines define the bibliography style to be used, and
%% the bibliography file.
% \section*{Supplementary Materials}
\label{sec:supp}

\subsection*{1. S2I+I2V \textbf{\textit{vs}} S2V}
\label{sec:supp:s2i2v}
\begin{figure}[h]
	\centering
	\includegraphics[width=0.98\columnwidth]{figures/s2i2v.pdf} 
	\caption{Comparison of subject-to-image-to-video \cite{dreamina} and subject-to-video (ours).}
	\label{fig:s2i2v}
\end{figure}

As mentioned in the main text, combining subject-to-image (S2I) and image-to-video (I2V) can achieve similar effects to subject-to-video (S2V), but there are some difficult limitations. Firstly, existing methods \cite{guo2024pulid, huang2024realcustom, dreamina} for generating subject-consistent images or ID-consistent images still exhibit noticeable artificial artifacts, and there is significant room for improvement in the dimension of subject consistency. Equally important, I2V cannot ensure consistency of the subject during motion. As illustrated in Figure \ref{fig:s2i2v}, when inputting a reference portrait, S2I first generates a reference image for the initial frame of I2V. If the initial frame includes a back view or occlusions, I2V may "imagine" a false ID during the process of removing the occlusion, leading to a failure in maintaining consistency.


\subsection*{2. Copy-paste problem}
\label{sec:supp:copypaste}

\begin{figure}[h]
	\centering
	\includegraphics[width=0.98\columnwidth]{figures/copypaste.pdf} 
	\caption{Intuitive cases of copy-paste problems. The red font in the text prompt does not function as intended.}
	\label{fig:copypaste}
\end{figure}

In the field of video generation, the copy-paste issue is particularly prominent, manifesting as the leakage of image content into the generated video. Some methods sample keyframes from a video and use them as image conditions to reconstruct the video. However, this approach allows the model to employ shortcut learning strategies, simplifying the content understanding process. Figure \ref{fig:copypaste} shows examples of the copy-paste issue, sampling from the initial, middle, and final frames: In the first row, the girl's expression remains unchanged, ignoring the text prompt. In the second row, the cartoon character's movements remain stiff and identical to the reference. The third row illustrates a common case where the generated video is too similar to I2V, diminishing the effectiveness of scene-related text and reducing content diversity. To address this, we focus on constructing cross-video multi-subject pairings, ensuring subjects match in content while allowing for non-rigid deformations and changes in color distribution, thereby avoiding the copy-paste problem.

\begin{figure*}[t]
	\centering
	\includegraphics[width=\textwidth]{figures/data_dist.pdf} 
	\caption{Distribution of object frequencies and class.}
	\label{fig:data_dist}
\end{figure*}


\subsection*{3. Ablation study supplement}
\begin{figure}[t]
	\centering
	\includegraphics[width=0.98\columnwidth]{figures/confused_case.pdf} 
	\caption{Examples of multi-subject confusion: On the left are the multi-subject reference images, while the right columns present the cases of confusion and the successful cases after improvement.}
	\label{fig:confused}
\end{figure}

\begin{table}[b]
	\centering
	\resizebox{\columnwidth}{!}{
	\begin{tabular}{lcc}
		\toprule
		& \textbf{\textit{w/o}} text-image alignment & \textbf{\textit{w/}} text-image alignment \\
		\midrule
		Success rate & 65\% & 95\% \\
		\bottomrule
	\end{tabular}
	}
	\caption{Success rate of multi-subject generation with and without text-image alignment.}
	\label{tab:success_rate}
\end{table}

\textbf{Multi-subject confusion issue.} When multiple reference subjects are input simultaneously, appearance confusion may occur. Our solution aligns text descriptions with video subjects during training, ensuring distinct descriptions for each subject. During inference, a rephraser adjusts the input text prompts to align with the training data format. For example, in the first row of Figure \ref{fig:confused}, the original prompt "A family of three is having a meal at the table" caused confusion. The rephrased prompt "a woman in black, a young girl in white, and an elderly man in a suit eating together at the table" resolved this issue. In the second row of Figure \ref{fig:confused}, the original prompt "a girl in casual clothes walking by the beach" failed to match the reference. The rephrased prompt "a girl in a white T-shirt and jeans walking by the beach" successfully matched the reference.
Quantitative analysis, shown in Table \ref{tab:success_rate}, indicates a significant increase in the success rate of subject-consistent generation with this method.
Aligning image and text is crucial for multi-subject generation tasks. This approach, which requires no additional complex data structures or model designs, significantly optimizes the multi-subject confusion problem.


\subsection*{4. Data pipeline for face ID }
\label{sec:supp:facedata}

\begin{figure}[h]
	\centering
	\includegraphics[width=0.98\columnwidth]{figures/face_pipe.pdf} 
	\caption{Facial data processing pipeline for constructing ID cross-pair}
	\label{fig:facepipe}
\end{figure}

To enhance facial ID consistency, we developed an additional data pipeline for processing facial data. As shown in Figure \ref{fig:facepipe}, the facial data pipeline reuses the scene segmentation, video filtering, and annotation steps from the general subject pipeline. During the detection stage, we use an internal facial detection tool to identify each face in the video reference frames and calibrate it with the VLM \cite{Qwen2.5-VL} results from the captions using IOU (Intersection Over Union). In the matching stage, we calculate facial similarity using Arcface \cite{deng2019arcface} features and add a deduplication operator \cite{idealods2019imagededup} to further calibrate the recognition results.


\subsection*{5. Data distribution}
\textbf{Distribution of video object quantities.} We sample three frames at [0.05, 0.5, 0.95] of the video timeline and perform object detection on these frames. We filter out objects that meet the following criteria: (1) objects that are small in size or occupy a small proportion of the frame; (2) objects with a high degree of overlap with other objects; and (3) incomplete objects judged by the VLM \cite{Qwen2.5-VL}. The final distribution of the number of objects per video is shown in the table on the left side of Figure \ref{fig:data_dist}.

\noindent \textbf{Distribution of video object types.} We use LLM \cite{GPT4} to classify the noun fields in all captions into the following categories: human, animal, clothes, product, landmark, IP character, and others. The distribution is shown in the accompanying Figure \ref{fig:data_dist}, with human, clothes, and product categories accounting for the majority.

\subsection*{6. Model architecture}
\label{sec:supp:model}

\begin{figure}[t]
	\centering
	\includegraphics[width=\columnwidth]{figures/architecture.pdf} 
	\caption{The supplementary diagram of the Phantom framework.}
	\label{fig:architecture}
\end{figure}
The architecture of the \textit{Phantom} model is shown in Figure \ref{fig:architecture}, which supplements the missing details in the main text. As illustrated, it integrates the VAE and CLIP encoders to process reference images, while the text encoder handles captions. The encoded features are combined with added noise and processed through multiple MMDiT blocks, resulting in the final output. This design ensures a balance between detailed reconstruction and high-level information alignment while also guaranteeing a unified training paradigm for single and multiple subject inputs.


\subsection*{7. Qualitative analysis}
\label{sec:supp:qualitative}

Qualitative comparison results of single-subject consistency generation are shown in Figure \ref{fig:supp_sip}. Firstly, Vidu \cite{Vidu} performs well in both image consistency and text following for the first two cases but fails in the third shoe case with two different seeds. The effectiveness of Pika \cite{Pika} is evident, as the first two cases show significant disadvantages in maintaining subject consistency, tending towards a cartoonish appearance. The major issue with Kling \cite{Keling} is that most cases resemble the I2V mode, where the initial frame directly replicates the reference image (as indicated by the red box in Figure \ref{fig:supp_sip}), followed by subject motion generated based on text, thereby limiting the effectiveness of textual descriptions.

Figure \ref{fig:supp_mip} displays some qualitative comparisons of multi-subject consistency generation. Firstly, Kling still reflects the I2V pattern, appearing unnatural transitions in the first few frames of the video. Additionally, in the second example with three reference images of persons, confusion issues are evident in all methods except ours. Vidu shows the first man's clothes and the second man's face, and includes a person unrelated to the reference images. Pika misses one person, and Kling also lacks one person and shows the same issue as Vidu. The final case demonstrates that Vidu and Pika appear more realistic, indicating that their text responsiveness is stronger than their subject consistency.

\begin{figure*}[t]
	\centering
	\includegraphics[width=0.78\textwidth]{figures/supp_sip.pdf} 
	\caption{Comparative results of single reference subject-to-video generation.}
	\label{fig:supp_sip}
\end{figure*}


\begin{figure*}[t]
	\centering
	\includegraphics[width=0.78\textwidth]{figures/supp_mip.pdf} 
	\caption{Comparative results of multi-reference subject-to-video generation.}
	\label{fig:supp_mip}
\end{figure*}


\subsection*{8. Limitations and future work}
\label{sec:supp:limitations}
\textbf{Limitations.} While \textit{Phantom} demonstrates strong performance in subject-consistent video generation, several challenges persist. First, handling uncommon subjects (e.g., rare animals or niche objects) remains difficult due to biases in training data coverage. Second, complex multi-subject interactions (e.g., overlapping movements or fine-grained spatial relationships) often lead to partial confusion or inconsistent relative subject sizes. Third, generating videos that strictly adhere to intricate text responses (e.g., precise spatial layouts or nuanced temporal dynamics) is limited by the current cross-modal alignment mechanism. These issues stem from three core factors: (1) gaps in dataset diversity, particularly for non-human-centric scenarios; (2) the inherent rigidity of the reference image injection strategy, which struggles to disentangle entangled features from multiple subjects; and (3) biases inherited from pre-trained base models and visual encoders, such as CLIP’s semantic oversimplification and VAE’s over-referenced details. 

\noindent \textbf{Future work.} Addressing these limitations will require multi-faceted innovations, and we propose the following directions: 

\begin{itemize}
    \item Enhanced Cross-Modal Alignment: Develop adaptive injection mechanisms that dynamically prioritize text or image conditions based on task requirements, reducing content leakage and improving text responsiveness. 
    
    \item Spatiotemporal Disentanglement: Integrate spatial-aware attention modules and physics-inspired motion priors to better model multi-subject interactions and enforce consistent relative scales. 
    
    \item Bias-Aware Training: Mitigate dataset and model biases through adversarial debiasing techniques and synthetic data augmentation for underrepresented subjects. 
    
    \item Granular Control: Explore auxiliary control signals (e.g., depth maps, segmentation masks) to complement text prompts, enabling precise alignment with complex instructions. 
    
    \item Foundation Model Adaptation: Fine-tune pre-trained encoders on domain-specific data (e.g., medical imaging, animation) to broaden \textit{Phantom}’s applicability while preserving generalization. 
\end{itemize}

By advancing these areas, \textit{Phantom} could evolve into a versatile tool for industrial applications such as virtual try-ons, interactive storytelling, and educational content creation, ultimately narrowing the gap between academic research and real-world demands.
\bibliographystyle{ACM-Reference-Format}
\bibliography{sample-base}




%%

\end{document}
\endinput
%%
%% End of file `sample-acmtog.tex'.
