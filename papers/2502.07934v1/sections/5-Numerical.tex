
\section{Numerical Results}\label{numerical}

In this section, we present numerical results to validate the theoretical analysis and optimization model developed in the previous sections. For clarity and better visualization, we consider a system with two sensors and two processes in our numerical experiments. However, it is important to note that all analyses can be generalized to systems with more sensors and processes. We vary different system parameters to verify our theoretical results in Section \ref{aoi-S}. After verifying the theoretical analysis, we investigate the optimization model described in Section \ref{aoi-opt} under different conditions and provide the results.

%\ali{I don't think we should talk about 2D grid search...people will say why did you do the analysis if u want to do grid search. Take this example back to after Theorem 2. And just say that we optimize the system and provide the results, and that's it. }, using a 2D grid search with a step size of 0.002 for both preemption probabilities, $\np_1$ and $\np_2$. While Theorem \ref{Theo2} provides an upper bound requiring at most 82 iterations for an example case with $M = 10$, $\mu = 5$, $\lambda_C = 15$, and $\epsilon_0 = 0.01$ that is not excessively high. However, the grid search is particularly suitable here due to the low system complexity.

Figure \ref{fig:exp_res} compares analytical results with experimental results for varying service rates, arrival rates, and preemption probabilities. Our simulations are unit-time-based and were run for 1 million units of time. The lowest arrival rate is 0.5 arrivals per unit time, ensuring at least $5 \times 10^5$ arrivals for each process to guarantee convergence. Simulation results are represented as circles, while the theoretical results derived from our analysis are depicted as solid lines in the figures. The strong alignment between the two verifies the validity of our analysis.


\begin{figure}[t!]
    \centering
    % Top-left subfigure
    \begin{subfigure}[t]{0.22\textwidth}
        \centering
        \includegraphics[width=1\textwidth]{figures/1.png}
        \caption{AoI versus $\lambda_1$ for different $\np_1=\np_2$ values with $\mu=2, \lambda_2=1,$ $\bfc= \begin{bmatrix}
1 & 0.5 \\
0.5 & 1
\end{bmatrix}$.}
        \label{fig:exp1}
    \end{subfigure}
    \hfill
    % Top-right subfigure
    \begin{subfigure}[t]{0.22\textwidth}
        \centering
        \includegraphics[width=1\textwidth]{figures/2.png}
        \caption{AoI versus $\mu$ for different $\np_1=\np_2$ values with $\lambda_1=1, \lambda_2=6,$ $\bfc= \begin{bmatrix}
1 & 0.5 \\
0.5 & 1
\end{bmatrix}.$}
        \label{fig:exp2}
    \end{subfigure}
    \caption{Simulation results vs theoretical findings.}
    \vspace{-14pt}
    \label{fig:exp_res}
\end{figure}


% \ali{Fix the figures after we decide on preemption probabilities; }

%\suresh{The captions for Fig. 3 (a) and (b) seem to be reversed.}

After that, we evaluate the optimal probabilistic preemption strategy for various correlations, arrival rates, and service rates. Figures \ref{l_res} and \ref{mu_res} illustrate the optimal probability values as well as their corresponding AoI values.  Additionally, the figures compare these optimal AoI results to the AoI values achieved under no preemption and full preemption scenarios. 

%\suresh{Clarify what you mean by aggregate updates.}


In Figure (\ref{fig:l1}), when the correlation matrix is an identity matrix, the preemption probability of Sensor 2 increases with $\lambda_1$ until reaching full preemption. On the other hand, Sensor 1’s probability starts high but decreases as $\lambda_1$ increases. This reflects a shift in importance toward Sensor 2 when Sensor 1’s arrival rate is higher. This behavior illustrates that packets with low arrival rates should preempt those with higher arrival rates when the correlation effect is eliminated. On the other hand, Figure (\ref{fig:l2}) shows the effect of the correlation matrix. Sensor 1 fully dominates due to its complete information about both processes and even at high \(\lambda_1\), preempting other packets with a packet from Sensor 2 is not preferable since it lacks any information about Process 1. In addition, Figure (\ref{fig:theta}) illustrates how optimal preemption evolves when each sensor fully tracks one process and partially observes the other with probability $\theta$. When $\theta$ is small, the sensor with the lower arrival rate is given higher preemption priority. However, as $\theta$ increases, the sensors become more similar in the information they provide. This reduces the distinction between them, leading to an optimal strategy of preempting every packet. These results underscore how the correlation matrix governs the preemption strategy and sensor roles. 






\begin{figure}[h!]
    \centering
    % Top-left subfigure
    \begin{subfigure}[b]{0.24\textwidth}
        \centering
        \includegraphics[width=1\textwidth]{figures/7_2.png}
        \caption{$\frac{}{}$ $\frac{}{}$$\frac{}{}$ $\frac{}{}$$\frac{}{}$$\frac{}{}$ $\frac{}{}$$\frac{}{}$ $\frac{}{}$}
        \label{fig:l1}
    \end{subfigure}
    \hfill
    % Top-right subfigure
    \begin{subfigure}[b]{0.24\textwidth}
        \centering
        \includegraphics[width=1\textwidth]{figures/8_2.png}
        \caption{$\frac{}{}$ $\frac{}{}$$\frac{}{}$ $\frac{}{}$$\frac{}{}$$\frac{}{}$ $\frac{}{}$$\frac{}{}$ $\frac{}{}$}
        \label{fig:l2}
    \end{subfigure}
% \vspace{-10pt}
%     \vskip\baselineskip
%     % Bottom-left subfigure
%     \begin{subfigure}[b]{0.22\textwidth}
%         \centering
%         \includegraphics[width=0.9\textwidth]{figures/9.png}
%         \caption{}
%         \label{fig:l3}
%     \end{subfigure}
%     \hfill
%     % Bottom-right subfigure
%     \begin{subfigure}[b]{0.22\textwidth}
%         \centering
%         \includegraphics[width=0.9\textwidth]{figures/10.png}
%         \caption{}
%         \label{fig:l4}
%     \end{subfigure}
    \caption{Optimal preemption probabilities under varying $\lambda_1$ to show the effect of correlation matrix.}
    \label{l_res}
    \vspace{-8pt}
\end{figure}

Lastly, in Figure (\ref{fig:div}), we see that Sensor 2 has more priority (higher preemption probability) than Sensor 1, even though Sensor 1 has more aggregate updates on average, i.e., 
the sum of the elements in the corresponding row of the correlation matrix is larger for Sensor 1. This stems from Sensor 2 having more diverse updates and playing a bigger role in minimizing the sum AoI. Note that in a single-source memoryless system, preempting packets is generally beneficial for reducing the AoI. Specifically, self-preempting, where a packet preempts another packet with information from the same processes, is profitable in our system. An increase in preemption probability results in a gain from self-preemption. However, there is a trade-off because it also increases the probability of preempting an informative packet for additional or different processes that can negatively impact the AoI. When $\lambda_1$ is low, the benefit of self-preemption for Sensor 1 is higher than the loss of preempting a packet from Sensor 2. However, as $\lambda_1$ increases, the risk of losing informative packets from other processes also rises when a packet from Sensor 1 preempts. Consequently, the optimal preemption probability decreases as $\lambda_1$ grows. Notably, this decrease begins even when $\lambda_1$ is smaller than $\lambda_2$, making packets from Sensor 1 relatively rare. Nevertheless, the diversity of information provided by Sensor 2 makes its packets more valuable, even when the aggregate correlation is low.
\vspace{-12pt}

%it is better to let only packets from Sensor 2 preempt because there might be a risk of losing a packet with more information when a packet from Sensor 1 preempts. With the increase in $\mu$, service time reduces, and packets from Sensor 1 start preempting with some probability to obtain the optimal AoI because it reduces to AoI by preempting their own packets. However, this probability decreases after some threshold and goes to zero again. The reason is that when $\mu$ is high, the gain of self-preemption is lower than the loss of preempting a packet with more information.


\begin{figure}[h!]

    \centering
    % Top-left subfigure
     \begin{subfigure}[b]{0.24\textwidth}
        \centering
        \includegraphics[width=1\textwidth]{figures/3_2.png}
        \caption{$\frac{}{}$$\frac{}{}$$\frac{}{}$$\frac{}{}$$\frac{}{}$$\frac{}{}$ $\frac{}{}$$\frac{}{}$ $\frac{}{}$}
        \label{fig:theta}
    \end{subfigure}
    \hfill
    % Top-right subfigure
    \begin{subfigure}[b]{0.24\textwidth}
        \centering
        \includegraphics[width=1\textwidth]{figures/4_2.png}
        \caption{$\frac{}{}$$\frac{}{}$ $\frac{}{}$$\frac{}{}$$\frac{}{}$$\frac{}{}$ $\frac{}{}$$\frac{}{}$ $\frac{}{}$}
        \label{fig:div}
    \end{subfigure}   
\vspace{-15pt}
    \caption{Optimal preemption probabilities under different conditions.}
\vspace{-10pt}
    \label{mu_res}
\end{figure}

%\suresh{Make the lines bolder.}

%\suresh{I'm not sure this figure is interesting anymore; the diff between full-pre and optimal-pre AoI is tiny.}



%\suresh{There is so much theory on managing the complexity of the optimization. Would be good to present some results of that here. It's not even clear what the $\epsilon_0$ you used to get the results and how many iterations were needed.} \egemen{You are right. However, we use grid search to find the optimum here. How should I mention it?}\suresh{I'm not sure. How many iterations does gridsearch take and how does it relate to the UB you derived. Seems that your optimization is disconnected from the result of Thm 2. Is that the case?} \egemen{Yes, the UB is for a branch and bound algorithm. However, for this experiment, we check every possible point with step size 0.002 to find the optimal probability.}







