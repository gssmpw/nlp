\section{System Model}\label{system-model}

We consider a sensor network in which \(N\) sensors monitor \(M\) information processes. 
Each process is dynamic, with its state evolving over time. To ensure the monitor remains updated, each sensor generates status updates and transmits them through a shared server, as illustrated in Figure \ref{fig:system_model}. 
\begin{figure}[!t]
  \centering
  \includegraphics[width=0.35\textwidth]{figures/system_model.png}
  \caption{Illustration of our system model.}
  \vspace{-14pt}\label{fig:system_model}
\end{figure}
We assume that the service time of each packet follows an exponential distribution with a service rate \(\mu\). Additionally, sensor \(i\) generates packets following a Poisson process with an arrival rate of \(\lambda_i\). We adopt a zero-buffer with probabilistic preemption assumption, motivated by its effectiveness in minimizing AoI in systems with preemption under specific conditions \cite{bedewy}. While this may not strictly apply to our scenario, our findings, \cite{erbayat2024}, show that buffers do not consistently improve performance when there is no preemption. Furthermore, the zero-buffer assumption is intuitive with preemption, as arriving packets either preempt the one in service or are dropped. Thus, we maintain this assumption throughout our analysis, implying that an arriving packet finding the server busy is either dropped or preempts the packet in the server \cite{dataNetworks:book}, depending on the adopted preemption policy.
%\ali{I feel like we talk a lot here, let us reduce this a bit and just reference our previous work}

With these assumptions, we define \(\boldsymbol{\lambda}\) as the vector of arrival rates from different sensors, where \(\lambda_i\) represents the arrival rate from sensor \(i\) for \(i=1,\ldots,N\). Specifically, we express \(\boldsymbol{\lambda}\) as:
\begin{equation}
\boldsymbol{\lambda}^T = \begin{bmatrix}
\lambda_{1} & \lambda_{2} & \dots & \lambda_{N}
\end{bmatrix}.
\end{equation}

To model the correlation between sensor observations, we assume that a packet generated by sensor \(i\) contains information about process \(j\) with a correlation probability \(\nc_{ij}\). This information reflects the state of the process at the time of packet generation. The correlation matrix \( \bfc \in [0,1]^{N \times M}\) is defined as
\begin{equation} 
\bfc = \begin{bmatrix}
\nc_{11} & \nc_{12} & \dots & \nc_{1M} \\
\nc_{21} & \nc_{22} & \dots & \nc_{2M} \\
\vdots & \vdots & \ddots & \vdots \\
\nc_{N1} & \nc_{N2} & \dots & \nc_{NM}
\end{bmatrix}.
\end{equation}

Each sensor is also associated with a preemption probability, which defines the likelihood that a packet arriving from sensor \(i\) preempts the packet currently being served when the server is busy.\footnote{To be more general, the preemption probabilities may depend on both the sensor that generated the arriving packet and the sensor that generated the packet in service. We defer the examination of this case for future work.} To represent the preemption probabilities of all sensors, we use the vector \(\bfp \in [0,1]^{N}\), where the \(i\)-th entry, \(\np_i\), denotes the preemption probability corresponding to sensor \(i\). The vector \(\bfp\) is expressed as
\begin{equation}
\bfp^T = \begin{bmatrix}
\np_{1} & \np_{2} & \dots & \np_{N} 
\end{bmatrix}.
\end{equation}
%\ali{Nx1 is a vector not a matrix. Let use a vector instead. Also, is correlation fixed by sensor, and not for each sensor i and process m there $r_{im}$?} \suresh{Ali, each packet contains correlated info from multiple processes, so the preemption prob is associated with the sensor. Egemen, could we improve it by making the pre-prob dependent on both the new packet sensor and the sensor of the packet-in-service? We should at least acknowledge that possibility even if we don't do it in this paper. Also, the system model says almost nothing about the process and its states. I think you should elaborate on that a bit.} \egemen{Correlation is not fixed by sensor that is $\nc_{ij}$. The preemption probability $\np_i$ is fixed by the sensor. We can make it pre-prob dependent and define a $NxN$ matrix. What should be the best way to include this possibility because I do not have an analysis for it?. I have just added a sentence indicating that the process' state evolves over time because we do not consider error. Should I elaborate more? }\suresh{This is fine. I added a footnote for that case.}
In the next section, we present the closed-form derivation of the AoI under the described system model.


