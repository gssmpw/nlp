%% LaTeX Template for ISIT 2025
%%
%% by Stefan M. Moser, October 2017
%% (with minor modifications by Tobias Koch, November 2023 and Michèle Wigger, November 2024)
%% 
%% derived from bare_conf.tex, V1.4a, 2014/09/17, by Michael Shell
%% for use with IEEEtran.cls version 1.8b or later
%%
%% Support sites for IEEEtran.cls:
%%
%% http://www.michaelshell.org/tex/ieeetran/
%% http://moser-isi.ethz.ch/manuals.html#eqlatex
%% http://www.ctan.org/tex-archive/macros/latex/contrib/IEEEtran/
%%

\documentclass[conference,letterpaper]{IEEEtran}
%\renewcommand{\baselinestretch}{0.97}

%% depending on your installation, you may wish to adjust the top margin:
\addtolength{\topmargin}{9mm}

%%%%%%
%% Packages:
%% Some useful packages (and compatibility issues with the IEEE format)
%% are pointed out at the very end of this template source file (they are 
%% taken verbatim out of bare_conf.tex by Michael Shell).
%
% *** Do not adjust lengths that control margins, column widths, etc. ***
% *** Do not use packages that alter fonts (such as pslatex).         ***
%

\usepackage[utf8]{inputenc} 
\usepackage[T1]{fontenc}
\usepackage{url}
\usepackage{color}
\usepackage{amsthm}
\usepackage{ifthen}
\usepackage{cite}
\usepackage{multirow}
\usepackage{array}
\usepackage{booktabs}
\usepackage{graphicx}
\usepackage{subcaption}
\usepackage{ifthen}

\usepackage[cmex10]{amsmath} 
\DeclareMathAlphabet\mathbfcal{OMS}{cmsy}{b}{n}
\DeclareFontFamily{OT1}{pzc}{}
\DeclareFontShape{OT1}{pzc}{m}{it}{<-> s * [1.10] pzcmi7t}{}
\DeclareMathAlphabet{\mathpzc}{OT1}{pzc}{m}{it}
\DeclareFontFamily{OT1}{bfpzc}{}
\DeclareFontShape{OT1}{bfpzc}{m}{it}{<-> s * [1.10] pzcmi7t}{}
\DeclareMathAlphabet{\mathbfpzc}{OT1}{bfpzc}{b}{it}

\newcommand{\bfc}
{\mathbfcal{C}}
\newcommand{\nc}
{\mathpzc{c}}
\newcommand{\bfp}
{\mathbf{r}}
\newcommand{\np}
{\mathpzc{r}}
\newtheorem{remark}{Remark} \newtheorem{Theorem}{Theorem}
\newtheorem{Corollary}{Corollary}
\newtheorem{Lemma}{Lemma}
\newtheorem{Proposition}{Proposition}
\newtheorem{Definition}{Definition}
\newtheorem{assumption}{Assumption}% Use the [cmex10] option to ensure compliance
                             % with IEEE Xplore (see bare_conf.tex)

%% Please note that the amsthm package must not be loaded with
%% IEEEtran.cls because IEEEtran provides its own versions of
%% theorems. Also note that IEEEXplore does not accepts submissions
%% with hyperlinks, i.e., hyperref cannot be used.
\newcommand{\ali}[1]{\textcolor{magenta}{Ali: #1}}
\newcommand{\suresh}[1]{\textcolor{red}{Suresh: #1}}
\newcommand{\egemen}[1]{\textcolor{blue}{Egemen: #1}}
\newboolean{withappendix}
\setboolean{withappendix}{true}


\interdisplaylinepenalty=2500 % As explained in bare_conf.tex


%%%%%%
% correct bad hyphenation here
\hyphenation{}

% ------------------------------------------------------------
\begin{document}
\title{Age of Information Optimization with Preemption Strategies for Correlated Systems} 

% %%% Single author, or several authors with same affiliation:
% \author{%
%  \IEEEauthorblockN{Author 1 and Author 2}
% \IEEEauthorblockA{Department of Statistics and Data Science\\
%                    University 1\\
 %                   City 1\\
  %                  Email: author1@university1.edu}% }


%%% Several authors with up to three affiliations:

\author{\IEEEauthorblockN{Egemen Erbayat$^1$, Ali Maatouk$^2$, Peng Zou$^3$, Suresh Subramaniam$^1$}
\IEEEauthorblockA{
\textit{$^{1}$The George Washington University}, \textit{$^{2}$Yale University}, \textit{$^{3}$Nanjing University of Information Science and Technology}\\
$^1$\textit{\{erbayat, suresh\}@gwu.edu}, $^2$\textit{ali.maatouk@yale.edu}, $^3$\textit{003967@nuist.edu.cn}} 
}





\maketitle



\begin{abstract}
In this paper, we examine a multi-sensor system where each sensor monitors multiple dynamic information processes and transmits updates over a shared communication channel. These updates may include correlated information across the various processes. In this type of system, we analyze the impact of preemption, where ongoing transmissions are replaced by newer updates, on minimizing the Age of Information (AoI). While preemption is optimal in some scenarios, its effectiveness in multi-sensor correlated systems remains an open question. To address this, we introduce a probabilistic preemption policy, where the source sensor preemption decision is stochastic. We derive closed-form expressions for the AoI and frame its optimization as a sum of linear ratios problem, a well-known NP-hard problem. To navigate this complexity, we establish an upper bound on the iterations using a branch-and-bound algorithm by leveraging a reformulation of the problem. This analysis reveals linear scalability with the number of processes and a logarithmic dependency on the reciprocal of the error that shows the optimal solution can be efficiently found. Building on these findings, we show how different correlation matrices can lead to distinct optimal preemption strategies. Interestingly, we demonstrate that the diversity of processes within the sensors' packets, as captured by the correlation matrix, plays a more significant role in preemption priority than the number of updates.


\end{abstract}








\section{Introduction}

\subsection{Background and Motivation}
Integrating Deep Reinforcement Learning (DRL) in financial market analysis significantly evolved investment analysis with Deep Learning. DRL combines deep learning and reinforcement learning to offer a sophisticated framework for adapting strategies in the dynamic financial domain. It allows a deep learning model to effectively decipher complex patterns in historical market data often overlooked by traditional quantitative models.
It is no secret that financial markets are inherently complex and influenced by economic trends and geopolitical events. Therefore, traditional financial modeling often struggles to adapt to these ever-changing conditions. However, with its direct learning from data and adaptive strategies, DRL presents a promising solution to these challenges. With its autonomous learning ability and continual adaptation to the financial environment, it leverages historical market data to identify complex relationships and patterns.


\subsection{Overview of Our Previous Work}
In recent years, significant progress has been made in applying deep reinforcement learning (DRL) to stock trading strategies. For instance, Wang et al. proposed a parallel multi-module DRL algorithm that effectively captures both current market conditions and long-term dependencies using fully connected and LSTM layers \cite{parallel_drl_stock_trading}. Zhang et al. introduced an automated stock trading system based on the Proximal Policy Optimization algorithm, modeling trading as a Markov decision process \cite{novel_drl_stock_trading}. Additionally, Huang et al. demonstrated the importance of integrating market sentiment data to enhance the performance of DRL models in trading \cite{market_sentiment_drl_stock_trading}. Liu et al. developed an end-to-end trading strategy using a multi-view environment representation neural network, incorporating a Long Memory mechanism to improve decision-making \cite{drl_end_to_end_stock_trading}. Lastly, Li et al. focused on adaptive trading strategies using Gated Recurrent Units to capture time-series data effectively \cite{adaptive_drl_stock_trading}. These studies collectively highlight the potential of DRL in creating robust and adaptive trading strategies.

Liu et al. significantly advanced Deep Reinforcement Learning in Finance by developing platforms such as FinRL-Meta \cite{Liu2022FinRL}. This platform is a comprehensive tool for training and evaluating data-driven reinforcement learning agents within several simulated financial markets, offering a robust benchmarking system for algorithm comparison and facilitating the simulation of complex market conditions. The FinRL platform enables researchers to refine and test the efficacy of various DRL strategies, and it has been pivotal in democratizing access to sophisticated financial simulation tools and propelling research in financial analysis.

FinRL uses environments that offer broad simulation capabilities. These specialized environments, such as ABIDES-Gym \cite{Vyawahare2020}, provide the necessary infrastructure that allows FinRL to create discrete event simulations explicitly tailored for financial markets. ABIDES-Gym extends the OpenAI Gym interface to accommodate the complex dynamics of financial trading, allowing for a nuanced replication of market mechanisms and agent interactions. This level of detail will enable researchers and practitioners to explore the impact of individual agent behaviors and market responses, thus enhancing the understanding of market microstructure and agent-based modeling. The framework also streamlines the model training process on financial datasets, epitomizing the intersection of DRL and high-performance computing. It Leverages distributed computing resources to reduce training times significantly and optimizes computational workflows to enable the application of complex DRL models to extensive financial tasks. Their efforts have led to the creation of scalable and efficient financial models.

Our previous work \cite{Montazeri2023} demonstrated the efficiency and capability of CNNs when used as policies for deep reinforcement learning. We utilized the FinRL platform to conduct experiments on CNNs as a significantly improved policy to FinRL's original proposition. We also showed \cite{Montazeri2024, Montazeri2024GradientRC} that rearranging the stock market features used in the FinRL platform to group them per company could benefit the mode's performance. This study also utilizes the FinRL platform with its original dataset, containing features generated through traditional Technical Analysis used in Finance. It also uses the new dataset introduced in FinRL Meta, which contains statistically engineered features such as Simple Moving Average (SMA), momentum, and rate of change.

Building upon these foundational studies, our research aims to bridge the gap between CNN architecture optimization and financial market analysis. By introducing a systematic approach to temporal window selection, we seek to enhance the adaptability and performance of DRL models in capturing complex market dynamics.
    
\section{Objectives of the Current Study}
So far, we have presented the literature and the setting in which our study operates. The primary objective of our research is to explore the effects of changing the temporal window of a Convolutional Neural Network (CNN) used as a policy in a FinRL. By progressively expanding the observation period, beginning with a concise two-week window and incrementally enlarging it by two weeks in each iteration and culminating in twelve weeks, we aim to observe and analyze the performance of our model as its temporal window changes in the FinRL platform. This iterative window expansion is designed to examine the impact of different temporal scales on the model's performance. This process enables a comprehensive analysis of how varying lengths of financial data affect the model's predictive capabilities, offering insights and an opportunity to optimize the temporal granularity for financial market analysis. Our study also examines the arrangement of feature vectors within these expanding windows to better understand the model-market dynamics.

Furthermore, we contrast the model's performance across these different temporal windows to discern patterns in market behavior and model performance. In our study, short-term windows, particularly the initial two-week period, are hypothesized to be critical for understanding the model's ability to capture immediate market changes and short-term trends, which are essential for timely and accurate trading predictions. As the window expands, the model is expected to integrate a broader spectrum of market conditions, capturing longer-term trends and patterns. This bi-weekly expansion strategy is designed to balance the benefits of short-term immediacy and long-term historical perspective, ensuring the model remains adaptable and responsive to transient market fluctuations and enduring trends. We hope to contribute to financial analytics by demonstrating the efficacy of CNNs in a DRL setting and by providing new insights into the role of temporal dynamics in financial modeling.
\section{System Model}\label{system-model}

We consider a sensor network in which \(N\) sensors monitor \(M\) information processes. 
Each process is dynamic, with its state evolving over time. To ensure the monitor remains updated, each sensor generates status updates and transmits them through a shared server, as illustrated in Figure \ref{fig:system_model}. 
\begin{figure}[!t]
  \centering
  \includegraphics[width=0.35\textwidth]{figures/system_model.png}
  \caption{Illustration of our system model.}
  \vspace{-14pt}\label{fig:system_model}
\end{figure}
We assume that the service time of each packet follows an exponential distribution with a service rate \(\mu\). Additionally, sensor \(i\) generates packets following a Poisson process with an arrival rate of \(\lambda_i\). We adopt a zero-buffer with probabilistic preemption assumption, motivated by its effectiveness in minimizing AoI in systems with preemption under specific conditions \cite{bedewy}. While this may not strictly apply to our scenario, our findings, \cite{erbayat2024}, show that buffers do not consistently improve performance when there is no preemption. Furthermore, the zero-buffer assumption is intuitive with preemption, as arriving packets either preempt the one in service or are dropped. Thus, we maintain this assumption throughout our analysis, implying that an arriving packet finding the server busy is either dropped or preempts the packet in the server \cite{dataNetworks:book}, depending on the adopted preemption policy.
%\ali{I feel like we talk a lot here, let us reduce this a bit and just reference our previous work}

With these assumptions, we define \(\boldsymbol{\lambda}\) as the vector of arrival rates from different sensors, where \(\lambda_i\) represents the arrival rate from sensor \(i\) for \(i=1,\ldots,N\). Specifically, we express \(\boldsymbol{\lambda}\) as:
\begin{equation}
\boldsymbol{\lambda}^T = \begin{bmatrix}
\lambda_{1} & \lambda_{2} & \dots & \lambda_{N}
\end{bmatrix}.
\end{equation}

To model the correlation between sensor observations, we assume that a packet generated by sensor \(i\) contains information about process \(j\) with a correlation probability \(\nc_{ij}\). This information reflects the state of the process at the time of packet generation. The correlation matrix \( \bfc \in [0,1]^{N \times M}\) is defined as
\begin{equation} 
\bfc = \begin{bmatrix}
\nc_{11} & \nc_{12} & \dots & \nc_{1M} \\
\nc_{21} & \nc_{22} & \dots & \nc_{2M} \\
\vdots & \vdots & \ddots & \vdots \\
\nc_{N1} & \nc_{N2} & \dots & \nc_{NM}
\end{bmatrix}.
\end{equation}

Each sensor is also associated with a preemption probability, which defines the likelihood that a packet arriving from sensor \(i\) preempts the packet currently being served when the server is busy.\footnote{To be more general, the preemption probabilities may depend on both the sensor that generated the arriving packet and the sensor that generated the packet in service. We defer the examination of this case for future work.} To represent the preemption probabilities of all sensors, we use the vector \(\bfp \in [0,1]^{N}\), where the \(i\)-th entry, \(\np_i\), denotes the preemption probability corresponding to sensor \(i\). The vector \(\bfp\) is expressed as
\begin{equation}
\bfp^T = \begin{bmatrix}
\np_{1} & \np_{2} & \dots & \np_{N} 
\end{bmatrix}.
\end{equation}
%\ali{Nx1 is a vector not a matrix. Let use a vector instead. Also, is correlation fixed by sensor, and not for each sensor i and process m there $r_{im}$?} \suresh{Ali, each packet contains correlated info from multiple processes, so the preemption prob is associated with the sensor. Egemen, could we improve it by making the pre-prob dependent on both the new packet sensor and the sensor of the packet-in-service? We should at least acknowledge that possibility even if we don't do it in this paper. Also, the system model says almost nothing about the process and its states. I think you should elaborate on that a bit.} \egemen{Correlation is not fixed by sensor that is $\nc_{ij}$. The preemption probability $\np_i$ is fixed by the sensor. We can make it pre-prob dependent and define a $NxN$ matrix. What should be the best way to include this possibility because I do not have an analysis for it?. I have just added a sentence indicating that the process' state evolves over time because we do not consider error. Should I elaborate more? }\suresh{This is fine. I added a footnote for that case.}
In the next section, we present the closed-form derivation of the AoI under the described system model.



\section{Age of Information Analysis} \label{aoi-S}



%In this section, we consider the AoI measure introduced in \cite{yates2012} as the performance metric to evaluate the timeliness of updates at the monitor. 


For analytical convenience, we first simplify the system by focusing solely on the relevance of the updates rather than their origin. In the considered system, the source sensor of the packet containing information about any arbitrary process $j$ is irrelevant from the monitor's perspective. Instead, what matters is whether the served packet contains information about process $j$, regardless of which sensor provided the update to track AoI. To formalize this, we label a status update as informative for process $j$ if it contains information on process $j$. Otherwise, it is labeled as uninformative. We build on this concept by defining two types of informative arrivals:
\begin{itemize}
    \item Informative arrivals that can preempt ongoing service.
    \item Informative arrivals that cannot preempt ongoing service.
\end{itemize}
Using this distinction, we define the informative arrival rate vectors as follows
\begin{equation}
\boldsymbol{\Tilde{\lambda}}^T = \begin{bmatrix}
\Tilde{\lambda}_{1} & \Tilde{\lambda}_{2} & \dots & \Tilde{\lambda}_{M}
\end{bmatrix} = (\boldsymbol{\lambda}^T \odot \bfp^T) \bfc,
\end{equation}
\begin{equation}
\boldsymbol{\dot{\lambda}}^T = \begin{bmatrix}
\dot{\lambda}_{1} & \dot{\lambda}_{2} & \dots & \dot{\lambda}_{M}
\end{bmatrix} = (\boldsymbol{\lambda}^T \odot (1-\bfp^T)) \bfc,
\end{equation}
%
where \(\boldsymbol{\Tilde{\lambda}}\) represents the informative arrival rate vector for packets that can preempt ongoing service, and \(\boldsymbol{\dot{\lambda}}\) represents the informative arrival rate vector for packets that cannot preempt ongoing service. The total channel arrival rate is given as:
\begin{equation}
\lambda_C = \Tilde{\lambda}_C + \dot{\lambda}_C,
\end{equation}
where
\begin{equation}
\Tilde{\lambda}_C = \sum_{i=1}^{N} \lambda_i \np_{i} \text{ and }
\dot{\lambda}_C = \sum_{i=1}^{N} \lambda_i (1 - \np_{i}).
\end{equation}
represent the channel arrival rates for packets that can and cannot preempt, respectively.

With the above quantities in mind, we analyze the system by reducing it to $M$ independent systems, each consisting of two sources as depicted in Figure \ref{fig:equiv_model}. The independence of these $M$ systems arises from the Poisson nature of packet arrivals. For any single process $j$, the arrivals of both informative and uninformative packets from all other processes collectively form Poisson streams, as shown in \ifthenelse{\boolean{withappendix}}
{Appendix~\ref{reduction-P}}
{Appendix A in \cite{technicalNote}}.%\ali{Can we put this in appendix instead of citing our work?}



\begin{figure}[!t]
  \centering
  \includegraphics[width=0.35\textwidth]{figures/simplification.png}
  \caption{Equivalent system model from process $j$'s perspective.}
  \vspace{-12pt}\label{fig:equiv_model}
\end{figure}

With this setup established, we now proceed to evaluate the evolution of AoI for a single process. The AoI of process \(j\) at time \(t\), denoted by \(\Delta_j(t)\), is defined as
\begin{equation}
\Delta_j(t) = t - T_j, \quad j=1,\ldots,M,
\end{equation}
where \(T_j\) represents the time at which the most recent \textit{informative} packet for process \(j\) was generated. The AoI for each process \(j\) evolves as follows: it increases linearly over time until an informative status update is successfully received, at which point a drop in the AoI occurs. However, whether an incoming packet contributes to reducing the AoI of a specific process depends on two key factors: (1) whether the packet contains information about process \(j\), and (2) the server's preemption dynamics.


To analyze this further, we model the server's operation through three states: \(0\) (idle), \(1\) (busy processing an informative packet), and \(2\) (busy processing an uninformative packet). In state \(0\), the server is not processing any packets, and the AoI for process \(j\) increases linearly due to the absence of updates. Upon transitioning to state \(1\), the server processes a packet containing relevant information for process \(j\), resulting in a decrease in the AoI after the packet's service time. Conversely, in state \(2\), the server is busy processing a packet that lacks relevant information, so the AoI for process \(j\) continues to increase linearly. Furthermore, when a new packet arrives at the busy server, incoming packets can interrupt ongoing service with a probability determined by the preemption matrix \(\bfp\). A transition from state \(2\) to state \(1\) via preemption leads to a linear increase in AoI during service time, followed by a decrease in AoI if the informative packet is successfully served, while a transition from state \(1\) to state \(2\) leads to a linear increase in AoI. The interaction between service states and preemption dynamics thus determines the AoI behavior over time. Lemma \ref{Lem1} provides the stationary distribution of these states and forms the foundation for deriving the closed-form expression of the AoI. %\ali{it feels like so much words, can we make this shorter? Also, why the results are given as a remark instead of a Lemma?}




%As per our system model, a served packet may or may not have information about process \(j\). If the served packet contains information on process \(j\), the AoI for process \(j\) decreases just after the service time of that packet. Conversely, if the served packet lacks information about process \(j\), the AoI for process \(j\) continues to increase linearly.

%Furthermore, the preemption mechanism impacts the AoI dynamics. Wrhen a new packet arrives at the server, it can preempt the currently served packet with a probability determined by the preemption matrix \(\bfp\). If preemption occurs, the ongoing service is interrupted, and the new packet is served instead. This can potentially reduce the AoI more effectively if the preempting packet contains relevant information for process \(j\). However, if the preempting packet lacks such information, the AoI will remain unaffected and continue its linear growth.



\begin{Lemma}\label{Lem1}
The stationary distribution of the states (\(0\), \(1\), \(2\)) can be derived as follows:
\begin{align}
\pi_0 = \frac{\mu}{(\lambda_C + \mu)}, \quad
\pi_1 = \frac{\lambda_C\Tilde{\lambda}_{1} + \dot{\lambda}_{1}\mu + \Tilde{\lambda}_{1}\mu}{(\lambda_C + \mu)(\Tilde{\lambda}_{C} + \mu)}, \\
\pi_2 = \frac{\Tilde{\lambda}_{C}\lambda_C + \lambda_C\mu -\lambda_C\Tilde{\lambda}_{1}  - \dot{\lambda}_{1}\mu - \Tilde{\lambda}_{1}\mu}{(\lambda_C + \mu)(\Tilde{\lambda}_{C} + \mu)}. 
\label{pi-distributions}
\end{align}
\begin{proof}
The illustration of the Markov chain states and the details of the proof can be found in \ifthenelse{\boolean{withappendix}}
{Appendix~\ref{spv-appendix}}
{Appendix B in \cite{technicalNote}}. %\ali{Perhaps you can mention here that an illustration of the Markov chain can also be found in the appendix}
\end{proof}
\end{Lemma}
With this model in place, we derive a closed-form expression for the AoI of each process, accounting for both informative and uninformative status updates with probabilistic preemption to comprehensively evaluate their impact on the system's dynamics. %as explained in Theorem \ref{The1}.

\begin{Theorem}\label{The1}
In the considered M/M/1/1 system, the average AoI for process $j$, denoted as $\Delta_j$, is

\footnotesize
\begin{align}
\Delta_j= \frac{\mu(\mu+\lambda_C)^2 + \sum_{i=1}^{N} \left(\mu\lambda_C \nc_{ij}(1-\np_{i}) + (\mu + \lambda_C)^2\np_{i}\right)\lambda_{i}}{
        \mu \sum_{i=1}^{N} \left(\mu + \lambda_C)(\lambda_C \np_{i} + \mu \right) \nc_{ij}\lambda_{i}}.
    \end{align}
\normalsize
    \end{Theorem}
\begin{proof} The proof leverages stochastic hybrid system modeling, focusing on state transitions (idle, busy with informative, or uninformative packets). Full details are provided in \ifthenelse{\boolean{withappendix}}
{Appendix~\ref{aoi-appendix}}
{Appendix C in \cite{technicalNote}}. 
\end{proof}
%\ali{both remark1 and theorem 1 have the same appendix?} \egemen{yes.}
Leveraging the above results, we examine two specific scenarios of interest:
\begin{itemize}

    \item \textbf{Preempt every packet} (\( \bfp = 1 \)):
    \begin{align}
    \Delta_j = \frac{{\lambda}_C+\mu}{\mu \tilde{\lambda}_j} .
    \end{align}
    
    \item \textbf{Preempt nothing} (\( \bfp = 0 \)):
    \begin{align}
    \Delta_j = \frac{{\lambda}_C}{\mu ({\lambda}_C + \mu)} + \frac{{\lambda}_C + \mu}{\mu \dot{\lambda}_j}.
    \end{align}


\end{itemize}


For these specific cases, both $\dot{\lambda}_j$ and $\tilde{\lambda}_j$ are equal to $\sum_{i=1}^{N} \nc_{ij}\lambda_i$. Therefore, the AoI in the no-preemption scenario is equal to the sum of a positive constant and the AoI in the full-preemption scenario. Thus, regardless of the correlation, preempting every packet guarantees a lower AoI than the no-preemption strategy. As the service rate approaches infinity, the average AoI difference between the two scenarios decreases because the system can accommodate updates almost instantaneously, which minimizes the necessity for preemption. Beyond these two special cases, we investigate the AoI-optimal preemption policy for our system in the following section.


\section{Average Age Optimization}\label{aoi-opt}


Preemption is a possible strategy for minimizing AoI in networked systems, but its effectiveness depends on the specific context, such as sensor arrival rates and the correlation between updates. In multi-sensor systems with correlated processes, the goal is to identify when and how preemption can provide benefits. Thus, our objective is to find optimal preemption probabilities that minimize AoI for a given setup.


%\ali{Let us reduce this; we speak too much. Just say, the goal is to find out in what scenario is preemption going to help in this multi-sensors correlated systems, etc.}
%\suresh{I actually like this rationale but would suggest making it more concise. And it sounds like this is motivation and belongs in the Intro.}

To that end, the objective is to minimize the sum of AoI, $\Delta_{\text{sum}}$, as follows:
\begin{align}
\min_{\bfp \in [0,1]^{N}} \sum_{j=1}^{M}\Delta_j = \Delta_{\text{sum}}.
\end{align}

%\suresh{Pay attention to notation, you have an $NxM$ matrix for $R$ here. Also, did you define $\Delta_{\text{sum}}$}

%\ali{align it better}
The objective function can be reformulated as:
\begin{align}\label{objective_func}
    \min_{\bfp \in [0,1]^{N}}\Delta_{\text{sum}} =& 
    \sum_{j=1}^{M} \frac{\mathbf{g_j}^T \bfp + g_{j0}}{\mathbf{f_j}^T \bfp + f_{j0}} = \sum_{j=1}^{M} \frac{G_j(\bfp)}{F_j(\bfp)}
    \\ \nonumber
    &\text{subject to } 0 < \bfp < 1,
\end{align}
where
\vspace{-15pt}
\begin{align}
    g_{j0} &= \mu(\mu + \lambda_C)^2 + \sum_{i=1}^{N} \lambda_{i}\mu\lambda_C \nc_{ij}, \\
    g_{ji} &= ((\mu + \lambda_C)^2 - \mu\lambda_C \nc_{ij})\lambda_{i}, \\
    f_{j0} &= (\mu + \lambda_C) \mu^2 \sum_{i=1}^{N} \nc_{ij} \lambda_{i}, \\
    f_{ji} &= \lambda_C \mu (\mu + \lambda_C) \nc_{ij} \lambda_{i}.
\end{align} 

This is a classical sum of linear ratios problem studied in the literature. The problem is well known for its computational challenges \cite{schaible2003fractional}, and it is NP-hard, in general \cite{freund2001solving}. However, given the special case of our problem, as we will show afterward, we can achieve a global optimum efficiently using a branch-and-bound algorithm with a finite number of iterations \cite{JIAO2022112701}.

%\ali{say that it is NP-hard in general. also, don't say progress has been made, etc. Just say, however, given the special case of our problem, as we will show afterwards, we can achieve a global optimum in an efficient way.}

To address the global minimization of the objective function in (\ref{objective_func}), an essential step is the reformulation of the original problem into an equivalent problem (EP). This reformulation allows for more tractable computation while preserving the global properties of the original problem.

For clarity, consider the following notations: for each \( i = 1, 2, \dots, M \), define
\[
\begin{aligned}
    \bar{l}_i &= \min_{\bfp \in [0,1]^{N}} F_i(\bfp), \quad 
    \bar{u}_i = \max_{\bfp \in [0,1]^{N}} F_i(\bfp), \\
    \bar{L}_i &= \min_{\bfp \in [0,1]^{N}} G_i(\bfp), \quad 
    \bar{U}_i = \max_{\bfp \in [0,1]^{N}} G_i(\bfp).
\end{aligned}
\]

% \ali{can you explain what linear problems are there?} 
These bounds can be determined by solving \( 4 \times M \) linear programs, where each program involves optimizing \( F_i(\bfp) \) and \( G_i(\bfp) \) over the feasible region \( \bfp \in [0,1]^{N} \), separately minimizing and maximizing each function to yield the values of \( \bar{l}_i \), \( \bar{u}_i \), \( \bar{L}_i \), and \( \bar{U}_i \). Using these results, we define the feasible region $
\Omega = \{ (\beta, \alpha) \in \mathbf{R}^{2M} \mid \bar{l}_i \leq \beta_i \leq \bar{u}_i, \; \bar{L}_i \leq \alpha_i \leq \bar{U}_i, \; i = 1, 2, \dots, M \}.
$
%\suresh{Wouldn't it be more natural to call it $(\alpha, \beta)$?} \egemen{I followed the notation in the cited paper. Do you want me to change?}\suresh{No; didn't know that was how it was done in that paper.}
The problem can then be reformulated as the following EP:

\[
\text{EP}:
\begin{cases}
    \min \; h(\beta, \alpha) = \sum_{i=1}^M \frac{\beta_i}{\alpha_i}, \\
    \text{s.t. } \quad F_i(\bfp) - \beta_i = 0, \; i = 1, 2, \dots, M, \\
    \quad G_i(\bfp) - \alpha_i = 0, \; i = 1, 2, \dots, M, \\
    \quad \bfp \in [0,1]^N, \; (\beta, \alpha) \in \Omega.
\end{cases}
\]

The feasible region of the EP, denoted as $
Z = \{ F_i(\bfp) - \beta_i = 0, \; G_i(\bfp) - \alpha_i = 0, \; i = 1, 2, \dots, M, \; \bfp \in [0,1]^N, \; (\beta, \alpha) \in \Omega \},
$
is a bounded compact set. Importantly, \( Z \neq \emptyset \) holds if and only if \( [0,1]^N \neq \emptyset \).
%
If a solution is globally optimal for the EP, it can be converted into a globally optimal solution for the original problem, and vice versa, making the EP sufficient for addressing the original optimization problem \cite{JIAO2022112701}.

Moreover, reformulating the problem as the EP allows us to analyze its computational complexity. Specifically, leveraging the EP's structure, we derive an upper bound on the iterations required for a branch-and-bound algorithm to find a global solution. The upper bound on the number of iterations required for a branch-and-bound algorithm to solve the EP is established in Theorem~\ref{Theo2}, as follows.

%\ali{This paragraph and the one afterward, it feels we are speaking too much; let us make things more compact with the major stuff that makes us appear that we have done quite some stuff here.} 

\begin{Theorem}\label{Theo2}
For any given positive error \( \epsilon_0 \in (0, 1) \), the outer space accelerating branch-and-bound algorithm can find a global \( \epsilon_0 \)-optimal solution in at most 
\begin{equation}
M \cdot \left\lceil \log_2 \frac{4M (\mu+\lambda_C)^2\lambda_C^2}{\epsilon_0\mu^3\hat{\lambda}_{\min}^{2}} \right\rceil
\end{equation}
iterations, where \( \hat{\lambda}_{\min} = \min(\boldsymbol{\lambda}^T \bfc) \) denotes the minimum element of \( \boldsymbol{\lambda}^T \bfc \) over the feasible set.
\end{Theorem}

\begin{proof} 
The proof uses the compact and bounded feasible region properties of the reformulated optimization problem to derive the computational complexity. Full details are provided in \ifthenelse{\boolean{withappendix}}
{Appendix~\ref{iteration-appendix}}
{Appendix D in \cite{technicalNote}}.
\end{proof}

This upper bound highlights the effect of parameters on the iteration count. Increasing $\mu$ reduces the number of required iterations because it also decreases the AoI value. Therefore, there is a need for a reduction in $\epsilon_0$ to achieve a similar level of precision. Likewise, smaller $\hat{\lambda}_{\min}$ and larger $\lambda_C$ lead to higher AoI values, necessitating more iterations to maintain the same error tolerance $\epsilon_0$ as expected. For an example case with $M = 10$, $\mu = 5$, $\lambda_C = 15$, and $\epsilon_0 = 0.01$, the upper bound requires at most 82 iterations, which is not excessively high. %\ali{You did not give a small example to showcase how fast the algorithm converge for typical values. Let us make our case stronger here}


% This problem, known as the sum of linear ratios problem, is generally challenging to solve \cite{schaible2003fractional}. However, it has been demonstrated that the globally optimal solution can be achieved in a finite number of steps, and various algorithms have been proposed, some of them providing upper bounds on the required number of iterations \cite{falk1992optimizing,shen2006global,gao2013global,JIAO2022112701}. 


% To globally minimize the problem in (\ref{objective_func}), the major work is to globally minimize an equivalent problem EP.

% Without losing generality, for each $i = 1, 2, \dots, M$, we let
% \[
% \begin{aligned}
%     \bar{l}_i &= \min_{\bfp \in [0,1]^{N}} F_i(\bfp), \quad 
%     \bar{u}_i = \max_{\bfp \in [0,1]^{N}} F_i(\bfp), \\
%     \bar{L}_i &= \min_{\bfp \in [0,1]^{N}} G_i(\bfp), \quad 
%     \bar{U}_i = \max_{\bfp \in [0,1]^{N}} G_i(\bfp).
% \end{aligned}
% \]
% Obviously, by solving $4 \times M$ linear programs, we can compute the values of $\bar{l}_i$, $\bar{u}_i$, $\bar{L}_i$, and $\bar{U}_i$. Let $
% \Omega = \{ (\beta, \alpha) \in \mathbf{R}^{2M} \mid \bar{l}_i \leq \beta_i \leq \bar{u}_i, \; \bar{L}_i \leq \alpha_i \leq \bar{U}_i, \; i = 1, 2, \dots, M \}.$
% Then, we convert our problem into the following equivalence problem:
% \[
% \text{EP}:
% \begin{cases}
%     \min \; h(\beta, \alpha) = \sum_{i=1}^p \frac{\beta_i}{\alpha_i}, \\
%     \text{s.t. } \quad F_i(\bfp) - \beta_i = 0, \; i = 1, 2, \dots, M, \\
%     \quad G_i(\bfp) - \alpha_i = 0, \; i = 1, 2, \dots, M, \\
%     \quad \bfp \in [0,1]^N, \; (\beta, \alpha) \in \Omega.
% \end{cases}
% \]

% Obviously, the feasible region $Z = \{ F_i(\bfp) - \beta_i = 0, \; G_i(\bfp) - \alpha_i = 0, \; i = 1, 2, \dots, M, \; \bfp \in [0,1]^N, \; (\beta, \alpha) \in \Omega \}$ of the EP is a bounded compact set, and $Z \neq \emptyset$ holds if and only if $[0,1]^N \neq \emptyset$ holds.



% Theorem \ref{Theo2} below provides an upper bound on the number of iterations required for a branch-and-bound algorithm from the literature \cite{JIAO2022112701}.

% \begin{Theorem}\label{Theo2}
% For any given positive error $\epsilon_0 \in (0, 1)$, the outer space accelerating branch-and-bound algorithm can seek out a global $\epsilon_0$-optimum solution in at most 
% \begin{equation}
% M \cdot \left\lceil \log_2 \frac{4M (\mu+\lambda_C)^2\lambda_C^2}{\epsilon_0\mu^3\hat{\lambda}_{\min}^{2}} \right\rceil
% \end{equation}

% iterations, where $\hat{\lambda}_{\min} = \min(\boldsymbol{\lambda}^T \bfc)$ denotes the minimum element of $\boldsymbol{\lambda}^T \bfc$ over the feasible set.
% \end{Theorem}
% \begin{proof} The details can be found in Appendix \ref{iteration-appendix}.
% \end{proof}


\section{Numerical Results}\label{numerical}

In this section, we present numerical results to validate the theoretical analysis and optimization model developed in the previous sections. For clarity and better visualization, we consider a system with two sensors and two processes in our numerical experiments. However, it is important to note that all analyses can be generalized to systems with more sensors and processes. We vary different system parameters to verify our theoretical results in Section \ref{aoi-S}. After verifying the theoretical analysis, we investigate the optimization model described in Section \ref{aoi-opt} under different conditions and provide the results.

%\ali{I don't think we should talk about 2D grid search...people will say why did you do the analysis if u want to do grid search. Take this example back to after Theorem 2. And just say that we optimize the system and provide the results, and that's it. }, using a 2D grid search with a step size of 0.002 for both preemption probabilities, $\np_1$ and $\np_2$. While Theorem \ref{Theo2} provides an upper bound requiring at most 82 iterations for an example case with $M = 10$, $\mu = 5$, $\lambda_C = 15$, and $\epsilon_0 = 0.01$ that is not excessively high. However, the grid search is particularly suitable here due to the low system complexity.

Figure \ref{fig:exp_res} compares analytical results with experimental results for varying service rates, arrival rates, and preemption probabilities. Our simulations are unit-time-based and were run for 1 million units of time. The lowest arrival rate is 0.5 arrivals per unit time, ensuring at least $5 \times 10^5$ arrivals for each process to guarantee convergence. Simulation results are represented as circles, while the theoretical results derived from our analysis are depicted as solid lines in the figures. The strong alignment between the two verifies the validity of our analysis.


\begin{figure}[t!]
    \centering
    % Top-left subfigure
    \begin{subfigure}[t]{0.22\textwidth}
        \centering
        \includegraphics[width=1\textwidth]{figures/1.png}
        \caption{AoI versus $\lambda_1$ for different $\np_1=\np_2$ values with $\mu=2, \lambda_2=1,$ $\bfc= \begin{bmatrix}
1 & 0.5 \\
0.5 & 1
\end{bmatrix}$.}
        \label{fig:exp1}
    \end{subfigure}
    \hfill
    % Top-right subfigure
    \begin{subfigure}[t]{0.22\textwidth}
        \centering
        \includegraphics[width=1\textwidth]{figures/2.png}
        \caption{AoI versus $\mu$ for different $\np_1=\np_2$ values with $\lambda_1=1, \lambda_2=6,$ $\bfc= \begin{bmatrix}
1 & 0.5 \\
0.5 & 1
\end{bmatrix}.$}
        \label{fig:exp2}
    \end{subfigure}
    \caption{Simulation results vs theoretical findings.}
    \vspace{-14pt}
    \label{fig:exp_res}
\end{figure}


% \ali{Fix the figures after we decide on preemption probabilities; }

%\suresh{The captions for Fig. 3 (a) and (b) seem to be reversed.}

After that, we evaluate the optimal probabilistic preemption strategy for various correlations, arrival rates, and service rates. Figures \ref{l_res} and \ref{mu_res} illustrate the optimal probability values as well as their corresponding AoI values.  Additionally, the figures compare these optimal AoI results to the AoI values achieved under no preemption and full preemption scenarios. 

%\suresh{Clarify what you mean by aggregate updates.}


In Figure (\ref{fig:l1}), when the correlation matrix is an identity matrix, the preemption probability of Sensor 2 increases with $\lambda_1$ until reaching full preemption. On the other hand, Sensor 1’s probability starts high but decreases as $\lambda_1$ increases. This reflects a shift in importance toward Sensor 2 when Sensor 1’s arrival rate is higher. This behavior illustrates that packets with low arrival rates should preempt those with higher arrival rates when the correlation effect is eliminated. On the other hand, Figure (\ref{fig:l2}) shows the effect of the correlation matrix. Sensor 1 fully dominates due to its complete information about both processes and even at high \(\lambda_1\), preempting other packets with a packet from Sensor 2 is not preferable since it lacks any information about Process 1. In addition, Figure (\ref{fig:theta}) illustrates how optimal preemption evolves when each sensor fully tracks one process and partially observes the other with probability $\theta$. When $\theta$ is small, the sensor with the lower arrival rate is given higher preemption priority. However, as $\theta$ increases, the sensors become more similar in the information they provide. This reduces the distinction between them, leading to an optimal strategy of preempting every packet. These results underscore how the correlation matrix governs the preemption strategy and sensor roles. 






\begin{figure}[h!]
    \centering
    % Top-left subfigure
    \begin{subfigure}[b]{0.24\textwidth}
        \centering
        \includegraphics[width=1\textwidth]{figures/7_2.png}
        \caption{$\frac{}{}$ $\frac{}{}$$\frac{}{}$ $\frac{}{}$$\frac{}{}$$\frac{}{}$ $\frac{}{}$$\frac{}{}$ $\frac{}{}$}
        \label{fig:l1}
    \end{subfigure}
    \hfill
    % Top-right subfigure
    \begin{subfigure}[b]{0.24\textwidth}
        \centering
        \includegraphics[width=1\textwidth]{figures/8_2.png}
        \caption{$\frac{}{}$ $\frac{}{}$$\frac{}{}$ $\frac{}{}$$\frac{}{}$$\frac{}{}$ $\frac{}{}$$\frac{}{}$ $\frac{}{}$}
        \label{fig:l2}
    \end{subfigure}
% \vspace{-10pt}
%     \vskip\baselineskip
%     % Bottom-left subfigure
%     \begin{subfigure}[b]{0.22\textwidth}
%         \centering
%         \includegraphics[width=0.9\textwidth]{figures/9.png}
%         \caption{}
%         \label{fig:l3}
%     \end{subfigure}
%     \hfill
%     % Bottom-right subfigure
%     \begin{subfigure}[b]{0.22\textwidth}
%         \centering
%         \includegraphics[width=0.9\textwidth]{figures/10.png}
%         \caption{}
%         \label{fig:l4}
%     \end{subfigure}
    \caption{Optimal preemption probabilities under varying $\lambda_1$ to show the effect of correlation matrix.}
    \label{l_res}
    \vspace{-8pt}
\end{figure}

Lastly, in Figure (\ref{fig:div}), we see that Sensor 2 has more priority (higher preemption probability) than Sensor 1, even though Sensor 1 has more aggregate updates on average, i.e., 
the sum of the elements in the corresponding row of the correlation matrix is larger for Sensor 1. This stems from Sensor 2 having more diverse updates and playing a bigger role in minimizing the sum AoI. Note that in a single-source memoryless system, preempting packets is generally beneficial for reducing the AoI. Specifically, self-preempting, where a packet preempts another packet with information from the same processes, is profitable in our system. An increase in preemption probability results in a gain from self-preemption. However, there is a trade-off because it also increases the probability of preempting an informative packet for additional or different processes that can negatively impact the AoI. When $\lambda_1$ is low, the benefit of self-preemption for Sensor 1 is higher than the loss of preempting a packet from Sensor 2. However, as $\lambda_1$ increases, the risk of losing informative packets from other processes also rises when a packet from Sensor 1 preempts. Consequently, the optimal preemption probability decreases as $\lambda_1$ grows. Notably, this decrease begins even when $\lambda_1$ is smaller than $\lambda_2$, making packets from Sensor 1 relatively rare. Nevertheless, the diversity of information provided by Sensor 2 makes its packets more valuable, even when the aggregate correlation is low.
\vspace{-12pt}

%it is better to let only packets from Sensor 2 preempt because there might be a risk of losing a packet with more information when a packet from Sensor 1 preempts. With the increase in $\mu$, service time reduces, and packets from Sensor 1 start preempting with some probability to obtain the optimal AoI because it reduces to AoI by preempting their own packets. However, this probability decreases after some threshold and goes to zero again. The reason is that when $\mu$ is high, the gain of self-preemption is lower than the loss of preempting a packet with more information.


\begin{figure}[h!]

    \centering
    % Top-left subfigure
     \begin{subfigure}[b]{0.24\textwidth}
        \centering
        \includegraphics[width=1\textwidth]{figures/3_2.png}
        \caption{$\frac{}{}$$\frac{}{}$$\frac{}{}$$\frac{}{}$$\frac{}{}$$\frac{}{}$ $\frac{}{}$$\frac{}{}$ $\frac{}{}$}
        \label{fig:theta}
    \end{subfigure}
    \hfill
    % Top-right subfigure
    \begin{subfigure}[b]{0.24\textwidth}
        \centering
        \includegraphics[width=1\textwidth]{figures/4_2.png}
        \caption{$\frac{}{}$$\frac{}{}$ $\frac{}{}$$\frac{}{}$$\frac{}{}$$\frac{}{}$ $\frac{}{}$$\frac{}{}$ $\frac{}{}$}
        \label{fig:div}
    \end{subfigure}   
\vspace{-15pt}
    \caption{Optimal preemption probabilities under different conditions.}
\vspace{-10pt}
    \label{mu_res}
\end{figure}

%\suresh{Make the lines bolder.}

%\suresh{I'm not sure this figure is interesting anymore; the diff between full-pre and optimal-pre AoI is tiny.}



%\suresh{There is so much theory on managing the complexity of the optimization. Would be good to present some results of that here. It's not even clear what the $\epsilon_0$ you used to get the results and how many iterations were needed.} \egemen{You are right. However, we use grid search to find the optimum here. How should I mention it?}\suresh{I'm not sure. How many iterations does gridsearch take and how does it relate to the UB you derived. Seems that your optimization is disconnected from the result of Thm 2. Is that the case?} \egemen{Yes, the UB is for a branch and bound algorithm. However, for this experiment, we check every possible point with step size 0.002 to find the optimal probability.}








\section{Conclusion}

This work analysed the results of evolutionary wrapper approaches using decision tree based models as proxies and compared them with common \gls{FE} techniques on a \gls{HL} detection problem. Three experiments were conducted using the proposed framework, each employing different proxy models.

When comparing the three experiments, an interesting behaviour of the framework was discovered, when changing the proxy model. The \gls{DT} experiment drastically reduced the number of features, while the other models did not. To further reduce the number of features, one could bias the grammar or apply some penalty in the fitness function for the individuals that use a large number of features. This might not change the behaviour when using different models other than a \gls{DT}, but it forcefully reduces the number of features.  

The results confirm that FEDORA can reduce the dimensionality of the data while statistically maintaining baseline performance, in every experiment. The framework consistently outperforms the remaining \gls{FE} methods, with statistical significance and large effect sizes, proving itself as a viable alternative.

The best result obtained is 76.2\% balanced accuracy using an individual from the \gls{RF} experiment, and a \gls{XGB} algorithm as the testing model, using 57 total features (45 Original, 6 Engineered and 6 Complex) out of the 60 original ones. When using the least amount of features, the best result is 72,8\% balanced accuracy using an individual from the \gls{DT} experiment and a \gls{RF} algorithm as the testing model, using a single complex feature.

In future work, exploring the above-mentioned behaviours might be relevant to better understanding them, namely when biasing the grammar or penalizing the use of many features in the fitness function. Concerning the explainability of the FEDORA transformations, researching meaningful grammar operators might prove useful in addressing problem-specific needs. In this case, having logical operators for the boolean features, which have values of "yes" or "no", and the choice of a simple decision algorithm as the proxy, may increase explainability. Additionally, the previous study has identified several areas for future research, yet to be addressed. For instance, comparing the framework with other common and more complex methods and completing the full \gls{ML} pipeline through the use of a method that addresses the \gls{CASH}, such as \cite{assunccao2020evolution}, and comparing it to other full pipeline frameworks, could be beneficial for contextualizing and evaluating the framework within the \gls{AutoML} and \gls{EC} domains. The framework still needs to be analysed with different datasets to properly assess its generalization capabilities.

\section*{Acknowledgment}
This work was supported in part by NSF grant CNS-2219180.
\newpage

\bibliographystyle{ieeetr}
\bibliography{references}
\ifthenelse{\boolean{withappendix}}
{\appendices
\section{} 
\label{reduction-P}

To prove our argument, we apply the splitting property of the Poisson process. Let \( N(t) \) be a Poisson process with rate parameter \( \lambda \). If events are split into two groups with probabilities \( p \) and \( 1-p \), then the resulting processes \( N_1(t) \) and \( N_2(t) \) are independent Poisson processes with rate parameters \( p\lambda \) and \( (1-p)\lambda \) respectively \cite{splitting_poisson}.

From process \( j \)'s perspective, we can split arrivals from sensor \( i \) into two groups: informative and uninformative arrivals with probabilities \( \nc_{ij} \) and \( 1-\nc_{ij} \), respectively. The rate of arrivals from sensor \( i \) is \( \lambda_i \), so the rate of informative arrivals for process \( j \) from sensor \( i \) is \( \nc_{ij}\lambda_i \). Additionally, we can further split the informative arrivals based on whether they can preempt ongoing service. The rate of informative arrivals that can preempt ongoing service for process \( j \) from sensor \( i \) is \( \np_{i}\nc_{ij}\lambda_i \) and the rate of informative arrivals that can not preempt ongoing service for process \( j \) from sensor \( i \) is \( (1-\np_{i})\nc_{ij}\lambda_i \). Since all these arrivals are Poisson, we can merge them into a single process. The total arrival rate of informative packets that can preempt ongoing service for process \( j \) is given by

\begin{equation}
\Tilde{\lambda}_j = \sum_{i=1}^{N} \np_{i}\nc_{ij}\lambda_i
\end{equation}

Similarly, the total arrival rate of informative packets that can not preempt ongoing service for process \( j \) is

\begin{equation}
\Tilde{\lambda}_j = \sum_{i=1}^{N} (1-\np_{i})\nc_{ij}\lambda_i
\end{equation}

We can express these rates in vector form as follows:

\begin{equation}
\boldsymbol{\Tilde{\lambda}}^T = \begin{bmatrix}
\Tilde{\lambda}_{1} & \Tilde{\lambda}_{2} & \dots & \Tilde{\lambda}_{M}
\end{bmatrix} = (\boldsymbol{\lambda}^T \odot \bfp^T) \bfc,
\end{equation}
\begin{equation}
\boldsymbol{\dot{\lambda}}^T = \begin{bmatrix}
\dot{\lambda}_{1} & \dot{\lambda}_{2} & \dots & \dot{\lambda}_{M}
\end{bmatrix} = (\boldsymbol{\lambda}^T \odot (1-\bfp^T)) \bfc,
\end{equation}


The importance of the packet is whether it has information of process $j$ so  we can say that The system with $N$ sensors and arrival rates $\boldsymbol{\lambda}$ shown in Figure \ref{fig:system_model} equivalents to the system with two sources as shown in Figure \ref{fig:equiv_model} from process $j$'s perspective.


\section{}\label{spv-appendix}


We adopt the stochastic hybrid system (SHS) model as defined in \cite{yates2019}, with a key distinction: our model incorporates probabilistic preemption. The system dynamics are depicted in Figure \ref{fig:equiv_model} so we can analyze the AoI for any process $i$ and generalize it. First, the discrete state is denoted as $q(t) = q \in Q = \{0, 1, 2\}$, where $q = 0$ represents an idle server, and $q \in \{1, 2\}$ signifies that an update packet is currently being serviced. The continuous state is described as $x(t) = [x_0(t), x_1(t)]$, where $x_0(t)$ represents the current age of the process, and $x_1(t)$ captures the potential age if the packet in service is successfully delivered. Notably, $x_1(t)$ is irrelevant in state $0$ since no packet is in service. In state $1$, $x_1(t)$ corresponds to the age of the informative update being serviced. Conversely, in state $2$, where an uninformative update is in service, the completion of this update does not affect the process age, rendering $x_1(t)$ irrelevant in this state as well.

\begin{table}[h]
\centering
\caption{Table of Transitions for the Markov Chain in Figure \ref{fig:shs}.}
\begin{tabular}{c c c c c c}
\toprule
$l$ & $q_l \rightarrow q'_l$ & $\lambda^{(l)}$ & $\mathbf{xA}_l$ & $\mathbf{A}_l$ & $\mathbf{v}_{q_l}\mathbf{A}_l$ \\
\midrule
1 & $0 \rightarrow 1$ & $\Tilde{\lambda}_{1}+\dot{\lambda}_{1}$ & $\begin{bmatrix} x_0 & 0 \end{bmatrix}$ & \small $\begin{bmatrix} 1 & 0 \\ 0 & 0 \end{bmatrix}$ \normalsize & $\begin{bmatrix} v_{00} & 0 \end{bmatrix}$ \\
2 & $0 \rightarrow 2$ & $\lambda_{C}-\Tilde{\lambda}_{1}-\dot{\lambda}_{1}$ & $\begin{bmatrix} x_0 & 0 \end{bmatrix}$ & \small $\begin{bmatrix} 1 & 0 \\ 0 & 0 \end{bmatrix}$ \normalsize & $\begin{bmatrix} v_{00} & 0 \end{bmatrix}$ \\
3 & $1 \rightarrow 0$ & $\mu$        & $\begin{bmatrix} x_1 & 0 \end{bmatrix}$ & \small$\begin{bmatrix} 0 & 0 \\ 1 & 0 \end{bmatrix}$ \normalsize & $\begin{bmatrix} v_{11} & 0 \end{bmatrix}$ \\
4 & $1 \rightarrow 1$ & $\Tilde{\lambda}_{1}
$  & $\begin{bmatrix} x_0 & 0 \end{bmatrix}$ & \small$\begin{bmatrix} 1 & 0 \\ 0 & 0 \end{bmatrix}$\normalsize & $\begin{bmatrix} v_{10} & 0 \end{bmatrix}$ \\
5 & $1 \rightarrow 2$ & $\Tilde{\lambda}_{C}-\Tilde{\lambda}_{1}$  & $\begin{bmatrix} x_0 & 0 \end{bmatrix}$ & \small$\begin{bmatrix} 1 & 0 \\ 0 & 0 \end{bmatrix}$ \normalsize & $\begin{bmatrix} v_{10} & 0 \end{bmatrix}$ \\
6 & $2 \rightarrow 0$ & $\mu$        & $\begin{bmatrix} x_0 & 0 \end{bmatrix}$ & \small$\begin{bmatrix} 0 & 0 \\ 1 & 0 \end{bmatrix}$ \normalsize & $\begin{bmatrix} v_{20} & 0 \end{bmatrix}$ \\
7 & $2 \rightarrow 1$ & $\Tilde{\lambda}_{1}$  & $\begin{bmatrix} x_0 & 0 \end{bmatrix}$ & \small$\begin{bmatrix} 1 & 0 \\ 0 & 0 \end{bmatrix}$ \normalsize & $\begin{bmatrix} v_{20} & 0 \end{bmatrix}$ \\
\bottomrule
\end{tabular}
\label{shs_table}
\end{table}


\begin{figure}
    \centering
    \includegraphics[width=0.5\linewidth]{figures/shs.png}
    \caption{The Markov chain for updates.}
    \label{fig:shs}
\end{figure}

A Markov chain representing the discrete state $q(t)$ is depicted in Figure~\ref{fig:shs}. The corresponding transitions of the SHS at state $q_l$ are detailed in Table~\ref{shs_table}. In the figure, a directed edge $l$ from node $q$ to node $q'$ indicates that transitions from state $q$ to state $q'$ occur at an exponential rate $\lambda^{(l)}$, as specified in the table. 





%\suresh{Incomplete.}

We first show that the stationary probability vector $\pi$ satisfies $
\mathbf{\pi D} = \mathbf{\pi Q} \quad \text{with}$ 
\begin{align}
\quad
\mathbf{D} = \text{diag}[\lambda_{C}, \mu + \Tilde{\lambda}_{C}, \mu + \Tilde{\lambda}_{1}], \quad  \\ \mathbf{Q} = 
\begin{bmatrix}
0 & \Tilde{\lambda}_{1}+\dot{\lambda}_{1} & \lambda_{C}-\Tilde{\lambda}_{1}-\dot{\lambda}_{1} \\
\mu & \Tilde{\lambda}_{1} & \Tilde{\lambda}_{C}-\Tilde{\lambda}_{1} \\
\mu & \Tilde{\lambda}_{1} & 0
\end{bmatrix}.
\end{align}
Applying $\sum_{i=0}^{2} \pi_i = 1$, the stationary probabilities are 
\begin{equation}
\pi_0 = \frac{\mu}{(\lambda_C + \mu)}, \label{pi0}
\end{equation}
\begin{equation}
\pi_1 = \frac{\lambda_C\Tilde{\lambda}_{1} + \dot{\lambda}_{1}\mu + \Tilde{\lambda}_{1}\mu}{(\lambda_C + \mu)(\Tilde{\lambda}_{C} + \mu)}, \label{pi1}
\end{equation}
\begin{equation}
\pi_2 = \frac{\Tilde{\lambda}_{C}\lambda_C + \lambda_C\mu -\lambda_C\Tilde{\lambda}_{1}  - \dot{\lambda}_{1}\mu - \Tilde{\lambda}_{1}\mu}{(\lambda_C + \mu)(\Tilde{\lambda}_{C} + \mu)} . \label{pi2}
\end{equation}

\section{}\label{aoi-appendix}





Given the SHS model and $\pi$ in Appendix \ref{spv-appendix}, we can evaluate $\bar{v}$ to find the AoI. Let 
\begin{equation}
\mathbf{\bar{v}} = [\mathbf{\bar{v}_0} \ \mathbf{\bar{v}_1} \ \mathbf{\bar{v}_2}] = [\bar{v}_{00} \ \bar{v}_{01} \ \bar{v}_{10} \ \bar{v}_{11} \ \bar{v}_{20} \ \bar{v}_{21}].   
\end{equation}
It follows that
\begin{equation}
\mathbf{\bar{v}D} = \mathbf{\pi B} +  \mathbf{\bar{v}R},
\end{equation}
where 
\begin{equation}
\mathbf{D} = \text{diag}[\lambda_C, \lambda_C, \mu + \Tilde{\lambda}_{C}, \mu + \Tilde{\lambda}_{C}, \mu + \Tilde{\lambda}_{1}, \mu + \Tilde{\lambda}_{1}],
\end{equation}
\begin{equation}
\mathbf{B} =
\begin{bmatrix}
1 & 0 & 0 & 0 & 0 & 0 \\
0 & 0 & 1 & 1 & 0 & 0 \\
0 & 0 & 0 & 0 & 1 & 0
\end{bmatrix},
\end{equation}
and
\begin{equation}
\mathbf{R} = 
\begin{bmatrix}
0 & 0 & \Tilde{\lambda}_{1}+\dot{\lambda}_{1}  & 0 & \lambda_{C}-\Tilde{\lambda}_{1}-\dot{\lambda}_{1} & 0 \\
0 & 0 & 0 & 0 & 0 & 0 \\
0 & 0 & \Tilde{\lambda}_{1} & 0 & \Tilde{\lambda}_{C}-\Tilde{\lambda}_{1} & 0 \\
\mu & 0 & 0 & 0 & 0 & 0 \\
\mu & 0 & \Tilde{\lambda}_{1} & 0 & 0 & 0 \\
0 & 0 & 0 & 0 & 0 & 0
\end{bmatrix}.
\end{equation}

Then, we obtain $\bar{v}_{01}=\bar{v}_{21} = 0 $ and 
\begin{align}
&
\label{pi_v}
\begin{bmatrix}
\bar{\pi}_0 & \bar{\pi}_1 & \bar{\pi}_1 & \bar{\pi}_2
\end{bmatrix}
= \\ \nonumber \hat{\mathbf{v}}&
\begin{bmatrix}
\lambda_{C} & -\Tilde{\lambda}_{1}-\dot{\lambda}_{1} & 0 & \Tilde{\lambda}_{1}+\dot{\lambda}_{1}-\lambda_{C} \\
0 & \mu + \Tilde{\lambda}_{C}-\Tilde{\lambda}_{1} & 0 & \Tilde{\lambda}_{1} - \Tilde{\lambda}_{C} \\
-\mu & 0 & \mu + \Tilde{\lambda}_{C} & 0 \\
-\mu & -\Tilde{\lambda}_{1} & 0 & \mu + \Tilde{\lambda}_{1}
\end{bmatrix}, \\
\text{where } 
\hat{\mathbf{v}} &= 
\begin{bmatrix}
\bar{v}_{00} & \bar{v}_{10} & \bar{v}_{11} & \bar{v}_{20}
\end{bmatrix}. \nonumber
\end{align}

After solving eq. (\ref{pi_v}) using eqs. (\ref{pi0}), (\ref{pi1}), and (\ref{pi2}), we determine $\mathbf{\bar{v}}$. Later, we find the average age of information using the formula for a single process $j$ $\Delta_j = \sum_{q=0}^2 \bar{v}_{10}$ as follows:

%\suresh{Is this what you defined as $\Delta_{\rm sum}$ earlier?}

\footnotesize
\begin{align}
\Delta_j = \frac{\lambda_{C}^{2} \tilde{\lambda}_C + \lambda_{C}^{2} \mu + \lambda_{C} \dot{\lambda}_1 \mu + 2 \lambda_{C} \tilde{\lambda}_C \mu + 2 \lambda_{C} \mu^{2} + \tilde{\lambda}_C \mu^{2} + \mu^{3}}{\mu \left(\lambda_{C}^{2} \tilde{\lambda}_1 + \lambda_{C} \dot{\lambda}_1 \mu + 2 \lambda_{C} \tilde{\lambda}_1 \mu + \dot{\lambda}_1 \mu^{2} + \tilde{\lambda}_1 \mu^{2}\right)}
\end{align}
\normalsize

\section{}\label{iteration-appendix}


In this section, we discuss the upper bound on the number of iterations required by the outer space accelerating branch-and-bound algorithm to achieve a global $\epsilon_0$-optimal solution. According to Theorem 5 in \cite{JIAO2022112701}, for any given positive error $\epsilon_0 \in (0, 1)$, the algorithm converges to the desired solution in at most
\begin{equation}
p \cdot \left\lceil \log_2 \frac{p\tau \delta(\Omega)}{\epsilon_0} \right\rceil 
\end{equation}
iterations.

Here, the symbols used in the theorem are defined as follows:

\begin{itemize}
    \item $\Omega \subseteq \mathbf{R}^p$ is a compact hyper-subrectangle, and $\delta(\Omega)$ is defined as:
    \begin{equation}
    \delta(\Omega) = \max_{i=1,2,\dots,p} \{ \bar{U}_i - \bar{L}_i \},    
    \end{equation}
    where $\bar{U}_i$ and $\bar{L}_i$ represent the upper and lower bounds of the $i$-th dimension of the rectangle $\Omega$.

    \item $\tau$ is defined as:
    \begin{equation}\label{tau_eq}
    \tau = \max_{i=1,\dots,p} \frac{4 \max\{|\bar{l}_i|, |\bar{u}_i|\}}{\min\{\bar{L}_i, \bar{U}_i, \bar{L}_i^2, \bar{U}_i^2\}},
    \end{equation}
    where the terms are determined as follows:
    \begin{align}
        \bar{l}_i &= \min_{y \in \Theta} n_i(y), \quad \bar{u}_i = \max_{y \in \Theta} n_i(y), \nonumber \\
        \bar{L}_i &= \min_{y \in \Theta} d_i(y), \quad \bar{U}_i = \max_{y \in \Theta} d_i(y).
    \end{align}

    \item The terms $n_i(y)$ and $d_i(y)$ come from the problem defined as:
    \begin{align}
    \quad \min f(y) = \sum_{i=1}^p \frac{n_i(y)}{d_i(y)}, \quad \nonumber \\ \text{s.t.} \; y \in \Theta = \{y \in \mathbf{R}^n \mid Ay \leq b \}. 
    \end{align}
    \end{itemize}


We can reformulate our problem to determine the upper bound using these definitions. The variable in our problem is $\bfp$, and the objective is specified in (\ref{objective_func}). There are $M$ different linear fractions in the objective. The numerators of these fractions increase as any element of $\bfp$ increases. Consequently, we obtain $\bar{l}_i$ when $\bfp = 0$ and $\bar{u}_i$ when $\bfp = 1$ as follows:
\begin{align}
            \bar{l}_i &= \mu(\mu + \lambda_C)^2 + \sum_{i=1}^{N} \lambda_{i}\mu\lambda_C \nc_{ij},\quad \bar{u}_i = (\mu + \lambda_{C})^3
\end{align}

 Similarly, the denominators of these fractions decrease as any element of $\bfp$ increases, leading to $\bar{L}_i$ when $\bfp = 0$ and $\bar{U}_i$ when $\bfp = 1$.
 \begin{align}
            \bar{L}_i &=  (\mu + \lambda_C) \mu^2 \sum_{i=1}^{N} \nc_{ij} \lambda_{i}, \quad \bar{U}_i =  (\mu + \lambda_C)^2 \mu \sum_{i=1}^{N} \nc_{ij} \lambda_{i}.
\end{align}

After that, $\delta(\Omega)$ becomes: 

\begin{align}
        \delta(\Omega) = \max_{i=1,2,\dots,M} \{(\mu + \lambda_C)\lambda_C \mu \sum_{i=1}^{N} \nc_{ij} \lambda_{i}\} \leq (\mu + \lambda_C)\lambda_C^2 \mu , 
\end{align}

Last, we find $\tau$. In our problem, all parameters and variables are positive, so both the nominators and the denominators are positive, which can help us simplify eq. (\ref{tau_eq}) and obtain $\tau$ as follows:

    \begin{align}
    \tau = \max_{i=1,\dots,M} \frac{4 \bar{u}_i}{\bar{L}_i^2} = \frac{4 (\mu + \lambda_{C})}{\mu^4 \hat{\lambda}_{\min}^2},\\ \nonumber
    \text{where } \hat{\lambda}_{\min} = \min(\boldsymbol{\lambda}^T \bfc)
    \end{align}

Putting all together, for any given positive error $\epsilon_0 \in (0, 1)$, the outer space accelerating branch-and-bound algorithm can seek out a global $\epsilon_0$-optimum solution in at most 
\begin{equation}
M \cdot \left\lceil \log_2 \frac{4M (\mu+\lambda_C)^2\lambda_C^2}{\epsilon_0\mu^3\hat{\lambda}_{\min}^{2}} \right\rceil
\end{equation}
iterations as shown in Theorem \ref{Theo2}.}
{}


\end{document}

