\section{Introduction}

The growing interest in blockchain and distributed ledger technology has resulted in many research advances in the field, ranging from improvements on the consensus layer~\cite{kauri,narwahl} to sharding~\cite{omniledger} and parallel transaction execution~\cite{blockstm,chiron}. 
As most blockchains still execute transactions sequentially, parallel smart contract execution engines that take advantage of modern multi-core architectures are considered a crucial building block to scale blockchain transaction throughput~\cite{blockstm}.

Existing approaches to parallel execution can be roughly divided into two categories: optimistic and guided. Optimistic approaches, such as Block-STM~\cite{blockstm}, are designed to execute transactions in parallel, detect conflicts as they arise, and re-execute affected transactions. 
However, in blockchain environments characterized by highly contended workloads~\cite{etheng,chiron}, conflicts arise more often, requiring more frequent re-executions of transactions.

In contrast to optimistic approaches, guided approaches strictly limit read/write access by requiring transactions to pre-declare an exhaustive list of resources (i.e., addresses) that will be accessed during execution. This allows the scheduler to identify independent transactions and execute them concurrently. Examples of this approach include FuelVM, Solana, or Sui~\cite{fuelvm,solana,sui}. While this avoids the re-execution overhead in settings with high contention, it puts additional load on the application developers. Furthermore, in some cases, it may not be possible to precisely predict at transaction creation time which resources will be accessed during execution, as the application state might change in the meantime. Then, an overly pessimistic approach is required, locking a wider range of resources, and potentially resulting in the sequential execution of transactions that otherwise could have been executed concurrently.

Combining both approaches, Polygon recently introduced an update~\cite{polygonupdate} that extracts transaction dependencies during block creation and includes this dependency tree as metadata in the block. This approach allows to optimize scheduling during the execution phase, avoiding unnecessary re-executions and pessimistic locking~\cite{polygonupdate}. Nonetheless, this approach requires executing transactions on the critical path of consensus during block creation, crippling the potential throughput.
A similar approach is \basesys~\cite{chiron} which leverages execution hints to speed up execution on struggling validators and full nodes. Chiron guarantees safety in the presence of invalid hints by utilizing the validation step of Block-STM, which identifies conflicting resource accesses and reschedules transactions that potentially accessed shared resources in parallel for re-execution~\cite{blockstm}.

However, as outlined in \cite{chiron}, due to the characteristics of blockchain workloads, transaction execution remains a significant bottleneck.
This is the case, as transactions that access the same resources must be executed sequentially and, as several recent studies have shown, in practice a significant portion of the transactions access the same resources, resulting in a long sequential path of transactions slowing down the system~\cite{ethparallelimpro,chiron}.
As such, the performance is currently limited by the workload.

Due to the nature of the problem, a single popular application can bottleneck the execution engine and cripple the throughput of the system~\cite{chiron}.
This could be a newly launched NFT, a popularly traded token, or even the on/off-boarding of a popular layer-2 smart contract.
This is further aggravated by the fact that most existing blockchains that support parallel execution currently have no pricing mechanisms to charge clients for accessing popular resources causing system bottlenecks.

Most blockchains such as Ethereum~\cite{ethereum} prevent extensive execution times by limiting the combined execution complexity in gas of each given block.
However, a single parameter is insufficient in the context of parallel execution, as it does not take transaction dependencies and potential parallelization into account. Therefore, a novel approach is necessary to make block assembly sensitive to transaction dependencies and execution complexity, charging clients for accessing popular resources and delaying transactions that would otherwise bottleneck the execution. 

In this paper, we propose \sys, a novel approach to construct blocks that takes both the execution complexity in gas and the distribution of resource accesses into account to construct ''Good Blocks'' that can be executed efficiently in parallel. We evaluate \sys extensively under a series of realistic workloads, showing a consistent speed-up up to \sysmax compared to native parallel execution.
\sys not only vastly improves the execution performance but also prevents popular or malicious applications from bottlenecking the system, eliminating a performance attack scenario.
\sys provides different latency paths between transactions accessing congested and not congested resources. Transactions can still be fast-tracked by paying higher transaction fees, resulting in a price that more closely reflects its resource consumption. We discuss this further in Section~\ref{sec:discussion}.

Moreover, thanks to its modular design, \sys can be integrated into any state-of-the-art blockchain seamlessly, without the need for a hard fork or modifications to the execution engine or consensus mechanism. \sys operates stateless and only requires execution hints such as the resources that will be accessed during execution. In blockchains such as Sui and Solana~\cite{sui,solana} these hints are already present during block construction, while in blockchains such as Aptos or Ethereum~\cite{aptos,ethereum} these hints could either be simulated in a pre-execution step or generated at the full nodes.

In summary, we provide the following contributions:

\begin{itemize}
    \item We propose \sys, a novel and modular block construction algorithm and approach to speed up parallel execution without security tradeoffs.
    \item We evaluate \sys integrated with both an optimistic and a guided execution engine under the \basesys benchmarks resulting in a significant speed-up in almost all settings.
\end{itemize}

In Section~\ref{sec:systemmodel}, we present the System Model of \sys, followed by a detailed overview of \sys in Section~\ref{sec:overview}. Next, we describe the implementation and evaluation in Section~\ref{sec:evaluation}. Related work is reviewed in Section~\ref{sec:relatedwork}, and potential drawbacks, along with their solutions, are discussed in Section~\ref{sec:discussion}. Finally, we conclude the paper in Section~\ref{sec:conclusion}.