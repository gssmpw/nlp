\section{Discussion}
\label{sec:discussion}

While \sys can achieve a performance boost of over \sysmax, there are tradeoffs. In this section, we discuss these tradeoffs and potential solutions.

\subsection{Malicious Leader}

While a correct leader can construct blocks that significantly speed up the system, the opposite is true for malicious leaders. In \sys the leader constructs the block sensitive to the number of cores and a gas per core measure, however, no mechanism in \sys enforces the leader to construct a block following this blueprint. Even though this might seem like an oversight, it is impossible to distinguish between a correct leader handling a fully sequential workload and a malicious leader deliberately constructing a sequential block.

This issue is inherent to blockchains that support parallel execution, such as Solana, Sui, or Aptos~\cite{solana,sui,aptos}. Two potential approaches could mitigate this challenge. One approach involves an expensive combination of a fair ordering protocol~\cite{fairordering} and a pre-execution stage on the critical path of consensus such as Pompe~\cite{zhang2020byzantine}. Alternatively, leaders could be incentivized through a game-theoretic framework to construct highly parallelizable blocks, with penalties imposed for creating overly sequential ones. However, a detailed analysis of these frameworks is beyond the scope of this work and is left for future research.

Therefore, while \sys extends the capabilities of correct nodes to improve the system throughput and empowers them to prevent clients from bottlenecking the system, it does not alter the role a malicious validator could play in the system compared to the state of the art. In fact, due to its modular nature, individual validators on many blockchains could already plug \sys into their stack without requiring a hard fork.

%However, while \sys does not alter the the power of the leader when it comes to performance attacks, it might make it harder to detect short-term censorship attacks.

\subsection{Censorship Resistance}

A common concern for leader-based protocols is censorship resistance. In \sys, the leader has, as part of the protocol, the power to delay some transactions to speed up the overall system. 
However, as the leaders in existing protocols such as Ethereum or Aptos already have this power as there is no mechanism that controls this, \sys would not hand the leader stronger censorship powers compared to the state of the art. 

Nonetheless, protocols focused on short-term censorship resistance such as \cite{xue2023bigdipper} might not work out of the box with \sys.
Thus, adjustments to the protocol would be necessary, to only allow a leader to delay a given transaction up to some bounds.
This presents a direct trade-off between performance and short-term censorship resistance. 

Furthermore, protocols focused on fair ordering, such as \cite{8526804} often require the transaction and metadata to be encrypted which strips \sys of the capability to use transaction meta-data to construct ``Good Blocks''. Nonetheless, these approaches are also generally incompatible with hint-based execution schemes as used in \basesys, Sui, or Solana.

\subsection{Transaction Fees \& Client Incentives}

While a malicious leader can arbitrarily delay a client transaction, in \sys correct nodes might also delay client transactions to improve the overall system throughput. However, in some cases, a client might want their transaction to be included with higher urgency even if it accesses very hot resources, e.g. when a bidding process is approaching the time limit.

Integrating a mechanism with \sys that allows client transactions to be included with higher priority is fairly straightforward. Blockchains such as Bitcoin and Ethereum~\cite{bitcoin,ethereum} already use pricing mechanisms to prioritize transaction inclusion. Therefore, a transaction with a higher fee could be transferred to the beginning of the first batch in the batch handler to guarantee its inclusion in the next block.
In fact, similar to the local fee markets in Solana~\cite{solana}, this kind of pricing scheme, in combination with \sys would naturally result in a higher price for accessing hot resources, incentivizing smart contract developers to design their smart contracts with concurrency in mind and incentivizing users to avoid hot resources during system congestion times. This can help to balance the system beyond the already existing throughput advantages of \sys.