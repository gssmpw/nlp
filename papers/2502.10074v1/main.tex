
\documentclass[runningheads]{llncs}
\usepackage[T1]{fontenc}
\usepackage{xcolor}
%\usepackage{tikz}
\usepackage{amsmath}
%\usepackage[compact]{titlesec}
\usepackage{graphicx}
%\usepackage{textcomp}
\usepackage{xspace}
%\usepackage{hyperref}
%\usepackage{xurl}
%\usepackage{subcaption}
\usepackage{algorithm, algpseudocode}
%\usepackage[normalem]{ulem}
\usepackage{pifont}
\newcommand{\cmark}{\ding{51}}%
\newcommand{\xmark}{\ding{55}}%
\usepackage{subcaption}
\captionsetup{compatibility=false}

\algtext*{EndIf}% Remove "end if" text
\algtext*{EndFor}% Remove "end for" text
\algtext*{EndProcedure}% Remove "end for" text
\algrenewcommand\algorithmicthen{}

%\newcommand{\lef}[1]{{{\color{blue}\textbf{Lef: }#1}\normalcolor}}
%\newcommand{\ray}[1]{{{\color{orange}#1}\normalcolor}}

\newcommand{\basesys}{Chiron\xspace}
\newcommand{\sys}{\textsc{Anthemius}\xspace}
\newcommand{\sysmax}{240\%\xspace}

%-------------------------------------------------------------------------------
\begin{document}
%-------------------------------------------------------------------------------

%don't want date printed

% make title bold and 14 pt font (Latex default is non-bold, 16 pt)
\title{\sys: Efficient \& Modular Block Assembly for Concurrent Execution}

\author{Ray Neiheiser\inst{1} \and
Eleftherios Kokoris-Kogias\inst{2}}
%Third Author\inst{3}\orcidID{2222--3333-4444-5555}}
%
\authorrunning{R. Neiheiser \and E. Kokoris-Kogias}
% First names are abbreviated in the running head.
% If there are more than two authors, 'et al.' is used.
%
\institute{ISTA, Klosterneuburg, Austria \and
Mysten Labs, Athens, Greece}
%\email{lncs@springer.com}\\
%\url{http://www.springer.com/gp/computer-science/lncs} \and
%ABC Institute, Rupert-Karls-University Heidelberg, Heidelberg, Germany\\
%\email{\{abc,lncs\}@uni-heidelberg.de}}



\maketitle

%-------------------------------------------------------------------------------
\begin{abstract}
%-------------------------------------------------------------------------------

Many blockchains such as Ethereum execute all incoming transactions sequentially significantly limiting the potential throughput.
A common approach to scale execution is parallel execution engines that fully utilize modern multi-core architectures.
Parallel execution is then either done optimistically, by executing transactions in parallel and detecting conflicts on the fly, or guided, by requiring exhaustive client transaction hints and scheduling transactions accordingly.

However, recent studies have shown that the performance of parallel execution engines depends on the nature of the underlying workload.
In fact, in some cases, only a 60\% speed-up compared to sequential execution could be obtained. This is the case, as transactions that access the same resources must be executed sequentially. For example, if 10\% of the transactions in a block access the same resource, the execution cannot meaningfully scale beyond 10 cores. Therefore, a single popular application can bottleneck the execution and limit the potential throughput.

In this paper, we introduce \sys, a block construction algorithm that optimizes parallel transaction execution throughput. We evaluate \sys exhaustively under a range of workloads, and show that \sys enables the underlying parallel execution engine to process over twice as many transactions.
\keywords{Blockchain \and Parallel Execution \and Smart Contracts \and Distributed Ledger Technology.}

\end{abstract}


\section{Introduction}\label{sec:introduction}
% -- Outline
% ---- LLMs are popular
% ---- There're many stakeholders in the training and inference loop
% ---- Adversaries in the training loop are a problem -- malpractice, poisoning
% ---- Also, showing compliance
% ---- Need a framework to prove the integrity of the pipeline
% ---- Enter Atlas

% ---- LLMs are popular
In recent years, machine learning (ML) models, have become increasingly popular.
The pervasive use of large language models (LLMs), in particular, and multi-stakeholder
involvement in model creation and deployment exacerbate security and privacy risks.
These considerations are emphasized by the global nature and the complexity of
large-scale ML deployments with different lifecycle stages:
%gathering and sanitizing the data from different sources,
%training and inferencing across many data centers,
%compliance with local laws or corporate policies.

% ---- There're many stakeholders in the training and inference loop
%Additionally, different stages of the ML development pipeline come with their own stakeholders:
\begin{enumerate}[label=\arabic*)]
    \item Collection and sanitation of a \emph{training} dataset from several public and proprietary sources.
    %\item Solicitation and facilitation of training.
    \item Provisioning of the training environment (hardware and software).
    \item Execution of training across many data centers.
    \item Construction of a \emph{testing} dataset from several sources, and the evaluation.
    \item Deployment and use of the model for inference that is compliant with local laws or corporate policies.
    %\item Use of the model in compliance with local laws or corporate policies.
\end{enumerate}

% ---- Adversaries in the training loop are a problem -- malpractice, poisoning
Each of these stages is vulnerable to malicious or dishonest parties.
For example, data can be poisoned~\cite{biggio2012poisoning,carlini2024poisoning} during collection or training.
Service providers executing outsourced training can shorten or omit critical steps to reduce their cost.
Model providers can serve smaller models in SaaS, or even distribute malicious ones.

% ---- Also, showing compliance
On the other hand, responsible model builders and other stakeholders may be incentivised or required to provide security and trust guarantees.
They may want to prove low bias in their training data, offer easily verifiable performance claims, or guarantee end-to-end integrity of the model creation in high risk domains.

% ---- Need a framework to prove the integrity of the pipeline
To address these challenges, it is necessary to guarantee the integrity of the entire ML lifecycle --
beginning with the data, through the training, and finally, the evaluation and deployment.
Was the data modified?
Did the hardware and software environment adhere to the specification?
Did the contractor follow the specified training procedure?
Can I trust the evaluation?
How can I guarantee that I am interacting with the intended model?
These are example questions that showcase the breadth of the involved challenges that must be tackled to provide end-to-end security.

% --- Enter Atlas
In this work, we introduce \atlas, a framework for enhancing the security and transparency of the lifecycle of ML models.
\atlas establishes the baseline of fundamental components and capabilities needed for comprehensive provenance tracking
at each stage of the ML lifecycle.
Subsequently, \atlas defines the core integrity requirements for verifiable ML lifecycle transparency.
We provide a reference implementation that instantiates \atlas using hardware-based security mechanisms -- with trusted execution environment (TEE),
including attestations.% , and comprehensive metadata-based provenance tracking.
%Our implementation satisfies all \atlas requirements.

We claim the following contributions:
\begin{enumerate}[label=\arabic*.]\label{sec:introduction:contributions}
    \item We introduce \atlas, a framework designed for end-to-end ML lifecycle transparency.
    \item We instantiate \atlas using TEEs and metadata-based provenance tracking.
    \item We evaluate our \atlas prototype through two case studies:
        \begin{enumerate*}[label=\arabic*)]
            \item fine-tuning of a BERT model~\cite{lin2023metabert, lin2023metabertimpl};
            \item fine-tuning of a bge-reranker model~\cite{chen2023bge}
        \end{enumerate*}.
\end{enumerate}

%\msm{revise: Integrate this motivation into intro}
%Organizations frequently leverage pre-trained models, outsource training processes, and integrate components from multiple sources,
%making it difficult to verify the authenticity and trustworthiness of their ML systems. This complexity is further compounded
%by the potential for malicious modifications at various stages of the model lifecycle, from data preparation through deployment.
%The involvement of various third parties in ML model development and deployment
%creates critical challenges in ensuring supply chain integrity.
%
%While Software Bills of Materials (SBOMs) and AI Bills of Materials (AI BOMs) provide basic inventory tracking for model components,
%they fall short in addressing the dynamic nature of ML pipelines. These approaches typically offer point-in-time snapshots but
%fail to capture the complex transformations, fine-tuning operations, and runtime modifications that characterize modern ML workflows.
%Additionally, they lack cryptographic guarantees about the integrity of recorded information and cannot effectively track the provenance
% of model weights and training data.
%
% These approaches demonstrate the growing importance of ML supply chain security.
% However, they are typically applied in an ad-hoc fashion, highlighting the need
% for a more integrated approach that combines comprehensive lineage tracking,
% strong cryptographic properties, and practical integration capabilities with existing ML development and deployment pipelines.
%
%A comprehensive solution requires not just documentation of components, but verifiable evidence of their origins,
%transformations, and integrity throughout the entire model lifecycle. This need has driven interest in more robust
%provenance tracking mechanisms that can:
%
%\begin{itemize}
%\item Provide cryptographic proof of model lineage
%\item Track and verify all pipeline transformations
%\item Maintain tamper-evident records of training processes
%\item Ensure integrity of model artifacts across organizational boundaries
%\end{itemize}
%
%Several existing tools and frameworks
%commonly focusing on different components of the model lifecycle and provenance tracking.
%While these solutions offer valuable capabilities, they often address only specific parts of the end-to-end ML
%supply chain rather than providing comprehensive coverage.
%\msm{end-revise}
%
%\todo{add discussion of EU-CRA AI Act requirements for model documentation and FDA guidelines for AI/ML in healthcare}

%The remainder of this paper is organized as follows:
%in Section~\ref{sec:background-related} we provide an overview of the necessary background, and the related work;
%Section~\ref{sec:problem} presents the challenge of providing integrity in the ML pipeline, the threat model, and the system assumptions;
%in Section~\ref{sec:framework} we present \atlas -- our framework for providing ML integrity;
%Section~\ref{sec:implementation} covers implementation details;
%in Section~\ref{sec:eval}, we show that \atlas is effective across three dimensions: training overhead $<8\%$, the verification time increases linearly with the size of the model, and it is compatible with PyTorch and Tensorflow;
%in Section~\ref{sec:casestudies} we present the case studies;
%in Section~\ref{sec:discussion} we discuss additional considerations for \atlas,
%and Section~\ref{sec:conclusion} concludes the paper and provides directions for future work.


\section{System Model}
\label{sec:systemmodel}

We assume a blockchain environment consisting of $N$ server processes $p_1, p_2,..,p_N$ and $I$ client processes $c_1, c_2,..,c_I$. Clients send signed transactions to the server processes to be included in a future block.
The blockchain functions as the Public Key Infrastructure where the identifier of a client is its public key, and clients use their private keys to sign their transactions.
%Client transactions may include sequence numbers that enforce a strict order on the inclusion of the transactions in a given block.

We assume a consensus abstraction as a blackbox, where one or more processes construct blocks of transactions and propose them to the consensus mechanism. As a result, the consensus abstraction outputs an ordered sequence of blocks $b_1, b_2, \dots, b_n$, which is then processed by the execution engine. Additionally, we assume an execution engine abstraction as a blackbox that receives this ordered sequence of blocks from the consensus abstraction and executes them deterministically.

In the context of this work, we make no assumptions regarding the coupling between the consensus and execution layers. The interaction between consensus and execution may either follow a modular, decoupled approach, as in Sui and Aptos~\cite{sui,aptos}, or operate in a tightly coupled, sequential manner, as in Ethereum~\cite{ethereum}.

Client transactions might range from simple peer-to-peer transactions to complex application logic with the help of smart contracts. As applications might access arbitrary resources (i.e., addresses) that can not easily be deduced, we assume the existence of a system that provides hints about the resources a transaction will access during execution to the block producer. This can either be in the form of client hints as in Solana or Sui~\cite{solana,sui}, or in the form of an optimistic pre-execution step that determines these hints locally as in Polygon~\cite{polygonupdate}. However, we do not assume the list of hints to be exhaustive or correct. Transactions with incomplete or incorrect hints might trigger re-executions if the execution engine is Block-STM or a derivative~\cite{blockstm,chiron}, or aborted in Solana or Sui~\cite{solana,sui}.


\section{Anthemius}
\label{sec:overview}

The primary objective of \sys is to redesign the block-assembly approach in blockchains that offer parallel execution to improve the overall system throughput and prevent popular applications from creating bottlenecks by factoring in transaction dependencies and execution time.

At the time of writing, most blockchains that support parallel transaction execution use a single parameter such as the computational complexity in gas, the raw block size in bytes, or the number of transactions to limit the block size~\cite{aptos,sui,solana}. However, in the context of parallel transaction execution, a single parameter does not reflect the execution complexity of a block. If all transactions in the block access the same resource, the execution time is the sum of the runtime of all transactions. In contrast, if none of the transactions access conflicting resources, the runtime depends on the number of cores.

%For example, if the maximum block size is chosen in relation to the number of transactions or raw transaction size, a block that contains too many computationally heavy transactions could result in an unreasonably long execution time and slow down the system significantly.

%However, user estimates on execution time cannot be trusted because they tend to purposefully overestimate the execution cost to guarantee that their transaction will be executed and, as such, taking this as a basis would result in a strong performance degradation. In a preliminary analysis we identified that clients on the Aptos network estimate an on average 25 times higher gas cost, compared to Ethereum or Polygon with around 2 times the gas cost, which would cut the potential throughput by the same factor.
%While it is theoretically possible to create incentives for clients to estimate the gas cost more accurately, this degrades the user experience significantly as this could increase abort rates.

%However, even given accurate transaction run-time predictions, there is another problem. While in Ethereum transactions are executed sequentially and, as such, the sum of the accurate run-time predictions of the transactions also perfectly mirrors the actual execution time of the block, the moment we introduce parallel transaction execution, the total run-time heavily depends on how transactions are scheduled and their interdependencies. If all transactions in a given block depend on each other, the best-case total execution time is equal to serial execution. However, in the worst case, if executors are unaware of transaction dependencies, concurrent accesses have to be detected and transactions have to be re-executed to avoid diverging states, resulting in worse than serial execution time~\cite{blockstm}.

%As such, a single heavily sequential application can bottleneck the entire system. This can happen due to bad engineering practices when designing the smart contract, but may also be a result of malicious design.

Therefore, as a first step to begin constructing ``Good Blocks'', we need parameters that allow us to quantify this. We deploy two parameters to address this. First a transaction complexity parameter in Gas, similar to Ethereum, and second a concurrency parameter $c$ describing the system's ability to execute transactions in parallel (i.e. number of cores). As a result, the total maximum capacity of each block is $c * Gas$.

In the next sections, we first discuss where \sys fits into existing blockchain architectures. Following that, we outline the design of the block construction algorithm that considers both parameters and constructs blocks sensitive to transaction dependencies and their execution time to speed up the parallel execution of the block.

\subsection{Architecture}

\begin{figure}
\begin{center}
	\includegraphics[width=1\columnwidth]{figures/architecture.pdf}
\end{center}
\caption{\sys is inserted between the Mempool and Consensus}
\label{fig:architecture}
\end{figure}

Figure~\ref{fig:architecture} shows where \sys fits into the existing protocol stack of a blockchain. \sys is a modular layer that can be inserted between the consensus layer and the mempool where client transactions are stored and handled. 
In \sys, instead of fetching transactions directly from the mempool, the consensus layer fetches blocks of transactions through \sys. In turn, \sys obtains its transactions from the mempool, divides transactions into batches, and constructs the block to return to consensus.
Following that, the block is proposed in consensus which outputs an ordered list of blocks to the execution engine.

\sys requires the read and write sets of transactions, as well as an estimation of their execution time, to assess dependencies between transactions and construct blocks that can be executed efficiently in parallel. This information is already available in blockchains such as Solana~\cite{solana} and Sui~\cite{sui}, where transactions must declare all resource addresses they access during execution. In other blockchains, such as Ethereum~\cite{ethereum}, this information can be obtained, for example, by executing the transactions.

This design allows \sys to be seamlessly integrated into any existing blockchain stack with minimal architectural and system changes, and without changing the block structure. Furthermore, since \sys operates solely on the set of transactions, their read and write sets, and their gas footprints, it remains essentially stateless. This makes \sys particularly suitable for deployment in modular architectures, such as Narwhal, where only the execution layer is stateful~\cite{narwahl}.  

\subsection{Block Construction}

An important problem that has to be tackled when constructing good blocks is the absence of information regarding the structure of the current workload. If all transactions in the mempool access the same resources, attempting to schedule them efficiently can further slow down an already bottlenecked system. Similarly, if the algorithm is too strict in situations where a large percentage of transactions access the same resources, the synergetic effects of executing larger batches of transactions are lost. This is the case, as, for each block, the system has to instantiate the executor and worker threads, set up the virtual machine, extract the execution results, etc.

Therefore, as a first step, we divide \sys into two modular elements. First, the \textit{batch handler}, which polls batches of transactions from the mempool and hands the batches to the \textit{batch scheduler} in a batch-by-batch fashion. Second, the \textit{batch scheduler}, that attempts to include a given batch into the current block and provides feedback to the \textit{batch handler} about the success rate.
Subsequently, based on the feedback, the batch handler can adjust the inclusion policy to prevent too small blocks and also avoid wasting scheduling time on difficult-to-schedule workloads.

\paragraph{Batch handler.}

\begin{algorithm}[t]

\centering
\caption{\textsl{Batch Handler}}
\label{algo:batchhandler}
\begin{algorithmic}[1]

\Procedure{CreateGoodBlock}{$block, maxgas, c$}
\State{$seqlimit = \frac{maxgas}{c}$} \Comment{Limit on the sequential path}
\State{$resmap \gets \emptyset$} \Comment{Map to track transaction dependencies}
\State{$skippedclients \gets \emptyset$} \Comment{Set to track clients with skipped transactions}
\State{$numrelax \gets 0$} \Comment{Number of times inclusion rate was relaxed}
\For{\textbf{all} $\textit{batch} \in mempool$}
\State{$incrate \gets$ \textsc{schedule($block, batch, seqlimit, c, resmap, skippedclients$)}} \label{algo:batchhandler:schedule}
\If{$incrate < \textsc{targetincrate} $}
\If{$numrelax \geq \textsc{maxrelaxnum} \lor (incrate = 0 \land batch.isfull)$} \label{algo:batchhandler:check}
\State{\textbf{return}}
\EndIf
\State{$seqlimit = \frac{maxgas}{c} * \textsc{min}(\textsc{maxrelaxrate}, incrate*\textsc{targetincrate})$} \label{algo:batchhandler:relax}
\State{$numrelax++$}
\EndIf
\EndFor

\EndProcedure
\end{algorithmic}
\end{algorithm}

The functionality of the Batch Handler is outlined in Algorithm~\ref{algo:batchhandler}. The batch handler receives a $block$ to fill, the global concurrency parameter $c$, and the maximum gas limit. It then calculates a limit on the sequential path $seqlimit$ and initiates a map to track the transaction dependencies $resmap$ as well as a set of clients with skipped transactions $skippedclients$. 

Next, the batch handler retrieves transaction batches from the mempool and hands them to the batch scheduler alongside the block, the limit on the gas, the number of cores, the transaction resource dependencies $resmap$, and $skippedclients$ set in Line~\ref{algo:batchhandler:schedule}.
The batch scheduler responds with the transaction inclusion rate $incrate$.

Depending on the workload, as mentioned, the $seqlimit$ may be very strict which can result in very few transactions being included in a block.
Therefore, if the inclusion rate $incrate$ is smaller than some $\textsc{targetincrate}$, we relax the gas limit relative to the inclusion rate, up to some \textsc{maxrelaxrate} (Line~\ref{algo:batchhandler:relax}).

However, if the inclusion rate was too small for several consecutive attempts (i.e. $numrelax \geq \textsc{maxrelaxnum}$), we exit scheduling to avoid building a heavily sequential block again.
Furthermore, if there was an attempt to schedule a full batch and no transaction of this batch was successfully included in the current block ($incrate = 0$) we also stop scheduling (Line~\ref{algo:batchhandler:check}) as this indicates that at this point transactions are only included at a high cost to execution performance and scheduling latency. The rest of the transactions are then only included in a later block.%(i.e. marginal scheduling time increases with each added transaction significantly).

%\lef{what happenst to these transactions? I am guessing the skipped eventually forces them to be included?}

\paragraph{Batch Scheduler.}

Scheduling transactions with interdependencies and varying runtimes is a known NP-complete problem~\cite{BAKER1996225} where approximate solutions can construct near optimal schedules in polynomial time. However, polynomial runtime, particularly when executed on the critical path of consensus, may lead to a construction time that outweighs the performance gains achieved from producing ''Good Blocks.''

Fortunately, our first insight is that a near-optimal schedule for block construction is unnecessary. Instead, our main objective is to prevent popular resources and applications from creating a bottleneck while maximizing the parallel execution. We can achieve this by iterating over the set of resources each transaction accesses, recording the cost of the sequential path leading up to the transaction, and deciding if the transaction should be included in the current block by comparing the cost of the path with the gas per core parameter.
Furthermore, we also want to delay transactions that access multiple hot resources as they make it harder to schedule subsequent transactions.

As a result, the complexity of the block construction is of $O(N*k)$ where $N$ is the number of transactions and $k$ is the average number of resource accesses per transaction.

\begin{algorithm}[t]

\centering

\caption{\textsl{Batch Scheduler - Called in Line~\ref{algo:batchhandler:schedule} of Algorithm~\ref{algo:batchhandler}}}
\label{algo:batchscheduler}
\begin{algorithmic}[1]

\Procedure{schedule}{$block, batch, seqlimit, c, resmap, skippedclients$}
\For{\textbf{all} $\textit{tx} \in batch$} \Comment{Iterate over transactions} \label{algo:batchscheduler:iterate}
     \If{$tx.sender$ \textbf{in} $skippedclients$}
            \State{\textbf{continue}} \Comment{Skip transaction inclusion} \label{algo:batchscheduler:skipuser}
    \EndIf
    \State{$chaincost \gets 0$} \Comment{Longest chain length}
    \State{$hotresources \gets 0$}
    \For{\textbf{all} $\textit{readres} \in tx.readset$} \Comment{Iterate over readset} \label{algo:batchscheduler:readloop}
        \If{$readres \in resmap$} \Comment{Find longest chain}
            \If{$resmap[readres] > chaincost$} \Comment{Find read with largest cost}
                \State{$chaincost \gets resmap[readres]$}
            \EndIf
            \If{$resmap[readres] > \frac{block.gas}{c}$} \Comment{Check if read exceeds limit}
                \State{$hotresources++$} \label{algo:batchscheduler:hotread}
            \EndIf
        \EndIf
    \EndFor
    \If{$hotresources \geq \textsc{maxhotr} \land (|block| > \textsc{lim} \lor |block| < \textsc{maxlen}-\textsc{lim})$}  \label{algo:batchscheduler:hotreadcheck}
        \State{$skippedclients \gets skippedclients \cup tx.sender$}
        \State{\textbf{continue}} \Comment{Skip transaction inclusion}
    \EndIf
    
    \If{$chaincost + tx.gas > seqlimit \lor block.gas + tx.gas > seqlimit*c$}  \label{algo:batchscheduler:gascheck}
            \State{$skippedclients \gets skippedclients \cup tx.sender$}
            \State{\textbf{continue}} \Comment{Skip transaction inclusion}
    \EndIf
    \State{$block \gets block \cup tx$} \Comment{Add tx to Block}
    \For{\textbf{all} $\textit{writeres} \in tx.writeset$} \Comment{Iterate over writeset}
        \If{$writeres \notin resmap \lor resmap[writeres] < chaincost$}
         \State{$resmap[writeres] \gets chaincost$} \Comment{Note new chain length}
        \EndIf
    \EndFor
\EndFor
\State{\textbf{return}$(\frac{numscheduled}{|batch|})$}
\EndProcedure
\end{algorithmic}
\end{algorithm}


Algorithm~\ref{algo:batchscheduler} shows how we achieve this. The algorithm starts with the call of the \textsc{schedule} method, which receives the block to include the transactions in, the batch of transactions to schedule, the maximum gas per core $seqlimit$, the concurrency parameter $c$, the map of resources and the skipped clients. Following that, it starts iterating over all transactions in the batch (line~\ref{algo:batchscheduler:iterate}).
First, to maintain the order clients specified (e.g. through sequence numbers), after a client had a transaction skipped, the client is added to the $skippedclients$ set and no further transaction from this client will be included in this block.(Line~\ref{algo:batchscheduler:skipuser}).
Following that, we iterate over all reads in the transaction read-set and attempt to calculate the read with the longest path in gas leading up to this transaction (Line~\ref{algo:batchscheduler:readloop}).
In parallel, we count the number of \textit{hot} reads. A hot read is a read on a resource that is accessed significantly more often than other resources. 
%We approximate this by checking if it has a longer path than the total gas in the block so far, divided by the number of cores.

After this, we check whether the number of hot reads exceeds a predefined threshold, \textsc{maxhotr}. If this condition is met and the transaction is not within the first or last \textsc{lim}(i.e. 10\%) transactions, we skip the transaction (Line~\ref{algo:batchscheduler:hotreadcheck}).
We delay transactions with too many hot reads as they unify several critical paths of transactions which can severely bottleneck the execution.
However, we initially allow any transactions to be included up to some threshold \textsc{lim} to accumulate sufficient data to assess the complexity of reads and to guarantee that transactions that access several hot resources are eventually included. Furthermore, we also allow including transactions with multiple hot reads towards the end of the block as the block is almost full already and they are less likely to cause scheduling problems at this point.

Following that, we check if the transaction cost itself is larger than the max gas per core $seqlimit$ or if the current transaction exceeds the total gas limit of the block. If so, we also skip the transaction (Line~\ref{algo:batchscheduler:gascheck}).

Finally, we include the transaction in the block, iterate over its write set, and record the transaction path cost in the resource map $resmap$ if its writes increase the critical path.
This results in an algorithm that is linear to the number of transactions per block, as the map accesses are $O(1)$ and we check each transaction at most once per block.





\section{Evaluation Metrics}\label{sec:evaluation}

\subsection{Game Creation Evaluation}\label{sec:evaluation_gc}

In GC, we evaluate an LLM's capability to create games that have good mechanics. This task requires complex reasoning over event-state interactions that is very challenging even for human. Section~\ref{sec:gdc} offered a broad overview of the GC task. We now define it more precisely.

\noindent\textbf{Task Definition [Game Creation]}  
Given a fictional character \(\mathcal{C}\) and related Wikipedia information \(\mathcal{R}\), an LLM must create a game \(\mathcal{G}\) that follows a predefined format \(\mathcal{J}\).

In \benchmark{}, 100 fictional characters are used, each with an associated Wikipedia page (\(\mathcal{R}\)), facilitating future expansion of the character pool. The game \(\mathcal{G}\) must conform to the structure \(\mathcal{J}\) given in Section~\ref{sec:gd}. We provide each LLM with a 5-shot prompt to generate one game per character. 

\begin{algorithm}[!ht]
   \caption{BFS Validity Checker}
   \label{alg:bfs-checker}
\begin{algorithmic}
   \STATE \textbf{Input:} Events \(\mathcal{E}\), each with entering and success conditions, plus success and fail effects; A state \(S_0\) with initial values for all variables; An integer \(M\) indicating the maximum number of states to be explored.

   \FUNCTION{\(\textit{isValid}(\mathcal{E},S_0,M)\)}
   \STATE Initialize a queue \(\mathcal{Q}\) and enqueue \(S_0\).
   \STATE Initialize a visited set \(\mathcal{V} = \{S_0\}\).
   \STATE Initialize a triggered-event set \(\mathcal{T} = \varnothing\).
   \STATE \(\textit{successFound} = \texttt{False};~ \textit{loseFound} = \texttt{false}\)

   \REPEAT
      \STATE \(S = \mathcal{Q}.\text{dequeue}()\)
      \IF{\(|\mathcal{V}| > M\)}
         \STATE \textbf{break} \quad\# \textit{Reached maximum search limit}
      \ENDIF
      
      \STATE \(\textit{availableEvents} = \{ e \in \mathcal{E} : e.\mathit{enterCond}(S) \}\)
      \FOR{each \(e \in \textit{availableEvents}\)}
         \STATE \(\mathcal{T} = \mathcal{T} \cup \{e\}\) \quad\# \textit{Mark event as triggered}

         \STATE \(S' = \textit{e.applyEffect}(S,\mathit{e.successCond}(S))\)
         
        \STATE \(\textit{successFound} \;|= \textit{e.isSuccessTermination}(S')\)
        \STATE \(\textit{loseFound} \;|= \textit{e.isLosingTermination}(S')\)

         \IF{\(S' \notin \mathcal{V}\)}
            \STATE \(\mathcal{Q}.\text{enqueue}(S')\);~\(\mathcal{V} = \mathcal{V} \cup \{S'\}\)
         \ENDIF
      \ENDFOR
   \UNTIL{\(\mathcal{Q}\) is empty}

   \STATE \textbf{return} 
      \(\bigl(\mathcal{T} = \mathcal{E}\bigr) \;|\; \textit{successFound} \;|\; \textit{loseFound}\)
   \ENDFUNCTION
\end{algorithmic}
\end{algorithm}

\paragraph{BFS Validity Checker}  
Once the output is confirmed to be valid JSON, we perform a BFS-based validity check (Algorithm~\ref{alg:bfs-checker}). Based on our event--state design, we employ BFS to decide if a game is valid. Starting from the initial state, we repeatedly check which events are available , apply success or failure effects accordingly, and track whether at least one success and one losing state can be reached. We stop when no new states can be discovered or when the search exceeds 10{,}000{,}000 states. A game is valid if every event is triggered at least once, and both success and losing termination conditions are achievable.

\paragraph{Metrics}  
For GC evaluation, we report the format-check pass rate (\textbf{FCR}) and the valid-check pass rate (\textbf{VCR}) as our main metrics, reflecting how reliably LLMs follow the prescribed JSON format and produce valid game mechanics. In order to examine fine-grained failures for the validity check, we include three additional ratios:
\begin{equation*}
\begin{aligned}
     \textbf{w. Success} &=\frac{\textit{\# games with successFound}}{\textit{\# games pass the format check}}\\
     \textbf{w. Lose}&=\frac{\textit{\# games with failFound}}{\textit{\# games pass the format check}}\\
     \textbf{Reachability}&=\frac{\textit{\# games without unreachable events}}{\textit{\# games pass the format check}}
\end{aligned}
\end{equation*}

\subsection{Game Simulation Evaluation}

Given a valid game, the GS task requires an LLM to simulate the game for a player. We introduce a multi-round simulation framework, based on which a comprehensive description of evaluation metrics is presented.

\paragraph{Game Simulation Framework}  
The simulation proceeds in multiple rounds of interaction with a (real or simulated) player. Before the first round, the LLM is given the complete game information and output instructions. Each round thereafter, the LLM outputs:
\begin{enumerate}
    \item \textbf{Event Plan}: A list of events occurring this round. Each entry specifies whether the event is starting (\texttt{start}) or ending (\texttt{end}); if ending, an \texttt{outcome} is either \texttt{success} or \texttt{failure}.
    \item \textbf{Game Narration}: A narrative description of the current round, concluding with three candidate actions for the player character. We prompt models to follow a play-script format for readability but do not enforce it during evaluation.
    \item \textbf{Game State}: The updated state variables after applying effects of any events that ended this round.
\end{enumerate}

\paragraph{Evaluation Metrics}  
Our evaluation covers multiple dimensions, scored over the trajectory of interactions. A simulated player selects one of the candidate actions at random each round.

\begin{enumerate}
    \item \textbf{Length}: We count words in the game narration (excluding candidate actions). Although no ideal length is defined, our prompt suggests remaining under 200 words to maintain brevity without sacrificing creativity. We report the average length per round.
    \item \textbf{Action Quality}: Using an LLM judge (prompt in Appendix~\ref{app:eval_prompt}), we rate the three candidate actions based on diversity, relevance, and clarity. The judge outputs a 1--5 score, normalized to \([0,1]\) via \(\frac{s-1}{4}\). We average scores across all rounds.
    \item \textbf{Interestingness}: An LLM judge evaluates how engaging the round’s narration is, assigning a 1--5 score also normalized to \([0,1]\). We average this score across the entire trajectory.
    \item \textbf{Role-Playing Factual Consistency}: We compare the game narration against each fact in the main NPC's fact list. An LLM judge labels each fact as \textit{align}, \textit{contradict}, or \textit{neutral}. We report the ratio \(\frac{\#\textit{align}}{\#\textit{align} + \#\textit{contradict}}\).
    \item \textbf{Role-Playing Personality Consistency}: We prompt an LLM to infer the main NPC's Big Five traits from the generated content, then compare these to the game definition. We employ the Ten-Item Personality Inventory (TIPI)~\cite{gosling2003very}, following previous work on eliciting LLM-based personality assessments of public figures~\cite{cao2024large}. In addition to TIPI, we also considered a direct approach that explicitly evaluates alignment between the game’s narrative and the NPC’s predefined traits. We use TIPI-based score in the main paper, with details on the direct approach and comparisons in the Appendix~\ref{app:eval_prompt} and ~\ref{app:human_eval}.
    \item \textbf{Game Mechanics}: We perform a fully automatic check for the following errors:
    \begin{enumerate}
        \item \textbf{Event Condition Error}: An event triggers when its entering condition is not met, or the outcome (success/failure) does not match the current state.
        \item \textbf{Variable Update Error}: The state variables do not update according to event effects.
    \end{enumerate}
    The main game mechanic metric we adopt is the round-level accurate rate $\textbf{MEC} = \frac{\# \textit{Rounds with no errors}}{\# \textit{Rounds}}.$
    % \begin{equation}
    %     \textbf{MechanicScore} = \frac{\# \textit{Rounds with no errors}}{\# \textit{Rounds}}.
    % \end{equation}
    We average the mechanic score over all games.
    
    For a more fine-grained analysis, we process events in the \textit{Event Plan} sequentially at each round and calculate an error rate for each error type,
    \begin{equation}
    \begin{aligned}
        \textbf{ECE}_t &= \frac{\# \textit{Event condition errors}}{\# \textit{Events}}\\
        \textbf{VUE}_t &= \frac{\# \textit{State variables incorrectly updated}}{\# \textit{State variables}}
    \end{aligned}
    \end{equation}    
    We average ECE and VUE over all rounds of all games. By design, all these metrics require no LLM judge. 
\end{enumerate}



\input{6motivationandrelatedwork}

Reward shaping is not applicable to DPO~\cite{DPO}, as it does not require a reward model during training. We also explore online DPO, which employs the policy model to generate two responses, and the reward model selects the response with the higher reward as the chosen response and the lower reward as the rejected response. However, since most reward shaping techniques are monotonic, they do not alter the binary preference and therefore, they do not influence the training procedure of online DPO.

For GRPO~\cite{shao2024deepseekmathpushinglimitsmathematical}, we argue that its advantage calculation inherently normalizes the proxy reward, making linear transformations (e.g., Minmax and mean\_std) ineffective. However, our non-linear PAR demonstrates slightly better performance than Vanilla GRPO in later stages (see Appendix~\ref{section:rsnotavail}).
An important observation is that GRPO does not exhibit the reward hacking problem during training, primarily because its advantage calculation effectively normalizes the rewards. Although the win rate decreases in the later stages, the proxy rewards also decrease proportionally, maintaining alignment between the optimization objective and the desired outcomes.


\section{Conclusion}
\label{sec:conclusion}

In this work, we presented \sys, a framework, and algorithm to construct highly parallelizable blocks in the context of parallel smart contract execution.
We evaluated \sys extensively under a series of realistic workloads, demonstrating a throughput improvement of up to \sysmax.   Furthermore, in most workloads, this approach leads to lower latency for the majority of transactions, while only delaying those that access hot resources and cause bottlenecks. %todo mention discussion point again.
Moreover, \sys not only improves the throughput of the underlying execution engine but also protects blockchains from being bottlenecked by popular applications.
Finally, \sys is highly modular and can be easily integrated into any production blockchain without any security tradeoffs.

%-------------------------------------------------------------------------------
%\section*{Availability}
%-------------------------------------------------------------------------------

%The code of \sys as well as the real-world workloads and the data on which they are based will be made public upon acceptance of the paper.

%-------------------------------------------------------------------------------
\bibliographystyle{splncs04}
\bibliography{main}

%%%%%%%%%%%%%%%%%%%%%%%%%%%%%%%%%%%%%%%%%%%%%%%%%%%%%%%%%%%%%%%%%%%%%%%%%%%%%%%%
\end{document}
%%%%%%%%%%%%%%%%%%%%%%%%%%%%%%%%%%%%%%%%%%%%%%%%%%%%%%%%%%%%%%%%%%%%%%%%%%%%%%%%

%%  LocalWords:  endnotes includegraphics fread ptr nobj noindent
%%  LocalWords:  pdflatex acks