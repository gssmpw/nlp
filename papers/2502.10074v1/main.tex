
\documentclass[runningheads]{llncs}
\usepackage[T1]{fontenc}
\usepackage{xcolor}
%\usepackage{tikz}
\usepackage{amsmath}
%\usepackage[compact]{titlesec}
\usepackage{graphicx}
%\usepackage{textcomp}
\usepackage{xspace}
%\usepackage{hyperref}
%\usepackage{xurl}
%\usepackage{subcaption}
\usepackage{algorithm, algpseudocode}
%\usepackage[normalem]{ulem}
\usepackage{pifont}
\newcommand{\cmark}{\ding{51}}%
\newcommand{\xmark}{\ding{55}}%
\usepackage{subcaption}
\captionsetup{compatibility=false}

\algtext*{EndIf}% Remove "end if" text
\algtext*{EndFor}% Remove "end for" text
\algtext*{EndProcedure}% Remove "end for" text
\algrenewcommand\algorithmicthen{}

%\newcommand{\lef}[1]{{{\color{blue}\textbf{Lef: }#1}\normalcolor}}
%\newcommand{\ray}[1]{{{\color{orange}#1}\normalcolor}}

\newcommand{\basesys}{Chiron\xspace}
\newcommand{\sys}{\textsc{Anthemius}\xspace}
\newcommand{\sysmax}{240\%\xspace}

%-------------------------------------------------------------------------------
\begin{document}
%-------------------------------------------------------------------------------

%don't want date printed

% make title bold and 14 pt font (Latex default is non-bold, 16 pt)
\title{\sys: Efficient \& Modular Block Assembly for Concurrent Execution}

\author{Ray Neiheiser\inst{1} \and
Eleftherios Kokoris-Kogias\inst{2}}
%Third Author\inst{3}\orcidID{2222--3333-4444-5555}}
%
\authorrunning{R. Neiheiser \and E. Kokoris-Kogias}
% First names are abbreviated in the running head.
% If there are more than two authors, 'et al.' is used.
%
\institute{ISTA, Klosterneuburg, Austria \and
Mysten Labs, Athens, Greece}
%\email{lncs@springer.com}\\
%\url{http://www.springer.com/gp/computer-science/lncs} \and
%ABC Institute, Rupert-Karls-University Heidelberg, Heidelberg, Germany\\
%\email{\{abc,lncs\}@uni-heidelberg.de}}



\maketitle

%-------------------------------------------------------------------------------
\begin{abstract}
%-------------------------------------------------------------------------------

Many blockchains such as Ethereum execute all incoming transactions sequentially significantly limiting the potential throughput.
A common approach to scale execution is parallel execution engines that fully utilize modern multi-core architectures.
Parallel execution is then either done optimistically, by executing transactions in parallel and detecting conflicts on the fly, or guided, by requiring exhaustive client transaction hints and scheduling transactions accordingly.

However, recent studies have shown that the performance of parallel execution engines depends on the nature of the underlying workload.
In fact, in some cases, only a 60\% speed-up compared to sequential execution could be obtained. This is the case, as transactions that access the same resources must be executed sequentially. For example, if 10\% of the transactions in a block access the same resource, the execution cannot meaningfully scale beyond 10 cores. Therefore, a single popular application can bottleneck the execution and limit the potential throughput.

In this paper, we introduce \sys, a block construction algorithm that optimizes parallel transaction execution throughput. We evaluate \sys exhaustively under a range of workloads, and show that \sys enables the underlying parallel execution engine to process over twice as many transactions.
\keywords{Blockchain \and Parallel Execution \and Smart Contracts \and Distributed Ledger Technology.}

\end{abstract}


\section{Introduction}

In today’s rapidly evolving digital landscape, the transformative power of web technologies has redefined not only how services are delivered but also how complex tasks are approached. Web-based systems have become increasingly prevalent in risk control across various domains. This widespread adoption is due their accessibility, scalability, and ability to remotely connect various types of users. For example, these systems are used for process safety management in industry~\cite{kannan2016web}, safety risk early warning in urban construction~\cite{ding2013development}, and safe monitoring of infrastructural systems~\cite{repetto2018web}. Within these web-based risk management systems, the source search problem presents a huge challenge. Source search refers to the task of identifying the origin of a risky event, such as a gas leak and the emission point of toxic substances. This source search capability is crucial for effective risk management and decision-making.

Traditional approaches to implementing source search capabilities into the web systems often rely on solely algorithmic solutions~\cite{ristic2016study}. These methods, while relatively straightforward to implement, often struggle to achieve acceptable performances due to algorithmic local optima and complex unknown environments~\cite{zhao2020searching}. More recently, web crowdsourcing has emerged as a promising alternative for tackling the source search problem by incorporating human efforts in these web systems on-the-fly~\cite{zhao2024user}. This approach outsources the task of addressing issues encountered during the source search process to human workers, leveraging their capabilities to enhance system performance.

These solutions often employ a human-AI collaborative way~\cite{zhao2023leveraging} where algorithms handle exploration-exploitation and report the encountered problems while human workers resolve complex decision-making bottlenecks to help the algorithms getting rid of local deadlocks~\cite{zhao2022crowd}. Although effective, this paradigm suffers from two inherent limitations: increased operational costs from continuous human intervention, and slow response times of human workers due to sequential decision-making. These challenges motivate our investigation into developing autonomous systems that preserve human-like reasoning capabilities while reducing dependency on massive crowdsourced labor.

Furthermore, recent advancements in large language models (LLMs)~\cite{chang2024survey} and multi-modal LLMs (MLLMs)~\cite{huang2023chatgpt} have unveiled promising avenues for addressing these challenges. One clear opportunity involves the seamless integration of visual understanding and linguistic reasoning for robust decision-making in search tasks. However, whether large models-assisted source search is really effective and efficient for improving the current source search algorithms~\cite{ji2022source} remains unknown. \textit{To address the research gap, we are particularly interested in answering the following two research questions in this work:}

\textbf{\textit{RQ1: }}How can source search capabilities be integrated into web-based systems to support decision-making in time-sensitive risk management scenarios? 
% \sq{I mention ``time-sensitive'' here because I feel like we shall say something about the response time -- LLM has to be faster than humans}

\textbf{\textit{RQ2: }}How can MLLMs and LLMs enhance the effectiveness and efficiency of existing source search algorithms? 

% \textit{\textbf{RQ2:}} To what extent does the performance of large models-assisted search align with or approach the effectiveness of human-AI collaborative search? 

To answer the research questions, we propose a novel framework called Auto-\
S$^2$earch (\textbf{Auto}nomous \textbf{S}ource \textbf{Search}) and implement a prototype system that leverages advanced web technologies to simulate real-world conditions for zero-shot source search. Unlike traditional methods that rely on pre-defined heuristics or extensive human intervention, AutoS$^2$earch employs a carefully designed prompt that encapsulates human rationales, thereby guiding the MLLM to generate coherent and accurate scene descriptions from visual inputs about four directional choices. Based on these language-based descriptions, the LLM is enabled to determine the optimal directional choice through chain-of-thought (CoT) reasoning. Comprehensive empirical validation demonstrates that AutoS$^2$-\ 
earch achieves a success rate of 95–98\%, closely approaching the performance of human-AI collaborative search across 20 benchmark scenarios~\cite{zhao2023leveraging}. 

Our work indicates that the role of humans in future web crowdsourcing tasks may evolve from executors to validators or supervisors. Furthermore, incorporating explanations of LLM decisions into web-based system interfaces has the potential to help humans enhance task performance in risk control.







\section{System Model}
\label{sec:systemmodel}

We assume a blockchain environment consisting of $N$ server processes $p_1, p_2,..,p_N$ and $I$ client processes $c_1, c_2,..,c_I$. Clients send signed transactions to the server processes to be included in a future block.
The blockchain functions as the Public Key Infrastructure where the identifier of a client is its public key, and clients use their private keys to sign their transactions.
%Client transactions may include sequence numbers that enforce a strict order on the inclusion of the transactions in a given block.

We assume a consensus abstraction as a blackbox, where one or more processes construct blocks of transactions and propose them to the consensus mechanism. As a result, the consensus abstraction outputs an ordered sequence of blocks $b_1, b_2, \dots, b_n$, which is then processed by the execution engine. Additionally, we assume an execution engine abstraction as a blackbox that receives this ordered sequence of blocks from the consensus abstraction and executes them deterministically.

In the context of this work, we make no assumptions regarding the coupling between the consensus and execution layers. The interaction between consensus and execution may either follow a modular, decoupled approach, as in Sui and Aptos~\cite{sui,aptos}, or operate in a tightly coupled, sequential manner, as in Ethereum~\cite{ethereum}.

Client transactions might range from simple peer-to-peer transactions to complex application logic with the help of smart contracts. As applications might access arbitrary resources (i.e., addresses) that can not easily be deduced, we assume the existence of a system that provides hints about the resources a transaction will access during execution to the block producer. This can either be in the form of client hints as in Solana or Sui~\cite{solana,sui}, or in the form of an optimistic pre-execution step that determines these hints locally as in Polygon~\cite{polygonupdate}. However, we do not assume the list of hints to be exhaustive or correct. Transactions with incomplete or incorrect hints might trigger re-executions if the execution engine is Block-STM or a derivative~\cite{blockstm,chiron}, or aborted in Solana or Sui~\cite{solana,sui}.


\section{Anthemius}
\label{sec:overview}

The primary objective of \sys is to redesign the block-assembly approach in blockchains that offer parallel execution to improve the overall system throughput and prevent popular applications from creating bottlenecks by factoring in transaction dependencies and execution time.

At the time of writing, most blockchains that support parallel transaction execution use a single parameter such as the computational complexity in gas, the raw block size in bytes, or the number of transactions to limit the block size~\cite{aptos,sui,solana}. However, in the context of parallel transaction execution, a single parameter does not reflect the execution complexity of a block. If all transactions in the block access the same resource, the execution time is the sum of the runtime of all transactions. In contrast, if none of the transactions access conflicting resources, the runtime depends on the number of cores.

%For example, if the maximum block size is chosen in relation to the number of transactions or raw transaction size, a block that contains too many computationally heavy transactions could result in an unreasonably long execution time and slow down the system significantly.

%However, user estimates on execution time cannot be trusted because they tend to purposefully overestimate the execution cost to guarantee that their transaction will be executed and, as such, taking this as a basis would result in a strong performance degradation. In a preliminary analysis we identified that clients on the Aptos network estimate an on average 25 times higher gas cost, compared to Ethereum or Polygon with around 2 times the gas cost, which would cut the potential throughput by the same factor.
%While it is theoretically possible to create incentives for clients to estimate the gas cost more accurately, this degrades the user experience significantly as this could increase abort rates.

%However, even given accurate transaction run-time predictions, there is another problem. While in Ethereum transactions are executed sequentially and, as such, the sum of the accurate run-time predictions of the transactions also perfectly mirrors the actual execution time of the block, the moment we introduce parallel transaction execution, the total run-time heavily depends on how transactions are scheduled and their interdependencies. If all transactions in a given block depend on each other, the best-case total execution time is equal to serial execution. However, in the worst case, if executors are unaware of transaction dependencies, concurrent accesses have to be detected and transactions have to be re-executed to avoid diverging states, resulting in worse than serial execution time~\cite{blockstm}.

%As such, a single heavily sequential application can bottleneck the entire system. This can happen due to bad engineering practices when designing the smart contract, but may also be a result of malicious design.

Therefore, as a first step to begin constructing ``Good Blocks'', we need parameters that allow us to quantify this. We deploy two parameters to address this. First a transaction complexity parameter in Gas, similar to Ethereum, and second a concurrency parameter $c$ describing the system's ability to execute transactions in parallel (i.e. number of cores). As a result, the total maximum capacity of each block is $c * Gas$.

In the next sections, we first discuss where \sys fits into existing blockchain architectures. Following that, we outline the design of the block construction algorithm that considers both parameters and constructs blocks sensitive to transaction dependencies and their execution time to speed up the parallel execution of the block.

\subsection{Architecture}

\begin{figure}
\begin{center}
	\includegraphics[width=1\columnwidth]{figures/architecture.pdf}
\end{center}
\caption{\sys is inserted between the Mempool and Consensus}
\label{fig:architecture}
\end{figure}

Figure~\ref{fig:architecture} shows where \sys fits into the existing protocol stack of a blockchain. \sys is a modular layer that can be inserted between the consensus layer and the mempool where client transactions are stored and handled. 
In \sys, instead of fetching transactions directly from the mempool, the consensus layer fetches blocks of transactions through \sys. In turn, \sys obtains its transactions from the mempool, divides transactions into batches, and constructs the block to return to consensus.
Following that, the block is proposed in consensus which outputs an ordered list of blocks to the execution engine.

\sys requires the read and write sets of transactions, as well as an estimation of their execution time, to assess dependencies between transactions and construct blocks that can be executed efficiently in parallel. This information is already available in blockchains such as Solana~\cite{solana} and Sui~\cite{sui}, where transactions must declare all resource addresses they access during execution. In other blockchains, such as Ethereum~\cite{ethereum}, this information can be obtained, for example, by executing the transactions.

This design allows \sys to be seamlessly integrated into any existing blockchain stack with minimal architectural and system changes, and without changing the block structure. Furthermore, since \sys operates solely on the set of transactions, their read and write sets, and their gas footprints, it remains essentially stateless. This makes \sys particularly suitable for deployment in modular architectures, such as Narwhal, where only the execution layer is stateful~\cite{narwahl}.  

\subsection{Block Construction}

An important problem that has to be tackled when constructing good blocks is the absence of information regarding the structure of the current workload. If all transactions in the mempool access the same resources, attempting to schedule them efficiently can further slow down an already bottlenecked system. Similarly, if the algorithm is too strict in situations where a large percentage of transactions access the same resources, the synergetic effects of executing larger batches of transactions are lost. This is the case, as, for each block, the system has to instantiate the executor and worker threads, set up the virtual machine, extract the execution results, etc.

Therefore, as a first step, we divide \sys into two modular elements. First, the \textit{batch handler}, which polls batches of transactions from the mempool and hands the batches to the \textit{batch scheduler} in a batch-by-batch fashion. Second, the \textit{batch scheduler}, that attempts to include a given batch into the current block and provides feedback to the \textit{batch handler} about the success rate.
Subsequently, based on the feedback, the batch handler can adjust the inclusion policy to prevent too small blocks and also avoid wasting scheduling time on difficult-to-schedule workloads.

\paragraph{Batch handler.}

\begin{algorithm}[t]

\centering
\caption{\textsl{Batch Handler}}
\label{algo:batchhandler}
\begin{algorithmic}[1]

\Procedure{CreateGoodBlock}{$block, maxgas, c$}
\State{$seqlimit = \frac{maxgas}{c}$} \Comment{Limit on the sequential path}
\State{$resmap \gets \emptyset$} \Comment{Map to track transaction dependencies}
\State{$skippedclients \gets \emptyset$} \Comment{Set to track clients with skipped transactions}
\State{$numrelax \gets 0$} \Comment{Number of times inclusion rate was relaxed}
\For{\textbf{all} $\textit{batch} \in mempool$}
\State{$incrate \gets$ \textsc{schedule($block, batch, seqlimit, c, resmap, skippedclients$)}} \label{algo:batchhandler:schedule}
\If{$incrate < \textsc{targetincrate} $}
\If{$numrelax \geq \textsc{maxrelaxnum} \lor (incrate = 0 \land batch.isfull)$} \label{algo:batchhandler:check}
\State{\textbf{return}}
\EndIf
\State{$seqlimit = \frac{maxgas}{c} * \textsc{min}(\textsc{maxrelaxrate}, incrate*\textsc{targetincrate})$} \label{algo:batchhandler:relax}
\State{$numrelax++$}
\EndIf
\EndFor

\EndProcedure
\end{algorithmic}
\end{algorithm}

The functionality of the Batch Handler is outlined in Algorithm~\ref{algo:batchhandler}. The batch handler receives a $block$ to fill, the global concurrency parameter $c$, and the maximum gas limit. It then calculates a limit on the sequential path $seqlimit$ and initiates a map to track the transaction dependencies $resmap$ as well as a set of clients with skipped transactions $skippedclients$. 

Next, the batch handler retrieves transaction batches from the mempool and hands them to the batch scheduler alongside the block, the limit on the gas, the number of cores, the transaction resource dependencies $resmap$, and $skippedclients$ set in Line~\ref{algo:batchhandler:schedule}.
The batch scheduler responds with the transaction inclusion rate $incrate$.

Depending on the workload, as mentioned, the $seqlimit$ may be very strict which can result in very few transactions being included in a block.
Therefore, if the inclusion rate $incrate$ is smaller than some $\textsc{targetincrate}$, we relax the gas limit relative to the inclusion rate, up to some \textsc{maxrelaxrate} (Line~\ref{algo:batchhandler:relax}).

However, if the inclusion rate was too small for several consecutive attempts (i.e. $numrelax \geq \textsc{maxrelaxnum}$), we exit scheduling to avoid building a heavily sequential block again.
Furthermore, if there was an attempt to schedule a full batch and no transaction of this batch was successfully included in the current block ($incrate = 0$) we also stop scheduling (Line~\ref{algo:batchhandler:check}) as this indicates that at this point transactions are only included at a high cost to execution performance and scheduling latency. The rest of the transactions are then only included in a later block.%(i.e. marginal scheduling time increases with each added transaction significantly).

%\lef{what happenst to these transactions? I am guessing the skipped eventually forces them to be included?}

\paragraph{Batch Scheduler.}

Scheduling transactions with interdependencies and varying runtimes is a known NP-complete problem~\cite{BAKER1996225} where approximate solutions can construct near optimal schedules in polynomial time. However, polynomial runtime, particularly when executed on the critical path of consensus, may lead to a construction time that outweighs the performance gains achieved from producing ''Good Blocks.''

Fortunately, our first insight is that a near-optimal schedule for block construction is unnecessary. Instead, our main objective is to prevent popular resources and applications from creating a bottleneck while maximizing the parallel execution. We can achieve this by iterating over the set of resources each transaction accesses, recording the cost of the sequential path leading up to the transaction, and deciding if the transaction should be included in the current block by comparing the cost of the path with the gas per core parameter.
Furthermore, we also want to delay transactions that access multiple hot resources as they make it harder to schedule subsequent transactions.

As a result, the complexity of the block construction is of $O(N*k)$ where $N$ is the number of transactions and $k$ is the average number of resource accesses per transaction.

\begin{algorithm}[t]

\centering

\caption{\textsl{Batch Scheduler - Called in Line~\ref{algo:batchhandler:schedule} of Algorithm~\ref{algo:batchhandler}}}
\label{algo:batchscheduler}
\begin{algorithmic}[1]

\Procedure{schedule}{$block, batch, seqlimit, c, resmap, skippedclients$}
\For{\textbf{all} $\textit{tx} \in batch$} \Comment{Iterate over transactions} \label{algo:batchscheduler:iterate}
     \If{$tx.sender$ \textbf{in} $skippedclients$}
            \State{\textbf{continue}} \Comment{Skip transaction inclusion} \label{algo:batchscheduler:skipuser}
    \EndIf
    \State{$chaincost \gets 0$} \Comment{Longest chain length}
    \State{$hotresources \gets 0$}
    \For{\textbf{all} $\textit{readres} \in tx.readset$} \Comment{Iterate over readset} \label{algo:batchscheduler:readloop}
        \If{$readres \in resmap$} \Comment{Find longest chain}
            \If{$resmap[readres] > chaincost$} \Comment{Find read with largest cost}
                \State{$chaincost \gets resmap[readres]$}
            \EndIf
            \If{$resmap[readres] > \frac{block.gas}{c}$} \Comment{Check if read exceeds limit}
                \State{$hotresources++$} \label{algo:batchscheduler:hotread}
            \EndIf
        \EndIf
    \EndFor
    \If{$hotresources \geq \textsc{maxhotr} \land (|block| > \textsc{lim} \lor |block| < \textsc{maxlen}-\textsc{lim})$}  \label{algo:batchscheduler:hotreadcheck}
        \State{$skippedclients \gets skippedclients \cup tx.sender$}
        \State{\textbf{continue}} \Comment{Skip transaction inclusion}
    \EndIf
    
    \If{$chaincost + tx.gas > seqlimit \lor block.gas + tx.gas > seqlimit*c$}  \label{algo:batchscheduler:gascheck}
            \State{$skippedclients \gets skippedclients \cup tx.sender$}
            \State{\textbf{continue}} \Comment{Skip transaction inclusion}
    \EndIf
    \State{$block \gets block \cup tx$} \Comment{Add tx to Block}
    \For{\textbf{all} $\textit{writeres} \in tx.writeset$} \Comment{Iterate over writeset}
        \If{$writeres \notin resmap \lor resmap[writeres] < chaincost$}
         \State{$resmap[writeres] \gets chaincost$} \Comment{Note new chain length}
        \EndIf
    \EndFor
\EndFor
\State{\textbf{return}$(\frac{numscheduled}{|batch|})$}
\EndProcedure
\end{algorithmic}
\end{algorithm}


Algorithm~\ref{algo:batchscheduler} shows how we achieve this. The algorithm starts with the call of the \textsc{schedule} method, which receives the block to include the transactions in, the batch of transactions to schedule, the maximum gas per core $seqlimit$, the concurrency parameter $c$, the map of resources and the skipped clients. Following that, it starts iterating over all transactions in the batch (line~\ref{algo:batchscheduler:iterate}).
First, to maintain the order clients specified (e.g. through sequence numbers), after a client had a transaction skipped, the client is added to the $skippedclients$ set and no further transaction from this client will be included in this block.(Line~\ref{algo:batchscheduler:skipuser}).
Following that, we iterate over all reads in the transaction read-set and attempt to calculate the read with the longest path in gas leading up to this transaction (Line~\ref{algo:batchscheduler:readloop}).
In parallel, we count the number of \textit{hot} reads. A hot read is a read on a resource that is accessed significantly more often than other resources. 
%We approximate this by checking if it has a longer path than the total gas in the block so far, divided by the number of cores.

After this, we check whether the number of hot reads exceeds a predefined threshold, \textsc{maxhotr}. If this condition is met and the transaction is not within the first or last \textsc{lim}(i.e. 10\%) transactions, we skip the transaction (Line~\ref{algo:batchscheduler:hotreadcheck}).
We delay transactions with too many hot reads as they unify several critical paths of transactions which can severely bottleneck the execution.
However, we initially allow any transactions to be included up to some threshold \textsc{lim} to accumulate sufficient data to assess the complexity of reads and to guarantee that transactions that access several hot resources are eventually included. Furthermore, we also allow including transactions with multiple hot reads towards the end of the block as the block is almost full already and they are less likely to cause scheduling problems at this point.

Following that, we check if the transaction cost itself is larger than the max gas per core $seqlimit$ or if the current transaction exceeds the total gas limit of the block. If so, we also skip the transaction (Line~\ref{algo:batchscheduler:gascheck}).

Finally, we include the transaction in the block, iterate over its write set, and record the transaction path cost in the resource map $resmap$ if its writes increase the critical path.
This results in an algorithm that is linear to the number of transactions per block, as the map accesses are $O(1)$ and we check each transaction at most once per block.




\section{Evaluation}
\label{sec:evaluation}

%\subsection{Implementation}
%\label{sec:implementation}

We implemented \sys on top of Block-STM~\cite{blockstm} and \basesys~\cite{chiron} in Rust to evaluate its performance impact on both an optimistic execution engine and a guided execution engine, covering two of the most widely adopted approaches to parallel execution in the blockchain space.
The implementation is publicly available on Github~\footnote{https://github.com/ISTA-SPiDerS/Anthemius}.
As \basesys is built on top of Block-STM, this simplifies the implementation and allows for an easier comparison of the results.
Furthermore, we use the parallel execution benchmarks proposed in \basesys~\cite{chiron}. 

%We picked \basesys to represent guided execution engines as it is built on top of BlockSTM which makes comparing the two approaches easier.
%Therefore, aside from evaluating how BlockSTM performs with our block scheduling approach, we also leverage the implementation of \basesys~\cite{chiron} to evaluate the performance of guided execution engines which can be found in blockchains such as Solana~\cite{solana} or Sui~\cite{sui}. 

%Due to this, we can evaluate how \sys affects the performance of two popular approaches to parallel smart contract execution.

Finally, we implemented the batch handler ($\sim70$ lines of code) and the batch scheduler ($\sim120$ lines of code) to assemble blocks and then forward these blocks to the respective execution engines.


%% LIMIT is 1000 (10%)
% MAX relax num = 2, max relax rate = 100
% TARGETINCRATE = At least double of numtx/c
% MAXLEN = 10k
% hot read limit = 4

%The constructed blocks are then passed to the execution engine.

%Depending on the approach to consensus, the block construction in \sys can be optimized.
%For instance, if the same leader proposes multiple consecutive blocks, as in Narwahl~\cite{narwahl}, Kauri~\cite{kauri}, or PBFT~\cite{pbft}, the leader can optimize scheduling by tracking whether a batch primarily consisted of contended transactions at a given point and then skipping that batch in subsequent rounds. The leader then waits until the batch becomes the first batch, to start including the transactions again.

\subsection{Benchmark}

The experiments were executed on a Debian GNU/Linux 12 server with two AMD EPYC 7763 64-Core Processors and 1024 GB of RAM. We generated batches of transactions with different distributions of read/write-accesses and different user distributions with the help of \basesys~\cite{chiron} for all five proposed workloads. Namely, one peer-to-peer workload (P2PTX), two Decentralized Exchange Workloads (DEXAVG and DEXBURSTY), one NFT workload (NFT), and one mixed workload (MIXED). 
These workloads are derived from real-world data from Ethereum and Solana and are designed to evaluate parallel transaction execution engines under realistic levels of contention. 
Each workload has a unique and realistic resource access pattern, along with a varying count of read and write operations per transaction.

Each experiment was executed a total of 10 times and the results we outline in this section present the average of all 10 runs. Furthermore, in each workload, we vary the number of worker threads from 4 to 32 in increments of 4.
Finally, we are interested in two key metrics: throughput, to assess the performance improvement introduced by \sys, and latency, to determine the average delay introduced by \sys.


We set the following parameters for the batch handler and batch scheduler:
First, we evaluate the execution engines using blocks of up to $\textsc{maxlen} = 10{,}000$ transactions, as this block size represents a sweet spot for both engines, where the execution setup overhead (e.g., virtual machine initialization) becomes negligible. Accordingly, we configured the batch size to match the target block size, as smaller batch sizes increase block construction overhead, while larger batch sizes reduce the batch handler's flexibility to adapt to the workload's characteristics.

Next, to minimize tail latency for transactions accessing hot resources, we allow the first and last $\textsc{lim} = 1{,}000$ transactions to be included freely without restrictions.  
Furthermore, we permit up to $\textsc{maxrelaxnum} = 2$ relaxations of the inclusion rate as we observed diminishing returns from additional relaxations and large scheduling costs beyond this point.  
We set the relaxation rate to a maximum of $\textsc{maxrelaxrate} = 100$, targeting an inclusion rate of $\textsc{targetincrate} = 2\frac{\text{maxlen}}{c}$. This accounts for the higher returns from a more aggressive target inclusion rate as the concurrency potential increases.  
Finally, we configure $\textsc{maxhotr} = 4$ to avoid uniting too many critical paths of transactions, ensuring manageable contention levels.




%As previously stated, we evaluate the performance of \sys with Optimistic Parallel Execution using Block-STM~\cite{blockstm} and Guided Parallel Execution using \basesys~\cite{chiron}. 


%Given the number of worker threads, we created an equal number of batches for the batch handler.

%The batch handler first tries to include transactions from the first batch and then attempts to include transactions from the following batches. Following that, the execution finishes once the first batch is fully exhausted. 
%Thus, each run of \sys presents the average run overall for several blocks with varying block sizes up to 10.000 transactions. For vanilla BlockSTM and \basesys, there is a stable block size of 10.000 transactions.
%To measure the average latency, we only evaluated the latency for the transactions in the first batch, adding up the execution time and scheduling time of all blocks until the given transaction was successfully included in a block and executed.

\subsection{Throughput}



As \sys delays the inclusion of some transactions in favor of others to enhance system performance, we provide the batch handler with several batches of $10{,}000$ transactions to saturate the system and measure the maximum throughput. Each batch is generated with the same distribution of resource accesses, both within and across batches. We then evaluate \sys by passing all batches to the batch handler and run \sys until all transactions from the first batch are successfully executed. Consequently, the evaluation for \sys spans multiple blocks, where the reported throughput represents the average throughput over the entire runtime and accounts for scheduling and execution time.
For the baseline versions of Block-STM and \basesys, we use a single block containing $10{,}000$ transactions that also fully saturates the system, with runtime variations dependent solely on the specific workload.  


As blockchains such as Aptos or Sui decouple consensus from execution, block scheduling could be moved outside of the critical path of consensus. This can significantly reduce the overhead, as scheduling requires only a single thread and only has to be done at the proposer node. Due to this, we display two lines for \sys. First, one that serves as a ceiling on performance, where we assume that there is an idle thread that can be used for scheduling outside of the critical path of consensus, denoted \textit{Decoupled \sys}. Second, one that serves as a floor on performance where we count the full scheduling overhead on the critical path of consensus, referred to as \textit{\sys}.

\begin{figure*}[t]
\begin{subfigure}{0.5\textwidth}
\includegraphics[width=1\linewidth]{figures/tputpygoodblock.pdf} 
\caption{Throughput per Second - \basesys}
\label{fig:tputpythia}
\end{subfigure}
\begin{subfigure}{0.5\textwidth}
\includegraphics[width=1\linewidth]{figures/tputstmgoodblock.pdf}
\caption{Throughput per Second - Block-STM}
\label{fig:tputblockstm}
\end{subfigure}
\caption{Throughput per Second}
\label{fig:image2}
\end{figure*}


The results for \sys with \basesys are shown in Figure~\ref{fig:tputpythia}, with the throughput in transactions per second on the y-axis and the number of worker threads on the x-axis. With the NFT workload, we only see a small speedup from creating good blocks. This is due to the account distribution in this workload, where transactions from users appear very frequently in several batches. Due to this, once a transaction of a given user is skipped, the following transactions also have to be skipped, resulting in long scheduling times and leaving very few transactions behind that can be included in the block. In comparison, in the peer-to-peer workload there is already a significant improvement, where with an increasing number of worker threads, we can reach almost twice the initial throughput. Following that, with increasing contention and less repetitive users, the decentralized exchange workloads reach over 240\% performance boost compared to vanilla \basesys. While in the average DEX workload, the scheduling overhead is very small, with increasing contention and increasing number of worker threads we can also see an increased scheduling overhead.
Finally, in the mixed workload, we also see a large performance advantage. This is also due to the much higher overall execution complexity compared to the scheduling overhead. Due to the complexity of the workload, the overhead is constant after 12 cores, but \sys under this workload shows over 200\% performance advantage compared to vanilla \basesys.


The throughput results for \sys with Block-STM are shown in Figure~\ref{fig:tputblockstm}, with the throughput in transactions per second on the y-axis and the number of worker threads on the x-axis. 
Compared to the results with \basesys, the results for Block-STM vary more as the high contention within each block results in a large re-execution overhead. 
As such, even when we build better blocks with \sys, the contention in the block is still so high, that Block-STM struggles to take advantage of that.
We can still see the largest disadvantage in the NFT workload, due to the user distribution preventing us from building better blocks. Furthermore, we can see that in the peer-to-peer workload, once we reach 20 threads, \sys is starting to be able to compensate for the re-execution overhead of Block-STM and reach a speed-up of up to 25\%. Similarly, for the DEX workloads, there is an initial performance drop due to the re-execution overhead, which is only compensated with more worker threads later.
Finally, in the MIXED workload, \sys shows a constant speed up compared to vanilla Block-STM up to 200\% the original performance.

%For both approaches the total scheduling overhead was only between 5m and 10m per batch

\subsection{Latency}

\begin{figure}[t]
\includegraphics[width=1\linewidth]{figures/anthemiuslatency.pdf} 
\caption{Tail Latency for \basesys and Block-STM}
\label{fig:latency}
\end{figure}


As we are delaying the inclusion of some transactions that access hot resources, we expect a latency overhead increase at the tail. 
Similarly to the throughput evaluation, we send several batches of transactions to the batch handler. To fully assess the effect of \sys, we evaluate how the tail latency develops when awaiting the finished execution of up to five batches for all workloads with a fixed number of $16$ cores.
The results of this evaluation are shown in Figure~\ref{fig:latency}, where the yellow line indicates the 50th percentile (median), the box represents the 25th and 75th percentiles (interquartile range), and the whiskers denote the 10th and 90th percentiles.

The results mirror what we saw in the throughput evaluation where in almost all workloads and configurations where \sys shows a significant speedup the average transaction latency is significantly lower. Furthermore, thanks to the large throughput advantage in these settings, especially when paired with \textit{Chiron}, \sys has a latency advantage for up to the 90\% percentile of transactions.

On the other hand, as expected, \sys shows a growing tail latency with an increasing number of batches. This
is expected since the congestion caused by the highly contended workloads results in different scheduling decisions. Nevertheless, we can see that the growing tail latency affects not only \sys but also the reference systems, although for certain workloads the effects of \sys are more prominent at the p90 percentile. 

This is a tradeoff the blockchain needs to take into account based on their expected workload and tune \sys parameters to better match the chracteristics of the transactions expected.




%We evaluate the tail latency of \sys compared to the vanilla single block latency in Block-STM and \basesys. We split transactions into three categories: p50, p75, and p90, relative to the latency percentile they were in. As such, the presented graphs show the latency for the 3 given percentiles.

%The results for \basesys are displayed in Figure~\ref{fig:latpythia}, with the latency in seconds on the y-axis and the different latency percentiles on the x-axis. We can see that for all workloads excluding the mixed workload, 50\% of all transactions were executed faster than in vanilla \basesys. This is the case as they were paired with less contended transactions such that one or even multiple blocks can be executed in the same time frame as a single block in vanilla \basesys.
%In the mixed workload, the latency is mostly on par with a slight latency overhead. 

%In p75 the results vary more. In the P2P and DEX workloads, the latency shows an advantage or is on par with vanilla \basesys. While in the MIXED and NFT workloads, there is a slight latency overhead.

%Finally, in the p90 group, there is a latency overhead in almost all settings. This is expected as we are delaying some transactions to improve the overall throughput and to improve the latency of transactions that are not accessing hot resources.
%However, we would like to point out that even in the p90, none of the transactions took more than twice as long in any setting. This indicates that for \basesys, \sys only presents a small tradeoff from system throughput to tail latency overhead.



%The results for Block-STM are displayed in Figure~\ref{fig:latpythia} and are very similar to the ones of \basesys. For the p50 group, we see an advantage compared to vanilla execution, and, analogous to \basesys, in the p75 group we see some workloads with a small advantage while others show a small overhead.
%However, finally, in the p90 group, for workloads with very high contention, we can see a more significant latency overhead. Nonetheless, in no setting, this is significantly larger than three times the vanilla latency.

\subsection{Summary}

In this section, we evaluated the throughput improvement \sys can provide across different execution engines. Our findings demonstrate that while \sys improves throughput for both types of execution engines under several of the workloads, its impact is significantly larger when combined with guided execution engines. In this case, \sys provides a large throughput improvement across all but one of the workloads. The only exception is the NFT workload, where many high-frequency users appear across multiple blocks, preventing \sys from effectively rescheduling their transactions.

When it comes to latency, we analyzed the tail latency percentiles of delayed transactions. Our results show that for most workloads the majority of transactions (over 75\%) have lower or similar latency compared to the vanilla execution, while only the slowest 25\% of transactions sustain a latency overhead.
This indicates that \sys can be a valuable addition to any blockchain with a parallel execution engine where the workload does not primarily stem from a very small set of users.


\section{Related Work}
\label{sec:relatedwork}

To the best of our knowledge, there is no academic work proposing algorithms to construct blocks sensitive to parallel execution efficiency. 
While the problem is an NP-Complete scheduling problem which is explored in theoretical computer science~\cite{BAKER1996225}, the greedy version of these algorithms still requires polynomial time which would present a large overhead and negate most of the positive effects.
By relaxing the optimality requirement, instead, \sys achieves a linear complexity relative to the number of transactions per block.

In the database literature, there are numerous approaches to re-order transactions for reduced abort rates.
Most of the work in this context reorders transactions after execution to increase the goodput. Examples of this approach are Aria~\cite{aria}, where an efficient algorithm reorders transactions after execution based on the read and write sets to reduce the number of aborted transactions. Similarly, Sharma et al.~\cite{blurring} focus on execute-order blockchains where transactions are reordered during block construction.
While these approaches are efficient and can increase the goodput, none of them consider the parallel execution setting.

Eve~\cite{eve} is the most similar approach to \sys. In Eve, transactions are organized into batches such that, with high probability, no two transactions within the same batch access the same resource. This allows the execution engine to execute the block concurrently without having to worry about concurrent accesses during execution. Although the scheduling is very efficient, this approach is unsuitable for blockchain ecosystems where we are expecting a large percentage of transactions to overlap~\cite{chiron} and already have execution engines that can process transactions with dependencies efficiently.

\begin{table}[t]
\centering
{
\caption{Comparison of existing Block Production Approaches.}
\label{tab:comparison}
\begin{tabular}{|l|c|c|c|c|}
\hline
\textbf{Approaches} &  \textbf{Parallel} &  \textbf{Two-Dimensional} & \textbf{Dependency} & \textbf{Execution-Time} \\
 
 & \textbf{Execution} & \textbf{Gas Parameter} & \textbf{Sensitive}  & \textbf{Aware} \\

 \hline
  Ethereum~\cite{ethereum}  & \xmark  & \xmark  & \xmark & \cmark  \\ 
  Polygon~\cite{polygonupdate}  & \cmark  & \xmark  & \xmark & \cmark \\ 
   Aptos~\cite{aptos}  & \cmark  & \xmark  & \xmark & \xmark  \\ 
   Solana~\cite{solana} & \cmark   & \xmark   & \xmark & \cmark  \\ 
  %Sui~\cite{sui} & \cmark  & \xmark  & \cmark& \cmark  \\ 
  \sys~ & \cmark   & \cmark   & \cmark& \cmark \\ 
\hline
\end{tabular}
}
\end{table}

We, therefore, focus on the current state of block assembly in production blockchains. The discussion is summarized in Table~\ref{tab:comparison}.
While Ethereum~\cite{ethereum} does not natively support parallel execution at this moment, it constructs its blocks sensitive to the execution complexity of the smart contracts. The version of Polygon~\cite{polygonupdate} with Block-STM integration supports parallel execution and takes the execution complexity into account. However, it only has a one-dimensional gas parameter and does not take dependencies into account. Aptos~\cite{aptos} supports parallel execution but is unaware of the execution complexity of the transactions at block construction time and only takes the number of transactions and byte size into account. In comparison, Finally, Solana~\cite{solana} also offers parallel execution and takes the execution complexity into account. However, Solana does not take dependencies into account and only limits the combined computational complexity of transactions of a given client. 

%Finally, Sui~\cite{sui} recently introduced an update~\cite{suiupdate} where validators can delay and even abort transactions if they result in too-long sequential paths. However, this happens during consensus commit on the critical path of consensus. 
%In comparison, \sys is fully modular and replaces transactions accessing hot resources with less contended transactions during block creation, to reduce the network overhead and create larger batches for execution.
%Additionally, \sys prevents transactions that access several hot resources from uniting the sequential path of several resources, reducing the number of transactions that have to be delayed.
% It doesn't consider a given transaction accessing several hot resources which makes non sequential paths sequential and only limits tx at the tail

Therefore, to the best of our knowledge, \sys is the first work proposing a modular and practical algorithm for ``Good Block'' construction in the context of parallel smart contract execution.


\section{Discussion \& Conclusion}\label{sec:discussion}

\fauxsection{Auditing with hard prompts.}
Attacks such as greedy coordinate gradient~\cite{zou2023gcg} optimize the attack prompt in the \emph{hard} token space instead of the soft token space.
Hence, they are weaker at eliciting completions.
On one hand, this might make them more suitable for auditing unlearning.
On the other hand, due to their computational requirements, they are often used to force only the beginning of a harmful completion (e.g. \textit{Sure, here's how to build a bomb...}) with the hope that the LLM follows.
It is unclear whether this would be sufficient to produce specific unlearned passages.
We see it as an interesting direction for future work.

\fauxsection{Unlearning vs jail-breaking.}
Our findings are applicable to the jail-breaking community as well.
Prior work~\cite{zhang2024safe} hinted that unlearning and preventing harmful outputs can be viewed as the same task -- removing or suppressing particular information.
\sta{s} and fine-tuning attacks~\cite{hu2024jogging} are useful tools for evaluating LLMs in powerful threat models.
It was shown that fine-tuning on benign data, or data unrelated to the unlearning records (for jail-breaking and unlearning respectively) can restore undesirable behavior~\cite{lucki2024adversarial}.

\fauxsection{Variation in gradient-based learning.}
Prior work showed that removing training records from the training set,
and repeating the training can result in the same final model~\cite{thudi2022auditunlearning} depending on the random seed.
Even though a record was part of the training run, its influence might be minimal, making unlearning unnecessary.
Similarly, it was shown that SGD has intrinsic privacy guarantees, assuming there exists a group of similar records~\cite{hyland2022empiricalsgdprivacy}.
Thus, algorithmic auditing of unlearning might not be possible, and one would have to rely on verified or attested procedures instead~\cite{eisenhofer2023verifiedunlearning}, regardless of their impact on the model.

\fauxsection{Distinguishing learned soft tokens.}
Even though, in most our results, the number of soft tokens required to elicit a completion is the same,
we attempted to distinguish between them.
To this end, we take all single-token \sta{s} optimized for \tofu (Table~\ref{tab:attack-unlearning-tofu}) and assign a label $y=\{0, 1\}$: $y=0$ for $f_\emptyset$, and $y=1$ for $f_{ft}$ and the unlearned models.
We then train a binary classifier using $f_\emptyset$ and $f_{ft}$.
While we are able to overfit it and distinguish between $f_\emptyset$ and $f_{ft}$,
we were not able to train a model that would generalize to the unlearned models, and decisively assign a class. 
Our approach is similar to Dataset Inference~\cite{maini2021di,maini2024dillms} which showed there can be distributional differences between the models, depending on the data they were trained on.
Further investigation into \emph{what} soft tokens are learned during the audit is an interesting direction for future work.

\section{Conclusion}\label{sec:conclusion}

In this work, we show that soft token attacks (\sta{s}) cannot reliably distinguish between base, fine-tuned, and unlearned models.
In all cases, the auditor can elicit all unlearned information by appending optimized soft prompts to the base prompt.
Additionally, we show that \sta with a single soft token can elicit $150$ random characters, and over $400$ with soft tokens.

Our work demonstrates that machine unlearning in LLMs needs better evaluation frameworks.
While many unlearning methods can be broken by simple paraphrasing of original prompts, or by fine-tuning on partial unlearned data or even \emph{unrelated data}, 
\sta misrepresents their efficacy.

\section{Limitations \& ethical considerations}\label{sec:limitations}

\fauxsection{Limitations.}
Due to computational constraints our work is limited to 7-8 billion parameter models.
Nevertheless, given that LLMs' expressive power increases with size~\cite{kaplan2020scalinglaws}, our results should hold for larger LLMs.
Our evaluation with random strings could be extended to verify if there is a clear and generalizable dependency between the number of soft tokens and the maximum number of generated characters.

\fauxsection{Ethical considerations.}
In this work, we show that an auditor (a user) with white-box access to the model, and sufficient compute can elicit any text from the LLM.
While it does require knowing the target completion for a given prompt, it is likely that partial completions might be enough, thus allowing the user to elicit harmful information.
This may be particularly dangerous in settings where the user has approximate knowledge of the information that had been scrubbed off the LLM.


\section{Conclusion}
\label{sec:conclusion}

In this work, we presented \sys, a framework, and algorithm to construct highly parallelizable blocks in the context of parallel smart contract execution.
We evaluated \sys extensively under a series of realistic workloads, demonstrating a throughput improvement of up to \sysmax.   Furthermore, in most workloads, this approach leads to lower latency for the majority of transactions, while only delaying those that access hot resources and cause bottlenecks. %todo mention discussion point again.
Moreover, \sys not only improves the throughput of the underlying execution engine but also protects blockchains from being bottlenecked by popular applications.
Finally, \sys is highly modular and can be easily integrated into any production blockchain without any security tradeoffs.

%-------------------------------------------------------------------------------
%\section*{Availability}
%-------------------------------------------------------------------------------

%The code of \sys as well as the real-world workloads and the data on which they are based will be made public upon acceptance of the paper.

%-------------------------------------------------------------------------------
\bibliographystyle{splncs04}
\bibliography{main}

%%%%%%%%%%%%%%%%%%%%%%%%%%%%%%%%%%%%%%%%%%%%%%%%%%%%%%%%%%%%%%%%%%%%%%%%%%%%%%%%
\end{document}
%%%%%%%%%%%%%%%%%%%%%%%%%%%%%%%%%%%%%%%%%%%%%%%%%%%%%%%%%%%%%%%%%%%%%%%%%%%%%%%%

%%  LocalWords:  endnotes includegraphics fread ptr nobj noindent
%%  LocalWords:  pdflatex acks