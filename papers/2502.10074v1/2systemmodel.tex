\section{System Model}
\label{sec:systemmodel}

We assume a blockchain environment consisting of $N$ server processes $p_1, p_2,..,p_N$ and $I$ client processes $c_1, c_2,..,c_I$. Clients send signed transactions to the server processes to be included in a future block.
The blockchain functions as the Public Key Infrastructure where the identifier of a client is its public key, and clients use their private keys to sign their transactions.
%Client transactions may include sequence numbers that enforce a strict order on the inclusion of the transactions in a given block.

We assume a consensus abstraction as a blackbox, where one or more processes construct blocks of transactions and propose them to the consensus mechanism. As a result, the consensus abstraction outputs an ordered sequence of blocks $b_1, b_2, \dots, b_n$, which is then processed by the execution engine. Additionally, we assume an execution engine abstraction as a blackbox that receives this ordered sequence of blocks from the consensus abstraction and executes them deterministically.

In the context of this work, we make no assumptions regarding the coupling between the consensus and execution layers. The interaction between consensus and execution may either follow a modular, decoupled approach, as in Sui and Aptos~\cite{sui,aptos}, or operate in a tightly coupled, sequential manner, as in Ethereum~\cite{ethereum}.

Client transactions might range from simple peer-to-peer transactions to complex application logic with the help of smart contracts. As applications might access arbitrary resources (i.e., addresses) that can not easily be deduced, we assume the existence of a system that provides hints about the resources a transaction will access during execution to the block producer. This can either be in the form of client hints as in Solana or Sui~\cite{solana,sui}, or in the form of an optimistic pre-execution step that determines these hints locally as in Polygon~\cite{polygonupdate}. However, we do not assume the list of hints to be exhaustive or correct. Transactions with incomplete or incorrect hints might trigger re-executions if the execution engine is Block-STM or a derivative~\cite{blockstm,chiron}, or aborted in Solana or Sui~\cite{solana,sui}.
