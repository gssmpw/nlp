\section{Related Work}
\label{sec:relatedwork}

To the best of our knowledge, there is no academic work proposing algorithms to construct blocks sensitive to parallel execution efficiency. 
While the problem is an NP-Complete scheduling problem which is explored in theoretical computer science~\cite{BAKER1996225}, the greedy version of these algorithms still requires polynomial time which would present a large overhead and negate most of the positive effects.
By relaxing the optimality requirement, instead, \sys achieves a linear complexity relative to the number of transactions per block.

In the database literature, there are numerous approaches to re-order transactions for reduced abort rates.
Most of the work in this context reorders transactions after execution to increase the goodput. Examples of this approach are Aria~\cite{aria}, where an efficient algorithm reorders transactions after execution based on the read and write sets to reduce the number of aborted transactions. Similarly, Sharma et al.~\cite{blurring} focus on execute-order blockchains where transactions are reordered during block construction.
While these approaches are efficient and can increase the goodput, none of them consider the parallel execution setting.

Eve~\cite{eve} is the most similar approach to \sys. In Eve, transactions are organized into batches such that, with high probability, no two transactions within the same batch access the same resource. This allows the execution engine to execute the block concurrently without having to worry about concurrent accesses during execution. Although the scheduling is very efficient, this approach is unsuitable for blockchain ecosystems where we are expecting a large percentage of transactions to overlap~\cite{chiron} and already have execution engines that can process transactions with dependencies efficiently.

\begin{table}[t]
\centering
{
\caption{Comparison of existing Block Production Approaches.}
\label{tab:comparison}
\begin{tabular}{|l|c|c|c|c|}
\hline
\textbf{Approaches} &  \textbf{Parallel} &  \textbf{Two-Dimensional} & \textbf{Dependency} & \textbf{Execution-Time} \\
 
 & \textbf{Execution} & \textbf{Gas Parameter} & \textbf{Sensitive}  & \textbf{Aware} \\

 \hline
  Ethereum~\cite{ethereum}  & \xmark  & \xmark  & \xmark & \cmark  \\ 
  Polygon~\cite{polygonupdate}  & \cmark  & \xmark  & \xmark & \cmark \\ 
   Aptos~\cite{aptos}  & \cmark  & \xmark  & \xmark & \xmark  \\ 
   Solana~\cite{solana} & \cmark   & \xmark   & \xmark & \cmark  \\ 
  %Sui~\cite{sui} & \cmark  & \xmark  & \cmark& \cmark  \\ 
  \sys~ & \cmark   & \cmark   & \cmark& \cmark \\ 
\hline
\end{tabular}
}
\end{table}

We, therefore, focus on the current state of block assembly in production blockchains. The discussion is summarized in Table~\ref{tab:comparison}.
While Ethereum~\cite{ethereum} does not natively support parallel execution at this moment, it constructs its blocks sensitive to the execution complexity of the smart contracts. The version of Polygon~\cite{polygonupdate} with Block-STM integration supports parallel execution and takes the execution complexity into account. However, it only has a one-dimensional gas parameter and does not take dependencies into account. Aptos~\cite{aptos} supports parallel execution but is unaware of the execution complexity of the transactions at block construction time and only takes the number of transactions and byte size into account. In comparison, Finally, Solana~\cite{solana} also offers parallel execution and takes the execution complexity into account. However, Solana does not take dependencies into account and only limits the combined computational complexity of transactions of a given client. 

%Finally, Sui~\cite{sui} recently introduced an update~\cite{suiupdate} where validators can delay and even abort transactions if they result in too-long sequential paths. However, this happens during consensus commit on the critical path of consensus. 
%In comparison, \sys is fully modular and replaces transactions accessing hot resources with less contended transactions during block creation, to reduce the network overhead and create larger batches for execution.
%Additionally, \sys prevents transactions that access several hot resources from uniting the sequential path of several resources, reducing the number of transactions that have to be delayed.
% It doesn't consider a given transaction accessing several hot resources which makes non sequential paths sequential and only limits tx at the tail

Therefore, to the best of our knowledge, \sys is the first work proposing a modular and practical algorithm for ``Good Block'' construction in the context of parallel smart contract execution.