\section{Related Work}
Traditional studies on healthcare access primarily rely on surveys and administrative data, which, while rigorous, often lack the temporal resolution and spatial granularity needed to track evolving public health challenges \cite{gao2016assessment}. In contrast, crowdsourced data from digital platforms enable large-scale, passive data collection, offering real-time insights into public experiences \cite{wazny2017crowdsourcing}. Plus, advancements in natural language processing and machine learning have significantly enhanced the ability to analyze vast amounts of unstructured text (crowdsourced) data, enabling more precise and systematic analysis of user-generated content \cite{khan2023exploring,devlin2018bert}.

Previous research has highlighted the value of user-generated crowdsourced data in health crisis management and even health-related decision-making. For example, large-scale social media data has been leveraged to monitor disease outbreaks \cite{gui2017managing} and track public sentiment on vaccines \cite{salathe2011vaccine,broniatowski2018weaponized}. However, social media data often lacks the location-specific granularity and structured context, which are needed to assess health resource accessibility at a community level. In contrast, online review platforms, such as Google Maps, provide richer, geo-tagged insights into real-world experiences with pharmacies as well as other healthcare providers. Despite this advantage, the potential of crowdsourced data to capture dynamic changes, spatial and temporal disparities, and evolving public perceptions of health resource accessibility has received little attention—particularly in crises where real-time and precise management is most critical.

This study addresses this gap by leveraging Google Maps reviews to analyze spatial and temporal disparities in perceived health resource accessibility. Unlike traditional survey-based surveillance methods, which often lack real-time responsiveness, and social media data, which may be unstructured and less location-specific, online review platforms like Google Maps provide a unique, geo-tagged perspective on public experiences with healthcare services. Utilizing location-specific, crowdsourced data, our approach serves as both a scalable, real-time complement to traditional surveillance and an enhancement over existing crowdsourced data sources. These findings offer actionable, data-driven insights for policymakers and public health officials, helping to shape more equitable and responsive health policies.