%%%% ijcai25.tex

\typeout{IJCAI--25 Instructions for Authors}

% These are the instructions for authors for IJCAI-25.

\documentclass{article}
\pdfpagewidth=8.5in
\pdfpageheight=11in

% The file ijcai25.sty is a copy from ijcai22.sty
% The file ijcai22.sty is NOT the same as previous years'
\usepackage{ijcai25}

% Use the postscript times font!
\usepackage{times}
\usepackage{soul}
\usepackage{url}
\usepackage[hidelinks]{hyperref}
\usepackage[utf8]{inputenc}
\usepackage[small]{caption}
\usepackage{graphicx}
\usepackage{amsmath}
\usepackage{amsthm}
\usepackage{booktabs}
\usepackage{algorithm}
\usepackage{algorithmic}
\usepackage[switch]{lineno}
\usepackage[T1]{fontenc}
\usepackage{amssymb}

\usepackage{wrapfig}
\usepackage{float} 
\usepackage{array}
\usepackage[a4paper,margin=1in]{geometry}
\usepackage{siunitx}

% Comment out this line in the camera-ready submission
% \linenumbers

\urlstyle{same}

% the following package is optional:
\usepackage{latexsym}

% See https://www.overleaf.com/learn/latex/theorems_and_proofs
% for a nice explanation of how to define new theorems, but keep
% in mind that the amsthm package is already included in this
% template and that you must *not* alter the styling.
\newtheorem{example}{Example}
\newtheorem{theorem}{Theorem}

% Following comment is from ijcai97-submit.tex:
% The preparation of these files was supported by Schlumberger Palo Alto
% Research, AT\&T Bell Laboratories, and Morgan Kaufmann Publishers.
% Shirley Jowell, of Morgan Kaufmann Publishers, and Peter F.
% Patel-Schneider, of AT\&T Bell Laboratories collaborated on their
% preparation.

% These instructions can be modified and used in other conferences as long
% as credit to the authors and supporting agencies is retained, this notice
% is not changed, and further modification or reuse is not restricted.
% Neither Shirley Jowell nor Peter F. Patel-Schneider can be listed as
% contacts for providing assistance without their prior permission.

% To use for other conferences, change references to files and the
% conference appropriate and use other authors, contacts, publishers, and
% organizations.
% Also change the deadline and address for returning papers and the length and
% page charge instructions.
% Put where the files are available in the appropriate places.


% PDF Info Is REQUIRED.

% Please leave this \pdfinfo block untouched both for the submission and
% Camera Ready Copy. Do not include Title and Author information in the pdfinfo section
\pdfinfo{
/TemplateVersion (IJCAI.2025.0)
}

\title{Toward Equitable Access: Leveraging Crowdsourced Reviews to Investigate Public Perceptions of Health Resource Accessibility}

% \author{Anonymous Author(s)}

\author{
Zhaoqian Xue$^{1,*}$ \and
Guanhong Liu$^2$ \and
Kai Wei$^3$ \and
Chong Zhang$^4$ \and
Qingcheng Zeng$^5$ \and
Songhua Hu$^6$ \and
Wenyue Hua$^7$ \and
Lizhou Fan$^8$ \and
Yongfeng Zhang$^9$ \and
Lingyao Li$^{10,\dagger}$\
\affiliations
$^1$Georgetown University\\
$^2$University of Chicago\\
$^4$University of Michigan\\
$^3$University of Liverpool\\
$^5$Northwestern University\\
$^6$Massachusetts Institute of Technology\\
$^7$University of California, Santa Barbara\\
$^8$Harvard Medical School\\
$^9$Rutgers University, New Brunswick\\
$^{10}$University of South Florida\\
\emails
zx136@georgetown.edu,
guanhongliu@uchicago.edu,
weikai@umich.edu,
pcczc15@gmail.com,
qingchengzeng2027@u.northwestern.edu,
hsonghua@mit.edu,
wenyuehua@ucsb.edu,
lfan8@bwh.harvard.edu,
yongfeng.zhang@rutgers.edu,
lingyaol@usf.edu
}

\begin{document}

\maketitle

\begin{abstract}
Access to health resources is a critical determinant of public well-being and societal resilience, particularly during public health crises when demand for medical services and preventive care surges. However, disparities in accessibility persist across demographic and geographic groups, raising concerns about equity. Traditional survey methods often fall short due to limitations in coverage, cost, and timeliness. This study leverages crowdsourced data from Google Maps reviews, applying advanced natural language processing techniques, specifically ModernBERT, to extract insights on public perceptions of health resource accessibility in the United States during the COVID-19 pandemic. Additionally, we employ Partial Least Squares regression to examine the relationship between accessibility perceptions and key socioeconomic and demographic factors—including political affiliation, racial composition, educational attainment and so on. Our findings reveal that public perceptions of health resource accessibility varied significantly across the U.S., with disparities peaking during the pandemic and slightly easing post-crisis. Political affiliation, racial demographics, and education levels emerged as key factors shaping these perceptions. These findings underscore the need for targeted interventions and policy measures to address inequities, fostering a more inclusive healthcare infrastructure that can better withstand future public health challenges.
\end{abstract}

\section{Introduction}

Access to essential health resources is fundamental to public well-being, particularly during health crises when demand surges for medical services and preventive care \cite{who2020}. Equitable distribution of critical supplies—such as medications, personal protective equipment (PPE), and testing kits—is vital for controlling disease spread and minimizing mortality \cite{sodhi2023research,roozenbeek2020susceptibility}. However, disparities in health resource availability persist, exacerbating existing social and economic inequalities \cite{baum2016health,detels2021socioeconomic}.

Traditional methods for assessing health resource accessibility—such as surveys and administrative records—provide valuable insights but are often constrained by time lags, high costs, and limited spatial coverage \cite{keppel2005methodological}. These limitations hinder timely interventions, particularly during rapidly evolving health crises. With the rise of digital platforms, crowdsourced data provides an alternative means to assess public perceptions of health resource accessibility \cite{kim2017problems}. Platforms like Google Maps offer granular, real-time insights into how people experience and perceive access to healthcare resources in their local communities \cite{jia2021online}.

This study examines the potential of crowdsourced data in analyzing social inequality by leveraging Google Maps reviews (2018–2021) to investigate how public perceptions of health resource accessibility varied across the U.S. during the COVID-19 pandemic. Specifically, we aim to answer:

\begin{itemize} 
\item[\textbf{RQ1:}] Do health resource disparities, as perceived by the public, exist across different regions in the United States? 
\item[\textbf{RQ2:}] Are these perceived health resource disparities correlated with the socioeconomic or demographic characteristics of local communities? 
\item[\textbf{RQ3:}] Did the COVID-19 pandemic exacerbate health resource disparities as perceived by the public? \end{itemize}

To address these questions, we apply state-of-the-art Natural Language Processing (NLP) techniques, typically ModernBERT, for text analysis and Partial Least Squares (PLS) regression to examine the relationship between perceived accessibility and socioeconomic factors. Our findings reveal significant disparities in public perceptions of health resource accessibility, which peaked during the crisis and partially eased afterward. Political affiliation, racial composition, and educational attainment emerged as key drivers of these perceptions. These results underscore the need for targeted policy interventions to promote a more equitable healthcare infrastructure.

\section{Related Work}

Traditional studies on healthcare access primarily rely on surveys and administrative data, which, while rigorous, often lack the temporal resolution and spatial granularity needed to track evolving public health challenges \cite{gao2016assessment}. In contrast, crowdsourced data from digital platforms enable large-scale, passive data collection, offering real-time insights into public experiences \cite{wazny2017crowdsourcing}. Plus, advancements in natural language processing and machine learning have significantly enhanced the ability to analyze vast amounts of unstructured text (crowdsourced) data, enabling more precise and systematic analysis of user-generated content \cite{khan2023exploring,devlin2018bert}.

Previous research has highlighted the value of user-generated crowdsourced data in health crisis management and even health-related decision-making. For example, large-scale social media data has been leveraged to monitor disease outbreaks \cite{gui2017managing} and track public sentiment on vaccines \cite{salathe2011vaccine,broniatowski2018weaponized}. However, social media data often lacks the location-specific granularity and structured context, which are needed to assess health resource accessibility at a community level. In contrast, online review platforms, such as Google Maps, provide richer, geo-tagged insights into real-world experiences with pharmacies as well as other healthcare providers. Despite this advantage, the potential of crowdsourced data to capture dynamic changes, spatial and temporal disparities, and evolving public perceptions of health resource accessibility has received little attention—particularly in crises where real-time and precise management is most critical.

This study addresses this gap by leveraging Google Maps reviews to analyze spatial and temporal disparities in perceived health resource accessibility. Unlike traditional survey-based surveillance methods, which often lack real-time responsiveness, and social media data, which may be unstructured and less location-specific, online review platforms like Google Maps provide a unique, geo-tagged perspective on public experiences with healthcare services. Utilizing location-specific, crowdsourced data, our approach serves as both a scalable, real-time complement to traditional surveillance and an enhancement over existing crowdsourced data sources. These findings offer actionable, data-driven insights for policymakers and public health officials, helping to shape more equitable and responsive health policies.

\section{Dataset and Methodology}

The research started with data collection and preparation, involving the aggregation of Google Maps reviews related to stores across the United States from 2018 to 2021 (Section \ref{sec:dataset}). We then applied a keyword filtering process to identify reviews concerning health resources (Section \ref{sec:keywords}). To determine the public perceptions of health resource accessibility, we developed text classification models using natural language processing (NLP) and machine learning techniques (Section \ref{sec:textcla}). After classification, we calculated perception scores based on the classification results (Section \ref{sec:measuring}). Finally, we employed Partial Least Squares (PLS) regression to explore the relationships between perception scores and both socioeconomic and demographic factors across different time periods (Section \ref{sec:pls}).

\subsection{Data Collection}
\label{sec:dataset}

Google Maps, a widely used location-based platform enriched with user-generated reviews, serves as a valuable data source for analyzing consumer experiences \cite{mehta2019google}. To investigate health resource disparities during the COVID-19 pandemic, we compiled a dataset of 4,569 Points of Interest (POIs) across the United States from 2018 to 2021. These POIs include pharmacies, groceries, and other medical resource providers, offering insights into the availability and accessibility of essential medical supplies.

The POI dataset was sourced from Google Local Data \cite{li2022uctopic,yan2023personalized}, which is a large-scale repository containing 666,324,103 reviews, 113,643,107 users, and 4,963,111 businesses in JSON format. This extensive database provides real-time, location-specific information on consumer experiences, enabling us to track the availability of critical health supplies such as personal protective equipment (PPE) and over-the-counter medications. By leveraging this dataset, our study captures public perceptions of health resource accessibility with a level of granularity and timeliness not easily attainable through traditional data sources.

To gain deeper insights into public perceptions of health resource accessibility, we segmented the dataset into three distinct time periods.

\begin{itemize}
    \item \textbf{Pre-Pandemic Period}: January 1, 2018 -- January 31, 2020
    \item \textbf{Peak-Pandemic Period}: February 1, 2020 -- May 31, 2020
    \item \textbf{Post-Peak Period}: June 1, 2020 -- May 31, 2021
\end{itemize}

On February 1, 2020, the World Health Organization (WHO) declared COVID-19 a global health emergency \cite{Cucinotta2020}, triggering severe shortages of critical health supplies \cite{trump2020proclamation}. In response, on March 18, 2020, President Trump invoked the Defense Production Act (DPA) to accelerate medical supply production \cite{peters2020dpa}, later extending it to protect food supply chains \cite{trump2020dpa}. By early May 2020, supply shortages began stabilizing \cite{gao2021report}, marking the end of the Peak-Pandemic period on May 31, 2020. The subsequent Post-Peak phase spanned until May 31, 2021, allowing us to assess the early recovery period. This segmentation provided a nuanced comparison of public perceptions before, during, and after the crisis.

We enhanced our analysis by incorporating three supplementary datasets from the U.S. Census Bureau. First, the Household Pulse Survey \cite{census2021pulse} offered a government-led benchmark to validate user-reported perceptions, comparing experiences of resource scarcity with official self-reported delays. Second, county-level administrative shapefiles \cite{census2023tiger} delineated geographic boundaries for spatial analysis. Third, socioeconomic metrics \cite{census2023tiger} facilitated regression modeling to understand how income, education, and race influenced public perceptions of health resource availability. By merging these sources, we aimed to ensure analytical rigor and robust insight into the pandemic’s varying impacts on health resource access across diverse regions.

\subsection{Keywords Ontology and Development}
\label{sec:keywords}

To identify reviews related to health resources, we developed a comprehensive keyword ontology informed by public health literature and consumer behavior research. Our process began with critical items designated by the Centers for Disease Control and Prevention (CDC) and WHO, such as “sanitizer,” “mask,” and “thermometer” \cite{liang2020development}. We then expanded the list to include common over-the-counter medications and household supplies (e.g., “Tylenol,” “Advil,” “Lysol spray,” “Vick's Vaporub”) \cite{keller2020consumer}, as well as COVID-19-specific terms, such as “N95,” “test kit,” and “home test” \cite{andrejko2022effectiveness}. Additionally, we incorporated controversial items (e.g., “hydroxychloroquine”), which sparked significant public debate during the pandemic \cite{ben2012hydroxychloroquine,rea2021open,self2020effect}. The full keyword list is presented in Table~\ref{tab:keywords}.

We then conducted iterative adjustments using insights from social media analyses and consumer health research \cite{awan2023neoteric} to refine our ontology, ensuring that our dataset captured colloquial terms, brand names, and emerging pandemic-related jargon \cite{amur2023unlocking}. This approach filtered 289,919 relevant reviews from 102,452 businesses, yielding a robust dataset that reflects public perceptions of health resource availability across different regions and time periods.

\begin{table}[htbp]
\centering
\caption{Keyword ontology for health resources}
\label{tab:keywords}
\centering
\begin{tabular}{>{\raggedright\arraybackslash}p{3cm}p{4cm}}
\toprule
\textbf{Category} & \textbf{Keywords} \\
\midrule
Essential health supplies & sanitizer, soap, toilet paper, mask, disinfectant, gloves, thermometer, tissues, wipes, face shield, hand wash, respirators, alcohol \\
\midrule
Over-the-counter medications & acetaminophen, tylenol, advil, motrin, ibuprofen, dayquil, nyquil, mucinex, robitussin, sudafed, pepto-bismol, tums, vick's vaporub \\
\midrule
Preventive healthcare items & vitamins, zinc, pedialyte, gatorade \\
\midrule
Diagnostic tools & test kit, home test, self test \\
\midrule
COVID-19 specific items & N95, hydroxychloroquine \\
\midrule
Household sanitization products & lysol spray, disinfectant wipes \\
\bottomrule
\end{tabular}
\end{table}

\subsection{Text Classification}
\label{sec:textcla}

We developed classification models to categorize reviews into three classes: Class -1 (shortage of health resources), Class 1 (no shortage), and Class 9 (unrelated). To ensure balance and minimize bias, we employed a sequential annotation approach, manually labeling reviews until each class contained approximately 500 instances, yielding a 1,500-review dataset.

To maintain annotation quality, three public health experts labeled the data following detailed guidelines. Each review was independently annotated by two experts, achieving a Cohen’s Kappa of 0.85 \cite{cohen1960coefficient}, with a third expert resolving any disagreements. The final dataset was split 80\% for training and 20\% for testing.

We vectorized the text using TF-IDF \cite{xiang2022application} and ModernBERT embeddings. Three classifiers—Random Forest \cite{breiman2001random}, Support Vector Machine (SVM) \cite{vapnik2013nature}, and Logistic Regression \cite{king2001logistic}—were trained on TF-IDF features, alongside a standalone ModernBERT model. After 5-fold cross-validation and grid search hyperparameter tuning, we compared models based on accuracy, precision, recall, and F1-Score \cite{krasnodkebska2024advancing}.

As shown in Figure~\ref{fig:classfiers} in appendix, ModernBERT outperformed TF-IDF-based models, achieving an overall accuracy of 85.2\% and superior F1-Scores. Consequently, we selected ModernBERT as the final classifier and applied it to all reviews, retaining only Class -1 and Class 1 reviews while discarding Class 9.

\subsection{Measuring Public Perceptions of Health Resource Availability}
\label{sec:measuring}

To assess public perceptions of health resource accessibility over time and across regions, we derived a county-level perception score from our classification results (\autoref{sec:textcla}). This score serves as a proxy for community experiences, capturing the balance between reported shortages and adequacy. Formally, for each county $c$ and period $t$:

\begin{equation}
S_{c,t} = \frac{1}{N_{c,t}} \sum_{i=1}^{N_{c,t}} L_i
\end{equation}

where $N_{c,t}$ is the number of reviews labeled as shortage (Class -1) or no shortage (Class 1), and $L_i$ is the label for review $i$. Scores approaching $-1$ indicate widespread shortage complaints, values near $-1$ suggest perceived resource adequacy, and scores around 0 reflect a mix of both perceptionss. Thus, $S_{c,t}$ provides a quantitative measure of health resource perceptions at the county level.

To ensure reliability, we restricted analysis to counties with at least 10 relevant reviews per period (Section~\ref{sec:dataset}). This threshold, widely used in social media analytics, helps reduce noise and improve representativeness \cite{faber2014sample}.

\begin{table}[htbp]
\centering
\caption{Variance Inflation Factor (VIF) for features in the model}
\begin{tabular}{lcc}
\toprule
\textbf{Feature} & \textbf{VIF} \\ 
\midrule
Democratic Rate               & 41.104 \\
Republican Rate             & 35.098 \\
Total Population            & 1.550 \\
Median Income               & 11.411 \\
GINI                        & 3.548 \\
No Insurance Rate           & 3.000 \\
Household Below Poverty Rate & 8.924 \\
HISPANIC LATINO Rate        & 5.771 \\
White Rate                  & 24.026 \\
Black Rate                  & 17.701 \\
Indian Rate                 & 1.894 \\
Asian Rate                  & 6.624 \\
Under 18 Rate               & \textit{inf} \\
Between 18 and 44 Rate      & \textit{inf} \\
Between 45 and 64 Rate      & \textit{inf} \\
Over 65 Rate                & \textit{inf} \\
Male Rate                   & 2.037 \\
Bachelor Rate               & 17.800 \\
Education Degree Rate       & 27.328 \\
Population Density          & 1.374 \\
Unemployed Rate             & 3.165 \\
\bottomrule
\end{tabular}
\label{tab:vif_table}
\end{table}

\subsection{Partial Least Squares (PLS) Regression}
\label{sec:regression}

After calculating the perception scores, we performed Partial Least Squares (PLS) regression at the county level to explore the relationship between average perception scores and socioeconomic factors. Ultimately, 530 counties were included in the final regression analysis.

PLS regression was selected to address the multicollinearity observed among several independent variables, such as Democratic and Republican Rate, all exhibiting variance inflation factors (VIF) exceeding 5 (Table~\ref{tab:vif_table}). A common solution to multicollinearity is to remove collinear variables, but this approach risks excluding key predictors. PLS regression, however, addresses this challenge by decomposing both dependent and independent variables into orthogonal scores and loadings during coefficient estimation \cite{deJong1993simpls}. The key equations of PLS regression are presented below:

\begin{equation}
    \begin{array}{l}
        X=T P^T+E \\
        Y=U Q^T+F \\
        Y=X K^T+\Theta
    \end{array}
\end{equation}
\noindent

where $T$, $U$, $P$, and $Q$ are orthogonal scores and loadings, $K$ represents the regression coefficients, and error terms $E$ and $F$ are independently and identically distributed. To address RQ2, we conducted three regressions on the average perception scores (Pre-, Peak-, and Post-Pandemic), and for RQ3, two additional regressions examined score differences between periods. Model fit was assessed using $R^2$ and RMSE.

Since standard PLS model does not inherently produce p-values, we employed a permutation test \cite{Afthanorhan2015}, comparing observed coefficients with those from permuted samples to derive robust significance estimates.

\section{Results}

This section presents the results of our analysis, addressing the three key research questions outlined in the introduction. First, we analyzed the geographic distribution of perception scores derived from Google Maps reviews to examine whether public perceptions of health resource accessibility varied across regions in the United States (Section \ref{lab:RQ1}). We visualized spatial patterns and quantified the degree of spatial clustering using Moran's I statistic. Next, we investigated the relationship between the socioeconomic characteristics of local communities (counties) and perceptions of health resource accessibility, across three time periods: Pre-Pandemic, Peak-Pandemic, and Post-Peak, through PLS regression (Section \ref{lab:RQ2}). Finally, we evaluated whether the pandemic exacerbated existing health resource disparities by investigating the changes in perception scores between periods (Section \ref{lab:RQ3}).

\begin{figure}[htbp] 
    \centering 
    \includegraphics[width=0.475\textwidth]{validation.png} 
    \caption{Comparison between weighted delayed ratio reported by US Census Bureau and average perception scores by online reviews.} 
    \label{fig:validation} 
\end{figure}

Before addressing the research questions, we validated the public perception of health resource accessibility---calculated using online reviews grouped by each month---by comparing it with data from the Household Pulse Survey conducted by the US Census Bureau from April 2020 to April 2021 (Figure \ref{fig:validation}). The survey measured the proportion of respondents reporting delays in securing health resources at the state level. To evaluate the alignment between the two measures, we calculated the correlation between the perceived scores (state-level averages weighted by perception scores) and the proportion of respondents reporting delays. To ensure robust parameter estimates, we employed Cook’s Distance \cite{Nurunnabi2015} to identify and remove observations that were both outliers and high-leverage points, as such data points could significantly distort the true correlation trend \cite{psu462}. 

As shown in Figure \ref{fig:validation}, the red dashed line shows the trend with a negative correlation ($r$ = -0.255). This suggests that states with higher public perception scores of health resource accessibility tended to have fewer reported delays in securing health resources. Although the correlation is slight, it implies that online reviews reflect the situation in terms of health resource accessibility.

\begin{figure}[htbp]
    \centering
    \includegraphics[width=0.475\textwidth]{sentiment_boxplot.pdf}
    \vspace{-5pt}
    \caption{Health resource availability trends across pandemic periods.}
    \label{fig:sentiment}
\end{figure}

\subsection{RQ1: Do health resource disparities, as perceived by the public, exist across different regions in the United States?}
\label{lab:RQ1}

To address RQ1, we analyzed county-level perception scores of health resource accessibility across the United States during the Pre-Pandemic, Peak-Pandemic, and Post-Peak periods. These scores range from -1 (indicating extremely negative perceptions) to 1 (indicating extremely positive perceptions). As illustrated in Figure~\ref{fig:sentiment}, the average perception scores dropped significantly during the Peak-Pandemic period and remained relatively low in the Post-Peak period, underscoring persistent challenges in health resource accessibility. These trends were further visualized using GIS maps (Figure~\ref{fig:sentiment_maps}), where red denotes negative perceptions and blue represents positive perceptions.

During the Pre-Pandemic period (Figure~\ref{fig:sentiment_maps}a), negative perceptions were particularly pronounced across parts of the Western states, while the Eastern regions exhibited a more nuanced mix of positive and negative perceptions, especially in densely populated areas. During the Peak-Pandemic period (Figure~\ref{fig:sentiment_maps}b), these regional disparities in public perceptions became more pronounced. Western states, notably California and its surrounding areas, experienced a significant surge in negative perceptions, as reflected by larger red clusters. Meanwhile, the Eastern region maintained its heterogeneous pattern but exhibited a noticeable shift toward more negative perceptions. By the Post-Peak period (Figure~\ref{fig:sentiment_maps}c), a substantial transition toward neutral perceptions was observed nationwide, as indicated by the predominance of light gray tones on the map.

\begin{figure}[htpb]
    \centering
    \includegraphics[width=0.475\textwidth]{GIS.png}
    \vspace{-5pt}
    \caption{Geographic patterns of the perceived health resource disparities across the United States.}
    \label{fig:sentiment_maps}
\end{figure}

We used Moran's I statistic to perform the spatial autocorrelation analysis and quantify the patterns of public perceptions, as shown in Figure~\ref{fig:morans_i}. Before the pandemic, perception scores showed virtually no spatial clustering among neighboring counties, as indicated by a low, non-significant Moran's I ($0.001$, $p = 0.421$). As the pandemic reached its peak, however, we observed the emergence of weak yet statistically significant spatial patterns in public perception (Moran's I $= 0.016$, $p = 0.048$). In the Post-Peak period, spatial autocorrelation became more pronounced and statistically significant (Moran's I $= 0.022$, $p = 0.012$), indicating that counties with similar perception scores were increasingly clustered geographically. This trend is also evident in Figure~\ref{fig:sentiment_maps}, which shows that most investigated counties display similar sentiment values. Such a progression suggests that the pandemic may have exposed and potentially exacerbated underlying regional inequalities in health resource availability.

\begin{figure}[htbp]
    \centering
    \includegraphics[width=0.475\textwidth]{moran_scatterplots.pdf}
    \vspace{-5pt}
    \caption{Moran's I scatterplots across pandemic periods.}
    \label{fig:morans_i}
\end{figure}

\begin{table*}[htbp]
\setlength{\tabcolsep}{4pt}
\captionsetup{width=\textwidth}
\caption{PLS regression results of socioeconomic factors on perception scores across pandemic periods.}
\label{tab:regression-results}
\scriptsize
\begin{tabular*}{\textwidth}{@{\extracolsep{\fill}}lccccccccc@{}}
\toprule
\textbf{Variable} & \multicolumn{3}{c}{\textbf{Pre-Pandemic}} & \multicolumn{3}{c}{\textbf{Peak-Pandemic}} & \multicolumn{3}{c}{\textbf{Post-Peak}} \\
\cmidrule(lr){2-4} \cmidrule(lr){5-7} \cmidrule(lr){8-10}
& {Coeffs} & {P-Value} & {Std. err.} & {Coeffs} & {P-Value} & {Std. err.} & {Coeffs} & {P-Value} & {Std. err.} \\
\midrule
Democratic Rate & 0.023 & 0.057 & 0.044 & 0.008 & 0.005** & 0.045 & -0.003 & 0.022* & 0.044 \\
Republican Rate & -0.025 & 0.025* & 0.045 & -0.017 & 0.002** & 0.046 & -0.001 & 0.002** & 0.045 \\
Total Population & 0.004 & 0.701 & 0.044 & 0.008 & 0.779 & 0.044 & -0.001 & 0.595 & 0.045 \\
Median Income & -0.018 & 0.599 & 0.044 & -0.013 & 0.010** & 0.044 & 0.002 & 0.005** & 0.046 \\
GINI & 0.016 & 0.064 & 0.046 & -0.015 & 0.865 & 0.046* & 0.002 & 0.116 & 0.043 \\
No Insurance Rate & -0.020 & 0.000** & 0.046 & -0.023 & 0.000** & 0.045 & -0.005 & 0.001** & 0.046 \\
Household Below Poverty Rate & -0.007 & 0.990 & 0.045 & 0.001 & 0.006** & 0.044 & -0.001 & 0.099 & 0.045 \\
HISPANIC LATINO Rate & -0.024 & 0.056 & 0.045 & -0.018 & 0.000** & 0.045 & 0.002 & 0.287 & 0.044 \\
White Rate & 0.024 & 0.000** & 0.044 & 0.001 & 0.674 & 0.044 & -0.002 & 0.632 & 0.044 \\
Black Rate & -0.045 & 0.000** & 0.045 & -0.008 & 0.240 & 0.046 & -0.003 & 0.358 & 0.044 \\
Indian Rate & 0.003 & 0.536 & 0.043 & 0.017 & 0.126 & 0.047 & -0.006 & 0.009** & 0.044 \\
Asian Rate & 0.007 & 0.679 & 0.045 & -0.001 & 0.055 & 0.047 & 0.001 & 0.002** & 0.044 \\
Under 18 Rate & 0.002 & 0.000** & 0.046 & 0.003 & 0.000** & 0.045 & -0.004 & 0.011* & 0.045 \\
Between 18 and 44 Rate & 0.001 & 0.849 & 0.044 & -0.009 & 0.810 & 0.044 & 0.003 & 0.013* & 0.044 \\
Over 65 Rate & 0.005 & 0.016* & 0.045 & 0.013 & 0.072 & 0.044 & 0.001 & 0.548 & 0.043 \\
Male Rate & -0.019 & 0.878 & 0.045 & -0.012 & 0.095 & 0.043 & -0.002 & 0.941 & 0.045 \\
Bachelor Rate & 0.004 & 0.185 & 0.045 & 0.012 & 0.000** & 0.046 & 0.002 & 0.000** & 0.045 \\
Education Degree Rate & -0.011 & 0.147 & 0.045 & 0.011 & 0.000** & 0.044 & 0.001 & 0.000** & 0.045 \\
Population Density & -0.012 & 0.794 & 0.044 & 0.001 & 0.320 & 0.047 & 0.001 & 0.086 & 0.045 \\
Unemployed Rate & 0.005 & 0.554 & 0.046 & -0.003 & 0.016* & 0.044 & 0.002 & 0.120 & 0.047 \\
\midrule
\textbf{Model Goodness-of-fit} & {$R^2$: 0.135} & {RMSE: 0.171} & {} & {$R^2$: 0.142} & {RMSE: 0.125} & {} & {$R^2$: 0.087} & {RMSE: 0.038} & {} \\
\bottomrule
\end{tabular*}
\captionsetup{justification=raggedright}
\caption*{Note: Significance codes: ** $<$ 0.01, * $<$ 0.05}
\end{table*}

\subsection{RQ2: Are these perceived health resource disparities correlated with the socioeconomic and demographic factors?}
\label{lab:RQ2}

The PLS regression overall revealed a strong correlation between public perceptions and socioeconomic characteristics (Table \ref{tab:regression-results}), with significant influences observed from factors such as race, income inequality, education, and even age. 

Before the outbreak of COVID-19, the White Rate (coeff = 0.024, $p$ = 0.000) exhibited a positive association with perception scores, indicating better access to health resources in counties with a higher proportion of White residents. Conversely, counties with a larger proportion of Black residents (Black Rate: coeff = -0.045, $p$ = 0.000) tended to have less favorable access. Additionally, a higher Republican population rate (Republican Rate: coeff = -0.025, $p$ = 0.025) and a higher uninsured population rate (No Insurance Rate: coeff = -0.020, $p$ = 0.000) were negatively associated with health resource accessibility, highlighting disparities in Republican-leaning regions and among uninsured populations.

During the peak of the pandemic, new factors emerged as significant. Politically liberal regions (Democratic Rate: coeff = 0.008, $p$ = 0.005) exhibited better accessibility to health resources, whereas politically conservative regions (Republican Rate: coeff = -0.017, $p$ = 0.002) faced challenges. Surprisingly, counties with lower median income levels (Median Income: coeff = -0.013, $p$ = 0.010) and higher poverty rates (Household Below Poverty Rate: coeff = 0.001, $p$ = 0.006) demonstrated resilience in addressing the pandemic’s challenges. Conversely, counties with higher unemployment rates (Unemployed Rate: coeff = -0.003, $p$ = 0.016) experienced more pronounced resource strain. Racial disparities also became evident, with counties with higher proportions of Hispanic/Latino populations (coeff = -0.018, $p$ = 0.000) showing severe resource pressure. While no clear trend emerged for the White population, educational attainment appeared to be a positive factor. Counties with higher educational attainment levels (Bachelor Rate: coeff = 0.012, $p$ = 0.000; Education Degree Rate: coeff = 0.011, $p$ = 0.000) were associated with better access to health resources.

During the Post-Peak recovery period, although the overall impact diminished, certain factors remained significant. Educational attainment continued to play a significant role, with the Bachelor Rate (coeff = 0.002, $p$ = 0.000) and Education Degree Rate (coeff = 0.001, $p$ = 0.000) exhibiting positive associations, albeit weaker compared to the pandemic peak. Surprisingly, dissatisfaction with health resource accessibility was observed among both Democrats and Republicans, with Democrats expressing more pronounced negative perceptions. Economic factors also gained prominence during recovery. Counties with higher median income levels (Median Income: coeff = 0.002, $p$ = 0.005) and lower uninsured rates (No Insurance Rate: coeff = -0.005, $p$ = 0.001) experienced faster recovery. Demographically, counties with higher Asian populations (coeff = 0.001, $p$ = 0.002) and middle-aged populations (Between 18 and 44 Rate: coeff = 0.003, $p$ = 0.013) recovered more quickly. Conversely, counties with higher proportions of Indian populations (Indian Rate: coeff = -0.006, $p$ = 0.009) and younger populations (Under 18 Rate: coefficient = -0.004, $p$ = 0.011) experienced slower recovery. \\

\subsection{RQ3: Did the COVID-19 pandemic exacerbate health resource accessibility as perceived by the public?}
\label{lab:RQ3}

The results presented in Tables \ref{tab:pre-peak} and \ref{tab:peak-post} highlight the significant shifts in health resource accessibility reflected by the online reviews during the peak of the pandemic and subsequent recovery period, offering valuable insights into the evolving disparities over time. Tables \ref{tab:pre-peak} demonstrates that the pandemic exacerbated pre-existing inequalities. Predominantly White communities experienced more pronounced declines in perception scores (coeff = -0.026, $p$ = 0.003), potentially due to their historically better access to health resources, which created a larger margin for potential deterioration. In contrast, Black communities exhibited greater resilience (coeff = 0.048, $p$ = 0.004), potentially reflecting their already limited access to resources before the pandemic, which left leaving less room for further decline.


Table \ref{tab:peak-post} highlights persistent inequalities in recovery are evident. Counties with higher Hispanic/Latino Rates (coeff = 0.019, $p$ = 0.000) experienced significant improvements in health resource accessibility, whereas Indian communities (coeff = -0.022, $p$ = 0.018) faced slower recovery, highlighting ongoing challenges for certain demographic groups. The No Insurance Rate (coeff = 0.018, $p$ = 0.000) was positively associated with delayed recovery, emphasizing the critical role of insurance coverage in improving access to health resources. Notably, political affiliations also demonstrated influence during the recovery phase. Counties with higher Republican Rates (coeff = 0.017, $p$ = 0.014) showed relatively greater improvements, while those with higher Democratic Rates (coeff = -0.010, $p$ = 0.047) experienced less favorable outcomes, suggesting nuanced differences in recovery trajectories linked to political demographics. Additionally, educational attainment, reflected in the negative associations of the Bachelor Rate (coeff = -0.010, $p$ = 0.001) and Education Degree Rate (coeff = -0.010, $p$ = 0.000), indicates that counties with lower education levels faced prolonged barriers to resource accessibility during the recovery phase. 

These findings demonstrate that the pandemic not only intensified existing disparities but also resulted in uneven recovery trajectories, shaped by a complex interplay of demographic, socioeconomic, and political factors.

\section{Discussion}

The study highlights clear disparities in public perceptions of health resource accessibility across various socioeconomic and demographic communities in the United States. Several key observations and practical implications are listed below. 

\subsection{Implications}% in Health Resource Accessibility}

\textbf{Geographic disparities.} The geographic disparities of public perception revealed through county-level analysis and Moran's I statistics indicate that health resource availability could vary significantly across regions in the United States, with notable shifts during the pandemic. Pre-Pandemic, Western states experienced more pronounced negative perceptions, while the Eastern regions displayed a mix of perceptions. The Peak-Pandemic period amplified these disparities, with Western states, particularly California, facing severe shortages. The post-pandemic period saw a return to more neutral perceptions. These observations underscore the importance of regional strategies in addressing health resource disparities. Policies should prioritize areas with historically negative perceptions, such as certain Western counties, ensuring equitable access during routine healthcare delivery and emergencies. 

\begin{table}[htbp]
\centering
\setlength{\tabcolsep}{4pt}
\caption{Changes in PLS regression coefficients between Pre-Pandemic and Peak-Pandemic periods (Peak-Pandemic minus Pre-Pandemic).}
\captionsetup{justification=centering} 
\label{tab:pre-peak}
\small
\resizebox{0.45\textwidth}{!}{
\begin{tabular}{l r @{\hspace{6pt}} r @{\hspace{6pt}} r}
\toprule
\textbf{Variable} & \textbf{Coeffs} & \textbf{P-Value} & \textbf{Std. err.} \\ \midrule
Democratic Rate & -0.021 & 0.963 & 0.045 \\
Republican Rate & 0.006 & 0.937 & 0.046 \\
Total Population & 0.010 & 0.646 & 0.046 \\
Median Income & -0.001 & 0.056 & 0.045 \\
GINI & -0.048 & 0.075 & 0.044 \\
No Insurance Rate & -0.012 & 0.777 & 0.047 \\
Household Below Poverty Rate & 0.026 & 0.090 & 0.045 \\
HISPANIC LATINO Rate & 0.020 & 0.282 & 0.045 \\
White Rate & -0.026 & 0.003** & 0.045 \\
Black Rate & 0.048 & 0.004** & 0.045 \\
Indian Rate & 0.007 & 0.604 & 0.045 \\
Asian Rate & -0.013 & 0.383 & 0.046 \\
Under 18 Rate & -0.001 & 0.221 & 0.044 \\
Between 18 and 44 Rate & -0.016 & 0.754 & 0.045 \\
Between 45 and 64 Rate & 0.001 & 0.391 & 0.047 \\
Over 65 Rate & 0.016 & 0.325 & 0.044 \\
Male Rate & 0.012 & 0.252 & 0.045 \\
Bachelor Rate & 0.010 & 0.073 & 0.043 \\
Education Degree Rate & 0.040 & 0.060 & 0.044 \\
Population Density & 0.471 & 0.390 & 0.044 \\
Unemployed Rate & -0.014 & 0.321 & 0.044 \\ \midrule
\multicolumn{4}{l}{\textbf{Model Goodness-of-fit:} $R^2$: 0.069, RMSE: 0.207} \\
\bottomrule
\end{tabular}
}
\end{table}

\textbf{Socioeconomic and demongraphic factors.} The study’s PLS regression results confirm that socioeconomic and demongaphic characteristics play a pivotal role in shaping health resource accessibility. Pre-Pandemic advantages observed in predominantly White and insured communities underscore inequities favoring wealthier, better-educated populations. During the pandemic, compounded challenges in communities with higher poverty rates, lower median incomes, and significant Hispanic/Latino or Asian populations reflected a deepening of these inequities. Post-pandemic, the persistence of disparities in uninsured and less-educated communities further underscores the entrenched nature of these challenges.

Policy measures should address these socioeconomic determinants by expanding health resource access for uninsured populations and investing in education to enhance long-term resilience, which aligns well with prior studies' findings \cite{khairat2019advancing}. For instance, enhancing community health education and subsidizing insurance coverage could help alleviate disparities and improve resource distribution during future crises. The disparities observed in Hispanic/Latino and Indian communities also suggest the need for culturally tailored interventions to bridge gaps in accessibility.

\begin{table}[htbp]
\centering
\setlength{\tabcolsep}{4pt}
\caption{Changes in PLS regression coefficients between Peak-Pandemic and Post-Peak periods (Post-Peak minus Peak-Pandemic).}
\captionsetup{justification=centering} 
\label{tab:peak-post}
\small
\resizebox{0.45\textwidth}{!}{
\begin{tabular}{l r @{\hspace{6pt}} r @{\hspace{6pt}} r}
\toprule
\textbf{Variable} & \textbf{Coeffs} & \textbf{P-Value} & \textbf{Std. err.} \\ \midrule
Democratic Rate & -0.010 & 0.047* & 0.044 \\
Republican Rate & 0.017 & 0.014* & 0.043 \\
Total Population & -0.010 & 0.904 & 0.045 \\
Median Income & 0.015 & 0.091 & 0.044 \\
GINI & 0.017 & 0.517 & 0.045 \\
No Insurance Rate & 0.018 & 0.000** & 0.046 \\
Household Below Poverty Rate & -0.001 & 0.023* & 0.043 \\
HISPANIC LATINO Rate & 0.019 & 0.000** & 0.045 \\
White Rate & -0.002 & 0.584 & 0.046 \\
Black Rate & 0.005 & 0.351 & 0.045 \\
Indian Rate & -0.022 & 0.018* & 0.045 \\
Asian Rate & 0.001 & 0.263 & 0.045 \\
Under 18 Rate & -0.007 & 0.004** & 0.046 \\
Between 18 and 44 Rate & 0.013 & 0.340 & 0.046 \\
Between 45 and 64 Rate & 0.009 & 0.058 & 0.044 \\
Over 65 Rate & -0.013 & 0.062 & 0.045 \\
Male Rate & 0.011 & 0.129 & 0.046 \\
Bachelor Rate & -0.010 & 0.001** & 0.044 \\
Education Degree Rate & -0.010 & 0.000** & 0.044 \\
Population Density & -0.001 & 0.643 & 0.046 \\
Unemployed Rate & 0.005 & 0.060 & 0.045 \\ \bottomrule
\multicolumn{4}{l}{\textbf{Model Goodness-of-fit:} $R^2$: 0.119, RMSE: 0.127} \\
\bottomrule
\end{tabular}
}
\end{table}

\textbf{Pandemic’s impact.} The analysis of temporal changes in perception scores reveals that the pandemic exacerbated pre-existing health resource disparities. Prior research also shows that public health crises could disproportionately affect different demographic communities \cite{connor2020health}. In our study, White communities experienced sharper declines in perception scores during the pandemic, potentially due to previously better access that was strained under increased demand. Conversely, the resilience of Black communities suggests a unique dynamic where limited access Pre-Pandemic left less room for perceived deterioration. The role of income inequality, as evidenced by the GINI coefficient, further illustrates how structural disparities became more pronounced during the pandemic.

Post-pandemic recovery trajectories reflect ongoing inequities, with some communities, such as Hispanic/Latino populations, showing notable improvements, while others, like Indian communities, experienced slower recovery. These findings emphasize the pandemic’s role in exacerbating inequalities and highlighting the need for tailored interventions. Policies aimed at bolstering resource distribution for underserved communities and addressing income inequality could play a transformative role in achieving equitable recovery.

\subsection{Opportunities for Future Work}

This study highlights several promising directions for future research. The first opportunity lies in improving model classification performance. While the current ModernBERT model has demonstrated satisfactory performance in detecting perceptions, future research could focus on leveraging more advanced models. For instance, using large language models (LLMs) like LLaMA could enable the identification of more fine-grained perceptions from online reviews. Additionally, expanding the datasets for training and testing through a more comprehensive annotation process could enhance the robustness of the model.

A second potential research direction involves data fusion. Social media platforms often represent specific subsets of users, and as noted in prior studies, social media data can overrepresent certain demographics such as educated or urban populations \cite{wang2019demographic}. For example, as shown in Figure~\ref{fig:sentiment_maps}, very few messages have been posted from less-represented areas. To address these limitations, future work could incorporate reviews from other platforms, such as Yelp or TripAdvisor, to explore how web data might complement traditional surveys. Such an approach could yield a broader and more representative understanding of public perceptions.


\section{Conclusions}
This study demonstrates the value of crowdsourced data from platforms like Google Maps reviews in identifying disparities in public perceptions of health resource accessibility. During the COVID-19 pandemic, these disparities intensified at the peak and showed improvement in the Post-Peak period, shaped primarily by socioeconomic and demographic factors. Geospatial and regression analyses revealed more positive perceptions among White, insured, wealthy, and educated communities, underscoring a persistent equity gap. These findings emphasize the urgency for targeted policies that address structural inequities and ensure equitable access to healthcare resources. By adopting data-driven strategies and collaborative partnerships, stakeholders can foster more inclusive, resilient communities prepared to withstand future public health challenges.

\clearpage

%% The file named.bst is a bibliography style file for BibTeX 0.99c
\bibliographystyle{named}
\bibliography{main}

\clearpage

%% If your work has an appendix, this is the place to put it.
\appendix

\section{Appendix}

\subsection{Classification performance}
\label{sec:classification}
Figure~\ref{fig:classfiers} shows the classification results by different text classification models.

\begin{figure}[htbp]
    \centering
    \includegraphics[width=0.5\textwidth]{classifiers.pdf}
    \vspace{-5pt}
    \caption{Classification performance of candidate models: (a) Precision, (b) Recall, (c) F1-score, and (d) Training and testing accuracy.}
    \label{fig:classfiers}
\end{figure}

\subsection{Permutation test method in PLS regression}
The key equations of Permutation tests are:
\label{sec:pls}

\begin{equation}
\begin{aligned}
\Theta^{\text{obs}} & = \text{Coefficient derived from the original dataset}, \\
\Theta^{(k)} & = \text{Coefficient derived from the $k$-th permutation}, \\
p_i & = \frac{1}{n_{\text{perm}}} \sum_{k=1}^{n_{\text{perm}}} \mathbb{I}(|\Theta_i^{(k)}| \geq |\Theta_i^{\text{obs}}|), \\
SE(\Theta_i) & = \sqrt{\frac{\sum_{k=1}^{n_{\text{perm}}} (\Theta_i^{(k)} - \bar{\Theta}_i)^2}{n_{\text{perm}} - 1}}.
\end{aligned}
\end{equation}
\noindent
where $\Theta^{\text{obs}}$ represents the observed regression coefficients calculated from the original dataset using the SIMPLS algorithm, and $\Theta^{(k)}$ denotes the coefficients derived from the $k$-th permutation of the dependent variable $\mathbf{Y}$, with $n_{\text{perm}}$ representing the total number of permutations (1000 in this study). The p-value ($p_i$) for each $\Theta_i$ is computed by comparing the observed $\Theta_i^{\text{obs}}$ with the null distribution generated by permuted $\Theta_i^{(k)}$. Specifically, $p_i$ measures the proportion of permuted coefficients that are as extreme as or more extreme than the observed coefficient.

In addition, the standard error $SE(\Theta_i)$ is calculated as the standard deviation of the permuted coefficients $\Theta_i^{(k)}$ around their mean $\bar{\Theta}_i$. 

% \subsection{VIF result}
% \label{sec:vif}
% Table~\ref{tab:vif_table} presents the VIF results for the independent variables. 

% \begin{table}[htbp]
% \centering
% \caption{Variance Inflation Factor (VIF) for features in the model}
% \begin{tabular}{lcc}
% \toprule
% \textbf{Feature} & \textbf{VIF} \\ 
% \midrule
% Democratic Rate               & 41.104 \\
% Republican Rate             & 35.098 \\
% Total Population            & 1.550 \\
% Median Income               & 11.411 \\
% GINI                        & 3.548 \\
% No Insurance Rate           & 3.000 \\
% Household Below Poverty Rate & 8.924 \\
% HISPANIC LATINO Rate        & 5.771 \\
% White Rate                  & 24.026 \\
% Black Rate                  & 17.701 \\
% Indian Rate                 & 1.894 \\
% Asian Rate                  & 6.624 \\
% Under 18 Rate               & \textit{inf} \\
% Between 18 and 44 Rate      & \textit{inf} \\
% Between 45 and 64 Rate      & \textit{inf} \\
% Over 65 Rate                & \textit{inf} \\
% Male Rate                   & 2.037 \\
% Bachelor Rate               & 17.800 \\
% Education Degree Rate       & 27.328 \\
% Population Density          & 1.374 \\
% Unemployed Rate             & 3.165 \\
% \bottomrule
% \end{tabular}
% \label{tab:vif_table}
% \end{table}

\subsection{Cook's Distance method to remove extreme points}
\label{sec:validation}
For an observation \( i \), Cook's Distance is calculated as:

\begin{equation}
D_i = \frac{\sum_{j=1}^n \left( \hat{y}_j^{(-i)} - \hat{y}_j \right)^2}{p \cdot \text{MSE}}
\end{equation}
\noindent
where: \( \hat{y}_j^{(-i)} \) is the predicted value for observation \( j \) when observation \( i \) is excluded, \( \hat{y}_j \) is the predicted value using the full dataset, \( p \) is the number of model parameters (including the intercept), and \( \text{MSE} \) is the mean squared error of the regression model.

To identify influential observations, we applied a threshold defined as:

\begin{equation}
\text{Threshold} = \frac{4}{n}
\end{equation}
\noindent
where \( n \) is the total number of observations in the dataset. Observations with \( D_i > \text{Threshold} \) were flagged as extreme observations, and they will be removed.


\subsection{Representative examples}
\label{sec:annotation}
Table~\ref{tab:google-reviews} shows the Representative Google Maps reviews for public perception of health resource accessibility.

\begin{table*}[htbp]
\centering
\caption{Representative Google Maps reviews for public perception of health resource accessibility}
\label{tab:google-reviews}
\small
\begin{tabular}{p{0.45\textwidth}p{0.35\textwidth}p{0.15\textwidth}}
\toprule
\textbf{Google Maps review} & \textbf{Targeted text} & \textbf{Attitude} \\
\midrule
Just absolutely crazy! There there was no hamburger and no toilet paper, and not hardly no potato chips on the shelves. People were grabbing up stuff like this was the end of the world. & There there was no hamburger and no toilet paper... & Shortage \\
\midrule
Ran out of a lot of paper goods - toilet paper, paper towels. Ended up buying the more expensive paper towels and toilet paper, which I really could not afford, but I needed it. & Ran out of a lot of paper goods - toilet paper, paper towels. & Shortage \\
\midrule
I was able to buy toilet paper there at the height of the shortage, and they sanitized every cart before use. & I was able to buy toilet paper there at the height of the shortage... & No Shortage \\
\midrule
I was thankfully able to visit the Dollar General store location off of Donaghey, and they had plenty of paper products, as well as hand soaps. & ...and they had plenty of paper products, as well as hand soaps. & No Shortage \\
\midrule
The only major store that requires you to wear a mask to shop, but the employees near the entrance having masks hanging below their nose makes it pointless to wear a mask. & The only major store that requires you to wear a mask to shop...having masks hanging below their nose makes it pointless to wear a mask. & Unrelated \\
\midrule
Boards are in place between customers and employees at the payment counter to encourage social distancing, but then only 1 out of the 3 staff present seem to know how to wear a mask properly. & ...but then only 1 out of the 3 staff present seem to know how to wear a mask properly. & Unrelated \\
\bottomrule
\end{tabular}
\captionsetup{justification=raggedright}
\end{table*}

% \subsection{Distribution Analysis of Health Resource Availability Perceptions}
% \label{sec:perception_boxplot}

% Figure~\ref{fig:sentiment} presents the temporal progression of public perception scores regarding health resource accessibility across pandemic phases. The pre-pandemic period exhibited relatively positive perceptions. During the peak-pandemic phase, perception scores showed a marked decline. The post-peak period demonstrated a slight recovery in perceptions, though not returning to pre-pandemic levels. Notably, the variance in perception scores decreased across successive periods, suggesting a convergence in public sentiment about health resource accessibility. 

% \begin{figure}[h]
%     \centering
%     \includegraphics[width=0.45\textwidth]{sentiment_boxplot.pdf}
%     \vspace{-5pt}
%     \caption{Health resource availability trends across pandemic periods.}
%     \label{fig:sentiment}
% \end{figure}


% \subsection{Spatial Autocorrelation Analysis}
% \label{sec:morans_i}
% Figure~\ref{fig:morans_i} presents the Moran's I scatterplots illustrating the spatial autocorrelation patterns of public perception scores across three pandemic periods. The analysis reveals an evolution from initially random spatial distribution (Moran's I = 0.001, p = 0.421) in the pre-pandemic period to increasingly clustered patterns during peak (Moran's I = 0.016, p = 0.048) and post-peak periods (Moran's I = 0.022, p = 0.012). This progression of spatial autocorrelation values corresponds with the geographic clustering patterns visible in Figure~\ref{fig:sentiment_maps}.
% \begin{figure}[h]
%     \centering
%     \includegraphics[width=0.5\textwidth]{moran_scatterplots.pdf}
%     \vspace{-5pt}
%     \caption{Moran's I scatterplots across pandemic periods.}
%     \label{fig:morans_i}
% \end{figure}

\clearpage

\end{document}

