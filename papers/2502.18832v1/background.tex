\section{Safety of Kernel Extensions}
\label{sec:background}
\vspace{-5pt}
%\djw{thinking about this being ``background and motivation'' with this being the first half and 3 being the section half}

%% \subsection{eBPF}
%% eBPF is a Linux kernel subsystem that allows for safe and dynamic kernel extension.
%% Developers write programs that get compiled to eBPF bytecode before being verified by an in-kernel verifier.
%% The verifier ensures certain safety properties about the programs such as termination and memory safety.
%% The execution of eBPF programs follows an event based mechanism, where control flow will transfer to the extension when certain events happen in the system.
%% eBPF programs also have access to a set of helper functions inside the kernel, which allow them to interact with kernel state.
%% Recent work has argued that the safety guarantees of the eBPF verifier are not as strong as claimed, especially in respect to the helper function interface \cite{untenableVerification}.

%% %\begin{itemize}
%% %    \item extension program model
%% %    \item verification
%% %    \item helpers
%% %    \item eBPF problems, echo the HotOS paper
%% %\end{itemize}

%% \subsection{Rust}

%% \juowen{TODO: maybe add some content related to Rust's syntactic sugar}
%% \begin{itemize}
%%     \item language-based safety
%%     \item expressiveness
%% \end{itemize}
%% \hubertus{you have to describe what "safe Rust" vs. "regular Rust" is, you should also highlight the popularity of Rust "https://medium.com/@codilime/the-future-of-rust-characteristics-popularity-and-challenges-7de4db5ebd67"}
%% % \subsection{Rust v.s. eBPF}
%% %
%% % \begin{itemize}
%% %     \item Probably has a better place
%% %     \item Sets and examples we discussed
%% %         \begin{itemize}
%% %             \item Rust and eBPF: Small, simple programs
%% %             \item eBPF but not Rust: certain code is unsafe in Rust but safe in
%% %                 eBPF, e.g. reinterpreting bytes in a packet into another struct
%% %             \item Rust but not eBPF: Expressiveness argument
%% %         \end{itemize}
%% % \end{itemize}


%% \subsection{Threat Model for Safe Kernel Extensions}
%% Part of the success of eBPF in the industry stems from its claim of
%% safety, achieved by the in-kernel verifier.  For example, as opposed
%% to Linux kernel modules (written in C), safety in eBPF lowers the
%% barrier to entry and trust required for system administrators to
%% install an extension into a production system.  However, there are
%% ongoing debates in industry~\cite{unprivileged-ebpf} and
%% academia~\cite{untenableVerification} about the quality of eBPF's safety
%% guarantees and in what circumstances they can be relied on.  Here we
%% specify the threat model and expectations for safety of kernel
%% extensions for the scope of this paper.

%% We assume that extensions are installed by a system administrator with
%% root privileges on the system.  Extensions are written by well-meaning
%% but imperfect developers and thus may contain programming mistakes.
%% We assume that the extension is not actively malicious.  So called
%% ``guarantees'' of safety therefore consist of the best-effort catching
%% of common mistakes.  While some in the community may be interested in
%% exploring stronger notions of
%% safety~\cite{unprivileged-ebpf,jia2023}, we note that this
%% definition of safety is consistent with the current eBPF ecosystem,
%% where actively malicious extension writers can crash or hang the
%% system~\cite{untenableVerification,ebpf-stackoverflow,ebpf-termination} (e.g.,
%% via helper functions), but the verifier prevents obvious mistakes (e.g.,
%% dereferencing a NULL pointer).




% Kernel extensions allow code to be added to the kernel dynamically at
% runtime for use cases ranging from observability and monitoring to
% customization and acceleration.  Whereas safety of extensions was once
% only a concern of the research community (e.g., SPIN~\cite{spin},
% VINO~\cite{vino}, etc.), the shortening space between development,
% debugging and deployment in today's industry landscape has created a
% need for extending systems in production, where any system instability
% caused by extensions is difficult to tolerate.
%
% As a result, the past decade has seen the re-emergence of safety
% concerns for kernel extensions, embodied in Linux by the eBPF
% ecosystem.  Unlike the Linux kernel module framework, where modules
% are written in C and programmer errors (e.g., errant pointer
% dereferences) will crash the kernel, the eBPF
% framework attempts to limit the damage from extension programmer error
% by compiling extensions written in a high-level language like C or
% Rust to a bytecode representation that is checked at extension load
% time by an in-kernel verifier.  This workflow is shown in
% Figure~\ref{fig:gap}.

% Safety is important in kernel extensions
%   - kernel extensions such as eBPF and LKMs directly run in kernel with same
%     capabilities as other kernel code
%   - a bug in kernel extension code easily crash the whole kernel
% Further highlighted by "the shortening space between development..."
%   (reuse previous sentence)
% To this extent, traditional linux kernel modules do not provide the needed
%  safety guarrantee
%   - programmer errors (e.g., errant pointer dereferences) remain unchecked
%     and will crash the kernel
%   - recent effort on writing kernel modules with Rust still does not provide
%     strong enough safety
%     - safety checks can still be bypassed with unsafe rust
% eBPF is a safe kernel extension
%   - Brings out eBPF "the past decade has seen ..."

Safety is critical to OS kernel extensions---extension 
  code runs directly in kernel space, and bugs % in kernel extensions
  can directly crash a running kernel. 
%Therefore, a bug in kernel extension code could easily crash the kernel.
% The importance of safety is further highlighted by the shortening space between
%    development, debugging and deployment in today's industry landscape, which
%    has created a need for extending systems in production, and that any system
%    instability caused by extensions is difficult to tolerate.
% To this extent, Linux kernel modules, including Rust kernel modules, 
%  provide few safety guarrantee.
% LKMs are traditionally written in C with no protection against programming errors
%    (e.g., errant pointer dereferences).
% Recent effort~\cite{rust-for-linux-doc} brings Rust as a safer
%   language alternative for kernel module development.
% Unfortunately, safety guarrantee of
%    Rust kernel modules is very limited due to the allowance 
%    of using {\it unsafe} Rust and the unbounded interface 
%    to the rest of the kernel.   
%    
% Different from kernel modules, 
% eBPF checks safety of kernel extensions using static verification.
% and its proposition 
%  on program safety won it significant traction with many use cases.
The eBPF verifier checks safety properties of  
  extension programs in bytecode % (compiled from high-level languages like C and Rust)
  before loading them into the kernel to prevent
% The eBPF verifier uses a form of symbolic execution to check predefined safety properties
%  against all possible execution paths.
% If the verifier fails to verify the extensions, it rejects the
%  extension.
  programming errors such as illegal memory access. % are prevented in the first place.
% As a result, the past decade has seen the re-emergence of safe kernel
%    extensions, embodied in Linux by the eBPF ecosystem.
The verifier also checks the extension's interactions with the kernel via a 
  bounded interface, defined by eBPF {\em helper functions},
  to prevent resource leaks and deadlocks. 
%% \begin{figure}
%%     \includegraphics[width=1.0\linewidth]{figs/bpf_intro.pdf}
%%     \centering
%%     \vspace{-10pt}
%%     \caption{Overview of eBPF}
%%     \label{fig:ebpf}
%%     \vspace{-10pt}
%% \end{figure}
We summarize the safety properties targeted by eBPF as follows:
\begin{packed_itemize}
  \vspace{-5pt}
\item {\bf Memory safety.} Kernel extensions can only access pre-allocated memory
  via explicit context arguments or kernel
  interface (helper functions), preventing NULL pointer dereferencing
  and corruption of kernel data structures.
\item {\bf Type safety.} When accessing data in memory, kernel extensions must use
  the correct types of data, avoiding misinterpretation of the data
  and memory corruption.
% When invoking kernel interfaces, the
%   extension must supply arguments of the appropriate type.  This helps
%   avoid kernel crashes in helper functions.
\item {\bf Resource safety.} When acquiring kernel resources
  (e.g., locks, memory objects, etc.) through helper functions,
  kernel extensions must eventually invoke the appropriate interface
  to release the resources, preventing memory leaks or
  deadlocks that can crash or hang the kernel.
%  interface on the object before completion.  This prevents memory
%  leaks or deadlocks that can hang or crash the kernel.
\item {\bf Runtime safety.} Kernel extensions must terminate, with no
  infinite loops that can hang the kernel indefinitely.
\item {\bf No undefined behavior.} Kernel extensions must never
  exhibit undefined behavior, such as integer errors (e.g.,
  divide by zero) that cause kernel crashes. % This prevents runtime kernel crashes.
\item {\bf Stack safety.} % A unique safety requirement for kernel exntensions.
  Kernel extensions must not overflow the limited and fixed-size kernel stack,
  avoiding kernel crashes or kernel memory corruptions.\vspace{-5pt}
\end{packed_itemize}

Note that the above notion of safety in eBPF focuses on preventing 
  programming errors that may crash or hang the kernel.
Despite the discussions on whether security is a reasonable 
  target~\cite{beebox-security24,safebpf-thomas},
  in practice, eBPF and other extension frameworks (e.g., KFlex~\cite{dwivedi-sosp24}) no longer pursue
  unprivileged use cases due to its inherent limitations (see detailed discussion in \S\ref{sec:safety_model})~\cite{reconsider-unpriv-ebpf-lwn,ebpf-sec-lwn,pawan-8a03e56b253e}. %,ebpf-stackoverflow,ebpf-termination}.
Our work follows this safety model.
% lowers the barrier to entry and trust
% required %for system administrators
% to install an extension into a
% production system.  However, there have been 
%  on the quality of eBPF's
%  safety guarantees and in what circumstances they can be relied
%  on.
%\tianyin{One consensus is... }
% One consensus is that these safety guarrantees are suitable for a
%    non-adversarial safety model, but not for the case where extension programs
%    are actively malicious -- past works have demonstrated such programs can
%    still crash or hang the
%    system~\cite{untenableVerification,ebpf-stackoverflow,ebpf-termination}.
%  Rax follows the same safety model and expectations of eBPF extensions
%  (see \S\ref{sec:safety_model}).
% Here we specify the safety model and expectations for safety of kernel
% extensions for the scope of this paper.

% How eBPF verification process work
% eBPF verification process is core to its safety
% write C/Rust/etc, compile to byte code
% verification runs on bytecode
%
% An important msg here is that the compilation and verification is separated.
