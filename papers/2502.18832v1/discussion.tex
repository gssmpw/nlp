\vspace{-8pt}
\section{Discussion}
\label{sec:discussion}
\vspace{-5pt}

% \begin{itemize}
%     \item Unsafe code in rt crate
%     \item Verified Rust extension using Verus
%     \item Dynamic allocation (if we end up not doing it)
%     \item Cross-Language Attacks from NDSS 2022
%     \item Rust memory ordering in kernel (\url{https://lwn.net/SubscriberLink/967049/66bfb6f365d164aa/})
%     \item Compatibility \projname{} across kernel versions.
%     \item Memory usage for usecases with loads of programs defined in the same
%         file (e.g. KCFI)
% \end{itemize}

{\bf Verification without language-verifier gaps.} Rax currently uses
    language features of Rust to ensure safety of kernel extensions.
This approach defers the checking of some safety properties to the runtime
(e.g., termination, integer errors).
It may be possible to minimize the amount of runtime errors by incorporating
    Rust-based verification
    techniques, e.g., ensuring freedom of panics~\cite{verus,verus-sosp24,paniccheck,rvt,kani-rust}.
Certainly, push-button verification techniques that use symbolic execution
    such as PanicCheck~\cite{paniccheck} are likely to
    re-introduce the \gap{}.  We suspect that using
    verification techinques that combine proofs and programming~\cite{coq,dafny,verus-sosp24,verus},
    such as Verus for Rust may allow \projname{} to reduce
    runtime errors \textit{without} the \gap{}.

\para{Dynamic memory allocation.}
% eBPF supports dynamic allocation since 6.2
%   - dedicated all-context memory allocator
%   - as kfunc
%   - alloc and free need to match
% Rax currently does not support dynamic allocation
%   - still possible to back the Rust memory allocator using eBPF allocator
%   - allows more advanced use cases
%   - makes more components of Rust standard library, e.g. collection types
%   - usability: drop is handled automatically by compiler
eBPF has recently supported dynamic allocation~\cite{Dwivedi-958cf2e273f0}
    that allows extension programs to request memory
    from the kernel using allocation kfuncs~\cite{kfuncs}.
% The verifier requires any allocated memory to be manually freed.
% The allocation API is based on the dedicated, all-context eBPF memory
%     allocator~\cite{bpf-mempool-lwn}. %which internally uses a memory pool
    %implementation.
\projname{} currently does not support dynamic memory allocation.
We plan to integrate memory
    allocation~\cite{rust-alloc} of Rust with the eBPF all-context
    allocator~\cite{bpf-mempool-lwn}, granting \projname{} dynamic allocation.
Dynamic allocation enhances programmability of extension programs and opens
    up more advanced use cases~\cite{dwivedi-sosp24}.
It also makes more components of the Rust standard library available, notably
    the collection and smart pointer types with automatic memory management.

%\tianyin{need a discussion on the security part of the eBPF verifier, e.g.,
%    Spectre mitigation etc.}

%\ayushb{\cite{hive-ebpf-sandbox} Hive paper addition in the security part as a general guideline. Removes the need of verification itself and also allows usage of unprivileged mode.}

% Although eBPF and Rax assume a non-adversarial model for developers and admins
%   users can still trigger the program and be malicious
% eBPF and kernel tries to mitigate transient execution attacks
%   relies on compiler
%   uses verifier and JIT
% Also other works that tries to harden eBPF against transient execution attacks
% Rax currently does not support transient execution attack mitigations
% We speculate that the same set of compile-time hardening could be applied
%
% Discussed with Di:
% - retpolines (which privileged eBPF does) are easy to intergrate
% - spectre V1 and V4 mitigations (which unprivileged eBPF adds on top of
%   retpoline) are much harder for a general language like Rust, as the defense
%   builds on top of the verifier analysis.
% - If desired the more advanced techinques such as BeeBox and Swivel can be
%   applied.

\begin{comment}
\para{Extension security.}
Under the safety model of \projname{},
    attackers % unprivileged users may still be able to
    may trigger the execution of extensions (e.g.,
    via a system call) and perform exploits like return
    oriented programming~\cite{shacham-rop} and Spectre~\cite{Kocher2018spectre}.
% eBPF defends against such attacks with its verifier and JIT compiler if
%    mitigations in the kernel are enabled.
eBPF employs Spectre-v2 mitigations~\cite{bpf-spectre-popl22} that
    applies retpolines~\cite{retpoline-intel} to every indirect branch; it also
    generates instruction sequences for indirect branch tracking
    (IBT)~\cite{ibt-lwn} and control flow integrity (CFI)~\cite{cfi-lwn}.
%    mechanisms~\cite{zijlstra-589127105588,zijlstra-4f9087f16651}.
% In unprivileged mode, eBPF also implements mitigations against Spectre V1 and V4.
We have not implemented those mitigations in \projname{} yet.
We believe that the use of retpolines, IBT, and CFI by
    eBPF can be applied to \projname{} via the compiler.
% Further mitigations available in unprivileged eBPF are more challenging to apply
%     on a general language like Rust, as they build on top of the verifier
%     analysis and the restrictive eBPF bytecode.
Other techniques as in BeeBox~\cite{beebox-security24} and
    Swivel~\cite{swivel-security21} can also be used in \projname{}
    to provide mitigations against Spectre.
\end{comment}

% \jinghao{
% I think JIT-hardening is out of scope here, as it is employed against malicious
%    exntensions (tho it can be configured to apply to privileged BPF), similarly
%    spectre v1 and v4 (only applied to unprivileged eBPF).
% Spectre V2, IBT, and CFI are the mitigations that does not rely on the
%    restrictive verification and JIT process (while mitigations against spectre
%    v1 and v4 are specific to eBPF).
% They are similar in a way that if the kernel enable the configuration then the
%    JIT generates the corresponding code, so I think we should put them
%    together.
% }

\para{Kernel crate maintenance.}
The \projname{} kernel crate inevitably needs to use unsafe Rust, as it
    directly interacts with kernel functions and variables. % to create the safe kernel abstraction for \projname{} extensions.
% Unsafe Rust omits many safety checks and thus are subject to bugs.
% and thus a bug in the unsafe kernel crate
%    code could cause safety voilations in extension programs.
As a principle, unsafe Rust code must not be used for escaping
    safety checks but only when it is the last resort
    (mostly for foreign function interface, FFI).
This keeps the scope of unsafe Rust at its minimum---the \projname{}
    kernel crate only leverages unsafe Rust necessary for FFI
    interaction and contributes to about 10\% (360 lines) of kernel crate code.
As unsafe code is isolated from extension programs and managed at a central
    location by trusted maintainers, we are not particularly concerned about
    its maintainability.
% By containing all unsafe code in the kernel crate, \projname{} not only isolates
%    extension programs from unsafe code so that they can be implemented with
%    maximum safety, but also facilitates code correctness and maintenance as
%    all unsafeness is centralized and managed by trusted maintainers.
% In addition, recent work~\cite{rudra-sosp21,rust-belt} provides tooling for checking unsafe Rust code.

% What Rust-for-Linux is
% The key difference between Rust kernel modules and Rax programs
%   Safety:
%     a enforced guarrantee in Rax, is somewhat optional in Rust modules
%     Rax: the developers are trusted with good intent, but may make mistakes
%     Rust kernel module: Developers are trusted to be somehow perfect
%   Interface:
%     Rax provides a small, manageable interface that is easy to maintain and
%       reason about safety properties
%     Kernel modules provides a huge, complex interface that it's hard to
%       maintain and reason about semantics and safety properties.

\begin{comment}
\para{Rethinking kernel interface for extensions.}
\projname{} builds its safe kernel interface on top of the eBPF helper function interface;
hence, its safe interactions depend on the implementation of eBPF helper functions.
However, helpers are written in unsafe C code with no correctness
    guarantee~\cite{untenableVerification};
    bugs in helper code may impair safety guarantees of Rax.
Certainly, Rax reduces existing eBPF helper interface (\S\ref{sec:impl}).
Rax also makes assumptions on the interface.
    %has to make assumptions regarding the interface.
% However, helper functions could violate critical safety properties as they are
%    complicated kernel C code that does not enjoy checks from the Rust compiler
%    and the \projname{} runtime.
For example, Rax's termination guarantee expects a helper
    call to eventually return (\S\ref{principle:termination}).
However, this is not guaranteed---what if a spinlock helper triggers a deadlock
    and never returns to the extension?
We plan to rethink the kernel interface for Rax. A promising direction
    is to implement certain helpers in the Rax kernel crate to ensure
    Rax's assumptions to always hold.
\end{comment}

% We argue that the eBPF helper function interface may not be the correct
%    interface for safe kernel extensions and a different interface may be needed
%    to provide the required safety.
% The above example can be solved by re-implementing the spinlock helpers in the
%     \projname{} kernel crate.

%Although the interface is compact and designed for safety in eBPF, the helper
%    functions themselves are complicated kernel C code that does not enjoy
%    safety checks from the eBPF verifier or the Rust
%    compiler~\cite{untenableVerification}.
%To make it worse, the limitation from static verification forces eBPF to add
%    more helper functions to make up for expressiveness.
%We argue that the eBPF helper function interface may not be the correct
%    interface for safe kernel extensions.

%\projname{}'s termination relies on the assumption that helper functions
%    will eventually return in order to properly process timeouts amid a helper
%    invocation (\S\ref{principle:termination}).
%This poses challenges on handling terminations with helpers that may not return
%    e.g., a spinlock helper will not return if the program causes a deadlock.
%We envision that this challenge could be solved by re-implementing such helpers
%    in the \projname{} kernel crate.
    % out of the kernel~\cite{dwivedi-sosp24} (
%    or make the helper implementation aware of \projname{}'s termination
%    mechanism.

% \para{Difference from Rust kernel modules.}
% Starting from version 6.1, the Linux kernel supports implementing
%    kernel modules in Rust~\cite{rust-for-linux-doc}.
% Rust kernel modules and \projname{} differs in many aspects.
% Rust kernel modules are permitted to have \textit{unsafe} Rust code,
%    in which many safety guarantees are missing.
% In contrast, safety is strictly enforced in Rax---the extension program cannot
%    contain any \textit{unsafe} Rust code.
% At the same time, Rust kernel modules inherit the large and arbitrary interface
%     the kernel exports to the traditional C modules.
% This vast interface creates significant challenges to systematically
%    construct a Rust abstraction for safe kernel interactions.
%On the other hand, Rax interacts with the kernel via the compact and tractable
%    interface inherited from eBPF, which facilitates reasoning against semantics
%    and safety, and makes it easy to create and maintain a safe kernel
%    programming interface.
% On the other hand, Rax builds upon the kernel interface of eBPF, which
%    is compact and already designed for safety.
% These properties facilitate \projname{} to encode the existing safety rules
%    with Rust's language safety, and thus make it easy to create and maintain a
%    safe kernel programming interface.

%In this paper, we have argued to
%close the language-verifier gap by leveraging inherent safety
%properties of the Rust language.  However, this approach defers the
%checking of some safety properties to runtime (e.g., termination,
%integer errors).

% Unsafe code is required to interact with low-level kernel code/data and be
%   compliant with certain kernel ABIs
%   - Kernel helper functions requires FFI call
%   - converting unsafe C types into safe Rust types
%   - accessing kernel per-CPU definition
% All unsafe Rust code presents in the kernel crate
%   - set to be developed by trusted and experienced maintainers
%   -
% The program is always implemented by only safe Rust, thereby making the
%   program subject to all safety checks in Rust.
% Recent works like verus employs formal verification on Rust code and provides
%   further guarrantees.

    % through the FFI interface and
    % when accessing low-level kernel code and data.
%As \projname{} kernel crate serves as an interface to the kernel for the
%    extension programs, it inevitably employs unsafe Rust code.
%This is needed because interacting with low-level kernel code and data requires
%    kernel crate to access unsafe C types and invoking kernel functions through
%    the FFI interface.
%Furthermore, certain ABIs of the Linux kernel, such as accessing per-CPU data,
%    also make unsafe code a necessity.
%    -- accessing per-CPU data requires inline assembly code as well
%    as direct pointer arithmetics.
%We note that, \projname{} limits all unsafe code to the kernel crate
%    implemented by trusted and experienced maintainers.
%Extension programs must be exclusively written in safe Rust to ensure safety.
%The extension program itself is always implemented only in safe Rust, thereby
%    making the program subject to all safety checks in Rust.
% Recent work~\cite{verus} has explored formal verification
%     techniques on Rust code to provide additional safety guarantees for
%     unsafe code, beyond what is provided by the compiler.
%    which provides additional safety guarantees
%    for the correctness of unsafe code beyond these provided by the compiler.

%\jinghao{Thinking about removing this unprivileged extensions paragraph, I feel
%    that there is not much interesting we can talk about here.
%We stressed several times about our safety model, and therefore this pretty
%    much goes out of scope.}

%\new{
%\para{Unprivileged extensions:}
%There have been proposals to support untrusted, unprivileged extensions for
%    various use cases, such as system call filtering~\cite{jia2023}.
%Because of \projname{}'s design around its privileged and non-adversarial safety
%    model, it cannot support such use cases.
%Properly guaranteeing the security of unprivileged extensions is challenging, as
%    defects in the protection mechanisms now bear direr consequencies---eBPF has
%    also backed off from unprivileged extensions~\cite{ebpf-sec-lwn}.
%}
%In our safety model, extensions are
%installed by a privileged user; they are trusted but may be buggy.
%Though these scenarios will also
%suffer from the \gap{}, it is not clear that a
%language-based approach like \projname{} is an adequate solution as it
%increases the size of the trusted computing base.
%Using the Rust compiler and toolchain increases the size of
%the trusted computing base.form a larger trusted
%computing base than an in-kernel verifier.
%We note that, even with the in-kernel verifier,
%guaranteeing safety remains a
%challenge~\cite{untenableVerification,ebpf-stackoverflow,ebpf-termination}.

% \jinghao{The two paragraphs below can be removed for space}

% With these types the extension developer no longer needs to manage memory
%     manually, as the compiler will handle the lifetime of these objects and
%     drop them automatically.

% \label{discussion:versions}\para{Version Mismatches:}
% % rax fixes mismatches between compiler and verifier by removing the verifier
% % for mismatches between kernel versions, Rax's kernel crate itself does not
% %   contain kernel version-specific code
% % rax's model requires the programs to be compiled on the target kernel,
% %   and the build process generates kernel version specific bindings for the
% %   programs
% There are three sources of versioning in eBPF that may lead to issues: kernel interface, verifier, and the compiler.
% Kernel interface version problems are resolved by the use of BTF and CO-RE~\cite{bpf-core} in BPF.
% However, the versioning issues between the compiler and verifier are not resolved.
% %There will always be gaps between the compiler and verifier as their development is separate.
% %The compiler and verifier are decoupled in their development, so there will
% %    always be gaps between them.
% \projname{}'s elimination of the in-kernel verifier effectively eliminates the
%     mismatch issue between the compiler and the verifier.
% As for the problem of API changes across kernel versions, \projname{}'s kernel
%     crate does not contain any kernel-version-specific code.
% \projname{} currently requires programs to be compiled on the
%     target kernel and the build process generates Rust bindings specific to the
%     target kernel.

%In the current eBPF system, the compiler and the verifier are decoupled.
%As development proceeds, implementation details about the compiler and verifier change.
%Problems can also arise from different verifier versions.
%If an eBPF program verifies on one version, there is no guarantee that it verifies on a different version.
%Examples of this include backwards compatibility for features like loops.
%There is no way to fully resolve the issues of versioning in the eBPF system as
%    each component is a constantly moving target.

% \para{Static memory efficiency:}
% % from eval: memory is more efficient when there are more programs defined in
% %   the same program
% % Use cases like KCFI require a lot of programs, rax's ability of pack the code
% %   together without forced code duplication is more effecient (we expect)
% As discussed in \S\ref{eval:mem-footprint}, the static memory footprint of
%     \projname{} is more efficient when more programs are defined togetherc in
%     the same project.
% Recent work has proposed use cases that require potentially a large number of
%     programs~\cite{ebpf-kcfi}, where different programs share a common, central
%     logic.
% We expect \projname{} to be more effecient in static memory footprint under
%     such conditions, given its ability of packing the program and data together
%     without forced code duplication that happens in eBPF.
