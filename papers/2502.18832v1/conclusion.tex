\vspace{-8pt}
\section{Conclusion}
\label{sec:conclusion}
\vspace{-5pt}

We build Rax, a new kernel extension framework that closes the \gap{}. 
We believe that closing the gap is essential to  
    programming experience and maintainability of kernel extensions, especially
    those that embody large, complex programs for advanced features.
Rax provides a solution that allows kernel extensions to be developed and maintained
    in a high-level language, while providing desired safety guarantees 
    as the existing framework like eBPF.
% This paper identifies the fundamental \gap{} in the current Linux eBPF
%    infrastructure that causes usability challenges.
% We suggest that by using and
%    pushing safety checks to the compiler,
%    the \gap{} can be closed.
% We present \projname{}, a new, Rust-based safe kernel extension
%    framework for the Linux kernel that leverages language-based safety from
%    Rust supplemented by a light-weight runtime.
% We evaluate \projname{} on both usability and performance, demonstrating that
%   \projname{} is able to address most of the existing usability challenges
%    from eBPF without performance costs.

\vspace{-8pt}
\section*{Acknowledgement}
\vspace{-5pt}

This work was funded in part by NSF CNS-1956007, NSF
CNS-2236966, and an IBM-Illinois Discovery Accelerator Institute (IIDAI) grant.
We used Rax to develop Machine Problems (MPs) for CS 423 (Operating System Design)
at the University of Illinois Urbana-Champaign in Fall 2022, 2023, and 2024.
We thank students in CS 423 for being the beta users of Rax, which 
    provides valuable feedback for us to improve the usability of Rax.
We thank Minh Phan, Manvik Nanda, and Quan Hao Ng for their participation
    in the project.
We also thank Jiyuan Zhang, Di Jin, Hao Lin, James Bottomley, and Darko Marinov for their
    feedback and discussion.

\begin{comment}
We have used Rax to develop three projects for an undergraduate OS course at the CS
    department in our university for two semesters.
The students were asked to implement a packet filter and a kprobe program using Rax.
In total, 67 students attempted one of the two projects, and 32 correctly implemented the Rax extension programs
    that pass all of our grader tests;
65 students implemented working Rax extensions with partial functionality (earning partial credit).
As OS learners,
    students found Rax easy to use and program with a low learning curve
    over Rust programming.
With positive feedback, we are evolving the projects to larger, more complex Rax extension programs.
\end{comment}
