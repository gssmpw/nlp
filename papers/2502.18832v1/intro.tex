\section{Introduction}
\vspace{-5pt}

%% Extending the operating system (OS) functionality at runtime is an
%%     important and powerful capability.
%% Modern systems provide such capability in the form of kernel extensions and
%%     loadable kernel modules.
%% Compared to loadable kernel modules, kernel extensions generally provide such
%%     capability in a safer and more light-weighted way.

Kernel extensibility is an essential capability of modern
% The ability to extend kernel functionality at runtime has long become an
    Operating Systems (OSes).
Kernel extensions allow users with diverse needs to customize the OS without
    adding complexity to core kernel code or introducing disruptive
    kernel reboots.
% Comtemporary OSs, such as Linux, often provide such capability in the form of
%     loadable kernel modules and kernel extensions.
% Loadable kernel modules have long seen extensive uses in OS kernels.
% On Linux, more than half of the kernel source is consisted of drivers that can
%     be built as loadable kernel modules~\cite{linus-and-dirk-lwn}.
% Kernel extensions, on the other hand, are increasingly leveraged in the domains
%     of kernel tracing~\cite{bpf_tracing}, file systems~\cite{Zhong:osdi:2022},
%     networking~\cite{BMC,Electrode,DINT,Hoiland-Jorgensen:conext:2018}, and
%     security~\cite{jia2023,ebpf-kcfi}.

% LKMs are unsafe form of extension in Linux <- qualify the scope here
% LKM code are compiled in the same way as core kernel code w/ additional
%   safety checks
% LKM code has almost the same capability as core kernel code at runtime and
%   does not have fault isolation
% If a developer is not careful enough, a bug in an LKM would have the same
%   devastating consequence as one that is in the core kernel code.
In Linux, kernel extensibility has traditionally taken the form of loadable kernel
    modules. However, kernel modules are inherently {\it unsafe}---simple programming
        errors can crash the kernel.
Despite the support of safe languages like Rust~\cite{rust-for-linux-doc},
    there is no systematic support to ensure the safety of kernel modules---\textit{unsafe} Rust code % (via the \texttt{unsafe} keyword in Rust)
    is allowed in kernel modules %, opening up escape hatches.
    wherein checks to prevent errors are non-existent.
%    \new{
%        In the last couple of years, to mitigate the weaknesses associated
%        with the use of unsafe language and add more safety to the kernel
%        modules, Rust \cite{rust-for-linux-doc} first appeared in Linux v6.1.
%        However, the goal
%        of safety still needs to be solved since \textit{unsafe} Rust code
%        is allowed in Rust kernel modules, wherein checks to prevent
%        errors are non-existent. Thus, adding a potentially safe
%        language without actually using its safety features
%        led us back to square one.
%        Kernel modules are mighty kernel extensions, allowing us
%        to write arbitrary code. However, simultaneously, the vast
%        arbitrary interface exposed to kernel modules creates significant
%        challenges to systematically providing a
%        safe language kernel abstraction \cite{Miller-hotos19}.
%    }
% Kernel modules are written in unsafe languages like C % in the same way as the core kernel code,
%    with no safety guarantee or fault isolation.
% bugs in kernel modules are as devastating as those in core kernel code.
% At runtime, kernel modules code generally has the
%    same capability as the core kernel code, usually without fault isolation.
% These properties colletively make bugs in loadable kernel modules as devastating
%    as those in the core kernel code, if module developers ever make mistakes.
% To make kernel modules safer,
% Linux supports Rust kernel modules
% leverage the safety from language
% Several problems the cause it to not harvest the safety guarrantee at its
%   entirety:
%   - Unsafe Rust that does not enjoy the safety checks is still allowed in
%     kernel module code, acting as escape hatches (not enforced, merely a
%     guideline)
%   - LKMs have a huge interface against the kernel, which makes it hard to
%     implement safe abstraction for all of them
% At the same time, the Rust-for-Linux project haven't seen a practical use
%   in the kernel source tree except some reference drivers that duplicates
%   functionality from the C driver.
% Linux recently added support for kernel modules written in Rust~\cite{rust-for-linux-doc},
%    a language known for its safety promises.
% Aiming to address this safety problem, Linux has added
%    support~\cite{rust-for-linux-doc} for developing kernel modules using
%    Rust~\cite{rust-lang}, a language known for its safety promises.
%The Rust compiler performs static analysis on extension source code and flags
%    unsafe code with undefined behavior.
%Unfortunately, Rust kernel modules cannot harvest the safety guarantee
%    from the language in its entirety.
% Safety is merely a guideline rather than an enforcement for Rust kernel modules.
% At the same time, kernel modules have a huge interface against the core kernel.
%     which, in the case of Rust kernel modules, is supposed to be
%     wrapped in \textit{safe} Rust abstractions.
Moreover, the vast, arbitrary interface exposed to kernel modules creates
    significant challenges in providing a safe Rust kernel
    abstraction to enforce safe Rust code~\cite{Miller-hotos19,rust-module-dev-quit-lwn}.
% on the interface?}
% Until this point, Rust kernel modules have not seen a practical use case in
%    Linux except certain reference drivers~\cite{rust-ref-driver}.

% Kernel extension has become an essential component of modern OSs and
% been increasingly leveraged in the domains of kernel tracing~\cite{bpf_tracing},
% file systems~\cite{Zhong:osdi:2022},
% networking~\cite{BMC,Electrode,DINT,Hoiland-Jorgensen:conext:2018}, and
% security~\cite{jia2023,ebpf-kcfi}. Crucially, many
% kernel extension use cases involve extending {\em production} systems,
% which has driven the industry towards {\em safe kernel extensions},
% where programmer errors in an extension should be prevented from
% crashing or hanging the (production) kernel.

Recently, eBPF extensions have gained significant traction and
    become the {\it de facto}
    kernel extensions~\cite{cilium-docs,ebpf-windows}.
Core to eBPF's value proposition is a promise of {\it safety} of kernel extensions,
    enforced by the in-kernel verifier.
The verifier
% In order to achieve a safe extension system, the most popular current
    statically analyzes extension programs in eBPF bytecode,
    compiled from high-level languages (C and Rust).
It performs symbolic execution against every possible code path in the bytecode
    to check safety properties (e.g.,
    memory safety, type safety, termination, etc).
% The de-facto safe kernel
% extension ecosystem is built around the Extended Berkeley Packet
% Filter (eBPF).  eBPF extension programs are written in a high-level
% language, like C or Rust, that is compiled into the eBPF bytecode and
% loaded into the kernel.  Upon load, an in-kernel verifier statically
The kernel rejects any extension the verifier fails to verify.
% analyzes every possible program path through the bytecode to check the
% safety properties and rejects any program that violates them.
Today, eBPF extensions have evolved far beyond simple packet filters (its original use cases~\cite{pf,bsdpf})
    and are increasingly being
    used to construct large, complex programs that customize storage~\cite{BMC,Zhong:osdi:2022,fetchbpf,lambda-io},
    networking~\cite{Electrode,DINT},
    CPU scheduling~\cite{ghost-scheduler,ghost-scheduler-lpc,sched-ext}, etc. % and memory management~\cite{lpc-24-bpfmm,ebpf-mm}.

% This promise of safety eBPF provides causes it to be increasingly leveraged in
%    the domains of kernel tracing~\cite{bpf_tracing},
%    file systems~\cite{Zhong:osdi:2022},
%    networking~\cite{BMC,Electrode,DINT,Hoiland-Jorgensen:conext:2018}, and
%    security~\cite{jia2023,ebpf-kcfi}.

% Kernel extension is a powerful technique to improve performance and
%     observability in operating systems.
% eBPF is the current way of safely extending the Linux kernel.
% The major use cases of eBPF today include networking, tracing and
%     observability, and security.
% eBPF has also attracted attention from the research community as
%     a tool to speed up applications by reducing kernel crossings \cite{BMC,Electrode,bpfxrp}.
%% The core value proposition of eBPF is its promise of safety.
%% This promise makes eBPF superior to loadable kernel modules, which are
%%     generally implemented in unsafe C code.
%% The safety promise of eBPF is provided by its in-kernel verifier:
%% when the program is loaded into the kernel, the verifier employs a form of symbolic
%%     execution on the program bytecode and checks against a list of safety
%%     properties, including memory safety, safe kernel resource management, and
%%     program termination. \milo{We never mention anything about bytecode and it just shows up here}
%% Programs the verifier deems unsafe would be denied from loading.

% A main attractor of eBPF is its use of static verification to ensure extension safety.
% This allows eBPF programs to be deployed in production to understand system performance,
%     as well providing a programmable way to extend the kernel.
However, we observe that eBPF's static verifier introduces
    significant usability issues, %\ayushb{Can we make such a general statement ? We observe that current eBPF static verifier has usability issues; in general using a good static verifier which does not put an excess overhead on the user can make things better, but writing and maintaining one is a herculean task},
    making eBPF extensions
    hard to develop and maintain, especially for large, complex programs.
For example, the eBPF verifier often incorrectly
    rejects {\it safe} extension code due to fundamental limitations of static verification
    and defects in the verifier implementation.
When such false rejections happen,
    developers have no choice but to refactor or rewrite extension programs
    in ways that ``please'' the verifier.
Such efforts range from breaking an extension program into multiple small ones,
    nudging compilers to generate verifier-friendly code, tweaking code
    to assist verification, etc (see \S\ref{sec:motivation}).
Some of the efforts also involve reinventing wheels and hacking eBPF bytecode,
    which creates significant cognitive overheads and
    makes maintenance difficult.

% places
% unnecessary constraints on the expressiveness of extension programs
% and also makes them harder to develop and use.
% Through analysis of
% commit messages from large eBPF open source projects like
% Cilium~\cite{cilium} and experiences reported from new systems
% utilizing eBPF like BMC~\cite{BMC}, we compile 75 concrete examples of
% developers struggling unnecessarily against the verifier and
% categorize them by the workarounds employed, ranging from splitting
% eBPF programs to hand-written bytecode sequences.
% Cilium~\cite{cilium} has to use hand-written eBPF bytecode in order to
%    workaround the verification checks.
% The BMC~\cite{BMC} program is forced to be split into seven programs, where
%    ideally only two programs are needed, and needs to incorporate additional
%    condition checks that are irrelevant to safety and correctness.

%% eBPF projects often contain a large number of workaround fixes to bypass the
%%     verifier.
%% Developers have to heavily massage their code - splitting the program
%%     into small pieces and implementing cumbersome conditions, in order to
%%     make the code accepted.
%% Cilium~\cite{cilium} has to use hand-written eBPF bytecode in order to
%%     workaround the verification checks.
%% The BMC~\cite{BMC} program is forced to be split into seven
%%     programs, where ideally only two programs are needed, and needs to
%%     incorporate additional condition checks that are irrelevant to safety and
%%     correctness.

% and complexities that arise from the gap between the programmer, compiler, and verifier.
We argue that these usability issues are rooted in the gap
    between the programming language and the eBPF verifier, which we term
    the {\em \gap{}}.
When writing eBPF programs, developers
    interact with the high-level language and naturally obey a
    {\em language contract} to align with the safety requirements of the language.
The compiler also adheres to the language contract.
Unfortunately, the verifier is not part of the language contract and
has different expectations.  As a result, verifier rejections may be
surprising; the feedback (verifier log) is at the bytecode level and is
    hard to map to source code.
As a result,
    developing eBPF extensions requires not only a deep knowledge
    of the high-level language and safety properties of kernel extensions
but also a deep understanding of implementation details and quirks of
    the verifier.
%\ayushb{"Various limitations on the programs being written" - Adding another point to solving \gap{} about increasing the set of programs also being accepted, which were safe but were rejected, since the limitations on the programs have been removed.}

Unfortunately, recent efforts to improve the eBPF verifier (e.g., via testing and verification~\cite{formal-verifier-ebpf,lpc-24-agni})
    cannot fundamentally
    close the \gap{} because
    (1) they do not address scalability issues of symbolic execution, so
        extension programs have to ill-fit the verifier's internal limits,
    and
    (2) it is unlikely that the language compiler (e.g., LLVM) and the eBPF verifier
        are always in synchronization, given their independent developments.
% \tianyin{For the same reasons, recent work that proposes to simplify the verifier (e.g., KFlex~\cite{lpc-24-kflex})
%    may mitigate the \gap{} but is hard to fundamentally close it.}
Recent efforts to improve extension expressiveness via techniques
    like software fault isolation (e.g., KFlex~\cite{dwivedi-sosp24})
% While KFlex can potentially reduce the verifier burden by skipping state tracking on heap pointers,
largely inherit the eBPF verifier and, therefore, do not address the \gap{}.
% \tianyin{need to comment on PCC perhaps}

% Throughout the recent years, there had been various efforts trying to improve
%    the eBPF verification process.
% The Linux kernel contributors have upstreamed large number of patches to the
%    current verifier in hope to fix its known defects.
% At the same time, academia has proposed other techniques that ranges from
%    formal methods including verifiying the eBPF verifier
%    itself~\cite{formal-verifier-ebpf,lpc-24-agni} as well as applying
%    proof-carrying code to the verification
%    process~\cite{proof-carrying-verifier,lpc-24-lazy-refinement}, to software
%    fault isolation (SFI) techniques to reduce the verification
%    burden~\cite{lpc-24-kflex,sandbpf}.
% However, these efforts fails to fundementally solve the problem of the \gap{},
%    which lies between the programmer and the verifier, given that none of them
%    avoids using a verifier completely.

%% This is, unfortunately, not enough for eBPF programming, as the code deemed
%%     safe by the compiler may not be safe from the verifier's perspective.
%% For example, statically unbounded loops are generally not regarded as a problem
%%     in high-level languages but will be rejected by the verifier because it
%%     cannot reason about the termination guarantee of programs.
%% One must know the requirements of the eBPF verifier well in order to
%%     avoid the hidden traps that cause verifier rejection.

% In this paper we present a novel extension system for the Linux kernel called \projname{}.
% \projname{} uses a combination of the Rust compiler and dynamic mechanisms
%     to ensure safety properties about kernel extensions.
% At the same time, the use of Rust closes the gap between the programmer and verifier,
%     as well as alleviates the expressiveness and usability challenges associated with
%     the verifier.

We present Rax, a new kernel extension framework that
    closes the \gap{} and effectively improves the usability of kernel extensions,
    in terms of programming experience and maintainability.
Rax builds up safety guarantees for kernel extensions based on safe language features.
With Rax, safety properties are checked by the language compiler within the language contract.
Rax drops the need for an extra verification layer and closes
    the \gap{}.
We choose Rust as the safe language, as it is already supported by Linux~\cite{rust-for-linux-doc}
    and offers desired language-based safety for practical systems programming~\cite{Miller-hotos19,rustSystems,theseus,redleaf}.

% and use them in place of eBPF programs.

%% by using a safe, high-level language that directly imple-
%% ments the required safety properties of kernel extensions
%% combined with a light-weight runtime,
%
%% In this paper, we make the observation that both the enhanced usability for
%%     developers and the needed safety properties for the kernel can be achieved
%%     at the same time via a safe language.
%% Under such a model, safety properties from the programming language and its
%%     compiler replace that of the in-kernel verifier.
%% This allows elimination of the verifier, which effectively closes the gap
%%     between itself and the programming language.
%% Developers of the extension programs only need to spend effort on fitting the
%%     their code in the safety requirements of the language.
%
% We present \projname{}, a new, Rust-based kernel extension framework that

%% \projname{} exploits the safety promise of Rust and combines it with
%%     light-weight runtime safety checks to ensure the programs can execute
%%     safely in the kernel.

%% Hoisting safety checks from the verifier into the compiler closes the existing
%%     gap in Linux eBPF that is the root of its usability problem.

% Technical contributions of Rax
% - Small, Safe kernel programming interface via static safety features from Rust
%   - wraps unsafe FFI kernel interface with safe Rust wrappers and bindings
%   - requires the program itself to be implemented only in safe code
% - Runtime safety checks
%   - Stack unwinding and resource cleanup for panic-exit
%   - Stack usage checks
%   - Termination enforcement

Rax kernel extensions are strictly written in {\it safe} Rust with selected
    features (unsafe Rust
    code is forbidden in Rax extensions).
% \ayushb{"Need a good way to introduce Rust here" - Rust kind of comes suddenly here,
% we have made an argument towards using a safe language until now to bridge the language-verifier gap,
% we can ease into why Rust is the ideal choice here, few supporting points might be -
% it's already adopted in kernel, eBPF programs can be written in Rust due to presence of Rust library and supports almost all the safety guarantees that we are looking for.}
Rax transforms the promises of Rust into safety guarantees for
    extension programs with the following endeavors.
% \ayushb{The kernel modules in Rust info, added in the 2nd paragraph of the intro, can be moved here and then we can continue building into the fact that how we move towards a far better solution, kind of brings the story together, since we introduce why we are thinking of Rust here and why current Rust solution is not enough and what we will do to make it better.}
% Unlike Rust kernel modules, safety is strictly enforced by \projname{},
%    forbidding  code in Rax extensions.
First, to enable Rax extensions to be written entirely in safe Rust
    in the context of kernel extension,
    Rax develops a kernel crate and offers a safe kernel interface that
    wraps the existing eBPF kernel interface (eBPF
    helper functions and data types) with safe Rust wrappers and bindings.
The kernel crate enforces memory safety, extends type safety,
    and ensures safe interactions with the kernel.
% \tianyin{moreover, it enforces ... [[one sentence for 5.1]]}
\projname{} further enforces only safe Rust features through its compiler toolchains.

%    \projname{} exposes
% that ensures
%    memory safety, type safety, and safe management of kernel resources.
% The interface
%   wraps 1) unsafe foreign function interface (FFI) to the compact eBPF kernel
%    helper functions with safe Rust wrappers and 2) the kernel data types
%    with Rust bindings,
%    implemented in purely safe Rust.

Moreover, \projname{} employs a lightweight extralingual runtime for
    safety properties that are hard to guarantee by static analysis.
% provides a light-weight runtime safety mechanism
%    by extending from the existing runtime checks (panics) in Rust.
Specifically, \projname{} supports safe stack unwinding and resource cleanup
    upon Rust panics at runtime.
\projname{} also checks kernel stack usage and uses
    watchdog timers to ensure termination with a safe mechanism.
The Rax runtime is engineered with minimal overhead to achieve
    high performance. % kernel extensions.
% We use our system to reimplement the BPF Memcached Cache~\cite{BMC} and achieve
%     comparable performance to the eBPF implementation, while avoiding the
%     usability challenges of eBPF.

We evaluate \projname{} on both its usability and performance.
We show that by closing the \gap{} and offering Rust's rich
    built-in functionality, Rax effectively rules out the usability
    issues in eBPF.
We further evaluate the usability by implementing the BPF Memcached Cache (BMC)~\cite{BMC}
    (a complex, performance-critical program written in eBPF)
    using Rax and show that Rax leads to cleaner, simpler extension code.
% Our classroom experience of Rax with undergraduate students
% \new{(the majority of
%    which has little kernel programming experience prior to the course)}
%    confirms its usability.
We also conduct extensive macro and micro benchmarks.
Rax extensions deliver the same level of performance as eBPF extensions---the
    enhanced usability does not come with a performance penalty.

% \ayushb{Can we comment on the fact that proposed safety guarantees will/may be stronger than what eBPF currently provides (or maybe even KFlex)}

% We implement , a complex and
%    performance-critical eBPF program representing a real world use case, in
%    \projname{}, and demonstrate that the enhanced usability of \projname{}
%    does not come at a cost of performance.
% \projname{}-BMC is able to achieve a throughput of 1.99M Memcached requests
%    per second, which is slightly higher than that of eBPF at 1.91M requests per second.

\vspace{-2.5pt}
\para{Limitations.} %Rax currently only supports (safe) Rust and thus limits
    %the language choices.
% \projname{} chooses to leverage the language-based safety from Rust.
% This choice enables \projname{} to close the \gap{} for kernel extensions
%    without losing safety.
\projname{}'s design comes with tradeoffs.
To close the \gap{} and improve usability, Rax requires kernel extensions to be written in Rust,
    though its design principles apply to other safe languages.
Rax brings the Rust toolchain into the Trusted Computing Base (TCB)
    and adds additional runtime complexity.
Note that Rax extensions and eBPF extensions can co-exist---Rax and eBPF
    represent different tradeoffs.
Rax targets large, complex kernel extensions for which usability
    and maintainability are critical.
%Rax includes the Rust compiler in the Trusted Computing Base (TCB), which
%    is several times larger than the eBPF verifier and thus likely contains more bugs.
% \mvle{this point is a bit hard to follow without more context later on. I'm also not sure this is a limitation.}
% Additionally, Rax's runtime termination mechanism
%    uses hard interrupts and thus does not apply to extensions
%    that must run in hard or non-maskable interrupts; such use cases
%    are not targets of Rax extensions (\S\ref{principle:termination}).
% \mvle{I also feel the TCB issue should be embedded in the text rather than have
% it's own nip. i don't feel like it's a limitation but a design tradeoff to get safety and flexibility at the cost of TCB.
% On that note, i think it would help to have a very succint para or a few sentences that lay out the benefit/advantage
% of Rax and the tradeoffs. e.g., Rax provides flexibility, closes the language-verifier gap, better safety, etc. On the other
% hand, Rax has a larger TCB and adds some additional runtime complexity. But we think the tradeoffs are justified because blah.
% Alot of this text is already there in intro, just need to bring it together in one spot.}

\para{Contributions.} We make the following main contributions:
\begin{packed_itemize}
    \item A discussion of the \gap{}
      and its impact on the usability and maintainability of safe kernel extensions;
    \item Design and implementation of the \projname{} kernel extension framework,
        which closes the \gap{} by realizing safe kernel extensions
        upon language-based safety,
        together with efficient runtime techniques;
%    \item A qualitative evaluation of the usability of \projname{} and
%      a performance evaluation that quantifies overheads.
    \item The Rax project is at \url{https://github.com/rex-rs}.
\end{packed_itemize}
