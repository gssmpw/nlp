%\documentclass[sigplan,screen,nonacm]{acmart}
%\documentclass[sigplan,10pt]{acmart}
%\settopmatter{printfolios=true,printccs=false,printacmref=false}
%\renewcommand\footnotetextcopyrightpermission[1]{}
%\documentclass[sigplan,twocolumn,review,anonymous,nonacm]{acmart}
%\acmSubmissionID{366}
%\renewcommand\footnotetextcopyrightpermission[1]{}
\documentclass[letterpaper,twocolumn,10pt]{article}

%
\setlength\unitlength{1mm}
\newcommand{\twodots}{\mathinner {\ldotp \ldotp}}
% bb font symbols
\newcommand{\Rho}{\mathrm{P}}
\newcommand{\Tau}{\mathrm{T}}

\newfont{\bbb}{msbm10 scaled 700}
\newcommand{\CCC}{\mbox{\bbb C}}

\newfont{\bb}{msbm10 scaled 1100}
\newcommand{\CC}{\mbox{\bb C}}
\newcommand{\PP}{\mbox{\bb P}}
\newcommand{\RR}{\mbox{\bb R}}
\newcommand{\QQ}{\mbox{\bb Q}}
\newcommand{\ZZ}{\mbox{\bb Z}}
\newcommand{\FF}{\mbox{\bb F}}
\newcommand{\GG}{\mbox{\bb G}}
\newcommand{\EE}{\mbox{\bb E}}
\newcommand{\NN}{\mbox{\bb N}}
\newcommand{\KK}{\mbox{\bb K}}
\newcommand{\HH}{\mbox{\bb H}}
\newcommand{\SSS}{\mbox{\bb S}}
\newcommand{\UU}{\mbox{\bb U}}
\newcommand{\VV}{\mbox{\bb V}}


\newcommand{\yy}{\mathbbm{y}}
\newcommand{\xx}{\mathbbm{x}}
\newcommand{\zz}{\mathbbm{z}}
\newcommand{\sss}{\mathbbm{s}}
\newcommand{\rr}{\mathbbm{r}}
\newcommand{\pp}{\mathbbm{p}}
\newcommand{\qq}{\mathbbm{q}}
\newcommand{\ww}{\mathbbm{w}}
\newcommand{\hh}{\mathbbm{h}}
\newcommand{\vvv}{\mathbbm{v}}

% Vectors

\newcommand{\av}{{\bf a}}
\newcommand{\bv}{{\bf b}}
\newcommand{\cv}{{\bf c}}
\newcommand{\dv}{{\bf d}}
\newcommand{\ev}{{\bf e}}
\newcommand{\fv}{{\bf f}}
\newcommand{\gv}{{\bf g}}
\newcommand{\hv}{{\bf h}}
\newcommand{\iv}{{\bf i}}
\newcommand{\jv}{{\bf j}}
\newcommand{\kv}{{\bf k}}
\newcommand{\lv}{{\bf l}}
\newcommand{\mv}{{\bf m}}
\newcommand{\nv}{{\bf n}}
\newcommand{\ov}{{\bf o}}
\newcommand{\pv}{{\bf p}}
\newcommand{\qv}{{\bf q}}
\newcommand{\rv}{{\bf r}}
\newcommand{\sv}{{\bf s}}
\newcommand{\tv}{{\bf t}}
\newcommand{\uv}{{\bf u}}
\newcommand{\wv}{{\bf w}}
\newcommand{\vv}{{\bf v}}
\newcommand{\xv}{{\bf x}}
\newcommand{\yv}{{\bf y}}
\newcommand{\zv}{{\bf z}}
\newcommand{\zerov}{{\bf 0}}
\newcommand{\onev}{{\bf 1}}

% Matrices

\newcommand{\Am}{{\bf A}}
\newcommand{\Bm}{{\bf B}}
\newcommand{\Cm}{{\bf C}}
\newcommand{\Dm}{{\bf D}}
\newcommand{\Em}{{\bf E}}
\newcommand{\Fm}{{\bf F}}
\newcommand{\Gm}{{\bf G}}
\newcommand{\Hm}{{\bf H}}
\newcommand{\Id}{{\bf I}}
\newcommand{\Jm}{{\bf J}}
\newcommand{\Km}{{\bf K}}
\newcommand{\Lm}{{\bf L}}
\newcommand{\Mm}{{\bf M}}
\newcommand{\Nm}{{\bf N}}
\newcommand{\Om}{{\bf O}}
\newcommand{\Pm}{{\bf P}}
\newcommand{\Qm}{{\bf Q}}
\newcommand{\Rm}{{\bf R}}
\newcommand{\Sm}{{\bf S}}
\newcommand{\Tm}{{\bf T}}
\newcommand{\Um}{{\bf U}}
\newcommand{\Wm}{{\bf W}}
\newcommand{\Vm}{{\bf V}}
\newcommand{\Xm}{{\bf X}}
\newcommand{\Ym}{{\bf Y}}
\newcommand{\Zm}{{\bf Z}}

% Calligraphic

\newcommand{\Ac}{{\cal A}}
\newcommand{\Bc}{{\cal B}}
\newcommand{\Cc}{{\cal C}}
\newcommand{\Dc}{{\cal D}}
\newcommand{\Ec}{{\cal E}}
\newcommand{\Fc}{{\cal F}}
\newcommand{\Gc}{{\cal G}}
\newcommand{\Hc}{{\cal H}}
\newcommand{\Ic}{{\cal I}}
\newcommand{\Jc}{{\cal J}}
\newcommand{\Kc}{{\cal K}}
\newcommand{\Lc}{{\cal L}}
\newcommand{\Mc}{{\cal M}}
\newcommand{\Nc}{{\cal N}}
\newcommand{\nc}{{\cal n}}
\newcommand{\Oc}{{\cal O}}
\newcommand{\Pc}{{\cal P}}
\newcommand{\Qc}{{\cal Q}}
\newcommand{\Rc}{{\cal R}}
\newcommand{\Sc}{{\cal S}}
\newcommand{\Tc}{{\cal T}}
\newcommand{\Uc}{{\cal U}}
\newcommand{\Wc}{{\cal W}}
\newcommand{\Vc}{{\cal V}}
\newcommand{\Xc}{{\cal X}}
\newcommand{\Yc}{{\cal Y}}
\newcommand{\Zc}{{\cal Z}}

% Bold greek letters

\newcommand{\alphav}{\hbox{\boldmath$\alpha$}}
\newcommand{\betav}{\hbox{\boldmath$\beta$}}
\newcommand{\gammav}{\hbox{\boldmath$\gamma$}}
\newcommand{\deltav}{\hbox{\boldmath$\delta$}}
\newcommand{\etav}{\hbox{\boldmath$\eta$}}
\newcommand{\lambdav}{\hbox{\boldmath$\lambda$}}
\newcommand{\epsilonv}{\hbox{\boldmath$\epsilon$}}
\newcommand{\nuv}{\hbox{\boldmath$\nu$}}
\newcommand{\muv}{\hbox{\boldmath$\mu$}}
\newcommand{\zetav}{\hbox{\boldmath$\zeta$}}
\newcommand{\phiv}{\hbox{\boldmath$\phi$}}
\newcommand{\psiv}{\hbox{\boldmath$\psi$}}
\newcommand{\thetav}{\hbox{\boldmath$\theta$}}
\newcommand{\tauv}{\hbox{\boldmath$\tau$}}
\newcommand{\omegav}{\hbox{\boldmath$\omega$}}
\newcommand{\xiv}{\hbox{\boldmath$\xi$}}
\newcommand{\sigmav}{\hbox{\boldmath$\sigma$}}
\newcommand{\piv}{\hbox{\boldmath$\pi$}}
\newcommand{\rhov}{\hbox{\boldmath$\rho$}}
\newcommand{\upsilonv}{\hbox{\boldmath$\upsilon$}}

\newcommand{\Gammam}{\hbox{\boldmath$\Gamma$}}
\newcommand{\Lambdam}{\hbox{\boldmath$\Lambda$}}
\newcommand{\Deltam}{\hbox{\boldmath$\Delta$}}
\newcommand{\Sigmam}{\hbox{\boldmath$\Sigma$}}
\newcommand{\Phim}{\hbox{\boldmath$\Phi$}}
\newcommand{\Pim}{\hbox{\boldmath$\Pi$}}
\newcommand{\Psim}{\hbox{\boldmath$\Psi$}}
\newcommand{\Thetam}{\hbox{\boldmath$\Theta$}}
\newcommand{\Omegam}{\hbox{\boldmath$\Omega$}}
\newcommand{\Xim}{\hbox{\boldmath$\Xi$}}


% Sans Serif small case

\newcommand{\Gsf}{{\sf G}}

\newcommand{\asf}{{\sf a}}
\newcommand{\bsf}{{\sf b}}
\newcommand{\csf}{{\sf c}}
\newcommand{\dsf}{{\sf d}}
\newcommand{\esf}{{\sf e}}
\newcommand{\fsf}{{\sf f}}
\newcommand{\gsf}{{\sf g}}
\newcommand{\hsf}{{\sf h}}
\newcommand{\isf}{{\sf i}}
\newcommand{\jsf}{{\sf j}}
\newcommand{\ksf}{{\sf k}}
\newcommand{\lsf}{{\sf l}}
\newcommand{\msf}{{\sf m}}
\newcommand{\nsf}{{\sf n}}
\newcommand{\osf}{{\sf o}}
\newcommand{\psf}{{\sf p}}
\newcommand{\qsf}{{\sf q}}
\newcommand{\rsf}{{\sf r}}
\newcommand{\ssf}{{\sf s}}
\newcommand{\tsf}{{\sf t}}
\newcommand{\usf}{{\sf u}}
\newcommand{\wsf}{{\sf w}}
\newcommand{\vsf}{{\sf v}}
\newcommand{\xsf}{{\sf x}}
\newcommand{\ysf}{{\sf y}}
\newcommand{\zsf}{{\sf z}}


% mixed symbols

\newcommand{\sinc}{{\hbox{sinc}}}
\newcommand{\diag}{{\hbox{diag}}}
\renewcommand{\det}{{\hbox{det}}}
\newcommand{\trace}{{\hbox{tr}}}
\newcommand{\sign}{{\hbox{sign}}}
\renewcommand{\arg}{{\hbox{arg}}}
\newcommand{\var}{{\hbox{var}}}
\newcommand{\cov}{{\hbox{cov}}}
\newcommand{\Ei}{{\rm E}_{\rm i}}
\renewcommand{\Re}{{\rm Re}}
\renewcommand{\Im}{{\rm Im}}
\newcommand{\eqdef}{\stackrel{\Delta}{=}}
\newcommand{\defines}{{\,\,\stackrel{\scriptscriptstyle \bigtriangleup}{=}\,\,}}
\newcommand{\<}{\left\langle}
\renewcommand{\>}{\right\rangle}
\newcommand{\herm}{{\sf H}}
\newcommand{\trasp}{{\sf T}}
\newcommand{\transp}{{\sf T}}
\renewcommand{\vec}{{\rm vec}}
\newcommand{\Psf}{{\sf P}}
\newcommand{\SINR}{{\sf SINR}}
\newcommand{\SNR}{{\sf SNR}}
\newcommand{\MMSE}{{\sf MMSE}}
\newcommand{\REF}{{\RED [REF]}}

% Markov chain
\usepackage{stmaryrd} % for \mkv 
\newcommand{\mkv}{-\!\!\!\!\minuso\!\!\!\!-}

% Colors

\newcommand{\RED}{\color[rgb]{1.00,0.10,0.10}}
\newcommand{\BLUE}{\color[rgb]{0,0,0.90}}
\newcommand{\GREEN}{\color[rgb]{0,0.80,0.20}}

%%%%%%%%%%%%%%%%%%%%%%%%%%%%%%%%%%%%%%%%%%
\usepackage{hyperref}
\hypersetup{
    bookmarks=true,         % show bookmarks bar?
    unicode=false,          % non-Latin characters in AcrobatÕs bookmarks
    pdftoolbar=true,        % show AcrobatÕs toolbar?
    pdfmenubar=true,        % show AcrobatÕs menu?
    pdffitwindow=false,     % window fit to page when opened
    pdfstartview={FitH},    % fits the width of the page to the window
%    pdftitle={My title},    % title
%    pdfauthor={Author},     % author
%    pdfsubject={Subject},   % subject of the document
%    pdfcreator={Creator},   % creator of the document
%    pdfproducer={Producer}, % producer of the document
%    pdfkeywords={keyword1} {key2} {key3}, % list of keywords
    pdfnewwindow=true,      % links in new window
    colorlinks=true,       % false: boxed links; true: colored links
    linkcolor=red,          % color of internal links (change box color with linkbordercolor)
    citecolor=green,        % color of links to bibliography
    filecolor=blue,      % color of file links
    urlcolor=blue           % color of external links
}
%%%%%%%%%%%%%%%%%%%%%%%%%%%%%%%%%%%%%%%%%%%


% Optional: Remove the ACM reference between the abstract and the main text.
%\settopmatter{printfolios=true,printacmref=false}
% Optional: Comment out the CCS concepts and keywords.
%
%
\setlength\unitlength{1mm}
\newcommand{\twodots}{\mathinner {\ldotp \ldotp}}
% bb font symbols
\newcommand{\Rho}{\mathrm{P}}
\newcommand{\Tau}{\mathrm{T}}

\newfont{\bbb}{msbm10 scaled 700}
\newcommand{\CCC}{\mbox{\bbb C}}

\newfont{\bb}{msbm10 scaled 1100}
\newcommand{\CC}{\mbox{\bb C}}
\newcommand{\PP}{\mbox{\bb P}}
\newcommand{\RR}{\mbox{\bb R}}
\newcommand{\QQ}{\mbox{\bb Q}}
\newcommand{\ZZ}{\mbox{\bb Z}}
\newcommand{\FF}{\mbox{\bb F}}
\newcommand{\GG}{\mbox{\bb G}}
\newcommand{\EE}{\mbox{\bb E}}
\newcommand{\NN}{\mbox{\bb N}}
\newcommand{\KK}{\mbox{\bb K}}
\newcommand{\HH}{\mbox{\bb H}}
\newcommand{\SSS}{\mbox{\bb S}}
\newcommand{\UU}{\mbox{\bb U}}
\newcommand{\VV}{\mbox{\bb V}}


\newcommand{\yy}{\mathbbm{y}}
\newcommand{\xx}{\mathbbm{x}}
\newcommand{\zz}{\mathbbm{z}}
\newcommand{\sss}{\mathbbm{s}}
\newcommand{\rr}{\mathbbm{r}}
\newcommand{\pp}{\mathbbm{p}}
\newcommand{\qq}{\mathbbm{q}}
\newcommand{\ww}{\mathbbm{w}}
\newcommand{\hh}{\mathbbm{h}}
\newcommand{\vvv}{\mathbbm{v}}

% Vectors

\newcommand{\av}{{\bf a}}
\newcommand{\bv}{{\bf b}}
\newcommand{\cv}{{\bf c}}
\newcommand{\dv}{{\bf d}}
\newcommand{\ev}{{\bf e}}
\newcommand{\fv}{{\bf f}}
\newcommand{\gv}{{\bf g}}
\newcommand{\hv}{{\bf h}}
\newcommand{\iv}{{\bf i}}
\newcommand{\jv}{{\bf j}}
\newcommand{\kv}{{\bf k}}
\newcommand{\lv}{{\bf l}}
\newcommand{\mv}{{\bf m}}
\newcommand{\nv}{{\bf n}}
\newcommand{\ov}{{\bf o}}
\newcommand{\pv}{{\bf p}}
\newcommand{\qv}{{\bf q}}
\newcommand{\rv}{{\bf r}}
\newcommand{\sv}{{\bf s}}
\newcommand{\tv}{{\bf t}}
\newcommand{\uv}{{\bf u}}
\newcommand{\wv}{{\bf w}}
\newcommand{\vv}{{\bf v}}
\newcommand{\xv}{{\bf x}}
\newcommand{\yv}{{\bf y}}
\newcommand{\zv}{{\bf z}}
\newcommand{\zerov}{{\bf 0}}
\newcommand{\onev}{{\bf 1}}

% Matrices

\newcommand{\Am}{{\bf A}}
\newcommand{\Bm}{{\bf B}}
\newcommand{\Cm}{{\bf C}}
\newcommand{\Dm}{{\bf D}}
\newcommand{\Em}{{\bf E}}
\newcommand{\Fm}{{\bf F}}
\newcommand{\Gm}{{\bf G}}
\newcommand{\Hm}{{\bf H}}
\newcommand{\Id}{{\bf I}}
\newcommand{\Jm}{{\bf J}}
\newcommand{\Km}{{\bf K}}
\newcommand{\Lm}{{\bf L}}
\newcommand{\Mm}{{\bf M}}
\newcommand{\Nm}{{\bf N}}
\newcommand{\Om}{{\bf O}}
\newcommand{\Pm}{{\bf P}}
\newcommand{\Qm}{{\bf Q}}
\newcommand{\Rm}{{\bf R}}
\newcommand{\Sm}{{\bf S}}
\newcommand{\Tm}{{\bf T}}
\newcommand{\Um}{{\bf U}}
\newcommand{\Wm}{{\bf W}}
\newcommand{\Vm}{{\bf V}}
\newcommand{\Xm}{{\bf X}}
\newcommand{\Ym}{{\bf Y}}
\newcommand{\Zm}{{\bf Z}}

% Calligraphic

\newcommand{\Ac}{{\cal A}}
\newcommand{\Bc}{{\cal B}}
\newcommand{\Cc}{{\cal C}}
\newcommand{\Dc}{{\cal D}}
\newcommand{\Ec}{{\cal E}}
\newcommand{\Fc}{{\cal F}}
\newcommand{\Gc}{{\cal G}}
\newcommand{\Hc}{{\cal H}}
\newcommand{\Ic}{{\cal I}}
\newcommand{\Jc}{{\cal J}}
\newcommand{\Kc}{{\cal K}}
\newcommand{\Lc}{{\cal L}}
\newcommand{\Mc}{{\cal M}}
\newcommand{\Nc}{{\cal N}}
\newcommand{\nc}{{\cal n}}
\newcommand{\Oc}{{\cal O}}
\newcommand{\Pc}{{\cal P}}
\newcommand{\Qc}{{\cal Q}}
\newcommand{\Rc}{{\cal R}}
\newcommand{\Sc}{{\cal S}}
\newcommand{\Tc}{{\cal T}}
\newcommand{\Uc}{{\cal U}}
\newcommand{\Wc}{{\cal W}}
\newcommand{\Vc}{{\cal V}}
\newcommand{\Xc}{{\cal X}}
\newcommand{\Yc}{{\cal Y}}
\newcommand{\Zc}{{\cal Z}}

% Bold greek letters

\newcommand{\alphav}{\hbox{\boldmath$\alpha$}}
\newcommand{\betav}{\hbox{\boldmath$\beta$}}
\newcommand{\gammav}{\hbox{\boldmath$\gamma$}}
\newcommand{\deltav}{\hbox{\boldmath$\delta$}}
\newcommand{\etav}{\hbox{\boldmath$\eta$}}
\newcommand{\lambdav}{\hbox{\boldmath$\lambda$}}
\newcommand{\epsilonv}{\hbox{\boldmath$\epsilon$}}
\newcommand{\nuv}{\hbox{\boldmath$\nu$}}
\newcommand{\muv}{\hbox{\boldmath$\mu$}}
\newcommand{\zetav}{\hbox{\boldmath$\zeta$}}
\newcommand{\phiv}{\hbox{\boldmath$\phi$}}
\newcommand{\psiv}{\hbox{\boldmath$\psi$}}
\newcommand{\thetav}{\hbox{\boldmath$\theta$}}
\newcommand{\tauv}{\hbox{\boldmath$\tau$}}
\newcommand{\omegav}{\hbox{\boldmath$\omega$}}
\newcommand{\xiv}{\hbox{\boldmath$\xi$}}
\newcommand{\sigmav}{\hbox{\boldmath$\sigma$}}
\newcommand{\piv}{\hbox{\boldmath$\pi$}}
\newcommand{\rhov}{\hbox{\boldmath$\rho$}}
\newcommand{\upsilonv}{\hbox{\boldmath$\upsilon$}}

\newcommand{\Gammam}{\hbox{\boldmath$\Gamma$}}
\newcommand{\Lambdam}{\hbox{\boldmath$\Lambda$}}
\newcommand{\Deltam}{\hbox{\boldmath$\Delta$}}
\newcommand{\Sigmam}{\hbox{\boldmath$\Sigma$}}
\newcommand{\Phim}{\hbox{\boldmath$\Phi$}}
\newcommand{\Pim}{\hbox{\boldmath$\Pi$}}
\newcommand{\Psim}{\hbox{\boldmath$\Psi$}}
\newcommand{\Thetam}{\hbox{\boldmath$\Theta$}}
\newcommand{\Omegam}{\hbox{\boldmath$\Omega$}}
\newcommand{\Xim}{\hbox{\boldmath$\Xi$}}


% Sans Serif small case

\newcommand{\Gsf}{{\sf G}}

\newcommand{\asf}{{\sf a}}
\newcommand{\bsf}{{\sf b}}
\newcommand{\csf}{{\sf c}}
\newcommand{\dsf}{{\sf d}}
\newcommand{\esf}{{\sf e}}
\newcommand{\fsf}{{\sf f}}
\newcommand{\gsf}{{\sf g}}
\newcommand{\hsf}{{\sf h}}
\newcommand{\isf}{{\sf i}}
\newcommand{\jsf}{{\sf j}}
\newcommand{\ksf}{{\sf k}}
\newcommand{\lsf}{{\sf l}}
\newcommand{\msf}{{\sf m}}
\newcommand{\nsf}{{\sf n}}
\newcommand{\osf}{{\sf o}}
\newcommand{\psf}{{\sf p}}
\newcommand{\qsf}{{\sf q}}
\newcommand{\rsf}{{\sf r}}
\newcommand{\ssf}{{\sf s}}
\newcommand{\tsf}{{\sf t}}
\newcommand{\usf}{{\sf u}}
\newcommand{\wsf}{{\sf w}}
\newcommand{\vsf}{{\sf v}}
\newcommand{\xsf}{{\sf x}}
\newcommand{\ysf}{{\sf y}}
\newcommand{\zsf}{{\sf z}}


% mixed symbols

\newcommand{\sinc}{{\hbox{sinc}}}
\newcommand{\diag}{{\hbox{diag}}}
\renewcommand{\det}{{\hbox{det}}}
\newcommand{\trace}{{\hbox{tr}}}
\newcommand{\sign}{{\hbox{sign}}}
\renewcommand{\arg}{{\hbox{arg}}}
\newcommand{\var}{{\hbox{var}}}
\newcommand{\cov}{{\hbox{cov}}}
\newcommand{\Ei}{{\rm E}_{\rm i}}
\renewcommand{\Re}{{\rm Re}}
\renewcommand{\Im}{{\rm Im}}
\newcommand{\eqdef}{\stackrel{\Delta}{=}}
\newcommand{\defines}{{\,\,\stackrel{\scriptscriptstyle \bigtriangleup}{=}\,\,}}
\newcommand{\<}{\left\langle}
\renewcommand{\>}{\right\rangle}
\newcommand{\herm}{{\sf H}}
\newcommand{\trasp}{{\sf T}}
\newcommand{\transp}{{\sf T}}
\renewcommand{\vec}{{\rm vec}}
\newcommand{\Psf}{{\sf P}}
\newcommand{\SINR}{{\sf SINR}}
\newcommand{\SNR}{{\sf SNR}}
\newcommand{\MMSE}{{\sf MMSE}}
\newcommand{\REF}{{\RED [REF]}}

% Markov chain
\usepackage{stmaryrd} % for \mkv 
\newcommand{\mkv}{-\!\!\!\!\minuso\!\!\!\!-}

% Colors

\newcommand{\RED}{\color[rgb]{1.00,0.10,0.10}}
\newcommand{\BLUE}{\color[rgb]{0,0,0.90}}
\newcommand{\GREEN}{\color[rgb]{0,0.80,0.20}}

%%%%%%%%%%%%%%%%%%%%%%%%%%%%%%%%%%%%%%%%%%
\usepackage{hyperref}
\hypersetup{
    bookmarks=true,         % show bookmarks bar?
    unicode=false,          % non-Latin characters in AcrobatÕs bookmarks
    pdftoolbar=true,        % show AcrobatÕs toolbar?
    pdfmenubar=true,        % show AcrobatÕs menu?
    pdffitwindow=false,     % window fit to page when opened
    pdfstartview={FitH},    % fits the width of the page to the window
%    pdftitle={My title},    % title
%    pdfauthor={Author},     % author
%    pdfsubject={Subject},   % subject of the document
%    pdfcreator={Creator},   % creator of the document
%    pdfproducer={Producer}, % producer of the document
%    pdfkeywords={keyword1} {key2} {key3}, % list of keywords
    pdfnewwindow=true,      % links in new window
    colorlinks=true,       % false: boxed links; true: colored links
    linkcolor=red,          % color of internal links (change box color with linkbordercolor)
    citecolor=green,        % color of links to bibliography
    filecolor=blue,      % color of file links
    urlcolor=blue           % color of external links
}
%%%%%%%%%%%%%%%%%%%%%%%%%%%%%%%%%%%%%%%%%%%



%%
%% \BibTeX command to typeset BibTeX logo in the docs
\AtBeginDocument{%
  \providecommand\BibTeX{{%
    Bib\TeX}}}


%%
%% end of the preamble, start of the body of the document source.
\begin{document}
%%
%% The "title" command has an optional parameter,
%% allowing the author to define a "short title" to be used in page headers.
%\title{Rax: safe kernel extension can be usable}
%\title[Rax: Addressing the Language-Verifier Gap with Safe and Usable Rust-based Kernel Extensions]{Rax: Addressing the Language-Verifier Gap with\\ Safe and Usable Rust-based Kernel Extensions}
%\title[Rax: Safe and Usable Rust-based Kernel Extensions without Language-Verifier Gaps]{Rax: Safe and Usable Rust-based Kernel Extensions without Language-Verifier Gaps}
%\title[Rax: Safe and Usable Rust-based Kernel Extensions with No Language-Verifier Gap]{Rax: Safe and Usable Rust-based Kernel Extensions with No Language-Verifier Gap}
%\title[Rax: Closing the Language-Verifier Gap with Safe and Usable Rust-based Kernel Extensions]{Rax: Closing the Language-Verifier Gap with Safe and Usable Rust-based Kernel Extensions}
%\title[Rax: Closing the Language-Verifier Gap with Safe and Usable Kernel Extensions]{Rax: Closing the Language-Verifier Gap with Safe and Usable Kernel Extensions}
%\title[Rax: Closing the language-verifier gap with safe and usable kernel extensions]{Rax: Closing the language-verifier gap with\\ safe and usable kernel extensions}
%\title{\bf Rax: Safe and usable kernel extensions in Rust}
\title{\bf Safe and usable kernel extensions with Rax}

\author{
  Jinghao Jia$^{*}$, Ruowen Qin$^{*}$, Milo Craun$^{\dagger}$, Egor Lukiyanov$^{\dagger}$, Ayush Bansal$^{*}$,\\[3pt]
  Michael V. Le$^{\ddagger}$, Hubertus Franke$^{\ddagger}$, Hani Jamjoom$^{\ddagger}$, Tianyin Xu$^{*}$, Dan Williams$^{\dagger}$\\[7.5pt]
  $^{*}$University of Illinois Urbana-Champaign\ \ \ \  $^{\dagger}$Virginia Tech\ \ \ \  $^{\ddagger}$IBM Research\\[17.5pt]
}


%%
%% The "author" command and its associated commands are used to define
%% the authors and their affiliations.
%% Of note is the shared affiliation of the first two authors, and the
%% "authornote" and "authornotemark" commands
%% used to denote shared contribution to the research.

%%
%% By default, the full list of authors will be used in the page
%% headers. Often, this list is too long, and will overlap
%% other information printed in the page headers. This command allows
%% the author to define a more concise list
%% of authors' names for this purpose.

%%
%% The code below is generated by the tool at http://dl.acm.org/ccs.cfm.
%% Please copy and paste the code instead of the example below.
%%

%%
%% Keywords. The author(s) should pick words that accurately describe
%% the work being presented. Separate the keywords with commas.

%%
%% This command processes the author and affiliation and title
%% information and builds the first part of the formatted document.
\maketitle
% \pagestyle{plain}

%%
%% The abstract is a short summary of the work to be presented in the
%% article.
\begin{abstract}  
Test time scaling is currently one of the most active research areas that shows promise after training time scaling has reached its limits.
Deep-thinking (DT) models are a class of recurrent models that can perform easy-to-hard generalization by assigning more compute to harder test samples.
However, due to their inability to determine the complexity of a test sample, DT models have to use a large amount of computation for both easy and hard test samples.
Excessive test time computation is wasteful and can cause the ``overthinking'' problem where more test time computation leads to worse results.
In this paper, we introduce a test time training method for determining the optimal amount of computation needed for each sample during test time.
We also propose Conv-LiGRU, a novel recurrent architecture for efficient and robust visual reasoning. 
Extensive experiments demonstrate that Conv-LiGRU is more stable than DT, effectively mitigates the ``overthinking'' phenomenon, and achieves superior accuracy.
\end{abstract}  

\section{Introduction}


\begin{figure}[t]
\centering
\includegraphics[width=0.6\columnwidth]{figures/evaluation_desiderata_V5.pdf}
\vspace{-0.5cm}
\caption{\systemName is a platform for conducting realistic evaluations of code LLMs, collecting human preferences of coding models with real users, real tasks, and in realistic environments, aimed at addressing the limitations of existing evaluations.
}
\label{fig:motivation}
\end{figure}

\begin{figure*}[t]
\centering
\includegraphics[width=\textwidth]{figures/system_design_v2.png}
\caption{We introduce \systemName, a VSCode extension to collect human preferences of code directly in a developer's IDE. \systemName enables developers to use code completions from various models. The system comprises a) the interface in the user's IDE which presents paired completions to users (left), b) a sampling strategy that picks model pairs to reduce latency (right, top), and c) a prompting scheme that allows diverse LLMs to perform code completions with high fidelity.
Users can select between the top completion (green box) using \texttt{tab} or the bottom completion (blue box) using \texttt{shift+tab}.}
\label{fig:overview}
\end{figure*}

As model capabilities improve, large language models (LLMs) are increasingly integrated into user environments and workflows.
For example, software developers code with AI in integrated developer environments (IDEs)~\citep{peng2023impact}, doctors rely on notes generated through ambient listening~\citep{oberst2024science}, and lawyers consider case evidence identified by electronic discovery systems~\citep{yang2024beyond}.
Increasing deployment of models in productivity tools demands evaluation that more closely reflects real-world circumstances~\citep{hutchinson2022evaluation, saxon2024benchmarks, kapoor2024ai}.
While newer benchmarks and live platforms incorporate human feedback to capture real-world usage, they almost exclusively focus on evaluating LLMs in chat conversations~\citep{zheng2023judging,dubois2023alpacafarm,chiang2024chatbot, kirk2024the}.
Model evaluation must move beyond chat-based interactions and into specialized user environments.



 

In this work, we focus on evaluating LLM-based coding assistants. 
Despite the popularity of these tools---millions of developers use Github Copilot~\citep{Copilot}---existing
evaluations of the coding capabilities of new models exhibit multiple limitations (Figure~\ref{fig:motivation}, bottom).
Traditional ML benchmarks evaluate LLM capabilities by measuring how well a model can complete static, interview-style coding tasks~\citep{chen2021evaluating,austin2021program,jain2024livecodebench, white2024livebench} and lack \emph{real users}. 
User studies recruit real users to evaluate the effectiveness of LLMs as coding assistants, but are often limited to simple programming tasks as opposed to \emph{real tasks}~\citep{vaithilingam2022expectation,ross2023programmer, mozannar2024realhumaneval}.
Recent efforts to collect human feedback such as Chatbot Arena~\citep{chiang2024chatbot} are still removed from a \emph{realistic environment}, resulting in users and data that deviate from typical software development processes.
We introduce \systemName to address these limitations (Figure~\ref{fig:motivation}, top), and we describe our three main contributions below.


\textbf{We deploy \systemName in-the-wild to collect human preferences on code.} 
\systemName is a Visual Studio Code extension, collecting preferences directly in a developer's IDE within their actual workflow (Figure~\ref{fig:overview}).
\systemName provides developers with code completions, akin to the type of support provided by Github Copilot~\citep{Copilot}. 
Over the past 3 months, \systemName has served over~\completions suggestions from 10 state-of-the-art LLMs, 
gathering \sampleCount~votes from \userCount~users.
To collect user preferences,
\systemName presents a novel interface that shows users paired code completions from two different LLMs, which are determined based on a sampling strategy that aims to 
mitigate latency while preserving coverage across model comparisons.
Additionally, we devise a prompting scheme that allows a diverse set of models to perform code completions with high fidelity.
See Section~\ref{sec:system} and Section~\ref{sec:deployment} for details about system design and deployment respectively.



\textbf{We construct a leaderboard of user preferences and find notable differences from existing static benchmarks and human preference leaderboards.}
In general, we observe that smaller models seem to overperform in static benchmarks compared to our leaderboard, while performance among larger models is mixed (Section~\ref{sec:leaderboard_calculation}).
We attribute these differences to the fact that \systemName is exposed to users and tasks that differ drastically from code evaluations in the past. 
Our data spans 103 programming languages and 24 natural languages as well as a variety of real-world applications and code structures, while static benchmarks tend to focus on a specific programming and natural language and task (e.g. coding competition problems).
Additionally, while all of \systemName interactions contain code contexts and the majority involve infilling tasks, a much smaller fraction of Chatbot Arena's coding tasks contain code context, with infilling tasks appearing even more rarely. 
We analyze our data in depth in Section~\ref{subsec:comparison}.



\textbf{We derive new insights into user preferences of code by analyzing \systemName's diverse and distinct data distribution.}
We compare user preferences across different stratifications of input data (e.g., common versus rare languages) and observe which affect observed preferences most (Section~\ref{sec:analysis}).
For example, while user preferences stay relatively consistent across various programming languages, they differ drastically between different task categories (e.g. frontend/backend versus algorithm design).
We also observe variations in user preference due to different features related to code structure 
(e.g., context length and completion patterns).
We open-source \systemName and release a curated subset of code contexts.
Altogether, our results highlight the necessity of model evaluation in realistic and domain-specific settings.





\section{Background}\label{sec:backgrnd}

\subsection{Cold Start Latency and Mitigation Techniques}

Traditional FaaS platforms mitigate cold starts through snapshotting, lightweight virtualization, and warm-state management. Snapshot-based methods like \textbf{REAP} and \textbf{Catalyzer} reduce initialization time by preloading or restoring container states but require significant memory and I/O resources, limiting scalability~\cite{dong_catalyzer_2020, ustiugov_benchmarking_2021}. Lightweight virtualization solutions, such as \textbf{Firecracker} microVMs, achieve fast startup times with strong isolation but depend on robust infrastructure, making them less adaptable to fluctuating workloads~\cite{agache_firecracker_2020}. Warm-state management techniques like \textbf{Faa\$T}~\cite{romero_faa_2021} and \textbf{Kraken}~\cite{vivek_kraken_2021} keep frequently invoked containers ready, balancing readiness and cost efficiency under predictable workloads but incurring overhead when demand is erratic~\cite{romero_faa_2021, vivek_kraken_2021}. While these methods perform well in resource-rich cloud environments, their resource intensity challenges applicability in edge settings.

\subsubsection{Edge FaaS Perspective}

In edge environments, cold start mitigation emphasizes lightweight designs, resource sharing, and hybrid task distribution. Lightweight execution environments like unikernels~\cite{edward_sock_2018} and \textbf{Firecracker}~\cite{agache_firecracker_2020}, as used by \textbf{TinyFaaS}~\cite{pfandzelter_tinyfaas_2020}, minimize resource usage and initialization delays but require careful orchestration to avoid resource contention. Function co-location, demonstrated by \textbf{Photons}~\cite{v_dukic_photons_2020}, reduces redundant initializations by sharing runtime resources among related functions, though this complicates isolation in multi-tenant setups~\cite{v_dukic_photons_2020}. Hybrid offloading frameworks like \textbf{GeoFaaS}~\cite{malekabbasi_geofaas_2024} balance edge-cloud workloads by offloading latency-tolerant tasks to the cloud and reserving edge resources for real-time operations, requiring reliable connectivity and efficient task management. These edge-specific strategies address cold starts effectively but introduce challenges in scalability and orchestration.

\subsection{Predictive Scaling and Caching Techniques}

Efficient resource allocation is vital for maintaining low latency and high availability in serverless platforms. Predictive scaling and caching techniques dynamically provision resources and reduce cold start latency by leveraging workload prediction and state retention.
Traditional FaaS platforms use predictive scaling and caching to optimize resources, employing techniques (OFC, FaasCache) to reduce cold starts. However, these methods rely on centralized orchestration and workload predictability, limiting their effectiveness in dynamic, resource-constrained edge environments.



\subsubsection{Edge FaaS Perspective}

Edge FaaS platforms adapt predictive scaling and caching techniques to constrain resources and heterogeneous environments. \textbf{EDGE-Cache}~\cite{kim_delay-aware_2022} uses traffic profiling to selectively retain high-priority functions, reducing memory overhead while maintaining readiness for frequent requests. Hybrid frameworks like \textbf{GeoFaaS}~\cite{malekabbasi_geofaas_2024} implement distributed caching to balance resources between edge and cloud nodes, enabling low-latency processing for critical tasks while offloading less critical workloads. Machine learning methods, such as clustering-based workload predictors~\cite{gao_machine_2020} and GRU-based models~\cite{guo_applying_2018}, enhance resource provisioning in edge systems by efficiently forecasting workload spikes. These innovations effectively address cold start challenges in edge environments, though their dependency on accurate predictions and robust orchestration poses scalability challenges.

\subsection{Decentralized Orchestration, Function Placement, and Scheduling}

Efficient orchestration in serverless platforms involves workload distribution, resource optimization, and performance assurance. While traditional FaaS platforms rely on centralized control, edge environments require decentralized and adaptive strategies to address unique challenges such as resource constraints and heterogeneous hardware.



\subsubsection{Edge FaaS Perspective}

Edge FaaS platforms adopt decentralized and adaptive orchestration frameworks to meet the demands of resource-constrained environments. Systems like \textbf{Wukong} distribute scheduling across edge nodes, enhancing data locality and scalability while reducing network latency. Lightweight frameworks such as \textbf{OpenWhisk Lite}~\cite{kravchenko_kpavelopenwhisk-light_2024} optimize resource allocation by decentralizing scheduling policies, minimizing cold starts and latency in edge setups~\cite{benjamin_wukong_2020}. Hybrid solutions like \textbf{OpenFaaS}~\cite{noauthor_openfaasfaas_2024} and \textbf{EdgeMatrix}~\cite{shen_edgematrix_2023} combine edge-cloud orchestration to balance resource utilization, retaining latency-sensitive functions at the edge while offloading non-critical workloads to the cloud. While these approaches improve flexibility, they face challenges in maintaining coordination and ensuring consistent performance across distributed nodes.


\section{Bellman Error Centering}

Centering operator $\mathcal{C}$ for a variable $x(s)$ is defined as follows:
\begin{equation}
\mathcal{C}x(s)\dot{=} x(s)-\mathbb{E}[x(s)]=x(s)-\sum_s{d_{s}x(s)},
\end{equation} 
where $d_s$ is the probability of $s$.
In vector form,
\begin{equation}
\begin{split}
\mathcal{C}\bm{x} &= \bm{x}-\mathbb{E}[x]\bm{1}\\
&=\bm{x}-\bm{x}^{\top}\bm{d}\bm{1},
\end{split}
\end{equation} 
where $\bm{1}$ is an all-ones vector.
For any vector $\bm{x}$ and $\bm{y}$ with a same distribution $\bm{d}$,
we have
\begin{equation}
\begin{split}
\mathcal{C}(\bm{x}+\bm{y})&=(\bm{x}+\bm{y})-(\bm{x}+\bm{y})^{\top}\bm{d}\bm{1}\\
&=\bm{x}-\bm{x}^{\top}\bm{d}\bm{1}+\bm{y}-\bm{y}^{\top}\bm{d}\bm{1}\\
&=\mathcal{C}\bm{x}+\mathcal{C}\bm{y}.
\end{split}
\end{equation}
\subsection{Revisit Reward Centering}


The update (\ref{src3}) is an unbiased estimate of the average reward
with  appropriate learning rate $\beta_t$ conditions.
\begin{equation}
\bar{r}_{t}\approx \lim_{n\rightarrow\infty}\frac{1}{n}\sum_{t=1}^n\mathbb{E}_{\pi}[r_t].
\end{equation}
That is 
\begin{equation}
r_t-\bar{r}_{t}\approx r_t-\lim_{n\rightarrow\infty}\frac{1}{n}\sum_{t=1}^n\mathbb{E}_{\pi}[r_t]= \mathcal{C}r_t.
\end{equation}
Then, the simple reward centering can be rewrited as:
\begin{equation}
V_{t+1}(s_t)=V_{t}(s_t)+\alpha_t [\mathcal{C}r_{t+1}+\gamma V_{t}(s_{t+1})-V_t(s_t)].
\end{equation}
Therefore, the simple reward centering is, in a strict sense, reward centering.

By definition of $\bar{\delta}_t=\delta_t-\bar{r}_{t}$,
let rewrite the update rule of the value-based reward centering as follows:
\begin{equation}
V_{t+1}(s_t)=V_{t}(s_t)+\alpha_t \rho_t (\delta_t-\bar{r}_{t}),
\end{equation}
where $\bar{r}_{t}$ is updated as:
\begin{equation}
\bar{r}_{t+1}=\bar{r}_{t}+\beta_t \rho_t(\delta_t-\bar{r}_{t}).
\label{vrc3}
\end{equation}
The update (\ref{vrc3}) is an unbiased estimate of the TD error
with  appropriate learning rate $\beta_t$ conditions.
\begin{equation}
\bar{r}_{t}\approx \mathbb{E}_{\pi}[\delta_t].
\end{equation}
That is 
\begin{equation}
\delta_t-\bar{r}_{t}\approx \mathcal{C}\delta_t.
\end{equation}
Then, the value-based reward centering can be rewrited as:
\begin{equation}
V_{t+1}(s_t)=V_{t}(s_t)+\alpha_t \rho_t \mathcal{C}\delta_t.
\label{tdcentering}
\end{equation}
Therefore, the value-based reward centering is no more,
 in a strict sense, reward centering.
It is, in a strict sense, \textbf{Bellman error centering}.

It is worth noting that this understanding is crucial, 
as designing new algorithms requires leveraging this concept.


\subsection{On the Fixpoint Solution}

The update rule (\ref{tdcentering}) is a stochastic approximation
of the following update:
\begin{equation}
\begin{split}
V_{t+1}&=V_{t}+\alpha_t [\bm{\mathcal{T}}^{\pi}\bm{V}-\bm{V}-\mathbb{E}[\delta]\bm{1}]\\
&=V_{t}+\alpha_t [\bm{\mathcal{T}}^{\pi}\bm{V}-\bm{V}-(\bm{\mathcal{T}}^{\pi}\bm{V}-\bm{V})^{\top}\bm{d}_{\pi}\bm{1}]\\
&=V_{t}+\alpha_t [\mathcal{C}(\bm{\mathcal{T}}^{\pi}\bm{V}-\bm{V})].
\end{split}
\label{tdcenteringVector}
\end{equation}
If update rule (\ref{tdcenteringVector}) converges, it is expected that
$\mathcal{C}(\mathcal{T}^{\pi}V-V)=\bm{0}$.
That is 
\begin{equation}
    \begin{split}
    \mathcal{C}\bm{V} &= \mathcal{C}\bm{\mathcal{T}}^{\pi}\bm{V} \\
    &= \mathcal{C}(\bm{R}^{\pi} + \gamma \mathbb{P}^{\pi} \bm{V}) \\
    &= \mathcal{C}\bm{R}^{\pi} + \gamma \mathcal{C}\mathbb{P}^{\pi} \bm{V} \\
    &= \mathcal{C}\bm{R}^{\pi} + \gamma (\mathbb{P}^{\pi} \bm{V} - (\mathbb{P}^{\pi} \bm{V})^{\top} \bm{d_{\pi}} \bm{1}) \\
    &= \mathcal{C}\bm{R}^{\pi} + \gamma (\mathbb{P}^{\pi} \bm{V} - \bm{V}^{\top} (\mathbb{P}^{\pi})^{\top} \bm{d_{\pi}} \bm{1}) \\  % 修正双重上标
    &= \mathcal{C}\bm{R}^{\pi} + \gamma (\mathbb{P}^{\pi} \bm{V} - \bm{V}^{\top} \bm{d_{\pi}} \bm{1}) \\
    &= \mathcal{C}\bm{R}^{\pi} + \gamma (\mathbb{P}^{\pi} \bm{V} - \bm{V}^{\top} \bm{d_{\pi}} \mathbb{P}^{\pi} \bm{1}) \\
    &= \mathcal{C}\bm{R}^{\pi} + \gamma (\mathbb{P}^{\pi} \bm{V} - \mathbb{P}^{\pi} \bm{V}^{\top} \bm{d_{\pi}} \bm{1}) \\
    &= \mathcal{C}\bm{R}^{\pi} + \gamma \mathbb{P}^{\pi} (\bm{V} - \bm{V}^{\top} \bm{d_{\pi}} \bm{1}) \\
    &= \mathcal{C}\bm{R}^{\pi} + \gamma \mathbb{P}^{\pi} \mathcal{C}\bm{V} \\
    &\dot{=} \bm{\mathcal{T}}_c^{\pi} \mathcal{C}\bm{V},
    \end{split}
    \label{centeredfixpoint}
    \end{equation}
where we defined $\bm{\mathcal{T}}_c^{\pi}$ as a centered Bellman operator.
We call equation (\ref{centeredfixpoint}) as centered Bellman equation.
And it is \textbf{centered fixpoint}.

For linear value function approximation, let define
\begin{equation}
\mathcal{C}\bm{V}_{\bm{\theta}}=\bm{\Pi}\bm{\mathcal{T}}_c^{\pi}\mathcal{C}\bm{V}_{\bm{\theta}}.
\label{centeredTDfixpoint}
\end{equation}
We call equation (\ref{centeredTDfixpoint}) as \textbf{centered TD fixpoint}.

\subsection{On-policy and Off-policy Centered TD Algorithms
with Linear Value Function Approximation}
Given the above centered TD fixpoint,
 mean squared centered Bellman error (MSCBE), is proposed as follows:
\begin{align*}
    \label{argminMSBEC}
 &\arg \min_{{\bm{\theta}}}\text{MSCBE}({\bm{\theta}}) \\
 &= \arg \min_{{\bm{\theta}}} \|\bm{\mathcal{T}}_c^{\pi}\mathcal{C}\bm{V}_{\bm{{\bm{\theta}}}}-\mathcal{C}\bm{V}_{\bm{{\bm{\theta}}}}\|_{\bm{D}}^2\notag\\
 &=\arg \min_{{\bm{\theta}}} \|\bm{\mathcal{T}}^{\pi}\bm{V}_{\bm{{\bm{\theta}}}} - \bm{V}_{\bm{{\bm{\theta}}}}-(\bm{\mathcal{T}}^{\pi}\bm{V}_{\bm{{\bm{\theta}}}} - \bm{V}_{\bm{{\bm{\theta}}}})^{\top}\bm{d}\bm{1}\|_{\bm{D}}^2\notag\\
 &=\arg \min_{{\bm{\theta}},\omega} \| \bm{\mathcal{T}}^{\pi}\bm{V}_{\bm{{\bm{\theta}}}} - \bm{V}_{\bm{{\bm{\theta}}}}-\omega\bm{1} \|_{\bm{D}}^2\notag,
\end{align*}
where $\omega$ is is used to estimate the expected value of the Bellman error.
% where $\omega$ is used to estimate $\mathbb{E}[\delta]$, $\omega \doteq \mathbb{E}[\mathbb{E}[\delta_t|S_t]]=\mathbb{E}[\delta]$ and $\delta_t$ is the TD error as follows:
% \begin{equation}
% \delta_t = r_{t+1}+\gamma
% {\bm{\theta}}_t^{\top}\bm{{\bm{\phi}}}_{t+1}-{\bm{\theta}}_t^{\top}\bm{{\bm{\phi}}}_t.
% \label{delta}
% \end{equation}
% $\mathbb{E}[\delta_t|S_t]$ is the Bellman error, and $\mathbb{E}[\mathbb{E}[\delta_t|S_t]]$ represents the expected value of the Bellman error.
% If $X$ is a random variable and $\mathbb{E}[X]$ is its expected value, then $X-\mathbb{E}[X]$ represents the centered form of $X$. 
% Therefore, we refer to $\mathbb{E}[\delta_t|S_t]-\mathbb{E}[\mathbb{E}[\delta_t|S_t]]$ as Bellman error centering and 
% $\mathbb{E}[(\mathbb{E}[\delta_t|S_t]-\mathbb{E}[\mathbb{E}[\delta_t|S_t]])^2]$ represents the the mean squared centered Bellman error, namely MSCBE.
% The meaning of (\ref{argminMSBEC}) is to minimize the mean squared centered Bellman error.
%The derivation of CTD is as follows.

First, the parameter  $\omega$ is derived directly based on
stochastic gradient descent:
\begin{equation}
\omega_{t+1}= \omega_{t}+\beta_t(\delta_t-\omega_t).
\label{omega}
\end{equation}

Then, based on stochastic semi-gradient descent, the update of 
the parameter ${\bm{\theta}}$ is as follows:
\begin{equation}
{\bm{\theta}}_{t+1}=
{\bm{\theta}}_{t}+\alpha_t(\delta_t-\omega_t)\bm{{\bm{\phi}}}_t.
\label{theta}
\end{equation}

We call (\ref{omega}) and (\ref{theta}) the on-policy centered
TD (CTD) algorithm. The convergence analysis with be given in
the following section.

In off-policy learning, we can simply multiply by the importance sampling
 $\rho$.
\begin{equation}
    \omega_{t+1}=\omega_{t}+\beta_t\rho_t(\delta_t-\omega_t),
    \label{omegawithrho}
\end{equation}
\begin{equation}
    {\bm{\theta}}_{t+1}=
    {\bm{\theta}}_{t}+\alpha_t\rho_t(\delta_t-\omega_t)\bm{{\bm{\phi}}}_t.
    \label{thetawithrho}
\end{equation}

We call (\ref{omegawithrho}) and (\ref{thetawithrho}) the off-policy centered
TD (CTD) algorithm.

% By substituting $\delta_t$ into Equations (\ref{omegawithrho}) and (\ref{thetawithrho}), 
% we can see that Equations (\ref{thetawithrho}) and (\ref{omegawithrho}) are formally identical 
% to the linear expressions of Equations (\ref{rewardcentering1}) and (\ref{rewardcentering2}), respectively. However, the meanings 
% of the corresponding parameters are entirely different.
% ${\bm{\theta}}_t$ is for approximating the discounted value function.
% $\bar{r_t}$ is an estimate of the average reward, while $\omega_t$ 
% is an estimate of the expected value of the Bellman error.
% $\bar{\delta_t}$ is the TD error for value-based reward centering, 
% whereas $\delta_t$ is the traditional TD error.

% This study posits that the CTD is equivalent to value-based reward 
% centering. However, CTD can be unified under a single framework 
% through an objective function, MSCBE, which also lays the 
% foundation for proving the algorithm's convergence. 
% Section 4 demonstrates that the CTD algorithm guarantees 
% convergence in the on-policy setting.

\subsection{Off-policy Centered TDC Algorithm with Linear Value Function Approximation}
The convergence of the  off-policy centered TD algorithm
may not be guaranteed.

To deal with this problem, we propose another new objective function, 
called mean squared projected centered Bellman error (MSPCBE), 
and derive Centered TDC algorithm (CTDC).

% We first establish some relationships between
%  the vector-matrix quantities and the relevant statistical expectation terms:
% \begin{align*}
%     &\mathbb{E}[(\delta({\bm{\theta}})-\mathbb{E}[\delta({\bm{\theta}})]){\bm{\phi}}] \\
%     &= \sum_s \mu(s) {\bm{\phi}}(s) \big( R(s) + \gamma \sum_{s'} P_{ss'} V_{\bm{\theta}}(s') - V_{\bm{\theta}}(s)  \\
%     &\quad \quad-\sum_s \mu(s)(R(s) + \gamma \sum_{s'} P_{ss'} V_{\bm{\theta}}(s') - V_{\bm{\theta}}(s))\big)\\
%     &= \bm{\Phi}^\top \mathbf{D} (\bm{TV}_{\bm{{\bm{\theta}}}} - \bm{V}_{\bm{{\bm{\theta}}}}-\omega\bm{1}),
% \end{align*}
% where $\omega$ is the expected value of the Bellman error and $\bm{1}$ is all-ones vector.

The specific expression of the objective function 
MSPCBE is as follows:
\begin{align}
    \label{MSPBECwithomega}
    &\arg \min_{{\bm{\theta}}}\text{MSPCBE}({\bm{\theta}})\notag\\ 
    % &= \arg \min_{{\bm{\theta}}}\big(\mathbb{E}[(\delta({\bm{\theta}}) - \mathbb{E}[\delta({\bm{\theta}})]) \bm{{\bm{\phi}}}]^\top \notag\\
    % &\quad \quad \quad\mathbb{E}[\bm{{\bm{\phi}}} \bm{{\bm{\phi}}}^\top]^{-1} \mathbb{E}[(\delta({\bm{\theta}}) - \mathbb{E}[\delta({\bm{\theta}})]) \bm{{\bm{\phi}}}]\big) \notag\\
    % &=\arg \min_{{\bm{\theta}},\omega}\mathbb{E}[(\delta({\bm{\theta}})-\omega) \bm{\bm{{\bm{\phi}}}}]^{\top} \mathbb{E}[\bm{\bm{{\bm{\phi}}}} \bm{\bm{{\bm{\phi}}}}^{\top}]^{-1}\mathbb{E}[(\delta({\bm{\theta}}) -\omega)\bm{\bm{{\bm{\phi}}}}]\\
    % &= \big(\bm{\Phi}^\top \mathbf{D} (\bm{TV}_{\bm{{\bm{\theta}}}} - \bm{V}_{\bm{{\bm{\theta}}}}-\omega\bm{1})\big)^\top (\bm{\Phi}^\top \mathbf{D} \bm{\Phi})^{-1} \notag\\
    % & \quad \quad \quad \bm{\Phi}^\top \mathbf{D} (\bm{TV}_{\bm{{\bm{\theta}}}} - \bm{V}_{\bm{{\bm{\theta}}}}-\omega\bm{1}) \notag\\
    % &= (\bm{TV}_{\bm{{\bm{\theta}}}} - \bm{V}_{\bm{{\bm{\theta}}}}-\omega\bm{1})^\top \mathbf{D} \bm{\Phi} (\bm{\Phi}^\top \mathbf{D} \bm{\Phi})^{-1} \notag\\
    % &\quad \quad \quad \bm{\Phi}^\top \mathbf{D} (\bm{TV}_{\bm{{\bm{\theta}}}} - \bm{V}_{\bm{{\bm{\theta}}}}-\omega\bm{1})\notag\\
    % &= (\bm{TV}_{\bm{{\bm{\theta}}}} - \bm{V}_{\bm{{\bm{\theta}}}}-\omega\bm{1})^\top {\bm{\Pi}}^\top \mathbf{D} {\bm{\Pi}} (\bm{TV}_{\bm{{\bm{\theta}}}} - \bm{V}_{\bm{{\bm{\theta}}}}-\omega\bm{1}) \notag\\
    &= \arg \min_{{\bm{\theta}}} \|\bm{\Pi}\bm{\mathcal{T}}_c^{\pi}\mathcal{C}\bm{V}_{\bm{{\bm{\theta}}}}-\mathcal{C}\bm{V}_{\bm{{\bm{\theta}}}}\|_{\bm{D}}^2\notag\\
    &= \arg \min_{{\bm{\theta}}} \|\bm{\Pi}(\bm{\mathcal{T}}_c^{\pi}\mathcal{C}\bm{V}_{\bm{{\bm{\theta}}}}-\mathcal{C}\bm{V}_{\bm{{\bm{\theta}}}})\|_{\bm{D}}^2\notag\\
    &= \arg \min_{{\bm{\theta}},\omega}\| {\bm{\Pi}} (\bm{\mathcal{T}}^{\pi}\bm{V}_{\bm{{\bm{\theta}}}} - \bm{V}_{\bm{{\bm{\theta}}}}-\omega\bm{1}) \|_{\bm{D}}^2\notag.
\end{align}
In the process of computing the gradient of the MSPCBE with respect to ${\bm{\theta}}$, 
$\omega$ is treated as a constant.
So, the derivation process of CTDC is the same 
as for the TDC algorithm \cite{sutton2009fast}, the only difference is that the original $\delta$ is replaced by $\delta-\omega$.
Therefore, the updated formulas of the centered TDC  algorithm are as follows:
\begin{equation}
 \bm{{\bm{\theta}}}_{k+1}=\bm{{\bm{\theta}}}_{k}+\alpha_{k}[(\delta_{k}- \omega_k) \bm{\bm{{\bm{\phi}}}}_k\\
 - \gamma\bm{\bm{{\bm{\phi}}}}_{k+1}(\bm{\bm{{\bm{\phi}}}}^{\top}_k \bm{u}_{k})],
\label{thetavmtdc}
\end{equation}
\begin{equation}
 \bm{u}_{k+1}= \bm{u}_{k}+\zeta_{k}[\delta_{k}-\omega_k - \bm{\bm{{\bm{\phi}}}}^{\top}_k \bm{u}_{k}]\bm{\bm{{\bm{\phi}}}}_k,
\label{uvmtdc}
\end{equation}
and
\begin{equation}
 \omega_{k+1}= \omega_{k}+\beta_k (\delta_k- \omega_k).
 \label{omegavmtdc}
\end{equation}
This algorithm is derived to work 
with a given set of sub-samples—in the form of 
triples $(S_k, R_k, S'_k)$ that match transitions 
from both the behavior and target policies. 

% \subsection{Variance Minimization ETD Learning: VMETD}
% Based on the off-policy TD algorithm, a scalar, $F$,  
% is introduced to obtain the ETD algorithm, 
% which ensures convergence under off-policy 
% conditions. This paper further introduces a scalar, 
% $\omega$, based on the ETD algorithm to obtain VMETD.
% VMETD by the following update:
% \begin{equation}
% \label{fvmetd}
%  F_t \leftarrow \gamma \rho_{t-1}F_{t-1}+1,
% \end{equation}
% \begin{equation}
%  \label{thetavmetd}
%  {{\bm{\theta}}}_{t+1}\leftarrow {{\bm{\theta}}}_t+\alpha_t (F_t \rho_t\delta_t - \omega_{t}){\bm{{\bm{\phi}}}}_t,
% \end{equation}
% \begin{equation}
%  \label{omegavmetd}
%  \omega_{t+1} \leftarrow \omega_t+\beta_t(F_t  \rho_t \delta_t - \omega_t),
% \end{equation}
% where $\rho_t =\frac{\pi(A_t | S_t)}{\mu(A_t | S_t)}$ and $\omega$ is used to estimate $\mathbb{E}[F \rho\delta]$, i.e., $\omega \doteq \mathbb{E}[F \rho\delta]$.

% (\ref{thetavmetd}) can be rewritten as
% \begin{equation*}
%  \begin{array}{ccl}
%  {{\bm{\theta}}}_{t+1}&\leftarrow& {{\bm{\theta}}}_t+\alpha_t (F_t \rho_t\delta_t - \omega_t){\bm{{\bm{\phi}}}}_t -\alpha_t \omega_{t+1}{\bm{{\bm{\phi}}}}_t\\
%   &=&{{\bm{\theta}}}_{t}+\alpha_t(F_t\rho_t\delta_t-\mathbb{E}_{\mu}[F_t\rho_t\delta_t|{{\bm{\theta}}}_t]){\bm{{\bm{\phi}}}}_t\\
%  &=&{{\bm{\theta}}}_t+\alpha_t F_t \rho_t (r_{t+1}+\gamma {{\bm{\theta}}}_t^{\top}{\bm{{\bm{\phi}}}}_{t+1}-{{\bm{\theta}}}_t^{\top}{\bm{{\bm{\phi}}}}_t){\bm{{\bm{\phi}}}}_t\\
%  & & \hspace{2em} -\alpha_t \mathbb{E}_{\mu}[F_t \rho_t \delta_t]{\bm{{\bm{\phi}}}}_t\\
%  &=& {{\bm{\theta}}}_t+\alpha_t \{\underbrace{(F_t\rho_tr_{t+1}-\mathbb{E}_{\mu}[F_t\rho_t r_{t+1}]){\bm{{\bm{\phi}}}}_t}_{{b}_{\text{VMETD},t}}\\
%  &&\hspace{-7em}- \underbrace{(F_t\rho_t{\bm{{\bm{\phi}}}}_t({\bm{{\bm{\phi}}}}_t-\gamma{\bm{{\bm{\phi}}}}_{t+1})^{\top}-{\bm{{\bm{\phi}}}}_t\mathbb{E}_{\mu}[F_t\rho_t ({\bm{{\bm{\phi}}}}_t-\gamma{\bm{{\bm{\phi}}}}_{t+1})]^{\top})}_{\textbf{A}_{\text{VMETD},t}}{{\bm{\theta}}}_t\}.
%  \end{array}
% \end{equation*}
% Therefore, 
% \begin{equation*}
%  \begin{array}{ccl}
%   &&\textbf{A}_{\text{VMETD}}\\
%   &=&\lim_{t \rightarrow \infty} \mathbb{E}[\textbf{A}_{\text{VMETD},t}]\\
%   &=& \lim_{t \rightarrow \infty} \mathbb{E}_{\mu}[F_t \rho_t {\bm{{\bm{\phi}}}}_t ({\bm{{\bm{\phi}}}}_t - \gamma {\bm{{\bm{\phi}}}}_{t+1})^{\top}]\\  
%   &&\hspace{1em}- \lim_{t\rightarrow \infty} \mathbb{E}_{\mu}[  {\bm{{\bm{\phi}}}}_t]\mathbb{E}_{\mu}[F_t \rho_t ({\bm{{\bm{\phi}}}}_t - \gamma {\bm{{\bm{\phi}}}}_{t+1})]^{\top}\\
%   &=& \lim_{t \rightarrow \infty} \mathbb{E}_{\mu}[{\bm{{\bm{\phi}}}}_tF_t \rho_t ({\bm{{\bm{\phi}}}}_t - \gamma {\bm{{\bm{\phi}}}}_{t+1})^{\top}]\\   
%   &&\hspace{1em}-\lim_{t \rightarrow \infty} \mathbb{E}_{\mu}[ {\bm{{\bm{\phi}}}}_t]\lim_{t \rightarrow \infty}\mathbb{E}_{\mu}[F_t \rho_t ({\bm{{\bm{\phi}}}}_t - \gamma {\bm{{\bm{\phi}}}}_{t+1})]^{\top}\\
%   && \hspace{-2em}=\sum_{s} d_{\mu}(s)\lim_{t \rightarrow \infty}\mathbb{E}_{\mu}[F_t|S_t = s]\mathbb{E}_{\mu}[\rho_t\bm{{\bm{\phi}}}_t(\bm{{\bm{\phi}}}_t - \gamma \bm{{\bm{\phi}}}_{t+1})^{\top}|S_t= s]\\   
%   &&\hspace{1em}-\sum_{s} d_{\mu}(s)\bm{{\bm{\phi}}}(s)\sum_{s} d_{\mu}(s)\lim_{t \rightarrow \infty}\mathbb{E}_{\mu}[F_t|S_t = s]\\
%   &&\hspace{7em}\mathbb{E}_{\mu}[\rho_t(\bm{{\bm{\phi}}}_t - \gamma \bm{{\bm{\phi}}}_{t+1})^{\top}|S_t = s]\\
%   &=& \sum_{s} f(s)\mathbb{E}_{\pi}[\bm{{\bm{\phi}}}_t(\bm{{\bm{\phi}}}_t- \gamma \bm{{\bm{\phi}}}_{t+1})^{\top}|S_t = s]\\   
%   &&\hspace{1em}-\sum_{s} d_{\mu}(s)\bm{{\bm{\phi}}}(s)\sum_{s} f(s)\mathbb{E}_{\pi}[(\bm{{\bm{\phi}}}_t- \gamma \bm{{\bm{\phi}}}_{t+1})^{\top}|S_t = s]\\
%   &=&\sum_{s} f(s) \bm{\bm{{\bm{\phi}}}}(s)(\bm{\bm{{\bm{\phi}}}}(s) - \gamma \sum_{s'}[\textbf{P}_{\pi}]_{ss'}\bm{\bm{{\bm{\phi}}}}(s'))^{\top}  \\
%   &&-\sum_{s} d_{\mu}(s) {\bm{{\bm{\phi}}}}(s) * \sum_{s} f(s)({\bm{{\bm{\phi}}}}(s) - \gamma \sum_{s'}[\textbf{P}_{\pi}]_{ss'}{\bm{{\bm{\phi}}}}(s'))^{\top}\\
%   &=&{\bm{\bm{\Phi}}}^{\top} \textbf{F} (\textbf{I} - \gamma \textbf{P}_{\pi}) \bm{\bm{\Phi}} - {\bm{\bm{\Phi}}}^{\top} {d}_{\mu} {f}^{\top} (\textbf{I} - \gamma \textbf{P}_{\pi}) \bm{\bm{\Phi}}  \\
%   &=&{\bm{\bm{\Phi}}}^{\top} (\textbf{F} - {d}_{\mu} {f}^{\top}) (\textbf{I} - \gamma \textbf{P}_{\pi}){\bm{\bm{\Phi}}} \\
%   &=&{\bm{\bm{\Phi}}}^{\top} (\textbf{F} (\textbf{I} - \gamma \textbf{P}_{\pi})-{d}_{\mu} {f}^{\top} (\textbf{I} - \gamma \textbf{P}_{\pi})){\bm{\bm{\Phi}}} \\
%   &=&{\bm{\bm{\Phi}}}^{\top} (\textbf{F} (\textbf{I} - \gamma \textbf{P}_{\pi})-{d}_{\mu} {d}_{\mu}^{\top} ){\bm{\bm{\Phi}}},
%  \end{array}
% \end{equation*}
% \begin{equation*}
%  \begin{array}{ccl}
%   &&{b}_{\text{VMETD}}\\
%   &=&\lim_{t \rightarrow \infty} \mathbb{E}[{b}_{\text{VMETD},t}]\\
%   &=& \lim_{t \rightarrow \infty} \mathbb{E}_{\mu}[F_t\rho_tR_{t+1}{\bm{{\bm{\phi}}}}_t]\\
%   &&\hspace{2em} - \lim_{t\rightarrow \infty} \mathbb{E}_{\mu}[{\bm{{\bm{\phi}}}}_t]\mathbb{E}_{\mu}[F_t\rho_kR_{k+1}]\\  
%   &=& \lim_{t \rightarrow \infty} \mathbb{E}_{\mu}[{\bm{{\bm{\phi}}}}_tF_t\rho_tr_{t+1}]\\
%   &&\hspace{2em} - \lim_{t\rightarrow \infty} \mathbb{E}_{\mu}[  {\bm{{\bm{\phi}}}}_t]\mathbb{E}_{\mu}[{\bm{{\bm{\phi}}}}_t]\mathbb{E}_{\mu}[F_t\rho_tr_{t+1}]\\ 
%   &=& \lim_{t \rightarrow \infty} \mathbb{E}_{\mu}[{\bm{{\bm{\phi}}}}_tF_t\rho_tr_{t+1}]\\
%   &&\hspace{2em} - \lim_{t \rightarrow \infty} \mathbb{E}_{\mu}[ {\bm{{\bm{\phi}}}}_t]\lim_{t \rightarrow \infty}\mathbb{E}_{\mu}[F_t\rho_tr_{t+1}]\\  
%   &=&\sum_{s} f(s) {\bm{{\bm{\phi}}}}(s)r_{\pi} - \sum_{s} d_{\mu}(s) {\bm{{\bm{\phi}}}}(s) * \sum_{s} f(s)r_{\pi}  \\
%   &=&\bm{\bm{\bm{\Phi}}}^{\top}(\textbf{F}-{d}_{\mu} {f}^{\top}){r}_{\pi}.
%  \end{array}
% \end{equation*}


% Safety model:
% - non-adversarial
% - only load by privileged users
% - developers are not malicious, but can make mistakes
%   - do not want to skip safety checks and want to be 100% sure that extensions
%     do not cause problems
% - safety checks therefore catch programming mistakes
% The safety model that is the same as that of eBPF
% The kernel has moved away from unprivileged eBPF and major distros ship the
%   kernel without unprivileged eBPF support.
% eBPF now can only be loaded by privileged users, the model is no longer
%   adversarial.
% - privileged users are always trusted to be not malicious
% - a privileged attacker almost always has other easier ways to attack than
%   using eBPF.
\section{Key Idea and Safety Model}
\label{sec:safety_model}
\vspace{-5pt}

The key idea of Rax is to realize {\it safe} kernel extensions without a separate layer of
    static verification.
% Rax targets large, complex kernel extensions that are increasingly suffering from the \gap{}.
Our insight is that the desired safety properties of kernel extensions
    can be built on the foundation of language-based properties of
    a safe programming language like Rust,
    together with extralingual runtime checks.
In this way, the in-kernel verifier can be dropped, and
    the \gap{} can be closed.
Rax extensions are strictly written in a {\it safe} subset of Rust.
We choose Rust as the safe language for kernel extensions
% Similar to related work~\cite{redleaf,theseus,tockos},
    (instead of other languages like Modula-3~\cite{spin}
    and Sing\#~\cite{lang-sing})
    because Rust is already supported
    by Linux~\cite{rust-for-linux-lwn} and offers
    desired language features for practical kernel code~\cite{redleaf,theseus,tockos}.
\new{Rax enforces the same set of safety properties eBPF enforces (\S\ref{sec:background}).}
Hence, Rax extensions fundamentally differ from unsafe kernel modules.
% As discussed in \S\ref{sec:motivation}, the current problem of usability of
%    eBPF is the large \gap{}. % between the programer, who works on
    % high-level languages, and the verifier, which operates on compiled bytecode.
% Our insight is that both the needed expressiveness and safety can be obtained
%     from a safe programming language without the verifier.

\para{Safety Model.}
Rax follows eBPF's non-adversarial safety model---the
    safety properties focus
    on preventing programming errors from crashing/hanging the kernel instead of
    malicious attacks.
Like eBPF, Rax extensions are installed from a trusted context with root
    privileges on the system.
Rax extensions can only be written in safe Rust with selected features
    and language-based safety is enforced out by a trusted Rust compiler (\S\ref{sec:lang_subset}). % and is strictly enforced;
Unlike Rust kernel modules that can use % bypass the compiler's safety checks
    unsafe Rust, the language-based safety of Rax extensions is strictly enforced.
Other safety properties that are not covered by language-based safety (e.g., termination)
    are checked and enforced by the lightweight Rax runtime.

While historically eBPF supported unprivileged mode~\cite{reconsider-unpriv-ebpf-lwn} and
    there are research efforts in supporting
    unprivileged use cases for kernel extensions~\cite{beebox-security24,safebpf-thomas,jia2023},
in practice, eBPF and other frameworks (e.g., KFlex~\cite{dwivedi-sosp24}) no longer pursue
    it~\cite{ebpf-sec-lwn,pawan-8a03e56b253e}.
The reasons come from inherent limitations of securing eBPF
    or kernel extensions in general.

First, it is hard for the eBPF verifier to prevent transient execution attacks like Spectre attacks
    completely, without major performance and compatibility overheads (see~\cite{ebpf-sec-lwn}).
Specifically, new Spectre variants are being discovered; though many of them are bugs in hardware,
    they cannot be easily detected and fixed by static analysis~\cite{perspective_isca}.
Sandboxing techinques cannot completely prevent Spectre attacks either,
    e.g., SafeBPF~\cite{safebpf-thomas} only prevents memory vulnerabilities,
    while BeeBox~\cite{beebox-security24} only focuses on two Spectre variants and requires manual instrumentation of helper functions.
For these reasons, the Linux kernel and major distributions also have moved away from unprivileged
    eBPF~\cite{pawan-8a03e56b253e,unpriv-ebpf-ubuntu,unpriv-ebpf-suse}.
%    effectively creating a non-adversarial safety model. % for \projname{} kernel extensions:

Second, eBPF chose not to be a sandbox environment (like WebAssembly or JavaScript)
    that does not know what code will be run~\cite{ebpf-sec-lwn}.
Instead, the development of eBPF assumes that ``{\it the intent of a BPF program is known}~\cite{ebpf-sec-lwn}.''

Lastly, the constantly reported verifier vulnerabilities~\cite{untenableVerification,ebpf-stackoverflow,ebpf-termination}
    indicate that a bug-free verifier is hard in practice.

    % Extensions are written by well-meaning but imperfect developers and thus the
%    code may contain programming mistakes but is not actively malicious.
% The imperfect developers want to be absolutely sure that their extension
%    programs does not lead to safety issues in the kernel and need the
%   programming mistakes to be detected beforehand.
% The so called ``guarantees'' of safety therefore consist of the best-effort
%    catching of common mistakes.

% We discuss how Rax enforces each of its safety properties in \S\ref{sec:principle}.




% \tianyin{TODO: map it to the safety defined in \S\ref{sec:motivation}}
% and is strictly enforced---the extensions cannot
%    bypass the checks.

%actively malicious extension writers can crash
%or hang the
%system~\cite{untenableVerification,ebpf-stackoverflow,ebpf-termination}
%(e.g., via helper functions), but the verifier prevents obvious
%mistakes (e.g., dereferencing a NULL pointer).

% TCB of Rax:
%   - Rust compiler
%   - Rax kernel crate
%   - Rax runtime
%
% Need to trust the correctness of the Rust compiler to enjoy the safety
%   guarrantee it provides
% - Core compiler is not too large, 72k loc vs. clang's 229k loc w/ tests
% - there are active efforts on ensuring correctness of Rust code that are
%   applicable to the Rust toolchain itself
%   - formal methods (Rustbelt)
%   - program analysis (Rudra)
%   - fuzzing (cargo-fuzz)
% This is aligned with the position of previous works (RedLeaf and Theseus)

\para{Trusted Computing Base (TCB).}
% \jinghao{TODO: decide whether to say ``toolchain'' here}
With Rax's safety model, the TCB consists of the Rust toolchain, the
    \projname{} kernel crate, and the \projname{} runtime.
\projname{} has to trust the Rust toolchain for its correctness to deliver
    language-based safety.
%    extension programs to enjoy the safety guarantees from Rust.
We believe the need to trust the Rust toolchain is acceptable
    and does not come with high risks with our safety model.
Recent work on safe OS kernels~\cite{theseus,redleaf,Miller-hotos19} makes the same decision
    to establish language-based safety by trusting the Rust toolchains.
The active effort on extensive fuzzing and formal verification of the Rust
    compiler~\cite{rust-belt,stacked-borrows-popl19,verus,verus-sosp24,rvt,rustc-fuzzing}
    may further reduce the risk.
Certainly, we acknowledge that the existing Rust compiler, such as rustc~\cite{rustc},
    is larger than the eBPF verifier.

% On the one hand, rustc~\cite{rustc}, the core compiler code of Rust is not
%    large, making it easier to reason about and ensure correctness than other
 %   larger compilers (e.g., Clang~\cite{clang}, also an LLVM frontend, contains
 %   three times as much code as Rustc). \jinghao{Not sure if this is a fair
 %   comparison}

% On the other hand, there have been various active efforts on ensuring
%    correctness of Rust code, especially unsafe Rust.
% These works ranges from applying formal methods on Rust
%    code~\cite{verus,rust-belt,stacked-borrows-popl19,astrauskas-oopsla19}, to
%   program analysis techinques~\cite{rudra-sosp21,qin-pldi20}, and to dynamic
%   approaches such as fuzzing~\cite{cargo-fuzz}.
% We expect these efforts to continue reducing the bugs and unsoundness in the
%    Rust toolchain and make it more trust-able.

% This is aligned with the position of previous works that also leverage the
%    safety promise from Rust~\cite{theseus,redleaf}.

% Rust-for-Linux does not satisfy the safety model
% - safety is merely a guideline, not a enforcement
% - unsafe code is allowed in module
% - kernel modules could still cause safety problems
% Note that the \projname{} safety model is different from that
%    of the Rust kernel modules, which is more relaxed and therefore cannot
%    satisfy our safety requirement.
% In Rust kernel modules, safety is not a property that is enforced, but merely a
%    guideline.
% This is because the full safety checks from the Rust compiler can be bypassed by
%    resorting to unsafe Rust, which is still allowed in the case of
%    kernel modules.
% The presence of such unsafe implementation could still cause safety problems and
%    undermine the safety guarrantees from Rust.

\vspace{-8pt}
\section{\projname{} Design}
\label{sec:principle}
\vspace{-5pt}
% - no clone impl for objects returned by RT crate to ensure uniqueness
% - difference between userspace Rust: there are things the compiler cannot see
%   e.g., we cannot have something like Mutex::get_mut in bpf_spinlock to
%   access a value w/o lock using the single-mutable-reference rule

\begin{figure}
    \centering
    \includegraphics[width=1.0\linewidth]{figs/rex-overview.pdf}
    \vspace{-15pt}
    \caption{Overview of the \projname{} kernel extension framework.
    The gray boxes are Rax components.}
    \label{fig:rax-overview}
    \vspace{-5pt}
\end{figure}

The key challenge of the Rax design is to provide safety guarantees of kernel extensions
    (listed in \S\ref{sec:background})
    on top of Rust's safe language features (adopting a safe language alone is insufficient,
    as in Rust kernel modules).

Figure~\ref{fig:rax-overview} gives an overview of the \projname{} framework.
To realize language-based safety, Rax enforces kernel extensions to be strictly
    written in {\it safe} Rust with selected features.
The \projname{} compiler toolchain rejects any Rax program that uses unsafe
    language features.
% and performs \projname{}-specific check on kernel stack usage.
% A Rax kernel extension must be  written in {\it safe} Rust
%    (no unsafe Rust code is allowed).
% Safe Rust provides inherent language-based safety, freeing Rax kernel
%     extensions of undefined behavior.
\new{Although this safe subset of Rust already provides inherent language-based
    safety within Rax extensions, eliminating undefined behaviors,
    safety of extensions is only achieved with the presence of safe kernel
    interactions provided by the \emph{\projname{} kernel crate}.
% Rax provides a specialized \emph{kernel crate} for kernel extensions.
% This is complemented with safe kernel interactions through the specialized
%     \projname{} \emph{kernel crate}.
The kernel crate is trusted and
%    contains a mixture of safe and unsafe Rust code
    % acts as a bridge between Rax extensions and unsafe kernel code.
    bridges Rax extensions with unsafe kernel code.}
Rax builds on top of the eBPF helper functions interface to provide a safe
    kernel interface for Rax extensions to interact with the kernel
    using safe Rust wrappers and bindings.
The safe interface encapsulates the interaction across the foreign
    function interface in the kernel crate.
\new{We reuse the eBPF helper interface, because it is designed
    for kernel extensions with a clearly defined programming contract and
    separates extensions from kernel's internal housekeeping
    (e.g., RCU~\cite{rcu}).}
% Rax relies on
%    a trusted Rust compiler which ensures the safety properties
%    of Rust. % and runs a \projname{}-specific pass for kernel stack safety.
%    for static analysis (e.g., lifetime and ownership)
%    and instrumentation (e.g. array bounds checks).
% \projname{} uses Rust as the safe language for extension programs.
% Rust is a system programming language that aims to provide
%    \emph{language-based} safety, promising programs free of \emph{undefined behaviors}.
%Rust is a system programming language that aims to provide
%    \emph{language-based} safety, which promises programs to be free of
%    \emph{undefined behaviors}.
% The compiler both performs static analysis (e.g., lifetime and ownership
%    models) and inserts dynamic checks (e.g. array bounds checks) in
%    Rust programs to guarantee safety.
%fulfill its safety promises.
% To facilitate low-level code that may be hard to fit into its safety model (e.g.
%    FFI calls), the language also provides \emph{unsafe} Rust, which is more
%    permissive and provides a weaker safety guarantee compared to
%    \emph{safe} Rust.


% Utilization of the language-based safety of Rust in kernel contexts has been
%    explored by previous works, and is also incorporated
%    into the Linux kernel~\cite{rust-for-linux-lwn}.
% This is because Rust happens to provide the desired properties -- its
%     language-based safety can be leveraged for safe kernel extensions without a
%     verifier.
% \new{
% In the context of safety for kernel extensions (\S\ref{sec:background}), Rust
%     provides both the desired properties directly and the building blocks to
%    ensure the important safety properties for extensions programs in
%    \projname{}.
% }
%In \projname{}, Rust provides both the desired properties directly and
%    building blocks to ensure the important safety properties for
%    kernel extensions (\S\ref{sec:background}).

% Here, we list the important safety properties
% (\jinghao{may need to move to \S\ref{sec:background}})
% \begin{itemize}
%     \item Memory safety
%     \item Type safety
%     \item Safe resource management
%     \item Exception-based runtime safety
%     \item Stack safety
%     \item Program termination
% \end{itemize}

% \jinghao{The runtime exception handling is not necessarily an independent one,
% it actually covers some of the type safety (arrays and slices does runtime
% checks that may trigger exceptions)}

% While the use of Rust automatically provides expressiveness (as it is
%     Turing-complete), it does not supply the safety out-of-box, and the use of
%     Rust, especially unsafe Rust, can still exhibit undesired behaviors.
% We require all extension programs to be implemented \emph{only} in safe Rust.
% On top of that, we discuss how the safety properties from Rust can be leveraged
%     and applied to the context of kernel extensions to provide a safe
%     programming interface.


Rax employs a lightweight extralingual in-kernel runtime that checks safety properties
    that are hard for static analysis. % statically in a sound way.
% The runtime enforces program termination and safe stack unwinding to
%     handle runtime exceptions (e.g., Rust panics).
\new{The runtime enforces program termination, kernel stack safety, and safe
    handling of runtime exceptions (e.g., Rust panics).}



% no unsafe
% no internal features
%   compiler intrinsics
% no unstable features
%   core::simd (portable simd)
% no function that change the resource management behavior
%   core::mem::forget
%   core::mem::ManuallyDrop
% no alloc
% no std
% also restricts the use of third-party crates
\vspace{-8pt}
% \subsection{Language-based safety}
\subsection{Safe Rust in \projname{}}
\label{sec:lang_subset}
\vspace{-3pt}

% principles
% 3 categories cannot be supported
%   1. unsafe code
%   2. features that interfere with compiler's management of object lifetimes
%   3. features cannot be supported in kernel space
\projname{} only allows language features that are safe in the context of
    kernel extensions.
First, Rax excludes any \textit{unsafe} Rust code as it misses
    important safety checks
    from the Rust compiler and can voilate various safety properties (\S\ref{sec:background}).
Second, \projname{} also forbids Rust
    features that interfere with Rust's automatic management of object
    lifetimes, which include \texttt{\small core::mem::\{forget,ManuallyDrop\}}
    and the \texttt{\small forget} intrinsic.
\new{These features are considered safe in Rust but
    violate resource safety of kernel extensions by facilitating resource
    leakage.}
Third, language features that cannot be supported in the kernel extension context are
    excluded by \projname{}.
This group contains the \texttt{\small std}~\cite{rust-std} library and
    dynamic allocation support (not available in \texttt{\small no\_std} build~\cite{rust-nostd}), the
    floating point and SIMD support (generally cannot be used in kernel space),
    and the \texttt{\small abort} intrinsic (triggers an invalid instruction).
Note that dynamic allocation may be supported by hooking
    \texttt{\footnotesize alloc}~\cite{rust-alloc} crate to the kernel
    allocator.
\new{
We plan to explore the use of dynamic allocation in kernel extensions in
    future works (\S\ref{sec:discussion}).
}

% Enforcement
% unsafe: -Funsafe_code
% mem::forget and abort: clippy:disallowed_{methods,types}
% std and alloc: already not available
% SIMD and float: -Funstable_feature and remove from target features

To enforce the restrictions, \projname{} configures the Rust compiler and linter
    to reject the use of prohibited features.
Specifically, we set compiler flags~\cite{rustc-lint} to forbid unsafe code and
    unstable Rust features which includes SIMD and intrinsics.
For individual langauge items such as \texttt{\small core::mem::forget}, we
    configure the Rust linter~\cite{clippy-lints} to detect and deny their usage.
We further remove floating point support by setting the target
    features~\cite{rustc-codegen} in \projname{} compilation.
The \texttt{\small std} library and dynamic allocation are already unavailable
    in \texttt{\small no\_std} environment used by \projname{} and therefore
    warrant no further enforcement.

% This subset contains exclusively safe Rust, thereby eliminates safety issues
%     from the unsafe counterpart.
% \projname{}'s nature as a kernel extension requires it to be built as a
%     standalone program, and therefore it cannot use the
%     \texttt{\small std}~\cite{rust-std} crate in Rust.
% \projname{} does not support dynamic allocation, and the
%     \texttt{\small alloc}~\cite{rust-alloc} crate cannot be used, either.
% %The \texttt{\small alloc} crate cannot be used, either, due to the fact that
% %    \projname{} does not hook it to the kernel allocation function.
% %We do not hook \texttt{\small alloc} with the kernel allocation function and it
% %    cannot be used, either.
% The \texttt{\small core}~\cite{rust-core} crate remains largely available to
%     \projname{} programs, with a few exceptions:
% \projname{} prohibits the use of features in \texttt{\small core} that are
%     unstable or internal to the compiler and library, such as
%     SIMD~\cite{rust-core-simd}, which is not be used in the kernel, and
%     intrinsics~\cite{rust-core-intrinsics}, which are too low-level and hard to
%     be used correctly\footnote{
% Many of the intrinsics already have high-level wrappers in the
%     \texttt{\footnotesize core} library, e.g. atomics, which can still be used by the
%     program.}.
% We further restricts the usage of
%     \texttt{\small core::mem::\{forget,ManuallyDrop\}}~\cite{rust-core-mem-forget,rust-core-mem-manuallydrop}
%     because they lead to risks of resource leaks by not automatically invoking
%     object destructors.
% Lastly, \projname{} restricts the use of third-party crates other than the
%     \projname{} kernel crate, since it is challenging to ensure safety on these
%     crates.
% To enforce the requirements, we use linter options from the Rust
%     toolchain~\cite{rustc-lint,clippy-lints} to restrict language features, and
%     check the project cnofiguration file to eliminate third-party dependencies.

% The key challenge of developing Rax is ensuring safety properties of
%    Rax kernel extensions effectively and efficiently.
% In the remainder of this section, we discuss how each safety property is
%    realized in Rax.

% It also provides a custom Rust panic handler to support panic-based runtime
%    safety checks in Rust.
% The program links with the kernel crate at compile time and
% Rax runs in a
%    light-weight runtime environment implemented in the kernel, which provides
%    program termination and stack unwinding support.

% We now discuss how the language-based safety from Rust can be applied to the
%    context of kernel extensions to provide a safe programming interface.

\vspace{-4pt}
\subsection{Memory safety}
\label{principle:memsafety}
\vspace{-3pt}

Rax enforces extensions to access kernel memory safely.
There are two common memory access patterns, depending
    on the ownership of the memory region:
    (1) memory owned by the extension (e.g., a stack buffer) is sent to the kernel
    through helper functions,
    and (2) memory owned by the kernel (e.g., a kernel struct) is accessed by
    the extension.

\para{Memory owned by extensions.}
A Rax extension can allocate memory on the stack
    and send it to the kernel (e.g., asking the kernel to fill a
    stack buffer with data) via existing eBPF helper functions.
%An unsafe memory access could result in corruption of stack data and possibly
%    trigger a kernel crash if the return address is overwritten to a non-present
%    page.
Rax ensures no unsafe memory access and thus prevents stack buffer overflow and
    kernel crash (e.g., corruption of the return address on the stack).
%An unsafe memory access, for example, an out-of-bounds write to the on-stack
%    buffer, could result in corruption of stack data and possibly trigger a
%    kernel crash if the return address is overwritten to a non-present page.

% In eBPF, the verifier validates the memory region with its size via the helper function interface
%    at load time to ensure no erroneous access at runtime.
% The kernel operates on the memory region with the knowledge of
%    its bounds from the helper function argument, avoiding unsafe memory accesses.
%For memory buffers sent to the kernel through the helper function interface,
%    the corresponding size is also sent as an argument to the helper.
% Memory buffers sent to the kernel via the helper function interface have to be
%    paired with its size as an argument.
%The verifier, at program load time, checks to ensure that the size is exactly
%    that of the memory buffer.
% Specifically, the verifier verifies at load time that the size matches that of the
%    memory buffer.

Unlike eBPF that checks a memory region with its size of
    every invocation of helper functions,
in \projname{}, the strict type system of Rust already prevents unsafe access.
\projname{} leverages the generic programming feature of
    Rust~\cite{rust-generics} to ensure that the size sent through the helper
    function interface is always valid.
% In order to make sure that the size sent to the kernel is always correct, which
%     is vital for the correctness of memory access from the kernel, we leverage
%     the generic programming support of Rust.
For helper functions that take in pointer and size as inputs,
    the \projname{} kernel crate creates an
    adaptor interface that parametrizes the pointer type as a generic type parameter.
The interface queries the size of the generic type from the compiler
    and invokes the kernel interface with this size as an argument.
Since Rust uses {\it monomorphization}~\cite{rustc-monomorphize}, the concrete
    type and its size are resolved at compile time, adding no runtime overhead.
In this way, the size is guaranteed to match the type statically and the
    kernel will never make an out-of-bound access.
This works for both scalar types and array types.
We use Rust's {\it const generics} to allow a constant to be used as a
    generic parameter~\cite{rust-generics} to encode array lengths.

\para{Memory owned by the kernel.} The kernel can provide
    extensions with a pointer to
    kernel memory (e.g., map value
    pointers and packet pointers).
The extension must not have out-of-bound memory access. %, as doing so risks corrupting kernel memory.
%In eBPF, for accesses on pointers with a static size, e.g. map value pointers,
%    as maps store the size of its value types, the verifier can independently
%    verify the validity.
In eBPF, the verifier checks uses of kernel pointers with a static size,
    e.g. map value pointers (maps store the size of values);
for pointers without a static size like packet data pointers, the verifier
    requires extensions to explicitly check memory boundaries.

In \projname{}, pointers with static sizes are handled through the Rust type
    system.
The kernel map interface of \projname{} encodes the key and value
    types through generics and returns such pointers to extension programs as
    safe Rust references.
%The type system forces the pointer to be a safe Rust reference of
%    the particular type.
%Pointers referring to a memory region without a static size are generally
%    dynamic-sized buffers.
To manage pointers referring to dynamically sized memory regions,
    the \projname{} kernel crate abstracts such pointers into a Rust
    \emph{slice} with dynamic size.
Rust slice provides runtime bounds checks (\S\ref{principle:eh}), which allows
    the check to happen %automatically
    without explicit handling by the
    extension.

% dynptr
%   ptr to a dynamically sized data region with metadata (size, type, r/w)
% Differences:
%   - access to the data region must have a known size
%   - accessing through read/write helpers does not provide zero copy
%     implementation
%   - obtained slice with data or slice helpers still requires explicit size
%     check from program

Rust slices are in principle similar to \texttt{\small dynptr}
    in eBPF~\cite{ebpf-dynptr-lwn}, but provide more flexibility.
eBPF \texttt{\small dynptr}s are pointers to dynamically sized data regions
    with metadata (size, type, etc); however,
% Extensions can access \texttt{\small dynptr}s via helpers and kfuncs, or obtain a
%    data slice to operate on it directly.
    access to the \texttt{\small dynptr}'s referred memory must be of a static size.
Rust slices allow dynamically sized access to the underlying
    memory, benefiting from its runtime bounds checks.
Moreover, the \texttt{\small bpf\_dynptr\_\{read,write\}} helpers do not
    implement a zero-copy interface available in Rust slices.
While \texttt{\small bpf\_dynptr\_\{data,slice\}} helpers
    allow extensions to obtain data slices without copying, they
    again require explicit checks of the bound of the slice.
As a tradeoff, eBPF \texttt{\small dynptr}s avoids runtime overheads of
    dynamic bounds checks, which we find negligible in our evaluation (\S\ref{eval:macro}).

\vspace{-4pt}
\subsection{Extended type safety}
\vspace{-3pt}

Rax extends Rust's type safety to allow extension programs to safely convert a byte stream
    into typed data.
% Kernel extensions may contain certain paradigms that are beyond the safe
%    type system of Rust.
% One challenge is to allow extension programs to safely reinterpret a stream
%    of bytes into useful data.
The pattern is notably found in networking use cases, where extensions need
    to extract the protocol header from a byte buffer in the packet as a struct.
Safety of such transformations is beyond Rust's native type safety
    because they inevitably involve unsafe type casting.
% For example, a XDP program using direct packet access might need to extract
%     the ethernet header information from the bytes in the packet.
eBPF allows pointer casting;
% Currently, eBPF allows the program to freely interpret the packet data into
%    other data types via pointer casting.
the verifier ensures: (1) the program does not make a
    pointer from a scalar value, and (2) the new type fits the memory boundary.

\projname{} also enforces the above two properties so that
    the reinterpreting cast (dubbed ``transmute'' in Rust) is safe.
\projname{} extends Rust's type safety to cover such casts.
To do so,
% In the case of Rust, the reinterpret cast (dubbed ``transmute'' in Rust) is an
%     unsafe operation, particularly because Rust does not prevent making
%     pointers from scalar values.
% To allow extension programs to safely reinterpret the bytes into useful
%    data types,
\projname{} defines a set of primitive scalar types that are
    considered safe as targets for casting.
\projname{} requires the target type of casting to be of either one of the safe types
    or a structure type in which all the members are of the safe types.
The safe types are specified by implementing the \texttt{\small Rax::SafeTransmute}
    trait, which is sealed and only implementable from within the kernel crate.
We use the \emph{procedural macros}~\cite{rust-proc-macro} feature of Rust to enforce
    this constraint and generate the safe transmute interface at compile time on
    extended types.
%The developer is still responsible for using the correct scalar types (e.g.,
%    endianness).
Under the safe-transmute contraints, mismatched scalar types can only cause
    logical errors, but do not constitute safety violation.

% Like conventional C macros, proc-macros performs transformation on the program,
%     albeit on the abstract syntax tree level.
% \mvle{i suggest cutting the rest of the paragraph and putting it in appendix.}
% Our proc-macro, when applied on a structure type, generates code that tries
%     to treat each field of the structure as an instance of
%     \texttt{Rax::SafeTransmute} using Rust trait bounds, followed by the actual
%     transmute operation to perform the unsafe cast.
% If one of the fields in the structure does not implement
%     \texttt{Rax::SafeTransmute}, the Rust compiler will issue a compile error.
% At the same time, if the program tries to directly transmute the bytes into a
%     structure without using the proc-macro, the compiler will also emit an
%     error because transmute belongs to unsafe Rust, which is forbidden in
%     \projname{} programs.
% Since proc-macro transformations happen after the linting of unsafe operations,
%     the transmute code generated by the proc-macro will not be rejected and,
%     therefore, allows programs to perform transmutes in a safe, and controlled
%     way.
% \jinghao{Shall we remove this sentence, people may ask whether programs can
% use proc-macros to get around the safe Rust requirement}

\vspace{-4pt}
\subsection{Safe resource management}
\vspace{-3pt}

%Executing in the kernel,
Rax extensions are ensured to acquire and release
    resources properly to avoid leaks of kernel resources (e.g., refcounts
    and spinlocks).
Different from eBPF where the verifier checks all possible code paths
    to ensure the release of acquired resources,
% Some kernel helper functions return ``referenced'' kernel resources that require
%    explicit release after use through the paired helper function calls.
% Failing to do so will result in leakage of kernel resources such as .
% In eBPF, some kernel helper functions returns kernel resources that requires
%     explicit release after use (e.g., reference counts of kernel objects,
%     locks).
\projname{} uses Rust's Resource-Acquisition-Is-Initialization
    (RAII) pattern~\cite{rust-raii}---for every kernel resource
    a Rax extension may acquire, the \projname{} kernel crate defines an RAII wrapper type
    that ties the resource to the lifetime of the wrapper object.
% \new{
% The same approach is also adopted by other Rust-based operating system and
%    kernel modules~\cite{rust-for-linux-doc,theseus}.
% }
% \tianyin{Is this the same as Rust kernel module (see the review)? If not, what's the difference?}

For example, when the program obtains a spinlock from the kernel, the
    \projname{} kernel crate constructs and returns a \emph{lock guard}.
The lock guard implements the RAII semantics through the
    \texttt{\small Drop} trait~\cite{rust-drop} in
    Rust, which defines the operation to perform when the object is destroyed.
In the case of lock guard, its \texttt{\small drop} handler releases the lock.
\new{
\projname{} uses compiler-inserted \texttt{\small drop} calls at the end of
    object lifetime during normal execution, and implements its own resource
    cleanup mechanism (\S\ref{principle:eh}) for exception handling.
}
The use of RAII automatically manages kernel resources to ensure
    safe acquisition and release.
Extension programs do not need to explicitly release the lock or drop the lock
    guard.
% Instead, the compiler inserts an implicit \texttt{\small drop} at the end
%    of the current scope, which effectively releases the lock when the lock
%    guard goes out of scope.

\vspace{-4pt}
\subsection{Safe exception handling}
\label{principle:eh}
\vspace{-3pt}

While certain Rust safety properties are enforced statically by the compiler,
    the others are checked at runtime and their violations trigger exceptions (i.e., Rust panics).
%    which is different from eBPF.
To handle exceptions in userspace, Rust uses the Itanium exception handling ABI~\cite{itanium-abi} to
    unwind the stack.
A Rust panic transfers the control flow to the stack unwinding
    library (e.g., llvm-libunwind), which backtracks the call stack and executes
    cleanup code and catch clauses for each call frame.
Unfortunately, this ABI is unsuitable for kernel extensions:
% \tianyin{How is it different from Theseus?}
% for the
%    following reasons as mentioned by~\cite{untenableVerification}:
%When an exception is triggered, the control flow is transferred to the Rust
%    panic handler in its standard library, which in turn calls into the unwind
%    library to perform stack unwinding and resource cleanups for each stack
%    frame.
%However, the Itanium ABI is complex and not suitable for kernel extensions,
%    making exception handling a challenge in \projname{} extension programs:
%\jinghao{Took these directly from the HotOS paper, shall we paraphrase?}
\begin{packed_itemize}
%     \item The Itanium ABI-based exception handling is too complicated, as the
%         userspace unwind libraries are not direcly usable.
    \item Unlike in userspace
        where failures during stack unwinding
        crash the program,\footnote{Theseus~\cite{theseus} implements stack unwinding in the kernel.
        But, it assumes that unwinding never fails; faults in unwinding result in kernel failures.}
%        \jinghao{it looks like is
%        unwinding fails it would just stop unwinding and directly kill the
%        failed task. Although it does not kill the kernel, this means incomplete
%        cleanups.}}
    stack unwinding in kernel extensions cannot fail---kernel
        % Failures during unwinding, which are permissible in userspace, cannot
        extensions must not crash the kernel and must not leak kernel resources.
%    \item ABI-based unwinding typically requires dynamic allocation, which
%        creates challenges for extensions in interrupt contexts, in which an
%        allocator may not be available~\cite{bpf-mempool-lwn}. \jinghao{I'm
%        thinking about removing this one since bpf now has an allocator.}
    \item Unwinding generally executes destructors for all existing objects on
        the stack, but executing untrusted, user-defined destructors (via the
        \texttt{\small Drop} trait~\cite{rust-drop} in Rust) is unsafe.
\end{packed_itemize}
Rax implements its own exception handling framework with two main components: (1) graceful exit
    upon exceptions, which resets the context, and (2) resource cleanup to
    ensure release of kernel resources (e.g., reference counts and locks).

\para{Graceful exit.}
To ensure a graceful exit from an exception, \projname{} implements a small
    runtime (Figure~\ref{fig:eh-overview}) in the kernel, which
    consists of a program dispatcher, a panic handler, and a landingpad.
The dispatcher takes the duty of executing the extension program
    (like the eBPF dispatcher).
It saves the stack pointer of the current context into per-CPU
    memory, switches to the dedicated program stack (\S\ref{principle:stack}),
    sets the termination state (\S\ref{principle:termination}), and
    then calls into the program.
If the program exits normally, it
    returns to the dispatcher, which switches the stack back and clears the
    termination state.
Under exceptional cases where a Rust panic is triggered, the panic handler
    releases kernel resources currently allocated by the extension, and
    transfer control to the in-kernel landingpad to print
    debug information to the
    kernel ring buffer and return a default error code to the kernel.
%     \new{and set a default return value depending on the
%    program type}.
Then, the landingpad redirects control flow to a pre-defined label
    in the middle of the dispatcher, where it restores the old value
    of the stack pointer from the per-CPU storage.
This effectively unwinds the stack and resets the context as if the extension returned successfully.
% We implement the program dispatcher function with panic handling in the kernel
%     in 28 lines of x86 assembly code.


\para{Resource cleanup.}
Correct handling of Rust panics requires cleaning up resources acquired
    by the extension.
However, static approaches that rely on the verifier to pre-compute resources
    to be released during verification
    (e.g., object table in~\cite{dwivedi-sosp24}) do not apply to Rax due to the
    \gap{}.
% Rax aims to avoid
%    artificial constraints for static verification (e.g.,
%    resources acquired in a loop iteration must be released by the end of the
%    iteration~\cite{dwivedi-sosp24}), as they exhibit the \gap{}.
% \jinghao{I think we need to remove this sentence, since now we say we only support single lock.}
%    resource acquisition \new{without additional constraints
%    other than that of Rust language}). \tianyin{why more flexible?}
%    \jinghao{KFlex still requires that, in a loop, the acquisition and cleanup
%    of a resource must happen in the same iteration. (Last paragraph in their
%    3.1). This is what allows them to statically compute the obj table}
% KFlex~\cite{dwivedi-sosp24} uses static "object table" approach, where the
%    list of resources to be released is generated .
% This is not suitable for \projname{}, as \projname{} is more relaxed on
%    resource acquisition, which makes it impossbile to static compute the list
%    of resources to be released.

%Light-weight mechanisms can be effective for resource cleanup in \projname{}.
Our insight is that extensions can only obtain
    resources by explicitly invoking helper functions. %; so, only these resources need to be released.
So, Rax records the allocated kernel resources
    during execution in a per-CPU buffer, which is in principle like the global
    heap registry in~\cite{redleaf}.
% \jinghao{Removed the static buffer part here, since it attracts reviewer's
% attacks. But at the same time, it is also hard for us to say it's growable,
% as that will get scrutiny on the memory alloc part...}
%Our current implementation supports up to 64
%    instances of kernel resources during a single run.
Upon a panic, the panic handler takes the responsibility to correctly
    release kernel resources, which involves traversing the
    buffer and dropping recorded resources.
% performing cleanup for each of the recorded resources.

\begin{figure}
    \includegraphics[width=0.95\linewidth]{figs/exception_handling.pdf}
    \centering
    \caption{Exception handling control flow in \projname{}}
    \label{fig:eh-overview}
    \vspace{-5pt}
\end{figure}




% \jinghao{It seems that nesting should be put somewhere, due to the fact that
% these per-CPU tricks depend on no-nesting.}
\new{
We implement the cleanup code as part of the panic handler in the \projname{}
    kernel crate, as it
    is responsible for coordinating helper function calls
    that obtain kernel resources.
% In this way, the resource cleanup is completely transparent to the extension
%     programs, adding no additional programming burden like the Itanium ABI.
Implementing the cleanup mechanism in the kernel crate ensures safety:
    as the code is called upon panic, it must not trigger deadlocks or yet
    another Rust panic to fail panic handling.
The careful design of the \projname{} kernel crate frees the cleanup code and
    \texttt{\small drop} handlers of locks and panic-triggering code.
% We carefully implement the cleanup code and \texttt{\small drop} handlers in the
%     kernel crate such that they do not trigger panics or hold locks, which may
%     hinder successful panic handling.
Kernel functions invoked by such code may still hold locks internally, but they
    are self-contained and do not propagate to \projname{} (deadlocks in kernel
    functions is out of the scope of \projname{}).
% For this reason,
\projname{} does not execute user-supplied \texttt{\small drop} handlers
    upon panic, as they are not guaranteed to be safe under panic handling
    context.
}
%and programs cannot allocate resources without helper function
%    calls.

\projname{} implements a crash-stop failure model---a panicked
    extension is removed from the kernel.
Any used maps and other \projname{} extensions sharing the maps
    will also be removed recursively.
% \mvle{should define what it means to be associated, e.g., different instances or have access to same resources}
This prevents extensions
    sharing the maps from running in a potentially inconsistent
    state---exception handling in \projname{} already ensures the kernel is
    left in a good state.

\vspace{-4pt}
\subsection{Kernel stack safety}
\label{principle:stack}
\vspace{-3pt}

Kernel extensions should never overflow the kernel stack.
% One unique safety requirement that kernel extension programs face is the
%     usage of kernel stack.
Unlike userspace stacks which grow on demand with a large maximum size,
    the stack in kernel space has a fixed size (4 pages on x86-64).
% Overflowing the kernel stack may result in memory corruptions or kernel panics.
The eBPF verifier checks stack safety by calculating stack size
    via symbolic execution.
However, it is reported that stack safety is broken in eBPF due to the difficulties
    of statically analyzing indirect tail calls~\cite{ebpf-stackoverflow}
    and uncontrolled program nestings~\cite{chintamaneni-ebpf24}.\footnote{Rax currently does not
    support program nesting (same as eBPF)}
%    even in eBPF it is unclear how to support nesting with safety.}
% \tianyin{how does KFlex does that? Is it also broken?}
% \jinghao{This is not explicitly mentioned in the paper. Based on my
%    understanding of the work, only heap access and termination is split out
%    and use SFI instead, so I would say the stack is still handled by the
%    verifier and it should be still broken.
% }

Our insight is that stack safety can be enforced
    at compile time to avoid runtime overhead if the extension program has no indirect or
    recursive calls, as %the global call graph is a directed acyclic graph (DAG)
    the stack usage can be statically computed.
Otherwise,
% When the program does employ indirect or recursive calls,
    it is easy to check stack safety at
    runtime. %when the program does employ indirect or recursive calls.
\projname{}, therefore, takes a hybrid approach and selects between static and dynamic checks based on the situation.
%    utilizes static checks for programs without indirect or recursive calls,
%    and runtime checks for programs that have such calls.

\para{Static check.}
The static check is done by a \projname{}-specific compiler pass (\S\ref{sec:impl}).
% \mvle{forward reference to implementation}
%For each program being compiled, the \projname{}-specific pass in the compiler
\new{
If the extension has no indirect or recursive calls,
    its total stack usage can be calculated by traversing its global static
    callgraph and sum up the size of each call frame.
% The path with the highest stack usage is the total stack usage of the program.
We turn on fat LTO and use a single Rust codegen unit~\cite{rustc-codegen} for
    \projname{} programs to ensure the compiler always has a global view across
    all translation units.
}
% If the total stack usage of the program exceeds total amount of stack
%    available, the \projname{} compiler pass will generate an error and reject
%    the program.

\para{Runtime check.}
For extensions with indirect or recursive calls, it is hard to calculate the
    % path with the highest
    stack usage from the callgraph due to the
    presence of unknown edges (indirect calls) and cycles (recursive calls).
%  are hard to check  for stack usage statically, as is the case in the eBPF verifier.
Under these cases, \projname{} performs runtime checks.
The \projname{} compiler pass first ensures each function in the program takes
    less than one page (4K) of stack.
This is more relaxed than the frame size warning threshold (2K) in Linux
    and ensures enough stack to handle Rust panics.
% \ayushb{A supporting argument on why the function size is being limited to one page might be helpful here, the reasoning behind this is not known to everyone and might be non-trivial assumption.}
Before each function call in the extension, the compiler inserts a
    call to the \texttt{\small rax\_check\_stack} function from the kernel crate to check the
    current stack usage: if the stack usage exceeds the
    threshold, it will trigger a Rust panic and terminate the
    program safely (\S\ref{principle:eh}).

% In order for \texttt{\small rax\_check\_stack}
To manage stack usage of Rax extensions effectively, \projname{}
    implements a dedicated kernel stack for each
    extension.
The dedicated stacks are allocated per-CPU and virtually mapped at
    kernel boot time with a size of eight pages.
% the same size as the kernel IRQ and task stacks.
Before executing a Rax extension,
    the dispatcher (Figure~\ref{fig:eh-overview}) saves the stack
    pointer of the current context, and
%In the dispatcher function (Figure~\ref{fig:eh-overview}), before calling into
%    the \projname{} program, the stack and frame pointers of the current
%    context are saved.
    then sets the stack and the frame pointer
    (already saved with other callee-saved registers) to the
    top of the dedicated stack.
When the extension exits, the original stack and frame pointers are restored.
% Upon exit of the extension, the original stack and frame pointers are restored, no
%    matter whether a exception is triggered by the extension.

%\projname{} sets the stack usage threshold to be two pages less than the
%    total stack available, with the following considerations:

\projname{} sets the stack usage threshold to be four pages for extension
    code; it reserves the next four pages with following considerations:
(1) helper functions are not visible at compile time but they
    also account for stack usage during execution;
%    giving extra spaces accommodates helper functions, and
    we use four pages as the de facto stack size used by the kernel itself, and
(2) since stack usage of each function is limited to
    one page of stack, in the worse case, the remaining stack space is at least
    three pages when \texttt{\small rax\_check\_stack} triggers a Rust panic.
\new{
Since the panic handler is implemented in the kernel crate and does not change
    with programs, this worse-case guarantee empirically leaves enough space for
    panic handling and stack unwinding.
}
\projname{}'s dynamic approach achieves stronger stack safety than that of
    eBPF.

\vspace{-4pt}
\subsection{Termination}
\label{principle:termination}
\vspace{-3pt}
% A bounded program runtime is important for kernel safety
%   - a potentially long running program can hold the CPU and cause kernel
%     lockups
% eBPF handles this by imposing verification limit (ref S3)
%   (Side note:
%    First, this is really ``bounded runtime'' rather than termination, because
%      a long-running program that eventually terminate is almost equally bad
%    Why would programs ever hit that limit if they are supposed to have an
%    acceptable runtime? Is it because of these two reasons:
%      1. The verifier misses some key information that can reduce the search
%         space drastically, as is the case of unbounded loops?
%      2. If the program is meant to execute that many instructions (current
%         limit is 1M), is it because the current limitation is still too small
%         or is it that the program is badly implemented and actually runs
%         longer? (Also, how long would 1M eBPF instructions run with
%         interpretation and JIT?)
%   )
%   - A program that goes over the limit gets rejected, no matter whether it will
%     actually run long
%   - creates usability issues
%   - previous works has also demonstrated ways of creating long running programs
%     without going over the limit (cite raj-lpc and jinghao-hotos)
%
% Given the ineffectiveness and usability issues associated with the static
%   verification approach, Rax uses a dynamic mechanism that interrupts and
%   terminates programs.
% When a program goes over its time limit, it will be terminated.
% The dynamic termination in Rax limits the runtime of the program through a
%   timer.
% When the timer expires, it issues an IPI to the CPU the program is running.
% The IPI effectively suspends the program and saves its registers to the stack.
% Inside the IPI handler, the timeout handler for Rax is executed, which
%   modifies the instruction pointer stored on the stack to the panic handler
%   of rax programs.
% After returning from the IPI the program will be executing the panic handler
%   to gracefully exit.
%
% Challenge: no termination in helpers/panic handlers
%
% We use a per-CPU flag to solve this

Termination is an important property of kernel extensions.
% critical to the
%    safety of the kernel, as non-terminating extension programs could hold the
%    processor for a long time and cause kernel lockups.
% }
%eBPF ensures termination by imposing a static limit on
%    the number of instructions the verifier explores. % go through.
% eBPF ensures termination by disallowing back edges in the program and by
%    imposing a static limit on the number of instructions the verifier explores.
In eBPF, an extension with a back edge or exceeds the instruction limit will be
    rejected, regardless whether it eventually terminates. %no matter whether the program will actually run for a long time.
% Such practices create many false positives and greatly hinders the
%    expressiveness of extension programs.
%is one of the main sources of
%    usability issues in eBPF (\S\ref{motivation:restructure}).
KFlex~\cite{dwivedi-sosp24} lifts the back edge restriction by inserting
    cancellation points in eBPF bytecode on all back edges during
    verification, which triggers termination at runtime.
However, back edge analysis is non-trivial outside
    eBPF bytecode and is unreliable for general Rust programs.
% poses challenges for \projname{}, which does not rely on the eBPF bytecode.
%\ayushb{eBPF static verifier is able to do back-edge analysis and thus prevent loops into the program, giving up on that poses issues specially for programs executing in non-maskable interrupts. This might pose a problem from the usability perspective itself, something that we are trying to solve. Maybe we can consult with Xudong as well on how big of a problem can this be.}
%\jinghao{Thinking about this again, yes we lose some analysis, but I don't think
%    the problem with hardirq has anything to do with the usability issues we try
%    to solve -- we never reject a correct program (and I think a
%    spanning-tree-like algorithm might make synchronous instrumentation possible.
%    We can keep the async way for anything that runs outside hardirq)}

%Previous works~\cite{ebpf-termination,untenableVerification} have also
%    demonstrated creation of long-running eBPF programs without going over the
%    verification limits, further weakening the runtime guarantee from the
%    verifier.
% \jinghao{Removed the discussion on exploiting long-running eBPF programs, since
%    1) I think termination (more relaxed than bounded prog runtime) is what we
%    are looking for, and 2) reviewer B has problems with our interpretation of
%    the verifier limits (``nothing to do with the runtime limits of the program'')
%    so I think the only thing this limit tells us is about the termination
%    property (programs within the verifier limit is guaranteed to terminate).}

% This is still missing the discussion of the design decisions, specifically:
%   Why hrtimer
%     because software timer wheel does not interrupt softirq
%   Why not arming timer at program entry rather than a periodic timer
%     exceesive overhead on hot path for timer setup (as pointed out in reviews)
%   Why not just use a single timer for all CPUs?
%     doing so requires sending IPIs to other CPUs, and sending them from
%     hardirq risks kernel deadlock
\projname{} employs a runtime that
    interrupts and terminates extensions that run for too long.
% \jinghao{Need to discuss: how should we argue that this leads to fewer FPs and is more usable?}
\projname{} limits the run time of extensions by
    leveraging kernel timers as watchdogs.
% A program will be terminated if it reaches a timeout implemented by the timers.
% Design decisions on how to leverage the hrtimer
Rax builds the runtime on the high resolution timer
    (\texttt{\small hrtimer}) subsystem in Linux~\cite{linux-hrtimer}.
Since \texttt{\small hrtimer} callbacks execute in hardware timer interrupts,
    they are capable of interrupting the contexts in which most extensions
    execute (soft interrupts and task
    context~\cite{elce-16-chaiken}).
Since hardware timer interrupts are periodically raised
    by the processor, regardless whether an \texttt{\small hrtimer} is present,
    executing timer callbacks in this existing hardware timer interrupts adds
    no extra interrupt or context switch, keeping the
    watchdog overhead minimal.

Rax sets one timer for each CPU to
    avoid inter-core communication, in contrast to using a
    single, global timer to handle programs from all CPUs.
Each timer only needs to monitor extensions running on the core.
Rax arms the timers at kernel boot time, which are triggered periodically
    with a constant timeout, and re-armed each time after
    firing.\footnote{\new{Disarming the timer when no extension is running saves
    CPU cycles, but
%     Arming the timer right before starting each extension
    incurs high
    overhead due to timer setup on the hot path of extension execution,
    especially for frequently triggered extensions (e.g., XDP
    extensions~\cite{cilium-docs})}}

\projname{} implements its watchdog logic in the timer handlers.
When a timer fires, its handler
%be executed in a hard interrupt (hard-IRQ)
%    on the current CPU, which
    suspends any soft interrupt or task context, and saves its
    registers.
The handler then checks the current CPU on whether the termination timeout
    of the \projname{} extension in the stopped context has been reached.
This is done by comparing the extension start time (stored as a per-CPU state as
    shown in Figure~\ref{fig:eh-overview}) with the current time.
% If the program exceeds the runtime threshold, an inter-processor interrupt (IPI)
%     is sent to the CPU the program is running on.
% This IPI effectively suspends the program and pushes its registers onto the
%     stack.
% Inside the IPI handler, the timeout handler of \projname{} is
%    executed, which sets the saved instruction pointer register to the
%    panic handler of the program.
If the extension exceeds the threshold, the timer handler overwrites the
    saved instruction pointer register to the panic handler (\S\ref{principle:eh}).
After returning from the timer interrupt, the extension executes its
    panic handler, which cleans up kernel resources and gracefully exits.
\projname{} sets both the timer period and runtime threshold to
    % to 21 seconds, which is
    the default RCU CPU stall timeout (Rax
    extensions run in RCU locks as they use eBPF hook points). % as they reuse the same hook points of eBPF.
% The termination logic uses the Linux kernel's Inter-Process Interrupt (IPI) mechanism
% to raise an interrupt on the target \projname{} program's CPU.
% Depending on the use-case, the IPI can be triggered by an operator or a timer within the system
% that is configured to notify when the \projname{} exceeds a certain runtime threshold.
% Note that we are assuming the availability of a free CPU for an operator to be able to invoke
% an IPI.
% In uniprocessor machines, where the only CPU will be busy running the extension, the operator
% won't be able to issue termination request and has to rely on installing timers.
% We find this limitation acceptable due to the current cloud infrastructure predominantly being SMP.

% To safely terminate a \projname{} extension, we need to ensure the following :
% Note that a Rax extension should not be terminated while executing a kernel helper
%    function or the panic handler, as doing so disrupts internal bookkeeping of
%    the kernel (e.g. acquired resources) or safe exception handling.
\projname{} defers termination when the extension is
    running kernel helper functions to avoid disrupting the kernel's internal resource bookkeeping;
    it also does not terminate an extension if it
    is in the panic handler. % on the exit path.
\projname{} uses a per-CPU tristate flag to track the state of an
    extension: % (Figure~\ref{fig:rax-termination}).
    (1) executing extension code, (2) executing kernel helpers or
    panic handlers, and (3) termination requested.
A helper call changes the state from 1 to 2.
When executing the timer handler, if the flag is at state 2,
%    which indicates the program is under a context that cannot be safely terminated,
    the termination handler modifies it to state 3 without
    changing the instruction pointer.
When a helper returns, if the flag is at state 3,
    the panic handler is called to gracefully exit. % the extension.

\new{
A corner case of this design is deadlock. Since spinlock acquisition in
    \projname{} is implemented by a kernel helper function, a deadlocked program
    will never return from the helper, and therefore will never be
    terminated properly.
\projname{} follows eBPF's solution toward deadlocks, where a program can only take one lock at a time.
This is achieved by using a per-CPU variable to track whether the program
    currently holds a lock---a program trying to acquire a second lock will
    trigger a Rust panic.
We note that if the ability of holding multiple locks at the same time is
    desired, the kernel spinlock logic can be modified to check the termination
    state of \projname{} programs during spinning and terminate a deadlocked
    program accordingly.
}
% \begin{enumerate}
%     \item A \projname{} program should never be terminated while executing a
%         kernel helper function or the panic handler, as doing so disrupts
%         internal bookkeeping of the kernel (e.g. acquired resources) and safe
%         exception handling process of \projname{}.
%         This is neccesary to ensure kernel objects acquired during a helper call
%         are freed.
%     \item Kernel resources allocated within the extension program cannot be
%         left unreleased and cause resource leaks.
% \end{enumerate}

% The raised interrupt handler first detaches the \projname{} program from its hookpoint to prevent
% further invocation.
% The handler then changes the saved registers(in the interrupt stack) from the \projname{} context
%  to point to the \projname{} panic handler(section \ref{principle:eh}).
% The CPU returns the execution to the panic handler which performs the cleanup of
% any kernel objects that were live at the time of termination.
% To address the case when the target \projname{} program could be inside a helper/panic handler,
% \begin{figure}
%     \includegraphics[width=1.0\linewidth]{figs/BPF_termination_state_diagram.pdf}
%     \centering
%     %\vspace{-10pt}
%     \caption{Termination state diagram for \projname{} extensions}
%     \label{fig:rax-termination}
%     %\vspace{-30pt}
% \end{figure}


% This mechanism is used to defer a panic invocation until the end of a helper execution.
% \jinghao{Shall we have a state diagram?}

% Rax timeout check runs in hardirq, as such it cannot interrupt programs
% running in hardirq.
% No protection for these programs
% could still choose eBPF, as Rax and eBPF are not mutually exclusive.
\para{Limitation.}
\projname{} uses
    hard interrupts, and thus cannot
    interrupt extensions that are already executing in hard or
    non-maskable interrupts~\cite{elce-16-chaiken} (e.g., hardware perf-event programs).
Such extensions are not targeted by Rax, as they are supposed to be small, simple, and
less likely to encounter the \gap{}.
%better fit eBPF's model.
%\ayushb{Good Point for us as well: Limitation of async for hard IRQ, this can be justified as a tradeoff of feature/safety vs performance/usability as well! KFlex does in sync but having async in our case could be better for performance aspect.}
%\jinghao{Actaully this turns out to be not that easy to argue --- we need to track
%    helper executions through the per-cpu flag, which requires accessing it
%    both before and after a helper call, and it is hard to reason about the
%    number of helper call vs the number of back edges in the execution.}
Note that Rax extensions and eBPF extensions are not mutually exclusive and
    can co-exist.
%\mvle{I feel this is a bigger point that should be surfaced in the beginning somewhere
%as we are saying some simple programs should actually still be written in eBPF.}
%\ayushb{Having the co-exist argument is something that should be added in intro as well, since this highlights the major point that we are not arguing removing the current verifier itself.}
%\jinghao{The following paragraphs are tentative --- I am okay with not adding
%    these if these are too much}
%\mvle{I think they should be included but should be dramatically shortened.
%Remove the details.}
%\new{
%\projname{} currently does not terminate a program if it is already interrupted.
%The situation can happen when a program under task context is interrupted by a
%    soft interrupt, which is in turn interrupted by the \projname{} timer.
%\projname{} could correctly determine that a program is executing on the CPU
%    (from the program pointer check) and the context interrupted by the timer is
%    not the program (from the stack pointer check).
%However, the lack of access to the program regsters will hinder the termination
%    effort, and currently, \projname{} has to wait for a triggering of the timer
%    that directly interrupts the program itself.
%}

Moreover, the termination of a timed-out \projname{} extension can be delayed
    if the extension is already interrupted by another event when the timer
    triggers (the registers will not be available to the timer
    handler). \projname{} needs to wait for a triggering of the timer that
    directly interrupts the extension.

%\new{
%At the same time, due to \projname{}'s use of the eBPF spinlock helper,
%    deadlocked programs cannot be correctly terminated.
%\projname{} defers termination if a program is executing kernel
%    helpers and waits for the program to return before it can terminate.
%    However, because the spinlock acquisition logic resides in the kernel,
%    deadlocked programs would spin in the kernel without returning.
%This effectively prevents deadlocked programs from being terminated.
%While \projname{} currently does not implement a solution, we envision two
%    possible solutions 1) use a custom spinlock that work entirely in
%    \projname{} without entering the kernel, the same solution is employed by
%    KFlex~\cite{dwivedi-sosp24}, and 2) add the termination flag check to the
%    spinning logic so that the extension program stop spinning immediately after
%    termination is requested.
%}


% These programs would recieve no termination guarrantee from the \projname{}
%    runtime.
% Developers that aim for such guarrantee could still use eBPF for their
%    use case, as Rax and eBPF are not mutually exclusive, although doing so
%    means handling the \gap{}.

\section{Implementation Choices}
\label{sec:impl}

In this section, we briefly discuss the design choices made in our implementation of \lithe.

\vspace{-0.1in}
\subsection{\lithe Parameter Settings}
\label{sec:llm-params}

The \emph{``temperature''} parameter of \gpt, which ranges over [0,1], controls the randomness of the model's response.
While a higher temperature can be useful for creative writing where one would seek diverse and exploratory answers, in our case we want a focused and deterministic answer as far as possible. Hence we set the temperature to 0 which forces the model to greedily sample the next token.


The hyperparameters used by \lithe for MCTS are as follows: The maximum number of iterations $iter_{max}$ is set to 8,  expansion threshold $\theta$ is 0.7, and number of expansions $k$ is 2.
The values of $c_{base}$ and $c$ were set to 10 and 4, respectively.
%
These settings were determined after an empirical evaluation of the various trade-offs, providing a robust balance between efficiency and quality.
%

Finally, we try a maximum of 5 times to fix, via prompt corrections, any rewrite that exhibits syntax errors (Section~\ref{sec:lithe-architecture}).

\vspace{-0.1in}
\subsection{Query Equivalence Testing}
\label{sec:sql-equivalence}
We use a multi-stage approach, described below, to test semantic equivalence between the original query and a candidate rewrite.

\myparagraph{1. Logic-based Equivalence.}
Although verifying the equivalence of a general pair of SQL queries is NP-complete~\cite{queryequivalence}, a variety of logic-based tools (e.g. Cosette\cite{Cosette}, SQL-Solver~\cite{SQLSolver}, VeriEQL~\cite{verieql}, QED~\cite{QED}) are available for proving equivalence over restricted classes of queries, as mentioned in the Introduction. 
%
In \lithe, we use the recently proposed QED~\cite{QED} since it was found to cover a larger set of queries compared to the alternatives. 
%
The advantage of such a logic-based approach is that it is definitive in outcome and relatively inexpensive. 

\myparagraph{2. Result Equivalence via Sampling.}
%
If the original query is not within the QED scope, we alternatively use a sampling-based approach to test equivalence. The idea here is to execute the queries on several small samples of the database and verify equivalence based on the sample results. 
%
However, while this test is a necessary condition for query equivalence, it is not sufficient. That is,  false positives may be present because the sampled database may not cover all the predicates featured in the query. To minimize this likelihood, we use a combination of (1) \textit{correlated sampling}~\cite{cs2} for maintaining join integrity in the sample, (2) adding synthetic tuples in the sample to distinguish outer and inner joins, and (3) adjusting constants in the filter predicates to produce populated results -- the complete details are in the Section~\ref{app:sampling-eq}. 

\myparagraph{3. Result Equivalence on the Entire Database.}
%
Result equivalence could also be evaluated, in principle, on the entire database itself. Of course, this could turn out to be prohibitively expensive, especially if the queries themselves are time-consuming (e.g. due to the scale of the underlying database) and/or if the candidate rewrites happen to be regressions. Therefore, we leave this check as an optional choice for the DBA.



\section{Implementation and Evaluation}
\label{sec:evaluation}

We prototype our proposal into a tool \toolName, using approximately 5K lines of OCaml (for the program analysis) and 5K lines of Python code (for the repair). 
In particular, we employ Z3~\cite{DBLP:conf/tacas/MouraB08} as the SMT solver, clingo~\cite{DBLP:books/sp/Lifschitz19} as the ASP solver, and Souffle~\cite{scholz2016fast} as the Datalog engine. %, respectively.
To show the effectiveness, 
we design the experimental evaluation to answer the 
following research questions (RQ):
(Experiments ran on a server with an Intel® Xeon® Platinum 8468V, 504GB RAM, and 192 cores. All the dataset are publicly available from \cite{zenodo_benchmark})

\begin{itemize}[align=left, leftmargin=*,labelindent=0pt]
\item \textbf{RQ1:} How effective is \toolName in verifying CTL properties for relatively small but complex programs, compared to the state-of-the-art tool  \function~\cite{DBLP:conf/sas/UrbanU018}?


\item \textbf{RQ2:} What is the effectiveness of \toolName in detecting real-world bugs, which can be encoded using both CTL and linear temporal logic (LTL), such as non-termination gathered from GitHub \cite{DBLP:conf/sigsoft/ShiXLZCL22} and unresponsive behaviours in protocols  \cite{DBLP:conf/icse/MengDLBR22}, compared with \ultimate~\cite{DBLP:conf/cav/DietschHLP15}?

\item \textbf{RQ3:} How effective is \toolName in repairing CTL violations identified in RQ1 and RQ2? which has not been achieved by any existing tools. 


 

\end{itemize}



% \begin{itemize}[align=left, leftmargin=*,labelindent=0pt]
% \item \textbf{RQ1:} What is the effectiveness of \toolName in verifying CTL properties in a set of relatively small yet challenging programs, compared to the state-of-the-art tools, T2~\cite{DBLP:conf/fmcad/CookKP14},  \function~\cite{DBLP:conf/sas/UrbanU018}, and \ultimate~\cite{DBLP:conf/cav/DietschHLP15}?


% \item \textbf{RQ2:} What is the effectiveness of \toolName in finding  real-world bugs, which can be encoded using CTL properties, such as non-termination 
% gathered from GitHub \cite{DBLP:conf/sigsoft/ShiXLZCL22} and unresponsive behaviours in protocol implementations \cite{DBLP:conf/icse/MengDLBR22}?

% \item \textbf{RQ3:} What is the effectiveness of \toolName in repairing CTL bugs from RQ1--2?

% \end{itemize}

%The benchmark programs are from various sources. More specifically, termination bugs from real-world projects \cite{DBLP:conf/sigsoft/ShiXLZCL22} and CTL analysis \cite{DBLP:conf/fmcad/CookKP14} \cite{DBLP:conf/sas/UrbanU018}, and temporal bugs in real-world protocol implementations \cite{DBLP:conf/icse/MengDLBR22}. 



% \ly{are termination bugs ok? Do we need to add new CTL bugs?}
\subsection{RQ1: Verifying CTL Properties}

% Please add the following required packages to your document preamble:
%  \Xhline{1.5\arrayrulewidth}

\hide{\begin{figure}[!h]
\vspace{-8mm}
\begin{lstlisting}[xleftmargin=0.2em,numbersep=6pt,basicstyle=\footnotesize\ttfamily]
(*@\textcolor{mGray}{//$EF(\m{resp}{\geq}5)$}@*)
int c = *; int resp = 0;
int curr_serv = 5; 
while (curr_serv > 0){ 
 if (*) {  
   c--; 
   curr_serv--;
   resp++;} 
 else if (c<curr_serv){
   curr_serv--; }}
\end{lstlisting} 
\vspace{-2mm}
\caption{A possibly terminating loop} 
\label{fig:terminating_loop}
\vspace{-2mm}
\end{figure}}


%loses precision due to a \emph{dual widening} \cite{DBLP:conf/tacas/CourantU17}, and 

The programs listed in \tabref{tab:comparewithFuntionT2} were obtained from the evaluation benchmark of \function, which includes a total of 83 test cases across over 2,000 lines of code. We categorize these test cases into six groups, labeled according to the types of CTL properties. 
These programs are short but challenging, as they often involve complex loops or require a more precise analysis of the target properties. The \function tends to be conservative, often leading it to return ``unknown" results, resulting in an accuracy rate of 27.7\%. In contrast, \toolName demonstrates advantages with improved accuracy, particularly in \ourToolSmallBenchmark. 
%achieved by the novel loop summaries. 
The failure cases faced by \toolName highlight our limitations when loop guards are not explicitly defined or when LRFs are inadequate to prove termination. 
Although both \function and \toolName struggle to obtain meaningful invariances for infinite loops, the benefits of our loop summaries become more apparent when proving properties related to termination, such as reachability and responsiveness.  




\begin{table}[!t]
\vspace{1.5mm}
\caption{Detecting real-world CTL bugs.}
\normalsize
\label{tab:comparewithCook}
\renewcommand{\arraystretch}{0.95}
\setlength{\tabcolsep}{4pt}  
\begin{tabular}{c|l|c|cc|cc}
\Xhline{1.5\arrayrulewidth}
\multicolumn{1}{l|}{\multirow{2}{*}{\textbf{}}} & \multirow{2}{*}{\textbf{Program}}        & \multirow{2}{*}{\textbf{LoC}} & \multicolumn{2}{c|}{\textbf{\ultimateshort}}   & \multicolumn{2}{c}{\textbf{\toolName}}             \\ \cline{4-7} 
  \multicolumn{1}{l|}{}                           &                                          &                               & \multicolumn{1}{c|}{\textbf{Res.}} & \textbf{Time} & \multicolumn{1}{c|}{\textbf{Res.}} & \textbf{Time} \\ \hline
  1 \xmark                                      & \multirow{2}{*}{\makecell[l]{libvncserver\\(c311535)}}   & 25                            & \multicolumn{1}{c|}{\xmark}      & 2.845         & \multicolumn{1}{c|}{\xmark}      & 0.855         \\  
  1 \cmark                                      &                                          & 27                            & \multicolumn{1}{c|}{\cmark}      & 3.743         & \multicolumn{1}{c|}{\cmark}      & 0.476         \\ \hline
  2 \xmark                                      & \multirow{2}{*}{\makecell[l]{Ffmpeg\\(a6cba06)}}         & 40                            & \multicolumn{1}{c|}{\xmark}      & 15.254        & \multicolumn{1}{c|}{\xmark}      & 0.606         \\  
  2 \cmark                                      &                                          & 44                            & \multicolumn{1}{c|}{\cmark}      & 40.176        & \multicolumn{1}{c|}{\cmark}      & 0.397         \\ \hline
  3 \xmark                                      & \multirow{2}{*}{\makecell[l]{cmus\\(d5396e4)}}           & 87                            & \multicolumn{1}{c|}{\xmark}      & 6.904         & \multicolumn{1}{c|}{\xmark}      & 0.579         \\  
  3 \cmark                                      &                                          & 86                            & \multicolumn{1}{c|}{\cmark}      & 33.572        & \multicolumn{1}{c|}{\cmark}      & 0.986         \\ \hline
  4 \xmark                                      & \multirow{2}{*}{\makecell[l]{e2fsprogs\\(caa6003)}}      & 58                            & \multicolumn{1}{c|}{\xmark}      & 5.952         & \multicolumn{1}{c|}{\xmark}      & 0.923         \\  
  4 \cmark                                      &                                          & 63                            & \multicolumn{1}{c|}{\cmark}      & 4.533         & \multicolumn{1}{c|}{\cmark}      & 0.842         \\ \hline
  5 \xmark                                      & \multirow{2}{*}{\makecell[l]{csound-an...\\(7a611ab)}} & 43                            & \multicolumn{1}{c|}{\xmark}      & 3.654         & \multicolumn{1}{c|}{\xmark}      & 0.782         \\  
  5 \cmark                                      &                                          & 45                            & \multicolumn{1}{c|}{TO}          & -             & \multicolumn{1}{c|}{\cmark}      & 0.648         \\ \hline
  6 \xmark                                      & \multirow{2}{*}{\makecell[l]{fontconfig\\(fa741cd)}}     & 25                            & \multicolumn{1}{c|}{\xmark}      & 3.856         & \multicolumn{1}{c|}{\xmark}      & 0.769         \\  
  6 \cmark                                      &                                          & 25                            & \multicolumn{1}{c|}{Error}       & -             & \multicolumn{1}{c|}{\cmark}      & 0.651         \\ \hline
  7 \xmark                                      & \multirow{2}{*}{\makecell[l]{asterisk\\(3322180)}}       & 22                            & \multicolumn{1}{c|}{\unk}        & 12.687        & \multicolumn{1}{c|}{\unk}        & 0.196         \\  
  7 \cmark                                      &                                          & 25                            & \multicolumn{1}{c|}{\unk}        & 11.325        & \multicolumn{1}{c|}{\unk}        & 0.34          \\ \hline
  8 \xmark                                      & \multirow{2}{*}{\makecell[l]{dpdk\\(cd64eeac)}}          & 45                            & \multicolumn{1}{c|}{\xmark}      & 3.712         & \multicolumn{1}{c|}{\xmark}      & 0.447         \\  
  8 \cmark                                      &                                          & 45                            & \multicolumn{1}{c|}{\cmark}      & 2.97          & \multicolumn{1}{c|}{\unk}        & 0.481         \\ \hline
  9 \xmark                                      & \multirow{2}{*}{\makecell[l]{xorg-server\\(930b9a06)}}   & 19                            & \multicolumn{1}{c|}{\xmark}      & 3.111         & \multicolumn{1}{c|}{\xmark}      & 0.581         \\  
  9 \cmark                                      &                                          & 20                            & \multicolumn{1}{c|}{\cmark}      & 3.101         & \multicolumn{1}{c|}{\cmark}      & 0.409         \\ \hline
  10 \xmark                                      & \multirow{2}{*}{\makecell[l]{pure-ftpd\\(37ad222)}}      & 42                            & \multicolumn{1}{c|}{\cmark}      & 2.555         & \multicolumn{1}{c|}{\xmark}      & 0.933         \\  
  10 \cmark                                      &                                          & 49                            & \multicolumn{1}{c|}{\cmark}        & 2.286         & \multicolumn{1}{c|}{\cmark}      & 0.383         \\ \hline
  11 \xmark  & \multirow{2}{*}{\makecell[l]{live555$_a$\\(181126)}} & 34  & \multicolumn{1}{c|}{\cmark} &  2.715         & \multicolumn{1}{c|}{\xmark}    & 0.513   \\  
  11 \cmark  &     &   37    & \multicolumn{1}{c|}{\cmark} &  2.837         & \multicolumn{1}{c|}{\cmark}      & 0.341 \\ \hline
  12 \xmark  & \multirow{2}{*}{\makecell[l]{openssl\\(b8d2439)}} & 88  & \multicolumn{1}{c|}{\xmark} &  4.15          & \multicolumn{1}{c|}{\xmark}    & 0.78   \\
  12 \cmark  &     &  88     & \multicolumn{1}{c|}{\cmark} &  3.809         & \multicolumn{1}{c|}{\cmark}      & 0.99 \\ \hline
  13 \xmark  & \multirow{2}{*}{\makecell[l]{live555$_b$\\(131205)}} & 83  & \multicolumn{1}{c|}{\xmark} & 2.838         & \multicolumn{1}{c|}{\xmark}    & 0.602     \\  
  13 \cmark  &    &   84     & \multicolumn{1}{c|}{\cmark} &  2.393         & \multicolumn{1}{c|}{\cmark}      & 0.565 \\ \Xhline{1.5\arrayrulewidth}
                                                   & {\bf{Total}}                                  & 1249  & \multicolumn{1}{c|}{\bestBaseLineReal}          & $>$180       & \multicolumn{1}{c|}{\ourToolRealBenchmark}              & 16.01        \\ \Xhline{1.5\arrayrulewidth}
  \end{tabular}
  \end{table}

\subsection{RQ2: CTL Analysis on  Real-world Projects}




Programs in \tabref{tab:comparewithCook} are from real-world repositories, each associated with a Git commit number where developers identify and fix the bug manually. 
In particular, the property used for programs 1-9 (drawn from \cite{DBLP:conf/sigsoft/ShiXLZCL22}) is  \code{AF(Exit())}, capturing non-termination bugs. The properties used for programs 10-13 (drawn from \cite{DBLP:conf/icse/MengDLBR22}) are of the form \code{AG(\phi_1{\rightarrow}AF(\phi_2))}, capturing unresponsive behaviours from the protocol implementation. 
We extracted the main segments of these real-world bugs into smaller programs (under 100 LoC each), preserving features like data structures and pointer arithmetic. Our evaluation includes both buggy (\eg 1\,\xmark) and developer-fixed (\eg 1\,\cmark) versions.
After converting the CTL properties to LTL formulas, we compared our tool with the latest release of UltimateLTL (v0.2.4), a regular participant in SV-COMP \cite{svcomp} with competitive performance. 
Both tools demonstrate high accuracy in bug detection, while \ultimateshort often requires longer processing time. 
This experiment indicates that LRFs can effectively handle commonly seen real-world loops, and \toolName performs a more lightweight summary computation without compromising accuracy. 



%Following the convention in \cite{DBLP:conf/sigsoft/ShiXLZCL22}, t
%Prior works \cite{DBLP:conf/sigsoft/ShiXLZCL22} gathered such examples by extracting 
%\toolName successfully identifies the majority of buggy and correct programs, with the exception of programs 7 and 8. 







{
\begin{table*}[!h]
  \centering
\caption{\label{tab:repair_benchmark}
{Experimental results for repairing CTL bugs. Time spent by the ASP solver is separately recorded. 
}
}
\small
\renewcommand{\arraystretch}{0.95}
  \setlength{\tabcolsep}{9pt}
\begin{tabular}{l|c|c|c|c|c|c|c|c}
  \Xhline{1.5\arrayrulewidth}
  \multicolumn{1}{c|}{\multirow{2}{*}{\textbf{Program}}} & \multicolumn{1}{c|}{\multirow{2}{*}{\shortstack{\textbf{LoC}\\\textbf{(Datalog)}}}} & \multicolumn{3}{c|}{\textbf{Configuration}}                                 & \multicolumn{1}{c|}{\multirow{2}{*}{\textbf{Fixed}}} & \multicolumn{1}{c|}{\multirow{2}{*}{\textbf{\#Patch}}} & \multicolumn{1}{c|}{\multirow{2}{*}{\textbf{ASP(s)}}} & \multirow{2}{*}{\textbf{Total(s)}} \\ \cline{3-5}

  \multicolumn{1}{c|}{}                                  & \multicolumn{1}{c|}{}                              & \multicolumn{1}{c|}{\textbf{Symbols}} & \multicolumn{1}{c|}{\textbf{Facts}} & \multicolumn{1}{c|}{\textbf{Template}} & \multicolumn{1}{c|}{} & \multicolumn{1}{c|}{} & \multicolumn{1}{c|}{}  &                                      \\ \hline

AF\_yEQ5 (\figref{fig:first_Example})                                           & 115                           & 3+0                   & 0+1                & Add                & \cmark     & 1                   & 0.979                              & 1.593                                \\
test\_until.c                                         & 101                            & 0+3                   & 1+0                & Delete                & \cmark     & 1                   & 0.023                              & 0.498                                \\
next.c                                                & 87                            & 0+4                   & 1+0                & Delete                & \cmark     & 1                   & 0.023                              & 0.472                                \\
libvncserver                                          & 118                            & 0+6                   & 1+0                & Delete                & \cmark     & 3                   & 0.049                              & 1.081                                \\
Ffmpeg                                                & 227                           & 0+12                  & 1+0                & Delete                & \cmark     & 4                   & 13.113                              & 13.335                                \\
cmus                                                  & 145                           & 0+12                  & 1+0                & Delete                & \cmark     & 4                   & 0.098                              & 2.052                                \\
e2fsprogs                                             & 109                           & 0+8                   & 1+0                & Delete                & \cmark     & 2                   & 0.075                              & 1.515                                \\
csound-android                                        & 183                           & 0+8                   & 1+0                & Delete                & \cmark     & 4                   & 0.076                              & 1.613                                \\
fontconfig                                            & 190                           & 0+11                  & 1+0                & Delete                & \cmark     & 6                   & 0.098                              & 2.507                                \\
dpdk                                                  & 196                           & 0+12                  & 1+0                & Delete                & \cmark     & 1                   & 0.091                              & 2.006                                \\
xorg-server                                           & 118                            & 0+2                   & 1+0                & Delete                & \cmark     & 2                   & 0.026                              & 0.605                                \\
pure-ftpd                                             & 258                           & 0+21                  & 1+0                & Delete                & \cmark     & 2                   & 0.069                              & 3.590                               \\
live$_a$                                              & 112                            & 3+4                   & 1+1                & Update                & \cmark     & 1                   & 0.552                              & 0.816                                \\
openssl                                               & 315                           & 1+0                   & 0+1                & Add.                & \cmark     & 1                   & 1.188                              & 2.277                                \\
live$_b$                                              & 217                           & 1+0                   & 0+1                & Add                & \cmark     & 1                   & 0.977                              & 1.494                                 \\
  \Xhline{1.5\arrayrulewidth}
\textbf{Total}                                                 & 2491                          &                       &                    &                   &           &                     & 17.437                              & 35.454                               \\ 
  \Xhline{1.5\arrayrulewidth}           
\end{tabular}

\vspace{-2mm}
\end{table*}
}


\subsection{RQ3: Repairing CTL Property Violations} 


\tabref{tab:repair_benchmark} gathers all the program instances (from \tabref{tab:comparewithFuntionT2} and \tabref{tab:comparewithCook}) that violate their specified CTL properties and are sent to \toolName for repair.   
The \textbf{Symbols} column records the number of symbolic constants + symbolic signs, while the \textbf{Facts} column records the number of facts allowed to be removed + added. 
We gradually increase the number of symbols and the maximum number of facts that can be added or deleted. 
The \textbf{Configuration} column shows the first successful configuration that led to finding patches, and we record the total searching time till reaching such configurations. 
We configure \toolName to apply three atomic templates in a breadth-first manner with a depth limit of 1, \ie, \tabref{tab:repair_benchmark} records the patch result after one iteration of the repair. 
The templates are applied sequentially in the order: delete, update, and add. The repair process stops when a correct patch is found or when all three templates have been attempted. 
%without success. 
% Because of this configuration, \toolName only finds one patch for Program 1 (AF\_yEQ5). 
% The patch inserting \plaincode{if (i>10||x==y) \{y=5; return;\}} mentioned in \figref{fig:Patched-program} cannot be found in current configuration, as it requires deleting facts then adding new facts on the updated program.
% The `Configuration' column in \tabref{tab:repair_benchmark} shows the number of symbolic constants and signs, the number of facts allowed to be removed and added, and the template used when a patch is found.

Due to the current configuration, \toolName only finds patch (b) for Program 1 (AF\_yEQ5), while the patch (a) shown in \figref{fig:Patched-program} can be obtained by allowing two iterations of the repair: the first iteration adds the conditional than a second iteration to add a new assignment on the updated program. 
Non-termination bugs are resolved within a single iteration by adding a conditional statement that provides an earlier exit. 
For instance, \figref{fig:term-Patched-program} illustrates the main logic of 1\,\xmark, which enters an infinite loop when \code{\m{linesToRead}{\leq}0}. 
\toolName successfully 
provides a fix that prevents \code{\m{linesToRead}{\leq}0} from occurring before entering the loop. Note that such patches are more desirable which fix the non-termination bug without dropping the loops completely. 
%much like the example shown in  \figref{fig:term-Patched-program}. At the same time, 
Unresponsive bugs involve adding more function calls or assignment modifications. 
%Most repairs were completed within seconds. 

On average, the time taken to solve ASP accounts for 49.2\% (17.437/35.454) of the total repair time. We also keep track of the number of patches that successfully eliminate the CTL violations. More than one patch is available for non-termination bugs, as some patches exit the entire program without entering the loop. 
While all the patches listed are valid, those that intend to cut off the main program logic can be excluded based on the minimum change criteria. 
After a manual inspection of each buggy program shown in \tabref{tab:repair_benchmark}, we confirmed that at least one generated patch is semantically equivalent to the fix provided by the developer. 
As the first tool to achieve automated repair of CTL violations, \toolName successfully resolves all reported bugs. 



\begin{figure}[!t]
\begin{lstlisting}[xleftmargin=6em,numbersep=6pt,basicstyle=\footnotesize\ttfamily]
void main(){ //AF(Exit())
  int lines ToRead = *;
  int h = *;
  (*@\repaircode{if ( linesToRead <= 0 )  return;}@*)
  while(h>0){
    if(linesToRead>h)  
        linesToRead=h; 
    h-=linesToRead;} 
  return;}
\end{lstlisting}
\caption{Fixing a Possible Hang Found in libvncserver \cite{LibVNCClient}}
\label{fig:term-Patched-program}
\end{figure}


\section{Discussion of Assumptions}\label{sec:discussion}
In this paper, we have made several assumptions for the sake of clarity and simplicity. In this section, we discuss the rationale behind these assumptions, the extent to which these assumptions hold in practice, and the consequences for our protocol when these assumptions hold.

\subsection{Assumptions on the Demand}

There are two simplifying assumptions we make about the demand. First, we assume the demand at any time is relatively small compared to the channel capacities. Second, we take the demand to be constant over time. We elaborate upon both these points below.

\paragraph{Small demands} The assumption that demands are small relative to channel capacities is made precise in \eqref{eq:large_capacity_assumption}. This assumption simplifies two major aspects of our protocol. First, it largely removes congestion from consideration. In \eqref{eq:primal_problem}, there is no constraint ensuring that total flow in both directions stays below capacity--this is always met. Consequently, there is no Lagrange multiplier for congestion and no congestion pricing; only imbalance penalties apply. In contrast, protocols in \cite{sivaraman2020high, varma2021throughput, wang2024fence} include congestion fees due to explicit congestion constraints. Second, the bound \eqref{eq:large_capacity_assumption} ensures that as long as channels remain balanced, the network can always meet demand, no matter how the demand is routed. Since channels can rebalance when necessary, they never drop transactions. This allows prices and flows to adjust as per the equations in \eqref{eq:algorithm}, which makes it easier to prove the protocol's convergence guarantees. This also preserves the key property that a channel's price remains proportional to net money flow through it.

In practice, payment channel networks are used most often for micro-payments, for which on-chain transactions are prohibitively expensive; large transactions typically take place directly on the blockchain. For example, according to \cite{river2023lightning}, the average channel capacity is roughly $0.1$ BTC ($5,000$ BTC distributed over $50,000$ channels), while the average transaction amount is less than $0.0004$ BTC ($44.7k$ satoshis). Thus, the small demand assumption is not too unrealistic. Additionally, the occasional large transaction can be treated as a sequence of smaller transactions by breaking it into packets and executing each packet serially (as done by \cite{sivaraman2020high}).
Lastly, a good path discovery process that favors large capacity channels over small capacity ones can help ensure that the bound in \eqref{eq:large_capacity_assumption} holds.

\paragraph{Constant demands} 
In this work, we assume that any transacting pair of nodes have a steady transaction demand between them (see Section \ref{sec:transaction_requests}). Making this assumption is necessary to obtain the kind of guarantees that we have presented in this paper. Unless the demand is steady, it is unreasonable to expect that the flows converge to a steady value. Weaker assumptions on the demand lead to weaker guarantees. For example, with the more general setting of stochastic, but i.i.d. demand between any two nodes, \cite{varma2021throughput} shows that the channel queue lengths are bounded in expectation. If the demand can be arbitrary, then it is very hard to get any meaningful performance guarantees; \cite{wang2024fence} shows that even for a single bidirectional channel, the competitive ratio is infinite. Indeed, because a PCN is a decentralized system and decisions must be made based on local information alone, it is difficult for the network to find the optimal detailed balance flow at every time step with a time-varying demand.  With a steady demand, the network can discover the optimal flows in a reasonably short time, as our work shows.

We view the constant demand assumption as an approximation for a more general demand process that could be piece-wise constant, stochastic, or both (see simulations in Figure \ref{fig:five_nodes_variable_demand}).
We believe it should be possible to merge ideas from our work and \cite{varma2021throughput} to provide guarantees in a setting with random demands with arbitrary means. We leave this for future work. In addition, our work suggests that a reasonable method of handling stochastic demands is to queue the transaction requests \textit{at the source node} itself. This queuing action should be viewed in conjunction with flow-control. Indeed, a temporarily high unidirectional demand would raise prices for the sender, incentivizing the sender to stop sending the transactions. If the sender queues the transactions, they can send them later when prices drop. This form of queuing does not require any overhaul of the basic PCN infrastructure and is therefore simpler to implement than per-channel queues as suggested by \cite{sivaraman2020high} and \cite{varma2021throughput}.

\subsection{The Incentive of Channels}
The actions of the channels as prescribed by the DEBT control protocol can be summarized as follows. Channels adjust their prices in proportion to the net flow through them. They rebalance themselves whenever necessary and execute any transaction request that has been made of them. We discuss both these aspects below.

\paragraph{On Prices}
In this work, the exclusive role of channel prices is to ensure that the flows through each channel remains balanced. In practice, it would be important to include other components in a channel's price/fee as well: a congestion price  and an incentive price. The congestion price, as suggested by \cite{varma2021throughput}, would depend on the total flow of transactions through the channel, and would incentivize nodes to balance the load over different paths. The incentive price, which is commonly used in practice \cite{river2023lightning}, is necessary to provide channels with an incentive to serve as an intermediary for different channels. In practice, we expect both these components to be smaller than the imbalance price. Consequently, we expect the behavior of our protocol to be similar to our theoretical results even with these additional prices.

A key aspect of our protocol is that channel fees are allowed to be negative. Although the original Lightning network whitepaper \cite{poon2016bitcoin} suggests that negative channel prices may be a good solution to promote rebalancing, the idea of negative prices in not very popular in the literature. To our knowledge, the only prior work with this feature is \cite{varma2021throughput}. Indeed, in papers such as \cite{van2021merchant} and \cite{wang2024fence}, the price function is explicitly modified such that the channel price is never negative. The results of our paper show the benefits of negative prices. For one, in steady state, equal flows in both directions ensure that a channel doesn't loose any money (the other price components mentioned above ensure that the channel will only gain money). More importantly, negative prices are important to ensure that the protocol selectively stifles acyclic flows while allowing circulations to flow. Indeed, in the example of Section \ref{sec:flow_control_example}, the flows between nodes $A$ and $C$ are left on only because the large positive price over one channel is canceled by the corresponding negative price over the other channel, leading to a net zero price.

Lastly, observe that in the DEBT control protocol, the price charged by a channel does not depend on its capacity. This is a natural consequence of the price being the Lagrange multiplier for the net-zero flow constraint, which also does not depend on the channel capacity. In contrast, in many other works, the imbalance price is normalized by the channel capacity \cite{ren2018optimal, lin2020funds, wang2024fence}; this is shown to work well in practice. The rationale for such a price structure is explained well in \cite{wang2024fence}, where this fee is derived with the aim of always maintaining some balance (liquidity) at each end of every channel. This is a reasonable aim if a channel is to never rebalance itself; the experiments of the aforementioned papers are conducted in such a regime. In this work, however, we allow the channels to rebalance themselves a few times in order to settle on a detailed balance flow. This is because our focus is on the long-term steady state performance of the protocol. This difference in perspective also shows up in how the price depends on the channel imbalance. \cite{lin2020funds} and \cite{wang2024fence} advocate for strictly convex prices whereas this work and \cite{varma2021throughput} propose linear prices.

\paragraph{On Rebalancing} 
Recall that the DEBT control protocol ensures that the flows in the network converge to a detailed balance flow, which can be sustained perpetually without any rebalancing. However, during the transient phase (before convergence), channels may have to perform on-chain rebalancing a few times. Since rebalancing is an expensive operation, it is worthwhile discussing methods by which channels can reduce the extent of rebalancing. One option for the channels to reduce the extent of rebalancing is to increase their capacity; however, this comes at the cost of locking in more capital. Each channel can decide for itself the optimum amount of capital to lock in. Another option, which we discuss in Section \ref{sec:five_node}, is for channels to increase the rate $\gamma$ at which they adjust prices. 

Ultimately, whether or not it is beneficial for a channel to rebalance depends on the time-horizon under consideration. Our protocol is based on the assumption that the demand remains steady for a long period of time. If this is indeed the case, it would be worthwhile for a channel to rebalance itself as it can make up this cost through the incentive fees gained from the flow of transactions through it in steady state. If a channel chooses not to rebalance itself, however, there is a risk of being trapped in a deadlock, which is suboptimal for not only the nodes but also the channel.

\section{Conclusion}
This work presents DEBT control: a protocol for payment channel networks that uses source routing and flow control based on channel prices. The protocol is derived by posing a network utility maximization problem and analyzing its dual minimization. It is shown that under steady demands, the protocol guides the network to an optimal, sustainable point. Simulations show its robustness to demand variations. The work demonstrates that simple protocols with strong theoretical guarantees are possible for PCNs and we hope it inspires further theoretical research in this direction.
\putsec{related}{Related Work}

\noindent \textbf{Efficient Radiance Field Rendering.}
%
The introduction of Neural Radiance Fields (NeRF)~\cite{mil:sri20} has
generated significant interest in efficient 3D scene representation and
rendering for radiance fields.
%
Over the past years, there has been a large amount of research aimed at
accelerating NeRFs through algorithmic or software
optimizations~\cite{mul:eva22,fri:yu22,che:fun23,sun:sun22}, and the
development of hardware
accelerators~\cite{lee:cho23,li:li23,son:wen23,mub:kan23,fen:liu24}.
%
The state-of-the-art method, 3D Gaussian splatting~\cite{ker:kop23}, has
further fueled interest in accelerating radiance field
rendering~\cite{rad:ste24,lee:lee24,nie:stu24,lee:rho24,ham:mel24} as it
employs rasterization primitives that can be rendered much faster than NeRFs.
%
However, previous research focused on software graphics rendering on
programmable cores or building dedicated hardware accelerators. In contrast,
\name{} investigates the potential of efficient radiance field rendering while
utilizing fixed-function units in graphics hardware.
%
To our knowledge, this is the first work that assesses the performance
implications of rendering Gaussian-based radiance fields on the hardware
graphics pipeline with software and hardware optimizations.

%%%%%%%%%%%%%%%%%%%%%%%%%%%%%%%%%%%%%%%%%%%%%%%%%%%%%%%%%%%%%%%%%%%%%%%%%%
\myparagraph{Enhancing Graphics Rendering Hardware.}
%
The performance advantage of executing graphics rendering on either
programmable shader cores or fixed-function units varies depending on the
rendering methods and hardware designs.
%
Previous studies have explored the performance implication of graphics hardware
design by developing simulation infrastructures for graphics
workloads~\cite{bar:gon06,gub:aam19,tin:sax23,arn:par13}.
%
Additionally, several studies have aimed to improve the performance of
special-purpose hardware such as ray tracing units in graphics
hardware~\cite{cho:now23,liu:cha21} and proposed hardware accelerators for
graphics applications~\cite{lu:hua17,ram:gri09}.
%
In contrast to these works, which primarily evaluate traditional graphics
workloads, our work focuses on improving the performance of volume rendering
workloads, such as Gaussian splatting, which require blending a huge number of
fragments per pixel.

%%%%%%%%%%%%%%%%%%%%%%%%%%%%%%%%%%%%%%%%%%%%%%%%%%%%%%%%%%%%%%%%%%%%%%%%%%
%
In the context of multi-sample anti-aliasing, prior work proposed reducing the
amount of redundant shading by merging fragments from adjacent triangles in a
mesh at the quad granularity~\cite{fat:bou10}.
%
While both our work and quad-fragment merging (QFM)~\cite{fat:bou10} aim to
reduce operations by merging quads, our proposed technique differs from QFM in
many aspects.
%
Our method aims to blend \emph{overlapping primitives} along the depth
direction and applies to quads from any primitive. In contrast, QFM merges quad
fragments from small (e.g., pixel-sized) triangles that \emph{share} an edge
(i.e., \emph{connected}, \emph{non-overlapping} triangles).
%
As such, QFM is not applicable to the scenes consisting of a number of
unconnected transparent triangles, such as those in 3D Gaussian splatting.
%
In addition, our method computes the \emph{exact} color for each pixel by
offloading blending operations from ROPs to shader units, whereas QFM
\emph{approximates} pixel colors by using the color from one triangle when
multiple triangles are merged into a single quad.


%\section{Experience}
% \begin{itemize}
%     \item XDP/Sched-cls kernel crate problem: commit a70dffa0
%     \item Triggering Rust exception when implementing BMC, but kernel is still
%         alive.
% \end{itemize}
\para{Rax Panic Handler}
During the debugging of cache miss rate issues in BMC, we observed that the
\texttt{bmc\_invalidate\_cache} function in the Rax version failed to deliver expected
    results via the statistical records.
This discrepancy indicated a potential flaw in that function's implementation
    or its interaction with the cache map.

    Then we attempted to utilize \texttt{bpf\_printk (Rax version)} for diagnostic, but
    the logging results was absence from the tracing files,
    and the \projname{}-BMC has a significant degradation in benchmark performance.
Reverting these changes restored the original performance levels and logging capabilities.

Afterwards, examination of the kernel logs showed an shocking number of kernel error messages.
With further investigation of these log along with the code, an out-of-bounds access error
emerged from the \texttt{bmc\_invalidate\_cache} function.
This mistake was the cause of kernel panics observed during benchmark.

Remarkably, even with the numerous kernel panics, the overall system stability
    and performance were not conspicuously impacted.
We have run several benchmarks in this kernel which already has a lot of Rax panic
    and this kernel continued to function well.
This resilience highlights an robustness in the unwind implementation of Rax.


\section{Conclusion}
In this work, we propose a simple yet effective approach, called SMILE, for graph few-shot learning with fewer tasks. Specifically, we introduce a novel dual-level mixup strategy, including within-task and across-task mixup, for enriching the diversity of nodes within each task and the diversity of tasks. Also, we incorporate the degree-based prior information to learn expressive node embeddings. Theoretically, we prove that SMILE effectively enhances the model's generalization performance. Empirically, we conduct extensive experiments on multiple benchmarks and the results suggest that SMILE significantly outperforms other baselines, including both in-domain and cross-domain few-shot settings.
%%
%% The next two lines define the bibliography style to be used, and
%% the bibliography file.
{\small
\bibliographystyle{acm}
\bibliography{ref}
}

% \appendix
% \subsection{Lloyd-Max Algorithm}
\label{subsec:Lloyd-Max}
For a given quantization bitwidth $B$ and an operand $\bm{X}$, the Lloyd-Max algorithm finds $2^B$ quantization levels $\{\hat{x}_i\}_{i=1}^{2^B}$ such that quantizing $\bm{X}$ by rounding each scalar in $\bm{X}$ to the nearest quantization level minimizes the quantization MSE. 

The algorithm starts with an initial guess of quantization levels and then iteratively computes quantization thresholds $\{\tau_i\}_{i=1}^{2^B-1}$ and updates quantization levels $\{\hat{x}_i\}_{i=1}^{2^B}$. Specifically, at iteration $n$, thresholds are set to the midpoints of the previous iteration's levels:
\begin{align*}
    \tau_i^{(n)}=\frac{\hat{x}_i^{(n-1)}+\hat{x}_{i+1}^{(n-1)}}2 \text{ for } i=1\ldots 2^B-1
\end{align*}
Subsequently, the quantization levels are re-computed as conditional means of the data regions defined by the new thresholds:
\begin{align*}
    \hat{x}_i^{(n)}=\mathbb{E}\left[ \bm{X} \big| \bm{X}\in [\tau_{i-1}^{(n)},\tau_i^{(n)}] \right] \text{ for } i=1\ldots 2^B
\end{align*}
where to satisfy boundary conditions we have $\tau_0=-\infty$ and $\tau_{2^B}=\infty$. The algorithm iterates the above steps until convergence.

Figure \ref{fig:lm_quant} compares the quantization levels of a $7$-bit floating point (E3M3) quantizer (left) to a $7$-bit Lloyd-Max quantizer (right) when quantizing a layer of weights from the GPT3-126M model at a per-tensor granularity. As shown, the Lloyd-Max quantizer achieves substantially lower quantization MSE. Further, Table \ref{tab:FP7_vs_LM7} shows the superior perplexity achieved by Lloyd-Max quantizers for bitwidths of $7$, $6$ and $5$. The difference between the quantizers is clear at 5 bits, where per-tensor FP quantization incurs a drastic and unacceptable increase in perplexity, while Lloyd-Max quantization incurs a much smaller increase. Nevertheless, we note that even the optimal Lloyd-Max quantizer incurs a notable ($\sim 1.5$) increase in perplexity due to the coarse granularity of quantization. 

\begin{figure}[h]
  \centering
  \includegraphics[width=0.7\linewidth]{sections/figures/LM7_FP7.pdf}
  \caption{\small Quantization levels and the corresponding quantization MSE of Floating Point (left) vs Lloyd-Max (right) Quantizers for a layer of weights in the GPT3-126M model.}
  \label{fig:lm_quant}
\end{figure}

\begin{table}[h]\scriptsize
\begin{center}
\caption{\label{tab:FP7_vs_LM7} \small Comparing perplexity (lower is better) achieved by floating point quantizers and Lloyd-Max quantizers on a GPT3-126M model for the Wikitext-103 dataset.}
\begin{tabular}{c|cc|c}
\hline
 \multirow{2}{*}{\textbf{Bitwidth}} & \multicolumn{2}{|c|}{\textbf{Floating-Point Quantizer}} & \textbf{Lloyd-Max Quantizer} \\
 & Best Format & Wikitext-103 Perplexity & Wikitext-103 Perplexity \\
\hline
7 & E3M3 & 18.32 & 18.27 \\
6 & E3M2 & 19.07 & 18.51 \\
5 & E4M0 & 43.89 & 19.71 \\
\hline
\end{tabular}
\end{center}
\end{table}

\subsection{Proof of Local Optimality of LO-BCQ}
\label{subsec:lobcq_opt_proof}
For a given block $\bm{b}_j$, the quantization MSE during LO-BCQ can be empirically evaluated as $\frac{1}{L_b}\lVert \bm{b}_j- \bm{\hat{b}}_j\rVert^2_2$ where $\bm{\hat{b}}_j$ is computed from equation (\ref{eq:clustered_quantization_definition}) as $C_{f(\bm{b}_j)}(\bm{b}_j)$. Further, for a given block cluster $\mathcal{B}_i$, we compute the quantization MSE as $\frac{1}{|\mathcal{B}_{i}|}\sum_{\bm{b} \in \mathcal{B}_{i}} \frac{1}{L_b}\lVert \bm{b}- C_i^{(n)}(\bm{b})\rVert^2_2$. Therefore, at the end of iteration $n$, we evaluate the overall quantization MSE $J^{(n)}$ for a given operand $\bm{X}$ composed of $N_c$ block clusters as:
\begin{align*}
    \label{eq:mse_iter_n}
    J^{(n)} = \frac{1}{N_c} \sum_{i=1}^{N_c} \frac{1}{|\mathcal{B}_{i}^{(n)}|}\sum_{\bm{v} \in \mathcal{B}_{i}^{(n)}} \frac{1}{L_b}\lVert \bm{b}- B_i^{(n)}(\bm{b})\rVert^2_2
\end{align*}

At the end of iteration $n$, the codebooks are updated from $\mathcal{C}^{(n-1)}$ to $\mathcal{C}^{(n)}$. However, the mapping of a given vector $\bm{b}_j$ to quantizers $\mathcal{C}^{(n)}$ remains as  $f^{(n)}(\bm{b}_j)$. At the next iteration, during the vector clustering step, $f^{(n+1)}(\bm{b}_j)$ finds new mapping of $\bm{b}_j$ to updated codebooks $\mathcal{C}^{(n)}$ such that the quantization MSE over the candidate codebooks is minimized. Therefore, we obtain the following result for $\bm{b}_j$:
\begin{align*}
\frac{1}{L_b}\lVert \bm{b}_j - C_{f^{(n+1)}(\bm{b}_j)}^{(n)}(\bm{b}_j)\rVert^2_2 \le \frac{1}{L_b}\lVert \bm{b}_j - C_{f^{(n)}(\bm{b}_j)}^{(n)}(\bm{b}_j)\rVert^2_2
\end{align*}

That is, quantizing $\bm{b}_j$ at the end of the block clustering step of iteration $n+1$ results in lower quantization MSE compared to quantizing at the end of iteration $n$. Since this is true for all $\bm{b} \in \bm{X}$, we assert the following:
\begin{equation}
\begin{split}
\label{eq:mse_ineq_1}
    \tilde{J}^{(n+1)} &= \frac{1}{N_c} \sum_{i=1}^{N_c} \frac{1}{|\mathcal{B}_{i}^{(n+1)}|}\sum_{\bm{b} \in \mathcal{B}_{i}^{(n+1)}} \frac{1}{L_b}\lVert \bm{b} - C_i^{(n)}(b)\rVert^2_2 \le J^{(n)}
\end{split}
\end{equation}
where $\tilde{J}^{(n+1)}$ is the the quantization MSE after the vector clustering step at iteration $n+1$.

Next, during the codebook update step (\ref{eq:quantizers_update}) at iteration $n+1$, the per-cluster codebooks $\mathcal{C}^{(n)}$ are updated to $\mathcal{C}^{(n+1)}$ by invoking the Lloyd-Max algorithm \citep{Lloyd}. We know that for any given value distribution, the Lloyd-Max algorithm minimizes the quantization MSE. Therefore, for a given vector cluster $\mathcal{B}_i$ we obtain the following result:

\begin{equation}
    \frac{1}{|\mathcal{B}_{i}^{(n+1)}|}\sum_{\bm{b} \in \mathcal{B}_{i}^{(n+1)}} \frac{1}{L_b}\lVert \bm{b}- C_i^{(n+1)}(\bm{b})\rVert^2_2 \le \frac{1}{|\mathcal{B}_{i}^{(n+1)}|}\sum_{\bm{b} \in \mathcal{B}_{i}^{(n+1)}} \frac{1}{L_b}\lVert \bm{b}- C_i^{(n)}(\bm{b})\rVert^2_2
\end{equation}

The above equation states that quantizing the given block cluster $\mathcal{B}_i$ after updating the associated codebook from $C_i^{(n)}$ to $C_i^{(n+1)}$ results in lower quantization MSE. Since this is true for all the block clusters, we derive the following result: 
\begin{equation}
\begin{split}
\label{eq:mse_ineq_2}
     J^{(n+1)} &= \frac{1}{N_c} \sum_{i=1}^{N_c} \frac{1}{|\mathcal{B}_{i}^{(n+1)}|}\sum_{\bm{b} \in \mathcal{B}_{i}^{(n+1)}} \frac{1}{L_b}\lVert \bm{b}- C_i^{(n+1)}(\bm{b})\rVert^2_2  \le \tilde{J}^{(n+1)}   
\end{split}
\end{equation}

Following (\ref{eq:mse_ineq_1}) and (\ref{eq:mse_ineq_2}), we find that the quantization MSE is non-increasing for each iteration, that is, $J^{(1)} \ge J^{(2)} \ge J^{(3)} \ge \ldots \ge J^{(M)}$ where $M$ is the maximum number of iterations. 
%Therefore, we can say that if the algorithm converges, then it must be that it has converged to a local minimum. 
\hfill $\blacksquare$


\begin{figure}
    \begin{center}
    \includegraphics[width=0.5\textwidth]{sections//figures/mse_vs_iter.pdf}
    \end{center}
    \caption{\small NMSE vs iterations during LO-BCQ compared to other block quantization proposals}
    \label{fig:nmse_vs_iter}
\end{figure}

Figure \ref{fig:nmse_vs_iter} shows the empirical convergence of LO-BCQ across several block lengths and number of codebooks. Also, the MSE achieved by LO-BCQ is compared to baselines such as MXFP and VSQ. As shown, LO-BCQ converges to a lower MSE than the baselines. Further, we achieve better convergence for larger number of codebooks ($N_c$) and for a smaller block length ($L_b$), both of which increase the bitwidth of BCQ (see Eq \ref{eq:bitwidth_bcq}).


\subsection{Additional Accuracy Results}
%Table \ref{tab:lobcq_config} lists the various LOBCQ configurations and their corresponding bitwidths.
\begin{table}
\setlength{\tabcolsep}{4.75pt}
\begin{center}
\caption{\label{tab:lobcq_config} Various LO-BCQ configurations and their bitwidths.}
\begin{tabular}{|c||c|c|c|c||c|c||c|} 
\hline
 & \multicolumn{4}{|c||}{$L_b=8$} & \multicolumn{2}{|c||}{$L_b=4$} & $L_b=2$ \\
 \hline
 \backslashbox{$L_A$\kern-1em}{\kern-1em$N_c$} & 2 & 4 & 8 & 16 & 2 & 4 & 2 \\
 \hline
 64 & 4.25 & 4.375 & 4.5 & 4.625 & 4.375 & 4.625 & 4.625\\
 \hline
 32 & 4.375 & 4.5 & 4.625& 4.75 & 4.5 & 4.75 & 4.75 \\
 \hline
 16 & 4.625 & 4.75& 4.875 & 5 & 4.75 & 5 & 5 \\
 \hline
\end{tabular}
\end{center}
\end{table}

%\subsection{Perplexity achieved by various LO-BCQ configurations on Wikitext-103 dataset}

\begin{table} \centering
\begin{tabular}{|c||c|c|c|c||c|c||c|} 
\hline
 $L_b \rightarrow$& \multicolumn{4}{c||}{8} & \multicolumn{2}{c||}{4} & 2\\
 \hline
 \backslashbox{$L_A$\kern-1em}{\kern-1em$N_c$} & 2 & 4 & 8 & 16 & 2 & 4 & 2  \\
 %$N_c \rightarrow$ & 2 & 4 & 8 & 16 & 2 & 4 & 2 \\
 \hline
 \hline
 \multicolumn{8}{c}{GPT3-1.3B (FP32 PPL = 9.98)} \\ 
 \hline
 \hline
 64 & 10.40 & 10.23 & 10.17 & 10.15 &  10.28 & 10.18 & 10.19 \\
 \hline
 32 & 10.25 & 10.20 & 10.15 & 10.12 &  10.23 & 10.17 & 10.17 \\
 \hline
 16 & 10.22 & 10.16 & 10.10 & 10.09 &  10.21 & 10.14 & 10.16 \\
 \hline
  \hline
 \multicolumn{8}{c}{GPT3-8B (FP32 PPL = 7.38)} \\ 
 \hline
 \hline
 64 & 7.61 & 7.52 & 7.48 &  7.47 &  7.55 &  7.49 & 7.50 \\
 \hline
 32 & 7.52 & 7.50 & 7.46 &  7.45 &  7.52 &  7.48 & 7.48  \\
 \hline
 16 & 7.51 & 7.48 & 7.44 &  7.44 &  7.51 &  7.49 & 7.47  \\
 \hline
\end{tabular}
\caption{\label{tab:ppl_gpt3_abalation} Wikitext-103 perplexity across GPT3-1.3B and 8B models.}
\end{table}

\begin{table} \centering
\begin{tabular}{|c||c|c|c|c||} 
\hline
 $L_b \rightarrow$& \multicolumn{4}{c||}{8}\\
 \hline
 \backslashbox{$L_A$\kern-1em}{\kern-1em$N_c$} & 2 & 4 & 8 & 16 \\
 %$N_c \rightarrow$ & 2 & 4 & 8 & 16 & 2 & 4 & 2 \\
 \hline
 \hline
 \multicolumn{5}{|c|}{Llama2-7B (FP32 PPL = 5.06)} \\ 
 \hline
 \hline
 64 & 5.31 & 5.26 & 5.19 & 5.18  \\
 \hline
 32 & 5.23 & 5.25 & 5.18 & 5.15  \\
 \hline
 16 & 5.23 & 5.19 & 5.16 & 5.14  \\
 \hline
 \multicolumn{5}{|c|}{Nemotron4-15B (FP32 PPL = 5.87)} \\ 
 \hline
 \hline
 64  & 6.3 & 6.20 & 6.13 & 6.08  \\
 \hline
 32  & 6.24 & 6.12 & 6.07 & 6.03  \\
 \hline
 16  & 6.12 & 6.14 & 6.04 & 6.02  \\
 \hline
 \multicolumn{5}{|c|}{Nemotron4-340B (FP32 PPL = 3.48)} \\ 
 \hline
 \hline
 64 & 3.67 & 3.62 & 3.60 & 3.59 \\
 \hline
 32 & 3.63 & 3.61 & 3.59 & 3.56 \\
 \hline
 16 & 3.61 & 3.58 & 3.57 & 3.55 \\
 \hline
\end{tabular}
\caption{\label{tab:ppl_llama7B_nemo15B} Wikitext-103 perplexity compared to FP32 baseline in Llama2-7B and Nemotron4-15B, 340B models}
\end{table}

%\subsection{Perplexity achieved by various LO-BCQ configurations on MMLU dataset}


\begin{table} \centering
\begin{tabular}{|c||c|c|c|c||c|c|c|c|} 
\hline
 $L_b \rightarrow$& \multicolumn{4}{c||}{8} & \multicolumn{4}{c||}{8}\\
 \hline
 \backslashbox{$L_A$\kern-1em}{\kern-1em$N_c$} & 2 & 4 & 8 & 16 & 2 & 4 & 8 & 16  \\
 %$N_c \rightarrow$ & 2 & 4 & 8 & 16 & 2 & 4 & 2 \\
 \hline
 \hline
 \multicolumn{5}{|c|}{Llama2-7B (FP32 Accuracy = 45.8\%)} & \multicolumn{4}{|c|}{Llama2-70B (FP32 Accuracy = 69.12\%)} \\ 
 \hline
 \hline
 64 & 43.9 & 43.4 & 43.9 & 44.9 & 68.07 & 68.27 & 68.17 & 68.75 \\
 \hline
 32 & 44.5 & 43.8 & 44.9 & 44.5 & 68.37 & 68.51 & 68.35 & 68.27  \\
 \hline
 16 & 43.9 & 42.7 & 44.9 & 45 & 68.12 & 68.77 & 68.31 & 68.59  \\
 \hline
 \hline
 \multicolumn{5}{|c|}{GPT3-22B (FP32 Accuracy = 38.75\%)} & \multicolumn{4}{|c|}{Nemotron4-15B (FP32 Accuracy = 64.3\%)} \\ 
 \hline
 \hline
 64 & 36.71 & 38.85 & 38.13 & 38.92 & 63.17 & 62.36 & 63.72 & 64.09 \\
 \hline
 32 & 37.95 & 38.69 & 39.45 & 38.34 & 64.05 & 62.30 & 63.8 & 64.33  \\
 \hline
 16 & 38.88 & 38.80 & 38.31 & 38.92 & 63.22 & 63.51 & 63.93 & 64.43  \\
 \hline
\end{tabular}
\caption{\label{tab:mmlu_abalation} Accuracy on MMLU dataset across GPT3-22B, Llama2-7B, 70B and Nemotron4-15B models.}
\end{table}


%\subsection{Perplexity achieved by various LO-BCQ configurations on LM evaluation harness}

\begin{table} \centering
\begin{tabular}{|c||c|c|c|c||c|c|c|c|} 
\hline
 $L_b \rightarrow$& \multicolumn{4}{c||}{8} & \multicolumn{4}{c||}{8}\\
 \hline
 \backslashbox{$L_A$\kern-1em}{\kern-1em$N_c$} & 2 & 4 & 8 & 16 & 2 & 4 & 8 & 16  \\
 %$N_c \rightarrow$ & 2 & 4 & 8 & 16 & 2 & 4 & 2 \\
 \hline
 \hline
 \multicolumn{5}{|c|}{Race (FP32 Accuracy = 37.51\%)} & \multicolumn{4}{|c|}{Boolq (FP32 Accuracy = 64.62\%)} \\ 
 \hline
 \hline
 64 & 36.94 & 37.13 & 36.27 & 37.13 & 63.73 & 62.26 & 63.49 & 63.36 \\
 \hline
 32 & 37.03 & 36.36 & 36.08 & 37.03 & 62.54 & 63.51 & 63.49 & 63.55  \\
 \hline
 16 & 37.03 & 37.03 & 36.46 & 37.03 & 61.1 & 63.79 & 63.58 & 63.33  \\
 \hline
 \hline
 \multicolumn{5}{|c|}{Winogrande (FP32 Accuracy = 58.01\%)} & \multicolumn{4}{|c|}{Piqa (FP32 Accuracy = 74.21\%)} \\ 
 \hline
 \hline
 64 & 58.17 & 57.22 & 57.85 & 58.33 & 73.01 & 73.07 & 73.07 & 72.80 \\
 \hline
 32 & 59.12 & 58.09 & 57.85 & 58.41 & 73.01 & 73.94 & 72.74 & 73.18  \\
 \hline
 16 & 57.93 & 58.88 & 57.93 & 58.56 & 73.94 & 72.80 & 73.01 & 73.94  \\
 \hline
\end{tabular}
\caption{\label{tab:mmlu_abalation} Accuracy on LM evaluation harness tasks on GPT3-1.3B model.}
\end{table}

\begin{table} \centering
\begin{tabular}{|c||c|c|c|c||c|c|c|c|} 
\hline
 $L_b \rightarrow$& \multicolumn{4}{c||}{8} & \multicolumn{4}{c||}{8}\\
 \hline
 \backslashbox{$L_A$\kern-1em}{\kern-1em$N_c$} & 2 & 4 & 8 & 16 & 2 & 4 & 8 & 16  \\
 %$N_c \rightarrow$ & 2 & 4 & 8 & 16 & 2 & 4 & 2 \\
 \hline
 \hline
 \multicolumn{5}{|c|}{Race (FP32 Accuracy = 41.34\%)} & \multicolumn{4}{|c|}{Boolq (FP32 Accuracy = 68.32\%)} \\ 
 \hline
 \hline
 64 & 40.48 & 40.10 & 39.43 & 39.90 & 69.20 & 68.41 & 69.45 & 68.56 \\
 \hline
 32 & 39.52 & 39.52 & 40.77 & 39.62 & 68.32 & 67.43 & 68.17 & 69.30  \\
 \hline
 16 & 39.81 & 39.71 & 39.90 & 40.38 & 68.10 & 66.33 & 69.51 & 69.42  \\
 \hline
 \hline
 \multicolumn{5}{|c|}{Winogrande (FP32 Accuracy = 67.88\%)} & \multicolumn{4}{|c|}{Piqa (FP32 Accuracy = 78.78\%)} \\ 
 \hline
 \hline
 64 & 66.85 & 66.61 & 67.72 & 67.88 & 77.31 & 77.42 & 77.75 & 77.64 \\
 \hline
 32 & 67.25 & 67.72 & 67.72 & 67.00 & 77.31 & 77.04 & 77.80 & 77.37  \\
 \hline
 16 & 68.11 & 68.90 & 67.88 & 67.48 & 77.37 & 78.13 & 78.13 & 77.69  \\
 \hline
\end{tabular}
\caption{\label{tab:mmlu_abalation} Accuracy on LM evaluation harness tasks on GPT3-8B model.}
\end{table}

\begin{table} \centering
\begin{tabular}{|c||c|c|c|c||c|c|c|c|} 
\hline
 $L_b \rightarrow$& \multicolumn{4}{c||}{8} & \multicolumn{4}{c||}{8}\\
 \hline
 \backslashbox{$L_A$\kern-1em}{\kern-1em$N_c$} & 2 & 4 & 8 & 16 & 2 & 4 & 8 & 16  \\
 %$N_c \rightarrow$ & 2 & 4 & 8 & 16 & 2 & 4 & 2 \\
 \hline
 \hline
 \multicolumn{5}{|c|}{Race (FP32 Accuracy = 40.67\%)} & \multicolumn{4}{|c|}{Boolq (FP32 Accuracy = 76.54\%)} \\ 
 \hline
 \hline
 64 & 40.48 & 40.10 & 39.43 & 39.90 & 75.41 & 75.11 & 77.09 & 75.66 \\
 \hline
 32 & 39.52 & 39.52 & 40.77 & 39.62 & 76.02 & 76.02 & 75.96 & 75.35  \\
 \hline
 16 & 39.81 & 39.71 & 39.90 & 40.38 & 75.05 & 73.82 & 75.72 & 76.09  \\
 \hline
 \hline
 \multicolumn{5}{|c|}{Winogrande (FP32 Accuracy = 70.64\%)} & \multicolumn{4}{|c|}{Piqa (FP32 Accuracy = 79.16\%)} \\ 
 \hline
 \hline
 64 & 69.14 & 70.17 & 70.17 & 70.56 & 78.24 & 79.00 & 78.62 & 78.73 \\
 \hline
 32 & 70.96 & 69.69 & 71.27 & 69.30 & 78.56 & 79.49 & 79.16 & 78.89  \\
 \hline
 16 & 71.03 & 69.53 & 69.69 & 70.40 & 78.13 & 79.16 & 79.00 & 79.00  \\
 \hline
\end{tabular}
\caption{\label{tab:mmlu_abalation} Accuracy on LM evaluation harness tasks on GPT3-22B model.}
\end{table}

\begin{table} \centering
\begin{tabular}{|c||c|c|c|c||c|c|c|c|} 
\hline
 $L_b \rightarrow$& \multicolumn{4}{c||}{8} & \multicolumn{4}{c||}{8}\\
 \hline
 \backslashbox{$L_A$\kern-1em}{\kern-1em$N_c$} & 2 & 4 & 8 & 16 & 2 & 4 & 8 & 16  \\
 %$N_c \rightarrow$ & 2 & 4 & 8 & 16 & 2 & 4 & 2 \\
 \hline
 \hline
 \multicolumn{5}{|c|}{Race (FP32 Accuracy = 44.4\%)} & \multicolumn{4}{|c|}{Boolq (FP32 Accuracy = 79.29\%)} \\ 
 \hline
 \hline
 64 & 42.49 & 42.51 & 42.58 & 43.45 & 77.58 & 77.37 & 77.43 & 78.1 \\
 \hline
 32 & 43.35 & 42.49 & 43.64 & 43.73 & 77.86 & 75.32 & 77.28 & 77.86  \\
 \hline
 16 & 44.21 & 44.21 & 43.64 & 42.97 & 78.65 & 77 & 76.94 & 77.98  \\
 \hline
 \hline
 \multicolumn{5}{|c|}{Winogrande (FP32 Accuracy = 69.38\%)} & \multicolumn{4}{|c|}{Piqa (FP32 Accuracy = 78.07\%)} \\ 
 \hline
 \hline
 64 & 68.9 & 68.43 & 69.77 & 68.19 & 77.09 & 76.82 & 77.09 & 77.86 \\
 \hline
 32 & 69.38 & 68.51 & 68.82 & 68.90 & 78.07 & 76.71 & 78.07 & 77.86  \\
 \hline
 16 & 69.53 & 67.09 & 69.38 & 68.90 & 77.37 & 77.8 & 77.91 & 77.69  \\
 \hline
\end{tabular}
\caption{\label{tab:mmlu_abalation} Accuracy on LM evaluation harness tasks on Llama2-7B model.}
\end{table}

\begin{table} \centering
\begin{tabular}{|c||c|c|c|c||c|c|c|c|} 
\hline
 $L_b \rightarrow$& \multicolumn{4}{c||}{8} & \multicolumn{4}{c||}{8}\\
 \hline
 \backslashbox{$L_A$\kern-1em}{\kern-1em$N_c$} & 2 & 4 & 8 & 16 & 2 & 4 & 8 & 16  \\
 %$N_c \rightarrow$ & 2 & 4 & 8 & 16 & 2 & 4 & 2 \\
 \hline
 \hline
 \multicolumn{5}{|c|}{Race (FP32 Accuracy = 48.8\%)} & \multicolumn{4}{|c|}{Boolq (FP32 Accuracy = 85.23\%)} \\ 
 \hline
 \hline
 64 & 49.00 & 49.00 & 49.28 & 48.71 & 82.82 & 84.28 & 84.03 & 84.25 \\
 \hline
 32 & 49.57 & 48.52 & 48.33 & 49.28 & 83.85 & 84.46 & 84.31 & 84.93  \\
 \hline
 16 & 49.85 & 49.09 & 49.28 & 48.99 & 85.11 & 84.46 & 84.61 & 83.94  \\
 \hline
 \hline
 \multicolumn{5}{|c|}{Winogrande (FP32 Accuracy = 79.95\%)} & \multicolumn{4}{|c|}{Piqa (FP32 Accuracy = 81.56\%)} \\ 
 \hline
 \hline
 64 & 78.77 & 78.45 & 78.37 & 79.16 & 81.45 & 80.69 & 81.45 & 81.5 \\
 \hline
 32 & 78.45 & 79.01 & 78.69 & 80.66 & 81.56 & 80.58 & 81.18 & 81.34  \\
 \hline
 16 & 79.95 & 79.56 & 79.79 & 79.72 & 81.28 & 81.66 & 81.28 & 80.96  \\
 \hline
\end{tabular}
\caption{\label{tab:mmlu_abalation} Accuracy on LM evaluation harness tasks on Llama2-70B model.}
\end{table}

%\section{MSE Studies}
%\textcolor{red}{TODO}


\subsection{Number Formats and Quantization Method}
\label{subsec:numFormats_quantMethod}
\subsubsection{Integer Format}
An $n$-bit signed integer (INT) is typically represented with a 2s-complement format \citep{yao2022zeroquant,xiao2023smoothquant,dai2021vsq}, where the most significant bit denotes the sign.

\subsubsection{Floating Point Format}
An $n$-bit signed floating point (FP) number $x$ comprises of a 1-bit sign ($x_{\mathrm{sign}}$), $B_m$-bit mantissa ($x_{\mathrm{mant}}$) and $B_e$-bit exponent ($x_{\mathrm{exp}}$) such that $B_m+B_e=n-1$. The associated constant exponent bias ($E_{\mathrm{bias}}$) is computed as $(2^{{B_e}-1}-1)$. We denote this format as $E_{B_e}M_{B_m}$.  

\subsubsection{Quantization Scheme}
\label{subsec:quant_method}
A quantization scheme dictates how a given unquantized tensor is converted to its quantized representation. We consider FP formats for the purpose of illustration. Given an unquantized tensor $\bm{X}$ and an FP format $E_{B_e}M_{B_m}$, we first, we compute the quantization scale factor $s_X$ that maps the maximum absolute value of $\bm{X}$ to the maximum quantization level of the $E_{B_e}M_{B_m}$ format as follows:
\begin{align}
\label{eq:sf}
    s_X = \frac{\mathrm{max}(|\bm{X}|)}{\mathrm{max}(E_{B_e}M_{B_m})}
\end{align}
In the above equation, $|\cdot|$ denotes the absolute value function.

Next, we scale $\bm{X}$ by $s_X$ and quantize it to $\hat{\bm{X}}$ by rounding it to the nearest quantization level of $E_{B_e}M_{B_m}$ as:

\begin{align}
\label{eq:tensor_quant}
    \hat{\bm{X}} = \text{round-to-nearest}\left(\frac{\bm{X}}{s_X}, E_{B_e}M_{B_m}\right)
\end{align}

We perform dynamic max-scaled quantization \citep{wu2020integer}, where the scale factor $s$ for activations is dynamically computed during runtime.

\subsection{Vector Scaled Quantization}
\begin{wrapfigure}{r}{0.35\linewidth}
  \centering
  \includegraphics[width=\linewidth]{sections/figures/vsquant.jpg}
  \caption{\small Vectorwise decomposition for per-vector scaled quantization (VSQ \citep{dai2021vsq}).}
  \label{fig:vsquant}
\end{wrapfigure}
During VSQ \citep{dai2021vsq}, the operand tensors are decomposed into 1D vectors in a hardware friendly manner as shown in Figure \ref{fig:vsquant}. Since the decomposed tensors are used as operands in matrix multiplications during inference, it is beneficial to perform this decomposition along the reduction dimension of the multiplication. The vectorwise quantization is performed similar to tensorwise quantization described in Equations \ref{eq:sf} and \ref{eq:tensor_quant}, where a scale factor $s_v$ is required for each vector $\bm{v}$ that maps the maximum absolute value of that vector to the maximum quantization level. While smaller vector lengths can lead to larger accuracy gains, the associated memory and computational overheads due to the per-vector scale factors increases. To alleviate these overheads, VSQ \citep{dai2021vsq} proposed a second level quantization of the per-vector scale factors to unsigned integers, while MX \citep{rouhani2023shared} quantizes them to integer powers of 2 (denoted as $2^{INT}$).

\subsubsection{MX Format}
The MX format proposed in \citep{rouhani2023microscaling} introduces the concept of sub-block shifting. For every two scalar elements of $b$-bits each, there is a shared exponent bit. The value of this exponent bit is determined through an empirical analysis that targets minimizing quantization MSE. We note that the FP format $E_{1}M_{b}$ is strictly better than MX from an accuracy perspective since it allocates a dedicated exponent bit to each scalar as opposed to sharing it across two scalars. Therefore, we conservatively bound the accuracy of a $b+2$-bit signed MX format with that of a $E_{1}M_{b}$ format in our comparisons. For instance, we use E1M2 format as a proxy for MX4.

\begin{figure}
    \centering
    \includegraphics[width=1\linewidth]{sections//figures/BlockFormats.pdf}
    \caption{\small Comparing LO-BCQ to MX format.}
    \label{fig:block_formats}
\end{figure}

Figure \ref{fig:block_formats} compares our $4$-bit LO-BCQ block format to MX \citep{rouhani2023microscaling}. As shown, both LO-BCQ and MX decompose a given operand tensor into block arrays and each block array into blocks. Similar to MX, we find that per-block quantization ($L_b < L_A$) leads to better accuracy due to increased flexibility. While MX achieves this through per-block $1$-bit micro-scales, we associate a dedicated codebook to each block through a per-block codebook selector. Further, MX quantizes the per-block array scale-factor to E8M0 format without per-tensor scaling. In contrast during LO-BCQ, we find that per-tensor scaling combined with quantization of per-block array scale-factor to E4M3 format results in superior inference accuracy across models. 


\end{document}
\endinput
