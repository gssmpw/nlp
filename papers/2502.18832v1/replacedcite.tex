\section{Related Work}
\label{sec:related}
\vspace{-5pt}

{\bf Improving eBPF.} eBPF has evolved from simple
    use cases like packet filtering____
    into a general-purpose kernel extension language and programming framework
    that enables many innovative projects____.
Recent work is making active progress to improve the correctness and security
    of the eBPF infrastructure, including
    fuzzing and bug finding____,
    formal verification____,
    sandboxing____,
    and integrating with hardware protection mechanisms____.
eBPF's design, which relies on an in-kernel static verifier for extension safety,
    inevitably creates the \gap{} (\S\ref{sec:motivation}).
In contrast, Rax provides an alternative to develop and maintain large, complex kernel extensions directly with
    high-level language safety, avoiding the \gap{}.

% In recent years, as the limitations of eBPF's static verification begin to
%    surface due to the \gap{}, many works propose to address such limitations
%    while providing needed safety for extensions---(1) adding
%    runtime-checks____, (2) extending in-kernel verification
%    to user space____,

% HIVE aims to improve the security aspect of eBPF via
%    hardware protection mechanism on AArch64. By offloading a huge portion of
%    safety and security checks from the verifier to hardware mechanisms, HIVE
%    can partially close the \gap{} (verification complexity limits are
%    lifted). However, we expect the \gap{} to remain due to the checks still
%    present in the verifier.
% At the same time, HIVE is limited by its use of a specific architecture.

% Safety and usability of kernel extensions
%   SPIN and VINO
%   difference in philosophical design
%   modern context

\para{Other frameworks.} The idea of building safe OS components using
    safe languages was proposed by SPIN____
    and revisited by Singularity____, Tock____, and a few recent
    discussions____.
However, adopting them in practice is challenging as they are based on
    clean-slate OS designs.
Rax develops a practical kernel extension framework for Linux, taking the
    opportunity of recent support of Rust as a safe language for OS code.
It addresses the key challenges of enforcing safe code only, interfacing
    with unsafe C code, and providing safety guarantees beyond language-based safety.

KFlex____ is a recent kernel extension framework built on top of eBPF.
KFlex aims to improve the flexibility of eBPF to let developers express diverse functionality in extensions.
It employs an efficient runtime by co-designing it with the existing eBPF verifier: (1) its Software Fault Isolation (SFI)
    elides checks already done by the verifier for efficiency,
% to sanitize accesses from the
%    extension to its own memory and
    and (2) its termination mechanism uses the verifier to
    statically compute the kernel resources acquired by the extension.
Rax made the same design choice as KFlex to use a lightweight runtime for safety properties that are hard to
    check statically.
Unlike KFlex, which is co-designed with the eBPF verifier,
    Rax eliminates the verifier to close the \gap{}.

BCF____ is a recent proposal to enhance eBPF's in-kernel verification
    with help from user space, asking for proof when the
    verifier fails to reason about certain program properties.
% Like KFlex, BCF fails to fundamentally bridge the \gap{} by retaining
%    a separate verification step, leaving users responsible for selecting and
%    proving the refinement condition, adding up developer burden and program
%    loading time.
The idea echoes proof-carrying code____ which asks a program to attach a formal proof
    that its code obeys the safety policy.
BCF leverages the eBPF verifier's range analysis and symbolic execution for proof
    generation but still requires developers to specify safety conditions to aid the generation.
Its uses of the verifier still lead to the \gap{}.

% Isolation techniques____
% and (3) Proof Carrying Code (PCC) approaches ____.
% The PCC____ approach allows  The verifier
% can be validate this proof, and if validation succeeds, the code is
% guaranteed to respect the safety policy.
% The major steps in PCC are identifying a condition that would aid in satisfying the safety policy, generating a formal statement for the desired condition, and finally producing
% a formal proof for the statement that can be verified when shipped
% with the code.
% BCF has automated the
% generation of a formal statement using
% and the generation of proof from the formal statement
% using solvers in the user space. Identifying the desired condition
% still depends on the user and thus falls behind in closing the \gap{}.
% }

%BCF____ extends in-kernel range analysis
%using user space, providing a refinement condition for the program and
%requiring a verification proof. It doesn’t bridge the \gap{} and leaves users
%responsible for selecting the refinement condition. Additionally, continuous proof requests could have a huge impact on loading times for complex programs.
%}

% VINO____ and SLIC____ both uses software fault isolation
%    techniques and interposition provide safety guarantees for extensions at
%    runtime, trading off performance.




%Similarly, SLIC ____ uses interposition as a mechanism to allow trusted
%    extensions to be run with only minimal operating system changes.
%Nooks____ followed in the same vein while KSplit____ attempted to
%automate the process of isolating kernel drivers by identifying and synchronizing
%shared kernel and driver state using static analysis.

% \jinghao{I have some trouble articulating the difference in philosophical design}
% \mvle{my attempt}
% \jinghao{It seems that nooks and ksplit does not belong here.}
%We leverage and extend these bodies of work by utilizing a safe language combined with
%a thin run time to achieve a safe and efficient kernel extension framework that
%closes the \gap{}.
% \ayushb{no mention of KFlex in related work.}

% Rust-base safety
%   Rust-based OS
%   Linux kernel Rust support
%   not for kernel extensions

\para{Rust for OS kernels.} Rust has been embraced
    by modern OSes____ as
    practical language which leads to safer code.
Recent work shows the promises to build new OS kernels using Rust____.
We claim no novelty of using Rust as a language.
In fact, a safe language alone does not lead to system safety,
    as exemplified by Rust kernel modules____.
Rax shows an example of how to build upon language-based safety
    to enable and enforce safe kernel extension programs.

%While \projname{} also applies Rust to the kernel context, its goal is to use
%    Rust to create safe and usable kernel extensions and has to work with
%    unsafe kernel code, which differs from other works that explore an OS
%    written in safe language.

% \subsection{Rust-based system}
% \begin{itemize}
%     \item Theseus
%     \item Redleaf
%     \item Evolving Operating Systems Towards Secure Kernel-Driver Interfaces
%     \item Michael Swift works
%     \item Leveraging Rust for Lightweight OS Correctness
%     \item Cross-Language Attacks from NDSS 2022
% \end{itemize}