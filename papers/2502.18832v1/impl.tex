\vspace{-8pt}
\section{Implementation}
\label{sec:impl}
\vspace{-5pt}
%We now discuss how of \projname{} is implemented.
% The implementation of \projname{} consists of three parts, the \projname{}
%    kernel crate, the kernel support, and the compiler support.

\begin{figure*}[t]
    \includegraphics[width=\linewidth]{comparison.pdf}
    \centering
    \vspace{-15pt}
    \caption{Cache invalidation implementation of Rax-BMC and eBPF-BMC; Rax leads to cleaner, simpler code.}
    \vspace{-5pt}
    \label{fig:rust-code}
\end{figure*}

We implement \projname{} on Linux v6.11.
\projname{} currently supports five eBPF program types (tracepoint, kprobe,
    perf-event, XDP, and TC) and shares their in-kernel hookpoints.
% \projname{} builds its kernel interface on top of the eBPF helper function interface.
Rax only includes helpers for kernel interactions.
helpers introduced due to
    contraints posed by the eBPF verifier
    (e.g., \texttt{\small bpf\_loop}, \texttt{\small bpf\_strtol}, and
    \texttt{\small bpf\_strncmp}) are entirely excluded by \projname{}.
% \projname{} does not support eBPF kfuncs~\cite{kfuncs}. \tianyin{give the reason}
% (tianyin) I don't understand this point -- ain't you reuse eBPF helpers? what's going on?
% We are actively investigating  kernel interface for safe kernel extensions.
%     -- \projname{} has already implemented some simple helper functions
%    (e.g., \texttt{\small bpf\_get\_current\_task})partially
%    in the Rust kernel crate.

\para{Kernel crate.}
% src/panic.rs: 7 + 12 + 6 + 1 + 1 = 27
% src/spinlock.rs: 1 + 1 = 2
% src/per_cpu.rs: 8 + 8 + 8 + 8 + 8 + 8 + 8 + 4 = 60
% src/base_helper.rs: 9 + 8 + 1 + 8 + 1 + 4 + 1 + 1 + 7 + 1 + 5 + 1 + 6 + 1 + 6 + 1 + 1 + 5 + 1 + 6 + 1 + 1 + 1 + 1 + 1 + 1 + 1 + 9 + 1 + 3 + 3 + 3 + 1 + 3 = 104
% src/pt_regs.rs: 1 + 1 = 2
% src/task_struct.rs: 1 + 1 + 1 = 3
% src/xdp/xdp_impl.rs: 1 + 1 + 3 + 1 + 1 + 1 + 3 + 5 + 5 + 5 + 5 + 3 + 6 = 40
% src/utils.rs: 1 + 1 = 2
% src/sched_cls/sched_cls_impl.rs: 5 + 5 + 9 + 1 + 10 + 10 + 5 + 1 + 5 + 5 + 5 + 5 + 3 + 6 + 1 + 3 = 79
% src/tracepoint/tp_impl.rs: 3 + 3 = 6
% src/random32.rs: 1
% src/perf_event/perf_event_impl.rs: 1 + 1 + 1 + 1 + 3 + 1 + 3 = 11
% src/map.rs: 1 + 1 = 2
% src/kprobe/kprobe_impl.rs: 1 + 1 = 2
% src/debug.rs: 1
% below are the ones currently not compiled in, but we included in the loc
% src/sysctl.rs: 1 + 1 + 1 = 3
% src/timekeeping.rs: 1 + 1 + 1 + 1 + 1 = 5
% src/read_once.rs: 10
The Rax kernel crate is implemented in 3.5K lines of Rust code, %(a mixture of safe and unsafe Rust).
    among which 360 lines are unsafe Rust code.
% It serves as a safe interface for \projname{} programs to interact with the kernel.
% \new{
% We implement the kernel crate with 3.8k LoC, including 3.5k lines of Rust code
%    and ~300 lines of scripts and glue code.
The kernel crate contains the following components:
% }
%\tianyin{how many lines of code?}
%     \milo{Missing portion?}
\begin{packed_itemize}
\item {\it Helper function interface} in Rax is
    implemented on top of eBPF helpers, with
    wrapping code that allows Rax extensions to invoke helpers with safe Rust types.
% Doing so comes at a cost of losing load-time optimization (e.g. inlining) done by the eBPF
%    verifier on a few helpers, which we evaluate in
%    \S\ref{eval:map}.

\item {\it Kernel data-type bindings} are generated for the extension to access kernel
    data types defined in C.
\projname{} uses rust-bindgen~\cite{bindgen} to automatically generate
    kernel bindings and integrates it into the build process of extensions.
\new{
\projname{} programs need to be rebuilt for each kernel they
    target to account for ABI differences in kernel data types.
}

\item {\it Program context} in Rax is wrapped in a Rust struct, which marks
    the context as private and implements getter methods for its
    public fields.
%eBPF hides irrelevant fields of the kernel internal struct that serves as
%    the program argument.
% eBPF emulates the data encapsulation commonly found in object oriented languages
%    on the ``context'' struct programs take as argument and hides its private
%    fields.
%This done by providing the program a ``user'' struct with only needed fields and
%    rewrite accessing instructions at verification time to the kernel struct.
% This is done by exporting to the program a simplified stub struct that only
%    contains the public fields to program against.
% The compiler generates instructions that access fields in the stub struct; at
%     verification time, the verifier checks these accesses and rewrites them to
%     correctly reference the fields in the internal context struct.
%\projname{} encapsulates the generated binding of the
%    kernel internal struct in a Rust struct, in which each field accessible by
%    the extension has an associated getter method.
\end{packed_itemize}

\vspace{-5pt}
\para{Kernel support.}
\projname{} implements the extension load code and the runtime in the
    kernel in 2.2K lines of C code on vanilla Linux.
% (tianyin) does not sound like a good idea
% \projname{} extends the \texttt{\small bpf()} system
%    call to load a \projname{} program.
To load an extension, the kernel parses the ELF executable of the extension
    and locates all the \texttt{\small LOAD}
    segments in the executable.
It then allocates new pages and maps the \texttt{\small LOAD} segments into the kernel
    address space based on the size and permissions of the segments.
The load function is responsible for fixups on the program code to resolve
    referenced kernel helpers and eBPF maps.
The \projname{} runtime in the kernel consists of the stack
    unwinding mechanisms (\S\ref{principle:eh}), support for dedicated
    kernel stack (\S\ref{principle:stack}) and
    termination (\S\ref{principle:termination}).

\para{Compiler support.}
\projname{} implements a compiler pass for \projname{}-specific
    compile-time instrumentations on the stack (\S\ref{principle:stack}).
We take advantage of Rust's use of LLVM~\cite{llvm} as its default code
    generation backend and implement the pass in LLVM.
A \projname{}-specific compiler switch is also added to the Rust compiler
    frontend (rustc~\cite{rustc}) to gate the \projname{} compiler pass.
% \new{
% We implement the LLVM pass and the rustc swtich in 720 LoC and 37 LoC,
%    respectively.
% }
