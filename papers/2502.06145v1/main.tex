% CVPR 2025 Paper Template; see https://github.com/cvpr-org/author-kit

\documentclass[10pt,twocolumn,letterpaper]{article}

%%%%%%%%% PAPER TYPE  - PLEASE UPDATE FOR FINAL VERSION
\usepackage{cvpr}              % To produce the CAMERA-READY version
%\usepackage[review]{cvpr}      % To produce the REVIEW version
% \usepackage[pagenumbers]{cvpr} % To force page numbers, e.g. for an arXiv version

% Import additional packages in the preamble file, before hyperref
\newcommand{\CG}{\mathcal{G}\xspace}
\newcommand{\CV}{\mathcal{V}\xspace}
\newcommand{\CE}{\mathcal{E}\xspace}
\newcommand{\CA}{\mathcal{A}\xspace}
\newcommand{\CF}{\mathcal{F}\xspace}
\newcommand{\CR}{\mathcal{R}\xspace}
\newcommand{\CB}{\mathcal{B}\xspace}
\newcommand{\CX}{\mathcal{X}\xspace}
\newcommand{\CK}{\mathcal{K}\xspace}
\newcommand{\CM}{\mathcal{M}\xspace}
\newcommand{\CC}{\mathcal{C}\xspace}
\newcommand{\CL}{\mathcal{L}\xspace}
\newcommand{\CI}{\mathcal{I}\xspace}
\newcommand{\CQ}{\mathcal{Q}\xspace}
\newcommand{\CO}{\mathcal{O}\xspace}
\newcommand{\CP}{\mathcal{P}\xspace}
\newcommand{\CS}{\mathcal{S}\xspace}
\newcommand{\CT}{\mathcal{T}\xspace}
\newcommand{\CJ}{\mathcal{J}\xspace}
\usepackage[para]{footmisc}
\usepackage{subfig}
% \usepackage{subcaption}
% \usepackage{array}
% \usepackage{colortbl}



% It is strongly recommended to use hyperref, especially for the review version.
% hyperref with option pagebackref eases the reviewers' job.
% Please disable hyperref *only* if you encounter grave issues, 
% e.g. with the file validation for the camera-ready version.
%
% If you comment hyperref and then uncomment it, you should delete *.aux before re-running LaTeX.
% (Or just hit 'q' on the first LaTeX run, let it finish, and you should be clear).
\definecolor{cvprblue}{rgb}{0.21,0.49,0.74}
\usepackage[pagebackref,breaklinks,colorlinks,allcolors=cvprblue]{hyperref}

\usepackage{indentfirst}

%%%%%%%%% PAPER ID  - PLEASE UPDATE
\def\paperID{*****} % *** Enter the Paper ID here
\def\confName{CVPR}
\def\confYear{2025}

%%%%%%%%% TITLE - PLEASE UPDATE
\title{Animate Anyone 2: High-Fidelity Character Image Animation with Environment Affordance}

%%%%%%%%% AUTHORS - PLEASE UPDATE
% \author{First Author\\
% Institution1\\
% Institution1 address\\
% {\tt\small firstauthor@i1.org}
% % For a paper whose authors are all at the same institution,
% % omit the following lines up until the closing ``}''.
% % Additional authors and addresses can be added with ``\and'',
% % just like the second author.
% % To save space, use either the email address or home page, not both
% \and
% Second Author\\
% Institution2\\
% First line of institution2 address\\
% {\tt\small secondauthor@i2.org}
% }

\author{Li Hu$^*$ \quad Guangyuan Wang$^*$ \quad Zhen Shen \quad Xin Gao \quad Dechao Meng \quad Lian Zhuo \\
Peng Zhang \quad Bang Zhang \quad Liefeng Bo\\
Tongyi Lab, Alibaba Group\\
%\tt\small \{hooks.hl, zimu.gx, futian.zp, xisheng.sk, zhangbang.zb, liefeng.bo\}@alibaba-inc.com\\
\small \url{https://humanaigc.github.io/animate-anyone-2/}
}


\begin{document}

% \begin{abstract}

% Recent works to jointly reconstruct 3D human and object from a single RGB image, are mostly model-based, that fail to capture the fine details of the clothed human body and object surface. In this paper, we introduce ReCHOR, a novel, model-free, first-method to produce realistic clothed human-object reconstructions from a monocular view. This is extremely challenging due to human-object occlusions, diverse interactions and depth ambiguity, as it needs to infer both 3D spatial awareness and high resolution details. Our core idea is based on estimating neural implicit representations for human and object respectively by an attention-based neural implicit model that attends to pixel-aligned features from both the global human-object image for spatial awareness and  the local separate view of human and object images for high quality details. Additionally, the network is conditioned on semantic features from an initial estimated human-object pose prior and a generative diffusion model that inpaints occluded regions, thus enabling the retrieval of details from them.
% We also propose a synthetic dataset with rendered scenes of diverse, inter-occluded 3D human and object scans, to train our network. We evaluate our method on the synthetic and real world BEHAVE dataset. Our experiments show that our method outperforms the SOTA in achieving realistic clothed human-object reconstructions.
Recent approaches to jointly reconstruct 3D humans and objects from a single RGB image represent 3D shapes with template-based or coarse models, which fail to capture details of loose clothing on human bodies. In this paper, we introduce a novel implicit approach for jointly reconstructing realistic 3D clothed humans and objects from a monocular view. For the first time, we model both the human and the object with an implicit representation, allowing to capture more realistic details such as clothing. This task is extremely challenging due to human-object occlusions and the lack of 3D information in 2D images, often leading to poor detail reconstruction and depth ambiguity. To address these problems, we propose a novel attention-based neural implicit model that leverages image pixel alignment from both the input human-object image for a global understanding of the human-object scene and from local separate views of the human and object images to improve realism with, for example, clothing details. Additionally, the network is conditioned on semantic features derived from an estimated human-object pose prior, which provides 3D spatial information about the shared space of humans and objects. To handle human occlusion caused by objects, we use a generative diffusion model that inpaints the occluded regions, recovering otherwise lost details. For training and evaluation, we introduce a synthetic dataset featuring rendered scenes of inter-occluded 3D human scans and diverse objects. Extensive evaluation on both synthetic and real-world datasets demonstrates the superior quality of the proposed human-object reconstructions over competitive methods.
\end{abstract}    
% \section{Introduction}
\label{sec:intro}
% Image editing methods in diffusion models depend on user-defined control directions - users can unlock their creativity using these methods by specifying the desired manipulation through prompts~\cite{gandikota2023concept}, reference images~\cite{ruiz2022dreambooth, kumari2022customdiffusion, gal2022image, chen2024trainingfreeregionalpromptingdiffusion}, or attribute vectors~\cite{parmar2023zero,hertz2022prompt}. In this work, we ask a fundamentally different question: \emph{Can we automatically discover the underlying visual structure of a concept within diffusion model's knowledge?} %Rather than requiring user-specified controls, we aim to decompose the model's internal knowledge into meaningful directions.

% This question touches on a fundamental limitation in how we interact with diffusion models. Current control methods ~\cite{zhang2023addingconditionalcontroltexttoimage, gandikota2023concept, ye2023ipadaptertextcompatibleimage,ye2023ipadaptertextcompatibleimage, hertz2024stylealignedimagegeneration, li2023photomaker, shi2024instantbooth, chen2024trainingfreeregionalpromptingdiffusion} require users to specify their desired manipulations in advance, limiting interactive creativity. This contrasts with natural human artistic workflows, where creators dynamically explore creative ideas while jointly refining them toward meaningful artistic outcomes~\cite{hoffmann2016modeling}. This synergy between specification and exploration is not new to generative models. Early GAN architectures naturally developed disentangled latent spaces that enabled continuous\cite{harkonen2020ganspace,radford2015unsupervised, wu2021stylespace, shen2020interfacegan}, compositional control over generated images. Users could explore these spaces to discover interesting variations that would be difficult to describe in words~\cite{wu2021stylespace}, then combine them to achieve their creative goals~\cite{grabe2022towards}. 


% While diffusion models have largely superseded GANs in conditional image synthesis~\cite{dhariwal2021diffusion},  their underlying structure remains less understood. Diffusion models achieve remarkable diversity through high-dimensional latents, unlike GANs' compact latent spaces.  With a single prompt, diffusion models can generate radically different variations through different random initializations of input noise. We ask - Is it possible to discover interpretable structure within this vast space of variations?

Text-to-image diffusion models are capable of generating remarkable visual variations from a single prompt through different random initializations. However, this vast creative potential remains largely opaque to users---while we can generate diverse images, we lack understanding of the underlying structure of these variations. This presents a fundamental challenge: how can we discover and expose the latent visual capabilities encoded within these models?

\let\thefootnote\relax \footnote{$^{*}$Correspondence to \texttt{gandikota.ro@northeastern.edu}}

The challenge touches on a key limitation in how we interact with diffusion models today. Current control methods require users to explicitly specify their desired edits in advance through prompts~\cite{gandikota2023concept}, reference images~\cite{zhang2023addingconditionalcontroltexttoimage, chen2024trainingfreeregionalpromptingdiffusion, ruiz2022dreambooth,kumari2022customdiffusion, Ryu_lora, hu2021lora}, or attribute vectors~\cite{ye2023ipadaptertextcompatibleimage, hertz2024stylealignedimagegeneration, li2023photomaker, shi2024instantbooth,parmar2023zero,hertz2022prompt}. That contrasts sharply with natural human creative workflows, where artists dynamically explore creative ideas and jointly refine them toward meaningful artistic outcomes~\cite{hoffmann2016modeling}. The need for pre-specified controls creates a barrier between users and the full creative potential of these models.

Interestingly, earlier generative models like GANs~\cite{gans,karras2019style,brock2018large} naturally developed more interpretable internal structures. Their compact latent spaces often exhibited emergent disentanglement~\cite{harkonen2020ganspace,radford2015unsupervised, wu2021stylespace, shen2020interfacegan}, enabling continuous and compositional control over generated images. Users could explore these spaces to discover interesting variations that would be difficult to describe in words~\cite{wu2021stylespace}, then combine them to achieve their creative goals~\cite{grabe2022towards}.

Diffusion models have largely superseded GANs in conditional image synthesis~\cite{dhariwal2021diffusion}, achieving greater diversity through much higher-dimensional latents. And yet an understanding of the underlying structure of these larger latent spaces has remained elusive. In this work, we ask a fundamental question: \emph{Can we automatically discover the visual structure within a diffusion model's knowledge of a concept?} Rather than requiring user-specified controls, we aim to decompose the model's internal representations into expressive directions that users can explore and combine.

To address these needs, we present \textbf{SliderSpace}, a framework that brings systematic explorability to diffusion models. Given just a text prompt, SliderSpace discovers a canonical set of meaningful, diverse, and controllable directions within the model's knowledge of that concept. Each direction is implemented as a low-rank adapter~\cite{hu2021lora} that can be scaled and composed with others, allowing users to explore and smoothly combine different aspects of variation, as shown in Figure~\ref{fig:intro}.

We ground SliderSpace discovery in three key requirements for meaningful decomposition of a diffusion model's visual manifold: 
\begin{enumerate}
    \item \textbf{Unsupervised Discovery:} The decomposition process should emerge from the intrinsic structure of the model's learned representation, rather than being guided by predefined attributes. This ensures we capture the true topology of the model's knowledge space rather than projecting our assumptions onto it.
    
    \item \textbf{Semantic Orthogonality:} Each discovered control must represent a distinct semantic direction. This is enforced in a semantic feature space, like CLIP, where every slider has an orthogonal effect in embeddings. This prevents discovering multiple controls that create similar semantic effects, making the system more efficient and easier.
    
    \item \textbf{Distribution Consistency:} Directions must induce consistent transformations across both random seeds and prompt variations. 
\end{enumerate}

These requirements naturally lead to our proposed framework, which we formalize in Section~\ref{sec:method}. As we show in our experiments, SliderSpace is architecture-agnostic, working with both conventional U-Net based models like Stable Diffusion~\cite{rombach2022high, rombach2022sd20, podell2023sdxl, turbo, dmd} and recent transformer-based architectures like Flux~\cite{flux}.

We demonstrate the expressiveness of SliderSpace through three applications: First, we show how SliderSpace can decompose high-level concepts into diverse and expressive components, revealing the natural axes of variation in the model's understanding. Second, we explore artistic style variation, where SliderSpace discovers directions that match or exceed the diversity of manually curated artist lists while being judged more useful by human evaluators. Finally, we show how SliderSpace can help reverse the mode collapse commonly observed in distilled diffusion models, restoring diversity while maintaining generation speed.

Beyond providing practical creative control, SliderSpace opens new avenues for understanding and utilizing the latent capabilities of diffusion models. By mapping these models' visual potential into intuitive, composable directions, we take a step toward making their creative possibilities more accessible and interpretable to users.

% Image editing methods in diffusion models unlock the creativity of users. In this work we ask an alternate question: \emph{Can we organize and expose what of the diffusion model is already capable of?}.
% Existing methods for controlling image generation typically require users to manually specify edit directions for desired changes. This process is time-consuming, requires technical expertise, and limits the spontaneity of the creative process. For instance, if a user wants to adjust the smile of a generated person, they must explicitly request this edit, often through imprecise prompt engineering or model fine-tuning. This approach of predefined controls or manual specifications restricts users from fully exploring the latent capabilities of the model. There may be interesting stylistic variations or attributes that the model can generate, but users have no easy way to discover or utilize these.

% Natural visual disentanglement was an emergent property in the latent space of Generative Adversarial Models (GANs) \cite{harkonen2020ganspace,radford2015unsupervised, wu2021stylespace, shen2020interfacegan}. In particular, it has been observed that StyleGAN~\cite{karras2019style} stylespace neurons offer detailed control over many meaningful aspects of images that would be difficult to describe in words~\cite{wu2021stylespace}. However, diffusion models do not share such a compact latent space~\cite{park2023unsupervised}; and efforts to uncover such a space in the semantic embeddings of the text conditioning have met with limited success \nik{Nick - is there a specific citation you were thinking about?}.

% In this work we introduce \textbf{SliderSpace}, which takes a step towards uncovering an analogous low dimensional representation of diffusion models' visual breadth; in essence treating the diffusion model as many generators sharing parameters, where a particular generator is defined by a specific prompt. For a given prompt we sample many random seeds (and optionally prompt expansions using an LLM), generate the corresponding images, and apply an off the shelf feature extractor (in this work CLIP, but our method can be applied to any differentiable feature extractor). We use PCA to analyze these features, and for each of the leading $k$ principal components we train a LoRA \cite{} which causes the diffusion model to produces images which increase the feature magnitude along that component when passed back through the same feature extractor. This leads to a 'Slider' for each principal component, because each LoRA can be scaled and applied to the original diffusion model, continuously varying those visual features in the generated results (as measured, in our case, by CLIP).

% There are many other works that enhance the controllability of diffusion models. One common approach is enabling users to add spatial constraints to a generation either manually, or via a reference image \cite{zhang2023addingconditionalcontroltexttoimage, chen2024trainingfreeregionalpromptingdiffusion}, a second is leveraging more abstract embeddings (e.g. identity, style) extracted from a reference image \cite{ye2023ipadaptertextcompatibleimage, hertz2024stylealignedimagegeneration, li2023photomaker, shi2024instantbooth}, a third is finetuning a foundation model to better generate a concept important to the user \cite{ruiz2022dreambooth, kumari2022customdiffusion, Ryu_lora, hu2021lora}, and a fourth (most relevant to this work) is finding low-rank adaptors of the model based on a prompt or small training set which can be scaled to provide continous control over one aspect of generated image (e.g. night vs day, basic vs luxury, etc.) \cite{gandikota2023concept}. SliderSpace is complementary to all of these methods and offers something distinct. All of the other methods we are aware require the user (and / or model designer) to know in advance what type of control they want. In contrast SliderSpace assists users in discovering and controlling hidden capabilities present in the diffusion model's distribution of possible generations.

%We propose that truly intuitive creative control in a text-to-image model should meet three key criteria: \emph{discoverability}, \emph{intuitiveness}, and \emph{specificity}. The model should reveal controllable attributes that may not be immediately obvious, offer controls that are easy to understand and manipulate, and ensure each control affects a distinct attribute of the generated image.

% We demonstrate the utility and power of SliderSpace using three applications built on top of SDXL-DMD \cite{dmd}, because its fast generation speed lends itself well to the continuous control offered by SliderSpace.

% First, we study concept decomposition (Section \ref{sec:concept_exp}), where we learn sliders for a specific concept (e.g. 'monster', 'waterfall', 'car'). Through quantitative metrics of diversity and text alignment we demonstrate that the learned sliders dramatically boost the diversity of generations when randomly applied without harming text alignment; we also ask humans to qualitatively judge these results in a user study where they find the SliderSpace results to be more 'Diverse', 'Useful', and 'Creative' than our baselines.

% Second, we attempt to compare the automatic discoveries of SliderSpace to a large scale manual study of artistic styles (Section \ref{sec:art_exp}), open-sourced by ParrotZone \cite{parrotzone}. In this study SDXL was prompted with over 4300 artist names,  and based on visual inspection the cases of successful stylistic mimicry recorded. Quantitatively SliderSpace more closely matches the distribution of artistic variation discovered by ParrotZone than other baselines, and in our user studies was judged to be significantly more 'Diverse' and 'Useful' than the baselines. To our surprise humans even judged SliderSpace results to be slightly more 'Diverse' than the results generated by the manually discovered artist names of \cite{parrotzone}.

% Third, we attempt to use SliderSpace to reverse the mode collapse commonly observed in distilled few-step diffusion models relative to the original teacher model (Section \ref{sec:diverse_exp}). We quantitatively demonstrate that applying SliderSpace to SDXL-DMD leads to more closely matching the distribution of images by the original teacher, SDXL.

%Through extensive experiments on various state-of-the-art text-to-image models, we demonstrate that SliderSpace significantly enhances user control and creative expression in AI-assisted image generation tasks. Our method enables a range of applications, including concept decomposition and control, diversity improvement in generated images, customization dissection and edits, and the exploration of artistic styles inherent in the model.

% SliderSpace goes beyond providing a practical tool for enhanced creative control. By mapping the visual potential of diffusion models it can open new avenues for generative creativity and deepens our understanding of each model's hidden potential.
% \section{Related work}
\label{sec:formatting}

\subsection{Text-to-Video Generation}

T2V generation has made notable progress, evolving from early GAN-based models \cite{saito2017temporal,tulyakov2018mocogan,fu2023tell,li2018video,wu2022nuwa,yu2022generating} to newer transformer \cite{yan2021videogpt,arnab2021vivit,esser2021taming,ramesh2021zero,yu2022scaling} and diffusion models \cite{kirkpatrick2017overcoming,sohl2015deep,song2020denoising,zhang2022gddim}. Early efforts like MoCoGAN~\cite{tulyakov2018mocogan} focused on short video clips but faced issues with stability and coherence. The introduction of transformers improved sequential data handling, enhancing video generation, while diffusion models further improved video quality by progressively denoising the input. 
Despite these advances, T2V models still struggle to reflect human preferences, with the generated videos generally lacking aesthetic quality. Additionally, the scarcity of paired video preference data hinders effective model training and may lead to insufficient flexibility and poor quality in the generated videos.


\subsection{RLHF}

\iffalse
Aligning LLMs \cite{dai1901transformer,radford2019language,zhang2023opt} typically involves two steps: supervised fine-tuning followed by Reinforcement Learning with Human Feedback (RLHF) \cite{gao2023scaling,stiennon2020learning,rafailov2024direct}. Although effective, RLHF is computationally expensive and can lead to issues like reward hacking. Methods like DPO have streamlined alignment by leveraging feedback data directly, improving efficiency.

In contrast, diffusion model alignment is still evolving, focusing mainly on enhancing visual quality through curated datasets. Techniques like DOODL \cite{wallace2023end} and AlignProp \cite{prabhudesai2023aligning} target aesthetic improvements but face challenges with complex tasks such as text-image alignment. Reinforcement learning methods like DPOK \cite{fan2024reinforcement} and DDPO \cite{black2023training} improve reward optimization but struggle with scalability. DPO-SDXL integrates DPO into T2I generation, boosting both alignment and aesthetics.

However, aligning video generation remains a largely unaddressed challenge, especially when dealing with motion consistency and semantic coherence across frames.
\fi

RLHF \cite{gao2023scaling,stiennon2020learning,rafailov2024direct} is a method that utilizes human feedback to guide machine learning models. Early RLHF algorithms, such as DDPG~\cite{lillicrap2015continuous} and PPO~\cite{schulman2017proximal}, typically relied on complex reward models to quantify human feedback. These reward models require a large amount of annotated data and face challenges during tuning. As research has progressed, more efficient preference learning methods have emerged, among which DPO has become a new framework. DPO does not depend on a separate reward model; instead, it obtains human preferences through pairwise comparisons and directly optimizes these preferences. This shift not only simplifies the application of RLHF but also enhances the alignment of models with human values. Furthermore, DPO has been successfully introduced into T2I tasks~\cite{wallace2024diffusion,yang2024using}, providing new insights for generative models in addressing the alignment of human preferences and showcasing DPO's potential in the field of AIGC~\cite{shi2024instantbooth,
qing2024hierarchical,menapace2024snap,koley2024s}. However, there remains a gap in current research regarding the application of DPO strategies to T2V tasks. Effectively integrating DPO into T2V tasks presents a challenging endeavor.



% \section{Preliminary}
\label{sec:preliminary}
In this section, we first introduce the mathematical formulation of flow-based text-to-image generative models~\cite{Xingchao_2022,Lipman_2022}, which forms the foundation of modern T2I systems~\cite{sd3,sdxl,imagen3,imagen}. We then describe classifier-free guidance~\cite{ho2022classifier}, a key technique to control the generation process through text conditioning.

\subsection{Flow-based text-to-image generative models}
In state-of-the-art T2I models~\cite{sd3}, the image generation process is modeled by learning, through a neural network, a flow $\psi$ that generates a probability path $(p_t)_{0\le t\le 1}$ bridging the source distribution $p_0$ and the target distribution $p_1$ ~\cite{Xingchao_2022,Lipman_2022}. This framework encompasses diffusion models~\cite{sohl2015deep,ddpm} as a special case. In particular, a commonly used formulation sets a Gaussian distribution as the source: $p_0 = \mathcal{N}(\mathbf{0}, \mathbf{I})$ and a delta distribution centered on a sample $\mathbf{x}_1$ from the data distribution $q$ as the target: $p_1 = \delta_{\mathbf{x}_1}$.
Then, it incorporates an affine conditional flow $\psi_t(\mathbf{x} | \mathbf{x}_1) = a_t \mathbf{x}_1 + b_t \mathbf{x}$ with the boundary condition $(a_0, b_0) = (0, 1),\ (a_1, b_1) = (1, 0)$ to bridge them. The neural network typically approximates quantities such as velocity fields, $x_0$ prediction or $x_1$ prediction. In this modeling, these quantities can be viewed as affine transformations of the marginal probability path score $\nabla_{\mathbf{x}} \log p_t(\mathbf{x})$.

\subsection{Classifier-free guidance in flow-based models}
Classifier-free guidance~\cite{ho2022classifier} is a method for sampling from a model conditioned by a text input $\mathbf{y}$ by guiding an unconditional image generation model modeled using the score $\nabla_{\mathbf{x}} \log p_t(\mathbf{x})$. This enables the sampling from
\[
q_w(\mathbf{x}, \mathbf{y}) \propto q(\mathbf{x})q(\mathbf{y}|\mathbf{x})^w \propto q(\mathbf{x})^{1-w}q(\mathbf{x}|\mathbf{y})^w
\]
where $w \in \mathbb{R}$ is the guidance scale typically used with $w > 1$. The score satisfies
\[
\nabla_{\mathbf{x}} \log q_w(\mathbf{x}, \mathbf{y}) = (1-w)\nabla_{\mathbf{x}} \log q(\mathbf{x}) + w\nabla_{\mathbf{x}} \log q(\mathbf{x}|\mathbf{y})
\]
so by training the network to learn both the unconditional score $\nabla_{\mathbf{x}} \log q(\mathbf{x})$ and conditional score $\nabla_{\mathbf{x}} \log q(\mathbf{x}|\mathbf{y})$, flexible sampling from the conditional distribution can be achieved through a weighted sum of the network outputs.


\twocolumn[{%
\renewcommand\twocolumn[1][]{#1}%
\maketitle
\begin{center}
    \centering
    \captionsetup{type=figure}
    \includegraphics[width=1.0\textwidth]{./figure/fig1.pdf}
    \captionof{figure}{We propose \textit{Animate Anyone 2}, which differs from previous character image animation methods that solely utilize motion signals to animate characters. Our approach additionally extracts environmental representations from the driving video, thereby enabling character animation to exhibit environment affordance. The generated results demonstrate that, beyond maintaining character consistency, \textit{Animate Anyone 2} can produce high-fidelity results that seamlessly integrate characters with the surrounding environment.}
    \label{fig:f1}
\end{center}%
}]

\maketitle

\begin{abstract}
Recent character image animation methods based on diffusion models, such as Animate Anyone, have made significant progress in generating consistent and generalizable character animations. However, these approaches fail to produce reasonable associations between characters and their environments. To address this limitation, we introduce Animate Anyone 2, aiming to animate characters with environment affordance. Beyond extracting motion signals from source video, we additionally capture environmental representations as conditional inputs. The environment is formulated as the region with the exclusion of characters and our model generates characters to populate these regions while maintaining coherence with the environmental context. We propose a shape-agnostic mask strategy that more effectively characterizes the relationship between character and environment. Furthermore, to enhance the fidelity of object interactions, we leverage an object guider to extract features of interacting objects and employ spatial blending for feature injection. We also introduce a pose modulation strategy that enables the model to handle more diverse motion patterns. Experimental results demonstrate the superior performance of the proposed method.
\end{abstract}


\renewcommand{\thefootnote}{}
\footnotetext{$^*$Equal contribution}


\section{Introduction}

The objective of character image animation is to synthesize animated video sequences utilizing a reference character image and a sequence of motion signals. 
Recent developments predominantly adopt diffusion-based frameworks~\cite{dreampose,disco,magicanimate,aa,magicpose,champ,unianimate,mimicmotion}, achieving notable enhancements in appearance consistency, motion stability and character generalizability. 
These advancements exhibit substantial potential in areas such as filmmaking, advertising, and virtual character applications.


In recent cross-identity animation workflows, motion signals are typically extracted from disparate videos, while the character's contextual environments are derived from static images. This setting introduces critical limitations: the spatial relationships between animated characters and their environments often lack authenticity, and intrinsic human-object interactions are disrupted. Consequently, most existing methods are predominantly limited to animating simple actions (e.g., individual gestures or dances) without adequately capturing the complex spatial and interactive relationships between characters and their surroundings. These limitations significantly hinder the advancement of character animation techniques.

Recent attempts to integrate character animation with scenes and objects, while promising, face significant challenges in generation quality and adaptability. 
For instance, MovieCharacter\cite{moviecharacter} synthesizes character videos by cascading the outputs from multiple algorithms, which introduces noticeable artifacts and unnatural visual discontinuities. 
AnchorCrafter\cite{anchorcrafter} primarily focuses on human-object manipulation animation, with relatively simplistic character motion and object appearance. 
MIMO\cite{mimo} addresses this challenge by composing characters, pre-processed backgrounds and occlusions, which are disentangled via depth. Such formulation for defining the relationship between characters and environments is suboptimal, 
limiting the ability to handle complex interactions.


%To overcome these limitations, 
In this paper, 
we propose to expand the scope of character animation by introducing \textit{Character Image Animation with Environment Affordance}.
Specifically, we define the research problem as follows: given a character image and a source video, the generated character animation should: 1) inherit character motion desired by the source video. 2) accurately demonstrate character-environment relationship consistent with the source video. 
This setting introduces novel challenges for character animation, as it requires that the model should effectively handle diverse and complex character motions, while ensuring precise interaction between characters and their environments throughout the animation process.

To achieve this, we introduce a novel framework \textit{Animate Anyone 2}. 
As illustrated in Fig.\ref{fig:f1}, unlike previous character animation methods that solely utilize motion signals, we additionally capture environmental representations from the source video as conditional inputs, which enables the model to learn the intrinsic relationship between character and environment in an end-to-end manner.
We formulate the environment by removing the character regions and our model generates characters to populate these regions while maintaining coherence with the environmental context. We develop a shape-agnostic mask strategy that better represents the boundary relationship between character and their contextual scenes, enabling effective learning for character-context integration while mitigating shape leakage issues. 
Second, to enhance the fidelity of object interactions, we introduce additional processing for interactive object regions. We design a lightweight object guider to extract interactive object features and propose a spatial blending mechanism to inject these features into the generation process. It facilitates the preservation of intricate interaction dynamics in the source video. 
Lastly, we propose depth-wise pose modulation approach for character motion modeling, empowering the model to handle more diverse and complex character poses with enhanced robustness.



The results in Fig.\ref{fig:f1} exhibit both high-quality character animation performance and remarkable environment affordance, manifested through three key advantages: 1) seamless scene integration; 2) coherent object interaction; and 3) robust handling of diverse and complex motions. Our approach is evaluated on corresponding benchmarks, achieving superior character animation results compared to existing methods. 
In summary, we highlight three key contributions of our paper.
\begin{itemize}
\item We introduce \textit{Animate Anyone 2}, a framework capable of animating character with environment affordance, achieving robust performance.
\item We propose a novel environment formulation and object injection strategy to achieve seamless character-environment integration.
\item We propose pose modulation strategy to enhance model robustness in challenging action scenarios.
\end{itemize}




\begin{figure*}[!t]
\begin{center}
	\setlength{\fboxrule}{0pt}
	\fbox{\includegraphics[width=0.99\textwidth]{./figure/fig2.pdf}}
\end{center}
\vspace{-0.6cm}
\caption{The framework of \textit{Animate Anyone 2}. We capture environmental information from the source video. The environment is formulated as regions devoid of characters and incorporated as model input, enabling end-to-end learning of character-environment fusion. To preserve object interactions, we additionally inject features of objects interacting with the character. These object features are extracted by a lightweight object guider and merged into the denoising process via spatial blending. To handle more diverse motions, we propose a pose modulation approach to better represent the spatial relationships between body limbs. 
}
\vspace{-0.2cm}
\label{fig:overview}
\end{figure*}



\section{Related Works}

\subsection{Character Image Animation}
Distinguished from GAN-based\cite{gan,wgan,stylegan} approaches\cite{fomm,mraa,ren2020deep,tpsmm,siarohin2019animating,zhang2022exploring,bidirectionally,everybody}, diffusion-based image animation methods\cite{dreampose,disco,aa,magicanimate,magicpose,mimicmotion,champ,unianimate,tcan,tpc} have emerged as the current research mainstream. As the most representative approach, Animate Anyone\cite{aa} designs its framework based on Stable Diffusion\cite{ldm}, and the denoising network is structured as a 3D UNet\cite{align,animatediff} for temporal modeling. It proposes ReferenceNet, a symmetric UNet\cite{unet} architecture, to preserve appearance consistency and employs pose guider to incorporate skeleton information as driving signals for stable motion control. The Animate Anyone framework achieves robust and generalizable character animation, from which we extensively drew inspiration.

Some works propose improvements upon foundational frameworks. MimicMotion\cite{mimicmotion} leverages pretrained image-to-video capabilities of Stable Video Diffusion\cite{svd}, designing a PoseNet to inject skeleton information. UniAnimate\cite{unianimate} stacks reference images across temporal dimensions, utilizing mamba-based\cite{mamba} temporal modeling techniques. Some works explore different motion control signals. DisCo\cite{disco} and MagicAnimate\cite{magicanimate} utilizes DensePose\cite{densepose} as human body representations. Champ\cite{champ} employs the 3D parametric human model SMPL\cite{smpl}, integrating multi-modal information including depth, normal, and semantic signals derived from SMPL.





%\subsection{Affordance-Aware Image/Video Generation}
%\subsection{Scene/Object-conditioned Generation}
\subsection{Human-environment Affordance Generation}
Numerous studies leverage diffusion models to generate human image or video that contextually integrate with scenes or interactive objects. Some studies\cite{putting,environment-specific,addme,text2place,invi} investigate inserting or inpainting human into given scenes to achieve scene affordance. \cite{putting} applies video self-supervised training to inpaint person into masked region with correct affordances. Text2Place\cite{text2place} aims to place a person in background scenes by learning semantic masks using text guidance for localizing regions. InVi\cite{invi} achieves object insertion by first conducting image inpainting and subsequently generating frames using extended-attention mechanisms.

Several works focus on character animation with scene or object interactions. MovieCharacter\cite{moviecharacter} composites the animated character results into person-removed video sequence. AnchorCrafter\cite{anchorcrafter}, focusing on human-object interaction, first perceives HOI-appearances and injects HOI-motion to generate anchor-style product promotion videos. MIMO\cite{mimo} introduces spatial decomposed diffusion, decomposing videos into human, background and occlusion based on 3D depth and subsequently composing these elements to generate character video.







\section{Method}


In this section, we introduce \textit{Animate Anyone 2}. In \ref{sec:framework}, we first elaborate on the overall framework. In \ref{sec:scene}, we delineate the strategy for environment formulation. In \ref{sec:object}, we present the design of object injection. In \ref{sec:pose}, we provide a detailed exposition of pose modulation strategy.


\subsection{Framework}\label{sec:framework}

\noindent
\textbf{System Setting. }
The overall framework is illustrated in Fig.\ref{fig:overview}.
During training, we employ a self-supervised learning strategy. Given a reference video ${\mathit I}^{1:\mathit N}$ where $\mathit N$ denotes the number of frames, we disentangle character and environment via a formulated mask (detailed in \ref{sec:scene}), obtaining separate character sequence ${\mathit I}^{1:\mathit N}_{\mathit c}$ and environment sequence ${\mathit I}^{1:\mathit N}_{\mathit e}$. To facilitate more fidelity object interaction, we additionally extracted the sequence of objects ${\mathit I}^{1:\mathit N}_{\mathit o}$. 
%We also extract motion sequence ${\mathit I}^{1:\mathit N}_{\mathit m}$ from the character as driving signals.
Motion sequence ${\mathit I}^{1:\mathit N}_{\mathit m}$ is extracted as driving signals.
We randomly sample a character image ${\mathit I}_{\mathit c}$ from ${\mathit I}^{1:\mathit N}_{\mathit c}$ with center crop and composite it onto a random background. Given image ${\mathit I}_{\mathit c}$, motion sequence ${\mathit I}^{1:\mathit N}_{\mathit m}$, environment sequence ${\mathit I}^{1:\mathit N}_{\mathit e}$ and object sequence ${\mathit I}^{1:\mathit N}_{\mathit o}$ as inputs, our model reconstructs the reference video ${\mathit I}^{1:\mathit N}$. 
During inference, given a target character image and a driving video, our method can animate the character with consistent actions and environmental relationship corresponding to the driving video.


\noindent
\textbf{Diffusion Model. }
%Our approach builds upon Animate Anyone, a robust diffusion-based character image animation framework.
Our method is developed based on LDM\cite{ldm}. It employs a pretrained VAE\cite{vae,vqvae} to transform images from pixel space to latent space: $\mathbf z \mathcal = \mathcal E$($\mathbf x$). During training, random Gaussian noise $\epsilon$ is progressively added to image latents ${\mathbf z}_{t}$ at different timesteps,
The training objective can be formulated as follows:

%\vspace{-0.1cm}
\begin{equation}
\label{eq1}
    {\mathbf L} = {\mathbb E}_{{\mathbf z}_{t},c,{\epsilon},t}({||{\epsilon}-{{\epsilon}_{\theta}}({\mathbf z}_{t},c,t)||}^{2}_{2})
\end{equation}
%\vspace{-0.1cm}

\noindent
where ${\epsilon}_{\theta}$ represents the function of DenoisingNet. $\mathnormal c$ represents conditional inputs. During inference, noise latents are iteratively denoised\cite{denoising,ddim} and reconstructed into images through the decoder of VAE: ${\mathbf x}_{recon} \mathcal = \mathcal D$($\mathbf z$). 
The network design of DenoisingNet is derived from Stable Diffusion\cite{ldm}, inheriting its pretrained weights. We extend the original 2D UNet architecture to 3D UNet, incorporating the temporal layer design from AnimateDiff\cite{animatediff}.

\noindent
\textbf{Conditional Generation. }
We adopt the ReferenceNet architecture from \cite{aa} to extract appearance features of the character image ${\mathit I}_{\mathit c}$. 
%Specifically, ReferenceNet employs an identical model structure to the 2D denoising UNet, and the extracted appearance features are injected into the corresponding layers of the denoising UNet via spatial attention\cite{attention}. This approach significantly enhances the consistency of character appearance details. 
In our framework, we simplify the computational complexity by merging these features exclusively in the midblock and upblock of the DenoisingNet decoder via spatial attention\cite{attention}.
Besides, three conditional embeddings are extracted from the souce video: environment sequence ${\mathit I}^{1:\mathit N}_{\mathit e}$, motion sequence ${\mathit I}^{1:\mathit N}_{\mathit m}$, and object sequence ${\mathit I}^{1:\mathit N}_{\mathit o}$. For environment sequence ${\mathit I}^{1:\mathit N}_{\mathit e}$, we employ VAE encoder to encode the embedding and subsequently merge it with noise latents. For motion sequence ${\mathit I}^{1:\mathit N}_{\mathit m}$, we design pose modulation strategy (elaborated in \ref{sec:pose}) and the motion information is also merged into the noise latents. For object sequence ${\mathit I}^{1:\mathit N}_{\mathit o}$, after encoding via VAE encoder, we develop an object guider to extract multi-scale features and inject them into the DenoisingNet through spatial blending, which will be detailed in \ref{sec:object}.






\subsection{Environment Formulation}\label{sec:scene}

\noindent
\textbf{Motivation. }
In our framework, the environment is formulated as a region excluding characters. During training, the model generates characters to populate these regions while maintaining coherence with the environmental context. 
The boundary relationship between characters and the environment is crucial. Appropriate boundary guidance can facilitate the model in learning character-environment integration more effectively, while preserving character shape consistency and environmental information integrity.
%To achieve seamless integration of diverse characters and scenes, it is crucial to dynamically formulate the boundary relationships between characters and their contextual scenes.
Some studies\cite{putting,invi} leverage bounding boxes to represent generative regions. 
However, we observe artifacts or inconsistencies with the source video when dealing with complex scenes, due to insufficient conditioning.
%However, we experimentally observe that when dealing with complex scenes, this strategy often suffers from artifacts or inconsistencies with the source video due to insufficient conditioning.
Conversely, directly using precise masks is also suboptimal, potentially introducing shape leakage.
Due to the self-supervised training strategy, there exists strong correlation between character outlines and mask boundaries. Consequently, the model tends to use this information as additional guidance for animating character. However, during inference, when the target character differs from the source in body shape and clothing, the model may forcibly conform to the mask boundary, resulting in integration artifacts. 


\begin{figure}[!t]
\begin{center}
    \vspace{-0.3cm}
	\setlength{\fboxrule}{0pt}
	\fbox{\includegraphics[width=1\linewidth]{./figure/fig3.pdf}}
\end{center}
\vspace{-0.6cm}
\caption{Different coefficients for mask formulation.}
\vspace{-0.3cm}
\label{fig:mask}
\end{figure}


\noindent
\textbf{Shape-agnostic Mask. }
Therefore, we propose a shape-agnostic mask strategy for environment formulation, with the core idea of disrupting the correspondence between mask region and character outline during training. Specifically, for a character mask ${\mathit M}_{\mathit c}$ in its bounding box of size ${\mathit h} \times {\mathit w}$, we define two coefficients ${\mathit k}_{\mathit h}$ and ${\mathit k}_{\mathit w}$. 
% We randomly partition the character mask ${\mathit M}_{\mathit c}$ into ${\mathit k}_{\mathit h} \times {\mathit k}_{\mathit w}$ segments, where $1 < {\mathit k}_{\mathit h} < {\mathit h}$ and $1 < {\mathit k}_{\mathit w} < {\mathit w}$. 
We divided the character mask ${\mathit M}_{\mathit c}$ into ${\mathit k}_{\mathit h} \times {\mathit k}_{\mathit w}$ non-overlapping blocks, where ${\mathit k}_{\mathit h} \in (1,{\mathit h}), {\mathit k}_{\mathit w} \in (1,{\mathit w})$. 
We denote ${\mathit P}^{\mathit (k)}_{\mathit c}$ as the divided patches, where ${\mathit k}$ is the index. 
We reformulate the mask ${\mathit M}_{\mathit c}$ into a new mask ${\mathit M}_{\mathit f}$ by propagating the patch-wise maximum value:

    % {\mathit M}_{\mathit f}(i, j) = \max {\mathit P}^{\mathit (k)}_{\mathit c}(i, j)
    
    % {\mathit M}_{\mathit f}(i, j) = \max_{i'=i}^{i'+k_h-1} \max_{j'=j}^{j'+k_w-1} {\mathit P}_{\mathit c}(i', j')
%\vspace{-0.1cm}
\begin{equation}
\label{eq1}
{\mathit M}_{\mathit f}(i, j) = \max_{(i,j) \in  {\mathit P}^{\mathit (k)}_{\mathit c} } {\mathit P}^{\mathit (k)}_{\mathit c}(i, j)
\end{equation}
%\vspace{-0.1cm}


% We divided the character mask ${\mathit M}_{\mathit c}$ into non-overlapping blocks of size ${\mathit k}_{\mathit h} \times {\mathit k}_{\mathit w}$, where ${\mathit k}_{\mathit h} \in (1,{\mathit h}), {\mathit k}_{\mathit w} \in (1,{\mathit w})$. 
% The transformed mask ${\mathit M}_{\mathit f}$ can be difined as :
% \begin{equation}
% \label{eq1}
% {\mathit M}_{\mathit f}(i, j) = \begin{cases} 
% 1 & \text{if} \max_{i'=i}^{i'+k_h-1} \max_{j'=j}^{j'+k_w-1} {\mathit P}(i', j')=1 \\
% 0 & \text{else}
% \end{cases}
% \end{equation}
% \vspace{-0.2cm}
% where ${\mathit P}(i', j')$ represents the pixel value of original character mask.

\noindent
where ${\mathit P}^{\mathit (k)}_{\mathit c}(i, j)$ represents the value at position ${\mathit (i,j)}$.
The visualized process is presented in Fig.\ref{fig:mask}. By employing this strategy, the formulated mask dynamically generate different shapes that deviate from the character boundaries, thereby compelling the network to learn context integration more effectively, unencumbered by predefined boundary constraints. During inference, we set ${\mathit k}_{\mathit h} = {\mathit h} / 10$ and ${\mathit k}_{\mathit w} = {\mathit w} / 10$. 


\noindent
\textbf{Random Scale Augmentation. }
Moreover, since the formulated mask is inherently larger than the original mask, this introduces an inevitable bias that constrains the generated character to be necessarily smaller than the given mask. To mitigate this bias, we employ random scale augmentation on source videos. Specifically, we extract the character together with the interacting objects based on their masks and apply a random scaling operation. Subsequently, we recompose these scaled content back into the source video. This approach ensures that the formulated mask has a probabilistic chance of being smaller than the actual character region. During inference, the model is capable of animating the character flexibly without being constrained by the size of the mask.
%Experimental results demonstrate that this strategy not only enhances the fidelity of generated characters but also facilitates seamless character-scene integration without introducing significant scene context distortion.


\subsection{Object Injection}\label{sec:object}

\noindent
\textbf{Object Guider. }
%In this section, we focus on human-object interactions. 
%Due to the scene formulation strategy, object regions might be incomplete. 
The environment formulation strategy may potentially lead to distortion of object regions.
To enhance the preservation of object interactions, we propose to inject additional object-level features. 
Interactive objects can be extracted through two methods: 1) Leveraging VLM\cite{cogvlm,qwenvl} to obtain object localization; 2) Interactively confirming object positions via manual annotation. Then we employ SAM2\cite{sam,sam2} to extract object mask, obtaining corresponding object image and encode it into object latents via VAE encoder. A naive approach to merging object features is to directly concatenate scene and object features before feeding them into the network. However, due to the intricate relationship between characters and objects, such method struggles to handle complex human-object interactions, often falling short in capturing both human and object details. 
%Scenes typically exhibit only boundary-level fusion with human subjects, while objects frequently demonstrate spatially intricate overlapping contours, necessitating a more fine-grained feature fusion design.
Thus we design an object guider to extract object-level features. 
%Since the objects and their interaction relationships are inherently preserved from the original video, our approach does not require a dedicated modeling network equivalent to the main backbone, as typically employed for subject feature extraction. 
Unlike character features that require complex modeling, objects inherently preserve visual characteristics from the source video. 
Thus we implement object guider using a lightweight fully convolutional architecture. 
%The design of each convolutional block is inspired by the Pose Guider in Animate Anyone. The difference is that 
specifically, object latents are downsampled four times via $3 \times 3$ Conv2D to obtain multi-scale features. The channel dimensions of these features are aligned with those in the midblock and upblock of the DenoisingNet,  facilitating subsequent feature fusion.

\noindent
\textbf{Spatial Blending. }
To recover the spatial relationships of human-object interaction, we employ spatial blending to inject features extracted by object guider into the DenoisingNet. Specifically, during the denoising process, spatial blending layer is performed after spatial attention layer. For noise latents ${\mathit z}_{\mathit noise}$ and object latents ${\mathit z}_{\mathit object}$, we concatenate their features and compute the alpha weight ${\mathit \alpha}$ through a Conv2D-Sigmoid layer. The spatial blending process can be mathematically formulated as follows:

\begin{equation}
\label{eq1}
    {\mathit \alpha} = {\mathit F}(cat({\mathit z}_{\mathit noise},{\mathit z}_{\mathit object}))
\end{equation}
\begin{equation}
\label{eq2}
    {\mathit z}_{\mathit blend} = {\mathit \alpha} \cdot {\mathit z}_{\mathit object} + (1 - {\mathit \alpha}) \cdot {\mathit z}_{\mathit noise}
\end{equation}


\noindent
where ${\mathit F}$ denotes the Conv2D-Sigmoid layer, which is initialized through zero convolution. ${\mathit z}_{\mathit blend}$ denotes the new noise latents after spatial blending.  In each stage of the DenoisingNet decoder, we alternately apply spatial attention on character features and spatial blending of object features, enabling the generation of high-fidelity results with excellent details of character-object interactions.

\subsection{Pose Modulation}\label{sec:pose}

\noindent
\textbf{Motivation. }
%In this section, we introduce our approach to motion modeling. 
Animate Anyone\cite{aa} employs a skeleton representation to capture character motion and utilizes pose guider for feature modeling. However, the skeleton representation lacks explicit modeling of inter-limb spatial relationships and hierarchical dependencies. Some existing methods\cite{champ,mimo} adopt 3D mesh representations like SMPL to represent human bodies, but this tends to compromise the generalizability across characters and potentially introduces shape leakage due to its dense representation.

\noindent
\textbf{Depth-wise Pose Modulation. }
We propose to retain the skeleton signals while augmenting it with structured depth to enhance the representation of inter-limb spatial relationships. We refer to this approach as depth-wise pose modulation. For motion signals, we leverage Sapien\cite{sapiens} to extract the skeleton and depth information from the source video. The depth information is structurally processed via the skeleton to mitigate potential shape leakage in raw depth maps. Specifically, we first binarize the skeleton image to obtain skeleton mask, and subsequently extract the depth results within this masked region.
Then we employ Conv2D with the same architectural design as the pose guider\cite{aa} to process the skeleton map and structured depth map. Then we merge the structured depth information into the skeleton features through a cross-attention mechanism. The key insight behind this approach is to enable each limb to incorporate spatial characteristics from other limbs, thereby facilitating a more nuanced understanding of limb interaction relationships. 
Given that pose information extracted from wild videos may contain errors, we utilize Conv3D to model temporal motion information, enhancing inter-frame connections and mitigating the impact of erroneous signals on individual frames.
%Our experiments demonstrate that by implementing the proposed depth-wise pose modulation, our model can effectively handle complex actions in videos.


% \subsection{Comparison with Prior Works}
% % Recently, several works have explored generating animated human characters with scene/object interaction capabilities. 
% AnchorCraft can produce videos of characters holding objects, which requires complex attention-based computations to inject multi-view object features and additionally extract depth conditions from reference videos, yet remains limited to generating front-facing videos under identical object contexts. In contrast, our approach directly leverages object information from reference videos as conditions, employing lightweight feature fusion. Moreover, our method demonstrates superior generalizability across diverse human-object interaction scenarios.
% MIMO propose a composition-based approach for generating scene-character interactions by synthesizing characters, backgrounds and occlusions. However, its background generation relies on an auxiliary algorithm\cite{propainter}, and its direct incorporation of depth-based occlusion processing encounters challenges in complex human-object interactions. Our method offers an end-to-end generation pipeline with robust performance across more varied and intricate scenarios.



\begin{figure*}[!t]
\begin{center}
	\setlength{\fboxrule}{0pt}
	\fbox{\includegraphics[width=0.99\linewidth]{./figure/fig4.pdf}}
\end{center}
\vspace{-0.6cm}
\caption{Qualitative Results. \textit{Animate Anyone 2} achieves consistent character animation while enabling the integration and interaction between characters and their environments, thereby realizing environment affordance.}
\vspace{-0.2cm}
\label{fig:vis}
\end{figure*}



\section{Experiments}

\subsection{Implementations}

To validate the generalizability of our method across more diverse scenarios, we curated a dataset of 100,000 character videos collected from the internet, encompassing a broader range of scene types, action categories, and human-object interaction cases. Experiments are conducted on 8 NVIDIA A100 GPUs. The training involves 100k steps with batch size of 8 and the video length in a batch is 16. 
%During training, The weight of VAE Encoder and Decoder are kept fixed, while the remaining components of the network are trained in an end-to-end manner. 
Video frames are cropped at consistent positions to ensure that the character is fully contained within the 16-frame sequence. The reference image is randomly sampled from the entire video sequence. We perform center cropping and remove the original background, compositing it with a new random background. This approach enables the model to automatically recognize characters within the image during inference without requiring additional segmentation, thereby mitigating potential accuracy limitations inherent in segmentation processes.

During long video inference, the video is segmented into multiple video clips, and inference is performed on each clip sequentially. Inspired by the motion frame technique in \cite{emo}, we utilize the final frame of the previous video clip as the temporal reference to guide the transition between clips. This strategy ensures smooth transitions between different video clips, preventing appearance texture discontinuities or blurriness. 







\subsection{Qualitative Results}

Fig.~\ref{fig:vis} demonstrates that our approach not only animates diverse characters with high-fidelity performance, but also achieves remarkably seamless visual integration and interaction with their surrounding environments. This substantiates the versatility and robustness of our method, underscoring its significant potential for widespread applications.


\subsection{Comparisons}

\noindent
\textbf{Metrics. }
We follow the previous evaluation metrics for character image animation. Specifically, for single-frame quality assessment, we employ PSNR\cite{psnr}, SSIM\cite{ssim}, and LPIPS\cite{lpips}. For video fidelity, we utilize the Frechet Video Distance (FVD)\cite{fvd}. 



\begin{table}
    \setlength{\tabcolsep}{4pt}
	\centering
	\resizebox{0.45\textwidth}{!}{
        \begin{tabular}{@{}ccccc@{}}
\toprule
Method         & SSIM $\uparrow$                      & PSNR $\uparrow$                     & LPIPS $\downarrow$                    & FVD  $\downarrow$                      \\ \midrule
MRAA \cite{mraa}          & 0.672           & 29.39        & 0.672           & 284.82                     \\
DisCo \cite{disco}         &  0.668        &     29.03        & 0.292         & 292.80                     \\
MagicAnimate \cite{magicanimate}  & 0.714         & 29.16           & 0.239        & 179.07                     \\
Animate Anyone \cite{aa} & 0.718          & 29.56        & 0.285           & 171.90                      \\
Champ* \cite{champ}         & 0.802          &  29.91         & 0.234         & 160.82 \\
UniAnimate* \cite{unianimate} & 0.811           & 30.77            & 0.231        & 148.06   \\
Ours  & 0.778           & 29.82            & 0.248        & 158.97   \\
Ours*           & \textbf{0.812}            & \textbf{30.82}            & \textbf{0.223}            & \textbf{144.65}            \\ \bottomrule
\end{tabular}
    }
    \vspace{-0.2cm}
	\caption{Quantitative comparison on Tiktok benchmark. * means utilizing other video data for pretraining. }
    \vspace{-0.3cm}
	\label{table:tiktok}
\end{table}





\noindent
\textbf{Evaluation on TikTok Dataset. }
We conduct experiments on the TikTok Benchmark\cite{tiktok}. In this dataset, the video backgrounds are static.
Existing character animation approaches typically synthesize target videos with both characters and backgrounds by a single reference image. To ensure a fair comparison, we adjust the configuration of our method: instead of using the ground truth background, we employ the background from the reference image as the environmental input. This modification allows all methods to generate outputs conditioned exclusively on a single reference image.
We implement two training settings of our approach: 1) trained exclusively on the Tiktok training set, and 2) first trained on our custom dataset and subsequently fine-tuned on the Tiktok training set. As shown in Tab.~\ref{table:tiktok}, when trained solely on the Tiktok training set, our method outperforms Magicanimate\cite{magicanimate} and Animate Anyone\cite{aa}. After incorporating pre-trained video data, our approach further surpasses Champ\cite{champ} and UniAnimate\cite{unianimate}, achieving state-of-the-art performance.







\begin{figure}[!t]
\begin{center}
	\setlength{\fboxrule}{0pt}
	\fbox{\includegraphics[width=0.95\linewidth]{./figure/fig5.pdf}}
\end{center}
\vspace{-0.7cm}
\caption{Qualitative comparion for character animation. We normalize the background to a uniform color. }
\vspace{-0.01cm}
\label{fig:com1}
\end{figure}



\begin{table}
    \setlength{\tabcolsep}{4pt}
	\centering
        \resizebox{0.45\textwidth}{!}{
        \begin{tabular}{@{}ccccc@{}}
\toprule
Method         & SSIM $\uparrow$                      & PSNR $\uparrow$                     & LPIPS $\downarrow$                    & FVD  $\downarrow$                      \\ \midrule
Animate Anyone\cite{aa} & 0.761          & 28.41        & 0.324           & 228.53                      \\
Champ\cite{champ}          & 0.771          &  28.69         & 0.294         & 205.79 \\
MimicMotion\cite{mimicmotion}  & 0.767           & 28.52            & 0.307        & 212.48   \\
Ours           & \textbf{0.809}            & \textbf{29.24}            & \textbf{0.259}            & \textbf{172.54}            \\ \bottomrule
\end{tabular}
    }
    \vspace{-0.2cm}
	\caption{Quantitative comparison on our dataset. Our approach demonstrates superior performance across generalized scenarios.}
    \vspace{-0.2cm}
	\label{table:propose}
\end{table}





\noindent
\textbf{Evaluation on Proposed Dataset. }
Due to the limitations of existing benchmarks\cite{dwnet,tiktok,mraa} that exhibit domain proximity, these datasets cannot effectively evaluate the generalizability of models across diverse scenarios. Following ~\cite{champ}, we establish a testset comprising 100 character videos from real-world scenarios to conduct additional evaluation. Since other methods cannot generate dynamic environment, we standardize the background of input images to a uniform color, thus isolating the impact of environment variations on the evaluation. 
For fair comparison, we finetune these methods on our custom training dataset. 
The quantitative comparison is shown in Tab.~\ref{table:propose}. Qualitative comparison is shown in Fig.\ref{fig:com1}.
Our results significantly outperform alternative approaches, which can be attributed to two key factors: (1) our proposed motion modeling demonstrates robust generalization across diverse motion patterns, and (2) our decoupled environment and character generation strategy enables the model to focus more precisely on character dynamics, mitigating interference from environment variations.





\noindent
\textbf{Evaluation for character-environment affordance. }
%The aforementioned comparisons solely focus on character animation itself without evaluating character-environment affordance.
We further evaluate the performance of character-environment affordance on our proposed dataset. 
We construct a baseline algorithm by directly compositing character animation results onto the original video background, creating a pseudo character-environment integration, similar to MovieCharacter \cite{moviecharacter}. we leverage ProPainter\cite{propainter} to inpaint the character region. 
Quantitative evaluation is presented in Tab.~\ref{table:base}. 
We conduct qualitative comparison illustrated in Fig.~\ref{fig:mimo}. Our approach demonstrates superior performance in terms of enhanced character-environment integration.
We also compare our method with MIMO\cite{mimo}, which is the most relevant method to our task setting. Due to the absence of public source code, we conduct a qualitative comparison focused on character-environment integration performance. The result of MIMO are obtained from its official ModelScope link$^*$. As illustrated in Fig.~\ref{fig:mimo}. From the first group of the visualization, it can be observed that due to MIMO's reliance on additional pre-processing algorithms for background inpainting, it tends to leave noticeable preprocessing artifacts and establish erroneous relationships between the background and the animated characters. In contrast, our proposed approach effectively mitigates these issues, enabling superior scene and character integration. The second group further illustrates MIMO's limitations in handling relatively complex human-object interaction scenarios, whereas our method demonstrates enhanced robustness in intricate scenes.

\renewcommand{\thefootnote}{}
\footnotetext{$^*$https://modelscope.cn/studios/iic/MIMO}


\begin{figure}[!t]
\begin{center}
	\setlength{\fboxrule}{0pt}
	\fbox{\includegraphics[width=0.99\linewidth]{./figure/fig6.pdf}}
\end{center}
\vspace{-0.7cm}
\caption{Qualitative comparion. Our method demonstrates superior environment integration and object interaction.}
\vspace{-0.1cm}
\label{fig:mimo}
\end{figure}



\begin{table}
    \setlength{\tabcolsep}{4pt}
	\centering
        \resizebox{0.36\textwidth}{!}{
        \section{Baseline} \label{sec:splitgraph}

The baseline method for batch-$k$DP solves each query using flow-augmenting path-based methods, which rely on the concept of \textit{split-graphs}~\cite{baseline_moreverbose, baseline1step2, baselineOnlySplitP1}. 
% For each query, paths are iteratively found in a split-graph, which is updated after each iteration.
% A split-graph is constructed by two transformations of the original graph:
% (1) reversing result-set paths, simulating flow-augmentation, and 
% (2) splitting vertices within these paths, giving rise to the name ``split-graph."

\textbf{Definition: Split-Graph~\cite{baselineOnlySplitP1}} 
Given a graph \( G = (V, E) \) and a set \( P \) of disjoint paths from \( s \) to \( t \), the split-graph \( \iG_{G,P} = (\iV_{G,P}, \iE_{G,P}) \) is constructed as follows:
(1) Initializing \( \iV_{G,P} = V \) and \( \iE_{G,P} = E \).
(2) For each edge in \( E(P) \), reversing the corresponding edge in \( \iE_{G,P} \).
(3) Splitting vertices \(v \in V(P) \setminus \{s, t\}\) into \(v^{in}\) and \(v^{out}\), and connecting them accordingly.
(4) Replacing edges in \(\iE_{G,P}\) with updated vertex connections, preserving incoming and outgoing edges.

% \textbf{Example}: 
% Fig.~\ref{fig:eg_split} shows the split-graph construction for the graph \( G \) in Fig.~\ref{fig:g} with $P= \{p_1=\{a, e, d, h\}\}$. Changes are shown in red.


% \vspace{-10pt}
\begin{figure}[h!]
\newcommand{\mylinewidth}{\linewidth}
\centering
    \begin{subfigure}[t]{0.35\mylinewidth}
        \centering
        % \resizebox{\mylinewidth}{!}
        {\includegraphics[width=\linewidth]{pic/eg/g}}
        \caption{Disjoint paths for $(a, h)$.}
        \label{fig:g}
    \end{subfigure}
    \begin{subfigure}[t]{0.6\mylinewidth}
        \centering
        % \resizebox{\mylinewidth}{!}
        {\includegraphics[width=\linewidth]{pic/eg/steps_red_new.pdf}}
        \caption{Split-graph with $P= \{p_1=\{$a$, $e$, $d$, $h$\}\}$.}
        \label{fig:eg_split}
    \end{subfigure}
    \caption{Examples of disjoint paths and split-graph.}
    % \label{fig:fg_share_intuition}
\end{figure} 
% \vspace{-5pt}

% 删除 begin
Given a graph \( G \) and vertices \( s \) and \( t \), the algorithm proceeds as follows:
% (1) Initialize \( P = \emptyset \) and \( \iG_{G,P} = G \).
% (2) Find the first path \( p_1 \) using a path-finding algorithm (e.g., BFS) in \( \iG_{G,P} \) and update \( \iG_{G,P} \).
% (3) Find the second path \( p_2 \), update found paths following an approach similar to augmenting paths in the maximum flow problem~\cite{baseline_moreverbose}, then update \( \iG_{G,P} \). More paths are found in a similar manner.
(1) Initialize $P = \emptyset$ and $\iG_{G, P} = G$.
(2) Find the first path $p_1$ in $\iG_{G, P}$ using any path-finding algorithm (e.g., BFS), forming $P_1 = \{p_1\}$, and update $\iG_{G, P}$ to $\iG_{G, P_1}$.
(3) Search for $p_2$ in $\iG_{G, P_1}$, yielding $P_2 = \{p_1, p_2\}$, and adjust $P_2$ following an approach similar to augmenting flows~\cite{baseline_moreverbose}.
Then update $\iG_{G, P_1}$ to $\iG_{G, P_2}$.
(4) Search for $p_3$ in $\iG_{G, P_2}$. More paths are found in a similar manner.
% 删除 end
    }
    \vspace{-0.2cm}
	\caption{Quantitative comparison with baseline on our dataset. Baseline refers to the pseudo character-environment integration.}
    \vspace{-0.3cm}
	\label{table:base}
\end{table}





\subsection{Ablation Study}

\noindent
\textbf{Environment Formulation. }
To demonstrate the effectiveness of our proposed environment formulation strategy, we explore alternative designs, including: 1) utilizing precise character masks from the source video, and 2) employing bounding box regions. Qualitative results are shown in Fig.~\ref{fig:aba1}. Using accurate masks can constrain the animated character's shape within the predefined mask boundaries, potentially causing appearance deformation and inconsistency. Conversely, adopting bounding box regions may introduce scene context distortions and fusion artifacts in the proximity of character. 
%By employing our proposed scene formulation strategy, we ensure robust character animation consistency, effectively mitigating potential biasing from source video characters. 
Our method demonstrates superior capability in learning flexible character generation and environmental completion, achieving both character consistency and seamless character-scene integration.

\begin{figure}[!t]
\begin{center}
	\setlength{\fboxrule}{0pt}
	\fbox{\includegraphics[width=1\linewidth]{./figure/fig7.pdf}}
\end{center}
\vspace{-0.7cm}
\caption{Ablation study of environment formulation.}
\vspace{-0.1cm}
\label{fig:aba1}
\end{figure}





\begin{figure}[!t]
\begin{center}
	\setlength{\fboxrule}{0pt}
	\fbox{\includegraphics[width=0.8\linewidth]{./figure/fig8.pdf}}
\end{center}
\vspace{-0.7cm}
\caption{Qualitative ablation of object modeling method.}
\vspace{-0.2cm}
\label{fig:aba2}
\end{figure}












\noindent
\textbf{Object Modeling. }
%We analyzed the two critical components in our proposed object injection mechanism.
We conduct a comparison of different object modeling approaches: directly merging object features with noise latents without employing spatial blending. Quantitative result is shown in Tab~\ref{table:ablation}. We further demonstrate the visualization results. As shown in Fig.~\ref{fig:aba2}, 
in complex interaction scenarios, it fails to comprehensively preserve the intrinsic features of interactive objects, resulting in local distortions and consequently misinterpreting their interaction relationships.
The second comparison reveals that the interactions between characters and objects exhibit an artificial stitching effect, which consequently compromises the naturalness of their interactive relationships.
%Our proposed method effectively learns the intricate interactions and fusion relationships between characters and objects, demonstrating superior performance.

\noindent
\textbf{Pose Modulation. }
We evaluate the effectiveness of our proposed pose modulation strategy. Quantitative result is presented in Tab~\ref{table:ablation}. Qualitative result is shown in Fig~\ref{fig:aba3}. Without employing the pose modulation method, character limb relationships may suffer from misalignment and spatial inconsistencies. Consequently, the model's capability to generate accurate and plausible character poses becomes severely constrained. In contrast, our proposed approach, by incorporating depth-aware information, can more effectively learn and capture the complex spatial relationships between limbs, enabling robust performance across diverse and challenging motion scenarios.







\begin{figure}[!t]
\begin{center}
	\setlength{\fboxrule}{0pt}
	\fbox{\includegraphics[width=0.8\linewidth]{./figure/fig9.pdf}}
\end{center}
\vspace{-0.7cm}
\caption{Qualitative ablation of pose modulation.}
\vspace{-0.01cm}
\label{fig:aba3}
\end{figure}



\begin{table}
    \setlength{\tabcolsep}{4pt}
	\centering
        \resizebox{0.45\textwidth}{!}{
        \begin{table*}
  [t]
  \centering
  \resizebox{\textwidth}{!}{%
  \begin{tabular}{cccccccccccc}
    \toprule \multicolumn{2}{c}{Components}                                                             & \multicolumn{5}{c}{Re-executability Rate (\%)} & \multicolumn{5}{c}{Readability (\#)} \\
    \cmidrule(lr){1-2} \cmidrule(lr){3-7} \cmidrule(lr){8-12}        \hspace{8pt}\labelemoji\hspace{8pt}                                                                & \hspace{8pt}\toolemoji\hspace{8pt}                                      & O0                                 & O1             & O2             & O3             & AVG            & O0             & O1             & O2             & O3             & AVG            \\
    \hline
    \rowcolor[rgb]{0.93,0.93,0.93}\multicolumn{12}{c}{\textbf{Initialize with LLM4Decompile-End-6.7B~\citep{llm4decompile}}}   \\
    \xmark                                                                                              & \xmark                                    & 69.51                              & 46.95          & 50.61          & 46.34          & 53.35          & 3.98 & 3.41 & 3.44 & 3.38 & 3.55 \\
    \cmark                                                                                              & \xmark                                    & 75.61                              & 50.61          & 50.00          & 50.00          & 56.55          & 4.01 & 3.44 & 3.39 & \textbf{3.49} & 3.58 \\
    \xmark                                                                                              & \cmark                                    & 83.54                     & \textbf{56.10}          & 51.22          & 50.61 & 60.37 & 4.05 & 3.51 & 3.51 & 3.42 & 3.62 \\
    \cmark                                                                                              & \cmark                                    & \textbf{85.37}                            & \textbf{56.10}                     & \textbf{51.83} & \textbf{52.43}          & \textbf{61.43} & \textbf{4.13} & \textbf{3.60} & \textbf{3.54} & \textbf{3.49} & \textbf{3.69} \\

    \rowcolor[rgb]{0.93,0.93,0.93}\multicolumn{12}{c}{\textbf{Initialize with Deepseek-Coder-6.7B-base~\citep{deepseekcoder}}} \\
    \xmark                                                                                              & \xmark                                    & 59.15                              & 35.98          & 39.02          & 37.80          & 42.99          & 3.71 & 3.05 & 3.16 & 3.05 & 3.24 \\
    \cmark                                                                                              & \xmark                                    & 66.46                              & 41.46          & 38.41          & 36.59          & 45.73          & 3.76 & 3.17 & \textbf{3.21} & 3.08 & 3.31 \\
    \xmark                                                                                              & \cmark                                    & 70.73                              & 39.63          & 39.02          & 40.24          & 47.41          & 3.90 & 3.17 & 3.08 & 3.11 & 3.31 \\
    \cmark                                                                                              & \cmark                                    & \textbf{79.88}                     & \textbf{45.73} & \textbf{43.90} & \textbf{42.68} & \textbf{53.05} & \textbf{3.96} & \textbf{3.21} & 3.18 & \textbf{3.19} & \textbf{3.38} \\
    \bottomrule
  \end{tabular}%
  }
  \caption{The ablation study of different methods across four optimization levels
  (O0, O1, O2, O3), as well as their average scores (AVG). The results in bold represent the optimal performance. The ~\labelemoji~ and ~\toolemoji~ means Relabedling and Function Call. \textbf{Bold} denotes the best performance.}
  \label{tab:ablation}
\end{table*}
    }
    \vspace{-0.2cm}
	\caption{Quantitative ablation study.}
    \vspace{-0.1cm}
	\label{table:ablation}
\end{table}



% \begin{table}
%     \setlength{\tabcolsep}{4pt}
% 	\centering
%         \resizebox{0.4\textwidth}{!}{
%         

\begin{figure}
    \setlength{\tabcolsep}{1pt}
    \centering
    {\scriptsize
    \begin{tabular}{cccccc}
        & 0.1 & 0.3 & 0.4 & 0.5 & 0.7 \\
        \raisebox{19pt}{\rotatebox[origin=t]{90}{Input}} &
        \includegraphics[width=0.19\linewidth]{images/ipa_scale_ablation/cfg_7.5/animal_8/0.1/original.jpg} &
        \includegraphics[width=0.19\linewidth]{images/ipa_scale_ablation/cfg_7.5/animal_8/0.3/original.jpg} &
        \includegraphics[width=0.19\linewidth]{images/ipa_scale_ablation/cfg_7.5/animal_8/0.4/original.jpg} &
        \includegraphics[width=0.19\linewidth]{images/ipa_scale_ablation/cfg_7.5/animal_8/0.5/original.jpg} &
        \includegraphics[width=0.19\linewidth]{images/ipa_scale_ablation/cfg_7.5/animal_8/0.7/original.jpg} \\
        \raisebox{19pt}{\rotatebox[origin=t]{90}{Recon.}} &
        \includegraphics[width=0.19\linewidth]{images/ipa_scale_ablation/cfg_7.5/animal_8/0.1/recon.jpg} &
        \includegraphics[width=0.19\linewidth]{images/ipa_scale_ablation/cfg_7.5/animal_8/0.3/recon.jpg} &
        \includegraphics[width=0.19\linewidth]{images/ipa_scale_ablation/cfg_7.5/animal_8/0.4/recon.jpg} &
        \includegraphics[width=0.19\linewidth]{images/ipa_scale_ablation/cfg_7.5/animal_8/0.5/recon.jpg} &
        \includegraphics[width=0.19\linewidth]{images/ipa_scale_ablation/cfg_7.5/animal_8/0.7/recon.jpg} \\
        \raisebox{19pt}{\rotatebox[origin=t]{90}{``cowboy hat''}} &
        \includegraphics[width=0.19\linewidth]{images/ipa_scale_ablation/cfg_7.5/animal_8/0.1/cowboy_hat.jpg} &
        \includegraphics[width=0.19\linewidth]{images/ipa_scale_ablation/cfg_7.5/animal_8/0.3/cowboy_hat.jpg} &
        \includegraphics[width=0.19\linewidth]{images/ipa_scale_ablation/cfg_7.5/animal_8/0.4/cowboy_hat.jpg} &
        \includegraphics[width=0.19\linewidth]{images/ipa_scale_ablation/cfg_7.5/animal_8/0.5/cowboy_hat.jpg} &
        \includegraphics[width=0.19\linewidth]{images/ipa_scale_ablation/cfg_7.5/animal_8/0.7/cowboy_hat.jpg} \\
    \end{tabular}
    }
    \caption{Ablating the guidance scales used in IP-Adapter. Increasing the scale results in a better reconstruction and better preservation of the original image in the edited one. However, using an overly strong scale limits the capability to edit the image.}
    \label{fig:ablation2}
\end{figure}

%     }
%     \vspace{-0.2cm}
% 	\caption{Quantitative ablation of object modeling method.}
%     \vspace{-0.1cm}
% 	\label{table:ablation2}
% \end{table}


% \begin{table}
%     \setlength{\tabcolsep}{4pt}
% 	\centering
%         \resizebox{0.4\textwidth}{!}{
%         \input{table/ablation3}
%     }
%     \vspace{-0.2cm}
% 	\caption{Quantitative ablation of pose modulation.}
%     \vspace{-0.2cm}
% 	\label{table:ablation3}
% \end{table}




\section{Discussion and Conclusion}
\noindent
\textbf{Limitations. }
Our approach may introduce visual artifacts when dealing with complex hand-object interactions that occupy a relatively small pixel region. In intricate human-object interactions, deformation artifacts may emerge when source and target characters exhibit substantial shape discrepancies. The performance of object interaction is also influenced by SAM's segmentation capabilities.
%In intricate human-object interactions, pose retargeting may potentially compromise the fine-grained interaction details. When source and target characters exhibit substantial shape discrepancies, deformation artifacts can emerge.

\noindent
\textbf{Potential Impact. }
The proposed method may be used to produce fake videos of individuals, which can be detected using some anti-spoofing techniques\cite{anti_color,anti_deep,anti_search,demamba,dormant}. 

\noindent
\textbf{Conclusion. }
In this paper, we introduce \textit{Animate Anyone 2}, a novel framework that enables character animation to exhibit environment affordance. We extract environmental information from driving videos, enabling the animated character to preserve its original environment. We propose a novel environment formulation and object injection strategy, facilitating seamless character-environment integration. Moreover, we propose pose modulation that empowers the model to robustly handle diverse motion patterns. 
Experimental results demonstrate that \textit{Animate Anyone 2} achieves high-fidelity generation performance.


{
    \small
    \bibliographystyle{ieeenat_fullname}
    \bibliography{main}
}

% WARNING: do not forget to delete the supplementary pages from your submission 
% 
\clearpage
% \setcounter{page}{1}
% \maketitlesupplementary
\begin{center}
Supplementary Material
\end{center}

% {
%     \onecolumn
%     \centering
%     \Large
%     \textbf{\thetitle}\\
%     \vspace{0.5em}Supplementary Material \\
%     \vspace{1.0em}
% }

\section{Proof of \cref{theorem:dr}}
We require some additional regularity assumptions:
\begin{assumption} 1) The number of classes $C$ is bounded w.r.t the number of samples $N$, 2) the missingness mechanism $P(A=1|Y,\theta)$, as well as its estimated counterpart $P(A=1|Y,\theta)$, are bounded below by some constant $\epsilon > 0$, 3) the quantities $P(Y|X,\theta)$ and $P(A|Y,\theta)$ are estimated using auxiliary samples independent of samples used for the sample averaging.
\label{assumption:extra}
\end{assumption}
Assumptions 1 and 2 are natural. For the missingness mechanism, the ground truth being bounded means that there is a non-vanishing proportion of samples for every class. The boundedness of the estimate can be enforced by clipping the estimate. Assumption 3 is called sample splitting in \cite{kennedy-dr}.

For convenience we use operator $\E_N$ to denote the average of $N$ samples i.e. $\frac{1}{N}\sum_{i=1}^N$. Note that this is by itself a random variable, in contrast to $\E$ which is a fixed number.

\begin{proof}[Proof of \cref{theorem:dr}] Because $C$ is bounded (assumption \ref{assumption:extra}), we can fix a class $c$ and prove the theorem.
Let us define the influence function $\phi$, parameterized by $\theta$, as
\begin{equation}
\phi(O | \theta)(c) = P(Y=c|X,\theta) + \frac{\one(A=1)}{P(A=1|Y,\theta)} (\one(Y=c) - P(Y=c|X,\theta)) - P(Y=c)
\end{equation}
As we have done in the main text, we use $\phi(O)$ to denote the same function but all estimated quantities are replaced with their truths. In other words, we use $\phi(O)$ for $\phi(O|\theta_0)$ where $\theta_0$ is the truth, given that our model contains $\theta_0$ e.g. when the model is consistent.

Recall that:
\begin{equation}
\begin{aligned}
\Psi_{dr}(\theta)(c) &= \frac{1}{N}\sum_{i=1}^N \left\{P(Y=c|X,\theta) + \frac{\one(A=1)}{P(A=1|Y,\theta)} (\one(Y=c) - P(Y=c|X,\theta))\right\}\\
&= \E_N [\phi(O|\theta)(c)] + P(Y=c)
\end{aligned}
\end{equation}

We will show that:
\begin{equation}
\Psi_{dr}(\theta)(c) - P(Y=c) = (\E_N - \E)[\phi(O)(c)] + o_P(N^{-1/2})
\label{eq:proof-linearity}
\end{equation}
To do that, we use the following decomposition
\begin{equation}
\begin{aligned}
\Psi_{dr}(\theta)(c) - P(Y=c) &= \E_N [\phi(O|\theta)(c)] \\
&= (\E_N - \E)[\phi(O)(c)] + (\E_N - \E)[\phi(O|\theta)(c) - \phi(O)(c)] + \E[\phi(O|\theta)(c)]
% &+ (\E_n - \E)[\phi(O;\theta) - \phi(O)]\\
% &+ \E[P(Y=c|X,\theta)] - \E[P(Y=c|X)] + \E[\phi(O,\theta)]
\end{aligned}
\end{equation}
and analyze the second and third term. The third term is:
\begin{equation}
\begin{aligned}
\E[\phi(O|\theta)(c)] &= \E[P(Y=c|X,\theta)] + \E\left[\frac{\one(A=1)}{P(A=1|Y,\theta)}(\one(Y=c) - P(Y=c|X,\theta))\right]- P(Y=c) \\
&= \E\left[P(Y=c|X,\theta) + \frac{P(A=1|Y)}{P(A=1|Y,\theta)}(P(Y=c|X) - P(Y=c|X,\theta))\right] - \E[P(Y=c|X)]\\
&= \E\left[(P(Y=c|X,\theta) - P(Y=c|X)) (P(A=1|Y,\theta) -P(A=1|Y)) \frac{1}{P(A=1|Y,\theta)}\right]\\
\end{aligned}
\end{equation}
by Cauchy-Schwarz inequality:
\begin{equation}
\begin{aligned}
\E[\phi(O|\theta)(c)] &\le \frac{1}{\epsilon} \|P(A=1|Y,\theta) - P(A=1|Y)\|_2 \|P(Y=c|X,\theta) - P(Y=c|X)\|_{L_2(P)}\\
&= \frac{1}{\epsilon} o_P(N^{-1/4} N^{-1/4}) = o_P(N^{-1/2})
\end{aligned}
\end{equation}
by assumption \ref{assumption:4th-root-n} and that $P(A=1|Y,\theta) > \epsilon$ (assumption \ref{assumption:extra}). The second term can be bounded by Chebyshev inequality
% \begin{equation}
% \begin{aligned}
% \E[\E_N[\phi(O|\theta)(c) - \phi(O)(c)]] &= \E[\phi(O|\theta)(c) - \phi(O)(c)]\\
% \var[\E_N[\phi(O|\theta)(c) - \phi(O)(c)]] &= \frac{1}{N}\var[\phi(O|\theta)(c) - \phi(O)(c)] \le 
% \end{aligned}
% \end{equation}
\begin{equation}
P(|(\E_N - \E)[\phi(O|\theta)(c) - \phi(O)(c)]| \ge t) \le \frac{\var[\E_N[\phi(O|\theta)(c) - \phi(O)(c)]]}{t^2} = \frac{\var[\phi(O|\theta)(c) - \phi(O)(c)]}{Nt^2}
\end{equation}
note here that $\theta$ is independent of the samples used for $\E_N$ by assumption \ref{assumption:extra}. For any $\varepsilon > 0$, by picking $t = \frac{1}{\sqrt{N\varepsilon}}$ we get
\begin{equation}
P\left(\left|\frac{(\E_N - \E)[\phi(O|\theta)(c) - \phi(O)(c)]}{N^{-1/2}}\right| \ge \frac{1}{\sqrt{\varepsilon}}\right) \le \varepsilon \var[\phi(O|\theta)(c) - \phi(O)(c)]
\end{equation}
by the definition of $O_P$, we then get
\begin{equation}
(\E_N - \E)[\phi(O|\theta)(c) - \phi(O)(c)] = O_P(N^{-1/2}\var[\phi(O|\theta)(c) - \phi(O)(c)])
\end{equation}
Because $\phi$ is a continuous function of $P(Y|X,\theta)$ and $P(A|Y,\theta)$ (given $P(A|Y,\theta) > \epsilon$, assumption \ref{assumption:extra}), by the continuous mapping theorem and the fact that $P(Y|X,\theta)$ and $P(A|Y,\theta)$ are convergent in probability (assumption \ref{assumption:4th-root-n}), we get $\var[\phi(O|\theta)(c) - \phi(O)(c)] = o_P(1)$. This gives
\begin{equation}
(\E_N - \E)[\phi(O|\theta)(c) - \phi(O)(c)] = o_P(N^{-1/2})
\end{equation}
Therefore, we have shown that the second and third term are both $o_P(N^{-1/2})$, proving \cref{eq:proof-linearity}. As the final step, multiply both sides of this equation by $\sqrt{N}$ we get:
\begin{equation}
\sqrt{N}(\Psi_{dr}(\theta)(c) - P(Y=c)) = \sqrt{N} (\E_N - \E)[\phi(O)(c)] + o_P(1) \rightsquigarrow \mathcal{N}(0, \var[\phi(O)(c)])
\end{equation}
by the central limit theorem, and $\var[\phi(O)(c)] = \E[\phi(O)(c)^2]$ because $\E[\phi(O)(c)] = 0$.
\end{proof}

While we started with the definition of $\phi$, \cref{eq:proof-linearity} shows that $\phi$ is indeed an influence function. Now we show that $\phi$ is also the efficient influence function, by using the characterization of the model's tangent space \cite{tsiatis-missingdata}. Note that the joint probability factorizes as $P(X,A,Y) = P(X)P(Y|X)P(A|Y)$, therefore the tangent space $\mathcal{T}$ factorizes as $\mathcal{T} = \mathcal{T}_{X} \oplus \mathcal{T}_{Y|X} \oplus \mathcal{T}_{A|Y}$ where $\mathcal{T}_X = \{h(X): \E[h] = 0\}$, $\mathcal{T}_{Y|X} = \{h(X,Y): \E[h|X] = 0\}$, $\mathcal{T}_{A|Y} = \{h(A,Y): \E[h|Y] = 0\}$, and the 3 subspaces are pairwise orthogonal. All influence functions are orthogonal to the tangent space, but the influence function that is also in the tangent space has the smallest variance and is called the efficient influence function. As $\phi$ is already an influence function, we need only show that $\phi$ is in $\mathcal{T}$. We write $\phi$ as
\begin{equation}
\phi(O)(c) = (P(Y=c|X) - P(Y=c)) + \left[\frac{\one(A=1)}{P(A=1|Y)} - 1\right](\one(Y=c) - P(Y=c|X)) + (\one(Y=c) - P(Y=c|X))
\end{equation}
and note that the first, second and third term are in $\mathcal{T}_X$, $\mathcal{T}_{A|Y}$ and $\mathcal{T}_{Y|X}$ respectively. Therefore, $\phi$ is indeed in $\mathcal{T}$. The efficient influence function has the smallest variance of all influence function, and therefore our estimator being asymptotically linear in $\phi$ (\cref{eq:proof-linearity}) has the smallest mean squared error in a local asymptotic minimax sense \cite{kennedy-dr, asymptoticstatistics}

\section{Further background and related work}
\paragraph{Discussion on semi-supervised EM.}
It appears that semi-supervised EM was first used for parameter estimation when the missingness mechanism is non-ignorable in \cite{ibrahim1996parameter}, but has not been used for label shift estimation.
Perhaps this is because the semi-supervised situation where additional unlabeled data is available during training is rarer than the test-time adaptation case. EM is well suited to take advantage of the extra unlabeled data to improve the classifier under very scarce and long-tailed labeled data. While the connection between pseudo-labeling and EM has been explored before \cite{entropyminimization}, the situation with label shift has not until recently \cite{simpro}. Here the application of EM is much more interesting, because other than simply giving pseudo-labeling a rigorous formulation, EM also estimates the missingness mechanism (equivalently the label distribution shift), which is important for shift correction and thus high-quality pseudo-labels \cite{acr}. The application of confidence thresholding can be seen as a sparse variant of EM \cite{neal1998view}.

\paragraph{The doubly-robust risk.} 
\label{subsec:dr-risk}
A technique that also derives from the theory of semi-parametric efficiency is orthogonal statistical learning \citep{foster2023orthogonal}. The idea is to minimize the doubly-robust risk:
\label{subsec:method-dr-risk}
\begin{equation}
\label{eq:dr-risk}
\mathcal{R}(\theta_2) = \frac{1}{N} \sum_{i=1}^N \Bigg[ l(x_i, \hat y_i|\theta_2) + \frac{\one(a_i=1)}{P(A=a_i|Y=y_i, \theta_1)} (l(x_i, y_i | \theta_2) - l(x_i, \hat y_i | \theta_2))\Bigg]
\end{equation}
where $l(x,y|\theta) = -\sum_{c=1}^C [y]_c \log P(Y=c|X=x,\theta)$ is the negative cross-entropy. 
The notation $[y]_c$ means that we are using the $c$-entry in a C-dimension probability vector $y$. 
Thus, $y_i$ denotes the one-hot label of observation $i$, while $\hat y_i$ denotes the pseudo-label, which can be one-hot or all-zero. 
Finally, we use $\theta_1$ to denote that $P(a|y,\theta_1)$ is an estimation from a previous stage, but it can be estimated with $\theta_2$ as well. 
The risk $\mathcal{R}(\theta_2)$ can be used as a training loss in a straightforward fashion. 
Similar to the doubly robust estimation of $P(Y)$, the doubly robust risk provides approximately unbiased estimation of the risk. 
This property has been used in \citep{arelabelsinformative, onnonrandommissinglabels, drst} also in the semi-supervised learning setting.
More broadly, it is at the heart of one of the core techniques in heterogenous treatment effect estimation in causal estimation \cite{kennedy2023towards, foster2023orthogonal, wager2018estimation}. 
The focus here is not the estimation of $\mathcal{R}(\theta_2)$ per se, but the quality of the learned model \cite{foster2023orthogonal}.
By using the doubly-robust risk, we can achieve an optimality result similar in spirit to our theorem \cref{theorem:dr}, but for the generalization error.
While this is appealing, in practice there are 2 problems with this approach. First, the inverse probability weight $P(A=a_i|Y=y_i,\theta_1)$ can be very large if the class ratio is highly unlabeled, making training unstable \cite{kallus2020deepmatch, pham2023stable}. 
This problem exists for our estimation as well. However, it is much easier to control for estimation than for training because of the iterative nature of model update. Secondly, we can further write $\mathcal{R}$ as:
\begin{equation}
\mathcal{R}(\theta_2) = \frac{1}{N}\sum_{i=1}^N l\left(x_i, \hat y_i + \frac{\one(a_i=1)}{P(A=a_i|Y=y_i,\theta_1)} (y_i - \hat y_i)\Bigg\vert\theta_2\right)
\end{equation}
which is a cross-entropy loss with new meta-pseudo-labels. However, these labels are not meant to be learned exactly, and furthermore they can be negative. Thus, theoretical works have to put stringent assumptions on the models. In \cref{subsec:ablation-1}, we show that experimentally that the instability problem makes doubly-robust risk performance worse than our 2-stage approach.

\section{Training and hyperparameter settings.}
\label{subsec:training-setting}
For neural network training, we follow the implementation and hyperparameter settings of \cite{simpro}. In particular, we adapt the core code of SimPro for Supervised, MLE and EM. For MLE, we update $P(A|Y)$ using the Adam optimizer with learning rate 1e-3, while for EM we use a momentum update similar to SimPro's update of $P(Y|A)$ because it has a a closed-form solution at each mini-batch. We use Wide ResNet-28-2 on all methods and all datasets in this section, including Imagenet-127, because we are motivated by the fact that stage-1's goal is not classification accuracy but the estimation of a finite-dimensional parameter. When using Wide ResNet-28-2 for Imagenet-127, we use the hyperparameters of CIFAR-100, except we lower the batch size of unlabeled data to 2 times that of labeled data instead of 8 for memory reason. We do not perform additional hyperparameter tuning. All experiments can be performed on 1 A6000 RTX GPU, and are run 3 times. We report the total variation distance between the estimated and the ground truth unlabeled class distribution, similar to its usage in Theorem 3.1 of \cite{lsc}, and the top-1 classification accuracy.

In the second stage of our algorithm, we freeze our estimation and plug it in SimPro and BOAT.
We keep exactly the same hyperparameter settings that SimPro and BOAT use. In particular, for Imagenet-127, we now use ResNet-50 and run each experiment once.
In SimPro, we set the unlabeled class distribution $P(Y|A=0)$ at the E-step;  however, we still keep a running estimate of the class distribution $P(Y)$ in the logit adjustment loss \cref{eq:simpro-la-loss}. While it is possible to use the first stage estimate in the logit adjustment loss, we observe that doing so results in lower accuracy than using the the running average. This is conceptually consistent with the role of the running average - serving not as an accurate estimate of $P(Y)$ but to make the classifier's class distribution uniform through the logit adjustment loss, which is good for the test set. Similarly, in BOAT, we only replace $\Delta_c = \log P(Y|A=1) - \log P(Y|A=0)$ in equation (4) of \cite{boat}, which is adjusting a classifier's predictions from the labeled to the unlabeled class distribution, with our SimPro + DR estimate instead of their on-the-fly estimate. 


% \section{Additional experiments}
% % \begin{table*}[t]
\centering
\caption{Total Variation Distance on CIFAR-10-LT ($N_l = 500$, $M_l = 4000$) with different class imbalance ratios $\gamma_l$ and $\gamma_u$ under five different unlabeled class distributions.}
\label{tab:cifar10-tv}
\resizebox{\textwidth}{!}{
\begin{tabular}{lccccccccccc}
\toprule
& & \multicolumn{2}{c}{consistent} & \multicolumn{2}{c}{uniform} & \multicolumn{2}{c}{reversed} & \multicolumn{2}{c}{middle} & \multicolumn{2}{c}{head-tail} \\
\cmidrule(lr){3-4} \cmidrule(lr){5-6} \cmidrule(lr){7-8} \cmidrule(lr){9-10} \cmidrule(lr){11-12}
& & $\gamma_l = 150$ & $\gamma_l = 100$ & $\gamma_l = 150$ & $\gamma_l = 100$ & $\gamma_l = 150$ & $\gamma_l = 100$ & $\gamma_l = 150$ & $\gamma_l = 100$ & $\gamma_l = 150$ & $\gamma_l = 100$ \\
Model & Estimator & $\gamma_u = 150$ & $\gamma_u = 100$ & $\gamma_u = 1$ & $\gamma_u = 1$ & $\gamma_u = 1/150$ & $\gamma_u = 1/100$ & $\gamma_u = 150$ & $\gamma_u = 100$ & $\gamma_u = 150$ & $\gamma_u = 100$ \\
\midrule
Supervised & MLLS & 0.269 ± 0.252 & 0.038 ± 0.006 & 0.251 ± 0.046 & 0.255 ± 0.060 & 0.429 ± 0.028 & 0.493 ± 0.050 & 0.333 ± 0.042 & 0.320 ± 0.009 & 0.457 ± 0.034 & 0.444 ± 0.043 \\
Supervised & RLLS & 0.043 ± 0.001 & 0.044 ± 0.010 & 0.348 ± 0.034 & 0.305 ± 0.068 & 0.769 ± 0.016 & 0.678 ± 0.028 & 0.430 ± 0.008 & 0.368 ± 0.013 & 0.539 ± 0.018 & 0.503 ± 0.020 \\
\midrule
MLE & IPW & 0.027 ± 0.001 & 0.027 ± 0.000 & 0.319 ± 0.072 & 0.243 ± 0.010 & 0.674 ± 0.020 & 0.646 ± 0.041 & 0.438 ± 0.020 & 0.454 ± 0.026 & 0.547 ± 0.049 & 0.491 ± 0.059 \\
MLE & OR & 0.045 ± 0.004 & 0.042 ± 0.000 & 0.215 ± 0.026 & 0.203 ± 0.032 & 0.433 ± 0.017 & 0.395 ± 0.033 & 0.193 ± 0.006 & 0.209 ± 0.037 & 0.307 ± 0.147 & 0.249 ± 0.130 \\
MLE & DR & 0.090 ± 0.002 & 0.079 ± 0.000 & 0.407 ± 0.027 & 0.360 ± 0.007 & 0.425 ± 0.007 & 0.421 ± 0.029 & 0.256 ± 0.001 & 0.286 ± 0.031 & 0.435 ± 0.136 & 0.362 ± 0.122 \\
\midrule
EM & IPW & 0.035 ± 0.002 & 0.040 ± 0.001 & 0.021 ± 0.001 & 0.029 ± 0.015 & 0.303 ± 0.187 & 0.091 ± 0.010 & 0.119 ± 0.011 & 0.105 ± 0.022 & 0.104 ± 0.026 & 0.104 ± 0.051 \\
EM & OR & 0.037 ± 0.003 & 0.042 ± 0.002 & 0.016 ± 0.001 & 0.024 ± 0.012 & 0.269 ± 0.183 & 0.090 ± 0.008 & 0.122 ± 0.012 & 0.103 ± 0.022 & 0.072 ± 0.012 & 0.073 ± 0.024 \\
EM & DR & 0.034 ± 0.004 & 0.037 ± 0.001 & 0.014 ± 0.001 & 0.027 ± 0.020 & 0.264 ± 0.191 & 0.092 ± 0.005 & 0.111 ± 0.019 & 0.097 ± 0.026 & 0.077 ± 0.016 & 0.073 ± 0.028 \\
\midrule
SimPro & IPW & 0.070 ± 0.011 & 0.058 ± 0.000 & 0.046 ± 0.001 & 0.049 ± 0.005 & 0.254 ± 0.074 & 0.223 ± 0.098 & 0.097 ± 0.025 & 0.067 ± 0.002 & 0.105 ± 0.066 & 0.110 ± 0.079 \\
SimPro & OR & 0.071 ± 0.012 & 0.058 ± 0.000 & 0.045 ± 0.001 & 0.049 ± 0.006 & 0.040 ± 0.003 & 0.059 ± 0.017 & 0.074 ± 0.006 & 0.075 ± 0.002 & 0.033 ± 0.003 & 0.033 ± 0.003 \\
SimPro & DR & 0.017 ± 0.004 & 0.026 ± 0.001 & 0.019 ± 0.002 & 0.018 ± 0.003 & 0.039 ± 0.003 & 0.058 ± 0.025 & 0.091 ± 0.007 & 0.031 ± 0.001 & 0.015 ± 0.003 & 0.019 ± 0.007 \\
\bottomrule
\end{tabular}
}
\end{table*}
% 

\begin{table*}[t]
\centering
\caption{Total Variation Distance on CIFAR-100-LT ($N_l = 50$, $M_l = 400$) with different class imbalance ratios $\gamma_l$ and $\gamma_u$ under five different unlabeled class distributions.}
\label{tab:cifar100-tv}
\resizebox{\textwidth}{!}{
\begin{tabular}{lccccccccccc}
\toprule
& & \multicolumn{2}{c}{consistent} & \multicolumn{2}{c}{uniform} & \multicolumn{2}{c}{reversed} & \multicolumn{2}{c}{middle} & \multicolumn{2}{c}{head-tail} \\
\cmidrule(lr){3-4} \cmidrule(lr){5-6} \cmidrule(lr){7-8} \cmidrule(lr){9-10} \cmidrule(lr){11-12}
& & $\gamma_l = 20$ & $\gamma_l = 10$ & $\gamma_l = 20$ & $\gamma_l = 10$ & $\gamma_l = 20$ & $\gamma_l = 10$ & $\gamma_l = 20$ & $\gamma_l = 10$ & $\gamma_l = 20$ & $\gamma_l = 10$ \\
Model & Estimator & $\gamma_u = 20$ & $\gamma_u = 10$ & $\gamma_u = 1$ & $\gamma_u = 1$ & $\gamma_u = 1/20$ & $\gamma_u = 1/10$ & $\gamma_u = 20$ & $\gamma_u = 10$ & $\gamma_u = 20$ & $\gamma_u = 10$ \\
\midrule
Supervised & MLLS & 0.707 ± 0.016 & 0.313 ± 0.100 & 0.445 ± 0.172 & 0.309 ± 0.119 & 0.383 ± 0.075 & 0.397 ± 0.006 & 0.570 ± 0.001 & 0.373 ± 0.107 & 0.543 ± 0.009 & 0.231 ± 0.057 \\
Supervised & RLLS & 0.520 ± 0.007 & 0.133 ± 0.003 & 0.337 ± 0.125 & 0.253 ± 0.082 & 0.424 ± 0.060 & 0.463 ± 0.003 & 0.454 ± 0.021 & 0.306 ± 0.074 & 0.460 ± 0.028 & 0.241 ± 0.040 \\
\midrule
MLE & IPW & 0.075 ± 0.000 & 0.071 ± 0.001 & 0.229 ± 0.001 & 0.167 ± 0.002 & 0.565 ± 0.005 & 0.443 ± 0.007 & 0.415 ± 0.000 & 0.311 ± 0.005 & 0.343 ± 0.000 & 0.280 ± 0.001 \\
MLE & OR & 0.065 ± 0.002 & 0.061 ± 0.001 & 0.200 ± 0.007 & 0.143 ± 0.001 & 0.526 ± 0.011 & 0.399 ± 0.023 & 0.360 ± 0.003 & 0.256 ± 0.012 & 0.328 ± 0.003 & 0.266 ± 0.005 \\
MLE & DR & 0.149 ± 0.019 & 0.145 ± 0.010 & 0.243 ± 0.004 & 0.214 ± 0.019 & 0.568 ± 0.005 & 0.464 ± 0.014 & 0.403 ± 0.014 & 0.309 ± 0.012 & 0.365 ± 0.007 & 0.320 ± 0.004 \\
\midrule
EM & IPW & 0.097 ± 0.008 & 0.092 ± 0.004 & 0.239 ± 0.007 & 0.179 ± 0.003 & 0.478 ± 0.012 & 0.329 ± 0.020 & 0.262 ± 0.016 & 0.202 ± 0.003 & 0.312 ± 0.002 & 0.227 ± 0.001 \\
EM & OR & 0.121 ± 0.007 & 0.108 ± 0.005 & 0.261 ± 0.007 & 0.189 ± 0.004 & 0.489 ± 0.013 & 0.335 ± 0.020 & 0.274 ± 0.016 & 0.211 ± 0.004 & 0.336 ± 0.003 & 0.235 ± 0.001 \\
EM & DR & 0.125 ± 0.005 & 0.111 ± 0.004 & 0.269 ± 0.007 & 0.194 ± 0.005 & 0.497 ± 0.010 & 0.336 ± 0.024 & 0.281 ± 0.019 & 0.219 ± 0.008 & 0.336 ± 0.007 & 0.233 ± 0.004 \\
\midrule
SimPro & IPW & 0.125 ± 0.001 & 0.100 ± 0.005 & 0.166 ± 0.007 & 0.141 ± 0.009 & 0.353 ± 0.023 & 0.261 ± 0.008 & 0.202 ± 0.003 & 0.158 ± 0.005 & 0.277 ± 0.009 & 0.197 ± 0.003 \\
SimPro & OR & 0.133 ± 0.005 & 0.100 ± 0.004 & 0.160 ± 0.007 & 0.138 ± 0.010 & 0.322 ± 0.014 & 0.253 ± 0.008 & 0.202 ± 0.003 & 0.156 ± 0.005 & 0.269 ± 0.006 & 0.191 ± 0.004 \\
SimPro & DR & 0.122 ± 0.003 & 0.106 ± 0.006 & 0.188 ± 0.009 & 0.149 ± 0.006 & 0.343 ± 0.023 & 0.257 ± 0.007 & 0.219 ± 0.010 & 0.172 ± 0.002 & 0.279 ± 0.007 & 0.198 ± 0.004 \\
\bottomrule
\end{tabular}
}
\end{table*}

\end{document}
