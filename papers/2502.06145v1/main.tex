% CVPR 2025 Paper Template; see https://github.com/cvpr-org/author-kit

\documentclass[10pt,twocolumn,letterpaper]{article}

%%%%%%%%% PAPER TYPE  - PLEASE UPDATE FOR FINAL VERSION
\usepackage{cvpr}              % To produce the CAMERA-READY version
%\usepackage[review]{cvpr}      % To produce the REVIEW version
% \usepackage[pagenumbers]{cvpr} % To force page numbers, e.g. for an arXiv version

% Import additional packages in the preamble file, before hyperref
%
% --- inline annotations
%
\newcommand{\red}[1]{{\color{red}#1}}
\newcommand{\todo}[1]{{\color{red}#1}}
\newcommand{\TODO}[1]{\textbf{\color{red}[TODO: #1]}}
% --- disable by uncommenting  
% \renewcommand{\TODO}[1]{}
% \renewcommand{\todo}[1]{#1}



\newcommand{\VLM}{LVLM\xspace} 
\newcommand{\ours}{PeKit\xspace}
\newcommand{\yollava}{Yo’LLaVA\xspace}

\newcommand{\thisismy}{This-Is-My-Img\xspace}
\newcommand{\myparagraph}[1]{\noindent\textbf{#1}}
\newcommand{\vdoro}[1]{{\color[rgb]{0.4, 0.18, 0.78} {[V] #1}}}
% --- disable by uncommenting  
% \renewcommand{\TODO}[1]{}
% \renewcommand{\todo}[1]{#1}
\usepackage{slashbox}
% Vectors
\newcommand{\bB}{\mathcal{B}}
\newcommand{\bw}{\mathbf{w}}
\newcommand{\bs}{\mathbf{s}}
\newcommand{\bo}{\mathbf{o}}
\newcommand{\bn}{\mathbf{n}}
\newcommand{\bc}{\mathbf{c}}
\newcommand{\bp}{\mathbf{p}}
\newcommand{\bS}{\mathbf{S}}
\newcommand{\bk}{\mathbf{k}}
\newcommand{\bmu}{\boldsymbol{\mu}}
\newcommand{\bx}{\mathbf{x}}
\newcommand{\bg}{\mathbf{g}}
\newcommand{\be}{\mathbf{e}}
\newcommand{\bX}{\mathbf{X}}
\newcommand{\by}{\mathbf{y}}
\newcommand{\bv}{\mathbf{v}}
\newcommand{\bz}{\mathbf{z}}
\newcommand{\bq}{\mathbf{q}}
\newcommand{\bff}{\mathbf{f}}
\newcommand{\bu}{\mathbf{u}}
\newcommand{\bh}{\mathbf{h}}
\newcommand{\bb}{\mathbf{b}}

\newcommand{\rone}{\textcolor{green}{R1}}
\newcommand{\rtwo}{\textcolor{orange}{R2}}
\newcommand{\rthree}{\textcolor{red}{R3}}
\usepackage{amsmath}
%\usepackage{arydshln}
\DeclareMathOperator{\similarity}{sim}
\DeclareMathOperator{\AvgPool}{AvgPool}

\newcommand{\argmax}{\mathop{\mathrm{argmax}}}     



% It is strongly recommended to use hyperref, especially for the review version.
% hyperref with option pagebackref eases the reviewers' job.
% Please disable hyperref *only* if you encounter grave issues, 
% e.g. with the file validation for the camera-ready version.
%
% If you comment hyperref and then uncomment it, you should delete *.aux before re-running LaTeX.
% (Or just hit 'q' on the first LaTeX run, let it finish, and you should be clear).
\definecolor{cvprblue}{rgb}{0.21,0.49,0.74}
\usepackage[pagebackref,breaklinks,colorlinks,allcolors=cvprblue]{hyperref}

\usepackage{indentfirst}

%%%%%%%%% PAPER ID  - PLEASE UPDATE
\def\paperID{*****} % *** Enter the Paper ID here
\def\confName{CVPR}
\def\confYear{2025}

%%%%%%%%% TITLE - PLEASE UPDATE
\title{Animate Anyone 2: High-Fidelity Character Image Animation with Environment Affordance}

%%%%%%%%% AUTHORS - PLEASE UPDATE
% \author{First Author\\
% Institution1\\
% Institution1 address\\
% {\tt\small firstauthor@i1.org}
% % For a paper whose authors are all at the same institution,
% % omit the following lines up until the closing ``}''.
% % Additional authors and addresses can be added with ``\and'',
% % just like the second author.
% % To save space, use either the email address or home page, not both
% \and
% Second Author\\
% Institution2\\
% First line of institution2 address\\
% {\tt\small secondauthor@i2.org}
% }

\author{Li Hu$^*$ \quad Guangyuan Wang$^*$ \quad Zhen Shen \quad Xin Gao \quad Dechao Meng \quad Lian Zhuo \\
Peng Zhang \quad Bang Zhang \quad Liefeng Bo\\
Tongyi Lab, Alibaba Group\\
%\tt\small \{hooks.hl, zimu.gx, futian.zp, xisheng.sk, zhangbang.zb, liefeng.bo\}@alibaba-inc.com\\
\small \url{https://humanaigc.github.io/animate-anyone-2/}
}


\begin{document}

% \begin{abstract}


The choice of representation for geographic location significantly impacts the accuracy of models for a broad range of geospatial tasks, including fine-grained species classification, population density estimation, and biome classification. Recent works like SatCLIP and GeoCLIP learn such representations by contrastively aligning geolocation with co-located images. While these methods work exceptionally well, in this paper, we posit that the current training strategies fail to fully capture the important visual features. We provide an information theoretic perspective on why the resulting embeddings from these methods discard crucial visual information that is important for many downstream tasks. To solve this problem, we propose a novel retrieval-augmented strategy called RANGE. We build our method on the intuition that the visual features of a location can be estimated by combining the visual features from multiple similar-looking locations. We evaluate our method across a wide variety of tasks. Our results show that RANGE outperforms the existing state-of-the-art models with significant margins in most tasks. We show gains of up to 13.1\% on classification tasks and 0.145 $R^2$ on regression tasks. All our code and models will be made available at: \href{https://github.com/mvrl/RANGE}{https://github.com/mvrl/RANGE}.

\end{abstract}

    
% \section{Introduction}
Backdoor attacks pose a concealed yet profound security risk to machine learning (ML) models, for which the adversaries can inject a stealth backdoor into the model during training, enabling them to illicitly control the model's output upon encountering predefined inputs. These attacks can even occur without the knowledge of developers or end-users, thereby undermining the trust in ML systems. As ML becomes more deeply embedded in critical sectors like finance, healthcare, and autonomous driving \citep{he2016deep, liu2020computing, tournier2019mrtrix3, adjabi2020past}, the potential damage from backdoor attacks grows, underscoring the emergency for developing robust defense mechanisms against backdoor attacks.

To address the threat of backdoor attacks, researchers have developed a variety of strategies \cite{liu2018fine,wu2021adversarial,wang2019neural,zeng2022adversarial,zhu2023neural,Zhu_2023_ICCV, wei2024shared,wei2024d3}, aimed at purifying backdoors within victim models. These methods are designed to integrate with current deployment workflows seamlessly and have demonstrated significant success in mitigating the effects of backdoor triggers \cite{wubackdoorbench, wu2023defenses, wu2024backdoorbench,dunnett2024countering}.  However, most state-of-the-art (SOTA) backdoor purification methods operate under the assumption that a small clean dataset, often referred to as \textbf{auxiliary dataset}, is available for purification. Such an assumption poses practical challenges, especially in scenarios where data is scarce. To tackle this challenge, efforts have been made to reduce the size of the required auxiliary dataset~\cite{chai2022oneshot,li2023reconstructive, Zhu_2023_ICCV} and even explore dataset-free purification techniques~\cite{zheng2022data,hong2023revisiting,lin2024fusing}. Although these approaches offer some improvements, recent evaluations \cite{dunnett2024countering, wu2024backdoorbench} continue to highlight the importance of sufficient auxiliary data for achieving robust defenses against backdoor attacks.

While significant progress has been made in reducing the size of auxiliary datasets, an equally critical yet underexplored question remains: \emph{how does the nature of the auxiliary dataset affect purification effectiveness?} In  real-world  applications, auxiliary datasets can vary widely, encompassing in-distribution data, synthetic data, or external data from different sources. Understanding how each type of auxiliary dataset influences the purification effectiveness is vital for selecting or constructing the most suitable auxiliary dataset and the corresponding technique. For instance, when multiple datasets are available, understanding how different datasets contribute to purification can guide defenders in selecting or crafting the most appropriate dataset. Conversely, when only limited auxiliary data is accessible, knowing which purification technique works best under those constraints is critical. Therefore, there is an urgent need for a thorough investigation into the impact of auxiliary datasets on purification effectiveness to guide defenders in  enhancing the security of ML systems. 

In this paper, we systematically investigate the critical role of auxiliary datasets in backdoor purification, aiming to bridge the gap between idealized and practical purification scenarios.  Specifically, we first construct a diverse set of auxiliary datasets to emulate real-world conditions, as summarized in Table~\ref{overall}. These datasets include in-distribution data, synthetic data, and external data from other sources. Through an evaluation of SOTA backdoor purification methods across these datasets, we uncover several critical insights: \textbf{1)} In-distribution datasets, particularly those carefully filtered from the original training data of the victim model, effectively preserve the model’s utility for its intended tasks but may fall short in eliminating backdoors. \textbf{2)} Incorporating OOD datasets can help the model forget backdoors but also bring the risk of forgetting critical learned knowledge, significantly degrading its overall performance. Building on these findings, we propose Guided Input Calibration (GIC), a novel technique that enhances backdoor purification by adaptively transforming auxiliary data to better align with the victim model’s learned representations. By leveraging the victim model itself to guide this transformation, GIC optimizes the purification process, striking a balance between preserving model utility and mitigating backdoor threats. Extensive experiments demonstrate that GIC significantly improves the effectiveness of backdoor purification across diverse auxiliary datasets, providing a practical and robust defense solution.

Our main contributions are threefold:
\textbf{1) Impact analysis of auxiliary datasets:} We take the \textbf{first step}  in systematically investigating how different types of auxiliary datasets influence backdoor purification effectiveness. Our findings provide novel insights and serve as a foundation for future research on optimizing dataset selection and construction for enhanced backdoor defense.
%
\textbf{2) Compilation and evaluation of diverse auxiliary datasets:}  We have compiled and rigorously evaluated a diverse set of auxiliary datasets using SOTA purification methods, making our datasets and code publicly available to facilitate and support future research on practical backdoor defense strategies.
%
\textbf{3) Introduction of GIC:} We introduce GIC, the \textbf{first} dedicated solution designed to align auxiliary datasets with the model’s learned representations, significantly enhancing backdoor mitigation across various dataset types. Our approach sets a new benchmark for practical and effective backdoor defense.



% \section{Related work}
\label{sec:formatting}

\subsection{Text-to-Video Generation}

T2V generation has made notable progress, evolving from early GAN-based models \cite{saito2017temporal,tulyakov2018mocogan,fu2023tell,li2018video,wu2022nuwa,yu2022generating} to newer transformer \cite{yan2021videogpt,arnab2021vivit,esser2021taming,ramesh2021zero,yu2022scaling} and diffusion models \cite{kirkpatrick2017overcoming,sohl2015deep,song2020denoising,zhang2022gddim}. Early efforts like MoCoGAN~\cite{tulyakov2018mocogan} focused on short video clips but faced issues with stability and coherence. The introduction of transformers improved sequential data handling, enhancing video generation, while diffusion models further improved video quality by progressively denoising the input. 
Despite these advances, T2V models still struggle to reflect human preferences, with the generated videos generally lacking aesthetic quality. Additionally, the scarcity of paired video preference data hinders effective model training and may lead to insufficient flexibility and poor quality in the generated videos.


\subsection{RLHF}

\iffalse
Aligning LLMs \cite{dai1901transformer,radford2019language,zhang2023opt} typically involves two steps: supervised fine-tuning followed by Reinforcement Learning with Human Feedback (RLHF) \cite{gao2023scaling,stiennon2020learning,rafailov2024direct}. Although effective, RLHF is computationally expensive and can lead to issues like reward hacking. Methods like DPO have streamlined alignment by leveraging feedback data directly, improving efficiency.

In contrast, diffusion model alignment is still evolving, focusing mainly on enhancing visual quality through curated datasets. Techniques like DOODL \cite{wallace2023end} and AlignProp \cite{prabhudesai2023aligning} target aesthetic improvements but face challenges with complex tasks such as text-image alignment. Reinforcement learning methods like DPOK \cite{fan2024reinforcement} and DDPO \cite{black2023training} improve reward optimization but struggle with scalability. DPO-SDXL integrates DPO into T2I generation, boosting both alignment and aesthetics.

However, aligning video generation remains a largely unaddressed challenge, especially when dealing with motion consistency and semantic coherence across frames.
\fi

RLHF \cite{gao2023scaling,stiennon2020learning,rafailov2024direct} is a method that utilizes human feedback to guide machine learning models. Early RLHF algorithms, such as DDPG~\cite{lillicrap2015continuous} and PPO~\cite{schulman2017proximal}, typically relied on complex reward models to quantify human feedback. These reward models require a large amount of annotated data and face challenges during tuning. As research has progressed, more efficient preference learning methods have emerged, among which DPO has become a new framework. DPO does not depend on a separate reward model; instead, it obtains human preferences through pairwise comparisons and directly optimizes these preferences. This shift not only simplifies the application of RLHF but also enhances the alignment of models with human values. Furthermore, DPO has been successfully introduced into T2I tasks~\cite{wallace2024diffusion,yang2024using}, providing new insights for generative models in addressing the alignment of human preferences and showcasing DPO's potential in the field of AIGC~\cite{shi2024instantbooth,
qing2024hierarchical,menapace2024snap,koley2024s}. However, there remains a gap in current research regarding the application of DPO strategies to T2V tasks. Effectively integrating DPO into T2V tasks presents a challenging endeavor.



% \section{Preliminary}
\label{sec:preliminary}
In this section, we first introduce the mathematical formulation of flow-based text-to-image generative models~\cite{Xingchao_2022,Lipman_2022}, which forms the foundation of modern T2I systems~\cite{sd3,sdxl,imagen3,imagen}. We then describe classifier-free guidance~\cite{ho2022classifier}, a key technique to control the generation process through text conditioning.

\subsection{Flow-based text-to-image generative models}
In state-of-the-art T2I models~\cite{sd3}, the image generation process is modeled by learning, through a neural network, a flow $\psi$ that generates a probability path $(p_t)_{0\le t\le 1}$ bridging the source distribution $p_0$ and the target distribution $p_1$ ~\cite{Xingchao_2022,Lipman_2022}. This framework encompasses diffusion models~\cite{sohl2015deep,ddpm} as a special case. In particular, a commonly used formulation sets a Gaussian distribution as the source: $p_0 = \mathcal{N}(\mathbf{0}, \mathbf{I})$ and a delta distribution centered on a sample $\mathbf{x}_1$ from the data distribution $q$ as the target: $p_1 = \delta_{\mathbf{x}_1}$.
Then, it incorporates an affine conditional flow $\psi_t(\mathbf{x} | \mathbf{x}_1) = a_t \mathbf{x}_1 + b_t \mathbf{x}$ with the boundary condition $(a_0, b_0) = (0, 1),\ (a_1, b_1) = (1, 0)$ to bridge them. The neural network typically approximates quantities such as velocity fields, $x_0$ prediction or $x_1$ prediction. In this modeling, these quantities can be viewed as affine transformations of the marginal probability path score $\nabla_{\mathbf{x}} \log p_t(\mathbf{x})$.

\subsection{Classifier-free guidance in flow-based models}
Classifier-free guidance~\cite{ho2022classifier} is a method for sampling from a model conditioned by a text input $\mathbf{y}$ by guiding an unconditional image generation model modeled using the score $\nabla_{\mathbf{x}} \log p_t(\mathbf{x})$. This enables the sampling from
\[
q_w(\mathbf{x}, \mathbf{y}) \propto q(\mathbf{x})q(\mathbf{y}|\mathbf{x})^w \propto q(\mathbf{x})^{1-w}q(\mathbf{x}|\mathbf{y})^w
\]
where $w \in \mathbb{R}$ is the guidance scale typically used with $w > 1$. The score satisfies
\[
\nabla_{\mathbf{x}} \log q_w(\mathbf{x}, \mathbf{y}) = (1-w)\nabla_{\mathbf{x}} \log q(\mathbf{x}) + w\nabla_{\mathbf{x}} \log q(\mathbf{x}|\mathbf{y})
\]
so by training the network to learn both the unconditional score $\nabla_{\mathbf{x}} \log q(\mathbf{x})$ and conditional score $\nabla_{\mathbf{x}} \log q(\mathbf{x}|\mathbf{y})$, flexible sampling from the conditional distribution can be achieved through a weighted sum of the network outputs.


\twocolumn[{%
\renewcommand\twocolumn[1][]{#1}%
\maketitle
\begin{center}
    \centering
    \captionsetup{type=figure}
    \includegraphics[width=1.0\textwidth]{./figure/fig1.pdf}
    \captionof{figure}{We propose \textit{Animate Anyone 2}, which differs from previous character image animation methods that solely utilize motion signals to animate characters. Our approach additionally extracts environmental representations from the driving video, thereby enabling character animation to exhibit environment affordance. The generated results demonstrate that, beyond maintaining character consistency, \textit{Animate Anyone 2} can produce high-fidelity results that seamlessly integrate characters with the surrounding environment.}
    \label{fig:f1}
\end{center}%
}]

\maketitle

\begin{abstract}
Recent character image animation methods based on diffusion models, such as Animate Anyone, have made significant progress in generating consistent and generalizable character animations. However, these approaches fail to produce reasonable associations between characters and their environments. To address this limitation, we introduce Animate Anyone 2, aiming to animate characters with environment affordance. Beyond extracting motion signals from source video, we additionally capture environmental representations as conditional inputs. The environment is formulated as the region with the exclusion of characters and our model generates characters to populate these regions while maintaining coherence with the environmental context. We propose a shape-agnostic mask strategy that more effectively characterizes the relationship between character and environment. Furthermore, to enhance the fidelity of object interactions, we leverage an object guider to extract features of interacting objects and employ spatial blending for feature injection. We also introduce a pose modulation strategy that enables the model to handle more diverse motion patterns. Experimental results demonstrate the superior performance of the proposed method.
\end{abstract}


\renewcommand{\thefootnote}{}
\footnotetext{$^*$Equal contribution}


\section{Introduction}

The objective of character image animation is to synthesize animated video sequences utilizing a reference character image and a sequence of motion signals. 
Recent developments predominantly adopt diffusion-based frameworks~\cite{dreampose,disco,magicanimate,aa,magicpose,champ,unianimate,mimicmotion}, achieving notable enhancements in appearance consistency, motion stability and character generalizability. 
These advancements exhibit substantial potential in areas such as filmmaking, advertising, and virtual character applications.


In recent cross-identity animation workflows, motion signals are typically extracted from disparate videos, while the character's contextual environments are derived from static images. This setting introduces critical limitations: the spatial relationships between animated characters and their environments often lack authenticity, and intrinsic human-object interactions are disrupted. Consequently, most existing methods are predominantly limited to animating simple actions (e.g., individual gestures or dances) without adequately capturing the complex spatial and interactive relationships between characters and their surroundings. These limitations significantly hinder the advancement of character animation techniques.

Recent attempts to integrate character animation with scenes and objects, while promising, face significant challenges in generation quality and adaptability. 
For instance, MovieCharacter\cite{moviecharacter} synthesizes character videos by cascading the outputs from multiple algorithms, which introduces noticeable artifacts and unnatural visual discontinuities. 
AnchorCrafter\cite{anchorcrafter} primarily focuses on human-object manipulation animation, with relatively simplistic character motion and object appearance. 
MIMO\cite{mimo} addresses this challenge by composing characters, pre-processed backgrounds and occlusions, which are disentangled via depth. Such formulation for defining the relationship between characters and environments is suboptimal, 
limiting the ability to handle complex interactions.


%To overcome these limitations, 
In this paper, 
we propose to expand the scope of character animation by introducing \textit{Character Image Animation with Environment Affordance}.
Specifically, we define the research problem as follows: given a character image and a source video, the generated character animation should: 1) inherit character motion desired by the source video. 2) accurately demonstrate character-environment relationship consistent with the source video. 
This setting introduces novel challenges for character animation, as it requires that the model should effectively handle diverse and complex character motions, while ensuring precise interaction between characters and their environments throughout the animation process.

To achieve this, we introduce a novel framework \textit{Animate Anyone 2}. 
As illustrated in Fig.\ref{fig:f1}, unlike previous character animation methods that solely utilize motion signals, we additionally capture environmental representations from the source video as conditional inputs, which enables the model to learn the intrinsic relationship between character and environment in an end-to-end manner.
We formulate the environment by removing the character regions and our model generates characters to populate these regions while maintaining coherence with the environmental context. We develop a shape-agnostic mask strategy that better represents the boundary relationship between character and their contextual scenes, enabling effective learning for character-context integration while mitigating shape leakage issues. 
Second, to enhance the fidelity of object interactions, we introduce additional processing for interactive object regions. We design a lightweight object guider to extract interactive object features and propose a spatial blending mechanism to inject these features into the generation process. It facilitates the preservation of intricate interaction dynamics in the source video. 
Lastly, we propose depth-wise pose modulation approach for character motion modeling, empowering the model to handle more diverse and complex character poses with enhanced robustness.



The results in Fig.\ref{fig:f1} exhibit both high-quality character animation performance and remarkable environment affordance, manifested through three key advantages: 1) seamless scene integration; 2) coherent object interaction; and 3) robust handling of diverse and complex motions. Our approach is evaluated on corresponding benchmarks, achieving superior character animation results compared to existing methods. 
In summary, we highlight three key contributions of our paper.
\begin{itemize}
\item We introduce \textit{Animate Anyone 2}, a framework capable of animating character with environment affordance, achieving robust performance.
\item We propose a novel environment formulation and object injection strategy to achieve seamless character-environment integration.
\item We propose pose modulation strategy to enhance model robustness in challenging action scenarios.
\end{itemize}




\begin{figure*}[!t]
\begin{center}
	\setlength{\fboxrule}{0pt}
	\fbox{\includegraphics[width=0.99\textwidth]{./figure/fig2.pdf}}
\end{center}
\vspace{-0.6cm}
\caption{The framework of \textit{Animate Anyone 2}. We capture environmental information from the source video. The environment is formulated as regions devoid of characters and incorporated as model input, enabling end-to-end learning of character-environment fusion. To preserve object interactions, we additionally inject features of objects interacting with the character. These object features are extracted by a lightweight object guider and merged into the denoising process via spatial blending. To handle more diverse motions, we propose a pose modulation approach to better represent the spatial relationships between body limbs. 
}
\vspace{-0.2cm}
\label{fig:overview}
\end{figure*}



\section{Related Works}

\subsection{Character Image Animation}
Distinguished from GAN-based\cite{gan,wgan,stylegan} approaches\cite{fomm,mraa,ren2020deep,tpsmm,siarohin2019animating,zhang2022exploring,bidirectionally,everybody}, diffusion-based image animation methods\cite{dreampose,disco,aa,magicanimate,magicpose,mimicmotion,champ,unianimate,tcan,tpc} have emerged as the current research mainstream. As the most representative approach, Animate Anyone\cite{aa} designs its framework based on Stable Diffusion\cite{ldm}, and the denoising network is structured as a 3D UNet\cite{align,animatediff} for temporal modeling. It proposes ReferenceNet, a symmetric UNet\cite{unet} architecture, to preserve appearance consistency and employs pose guider to incorporate skeleton information as driving signals for stable motion control. The Animate Anyone framework achieves robust and generalizable character animation, from which we extensively drew inspiration.

Some works propose improvements upon foundational frameworks. MimicMotion\cite{mimicmotion} leverages pretrained image-to-video capabilities of Stable Video Diffusion\cite{svd}, designing a PoseNet to inject skeleton information. UniAnimate\cite{unianimate} stacks reference images across temporal dimensions, utilizing mamba-based\cite{mamba} temporal modeling techniques. Some works explore different motion control signals. DisCo\cite{disco} and MagicAnimate\cite{magicanimate} utilizes DensePose\cite{densepose} as human body representations. Champ\cite{champ} employs the 3D parametric human model SMPL\cite{smpl}, integrating multi-modal information including depth, normal, and semantic signals derived from SMPL.





%\subsection{Affordance-Aware Image/Video Generation}
%\subsection{Scene/Object-conditioned Generation}
\subsection{Human-environment Affordance Generation}
Numerous studies leverage diffusion models to generate human image or video that contextually integrate with scenes or interactive objects. Some studies\cite{putting,environment-specific,addme,text2place,invi} investigate inserting or inpainting human into given scenes to achieve scene affordance. \cite{putting} applies video self-supervised training to inpaint person into masked region with correct affordances. Text2Place\cite{text2place} aims to place a person in background scenes by learning semantic masks using text guidance for localizing regions. InVi\cite{invi} achieves object insertion by first conducting image inpainting and subsequently generating frames using extended-attention mechanisms.

Several works focus on character animation with scene or object interactions. MovieCharacter\cite{moviecharacter} composites the animated character results into person-removed video sequence. AnchorCrafter\cite{anchorcrafter}, focusing on human-object interaction, first perceives HOI-appearances and injects HOI-motion to generate anchor-style product promotion videos. MIMO\cite{mimo} introduces spatial decomposed diffusion, decomposing videos into human, background and occlusion based on 3D depth and subsequently composing these elements to generate character video.







\section{Method}


In this section, we introduce \textit{Animate Anyone 2}. In \ref{sec:framework}, we first elaborate on the overall framework. In \ref{sec:scene}, we delineate the strategy for environment formulation. In \ref{sec:object}, we present the design of object injection. In \ref{sec:pose}, we provide a detailed exposition of pose modulation strategy.


\subsection{Framework}\label{sec:framework}

\noindent
\textbf{System Setting. }
The overall framework is illustrated in Fig.\ref{fig:overview}.
During training, we employ a self-supervised learning strategy. Given a reference video ${\mathit I}^{1:\mathit N}$ where $\mathit N$ denotes the number of frames, we disentangle character and environment via a formulated mask (detailed in \ref{sec:scene}), obtaining separate character sequence ${\mathit I}^{1:\mathit N}_{\mathit c}$ and environment sequence ${\mathit I}^{1:\mathit N}_{\mathit e}$. To facilitate more fidelity object interaction, we additionally extracted the sequence of objects ${\mathit I}^{1:\mathit N}_{\mathit o}$. 
%We also extract motion sequence ${\mathit I}^{1:\mathit N}_{\mathit m}$ from the character as driving signals.
Motion sequence ${\mathit I}^{1:\mathit N}_{\mathit m}$ is extracted as driving signals.
We randomly sample a character image ${\mathit I}_{\mathit c}$ from ${\mathit I}^{1:\mathit N}_{\mathit c}$ with center crop and composite it onto a random background. Given image ${\mathit I}_{\mathit c}$, motion sequence ${\mathit I}^{1:\mathit N}_{\mathit m}$, environment sequence ${\mathit I}^{1:\mathit N}_{\mathit e}$ and object sequence ${\mathit I}^{1:\mathit N}_{\mathit o}$ as inputs, our model reconstructs the reference video ${\mathit I}^{1:\mathit N}$. 
During inference, given a target character image and a driving video, our method can animate the character with consistent actions and environmental relationship corresponding to the driving video.


\noindent
\textbf{Diffusion Model. }
%Our approach builds upon Animate Anyone, a robust diffusion-based character image animation framework.
Our method is developed based on LDM\cite{ldm}. It employs a pretrained VAE\cite{vae,vqvae} to transform images from pixel space to latent space: $\mathbf z \mathcal = \mathcal E$($\mathbf x$). During training, random Gaussian noise $\epsilon$ is progressively added to image latents ${\mathbf z}_{t}$ at different timesteps,
The training objective can be formulated as follows:

%\vspace{-0.1cm}
\begin{equation}
\label{eq1}
    {\mathbf L} = {\mathbb E}_{{\mathbf z}_{t},c,{\epsilon},t}({||{\epsilon}-{{\epsilon}_{\theta}}({\mathbf z}_{t},c,t)||}^{2}_{2})
\end{equation}
%\vspace{-0.1cm}

\noindent
where ${\epsilon}_{\theta}$ represents the function of DenoisingNet. $\mathnormal c$ represents conditional inputs. During inference, noise latents are iteratively denoised\cite{denoising,ddim} and reconstructed into images through the decoder of VAE: ${\mathbf x}_{recon} \mathcal = \mathcal D$($\mathbf z$). 
The network design of DenoisingNet is derived from Stable Diffusion\cite{ldm}, inheriting its pretrained weights. We extend the original 2D UNet architecture to 3D UNet, incorporating the temporal layer design from AnimateDiff\cite{animatediff}.

\noindent
\textbf{Conditional Generation. }
We adopt the ReferenceNet architecture from \cite{aa} to extract appearance features of the character image ${\mathit I}_{\mathit c}$. 
%Specifically, ReferenceNet employs an identical model structure to the 2D denoising UNet, and the extracted appearance features are injected into the corresponding layers of the denoising UNet via spatial attention\cite{attention}. This approach significantly enhances the consistency of character appearance details. 
In our framework, we simplify the computational complexity by merging these features exclusively in the midblock and upblock of the DenoisingNet decoder via spatial attention\cite{attention}.
Besides, three conditional embeddings are extracted from the souce video: environment sequence ${\mathit I}^{1:\mathit N}_{\mathit e}$, motion sequence ${\mathit I}^{1:\mathit N}_{\mathit m}$, and object sequence ${\mathit I}^{1:\mathit N}_{\mathit o}$. For environment sequence ${\mathit I}^{1:\mathit N}_{\mathit e}$, we employ VAE encoder to encode the embedding and subsequently merge it with noise latents. For motion sequence ${\mathit I}^{1:\mathit N}_{\mathit m}$, we design pose modulation strategy (elaborated in \ref{sec:pose}) and the motion information is also merged into the noise latents. For object sequence ${\mathit I}^{1:\mathit N}_{\mathit o}$, after encoding via VAE encoder, we develop an object guider to extract multi-scale features and inject them into the DenoisingNet through spatial blending, which will be detailed in \ref{sec:object}.






\subsection{Environment Formulation}\label{sec:scene}

\noindent
\textbf{Motivation. }
In our framework, the environment is formulated as a region excluding characters. During training, the model generates characters to populate these regions while maintaining coherence with the environmental context. 
The boundary relationship between characters and the environment is crucial. Appropriate boundary guidance can facilitate the model in learning character-environment integration more effectively, while preserving character shape consistency and environmental information integrity.
%To achieve seamless integration of diverse characters and scenes, it is crucial to dynamically formulate the boundary relationships between characters and their contextual scenes.
Some studies\cite{putting,invi} leverage bounding boxes to represent generative regions. 
However, we observe artifacts or inconsistencies with the source video when dealing with complex scenes, due to insufficient conditioning.
%However, we experimentally observe that when dealing with complex scenes, this strategy often suffers from artifacts or inconsistencies with the source video due to insufficient conditioning.
Conversely, directly using precise masks is also suboptimal, potentially introducing shape leakage.
Due to the self-supervised training strategy, there exists strong correlation between character outlines and mask boundaries. Consequently, the model tends to use this information as additional guidance for animating character. However, during inference, when the target character differs from the source in body shape and clothing, the model may forcibly conform to the mask boundary, resulting in integration artifacts. 


\begin{figure}[!t]
\begin{center}
    \vspace{-0.3cm}
	\setlength{\fboxrule}{0pt}
	\fbox{\includegraphics[width=1\linewidth]{./figure/fig3.pdf}}
\end{center}
\vspace{-0.6cm}
\caption{Different coefficients for mask formulation.}
\vspace{-0.3cm}
\label{fig:mask}
\end{figure}


\noindent
\textbf{Shape-agnostic Mask. }
Therefore, we propose a shape-agnostic mask strategy for environment formulation, with the core idea of disrupting the correspondence between mask region and character outline during training. Specifically, for a character mask ${\mathit M}_{\mathit c}$ in its bounding box of size ${\mathit h} \times {\mathit w}$, we define two coefficients ${\mathit k}_{\mathit h}$ and ${\mathit k}_{\mathit w}$. 
% We randomly partition the character mask ${\mathit M}_{\mathit c}$ into ${\mathit k}_{\mathit h} \times {\mathit k}_{\mathit w}$ segments, where $1 < {\mathit k}_{\mathit h} < {\mathit h}$ and $1 < {\mathit k}_{\mathit w} < {\mathit w}$. 
We divided the character mask ${\mathit M}_{\mathit c}$ into ${\mathit k}_{\mathit h} \times {\mathit k}_{\mathit w}$ non-overlapping blocks, where ${\mathit k}_{\mathit h} \in (1,{\mathit h}), {\mathit k}_{\mathit w} \in (1,{\mathit w})$. 
We denote ${\mathit P}^{\mathit (k)}_{\mathit c}$ as the divided patches, where ${\mathit k}$ is the index. 
We reformulate the mask ${\mathit M}_{\mathit c}$ into a new mask ${\mathit M}_{\mathit f}$ by propagating the patch-wise maximum value:

    % {\mathit M}_{\mathit f}(i, j) = \max {\mathit P}^{\mathit (k)}_{\mathit c}(i, j)
    
    % {\mathit M}_{\mathit f}(i, j) = \max_{i'=i}^{i'+k_h-1} \max_{j'=j}^{j'+k_w-1} {\mathit P}_{\mathit c}(i', j')
%\vspace{-0.1cm}
\begin{equation}
\label{eq1}
{\mathit M}_{\mathit f}(i, j) = \max_{(i,j) \in  {\mathit P}^{\mathit (k)}_{\mathit c} } {\mathit P}^{\mathit (k)}_{\mathit c}(i, j)
\end{equation}
%\vspace{-0.1cm}


% We divided the character mask ${\mathit M}_{\mathit c}$ into non-overlapping blocks of size ${\mathit k}_{\mathit h} \times {\mathit k}_{\mathit w}$, where ${\mathit k}_{\mathit h} \in (1,{\mathit h}), {\mathit k}_{\mathit w} \in (1,{\mathit w})$. 
% The transformed mask ${\mathit M}_{\mathit f}$ can be difined as :
% \begin{equation}
% \label{eq1}
% {\mathit M}_{\mathit f}(i, j) = \begin{cases} 
% 1 & \text{if} \max_{i'=i}^{i'+k_h-1} \max_{j'=j}^{j'+k_w-1} {\mathit P}(i', j')=1 \\
% 0 & \text{else}
% \end{cases}
% \end{equation}
% \vspace{-0.2cm}
% where ${\mathit P}(i', j')$ represents the pixel value of original character mask.

\noindent
where ${\mathit P}^{\mathit (k)}_{\mathit c}(i, j)$ represents the value at position ${\mathit (i,j)}$.
The visualized process is presented in Fig.\ref{fig:mask}. By employing this strategy, the formulated mask dynamically generate different shapes that deviate from the character boundaries, thereby compelling the network to learn context integration more effectively, unencumbered by predefined boundary constraints. During inference, we set ${\mathit k}_{\mathit h} = {\mathit h} / 10$ and ${\mathit k}_{\mathit w} = {\mathit w} / 10$. 


\noindent
\textbf{Random Scale Augmentation. }
Moreover, since the formulated mask is inherently larger than the original mask, this introduces an inevitable bias that constrains the generated character to be necessarily smaller than the given mask. To mitigate this bias, we employ random scale augmentation on source videos. Specifically, we extract the character together with the interacting objects based on their masks and apply a random scaling operation. Subsequently, we recompose these scaled content back into the source video. This approach ensures that the formulated mask has a probabilistic chance of being smaller than the actual character region. During inference, the model is capable of animating the character flexibly without being constrained by the size of the mask.
%Experimental results demonstrate that this strategy not only enhances the fidelity of generated characters but also facilitates seamless character-scene integration without introducing significant scene context distortion.


\subsection{Object Injection}\label{sec:object}

\noindent
\textbf{Object Guider. }
%In this section, we focus on human-object interactions. 
%Due to the scene formulation strategy, object regions might be incomplete. 
The environment formulation strategy may potentially lead to distortion of object regions.
To enhance the preservation of object interactions, we propose to inject additional object-level features. 
Interactive objects can be extracted through two methods: 1) Leveraging VLM\cite{cogvlm,qwenvl} to obtain object localization; 2) Interactively confirming object positions via manual annotation. Then we employ SAM2\cite{sam,sam2} to extract object mask, obtaining corresponding object image and encode it into object latents via VAE encoder. A naive approach to merging object features is to directly concatenate scene and object features before feeding them into the network. However, due to the intricate relationship between characters and objects, such method struggles to handle complex human-object interactions, often falling short in capturing both human and object details. 
%Scenes typically exhibit only boundary-level fusion with human subjects, while objects frequently demonstrate spatially intricate overlapping contours, necessitating a more fine-grained feature fusion design.
Thus we design an object guider to extract object-level features. 
%Since the objects and their interaction relationships are inherently preserved from the original video, our approach does not require a dedicated modeling network equivalent to the main backbone, as typically employed for subject feature extraction. 
Unlike character features that require complex modeling, objects inherently preserve visual characteristics from the source video. 
Thus we implement object guider using a lightweight fully convolutional architecture. 
%The design of each convolutional block is inspired by the Pose Guider in Animate Anyone. The difference is that 
specifically, object latents are downsampled four times via $3 \times 3$ Conv2D to obtain multi-scale features. The channel dimensions of these features are aligned with those in the midblock and upblock of the DenoisingNet,  facilitating subsequent feature fusion.

\noindent
\textbf{Spatial Blending. }
To recover the spatial relationships of human-object interaction, we employ spatial blending to inject features extracted by object guider into the DenoisingNet. Specifically, during the denoising process, spatial blending layer is performed after spatial attention layer. For noise latents ${\mathit z}_{\mathit noise}$ and object latents ${\mathit z}_{\mathit object}$, we concatenate their features and compute the alpha weight ${\mathit \alpha}$ through a Conv2D-Sigmoid layer. The spatial blending process can be mathematically formulated as follows:

\begin{equation}
\label{eq1}
    {\mathit \alpha} = {\mathit F}(cat({\mathit z}_{\mathit noise},{\mathit z}_{\mathit object}))
\end{equation}
\begin{equation}
\label{eq2}
    {\mathit z}_{\mathit blend} = {\mathit \alpha} \cdot {\mathit z}_{\mathit object} + (1 - {\mathit \alpha}) \cdot {\mathit z}_{\mathit noise}
\end{equation}


\noindent
where ${\mathit F}$ denotes the Conv2D-Sigmoid layer, which is initialized through zero convolution. ${\mathit z}_{\mathit blend}$ denotes the new noise latents after spatial blending.  In each stage of the DenoisingNet decoder, we alternately apply spatial attention on character features and spatial blending of object features, enabling the generation of high-fidelity results with excellent details of character-object interactions.

\subsection{Pose Modulation}\label{sec:pose}

\noindent
\textbf{Motivation. }
%In this section, we introduce our approach to motion modeling. 
Animate Anyone\cite{aa} employs a skeleton representation to capture character motion and utilizes pose guider for feature modeling. However, the skeleton representation lacks explicit modeling of inter-limb spatial relationships and hierarchical dependencies. Some existing methods\cite{champ,mimo} adopt 3D mesh representations like SMPL to represent human bodies, but this tends to compromise the generalizability across characters and potentially introduces shape leakage due to its dense representation.

\noindent
\textbf{Depth-wise Pose Modulation. }
We propose to retain the skeleton signals while augmenting it with structured depth to enhance the representation of inter-limb spatial relationships. We refer to this approach as depth-wise pose modulation. For motion signals, we leverage Sapien\cite{sapiens} to extract the skeleton and depth information from the source video. The depth information is structurally processed via the skeleton to mitigate potential shape leakage in raw depth maps. Specifically, we first binarize the skeleton image to obtain skeleton mask, and subsequently extract the depth results within this masked region.
Then we employ Conv2D with the same architectural design as the pose guider\cite{aa} to process the skeleton map and structured depth map. Then we merge the structured depth information into the skeleton features through a cross-attention mechanism. The key insight behind this approach is to enable each limb to incorporate spatial characteristics from other limbs, thereby facilitating a more nuanced understanding of limb interaction relationships. 
Given that pose information extracted from wild videos may contain errors, we utilize Conv3D to model temporal motion information, enhancing inter-frame connections and mitigating the impact of erroneous signals on individual frames.
%Our experiments demonstrate that by implementing the proposed depth-wise pose modulation, our model can effectively handle complex actions in videos.


% \subsection{Comparison with Prior Works}
% % Recently, several works have explored generating animated human characters with scene/object interaction capabilities. 
% AnchorCraft can produce videos of characters holding objects, which requires complex attention-based computations to inject multi-view object features and additionally extract depth conditions from reference videos, yet remains limited to generating front-facing videos under identical object contexts. In contrast, our approach directly leverages object information from reference videos as conditions, employing lightweight feature fusion. Moreover, our method demonstrates superior generalizability across diverse human-object interaction scenarios.
% MIMO propose a composition-based approach for generating scene-character interactions by synthesizing characters, backgrounds and occlusions. However, its background generation relies on an auxiliary algorithm\cite{propainter}, and its direct incorporation of depth-based occlusion processing encounters challenges in complex human-object interactions. Our method offers an end-to-end generation pipeline with robust performance across more varied and intricate scenarios.



\begin{figure*}[!t]
\begin{center}
	\setlength{\fboxrule}{0pt}
	\fbox{\includegraphics[width=0.99\linewidth]{./figure/fig4.pdf}}
\end{center}
\vspace{-0.6cm}
\caption{Qualitative Results. \textit{Animate Anyone 2} achieves consistent character animation while enabling the integration and interaction between characters and their environments, thereby realizing environment affordance.}
\vspace{-0.2cm}
\label{fig:vis}
\end{figure*}



\section{Experiments}

\subsection{Implementations}

To validate the generalizability of our method across more diverse scenarios, we curated a dataset of 100,000 character videos collected from the internet, encompassing a broader range of scene types, action categories, and human-object interaction cases. Experiments are conducted on 8 NVIDIA A100 GPUs. The training involves 100k steps with batch size of 8 and the video length in a batch is 16. 
%During training, The weight of VAE Encoder and Decoder are kept fixed, while the remaining components of the network are trained in an end-to-end manner. 
Video frames are cropped at consistent positions to ensure that the character is fully contained within the 16-frame sequence. The reference image is randomly sampled from the entire video sequence. We perform center cropping and remove the original background, compositing it with a new random background. This approach enables the model to automatically recognize characters within the image during inference without requiring additional segmentation, thereby mitigating potential accuracy limitations inherent in segmentation processes.

During long video inference, the video is segmented into multiple video clips, and inference is performed on each clip sequentially. Inspired by the motion frame technique in \cite{emo}, we utilize the final frame of the previous video clip as the temporal reference to guide the transition between clips. This strategy ensures smooth transitions between different video clips, preventing appearance texture discontinuities or blurriness. 







\subsection{Qualitative Results}

Fig.~\ref{fig:vis} demonstrates that our approach not only animates diverse characters with high-fidelity performance, but also achieves remarkably seamless visual integration and interaction with their surrounding environments. This substantiates the versatility and robustness of our method, underscoring its significant potential for widespread applications.


\subsection{Comparisons}

\noindent
\textbf{Metrics. }
We follow the previous evaluation metrics for character image animation. Specifically, for single-frame quality assessment, we employ PSNR\cite{psnr}, SSIM\cite{ssim}, and LPIPS\cite{lpips}. For video fidelity, we utilize the Frechet Video Distance (FVD)\cite{fvd}. 



\begin{table}
    \setlength{\tabcolsep}{4pt}
	\centering
	\resizebox{0.45\textwidth}{!}{
        \begin{tabular}{@{}ccccc@{}}
\toprule
Method         & SSIM $\uparrow$                      & PSNR $\uparrow$                     & LPIPS $\downarrow$                    & FVD  $\downarrow$                      \\ \midrule
MRAA \cite{mraa}          & 0.672           & 29.39        & 0.672           & 284.82                     \\
DisCo \cite{disco}         &  0.668        &     29.03        & 0.292         & 292.80                     \\
MagicAnimate \cite{magicanimate}  & 0.714         & 29.16           & 0.239        & 179.07                     \\
Animate Anyone \cite{aa} & 0.718          & 29.56        & 0.285           & 171.90                      \\
Champ* \cite{champ}         & 0.802          &  29.91         & 0.234         & 160.82 \\
UniAnimate* \cite{unianimate} & 0.811           & 30.77            & 0.231        & 148.06   \\
Ours  & 0.778           & 29.82            & 0.248        & 158.97   \\
Ours*           & \textbf{0.812}            & \textbf{30.82}            & \textbf{0.223}            & \textbf{144.65}            \\ \bottomrule
\end{tabular}
    }
    \vspace{-0.2cm}
	\caption{Quantitative comparison on Tiktok benchmark. * means utilizing other video data for pretraining. }
    \vspace{-0.3cm}
	\label{table:tiktok}
\end{table}





\noindent
\textbf{Evaluation on TikTok Dataset. }
We conduct experiments on the TikTok Benchmark\cite{tiktok}. In this dataset, the video backgrounds are static.
Existing character animation approaches typically synthesize target videos with both characters and backgrounds by a single reference image. To ensure a fair comparison, we adjust the configuration of our method: instead of using the ground truth background, we employ the background from the reference image as the environmental input. This modification allows all methods to generate outputs conditioned exclusively on a single reference image.
We implement two training settings of our approach: 1) trained exclusively on the Tiktok training set, and 2) first trained on our custom dataset and subsequently fine-tuned on the Tiktok training set. As shown in Tab.~\ref{table:tiktok}, when trained solely on the Tiktok training set, our method outperforms Magicanimate\cite{magicanimate} and Animate Anyone\cite{aa}. After incorporating pre-trained video data, our approach further surpasses Champ\cite{champ} and UniAnimate\cite{unianimate}, achieving state-of-the-art performance.







\begin{figure}[!t]
\begin{center}
	\setlength{\fboxrule}{0pt}
	\fbox{\includegraphics[width=0.95\linewidth]{./figure/fig5.pdf}}
\end{center}
\vspace{-0.7cm}
\caption{Qualitative comparion for character animation. We normalize the background to a uniform color. }
\vspace{-0.01cm}
\label{fig:com1}
\end{figure}



\begin{table}
    \setlength{\tabcolsep}{4pt}
	\centering
        \resizebox{0.45\textwidth}{!}{
        \begin{tabular}{@{}ccccc@{}}
\toprule
Method         & SSIM $\uparrow$                      & PSNR $\uparrow$                     & LPIPS $\downarrow$                    & FVD  $\downarrow$                      \\ \midrule
Animate Anyone\cite{aa} & 0.761          & 28.41        & 0.324           & 228.53                      \\
Champ\cite{champ}          & 0.771          &  28.69         & 0.294         & 205.79 \\
MimicMotion\cite{mimicmotion}  & 0.767           & 28.52            & 0.307        & 212.48   \\
Ours           & \textbf{0.809}            & \textbf{29.24}            & \textbf{0.259}            & \textbf{172.54}            \\ \bottomrule
\end{tabular}
    }
    \vspace{-0.2cm}
	\caption{Quantitative comparison on our dataset. Our approach demonstrates superior performance across generalized scenarios.}
    \vspace{-0.2cm}
	\label{table:propose}
\end{table}





\noindent
\textbf{Evaluation on Proposed Dataset. }
Due to the limitations of existing benchmarks\cite{dwnet,tiktok,mraa} that exhibit domain proximity, these datasets cannot effectively evaluate the generalizability of models across diverse scenarios. Following ~\cite{champ}, we establish a testset comprising 100 character videos from real-world scenarios to conduct additional evaluation. Since other methods cannot generate dynamic environment, we standardize the background of input images to a uniform color, thus isolating the impact of environment variations on the evaluation. 
For fair comparison, we finetune these methods on our custom training dataset. 
The quantitative comparison is shown in Tab.~\ref{table:propose}. Qualitative comparison is shown in Fig.\ref{fig:com1}.
Our results significantly outperform alternative approaches, which can be attributed to two key factors: (1) our proposed motion modeling demonstrates robust generalization across diverse motion patterns, and (2) our decoupled environment and character generation strategy enables the model to focus more precisely on character dynamics, mitigating interference from environment variations.





\noindent
\textbf{Evaluation for character-environment affordance. }
%The aforementioned comparisons solely focus on character animation itself without evaluating character-environment affordance.
We further evaluate the performance of character-environment affordance on our proposed dataset. 
We construct a baseline algorithm by directly compositing character animation results onto the original video background, creating a pseudo character-environment integration, similar to MovieCharacter \cite{moviecharacter}. we leverage ProPainter\cite{propainter} to inpaint the character region. 
Quantitative evaluation is presented in Tab.~\ref{table:base}. 
We conduct qualitative comparison illustrated in Fig.~\ref{fig:mimo}. Our approach demonstrates superior performance in terms of enhanced character-environment integration.
We also compare our method with MIMO\cite{mimo}, which is the most relevant method to our task setting. Due to the absence of public source code, we conduct a qualitative comparison focused on character-environment integration performance. The result of MIMO are obtained from its official ModelScope link$^*$. As illustrated in Fig.~\ref{fig:mimo}. From the first group of the visualization, it can be observed that due to MIMO's reliance on additional pre-processing algorithms for background inpainting, it tends to leave noticeable preprocessing artifacts and establish erroneous relationships between the background and the animated characters. In contrast, our proposed approach effectively mitigates these issues, enabling superior scene and character integration. The second group further illustrates MIMO's limitations in handling relatively complex human-object interaction scenarios, whereas our method demonstrates enhanced robustness in intricate scenes.

\renewcommand{\thefootnote}{}
\footnotetext{$^*$https://modelscope.cn/studios/iic/MIMO}


\begin{figure}[!t]
\begin{center}
	\setlength{\fboxrule}{0pt}
	\fbox{\includegraphics[width=0.99\linewidth]{./figure/fig6.pdf}}
\end{center}
\vspace{-0.7cm}
\caption{Qualitative comparion. Our method demonstrates superior environment integration and object interaction.}
\vspace{-0.1cm}
\label{fig:mimo}
\end{figure}



\begin{table}
    \setlength{\tabcolsep}{4pt}
	\centering
        \resizebox{0.36\textwidth}{!}{
        \newcommand{\boformat}[1]{$\mathbf{#1}$}
\newcommand{\blformat}[1]{\textcolor{blue}{$\mathbf{#1}$}}

\def\arraystretch{1.1}
\setlength{\tabcolsep}{0.5em} %

\begin{table}[t!] \centering
    \caption{Comparison of TD3 with existing dataset distillation techniques and heuristic sampling based methods.
    }
    \vspace{-8pt}
    \label{tab:baselines}
    \begin{center}
    \resizebox{0.48\textwidth}{!}{
    \begin{tabular}{cc|cc|cc|c}
        \toprule
        \multirow{2.8}{*}{Dataset} & \multirow{2.8}{*}{Metric} & \multicolumn{2}{c|}{Sampling} & \multicolumn{2}{c|}{Distillation} & \multirow{2.8}{*}{Full-Data} \\[3pt]
        
        \cmidrule{3-6}
        & & \multicolumn{1}{c}{Random.} & \multicolumn{1}{c|}{Longest.} & \multicolumn{1}{c}{Farzi} & \multicolumn{1}{c|}{\sampler} \\[2pt]
        
        \midrule
        \multirow{4}{*}{\STAB{Magazine\\\texttt{[30$\times$20]}}} 
        & HR@10 $\uparrow$     & 15.44 \std{$\pm$2.68} & 19.62 \std{$\pm$0.31} &      41.46 \std{$\pm$1.27} & \textbf{\ul{47.87} \std{$\pm$1.54}} & \textit{45.40} \\
        & HR@20 $\uparrow$     & 27.50 \std{$\pm$1.03} & 33.66 \std{$\pm$2.73} & \ul{58.70} \std{$\pm$0.13} & \textbf{\ul{61.17} \std{$\pm$1.83}} & \textit{56.98} \\
        & NDCG@10 $\uparrow$   &  7.75 \std{$\pm$1.16} &  9.58 \std{$\pm$0.58} &      24.27 \std{$\pm$0.20} & \textbf{\ul{27.90} \std{$\pm$0.88}} & \textit{25.63} \\
        & NDCG@20 $\uparrow$   & 10.75 \std{$\pm$0.77} & 13.10 \std{$\pm$0.30} & \ul{28.65} \std{$\pm$0.13} & \textbf{\ul{31.25} \std{$\pm$0.98}} & \textit{28.57} \\
        \midrule
        \multirow{4}{*}{\STAB{Epinions\\\texttt{[15$\times$30]}}} 
        & HR@10 $\uparrow$     & 10.67 \std{$\pm$0.90} & 10.24 \std{$\pm$0.21} &      18.99 \std{$\pm$0.30} & \textbf{\ul{19.86} \std{$\pm$0.10}} & \textit{19.13} \\
        & HR@20 $\uparrow$     & 20.25 \std{$\pm$1.25} & 20.01 \std{$\pm$0.41} &      29.82 \std{$\pm$0.39} & \textbf{\ul{31.06} \std{$\pm$0.19}} & \textit{30.09} \\
        & NDCG@10 $\uparrow$   &  4.93 \std{$\pm$0.36} &  4.79 \std{$\pm$0.06} & \ul{10.38} \std{$\pm$0.17} & \textbf{\ul{10.67} \std{$\pm$0.05}} & \textit{10.25} \\
        & NDCG@20 $\uparrow$   &  7.31 \std{$\pm$0.45} &  7.22 \std{$\pm$0.11} & \ul{13.09} \std{$\pm$0.07} & \textbf{\ul{13.49} \std{$\pm$0.08}} & \textit{13.00} \\
        \midrule
        \multirow{4}{*}{\STAB{ML-100k\\\texttt{[30$\times$50]}}} 
        & HR@10 $\uparrow$     & 10.85 \std{$\pm$2.30} & 13.36 \std{$\pm$0.45} & 62.92 \std{$\pm$1.39} & \textbf{66.14 \std{$\pm$1.16}} & \textit{68.93} \\
        & HR@20 $\uparrow$     & 21.49 \std{$\pm$3.98} & 25.59 \std{$\pm$0.64} & 77.84 \std{$\pm$0.30} & \textbf{81.37 \std{$\pm$0.43}} & \textit{83.78} \\
        & NDCG@10 $\uparrow$   &  4.81 \std{$\pm$1.10} &  5.97 \std{$\pm$0.32} & 34.92 \std{$\pm$0.71} & \textbf{38.56 \std{$\pm$0.86}} & \textit{40.97} \\
        & NDCG@20 $\uparrow$   &  7.48 \std{$\pm$1.28} &  9.02 \std{$\pm$0.36} & 38.72 \std{$\pm$0.40} & \textbf{42.44 \std{$\pm$0.72}} & \textit{44.76} \\
        \midrule
        \multirow{4}{*}{\STAB{ML-1M\\\texttt{[200$\times$50]}}} 
        & HR@10 $\uparrow$     & 15.88 \std{$\pm$0.22} & 16.60 \std{$\pm$0.51} & 38.01 \std{$\pm$0.98} & \textbf{70.52 \std{$\pm$0.36}} & \textit{79.32} \\
        & HR@20 $\uparrow$     & 28.09 \std{$\pm$0.31} & 30.93 \std{$\pm$0.45} & 56.10 \std{$\pm$1.24} & \textbf{82.52 \std{$\pm$0.25}} & \textit{87.60} \\
        & NDCG@10 $\uparrow$   &  7.40 \std{$\pm$0.11} &  7.77 \std{$\pm$0.16} & 19.85 \std{$\pm$0.49} & \textbf{45.93 \std{$\pm$0.37}} & \textit{58.82} \\
        & NDCG@20 $\uparrow$   & 10.47 \std{$\pm$0.10} & 11.36 \std{$\pm$0.14} & 24.41 \std{$\pm$0.55} & \textbf{48.97 \std{$\pm$0.30}} & \textit{60.93} \\
        \bottomrule
    \end{tabular}}
    \end{center}
\end{table}

    }
    \vspace{-0.2cm}
	\caption{Quantitative comparison with baseline on our dataset. Baseline refers to the pseudo character-environment integration.}
    \vspace{-0.3cm}
	\label{table:base}
\end{table}





\subsection{Ablation Study}

\noindent
\textbf{Environment Formulation. }
To demonstrate the effectiveness of our proposed environment formulation strategy, we explore alternative designs, including: 1) utilizing precise character masks from the source video, and 2) employing bounding box regions. Qualitative results are shown in Fig.~\ref{fig:aba1}. Using accurate masks can constrain the animated character's shape within the predefined mask boundaries, potentially causing appearance deformation and inconsistency. Conversely, adopting bounding box regions may introduce scene context distortions and fusion artifacts in the proximity of character. 
%By employing our proposed scene formulation strategy, we ensure robust character animation consistency, effectively mitigating potential biasing from source video characters. 
Our method demonstrates superior capability in learning flexible character generation and environmental completion, achieving both character consistency and seamless character-scene integration.

\begin{figure}[!t]
\begin{center}
	\setlength{\fboxrule}{0pt}
	\fbox{\includegraphics[width=1\linewidth]{./figure/fig7.pdf}}
\end{center}
\vspace{-0.7cm}
\caption{Ablation study of environment formulation.}
\vspace{-0.1cm}
\label{fig:aba1}
\end{figure}





\begin{figure}[!t]
\begin{center}
	\setlength{\fboxrule}{0pt}
	\fbox{\includegraphics[width=0.8\linewidth]{./figure/fig8.pdf}}
\end{center}
\vspace{-0.7cm}
\caption{Qualitative ablation of object modeling method.}
\vspace{-0.2cm}
\label{fig:aba2}
\end{figure}












\noindent
\textbf{Object Modeling. }
%We analyzed the two critical components in our proposed object injection mechanism.
We conduct a comparison of different object modeling approaches: directly merging object features with noise latents without employing spatial blending. Quantitative result is shown in Tab~\ref{table:ablation}. We further demonstrate the visualization results. As shown in Fig.~\ref{fig:aba2}, 
in complex interaction scenarios, it fails to comprehensively preserve the intrinsic features of interactive objects, resulting in local distortions and consequently misinterpreting their interaction relationships.
The second comparison reveals that the interactions between characters and objects exhibit an artificial stitching effect, which consequently compromises the naturalness of their interactive relationships.
%Our proposed method effectively learns the intricate interactions and fusion relationships between characters and objects, demonstrating superior performance.

\noindent
\textbf{Pose Modulation. }
We evaluate the effectiveness of our proposed pose modulation strategy. Quantitative result is presented in Tab~\ref{table:ablation}. Qualitative result is shown in Fig~\ref{fig:aba3}. Without employing the pose modulation method, character limb relationships may suffer from misalignment and spatial inconsistencies. Consequently, the model's capability to generate accurate and plausible character poses becomes severely constrained. In contrast, our proposed approach, by incorporating depth-aware information, can more effectively learn and capture the complex spatial relationships between limbs, enabling robust performance across diverse and challenging motion scenarios.







\begin{figure}[!t]
\begin{center}
	\setlength{\fboxrule}{0pt}
	\fbox{\includegraphics[width=0.8\linewidth]{./figure/fig9.pdf}}
\end{center}
\vspace{-0.7cm}
\caption{Qualitative ablation of pose modulation.}
\vspace{-0.01cm}
\label{fig:aba3}
\end{figure}



\begin{table}
    \setlength{\tabcolsep}{4pt}
	\centering
        \resizebox{0.45\textwidth}{!}{
        % \begin{table}[!t]
% \centering
% \scalebox{0.68}{
%     \begin{tabular}{ll cccc}
%       \toprule
%       & \multicolumn{4}{c}{\textbf{Intellipro Dataset}}\\
%       & \multicolumn{2}{c}{Rank Resume} & \multicolumn{2}{c}{Rank Job} \\
%       \cmidrule(lr){2-3} \cmidrule(lr){4-5} 
%       \textbf{Method}
%       &  Recall@100 & nDCG@100 & Recall@10 & nDCG@10 \\
%       \midrule
%       \confitold{}
%       & 71.28 &34.79 &76.50 &52.57 
%       \\
%       \cmidrule{2-5}
%       \confitsimple{}
%     & 82.53 &48.17
%        & 85.58 &64.91
     
%        \\
%        +\RunnerUpMiningShort{}
%     &85.43 &50.99 &91.38 &71.34 
%       \\
%       +\HyReShort
%         &- & -
%        &-&-\\
       
%       \bottomrule

%     \end{tabular}
%   }
% \caption{Ablation studies using Jina-v2-base as the encoder. ``\confitsimple{}'' refers using a simplified encoder architecture. \framework{} trains \confitsimple{} with \RunnerUpMiningShort{} and \HyReShort{}.}
% \label{tbl:ablation}
% \end{table}
\begin{table*}[!t]
\centering
\scalebox{0.75}{
    \begin{tabular}{l cccc cccc}
      \toprule
      & \multicolumn{4}{c}{\textbf{Recruiting Dataset}}
      & \multicolumn{4}{c}{\textbf{AliYun Dataset}}\\
      & \multicolumn{2}{c}{Rank Resume} & \multicolumn{2}{c}{Rank Job} 
      & \multicolumn{2}{c}{Rank Resume} & \multicolumn{2}{c}{Rank Job}\\
      \cmidrule(lr){2-3} \cmidrule(lr){4-5} 
      \cmidrule(lr){6-7} \cmidrule(lr){8-9} 
      \textbf{Method}
      & Recall@100 & nDCG@100 & Recall@10 & nDCG@10
      & Recall@100 & nDCG@100 & Recall@10 & nDCG@10\\
      \midrule
      \confitold{}
      & 71.28 & 34.79 & 76.50 & 52.57 
      & 87.81 & 65.06 & 72.39 & 56.12
      \\
      \cmidrule{2-9}
      \confitsimple{}
      & 82.53 & 48.17 & 85.58 & 64.91
      & 94.90&78.40 & 78.70& 65.45
       \\
      +\HyReShort{}
       &85.28 & 49.50
       &90.25 & 70.22
       & 96.62&81.99 & \textbf{81.16}& 67.63
       \\
      +\RunnerUpMiningShort{}
       % & 85.14& 49.82
       % &90.75&72.51
       & \textbf{86.13}&\textbf{51.90} & \textbf{94.25}&\textbf{73.32}
       & \textbf{97.07}&\textbf{83.11} & 80.49& \textbf{68.02}
       \\
   %     +\RunnerUpMiningShort{}
   %    & 85.43 & 50.99 & 91.38 & 71.34 
   %    & 96.24 & 82.95 & 80.12 & 66.96
   %    \\
   %    +\HyReShort{} old
   %     &85.28 & 49.50
   %     &90.25 & 70.22
   %     & 96.62&81.99 & 81.16& 67.63
   %     \\
   % +\HyReShort{} 
   %     % & 85.14& 49.82
   %     % &90.75&72.51
   %     & 86.83&51.77 &92.00 &72.04
   %     & 97.07&83.11 & 80.49& 68.02
   %     \\
      \bottomrule

    \end{tabular}
  }
\caption{\framework{} ablation studies. ``\confitsimple{}'' refers using a simplified encoder architecture. \framework{} trains \confitsimple{} with \RunnerUpMiningShort{} and \HyReShort{}. We use Jina-v2-base as the encoder due to its better performance.
}
\label{tbl:ablation}
\end{table*}
    }
    \vspace{-0.2cm}
	\caption{Quantitative ablation study.}
    \vspace{-0.1cm}
	\label{table:ablation}
\end{table}



% \begin{table}
%     \setlength{\tabcolsep}{4pt}
% 	\centering
%         \resizebox{0.4\textwidth}{!}{
%         \begin{table}[H]
\caption{Comparison of ClipScore with the selected LoRA integration methods under different number of LoRAs.}
\label{ablation2}
\begin{center}
\scalebox{0.8}{
\begin{tabular}{c|cccc}
\toprule
Model & ClipScore (N=2) & ClipScore (N=3) & ClipScore (N=4) & ClipScore (N=5) \\
\midrule
CMLoRA ($\text{Cache}_{D}$) & \textcolor{darkgreen}{\textbf{35.422}} & \textcolor{darkgreen}{\textbf{35.215}} & \textcolor{darkgreen}{\textbf{35.208}} & \textcolor{darkgreen}{\textbf{34.341}} \\
CMLoRA ($\text{Cache}_{c=2}$) & 35.241 & 34.953 & 34.910 & 34.141 \\
CMLoRA ($\text{Cache}_{c=3}$) & 34.825 & 34.628 & 34.720 & 33.885 \\
CMLoRA ($\text{Cache}_{c=5}$) & 34.499 & 34.864 & 34.516 & 33.174 \\
\bottomrule
\end{tabular}}
\end{center}
\end{table}

%     }
%     \vspace{-0.2cm}
% 	\caption{Quantitative ablation of object modeling method.}
%     \vspace{-0.1cm}
% 	\label{table:ablation2}
% \end{table}


% \begin{table}
%     \setlength{\tabcolsep}{4pt}
% 	\centering
%         \resizebox{0.4\textwidth}{!}{
%         \input{table/ablation3}
%     }
%     \vspace{-0.2cm}
% 	\caption{Quantitative ablation of pose modulation.}
%     \vspace{-0.2cm}
% 	\label{table:ablation3}
% \end{table}




\section{Discussion and Conclusion}
\noindent
\textbf{Limitations. }
Our approach may introduce visual artifacts when dealing with complex hand-object interactions that occupy a relatively small pixel region. In intricate human-object interactions, deformation artifacts may emerge when source and target characters exhibit substantial shape discrepancies. The performance of object interaction is also influenced by SAM's segmentation capabilities.
%In intricate human-object interactions, pose retargeting may potentially compromise the fine-grained interaction details. When source and target characters exhibit substantial shape discrepancies, deformation artifacts can emerge.

\noindent
\textbf{Potential Impact. }
The proposed method may be used to produce fake videos of individuals, which can be detected using some anti-spoofing techniques\cite{anti_color,anti_deep,anti_search,demamba,dormant}. 

\noindent
\textbf{Conclusion. }
In this paper, we introduce \textit{Animate Anyone 2}, a novel framework that enables character animation to exhibit environment affordance. We extract environmental information from driving videos, enabling the animated character to preserve its original environment. We propose a novel environment formulation and object injection strategy, facilitating seamless character-environment integration. Moreover, we propose pose modulation that empowers the model to robustly handle diverse motion patterns. 
Experimental results demonstrate that \textit{Animate Anyone 2} achieves high-fidelity generation performance.


{
    \small
    \bibliographystyle{ieeenat_fullname}
    \bibliography{main}
}

% WARNING: do not forget to delete the supplementary pages from your submission 
% \clearpage
\pagenumbering{gobble}
\maketitlesupplementary

\section{Additional Results on Embodied Tasks}

To evaluate the broader applicability of our EgoAgent's learned representation beyond video-conditioned 3D human motion prediction, we test its ability to improve visual policy learning for embodiments other than the human skeleton.
Following the methodology in~\cite{majumdar2023we}, we conduct experiments on the TriFinger benchmark~\cite{wuthrich2020trifinger}, which involves a three-finger robot performing two tasks: reach cube and move cube. 
We freeze the pretrained representations and use a 3-layer MLP as the policy network, training each task with 100 demonstrations.

\begin{table}[h]
\centering
\caption{Success rate (\%) on the TriFinger benchmark, where each model's pretrained representation is fixed, and additional linear layers are trained as the policy network.}
\label{tab:trifinger}
\resizebox{\linewidth}{!}{%
\begin{tabular}{llcc}
\toprule
Methods       & Training Dataset & Reach Cube & Move Cube \\
\midrule
DINO~\cite{caron2021emerging}         & WT Venice        & 78.03     & 47.42     \\
DoRA~\cite{venkataramanan2023imagenet}          & WT Venice        & 81.62     & 53.76     \\
DoRA~\cite{venkataramanan2023imagenet}          & WT All           & 82.40     & 48.13     \\
\midrule
EgoAgent-300M & WT+Ego-Exo4D      & 82.61    & 54.21      \\
EgoAgent-1B   & WT+Ego-Exo4D      & \textbf{85.72}      & \textbf{57.66}   \\
\bottomrule
\end{tabular}%
}
\end{table}

As shown in Table~\ref{tab:trifinger}, EgoAgent achieves the highest success rates on both tasks, outperforming the best models from DoRA~\cite{venkataramanan2023imagenet} with increases of +3.32\% and +3.9\% respectively.
This result shows that by incorporating human action prediction into the learning process, EgoAgent demonstrates the ability to learn more effective representations that benefit both image classification and embodied manipulation tasks.
This highlights the potential of leveraging human-centric motion data to bridge the gap between visual understanding and actionable policy learning.



\section{Additional Results on Egocentric Future State Prediction}

In this section, we provide additional qualitative results on the egocentric future state prediction task. Additionally, we describe our approach to finetune video diffusion model on the Ego-Exo4D dataset~\cite{grauman2024ego} and generate future video frames conditioned on initial frames as shown in Figure~\ref{fig:opensora_finetune}.

\begin{figure}[b]
    \centering
    \includegraphics[width=\linewidth]{figures/opensora_finetune.pdf}
    \caption{Comparison of OpenSora V1.1 first-frame-conditioned video generation results before and after finetuning on Ego-Exo4D. Fine-tuning enhances temporal consistency, but the predicted pixel-space future states still exhibit errors, such as inaccuracies in the basketball's trajectory.}
    \label{fig:opensora_finetune}
\end{figure}

\subsection{Visualizations and Comparisons}

More visualizations of our method, DoRA, and OpenSora in different scenes (as shown in Figure~\ref{fig:supp pred}). For OpenSora, when predicting the states of $t_k$, we use all the ground truth frames from $t_{0}$ to $t_{k-1}$ as conditions. As OpenSora takes only past observations as input and neglects human motion, it performs well only when the human has relatively small motions (see top cases in Figure~\ref{fig:supp pred}), but can not adjust to large movements of the human body or quick viewpoint changes (see bottom cases in Figure~\ref{fig:supp pred}).

\begin{figure*}
    \centering
    \includegraphics[width=\linewidth]{figures/supp_pred.pdf}
    \caption{Retrieval and generation results for egocentric future state prediction. Correct and wrong retrieval images are marked with green and red boundaries, respectively.}
    \label{fig:supp pred}
\end{figure*}

\begin{figure*}[t]
    \centering
    \includegraphics[width=0.9\linewidth]{figures/motion_prediction.pdf}
    \vspace{-0.5mm}
    \caption{Motion prediction results in scenes with minor changes in observation.}
    \vspace{-1.5mm}
    \label{fig:motion_prediction}
\end{figure*}

\subsection{Finetuning OpenSora on Ego-Exo4D}

OpenSora V1.1~\cite{opensora}, initially trained on internet videos and images, produces severely inconsistent results when directly applied to infer future videos on the Ego-Exo4D dataset, as illustrated in Figure~\ref{fig:opensora_finetune}.
To address the gap between general internet content and egocentric video data, we fine-tune the official checkpoint on the Ego-Exo4D training set for 50 epochs.
OpenSora V1.1 proposed a random mask strategy during training to enable video generation by image and video conditioning. We adopted the default masking rate, which applies: 75\% with no masking, 2.5\% with random masking of 1 frame to 1/4 of the total frames, 2.5\% with masking at either the beginning or the end for 1 frame to 1/4 of the total frames, and 5\% with random masking spanning 1 frame to 1/4 of the total frames at both the beginning and the end.

As shown in Fig.~\ref{fig:opensora_finetune}, despite being trained on a large dataset, OpenSora struggles to generalize to the Ego-Exo4D dataset, producing future video frames with minimal consistency relative to the conditioning frame. While fine-tuning improves temporal consistency, the moving trajectories of objects like the basketball and soccer ball still deviate from realistic physical laws. Compared with our feature space prediction results, this suggests that training world models in a reconstructive latent space is more challenging than training them in a feature space.


\section{Additional Results on 3D Human Motion Prediction}

We present additional qualitative results for the 3D human motion prediction task, highlighting a particularly challenging scenario where egocentric observations exhibit minimal variation. This scenario poses significant difficulties for video-conditioned motion prediction, as the model must effectively capture and interpret subtle changes. As demonstrated in Fig.~\ref{fig:motion_prediction}, EgoAgent successfully generates accurate predictions that closely align with the ground truth motion, showcasing its ability to handle fine-grained temporal dynamics and nuanced contextual cues.

\section{OpenSora for Image Classification}

In this section, we detail the process of extracting features from OpenSora V1.1~\cite{opensora} (without fine-tuning) for an image classification task. Following the approach of~\cite{xiang2023denoising}, we leverage the insight that diffusion models can be interpreted as multi-level denoising autoencoders. These models inherently learn linearly separable representations within their intermediate layers, without relying on auxiliary encoders. The quality of the extracted features depends on both the layer depth and the noise level applied during extraction.


\begin{table}[h]
\centering
\caption{$k$-NN evaluation results of OpenSora V1.1 features from different layer depths and noising scales on ImageNet-100. Top1 and Top5 accuracy (\%) are reported.}
\label{tab:opensora-knn}
\resizebox{0.95\linewidth}{!}{%
\begin{tabular}{lcccccc}
\toprule
\multirow{2}{*}{Timesteps} & \multicolumn{2}{c}{First Layer} & \multicolumn{2}{c}{Middle Layer} & \multicolumn{2}{c}{Last Layer} \\
\cmidrule(r){2-3}   \cmidrule(r){4-5}  \cmidrule(r){6-7}  & Top1           & Top5           & Top1            & Top5           & Top1           & Top5          \\
\midrule
32        &  6.10           & 18.20             & 34.04               & 59.50             & 30.40             & 55.74             \\
64        & 6.12              & 18.48              & 36.04               & 61.84              & 31.80         & 57.06         \\
128       & 5.84             & 18.14             & 38.08               & 64.16              & 33.44       & 58.42 \\
256       & 5.60             & 16.58              & 30.34               & 56.38              &28.14          & 52.32        \\
512       & 3.66              & 11.70            & 6.24              & 17.62              & 7.24              & 19.44  \\ 
\bottomrule
\end{tabular}%
}
\end{table}

As shown in Table~\ref{tab:opensora-knn}, we first evaluate $k$-NN classification performance on the ImageNet-100 dataset using three intermediate layers and five different noise scales. We find that a noise timestep of 128 yields the best results, with the middle and last layers performing significantly better than the first layer.
We then test this optimal configuration on ImageNet-1K and find that the last layer with 128 noising timesteps achieves the best classification accuracy.

\section{Data Preprocess}
For egocentric video sequences, we utilize videos from the Ego-Exo4D~\cite{grauman2024ego} and WT~\cite{venkataramanan2023imagenet} datasets.
The original resolution of Ego-Exo4D videos is 1408×1408, captured at 30 fps. We sample one frame every five frames and use the original resolution to crop local views (224×224) for computing the self-supervised representation loss. For computing the prediction and action loss, the videos are downsampled to 224×224 resolution.
WT primarily consists of 4K videos (3840×2160) recorded at 60 or 30 fps. Similar to Ego-Exo4D, we use the original resolution and downsample the frame rate to 6 fps for representation loss computation.
As Ego-Exo4D employs fisheye cameras, we undistort the images to a pinhole camera model using the official Project Aria Tools to align them with the WT videos.

For motion sequences, the Ego-Exo4D dataset provides synchronized 3D motion annotations and camera extrinsic parameters for various tasks and scenes. While some annotations are manually labeled, others are automatically generated using 3D motion estimation algorithms from multiple exocentric views. To maximize data utility and maintain high-quality annotations, manual labels are prioritized wherever available, and automated annotations are used only when manual labels are absent.
Each pose is converted into the egocentric camera's coordinate system using transformation matrices derived from the camera extrinsics. These matrices also enable the computation of trajectory vectors for each frame in a sequence. Beyond the x, y, z coordinates, a visibility dimension is appended to account for keypoints invisible to all exocentric views. Finally, a sliding window approach segments sequences into fixed-size windows to serve as input for the model. Note that we do not downsample the frame rate of 3D motions.

\section{Training Details}
\subsection{Architecture Configurations}
In Table~\ref{tab:arch}, we provide detailed architecture configurations for EgoAgent following the scaling-up strategy of InternLM~\cite{team2023internlm}. To ensure the generalization, we do not modify the internal modules in InternML, \emph{i.e.}, we adopt the RMSNorm and 1D RoPE. We show that, without specific modules designed for vision tasks, EgoAgent can perform well on vision and action tasks.

\begin{table}[ht]
  \centering
  \caption{Architecture configurations of EgoAgent.}
  \resizebox{0.8\linewidth}{!}{%
    \begin{tabular}{lcc}
    \toprule
          & EgoAgent-300M & EgoAgent-1B \\
          \midrule
    Depth & 22    & 22 \\
    Embedding dim & 1024  & 2048 \\
    Number of heads & 8     & 16 \\
    MLP ratio &    8/3   & 8/3 \\
    $\#$param.  & 284M & 1.13B \\
    \bottomrule
    \end{tabular}%
    }
  \label{tab:arch}%
\end{table}%

Table~\ref{tab:io_structure} presents the detailed configuration of the embedding and prediction modules in EgoAgent, including the image projector ($\text{Proj}_i$), representation head/state prediction head ($\text{MLP}_i$), action projector ($\text{Proj}_a$) and action prediction head ($\text{MLP}_a$).
Note that the representation head and the state prediction head share the same architecture but have distinct weights.

\begin{table}[t]
\centering
\caption{Architecture of the embedding ($\text{Proj}_i$, $\text{Proj}_a$) and prediction ($\text{MLP}_i$, $\text{MLP}_a$) modules in EgoAgent. For details on module connections and functions, please refer to Fig.~2 in the main paper.}
\label{tab:io_structure}
\resizebox{\linewidth}{!}{%
\begin{tabular}{lcl}
\toprule
       & \multicolumn{1}{c}{Norm \& Activation} & \multicolumn{1}{c}{Output Shape}  \\
\midrule
\multicolumn{3}{l}{$\text{Proj}_i$ (\textit{Image projector})} \\
\midrule
Input image  & -          & 3$\times$224$\times$224 \\
Conv 2D (16$\times$16) & -       & Embedding dim$\times$14$\times$14    \\
\midrule
\multicolumn{3}{l}{$\text{MLP}_i$ (\textit{State prediction head} \& \textit{Representation head)}} \\
\midrule
Input embedding  & -          & Embedding dim \\
Linear & GELU       & 2048          \\
Linear & GELU       & 2048          \\
Linear & -          & 256           \\
Linear & -          & 65536     \\
\midrule
\multicolumn{3}{l}{$\text{Proj}_a$ (\textit{Action projector})} \\
\midrule
Input pose sequence  & -          & 4$\times$5$\times$17 \\
Conv 2D (5$\times$17) & LN, GELU   & Embedding dim$\times$1$\times$1    \\
\midrule
\multicolumn{3}{l}{$\text{MLP}_a$ (\textit{Action prediction head})} \\
\midrule
Input embedding  & -          & Embedding dim$\times$1$\times$1 \\
Linear & -          & 4$\times$5$\times$17     \\
\bottomrule
\end{tabular}%
}
\end{table}


\subsection{Training Configurations}
In Table~\ref{tab:training hyper}, we provide the detailed training hyper-parameters for experiments in the main manuscripts.

\begin{table}[ht]
  \centering
  \caption{Hyper-parameters for training EgoAgent.}
  \resizebox{0.86\linewidth}{!}{%
    \begin{tabular}{lc}
    \toprule
    Training Configuration & EgoAgent-300M/1B \\
    \midrule
    Training recipe: &  \\
    optimizer & AdamW~\cite{loshchilov2017decoupled} \\
    optimizer momentum & $\beta_1=0.9, \beta_2=0.999$ \\
    \midrule
    Learning hyper-parameters: &  \\
    base learning rate & 6.0E-04 \\
    learning rate schedule & cosine \\
    base weight decay & 0.04 \\
    end weight decay & 0.4 \\
    batch size & 1920 \\
    training iters & 72,000 \\
    lr warmup iters & 1,800 \\
    warmup schedule & linear \\
    gradient clip & 1.0 \\
    data type & float16 \\
    norm epsilon & 1.0E-06 \\
    \midrule
    EMA hyper-parameters: &  \\
    momentum & 0.996 \\
    \bottomrule
    \end{tabular}%
    }
  \label{tab:training hyper}%
\end{table}%

\clearpage


\end{document}
