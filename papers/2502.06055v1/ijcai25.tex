%%%% ijcai25.tex

%\typeout{IJCAI--25 Instructions for Authors}

% These are the instructions for authors for IJCAI-25.

\documentclass{article}
\pdfpagewidth=8.5in
\pdfpageheight=11in

% The file ijcai25.sty is a copy from ijcai22.sty
% The file ijcai22.sty is NOT the same as previous years'
\usepackage{ijcai25}
\usepackage{xcolor}

\newcommand{\CB}[1]{\textbf{\textcolor{red}{CB: #1}}}
\newcommand{\BL}[1]{\textbf{\textcolor{blue}{BL: #1}}}

% Use the postscript times font!
\usepackage{times}
\usepackage{soul}
\usepackage{url}
\usepackage[hidelinks]{hyperref}
\usepackage[utf8]{inputenc}
\usepackage[small]{caption}
\usepackage{graphicx}
\usepackage{amsmath,amsthm,amstext,amssymb}
\usepackage{mathtools}
\usepackage{booktabs}
\usepackage{algorithm}
\usepackage{algorithmic}
\usepackage{xspace}
\usepackage{comment}
\usepackage{multirow} 
\usepackage{float}
\usepackage[switch]{lineno}
%\newcommand*{\Comb}[2]{{}^{#1}C_{#2}} % CB comment: this is a significantly less common notation
\newcommand*{\Comb}[2]{\binom{#1}{#2}}

\newcommand\MC{\textsc{Math\-Check}\xspace}
\newcommand\AMS{\textsc{Alpha\-Maple\-SAT}\xspace}
\newcommand{\TT}{\mathit{T\mspace{-2.5mu}T}}
\newcommand{\citein}[1]{\citeauthor{#1}~(\citeyear{#1})}

% Comment out this line in the camera-ready submission
%\linenumbers

\urlstyle{same}

% the following package is optional:
%\usepackage{latexsym}

% See https://www.overleaf.com/learn/latex/theorems_and_proofs
% for a nice explanation of how to define new theorems, but keep
% in mind that the amsthm package is already included in this
% template and that you must *not* alter the styling.
\newtheorem{example}{Example}
\newtheorem{theorem}{Theorem}
\newtheorem{lemma}{Lemma}

% Following comment is from ijcai97-submit.tex:
% The preparation of these files was supported by Schlumberger Palo Alto
% Research, AT\&T Bell Laboratories, and Morgan Kaufmann Publishers.
% Shirley Jowell, of Morgan Kaufmann Publishers, and Peter F.
% Patel-Schneider, of AT\&T Bell Laboratories collaborated on their
% preparation.

% These instructions can be modified and used in other conferences as long
% as credit to the authors and supporting agencies is retained, this notice
% is not changed, and further modification or reuse is not restricted.
% Neither Shirley Jowell nor Peter F. Patel-Schneider can be listed as
% contacts for providing assistance without their prior permission.

% To use for other conferences, change references to files and the
% conference appropriate and use other authors, contacts, publishers, and
% organizations.
% Also change the deadline and address for returning papers and the length and
% page charge instructions.
% Put where the files are available in the appropriate places.


% PDF Info Is REQUIRED.

% Please leave this \pdfinfo block untouched both for the submission and
% Camera Ready Copy. Do not include Title and Author information in the pdfinfo section
\pdfinfo{
/TemplateVersion (IJCAI.2025.0)
}

% Testing different titles
\title{Verified Certificates via SAT and Computer Algebra Systems for the \\ Ramsey $\boldsymbol{R(3,8)}$ and $\boldsymbol{R(3,9)}$ Problems}
%\title{Verifying Ramsey Problems via SAT and Computer Algebra}
%\title{A Verified SAT Solver + Computer Algebra System Tool for Ramsey $\boldsymbol{R(3,8)}$ and $\boldsymbol{R(3,9)}$ Problems}

% Multiple author syntax (remove the single-author syntax above and the \iffalse ... \fi here)

%\author{Anonymous}

\author{
Zhengyu Li$^1$\and
Conor Duggan$^2$\and
Curtis Bright$^3$\And
Vijay Ganesh$^1$\\
\affiliations
$^1$Georgia Institute of Technology, USA\\
$^2$University of Waterloo, Canada\\
$^3$University of Windsor, Canada
\emails
brian.li@gatech.edu,
c4duggan@uwaterloo.ca,
cbright@uwindsor.ca,
vganesh45@gatech.edu
}

\begin{document}

\maketitle

\begin{abstract}
The Ramsey problem $R(3,k)$ seeks to determine the smallest value of $n$ such that any red/blue edge coloring of the complete graph on $n$ vertices must either contain a blue triangle (3-clique) or a red clique of size~$k$.
Despite its significance, many previous computational results for the Ramsey $R(3,k)$ problem such as $R(3,8)$ and $R(3,9)$ lack formal verification.
To address this issue, we use the software \MC to generate certificates for Ramsey problems $R(3,8)$ and $R(3,9)$ (and symmetrically $R(8,3)$ and $R(9,3)$) by integrating a Boolean satisfiability (SAT) solver with a computer algebra system (CAS\@).
Our SAT+CAS approach significantly outperforms traditional SAT-only methods, demonstrating a significant improvement in runtime. For instance, our SAT+CAS approach solves $R(8,3)$ sequentially in 18.5 hours, while a SAT-only approach using the state-of-the-art CaDiCaL solver times out after 7 days. Additionally, in order to be able to scale to harder Ramsey problem like $R(9,3)$ we further optimized our SAT+CAS tool using a parallelized cube-and-conquer approach. Our results provide the first independently verifiable certificates for these Ramsey numbers, ensuring both correctness and completeness of the exhaustive search process of our SAT+CAS tool.
\end{abstract}

\section{Introduction}

Ramsey Theory, introduced by Frank P. Ramsey in \textit{On a problem of formal logic}~\cite{ramsey1987problem}, studies the existence of ordered substructures within sufficiently large structures. The classical Ramsey problem seeks the smallest integer $n$ (known as the Ramsey number) such that any red/blue edge coloring of the complete graph on $n$ vertices must contain either a blue triangle or a red $k$-clique. This problem is often framed as the ``party problem'': determining the minimum number of guests required to ensure that either $p$ guests all know each other or $q$ guests are mutual strangers. Despite its simple formulation, computing Ramsey numbers is notoriously difficult, with only nine non-trivial values known to date, despite extensive research~\cite{Ramseynumbers}.

Most contemporary methods of finding non-trivial Ramsey numbers rely heavily on the use of computer programs such as nauty~\cite{nauty2014} to enumerate Ramsey graphs exhaustively. However, nauty cannot generate a formal certificate certifying that the enumeration is indeed exhaustive, thus raising the possibility that these results may have errors in them. In this paper, we apply the software \MC~\cite{bright2016mathcheck2} that combines satisfiability solvers with computer algebra systems (SAT+CAS)~\cite{Bright2022} to not only produce formal certificates for Ramsey numbers but also achieve significant speedups compared to SAT-only approaches. A key component of our implementation is the use of the IPASIR-UP interface~\cite{fazekas2023ipasir}, which facilitates seamless integration of external learned clause addition within modern SAT solvers. Specifically, we leverage IPASIR-UP to implement the CaDiCaL+CAS framework as part of \MC. 
In this work, we focus on providing certificates of completeness for the exhaustive search components required to determine $R(3, k)$ and $R(k,3)$. It is important to emphasize that our approach does not constitute a formal proof of the Ramsey number. Instead, our certificates ensure correctness and completeness for the parts of the proof that involve exhaustive graph enumeration. 

% To follow the established convention in Ramsey theory, this paper begins by discussing the Ramsey number \( R(3, k) \). However, in later sections, we transition to \( R(k, 3) \), as it aligns better with our computational approach. Importantly, we remind the reader in the conclusion that \( R(3, k) = R(k, 3) \), as these are symmetric by definition.


\subsection{Our Contributions}

\textbf{1) Certified Ramsey Number:} In this paper, we extend the well-known SAT+CAS tool \MC~\cite{zulkoski2015mathcheck} (see Figure~\ref{fig:pipeline}) to solve Ramsey problems of type $R(3,k)$ and $R(k,3)$, and provide certificates of correctness.\footnote{All results can be reproduced at \url{https://github.com/ConDug/MathCheckRamsey}.} These certificates ensure the correctness of the exhaustive search process of our tool.\footnote{We cannot claim complete formal proofs of these Ramsey numbers since their SAT encodings have not been formally verified.} We verify the values of two Ramsey numbers for $k= 8$~\cite{mckay1992value} and $k= 9$~\cite{grinstead1982ramsey}, whose proofs rely heavily on graph enumeration. To our knowledge, these are the only two remaining known Ramsey numbers that have yet to be verified.

\noindent\textbf{2) Speedup over SAT-only approaches:} We show that our sequential and parallel SAT+CAS tools significantly outperform SAT-only approaches (see Table~\ref{tbl:sec6}). This speedup is achieved through the isomorph-free exhaustive generation technique employed by \MC, which reduces the search space by dynamically blocking symmetric branches. 

\noindent\textbf{3) Parallel Cube-and-Conquer SAT+CAS \MC:} We introduce a parallel cube-and-conquer SAT+CAS tool by integrating \AMS~\cite{alphamaplesat} as the cubing solver and \MC as the conquering solver. We solved and certified $R(8,3) = 28$ in 8 hours of wall clock time, and in doing so
decreased the total CPU time spent solving to 6.2 hours (while the sequential version spent 18.5 hours just solving), and solved and verified $R(9,3) = 36$ in 26 hours (while the sequential version timed out after 7 days).

\section{Preliminaries}

% We begin by introducing the SAT+CAS paradigm for graph enumeration, highlighting its advantages over traditional SAT-only approaches. We then present an overview of Ramsey problems, including key definitions and notations. Finally, we discuss the importance of formal verification in computer-assisted proofs.

\subsection{SAT+CAS for Combinatorial Problems}

A conflict-driven clause learning (CDCL) satisfiability (SAT) solver takes as input a Boolean formula in conjunctive normal form (CNF) and determines whether there exists an assignment of variables such that the formula evaluates to True, in which case the formula is satisfiable (SAT)\@. 
Otherwise, the formula is unsatisfiable (UNSAT)\@.
Thanks to the rich advancement made by the SAT community, state-of-the-art CDCL SAT solvers can solve many instances with millions of variables efficiently~\cite{ganesh2020unreasonable}.
However, SAT solvers face challenges when solving hard combinatorial problems such as the Ramsey problem due to the large amount of symmetry in the search space.

Computer Algebra Systems (CASs) such as Maple, Mathematica, Magma, and SageMath are storehouses of mathematical knowledge and contain state-of-the-art algorithms from a broad range of mathematical areas. Therefore, many mathematical properties can be easily expressed in a CAS, whereas an off-the-shelf SAT solver is limited to expressions in Boolean logic.
%\CB{Avoid one-sentence paragraphs.}

Both SAT and CAS have their drawbacks when solving combinatorial problems. SAT solvers can perform scalable searches but lack the mathematical domain knowledge required to prune out symmetries in the search space. On the other hand, CASs have rich mathematical capabilities but lack scalability when dealing with enormous search spaces.
To combine the best of both worlds, we leverage \MC to dynamically provide mathematical context to the SAT solver to only enumerate non-isomorphic graphs in the search space.
More specifically, we use a CAS to generate blocking clauses that are passed to the SAT solver dynamically via a programmatic interface~\cite{vijayprogram}.
These clauses block the SAT solver from exploring symmetric branches of the search tree.
We elaborate on this technique in detail in Section~\ref{sec:orderly}. %on orderly generation using SAT+CAS\@.
The SAT+CAS paradigm has been applied to solving hard combinatorial problems such as the Williamson conjecture~\cite{williamson}, Lam's problem~\cite{Lams}, Rota's basis conjecture~\cite{kirchweger2022sat}, the Erd\H{o}s-Faber-Lovász Conjecture~\cite{kirchweger2023sat}, the Kochen--Specker problem~\cite{li2023sat,kssms}, and integer factorization~\cite{Ajani2024}.
%\CB{Cite work of otheres, especially if they are likely to review the paper.}
We show that SAT+CAS is orders of magnitude faster than a SAT-only solver without compromising the verifiability of the result.
%For example, for $R(3,7)$ on 23 vertices, SAT+CAS solved the problem in 23 seconds and the SAT solver alone could not solve it in 24 hours on an i7-9750 processor running at 2.60~GHz with 8~GiB of RAM.

\begin{table}
\centering
\begin{tabular}{cccc}
$R(p,q)$ & $p=3$ & $p=4$ & $p=5$ \\ \hline
$q=3$ & 6 & & \\
$q=4$ & 9 & 18 & \\
$q=5$ & 14 & 25 & 43--46 \\
$q=6$ & 18 & 36--40 & 59--85 \\
$q=7$ & 23 & 49--58 & 80--133 \\
$q=8$ & \textbf{28} & 59--79 & 101--193 \\
$q=9$ & \textbf{36} & 73--105 & 133--282 \\
\end{tabular}
\caption{Exact Values and Bounds for Ramsey Numbers $R(p, q)$. The highlighted values of $R(3,8)$ and $R(3,9)$ are the Ramsey numbers verified in this paper. Some values are excluded from the table since $R(p, q) = R(q, p)$.}
\label{tbl:ramsey}
\end{table}


\subsection{Ramsey Problems}
The Ramsey Theorem states that for every $p$, $q \in \mathbb{Z}$, there exists an $n \in \mathbb{Z}$ such that every graph of order $n$ contains either a $p$-clique or an independent set of size $q$.
An $m$-clique is a complete subgraph of order $m$ and an independent set is a subset of mutually unconnected vertices.
The Ramsey problem is defined as finding the smallest integer $n$, denoted as $R(p, q)$, for some given input $p$ and~$q$. 
%This $n$ is not necessarily $R(p, q)$. 
A common and equivalent reformulation is as follows:
for every $p$, $q \in \mathbb{Z}$, there exists an $n \in \mathbb{Z}$ such that 
any red/blue coloring of the edges of the complete graph of order $n$, denoted $K_n$, contains a blue monochromatic $p$-clique or a red monochromatic $q$-clique. 
A $(p,q)$-graph is a graph without a $p$-clique and without an independent set of size~$q$.  
$(p,q;n)$-graphs and $(p,q;n;e)$-graphs are $(p,q)$-graphs on $n$ vertices and $(p,q)$-graphs on $n$ vertices and with $e$ edges, respectively.
All graphs are assumed to be simple and undirected unless stated otherwise.

\subsection{Correctness of Results}

Correctness of results is a long-standing problem in the field of computer-assisted proofs, particularly for results that require extensive enumeration
that cannot be checked by hand~\cite{Lam1990}.
Verification is of utmost importance as it provides a formal guarantee that the result is correct. Without verification, one has to trust the correctness of the program, and this could allow undetected software or administrative errors to compromise the validity of the proof.
For example, recent work uncovered consistency issues in previous computational 
resolutions of Lam's problem, highlighting the difficulty of relying on special purpose search code for nonexistent results~\cite{Lams}. For prominent combinatorial problems such as Ramsey problems that rely on extensive computation, formal verification is crucial in order to trust the results.

\begin{figure}
  \centering
  \includegraphics[scale=0.5]{RamseyMathCheck.pdf}
  \caption{Flowchart of the parallelized tool. \AMS is used as the cubing solver, and CaDiCaL + CAS is used as the conquering solver}
  \label{fig:pipeline}
\end{figure}

\section{Previous Work}

\subsection{Classical Ramsey Numbers}

In April 2024, Gauthier and Brown formally proved $R(4, 5) = 25$ using a SAT solver~\cite{r45_formal}, verifying a result originally obtained in 1995 that used unverified computational methods~\cite{r_45_mckay}.
Gauthier and Brown's approach combined an interactive theorem prover, a SAT solver, and gluing together generalizations of colored graphs (as described below).
\emph{Graph gluing} refers to inserting an $n \times n$ adjacency matrix and an $m \times m$ adjacency matrix along the main diagonal of a larger empty matrix, to form an $(m+n) \times (m+n)$ adjacency matrix. 
The off-diagonal is then filled in, subject to some constraints (in this case the Ramsey constraints) in an exhaustive way.
A \emph{generalization} of a colored graph is a colored graph with some edges uncolored. 
A $(p,q)$-graph generalization with one uncolored edge represents the $(p,q)$-graph where the uncolored edge can be colored red or blue.
They constructed exact covers for sets of $(3, 5; m)$-graphs and $(4, 4; n)$-graphs, reducing the number of graph gluings required. 
%\CB{Term ``gluings'' wasn't defined yet.} Resolved, definition added
Their method took over 2.5 years of CPU time, but they estimate it would have taken 44 years without generalizations. This represents a very recent advancement in the formal verification of results concerning Ramsey numbers, showcasing the potential of rigorous computational methods in this domain.
%\CB{Generalizations wasn't defined either.} Resolved, definition added

\citein{fujita2013scsat} used a soft-constraint approach to improve the lower bound of $R(4, 8)$ from 56 to 58. 
They introduced two types of soft constraints: zebra soft-constraints and unit soft-constraints. These could be iteratively removed, with their selection based on the number of conflicts relating to a soft-constraint.
%\CB{Is this sentence grammatically correct?} Resolved, added 'with their'
This method allows for efficient propagation of edge assignments and significantly reduced the search space.

SAT Modulo Symmetries (SMS) is a framework developed for graph generation and enumeration~\cite{pysms}.
It leverages the SAT solver CaDiCaL~\cite{Biere2024} with a dedicated symmetry propagator to check the canonicity of partial solutions. SMS is implemented using the IPASIR-UP interface~\cite{fazekas2023ipasir}, which allows the integration of user-defined symmetry propagators directly into the solving process, enabling efficient dynamic symmetry-breaking constraints.
%\CB{The IPASIR-UP paper needs to be cited, but currently there is no mention of it. Also, the journal version of the IPASIR-UP paper was just published and more functionality was added to the interface, like the ability to forget CAS-derived clauses. The new interface isn't backwards-compatible, so the code would have to be updated to conform to the newest interface, but at some point this should be done to see if efficiency can be improved in the latest version.}
This approach has been applied to verify smaller Ramsey numbers, such as $R(3,5)$ and $R(4,4)$~\cite{SMS2ramsey}, but has not yet been extended to larger instances. %In our studies, \MC creates more balanced cubes \CB{``more blanced cubes'' comes out of nowhere.  Is this compared to SMS?  Plus, it is not clear you are now talking about parallelization, as the reader may not know what a cube is.} \BL{I'm going to comment this out for now, I am not too confident about this since I am not sure if Conor ran this experiment, this might be an observation from KS. I can run some experiments after the deadline} when combined with \AMS~\cite{alphamaplesat}, thus leads to parallel solving and verification with better wall clock time. 

More recently, \citein{codelverified} introduced verified proof checking tools for the substitution redundancy (SR) proof system,
a powerful generalization of the propagation redundancy (PR) and resolution asymmetric tautology (RAT) proof systems.
Their work presents the first verified SR proof checker, implemented in the Lean theorem prover~\cite{deMoura2015},
and demonstrates SR's ability to produce significantly shorter proofs than RAT for certain problems.
They provide a concise 38-clause SR proof of $R(4,4) \leq 18$. Their experimental results show SR proofs are on average 99.6\% smaller than equivalent RAT proofs, while their verified checker performs comparably to existing fast PR checkers. Although no current SAT solvers support SR reasoning, this work lays the groundwork for potential advancements in SAT solving techniques, offering a path to more powerful reasoning and shorter proofs for complex problems.

\subsection{Tri-color Ramsey Problems}

An $(r_1,\dotsc,r_k; n)$ Ramsey colouring is a colouring of the edges of a complete graph $K_n$ with $k$ colours such that there is no monochromatic complete subgraph $K_{r_i}$ in colour~$i$ for each $1\leq i\leq k$.
The multicolor Ramsey number $R(r_1,\dotsc,r_k)$ is the minimal value of~$n$ such that every $k$-coloring of $K_n$ is a $(r_1,\dotsc,r_k; n)$ Ramsey colouring.
\citein{codishresult} solved the tri-color Ramsey problem $R(4, 3, 3)$ using a SAT solver in combination with symmetry breaking techniques and the graph isomorphism tool nauty~\cite{nauty2014}. 
%\CB{Multicolor Ramsey problems haven't been defined.}
They employed degree sequences and degree matrices to break down the problem into smaller, more manageable sub-problems. 
A \emph{degree sequence} of an undirected graph is a non-increasing sequence of its vertex degrees. 
A \emph{degree matrix} of a $k$-color graph of order~$n$ is an $n \times k$ matrix where entry $(i,j)$ is the number of $j$-colored edges on vertex $i$.
Based on prior theory, the authors knew a $(4,3,3)$-graph of order 30 must be $\langle a,b,c\rangle$ regular in one of the combinations of $\langle a,b,c\rangle$ shown below. Namely, each vertex in the coloring must have $a$ edges in the first color, $b$ edges in the second color, and $c$
edges in the third color:
%\CB{What does this angle bracket notation and ``regular in'' mean?}
%\BL{I briefly read the paper and provided my definition. Conor, feel free to improve on it.} Agreed, thought 'regular' was common enough in graph theory to be excluded, but not all readers will be familar with graph theory I suppose 
\begin{gather*}
\langle 13, 8, 8\rangle,
\langle14, 8, 7\rangle,
\langle15, 7, 7\rangle, \\
\langle15, 8, 6\rangle,
\langle16, 7, 6\rangle,
\text{ or } \langle16, 8, 5\rangle
\end{gather*}
They instantiated a problem by taking valid assignments of combinations of the above $\langle a,b,c\rangle$-regular graphs and inserting them along the main diagonal of a $30 \times 30$ matrix. 
This represents a partial coloring of the complete graph on 30 vertices in 3 colors. 
The remaining edge colors were determined by a SAT solver. After a total run time of over 350 hours, all combinations returned UNSAT except for $\langle13,8,8\rangle$.
%\CB{Not really clear; I don't know what a degree matrix is for example.} Resolved re-added definitions and expanded on their method
Following this, their approach involved generating all 3-colorings on 13 vertices without monochromatic triangles and using these to construct partial solutions for larger graphs. The combined computational runtime was 128 years, although the real-time was reduced through parallelization on 456 threads.

\subsection{Directed Ramsey Graphs}

A \emph{tournament} is a directed graph with exactly one edge in one of the two possible directions between each two vertices. A tournament is transitive if, for all triplets of vertices $u$, $v$, and $w$, when edges $uv$ and $vw$ exist, then edge~$uw$ must also exist. The directed Ramsey number $R(k)$ is the minimum number of vertices a tournament must have so that it must contain a transitive subtournament of size~$k$. \citein{heuleR7} improved the lower and upper bounds on $R(7)$ to 34 and 47 respectively using a SAT solver.
%\CB{What is $R(7)$?  This notation was not defined.}
%\BL{Similarly here, I read the paper and added the definition, Conor, feel free to improve it.} Agreed
For the purposes of this paper, we will only describe details and differences of directed Ramsey numbers where necessary. We define a tournament as an orientation of the complete graph, such that for all pairs of distinct vertices $u$ and $v$, exactly one of the edges $uv$ or~$vu$ is in the tournament. 
A subtournament on $k$ vertices, denoted~$\TT_k$, is called transitive if it is a subgraph of a tournament and for all vertices $u$, $v$, and~$w$, the existence of edges $uv$ and $vw$ implies the existence of edge $uw$. 
The directed Ramsey number $R(7)$ is the smallest integer~$n$ such that all tournaments on~$n$ vertices contain a transitive subtournament of size $k$.
%\CB{$R(7)$ never defined.} Resolved re-added definitions
\citein{heuleR7} explored various encodings of the directed $R(7)$ problem and found that employing self-subsuming resolution performed better, likely due to arc consistency being maintained. 
Their approach to finding the upper bound involved a combination of graph theory techniques and SAT solving, including cataloging $\TT_6$-free 
%\CB{I've never seen such a long hyphenated word. This can be rewritten.} Resolved, simpler with old definitions readded
tournaments on 23, 24, and 25 vertices. The lower bound was improved with a direct application of a SAT solver to their encoding.


\subsection{Previous Work on \texorpdfstring{$\boldsymbol{R(3, 8)}$}{\textbf{R(3,8)}} and \texorpdfstring{$\boldsymbol{R(3, 9)}$}{\textbf{R(3,9)}}}

\citein{mckay1992value} computationally showed that $R(3, 8) = 28$. Their method involved generating all graphs up to isomorphism on $20$--$22$ vertices without 3-cliques and without independent sets of size 7. They used a recursive generation procedure and the graph isomorphism tool nauty to remove isomorphic graphs.

\citein{GRAVER1968125} made significant progress on $R(3, 9)$ before the problem was ultimately solved by \citein{grinstead1982ramsey}. Graver and Yackel proved $R(3, 9) \geq 35$ and showed a $(3, 9; 36)$-graph must be regular of degree 8 and must contain a $(3, 8; 27; 80)$-subgraph. Grinstead and Roberts computationally showed that no $(3, 8; 27; 80)$-graphs exist, thereby proving $R(3, 9) = 36$. Their method involved a series of lemmas on the structure of various subgraphs of $(3, 8; 27; 80)$-graphs and used graph gluing techniques combined with computational searches.

The previous approaches to $R(3, 8)$ and $R(3, 9)$ relied on unverified computational methods for graph enumeration,
and therefore we present verifiable proofs using the SAT+CAS paradigm.

\section{Encoding Ramsey Problems}
\label{sec:constraints}

We now discuss the encodings we used to translate the Ramsey problem into formulas in conjunctive normal form (CNF)\@.
In addition to the naive encoding of Ramsey graphs, we introduce encodings that further narrow down the search space by limiting symmetries and the cardinality of edges and vertex degrees in the graph.


\subsection{Encoding Ramsey Graphs}
The Ramsey problem is encoded for a predefined $n$, $p$, and~$q$ by deriving a Boolean formula in conjunctive normal form asserting the existence of a $(p,q)$-graph of order~$n$.
The encoding enforces every $p$-clique to have at least one edge in the opposing (red) color and every $q$-clique to have at least one edge in the opposing (blue) color, i.e., 
\[ \bigwedge_{K_p \subseteq K_n} \bigvee_{e \in K_p} \neg e \quad\text{and}\quad \bigwedge_{K_q \subseteq K_n} \bigvee_{e \in K_q} e, \]
where the variable $e$ is assigned true~($\top$) when the corresponding edge is colored blue and is assigned false~($\bot$) when the corresponding edge is colored red.
%We refer to this as the `basic Ramsey' encoding.
A satisfying assignment of the encoding corresponds to finding a $(p,q)$-graph of order~$n$, and therefore $R(p,q)>n$.
Similarly, an unsatisfiable result means no such colorings exist for this particular~$n$, i.e., all colorings contain a blue $p$-clique or a red $q$-clique,
and therefore $R(p,q)\leq n$.

\subsection{Symmetry Breaking Constraints}

In order to break the symmetries of the problem, we first add static constraints that enforce a lexicographic ordering on rows of
the graph's adjacency matrix, then use the dynamic symmetry breaking capabilities of a CAS to break the rest of the symmetries. 

We encoded the partial static symmetry breaking constraints developed by \citein{codish2019constraints},
which enforces a lexicographical ordering on the rows of a graph's adjacency matrix. 
These block the solver from exploring certain symmetric portions of the search space before the CAS is called.
This is beneficial, as there is an overhead associated with calling the CAS\@.

%\CB{The statement is correct but doesn't flow with the rest of the paragraph.}, Resolved, added a paragraph break and introductory line
These clauses are constructed in the following manner:
for an adjacency matrix $A$ of a graph of order $n$, define $A_{i,j}$ as the $i$th row of A without columns $i$ and $j$.
The clauses enforce that $A_{i,j}$ is lexicographically equal or less than $A_{j,i}$ for all $1 \le i < j \le n$. 
These clauses introduce $O(n^3)$ auxiliary variables and clauses.
Based on our empirical evidence, these constraints provide a significant speed-up by breaking symmetries statically and were included in all instances.

\subsection{Cardinality Constraints}\label{sec:card}

Cardinality constraints are used to further reduce the search space by enforcing both the degree of vertices and the number of edges in Ramsey graphs. More specifically, it is known that every vertex~$v$ of a $(p,q;n)$-graph satisfies
\[ n - R(p,q - 1) \le \deg_b(v) \le R(p - 1,q) - 1 \]
where $\deg_b(v)$ is the number of blue edges on vertex~$v$~\cite{GRAVER1968125}. We also leverage theoretical results to restrict the number of edges allowed in the Ramsey graph when proving the value of $R(3,9)$ in Section~\ref{r39theory}.

We encoded these constraints using the totalizer encoding of \citein{b_b}.
The totalizer uses a binary tree to create relationships between auxiliary variables. Each node in the tree is assigned a value and a set of unique
variables. %---refer to the original paper for how these are assigned.
Suppose we wish to encode between $l$ and~$u$ of~$m$ variables are true.  We form a binary tree with
$m$ leaf nodes and associate each leaf node to one of these variables.
For a non-leaf node $r$ with children $a,b$, let $R= \{r_1,\dotsc,r_{m_0}\}$, $A=\{a_1,\dotsc,a_{m_1}\}$, and $B= \{b_1,\dotsc,b_{m_2}\}$ be the set of variables assigned to $r$, $a$ and $b$ respectively. 
The following conjunction of clauses is related to the node~$r$: 
%\[ \bigwedge_{\substack{0\le\alpha\le m_1 \\ 0\le \beta \leq m_2 \\ 0\le \sigma \le {m_0} \\ \alpha + \beta =\sigma}}C_1(\alpha,\beta , \sigma)\land C_2(\alpha,\beta , \sigma) \]
\[ \bigwedge_{\substack{0\le\alpha\le m_1 \\ 0\le \beta \leq m_2 \\ 0\le \sigma \le {m_0} \\ \alpha + \beta =\sigma}}
(\neg a_\alpha \lor\neg b_\beta\lor r_\sigma) \land
(a_{\alpha+1} \lor b_{\beta+1} \lor \neg r_{\sigma+1})
\]
with $a_0$, $b_0$, and $r_0$ assigned true
and $a_{m_{1}+1}$, $b_{m_{2}+1}$, and $r_{m_0+1}$ assigned false.
%with the following notation:
%\begin{gather*}
%a_0=b_0=r_0=\top,a_{m_{1}+1}=b_{m_{2}+1}=r_{m_0+1}=\bot . %\\
%C_1(\alpha,\beta , \sigma)=(\neg a_\alpha \lor\neg b_\beta\lor r_\sigma), \\
%C_2(\alpha,\beta , \sigma)=(a_{\alpha+1} \lor b_{\beta+1} \lor \neg r_{\sigma+1}).
%\end{gather*}
%$C_1(\alpha,\beta, \sigma)$ is the CNF representation of the relation $\alpha+\beta \le \sigma$ and $C_2(\alpha,\beta,\sigma)$ is the CNF representation of the relation $\alpha+\beta \ge \sigma$.
These clauses ensure that the number of variables assigned true in $R$
is equal to the number of variables assigned true in $A\cup B$.
Finally, the clauses
$\bigwedge_{1\le i\le l}c_i$ and $\bigwedge_{u+1\le i\le m}\neg c_i$
specify that between $l$ and $u$ of our original $m$ variables are true
where $c_1,\dotsc,c_m$ denote the variables associated with the root node of the tree.
%\CB{This encoding is not well-defined; it doesn't even depend on $u$\dots} resolved, re-added previously cut parts
This encoding uses $O(m\log m)$ new variables and $O(mu)$ clauses after applying unit propagation and removing satisfied clauses.

\section{Orderly Generation using SAT+CAS}\label{sec:orderly}

When searching for graphs using a SAT solver, the static symmetry breaking constraints introduced previously do not block all isomorphic copies---in fact, most symmetries remain in the search space.
Therefore, we use a symbolic computation method to block the remaining symmetries dynamically during the solving process.
More specifically, we implement an isomorph-free graph generation technique called orderly generation, developed independently by \citein{read1978every} and \citein{faradvzev1978constructive}. 

An adjacency matrix $A_G$ of a graph $G$ is canonical if every permutation of the graph's
vertices produces a matrix lexicographically greater than or equal to $A_G$,
where the lexicographical order is defined by concatenating the above-diagonal
entries of the columns of the adjacency matrix starting from the left.
In other words, the canonical matrix of a graph is the lexicographically-least (lex-least)
way of representing the graph's adjacency matrix.

An \emph{intermediate} matrix of $A_G$ is a square upper-left submatrix of $A_G$.  If $A_G$ is of order~$n$
then its intermediate matrix of order $n-1$ is said to be its \emph{parent},
and $A$ is said to be a descendant of its intermediate matrices.
The orderly generation method is based on the following two consequences of this definition of canonicity:
\begin{enumerate}
    \item[(1)] Every isomorphic class of graphs has exactly one canonical (lex-least) representative.
    \item[(2)] If a matrix is lex-least canonical, then its parent is also lex-least canonical.
\end{enumerate}

This particular definition of canonicity is particularly useful
as the contrapositive of the second property implies that if a matrix is not canonical,
then all of its descendants are also not canonical.
Therefore, any noncanonical intermediate matrix encountered during the search can be blocked immediately, as none of its descendants are canonical.

To implement the orderly generation algorithm in a SAT solver,
when the solver finds a partial assignment that corresponds to an intermediate matrix, the canonicity of this matrix is determined by a canonicity-checking routine implemented in the CAS (Figure \ref{fig:pipeline}). The CAS operates by exploring permutations of the graph's vertices and evaluating whether any of these permutations result in a lexicographically smaller adjacency matrix---if so, the matrix is noncanonical.
If the CAS finds a permutation that demonstrates the noncanonicity of the matrix,
then a blocking clause is learned which removes this matrix and all of its descendants from the search.
Otherwise, the matrix is canonical and the SAT solver proceeds as normal. 
When a matrix is noncanonical, the canonicity-checking routine provides a witness
(a permutation of the vertices producing a lex-smaller adjacency matrix),
which allows for the verification of \MC's result without trusting the CAS\@.

Even though $R(3,k)$ and $R(k,3)$ share the same value, in the context of CNF encoding, the $R(3,k)$ and $R(3,k)$ instances are different with $R(k,3)$ containing a much higher number of negative literals. An important observation in our experiments is that solving \( R(k,3) \) is consistently faster than solving \( R(3,k) \), despite these instances being equivalent due to symmetry. This behavior persists even when the CAS is turned off (Table~\ref{tbl:sec6}).
We solved both $R(3,k)$ and $R(k,3)$ for $k=8$ and~$9$ and included both runtimes in Section~\ref{sec:result}.
\begin{table}
\centering
\begin{tabular}{c cc cc}
    & \multicolumn{2}{c}{CaDiCaL + CAS} & \multicolumn{2}{c}{CaDiCaL only} \\ 
$k$ & $R(3, k)$ & $R(k, 3)$ & $R(3, k)$ & $R(k, 3)$ \\ \hline
7   & 14.3 s& 8.2 s& 564.3 s& 220.7 s\\ %updated
8   & 112.1 h& 18.5 h& $>$ 7 days & $>$ 7 days \\
\end{tabular}
\caption{{\bf Sequential SAT+CAS solver runtimes for \textit{R}(3,\,7) and \textit{R}(3,\,8) (and their symmetric versions \textit{R}(7,\,3) and \textit{R}(8,\,3)):} Comparison of sequential runtime for instances $R(3,k)$ and $R(k,3)$. ``CaDiCaL + CAS'' indicates solutions using CaDiCaL with CAS, while ``CaDiCaL only'' uses CaDiCaL without CAS\@. Cardinality constraints are excluded for $k=7$ to avoid making the instance too easy, but included for $k=8$. Experiments were conducted on Dual Xeon Gold 6226 processors running at 2.70 GHz.}
\label{tbl:sec6}
\end{table}

\section{Parallelization}\label{sec:c&c}

In order to scale our technique up to $k=9$, parallelization is used to reduce the wall clock time. We use the cube and conquer technique, which splits the CNF instance into tens of thousands of subproblems and solves them in parallel.

\subsection{Cube and Conquer}
Cube and conquer~\cite{March} is a parallelization technique whereby a set of simpler instances are solved, and the aggregate result is equivalent to solving the original instance.
Initially developed to solve SAT instances arising from computing van der Waerden numbers~\cite{Ahmed2010,Ahmed2014},
many combinatorial problems have since
been attacked using this technique,
such as Lam's Problem~\cite{Lams},
the Boolean Pythagorean Triples Problem~\cite{heule_pyth_trip},
Schur number five~\cite{10.5555/3504035.3504843},
and the Kochen--Specker problem~\cite{li2023sat}.
%Cube and conquer algorithms create a set of conjunctions of literals. 

Let $v_i$ be a variable appearing in the Boolean formula $F$. 
Then solving instances $F_1\coloneqq F \land v_i$ and $F_2\coloneqq F \land \neg v_i$ is equivalent to solving $F$ and
if either $F_1$ or $F_2$ are satisfiable, then $F$ is satisfiable.
We say $v_i$ is the \emph{splitting} variable and consider $F_1$ and $F_2$ as subinstances of $F$.
If a subinstance is still hard, an unassigned variable (subject to some selection criteria) in the subinstance can be split on. 
This splitting can be repeated until some stopping criteria are reached.

When choosing the appropriate cubing solver, there are two main factors to consider: one is the time it takes to generate all cubes,
and the other is the quality of the cubes, which can be measured by the time required to solve the hardest cube.
%\CB{This previously said ``wall clock time''.  But that depends on the number of processors you have available.} \BL{``by the time to solve the hardest cube'' might be more accurate?}
Two cubing solvers we tried are \textsc{march\_cu}~\cite{March} and \AMS~\cite{alphamaplesat}.
We use \AMS as it generates a large set of cubes faster without compromising the quality of the cubes.

\AMS introduces a novel approach to cube challenging combinatorial instances. At its core, \AMS employs a Monte Carlo Tree Search (MCTS) based lookahead cubing technique, which sets it apart from traditional cube-and-conquer solvers. This method allows for a deeper heuristic search to identify effective cubes while maintaining low computational costs. 

We define eliminated variables to be variables that are either assigned a $\top$/$\bot$ value or propagated to be $\top$/$\bot$ by the simplification solver 
(CaDiCaL with orderly generation). To determine when to stop cubing the instance further, we used the number of eliminated variables as the stopping criteria. If the number of eliminated variables is greater than a predefined threshold, the cubing stops for this subinstance. Given the original formula $\phi$, each time \AMS chooses an edge variable $x$ to split on, the formula $\phi \land x$ and $\phi \land \lnot x$ are generated and simplified using CaDiCaL + CAS, so that \AMS can choose the next splitting variable based on the simplified formula, which contains clauses derived from the computer algebra system, therefore incorporate blocking clauses from orderly generation during the cubing process.

Following cubing, the cubes are ``conquered'', i.e., solved by a SAT+CAS solver.
%\CB{There seems to be something missing or an extra `and'.} resolved, removed 'and'
Ideally, the solver can solve all cubes.
However, when our solver could not solve an instance without producing a proof file larger than 7 GiB, the instance was passed back to \AMS to be cubed again.
We chose 7 GiB as the maximum allowed file size because we found that the DRAT-trim proof checker~\cite{wetzler2014drat} could verify such proofs using at most 4~GiB of memory. 
This process of alternating cubing and conquering continues for difficult instances until all cubes can be solved with proof files smaller than 7 GiB. To solve cubes efficiently across multiple CPUs, we utilized the Python multiprocessing library.


\begin{table*}[ht]
\centering
\begin{tabular}{c c c c c c c c c}
\textbf{Instance} & \textbf{Cubing Time} & \textbf{Simplification Time} & \textbf{Solving Time} & \textbf{Verification Time} & \textbf{Wall Clock Time} \\ \hline
% Placeholder rows
$R(8,3)$  & \phantom01,360 s & \phantom01,217 s & \phantom019,811 s & \phantom022,328 s & \phantom08 hrs   \\ 
$R(9,3)$  &         15,530 s &         42,482 s &         697,575 s &         473,874 s &         26 hrs   \\ 
\end{tabular}
\caption{Summary of experimental results for solving \( R(8,3) \) and \( R(9,3) \) in parallel.}
\label{tbl:results}
\end{table*}


\section{Theory of \texorpdfstring{$\boldsymbol{R(3,9)}$ and $\boldsymbol{R(9,3)}$}{Theory of R(3,9) and R(9,3)}}\label{r39theory}
Directly applying a SAT+CAS solver to the $R(9,3)$ problem is difficult.  
As the problem is significantly harder than the $R(8,3)$ problem, we leverage theoretical results to reformulate the problem.
\citein{grinstead1982ramsey} combined theory and computation to show that $R(3,9)=36$, leveraging theoretical results from \citein{GRAVER1968125}. This section highlights key theoretical results from Graver and Yackel's paper. 
The following lemma, from Section~3 of \citein{GRAVER1968125}, forms the basis for the original proof of $R(3,9)=36$.
We also use it in our proof and verification, since the lemma does not rely on computation and can be proved mathematically.

\begin{lemma}\label{lem:3-9-36}
    A $(3,9;36)$-graph contains a $(3,8;27;80)$-graph. A $(9,3;36)$-graph contains a $(8,3;27;271)$-graph.
\end{lemma}
%\CB{Lemma~\ref{lem:3-9-36} appears suddenly, without warning.}
%\citein{GRAVER1968125} originally used equation~\eqref{formula:graver} below to derive this result.
%The value of $R(3,8)$ was unknown when this lemma was proven. Using $R(3,8)=28$ (which we have computationally verified), in conjunction with the degree bounds, this lemma is easier to prove.
%\CB{Okay, but is that relevant?  We just need to know it is true and can be checked by hand.} \BL{Not very relevant, removing it}

%\CB{It should be made clear if this lemma and proof originate with us or not.  If not, the proof probably shouldn't be in the paper---it should be cited instead.} Resolved, deleted proof and clarified the original proof writers

Graver and Yackel improved the bounds on many Ramsey numbers and on the minimum number of edges in Ramsey graphs. They constructed a $(3,9)$-graph on 35 vertices, thereby showing $R(3,9)>35$.
Specifically for $R(3,9)$, Grinstead and Roberts derived a sequence of lemmas on the structures of various subgraphs of $(3,8;27;80)$-graphs. 
They computationally proved structures in $(3,8;27;80)$-graphs cannot exist in order to show $R(3,9)\le36$ via Lemma~\ref{lem:3-9-36}.
They estimated $5 \times 10^{10}$ machine operations and $2.5 \times 10^4$ seconds of computation,
%\CB{Only seven hours?? Seems very surprising.} Resolved, Paper directly says this number for R(3,9)
but note the time could be improved using machines with more efficient bitstring operations.
Computations were performed on a Honeywell Level 66 computer. The second part of the lemma is derived from the fact that $\Comb{27}{2} - 80 = 271$, thus a $(9,3;36)$-graph contains a $(8,3;27;271)$-graph.

We apply a parallelized SAT+CAS tool to a CNF encoding asserting the existence of a $(8,3;27;271)$-graph.
We obtained an UNSAT result (Section~\ref{sec:r39}), therefore showing that no $(8,3;27;271)$-graphs exist.
By Lemma~\ref{lem:3-9-36}, this implies $R(9,3)\leq36$,
and since a $(3,9;35)$-graph exists, this implies $R(9,3) = R(3,9)=36$.
In doing so, we do not rely on the method employed by the authors of the original proof to show the
nonexistence of $(3,8;27;80)$-graphs, %\CB{What graphs?} 
since they rely on unverified computational results.
To simplify our encoding, we do not rely
on any additional theoretical results on $(3,8;27;80)$-graphs or $(8,3;27;271)$-graphs, aside from the vertex degree constraints mentioned in Section~\ref{sec:card}.

\section{Results and Verification}\label{sec:result}

\subsection{Experimental Setup}

We performed both SAT+CAS solving and verification in parallel by integrating \AMS with the \MC tool.
The \( R(8,3) \) result was obtained on a cluster of Dual Xeon Gold 6226 processors @ 2.70 GHz and used 24 CPUs,
while the \( R(9,3) \) result was computed on a cluster of Dual AMD Epyc 7713 CPUs @ 2.0 GHz and used 128 CPUs.

By default, \MC uses a pseudo-canonical check and stops the canonical check
early if the CAS takes too long to determine canonicity.
To optimize solving performance, we enabled full canonicity checking and  
this adjustment resulted in a 2$\times$ improvement in solving time.
Even after enabling a full canonical check, a minority of the total solving time is spent in the CAS---%\CB{How much?}
%More precisely, we observe that when we enable the full canonicity check,
for $R(8,3)$, the CAS accounts for 7\% of the total solving time, while for $R(9,3)$, it accounts for 38\%.
Although a full canonicity check is more expensive, it generates more symmetry-blocking clauses
and ultimately enhances the solver's efficiency for this problem.

Table~\ref{tbl:results} includes ``simplification time'' for our cube-and-conquer tool. During cubing, each variable split triggers a brief CaDiCaL+CAS simplification (10,000 conflicts), with the simplified instance passed back to AMS. This tracks eliminated variables and crucially allows AMS to leverage SAT+CAS-derived clauses for balanced cube generation.

\subsection{Solving \texorpdfstring{\( \boldsymbol{R(3,8)} \) and \( \boldsymbol{R(8,3)} \)}{\textbf{R(3,8) and R(8,3)}}}

The Ramsey number \( R(3,8) = 28 \) was confirmed by obtaining an UNSAT result for the encoding asserting the existence of a 28-vertex \( (8,3) \)-graph and a SAT result in 288 seconds for the encoding asserting the existence of a 27-vertex \( (8,3) \)-graph. Solving \( R(8,3) \) on 28 vertices can be done sequentially (as shown in Table~\ref{tbl:sec6}) or in parallel.

%To make the instance easier to solve and search space more constrained, we imposed cardinality constraints, restricting each vertex to have a degree between 20 and 22 (as derived from Section~\ref{sec:card}.
For $R(8,3)$, we used the cardinality constraints from Section~\ref{sec:card}, implying
that the degree of each vertex is between 20 and~22.
%\CB{Why ``optimize parallel solving''? Aren't these just the cardinality constraints from Sec.~4.3?}
Cubing was performed until 120 edge variables were eliminated. If a cube was not solved after producing a 7~GiB certificate, further cubing was applied, eliminating an additional 40 variables.

A total of 41 cubes were generated, each returning UNSAT\@. With parallelization, the problem was solved in approximately 8 hours of wall clock time. The combined certificate files amounted to 5.8 GiB. We applied the same techniques to solve $R(3,8)$, where each vertex has a degree between 5 and~7. We observed that solving the $R(8,3)$ instance is about 6$\times$ faster than solving $R(3,8)$ sequentially.

\subsection{Solving \texorpdfstring{\( \boldsymbol{R(3,9)} \) and \( \boldsymbol{R(9,3)} \)}{\textbf{R(3,9) and R(9,3)}}}\label{sec:r39}

Solving \( R(9,3) \) corresponds to solving the \( R(8,3;27;271) \) problem, as mentioned in Section~\ref{r39theory}. This instance must be solved using the parallelized MathCheck tool, as \( R(8,3;27;271) \) is much more challenging to solve sequentially in a reasonable amount of wall clock time.

Cardinality constraints were applied similarly, enforcing that each vertex must have a degree between 19 and 22, and that the graph must contain exactly 271 edges. Cubing continued until 100 edge variables were eliminated. If a cube was not solved after producing a 7~GiB certificate, further cubing eliminated an additional 40 variables.

A total of 2486 cubes were generated, each returning UNSAT\@. With parallelization, the problem was solved in approximately 26 hours of wall clock time. The combined proof files amounted to 289~GiB. Similarly, $R(3,9)$ is solved using the same cubing cutoff criteria. We witnessed that solving the $R(9,3)$ instance is about 6$\times$ faster than solving $R(3,9)$.

\subsection{Verification of Results}

The SAT+CAS approach produces verifiable certificates, enabling an independent third party to confirm the solver's results. These certificates ensure that the SAT solver's search is exhaustive and that the learned clauses provided by the CAS are correct. As a result, only the correctness of the proof verifier—a relatively simple piece of software—needs to be trusted, rather than the SAT solver or the CAS itself.

Verification was performed using the DRAT-trim proof checker~\cite{wetzler2014drat} modified to support clauses specified to be trusted. Clauses derived from the CAS were prefixed with the character `\texttt{t}' to mark them as trusted, and they were verified separately. A Python script applied the witness permutations to confirm that each noncanonical adjacency matrix produced a lexicographically smaller matrix, verifying the correctness of the CAS-derived clauses.

When using the cube-and-conquer approach, it is critical to ensure that the generated cubes collectively partition the search space. Completeness was verified by recursively checking that for any literal \( x \) forming a cube \( \phi \land x \), all extensions of the complementary literal \( \phi \land \neg x \) were covered by the set of generated cubes. This process confirmed that \(R(8,3) = R(3,8) = 28 \) and \(R(9,3) = R(3,9) = 36 \). In addition, we ran $R(8,3)$ on the same machine sequentially and obtained a runtime of 66,504 seconds (Table~\ref{tbl:results}), however, the total CPU time combining cubing, simplification, and solving was only 22,388 seconds. Therefore, cube-and-conquer achieved a reduction in total CPU time compared to sequential solving. This mirrors the efficiency gains observed in the boolean Pythagorean Triples problem~\cite{Heule2016SolvingAV}, where cube-and-conquer reduced the total solving time from 2125 days to 2 days using 800 cores, effectively reducing the computational effort to 1600 CPU days. %\CB{How does it align with an almost linear speedup? I thought Marijn had superlinear speedups as well.} \BL{Yeah I did the math and it is superlinear, I am not sure why Marijn said ``almost linear'' in his paper.}

\begin{comment}
\begin{table}
\centering
\begin{tabular}{cccc}
 & \MC & SMS & SAT-only \\ \hline
$R(3,7) \leq23$ & $23$ s & $12$ s    &  $>86{,}400$ s\\
$R(3,8) \leq28$ & $57$ h & $31$ h    & $>100$ h \\ 
$R(3,9) \leq36$ & 1918 h & $>100 $ h & $>100$ h
\end{tabular}
\caption{CPU time comparison for different verifiable tools at solving UNSAT instances. $R(3,9)$ is run using the parallel \MC pipeline, therefore the reported time includes CPU time for cubing, simplification, and solving.}
%\BL{I have changed the equality to $\leq$, since technically that is what the UNSAT instance represents, I hope this is better.}}
%\CB{Not cited until the next page.}}
\label{tbl:comparison}
\end{table}
\end{comment}

\section{Conclusion}
%======================================================================
Using SAT+CAS, we significantly improve the efficiency of a SAT solver on Ramsey problems and
provide the first independently-checkable proof of the result $R(3,8)=28$ of McKay and Min from 1992.
% Further, verification is a key aspect of computer assisted proofs. Using our SAT+CAS methodology, we cross-verify the result of McKay and Min and provide the first certificate of an independently verified proof that $R(3, 8) = 28$.
%The certificate is 31~GiB and can be verified in 89 hours.
%Using a parallelized SAT+CAS solver, and with less dependency on theoretical results,
We verify that $R(3,9)\le36$ (and therefore $R(3,9) = 36$)
with less dependency on theoretical results than previous approaches.  This
result was first shown by Grinstead and Roberts in 1982 but was never before verified
by independently-checkable certificates.

It is important to note that while our approach generates certificates of completeness for exhaustive search processes, these certificates do not provide formal proof of the Ramsey number itself. Instead, they complement formal proof by verifying computationally intensive parts of the proof. This contribution ensures reliability and reproducibility in the search for larger Ramsey instances.

SAT+CAS has been demonstrated to be an effective problem-solving and verifying tool for Ramsey problems.
When combined with previously known domain knowledge about Ramsey problems,
the search space can be reduced and thus the effectiveness of the SAT+CAS method is improved. 
More difficult Ramsey problems often have larger encodings, which require an excessive amount of memory. 
Thus, reducing problems using domain knowledge has the additional benefit of reducing the encoding size.
Future applications of SAT+CAS to Ramsey problems include the verification of all known Ramsey numbers and determining the values of
$R(3,10)$, $R(4,6)$, or $R(5,5)$, which remain open problems.

\newpage
% References and End of Paper
% These lines must be placed at the end of your paper
\bibliographystyle{named}
\bibliography{ijcai25}

\end{document}

