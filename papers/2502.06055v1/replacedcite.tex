\section{Previous Work}
\subsection{Classical Ramsey Numbers}

In April 2024, Gauthier and Brown formally proved $R(4, 5) = 25$ using a SAT solver____, verifying a result originally obtained in 1995 that used unverified computational methods____.
Gauthier and Brown's approach combined an interactive theorem prover, a SAT solver, and gluing together generalizations of colored graphs (as described below).
\emph{Graph gluing} refers to inserting an $n \times n$ adjacency matrix and an $m \times m$ adjacency matrix along the main diagonal of a larger empty matrix, to form an $(m+n) \times (m+n)$ adjacency matrix. 
The off-diagonal is then filled in, subject to some constraints (in this case the Ramsey constraints) in an exhaustive way.
A \emph{generalization} of a colored graph is a colored graph with some edges uncolored. 
A $(p,q)$-graph generalization with one uncolored edge represents the $(p,q)$-graph where the uncolored edge can be colored red or blue.
They constructed exact covers for sets of $(3, 5; m)$-graphs and $(4, 4; n)$-graphs, reducing the number of graph gluings required. 
%\CB{Term ``gluings'' wasn't defined yet.} Resolved, definition added
Their method took over 2.5 years of CPU time, but they estimate it would have taken 44 years without generalizations. This represents a very recent advancement in the formal verification of results concerning Ramsey numbers, showcasing the potential of rigorous computational methods in this domain.
%\CB{Generalizations wasn't defined either.} Resolved, definition added

\citein{fujita2013scsat} used a soft-constraint approach to improve the lower bound of $R(4, 8)$ from 56 to 58. 
They introduced two types of soft constraints: zebra soft-constraints and unit soft-constraints. These could be iteratively removed, with their selection based on the number of conflicts relating to a soft-constraint.
%\CB{Is this sentence grammatically correct?} Resolved, added 'with their'
This method allows for efficient propagation of edge assignments and significantly reduced the search space.

SAT Modulo Symmetries (SMS) is a framework developed for graph generation and enumeration____.
It leverages the SAT solver CaDiCaL____ with a dedicated symmetry propagator to check the canonicity of partial solutions. SMS is implemented using the IPASIR-UP interface____, which allows the integration of user-defined symmetry propagators directly into the solving process, enabling efficient dynamic symmetry-breaking constraints.
%\CB{The IPASIR-UP paper needs to be cited, but currently there is no mention of it. Also, the journal version of the IPASIR-UP paper was just published and more functionality was added to the interface, like the ability to forget CAS-derived clauses. The new interface isn't backwards-compatible, so the code would have to be updated to conform to the newest interface, but at some point this should be done to see if efficiency can be improved in the latest version.}
This approach has been applied to verify smaller Ramsey numbers, such as $R(3,5)$ and $R(4,4)$____, but has not yet been extended to larger instances. %In our studies, \MC creates more balanced cubes \CB{``more blanced cubes'' comes out of nowhere.  Is this compared to SMS?  Plus, it is not clear you are now talking about parallelization, as the reader may not know what a cube is.} \BL{I'm going to comment this out for now, I am not too confident about this since I am not sure if Conor ran this experiment, this might be an observation from KS. I can run some experiments after the deadline} when combined with \AMS____, thus leads to parallel solving and verification with better wall clock time. 

More recently, \citein{codelverified} introduced verified proof checking tools for the substitution redundancy (SR) proof system,
a powerful generalization of the propagation redundancy (PR) and resolution asymmetric tautology (RAT) proof systems.
Their work presents the first verified SR proof checker, implemented in the Lean theorem prover____,
and demonstrates SR's ability to produce significantly shorter proofs than RAT for certain problems.
They provide a concise 38-clause SR proof of $R(4,4) \leq 18$. Their experimental results show SR proofs are on average 99.6\% smaller than equivalent RAT proofs, while their verified checker performs comparably to existing fast PR checkers. Although no current SAT solvers support SR reasoning, this work lays the groundwork for potential advancements in SAT solving techniques, offering a path to more powerful reasoning and shorter proofs for complex problems.

\subsection{Tri-color Ramsey Problems}

An $(r_1,\dotsc,r_k; n)$ Ramsey colouring is a colouring of the edges of a complete graph $K_n$ with $k$ colours such that there is no monochromatic complete subgraph $K_{r_i}$ in colour~$i$ for each $1\leq i\leq k$.
The multicolor Ramsey number $R(r_1,\dotsc,r_k)$ is the minimal value of~$n$ such that every $k$-coloring of $K_n$ is a $(r_1,\dotsc,r_k; n)$ Ramsey colouring.
\citein{codishresult} solved the tri-color Ramsey problem $R(4, 3, 3)$ using a SAT solver in combination with symmetry breaking techniques and the graph isomorphism tool nauty____. 
%\CB{Multicolor Ramsey problems haven't been defined.}
They employed degree sequences and degree matrices to break down the problem into smaller, more manageable sub-problems. 
A \emph{degree sequence} of an undirected graph is a non-increasing sequence of its vertex degrees. 
A \emph{degree matrix} of a $k$-color graph of order~$n$ is an $n \times k$ matrix where entry $(i,j)$ is the number of $j$-colored edges on vertex $i$.
Based on prior theory, the authors knew a $(4,3,3)$-graph of order 30 must be $\langle a,b,c\rangle$ regular in one of the combinations of $\langle a,b,c\rangle$ shown below. Namely, each vertex in the coloring must have $a$ edges in the first color, $b$ edges in the second color, and $c$
edges in the third color:
%\CB{What does this angle bracket notation and ``regular in'' mean?}
%\BL{I briefly read the paper and provided my definition. Conor, feel free to improve on it.} Agreed, thought 'regular' was common enough in graph theory to be excluded, but not all readers will be familar with graph theory I suppose 
\begin{gather*}
\langle 13, 8, 8\rangle,
\langle14, 8, 7\rangle,
\langle15, 7, 7\rangle, \\
\langle15, 8, 6\rangle,
\langle16, 7, 6\rangle,
\text{ or } \langle16, 8, 5\rangle
\end{gather*}
They instantiated a problem by taking valid assignments of combinations of the above $\langle a,b,c\rangle$-regular graphs and inserting them along the main diagonal of a $30 \times 30$ matrix. 
This represents a partial coloring of the complete graph on 30 vertices in 3 colors. 
The remaining edge colors were determined by a SAT solver. After a total run time of over 350 hours, all combinations returned UNSAT except for $\langle13,8,8\rangle$.
%\CB{Not really clear; I don't know what a degree matrix is for example.} Resolved re-added definitions and expanded on their method
Following this, their approach involved generating all 3-colorings on 13 vertices without monochromatic triangles and using these to construct partial solutions for larger graphs. The combined computational runtime was 128 years, although the real-time was reduced through parallelization on 456 threads.

\subsection{Directed Ramsey Graphs}

A \emph{tournament} is a directed graph with exactly one edge in one of the two possible directions between each two vertices. A tournament is transitive if, for all triplets of vertices $u$, $v$, and $w$, when edges $uv$ and $vw$ exist, then edge~$uw$ must also exist. The directed Ramsey number $R(k)$ is the minimum number of vertices a tournament must have so that it must contain a transitive subtournament of size~$k$. \citein{heuleR7} improved the lower and upper bounds on $R(7)$ to 34 and 47 respectively using a SAT solver.
%\CB{What is $R(7)$?  This notation was not defined.}
%\BL{Similarly here, I read the paper and added the definition, Conor, feel free to improve it.} Agreed
For the purposes of this paper, we will only describe details and differences of directed Ramsey numbers where necessary. We define a tournament as an orientation of the complete graph, such that for all pairs of distinct vertices $u$ and $v$, exactly one of the edges $uv$ or~$vu$ is in the tournament. 
A subtournament on $k$ vertices, denoted~$\TT_k$, is called transitive if it is a subgraph of a tournament and for all vertices $u$, $v$, and~$w$, the existence of edges $uv$ and $vw$ implies the existence of edge $uw$. 
The directed Ramsey number $R(7)$ is the smallest integer~$n$ such that all tournaments on~$n$ vertices contain a transitive subtournament of size $k$.
%\CB{$R(7)$ never defined.} Resolved re-added definitions
\citein{heuleR7} explored various encodings of the directed $R(7)$ problem and found that employing self-subsuming resolution performed better, likely due to arc consistency being maintained. 
Their approach to finding the upper bound involved a combination of graph theory techniques and SAT solving, including cataloging $\TT_6$-free 
%\CB{I've never seen such a long hyphenated word. This can be rewritten.} Resolved, simpler with old definitions readded
tournaments on 23, 24, and 25 vertices. The lower bound was improved with a direct application of a SAT solver to their encoding.


\subsection{Previous Work on \texorpdfstring{$\boldsymbol{R(3, 8)}$}{\textbf{R(3,8)}} and \texorpdfstring{$\boldsymbol{R(3, 9)}$}{\textbf{R(3,9)}}}

\citein{mckay1992value} computationally showed that $R(3, 8) = 28$. Their method involved generating all graphs up to isomorphism on $20$--$22$ vertices without 3-cliques and without independent sets of size 7. They used a recursive generation procedure and the graph isomorphism tool nauty to remove isomorphic graphs.

\citein{GRAVER1968125} made significant progress on $R(3, 9)$ before the problem was ultimately solved by \citein{grinstead1982ramsey}. Graver and Yackel proved $R(3, 9) \geq 35$ and showed a $(3, 9; 36)$-graph must be regular of degree 8 and must contain a $(3, 8; 27; 80)$-subgraph. Grinstead and Roberts computationally showed that no $(3, 8; 27; 80)$-graphs exist, thereby proving $R(3, 9) = 36$. Their method involved a series of lemmas on the structure of various subgraphs of $(3, 8; 27; 80)$-graphs and used graph gluing techniques combined with computational searches.

The previous approaches to $R(3, 8)$ and $R(3, 9)$ relied on unverified computational methods for graph enumeration,
and therefore we present verifiable proofs using the SAT+CAS paradigm.