\subsection{Dataset}

We used public data from the PI-CAI challenge\footnote{\url{https://pi-cai.grand-challenge.org/DATA/}} for training and testing.
The dataset originally contained 1500 cases, which we filtered down to 1411 cases by excluding cases from the same patients to avoid data leakage.
We ensure that all testing cases are biopsy-confirmed.
Our AI model was trained on 1211 cases, including 365 $(30.1\%)$ clinically significant prostate cancer (csPCa) cases. 
For study 1, the testing set includes 75 cases, of which 23 $(30.6\%)$ are csPCa.
Study 2 consists of 100 cases, with 32 $(32\%)$ being csPCa. 
For each patient case, we used T2-weighted (T2W), diffusion-weighted imaging (DWI), and apparent diffusion coefficient (ADC) sequences as inputs for both AI and human studies. 
50 cases were shared between study 1 and study 2, which allows us to directly compare performance metrics across both studies on this shared subset.


\para{Labels/annotations.} Case labels were obtained from three sources: biopsy-confirmed results (from systematic, magnetic resonance-guided biopsy, or prostatectomy), human-expert annotations, and AI-derived annotations~\cite{bosma2021annotation}. Out of the original 1500 cases, 1001 has biopsy confirmed case-level labels. 
Out of the 425 positive cases, 220 have human expert annotations, with the remaining annotated by AI. We prioritized human expert annotations when available, defaulting to AI annotations otherwise.
Ground truth case-level labels are approximately accurate, with 66.7\% (1001/1500) cases having biopsy results. Lesion-level annotations are less accurate due to the practical challenges of annotating all lesions in the large dataset. 
For all of our testing patient cases, case-level labels are derived from biopsy results. Lesion-level annotations are derived by experts (trained investigators and resident, supervised by expert radiologists), using all available clinical data. This includes MRI scans, diagnostic reports (radiology and pathology), and whole-mount prostatectomy specimens or other biopsy results when available. 






