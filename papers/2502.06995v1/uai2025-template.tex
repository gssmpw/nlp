\documentclass[accepted]{uai2025} % for initial submission
%\documentclass[accepted]{uai2025} % after acceptance, for a revised version; 
% also before submission to see how the non-anonymous paper would look like 
                        
%% There is a class option to choose the math font
% \documentclass[mathfont=ptmx]{uai2025} % ptmx math instead of Computer
                                         % Modern (has noticeable issues)
% \documentclass[mathfont=newtx]{uai2025} % newtx fonts (improves upon
                                          % ptmx; less tested, no support)
% NOTE: Only keep *one* line above as appropriate, as it will be replaced
%       automatically for papers to be published. Do not make any other
%       change above this note for an accepted version.

%% Choose your variant of English; be consistent
\usepackage[american]{babel}
% \usepackage[british]{babel}

%% Some suggested packages, as needed:
\usepackage{natbib} % has a nice set of citation styles and commands
    \bibliographystyle{plainnat}
    \renewcommand{\bibsection}{\subsubsection*{References}}
\usepackage{mathtools} % amsmath with fixes and additions
% \usepackage{siunitx} % for proper typesetting of numbers and units
\usepackage{booktabs} % commands to create good-looking tables
\usepackage{tikz} % nice language for creating drawings and diagrams

\newcommand{\dataset}{{\cal D}}
\newcommand{\fracpartial}[2]{\frac{\partial #1}{\partial  #2}}

% Packages

\usepackage[utf8]{inputenc}
\usepackage{enumitem}  
\usepackage{bm,amsfonts,amsmath, mathtools, dsfont}
\usepackage{comment}
\usepackage{hyperref}
\usepackage{bbm}
\usepackage{subfig}
\usepackage{adjustbox}
\usepackage{amsthm}
\usepackage{makecell}
\usepackage{hyperref}
\usepackage{float}

\usepackage{color}
% ATENCAO ******************************
%**************************************
% eu mudei essa biblioteca, isso é padrao do UAI?
% \usepackage{algorithm2e}

\usepackage[ruled,vlined]{algorithm2e}

% Environments

\newtheorem{prop}{Proposition}
\newtheorem{thm}{Theorem}
\newtheorem{Assumption}{Assumption}
\newtheorem{Lemma}{Lemma}
\newtheorem{Corollary}{Corollary}
\newtheorem{Definition}{Definition}
\newtheorem{Remark}{Remark}
\newtheorem{Example}{Example}

\usepackage{amsmath, amssymb}
\DeclareMathOperator*{\argmin}{arg\,min}

\definecolor{forestgreen}{rgb}{0.0, 0.27, 0.13}
\newcommand{\guilherme}[1]{ \textbf{\color{magenta}[Guilherme: #1]}  } % old comments
\newcommand{\rafael}[1]{ \textbf{\color{blue}[Rafael: #1]}  } % old comments
\newcommand{\luben}[1]{ \textbf{\color{forestgreen}[Luben: #1]}} % old comments

\newcommand{\thiago}[1]{ \textbf{\color{cyan}[Thiago: #1]}} % old comments

\newcommand{\vagner}[1]{ \textbf{\color{purple}[Vagner: #1]}} % old comments


\newcommand{\add}[1]{ \textbf{\color{green}[Add: #1]}}
%\newcommand{\remove}[1]{\ifdraft \textbf{\color{green} [#1]} \else \fi}
\newcommand{\remove}[1]{ \textbf{\color{red}[remover: #1]}} % old 



\tikzset{every picture/.style={line width=0.75pt}} %set default line width to 0.75pt        

\newcommand{\epicscorescheme}{

\tikzset{every picture/.style={line width=0.75pt}} %set default line width to 0.75pt        

\begin{tikzpicture}[x=0.75pt,y=0.75pt,yscale=-1,xscale=1]
%uncomment if require: \path (0,383); %set diagram left start at 0, and has height of 383

%Right Arrow [id:dp22506809845274267] 
\draw  [color={rgb, 255:red, 74; green, 144; blue, 226 }  ,draw opacity=1 ][fill={rgb, 255:red, 74; green, 144; blue, 226 }  ,fill opacity=1 ] (136,23.5) -- (150.2,23.5) -- (150.2,21.08) -- (159.67,25.92) -- (150.2,30.75) -- (150.2,28.33) -- (136,28.33) -- cycle ;
%Right Arrow [id:dp43566076453433733] 
\draw  [color={rgb, 255:red, 74; green, 144; blue, 226 }  ,draw opacity=1 ][fill={rgb, 255:red, 74; green, 144; blue, 226 }  ,fill opacity=1 ] (366,23.5) -- (380.2,23.5) -- (380.2,21.08) -- (389.67,25.92) -- (380.2,30.75) -- (380.2,28.33) -- (366,28.33) -- cycle ;
%Image [id:dp6931317746433254] 
\draw (263.65,260.79) node  {\includegraphics[width=124.5pt,height=124.5pt]{figures/conformal_scores.png}};
%Right Arrow [id:dp3926293512157466] 
\draw  [color={rgb, 255:red, 74; green, 144; blue, 226 }  ,draw opacity=1 ][fill={rgb, 255:red, 74; green, 144; blue, 226 }  ,fill opacity=1 ] (455.07,71.32) -- (455.08,80.96) -- (458.66,80.96) -- (451.51,87.4) -- (444.34,80.98) -- (447.92,80.97) -- (447.91,71.33) -- cycle ;
%Right Arrow [id:dp12024575621574463] 
\draw  [color={rgb, 255:red, 74; green, 144; blue, 226 }  ,draw opacity=1 ][fill={rgb, 255:red, 74; green, 144; blue, 226 }  ,fill opacity=1 ] (546,23.5) -- (560.2,23.5) -- (560.2,21.08) -- (569.67,25.92) -- (560.2,30.75) -- (560.2,28.33) -- (546,28.33) -- cycle ;
%Image [id:dp0954851617021295] 
\draw (84,260.79) node  {\includegraphics[width=124.5pt,height=125.41pt]{figures/QR_fitted.png}};
%Image [id:dp8206955411263626] 
\draw (451.5,260.79) node  {\includegraphics[width=121.88pt,height=121.88pt]{figures/conformal_score_predictive.png}};
%Image [id:dp8994652736396633] 
\draw (621.25,260.79) node  {\includegraphics[width=119.25pt,height=122.04pt]{figures/predictive_band_EPIC.png}};
%Straight Lines [id:da04141214528677106] 
\draw [color={rgb, 255:red, 241; green, 19; blue, 19 }  ,draw opacity=1 ] [dash pattern={on 0.84pt off 2.51pt}]  (575.87,259.78) -- (564,350.39) ;
%Curve Lines [id:da9719556813126957] 
\draw [color={rgb, 255:red, 244; green, 9; blue, 37 }  ,draw opacity=1 ] [dash pattern={on 0.84pt off 2.51pt}]  (574,350.39) .. controls (583,314.78) and (622,237.39) .. (672,226.39) ;
%Straight Lines [id:da3044188563089476] 
\draw [color={rgb, 255:red, 237; green, 13; blue, 13 }  ,draw opacity=1 ] [dash pattern={on 0.84pt off 2.51pt}]  (656.4,262.2) -- (660.57,311.6) -- (661,353.39) ;
%Right Arrow [id:dp4374797856764401] 
\draw  [color={rgb, 255:red, 74; green, 144; blue, 226 }  ,draw opacity=1 ][fill={rgb, 255:red, 74; green, 144; blue, 226 }  ,fill opacity=1 ] (455.07,124.32) -- (455.08,133.96) -- (458.66,133.96) -- (451.51,140.4) -- (444.34,133.98) -- (447.92,133.97) -- (447.91,124.33) -- cycle ;

% Text Node
\draw (42.5,15.42) node [anchor=north west][inner sep=0.75pt]  [font=\large] [align=left] {{ \textbf{Base model}}};
% Text Node
\draw (63.5,98.4) node [anchor=north west][inner sep=0.75pt]  [font=\scriptsize]  {$q_{\alpha _{1}} ,\ q_{ \alpha _{2}}$};
% Text Node
\draw (205.65,15.42) node [anchor=north west][inner sep=0.75pt]  [font=\large] [align=left] {{\textbf{Conformal score}}};
% Text Node
\draw (171.65,98.4) node [anchor=north west][inner sep=0.75pt]  [font=\scriptsize]  {$s( x,\ y) =\max\{q_{\alpha _{1}}( x) -y,\ y-q_{\alpha _{2}}( x)\}$};
% Text Node
\draw (408,14.92) node [anchor=north west][inner sep=0.75pt]  [font=\large] [align=left] {{\fontfamily{pcr}\selectfont EPICSCORE}};
% Text Node
\draw (395,51) node [anchor=north west][inner sep=0.75pt]  [font=\scriptsize] [align=left] {\begin{minipage}[lt]{78.41pt}\setlength\topsep{0pt}
\begin{center}
$\displaystyle F( s( x,y) |x,D) \text{ in } \displaystyle \mathcal{D}_{\text{cal,1 }}$
\end{center}

\end{minipage}};
% Text Node
\draw (424.5,150) node [anchor=north west][inner sep=0.75pt]  [font=\scriptsize] [align=left] {cutoff $\displaystyle t_{1\ -\ \alpha }$};
% Text Node
\draw (576.25,15.42) node [anchor=north west][inner sep=0.75pt]  [font=\large,opacity=1 ] [align=left] {\textbf{Prediction}};
% Text Node
\draw (526.27,354.45) node [anchor=north west][inner sep=0.75pt]   [align=left] {\begin{minipage}[lt]{51.53pt}\setlength\topsep{0pt}
\begin{center}
{\scriptsize High epistemic }\\{\scriptsize uncertainty}
\end{center}

\end{minipage}};
% Text Node
\draw (661.4,372.45) node   [align=left] {\begin{minipage}[lt]{68pt}\setlength\topsep{0pt}
\begin{center}
{\scriptsize Low epistemic}\\{\scriptsize uncertainty}
\end{center}

\end{minipage}};
% Text Node
\draw (368,98.4) node [anchor=north west][inner sep=0.75pt]  [font=\scriptsize]  {$s'( x,y) \ =\ F( s( x,y) |x,D) \ \text{in} \ \mathcal{D}_{\text{cal, 2}}$};
% Text Node
\draw (544.75,98.4) node [anchor=north west][inner sep=0.75pt]  [font=\scriptsize]  {$ \begin{array}{l}
[q_{\alpha _{1}}(x_{n+1}) -F^{-1}(t_{1 - \alpha} |x_{n+1} ,D) ,\\
q_{\alpha _{2}} x_{n+1}) +F^{-1}(t_{1 -\alpha} |x_{n+1} ,D)]
\end{array}$};


\end{tikzpicture}
}

\newcommand{\imageexample}{

\begin{tikzpicture}[x=0.75pt,y=0.75pt,yscale=-1,xscale=1]
%uncomment if require: \path (0,623); %set diagram left start at 0, and has height of 623

%Image [id:dp04540270577738248] 
\draw (336,254.75) node  {\includegraphics[width=277.5pt,height=175.88pt]{figures/t_sne_figure.png}};
%Straight Lines [id:da956117260926385] 
\draw [color={rgb, 255:red, 208; green, 2; blue, 27 }  ,draw opacity=1 ] [dash pattern={on 0.84pt off 2.51pt}]  (485.5,379) -- (310,298) ;
%Straight Lines [id:da7817210183345735] 
\draw [color={rgb, 255:red, 208; green, 2; blue, 27 }  ,draw opacity=1 ] [dash pattern={on 0.84pt off 2.51pt}]  (489.5,176.5) -- (437.5,222) ;
%Straight Lines [id:da7125062275715932] 
\draw  [dash pattern={on 0.84pt off 2.51pt}]  (219.5,267.5) -- (179,193.5) ;
%Image [id:dp8892650292515647] 
\draw (574.5,423.66) node  {\includegraphics[width=125.25pt,height=101.25pt]{figures/inlier_image_1.png}};
%Straight Lines [id:da9813095471922089] 
\draw  [dash pattern={on 0.84pt off 2.51pt}]  (310.5,338.5) -- (189,390.5) ;
%Image [id:dp2503039319662854] 
\draw (95.1,160.5) node  {\includegraphics[width=128.25pt,height=94.5pt]{figures/outlier_image_0.png}};
%Image [id:dp1234749272718505] 
\draw (98.6,423.66) node  {\includegraphics[width=130.8pt,height=96.49pt]{figures/outlier_image_1.png}};
%Image [id:dp2291580865994045] 
\draw (572.5,160.5) node  {\includegraphics[width=131.63pt,height=96.75pt]{figures/inlier_image_0.png}};

% Text Node
\draw (13.16,26.97) node [anchor=north west][inner sep=0.75pt]  [font=\LARGE] [align=left] {{\tiny \textbf{High Epistemic Uncertainty}}};
% Text Node
\draw (485.46,26.97) node [anchor=north west][inner sep=0.75pt]  [font=\LARGE] [align=left] {{\tiny \textbf{\textcolor[rgb]{0.92,0.04,0.04}{Low Epistemic Uncertainty}}}};
% Text Node
\draw (1.75,232.5) node [anchor=north west][inner sep=0.75pt]  [font=\small] [align=left] {\begin{minipage}[lt]{121.24pt}\setlength\topsep{0pt}
\begin{center}
{\footnotesize {\fontfamily{ptm}\selectfont \textcolor[rgb]{0.25,0.46,0.02}{APS set: \{}\textcolor[rgb]{0.25,0.46,0.02}{\textbf{bear}}\textcolor[rgb]{0.25,0.46,0.02}{, beaver, catterpilar,}}}\\{\footnotesize \textcolor[rgb]{0.25,0.46,0.02}{{\fontfamily{ptm}\selectfont chimpanzee, crocodile, elephant, forest}}}\\{\footnotesize \textcolor[rgb]{0.25,0.46,0.02}{{\fontfamily{ptm}\selectfont palm\_tree, possum, rabbit, willow\_tree\}}}}
\end{center}

\end{minipage}};
% Text Node
\draw (7.25,274.2) node [anchor=north west][inner sep=0.75pt]  [font=\small] [align=left] {\begin{minipage}[lt]{113.85pt}\setlength\topsep{0pt}
\begin{center}
{\footnotesize {\fontfamily{ptm}\selectfont \textcolor[rgb]{0.82,0.01,0.11}{EPIC set: \{}\textcolor[rgb]{0.82,0.01,0.11}{\textbf{bear}}\textcolor[rgb]{0.82,0.01,0.11}{, beaver, caterpillar,}}}\\{\fontfamily{ptm}\selectfont {\footnotesize \textcolor[rgb]{0.82,0.01,0.11}{chimpanzee, crocodile, elephant, }}}\\{\fontfamily{ptm}\selectfont {\footnotesize \textcolor[rgb]{0.82,0.01,0.11}{flatfish, forest, lion, otter, }}}\\{\fontfamily{ptm}\selectfont {\footnotesize \textcolor[rgb]{0.82,0.01,0.11}{palm\_tree, porcupine, possum, rabbit,}}}\\{\fontfamily{ptm}\selectfont {\footnotesize \textcolor[rgb]{0.82,0.01,0.11}{skunk, whale, willow\_tree\}}}}
\end{center}

\end{minipage}};
% Text Node
\draw (487.3,232) node [anchor=north west][inner sep=0.75pt]  [font=\small] [align=left] {\begin{minipage}[lt]{139.73pt}\setlength\topsep{0pt}
\begin{center}
{\footnotesize \textcolor[rgb]{0.25,0.46,0.02}{{\fontfamily{ptm}\selectfont APS set: \{clock, cup, elephant,}}}\\{\footnotesize {\fontfamily{ptm}\selectfont \textcolor[rgb]{0.25,0.46,0.02}{\textbf{keyboard}}\textcolor[rgb]{0.25,0.46,0.02}{, lamp, lawn\_mower, road,}}}\\{\footnotesize \textcolor[rgb]{0.25,0.46,0.02}{{\fontfamily{ptm}\selectfont shrew, skyscrapper, snail, streetcar, telephone\}}}}
\end{center}

\end{minipage}};
% Text Node
\draw (504.8,268.9) node [anchor=north west][inner sep=0.75pt]  [font=\small] [align=left] {\begin{minipage}[lt]{115.76pt}\setlength\topsep{0pt}
\begin{center}
{\footnotesize {\fontfamily{ptm}\selectfont \textcolor[rgb]{0.82,0.01,0.11}{EPIC set: \{clock, elephant, }\textcolor[rgb]{0.82,0.01,0.11}{\textbf{keyboard}}\textcolor[rgb]{0.82,0.01,0.11}{,}}}\\{\footnotesize {\fontfamily{ptm}\selectfont \textcolor[rgb]{0.82,0.01,0.11}{lamp, skyscraper, telephone\}}}}
\end{center}

\end{minipage}};
% Text Node
\draw (471.75,502) node [anchor=north west][inner sep=0.75pt]  [font=\small] [align=left] {\begin{minipage}[lt]{141.12pt}\setlength\topsep{0pt}
\begin{center}
{\footnotesize \textcolor[rgb]{0.25,0.46,0.02}{{\fontfamily{ptm}\selectfont APS set: \{aquarium\_fish, bee, caterpillar, crab,}}}\\{\footnotesize {\fontfamily{ptm}\selectfont \textcolor[rgb]{0.25,0.46,0.02}{dinosaur, lizard, shark, }\textcolor[rgb]{0.25,0.46,0.02}{\textbf{trout}}\textcolor[rgb]{0.25,0.46,0.02}{, turtle\}}}}
\end{center}

\end{minipage}};
% Text Node
\draw (469.75,531.5) node [anchor=north west][inner sep=0.75pt]  [font=\small] [align=left] {\begin{minipage}[lt]{143.79pt}\setlength\topsep{0pt}
\begin{center}
{\footnotesize {\fontfamily{ptm}\selectfont \textcolor[rgb]{0.82,0.01,0.11}{EPIC set: \{aquarium\_fish, crab, dinosaur, lizard}}}\\{\footnotesize {\fontfamily{ptm}\selectfont \textcolor[rgb]{0.82,0.01,0.11}{\textbf{trout}}\textcolor[rgb]{0.82,0.01,0.11}{\}}}}
\end{center}

\end{minipage}};
% Text Node
\draw (4.85,534.9) node [anchor=north west][inner sep=0.75pt]  [font=\small] [align=left] {\begin{minipage}[lt]{149.5pt}\setlength\topsep{0pt}
\begin{center}
{\footnotesize {\fontfamily{ptm}\selectfont \textcolor[rgb]{0.82,0.01,0.11}{EPIC set: \{aquarium\_fish, boy, bridge, bus, can, }}}\\{\footnotesize {\fontfamily{ptm}\selectfont \textcolor[rgb]{0.82,0.01,0.11}{castle, cloud, cockroach, crab, dolphin, elephant, }}}\\{\footnotesize {\fontfamily{ptm}\selectfont \textcolor[rgb]{0.82,0.01,0.11}{flatfish, girl, house, lobster, maple\_tree, }}}\\{\footnotesize {\fontfamily{ptm}\selectfont \textcolor[rgb]{0.82,0.01,0.11}{mountain, oak\_tree, orchid, palm\_tree, pine\_tree, }}}\\{\footnotesize {\fontfamily{ptm}\selectfont \textcolor[rgb]{0.82,0.01,0.11}{ray, rocket, rose, shark, skyscraper, streetcar, }}}\\{\footnotesize {\fontfamily{ptm}\selectfont \textcolor[rgb]{0.82,0.01,0.11}{sunflower, \ sweet\_pepper, television, trout,}}}\\{\footnotesize {\fontfamily{ptm}\selectfont \textcolor[rgb]{0.82,0.01,0.11}{turtle, whale, }\textcolor[rgb]{0.82,0.01,0.11}{\textbf{willow\_tree}}\textcolor[rgb]{0.82,0.01,0.11}{, worm\}}}}
\end{center}

\end{minipage}};
% Text Node
\draw (10.85,497) node [anchor=north west][inner sep=0.75pt]  [font=\small] [align=left] {\begin{minipage}[lt]{141.44pt}\setlength\topsep{0pt}
\begin{center}
{\footnotesize \textcolor[rgb]{0.25,0.46,0.02}{{\fontfamily{ptm}\selectfont APS set: \{aquarium\_fish, bridge, castle, house,}}}\\{\footnotesize \textcolor[rgb]{0.25,0.46,0.02}{{\fontfamily{ptm}\selectfont maple\_tree, pine\_tree, ray, shark, streetcar,}}}\\{\footnotesize \textcolor[rgb]{0.25,0.46,0.02}{{\fontfamily{ptm}\selectfont sweet\_pepper, trout, turtle, whale, worm\}}}}
\end{center}

\end{minipage}};
% Text Node
\draw (533.75,76) node [anchor=north west][inner sep=0.75pt]  [font=\small] [align=left] {{\footnotesize {\fontfamily{ptm}\selectfont True label: keyboard}}};
% Text Node
\draw (538.5,340.5) node [anchor=north west][inner sep=0.75pt]  [font=\small] [align=left] {{\footnotesize {\fontfamily{ptm}\selectfont True label: trout}}\\};
% Text Node
\draw (56.75,76) node [anchor=north west][inner sep=0.75pt]  [font=\small] [align=left] {{\footnotesize {\fontfamily{ptm}\selectfont True label: bear}}};
% Text Node
\draw (51.25,344.5) node [anchor=north west][inner sep=0.75pt]  [font=\small] [align=left] {{\footnotesize {\fontfamily{ptm}\selectfont True label: willow\_tree}}};


\end{tikzpicture}



\end{tikzpicture}


}


\newcommand{\x}{\mathbf{x}}
\newcommand{\y}{\mathbf{y}}
\newcommand{\Y}{\mathbf{Y}}
\newcommand{\X}{\mathbf{X}}
\renewcommand{\S}{\mathbf{S}}
\newcommand{\s}{\mathbf{s}}
%\newcommand{\P}{\text{Pr}}
\renewcommand{\P}{\mathbb{P}}   
\newcommand{\I}{\mathbb{I}}
\newcommand{\E}{\mathbb{E}}
\newcommand{\V}{\mathbb{V}}
\newcommand{\R}{\mathbb{R}}
\newcommand{\N}{\mathbb{N}}
\newcommand{\p}{\mathbf{p}}
\newcommand{\pv}{\text{p-val}}
\newcommand{\ind}{\perp\!\!\!\!\perp}
\newcommand{\Ind}{\mathbf{1}}

\newcommand{\train}{\textrm{train}}
\newcommand{\calib}{\textrm{cal}}
\newcommand{\test}{\textrm{test}}

\newcommand{\mP}{\mathbb{P}}
\newcommand{\mE}{\mathbb{E}}
\newcommand{\mV}{\mathbb{V}}
\newcommand{\mW}{\mathcal{W}}
\newcommand{\mK}{\mathcal{K}}
\newcommand{\bW}{\textbf{W}}
\newcommand{\mT}{\mathcal{T}} 	
\newcommand{\mA}{\mathcal{A}} 
\newcommand{\tV}{\text{Var}}
\newcommand{\bx}{\mathbf{x}}
\newcommand{\bX}{\textbf{X}}
\newcommand{\bY}{\textbf{Y}}
\newcommand{\bZ}{\textbf{Z}}
\newcommand{\tr}{\text{tr}}
\newcommand{\IC}{\text{IC}}
\newcommand{\ourmethod}{\texttt{EPICSCORE}}
\newcommand{\ourmethodshort}{\textrm{EPIC}}

\newcommand{\bcdot}{\boldsymbol{\cdot}}



\newcommand{\tval}{t'}

%% Self-defined macros
\newcommand{\swap}[3][-]{#3#1#2} % just an example

\title{Epistemic Uncertainty in Conformal Scores: A Unified Approach}

% The standard author block has changed for UAI 2025 to provide
% more space for long author lists and allow for complex affiliations
%
% All author information is authomatically removed by the class for the
% anonymous submission version of your paper, so you can already add your
% information below.
%
% Add authors
\author[1,2]{\href{mailto:<lucruz45.cab@gmail.com>?Subject=Your UAI 2025 paper}{Luben~M.~C.~Cabezas}{}$^*$}
\author[1]{Vagner~S.~Santos$^*$}
\author[1]{Thiago~R.~Ramos}
\author[1]{Rafael~Izbicki}
% Add affiliations after the authors
\affil[1]{%
    Department of Statistics\\
    Federal University of S\~ao Carlos\\
    S\~ao Carlos, S\~ao Paulo, Brazil
}
\affil[2]{%
    Institute of Mathematics and Computer Science\\
    University of S\~ao Paulo\\
    S\~ao Carlos, S\~ao Paulo, Brazil
}

\affil[*]{Equal contribution.}

  \begin{document}
\maketitle


\begin{abstract}
Conformal prediction methods create prediction bands with distribution-free guarantees but do not explicitly capture epistemic uncertainty, which can lead to overconfident predictions in data-sparse regions.
Although recent conformal scores have been developed to address this limitation, they are typically designed for specific tasks, such as regression or quantile regression. Moreover, they rely on particular modeling choices for epistemic uncertainty, restricting their applicability. We introduce \ourmethod, a model-agnostic approach that enhances any conformal score by explicitly integrating epistemic uncertainty. Leveraging Bayesian techniques such as Gaussian Processes, Monte Carlo Dropout, or Bayesian Additive Regression Trees, \ourmethod \  adaptively expands predictive intervals in regions with limited data while maintaining compact intervals where data is abundant. As with any conformal method, it preserves finite-sample marginal coverage. Additionally,  it also achieves asymptotic conditional coverage. Experiments  demonstrate its good performance compared to existing methods. Designed for compatibility with any Bayesian model, but equipped with distribution-free guarantees, \ourmethod\  provides a general-purpose framework for uncertainty quantification in prediction problems.
\end{abstract}

\section{Introduction}\label{sec:intro}


 Machine learning models traditionally focus on point predictions, estimating target variables from input features. However, understanding prediction uncertainty is crucial in many applications \citep{horta2015potential,freeman2017unified,izbicki2017converting,Schmidt2020Photo-z,dalmasso2020conditional,csillag2023amnioml,mian2024literature,Valle2024,frohlich2024personalizedus}. This has led to increased interest in uncertainty quantification methods, particularly conformal prediction, which constructs predictive regions with finite-sample validity under mild i.i.d. assumptions \citep{vovk2005algorithmic,shafer2008tutorial}. Unlike probabilistic models that rely on asymptotic assumptions or priors, conformal methods provide a distribution-free framework with guaranteed coverage.
 
Conformal prediction works by designing a non-conformity score, $s(\x,y)$, which measures the degree to which a given label value $y$ aligns with the feature values $\x$ of an instance.  Given a new input $\x_{\text{new}}$, the method constructs a predictive region by inverting the non-conformity score at a given confidence level (see Section \ref{sec:review_conformal}). The choice of $s(\x,y)$ is critical, as it directly influences the shape and informativeness of the resulting predictive regions \citep{angelopoulos2021gentle}. For instance, in regression problems, a standard choice is $s(\x,y)=|y-g(\x)|$, where $g(\x)$ is a point prediction for $Y$, typically an estimate of the regression function $\E[Y|\x]$ \citep{Lei2018}. Another common option is $s(\x,y)=\max\{ \widehat q_{\alpha_1}(\x)-y,y- \widehat q_{\alpha_2}(\x)\}$,  where $\widehat q_{\alpha_1},\widehat q_{\alpha_2}$  are quantile estimates of $Y|\x$ \citep{romano2019}.   

Despite offering distribution-free guarantees, standard conformal scores primarily capture \emph{aleatoric uncertainty}, which arises from inherent randomness in the data generation process - specifically, the fact that $\x$ does not uniquely determine $y$. For example, in the cases discussed above, both $E[Y|\x]$ and $q_\alpha(\x)$ reflect this form of uncertainty. However, an equally important source of uncertainty is \emph{epistemic uncertainty}, which stems from limitations in training data and the resulting lack of knowledge about the true data-generating process \citep{hullermeier2021aleatoric}.

Since standard conformal prediction does not explicitly account for epistemic uncertainty, it can produce misleadingly narrow predictive intervals in regions with little or no training data. Figure \ref{fig::reg_split} illustrates this issue in a regression problem (see Appendix \ref{sec::technicalReg} for technical details): in the range $x \in (7,8)$, there is essentially no training data, so we expect predictive regions to widen, reflecting the increased uncertainty. However, standard conformal methods (e.g., regression-split) instead produce overconfident, narrow intervals in this region. To address this limitation, we propose a novel approach that augments any conformal score with a measure of epistemic uncertainty. As shown in the figure, our method (\ourmethod) successfully expands the predictive regions where data is scarce, providing a more reliable uncertainty quantification.

\begin{figure}[h]
    \centering
    \begin{adjustbox}{width=\columnwidth, center}
        \includegraphics[width=\columnwidth]{figures/reg_split_sim_example.png}
    \end{adjustbox}
\caption{A comparison of predictive intervals from standard split-conformal regression and our proposed \ourmethod\ approach. While all methods maintain valid marginal coverage, standard conformal prediction often produces overconfidently narrow intervals in the data-scarce region (e.g. $x \in (7,8)$). Our method explicitly accounts for epistemic uncertainty, resulting in appropriately widened predictive intervals that better reflect total uncertainty when extrapolating beyond the training distribution.}
    \label{fig::reg_split}
\end{figure}

This limitation also appears in classification tasks, where conventional conformal methods produce overconfident prediction sets for test instances outside the training distribution. Using a ResNet-34 pre-trained on ImageNet (see Appendix \ref{sec::technicalClass} for details and additional results), Figure \ref{fig::images} compares Adaptive Prediction Sets (APS, \citealt{romano2020classification}) and \ourmethod\ on CIFAR-100 images. Both maintain valid coverage, but \ourmethod\ explicitly quantifies epistemic uncertainty: prediction sets expand for outliers while remaining concise for in-distribution examples. \ourmethod\ also shows improved Size-Stratified Coverage (SSC; \citealt{angelopoulos2021gentle}) by 33\% over APS, thus producing better prediction sets.  %Set Size Cost

\begin{figure*}[ht]
    \centering
        \includegraphics[width = 0.8\textwidth]{figures/aps_comparisson_v2.pdf}
\caption{Prediction sets from Adaptive Prediction Sets (APS) versus the proposed \ourmethod\ approach on CIFAR-100 images. Both methods maintain valid coverage, but \ourmethod\ explicitly quantifies epistemic uncertainty, resulting in adaptively expanded prediction sets for outlier images (e.g., those in data-sparse regions) while remaining concise for in-distribution examples. }
    \label{fig::images}
\end{figure*}



\subsection{Novelty}

We introduce \ourmethod \ (Epistemic Conformal Score), a novel nonconformity score that explicitly integrates epistemic uncertainty into the conformal prediction framework. Our key innovation is modeling the epistemic uncertainty of any given conformal score $s(\x,y)$ using a Bayesian process. This allows us to refine uncertainty quantification, particularly in regions of the feature space where data are sparse.

\ourmethod\ is  flexible and can leverage various Bayesian approaches, including Gaussian Processes, Bayesian Additive Regression Trees, or even approximations such as Neural Networks with Monte Carlo Dropout. By incorporating these probabilistic models, our method dynamically adjusts predictive regions to account for both aleatoric and epistemic uncertainty.

Despite \ourmethod\ using Bayesian models, it preserves the finite-sample validity guarantees of conformal prediction and also achieves asymptotic conditional coverage, a property that many existing conformal approaches lack. Furthermore, \ourmethod\ is model-agnostic and can be applied on top of any existing conformal score, making it an enhancement rather than a replacement.

\subsection{Relation to Other Work}


Several recent frameworks have attempted to model epistemic uncertainty in machine learning predictions (see, e.g., \citealt{he2023survey,tyralis2024review,wang2025aleatoric,Izbicki2025} and references therein). However, these methods generally lack guarantees on the coverage of their estimates.  


In contrast, conformal prediction offers valid coverage guarantees under the relatively weak assumption of i.i.d. data. Since its introduction \citep{vovk2005algorithmic,shafer2008tutorial}, the method has advanced significantly in both theoretical foundations and practical applications \citep{Lei2018,angelopoulos2021gentle,fontana2023conformal,Manokhin2024}. A key challenge in this area has been enhancing the quality of predictive regions, particularly by achieving  conditional coverage, which ensures validity conditional on specific feature values.


Since exact conditional coverage is unattainable without strong assumptions \citep{lei2014distribution}, research has focused on two strategies: (1) locally tuning cutoffs to adapt to the data distribution \citep{bostrom2020mondrian,foygel2021limits,guan2023localized,cabezas2024distribution,cabezas2025regression}, and (2) designing conformal scores that achieve conditional coverage asymptotically \citep{romano2019,izbicki2020flexible,chernozhukov2021distributional,izbicki2022cd,dheur2024distribution,plassier2024conditionally}. Among these, 
the score introduced by \citet[Eq.~14]{dheur2025multi} is particularly relevant to our approach.
 This method
   transforms a nonconformity score $s$ via its estimated  cumulative distribution function, $s'(\x;y) = \widehat{F}(s(\x,y) | \x)$, improving conditional coverage. While $s'$ ensures asymptotic conditional coverage, it does not explicitly model epistemic uncertainty. Our approach builds on this idea by incorporating epistemic uncertainty in $s'$. In particular, we show that as the calibration sample grows, \ourmethod\ achieves asymptotic conditional coverage (Theorem~\ref{thm::conditional_coverage}).


Bayesian methods have  been explored as a means to incorporate epistemic uncertainty into conformal prediction. One approach is to use Bayesian predictive sets and subsequently apply conformal methods to adjust their marginal coverage. However, existing techniques typically do not leverage existing conformal scores explicitly, and do not lead to  asymptotic conditional coverage  \citep{vovk2005algorithmic,fong2021conformal,wasserman2011frasian}.

Other recent studies have attempted to incorporate epistemic uncertainty directly into existing conformal scores. 
\citet{cocheteux2025uncertainty} modifies the Weighted regression-split nonconformity score \citep{Lei2018},   $s(\x,y) = |y - g(\x)| / \sigma(\x)$, by redefining 
$\sigma(\x)$ to capture epistemic uncertainty about $Y$, rather than aleatoric uncertainty only as in the original formulation. This uncertainty 
is estimated via Monte Carlo Dropout, which we also employ in some experiments. However, our approach is more flexible, as it accommodates 
any nonconformity score and, unlike this method, ensures asymptotic conditional coverage.

Another approach is proposed by \citet{rossellini2024integrating}, who modify the conformal 
quantile regression (CQR; \citealt{romano2019}) score function, $s(\x,y) = \max\{\widehat{q}_{\alpha_1}(\x) - y, y - \widehat{q}_{\alpha_2}(\x)\}$, to
account for epistemic uncertainty in the quantile estimates. While this adaptation shows promising results, it is restricted to quantile regression 
models and does not generalize to other conformal scores. Moreover, it necessarily measures epistemic uncertainty using ensembles.

Other efforts to model epistemic uncertainty include \citet{jaber2024conformal} and \citet{pion2024gaussian}, which 
integrate conformal methods with Gaussian processes. Although these approaches yield calibrated sets for Gaussian processes, 
they are  tailored to this class of models and cannot be applied to other frameworks, such as Bayesian Additive Regression 
Trees \citep{chipman2010bart}. In contrast, \ourmethod\ is model-agnostic and can be applied to any Bayesian model for epistemic uncertainty.


\section{Methodology}\label{sec:methodology}

\subsection{Review of Conformal Prediction}
\label{sec:review_conformal}


Conformal prediction constructs valid prediction regions $R(\X_{n+1})$ under minimal assumptions. A widely used approach is split conformal prediction \citep{papadopoulos2002inductive,Lei2018}, which partitions the data into a training set $\mathcal{D}_\train$ and a calibration set $\mathcal{D}_\calib=\{(\X_1,Y_1), \dots, (\X_n,Y_n)\}$.  The training set is used to fit a nonconformity score $s(\x,y)$ such as those described in the introduction.
The conformal prediction region is given by
$$
R(\x_{n+1}) = \{ y : s(\x_{n+1}, y) \leq t_{1-\alpha} \}.
$$
The value $t_{1-\alpha}$ is set using the calibration set. Concretely,
$t_{1-\alpha}$ is set to be the $(1-\alpha)$-quantile of the calibration scores,
\[
t_{1-\alpha} = \text{Quantile}_{1-\alpha} \{ s(\X_i, Y_i) : (\X_i, Y_i) \in \mathcal{D}_\calib \}.
\]
This construction  ensures that the prediction set for a new observation $(\X_{n+1}, Y_{n+1})$ satisfies marginal coverage:
\[
\mathbb{P} (Y_{n+1} \in R(\X_{n+1})) \geq 1 - \alpha.
\]
 
 The choice of $s$ is critical in determining the shape and other properties of $R$. In the next section, we introduce a new conformal score that measures the epistemic uncertainty around any given score $s$. 

\subsection{Our Approach - \ourmethod}


We assume that a nonconformity score $s(\x, y)$ is already defined based on the training set.
Our starting point is to define a family of distributions
 that model the aleatoric uncertainty of $s(\X,Y)$ given $\X$, which is a set of distributions indexed by a parameter $\theta \in \Theta$. We denote this family by
  $\mathcal{F} = \{f(s|\x,\theta): \theta \in \Theta\}$. 
  This formulation is very general; $\Theta$ may even represent a nonparametric space. 
  
  
To construct \ourmethod, we adopt a Bayesian model that places a prior distribution over $\Theta$ (or equivalently, over $\mathcal{F}$). For simplicity, we assume this prior has a density $f(\theta)$, though the method is generally applicable.  This prior captures epistemic uncertainty in the data-generating process. In our experiments, we use Gaussian Processes, Bayesian Additive Regression Trees, and approximate Bayesian Mixture Density Neural Networks with Monte Carlo Dropout, but our approach supports any prior process.


We update the prior distribution \( f(\theta) \) using a subset of the calibration set \(\mathcal{D}_\calib\). Specifically, \(\mathcal{D}_\calib\) is split into two disjoint subsets: \(\mathcal{D}_{\calib, 1}\) and \(\mathcal{D}_{\calib, 2}\).  The first subset, \(\mathcal{D}_{\calib, 1}\) is transformed into the  dataset 
\[
D = \{(\mathbf{X}, S) : (\mathbf{X}, Y) \in \mathcal{D}_{\calib, 1}, \, S = s(\mathbf{X}, Y)\}.
\]
Using this transformed dataset, we compute the posterior distribution $f(\theta|D)$,   which reflects the updated epistemic uncertainty about the data-generating process after observing this data.
Our Bayesian model assumes that, given $\theta$, the data points $(\X,S)$ are independent and share the same conditional distribution $f(s|\x,\theta)$. 
Then, we derive the predictive cumulative distribution
$$F(s|\x,D)=\int F(s|\x,\theta)f(\theta|D)d\theta,$$
where   $F(s|\x,\theta)$ is the CDF given by model $\theta$.



Finally, our modified  nonconformity score, \ourmethod, is defined
as
\begin{align}
\label{eq:ourscore}
    s'(\x,y)=F(s(\x,y)|\x,D).
\end{align}
By construction, $s'$ incorporates epistemic uncertainty into $s$ by averaging the original score distribution, $F(s|\x,\theta)$, over the posterior $f(\theta|D)$, thus propagating the uncertainty about $\theta$ throughout the model.

Once $s'$ is computed, the prediction region for a new sample point is obtained using the standard split conformal method, with $s'$ serving as the nonconformity score. Specifically, $s'$ is evaluated for every sample in the second subset, $\mathcal{D}_{\calib, 2}$. The $(1-\alpha)$-quantile of these values, denoted as \( t_{1-\alpha} \), is then used to construct the prediction region
\[
R_{\ourmethodshort}(\x_{n+1}) = \{y: s'(\x_{n+1},y) \leq t_{1-\alpha} \},
\]
which, by the definition of $s'$, can be expressed in terms of the original nonconformity score $s$ as
\[
R_{\ourmethodshort}(\x_{n+1}) = \{y: s(\x_{n+1},y) \leq F^{-1}(t_{1-\alpha}|\x_{n+1},D) \}.
\] 

%As the task of computing the epistemic uncertainty of $s$ is harder than that of estimating the quantile $t_{1-\alpha}$, we typically choose $\mathcal{D}_{\calib, 1}$ to be smaller than $\mathcal{D}_{\calib, 2}$.

 \ourmethod{} is summarized in Algorithm \ref{alg:epicscore} and illustrated in Figure \ref{fig:EPICSCORE_scheme}. %Additional details on how the calibration set is split for fitting the predictive model and deriving the prediction-based cutoff can be found in Appendix XXX.
% \thiago{a biblioteca algorithm2e é padrao do UAI?}


\begin{figure*}[ht]
    \centering
    \includegraphics[width = 1.0\textwidth]{figures/EPICSCORE_scheme_v2.pdf}
    \caption{\ourmethod{} schematic illustration: Given a fitted base model (first panel), we begin by creating a nonconformity score and evaluating it over the calibration set (second panel). We then model the predictive distribution of the conformal score $s(\X, Y)$ using a specified family of models, integrating the epistemic uncertainty about the data-generating process. The predictive CDF of each original score defines a new conformal score, allowing threshold computation in the transformed space (third panel). Finally, leveraging these predictive-based cutoffs, we construct an adaptive prediction band that accounts for epistemic uncertainty (fourth panel).}
    \label{fig:EPICSCORE_scheme}
\end{figure*}


\begin{algorithm}[ht]
\caption{\ourmethod}
\label{alg:epicscore}
\KwIn{Data $\mathcal{D} = \{(\X_i,Y_i)\}_{i =1}^n$, conformal score $s(\X, Y)$, nominal level $\alpha$, test point $\X_{n + 1}$}

\textbf{Step I: Fit conformal scores} \\
1: Split data $\mathcal{D}$ into a training set $\mathcal{D}_{\text{train}}$ and a calibration set $\mathcal{D}_{\text{cal}}$. \\
2: Fit the conformal score $s(\X, Y)$ in $\mathcal{D}_{\text{train}}$.

\textbf{Step II: Fit the predictive function}\\
1: Split data $\mathcal{D}_{\text{cal}}$ into a training set $\mathcal{D}_{\text{cal},1}$ and a calibration set $\mathcal{D}_{\text{cal}, 2}$. \\
2. Fit predictive CDF $F(s|\x,D)$ using $\mathcal{D}_{\text{cal},1}$. \\
3. Compute \ourmethod{} conformal score $s'(\x,y)$ for all elements of $\mathcal{D}_{\text{cal}, 2}$ (Eq.~\eqref{eq:ourscore}) \\
4: Compute the $(1 - \alpha)$ empirical quantile $t_{1-\alpha}$ of the conformal scores. \\

\textbf{Step III: Compute prediction set}\\
3: Compute the set ${R}_{\ourmethodshort}(\X_{n + 1})$ as:
\vspace{-2.5mm}
   \begin{align*}
   {R}_{\ourmethodshort}(&\X_{n + 1}) = \{y \mid s'(\X_{n + 1}, y) \leq t_{1-\alpha} \} \\
   &= \{y \mid s(\X_{n + 1},y) \leq F^{-1}(t_{1- \alpha}|\X_{n + 1}, D) \}
   \end{align*}
\vspace{-3mm}
\end{algorithm}


\subsubsection{Special Cases}
\label{sec::special_cases}
We examine specific instances of conformal scores to provide further insight into how \ourmethod\ captures epistemic uncertainty.

\textbf{Regression}. 
If the original conformal score is $s(\x,y)=|y-g(\x)|$, 
the prediction regions given by \ourmethod\ have the form
% $$g(\x)\pm F^{-1}(t_{1-\alpha}|\x,D),$$ 
\[ g(\x_{n+1}) \pm F^{-1}(t_{1-\alpha}|\x_{n+1},D),\]
In particular, if $S|\x,D$ is modeled by a normal distribution with mean $\mu(\x,D)$ and standard deviation $\sigma(\x,D)$  the prediction sets will have the shape
$$\left( g(\x_{n+1})-\mu(\x_{n+1},D)\right) \pm \sigma (\x,D)\sqrt{2} \text{erf}^{-1}(2t_{1-\alpha} - 1),$$
where $\text{erf}^{-1}$ denotes the inverse error function. 
 This is equivalent to changing the original conformal score to
$|y-g(\x_{n+1})-\mu(\x_{n+1},D)|/\sigma(\x_{n+1},D)$, which is similar to the approach by \citet{cocheteux2025uncertainty}, although any process can be used to model the epistemic uncertainty in our version.

\textbf{Quantile Regression}. 
If the original conformal score \( s \) is given by Conformalized Quantile Regression (CQR) \citep{romano2019}, the prediction regions of \ourmethod\ have the form
\begin{equation*}
\begin{split}
[q_{\alpha_1}(\x_{n+1}) - F^{-1}(t_{1-\alpha}|\x_{n+1}, D), \\
q_{\alpha_2}(\x_{n+1}) + F^{-1}(t_{1-\alpha}|\x_{n+1}, D)].
\end{split}
\end{equation*}
Unlike the original CQR formulation, which expands or contracts the quantile regions $[q_{\alpha_1}(\x_{n+1}),q_{\alpha_2}(\x_{n+1})]$ by a constant factor \( t \), \ourmethod\ adjusts the regions dynamically based on the epistemic uncertainty at \( \x_{n+1} \).  
This approach is similar, but more flexible than previous methods, such as UACQR-S \citep{rossellini2024integrating}, which imposes a correction factor of the form \( t \times g(\x_{n+1}) \).  Also, any process can be used to model
$F^{-1}(t_{1-\alpha}|\x_{n+1}, D)$.

\textbf{Classification}.  
Let $s(\x,y)$ be any nonconformity score for classification. One example is the APS score
\begin{align}\label{eq:aps}s(\mathbf{x}, y) = \sum_{y' \in \mathcal{Y}: \widehat{\P}(y'|\mathbf{x}) > \widehat{\P}(y|\mathbf{x})} \widehat{\P}(y'|\mathbf{x}),
\end{align}
where $\widehat{\P}(y'|\mathbf{x})$ represents the predicted probabilities from any classifier. Another common choice is
\begin{align}\label{eq:cdsplit}s(\mathbf{x}, y) = - \widehat{\P}(y|\mathbf{x}),
\end{align} \citep{vovk2005algorithmic, valle2023quantifying,valle2024local}.
Since $Y$ is discrete, the score $s(\mathbf{x}, Y)$ is also discrete. Moreover, the cumulative distribution function of its predictive distribution can be computed using the predictive distribution of the labels, $\P(y|\x,D)$. In particular, \ourmethod\ is given by
\begin{align*}
    s'(\x,y)&= \P \left(s(\x,Y) \leq s(\x,y)|\x,D \right)\\
    &=\sum_{y'}\I(s(\x,y') \leq s(\x,y)) \P(y'|\x,D)\\
    &=\sum_{y': s(\x,y') \leq s(\x,y)} \P(y'|\x,D).
\end{align*}
This formulation reveals several key insights about \ourmethod\ for classification:
\begin{itemize}

\item If the initial classifier $\widehat \P(y|\x)$ is a neural network trained with Monte Carlo dropout or batch normalization, an approximation to $\P(y|\x,D)$ is readily available; one only needs to use the same technique at test time.  This is because these methods provide a variational approximation of Bayesian predictive distributions \citep{gal2016dropout, teye2018bayesian}, eliminating the need to compute the predictive distribution using a separate holdout set.

\item Both scoring functions in Eqs.~\ref{eq:aps} and \ref{eq:cdsplit} lead to the same \ourmethod\ score. Additionally, \ourmethod\ follows a similar structure to APS, with the key difference being that the estimates $\widehat{\P}(y|\mathbf{x})$ on the sum of Eq.~\ref{eq:aps} are replaced by the predictive distribution $\P(y|\x,D)$.

\item When $\x_{n+1}$ is located in a region with little training data, the estimated probabilities $\widehat{\P}(y|\mathbf{x})$ tend to be low for all labels, reflecting a high degree of uncertainty. Consequently, even the most conformal labels will have small probabilities, leading to lower values of $s'(\x_{n+1},y)$. This results in larger prediction sets $\{ y : s'(\x_{n+1}, y) \leq t_{1-\alpha} \}$ in areas with sparse data, effectively capturing uncertainty in underrepresented regions.
\item For large $D$, no epistemic uncertainty remains and, therefore, \ourmethod converges to the populational version of the APS score.
\end{itemize}


\section{Theory}\label{sec:theory}


Just like any split-conformal method, \ourmethod\ guarantees marginal coverage as long as the data is i.i.d., regardless of the chosen Bayesian model (see Appendix \ref{sec:proofs} for all proofs):

\begin{thm}
\label{thm::marginal}
Assuming that the data are independent and identically distributed (i.i.d.), the confidence region constructed by \ourmethod~ satisfies marginal coverage, that is,
\[
\P\left(Y \in {R}_{\ourmethodshort}(\X) \right) \geq  1 - \alpha.
\]
Moreover, if the fitted scores follow a continuous joint distribution, the upper bound also holds:
\[
\P\left(Y \in {R}_{\ourmethodshort}(\X) \right) \leq  1 - \alpha + \frac{1}{1 + |\mathcal{D}_{\calib, 2}|}.
\]
\end{thm}


We now analyze its conditional coverage properties.

As the sample size of the calibration set used to compute the posterior $D$ increases, the distribution function $F(s(\x,y)|\x,D)$ typically converges to $S(\X,Y)|\x,\theta^*$, where $\theta^*$ denotes the true parameter value \citep{schervish2012theory, bernardo2009bayesian}. Consequently, \ourmethod\ recovers the score proposed by \citet[Eq.~14]{dheur2025multi} in the limit of large calibration samples, which is exactly when epistemic uncertainty is negligible. This score is known to control asymptotic conditional coverage. We show that our proposed score exhibits the same property.

Formally, we assume that the 
predictive distribution converges to the true distribution of the conformal score \citep{bernardo2009bayesian,schervish2012theory}:
\begin{Assumption}\label{assumption:uniform_convergence}
For any $\varepsilon > 0$, we assume uniform convergence in probability over the randomness in $D$:
\begin{align*}
    \lim_{|D| \to \infty} \P\left( \sup_{s, \x} \left|F(s \mid \x, D) - F(s \mid \x, \theta^*)\right| > \varepsilon \right) = 0.
\end{align*}
\end{Assumption}


%Having established the uniform convergence assumption, we now turn to its key implication: the asymptotic validity of the confidence region constructed by \ourmethod. Intuitively, as the size of the calibration dataset grows, the preempirical estimates of the conditional distribution become increasingly accurate. This allows us to construct confidence regions that achieve the desired conditional coverage level in the limit.
Next, we show  that \ourmethod \ has  asymptotic conditional coverage:

\begin{thm}
Under Assumption \ref{assumption:uniform_convergence}, and assuming that the data are independent and identically distributed (i.i.d.), the confidence region constructed by \ourmethod~ satisfies the asymptotic conditional coverage condition, that is:
\label{thm::conditional_coverage}
\[
\lim_{|\mathcal{D}_{\calib}|\to \infty}\P\left(Y \in {R}_{\ourmethodshort}(\X) \mid \X = \x \right) = 1 - \alpha.
\]
\end{thm}



\section{Experiments}\label{sec:experiments}

In this section, we evaluate our framework against state-of-the-art baselines, applying \ourmethod\ to two initial conformal scores: (i) quantile-based and (ii) regression-based. Each version of \ourmethod\ is compared to appropriate baselines, which are detailed in the following subsections.


We consider three versions of \ourmethod, each using a different model for the predictive distribution (see implementation details in Appendix \ref{sec::comp_details}):  
\begin{itemize}
    \item \textbf{Bayesian Additive Regression Trees} \citep{chipman2010bart}: The BART model represents the score as a sum of regression trees:
    \begin{align*}
        s(Y, \X)|\X, \boldsymbol{\theta} \sim \phi\left( \sum_{i = 1}^m G_i(\X, T_i, M_i), \sigma \right) \; ,
    \end{align*}
    where $\phi$ denotes a probability distribution, $\sigma$ its associated scale, $G_i$ a binary tree with structure $T_i$ and leaf  values $M_i$. 
    %The set of all parameters, including the trees and their structures, is denoted by $\boldsymbol{\theta}$. 
    We set $\phi$ as a Normal distribution and $\sigma$ depending on $\X$ to incorporate heteroskedasticity \citep{pratola2020heteroscedastic}.
    
% For BART the posterior and predictive distributions are obtained using a Markov Chain Monte Carlo (MCMC) algorithm \citep{chipman2010bart}.
% To estimate the posterior distribution of parameters and obtain the predictive distribution, BART relies on a Markov Chain Monte Carlo (MCMC) algorithm \citep{chipman2010bart}. In this work, we leverage the scalable BART implementation available in \textit{pymc} \citep{quiroga2022bayesian}. 
% Depois separar alguns detalhes para o material suplementar
    \item \textbf{Gaussian Process (GP)} \citep{williams2006gaussian, schulz2018tutorial}: For the GP regression model, we assume the score follows the form
$
        s(Y,\x) = f(\x) + \varepsilon \;,
 $ with $\varepsilon \sim N(0, \sigma_{\varepsilon}^2)$ representing independent Gaussian noise, and $f(\x) \sim GP(m(\x), k(\x, \x'))$ is a Gaussian Process with mean function $m(\x)$ and covariance function $k(\x, \x')$.  
    %Since the GP prior is conjugate, the predictive score distribution is derived analytically, with updated uncertainty captured through the posterior mean and variance, both influenced by the kernel function. Given the high computational cost of matrix inversions, we
    We adopt variational approximations to the predictive distribution \citep{salimbeni2018natural}, which offer scalability.
    \item \textbf{Mixture Density Network with MC-Dropout \citep{bishop1994mixture, gal2016dropout}}: The Mixture Density Network (MDN) models the score distribution using a weighted sum of Gaussian components:
    \begin{align*}
        f(s(y,\x)|\x) = \sum_{k = 1}^K \pi_k(\x)N(s(y,\x) |\mu_k(\x), \sigma^2_k(\x)) \; ,
    \end{align*}
    where $N(\cdot)$ denotes the normal density and $\pi_k(\cdot)$, $\mu_k(\cdot)$, $\sigma_k(\cdot)$ are all modeled by neural networks, with $\sum_{k = 1}^K \pi_k(\x) = 1$. %The model is fitted by minimizing the negative log-likelihood function.
    %loss function:
    %\begin{align*}
     %   \mathcal{L} = -\sum_{i = 1}^N \log \left[ \sum_{k = 1}^K \pi_k(\x_i)N(y_i|\mu_k(\x_i), \sigma^2_k(\x_i)) \right] \; .
    %\end{align*}
    To derive a predictive distribution for the scores, we incorporate dropout at each MDN layer. By performing multiple stochastic forward passes using MC Dropout, we approximate the posterior distribution of the MDN parameters, thus  propagating uncertainty into the predictive score distribution \citep{gal2016dropout}.
\end{itemize}

 

Our comparisons are conducted using 13  datasets commonly employed for benchmarking in the conformal prediction literature: Airfoil \citep{dua2017uci}, Bike \citep{kaggle_bike_sharing_demand}, Concrete \citep{concrete_compressive_strength_165, dua2017uci}, Cycle \citep{combined_cycle_power_plant_294, dua2017uci}, Homes\cite{kaggle2016},Electric \citep{dua2017uci}, Meps19 \citep{romano2019}, Protein \citep{physicochemical_properties_of_protein_tertiary_structure_265, dua2017uci}, Star \citep{achilles2008tennessee}, SuperConductivity \citep{superconductivty_data_464}, WEC \citep{neshat2020optimisation}, WineRed \citep{wine_quality_186}, and WineWhite \citep{wine_quality_186}. Additional details on these datasets are provided in Table \ref{tab:realdata}  of the Appendix.

 We report the average performance across 50  runs, highlighting methods that achieve statistically significant improvements based on 95\% confidence interval of each evaluation metric. In each run, we randomly partition the data into  40\% for training, 40\% for calibration, and 20\% for testing.


 We use the Average Interval Score Loss (AISL) \citep{gneiting2007strictly} as our primary evaluation metric, as it balances coverage and interval length, favoring the narrowest valid prediction intervals. Additionally, we assess each method's efficiency by measuring (i) the average interval length, (ii) the marginal coverage, and (iii) the Pearson correlation between coverage and interval length on Appendix \ref{sec::additional_res}, with the latter serving as a proxy for the quality of conditional coverage \citep{feldman2021improving}. A detailed description of all evaluation metrics is provided in Appendix \ref{sec::evaluation_metrics}.

\subsection{Quantile-Regression Baselines}
 
 For quantile regression-based scores, we adopt CatBoost \citep{dorogush2018catboost} as the base quantile-regression model in all conformal methods. See Appendix \ref{sec::base_model_details} for details on hyperparameters. 

 %The performance metrics analyzed are AISL, Average Coverage, Interval Length, and Pearson Correlation, all calculated on the testing set.

%is a proper scoring rule for intervals and functions as a good approximate measure for conditional coverage\rafael{isso é argumentado em outros artigos? Pq a barber fala que usa ela?}

%The idea is to compare this new approach, \ourmethod, with existing methods. For this purpose, we evaluated their performance on different real databases. The datasets used in our evaluation include 

We compare \ourmethod\ to the following baselines:
\begin{itemize}
\item \textbf{CQR}  \citep{romano2019}, the conformal quantile regression method described in the introduction.
    \item \textbf{CQR-r}  \citep{sesia2020comparison}, which scales each derived cutoff by the interval width to produce adaptive intervals. As CQR, this approach accounts only for aleatoric uncertainty.
    \item \textbf{UACQR-P} and \textbf{UACQR-S} \citep{rossellini2024integrating}, which aims to integrate epistemic uncertainty into prediction intervals through ensemble-based statistics. We use  their default strategies, deriving UACQR-S correction factors from the ensemble standard deviation and computing UACQR-P percentiles using ensemble order statistics.
\end{itemize}
All baselines are fitted and evaluated using the implementation from \cite{rossellini2024integrating}.


%For the quantile regression setting, we use UACQR-P, UACQR-S \citep{rossellini2024integrating}, CQR \citep{romano2019}, and CQR-r \citep{sesia2020comparison} as comparison baselines. For UACQR-S and UACQR-P, we adopt their default ensemble-based strategies, computing UACQR-S correction factors using ensemble standard deviation and UACQR-P percentiles using ensemble order statistics. All baselines are fitted and evaluated using the implementation from \cite{rossellini2024integrating}.


\subsection{Regression baselines}

 
 For regression-based conformal scores, we use a neural network optimized with a penalized Mean Squared Loss.  Detailed descriptions of the architectures and hyperparameters used can be found in Appendix  \ref{sec::base_model_details}.


% For regression-based methods, we consider Regression Split \citep{lei2014distribution}, Locally-Weighted Regression Split \citep{Lei2018}, and Mondrian Conformal Regression \citep{bostrom2020mondrian} as baselines. For the Locally-Weighted and Mondrian approaches, we follow their default strategies for estimating conditional spread. In the Locally-Weighted method, we model the Mean Absolute Deviance by regressing the absolute residuals on $\X$ using the same model type as the base predictor. For Mondrian, we estimate conditional variance by fitting a Random Forest to the original data $(\X, Y)$ and computing an ensemble-based conditional variance. All baselines are implemented in our package \luben{citar repo anonimo}.
\begin{table*}[h]
\caption{Quantile regression AISL values for each method and dataset. The table reports the mean across 50 runs, with twice the standard deviation in brackets. Bold values indicate the best-performing method within a $95\%$ confidence interval. \ourmethod{} demonstrates strong performance across most datasets and consistently ranks among the top methods.}
\label{tab:aisl_quantile}
\centering
\begin{adjustbox}{max width=\textwidth}
\begin{tabular}{lccccccc}
\hline
\textbf{Dataset}  & \textbf{EPIC-BART}      & \textbf{EPIC-GP}        & \textbf{EPIC-MDN}       & \textbf{CQR}            & \textbf{CQR-r}          & \textbf{UACQR-P} & \textbf{UACQR-S}        \\ \hline
airfoil           & 19.361 (0.234)          & 19.704 (0.27)           & \textbf{18.799 (0.29)}  & 20.521 (0.234)          & 20.535 (0.236)          & 23.021 (0.337)   & 20.188 (0.3)            \\
bike $\times (10^1)$             & 44.722 (0.297)          & 47.818 (0.320)          & \textbf{43.858 (0.326)} & 45.628 (0.256)          & 45.638 (0.258)          & 53.413 (0.376)   & \textbf{43.815 (0.385)} \\
concrete          & \textbf{42.765 (0.723)} & 45.276 (0.764)          & 44.442 (0.8)            & 46.882 (0.681)          & 46.896 (0.683)          & 52.789 (1.097)   & 47.324 (1.349)          \\
cycle             & 34.435 (0.142)          & 35.054 (0.131)          & \textbf{34.077 (0.129)} & 39.218 (0.134)          & 39.408 (0.136)          & 43.775 (0.181)   & 35.346 (0.197)          \\
electric          & 0.099 (< 0.001)           & 0.096 (< 0.001)           & \textbf{0.082 (< 0.001)}  & 0.102 (0.001)           & 0.102 (0.001)           & 0.111 (0.001)    & 0.097 (< 0.001)           \\
homes $\times (10^5)$            & 7.739 (0.066)           & 8.098 (0.072)           & \textbf{7.225 (0.049)}  & 8.360 (0.075)           & 8.433 (0.078)           & 11.427 (0.131)   & 8.544 (0.107)           \\
meps19            & \textbf{65.085 (1.469)} & \textbf{64.907 (1.56)}  & \textbf{64.3 (1.528)}   & \textbf{64.239 (1.56)}  & \textbf{64.239 (1.56)}  & 71.015 (1.763)   & \textbf{63.737 (1.461)} \\
protein           & 17.687 (0.019)          & 18.096 (0.037)          & \textbf{17.417 (0.019)} & 17.7 (0.015)            & 17.7 (0.016)            & 18.149 (0.015)   & 17.691 (0.015)          \\
star $\times (10^1)$            & \textbf{98.466 (0.768)} & \textbf{98.033 (0.750)} & \textbf{98.725 (0.754)} & \textbf{97.770 (0.725)} & \textbf{97.791 (0.724)} & 99.782 (0.647)   & 99.809 (0.968)          \\
superconductivity & 74.37 (0.222)           & 80.278 (0.266)          & \textbf{70.212 (0.196)} & 75.496 (0.219)          & 75.508 (0.218)          & 87.929 (0.513)   & 73.971 (0.404)          \\
WEC $\times (10^5)$          &  2.925 (0.009)   & 2.665 (0.012)   & \textbf{2.374 (0.010)}  & 3.138 (0.009)         & 3.142 (0.009)           & 3.517 (0.010)   & 3.046 (0.010)           \\
winered           & \textbf{3.007 (0.058)}  & \textbf{3.009 (0.059)}  & \textbf{2.977 (0.05)}   & \textbf{2.979 (0.069)}  & \textbf{2.978 (0.069)}  & 3.059 (0.069)    & \textbf{2.999 (0.063)}  \\
winewhite         & 3.334 (0.03)            & 3.327 (0.034)           & \textbf{3.219 (0.03)}   & 3.316 (0.036)           & 3.315 (0.036)           & 3.378 (0.038)    & \textbf{3.2 (0.036)}    \\ \hline
\end{tabular}

\end{adjustbox}
\end{table*}

We compare \ourmethod\ to the following baselines: \begin{itemize} 
\item \textbf{Regression Split} \citep{lei2014distribution}, the conformal method based on residuals from a regression model described in the introduction. 
\item \textbf{Weighted Regression Split} \citep{Lei2018}, which multiplies the derived cutoff by a conditional Mean Absolute Deviance (MAD) estimate to yield adaptive intervals. The MAD is modeled by regressing the training set's absolute residuals on $\X$, using the same model architecture as the base predictor.
\item \textbf{Mondrian Conformal Regression} \citep{bostrom2020mondrian}, which enhances conditional coverage by adaptively partitioning the feature space using a binning scheme based on conditional variance. We estimate variance by fitting a Random Forest to $(\X, Y)$.
\end{itemize}



%\subsection{Metrics}
%\luben{Essa seção pode ser migrada para o material suplementar, deixei um espaço la já.}
%In real data, some limitations arise in evaluations due to not knowing the true distribution of the data. For this reason, we have adopted evaluation metrics that are widely used in the predictive interval literature. These do not require the distribution of the data, but are also applicable to simulated data. 

%\rafael{trocar para R para ficar igual ao resto do artigo}


\subsection{Results}

The mean average coverage is close to the nominal 90\% for all methods (Table \ref{tab:amc} and \ref{tab:amc_reg} of Appendix \ref{sec:additional}),  which is expected since all methods are conformal.

\begin{table*}[ht]
\caption{Regression AISL values for each method and dataset. The reported values represent the average across 50 runs, with two times the standard deviation in parentheses. Bolded values highlight the method with superior performance within a $95\%$ confidence interval. \ourmethod{} demonstrates competitive or superior performance compared to other methods.}
\label{tab:aisl_reg}
\centering
\begin{adjustbox}{max width=\textwidth}
\begin{tabular}{lcccccc}
\hline
\textbf{Dataset}  & \textbf{EPIC-BART}                & \textbf{EPIC-GP}          & \textbf{EPIC-MDN}                 & \textbf{Mondrian}       & \textbf{Reg-split}         & \textbf{Weighted}                 \\ \hline
airfoil           & \textbf{19.747 (0.767)}           & \textbf{20.287 (0.686)}   & \textbf{19.823 (0.675)}           & 21.532 (0.919)          & 21.201 (0.98)              & \textbf{20.276 (0.819)}           \\
bike $\times (10^1)$           & \textbf{36.381 (0.463)}           & 41.448 (0.575)            & \textbf{37.041 (0.452)}           & 38.190 (0.403)          & 43.918 (0.567)             & 37.773 (0.468)                    \\
concrete          & \textbf{52.098 (2.237)}           & \textbf{52.998 (2.359)}   & \textbf{51.648 (2.185)}           & 61.915 (2.815)          & \textbf{54.902 (2.634)}    & 58.399 (3.165)                    \\
cycle             & \textbf{19.418 (0.211)}           & \textbf{19.522 (0.221)}   & \textbf{19.436 (0.213)}           & \textbf{19.403 (0.226)} & \textbf{19.73 (0.208)}     & \textbf{19.49 (0.207)}            \\
electric          & \textbf{0.048 (\textless{}0.001)} & 0.049 (\textless{}0.001)  & \textbf{0.048 (\textless{}0.001)} & 0.05 (\textless{}0.001) & 0.05 (0.001)               & \textbf{0.048 (\textless{}0.001)} \\
homes $\times (10^5)$             & 5.921 (0.0716)                    & 6.192 (0.0689)            & \textbf{5.546 (0.0545)}           & 5.710 (0.053)           & 7.569 (0.098)              & 5.860 (0.056)                     \\
meps19            & 86.039 (2.421)                    & 87.086 (2.405)            & \textbf{75.061 (1.807)}           & 79.192 (1.821)          & 109.83 (2.695)             & 92.433 (3.259)                    \\
protein           & 18.885 (0.054)                    & 18.772 (0.065)            & 17.735 (0.055)                    & \textbf{17.586 (0.051)} & 19.423 (0.055)             & 18.314 (0.065)                    \\
star $\times (10^1)$ & \textbf{105.616 (1.255)}        & \textbf{106.112 (0.998)} & \textbf{106.368 (1.173)}        & 109.346 (1.119)       & \textbf{105.250 (1.038)} & 129.492 (1.657)   \\
superconductivity & 54.895 (0.364)                    & 59.16 (0.449)             & \textbf{53.406 (0.365)}           & 58.065 (0.313)          & 68.183 (0.418)             & 54.981 (0.345)                    \\
WEC $\times (10^5)$ & 1.437 (0.010) & 1.435 (0.011) & \textbf{1.283 (0.009)} & 
\textbf{1.294 (0.009)} & 1.620 (0.009) & 1.410 (0.009) \\
winered           & \textbf{3.152 (0.07)}             & \textbf{3.171 (0.064)}    & \textbf{3.101 (0.062)}            & 3.262 (0.069)           & \textbf{3.214 (0.063)}     & 3.415 (0.067)                     \\
winewhite         & \textbf{3.104 (0.027)}            & 3.187 (0.029)             & \textbf{3.129 (0.029)}            & \textbf{3.087 (0.023)}  & 3.181 (0.028)              & 3.189 (0.033)                     \\ \hline
\end{tabular}
\end{adjustbox}
\end{table*}

%Analyzing Table \ref{tab:aisl}, we observe a notable dominance of EPICSCORE using MDN with MC Dropout compared to the other methods under the AISL (Average Interval Score Loss) criterion, with results statistically significant at the 5\% level. The AISL metric evaluates the quality of prediction intervals by balancing two key aspects: the width of the intervals and the coverage of the true values. A lower AISL value indicates more precise and reliable prediction intervals, as it reflects both narrow intervals and high coverage accuracy. The superior performance of EPICSCORE with MDN and MC Dropout suggests that this method effectively captures uncertainty and provides well-calibrated prediction intervals, making it a robust choice for tasks requiring reliable probabilistic forecasts.

In the quantile regression setting, Table \ref{tab:aisl_quantile} highlights \ourmethod{}'s strong performance across all datasets, with the MDN-MC Dropout variant excelling in 12 out of 13 cases. Notably, \ourmethod{} outperforms all competitors on 7 datasets, effectively balancing coverage and interval precision. Additionally, Tables \ref{tab:il} and \ref{tab:pcorr} (Appendix \ref{sec:additional}) show that the MDN-MC Dropout variant produces narrower predictive intervals while maintaining good conditional coverage.


For regression, Table \ref{tab:aisl_reg} shows \ourmethod{}'s strong performance across most datasets. The MDN-MC Dropout variant excels in 12 out of 13 cases, while the BART version ranks among the top in 8 datasets. Additionally, Tables \ref{tab:il_reg} and \ref{tab:pcorr_reg} (Appendix \ref{sec:additional}) confirm its effectiveness in interval length, as well as approximate conditional coverage. These results highlight \ourmethod{}'s flexibility and robustness in capturing epistemic and aleatoric uncertainty across different conformal scores.

\section{Final Remarks}\label{sec:final}

We introduce \ourmethod, a novel conformal score that incorporates epistemic uncertainty into predictive regions. Using Bayesian modeling, \ourmethod\ dynamically adjusts any nonconformity score to account for epistemic uncertainty, ensuring coverage even in sparse regions. We prove it preserves marginal coverage and achieves asymptotic conditional coverage.
Empirical results show \ourmethod\ often outperforms alternatives, producing prediction regions that better reflect uncertainty while maintaining valid coverage. 


Unlike previous approaches that rely on specific modeling choices or task-dependent formulations, \ourmethod\ is fully model-agnostic. Any Bayesian model can estimate the epistemic uncertainty of a given conformal score, allowing practitioners to tailor the method to their application. This flexibility extends \ourmethod's applicability across regression, classification, and structured prediction problems.


Looking ahead, we will extend \ourmethod\ to settings with distribution shift, where capturing epistemic uncertainty is crucial due to data sparsity in some regions. By refining our approach for domain adaptation, we aim to maintain reliable predictive regions even when test distributions differ from training data.

Code to implement \ourmethod \ and reproduce the experiments is available at \url{https://github.com/Monoxido45/EPICSCORE}.

\begin{acknowledgements} % will be removed in pdf for initial submission,
						 % (without ‘accepted’ option in \documentclass)
                         % so you can already fill it to test with the
                         % ‘accepted’ class option


    L.M.C.C is grateful for the fellowship provided by São Paulo Research Foundation (FAPESP), grant 2022/08579-7. V. S. S. is grateful for the financial support from FAPESP (grant 2023/05587-1). R. I. is grateful for the financial support of FAPESP (grants 2019/11321-9 and 2023/07068-1) and 
CNPq (grants 422705/2021-7 and 305065/2023-8). 

\end{acknowledgements}

% References
\bibliography{uai2025-template}

\newpage

\onecolumn

\title{Epistemic Uncertainty in Conformal Scores: A Unified Approach\\(Appendix)}
\maketitle

\appendix

\section{Technical Details and Supplementary Results for the Introduction's Examples}  
\label{sec::technical}

\subsection{Regression}
\label{sec::technicalReg}


In this section, we detail the example presented by Figure \ref{fig::reg_split}. We simulate a scenario with two distinct dense regions exhibiting low aleatoric and epistemic uncertainty, separated by an intermediate, sparser region with high aleatoric and epistemic uncertainty. Given a sample size $n$, we first generate $\left \lfloor{0.425 \cdot n}\right \rfloor$ samples for each $X \sim U(0, 1.5)$ and $X \sim U(8, 10)$, ensuring that $85\%$ of the data comes from the two outer regions, reflecting low epistemic uncertainty. The corresponding response variable follows $Y \sim N(2\sin{X}, 0.1)$, which also reflects low aleatoric uncertainty. For the remaining $\left \lfloor{0.425 \cdot n}\right \rfloor$ samples, we draw $X$ from a transformed Beta distribution, $X \sim (\text{Beta}(8, 8) \cdot (8 - 1.5) ) + 1.5$, concentrating points in the intermediate region. Here, the response variable follows $Y \sim N(2 \sin{X}, 2.1)$, introducing high aleatoric uncertainty. Epistemic uncertainty is particularly elevated at the boundaries of this region.

In this setting, we use a K-nearest neighbors (KNN) algorithm with $k = 10$ as the regression base model $g(\x)$. We illustrate the difference between \ourmethod{} and established conformal prediction baselines for regression intervals, including Regression Split \citep{lei2014distribution}, Weighted Regression Split \citep{Lei2018}, and the Mondrian Conformal Regression \citep{bostrom2020mondrian}. For \ourmethod{}, we use the BART-based version, with $m = 100$ trees, default prior options for all parameters, and a heteroscedastic gamma distribution as the probability model for the conformal score (detailed in \ref{sec::bart_details}), which is appropriate given that the regression conformal score is non-negative and often asymmetric.

In terms of baselines, both the Weighted and Mondrian methods estimate the conditional spread to construct prediction intervals. The locally weighted approach models the Mean Absolute Deviance (MAD), $\E[|g(\X) - Y||\X]$, by regressing absolute residuals $|g(\x) - y|$ on $\X$ using the same model type as $g(\x)$. Meanwhile, the Mondrian method partitions the feature space using a binning scheme (or taxonomy) based on an estimation of conditional variance $\V[Y|\X]$, generally obtained using ensemble-based variance, commonly derived from an additionally fitted Random Forest. 

Visually, both baseline methods outperform regression split, providing adaptive prediction intervals that widen in regions with high aleatoric uncertainty and narrow in regions with low aleatoric uncertainty. However, both methods struggle to generate wider intervals in data-sparse regions, such as $x \in (1.5, 2)$ and $x \in (7,8)$. This limitation arises because these regions have low spread estimates due to insufficient data, leading to underestimated cutoffs. In contrast, \ourmethod{} offers widened predictive intervals in these regions, better capturing epistemic uncertainty, while still accurately representing uncertainty in data-rich areas.

\subsection{Classification}
\label{sec::technicalClass}
For the image classification example (introduced in Figure \ref{fig::images}), we used the publicly available CIFAR-100 dataset \citep{krizhevsky2009learning}, which consists of $60,000$ color images of size $32\times32$ spanning $100$ classes, with each class containing $600$ images. The dataset was split into training, calibration, and test sets, allocating $10\%$ for testing and $45\%$ each for training and calibration. We utilized a ResNet-34 model \citep{he2016deep} to extract 512-dimensional feature representations and trained a Random Forest classifier on the training set with default parameters except for the number of trees, which we fixed at $300$. The classifier reports an accuracy of $47\%$ in the test set. The Adaptive Prediction Set and \ourmethod{} were then applied to the calibration set, using the MDN MC-dropout variant of \ourmethod{}, which model's architecture is better detailed in \ref{sec::mc-dropout-architecture}. 
%botar seção so de arquitetura

In the classification setting, \ourmethod{} includes an alternative adaptation to handle the discrete nature of the conformal score (see Section \ref{sec::special_cases} for details). However, it can also be applied similarly to the regression and quantile regression settings by treating the conformal score as continuous, fitting a predictive distribution, and deriving adaptive cutoffs. This approach involves normalizing the score to better leverage the chosen models. Given the large number of classes in CIFAR-100, this approximation remains valid, as it produces more fine-grained probability vectors. In this example, we adopt this formulation of \ourmethod{}, using the APS score as the conformal score and deriving adaptive thresholds from its fitted predictive distribution. 

To provide both a broad and detailed performance comparison, we first assess each method’s coverage and set size using the size-stratified coverage (SSC) metric \citep{angelopoulos2021gentle}. Next, we examine and visualize the prediction set sizes for the top 150 outliers and inliers, evaluating how well each method captures epistemic uncertainty in outliers while remaining adaptive for inliers. For illustration purposes, we set $\alpha = 0.2$ across all analyses.

\subsubsection*{SSC metric}
The SSC metric aims to evaluate the calibration of prediction sets by stratifying them into $G$ bins $\{B_j\}_{j = 1}^G$ based on their cardinality. For $j < G$, the bin $B_j$ contains the prediction sets with cardinality $j$,  while $B_G$ includes all sets with at least $G$ elements. Formally, let  $I_j = \{i: R(\X_i) \in B_j\}$ denote the indices of prediction sets that fall into bin $B_j$. The SSC metric for a given prediction set method $R(\cdot)$ is then defined as:
\begin{align}
    SSC(R) = \min_{j \in \{1, \dots, G\}} \frac{1}{|I_j|} \sum_{i \in I_j} \I \left\{Y_i \in R(\X_i) \right\} \; .
\end{align}
Intuitively, this metric measures the minimum coverage of $R(\cdot)$ across different set sizes, assessing whether coverage remains stable despite changes in set cardinality. SSC values close to $1 - \alpha$ indicate strong coverage performance, while values farther from $1 - \alpha$ suggest greater violations of conditional coverage \citep{angelopoulos2021gentle}. For this analysis, we set $G = 15$. Table \ref{tab:ssc_methods} reports the SSC average values and 2 times the standard error across 10 runs for both methods.


% Intuitively, this metric captures the worst-case deviation of $R(\cdot)$ from the nominal coverage level $1-\alpha$ across different set sizes, ensuring that coverage consistency is maintained regardless of variability in set cardinality. For this example, we set $G = 30$. Table \ref{tab:sscv_methods} presents the computed SSCV values on the test set for both methods.

%\begin{align}
   % SSCV(R) = \sup_j \left| \frac{|\{i \in I_j: Y_i \in R(\X_i)\}|}{|I_j|} 
 % - (1 - \alpha)\right| \; .
% \end{align}

\begin{table}[!h]
\centering
\caption{Average SSC metric over 10 runs, with twice the standard deviation in brackets. \ourmethod{} achieves SSC values closer to the nominal level compared to APS.}
\label{tab:ssc_methods}
\begin{tabular}{lrr}
\toprule
\textbf{Method} & \textbf{SSC} & \textbf{2 * SE} \\
\midrule
\ourmethod-MDN & 0.734 & 0.014 \\
APS & 0.553  & 0.036 \\
\bottomrule
\end{tabular}
\end{table}

%\begin{tabular}{cc}
% \hline
% \textbf{Method}           & \textbf{SSC} \\ \hline
% \ourmethod-MDN & 0.110         \\
% APS                       & 0.300         \\ \hline
% \end{tabular}

These results show that \ourmethod-MDN achieves an SSC much closer to the nominal level of $0.8$ than APS, indicating more consistent coverage across different set sizes. In contrast, APS has a lower average SSC, reflecting greater deviations from the target coverage and potentially less reliable uncertainty quantification.

\subsubsection*{Outlier and inlier analysis}
To differentiate inliers from outliers for further analysis and comparison of each method's prediction sets, we first apply t-SNE \citep{van2008visualizing} to reduce the dimensionality of the feature space in the test set. We then use the Local Outlier Factor (LOF) method \citep{breunig2000lof} for outlier detection, leveraging a KNN-based density estimation to identify anomalies. The LOF score not only highlights outliers but also helps characterize typical (inlier) observations, offering a structured approach to assessing epistemic uncertainty across different regions of the data distribution. We fit LOF on the first two t-SNE dimensions, assuming a contamination rate of $10\%$ of the sample is contaminated (i.e., treating 
$10\%$ of the sample as outliers), and rank the top 150 outliers and inliers for analysis.
 
In general, we expect outliers to have wider prediction sets, as they are located in sparser regions of the feature space. In contrast, inliers are likely to have narrower sets, reflecting their position in denser, more typical regions of the data distribution. Beyond Figure \ref{fig::images}, Figures \ref{fig::outlier_images_prediction_sets} and \ref{fig::inlier_images_prediction_sets} provide additional examples that further illustrate this behavior, emphasizing how \ourmethod{} differentiates itself from APS. Additionally, Figure \ref{fig::outliers_inliers_set_sizes_boxplot} displays the distribution of set sizes for outliers and inliers across both methods. While both methods generate larger prediction sets for outliers than for inliers, \ourmethod{} shows higher set sizes and a more dispersed distribution for outliers compared to APS, while presenting a more concentrated distribution for inliers.

Furthermore, we observe that the APS set sizes do not exceed a cardinality of 20, highlighting its lack of adaptability. Overall, these results reinforce the flexibility of our method across regions with varying levels of epistemic uncertainty, showcasing its advantage over the standard APS by explicitly incorporating epistemic uncertainty into the cutoff derivation.


\begin{figure}[ht]
    \centering
    \includegraphics[width=0.8\linewidth]{figures/outlier_images_prediction_sets_extended.pdf}
    \caption{Additional outlier image prediction sets examples. \ourmethod{} consistently produces broader prediction sets for all selected outlier images, effectively capturing the high epistemic uncertainty associated with these observations.}
    \label{fig::outlier_images_prediction_sets}
\end{figure}



\begin{figure}[ht]
    \centering
    \includegraphics[width=0.8\linewidth]{figures/inlier_images_prediction_sets_extended.pdf}
    \caption{Additional inlier prediction sets examples. \ourmethod{} generates more compact prediction sets for all selected inliers while also preventing empty sets in one instance. This highlights its robustness and reliability in regions with low epistemic uncertainty.}
    \label{fig::inlier_images_prediction_sets}
\end{figure}

\begin{figure}[ht]
    \centering
    \includegraphics[width=0.675\linewidth]{figures/outliers_inlier_set_sizes_boxplot.pdf}
    \caption{Left: Prediction set sizes for the top 150 most outlying observations. Right: Prediction set sizes for the top 150 most typical (inlier) observations. Both methods consistently produce larger prediction sets for outliers compared to inliers, but \ourmethod{} shows a more dispersed boxplot with higher set sizes for outliers, and a more concentrated boxplot for inliers compared to APS.}
    \label{fig::outliers_inliers_set_sizes_boxplot}
\end{figure}

\newpage

\section{Additional Results and Details for Real Data Experiments}
\label{sec::additional_results}
In this section, we provide an overview of the evaluation metrics, dataset summaries, and additional results for both quantile and standard regression experiments. Additionally, we outline the architecture and hyperparameter configurations for each base model.

\subsection{Evaluation Metrics}
\label{sec::evaluation_metrics}

Let $R(\cdot)$ denote a generic prediction interval. Given a test set $(\mathbf{X}_1, Y_1), (\mathbf{X}_2, Y_2), \ldots, (\mathbf{X}_m, Y_m)$, we evaluate performance using the following metrics:

\begin{itemize}
    \item \textbf{Average Marginal Coverage (AMC)}:
    \begin{equation*}
        \text{AMC} = \frac{1}{m} \sum_{i=1}^{m} \I\left( Y_i \in R(\X_i) \right),
    \end{equation*}
    which is an estimate of the marginal coverage of $R$.
    
    \item \textbf{Average Interval Score Loss (AISL)} \citep{gneiting2007strictly}: %\rafael{citar referencia ao artigo original}
    \begin{align*}
        \text{AISL} = \frac{1}{m} \sum_{i=1}^{m} \Bigg[ &\left( \max \hat{R}(\X_i) - \min \hat{R}(\X_i)\right) \\
        &+ \frac{2}{\alpha} \cdot \left(\min \hat{R}(\X_i) - Y_i \right) \cdot \I \left\{Y_i < \min \hat{R}(\X_i) \right\} \\
        &+ \frac{2}{\alpha} \cdot \left(Y_i - \max \hat{R}(\X_i) \right)\cdot \I \left\{Y_i > \max \hat{R}(\X_i \right\} \Bigg] \; ,
    \end{align*}
    where $\min \hat{R}(\X)$ and  $\max \hat{R}(\X)$  represent the lower and upper bounds of the prediction interval, respectively, and $\alpha$ is the miscalibration level. The Interval Score Loss balances two key objectives: maintaining narrow prediction intervals while penalizing those that fail to cover $Y_i$,  with larger penalties for greater deviations. By averaging these scores across all instances, AISL provides a measure that prioritizes the shortest interval while ensuring sufficient coverage.

    \item \textbf{Interval Length (IL)}:
    \begin{equation*}
        \text{IL} =  \frac{1}{m} \sum_{i = 1}^{m} \max \hat{R}(\X_i) - \min \hat{R}(\X_i) \; ,
    \end{equation*}
    which measures the average interval length, reflecting the precision of the predictive intervals. Larger values correspond to wider, less informative intervals, while smaller values indicate more compact and precise intervals.
\item \textbf{Pearson Correlation between Coverage and Interval Length ($\rho$)} \citep{feldman2021improving}: This metric measures the correlation between the width of the prediction interval and its coverage, providing insight into potential conditional coverage violations. Specifically,
\begin{equation*} \rho = \left| \frac{\text{Cov}(\mathbf{C}, \mathbf{W})}{\sigma_{\mathbf{C}} \sigma_{\mathbf{W}}} \right| , \end{equation*}

where $\mathbf{C} = (C_1, \ldots, C_m)$ represents a binary vector, with $C_i = \I(Y_i \in R(\mathbf{X}_i))$ indicating whether the prediction interval $R(\mathbf{X}_i)$ covers $Y_i$, and $\mathbf{W} = (W_1, \ldots, W_m)$, where $W_i = \max \hat{R}(\mathbf{X}_i) - \min \hat{R}(\mathbf{X}_i)$.
A strong correlation between coverage and interval width suggests a potential violation of conditional coverage, which requires their independence \citep{feldman2021improving}. However, $\rho = 0$ does not guarantee conditional coverage, as non-adaptive methods like regression split can achieve zero correlation by maintaining constant-width intervals \citep{rossellini2024integrating}. Thus, while this metric provides a useful proxy for assessing conditional coverage, it is not a definitive measure.

%\citet{feldman2021improving} show that $\rho = 0$ when conditional coverage is achieved. Any deviation from zero indicates a violation of this coverage.
\end{itemize}

\subsection{Additional results}
\label{sec:additional}

The dataset details are provided in Table \ref{tab:realdata}. The evaluation results for quantile regression are summarized in Tables \ref{tab:amc}, \ref{tab:il}, and \ref{tab:pcorr}, while the corresponding results for standard regression are presented in Tables \ref{tab:amc_reg}, \ref{tab:il_reg}, and \ref{tab:pcorr_reg}.

%% dataset table
\begin{table*}[ht]
\caption{Summary of the datasets used in this paper, including the number of samples ($n$), features ($p$), and access links.}
\label{tab:realdata}
\centering
\begin{adjustbox}{max width=0.825\textwidth}
\begin{tabular}{lccccccccccccp{15mm}}
\hline
\textbf{Dataset} & n & p & Source& \textbf{Dataset} & n & p &Source  \\
\hline
Airfoil & 1503  & 5  & \href{https://archive.ics.uci.edu/dataset/291/airfoil+self+noise}{Airfoil (UCI)} & Protein  & 45730 & 8 & \href{http://archive.ics.uci.edu/dataset/265/physicochemical+properties+of+protein+tertiary+structure}{Protein (UCI)}   \\

Bike & 10885 & 12 & \href{https://www.kaggle.com/code/rajmehra03/bike-sharing-demand-rmsle-0-3194/input?select=train.csv}{Bike (Kaggle)} & Star & 2161  & 48 & \href{https://dataverse.harvard.edu/dataset.xhtml?persistentId=doi:10.7910/DVN/SIWH9F}{Star (Harvard Dataverse)} \\

Concrete & 1030  & 8 & \href{https://archive.ics.uci.edu/dataset/165/concrete+compressive+strength}{Concrete (UCI)} & SuperConductivity & 21263 & 81 & \href{http://archive.ics.uci.edu/ml/datasets/Superconductivty+Data}{Superconductivity (UCI)} \\

Cycle & 9568 & 4 & \href{http://archive.ics.uci.edu/dataset/294/combined+cycle+power+plant}{Cycle (UCI)}  & Wave Energy Converter & 54000  & 49 & {\href{https://archive.ics.uci.edu/dataset/882/large-scale+wave+energy+farm}{WEC (UCI)}}  \\

Homes & 21613 & 17  & \href{https://www.kaggle.com/datasets/harlfoxem/housesalesprediction}{Home (Kaggle)} & Winered & 4898  & 11 & \href{https://archive.ics.uci.edu/dataset/186/wine+quality}{Wine red (UCI)} \\

Eletric & 10000 & 12  & \href{http://archive.ics.uci.edu/ml/datasets/Electrical+Grid+Stability+Simulated+Data+}{Electric (UCI)}  & WineWhite& 1599  & 11 & \href{https://archive.ics.uci.edu/dataset/186/wine+quality}{Wine white (UCI)}  \\


Meps19& 15781  & 141 & Meps19 \href{https://meps.ahrq.gov/mepsweb/data_stats/download_data_files_detail.jsp?cboPufNumber=HC-181}{(AHRQ site)}) & &   & &  \\
\hline
\end{tabular}
\end{adjustbox}
\end{table*}

\label{sec::additional_res}
\begin{table*}[ht]
\caption{Quantile regression Mean Average Coverage values across different methods and datasets. The reported values represent the average over 50 runs, with two times the standard deviation in parentheses. As expected for conformal methods, all approaches achieve marginal coverage close to the nominal level of 0.9}
\label{tab:amc}
\centering
\begin{adjustbox}{max width=0.925\textwidth}
\begin{tabular}{lccccccc}
\hline
\textbf{Dataset}  & \textbf{EPIC-BART} & \textbf{EPIC-GP} & \textbf{EPIC-MDN} & \textbf{CQR}  & \textbf{CQR-r} & \textbf{UACQR-P} & \textbf{UACQR-S} \\ \hline
airfoil           & 0.9 (0.008)        & 0.9 (0.01)       & 0.896 (0.01)      & 0.901 (0.007) & 0.901 (0.007)  & 0.907 (0.009)    & 0.9 (0.007)      \\
bike              & 0.9 (0.003)        & 0.898 (0.003)    & 0.899 (0.003)     & 0.899 (0.002) & 0.899 (0.002)  & 0.9 (0.002)      & 0.9 (0.002)      \\
concrete          & 0.905 (0.008)      & 0.9 (0.009)      & 0.898 (0.01)      & 0.897 (0.007) & 0.897 (0.007)  & 0.914 (0.012)    & 0.895 (0.007)    \\
cycle             & 0.899 (0.003)      & 0.9 (0.003)      & 0.898 (0.004)     & 0.901 (0.003) & 0.902 (0.003)  & 0.901 (0.002)    & 0.9 (0.002)      \\
electric          & 0.9 (0.003)        & 0.901 (0.003)    & 0.902 (0.004)     & 0.901 (0.002) & 0.901 (0.002)  & 0.901 (0.002)    & 0.901 (0.002)    \\
homes             & 0.902 (0.003)      & 0.902 (0.003)    & 0.9 (0.003)       & 0.901 (0.002) & 0.901 (0.002)  & 0.901 (0.002)    & 0.901 (0.002)    \\
meps19            & 0.9 (0.003)        & 0.9 (0.003)      & 0.9 (0.003)       & 0.899 (0.002) & 0.899 (0.002)  & 0.901 (0.002)    & 0.899 (0.002)    \\
protein           & 0.897 (0.003)      & 0.897 (0.003)    & 0.897 (0.003)     & 0.9 (0.001)   & 0.9 (0.001)    & 0.901 (0.001)    & 0.9 (0.001)      \\
star              & 0.902 (0.006)      & 0.902 (0.006)    & 0.903 (0.006)     & 0.902 (0.004) & 0.901 (0.004)  & 0.93 (0.013)     & 0.902 (0.004)    \\
superconductivity & 0.898 (0.004)      & 0.898 (0.003)    & 0.898 (0.003)     & 0.9 (0.002)   & 0.9 (0.002)    & 0.9 (0.001)      & 0.9 (0.002)      \\
WEC               & 0.897 (0.003)      & 0.899 (0.003)    & 0.897 (0.004)     & 0.9 (0.001)   & 0.9 (0.001)    & 0.899 (0.001)    & 0.9 (0.001)      
\\

winered           & 0.906 (0.008)      & 0.905 (0.008)    & 0.904 (0.007)     & 0.897 (0.006) & 0.897 (0.006)  & 0.903 (0.009)    & 0.897 (0.006)    \\
winewhite         & 0.901 (0.004)      & 0.9 (0.005)      & 0.9 (0.004)       & 0.898 (0.003) & 0.898 (0.003)  & 0.908 (0.009)    & 0.898 (0.003)    \\ \hline
\end{tabular}
\end{adjustbox}
\end{table*}

\begin{table}[ht]
\caption{Quantile regression Interval Length values across different methods and datasets. The reported values represent the average over 50 runs, with two times the standard deviation in parentheses. Bolded values indicate the best-performing method within a 95\% confidence interval. Overall, \ourmethod{} consistently produces narrower intervals in most cases.}
\label{tab:il}
\centering
\begin{adjustbox}{max width=0.925\textwidth}
\begin{tabular}{lccccccc}
\hline
\textbf{Dataset}  & \textbf{EPIC-BART}      & \textbf{EPIC-GP}        & \textbf{EPIC-MDN}       & \textbf{CQR}            & \textbf{CQR-r}           & \textbf{UACQR-P}        & \textbf{UACQR-S}        \\ \hline
airfoil           & 16.521 (0.18)           & \textbf{16.395 (0.237)} & \textbf{16.02 (0.222)}  & 17.087 (0.127)          & 17.09 (0.124)            & 18.838 (0.367)          & 16.656 (0.386)          \\
bike $\times (10^1)$             & 37.042 (0.190)          & 38.413 (0.229)          & \textbf{36.250 (0.218)} & 41.164 (0.150)          & 41.125 (0.153)           & 43.627 (0.591)          & 37.415 (0.386)          \\
concrete          & \textbf{36.537 (0.375)} & 38.328 (0.618)          & 37.614 (0.651)          & 39.477 (0.353)          & 39.486 (0.36)            & 44.425 (1.307)          & 39.853 (1.536)          \\
cycle             & \textbf{30.975 (0.128)} & 31.587 (0.132)          & \textbf{30.714 (0.146)} & 35.235 (0.095)          & 35.346 (0.093)           & 38.292 (0.195)          & 31.045 (0.207)          \\
electric          & 0.088 (0.001)           & 0.084 (0.001)           & \textbf{0.072 (0.001)}  & 0.09 (0.001)            & 0.09 (0.001)             & 0.097 (0.001)           & 0.084 (0.001)           \\
homes $\times (10^5)$ & 5.888 (0.028)           & 5.739 (0.028)           & 5.816 (0.040)           & 6.313 (0.024)           & 6.259 (0.024)            & 6.750 (0.0302)          & \textbf{5.312 (0.0309)} \\
meps19            & 32.996 (0.314)          & 29.268 (0.262)          & 29.16 (0.266)           & 28.948 (0.249)          & 28.949 (0.249)           & \textbf{27.857 (0.314)} & 32.763 (0.815)          \\
protein           & 16.195 (0.039)          & 16.485 (0.052)          & \textbf{16.048 (0.034)} & 16.378 (0.011)          & 16.378 (0.011)           & 16.797 (0.019)          & 16.356 (0.017)          \\
star $\times (10^1)$             & \textbf{81.851 (1.029)} & \textbf{81.760 (1.042)} & \textbf{82.050 (1.083)} & \textbf{81.359 (0.508)} & \textbf{81.396 (0.5117)} & 82.521 (0.618)          & 83.253 (0.952)          \\
superconductivity & 66.906 (0.205)          & 70.631 (0.366)          & \textbf{64.805 (0.197)} & 69.51 (0.144)           & 69.482 (0.145)           & 78.478 (0.7)            & 67.046 (0.492)          \\
WEC $\times (10^5)$              & 2.401 (0.0111)          & 2.076 (0.011)          & \textbf{1.890 (0.009)}  & 2.708 (0.004)           & 2.709 (0.004)            & 2.843 (0.008)           & 2.547 (0.007)           \\

winered           & 2.098 (0.034)           & 2.096 (0.035)           & 2.11 (0.031)            & \textbf{1.906 (0.011)}  & \textbf{1.902 (0.01)}    & 2.031 (0.025)           & 2.077 (0.042)           \\
winewhite         & 2.31 (0.017)            & 2.212 (0.029)           & 2.253 (0.016)           & \textbf{2.12 (0.006)}   & \textbf{2.121 (0.006)}   & \textbf{2.124 (0.012)}  & 2.222 (0.017)           \\ \hline
\end{tabular}
\end{adjustbox}
\end{table}



\begin{table}[ht]
\caption{Quantile regression Pearson correlation values across different methods and datasets. The reported values represent the average over 50 runs, with two times the standard deviation in parentheses. Bolded values indicate the best-performing method within a $95\%$ confidence interval. Overall, \ourmethod{} exhibits low correlation in most cases, reflecting strong conditional coverage performance.}
\label{tab:pcorr}
\centering
\begin{adjustbox}{max width = 0.925\textwidth}
\begin{tabular}{lccccccc}
\hline
\textbf{Dataset}  & \textbf{EPIC-BART}     & \textbf{EPIC-GP}       & \textbf{EPIC-MDN}      & \textbf{CQR}           & \textbf{CQR-r}         & \textbf{UACQR-P}       & \textbf{UACQR-S}       \\ \hline
airfoil           & \textbf{0.06 (0.013)}  & 0.18 (0.016)           & \textbf{0.071 (0.016)} & 0.125 (0.017)          & 0.129 (0.017)          & 0.132 (0.033)          & 0.108 (0.02)           \\
bike              & 0.171 (0.013)          & 0.138 (0.013)          & 0.213 (0.012)          & \textbf{0.062 (0.013)} & \textbf{0.069 (0.013)} & 0.108 (0.016)          & 0.091 (0.012)          \\
concrete          & 0.101 (0.021)          & 0.147 (0.02)           & \textbf{0.068 (0.013)} & \textbf{0.081 (0.017)} & \textbf{0.082 (0.017)} & 0.121 (0.034)          & \textbf{0.088 (0.02)}  \\
cycle             & \textbf{0.046 (0.008)} & \textbf{0.045 (0.009)} & 0.085 (0.01)           & 0.27 (0.011)           & 0.292 (0.011)          & 0.255 (0.011)          & 0.192 (0.012)          \\
electric          & \textbf{0.023 (0.006)} & \textbf{0.044 (0.009)} & 0.159 (0.008)          & 0.075 (0.006)          & 0.071 (0.006)          & 0.123 (0.01)           & 0.134 (0.008)          \\
homes             & 0.122 (0.007)          & 0.143 (0.01)           & \textbf{0.048 (0.008)} & 0.126 (0.008)          & 0.15 (0.007)           & 0.271 (0.008)          & 0.221 (0.007)          \\
meps19            & \textbf{0.017 (0.004)} & 0.084 (0.01)           & 0.069 (0.009)          & 0.08 (0.006)           & 0.08 (0.006)           & 0.128 (0.006)          & 0.051 (0.006)          \\
protein           & \textbf{0.031 (0.004)} & 0.15 (0.01)            & 0.073 (0.004)          & 0.094 (0.003)          & 0.094 (0.003)          & 0.116 (0.004)          & 0.094 (0.004)          \\
star              & 0.07 (0.012)           & \textbf{0.042 (0.009)} & 0.085 (0.012)          & \textbf{0.046 (0.009)} & \textbf{0.047 (0.01)}  & \textbf{0.048 (0.011)} & \textbf{0.041 (0.009)} \\
superconductivity & 0.167 (0.006)          & 0.217 (0.008)          & \textbf{0.034 (0.007)} & 0.069 (0.007)          & 0.073 (0.006)          & 0.137 (0.008)          & 0.088 (0.005)          \\
WEC               & \textbf{0.122 (0.004)} & \textbf{0.119 (0.006)} & 0.149 (0.005)          & 0.136 (0.005)          & 0.147 (0.005)          & \textbf{0.13 (0.007)}  & 0.132 (0.005)          \\
winered           & \textbf{0.062 (0.013)} & \textbf{0.085 (0.016)} & \textbf{0.058 (0.011)} & 0.113 (0.015)          & 0.114 (0.016)          & 0.097 (0.019)          & \textbf{0.076 (0.016)} \\
winewhite         & \textbf{0.068 (0.016)} & 0.13 (0.016)           & \textbf{0.072 (0.012)} & 0.147 (0.011)          & 0.147 (0.011)          & 0.156 (0.014)          & 0.099 (0.011)          \\ \hline
\end{tabular}
\end{adjustbox}
\end{table}

\begin{table*}[ht]
\caption{Regression Mean Average Coverage values across different methods and datasets. The reported values represent the average over 50 runs, with two times the standard deviation in parentheses. As expected for conformal methods, all approaches maintain marginal coverage close to the nominal level of 0.9.}
\label{tab:amc_reg}
\centering
\begin{adjustbox}{max width = 0.925\textwidth}
\begin{tabular}{lcccccc}
\hline
\textbf{Dataset}  & \textbf{EPIC-BART} & \textbf{EPIC-GP} & \textbf{EPIC-MDN} & \textbf{Mondrian} & \textbf{Reg-split} & \textbf{Weighted} \\ \hline
airfoil           & 0.897 (0.01)      & 0.9 (0.008)     & 0.897 (0.009)    & 0.906 (0.006)     & 0.897 (0.007)      & 0.9 (0.007)       \\
bike              & 0.901 (0.003)     & 0.903 (0.003)   & 0.898 (0.003)    & 0.904 (0.002)     & 0.899 (0.002)      & 0.9 (0.002)       \\
concrete          & 0.897 (0.009)     & 0.902 (0.011)   & 0.907 (0.009)    & 0.929 (0.006)     & 0.901 (0.008)      & 0.896 (0.008)     \\
cycle             & 0.898 (0.004)     & 0.9 (0.004)     & 0.896 (0.004)    & 0.905 (0.003)     & 0.898 (0.003)      & 0.9 (0.002)       \\
electric          & 0.899 (0.003)     & 0.9 (0.003)     & 0.896 (0.003)    & 0.905 (0.002)     & 0.899 (0.003)      & 0.901 (0.003)     \\
homes             & 0.9 (0.003)       & 0.899 (0.003)   & 0.9 (0.004)      & 0.902 (0.002)     & 0.901 (0.002)      & 0.9 (0.002)       \\
meps19            & 0.899 (0.003)     & 0.9 (0.003)     & 0.897 (0.003)    & 0.902 (0.006)     & 0.9 (0.002)        & 0.9 (0.002)       \\
protein           & 0.9 (0.003)       & 0.901 (0.003)   & 0.899 (0.002)    & 0.9 (0.001)       & 0.9 (0.001)        & 0.899 (0.001)     \\
star              & 0.903 (0.005)     & 0.9 (0.006)     & 0.906 (0.006)    & 0.913 (0.005)     & 0.903 (0.004)      & 0.9 (0.004)       \\
superconductivity & 0.901 (0.003)     & 0.901 (0.003)   & 0.9 (0.004)      & 0.901 (0.002)     & 0.899 (0.002)      & 0.899 (0.002)     \\
WEC & 0.901 (0.002) & 0.898 (0.003) & 0.899 (0.003) & 0.9 (0.001) & 0.899 (0.001) & 0.9 (0.001) \\
winered           & 0.898 (0.009)     & 0.9 (0.008)     & 0.895 (0.008)    & 0.91 (0.005)      & 0.903 (0.006)      & 0.895 (0.005)     \\
winewhite         & 0.902 (0.004)     & 0.904 (0.004)   & 0.901 (0.004)    & 0.911 (0.003)     & 0.9 (0.004)        & 0.899 (0.003)     \\ \hline
\end{tabular}
\end{adjustbox}
\end{table*}

\begin{table*}[ht]
\caption{Regression Interval length values across different methods and datasets. The reported values represent the average over 50 runs, with two times the standard deviation in parentheses. Bold values indicate the best-performing method within a $95\%$ confidence interval. In general, our framework produces narrower intervals in most datasets.}
\label{tab:il_reg}
\centering
\begin{adjustbox}{max width = 0.925\textwidth}
\begin{tabular}{lcccccc}
\hline
\textbf{Dataset}  & \textbf{EPIC-BART}                & \textbf{EPIC-GP}         & \textbf{EPIC-MDN}                & \textbf{Mondrian}        & \textbf{Reg-split}       & \textbf{Weighted}        \\ \hline
airfoil           & \textbf{15.099 (0.592)}           & \textbf{15.223 (0.591)}  & \textbf{15.089 (0.56)}           & 16.671 (0.632)           & \textbf{15.325 (0.469)}  & \textbf{15.693 (0.521)}  \\
bike $\times (10^1)$       & \textbf{24.910 (0.302)}           & 26.616 (0.333)           & 25.665 (0.338)                   & 27.263 (0.264)           & 27.634 (0.271)           & 25.865 (0.249)           \\
concrete          & \textbf{39.025 (1.49)}            & \textbf{39.44 (1.575)}   & \textbf{40.475 (1.551)}          & 51.284 (1.904)           & \textbf{39.943 (1.204)}  & 43.053 (1.629)           \\
cycle             & \textbf{14.911 (0.183)}           & \textbf{14.851 (0.199)}  & \textbf{14.712 (0.174)}          & \textbf{15.015 (0.164)}  & \textbf{14.855 (0.159)}  & \textbf{14.833 (0.15)}   \\
electric          & \textbf{0.036 (\textless{}0.001)} & 0.037 (\textless{}0.001) & \textbf{0.036 (\textless 0.001)} & 0.038 (\textless{}0.001) & 0.037 (\textless{}0.001) & 0.037 (\textless{}0.001) \\
homes $\times (10^5)$  & \textbf{3.848 (0.043)}            & \textbf{3.758 (0.049)}   & 4.040 (0.059)                    & 4.235 (0.037)            & 4.014 (0.033)            & 3.984 (0.032)            \\
meps19            & \textbf{25.013 (0.87)}            & \textbf{26.605 (0.774)}  & 32.093 (1.446)                   & 38.904 (1.039)           & 28.899 (0.544)           & 29.555 (0.843)           \\
protein           & 14.572 (0.106)                    & 14.311 (0.12)            & \textbf{13.573 (0.103)}          & 14.102 (0.038)           & 15.261 (0.037)           & \textbf{13.692 (0.037)}  \\
star $\times (10^1)$    & \textbf{85.3 (1.015)}         & \textbf{85.148 (1.326)} & \textbf{86.499 (1.288)}        & 90.539 (0.762)    & \textbf{85.230 (0.792)} & 104.202 (1.306)   \\
superconductivity & \textbf{39.13 (0.419)}            & \textbf{39.283 (0.479)}  & \textbf{39.547 (0.5)}            & 42.204 (0.216)           & 46.14 (0.242)            & 40.115 (0.228)           \\
WEC $\times (10^{5})$ & 0.893 (0.011) & \textbf{0.858(0.011)} & 0.900 (0.012) & 0.903 (0.004) & 0.925 (0.006) & 0.879 (0.006) \\
winered           & \textbf{2.361 (0.065)}            & \textbf{2.37 (0.067)}    & \textbf{2.316 (0.054)}           & 2.576 (0.04)             & \textbf{2.39 (0.037)}    & 2.541 (0.051)            \\
winewhite         & \textbf{2.337 (0.032)}            & \textbf{2.4 (0.032)}     & \textbf{2.356 (0.031)}           & 2.445 (0.013)            & 2.361 (0.014)            & 2.387 (0.015)            \\ \hline
\end{tabular}
\end{adjustbox}
\end{table*}

\begin{table}[ht]
\caption{Regression Pearson correlation values across different methods and datasets. The reported values represent the average over 50 runs, with two times the standard deviation in parentheses. Bold values indicate the best-performing method within a $95\%$ confidence interval. The Pearson correlation for Regression Split is omitted, as its constant interval length results in an undefined correlation value. Overall, \ourmethod{} achieves low correlations in most cases, indicating strong conditional coverage performance.}
\label{tab:pcorr_reg}
\centering
\begin{adjustbox}{max width = 0.925\textwidth}
\begin{tabular}{lccccc}
\hline
\textbf{Dataset}  & \textbf{EPIC-BART}     & \textbf{EPIC-GP}      & \textbf{EPIC-MDN}      & \textbf{Mondrian}      & \textbf{Weighted}      \\ \hline
airfoil           & \textbf{0.056 (0.013)} & 0.125 (0.018)         & \textbf{0.054 (0.012)} & 0.148 (0.017)          & 0.124 (0.016)          \\
bike              & 0.164 (0.009)          & 0.172 (0.006)         & 0.054 (0.007)          & \textbf{0.028 (0.005)} & 0.043 (0.007)          \\
concrete          & \textbf{0.064 (0.015)} & 0.116 (0.019)         & \textbf{0.054 (0.011)} & 0.191 (0.02)           & 0.211 (0.022)          \\
cycle             & \textbf{0.022 (0.005)} & 0.075 (0.008)         & \textbf{0.023 (0.005)} & 0.043 (0.006)          & \textbf{0.025 (0.005)} \\
electric          & 0.052 (0.007)          & 0.128 (0.009)         & \textbf{0.024 (0.005)} & 0.047 (0.007)          & \textbf{0.029 (0.006)} \\
homes             & 0.135 (0.007)          & \textbf{0.19 (0.011)} & \textbf{0.019 (0.005)} & \textbf{0.016 (0.003)} & 0.038 (0.005)          \\
meps19            & 0.17 (0.012)           & 0.183 (0.015)         & \textbf{0.034 (0.008)} & \textbf{0.022 (0.006)} & 0.053 (0.016)          \\
protein           & 0.063 (0.003)          & 0.071 (0.005)         & 0.062 (0.003)          & \textbf{0.013 (0.003)} & 0.043 (0.005)          \\
star              & 0.076 (0.012)          & \textbf{0.037 (0.01)} & 0.073 (0.01)           & 0.156 (0.012)          & 0.335 (0.016)          \\
superconductivity & 0.072 (0.006)          & 0.254 (0.005)         & \textbf{0.016 (0.004)} & \textbf{0.019 (0.004)} & \textbf{0.025 (0.006)} \\
WEC &  \textbf{0.012 (0.002)} & 0.115 (0.004) & 0.21 (0.007) & \textbf{0.009 (0.002)} & 0.059 (0.007)\\
winered           & \textbf{0.05 (0.01)}   & 0.119 (0.016)         & \textbf{0.042 (0.009)} & 0.153 (0.018)          & 0.221 (0.019)          \\
winewhite         & \textbf{0.035 (0.007)} & 0.079 (0.011)         & \textbf{0.025 (0.005)} & 0.055 (0.009)          & 0.092 (0.011)          \\ \hline
\end{tabular}
\end{adjustbox}
\end{table}

%\begin{table}[!h]
 %   \centering
  %  \caption{An Interesting Table.} \label{tab:supp-data}
   % \begin{tabular}{rl}
    %    \toprule % from booktabs package
     %   \bfseries Dataset & \bfseries Result\\
      %  \midrule % from booktabs package
       % Data1 & 0.12345\\
       % Data2 & 0.67890\\
       % Data3 & 0.54321\\
       % Data4 & 0.09876\\
       % \bottomrule % from booktabs package
   % \end{tabular}
% \end{table}



\subsection{Base model hyperparameters}
\label{sec::base_model_details}
For the CatBoost quantile regression model \citep{dorogush2018catboost}, we set the number of iterations (trees) to 1,000 and the learning rate to 0.001, enabling early stopping after 50 rounds of no improvement. To mitigate overfitting even further, we limit each tree to a maximum depth of 6. All other parameters follow the default CatBoost settings. In the regression setting, we implemented a Neural Network with three hidden layers, consisting of 64, 32, and 16 neurons, respectively. Each layer utilizes ReLU activation, batch normalization \citep{ioffebatch2015}, and dropout rates \citep{srivastava2014dropout} of 0.2, 0.1, and 0.05, correspondingly. We train the model using a smooth L1 loss, as it provides a balance between mean absolute error (MAE) and mean squared error (MSE), making it more robust to outliers while maintaining stable gradient updates.

For optimization, we utilize the Adam optimizer \citep{kingma2014adam} with an initial learning rate of 0.01. A learning rate scheduler is incorporated to decrease the learning rate by a factor of 0.5 if there is no improvement after 10 epochs, accelerating convergence. All weights are initialized using Xavier normal initialization \citep{kumar2017weight}. We set aside 30\% of the training data for validation and set a batch size of $35$. Training proceeds for a maximum of 750 epochs, with early stopping triggered if there is no improvement on the validation set for 30 consecutive epochs. Additionally, feature scaling and target min-max normalization are applied to ensure stable training.


\section{\ourmethod{} Computational Details}
\label{sec::comp_details}
\label{}
This section outlines the specifications of each predictive model used within our framework.  

\subsection{Splitting strategy}
When splitting the calibration set $\mathcal{D}_{\text{cal}}$ for fitting the predictive model and deriving \ourmethod{}'s adaptive cutoffs in disjunct data subsets, we prioritize allocating the majority of data to model training, as cutoff derivation primarily involves a simpler quantile computation. Specifically, for small with $n \leq 3000$, we reserve 30\% of the calibration samples for cutoff computation. For larger datasets, this allocation is capped at 1000 samples to maintain computational efficiency. This approach ensures that \ourmethod{} achieves both accurate predictive distribution estimates and well-calibrated cutoffs.

\subsection{MDN MC-Dropout details}
\label{sec::mc-dropout-architecture}
The employed Mixture Density Network (MDN) consists of $2$ hidden layers, each with 64 neurons, ReLU activations, and batch normalization \citep{ioffebatch2015} to enhance training stability. The output layer predicts three parameters for each mixture component $k = \{1, \dots, K\}$: membership probability $\pi_k(\cdot)$,  the mixture mean $\mu_k(\cdot)$ and the mixture variance $\sigma^2_k(\cdot)$. We set the number of components to $K= 3$, outputting 9 parameters. The network is trained using the negative log-likelihood loss, applying a softmax activation to the mixture probabilities and a softplus activation to the variance estimates. 

Optimization relies on the Adam optimizer \citep{kingma2014adam} with a learning rate of 0.001 and a step learning rate scheduler that decays by 0.99 every 5 epochs. To model epistemic uncertainty, we incorporate MC dropout \citep{gal2016dropout}, applying a dropout rate of 0.5 to each hidden layer. All weights are initialized using PyTorch’s default settings. To monitor generalization performance, we reserve 30\% of the training data exclusively for validation. The model is trained for up to 2,000 epochs, with early stopping triggered after 50 epochs of no improvement on the validation set to prevent overfitting. Additionally, we apply feature scaling and target normalization to enhance numerical stability and improve parameter estimation.

Batch size selection is dataset-dependent: $40$ for small datasets $(n < 10000)$, $125$ for medium datasets ($n < 50000$), and $250$ for large-scaled datasets, such as WEC. For the image experiment, we introduce an additional hidden layer with $32$ neurons and maintain a dropout rate of $0.5$, while adjusting the batch size to $135$. To compute the predictive CDF $F(s(\X, Y)|\X, D)$ at $\X = \x$, we generate 500 samples of the mixture parameters using MC dropout forward passes. For each sampled mixture parameter set, score samples are drawn from the Gaussian Mixture Model. The final predictive CDF is obtained by computing the empirical distribution of $s(\X,Y)$ over these score samples.

\subsection{Variational GP details}
For the Variational Gaussian Process (GP), we implement the model using the \textit{gpytorch} package \citep{gardner2018gpytorch} and PyTorch. We first define the GP prior with a constant mean function and an RBF kernel for the covariance. To approximate the posterior, we employ a Natural Variational Distribution \citep{salimbeni2018natural}, using 15 inducing points for small datasets ($n < 10000$) and $50$ for medium and large-scale data. Training is performed by minimizing the negative variational ELBO, where the Natural Gradient Descent \citep{salimbeni2018natural} updates the variational parameters, while the Adam optimizer \citep{kingma2014adam} refines the GP kernel and noise variance hyperparameters. 

The model is trained for up to 2000 epochs, with early stopping triggered after 50 epochs of no improvement. Following the MDN predictive model,  we reserve 30\% of the training data exclusively for validation and adopt an adaptive batch size of $40$ for small datasets ($n < 10000$), $125$ for medium-sized datasets ($n < 50000$), and $250$ for large datasets. To ensure numerical stability, we apply feature scaling and target normalization. Using the learned variational gaussian posterior, we easily derive the predictive CDF $F(s(\X, Y)|\X, D)$ at $\X = \x$ by using gaussian conjugacy.


\subsection{BART details}
\label{sec::bart_details}
For the Bayesian Additive Regression Tree (BART) model, we use the base implementation from \textit{pymc3} \citep{quiroga2022bayesian} and adopt the heteroscedastic variant \citep{pratola2020heteroscedastic}. The conformal scores are modeled as a normal distribution, where the mean is determined by the sum of regression trees, and the variance depends on $\X$. We set the number of trees to 100, while keeping the prior hyperparameters at their default values in \textit{pymc3}. Target normalization is applied to enhance numerical stability and improve posterior estimation.

After obtaining BART posterior samples via MCMC \citep{chipman2010bart}, we derive the predictive CDF $F(S(\X, Y)|\X, D)$ for a given $\X = \x$ by simulating scores from the postulated distribution using the posterior samples. The corresponding empirical CDF for the score $S(\X, Y)$ is then computed. In the regression example presented in Figure \ref{fig::reg_split} and detailed in Appendix \ref{sec::technicalReg}, we used a modified version of BART. In this version, a heteroscedastic gamma distribution was assigned to the score instead of a normal distribution, which was necessary to account for the asymmetry in the regression conformal scores for that example.



\section{Proofs}
\label{sec:proofs}

\begin{proof}[Proof of Theorem \ref{thm::marginal}]
Follows immediately from the fact that  \ourmethod\ is a conformal score and \citet[Theorem 2]{Lei2018}.
\end{proof}


%The proof of Theorem \ref{thm::marginal} follows immediately from the fact that 
%\ourmethod\ is a conformal score and \citet[Theorem 2]{Lei2018}.

\begin{proof}[Proof of Theorem \ref{thm::conditional_coverage}]




Let $t'_{1-\alpha}$ be the empirical quantile obtained using $s'(\x,y) = F(s(\x,y)|\x,D)$, and let $t''_{1-\alpha}$ be the empirical quantile obtained using $s''(\x,y) = F(s(\x,y)|\x,\theta^*)$. 

By Assumption \ref{assumption:uniform_convergence}, for any $\varepsilon,\delta > 0$, if the calibration set is sufficiently large, then with probability at least $1 - \delta$ (for some event $\Omega$), we have
\[
\sup_{s, \x} \left|F(s \mid \x, D) - F(s \mid \x, \theta^*)\right| \leq \varepsilon,
\]
where the randomness is over $D$.

Let $\hat{\P}$ denote the empirical probability measure based on $\mathcal{D}_{\calib,2}$, i.e., for a given event \( A \) and any function $g$:
\[
\hat{\P}(g(s) \in A) = \frac{1}{|\mathcal{D}_{\calib,2}|} \sum_{(\X_i, Y_i) \in \mathcal{D}_{\calib,2}} \Ind(g(s(\X_i, Y_i)) \in A).
\]

Conditionally on the event $\Omega$, we obtain:
\begin{align*}
   1 - \alpha &\leq \hat{\P}(F(s | \x, D) \leq t'_{1-\alpha})\\
   &\leq \hat{\P}(F(s | \x, \theta^*) - \varepsilon \leq t'_{1-\alpha}),
\end{align*}
which, by the definition of the empirical quantile, implies that 
\[
t''_{1-\alpha} \leq t'_{1-\alpha} + \varepsilon.
\]
By a similar argument, we also have:
\begin{align*}
   1 - \alpha &\leq \hat{\P}(F(s | \x, \theta^*) \leq t''_{1-\alpha})\\
   &\leq \hat{\P}(F(s | \x, D) - \varepsilon \leq t''_{1-\alpha}).
\end{align*}
Therefore, under the event $\Omega$ of probability at least $1 - \delta$, we conclude that
\[
|t'_{1-\alpha} - t''_{1-\alpha}| \leq \varepsilon.
\]

Using this result and the fact that $\left|F(s \mid \x, D) - F(s \mid \x, \theta^*)\right| \leq \varepsilon$, we establish the following bound:
\begin{align*}
    \P(s'(\X,Y) \leq t'_{1-\alpha}|\X) &\leq \P(s'(\X,Y) \leq t'_{1-\alpha} \mid \X,\Omega) \cdot 1 + \delta\\
    & \leq \P(s'(\X,Y) \leq t''_{1-\alpha} + \varepsilon \mid \X,\Omega) + \delta\\
    &\leq \P(s''(\X,Y) - \varepsilon \leq t''_{1-\alpha} + \varepsilon|\X,\Omega) + \delta\\
    &=  \P(s''(\X,Y) \leq t''_{1-\alpha} + 2\varepsilon \mid \X) + \delta\\
    &= t''_{1-\alpha} + 2\varepsilon + \delta.
\end{align*}
In the fourth equality, we used the fact that \( s''(x,y) \) is non-random and that \( t''_{1-\alpha} \) depends only on \( \mathcal{D}_{\calib, 2} \), making it independent of \( \Omega \). In the last equality, we used the fact that the random variable $s''(\X,Y)|\X$ is uniform.


By a similar argument, we can show that:
\[
\left|\P(s'(\X,Y) \leq t'_{1-\alpha}|\X) - t''_{1-\alpha}\right| \leq 2\varepsilon + \delta.
\]
Thus, as $|\mathcal{D}_{\calib,1}| \to \infty$, we can take $\varepsilon, \delta \to 0$, which implies that
\[
\lim_{|\mathcal{D}_{\calib,1}| \to \infty} \P(s'(\X,Y) \leq t'_{1-\alpha}|\X) = t''_{1-\alpha}.
\]
Finally, by \cite[Lemma 2, Section D.2.2]{dheur2025multi}, we know that as $|\mathcal{D}_{\calib,2}| \to \infty$, we have $t''_{1-\alpha} \to 1-\alpha$, which concludes the proof.


\end{proof}


% % NOTE: necessary when ptmx or no mathfont class option is given
% \providecommand{\upGamma}{\Gamma}
% \providecommand{\uppi}{\pi}
% How math looks in equations is important:
% \begin{equation*}
%     F_{\alpha,\beta}^\eta(z) = \upGamma(\tfrac{3}{2}) \prod_{\ell=1}^\infty\eta \frac{z^\ell}{\ell} + \frac{1}{2\uppi}\int_{-\infty}^z\alpha \sum_{k=1}^\infty x^{\beta k}\mathrm{d}x.
% \end{equation*}
% However, one should not ignore how well math mixes with text:
% The frobble function \(f\) transforms zabbies \(z\) into yannies \(y\).
% It is a polynomial \(f(z)=\alpha z + \beta z^2\), where \(-n<\alpha<\beta/n\leq\gamma\), with \(\gamma\) a positive real number.


\end{document}
