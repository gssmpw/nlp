\section{Related works}
Research to solve CAPTCHA codes automatically using ML techniques has become a challenge for years. There are even commercial services available in Internet where everyone can buy a packet of Application Programming Interface (API) requests to solve CAPTCHAs in webpages which are visited. That is why new CAPTCHA types like Google’s reCAPTCHAv2 or reCAPTCHAv3 systems are still being developed to counteract against bots that are feasible to provide CAPTCHA resolves in real time. To solve CAPTCHAs based on sequential information (like text of different lengths or with complicated features of characters) one can also concatenate a CNN with a Recurrent Neural Network (RNN) \cite{Shu_Xu} that is able to correlate dependencies in CAPTCHA patterns. Another ML-based solution is Generative Adversarial Networks (GANs) \cite{Zhang2020} that allow providing synthetic CAPTCHA images as similar as possible to the real ones. A promising technique is Capsule Networks \cite{Mocanu2022} that can detect the spatial relationships between different characters in the CAPTCHA. In \cite{Zahra2020} a customized CNN called Deep-CAPTCHA was developed to solve both numerical and alphanumerical CAPTCHAs, leading to cracking accuracy of 98.94\% and 98.31\%, respectively. The same approach based on a CNN named CapNet as well as VGG-19 and AlexNet deep CNN models were selected in \cite{Walia2023}. The CapNet solver achieved accuracy of 96.08\% (with slight differences for each of 5 digits in alphanumeric characters of CAPTCHA challenges) using advanced pre-processing techniques like noise reduction filtering, Grey-scaling, resizing, normalization, one-hot encoding, and image size reduction. Pre-trained YOLO v8 models for image segmentation and classification were used in \cite{Plesner2024} for three types of CAPTCHAs (and 13 classes) provided within the reCAPTCHAv2 system. According to the normalized confusion matrix top 5 accuracy was 99.5\%. Reinforcement Learning techniques can be used to provide automatically upgrades for CAPTCHA-solving algorithms.