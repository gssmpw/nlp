\lstset{style=riscv-asm-style}

\begin{figure*}[tp]
  \centering

  %---------------------------------------------------------------------------
  % mssort example
  %---------------------------------------------------------------------------

  \begin{subfigure}[t]{\cw}
    \centering
    \includegraphics[width=\cw]{stream-sort-example.svg.pdf}
    %\vspace{-0.8cm}
\begin{lstlisting}[xleftmargin=0.025\tw]
# a0, a1: base addresses of input key & value arrays
# a2, a3: base addresses of output key & value arrays
# v0, v1: lengths of the 1st & 2nd input chunks
# v2, v3: pointers to the 1st & 2nd input chunks
# v4, v5: lengths of the 1st & 2nd output chunks
# v6, v7: pointers to the 1st & 2nd output chunks
# load keys & values from both input chunks
mlxe.t tr0, 0(a0), v2, v0
mlxe.t tr1, 0(a1), v2, v0
mlxe.t tr2, 0(a0), v3, v1
mlxe.t tr3, 0(a1), v3, v1
# sort keys & values from both input chunks
mssortk.tt tr0, tr2, v0, v1
mssortv.tt tr1, tr3, v0, v1
# get lengths of output chunks
mmv.vo v4, 0x0
mmv.vo v5, 0x1
# store sorted keys & values to both output chunks
msxe.t tr0, 0(a2), v6, v4
msxe.t tr1, 0(a3), v6, v4
msxe.t tr2, 0(a2), v7, v5
msxe.t tr3, 0(a3), v7, v5
\end{lstlisting}
    \vspace{-0.4cm}
    \caption{Sorting Chunks of Keys and Values}
    \label{fig-spz-mssort-code}
  \end{subfigure}
  \hfill
  %---------------------------------------------------------------------------
  % mszip example
  %---------------------------------------------------------------------------
  \begin{subfigure}[t]{\cw}
    \centering
    \includegraphics[width=\cw]{stream-zip-example.svg.pdf}
    %\vspace{-0.8cm}
\begin{lstlisting}[xleftmargin=0.025\tw]
# a0, a1: base addresses of input key & value arrays
# a2, a3: base addresses of output key & value arrays
# v0, v1: lengths of the 1st and 2nd input chunks
# v2, v3: pointers to the 1st and 2nd input chunks
# v4: lengths of the output chunks
# v5: pointers to the output chunks
# load keys & values from both input chunks
mlxe.t tr0, 0(a0), v2, v0
mlxe.t tr1, 0(a1), v2, v0
mlxe.t tr2, 0(a0), v3, v1
mlxe.t tr3, 0(a1), v3, v1
# merge keys & values from both input chunks
mszipk.tt tr0, tr2, v0, v1
mszipv.tt tr1, tr3, v0, v1
# get the number of merged elements from input chunks
mmv.vi v6, 0x0
mmv.vi v7, 0x1
# get the number of elements added to output chunks
mmv.vo v8, 0x0
mmv.vo v9, 0x1
# store merged keys & values to output chunks
msxe.t tr0, 0(a2), v5, v8
msxe.t tr1, 0(a3), v5, v8
vadd.vv v5, v5, v8  # bump output pointers
msxe.t tr2, 0(a2), v5, v9
msxe.t tr3, 0(a3), v5, v9
vadd.vv v5, v5, v9  # bump output pointers
\end{lstlisting}
    \vspace{-0.4cm}
    \caption{Merging Chunks of Keys and Values}
    \label{fig-spz-mszip-code}
  \end{subfigure}
  \caption{
    Examples of Using SparseZipper Instructions to Sort and Merge Key-Value Streams --
    \TT{a\{0..3\}}~=~scalar registers;
    \TT{v\{0..9\}}~=~vector registers;
    \TT{tr\{0..3\}}~=~matrix registers.
  }
  \label{fig-spz-code}
  \vspace{-0.3cm}
\end{figure*}
