%=========================================================================
% SparseZipper: Micro-architecture
%=========================================================================

\section{S\lowercase{parse}Z\lowercase{ipper} Micro-Architecture}
\label{sec-spz-uarch}

This section describes SparseZipper micro-architecture that extends a baseline
systolic array specialized for dense-dense GEMM to support the proposed
instructions for sparse-sparse GEMM presented in Section~\ref{sec-spz-isa}.
Each sorting/zipping instruction is decomposed into micro-operations.
Each input matrix row corresponding to a data stream is processed in one
micro-operation going through the systolic array in two passes: (1)
sorting/zipping and (2) compressing.  The execution of one micro-operation is
explained in Section~\ref{sec-spz-uarch-mssort} and~\ref{sec-spz-uarch-mszip}.
In Section~\ref{sec-spz-uarch-multi-rows}, we then explain how pipelining
happens across micro-operations of a single instruction and multiple
instructions.
Finally, in Section~\ref{sec-spz-uarch-ext}, we discuss hardware changes needed
to support SparseZipper.

\begin{figure*}[tp]
  \centering
  \begin{subfigure}{\textwidth}
      \includegraphics[width=\textwidth]{mssortk-systolic-exec-example.svg.pdf}
      \caption{Systolic Execution of \TT{mssortk} Instruction.}
      \label{fig:mssortk-systolic-exec}
  \end{subfigure}
  \hfill
  \vspace{0.01cm}
  \begin{subfigure}{\textwidth}
      \includegraphics[width=\textwidth]{mszipk-systolic-exec-example.svg.pdf}
      \caption{Systolic Execution of \TT{mszipk} Instruction.}
      \label{fig:mszipk-systolic-exec}
  \end{subfigure}
  \caption[Cycle-by-Cycle Systolic Execution of \TT{mssortk} in a 3$\times$3 Systolic Array for Two Unsorted Lists of Keys]{
    Cycle-by-Cycle Systolic Execution of \TT{mssortk} in a 3$\times$3 Systolic Array for Two Unsorted Lists of Keys --
    PE states: F~=~forward, X~=~switch, C~=~combine;
    W\_IC~=~west input counter;
    N\_IC~=~north input counter;
    E\_OC~=~east output counter;
    S\_OC~=~south output counter;
    d~=~duplicate key that is excluded;
    x~=~unmergeable key;
    Counters in red indicate they are being updated.
    PEs in gray are inactive.
    PEs in blue are merging keys.
    PEs in purple are compressing valid output keys.
    Keys in red come from the north input.
    Keys in green come from the west input.
    Keys in west and east sides are ordered from bottom to top.
    Keys in north and south sides are ordered from left to right.
  }
  \label{fig:systolic-exec}
  \vspace{-0.1cm}
\end{figure*}

%\lstset{style=riscv-asm-style}
\begin{figure*}[tp]
  \centering
  \includegraphics[width=\tw]{systolic-exec-all-rows.svg.pdf}
  \caption{
    Cycle-by-Cycle Systolic Execution of Sorting Multiple Key-Value Lists in a 3$\times$3 Systolic Array --
    PEs performing key-value sorting are annotated with letter \TT{s}.
    PEs performing key-value compression are annotated with letter \TT{c}.
    PEs in gray color are idle.
    Otherwise, the color of a PE refers to a set of rows in matrix  registers
    that the PE is processing.
  }
  \label{fig:systolic-exec-all-rows}
  \vspace{-0.3cm}
\end{figure*}


%----------------
% Systolic execution of mssort*
%----------------

%\begin{figure*}[tp]
    \centering
    \includegraphics[width=\tw]{mssortk-systolic-exec-example.svg.pdf}
    \caption[Cycle-by-Cycle Systolic Execution of \TT{mssortk} in a 3$\times$3 Systolic Array for Two Unsorted Lists of Keys]{
      \textbf{Cycle-by-Cycle Systolic Execution of \TT{mssortk} in a 3$\times$3 Systolic Array for Two Unsorted Lists of Keys} --
      PE states: F~=~forward, X~=~switch, C~=~combine;
      W\_IC~=~west input counter;
      N\_IC~=~north input counter;
      E\_OC~=~east output counter;
      S\_OC~=~south output counter;
      d~=~duplicate key that is excluded;
      Counters in red indicate they are updated in corresponding cycles.
      PEs in gray are inactive.
      PEs in blue are zipping keys.
      PEs in purple are compressing valid output keys.
      Keys in red come from the north input.
      Keys in green come from the west input.
      Keys in west and east sides are ordered from bottom to top.
      Keys in north and south sides are ordered from left to right.
    }
    \label{fig:mssortk-systolic-exec}
\end{figure*}

%\begin{figure*}[tp]
    \centering
    \includegraphics[width=\tw]{mszipk-systolic-exec-example.svg.pdf}
    \caption[Cycle-by-Cycle Systolic Execution of \TT{mszipk} in a 3$\times$3 Systolic Array for Two Sorted Lists of Keys]{
      \textbf{Cycle-by-Cycle Systolic Execution of \TT{mszipk} in a 3$\times$3 Systolic Array for Two Sorted Lists of Keys} --
      PE states: F~=~forward, X~=~switch, C~=~combine;
      W\_IC~=~west input counter;
      N\_IC~=~north input counter;
      E\_OC~=~east output counter;
      S\_OC~=~south output counter;
      d~=~duplicate key that is excluded;
      x~=~unmergeable key;
      Counters in red indicate they are updated in corresponding cycles.
      PEs in gray are inactive.
      PEs in blue are merging keys.
      PEs in purple are compressing valid output keys.
      Keys in red come from the north input.
      Keys in green come from the west input.
      Keys in west and east sides are ordered from bottom to top.
      Keys in north and south sides are ordered from left to right.
    }
    \label{fig:mszipk-systolic-exec}
\end{figure*}


\subsection{Systolic Execution of Sorting Key-Value Chunks \\ in a Single Stream}
\label{sec-spz-uarch-mssort}

% explain input/output and components in the systolic array
Figure~\ref{fig:mssortk-systolic-exec} shows a 3$\times$3 systolic array
executing \TT{mssortk} instruction to sort two unsorted chunks of keys over
multiple cycles.
Initially, the two input chunks are located in the west and north sides of the
array.
Similar to a typical systolic execution of dense-dense GEMM, inputs are
staggered into the array.
Outputs come out from the east and south sides.
The dataflow through the array is the same as in dense-dense GEMM.
Each PE receives inputs from the west and north sides and sends outputs to the
east and south sides.

% explain how the data flow for mssortk
Input keys flow through the array in two passes: sorting and compressing.
In the sorting pass, keys in a chunk are sorted in an ascending order.
If a chunk contains duplicates, the sorting pass combines them into one valid
output key, and the rest become invalid outputs to be excluded.
The invalid outputs may exist in between valid outputs after the sorting pass,
so the compressing pass places valid outputs consecutively starting from the
first position of an output chunk and moves invalid outputs to the end.
For example, in Figure~\ref{fig:mssortk-systolic-exec}, after the sorting pass,
the north-side inputs \{5,~8,~5\} come out in the east side as \{5,~\TT{d},~8\}
(i.e., \TT{d} indicates excluded duplicated key(s)).
In the compressing pass, the partial output chunk is pushed into the array from
the west side, and the final output chunk \{5,~8,~\TT{d}\} comes out from the
south side.

% explain PE states
When sorting keys, we need to store the key reordering so that their values can
be shuffled to correct positions later.
In each PE, we encode and store the direction in which west-side and north-side
keys are routed towards and whether the keys are duplicate.
Given a pair of input keys, there are four possible states: (1) initial (no
data routing), (2) forwarding, (3) switching, and (4) combining.
The forwarding state encodes a PE routes west-side and north-side keys to the
east and south sides respectively.
The switching state encodes a PE routes west-side and north-side keys to the
south and east sides respectively.
The combining state encodes that two input keys are duplicate and that they are
combined into one valid key routed to the south side.
Each PE needs to store the states for both sorting and compressing passes.

% explain the algorithm
The west- and north-side input keys are sorted independently using the
bottom-left and top-right half of the systolic array.
PEs on the main diagonal are hard-coded to always switch inputs so that data
from two input chunks are not intermixed.
In other PEs, two input keys are compared.
The larger key is routed to the east, and the smaller key is routed to the
south (e.g., cycle 2 in Figure~\ref{fig:mssortk-systolic-exec}).
If keys are duplicate (e.g., cycle 5 in
Figure~\ref{fig:mssortk-systolic-exec}), a single combined valid key is sent to
the south port, and the east output is tagged as invalid key.
In subsequent PEs, the invalid key is considered larger than any valid key, so
it is always forwarded to the east.

% explain the counters
Instruction \TT{mssortk} updates special-purpose input and output counter
vector registers: \TT{W\_IC} (for the west input), \TT{N\_IC} (for the north
input), \TT{E\_OC} (for the east output) and \TT{S\_OC} (for the south output)
as keys come out from the array as shown in
Figure~\ref{fig:mssortk-systolic-exec} to count the number of processed input
keys and valid output keys.
Input counters (\TT{W\_IC} and \TT{N\_IC}) are updated in the sorting pass
while output counters (\TT{E\_OC} and \TT{S\_OC}) are updated in the
compressing pass.

% mssortv
Instruction \TT{mssortv} shuffles values based on a key reordering produced by
\TT{mssortk}.
Values are also passed through the array in two passes and directed based on
the states captured in each PE during the execution of \TT{mssortk}.
If the state is combining, two values are accumulated, and the accumulated
value is forwarded to a PE's south side.
Instruction \TT{mssortv} does not update the input and output counters.

%----------------
% Systolic execution of mszip*
%----------------
\subsection{Systolic Execution of Merging Key-Value Chunks \\ in a Single Stream}
\label{sec-spz-uarch-mszip}

% explain input and output
Figure~\ref{fig:mszipk-systolic-exec} shows the systolic execution of
\TT{mszipk} instruction in a 3$\times$3 array to merge two sorted key-value
chunks from a single stream.
Initially, the two input chunks are placed in the west and north sides.
Keys in the west side are ordered from bottom to top in an ascending order
while keys in the north side are ordered from left to right.
The final output chunk is stored in two parts.
For example, the part with smaller keys \{2,~3,~5\} is located in the east side
while the second part with larger keys \{8\} is stored in the south side.

% systolic data flow
Keys flow through the systolic array in two passes: merging and compressing.
The merging pass generates a merged list of sorted keys with invalid outputs
(i.e., caused by duplicate keys) potentially located in between valid output
keys.
The compressing pass then places valid output keys consecutively.
Unlike the sorting pass, the merging pass intermixes keys from both input
chunks, so PEs on the main diagonal work the same as other PEs instead of being
hard-coded to always switch inputs.
Similarly, larger keys flow to the east while smaller keys flow to the south
side.
The compressing pass works exactly the same as in the sorting operation.

% exclude keys that are not merged
Not all input keys can be merged into the output chunk.
Keys from an input chunk that are greater than all keys from the other chunk
need to be excluded since we do not know yet their positions in the output
stream as discussed in Section~\ref{sec-spz-isa-merge-spgemm}.
For example, in Figure~\ref{fig:mszipk-systolic-exec}, the key 9 in the west
input chunk is excluded from the output chunk (\TT{x} in cycle 4) since
it is greater than every key from the north-side chunk.
To detect keys to be excluded, each key is tagged with two extra bits to track
(1) from which input side the key comes from (source bit) and (2) whether the
key has been compared with another larger or equal key from the other input
chunk yet (merge bit).
The merge bit of a key is initially set to false and flipped to true when a
PE detects a larger or equal key from the other input side.
After the merging pass, if the merge bit is still false, the key is excluded
from the output chunk.

% explain the counters
Input and output counters are updated in the same way as in the sorting
operation.
\TT{W\_IC} and \TT{N\_IC} count the numbers of merged keys for the
west-side and north-side input chunks respectively.
\TT{E\_OC} and \TT{S\_OC} count the numbers of valid output keys in the
east-side and south-side output chunks respectively.
Instruction \TT{mszipv.tt} shuffles values based on the reordering of their
corresponding keys.

%----------------
% Multiple input lists
%----------------
\subsection{Merging and Sorting Key-Value Chunks \\ across Multiple Streams}
\label{sec-spz-uarch-multi-rows}

The execution of multiple micro-operations mapping to different streams can
overlap in SparseZipper's systolic array.
Figure~\ref{fig:systolic-exec-all-rows} shows the cycle-by-cycle systolic
execution of sorting key-value chunks from multiple streams.
Keys and values from multiple streams are mapped to different rows in matrix
registers.
Input keys and values from adjacent matrix register rows enter the systolic
array back-to-back in consecutive cycles since there is no data dependency
between micro-operations of an instruction.
There are one-cycle stalls in cycle 4 and cycle 11 since the systolic array
takes one extra cycle to route data from the west and south sides at the end of
a sorting pass to the east and north sides at the beginning of a compressing
pass.
The execution of \TT{mssortk}/\TT{mszipk} and \TT{mssortk}/\TT{mszipv}
instructions in a pair can overlap.
Since the latency of a micro-operation through the systolic array is fixed
(i.e., 2N+1 where N is the number of PEs in a row/column of the array), the
array can schedule to start the following \TT{mssortv}/\TT{mszipv} as soon as
the top-left-corner PE finishes its last key-compressing operation (e.g., in
cycle 8 in Figure~\ref{fig:systolic-exec-all-rows}).
However, the execution of different key-value instruction pairs processing
different input matrices of key-value data streams do not overlap to avoid
overwriting the array's output counters.
The counters must be read out to a vector register before the array can start
executing a new key-value instruction pair.

%----------------
% Systolic array details
%----------------
\subsection{Hardware Changes to the Baseline Systolic Array}
\label{sec-spz-uarch-ext}
\begin{figure*}[tp]
  \centering
  \includegraphics[width=\tw]{sparse_zipper_uarch_o3_core.svg.pdf}
  \caption[SparseZipper Systolic Array Micro-architecture]{
    SparseZipper Systolic Array Micro-architecture --
    Components and wires added to support sparse computation are in red;
    %PE~=~processing element;
    popc~=~population counting logic;
    W\_IC \& N\_IC~=~west \& north input counters;
    E\_OC \& S\_OC~=~east \& south output counters.
  }
  \label{fig:spz-sparse-zipper-uarch}
  \vspace{-0.2cm}
\end{figure*}


Figure~\ref{fig:spz-sparse-zipper-uarch} shows micro-architectural changes to
support SparseZipper in the baseline systolic array specialized for dense-dense
GEMM.
We add a second write port to the matrix register file as the sorting and
merging instructions have two output operands.
Since each physical matrix register is quite large (e.g., 1KB for a
16$\times$16 32-bit-element matrix register), the matrix register file may
consist of multiple SRAM banks, one for each physical matrix register.
Therefore, adding an additional write port to the register file simply requires
an extra crossbar instead of adding an extra write port to each SRAM bank which
would incur significant area overheads.
In order to retrieve the east-side output data from the systolic array, we add
a second deskew buffer.

% additional control signals and the loop-back paths
Additional control bits (i.e., source, duplicate, and merge bits) are tagged
along with data flowing through the systolic array as described in
Section~\ref{sec-spz-uarch-mszip}.
We add a three-bit control between any two PEs to the existing data paths.
For routing data between a sorting/merging pass and a compressing pass, two
loop-back paths are added to connect east and south output sides to the west
and north input sides respectively.
Each path is pipelined via an extra register to account for its long distance
between two sides of the systolic array.

% popc logic and input/output vectors of counters
Four input and output vectors of N counters are added to track the number of
valid input and output elements for N rows of matrix registers.
Each counter counts up to N (i.e., the number of elements in a row), so a
vector of N counters is $N \times \log_2 N$ bit wide.
The population counting logic uses output control signals from the systolic
array and increments corresponding input/output counters.

% inside each PE
In each PE, we slightly modify the existing adder to support comparison of
input keys.
An additional control unit uses the comparison outcome to make routing
decisions (i.e., forwarding, switching, and combining) and route data from
input to output ports by controlling the two output multiplexers.
The control unit also updates the duplicate and merge bits based on the source
bit and the comparison outcome.
We use the same adder for adding up values for \TT{mssortv} and \TT{mszipv}
instructions in case of combining inputs.
We repurpose the weight register in each PE to store routing states.
Each state requires two bits to encode.
Each pair of rows from two input matrix registers needs to store two states for
their sorting/merging and compressing passes.
Therefore, for N pairs of rows, we need a total of $N \times 4$ bits for all
the routing states (e.g., 64 bits for a hardware vector length of 16
elements).
