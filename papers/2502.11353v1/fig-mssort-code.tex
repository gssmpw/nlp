%=============================================================================
% asm code for sorting chunks of VL elements
%=============================================================================

%\lstset{language=[RISC-V]{Assembler}}
\lstset{style=riscv-asm-style}

\begin{figure}[tp]

  \centering
  \includegraphics[width=0.9\cw]{stream-sort-example.svg.pdf}

\begin{lstlisting}[xleftmargin=0.025\tw]
# a0, a1: base address of input key and value arrays
# a2, a3: base address of output key and value arrays
# v0, v1: lengths of the 1st and 2nd input chunks
# v2, v3: pointers to the 1st and 2nd input chunks
# v4, v5: lengths of the 1st and 2nd output chunks
# v6, v7: pointers to the 1st and 2nd output chunks
# v8, v9: index of the current set of chunk pairs and the numbers of chunk pairs
loop:
  # load
  mlxe.t tr0, 0(a0), v2, v0   # load keys of the 1st input chunks
  mlxe.t tr1, 0(a1), v2, v0   # load values of the 1st input chunks
  mlxe.t tr2, 0(a0), v3, v1   # load keys of the 2nd input chunks
  mlxe.t tr3, 0(a1), v3, v1   # load values of the 2nd input chunks
  # sort
  mssortk.tt tr0, tr2, v0, v1 # sort keys of the 1st and 2nd input chunks
  mssortv.tt tr1, tr3, v0, v1 # sort values of the 1st and 2nd input chunks
  mmv.vo v4, 0x0              # get lengths of the 1st output chunks
  mmv.vo v5, 0x1              # get lengths of the 2nd output chunks
  # store
  msxe.t tr0, 0(a2), v6, v4   # store keys of the 1st output chunks
  msxe.t tr1, 0(a3), v6, v4   # store values of the 1st output chunks
  msxe.t tr2, 0(a2), v7, v5   # store keys of the 2nd output chunks
  msxe.t tr3, 0(a3), v7, v5   # store values of the 2nd output chunks
  # check if we finish all pairs
  vadd.vx v8, v8, 0x2
  vmsltu.vv v10, v8, v9
  vpopc.m a4, v10
  bnez a4, loop
\end{lstlisting}

\caption[Sorting Chunks of Keys and Values from Multiple Streams in Parallel]{
  \BF{Sorting Chunks of Keys and Values from Multiple Streams in Parallel
  Using the Proposed Matrix Instructions} --
  This code snippet only shows the most inner loop that iterating through the
  working set of chunks bordered by the dash lines.
  VL~=~vector length;
  \TT{a\{0..4\}}~=~scalar registers;
  \TT{v\{0..9\}}~=~vector registers;
  \TT{tr\{0..3\}}~=~matrix registers.
}
\label{fig-spz-mssort-code}
\end{figure}
