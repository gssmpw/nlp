%=========================================================================
% SparseZipper: Background
%=========================================================================

%-------------------------------------------------------------------------
% Background on SpGEMM
%-------------------------------------------------------------------------
% Take-aways:
%   - Row-wise dataflow is efficient (time and space) and parallelizable
%   - Different ways to implement the row-wise product dataflow
%       + Using an accumulator
%       + Using ESC method

\section{Background: Sparse General Matrix Multiplication}
\label{sec-spz-spgemm-background}

Sparse general matrix multiplication (SpGEMM) is a commonly used building block
in various application domains including graph
analytics~\cite{davis-graphblas-tmos2019,hoefler2011generic,shun-multicore-tc-2015},
machine
learning~\cite{reddi-mlperf-isca2020,naumov-dnn-model-arxiv2019,han-deep-compress-arxiv2015,jouppi-datacenter-isca2017,wu-ml-facebook-hpca2019},
and scientific computing~\cite{canning-sparse-sim-1996,galli-quantum-sim-1996}.
This section describes widely used data structures for representing sparse
matrices, various algorithms for multiplying two sparse matrices, and their
trade-offs.

%----------
% Matrix formats
%----------

\subsection{Sparse Matrix Formats}
\begin{figure}[tp]
    \centering
    \includegraphics[width=0.7\cw]{csr_format.svg.pdf}
    \caption[Compressed Sparse Row Format]{
      \textbf{Compressed Sparse Row Format}
    }
    \label{fig-spz-csr-format}
\end{figure}


Since most elements in a sparse matrix are zeros, it is efficient to store only
non-zero elements to minimize the amount of required storage and also avoid
ineffectual computation (i.e., multiplying with zero).
The most common data structures for storing a sparse matrix are coordinate
format (COO), compressed sparse row (CSR), and compressed sparse column (CSC).
They are the default data structures in many widely used linear algebra
libraries such as Intel Math Kernel Library (MKL)~\cite{wang-intel-mkl-2014},
CUSPARSE~\cite{naumov-cusparse-2010}, and
Matlab~\cite{gilbert-sparse-matlab-1992}.

COO format stores a list of tuples including row index, column index, and value
of non-zero elements in a sparse matrix.
Typically, the list is kept sorted by either row indices, column indices, or a
combination of both row and column indices for more efficient lookups and
memory accesses.
Further improving the storage efficiency, compared to COO, CSR format groups
per-row non-zero elements so that the row indices for elements in the same row
can be further compressed.
Figure~\ref{fig-spz-csr-format} shows an example of CSR format.
Non-zero elements in each row are represented by a stream of keys (i.e., column
indices) and values (i.e., element data).
Typically, elements in a stream are sorted by their keys, and streams of
adjacent rows are placed consecutively in memory to form the column index and
value arrays.
The row pointer array consists of pointers indicating where each row starts in
the column index and value arrays.
CSC format is similar to CSR format except that non-zero elements are
compressed by columns.

In addition to those general sparse matrix formats, there are numerous
other sparse matrix formats specialized for certain structures of non-zero
elements and/or
target architectures~\cite{borvstnik-dbcsr-2014,hu-graphlily-iccad2021,gomez-spmv-vector-2021,srivastava-matraptor-micro2020,chen-spgemm-taihulight-2018}.
However, such specialized matrix formats often incur format converting
overheads, and their performance is not portable across different matrices and
architectures.
In this chapter, we assume CSR/CSC which are storage-efficient and widely used
sparse matrix formats.

%----------
% Dataflows
%----------

\begin{figure}[tp]
    \centering
    \includegraphics[width=\cw]{spgemm_dataflows.svg.pdf}
    \caption[Different Sparse GEMM Dataflows]{
      \textbf{Different SpGEMM Dataflows} -- Non-zero elements in the input
      matrices (i.e., A and B) and the output matrix (i.e., C) are colored.
      Zero elements are left blank. In each computation step, involved matrix
      elements are shown within the dark borders.
    }
    \label{fig-spz-spgemm-dataflows}
\end{figure}

\subsection{SpGEMM Dataflows}

Figure~\ref{fig-spz-spgemm-dataflows} shows common dataflows for SpGEMM:
inner-product, outer-product, and row-wise-product (also known as Gustavson's
algorithm).

\BF{Inner-product dataflow --}
This dataflow computes each element in the output matrix one at a time by doing
a dot product between two vectors (i.e., a row in matrix A and a corresponding
column in matrix B), as shown in Figure~\ref{fig-spz-spgemm-dataflows}.
Since there is no data dependency among output elements, the inner-product
dataflow is highly parallelizable by computing multiple dot products for
different output elements in parallel.
In addition, this dataflow also has good data locality for the output matrix
but low data reuse for the input matrices.
One major downside of this approach is the inefficiency of doing a dot product
between two sparse vectors.
Since input matrices are sparse, the output matrix is likely to have few
non-zero elements.
Therefore, most of the dot products result in zeros.
Furthermore, the dot product of two sparse vectors requires intersecting two
lists of indices, and this index matching is highly inefficient due to the high
sparsity of a row and a column in the input matrices.

\BF{Outer-product dataflow --}
Unlike the inner-product dataflow, the outer-product dataflow computes partial
results for the entire output matrix at each step by doing an outer product
between a column in matrix A and a corresponding row in matrix B, as shown in
Figure~\ref{fig-spz-spgemm-dataflows}.
Partial results are combined into a single final output matrix C.
All non-zero elements in a column in matrix A and a row in matrix B are used to
generate non-zero elements in matrix C, so the outer-product dataflow avoids
the problem of ineffectual index matching operations in the inner-product
dataflow.
Multiple outer-product operations on different pairs of matrix A's columns and
matrix B's rows can happen in parallel.
However, combining partial outer-product results typically requires complex
synchronizations, and this step is often a performance bottleneck in this
approach.
Despite having good input data reuse, the outer-product dataflow has low output
data reuse, and the aggregate memory space of all partial output matrices is
often much larger than the final output matrix's size.

\BF{Row-wise-product dataflow --}
The row-wise-product dataflow computes each row of the output matrix at a time
by multiplying a row in matrix A (sparse vector) with the entire matrix
B, as shown in Figure~\ref{fig-spz-spgemm-dataflows}.
Similar to the outer-product dataflow, this row-wise-product dataflow is
work-efficient since it processes only non-zero input elements that contribute
to generating non-zero output elements.
In each step, a multiplication between a sparse vector and a sparse matrix may
require merging some partial output vectors into a final row in the output
matrix.
However, compared to the two-dimensional matrix merging step in the
outer-product dataflow, merging partial one-dimensional sparse vectors is much
less complex and requiring much less temporary memory space, which may fit in
on-chip caches.
The row-wise-product dataflow has relatively poor data reuse of matrix B since
column indices of non-zero elements in each row in matrix A are used to access
corresponding rows in matrix B.
Finally, unlike both the inner-product and outer-product approaches, the
row-wise-product dataflow does not require input matrices to be stored in two
different formats: CSR and CSC.
All input and output matrices can be consistently stored in CSR, so there is no
need for converting between sparse matrix formats.

\begin{table*}[tp]
  \centering
  \cbxsetfontsize{10pt}

  \begin{tabular}{lcccccccc}
  \toprule
  \multicolumn{1}{c}{\multirow{2}{*}{\textbf{Dataflow}}} & \multicolumn{3}{c}{\textbf{Data Access Efficiency}}                           & \multicolumn{3}{c}{\textbf{Sparse Format}}                                    & \multicolumn{2}{c}{\textbf{Algorithmic Efficiency}} \\
  \cmidrule{2-4}  \cmidrule{5-7} \cmidrule{8-9}
  \multicolumn{1}{c}{}                                   & \textbf{Mtx A} & \multicolumn{1}{l}{\textbf{Mtx B}} & \textbf{Mtx C} & \textbf{Mtx A} & \multicolumn{1}{l}{\textbf{Mtx B}} & \textbf{Mtx C} & \textbf{Time}            & \textbf{Space}           \\
  \midrule
  Inner-product                                          & Low               & High                                  & High              & CSR               & CSC                                   & Either            & Low                      & High                     \\
  Outer-product                                          & High              & High                                  & Low               & CSC               & CSR                                   & CSR               & High                     & Low                      \\
  Row-wise-product                                       & High              & Low                                   & High              & CSR               & CSR                                   & CSR               & High                     & Medium                   \\
  \bottomrule
  \end{tabular}

  \caption[Trade-offs Among Different SpGEMM Dataflows]{
    \textbf{Trade-offs Among Different SpGEM Dataflows}
  }
  \label{tab-spz-spgemm-dataflows}

\end{table*}


Table~\ref{tab-spz-spgemm-dataflows} summarizes trade-offs among the three
SpGEMM dataflows.
In this work, we consider only the row-wise-product dataflow since it is
relatively work-efficient by avoiding the extremes in both inner-product and
outer-product dataflows.

%----------
% Row-wise algorithms
%----------

\subsection{Row-Wise Product Algorithms}

There are three common algorithms to implement the row-wise-product dataflow
for SpGEMM: array-based, hash-based, and expand-sort-compress algorithms.

\begin{figure}[ht]
    \centering
    \includegraphics[width=0.8\cw]{spa-algorithm.svg.pdf}
    \caption[SpGEMM Algorithm Using Dense Array for Accumulating Sparse Values]{
      \BF{Array-Based SpGEMM Algorithm}
    }
    \label{fig-spz-spa-algorithm}
\end{figure}

\BF{Array-based row-wise SpGEMM --}
This is also known as sparse accumulator algorithm
(SPA)~\cite{gilbert-sparse-matlab-1992}.
Figure~\ref{fig-spz-spa-algorithm} shows an example of how this algorithm
works.
The algorithm uses a set of three dense arrays that have the same size as the
number of columns in the output matrix.
The first two arrays store accumulated values and valid flags.
When a key-value tuple is inserted, it accesses the two arrays by the key.
If the flag for the key is false (i.e., no valid value has been added for this
key yet), it overwrites the corresponding entry in the accumulated value array
and sets the flag to true.
Otherwise, the new value is added to the existing value.
The third array stores inserted keys in the order they are first added to the
list.
Those keys are also indices to the value and flag arrays.
After all key-value tuples are inserted, we can iterate through the key array
to construct an unsorted list of key-value output list.
The list is then sorted by keys to produce one row of the output matrix.
The flag array needs to be reset before processing the next matrix row.

\BF{Hash-based row-wise SpGEMM --}
Another common approach to accumulate sparse values in row-wise SpGEMM is using
a hash table~\cite{anh-hash-spgemm-2016,deveci-multithreaded-spgemm-2018}.
Key-value tuples are inserted into a hash table based on their hashed keys.
Similar to the array-based row-wise SpGEMM, key-value tuples in the hash table
are not sorted, so a final sorting step is needed to generate an output matrix
with per-row non-zero elements sorted by their column indices.

\begin{figure}[tp]
    \centering
    \includegraphics[width=\cw]{esc_algorithms.svg.pdf}
    \caption[Expand-Sort-Compress Algorithm]{
      \textbf{Expand-Sort-Compress Algorithm with the Block Size of Two Rows} --
      Non-zero elements are colored in the input and output matrices.
      Zero elements are left blank.
      A block of two adjacent rows in the matrix A and their corresponding rows
      in the matrix B are expanded, sorted, and compressed together to generate
      output values for the two rows in the output matrix C.
      All rows involved in the current computation step are shown within the
      dark borders.
    }
    \label{fig-spz-esc-algorithm}
\end{figure}

\BF{Expand-Sort-Compress (ESC) algorithm --}
ESC algorithm was initially proposed for performing SpGEMM on
GPUs~\cite{dalton-spgemm-gpu-2015,winter-adaptive-spgemm-gpu-ppopp2019} and
later adopted to vector
architectures~\cite{fevre-spgemm-rvv-2023,li-spgemm-vector-arch-mchpc2019}.
Figure~\ref{fig-spz-esc-algorithm} shows an example of how this ESC algorithm
works.
Unlike the array-based and hash-based SpGEMM algorithms, more than one row of
the input matrix A and output matrix C can be processed together to increase
the amount of work that can be parallelized or vectorized.
Intermediate results of multiplications are expanded in triples of row index,
column index, and value.
The list of triples are then sorted by their row and then column indices.
Triples with duplicate key (i.e., same row and column indices) are compressed
into one entry with the values being accumulated in the final output.
