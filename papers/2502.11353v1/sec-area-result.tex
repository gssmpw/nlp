\subsection{Area Evaluation}
\begin{table}[tp]
  \centering
  \cbxsetfontsize{8pt}
  \tabcolsep 3pt

  \caption{
    Post-synthesis Area Estimates of 16$\times$16 SparseZipper Systolic Array with 512-bit Datapath
  }

  \begin{tabular}{lrrr}
    \toprule
    % first row
    \multicolumn{1}{c}{\multirow{2}{*}{\BF{Component}}}   &
    \multicolumn{1}{c}{\BF{Area}}                         &
    \multicolumn{1}{c}{\multirow{2}{*}{\BF{Baseline}}}    &
    \multicolumn{1}{c}{\BF{Sparse}} \\
    % second row
    \multicolumn{1}{c}{}                  &
    \multicolumn{1}{c}{\BF{(k~$\um^2$)}}  &
    \multicolumn{1}{c}{}                  &
    \multicolumn{1}{c}{\BF{Zipper}}  \\
    \midrule
    Baseline PE (with a 32-bit MAC unit)      &  0.45  & $\times$~256   &               \\
    SparseZipper PE (with a 32-bit MAC unit)  &  0.51  &                & $\times$~256  \\
    Skew buffer (16-lane)                     &  3.16  & $\times$~2     & $\times$~2    \\
    Deskew buffer (16-lane)                   &  3.16  & $\times$~1     & $\times$~2    \\
    Matrix register (16 $\times$ 512b)        &  0.96  & $\times$~16    & $\times$~16   \\
    Popcount logic                                &  0.45  &                & $\times$~1    \\
    \midrule
    % last row
    \multicolumn{2}{l}{\BF{Total}}                              & 140.16 & 158.00       \\
    \multicolumn{2}{l}{\BF{SparseZipper vs. baseline overhead}} &        & \BF{12.72\%} \\
    \bottomrule
  \end{tabular}

    %PE~=~processing element;
    %MAC~=~multiply-accumulate;
    %16$\times$16 systolic array and 512-bit vector length.

  \label{tab-spz-area-result}
  \vspace{-0.3cm}
\end{table}

%    Baseline PE                         &  1.50  & $\times$~256   &               \\
%    SparseZipper PE                     &  1.61  &                & $\times$~256  \\
%    Skew buffer (16-lane)               &  6.32  & $\times$~2     & $\times$~2    \\
%    Deskew buffer (16-lane)             &  6.32  & $\times$~1     & $\times$~2    \\
%    Matrix register (16 $\times$ 512b)  &  0.96  & $\times$~16    & $\times$~16   \\
%    Popc logic                          &  0.45  &                & $\times$~1    \\
%    \midrule
%    % last row
%    \multicolumn{2}{l}{\BF{Total}}                              & 419.15 & 453.78       \\
%    \multicolumn{2}{l}{\BF{SparseZipper vs. baseline overhead}} &        & \BF{8.26\%}  \\


\BF{Methodology --}
We use a post-synthesis component-level area modeling methodology to evaluate
area overheads of hardware added to a baseline 16$\times$16 systolic array for
SparseZipper.
We implement area-significant components of the systolic array in RTL and
synthesize them using a 12nm standard-cell library.
Each PE includes a 32-bit single-precision floating-point MAC unit.
We model control logic added to a PE to support for the stream sorting and
merging operations.
Each skew/deskew buffer is used to stagger input and output data coming in and
going out of the systolic array.
We model each skew/deskew buffer as an array of 16 shift registers using
flip-flops with their sizes ranging from one to 16 entries.
We model 16 rows, each is 512-bit wide (16$\times$ 32-bit data elements), in a
SRAM-based matrix register and a total of 16 physical matrix registers.
Regarding the popcount logic, we implemented an array of 16 five-bit counters
(counting up to 16) and a list of counter vector registers (16 $\times$ 5 bits
per register).

\BF{Area overheads of SparseZipper --}
Table~\ref{tab-spz-area-result} shows the detailed area comparison between
SparseZipper and the baseline using our first-order component-based area
modeling methodology.
In overall, a 16$\times$16 SparseZipper implementation adds around 12.72\% area
overhead compared to the baseline implementation with the same systolic array's
dimensions.
When considering a complete system including an out-of-order core, its vector
engine, and its caches, we expect the percentage of extra area added to the
baseline systolic array for supporting SparseZipper to be much lower.
