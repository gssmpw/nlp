%=============================================================================
% asm code for merging paritions of VL elements
%=============================================================================

%\lstset{language=[RISC-V]{Assembler}}
\lstset{style=riscv-asm-style}

\begin{figure}[tp]

\centering
\includegraphics[width=\cw]{stream-zip-example.svg.pdf}

\begin{lstlisting}[xleftmargin=0.025\tw]
# a0, a1: base address of input key and value arrays
# a2, a3: base address of output key and value arrays
# v0, v1: lengths of the 1st and 2nd input paritions
# v2, v3: pointers to the 1st and 2nd input paritions
# v4: lengths of the output chunks
# v5: pointers to the output chunks
loop:
  # load
  mlxe.t tr0, 0(a0), v2, v0  # load chunks of keys from 1st input partitions
  mlxe.t tr1, 0(a1), v2, v0  # load chunks of values from 1st input partitions
  mlxe.t tr2, 0(a0), v3, v1  # load chunks of keys of 2nd input partitions
  mlxe.t tr3, 0(a1), v3, v1  # load chunks of values of 2nd input partitions
  # zip
  mszipk.tt tr0, tr2, v0, v1 # zip keys of 1st and 2nd input chunks
  mszipv.tt tr1, tr3, v0, v1 # zip values of 1st and 2nd input chunks
  mmv.vi v6, 0x0             # get number of zipped input elements (1st chunks)
  mmv.vi v7, 0x1             # get number of zipped input elements (2nd chunks)
  mmv.vo v8, 0x0             # get number of output elements (1st part)
  mmv.vo v9, 0x1             # get number of output elements (2nd part)
  vsub.vv v0, v0, v6         # update input lengths (1st chunks)
  vsub.vv v1, v1, v7         # update input lengths (2nd chunks)
  vadd.vv v2, v2, v6         # bump input pointers (1st chunks)
  vadd.vv v3, v3, v7         # bump input pointers (2nd chunks)
  # store
  msxe.t tr0, 0(a2), v5, v8  # store keys of 1st output chunks
  msxe.t tr1, 0(a3), v5, v8  # store values of 1st output chunks
  vadd.vv v5, v5, v8         # bump output pointers
  msxe.t tr2, 0(a2), v5, v9  # store keys of 2nd output chunks
  msxe.t tr3, 0(a3), v5, v9  # store values of 2nd output chunks
  vadd.vv v5, v5, v9         # bump output pointers
  ...
  vpopc.m a4, v10   # count the number of active pairs of partitions
  bnez a4, loop
\end{lstlisting}

\caption[Zipping Partitions of Keys and Values across Multiple Streams in Parallel]{
  \BF{Zipping Partitions of Keys and Values across Multiple Streams in Parallel
  Using the Proposed Matrix Instructions} --
  VL~=~vector length (i.e., four elements in this example);
  \TT{a\{0..4\}}~=~scalar registers;
  \TT{v\{0..10\}}~=~vector registers;
  \TT{tr\{0..3\}}~=~matrix registers.
}
\label{fig-spz-mszip-code}
\end{figure}
