\section{Related works}
%%%%%%%%%%%%%%%%%%%%%%%%%%%%%%%%%%%%%%%%%%%%%%%%%%%%%%%%%%%%%%%%%%%%%%%%%%%%%%%%%%%%%%
%% general intro
The state-of-the-art review is divided into two paragraphs, each addressing key research areas to which the DF aims to contribute. The first focuses on works related to fabrication evaluation in timber construction via scanning. In contrast, the second examines the open-source CAD-integrated point cloud processing plug-ins and free software available to this date. 

\subsection{Timber fabrication evaluation}
\label{sec:rel:timber_eval}
%% scanning in quality control for timber fabrication
Timber fabrication is not new to the use of scanning and sensing systems to ensure quality control.
%% glulam (since there was the most literature)
Free-form glulam production, similar to round wood manufacturing, is hindered by the lack of clear reference points. This called for early adoption of scanning techniques and point cloud, often live, processing to better align virtual models with the physical production of such elements. Consequently, this has led to the development of several point cloud-based sensing workflows [____].
%% subtractive
Digital assessment through scan-CAD can gauge differences between robotic milling of timber joinery against standard CNC tools [____], or whenever a 3D model exists for any manually worked element.
%% robotic assembly
Concerning robotic manufacturing and assembly, scan-to-CAD comparison methodologies can be employed to evaluate the accuracy of the process, as errors can occur due to specific material behavior or operations [____], human agents [____], or large scale prototyping [____].

\subsection{Existing scan processing tools}
\label{sec:rel:timber_eval}
%% tools in Rh/Gh
Rhino's Grasshopper (GH) environment has recently been populated with multiple scan processing plug-ins. Point Cloud Components [____] is one of the very first to propose point cloud processing and geometry-to-scan preliminary tools originally for landscape applications. Volvox [____] proposes more advanced methods such as cropping, merging, and sub-sampling, and it tackles generic scan-CAD comparisons for some study cases at the building scale. It is also worth mentioning Tarsier [____], a small-scale open-source project primarily utilized for point cloud visualization from sensing devices. Cockroach [____] is the first umbrella plug-in regrouping multiple external point cloud processing libraries, presenting an extensive, yet, as the rest of the mentioned plug-ins, very generic post-processing tool-set.
% outside CAD
Beyond the limitations of CAD, two of the most widely recognized free solutions for scan processing and comparison are CloudCompare [____] and MeshLab [____]. Both provide an extensive range of functionalities, with CloudCompare being particularly adept at detecting discrepancies between point clouds and other data types, such as meshes or between point clouds. While these tools offer flexibility through plug-in systems and Python wrappers, they lack seamless integration within CAD environments. When switching between multiple software, we lose crucial semantic information necessary to track specific features of our timber components, such as joint faces or individual assembly elements, which are important in the fabrication process.


%%%%%%%%%%%%%%%%%%%%%%%%%%%%%%%%%%%%%%%%%%%%%%%%%%%%%%%%%%%%%%%%%%%%%%%%%%%%%%%%%%%%%%