\section{Preliminaries}
% \mznote{1. bidding strategies or bids? 2. Should we use $b_1,...,b_n$ as the bids instead of 1,...,n?}
% \xtnote{add budget constraint} 
We consider the problem of finding the optimal bidding strategy in a multi-platform auction setting for a value-maximizer with \emph{budget} and \emph{return on spend} (ROS) constraints. There is a set $\platform$ consisting of $m$ platforms in the market.
We assume that for each platform, the advertiser can pick from $n$ different bids (note that this set can be different for different platforms), indexed by $0,1,2,\dots,n$, where bid $0$ is used to denote non-participation.
% Let $\strategy = \{0,1, 2, \dots, n\}$ \ganote{should this start from 1?} \xtnote{we are including $0$ so that the algorithm can always choose to not spend on some platform, so there also is a feasible solution.}
% \xtnote{for presentation purpose add why it is 0 to n} be the set of possible bidding strategies in each platform. 
Each platform $j \in \platform$ is described by a value and a cost function that map {bid indices} to a corresponding value and cost, respectively, i.e., $v_j:\strategy \rightarrow \R^{\geq 0}$ and $c_j: \strategy \rightarrow \R^{\geq 0}$. In other words, when a bidder chooses to bid according to bid $\mu \in \strategy$, they incur a cost of $c_j(\mu)$ and receive a value of $v_j(\mu)$.
We assume that there is a strict ordering of costs and values by bid index, i.e. $v_j(\mu) < v_j(\mu+1)$ and $c_j(\mu) < c_j(\mu+1)$ for all $\mu \in \strategy$. 
% \footnote{One can think of the set $\strategy$ as the set of multipliers to apply to the bid. \ganote{this is confusing, since 1...n may not make as much sense as multipliers}}
We refer to the mapping from bid to the cost and value of each platform as the \emph{landscape} of that platform. 
In addition, we define the \emph{marginal cost} of bidding $\mu\geq 1$ on platform $j$ as
% \begin{align}\label{eq:marginalcost}
%     \mc_j(\mu) = \lim_{\Delta \rightarrow 0}\frac{c_j(\mu+\Delta)-c_j(\mu)}{v_j(\mu+\Delta)-v_j(\mu)}. \tag{Marginal Cost}
% \end{align}
\begin{align}\label{eq:marginalcost}
    \mc_j(\mu) = \frac{c_j(\mu)-c_j(\mu-1)}{v_j(\mu)-v_j(\mu-1)}, \tag{Marginal Cost}
\end{align}
where $c(0)=0$ and $v(0)=0$.
We make standard convexity assumption that $\mc_j$ is non-decreasing for every platform $j$. 
% \xtnote{Any other assumptions we need for the functions?}

{Given the integral strategy set, we expand the bidding space by also considering the fractional solution between each integral bid, hence making the strategy space continuous. We use $S$ and $S^c$ to denote the integral and fractional strategy space, respectively. The cost, value and marginal functions of the continuous bidding space $[0,n]^m$ extend the discrete function by linear interpolation.\footnote{It can be viewed as bidding randomly between two adjacent bids.} Formally
\[v_j(\mu) = (\ceil{\mu} - \mu) \cdot v_j(\floor{\mu}) + (\mu -\floor{\mu})  \cdot v_j(\ceil{\mu}),\]
\[c_j(\mu) = (\ceil{\mu} - \mu) \cdot c_j(\floor{\mu}) +(\mu -\floor{\mu})  \cdot c_j(\ceil{\mu}),\]
%i.e., the cost and value of some non-integer strategy is the linear combination of its floor and ceiling strategies. 
Consequently, we have that $\mc_j(\mu) = \mc_j(\ceil{\mu})$.}

% \xtnote{We first define the fractional (randmized) version of our problem. Instead of having discrete bidding strategy $S = \{0,1,\dots, n\}$, we will have a continuous strategy space $S = [0,n]$, where the cost, value and marginal cost functions are defined as follows:
% \[v_j(\mu) = (\mu -\floor{\mu}) \cdot v_j(\floor{\mu}) + (\ceil{\mu} - \mu) \cdot v_j(\ceil{\mu}),\]
% \[c_j(\mu) = (\mu -\floor{\mu}) \cdot c_j(\floor{\mu}) + (\ceil{\mu} - \mu) \cdot c_j(\ceil{\mu}),\]
% i.e., the cost and value of some non-integer strategy is the linear combination of its floor and ceiling strategies. Consequently, we have that
% \[\mc_j(\mu) = \mc_j(\ceil{\mu}).\]}
% \mznote{Add note that finding the optimal integral solution is NP-hard.}
{We note that the problem of finding the optimal integral solution is NP-hard.\footnote{It is not hard to see that we can encode any knapsack problem as an instance of our problem with a budget constraint.}}
The objective of the bidder is therefore to find an optimal \emph{fractional}  bidding strategy $\bid = (\mu_1, \mu_2, \dots, \mu_m)$ {where $\mu_j \in [0,n]$} such that she maximizes the total value received by executing bidding strategy $\mu_j$ on each platform $j$, subject to {a budget constraint} and the ROS constraint \emph{across all platforms}.
Let {$B$ and} $T$ be the budget and target ROS of the bidder. We can formulate the problem as the following program:
{\begin{align}\label{eq:bidderproblem}
    \max_{\bid=(\mu_1, \mu_2 \dots, \mu_m)} &\sum_{j \in \platform} v_j(\mu_j) \nonumber\\
    s.t. & \sum_{j \in \platform} c_j(\mu_j) \leq T \cdot \sum_{j \in \platform} v_j(\mu_j),\\
    & \sum_{j \in \platform} c_j(\mu_j) \leq B. \nonumber 
\end{align}}
{throughout the paper, we denote $\optf$ the optimal (fractional) bidding strategy, and $\opti$ the floor of it, i.e., $\mu_j^* = \floor{\mu^o_j}$ for all platform $j$. Note that $\mu_j^* \in S$.}

%If the landscape ($v_j$ an $c_j$) of each platform is known, the bidder can compute the optimal solution. However, in practice, the bidder usually has no information prior to interacting with the platforms. Instead, she needs to \emph{query} each platform by playing a strategy to learn the corresponding values and costs.
We assume that the bidder only knows the set of possible bidding strategies, but has no information about the platforms' value and cost functions. Instead, the bidder can interact with platforms via \emph{bidding queries}: the bidder plays strategy $\mu$ on a platform $j$ to learn the value $v_j(\mu)$, the cost $c_j(\mu)$, and the marginal cost $\mc_j(\mu)$\footnote{When marginal cost is not part of the query output, it is still achievable by querying both the current and the previous bid, which increases the query complexity by a constant factor.}.
% \xtnote{define worst-case query complexity of an algorithm? include randomized one as well?}
Each such query is costly to the bidder, and the goal is to minimize the number of queries required to determine the optimal strategy. 

Given an instance $\mathcal{I}$ and
% \mznote{Modify the notation using $\mathcal{I}$ as instance.} 
an algorithm $\alg$, let $\alg(\mathcal{I})$ denote the number of queries needed to find the optimal strategy for that instance. Then the query complexity of the algorithm is defined as:
\[\max_{\mathcal{I}} \alg(\mathcal{I})\]

\paragraph{The learning-augmented framework} In this work, we adopt the learning-augmented framework and study how we can further reduce the query complexity by considering search algorithms that are equipped with a (potentially erroneous) prediction $\pred \in [0,n]^m$ of the  optimal fractional bidding strategy $\optf(\mathcal{I}) = (\mu^o_1, \mu^o_2,\dots, \mu^o_n)$. The \emph{error} of an predictions $\eta$ is defined to be the maximum point-wise deviation from $\optf$, formally:
\[\eta(\pred,\mathcal{I}) = \max_{j}|\hat{\mu}_j - \mu^o_j(\mathcal{I})|\]
We let the algorithm $\alg$ use both the instance $\mathcal{I}$ and the prediction $\pred$ as input. We evaluate the performance of such an algorithm using its \emph{robustness}, \emph{consistency} and the query complexity as a function of the prediction error. 

The robustness of an algorithm refers to the worst-case query complexity of the algorithm
given an adversarially chosen, possibly erroneous, prediction. Mathematically,
\[\text{robustness}(\alg) = \max_{\pred, \mathcal{I}} \alg(\pred, \mathcal{I})\]
The consistency of an algorithm refers to the worst-case query complexity
when the prediction that it is provided with is accurate, i.e., $\pred = \opti(\mathcal{I})$. Mathematically,
\[\text{consistency}(\alg) = \max_{\pred, \mathcal{I}: \pred = \opti(\mathcal{I})} \alg(\pred, \mathcal{I}).\]
Lastly, the query complexity of an algorithm given a prediction with error $\eta'$ is defined to be:
\[\max_{\pred, \mathcal{I}: \eta(\pred,\mathcal{I}) \leq \eta'} \alg(\pred, \mathcal{I}).\]