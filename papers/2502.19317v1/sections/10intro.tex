\section{Introduction}
%\xtnote{update to add budget constraint}
%\xtnote{add discussion of Yuan's paper}
Online advertisers often advertise across multiple platforms, such as Amazon, Bing, Google, Meta and TikTok, and face the challenging task of optimizing their bids across these platforms. The complexity comes not just from having to select a vector of bidding strategies (one for each platform), but also from the diversity of auctions used across platforms and the black-box nature of the detailed auction rules and the level of competition. 

To deal with the complexity, advertisers are increasingly using automated bidding agents (aka autobidding) to bid on their behalf. An autobidder allows an advertiser to specify constraints like Budget and Return-on-Spend (ROS), and bids on their behalf to maximize value subject to the constraints. 
%In the theoretical model, the optimal bidding strategy is tractable given a complete mapping from any possible bid submitted to each platform to the corresponding auction outcome. However in practice, due to the complexity of the auctions and the dynamic market environment, such a mapping is hard to achieve beforehand.
This has led to a lot of research interest in problems related to autobidding (see ~\citet{AutobiddingSurvey24} for a recent survey). In particular, the problem of designing bidding algorithms for the single platform setting is well-studied~\cite{ABM19}, including in the online learning setting (see Section 3 of the survey \cite{AutobiddingSurvey24}). However, there is not much work on the problem of bidding optimally across multiple platforms (see the related work section for what is known).
%With the rise of automated bidding, there has been extensive research effort on constructing bidding algorithms in classic auctions for value-maximizing advertisers. 
%Yet the bidding problem when advertisers participate in the multi-platform system is not well-studied. 

In this paper, we study the problem of finding the optimal bidding strategy in the multi-platform setting. In particular, an advertiser aims to maximize her total value subject to a global budget and return-on-spend (ROS) constraint across all platforms. %To capture the lack of transparency into platforms' auctions, %difficulty in practice,
To capture the black-box nature of auction mechanisms and level of competition, we assume that the advertiser has no prior knowledge about the mapping from bids to auction outcomes for any platform. Instead, the advertiser interacts with a platform's auction by submitting a bid to the platform and observing the corresponding cost and value (i.e., the user has ``query access'' to the mapping). We propose algorithms that find the optimal bidding strategy in this setting, %with no prior knowledge of each platform’s auction, 
and prove worse-case query complexity bounds for them. 

%\xtnote{paragraph about learning-augmented framework}
While worst-case results offer robustness and broad applicability, the guarantees they provide can sometimes be overly pessimistic. To address this, a new framework called "algorithms with predictions" has recently been introduced. This framework allows algorithms to incorporate potentially flawed machine-learned predictions as a guiding tool. The objective is twofold: to achieve improved performance guarantees when the prediction is accurate (a property known as \emph{consistency}) and to maintain good worst-case bounds even when the prediction is completely incorrect (a property called \emph{robustness}). This framework provides a natural way to integrate machine-learned predictions into the design of algorithms while preserving the essential robustness offered by worst-case analysis. In this work, we adopt the learning-augmented framework and explore the role of predictions in bidding strategy optimization. Specifically, we examine the scenario where the algorithm has access to a prediction $\hat{\bid}$ of the optimal bidding strategy, without any assumption regarding the prediction's accuracy. We propose algorithms that leverage the untrusted prediction to achieve improved query complexity bounds, which degrade gracefully based on the quality of the prediction.

\subsection{Our Results}
% \xtnote{discuss the technical difficulty as well, that it is not downward-closed}

%\agnote{We have not givenexplained what $m$ and $n$ are, so the reader will be confused what the model is. Would it be useful to give a short explanation: e.g., We model the problem as follows: there are $m$ channels. For each channel $j$, we are given a cost function $c_j$ and a value function $v_j$, such that $c_j(\mu_j)$ and $v_j(\mu_j)$ are respectively the cost incurred by bidding $\mu_j$ on channel $j$, and the value accrued from this bid. Our goal is to find a collection $\bid = (\mu_1, \ldots, \mu_m)$ such that (a) the ratio of the total value accrued to the total cost is at least some target threshold, and (b) the total value is maximized.} 

We model the problem of searching for the optimal bidding strategy as follows: there are $m$ platforms. For each platform $j$, we are given a cost function $c_j$ and a value function $v_j$, and $n$ different bidding strategies indexed $1$ through $n$ such that $c_j(\mu_j)$ and $v_j(\mu_j)$ are respectively the cost incurred by bidding $\mu_j$ on platform $j$, and the value accrued from this bidding strategy. 
% (We assume that these functions are piece-wise linear, with at most $n$ breakpoints.)
%\mznote{Do we want to explicitly assume this or say that there are n different bidding strategies?} \xtnote{currently in the prelim we are saying there are $n$ different bidding strategies} 
Our goal is to find a collection $\bid = (\mu_1, \ldots, \mu_m)$ s.t. (a) the ratio of the total value accrued to the total cost is at least some target threshold, (b) the total cost is less than the budget, and (c) the total value is maximized. 
% \ganote{added the budget constraint in the previous sentence}

%We first provide a characterization of the optimal fractional bidding strategy. \ganote{This makes it sound like a new characterization. Are we claiming that? This is in the APSZ paper, and maybe in some form earlier too} 
We propose a search algorithm, \mom, that determines the exact optimal bidding strategy using \(O(m \log (mn) \log n)\) queries. Our algorithm builds on a characterization of the optimal bidding strategy in the multi-platform setting first developed in \cite{APSZ24} under continuous strategy space.
Intuitively, the optimal strategy is to keep increasing bid on the platform that currently offers the highest marginal bang-per-buck (corresponding to the lowest marginal cost-per-unit-value) until
either the budget or 
the ROS constraint is about to be violated.
%\agnote{I commented out the budget constraint comment until we add it in.}
In other words, the optimal strategy aims to equalize the marginal cost-per-unit-value across all the platforms. 
%In fact, algorithm would, consider the process of gradually increasing spend. \ganote{reword this a bit. allocating funds works for budget, not so much for ROI constraint} 
%This aligns with the findings presented by \cite{APSZ24} in the continuous strategy space. This characterization gives us a clear structure of the optimal solution w.r.t. marginal costs (Lemma~\ref{cor:opt}).

%Using this characterization, we propose a search algorithm, \mom, that determines the exact optimal solution using \(O(m \log (mn) \log n)\) queries.
At a high level, the algorithm searches in the space of marginal costs. Initially, there are up to $mn$ candidate marginal costs.  The algorithm carefully selects a candidate marginal cost and finds the corresponding vector of bidding strategies, as defined in Lemma~\ref{lem:opt}, via Subroutine~\ref{sub:matchingmc}. Based on the outcomes (i.e. cost and value on each platform) of the corresponding strategy vector, the algorithm removes a constant fraction of candidate marginals from consideration, and recurses on the residual problem. 

{We complement this algorithmic result with an $\Omega(m \log n)$ lower bound and an $\Omega(\log mn)$ lower bound for this problem. The $\Omega(\log mn)$ lower bound reflects the difficulty of identifying the optimal marginal cost among the $mn$ possible candidates, while the $\Omega(m \log n)$ bound captures the complexity of determining the corresponding bidding strategy for that optimal marginal cost. Notably, these are the two key components of our algorithm, and our upper bound is the product of these two lower bounds.}%\ganote{this is different from the upper bound. do we want to say here in what sense is this a good bound?}%gagan:this looks good.

Next, we adopt the learning-augmented framework to improve the worst-case query complexity bound.
% \ganote{what does floor of the strategy mean?} 
We propose an algorithm with access to a prediction \(\hat{\bid}\) of the
% \xtnote{floor of the} 
optimal strategy \(\optf\). The algorithm, \bmom, starts with trying to the find the optimal solution in a small range around the predictions, and expands the search range if the search is unsuccessful. With the right sequence of expanding ranges, we show that the algorithm finds the optimal strategy \(\bid^*\) with \(O(m \log (m\eta) \log \eta)\) queries, where \(\eta = \max_{j}|\bid^*_j - \hat{\bid}_j|\) represents the prediction error. 
% \ganote{prediction error is the difference between the mu's or the difference between the indices of the mus?} \xtnote{$\mu$ is the indices of the strategies}. 
This means that the algorithm requires only \(O(m)\) queries when the predictions are accurate; this is the minimum number of queries needed to implement any bidding strategy. Moreover, since \(\eta \leq n\), the total number of queries never exceeds that of the \mom\ algorithm.

%Finally, we show that one can use the classic \emph{centroid} optimization algorithm to find a fractional bidding strategy that is \(\varepsilon\)-close to the optimal strategy using \(O(m^2 \log (G\sqrt{m}n/\varepsilon))\) queries, where \(G\) is the upper bound of the derivatives of the value functions. This provides an alternative to our \mom algorithm above, and is better in some parameter-regimes.

%advertisers with a greater variety of algorithms to choose from.

\subsection{Related Work}
\paragraph{Multi-platform mechanism design and autobidding.} Previous research has examined the multi-platform auction environment from both the auctioneer's and the bidders' perspectives. Regarding auction design, \citet{aggarwal_perlroth_zhao_ec23} analyzes a scenario where a single platform manages multiple channels, each selling queries via a second-price auction (SPA) with a reserve price. The authors assess the costs associated with each channel optimizing its own reserve price compared to a unified platform policy. Inspired by the Display Ad market, \citet{renato_balu_yifeng_www2020} explores a model in which multiple platforms vie for profit-maximizing bidders who must use the same bid across all platforms (which we refer to as a uniform bid). Their key finding indicates that the first-price auction (FPA) serves as the optimal auction format for these platforms. On the bidders' side, \citet{susan2023multi} investigates bidding strategies for utility-maximizing advertisers operating across multiple platforms while adhering to budget constraints. Meanwhile, \citet{deng2023multi} focuses on value-maximizing advertisers and reveals that optimizing ROS per platform can yield arbitrarily poor results when both ROS and budget constraints are in play. 

\citet{APSZ24} study a similar multi-platform setting with autobidders under ROI constraints but focus on the auction design problem from the platform's perspective. While first-price auctions are optimal in the absence of competition (\citet{deng2021towards}), they show that from the perspective of each separate platform, running a second-price auction can achieve larger revenue than first-price auction when there are two competing platforms. They also identify key factors influencing the platform’s choice of auction formats, including advertiser sensitivity to auction changes, competition and relative inefficiency of second-price auctions. In our paper, we focus on how advertisers can bid optimally in the multi-platform setting.

% \paragraph{Multi-dimensional Binary Search} 

\paragraph{Algorithms with predictions.} In recent years, the learning-aug\-mented framework has emerged as a prominent paradigm for the design and analysis of algorithms. For an overview of early contributions, we refer to \citep{MV22}, while \citep{alps} offers an up-to-date compilation of relevant papers in this area. This framework seeks to address the shortcomings of overly pessimistic worst-case analyses. In the last five years alone, hundreds of papers have explored traditional algorithmic challenges through this lens, with notable examples including online paging \citep{lykouris2018competitive}, scheduling \citep{PSK18}, optimization problems related to covering \citep{BMS20} and knapsack constraints \citep{IKQP21}, as well as topics like Nash social welfare maximization \citep{banerjee2020online}, the secretary problem \citep{AGKK23, DLLV21, KY23}, and various graph-related problems \citep{azar2022online}.

More closely related to our work, the research on learning-aug\-mented mechanisms interacting with strategic agents is recently initiated by \citet{ABGOT22} and \citet{XL22}. This area includes strategic facility location \citep{ABGOT22, XL22, IB22, BGT24}, strategic scheduling \citep{XL22, BGT223}, auctions \citep{MV17, XL22, LuWanZhang23, caragiannis2024randomized, BGTZ23}, bicriteria mechanism design (which seeks to optimize both social welfare and revenue) \citep{BPS23}, graph problems with private input \citep{CKST24}, metric distortion \citep{BFGT23}, and equilibrium analysis \citep{GKST22, IBB24}. Recently, \citet{CSV24} revisited mechanism design challenges by focusing on predictions about the outcome space rather than the input. While most of these studies concentrate on the mechanism design problem, our research emphasizes how predictions can assist agents in identifying optimal strategies.  For more information on this body of work, we recommend \citep{BGT23}.