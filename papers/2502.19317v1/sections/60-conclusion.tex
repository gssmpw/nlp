\section{Conclusion}
In this work, we addressed the challenge of finding the optimal bidding strategy for advertisers operating in a multi-platform auction environment with low query complexity. Our approach models competition within each platform through value and cost functions that map various bidding strategies to their respective outcomes. We introduced an efficient algorithm that achieves this goal with a query complexity of \(O(m \log (mn) \log n)\), where \(m\) represents the number of platforms and \(n\) denotes the number of potential bidding strategies available on each platform. 

To further enhance efficiency, we incorporated the learning-augmented framework, proposing an algorithm that leverages a potentially flawed prediction of the optimal bidding strategy. Our results provide a query complexity bound that degrades gracefully, achieving \(O(m)\) queries when accurate predictions are available and \(O(m \log (mn) \log n)\) even with completely incorrect predictions. This flexibility exemplifies a ``best-of-both-world'' scenario, providing advertisers with different options to effectively navigate the complexities of multi-platform bidding with minimal queries.

We believe that autobidding in multi-platform auction settings is understudied, and many intriguing questions remain unanswered. One immediate question for exploration is closing the gap between the upper and lower bounds established in our work, which would necessitate different tools and ideas. Additionally, it would be interesting to analyze the dynamics of the market if all the bidders adopted the approach presented in this study in determining their bidding strategies.

