\section{Characterization of Bidder's Optimal Bidding Strategy}
In this section, we present a characterization of the optimal bidding strategy $\optf$ that will be useful in designing the algorithm. To this end, we first prove a useful lemma about the ``ranking'' of integral strategies in $S$. We then argue how an ``almost-optimal'' integral solution can be used to determine the optimal fractional solution.
% \begin{definition} Given some positive number $a$, define $\bid^a = (\mu^a_1, \mu^a_2, \dots \mu^a_m)$ where
% \[\mu^a _j = \argmax_{\mu}\{\mc_j(\mu)\leq a\}.\]
% \end{definition}
\begin{lemma}\label{lem:opt}
Given some positive number $k$, and the $n$ discrete indices on each platform,
% \ganote{should this be S or {bid indices}? we are assuming here that the set S is the same for each platform, so perhaps bid indices is better. Or maybe we can say somewhere that we will refer to the bid indices as the set of strategies?},
define $\bid^k = (\mu^k_1, \mu^k_2, \dots \mu^k_m)$ where
\[\mu^k_j = \argmax_{\mu \in \strategy}\{\mc_j(\mu)\leq k\},\]
then there exist a $k^*$ such that for any $k \leq k^*$, $\bid^k$ is a feasible solution for program\eqref{eq:bidderproblem} and for any $k' > k^*$, $\bid^{k'}$ is not feasible.
\end{lemma}
We first show the following helper lemma. Intuitively, if we consider the landscape of each platform, and connect each bidding strategy with a straight line, the landscape would be convex, the lemma below is simply a property of a convex function. 
% Due to space limitations, we defer the proof to appendix~\ref{app:convexlemma}.
\begin{lemma}\label{lem:helper}
For any platform $j \in \platform$,
% , the ratio $\frac{c_j(\mu)}{v_j(\mu)}$ is weakly increasing, in addition,
$\frac{c_j(\mu)}{v_j(\mu)} \leq \mc_j(\mu)$.
\end{lemma}

\begin{proof}
For presentation purpose, we drop the subscript $j$ in this prove as it should hold for any platform. We define $c(0)/ v(0) = 0$ We prove the statement via induction. First consider the base case for $\mu = 1$, we have $c(1)/v(1) \geq 0 = c(0)/ v(0)$ since both the cost and the value functions weakly increase w.r.t $\mu$, we also have $c(1)/v(1) = \frac{c(1)- c(0)}{v(1) - v(0)} = \mc(1)$ by definition. The base case is therefore established.

Let $\frac{c(\mu)}{v(\mu)} = X_{\mu}$. Assume, for induction, that $X_{\mu'} \leq \mc(\mu')$ for any $\mu' < \mu$. We first show $X_{\mu-1} \leq X_{\mu}$ holds for $\mu\geq 2$. Consider
{\allowdisplaybreaks
\begin{align*}
     c(\mu) & = X_{\mu}\cdot v(\mu)\\
     c(\mu) - c(\mu-1) & =  X_{\mu}\cdot v(\mu) - c(\mu-1)\\
     \mc(\mu) \cdot (v(\mu) - v(\mu-1)) & = X_{\mu}\cdot v(\mu) - c(\mu-1)\\
     \mc(\mu-1) \cdot (v(\mu) - v(\mu-1)) & = X_{\mu}\cdot v(\mu) - c(\mu-1)\\
     X_{\mu-1}\cdot (v(\mu) - v(\mu-1)) &\leq X_{\mu}\cdot v(\mu) - c(\mu-1)\\
     X_{\mu-1}\cdot (v(\mu) - v(\mu-1)) &\leq X_{\mu}\cdot v(\mu) - X_{\mu-1} \cdot v(\mu-1)\\
     X_{\mu-1} \cdot v(\mu) & \leq X_{\mu}\cdot v(\mu)\\
     X_{\mu-1} &\leq X_{\mu},
\end{align*}}
where the third equality is by definition of $\mc$, the forth equality is by monotoncity of $\mc$, and the first inequality is by induction hypothesis $\mc(\mu-1)\geq X_{\mu-1}$.
In addition, consider the same set of equation again:
{\allowdisplaybreaks\begin{align*}
     c(\mu) & = X_{\mu}\cdot v(\mu)\\
     c(\mu) - c(\mu-1) & =  X_{\mu}\cdot v(\mu) - c(\mu-1)\\
    c(\mu) - c(\mu-1) & = X_{\mu}\cdot v(\mu) - X_{\mu-1} \cdot v(\mu-1)\\
    c(\mu) - c(\mu-1) & \geq X_{\mu}\cdot v(\mu) - X_{\mu} \cdot v(\mu-1)\\
    \mc(\mu) \cdot (v(\mu) - v(\mu-1))  & \geq X_\mu \cdot (v(\mu)-v(\mu-1))\\
    \mc(\mu) & \geq X_\mu
\end{align*}}
where as the first inequality is due to $X_{\mu-1} \leq X_{\mu}$, we therefore have $\mc(\mu) \geq X_\mu$, hence proved.
\end{proof}

\begin{proof}[Proof of Lemma~\ref{lem:opt}]
let $k$ be the smallest $k$ with infeasible $\bid^k$, if the infeasibility is due to the budget constraint, then for any $k' \geq k$ we trivially have that $\bid^{k'}$ violates the budget constraint as well since $\mu_j^{k'} \geq \mu_j^{k}$ and the cost functions are monotone.

If the infeasibility is due to the ROS constraint, i.e.,
\begin{align}\label{eq:smallestviolation}
   \sum_{j \in M}c_j(\mu_j^{k}) > T \cdot \sum_{j \in M}v_j(\mu_j^{k}),
\end{align}
proving the statement is equivalent to proving for any $k' \geq k$, $\bid^{k'}$ is also infeasible. To this end, we first show that the maximum marginals among the $\mu_j^{k}$ is strictly more than $T$, assume for contradiction, that $\mc_j(\mu_j^{k}) \leq T$ for all $j$, by lemma~\ref{lem:helper} we  would have the $c_j(\mu_j^{k})/v_j(\mu_j^{k}) \leq \mc_j(\mu_j^{k}) \leq T$, which contradicts with \eqref{eq:smallestviolation}. We therefore have
\begin{align}\label{eq:lowerbound}
    \max_{j \in \platform} \mc_j(\mu_j^{k}) > T
\end{align}
We now inductively prove that for any $k' \geq k$, we have $\bid^{k'}$ is infeasible. Consider increasing $k'$ starting from $k$, at the beginning we could have $\bid^{k'} = \bid^k$ (which is infeasible), consider the first point $k' \geq k$ such that $\bid^{k'} \neq \bid^k$, we know that:
\begin{enumerate}
    \item there exist at least one platform $j'$ such that $\mu_{j'}^{k'} = \mu_{j'}^k+1$
    \item $\mc_{j'}(\mu^{k'}_{j'}) > \max_{j \in \platform} \mc_j(\mu_j^{k}) > T,$
 \end{enumerate}
%begin{align*}
%     \text{there exist at least one platform $j'$ such that } \mu_{j'}^{k'} = \mu_{j'}^k+1\\
%     \mc_{j'}(\mu^{k'}_{j'}) > \max_{j \in \platform} \mc_j(\mu_j^{k}) > T,
% \end{align*}
where the first inequality is by definition of $\bid^{k'}$ and the second inequality is due to \eqref{eq:lowerbound}. Now consider:
{\allowdisplaybreaks\begin{align*}
    \sum_{j \in \platform}c_j(\mu_j^{k'}) &= \sum_{j \in M}c_j(\mu_j^{k}) + c_{j'}(\mu_{j'}^k+1) - c_{j'}(\mu_{j'}^k)\\
    & > T \cdot \sum_{j \in M}v_j(\mu_j^{k})+ c_{j'}(\mu_{j'}^k+1) - c_{j'}(\mu_{j'}^k)\\
    & > T \cdot \sum_{j \in M}v_j(\mu_j^{k})+ \mc_{j'}(\mu^k_{j'}+1)\cdot (v_{j'}(\mu_{j'}^k+1) - v_{j'}(\mu_{j'}^k)\\
    &> T \cdot \sum_{j \in M}v_j(\mu_j^{k})+ T \cdot (v_{j'}(\mu_{j'}^k+1) - v_{j'}(\mu_{j'}^k)\\
    &> T \cdot \sum_{j \in M}v_j(\mu_j^{k'})
\end{align*}}
Inductively apply this argument for each update of $\bid^{k'}$ proves the statement.
\end{proof}

% \begin{center}  % Center the figure within the column
% \resizebox{0.45\textwidth}{!}{
% \begin{tikzpicture}
%     % Draw the main line representing k
%     \draw[thick] (0,0) -- (10,0);

%     % Mark the strategy points (mu) along the k line
%     \foreach \x in {1, 3, 5, 7, 9} {
%         \draw[thick] (\x,0) -- (\x,0.4);
%     }

%     % Draw arrows pointing to the strategy points (mu values)
%     \draw[-] (1, 0.4) -- (1, 1) node[below] {$\mu_1$};
%     \draw[-] (3, 0.4) -- (3, 1) node[below] {$\mu_2$};
%     \draw[-] (5, 0.4) -- (5, 1) node[below] {$\mu_3$};
%     \draw[-] (7, 0.4) -- (7, 1) node[below] {$\mu_4$};
%     \draw[-] (9, 0.4) -- (9, 1) node[below] {$\mu_5$};

%     % Label for k
%     \node at (10,-0.3) {$k$};
% \end{tikzpicture}}
% \end{center}

% \xtnote{maybe add a plot here for the marginal line and the critical point and optimal bidding strategy.} 
% \mznote{Let's maybe call it "almost optimal" solution and point it to the specific solution you defined in Section 3. "near-optimal" sounds like any eps approximately-optimal solution. }

\subsection{The fractional optimal bidding strategy}
{We present the optimal fractional solution, which at a high level is achieved by the following greedy process: starting with an initial budget of $B$, we allocate an infinitesimal amount to the platform currently offering the best value-to-unit-cost ratio (equivalently, the smallest marginal cost). We continue this process until either the budget is exhausted or the Return on Spend (ROS) constraint becomes tight. The bids corresponding to the final cost/value ratio on each platform form the optimal fractional strategy.

For ease reference, we formally define the feasible $\bid^k$ with the largest $k$ as the ``almost-optimal'' integral solution.
\begin{definition}[Almost-Optimal Strategy]\label{def:almostoptimal}
We say an bidding strategy $\bid$ is almost-optimal if $\bid = \max_{k}\left[\bid^k \text{ is feasible}\right]$.
\end{definition}

Essentially, the almost-optimal integral strategy provides a close lower bound for the optimal fractional solution $\optf$. Specifically, let $\opti$ represent the largest feasible $\bid^k$; then, $\opti = \floor{\optf}$. We present a subroutine that takes the almost-optimal integral solution $\opti$ as input and returns the exact fractional optimal bidding strategy $\optf$ by greedily selecting the smallest marginal costs (breaking ties lexicographically with respect to a fixed ordering of platforms) until the constraint is tight.

Consider the following equation:
\begin{align}\label{eq:roundup}
    &\sum_{j}(x_j \cdot c_j(\mu^*_j) + (1 -x_j) \cdot c_j(\mu^*_j+1)) \nonumber\\
    = &\min\left[B, T \cdot \left(\sum_{j}(x_j \cdot v_j(\mu^*_j) + (1 -x_j) \cdot v_j(\mu*_j+1))\right)\right]
\end{align}
% \begin{align}\label{eq:roundup}
%     \sum_{j}(x_j \cdot c_j(\mu^*_j) + (1 -x_j) \cdot c_j(\mu^*_j+1)) \nonumber
%     = \min\left[B, T \cdot \left(\sum_{j}(x_j \cdot v_j(\mu^*_j) + (1 -x_j) \cdot v_j(\mu*_j+1))\right)\right]
% \end{align}


By definition of $\opti$, we have the $x_j \in [0,1]$ for all platform $j$. In the case where there are multiple set of solutions, we break ties by maximizing the $x_j$ with lower platform index first.

% In addition, it is not hard to see that $\opti = \floor{\optf}$, i.e., $\mu^*_j = \floor{\mu^o_j}$ for all platform $j$.

\renewcommand*{\algorithmcfname}{SUBROUTINE}
\begin{algorithm}[ht]
\DontPrintSemicolon
\LinesNumbered
\SetNoFillComment
\KwIn{almost-optimal integral solution $\opti$}
% \textbf{Initialize: $\mu_j \gets 0$ for all $j \in \platform$}
\For{$j \in \platform$}{
    % $\ell_j \gets 0$, $r_j \gets 1$
    $\mu'_j \gets \mu^*_j+1$
    
    $\mu^o_j \gets \mu^*_j$
    
    query each $\mu'_j$ to obtain $v_j(\mu'_j), c_j(\mu'_j)$ and $\mc_j(\mu'_j)$
}
re-index the platforms in non-decreasing order of $\mc_j(\mu'_j)$ s.t. if $i \leq j$, $\mc_i(\mu'_i) \leq \mc_j(\mu'_j)$

Solve for $x_j$ in \eqref{eq:roundup}

$\mu_j^o \gets \mu_j^o + x_j$

\Return{$\bid^o$}

% \While{$\sum_{j} c_j(\mu^o_j) < \mzedit{min\{T \cdot \sum_{j \in M} v_j(\mu^o_j), B\}}$}
% {
% $j \gets \min(M)$

% $\mu_j^o \gets \mu_j^o + \epsilon$

% \If{$\mu_j^o \geq \mu'_j$}
% {
% $M \gets M \setminus \{j\}$
% }
% }
% \Return{$\bid^o$}
% \SetAlgoRefName{1}
\caption{\roundup}
\label{sub:roudup}
\end{algorithm}
% \mznote{In subroutine 1 we increases the bid by eps for each step. This may give us an eps-optimal solution instead of exact optimal. Can we write in the way that: we stop whenever $\mu_j$ is feasible while $\mu_j+1$ become not feasible for some platform $j$ (as it goes through the platform in step 7). Then compute the right randomization prob. so that the constraint is exactly met.}

\begin{lemma}[Optimal Bidding Strategy]\label{cor:opt} Let $\opti$ be the almost-optimal integral solution. Then \roundup\ ($\opti$) is bidder's fractional optimal bidding strategy.
\end{lemma}
\begin{proof}
We prove optimality of $\bid' = $ \roundup\ ($\opti$) by contradiction. Assume there exist another solution $\bid^a$ that is optimal with a value strictly higher than $\bid'$. If $\bid^a$ obtains a better value than $\bid'$, there must be at least one platform $j$ such that $\mu^a_j > \mu'_j$. Consequently, there must be some other platform $i$, such that $\mu^a_j < \mu'_j$, since the constraints are tight of solution $\bid'$ by definition of the \roundup\ algorithm and $\bid^a$ is feasible by assumption.

we first argue that $\mc_j(\mu^a_j) > \mc_i(\mu'_i)$. To see this, first note that $\mc_j(\mu^a_j) \geq \mc_i(\floor{\mu'_i})$, since otherwise by definition of $\bid^k$ we have:
\[\mu'_j = \argmax_{\mu} [\mc_j(\mu) \leq k] \geq \argmax_{\mu} [\mc_j(\mu) \leq \mc_i(\floor{\mu'_i})] \geq \ceil{\mu^a_j},\]
where the last inequality is since $\mc(\mu) = \mc(\ceil{\mu})$ for any $\mu$, this contradicting with the assumption that $\mu'_j <\mu^a_j$. By the greedy natural, with a similar contradiction argument we can show that $\mc_j(\mu^a_j) > \mc_i(\mu'_i)$, we then have:
\begin{align}\label{eq:contradiction_fractionalopt}
    \mc_j(\mu^a_j) > \mc_i(\mu'_i) > \mc_i(\mu^a_i),
\end{align}
where the last inequality is due to the assumption that $\mu'_i \geq \mu^a_i$ and the monotonicity of $\mc$ functions. 

\begin{center}
\begin{tikzpicture}
% Platform i
\draw[thick] (0,0) -- (0,3);
\node[below] at (0,0) {Platform $i$};
\node[left] at (0,1) {$\mu^a_i$};
\draw[fill] (0,1) circle (1.5pt);

\node[left] at (0,2) {$\mu'_i$};
\draw[fill] (0,2) circle (1.5pt);

% Platform j
\draw[thick] (2,0) -- (2,3);
\node[below] at (2,0) {Platform $j$};
\node[right] at (2,1.5) {$\mu'_j$};
\draw[fill] (2,1.5) circle (1.5pt);

\node[right] at (2,2.5) {$\mu^a_j$};
\draw[fill] (2,2.5) circle (1.5pt);

%Relationship arrows
\draw[-{Latex[length=2.5mm]}, thick] (0,1.1) -- (0,1.9);
\draw[-{Latex[length=2.5mm]}, thick] (2,1.6) -- (2,2.4);% 

% Label the relationships
\node at (1,0.5) {$\mu^a_i < \mu'_i$};
\node at (1,2.9) {$\mu'_j < \mu^a_j$};
\end{tikzpicture}
\end{center}

Now we argue that $\bid^a$ can be further improved by an exchange argument, contradicting with the assumption that $\bid^a$ is optimal. Consider again the bidding strategy $\bid^a$, consider reduce $\mu^a_j$ by some $\epsilon$ amount, and increase on $\mu^a_i$ by the corresponding amount until the constraints are tight again, since $\mc_j(\mu^a_j) \geq \mc_i(\mu^a_i)$, the ``bang per buck'' for the exchange portion strictly increases, contradicting with the assumption that $\bid^a$ is optimal.
\end{proof}
}

