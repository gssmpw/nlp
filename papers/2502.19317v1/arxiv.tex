\documentclass[letterpaper, 11pt]{article}
\usepackage{amsmath,amsthm,graphicx,url}
\usepackage{amssymb}
\usepackage[margin=1in]{geometry}
\usepackage{booktabs} % For formal tables
% \usepackage[ruled,noline]{algorithm2e} % For algorithms
%  \renewcommand{\algorithmcfname}{ALGORITHM}
%  \SetKwBlock{Initialize}{Initialize:}{} \SetKwFor{From}{from}{do}{}
% \SetKwFor{ForAllInParallel}{for all}{in parallel do}{}
% \SetAlFnt{\small}
% \SetAlCapFnt{\small}
% \SetAlCapNameFnt{\small}
% \SetAlCapHSkip{0pt}
%\IncMargin{-\parindent}
\usepackage{authblk}
\newcommand*\samethanks[1][\value{footnote}]{\footnotemark[#1]}
\usepackage[numbers]{natbib}
% \newtheorem{definition}[theorem]{Definition}%[section]

% The packages, macros and environments are in the settings folder
% basic
%\usepackage{color,xcolor}
\usepackage{color}
\usepackage{epsfig}
\usepackage{graphicx}
\usepackage{algorithm,algorithmic}
% \usepackage{algpseudocode}
%\usepackage{ulem}

% figure and table
\usepackage{adjustbox}
\usepackage{array}
\usepackage{booktabs}
\usepackage{colortbl}
\usepackage{float,wrapfig}
\usepackage{framed}
\usepackage{hhline}
\usepackage{multirow}
% \usepackage{subcaption} % issues a warning with CVPR/ICCV format
% \usepackage[font=small]{caption}
\usepackage[percent]{overpic}
%\usepackage{tikz} % conflict with ECCV format

% font and character
\usepackage{amsmath,amsfonts,amssymb}
% \let\proof\relax      % for ECCV llncs class
% \let\endproof\relax   % for ECCV llncs class
\usepackage{amsthm} 
\usepackage{bm}
\usepackage{nicefrac}
\usepackage{microtype}
\usepackage{contour}
\usepackage{courier}
%\usepackage{palatino}
%\usepackage{times}

% layout
\usepackage{changepage}
\usepackage{extramarks}
\usepackage{fancyhdr}
\usepackage{lastpage}
\usepackage{setspace}
\usepackage{soul}
\usepackage{xspace}
\usepackage{cuted}
\usepackage{fancybox}
\usepackage{afterpage}
%\usepackage{enumitem} % conflict with IEEE format
%\usepackage{titlesec} % conflict with ECCV format

% ref
% commenting these two out for this submission so it looks the same as RSS example
% \usepackage[breaklinks=true,colorlinks,backref=True]{hyperref}
% \hypersetup{colorlinks,linkcolor={black},citecolor={MSBlue},urlcolor={magenta}}
\usepackage{url}
\usepackage{quoting}
\usepackage{epigraph}

% misc
\usepackage{enumerate}
\usepackage{paralist,tabularx}
\usepackage{comment}
\usepackage{pdfpages}
% \usepackage[draft]{todonotes} % conflict with CVPR/ICCV/ECCV format



% \usepackage{todonotes}
% \usepackage{caption}
% \usepackage{subcaption}

\usepackage{pifont}% http://ctan.org/pkg/pifont

% extra symbols
\usepackage{MnSymbol}



%
\setlength\unitlength{1mm}
\newcommand{\twodots}{\mathinner {\ldotp \ldotp}}
% bb font symbols
\newcommand{\Rho}{\mathrm{P}}
\newcommand{\Tau}{\mathrm{T}}

\newfont{\bbb}{msbm10 scaled 700}
\newcommand{\CCC}{\mbox{\bbb C}}

\newfont{\bb}{msbm10 scaled 1100}
\newcommand{\CC}{\mbox{\bb C}}
\newcommand{\PP}{\mbox{\bb P}}
\newcommand{\RR}{\mbox{\bb R}}
\newcommand{\QQ}{\mbox{\bb Q}}
\newcommand{\ZZ}{\mbox{\bb Z}}
\newcommand{\FF}{\mbox{\bb F}}
\newcommand{\GG}{\mbox{\bb G}}
\newcommand{\EE}{\mbox{\bb E}}
\newcommand{\NN}{\mbox{\bb N}}
\newcommand{\KK}{\mbox{\bb K}}
\newcommand{\HH}{\mbox{\bb H}}
\newcommand{\SSS}{\mbox{\bb S}}
\newcommand{\UU}{\mbox{\bb U}}
\newcommand{\VV}{\mbox{\bb V}}


\newcommand{\yy}{\mathbbm{y}}
\newcommand{\xx}{\mathbbm{x}}
\newcommand{\zz}{\mathbbm{z}}
\newcommand{\sss}{\mathbbm{s}}
\newcommand{\rr}{\mathbbm{r}}
\newcommand{\pp}{\mathbbm{p}}
\newcommand{\qq}{\mathbbm{q}}
\newcommand{\ww}{\mathbbm{w}}
\newcommand{\hh}{\mathbbm{h}}
\newcommand{\vvv}{\mathbbm{v}}

% Vectors

\newcommand{\av}{{\bf a}}
\newcommand{\bv}{{\bf b}}
\newcommand{\cv}{{\bf c}}
\newcommand{\dv}{{\bf d}}
\newcommand{\ev}{{\bf e}}
\newcommand{\fv}{{\bf f}}
\newcommand{\gv}{{\bf g}}
\newcommand{\hv}{{\bf h}}
\newcommand{\iv}{{\bf i}}
\newcommand{\jv}{{\bf j}}
\newcommand{\kv}{{\bf k}}
\newcommand{\lv}{{\bf l}}
\newcommand{\mv}{{\bf m}}
\newcommand{\nv}{{\bf n}}
\newcommand{\ov}{{\bf o}}
\newcommand{\pv}{{\bf p}}
\newcommand{\qv}{{\bf q}}
\newcommand{\rv}{{\bf r}}
\newcommand{\sv}{{\bf s}}
\newcommand{\tv}{{\bf t}}
\newcommand{\uv}{{\bf u}}
\newcommand{\wv}{{\bf w}}
\newcommand{\vv}{{\bf v}}
\newcommand{\xv}{{\bf x}}
\newcommand{\yv}{{\bf y}}
\newcommand{\zv}{{\bf z}}
\newcommand{\zerov}{{\bf 0}}
\newcommand{\onev}{{\bf 1}}

% Matrices

\newcommand{\Am}{{\bf A}}
\newcommand{\Bm}{{\bf B}}
\newcommand{\Cm}{{\bf C}}
\newcommand{\Dm}{{\bf D}}
\newcommand{\Em}{{\bf E}}
\newcommand{\Fm}{{\bf F}}
\newcommand{\Gm}{{\bf G}}
\newcommand{\Hm}{{\bf H}}
\newcommand{\Id}{{\bf I}}
\newcommand{\Jm}{{\bf J}}
\newcommand{\Km}{{\bf K}}
\newcommand{\Lm}{{\bf L}}
\newcommand{\Mm}{{\bf M}}
\newcommand{\Nm}{{\bf N}}
\newcommand{\Om}{{\bf O}}
\newcommand{\Pm}{{\bf P}}
\newcommand{\Qm}{{\bf Q}}
\newcommand{\Rm}{{\bf R}}
\newcommand{\Sm}{{\bf S}}
\newcommand{\Tm}{{\bf T}}
\newcommand{\Um}{{\bf U}}
\newcommand{\Wm}{{\bf W}}
\newcommand{\Vm}{{\bf V}}
\newcommand{\Xm}{{\bf X}}
\newcommand{\Ym}{{\bf Y}}
\newcommand{\Zm}{{\bf Z}}

% Calligraphic

\newcommand{\Ac}{{\cal A}}
\newcommand{\Bc}{{\cal B}}
\newcommand{\Cc}{{\cal C}}
\newcommand{\Dc}{{\cal D}}
\newcommand{\Ec}{{\cal E}}
\newcommand{\Fc}{{\cal F}}
\newcommand{\Gc}{{\cal G}}
\newcommand{\Hc}{{\cal H}}
\newcommand{\Ic}{{\cal I}}
\newcommand{\Jc}{{\cal J}}
\newcommand{\Kc}{{\cal K}}
\newcommand{\Lc}{{\cal L}}
\newcommand{\Mc}{{\cal M}}
\newcommand{\Nc}{{\cal N}}
\newcommand{\nc}{{\cal n}}
\newcommand{\Oc}{{\cal O}}
\newcommand{\Pc}{{\cal P}}
\newcommand{\Qc}{{\cal Q}}
\newcommand{\Rc}{{\cal R}}
\newcommand{\Sc}{{\cal S}}
\newcommand{\Tc}{{\cal T}}
\newcommand{\Uc}{{\cal U}}
\newcommand{\Wc}{{\cal W}}
\newcommand{\Vc}{{\cal V}}
\newcommand{\Xc}{{\cal X}}
\newcommand{\Yc}{{\cal Y}}
\newcommand{\Zc}{{\cal Z}}

% Bold greek letters

\newcommand{\alphav}{\hbox{\boldmath$\alpha$}}
\newcommand{\betav}{\hbox{\boldmath$\beta$}}
\newcommand{\gammav}{\hbox{\boldmath$\gamma$}}
\newcommand{\deltav}{\hbox{\boldmath$\delta$}}
\newcommand{\etav}{\hbox{\boldmath$\eta$}}
\newcommand{\lambdav}{\hbox{\boldmath$\lambda$}}
\newcommand{\epsilonv}{\hbox{\boldmath$\epsilon$}}
\newcommand{\nuv}{\hbox{\boldmath$\nu$}}
\newcommand{\muv}{\hbox{\boldmath$\mu$}}
\newcommand{\zetav}{\hbox{\boldmath$\zeta$}}
\newcommand{\phiv}{\hbox{\boldmath$\phi$}}
\newcommand{\psiv}{\hbox{\boldmath$\psi$}}
\newcommand{\thetav}{\hbox{\boldmath$\theta$}}
\newcommand{\tauv}{\hbox{\boldmath$\tau$}}
\newcommand{\omegav}{\hbox{\boldmath$\omega$}}
\newcommand{\xiv}{\hbox{\boldmath$\xi$}}
\newcommand{\sigmav}{\hbox{\boldmath$\sigma$}}
\newcommand{\piv}{\hbox{\boldmath$\pi$}}
\newcommand{\rhov}{\hbox{\boldmath$\rho$}}
\newcommand{\upsilonv}{\hbox{\boldmath$\upsilon$}}

\newcommand{\Gammam}{\hbox{\boldmath$\Gamma$}}
\newcommand{\Lambdam}{\hbox{\boldmath$\Lambda$}}
\newcommand{\Deltam}{\hbox{\boldmath$\Delta$}}
\newcommand{\Sigmam}{\hbox{\boldmath$\Sigma$}}
\newcommand{\Phim}{\hbox{\boldmath$\Phi$}}
\newcommand{\Pim}{\hbox{\boldmath$\Pi$}}
\newcommand{\Psim}{\hbox{\boldmath$\Psi$}}
\newcommand{\Thetam}{\hbox{\boldmath$\Theta$}}
\newcommand{\Omegam}{\hbox{\boldmath$\Omega$}}
\newcommand{\Xim}{\hbox{\boldmath$\Xi$}}


% Sans Serif small case

\newcommand{\Gsf}{{\sf G}}

\newcommand{\asf}{{\sf a}}
\newcommand{\bsf}{{\sf b}}
\newcommand{\csf}{{\sf c}}
\newcommand{\dsf}{{\sf d}}
\newcommand{\esf}{{\sf e}}
\newcommand{\fsf}{{\sf f}}
\newcommand{\gsf}{{\sf g}}
\newcommand{\hsf}{{\sf h}}
\newcommand{\isf}{{\sf i}}
\newcommand{\jsf}{{\sf j}}
\newcommand{\ksf}{{\sf k}}
\newcommand{\lsf}{{\sf l}}
\newcommand{\msf}{{\sf m}}
\newcommand{\nsf}{{\sf n}}
\newcommand{\osf}{{\sf o}}
\newcommand{\psf}{{\sf p}}
\newcommand{\qsf}{{\sf q}}
\newcommand{\rsf}{{\sf r}}
\newcommand{\ssf}{{\sf s}}
\newcommand{\tsf}{{\sf t}}
\newcommand{\usf}{{\sf u}}
\newcommand{\wsf}{{\sf w}}
\newcommand{\vsf}{{\sf v}}
\newcommand{\xsf}{{\sf x}}
\newcommand{\ysf}{{\sf y}}
\newcommand{\zsf}{{\sf z}}


% mixed symbols

\newcommand{\sinc}{{\hbox{sinc}}}
\newcommand{\diag}{{\hbox{diag}}}
\renewcommand{\det}{{\hbox{det}}}
\newcommand{\trace}{{\hbox{tr}}}
\newcommand{\sign}{{\hbox{sign}}}
\renewcommand{\arg}{{\hbox{arg}}}
\newcommand{\var}{{\hbox{var}}}
\newcommand{\cov}{{\hbox{cov}}}
\newcommand{\Ei}{{\rm E}_{\rm i}}
\renewcommand{\Re}{{\rm Re}}
\renewcommand{\Im}{{\rm Im}}
\newcommand{\eqdef}{\stackrel{\Delta}{=}}
\newcommand{\defines}{{\,\,\stackrel{\scriptscriptstyle \bigtriangleup}{=}\,\,}}
\newcommand{\<}{\left\langle}
\renewcommand{\>}{\right\rangle}
\newcommand{\herm}{{\sf H}}
\newcommand{\trasp}{{\sf T}}
\newcommand{\transp}{{\sf T}}
\renewcommand{\vec}{{\rm vec}}
\newcommand{\Psf}{{\sf P}}
\newcommand{\SINR}{{\sf SINR}}
\newcommand{\SNR}{{\sf SNR}}
\newcommand{\MMSE}{{\sf MMSE}}
\newcommand{\REF}{{\RED [REF]}}

% Markov chain
\usepackage{stmaryrd} % for \mkv 
\newcommand{\mkv}{-\!\!\!\!\minuso\!\!\!\!-}

% Colors

\newcommand{\RED}{\color[rgb]{1.00,0.10,0.10}}
\newcommand{\BLUE}{\color[rgb]{0,0,0.90}}
\newcommand{\GREEN}{\color[rgb]{0,0.80,0.20}}

%%%%%%%%%%%%%%%%%%%%%%%%%%%%%%%%%%%%%%%%%%
\usepackage{hyperref}
\hypersetup{
    bookmarks=true,         % show bookmarks bar?
    unicode=false,          % non-Latin characters in AcrobatÕs bookmarks
    pdftoolbar=true,        % show AcrobatÕs toolbar?
    pdfmenubar=true,        % show AcrobatÕs menu?
    pdffitwindow=false,     % window fit to page when opened
    pdfstartview={FitH},    % fits the width of the page to the window
%    pdftitle={My title},    % title
%    pdfauthor={Author},     % author
%    pdfsubject={Subject},   % subject of the document
%    pdfcreator={Creator},   % creator of the document
%    pdfproducer={Producer}, % producer of the document
%    pdfkeywords={keyword1} {key2} {key3}, % list of keywords
    pdfnewwindow=true,      % links in new window
    colorlinks=true,       % false: boxed links; true: colored links
    linkcolor=red,          % color of internal links (change box color with linkbordercolor)
    citecolor=green,        % color of links to bibliography
    filecolor=blue,      % color of file links
    urlcolor=blue           % color of external links
}
%%%%%%%%%%%%%%%%%%%%%%%%%%%%%%%%%%%%%%%%%%%





\title{Multi-Platform Autobidding with and without Predictions}
\author[a]{Gagan Aggarwal}
\author[b]{Anupam Gupta}
\author[c]{Xizhi Tan \thanks{This work was done while the author was visiting Google as a student researcher.}}
\author[a]{Mingfei Zhao}

\affil[a]{Google Research: \texttt{{\{gagana, mingfei\}@google.com}}}
\affil[b]{New York University: \texttt{anupam.g@nyu.edu}}
\affil[c]{Drexel University: \texttt{xizhi@drexel.edu}}

\date{}

\begin{document}
\begin{titlepage}

\maketitle


\begin{abstract}
We study the problem of finding the optimal bidding strategy for an advertiser in a multi-platform auction setting. The competition on a platform is captured by a value and a cost function, mapping bidding strategies to value and cost respectively. We assume a diminishing returns property, whereby the marginal cost is increasing in value. The advertiser uses an autobidder that selects a bidding strategy for each platform, aiming to maximize total value subject to budget and return-on-spend constraint. The advertiser
has no prior information and learns about the value and cost functions by querying a platform with a specific bidding strategy. Our goal is to design algorithms that find the optimal bidding strategy with a small number of queries.

We first present an algorithm that requires \(O(m \log (mn) \log n)\) queries, where $m$ is the number of platforms and $n$ is 
the number of possible bidding strategies in each platform. 
Moreover, we adopt the learning-augmented framework and propose an algorithm that utilizes a (possibly erroneous) prediction of the optimal bidding strategy. We provide a $O(m \log (m\eta) \log \eta)$ query-complexity bound on our algorithm as a function of the prediction error $\eta$. This guarantee gracefully degrades to \(O(m \log (mn) \log n)\). This achieves a ``best-of-both-worlds'' scenario: \(O(m)\) queries when given a correct prediction, and \(O(m \log (mn) \log n)\) even for an arbitrary incorrect prediction.

% Lastly, we employ the classic centroid method to provide an algorithm that finds a (randomized) bidding strategy that is \(\epsilon\)-close to the optimal strategy using \(O(m^2 \log \frac{Gn}{\epsilon})\) queries, where \(G\) is an upper bound on the derivative of the value functions. \mznote{Remove this paragraph.}
\end{abstract}
\thispagestyle{empty} 
\end{titlepage}

\section{Introduction}
%\xtnote{update to add budget constraint}
%\xtnote{add discussion of Yuan's paper}
Online advertisers often advertise across multiple platforms, such as Amazon, Bing, Google, Meta and TikTok, and face the challenging task of optimizing their bids across these platforms. The complexity comes not just from having to select a vector of bidding strategies (one for each platform), but also from the diversity of auctions used across platforms and the black-box nature of the detailed auction rules and the level of competition. 

To deal with the complexity, advertisers are increasingly using automated bidding agents (aka autobidding) to bid on their behalf. An autobidder allows an advertiser to specify constraints like Budget and Return-on-Spend (ROS), and bids on their behalf to maximize value subject to the constraints. 
%In the theoretical model, the optimal bidding strategy is tractable given a complete mapping from any possible bid submitted to each platform to the corresponding auction outcome. However in practice, due to the complexity of the auctions and the dynamic market environment, such a mapping is hard to achieve beforehand.
This has led to a lot of research interest in problems related to autobidding (see ~\citet{AutobiddingSurvey24} for a recent survey). In particular, the problem of designing bidding algorithms for the single platform setting is well-studied~\cite{ABM19}, including in the online learning setting (see Section 3 of the survey \cite{AutobiddingSurvey24}). However, there is not much work on the problem of bidding optimally across multiple platforms (see the related work section for what is known).
%With the rise of automated bidding, there has been extensive research effort on constructing bidding algorithms in classic auctions for value-maximizing advertisers. 
%Yet the bidding problem when advertisers participate in the multi-platform system is not well-studied. 

In this paper, we study the problem of finding the optimal bidding strategy in the multi-platform setting. In particular, an advertiser aims to maximize her total value subject to a global budget and return-on-spend (ROS) constraint across all platforms. %To capture the lack of transparency into platforms' auctions, %difficulty in practice,
To capture the black-box nature of auction mechanisms and level of competition, we assume that the advertiser has no prior knowledge about the mapping from bids to auction outcomes for any platform. Instead, the advertiser interacts with a platform's auction by submitting a bid to the platform and observing the corresponding cost and value (i.e., the user has ``query access'' to the mapping). We propose algorithms that find the optimal bidding strategy in this setting, %with no prior knowledge of each platform’s auction, 
and prove worse-case query complexity bounds for them. 

%\xtnote{paragraph about learning-augmented framework}
While worst-case results offer robustness and broad applicability, the guarantees they provide can sometimes be overly pessimistic. To address this, a new framework called "algorithms with predictions" has recently been introduced. This framework allows algorithms to incorporate potentially flawed machine-learned predictions as a guiding tool. The objective is twofold: to achieve improved performance guarantees when the prediction is accurate (a property known as \emph{consistency}) and to maintain good worst-case bounds even when the prediction is completely incorrect (a property called \emph{robustness}). This framework provides a natural way to integrate machine-learned predictions into the design of algorithms while preserving the essential robustness offered by worst-case analysis. In this work, we adopt the learning-augmented framework and explore the role of predictions in bidding strategy optimization. Specifically, we examine the scenario where the algorithm has access to a prediction $\hat{\bid}$ of the optimal bidding strategy, without any assumption regarding the prediction's accuracy. We propose algorithms that leverage the untrusted prediction to achieve improved query complexity bounds, which degrade gracefully based on the quality of the prediction.

\subsection{Our Results}
% \xtnote{discuss the technical difficulty as well, that it is not downward-closed}

%\agnote{We have not givenexplained what $m$ and $n$ are, so the reader will be confused what the model is. Would it be useful to give a short explanation: e.g., We model the problem as follows: there are $m$ channels. For each channel $j$, we are given a cost function $c_j$ and a value function $v_j$, such that $c_j(\mu_j)$ and $v_j(\mu_j)$ are respectively the cost incurred by bidding $\mu_j$ on channel $j$, and the value accrued from this bid. Our goal is to find a collection $\bid = (\mu_1, \ldots, \mu_m)$ such that (a) the ratio of the total value accrued to the total cost is at least some target threshold, and (b) the total value is maximized.} 

We model the problem of searching for the optimal bidding strategy as follows: there are $m$ platforms. For each platform $j$, we are given a cost function $c_j$ and a value function $v_j$, and $n$ different bidding strategies indexed $1$ through $n$ such that $c_j(\mu_j)$ and $v_j(\mu_j)$ are respectively the cost incurred by bidding $\mu_j$ on platform $j$, and the value accrued from this bidding strategy. 
% (We assume that these functions are piece-wise linear, with at most $n$ breakpoints.)
%\mznote{Do we want to explicitly assume this or say that there are n different bidding strategies?} \xtnote{currently in the prelim we are saying there are $n$ different bidding strategies} 
Our goal is to find a collection $\bid = (\mu_1, \ldots, \mu_m)$ s.t. (a) the ratio of the total value accrued to the total cost is at least some target threshold, (b) the total cost is less than the budget, and (c) the total value is maximized. 
% \ganote{added the budget constraint in the previous sentence}

%We first provide a characterization of the optimal fractional bidding strategy. \ganote{This makes it sound like a new characterization. Are we claiming that? This is in the APSZ paper, and maybe in some form earlier too} 
We propose a search algorithm, \mom, that determines the exact optimal bidding strategy using \(O(m \log (mn) \log n)\) queries. Our algorithm builds on a characterization of the optimal bidding strategy in the multi-platform setting first developed in \cite{APSZ24} under continuous strategy space.
Intuitively, the optimal strategy is to keep increasing bid on the platform that currently offers the highest marginal bang-per-buck (corresponding to the lowest marginal cost-per-unit-value) until
either the budget or 
the ROS constraint is about to be violated.
%\agnote{I commented out the budget constraint comment until we add it in.}
In other words, the optimal strategy aims to equalize the marginal cost-per-unit-value across all the platforms. 
%In fact, algorithm would, consider the process of gradually increasing spend. \ganote{reword this a bit. allocating funds works for budget, not so much for ROI constraint} 
%This aligns with the findings presented by \cite{APSZ24} in the continuous strategy space. This characterization gives us a clear structure of the optimal solution w.r.t. marginal costs (Lemma~\ref{cor:opt}).

%Using this characterization, we propose a search algorithm, \mom, that determines the exact optimal solution using \(O(m \log (mn) \log n)\) queries.
At a high level, the algorithm searches in the space of marginal costs. Initially, there are up to $mn$ candidate marginal costs.  The algorithm carefully selects a candidate marginal cost and finds the corresponding vector of bidding strategies, as defined in Lemma~\ref{lem:opt}, via Subroutine~\ref{sub:matchingmc}. Based on the outcomes (i.e. cost and value on each platform) of the corresponding strategy vector, the algorithm removes a constant fraction of candidate marginals from consideration, and recurses on the residual problem. 

{We complement this algorithmic result with an $\Omega(m \log n)$ lower bound and an $\Omega(\log mn)$ lower bound for this problem. The $\Omega(\log mn)$ lower bound reflects the difficulty of identifying the optimal marginal cost among the $mn$ possible candidates, while the $\Omega(m \log n)$ bound captures the complexity of determining the corresponding bidding strategy for that optimal marginal cost. Notably, these are the two key components of our algorithm, and our upper bound is the product of these two lower bounds.}%\ganote{this is different from the upper bound. do we want to say here in what sense is this a good bound?}%gagan:this looks good.

Next, we adopt the learning-augmented framework to improve the worst-case query complexity bound.
% \ganote{what does floor of the strategy mean?} 
We propose an algorithm with access to a prediction \(\hat{\bid}\) of the
% \xtnote{floor of the} 
optimal strategy \(\optf\). The algorithm, \bmom, starts with trying to the find the optimal solution in a small range around the predictions, and expands the search range if the search is unsuccessful. With the right sequence of expanding ranges, we show that the algorithm finds the optimal strategy \(\bid^*\) with \(O(m \log (m\eta) \log \eta)\) queries, where \(\eta = \max_{j}|\bid^*_j - \hat{\bid}_j|\) represents the prediction error. 
% \ganote{prediction error is the difference between the mu's or the difference between the indices of the mus?} \xtnote{$\mu$ is the indices of the strategies}. 
This means that the algorithm requires only \(O(m)\) queries when the predictions are accurate; this is the minimum number of queries needed to implement any bidding strategy. Moreover, since \(\eta \leq n\), the total number of queries never exceeds that of the \mom\ algorithm.

%Finally, we show that one can use the classic \emph{centroid} optimization algorithm to find a fractional bidding strategy that is \(\varepsilon\)-close to the optimal strategy using \(O(m^2 \log (G\sqrt{m}n/\varepsilon))\) queries, where \(G\) is the upper bound of the derivatives of the value functions. This provides an alternative to our \mom algorithm above, and is better in some parameter-regimes.

%advertisers with a greater variety of algorithms to choose from.

\subsection{Related Work}
\paragraph{Multi-platform mechanism design and autobidding.} Previous research has examined the multi-platform auction environment from both the auctioneer's and the bidders' perspectives. Regarding auction design, \citet{aggarwal_perlroth_zhao_ec23} analyzes a scenario where a single platform manages multiple channels, each selling queries via a second-price auction (SPA) with a reserve price. The authors assess the costs associated with each channel optimizing its own reserve price compared to a unified platform policy. Inspired by the Display Ad market, \citet{renato_balu_yifeng_www2020} explores a model in which multiple platforms vie for profit-maximizing bidders who must use the same bid across all platforms (which we refer to as a uniform bid). Their key finding indicates that the first-price auction (FPA) serves as the optimal auction format for these platforms. On the bidders' side, \citet{susan2023multi} investigates bidding strategies for utility-maximizing advertisers operating across multiple platforms while adhering to budget constraints. Meanwhile, \citet{deng2023multi} focuses on value-maximizing advertisers and reveals that optimizing ROS per platform can yield arbitrarily poor results when both ROS and budget constraints are in play. 

\citet{APSZ24} study a similar multi-platform setting with autobidders under ROI constraints but focus on the auction design problem from the platform's perspective. While first-price auctions are optimal in the absence of competition (\citet{deng2021towards}), they show that from the perspective of each separate platform, running a second-price auction can achieve larger revenue than first-price auction when there are two competing platforms. They also identify key factors influencing the platform’s choice of auction formats, including advertiser sensitivity to auction changes, competition and relative inefficiency of second-price auctions. In our paper, we focus on how advertisers can bid optimally in the multi-platform setting.

% \paragraph{Multi-dimensional Binary Search} 

\paragraph{Algorithms with predictions.} In recent years, the learning-aug\-mented framework has emerged as a prominent paradigm for the design and analysis of algorithms. For an overview of early contributions, we refer to \citep{MV22}, while \citep{alps} offers an up-to-date compilation of relevant papers in this area. This framework seeks to address the shortcomings of overly pessimistic worst-case analyses. In the last five years alone, hundreds of papers have explored traditional algorithmic challenges through this lens, with notable examples including online paging \citep{lykouris2018competitive}, scheduling \citep{PSK18}, optimization problems related to covering \citep{BMS20} and knapsack constraints \citep{IKQP21}, as well as topics like Nash social welfare maximization \citep{banerjee2020online}, the secretary problem \citep{AGKK23, DLLV21, KY23}, and various graph-related problems \citep{azar2022online}.

More closely related to our work, the research on learning-aug\-mented mechanisms interacting with strategic agents is recently initiated by \citet{ABGOT22} and \citet{XL22}. This area includes strategic facility location \citep{ABGOT22, XL22, IB22, BGT24}, strategic scheduling \citep{XL22, BGT223}, auctions \citep{MV17, XL22, LuWanZhang23, caragiannis2024randomized, BGTZ23}, bicriteria mechanism design (which seeks to optimize both social welfare and revenue) \citep{BPS23}, graph problems with private input \citep{CKST24}, metric distortion \citep{BFGT23}, and equilibrium analysis \citep{GKST22, IBB24}. Recently, \citet{CSV24} revisited mechanism design challenges by focusing on predictions about the outcome space rather than the input. While most of these studies concentrate on the mechanism design problem, our research emphasizes how predictions can assist agents in identifying optimal strategies.  For more information on this body of work, we recommend \citep{BGT23}.

\section{Preliminaries}
% \mznote{1. bidding strategies or bids? 2. Should we use $b_1,...,b_n$ as the bids instead of 1,...,n?}
% \xtnote{add budget constraint} 
We consider the problem of finding the optimal bidding strategy in a multi-platform auction setting for a value-maximizer with \emph{budget} and \emph{return on spend} (ROS) constraints. There is a set $\platform$ consisting of $m$ platforms in the market.
We assume that for each platform, the advertiser can pick from $n$ different bids (note that this set can be different for different platforms), indexed by $0,1,2,\dots,n$, where bid $0$ is used to denote non-participation.
% Let $\strategy = \{0,1, 2, \dots, n\}$ \ganote{should this start from 1?} \xtnote{we are including $0$ so that the algorithm can always choose to not spend on some platform, so there also is a feasible solution.}
% \xtnote{for presentation purpose add why it is 0 to n} be the set of possible bidding strategies in each platform. 
Each platform $j \in \platform$ is described by a value and a cost function that map {bid indices} to a corresponding value and cost, respectively, i.e., $v_j:\strategy \rightarrow \R^{\geq 0}$ and $c_j: \strategy \rightarrow \R^{\geq 0}$. In other words, when a bidder chooses to bid according to bid $\mu \in \strategy$, they incur a cost of $c_j(\mu)$ and receive a value of $v_j(\mu)$.
We assume that there is a strict ordering of costs and values by bid index, i.e. $v_j(\mu) < v_j(\mu+1)$ and $c_j(\mu) < c_j(\mu+1)$ for all $\mu \in \strategy$. 
% \footnote{One can think of the set $\strategy$ as the set of multipliers to apply to the bid. \ganote{this is confusing, since 1...n may not make as much sense as multipliers}}
We refer to the mapping from bid to the cost and value of each platform as the \emph{landscape} of that platform. 
In addition, we define the \emph{marginal cost} of bidding $\mu\geq 1$ on platform $j$ as
% \begin{align}\label{eq:marginalcost}
%     \mc_j(\mu) = \lim_{\Delta \rightarrow 0}\frac{c_j(\mu+\Delta)-c_j(\mu)}{v_j(\mu+\Delta)-v_j(\mu)}. \tag{Marginal Cost}
% \end{align}
\begin{align}\label{eq:marginalcost}
    \mc_j(\mu) = \frac{c_j(\mu)-c_j(\mu-1)}{v_j(\mu)-v_j(\mu-1)}, \tag{Marginal Cost}
\end{align}
where $c(0)=0$ and $v(0)=0$.
We make standard convexity assumption that $\mc_j$ is non-decreasing for every platform $j$. 
% \xtnote{Any other assumptions we need for the functions?}

{Given the integral strategy set, we expand the bidding space by also considering the fractional solution between each integral bid, hence making the strategy space continuous. We use $S$ and $S^c$ to denote the integral and fractional strategy space, respectively. The cost, value and marginal functions of the continuous bidding space $[0,n]^m$ extend the discrete function by linear interpolation.\footnote{It can be viewed as bidding randomly between two adjacent bids.} Formally
\[v_j(\mu) = (\ceil{\mu} - \mu) \cdot v_j(\floor{\mu}) + (\mu -\floor{\mu})  \cdot v_j(\ceil{\mu}),\]
\[c_j(\mu) = (\ceil{\mu} - \mu) \cdot c_j(\floor{\mu}) +(\mu -\floor{\mu})  \cdot c_j(\ceil{\mu}),\]
%i.e., the cost and value of some non-integer strategy is the linear combination of its floor and ceiling strategies. 
Consequently, we have that $\mc_j(\mu) = \mc_j(\ceil{\mu})$.}

% \xtnote{We first define the fractional (randmized) version of our problem. Instead of having discrete bidding strategy $S = \{0,1,\dots, n\}$, we will have a continuous strategy space $S = [0,n]$, where the cost, value and marginal cost functions are defined as follows:
% \[v_j(\mu) = (\mu -\floor{\mu}) \cdot v_j(\floor{\mu}) + (\ceil{\mu} - \mu) \cdot v_j(\ceil{\mu}),\]
% \[c_j(\mu) = (\mu -\floor{\mu}) \cdot c_j(\floor{\mu}) + (\ceil{\mu} - \mu) \cdot c_j(\ceil{\mu}),\]
% i.e., the cost and value of some non-integer strategy is the linear combination of its floor and ceiling strategies. Consequently, we have that
% \[\mc_j(\mu) = \mc_j(\ceil{\mu}).\]}
% \mznote{Add note that finding the optimal integral solution is NP-hard.}
{We note that the problem of finding the optimal integral solution is NP-hard.\footnote{It is not hard to see that we can encode any knapsack problem as an instance of our problem with a budget constraint.}}
The objective of the bidder is therefore to find an optimal \emph{fractional}  bidding strategy $\bid = (\mu_1, \mu_2, \dots, \mu_m)$ {where $\mu_j \in [0,n]$} such that she maximizes the total value received by executing bidding strategy $\mu_j$ on each platform $j$, subject to {a budget constraint} and the ROS constraint \emph{across all platforms}.
Let {$B$ and} $T$ be the budget and target ROS of the bidder. We can formulate the problem as the following program:
{\begin{align}\label{eq:bidderproblem}
    \max_{\bid=(\mu_1, \mu_2 \dots, \mu_m)} &\sum_{j \in \platform} v_j(\mu_j) \nonumber\\
    s.t. & \sum_{j \in \platform} c_j(\mu_j) \leq T \cdot \sum_{j \in \platform} v_j(\mu_j),\\
    & \sum_{j \in \platform} c_j(\mu_j) \leq B. \nonumber 
\end{align}}
{throughout the paper, we denote $\optf$ the optimal (fractional) bidding strategy, and $\opti$ the floor of it, i.e., $\mu_j^* = \floor{\mu^o_j}$ for all platform $j$. Note that $\mu_j^* \in S$.}

%If the landscape ($v_j$ an $c_j$) of each platform is known, the bidder can compute the optimal solution. However, in practice, the bidder usually has no information prior to interacting with the platforms. Instead, she needs to \emph{query} each platform by playing a strategy to learn the corresponding values and costs.
We assume that the bidder only knows the set of possible bidding strategies, but has no information about the platforms' value and cost functions. Instead, the bidder can interact with platforms via \emph{bidding queries}: the bidder plays strategy $\mu$ on a platform $j$ to learn the value $v_j(\mu)$, the cost $c_j(\mu)$, and the marginal cost $\mc_j(\mu)$\footnote{When marginal cost is not part of the query output, it is still achievable by querying both the current and the previous bid, which increases the query complexity by a constant factor.}.
% \xtnote{define worst-case query complexity of an algorithm? include randomized one as well?}
Each such query is costly to the bidder, and the goal is to minimize the number of queries required to determine the optimal strategy. 

Given an instance $\mathcal{I}$ and
% \mznote{Modify the notation using $\mathcal{I}$ as instance.} 
an algorithm $\alg$, let $\alg(\mathcal{I})$ denote the number of queries needed to find the optimal strategy for that instance. Then the query complexity of the algorithm is defined as:
\[\max_{\mathcal{I}} \alg(\mathcal{I})\]

\paragraph{The learning-augmented framework} In this work, we adopt the learning-augmented framework and study how we can further reduce the query complexity by considering search algorithms that are equipped with a (potentially erroneous) prediction $\pred \in [0,n]^m$ of the  optimal fractional bidding strategy $\optf(\mathcal{I}) = (\mu^o_1, \mu^o_2,\dots, \mu^o_n)$. The \emph{error} of an predictions $\eta$ is defined to be the maximum point-wise deviation from $\optf$, formally:
\[\eta(\pred,\mathcal{I}) = \max_{j}|\hat{\mu}_j - \mu^o_j(\mathcal{I})|\]
We let the algorithm $\alg$ use both the instance $\mathcal{I}$ and the prediction $\pred$ as input. We evaluate the performance of such an algorithm using its \emph{robustness}, \emph{consistency} and the query complexity as a function of the prediction error. 

The robustness of an algorithm refers to the worst-case query complexity of the algorithm
given an adversarially chosen, possibly erroneous, prediction. Mathematically,
\[\text{robustness}(\alg) = \max_{\pred, \mathcal{I}} \alg(\pred, \mathcal{I})\]
The consistency of an algorithm refers to the worst-case query complexity
when the prediction that it is provided with is accurate, i.e., $\pred = \opti(\mathcal{I})$. Mathematically,
\[\text{consistency}(\alg) = \max_{\pred, \mathcal{I}: \pred = \opti(\mathcal{I})} \alg(\pred, \mathcal{I}).\]
Lastly, the query complexity of an algorithm given a prediction with error $\eta'$ is defined to be:
\[\max_{\pred, \mathcal{I}: \eta(\pred,\mathcal{I}) \leq \eta'} \alg(\pred, \mathcal{I}).\]

\section{Characterization of Bidder's Optimal Bidding Strategy}
In this section, we present a characterization of the optimal bidding strategy $\optf$ that will be useful in designing the algorithm. To this end, we first prove a useful lemma about the ``ranking'' of integral strategies in $S$. We then argue how an ``almost-optimal'' integral solution can be used to determine the optimal fractional solution.
% \begin{definition} Given some positive number $a$, define $\bid^a = (\mu^a_1, \mu^a_2, \dots \mu^a_m)$ where
% \[\mu^a _j = \argmax_{\mu}\{\mc_j(\mu)\leq a\}.\]
% \end{definition}
\begin{lemma}\label{lem:opt}
Given some positive number $k$, and the $n$ discrete indices on each platform,
% \ganote{should this be S or {bid indices}? we are assuming here that the set S is the same for each platform, so perhaps bid indices is better. Or maybe we can say somewhere that we will refer to the bid indices as the set of strategies?},
define $\bid^k = (\mu^k_1, \mu^k_2, \dots \mu^k_m)$ where
\[\mu^k_j = \argmax_{\mu \in \strategy}\{\mc_j(\mu)\leq k\},\]
then there exist a $k^*$ such that for any $k \leq k^*$, $\bid^k$ is a feasible solution for program\eqref{eq:bidderproblem} and for any $k' > k^*$, $\bid^{k'}$ is not feasible.
\end{lemma}
We first show the following helper lemma. Intuitively, if we consider the landscape of each platform, and connect each bidding strategy with a straight line, the landscape would be convex, the lemma below is simply a property of a convex function. 
% Due to space limitations, we defer the proof to appendix~\ref{app:convexlemma}.
\begin{lemma}\label{lem:helper}
For any platform $j \in \platform$,
% , the ratio $\frac{c_j(\mu)}{v_j(\mu)}$ is weakly increasing, in addition,
$\frac{c_j(\mu)}{v_j(\mu)} \leq \mc_j(\mu)$.
\end{lemma}

\begin{proof}
For presentation purpose, we drop the subscript $j$ in this prove as it should hold for any platform. We define $c(0)/ v(0) = 0$ We prove the statement via induction. First consider the base case for $\mu = 1$, we have $c(1)/v(1) \geq 0 = c(0)/ v(0)$ since both the cost and the value functions weakly increase w.r.t $\mu$, we also have $c(1)/v(1) = \frac{c(1)- c(0)}{v(1) - v(0)} = \mc(1)$ by definition. The base case is therefore established.

Let $\frac{c(\mu)}{v(\mu)} = X_{\mu}$. Assume, for induction, that $X_{\mu'} \leq \mc(\mu')$ for any $\mu' < \mu$. We first show $X_{\mu-1} \leq X_{\mu}$ holds for $\mu\geq 2$. Consider
{\allowdisplaybreaks
\begin{align*}
     c(\mu) & = X_{\mu}\cdot v(\mu)\\
     c(\mu) - c(\mu-1) & =  X_{\mu}\cdot v(\mu) - c(\mu-1)\\
     \mc(\mu) \cdot (v(\mu) - v(\mu-1)) & = X_{\mu}\cdot v(\mu) - c(\mu-1)\\
     \mc(\mu-1) \cdot (v(\mu) - v(\mu-1)) & = X_{\mu}\cdot v(\mu) - c(\mu-1)\\
     X_{\mu-1}\cdot (v(\mu) - v(\mu-1)) &\leq X_{\mu}\cdot v(\mu) - c(\mu-1)\\
     X_{\mu-1}\cdot (v(\mu) - v(\mu-1)) &\leq X_{\mu}\cdot v(\mu) - X_{\mu-1} \cdot v(\mu-1)\\
     X_{\mu-1} \cdot v(\mu) & \leq X_{\mu}\cdot v(\mu)\\
     X_{\mu-1} &\leq X_{\mu},
\end{align*}}
where the third equality is by definition of $\mc$, the forth equality is by monotoncity of $\mc$, and the first inequality is by induction hypothesis $\mc(\mu-1)\geq X_{\mu-1}$.
In addition, consider the same set of equation again:
{\allowdisplaybreaks\begin{align*}
     c(\mu) & = X_{\mu}\cdot v(\mu)\\
     c(\mu) - c(\mu-1) & =  X_{\mu}\cdot v(\mu) - c(\mu-1)\\
    c(\mu) - c(\mu-1) & = X_{\mu}\cdot v(\mu) - X_{\mu-1} \cdot v(\mu-1)\\
    c(\mu) - c(\mu-1) & \geq X_{\mu}\cdot v(\mu) - X_{\mu} \cdot v(\mu-1)\\
    \mc(\mu) \cdot (v(\mu) - v(\mu-1))  & \geq X_\mu \cdot (v(\mu)-v(\mu-1))\\
    \mc(\mu) & \geq X_\mu
\end{align*}}
where as the first inequality is due to $X_{\mu-1} \leq X_{\mu}$, we therefore have $\mc(\mu) \geq X_\mu$, hence proved.
\end{proof}

\begin{proof}[Proof of Lemma~\ref{lem:opt}]
let $k$ be the smallest $k$ with infeasible $\bid^k$, if the infeasibility is due to the budget constraint, then for any $k' \geq k$ we trivially have that $\bid^{k'}$ violates the budget constraint as well since $\mu_j^{k'} \geq \mu_j^{k}$ and the cost functions are monotone.

If the infeasibility is due to the ROS constraint, i.e.,
\begin{align}\label{eq:smallestviolation}
   \sum_{j \in M}c_j(\mu_j^{k}) > T \cdot \sum_{j \in M}v_j(\mu_j^{k}),
\end{align}
proving the statement is equivalent to proving for any $k' \geq k$, $\bid^{k'}$ is also infeasible. To this end, we first show that the maximum marginals among the $\mu_j^{k}$ is strictly more than $T$, assume for contradiction, that $\mc_j(\mu_j^{k}) \leq T$ for all $j$, by lemma~\ref{lem:helper} we  would have the $c_j(\mu_j^{k})/v_j(\mu_j^{k}) \leq \mc_j(\mu_j^{k}) \leq T$, which contradicts with \eqref{eq:smallestviolation}. We therefore have
\begin{align}\label{eq:lowerbound}
    \max_{j \in \platform} \mc_j(\mu_j^{k}) > T
\end{align}
We now inductively prove that for any $k' \geq k$, we have $\bid^{k'}$ is infeasible. Consider increasing $k'$ starting from $k$, at the beginning we could have $\bid^{k'} = \bid^k$ (which is infeasible), consider the first point $k' \geq k$ such that $\bid^{k'} \neq \bid^k$, we know that:
\begin{enumerate}
    \item there exist at least one platform $j'$ such that $\mu_{j'}^{k'} = \mu_{j'}^k+1$
    \item $\mc_{j'}(\mu^{k'}_{j'}) > \max_{j \in \platform} \mc_j(\mu_j^{k}) > T,$
 \end{enumerate}
%begin{align*}
%     \text{there exist at least one platform $j'$ such that } \mu_{j'}^{k'} = \mu_{j'}^k+1\\
%     \mc_{j'}(\mu^{k'}_{j'}) > \max_{j \in \platform} \mc_j(\mu_j^{k}) > T,
% \end{align*}
where the first inequality is by definition of $\bid^{k'}$ and the second inequality is due to \eqref{eq:lowerbound}. Now consider:
{\allowdisplaybreaks\begin{align*}
    \sum_{j \in \platform}c_j(\mu_j^{k'}) &= \sum_{j \in M}c_j(\mu_j^{k}) + c_{j'}(\mu_{j'}^k+1) - c_{j'}(\mu_{j'}^k)\\
    & > T \cdot \sum_{j \in M}v_j(\mu_j^{k})+ c_{j'}(\mu_{j'}^k+1) - c_{j'}(\mu_{j'}^k)\\
    & > T \cdot \sum_{j \in M}v_j(\mu_j^{k})+ \mc_{j'}(\mu^k_{j'}+1)\cdot (v_{j'}(\mu_{j'}^k+1) - v_{j'}(\mu_{j'}^k)\\
    &> T \cdot \sum_{j \in M}v_j(\mu_j^{k})+ T \cdot (v_{j'}(\mu_{j'}^k+1) - v_{j'}(\mu_{j'}^k)\\
    &> T \cdot \sum_{j \in M}v_j(\mu_j^{k'})
\end{align*}}
Inductively apply this argument for each update of $\bid^{k'}$ proves the statement.
\end{proof}

% \begin{center}  % Center the figure within the column
% \resizebox{0.45\textwidth}{!}{
% \begin{tikzpicture}
%     % Draw the main line representing k
%     \draw[thick] (0,0) -- (10,0);

%     % Mark the strategy points (mu) along the k line
%     \foreach \x in {1, 3, 5, 7, 9} {
%         \draw[thick] (\x,0) -- (\x,0.4);
%     }

%     % Draw arrows pointing to the strategy points (mu values)
%     \draw[-] (1, 0.4) -- (1, 1) node[below] {$\mu_1$};
%     \draw[-] (3, 0.4) -- (3, 1) node[below] {$\mu_2$};
%     \draw[-] (5, 0.4) -- (5, 1) node[below] {$\mu_3$};
%     \draw[-] (7, 0.4) -- (7, 1) node[below] {$\mu_4$};
%     \draw[-] (9, 0.4) -- (9, 1) node[below] {$\mu_5$};

%     % Label for k
%     \node at (10,-0.3) {$k$};
% \end{tikzpicture}}
% \end{center}

% \xtnote{maybe add a plot here for the marginal line and the critical point and optimal bidding strategy.} 
% \mznote{Let's maybe call it "almost optimal" solution and point it to the specific solution you defined in Section 3. "near-optimal" sounds like any eps approximately-optimal solution. }

\subsection{The fractional optimal bidding strategy}
{We present the optimal fractional solution, which at a high level is achieved by the following greedy process: starting with an initial budget of $B$, we allocate an infinitesimal amount to the platform currently offering the best value-to-unit-cost ratio (equivalently, the smallest marginal cost). We continue this process until either the budget is exhausted or the Return on Spend (ROS) constraint becomes tight. The bids corresponding to the final cost/value ratio on each platform form the optimal fractional strategy.

For ease reference, we formally define the feasible $\bid^k$ with the largest $k$ as the ``almost-optimal'' integral solution.
\begin{definition}[Almost-Optimal Strategy]\label{def:almostoptimal}
We say an bidding strategy $\bid$ is almost-optimal if $\bid = \max_{k}\left[\bid^k \text{ is feasible}\right]$.
\end{definition}

Essentially, the almost-optimal integral strategy provides a close lower bound for the optimal fractional solution $\optf$. Specifically, let $\opti$ represent the largest feasible $\bid^k$; then, $\opti = \floor{\optf}$. We present a subroutine that takes the almost-optimal integral solution $\opti$ as input and returns the exact fractional optimal bidding strategy $\optf$ by greedily selecting the smallest marginal costs (breaking ties lexicographically with respect to a fixed ordering of platforms) until the constraint is tight.

Consider the following equation:
\begin{align}\label{eq:roundup}
    &\sum_{j}(x_j \cdot c_j(\mu^*_j) + (1 -x_j) \cdot c_j(\mu^*_j+1)) \nonumber\\
    = &\min\left[B, T \cdot \left(\sum_{j}(x_j \cdot v_j(\mu^*_j) + (1 -x_j) \cdot v_j(\mu*_j+1))\right)\right]
\end{align}
% \begin{align}\label{eq:roundup}
%     \sum_{j}(x_j \cdot c_j(\mu^*_j) + (1 -x_j) \cdot c_j(\mu^*_j+1)) \nonumber
%     = \min\left[B, T \cdot \left(\sum_{j}(x_j \cdot v_j(\mu^*_j) + (1 -x_j) \cdot v_j(\mu*_j+1))\right)\right]
% \end{align}


By definition of $\opti$, we have the $x_j \in [0,1]$ for all platform $j$. In the case where there are multiple set of solutions, we break ties by maximizing the $x_j$ with lower platform index first.

% In addition, it is not hard to see that $\opti = \floor{\optf}$, i.e., $\mu^*_j = \floor{\mu^o_j}$ for all platform $j$.

\renewcommand*{\algorithmcfname}{SUBROUTINE}
\begin{algorithm}[ht]
\DontPrintSemicolon
\LinesNumbered
\SetNoFillComment
\KwIn{almost-optimal integral solution $\opti$}
% \textbf{Initialize: $\mu_j \gets 0$ for all $j \in \platform$}
\For{$j \in \platform$}{
    % $\ell_j \gets 0$, $r_j \gets 1$
    $\mu'_j \gets \mu^*_j+1$
    
    $\mu^o_j \gets \mu^*_j$
    
    query each $\mu'_j$ to obtain $v_j(\mu'_j), c_j(\mu'_j)$ and $\mc_j(\mu'_j)$
}
re-index the platforms in non-decreasing order of $\mc_j(\mu'_j)$ s.t. if $i \leq j$, $\mc_i(\mu'_i) \leq \mc_j(\mu'_j)$

Solve for $x_j$ in \eqref{eq:roundup}

$\mu_j^o \gets \mu_j^o + x_j$

\Return{$\bid^o$}

% \While{$\sum_{j} c_j(\mu^o_j) < \mzedit{min\{T \cdot \sum_{j \in M} v_j(\mu^o_j), B\}}$}
% {
% $j \gets \min(M)$

% $\mu_j^o \gets \mu_j^o + \epsilon$

% \If{$\mu_j^o \geq \mu'_j$}
% {
% $M \gets M \setminus \{j\}$
% }
% }
% \Return{$\bid^o$}
% \SetAlgoRefName{1}
\caption{\roundup}
\label{sub:roudup}
\end{algorithm}
% \mznote{In subroutine 1 we increases the bid by eps for each step. This may give us an eps-optimal solution instead of exact optimal. Can we write in the way that: we stop whenever $\mu_j$ is feasible while $\mu_j+1$ become not feasible for some platform $j$ (as it goes through the platform in step 7). Then compute the right randomization prob. so that the constraint is exactly met.}

\begin{lemma}[Optimal Bidding Strategy]\label{cor:opt} Let $\opti$ be the almost-optimal integral solution. Then \roundup\ ($\opti$) is bidder's fractional optimal bidding strategy.
\end{lemma}
\begin{proof}
We prove optimality of $\bid' = $ \roundup\ ($\opti$) by contradiction. Assume there exist another solution $\bid^a$ that is optimal with a value strictly higher than $\bid'$. If $\bid^a$ obtains a better value than $\bid'$, there must be at least one platform $j$ such that $\mu^a_j > \mu'_j$. Consequently, there must be some other platform $i$, such that $\mu^a_j < \mu'_j$, since the constraints are tight of solution $\bid'$ by definition of the \roundup\ algorithm and $\bid^a$ is feasible by assumption.

we first argue that $\mc_j(\mu^a_j) > \mc_i(\mu'_i)$. To see this, first note that $\mc_j(\mu^a_j) \geq \mc_i(\floor{\mu'_i})$, since otherwise by definition of $\bid^k$ we have:
\[\mu'_j = \argmax_{\mu} [\mc_j(\mu) \leq k] \geq \argmax_{\mu} [\mc_j(\mu) \leq \mc_i(\floor{\mu'_i})] \geq \ceil{\mu^a_j},\]
where the last inequality is since $\mc(\mu) = \mc(\ceil{\mu})$ for any $\mu$, this contradicting with the assumption that $\mu'_j <\mu^a_j$. By the greedy natural, with a similar contradiction argument we can show that $\mc_j(\mu^a_j) > \mc_i(\mu'_i)$, we then have:
\begin{align}\label{eq:contradiction_fractionalopt}
    \mc_j(\mu^a_j) > \mc_i(\mu'_i) > \mc_i(\mu^a_i),
\end{align}
where the last inequality is due to the assumption that $\mu'_i \geq \mu^a_i$ and the monotonicity of $\mc$ functions. 

\begin{center}
\begin{tikzpicture}
% Platform i
\draw[thick] (0,0) -- (0,3);
\node[below] at (0,0) {Platform $i$};
\node[left] at (0,1) {$\mu^a_i$};
\draw[fill] (0,1) circle (1.5pt);

\node[left] at (0,2) {$\mu'_i$};
\draw[fill] (0,2) circle (1.5pt);

% Platform j
\draw[thick] (2,0) -- (2,3);
\node[below] at (2,0) {Platform $j$};
\node[right] at (2,1.5) {$\mu'_j$};
\draw[fill] (2,1.5) circle (1.5pt);

\node[right] at (2,2.5) {$\mu^a_j$};
\draw[fill] (2,2.5) circle (1.5pt);

%Relationship arrows
\draw[-{Latex[length=2.5mm]}, thick] (0,1.1) -- (0,1.9);
\draw[-{Latex[length=2.5mm]}, thick] (2,1.6) -- (2,2.4);% 

% Label the relationships
\node at (1,0.5) {$\mu^a_i < \mu'_i$};
\node at (1,2.9) {$\mu'_j < \mu^a_j$};
\end{tikzpicture}
\end{center}

Now we argue that $\bid^a$ can be further improved by an exchange argument, contradicting with the assumption that $\bid^a$ is optimal. Consider again the bidding strategy $\bid^a$, consider reduce $\mu^a_j$ by some $\epsilon$ amount, and increase on $\mu^a_i$ by the corresponding amount until the constraints are tight again, since $\mc_j(\mu^a_j) \geq \mc_i(\mu^a_i)$, the ``bang per buck'' for the exchange portion strictly increases, contradicting with the assumption that $\bid^a$ is optimal.
\end{proof}
}



\section{The Median of the Medians Algorithm}\label{sec:mom}
In this section, with the help of the characterization in the previous section, we present an algorithm 
%inspired by the median of medians algorithm 
with a worst-case query complexity of $O(m \log (mn) \log n)$. {Note that if our feasible region were downward-closed, there would be a straightforward algorithm to solve the problem: We could perform a high-dimensional binary search in the bidding space, cutting down the whole strategy space by a constant fraction each time we query a particular strategy. This would lead to \(m \cdot \log(n^m) = m^2 \log n\) queries (since querying one vector of strategies %\(\mathbf{bid}\) 
requires submitting a bidding strategy on each of the \(m\) platforms). Unfortunately, in the example below we show that our feasible region is not necessarily downward-closed. 

\begin{example}
Consider a simple example with two platforms, 1 and 2, both having the following cost and value functions:
\[
c_1(\mu) = \mu \quad v_1(\mu) = \frac{8}{3} \mu;
\]
\[
c_2(\mu) = \mu \quad v_2(\mu) = \mu.
\]
Suppose the constraints are \(B = 10\) and \(T = \frac{1}{2}\). First, observe that \(\mu_1 = \frac{3}{2}\) and \(\mu_2 = 1\) is a feasible solution since:
\[
2 \cdot \left( \frac{3}{2} + 1 \right) = \frac{8}{3} \cdot \frac{3}{2} + 1.
\]
However, if we reduce \(\mu_1\) to 1, we get:
\[
2 \cdot (1 + 1) > \frac{8}{3} + 1,
\]
indicating that the updated bidding strategy is no longer feasible. Therefore, the feasible region is not downward-closed.
\end{example}

Without a downward-closed feasible region, it is unclear which bidding strategy to try or how the search algorithm should proceed to minimize the number of attempts. To address this, we leverage the structure of integral bidding strategies shown in previous section and focus on identifying the \(k\) values corresponding to the maximum feasible \(\bid^k\) first. The potential \(k\) values are the set of marginal costs across all platforms (there are \(mn\) of them). We utilize the ``median of medians'' idea to ensure that we eliminate a constant fraction of marginal cost options with each round of probing.} 

First, given the characterization of the optimal solution, we provide two subroutines that are useful for our algorithm. We first provide a subroutine named \matchingmc, that given a $k$, finds the $\bid^k$ vector via binary search on each platform. We show that the query complexity of this subroutine is $O(m \log n)$. Whenever we call the subroutine, we would make sure that the $\mc_j(\ell_j) \leq k$, i.e., there is at least one strategy in the search range that is feasible. 
% Due to space limitations, we defer the following two proofs to appendix~\ref{app:matchingmc}
\renewcommand*{\algorithmcfname}{SUBROUTINE}
\begin{algorithm}[h]
\DontPrintSemicolon
\LinesNumbered
\SetNoFillComment
\KwIn{search range $[\ell_j, r_j]$ of each $j$, the target MC $k$}
% \textbf{Initialize: $\mu_j \gets 0$ for all $j \in \platform$}
\For{$j \in \platform$}{
    % $\ell_j \gets 0$, $r_j \gets 1$
    
    \While{$\mu^k_j =$ NULL}{
        $\mu_j \gets \frac{\ell_j + r_j}{2}$ for all $j \in \platform$ \tcp*{Binary search on each platform}
        
        \lIf{$\mc_j(\mu_j) \leq k$ }{
        $\ell_j \gets \mu_j$
        }
        
        \lIf{$\mc_j(\mu_j) > k$}{
        $r_j \gets \mu_j - 1$
        }
        \lIf{$r_j \leq \ell_j$}{
        $\mu_j^k \gets \ell_j$
        }
    }
}
\Return{$\bid^k = (\mu^k_1,\mu^k_2,\dots, \mu^k_m)$}
% \SetAlgoRefName{1}
\caption{\matchingmc}
\label{sub:matchingmc}
\end{algorithm}

\begin{lemma}\label{lem:matchingmc}
Given some $k \geq 0$, \matchingmc\ outputs the corresponding $\bid^k$ with at most 

$O(m\log \max_j(r_j - \ell_j))$ queries. 
% \xtnote{assuming $\mc_j(\ell_j) \leq k$ (not sure yet if we want to change the algorithm or keep the assumption in the lemma)}
\end{lemma}
\begin{proof}
The algorithm performs binary search on each platform to find the maximum $\mu$ such that $\mc_j(\mu) \leq k$, since binary search checks $O(\log (r_j - \ell_j))$ number of choice of $\mu$ on each platform and there are $m$ platforms, the query complexity is $O(m \log \max_j(r_j - \ell_j))$.

We now prove the correctness of the algorithm via case analysis, i.e., for each platform $j$, we have $\mu^k_j = \max_{\mu}(\mc_j(\mu) \leq k)$. Fix any arbitrary platform $j$, if $\mc_j(r_j) \leq k$, the algorithm will keep update $\ell_j$ until eventually $\ell_j = r_j$ and correctly set $\mu^k_j = \ell_j = r_j$. On the other hand, if $\mc_j(r_j) > k$, by the termination condition, we know that $\mc_j(\mu^k_j +1) > k$, $\mc_j(\mu^k_j)\leq k$, which corresponds to $\mu^k_j$ being the maximum bid with a marginal cost weakly less than $k$.
\end{proof}

We now provide a subroutine that check if a given integral bidding profile $\bid$ is the almost-optimal solution (defined in Definition~\ref{def:almostoptimal}) or not.
% \mznote{In particular this sentence sounds like we are checking if the solution is approximately optimal. In fact we want to verify if it's the specific solution defined in Section 3.} 
In addition, the subroutine can also check if a bidding profile is $\bid^k$ for some $k$ (defined in Lemma~\ref{lem:opt}). The worst-case query complexity is $O(m)$.

\begin{algorithm}[ht]
\DontPrintSemicolon
\LinesNumbered
\SetNoFillComment
\KwIn{some bidding strategy $\bid$}
% \textbf{Initialize: $\mu_j \gets 0$ for all $j \in \platform$}
query each platform $j$ strategy $\mu_j$, obtain $v_j(\mu_j)$, $c_j(\mu_j)$ and $\mc_j(\mu_j)$

\lIf{$\bid$ is infeasible}{
\Return{\texttt{INFEASIBLE}}
}
$\bar{\jmath} \gets \argmax_{j}[\mc_j(\mu_j)]$

$k \gets \mc_{\bar{\jmath} }(\mu_{\bar{\jmath}})$

\For{$j \in \platform$}{
    $\mu'_j \gets \mu_j + 1$ \tcp*{Check the next $\mc$ value for platform $j$}
    query $\mu'_j$ on platform $j$ obtain $v_j(\mu'_j)$, $c_j(\mu'_j)$ and $\mc_j(\mu'_j)$
    
    \If{$\mc_j(\mu'_j) \leq k$ for any $j \neq \bar{\jmath} $\label{line:counterk}}{
\Return{\texttt{NOT-$\bid^k$} \tcp*{$\bid$ is not $\bid^k$ for some $k$}\label{line:notmuktermination}}
}
  }  
    $j^* \gets \argmin_{j}(\mc_j(\mu'_j))$ \tcp*{find the minimum among the next points}
    
    $\mu_{j^*} \gets \mu'_{j^*}$
    
    \lIf{$\bid$ is infeasible \tcp*{the updated $\bid$ is not feasible}}{
    \Return{\texttt{ALMOST-OPTIMAL}}}
    \lElse{\Return{\texttt{NOT-OPTIMAL}}}

% \Return{$\bid^k = (\mu^k_1,\mu^k_2,\dots, \mu^k_m)$}
% \SetAlgoRefName{1}
\caption{\optcheck}
\label{sub:optcheck}
\end{algorithm}
\renewcommand*{\algorithmcfname}{ALGORITHM}

\begin{lemma}\label{lem:optcheck}
Given a bidding profile $\bid$, the subroutine \optcheck\ determines if the given $\bid$ is \texttt{INFEASIBLE}, \texttt{NOT $\bid^k$}, \texttt{NOT-OPTIMAL} or \texttt{ALMOST-OPTIMAL} with at most $O(m)$ queries.
\end{lemma}
\begin{proof}
Since \optcheck\ queries at most 2 strategies from each platform, the worst-case number of queries used is \(2m = O(m)\). We now prove the correctness of the subroutine for each different case. The \texttt{INFEASIBLE} case is trivial. For the \texttt{NOT-$\bid^k$} case, as indicated by line~\ref{line:counterk}, since there exists a platform where \(\mc_j(\mu_j + 1) \leq k\), we know that the given bidding profile \(\bid\) is not \(\bid^k\) for some \(k\) by definition. On the other hand, if \optcheck\ does not terminate in line~\ref{line:notmuktermination}, it means \(\bid\) is feasible and \(\bid = \bid^k\) for some \(k\). To check if the given profile is almost-optimal (the floor of $\optf$), by Lemma~\ref{cor:opt}, we just need to verify whether increasing \(k\) would make \(\bid^k\) infeasible. If the next immediate change would cause \(\bid\) to be infeasible, then \(\bid\) is \texttt{ALMOST-OPTIMAL}; otherwise, it is \texttt{NOT-OPTIMAL}. 
% \xtnote{maybe elaborate more on the last two cases}
\end{proof}
% \subsection{The \mom\ algorithm}
We are now ready to present our algorithm, \mom. This algorithm finds the \emph{almost-optimal integral solution} by searching within the marginal cost space and then converts this almost-optimal integral solution to the optimal fractional solution using the \roundup\ procedure. The search process is inspired by the median-of-medians algorithm. In each iteration, we first identify the median marginal cost for each platform, and then select the median that most evenly splits the space, i.e., ensuring that the number of marginals weakly smaller than this median is equal to the number of marginals weakly larger than it.

Next, we use \matchingmc\ to determine the corresponding bidding profile $\bid^k$ with the median-of-the-medians marginal as the $k$-value, and apply \optcheck\ to evaluate the quality of the bidding profile $\bid^k$. Based on the result from \optcheck($ \bid^k $), we can eliminate a constant fraction of the remaining candidates for the optimal marginal costs. This process is repeated iteratively until we find an almost-optimal solution. Finally, we apply the \roundup\ procedure to obtain the fractional optimal solution. For a formal description, please refer to Algorithm~\ref{alg:mom}.

\begin{figure}[h]
\begin{center}  % Center the figure within the column
    \resizebox{0.45\textwidth}{!}{  % Resize to fit within a single column
        \begin{tikzpicture}

            % Parameters for grid layout
            \def\circleRadius{0.05}   % Radius of each circle
            \def\dotRadius{0.01}     % Radius of the smaller dots
            \def\vspacing{0.4}       % Vertical spacing between circles
            \def\hspacing{0.5}       % Horizontal spacing between platforms
            \def\platforms{7}        % Number of platforms
            \def\circlesPerPlatform{5} % Circles per vertical line
            
            % Loop to draw platforms
            \foreach \col in {1,...,\platforms} {
                
                % Adjust for the first 4 platforms: skip the first and last circle
                \ifnum\col<5
                    % Draw the circles, skipping first and last
                    \foreach \row in {2,...,4} {
                        % Position of each circle
                        \node[circle, draw, fill=white, minimum size=\circleRadius cm] 
                        at (\col*\hspacing, \row*\vspacing) {};
                    }
                \else
                    % Draw all the circles for the remaining platforms
                     \foreach \row in {1,...,\circlesPerPlatform} {
                        % Check for the fifth platform (col = 5) and the third row (row = 3)
                        \ifnum\col=5\relax
                            \ifnum\row=3\relax
                                % Fill the 3rd circle of the fifth column with grey
                                \node[circle, draw, fill=gray, minimum size=\circleRadius cm] 
                                at (\col*\hspacing, \row*\vspacing) {};
                            \else
                                % Regular circles for the rest
                                \node[circle, draw, fill=white, minimum size=\circleRadius cm] 
                                at (\col*\hspacing, \row*\vspacing) {};
                            \fi
                        \else
                            % Regular circles for other columns
                            \node[circle, draw, fill=white, minimum size=\circleRadius cm] 
                            at (\col*\hspacing, \row*\vspacing) {};
                        \fi
                    }
                \fi
                
            }
        % Draw the upward arrow
    \draw[->, >=latex] (0,1.7) -- (0,0.7) node[above] {};
    
    % Add the vertical text "increasing mc" to the right of the arrow
    % \node[right] at (-0.2,0.4) [rotate=90] {\footnotesize increasing mc};
    
    \draw[->,>=latex] (1,0) -- (3,0) node[right] {};
    
    \node[below] at (2,0) {\tiny increasing median marginal cost};    
    
    % Draw a square with rounded corners and a pattern fill
    \filldraw[pattern=north east lines, rounded corners=5pt, thick] 
        (2.3,0.2) -- (3.7,0.2) -- (3.7,1.4) -- (2.3,1.4) -- cycle;
        
    \draw[rounded corners=5pt, thick] 
        (0.3,1) -- (2.7,1) -- (2.7,2.2) -- (0.3,2.2) -- cycle;
            
        \end{tikzpicture}
    }
\end{center}
\caption{Illustration of one round of \mom. Each column represents the current search region of a platform. The vertical arrow indicates the increasing direction of $\mu$ in each platform. The platforms are ranked by the median marginals as described in the algorithm. The grey point represents the queried $k$ value. If $\bid^k$ is infeasible, all strategies in the shaded round rectangle are removed; otherwise, all strategies in the non-shaded one are removed.}
\end{figure}





\begin{algorithm}[h]
\DontPrintSemicolon
\LinesNumbered
\SetNoFillComment
% \KwIn{}
\textbf{Initialize: $\ell_j \gets 1$, $r_j \gets n$ for all $j \in \platform$}

\While{TRUE}{
$\mu_j \gets \frac{\ell_j + r_j}{2}$ for all $j \in \platform$

query each platform $j$ strategy $\mu_j$, obtain $v_j(\mu_j)$, $c_j(\mu_j)$ and $\mc_j(\mu_j)$


rank the platforms in non-decreasing order of $\mu_j$ s.t. if $i \leq j$, $\mu_i \leq \mu_j$\label{line:ranking}

$j^* \gets \min_{j}(|\sum_{i \leq j} (r_i - \ell_i) - \sum_{i \geq j}(r_i - \ell_i)|)$ \tcp*{find the $j^*$ that equally split the search space}

%$k \gets \mc_{j^*}(\mu_{j^*})$

% $k \gets \text{Median}(\{\mc_j(\mu_j)| j \in \platform\})$

%$\bid* \gets \matchingmc([\ell_j, r_j] \text{ for all } j \in \platform, k)$ %\tcp*{$O(m \log \frac{1}{\epsilon})$ queries} 

$\bid^* \gets \matchingmc([1, n] \text{ for all } j, \mc_{j^*}(\mu_{j^*}))$

\uIf{$\optcheck(\bid^*)$ = \texttt{INFEASIBLE}}{
$r_j \gets \mu_j-1$ for all $j \geq j^*$ \tcp*{reduce the search space}\label{line:cutright}
}

\uElseIf{$\optcheck(\bid^*)$ = \texttt{NOT-OPTIMAL}}
{$\ell_j \gets \mu_j+1$ for all $j\leq j^*$}\label{line:cutleft}

\uElseIf{$\optcheck(\bid^*)$ = \texttt{ALMOST-OPTIMAL}}
{\Return{\roundup($\bid^*$)}
}
}
\SetAlgoRefName{1}
\caption{\mom}
\label{alg:mom}
\end{algorithm}
In the rest of the section, we prove the correctness and query complexity of the algorithm. 

\begin{theorem}\label{thm:medianofmedians}
Given any instance $\mathcal{I}$, the \mom\ algorithm finds the fractional optimal bidding strategy with at most $O(m \log mn \log n)$ queries.
\end{theorem}
\begin{proof}
We first prove the correctness of the algorithm. By Lemma~\ref{cor:opt} we know that the almost-optimal integral bidding strategy correspond to $\bid^{k^*}$ where $k^*$ is the maximum $k$ such that $\bid^k$ is feasible. 
We prove the correctness of the algorithm by first showing that during the execution of the algorithm, there always exist some $\mu \in [\ell_j, r_j]$, of which the $\bid^{\mc_j(\mu)} = \bid^{k^*}$. 
In other words, the algorithm can not eliminate the critical marginal cost $\mc_j(\mu)$ that corresponds to $\bid^{k^*}$.
Consider the possible updates of $\ell_j$ and $r_j$ for each platform $j$, i.e., Line~\ref{line:cutright} and Line~\ref{line:cutleft}. 
First consider any iteration such that $\optcheck(\bid^*) = \texttt{INFEASIBLE}$, and for all platforms $j \geq j^*$ w.r.t to the ranking defined in Line~\ref{line:ranking}, we have $\mc_j(\mu_j) \geq \mc_{j^*}(\mu_{j^*})$. 
By monotonicity, for any $\mu \geq \mu_j$ on platform $j$ we have:
\[\mc_j(\mu) \geq \mc_j(\mu_j) \geq \mc_{j^*}(\mu_{j^*}),\]
By Lemma~\ref{lem:opt}, since $\optcheck(\bid^*) = \texttt{INFEASIBLE}$, we would also have $\bid^{\mc_j(\mu)}$ is infeasible for $\mu \geq \mu_j$ for platforms $j \geq j^*$. 
Therefore, Line~\ref{line:cutright} does not remove any $\mu$ of which $\bid^{\mc_j(\mu)} = \bid^{k^*}$.

Next consider any iteration such that $\optcheck(\bid^*) = \texttt{NOT-OPTIMAL}$, and for all platforms $j \leq j^*$ w.r.t to the ranking defined in Line~\ref{line:ranking}, we have $\mc_j(\mu_j) \leq \mc_{j^*}(\mu_{j^*})$. Again by monotonicity, for any $\mu \leq \mu_j$ on platform $j$ we have:
\[\mc_j(\mu) \leq \mc_j(\mu_j) \leq \mc_{j^*}(\mu_{j^*}),\]
By Lemma~\ref{lem:opt}, since $\optcheck(\bid^*) = \texttt{NOT-OPTIMAL}$, we have $\bid^{\mc_j(\mu)}$ is also feasible and not optimal for $\mu \leq \mu_j$ for platforms $j \leq j^*$. Therefore Line~\ref{line:cutleft} does not remove any $\mu$ of which $\bid^{\mc_j(\mu)} = \bid^{k^*}$ as well. 
% Therefore the algorithm does not remove any $\mu$ of which $\bid^{\mc_j(\mu)} = \bid^{k^*}$ during the execution.
In addition, it is easy to see that $\bid^{\mc_j(\mu)} = \bid^{k^*}$ for some platform $j$ and some strategy $\mu$. (let $\mc_j(\mu) = \argmax (\mc_j(\mu^{k^*}_j))$). And since the set of bids is finite and getting strictly smaller in each round, the algorithm will eventually terminate with the almost-optimal integral bidding solution $\bid^{k^*}$, after which applying $\roundup$ would give us the fractional optimal solution.

We now prove the query complexity of the algorithm. In particular, we argue that the while loop would iterate no more then $O(\log (mn))$ times. Together with the $O(m \log n)$ query complexity of \matchingmc\ this would show that the query complexity of the algorithm is $O(m \log (mn) \log n)$. First note that there are in total $m\cdot n$ possible marginal costs ($\mc_j(\mu)$ for all $j$ and $\mu$). By definition of $j^*$, and $\mu_j$ for each platform $j$, we have that  
$\min(|\{\mc_i(\mu): i \leq j^* \text{ and } \mu \leq \mu_i\}|, |\{\mc_i(\mu): i \geq j^* \text{ and } \mu \geq \mu_i\}|) \geq \frac{\sum_{j}(r_j - \ell_j)}{4} - \min_j(r_j - \ell_j) = O(\sum_{j}(r_j - \ell_j))$. Since we remove a constant fraction of choices in each round, the number of queries is no more then $O(\log mn)$. Lastly note that \roundup\  makes at most $m$ queries, making the total queries needed for \mom\ $O(m \log (mn) \log n)$.
\end{proof}

\section{Lower Bounds on Query Complexity}\label{sec:lowerbound}
In this section, we provide some lower bounds on query complexity. We show that any algorithm needs to have a query complexity of $\Omega(m \log n)$ even if it knows the optimal marginal cost $k$. We also provide a lower bound of $\Omega(\log mn)$ for finding the optimal marginal cost $k$ even when the algorithm is given a single-query black-box oracle for $\matchingmc$. Note that our algorithm \mom~ finds the optimal solution essentially by searching for the optimal marginal cost using $O(\log mn)$ calls to $\matchingmc$ which itself costs $(m \log n)$ queries, and the query complexity upper bound is in fact the product of the two aforementioned lower bounds. This suggests that improving the query complexity upper bound further would require an algorithm that does not treat $\matchingmc$ as a black-box. 
% Due to space limitations, we defer the proofs in this section to Appendix~\ref{app:lowerbound}.

\begin{theorem}\label{thm:lowerbound1}
Any algorithm needs $\Omega(m \log n)$ queries to find the optimal bidding strategy, even if it knows the value $k$ for which $\bid^k$ is the almost-optimal integral bidding strategy.
\end{theorem}
\begin{proof}
Given any algorithm, assume it knows the correct value of $ k $. On each platform, finding the maximum $ \mu $ (therefore the $\ceil{\mu_j^o}$) such that $ \mc_j(\mu) \leq k $ takes at least $ \Omega(\log n) $ queries. We prove this via a decision tree argument similar to the $ \Omega(\log n) $ query complexity for the binary search problem. 

Fix an arbitrary platform $ j $; we want to determine the maximum index $ \mu $ such that $ \mc_j(\mu) \leq k $. We represent any algorithm as a decision tree as follows:

(1). Each query made to the platform is represented as a node in the decision tree, and each node has three children: one for $ \mc_j(\mu) \leq k $, one for $ \mc_j(\mu) > k $, and a third for cases not specified.
(2). The leaves of this tree represent the possible outcomes of the search: specifically, finding the maximum index $\mu$ such that $\mc_j(\mu) \leq k$.

There are $n + 1$ distinct outcomes, corresponding to the maximum value of $\mu$ being 0, 1, ..., or $ n$. In any decision tree with $x$ leaves, the minimum height $h$ is $\log x$. 

Moreover, the height $h$ of the decision tree corresponds to the number of queries made. Therefore, the minimum height of the decision tree is $\log(n + 1)$, implying that the number of queries needed to resolve the search will be at least $\Omega(\log(n + 1)) = \Omega(\log n)$. 

Lastly, since all platforms operate independently, the search on each platform requires $ \Omega(\log n)$ queries. Consequently, to complete the search across all platforms will require $ \Omega(m \log n) $ queries.
\end{proof}

\begin{theorem}\label{thm:lowerbound2}
Any algorithm needs $\Omega(\log (mn))$ queries to find the optimal bidding strategy, even if it has access to a black-box oracle of $\matchingmc$ that uses a single query. 
% \ganote{should this say "that uses a single query" rather than "without using any queries"?}
\end{theorem}

\begin{proof}
There are a total of \( mn \) distinct marginal costs, and our objective is to determine the marginal cost \( \mc_j(\mu) \) for a specified \( j \) and \( \mu \) such that \( \bid^{\mc_j(\mu)} \) represents the almost-optimal integral solution. We establish this by reducing the binary search problem to our problem. 

Consider a binary search scenario involving a single sorted array \( A \) with \( |A| = h \) and a target number \( a \) for which we are searching. Let \( i \) denote the index of \( a \) within this array. We can construct an instance of our problem featuring a global ordering of marginal costs. In this global ordering, the marginal costs \( \mc_j(\mu) \) located at index \( i \) correspond to the bidding strategy \( \bid^{\mc_j(\mu)} \), which serves as the almost-optimal integral solution. If we are able to identify the index of the optimal marginal cost in fewer than \( \Omega(\log mn) \) queries, it would consequently allow us to resolve the binary search problem in fewer than \( \Omega(\log h) \) queries. This outcome would contradict the established complexity bounds associated with binary search.
\end{proof}
\section{Learning-Augmented Algorithms}
In this section we aim to design searching algorithm that utilize a (possibility erroneous) prediction $\pred$ regarding the actual optimal fractional strategy $\optf$. The error of the prediction is measured by its distance to the optimal solution in the $\ell$-infinity norm, i.e.
\begin{align}\label{eq:errordef}
    \eta = \max_{j}|\hat{\mu}_j - \mu^o_j|.
\end{align}
We show the following algorithm, modified from \mom, achieves a query complexity of $O(m \log m\eta \log\eta)$, note that since $\eta \leq n$, this guarantee matches the query complexity of \mom\ even if the prediction is arbitrarily wrong.

The algorithm begins by checking whether the floor of the predicted bidding strategy, $\floor{\hat{\mu}_j}$, for all $j $, is \texttt{ALMOST-OPTIMAL} using \optcheck. If it is, the algorithm applies \roundup\ and returns the optimal solution. If not, the algorithm assumes the error is small and attempts to search for the optimal solution within a restricted range around the predicted strategy $ \hat{\mu}_j$ on each platform, following a similar approach to the \mom\ algorithm. 

If the optimal solution is still not found, the search range is expanded, and the search is repeated. This process continues until a almost-optimal solution is identified. By progressively expanding the search range as the square of the previous range, we show that the query complexity is at most  $O(m \log (m\eta) \log \eta)$. Please refer to Algorithm~\ref{alg:bmom} for a formal description.

\begin{algorithm}[h]
\DontPrintSemicolon
\SetAlgoLined
\LinesNumbered
\SetNoFillComment
% \KwIn{}
\textbf{Initialize: $\ell_j \gets \hat{\mu}_j$, $r_j \gets \hat{\mu}_j$ for all $j \in \platform$}

$\pred \gets \floor{\pred}$

\lIf{\optcheck($\pred$) == \texttt{ALMOST-OPTIMAL}}{\Return{\roundup($\pred$)}}
$i \gets 0$ \tcp*{initialize the counter for doubling process}

\While{TRUE}{
$\ell_j \gets \hat{\mu}_j - 2^{2^i}$, $r_j \gets \hat{\mu}_j+2^{2^i}$ for all $j \in \platform$

range-indicator $\gets$ TRUE \tcp*{assume range is correct}

\While{range-indicator == TRUE}{

$\mu_j \gets \frac{\ell_j + r_j}{2}$ for all $j \in \platform$

query each platform $j$ strategy $\mu_j$, obtain $v_j(\mu_j)$, $c_j(\mu_j)$ and $\mc_j(\mu_j)$

rank the platforms in non-decreasing order of $\mu_j$ s.t. if $i \leq j$, $\mu_i \leq \mu_j$

$j^* \gets \min_{j}(|\sum_{i \leq j} (r_i - \ell_i) - \sum_{i \geq j}(r_i - \ell_i)|)$ \tcp*{find the $j^*$ that equally split the search space}

$k \gets \mc_{j^*}(\mu_{j^*})$


% query each platform $j$ strategy $\ell_j$, obtain $v_j(\ell_j)$, $c_j(\ell_j)$ and $\mc_j(\ell_j)$ 

\If{there exist a $\mc_j(\hat{\mu}_j - 2^{2^i}) > k$}
{$\ell_j \gets \mu_j+1$ for $j \leq j^*$ \tcp*{$k$ is too small} \label{line:leftcheck}}

\Else{$\bid^k \gets \matchingmc([\hat{\mu}_j -2^{2^i}, \hat{\mu}_j+2^{2^i}]\ \forall j, k)$ %\tcp*{$O(m \log \frac{1}{\epsilon})$ queries} 

\If{$\optcheck(\bid^k)) == $ \texttt{NOT-$\bid^k$}}{$r_j \gets \mu_j-1$ for all $j \geq j^*$ \tcp*{$k$ is too large}\label{line:rightcheck}}

\ElseIf{$\optcheck(\bid^k)$ == \texttt{INFEASIBLE}}{
$r_j \gets \mu_j-1$ for all $j \geq j^*$ \tcp*{$k$ is too large}\label{line:infeasible}
}

\uElseIf{$\optcheck(\bid^k)$ == \texttt{ALMOST-OPTIMAL}}{\Return{\roundup($\bid^k$)}\label{line:opt}}
\Else{\tcp*{$\optcheck(\bid^k)$ == \texttt{NOT-OPTIMAL}}
$\ell_j \gets \mu_j+1$ for all $j \leq j^*$ \tcp*{$k$ is too small}
}\label{line:notoptimal}}
\If{there exist a platform with $\ell_j > r_j$}{
range-indicator $\gets$ FALSE \tcp*{search in the given range is complete}}
}


$i \gets i+1$ \tcp*{update the search range}
}
\SetAlgoRefName{2}
\caption{\bmom}
\label{alg:bmom}
\end{algorithm}

%\mznote{In Line 21 of the algorithm, if the return is not-mu-k, it must be the case that some platform hits the boundary. If it hits the lower bound, should we make $k$ larger instead of smaller?}
%\xtnote{I think the case you mentioned is in Line 15-16 so in line 21 the case is assumed away already, so we will only be hitting the upper bound)}

\begin{theorem}\label{thm:momwithpredictions}
Given any instance $\mathcal{I}$, and predicted optimal bidding strategy $\pred$ such that the error of the $\pred$ is $\eta$, the \bmom\ algorithm finds the optimal bidding strategy with at most $O(m \log m \eta \log \eta)$ queries.
\end{theorem}
\begin{proof}
We first argue the correctness of the algorithm. Let \( \eta \) denote the error of the prediction as defined in \eqref{eq:errordef}, and let \( i^* \) be the smallest \( i \) such that \( 2^{2^i} \geq (\eta + 1) \). We will show that the algorithm does not terminate in any round \( i < i^* \). Let \( \bid^* \) represent the almost-optimal integral bidding strategy. When \( i < i^* \), there exists at least one platform \( j \) such that \( \mu^*_j \notin [\ell_j, r_j] \) at the beginning of the \( i^* \)-th iteration of the outer while loop. By the correctness of \optcheck\ and since the algorithm only searches within the range \( [\ell_j, r_j] \), it cannot return a solution in earlier rounds.

Next, we argue that the algorithm will terminate in round \( i^* \) with the optimal solution. Since \( 2^{2^{i^*}} \geq (\eta + 1) \), we know that \( \mu^*_j \in [\ell_j, r_j] \) for all \( j \) at the start of the \( i^* \)-th iteration of the outer while loop. Now, consider the search process during this round. As in the proof of Theorem~\ref{thm:medianofmedians}, we show correctness by proving that during the execution of the \( i^* \)-th iteration, there always exists some \( \mu \in [\ell_j, r_j] \) such that \( \mu^{\mc_j(\mu)} = \bid^* \). Considering all possible updates to \( \ell_j \) and \( r_j \) for each platform, we will now show that none of these updates eliminate any such \( \mu \) values.

First, in Line~\ref{line:leftcheck}, the algorithm encounters a platform \( j \) where \( \mc_j(\hat{\mu}_j - 2^{2^{i^*}}) > k \), meaning the current candidate marginal cost \( k \) is smaller than the marginal cost of the smallest strategy within the current search range for that platform. Since \( \mu^*_j \in [\hat{\mu}_j - 2^{2^{i^*}}, \hat{\mu}_j + 2^{2^{i^*}}] \) for all platforms and \( \bid^* = \bid^k \) for some \( k \), this implies that the current marginal cost candidate \( k \), as well as all marginal costs weakly smaller than \( k \), cannot correspond to the optimal marginal cost \( \bid^* \). These marginal costs (and their corresponding strategies) are thus eliminated from the search range.

In Line~\ref{line:rightcheck}, the algorithm is in the case where \( \optcheck(\bid^k) == \texttt{NOT}-\bid^k \), indicating that \( \bid^k \) is not optimal. This implies that there exists a platform \( j \) such that: 
1. \( \mu^k_j = \hat{\mu}_j + 2^{2^{i^*}} \), i.e., the largest strategy, and 
2. for the same platform \( j \), \( \mc_j(\hat{\mu}_j + 2^{2^{i^*}}+1) \leq k \). 
This means that \( k \), along with all marginal costs weakly greater than \( k \), exceeds the optimal marginal cost corresponding to \( \bid^* \). These marginal costs (and their corresponding strategies) are therefore eliminated from the search range.

The remaining cases are handled in the same way as discussed in Theorem~\ref{thm:medianofmedians}. In Line~\ref{line:infeasible}, when \( \bid^k \) is infeasible, we eliminate all marginal costs weakly greater than the current one being tested. In Line~\ref{line:notoptimal}, when \( \bid^k \) is not optimal, we eliminate all marginal costs weakly smaller than the current one. Finally, in Line~\ref{line:opt}, once we find \( \bid^* \), we use \roundup\ to obtain the optimal fractional strategy.

We now prove the query complexity of the algorithm. Let \( i^* \) be the value of \( i \) when the algorithm terminates. First, we have \( i^* \leq \eta^2 \), where \( \eta \) is the error of the prediction as defined in \eqref{eq:errordef}. By Lemma~\ref{lem:matchingmc}, we know that the \matchingmc\ operation in iteration \( i \) takes \( m \log 2^{2^i} \) time. Additionally, by Theorem~\ref{thm:medianofmedians}, the while loop within this iteration will run \( \log (m 2^{2^i}) \) times. Since all other subroutines take \( O(m) \) queries, and the size of the search range is squared at each step, the algorithm terminates when the search space is weakly larger than \( n \).
{\allowdisplaybreaks
\begin{align*}
    \sum_{i = 0}^{\log \log i^*} m \log (m \cdot 2^{2^i}) \cdot \log 2^{2^i} 
    =& \sum_{i = 0}^{\log \log i^*} m (\log m + 2^i) 2^i\\
    =& m (\log m + 2^{\log \log i^*+1}) \cdot 2^{\log \log i^* +1}\\
     =& m (\log m + 2 \log i^*) \cdot 2 \log i^*\\
     \leq& m (\log m + 2 \log (\eta)^2) \cdot 2 \log (\eta)^2\\
     =& m (\log m +4\log (\eta)) \cdot 2 \log (\eta)\\
     =& O(m \log(m \eta)\log \eta)\qedhere
\end{align*}}
\end{proof}

As a corollary, we also achieved "best-of-both-worlds" results in terms of consistency and robustness. Specifically, if the provided prediction is correct (or even "almost correct," i.e., $\floor{\pred} = \floor{\optf}$), only $2m$ queries are required (note that even checking that a bidding profile is feasible requires $m$ queries).
%gagan: changed the above sentence
Since $\eta \leq n$ by definition, the total number of queries will never exceed $O(m \log (mn) \log n)$, which matches the query complexity of \mom:
%\xtnote{if we include the lower bounds discussion, we should include the statement below as well}
\begin{corollary}
The \bmom\ algorithm is $2m$-consistent and $O(m \log mn \log n)$ robust, where the robustness matches the query complexity of \mom.
\end{corollary}
% \section{The Centroid Method}
In this section, we provide an alternative method that utilize the classic centroid algorithm in convex optimization to obtain a solution that is $\varepsilon$ close to the optimal solution. Recall the bidder's problem is:
\begin{align*}
    \max_{\bid=(\mu_1, \mu_2 \dots, \mu_m)} &\sum_{j \in \platform} v_j(\mu_j) \nonumber\\
    s.t. & \sum_{j \in \platform} c_j(\mu_j) \leq T \cdot \sum_{j \in \platform} v_j(\mu_j). 
\end{align*}
\xtnote{double check the assumptions on the value and cost function} Since each of the value functions are concave, the sum of them is also a concave function. In addition, the feasible range defined by the ROS constraint is a convex region. \agnote{We should add a reason for this convexity.} Let $R$ be convex feasible range, we have $R \subseteq [0,n]^m$. For presentation purpose, we abuse notation and write $v(\bid) = \sum_j v_i(\mu_j)$. We want to approximately maximize the objective function, i.e., we want to output a bidding strategy $\bid'$ such that 
\[ v(\bid') \geq  v(\optf)-\varepsilon.\]
we assume a gradient oracle for the objective function\footnote{\xtnote{discuss why the assumption is reasonable}}.
The centroid \( c \) of a convex range \( R \subseteq \mathbb{R}^n \) can be expressed as:
\[
c = \frac{\int_R x \,dx}{\text{vol}(R)}, 
\]
where \(\text{vol}(R)\) is the volume of the range \( R \). This represents the ``center of mass'' of the range.
The following
result of the centroid is crucial to our analysis. 
\begin{lemma}
[Grünbaum's Lemma]For any compact convex set  $R \subseteq \mathbb{R}^n \text{ with a centroid } c \in \mathbb{R}^n,$ 
and for any halfspace \( H = \{ x \in \mathbb{R}^n \mid a^\top (x - c) \geq 0 \} \) whose supporting hyperplane passes through \( c \), the following holds:
\[
\frac{1}{e} \leq \frac{\text{vol}(K \cap H)}{\text{vol}(K)} \leq 1 - \frac{1}{e}.
\]
\end{lemma} 
In addition, we also utilize the strong separation oracle.
\begin{definition}
[strong seperating orcale]
For a convex set $K \subseteq \mathbb{R}^n$, a strong separation oracle for $K$ is an algorithm that takes a point $z \in \mathbb{R}^n$ and correctly outputs one of:
\begin{itemize}
    \item[(i)] \textbf{Yes} (i.e., $z \in K$), or
    \item[(ii)] \textbf{No} (i.e., $z \notin K$), as well as a separating hyperplane given by $a \in \mathbb{R}^n$, $b \in \mathbb{R}$ such that $K \subseteq \{ x \in \mathbb{R}^n \mid \langle a, x \rangle \leq b \}$ but $\langle a, z \rangle > b$.
\end{itemize}
\end{definition}
With these tools in hand, we present the following algorithm of our problem inspired by the centroid method for convex optimization. The \textsc{CentroidMethod} algorithm iteratively refines a search space by checking the centroid of the current search region in each iteration and queries the oracle to check if the centroid is feasible. If the point is infeasible, the oracle provides a separating hyperplane to update and reduce the search space; if feasible, the region is updated based on the value function's gradient. In particular, we consider the halfspace
of vectors positively correlated with the gradient, and continue. In either case, a constant fraction of the region is removed from the search space. This process repeats for \( T \) iterations, progressively narrowing the search space. After all iterations, the algorithm selects the best feasible point that minimizes the value function among the centroids checked, and returns it as the solution. Please refer to Algorithm~\ref{alg:centroid} for a formal description.

\begin{algorithm}
\SetAlgoRefName{3}
\SetAlgoLined
\KwIn{value function $v$, $T$}
% \KwOut{$\bid$}
\textbf{Initialize:} $K_1 \leftarrow [0,n]^m$\\
\For{$t = 1, \ldots, T$}{
    $\bid_t \leftarrow$ centroid of $K_t$\\
    query strong separation oracle on $\bid_t$\\
    \If{$\bid_t$ is not feasible}{
    $a_t \gets$ direction from strong separation oracle\\
    $K_{t+1} \leftarrow K_t \cap \{ x \mid \langle \nabla a_t, x - \bid_t \rangle\leq 0 \}$}
    \Else{
    $K_{t+1} \leftarrow K_t \cap \{ x \mid \langle \nabla v(c_t), x - \bid_t \rangle \geq 0 \}$}
}
$\bid \leftarrow \arg\min_{t \in \{1, \ldots, T|\ \bid_t \text{ is feasible}\}} v(\bid_t)$\\
\Return $\bid$
\caption{\textsc{CentroidMethod}}
\label{alg:centroid}
\end{algorithm}


% \begin{theorem}
% , the Centroid algorithm finds a (randomized) bidding strategy that achieves a value of at least opt-$\varepsilon$ with at most $O(m^2 \log \frac{G(n+1)}{\varepsilon})$,
% where $G$ is the upper bound of gradient of the objective function.
% \end{theorem}



\begin{theorem}
Given the feasible convex set $R \subseteq \mathbb{R}^m$ with $||x - y|| \leq r$ for any $x,y \in R$, and a convex function $v : R \rightarrow \mathbb{R}$ such that $\|\nabla v(x)\| \leq G$ for all $x \in R$. If $\bid$ is the result of the \textsc{CentroidMethod} algorithm,  $\optf = \arg\max_{x \in R} v(x)$, \xtnote{and  $\gamma = \frac{\text{vol}(R)}{\text{vol}[0,n]^m}$} then
\[
\xtnote{v(\optf) - v(\bid)  \leq \frac{4Gr}{\gamma} \cdot \exp(-T/3n)}.
\]
Hence, for any $\varepsilon \leq 1$, as long as $T \geq 3n \ln \frac{4Gr}{\varepsilon}$, we have
\[
 v(\optf) - v(\bid) \leq \varepsilon.
\]
\end{theorem}

\begin{proof}
For some $\delta \leq 1$, define the body
\[
R^\delta := \{(1 - \delta)\optf + \delta x \mid x \in R\}
\]
as a scaled-down version of $R$ centered at $\optf$. we have that:
\begin{enumerate}
    \item $\text{vol}(R^\delta) = \delta^m \cdot \text{vol}(R)$.
    \item For any points $x, y \in R$, integrating along the path from $x$ to $y$ and using the fact that the gradients are bounded by $G$ gives
    \begin{align*}
        v(x) - v(y)& = \int_{t=0}^1 \langle \nabla v(y + t(x - y)), x - y \rangle  dt\\
         &\leq \int_{t=0}^1 \|\nabla v(y + t(x - y))\| \|x - y\| dt\\
         & \leq G \|x - y\| \leq G \cdot (2r).
    \end{align*}
    \item The value of $v$ on any point $y = (1 - \delta)\optf + \delta x \in R^\delta$ is
    \begin{align*}
        v(y) & = v((1 - \delta)\optf + \delta x) \geq (1 - \delta)v(\optf) + \delta v(x)\\
        & = v(\optf) - \delta (v(\optf) - v(x)) \geq v(\optf) - 2\delta Gr,
    \end{align*}
where the first inequality is by concavity of the value function and the third inequality is by fact (2) above.
\end{enumerate}
Using Grünbaum's lemma, in our algorithm, the volume of the search range (may not necessarily be entirely feasible) falls by a constant factor in each iteration, so $\text{vol}(K_t) \leq \text{vol}(K_1) \cdot \left(1 - \frac{1}{e}\right)^t$, where $K_1 = [0,n]^m$ \xtnote{let $\gamma = \frac{\text{vol}(R)}{\text{vol}(K)}$ be the ratio between the volume of the total search region of the algorithm and the feasible region in it.
Let $\delta := 2/\gamma(1 - 1/e)^{T/m}$, then after $T$ steps the volume of $K_T$ is smaller than that of $R^\delta$}. Therefore, some point of $R^\delta$ must have been cut off.

Consider step $t$ with $R^\delta \subseteq K_t$ but $R^\delta \not\subseteq K_{t+1}$. Let $y \in K^\delta \cap (K_t \setminus K_{t+1})$ be a point that is ``cut off''. By concavity, we have
\[
v(y) \leq v(c_t) + \langle \nabla v(c_t), y - c_t \rangle;
\]
moreover, $\langle \nabla v(c_t), y - c_t \rangle < 0$ since the cut-off point $y \in K_t \setminus K_{t+1}$. 
In addition, since the hyperplane cross $R^\delta \subseteq R$, we also have that $c_t \in R$, since if $c_t \notin R$, the hyperplane defined by the strong separation orcale would not cut-off any feasible ranges.
Hence the corresponding centroid has value $v(c_t) > v(y) \geq v(\optf) - 2\delta Gr$. Since $\bid$ is the centroid with the smallest function value, we get

\[
\xtnote{f(\bid) - f(\optf) \leq 2Gr \cdot \frac{2}{\gamma}\left(1 - \frac{1}{e}\right)^{T/m} \leq \frac{4Gr}{\gamma} \exp(-T/3m).}
\]
The second claim follows by substituting \xtnote{$T \geq 3m \ln \frac{4Gr}{\gamma\varepsilon}$} into the first claim, and simplifying.
\end{proof}

As a corollary of the theorem, we can get the following query bound of the centroid metho of our algorithm:
\begin{corollary}
GIven any instance $M$, $S$, the \textsc{CentroidMethod} algorithm finds a bidding strategy such that
\[v(\bid) \geq v(\optf)-\varepsilon\]
with at most \xtnote{ $O(m^2\log \frac{4G\sqrt{m}n}{\gamma\varepsilon})$ }queries, where  $\|\nabla v(x)\| \leq G$ for all $x$ in the feasible range $R$.
\end{corollary}

\begin{proof}
Since the feasible region $R \subseteq [0,n]^m$, we have that $\|x-y\| < \sqrt{m}(n+1)$ for any $x, y \in R$. In addition, in each iteration, querying the strong separation oracle and obtaining the value of the centroid point requires $m$ queries. Given some $\varepsilon > 0$, we therefore have that the number of queries needed is 
\[
\xtnote{
m \cdot 3n \ln \frac{4GR}{\gamma\varepsilon} = O(m^2 \log \frac{G\sqrt{m}n}{\gamma\varepsilon}).}
\qedhere
\]
\end{proof}


% \begin{lemma}
% Given a fractional optimal solution, there should exist at most one platform such that the bid is not integral
% \end{lemma}

% \xtnote{conjecture: Can we translate the $\varepsilon$-fractional OPT to fractional OPT or discrete OPT 
% 	With some assumption on the minimum difference between adjacent value/marginals}

\section{Conclusion}
\label{sec:conclusion}

We propose using surrogate modeling to evaluate the throughput of different infrastructure designs. 
%
We train three model architectures using simulator data to predict different job observables.
%
From our evaluation results, the architecture choice does not significantly influence the accuracy of the predictions at the current stage of development.
%
All three architectures decrease the execution times by orders of magnitude compared to DCSim.

At the current stage of the models inaccuracies are observed.
We suspect a lack of input information given to the models as the predominant source.
While the models are able to predict the compute times of our heterogeneous jobs scenario, where some implicit information about the infrastructure can be extracted by the model, they fail to predict the transfer time of input files due to not being aware of the data infrastructure setup that is more complex. 

Future work can build on these results by incorporating platform information into the training data to improve the predictions.
This will become essential to ensure that the model performs accurately on arbitrary workload mixes.
%
Moreover, training on real-world data instead of simulator data could enhance the capabilities and applicability of the models, also providing valuable data in regimes where the simulation, due to its scaling behavior, is not able to feasibly produce large amounts of training data.


\bibliographystyle{abbrvnat}
\bibliography{biblio}



\end{document}