\section{Intersectionality}
\label{ch:intersectionality}%

In this section, we explore the concept of intersectionality, a framework that appreciates the complexity of human experiences within societal structures. We trace its origins, explore fundamental theories like Kimberlé Crenshaw's tripartite framework and Patricia Hill Collins' Matrix of Domination, and reflect on its use as an analytical tool to address social divisions. These discussions provide crucial insights into the role of intersectionality and its potential applications, such as in computer science, as we will see in the next section.

\subsection{Definition and origin of intersectionality}
Intersectionality is a conceptual framework that acknowledges the complexity and multiplicity of human experiences and social realities. It asserts that an individual's experiences and the societal structures they navigate cannot be reduced to a single factor. Instead, they are shaped by many complex facets that interact and influence each other. This perspective is particularly relevant when examining social inequality, where it becomes evident that people's lives and societal power structures are not shaped by a single axis of social division, such as race, gender, or class, but by multiple axes that work together \cite{collins2020intersectionality}.

The concept of intersectionality traces its roots back to the legal field, as revealed in the meticulous work of Kimberlé Crenshaw, an American civil rights advocate and key proponent of critical race theory. In her research \cite{crenshaw1989demarginalizing}, Crenshaw explores various legal cases illuminating intersectional discrimination.

One such foundational case is `DeGraffenreid v. General Motors' in 1969, where the complexity of intersectional discrimination first came to light. This lawsuit was instigated by five black women who claimed that the seniority system employed by General Motors sustained a history of discrimination against black women. In this case, the complainants were not merely advocating for black individuals or women but particularly highlighting the intersectional experiences of black women. Despite their request, the district court negated their claim, arguing that black women did not represent a distinct class prone to discrimination. Additionally, the court denied their attempt to amalgamate claims of race and sex discrimination, asserting they could not craft a `super-remedy' not initially foreseen by the authors of the relevant statutes.

Only in 2020 did the judicial landscape shift with the case of `Bostock v. Clayton County, Georgia' \cite{lund2020unleashed}. Here, the Supreme Court took a momentous step towards acknowledging intersectionality. They recognized the inextricable link between sex and sexual orientation or gender identity. This landmark ruling is viewed as a progression towards acknowledging intersectionality by the highest judicial body, potentially creating opportunities for more comprehensive claims of intersectional discrimination.

Intersectionality is being used as an analytical tool to address problems arising from various social divisions, such as race, gender, class, ethnicity, and more. For instance, in higher education, institutions have faced challenges related to inclusivity and fairness within their communities. The demographics within these institutions are increasingly diverse, including individuals from different racial, economic, and social backgrounds, each of whom brings unique experiences and needs.\\
Previously, the approach to addressing the needs of these diverse groups was often segmented, focusing on one group at a time. However, this approach proved inefficient as it needed to account for the intersecting identities of individuals who belonged to more than one social group. The concept of intersectionality thus emerges as a critical tool for understanding and strategizing ways to achieve equity within such diverse settings \cite{collins2020intersectionality}.

\subsection{Crenshaw's Theory} 
\label{sec:int_crensh}
In 1990 Kimberlé Crenshaw introduced in her seminal work \cite{crenshaw1990mapping} a tripartite framework of intersectionality: structural, political, and representational. This theory is particularly relevant to the experiences of women of color, who often find themselves at the intersection of multiple forms of discrimination.


% \subsubsection{Structural Intersectionality}
% \label{subsec:int_crensh_struct}

\emph{Structural intersectionality} refers to the unique experiences of women of color who exist within the intersection of race and gender. Crenshaw argues that these women often face distinct types of violence and discrimination that are qualitatively different from those experienced by white women. As shown in \cite{crenshaw1990mapping}, the experiences of women of color in domestic violence are often shaped by their socioeconomic status, which is a product of systemic racial and gender biases. These women often face additional challenges such as poverty, childcare responsibilities, and lack of job skills, which can further complicate their situations and limit their access to resources and support.

% \subsubsection{Political Intersectionality}
% \label{subsec:int_crensh_pol}

\emph{Political intersectionality} refers to how women of color can be marginalized within political systems and discourses. This marginalization can occur within both feminist and antiracist politics, which paradoxically can contribute to the invisibility of issues specifically affecting women of color. The unique experiences and challenges these women face are often overlooked or misunderstood due to the dominant focus on either gender or race, but rarely their intersection.

% \subsubsection{Representational Intersectionality}
% \label{subsec:int_crensh_rep}

The third form of intersectionality, \emph{representational intersectionality}, refers to the cultural construction and depiction of women of color. Crenshaw argues that these women are often misrepresented or underrepresented in popular culture, which can further contribute to their marginalization and disempowerment. This lack of accurate representation can perpetuate harmful stereotypes and biases, further complicating the intersectional experiences of women of color.

\subsection{Collins' Theory}
\label{sec:int_coll}

Sociology professor Patricia Hill Collins developed a theory known as the \emph{Matrix of Domination}~\cite{collins1990} in 1990 to elucidate the interconnected nature of social categorizations such as race, class, and gender. 
%
The Matrix of Domination suggests that multiple, intersecting layers of oppression influence individuals and groups. These layers are not isolated but are interlinked and mutually reinforcing, operating along axes such as race, class, and gender, and functioning at multiple levels: personal, cultural, and institutional. Each individual's experiences within this matrix are unique, shaped by their specific combination of identities and the ways these identities interact with societal structures and norms.
%
In this matrix, individuals can simultaneously be oppressors and oppressed, depending on the context. For instance, white women may be oppressed by their gender but privileged by their race. This highlights the complexity of intersectionality and the need for a nuanced understanding of individual experiences within the matrix.

\subsection{Criticisms}
\label{sec:int_criticisms}

Scholars Lisa Downing and Tommy J. Curry have criticized the concept of intersectionality despite its significant contributions to understanding social inequality.

Downing introduces a novel angle on intersectionality with her concept of \emph{identity category violations}~\cite{downing2018body}. She discusses how intersectionality categorizes individuals into predefined identity boxes, thus potentially overlooking those whose identities encompass less conventional mixtures. Downing claims that intersectionality's rigid categories might not fully encapsulate multifaceted identities.

Tommy J. Curry examines the way intersectionality handles the portrayal of black men \cite{curry2021decolonizing}, arguing that the theory often oversimplifies and homogenizes their experiences, failing to recognize their diversity. Moreover, Curry points out the problem of pathologizing black men within intersectionality, suggesting that the theory often reduces them to tragic figures associated with violence, which neglects their complex humanity. Curry demands a more comprehensive representation of black men within intersectionality, including their identities' positive and functional aspects.

%--------------------------------------------------------------------------------------
