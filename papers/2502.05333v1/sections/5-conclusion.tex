\section{Conclusion}
\label{ch:conclusions}%

% Our work initially examined the need for ethical and fair ranking systems. 
Rankings hold significant weight in multiple sectors, from recommendation systems to hiring processes. Despite the crucial role of these systems, we have identified consistent instances of bias that lead to discriminatory and prejudiced results.
In this respect, we have focused on ethical rankings and fairness in computer science. Addressing fairness and mitigating bias in algorithms, particularly regarding protected attributes such as race, gender, and socio-economic status, is critical.

Nevertheless, a significant gap in the current fairness-aware ranking methods lies in their tendency to focus on individual protected attributes, often ignoring the intertwined nature of these attributes that shape a person's identity.
The intersectionality perspective provides a more refined understanding of how bias and discrimination materialize in data-driven systems, particularly those impacting individuals belonging to multiple marginalized groups.

Our research has stressed the need to incorporate intersectionality into creating fairness-aware ranking algorithms. By considering multiple protected attributes together, the intersectional approach to fair rankings could achieve more equitable outcomes.
Our study offered a comparative analysis of existing literature on intersectional fair ranking in computer science through practical examples and provided a synoptic table to more easily compare the different methods.

We highlighted how fairness does not necessarily imply intersectionality, emphasizing the need for future research to explore and further develop intersectional approaches to fairness in ranking systems. More comprehensive techniques considering the intersection of multiple protected attributes could enable the development of ranking systems that more accurately reflect our diverse and intersectional society. Moreover, these approaches could contribute not only to more unbiased data-driven systems, but also to a more equitable representation of data in general, enhancing the integrity and reliability of such systems. 

% We foresee a more unbiased future for data-driven systems, where intersectionality is no longer an optional aspect of fair rankings but a fundamental requirement. This vision guides future research in this area, encouraging the continuous pursuit of innovative methodologies to ensure fair and unbiased outcomes.\\
% These studies show the potential and capability to shape a more equitable and ethically driven future in data-driven systems.
