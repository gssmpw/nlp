%\documentclass{article}
\documentclass[runningheads]{llncs}
\usepackage[utf8]{inputenc}
\usepackage{xcolor}
\usepackage{xcolor}
\usepackage{lineno} 
\usepackage{xcolor}
\usepackage{graphicx,multicol}

\usepackage{epic,eepic,epsfig}

\usepackage{amssymb}
\usepackage[square,numbers]{natbib}
\usepackage{amsmath} %new

%\usepackage{amsthm} 
\usepackage{cases}
\usepackage{authblk}
\usepackage{tikz}
\usepackage[upright]{fourier}
\usepackage{tkz-graph}
\usepackage{mathtools}


\newcommand{\rev}[1]{\textcolor{black}{#1}}

\newcommand{\wal}[1]{\textcolor{red}{#1}}
\newcommand{\ale}[1]{\textcolor{blue}{#1}}
\DeclarePairedDelimiter\ceil{\lceil}{\rceil}
\DeclarePairedDelimiter\floor{\lfloor}{\rfloor}

\def\LN{{\rlap{\rm I}\hskip0.15em\hbox{\rm N}}}
%---------------------------------------------------------------------
\definecolor{forestgreen}{rgb}{0.13, 0.55, 0.13}

%\newtheorem{theorem}{Theorem}[section]

%\newtheorem{lemma}[theorem]{Lemma}

%\newtheorem{proposition}[theorem]{Proposition}

%\newtheorem{observation}[theorem]{Observation}

%\newtheorem{definition}[theorem]{Definition}

%\newtheorem{corollary}[theorem]{Corollary}

%\newtheorem{conjecture}[theorem]{Conjecture}

%\newtheorem{question}[theorem]{Question}

%\newtheorem{problem}[theorem]{Problem}

\newtheorem{alg}[theorem]{Algorithm}

\begin{document}
%\newtheorem*{claim}{Claim}
\title{Hunting a rabbit is hard}
%\titlerunning{Abbreviated paper title}
% If the paper title is too long for the running head, you can set
% an abbreviated paper title here
%
\author{Walid Ben-Ameur\inst{1} \and
Harmender Gahlawat\inst{2} \and
Alessandro Maddaloni\inst{1}}
%
\authorrunning{W. Ben-Ameur et al.}
% First names are abbreviated in the running head.
% If there are more than two authors, 'et al.' is used.
%
\institute{SAMOVAR, Télécom SudParis, Institut Polytechnique de Paris, Palaiseau, France
\email{\{walid.benameur,alessandro.maddaloni\}@telecom-sudparis.eu}\\
%\url{http://www.springer.com/gp/computer-science/lncs} 
\and
LIMOS, Universite Clermont Auvergne, France\\
\email{harmendergahlawat@gmail.com}}
%
\maketitle        

%\title{Hunting a rabbit is hard}

%\author[1]{Walid Ben-Ameur}
%\author[2]{Harmender Gahlawat}
%\author[1]{Alessandro Maddaloni}

%\affil[1]{SAMOVAR, Télécom SudParis, Institut Polytechnique de Paris, Palaiseau, France}

%\affil[2]{LIMOS, Universite Clermont Auvergne, France}



%\begin{document}

\begin{abstract}
In the Hunters and Rabbit game, $k$ hunters attempt to  shoot an invisible rabbit on a given graph $G$. In each round, the hunters can choose $k$ vertices to {shoot at}, while the rabbit must move along an edge of $G$. The hunters win if  at any point the rabbit is shot. 
The hunting number of $G$, denoted $h(G)$, is the minimum $k$ for which $k$ hunters can win, regardless of the rabbit's moves. 
The complexity of computing $h(G)$ has been the longest standing open problem concerning the game and has been posed as an explicit open problem by several authors. 
The first contribution of this paper resolves this question by establishing that
computing $h(G)$ is NP-hard even for bipartite simple graphs. We also prove that the problem remains hard even when $h(G)$ is $O(n^{\epsilon})$ or when $n-h(G)$ is $O(n^{\epsilon})$, where $n$ is the order of $G$. Furthermore, we prove that it is NP-hard to additively approximate $h(G)$ within $O(n^{1-\epsilon})$. Finally, we give a characterization of graphs {with loops} for which $h(G)=1$ by means of forbidden subgraphs, extending a known characterization for simple graphs.
\end{abstract}

{\bf{Keywords:}}
Hunters and rabbit, Complexity, Inapproximability


\section{Introduction}

The hunters and rabbit game has been studied under several different names. Although we hold nothing against rabbits, we choose to use the terminology of hunters and rabbit, as it is the most widely adopted. \\
This game is played on an undirected graph $G$ with a positive integer $k$ representing the number of hunters. In each round or time step, the hunters shoot at $k$ vertices, while the rabbit occupies a vertex unknown to the hunters (until the rabbit is possibly shot). The rabbit can start the game in a given subset of vertices (usually all of them) and, if the rabbit is not shot, it must move to an adjacent vertex after each round. The rabbit wins if it can ensure that its position is never shot; otherwise, the hunters win.\\
As an example consider a complete graph on $n$ vertices: here $n-1$ hunters can shoot at the same $n-1$ vertices for two rounds and be sure the rabbit will be shot. On the other hand, on a path on $n>2$ vertices $v_1,...,v_n$, one hunter can win by
%shoot twice $v_2$ if $n=2$ or it 
 subsequently shooting at all vertices from $v_2$, to $v_{n-1}$ and then restart shooting backward $v_{n-1},...,v_2$. 
%: consider a $2$-coloring of the path, if the rabbit starts on a vertex whose color is the same as that of $v_2$, then it is shot in the forward shooting, otherwise it is shot in the backward shooting. Observe that hunters shooting need not follow graph edges. That is why, in the previous example, the hunter can shoot $v_{n-1}$ in two consecutive rounds.\\

The hunters and rabbit game was introduced in \cite{journals/combinatorics/BritnellW13} for the case $k=1$, where the authors show that one hunter wins on a tree if and only if it does not contain a $3$-spider ($H_1$ in Figure \ref{fig:forbid}) as a subgraph. Further, it was also shown that when one hunter can win, he can win in a number of rounds linear in the number of vertices. Very similar results were obtained also in \cite{HASLEGRAVE20141}.
%, as a subgraph, a $3$-spider, namely a tree obtained from a star with three leaves by subdividing each edge twice. It was also shown that when one hunter can win, he can make sure the rabbit is shot in a number of rounds linear in the number of vertices. Very similar results were obtained also in \cite{HASLEGRAVE20141}.
%Notice that, to win the game, the hunters must be at least as many as the minimum degree of the graph, while, if there is one hunter per vertex, they certainly win the game. 
The minimum number of hunters needed to win on a given graph $G$ is called the \textit{hunting number} of $G$, denoted $h(G)$. In \cite{ABRAMOVSKAYA201612} it is proven that the hunting number is upper bounded by the graph pathwidth plus $1$. %The hunting number is also studied in some  classes of graphs. For example in \cite{ABRAMOVSKAYA201612} 
It is also shown that the hunting number of an $(n\times m)$-grid is $\lfloor \frac{\min(n,m)}{2}\rfloor+1$ and that the hunting number of trees is $O(\log (n))$. On the other hand, there are trees for which the hunting number is $\Omega(\log (n))$ \cite{gruslys2015catching}. In \cite{BOLKEMA2019360} the hunting number of the $n$ dimensional hypercube is proven to be $1+\sum_{i=1}^{n-2}\binom{i}{\lfloor \frac{i}{2} \rfloor}$. Here the authors also show that $\max_{S \subseteq V(G)} \delta(D[S])$ {(i.e., the maximum through induced subgraphs of their minimum degree)} is a lower bound for the hunting number of any graph $G$. \\
In \cite{dissaux:hal-03995642} the authors provide a polynomial-time algorithm to determine the hunting number of split graphs. They also show that computing the hunting number on any graph is FPT parameterized by the size of a vertex cover. Moreover they prove that, if a monotone capture is required, the number of hunters must be at least the pathwidth of the graph and it is not possible to additively approximate the monotone hunting number within $O(n^{1-\epsilon})$. 

{A more general version of the hunters and rabbit problem  was also considered in \cite{ourpaper,benameur2024complexityresultscopsrobber} where the rabbit moves along the edges of a directed graph $D$ that might also contain loops.  
It is shown there that it is NP-hard to decide whether the hunting number of a digraph is $1$. Computing the hunting number is  proved to be FPT parametrized in some generalization of the vertex cover. When the digraph is a tournament, tractability is achieved with respect to the minimum size of a feedback vertex set. The hunting number is also proved to be less than or equal to $1$ + the directed pathwidth. An easy to compute  lower bound  is given by $\max_{S \subseteq V} \max(\delta^+(D[S],\delta^-(D[S])$ (i.e., the maximum through all induced subgraphs of the minimum indegree and the minimum outdegree). When a monotone capture is assumed,  the hunting number is proved to be greater than or equal to the directed pathwidth, while pathwidth plus $1$ is still a valid upper bound. Another result worth mentioning  from \cite{ourpaper} is related to the minimum number of shots (regardless of  the number of hunters) required to shoot the rabbit. It is proved that this number is easy to compute, and that the rabbit can always be shot before time step $n$ using this minimum number of shots. Some connections with the no-meet matroids of \cite{BENAMEUR2022}, as well as with the matrix mortality problem are also drawn in  \cite{benameur2024complexityresultscopsrobber}. }


The hunters and rabbit game falls within the broader category of cops and robber games, where different versions are defined based on factors such as the available moves for the cops, the robber's speed, and the robber's visibility to the cops. These kind of games are widely studied, for a review see the book \cite{Bontato}. The first cops and robber game was defined in \cite{Quillot} and \cite{NOWAKOWSKI1983235}. The difference between the hunters and rabbit and this game is that the cops must follow the edges of the graph and can always see the robber. Deciding whether $k$ cops can catch a robber in this version is EXPTIME-complete when $k$ is part of the input \cite{KINNERSLEY2015201}, but it is polynomial when $k$ is fixed.\\
Cops and robber games variants provide algorithmic interpretations of several graph (width) measures like treewidth \cite{SEYMOUR199322}, pathwidth \cite{Parsons1978PursuitevasionIA}, directed pathwidth \cite{Barat}, directed tree-width \cite{seymourdirtw}, DAG-width \cite{10.5555, bang2016dag}. These games have been intensively studied also due to their applications in numerous fields such as multi-agent systems \cite{AlejandroIsaza,Stern2019}, robotics \cite{chungrobots}, database theory \cite{GOTTLOB2003775}, distributed computing \cite{Nisse2019}.

\smallskip
\noindent\textbf{Contributions and paper organization.} \\
In this paper we deal with the hunters and rabbit game on undirected graphs and we also consider the case when those graphs contain loops. Our main result states that computing $h(G)$ is NP-hard. This confirms the sentiment emerging from the literature (e.g. in \cite{ABRAMOVSKAYA201612,dissaux:hal-03995642}). {In particular, we first provide a reduction from \textsc{3-partition} to {the problem of computing the minimum number of hunters, $h_S(G)$, required when the rabbit starts in a subset $S$}. We then provide a polynomial time reduction from computing $h_S(G)$ to computing $h(G)$ for graphs that may contain loops. We conclude by providing a reduction from $h(G)$ on graphs with loops to bipartite graphs.} We also prove that the problem remains hard even if $h(G)$ or $n-h(G)$ is as small as $O(n^{\epsilon})$, for any $\epsilon>0$. Approximating the hunting number of a graph within an additive error of $O(n^{1-\epsilon})$ is shown to be NP-hard too, for every $\epsilon >0$. Finally we extend, to graphs that can contain loops, the characterization from 
\cite{journals/combinatorics/BritnellW13,HASLEGRAVE20141}
of graphs $G$ such that $h(G)=1$.

The paper is organized as follows: notation and definitions are provided  in Section 2; in Section 3 we derive preliminary properties; in Section 4 we prove our main result and the other hardness results; finally, in Section 5, we characterize graphs (with loops) where one hunter wins; {the paper concludes with a few remarks in Section 6.}

\section{Notation and some terminology}
\label{sec:nota}
Let us start with some notation. $G=(V,E)$ is an undirected graph where $V(G):= V$ (resp. $E(G):=E$) denotes the set of vertices (resp. edges) of $G$. 
All graphs considered in this paper contain neither {isolated vertices} nor parallel edges.  One can then use $uv$ to denote an edge whose endpoints are $u$ and $v$. When it does not contain loops, the graph is said to be simple.
The number of vertices is generally denoted by $n(G):=|V|$, or simply $n$  when clear from the context.
Let $K_n$ be the complete graph on $n$ vertices without loops, while $\overline{K}_n$ denotes the complete graph with loops (so the number of edges is $n(n+1)/2$).
Given $A \subset V$, $G[A]$ is the subgraph of $G$ induced by $A$. Two disjoint subsets of vertices $A$ and $B$ are said to be \emph{fully connected} if each vertex of $A$ is adjacent to each vertex in $B$. 
We will use $[n]$ to denote the set of numbers $\{1,...,n \}$.

{The hunters and rabbit game is played on a graph $G=(V,E)$. We identify with $W_t \subseteq V$ the positions at which hunters are 
\emph{shooting} at time $t$,
 while  $R_t$ denotes the \emph{rabbit territory} (i.e., the set of possible positions of the {invisible} rabbit assuming that he was not yet shot). A \emph{hunter strategy} {$(W_t)_{t \geq 1}$} 
 %W_1, W_2, ..., W_T$ 
 is a  \emph{winning strategy}, if and only if, {there exists a finite $T$ such that}  $R_T = \emptyset$. {With a slight abuse of notation, such a winning strategy could be simply identified with the first $T$ sets $W_1,..., W_T$.}
 Observe that $(W_t)_{t \geq 1}$ %\ale{***here we are mixing finite and infinite sequences. I propose to keep everything finite but we mix also in the paper like proof of Lemma 2/3/4 } 
 is not a winning strategy, if and only if, there exists an \emph{escape walk} of the rabbit (i.e., $(v_t)_{t \geq 1}$ such that $v_t \notin W_t$, and $v_t v_{t+1} \in E(G)$, $\forall t \ge 1$).
If $R_t \cap A =\emptyset$, we say that $A$ is \emph{decontaminated} at time $t$. When {$k=\max_{t \geq 1} |W_t|$, we say that the strategy $(W_t)_{t \geq 1}$} uses $k$ hunters. The minimum integer $k$ such that there exist an integer $T$ and a  winning  strategy $W_1,...,W_T$ {such that $k=\max_{t \in [T]} |W_t|$} is called the \emph{hunting number} of $G$ and denoted by $h(G)$.  When the set of possible initial positions of the rabbit is restricted to some subset $S$, the hunting number is noted $h_S(G)$ (so $ h(G) = h_V(G)$).  Note that $R_1 = S \setminus W_1$. If $h(G) \leq k$, $G$ is said to be a \emph{$k$-hunterwin} graph. }
%{\ale{Alternative formulation: The hunters and rabbit game is played on a graph $G=(V,E)$. A hunter strategy is a sequence $W_1, W_2, ..., W_T$ with $W_t\subseteq V$ for $1\le t\le T$; a rabbit strategy or \emph{escape walk} is a walk $r_1...r_T$ such that $r_1\in V$, $r_tr_{t+1}\in E$ for every $1\le t\le T-1$ and $r_t \notin W_t$ for every $1\le t \le T$. We identify with $W_t$ the positions at which hunters are 
%\emph{shooting},
% while we indicate with $R_t$ the \emph{rabbit territory} (i.e. the set of vertices such that there exists an escape walk $r_1...r_t$, with $r_t\in R_t$): note that $R_1=S\setminus W_1$ and $W_t \cap R_t = \emptyset$ at any time step $t$. 
%A hunter strategy $W_1, W_2, ..., W_T$ is a  \emph{winning strategy} if and only if $R_T = \emptyset$. If $R_t \cap A =\emptyset$, we say that $A$ is \emph{decontaminated} at time $t$. When $k=\max_{t\in [T]} |W_t|$, we say that the strategy $W_1,...,W_T$ uses $k$ hunters. Note that a strategy using $|V|$ hunters is a winning one; the minimum integer $h$ such that there exist an integer $T$ and a  winning hunter strategy $W_1,...,W_T$ is called the \emph{hunting number} of $G$ with respect to $S$ denoted by $h_S(G)$. Let $\mathcal{P}_{S}$ be the problem of computing $h_S(G)$. When no assumption is made about the starting position (i.e. $S=V$), we denote the hunting number simply by $h(G)$, while $\mathcal{P}$ is the corresponding optimization problem.
%}
%Let $h_S(G)$ denote the minimum number of hunters required to shoot the rabbit assuming that the rabbit can only start at a subset $S$. Let $\mathcal{P}_{S}$ be the problem of computing $h_S(G)$.  Observe that $h_S(G) \leq |S|$.  
%When no assumption is made about the starting position (i.e., $S=V(G)$), we get the already mentioned  \emph{hunting number}  denoted by $h(G)$, while $\mathcal{P}$ is the corresponding optimization problem. \\
%At time $t \geq 1$, let $W_t$ be the set of positions at which hunters are 
%\emph{shooting},
 %while $R_t$ is the \emph{rabbit territory} (i.e., the set of possible positions of the rabbit assuming that he was not yet shot). We necessarily have $W_t \cap R_t = \emptyset$ at any time step $t$. When  $\mathcal{P}_{S}$ is considered, we can write that $R_1 = S \setminus W_1$.  A \emph{winning strategy} $W_1, W_2, ..., W_T$ (for some number $T$)  guarantees that $R_T = \emptyset $.  $(W_t)_{t \geq 1}$ is not a winning strategy, if and only if, there exists an \emph{escape walk} of the rabbit (i.e., $(v_t)_{t \geq 1}$ such that $v_t \notin W_t$, and $v_t v_{t+1} \in E(G)$, $\forall t \ge 1$).
%If $R_t \cap A =\emptyset$, we say that $A$ is \emph{decontaminated} at time $t$.}

%{The first reduction presented in the paper will be based on the \textsc{3-partition} problem. Remember that an instance of \textsc{3-partition} is a multiset $S$ of $n$ positive integers $\{a_1,...,a_n\}$ with $n=3m$ for which we aim to decide whether there is a partition $S_1,...,S_m$ of $S$ such that the sum of the elements in each $S_j$ equals $\beta = 1/m \sum_{i=1}^n a_i$.
%This problem is NP-hard even when the $a_i$ are bounded by a polynomial in $n$ and $\frac{\beta}{4} < a_i < \frac{\beta}{2}$ for $i=1,...,n$ \cite{gareyjohnson79}. Note that the latter condition implies that the $S_j$ are triplets.}


\section{Preliminary results}

Let $k$ represent the number of hunters ($|W_t| = k$, $\forall t \geq 1$) to be dealt with. From the definition of $h_S(G)$, we can ensure that if $k<h_S(G)$, then there are no  winning strategies using $k$ hunters if the rabbit starts at $S$. In other words, $R_t \neq \emptyset$ for any time $t$.  
The first lemma provides a reciprocal view that is mainly due to the undirected nature of the graph: using $k$ hunters with $k <h_S(G)$ and assuming that the rabbit can start anywhere (not only in $S$), then the rabbit territory will always intersect with $S$ (i.e., $S$ cannot be decontaminated). {Lemma \ref{nodeconaminationS} will be crucial to lift our hardness result  for computing $h_S(G)$ to hardness of computing $h(G)$ in general.}


\begin{lemma}\label{nodeconaminationS}
If $h_S(G) > k$, then any strategy using $k$ %cops 
{hunters} in $G$ is such that $R_{\tau} \cap S \neq \emptyset, \forall \tau \geq 1$.
\end{lemma}
\begin{proof}
Let  $(W_t)_{t \geq 1}$ be any strategy in $G$ using $k$ {hunters}. 
Let $\tau \geq 1$ be an integer and assume hunters shoot at $W_{\tau},...,W_1$ in this order: since $h_S(G) > k$, there must exist a rabbit walk $v_1,...,v_{\tau}$ with $v_1\in S$ that survives against $W_{\tau},...,W_1$ (i.e. $v_t\notin W_{\tau+1-t}$ for $t \in [\tau]$). But then $v_{\tau},...,v_1$ is a rabbit walk to $S$ that survives against $W_1,...,W_{\tau}$, namely  $v_{\tau+1-t}\notin W_t$ for $t \in [\tau]$ and $R_{\tau} \cap S \neq \emptyset$.%which is a contradiction.
\qed
\end{proof}


Let $G = (V,E)$ be a graph whose edge set might contain loops in addition to regular edges.   Let $B_G$ be the undirected  bipartite graph built from $G$ as follows: $V(B_G) = V \cup V'$ where $V'$ is a copy of $V$ and $E(B_G)$ contains edges $v'w$ and $w'v$ for any edge  $v w \in E$. A loop  $v v$ of  $E$ is then represented by one edge $v'v$ in $B_G$. $B_G$ can also be seen as the tensor product of $G$ with $K_2$. 
Observe that $B_G$ does not contain loops (see Figure \ref{fig:BC} for illustration). Next lemma states that $G$ and $B_G$ have the same hunting number. 

\begin{lemma}
$h(G) = h(B_G)$.
\label{lem:bipartite}
\end{lemma}
\begin{proof}
Let $W_1, W_2,..., W_T$ be a winning strategy in  $G$ using $h(G)$ hunters. A winning strategy $W^B$ of length $2T$ is built in $B_G$ as follows: for $1 \leq t \leq T$, let $W^B_t = \{v: v \in W_t \}$ if $t$ is odd and  $W^B_t = \{v': v \in W_t\}$ for even time $t$. For $T+1 \leq t \leq 2T$, we take $W^B_t = \{v: v \in W_{t-T}\}$ if $t$ is even and  $W^B_t = \{v': v \in W_{t-T}\}$ otherwise. If the rabbit was initially in $V$, then he will be shot during the first $T$ iterations, otherwise  this occurs between $T+1$ and $2T$. We consequently have $h(B_G) \leq h(G)$.\\
Conversely, if $(W^B_t)_{t \geq 1} $ is a winning strategy in $B_G$, then by simple projection on $V$ we get a winning strategy in $G$: let $W_t = \{v: v \in W^B_t\} \cup \{v: v' \in W^B_t\}$. This implies that $h(B_G) \leq h(G)$.
\qed\end{proof}



\begin{figure}[htbp]
\centering
\vspace{-15mm}
\includegraphics
[scale=0.30]{BC.pdf} \vspace{-15mm}
    \caption{Illustration of $B_G$ and $C_G^p$ (with $p = 2$)}
\label{fig:BC}
\end{figure}

For some number $p \geq 2$ and some graph $G=(V,E)$, let us build the graph $C_G^p$ as follows:  $p$ copies   $G^i = (V^i,E^i)_{1 \leq i \leq p}$ of  $G$ are considered; if some edge $vw \in E$ then $C_G^p$ contains also edges $v^i w^j$ for $1\leq i, j \leq p$ (see Figure \ref{fig:BC} for illustration).  
The graph $C_G^p$ can be seen as the tensor product of $G$ with $\overline{K}_p$. We show that $h(C_G^p)$ is simply $p \times h(G)$.

\begin{lemma}
$h(C_G^p) = p \times h(G)$.
\label{lem:C}
\end{lemma}
\begin{proof}
Consider a winning strategy $(W_t)_{t \geq 1}$ in $G$ using $h(G)$ hunters. We build a  strategy $(W^C_t)_{t \ge 1}$  in $C_G^p$ using $p \times h(G)$ hunters as follows: $W^C_t = \bigcup_{1 \leq i \leq p} \{v^i: v \in  W_t\}$. This strategy is obviously a winning one showing that $h(C_G^p) \leq p \times h(G) $.
%\ale{Shall we keep $C_G^p$ notation throughout the proof?} \wal{Yes, it was just mix of old and new notation...}
Assume that $h(C_G^p) <p \times h(G)$. Consider a winning strategy $(W^C_t)_{t\ge 1}$ using $h(C_G^p)$ hunters. Observe that at each time step $t$, there exists at least one index $i$ (denoted by $i(t)$) such that  $|W^C_t \cap V^i| < h(G)$. Consider the strategy $(W_t)_{t\ge 1}$ (in $G$) defined by $W_t = \{v: v^{i(t)} \in W^C_t\}$. 
Observe that $|W_t|<h(G)$ implying the existence of a rabbit {escape walk} $(v_t)_{t \geq 1}$ allowing him to survive the hunter strategy. The rabbit escape {walk} in $G$ can be transformed into an escape {walk} $(u_t)_{t \geq 1}$ in $C_G^p$ where  $u_t = v^{i(t)}_t$.  This leads to contradiction since $(W^C_t)_{t\ge 1}$ was assumed to be a winning strategy. Hence, $h(C_G^p)=p \times h(G)$. 
\qed\end{proof}

Given two graphs $G=(V(G),E(G))$ and $H = (V(H),E(H))$, let  
$G \nabla H$ be  {the join} graph obtained by considering the union of $G$ and $H$ and fully connecting   $V(G)$ and $V(H)$: $V(G \nabla H) = V(H) \cup V(G)$ and $E (G \nabla H) = E(G) \cup E(H) \cup \{ uv: u \in V(G), v \in V(H)  \}$. The next lemma states that $h(\cdot)$ is superadditive with respect to $\nabla$ and provides  an obvious upper bound. %\ale{***Might it be $ h(G \nabla H) = \min \left(h(G) + n(H), h(H) + n(G) \right)$   ?}% Assume wlog $H$ is not ``definitively decontaminated" before $G$ and let $t$ be the last time $V(H)\cap R_t \neq \emptyset$, since $V(G)\cap R_{t+1} \neq \emptyset$, then from $t+1$ until $G$ (and everything else) is decontaminated, $n(H)$ hunters must be on $H$. Moreover $R_t\cap V(G)=V(G)\setminus W_t$, therefore other $h(G)$ hunters are needed on $G$ to decontaminate it.}
%\wal{Possible...  If $h(G) + h(H) = n(G) + n(H)$ then your conjecture is true. Assume then that $h(G) + h(H) <n(G) + n(H)$, then from the upper bound we know that $ h(G \nabla H) < n(G) + n(H)$ implying that the rabbit cannot be captured in the first time slot, so  the capture time $T$ satisfies $T \geq 2$. It is also clear that at time $T - 1$ we cannot simultaneously have $R_{T-1} \cap V(G) \neq \emptyset$ and  $R_{T-1} \cap V(H) \neq \emptyset$  since it is not possible to shoot the rabbit in next time slot. So we know that one graph will be "decontaminated before the other". Assume that this will be the case for $H$, so there exists some time $t'$ such that $R_{t'} \cap V(H) \neq \emptyset$ and $R_{t\geq 1+t'} \cap V(H) = \emptyset$.  We can say that $R_{1+t' \leq  t\leq T - 1} \cap V(G) \neq \emptyset$...but the problem is that we might have $R_{t'} \cap V(G) = \emptyset$, so at time $t'+1$ we might not need to put $n(H)$ hunters in $V(H)$ and we have "free" hunters (more than $h(G)$ that are used in the $G$ side at time $t'+1$)....  }

\begin{lemma}
$h(G) + h(H)  \leq h(G \nabla H) \leq \min \left(h(G) + n(H), h(H) + n(G) \right)$.
\label{lem:comp}
\end{lemma}

\begin{proof}
Consider a winning strategy $(W_t)_{t \geq 1}$ in $G$ using $h(G)$ hunters. Then $(V(H) \cup W_t)_{t \ge 1}$ is obviously a winning strategy in  $G \nabla H$ showing that  $h(G) + n(H)$ is an upper bound for $h(G \nabla H)$. By symmetry, $h(H) + n(G)$ is also an upper bound.  The lower bound is proved using exactly the same technique already used in the proof of Lemma \ref{lem:C}. More precisely, given any strategy
$(W_t)_{t \geq 1}$ using strictly less than $h(G)+h(H)$ hunters, we have either $|W_t \cap V(G)| <h(G)$ or  $|W_t \cap V(H)| <h(H)$ allowing to build a rabbit escape strategy.  
\qed\end{proof}

Observe that Lemma \ref{lem:comp} implies that $h(G \nabla \overline{K}_k) = h(G) + k$ while two inequalities can be obtained if $H = K_k$:   
$h(G) + k -1 \leq h(G \nabla {K}_k) \leq h(G) + k$.
{Note that $h(K_k \nabla {K}_k)=h(K_{2k})=2k-1=h(K_k)+n(K_k)>h(K_k)+h(K_k)$ showing that the upper bound is sharp. 
%(clearly the two copies of $K_k$ are considered as distinct).
On the other hand, let $G(a,p)$ be a graph obtained as the union of  {$\overline{K}_{a+1}$ and a path $v_1...v_p$  on $p \geq 2$ vertices starting at the clique (so $v_1 \in \overline{K}_{a+1}$). Observe that $n(G(a,p)) = a+p$ and $h(G(a,p)) = a+1$.}   Now consider the join of two distinct copies of $G(a,p)$: we have $h(G(a,p) \nabla G(a,p))\le \max(2a+2,a+p+2)$. Indeed, two hunters can decontaminate the path on the first copy, while $a+p$ hunters {are shooting at} 
%remain stationed in 
the second copy, then, for one step, $2(a+1)$ hunters can cover the clique
%and $v_1$ 
on both copies.
At this point the %robber's 
{rabbit's} territory is reduced to $\{v_2,...,v_p\}$ on the second copy. From there $a+p$ hunters {shoot at} 
%stay in 
the  first copy, while two hunters decontaminate the path in the second copy. When {$2 \leq p\le a$, we have $ h(G(a,p) \nabla G(a,p))=2a+2=h(G(a,p))+h(G(a,p))<h(G(a,p))+n(G(a,p))$, implying that the lower bound is also  sharp.} 


\section{Complexity}

%For our reduction, we use the $3$-partition problem.

%\Pb{$3$-partition}{A multiset $\mathcal{S}$ of $n$ natural numbers $\{a_1,\ldots,a_n\}$ with $n=3m$ for some $m\in \mathbb{N}$.}{Question}{Does there exist a partition $S_1,\ldots,S_m$ of $\mathcal{S}$ such that sum of elements in each $S_j$ equals $\beta = \frac{\sum_{i=1}^{n} a_i}{n}$?}


We start with a reduction from \textsc{3-partition} to the problem of computing $h_S(G)$. Then the latter is reduced to the problem of computing $h(G)$ in a graph having loops. Finally, using Lemma \ref{lem:bipartite}, we deduce that computing the hunting number is NP-hard in a bipartite graph.\\
Remember that an instance of \textsc{3-partition} is a multiset $S$ of $n$ positive integers $\{a_1,...,a_n\}$ with $n=3m$ for which we aim to decide whether there is a partition $S_1,...,S_m$ of $S$ such that the sum of the elements in each $S_j$ equals $\beta = 1/m \sum_{i=1}^n a_i$.
This problem is NP-hard even when the $a_i$ are bounded by a polynomial in $n$ and $\frac{\beta}{4} < a_i < \frac{\beta}{2}$ for $i=1,...,n$ \cite{gareyjohnson79}. Note that the latter condition implies that the $S_j$ are triplets.}


\begin{figure}[htbp]
    \centering
    \vspace{-5mm}
    \includegraphics[scale=0.4]{ReductionRestricted.pdf}
    \caption{Illustration for the construction used in the Proof of Proposition~\ref{P:hard}. Here, each block of vertices other than $U$ is an independent set and $U$ is a clique. Further, $Y,Z_1,Z_2,Z_3, V_1,\ldots, V_{2m+3}$ contain $\beta$ vertices each, $U$ contains $\beta+2$ vertices, $X_i$ and $X'_i$ contain $a_i$ vertices (for $i\in [n]$). Finally, whenever two blocks, say $A$ and $B$, are illustrated to be connected by a red connection, each vertex of $A$ has an edge with each vertex of $B$ in $G$ (i.e., $A$ and $B$ are fully connected).}
    \label{fig:constrainedHardness}
\end{figure}

\begin{proposition}\label{P:hard}
It is  NP-hard to compute $h_S(G)$.
\end{proposition}
\begin{proof}
Let $\mathcal{S}$ be an instance of \textsc{3-partition}. We show that   $\mathcal{S}$ admits a \textsc{3-partition} if and only if $h_S(G) = \beta$ where $G$ and $S$ are described below and shown in Figure~\ref{fig:constrainedHardness}. The instance $\mathcal{S}$ defined by the numbers $(a_i)_{i \in [n]}$ is chosen as described above. 
%As mentioned in Section \ref{sec:nota}, we can assume that the numbers $(a_i)_{i \in [n]}$ defining the \textsc{3-partition} instance are polynomially bounded in $n$ and are between $\beta /4$ and $\beta / 2$ with $\beta = 1/m \sum_{i=1}^n a_i$ and $m = n/3$. 

\smallskip
\noindent\textbf{Construction of $G$.} Let $Y,Z_1,Z_2, Z_3,V_1,\ldots,V_{2m+3}$ each be a set of $\beta$ independent vertices. Further, let $U$ be a set of $\beta+2$ vertices that induces a clique in $G$. Finally, let $X_1,\ldots,X_n,X'_1,\ldots,X'_n$ be sets of independent vertices such that $|X_i|=|X'_i|=a_i$. %Now, for $i\in [2m+2]$, connect each vertex of $V_i$ to every vertex of $V_{i+1}$ (such that $G[V_i\cup V_{i+1}]$ induces a complete bipartite graph with both partitions containing $\beta$ vertices). Similarly, \wal{fully} connect 
% %each vertex of 
% $X_i$ 
% %to every vertex in 
% to $X'_i$, and 
% %connect each vertex  of 
% $Z_2$ 
% %to every vertex of 
% to $Z_3$. Moreover, 
% %connect each vertex in 
% $Z_3 \cup Z_1 \cup V_{2m+3}$ 
% %to every vertex of
% and $H$ \wal{are fully connected}, and for $i\in [n]$, 
% %connect each vertex of 
% each $X'_i$ 
% %to every vertex in 
% \wal{is fully connected to} $H$. Finally, let $S = Z_1\cup Z_2\cup V_1 \cup X_1 \cup \cdots \cup X_n$ be the allowed starting positions of the rabbit.


Now, for $i\in [2m+2]$, fully connect $V_i$ to $V_{i+1}$ (i.e., $G[V_i\cup V_{i+1}]$ induces a complete bipartite graph with both partitions containing $\beta$ vertices). Similarly, fully connect $X_i$ to $X'_i$, and $Z_2$ to $Z_3$. Moreover, $Z_3 \cup Z_1 \cup V_{2m+3}$ and $U$ are fully connected, and for $i\in [n]$, each $X'_i$ is fully connected to $U$. Finally, let $S = Z_1\cup Z_2\cup V_1 \cup X_1 \cup \cdots \cup X_n$ be the allowed starting positions of the rabbit. %\textcolor{green}{Finally, let $S_j$ denote the set  $\bigcup_{i | a_i \in S_j}X_i'$ of vertices.}


In one direction, suppose $\mathcal{S}$ admits a \textsc{3-partition} $S_1,\ldots,S_m$. Then, we claim that the hunter strategy $Z_1, Z_3, Y, S_1, Y,  S_2, Y, S_3, Y,\ldots,S_m, V_{2m+3}, V_{2m+2},\ldots, V_2$ is a winning one  (here with a slight abuse of notation we are indicating with $S_j$ the set $\bigcup_{i | a_i \in S_j}X_i'$). 
To ease the exposition, we provide a case by case analysis distinguished by the starting positions of the rabbit.
\begin{enumerate}
    \item The rabbit starts in $Z_1$: The rabbit is {shot} in the first round since $W_1 = Z_1$.
    \item The rabbit starts in $Z_2$: 
    %In this case, observe that $R_1 = Z_2$ and since $W_2 = Z_2$, 
    The rabbit is {shot}  in the second round.
    \item The rabbit starts in $X_i$ for some $i\in [n]$:  In this case, first we establish that the rabbit will be restricted to vertices in $X_1,\ldots,X_n, X'_1,\ldots,X'_n$ until round $2m+{2}$. To this end, observe that if the rabbit needs to leave these vertices, it needs to reach a vertex in $Y$, and it can only do so {at an odd time step greater than $1$. But this is not possible since $W_{t}=Y$ for odd rounds $3\le t \le 2m+1 $.\\%from then on, the hunters shoot at every vertex in $Y$ on every odd time step for $m$ odd turns (for turns $W_3, W_5,W_7, \ldots, W_{2m+1}$),    the rabbit is restricted to $X_1,\ldots,X_n, X'_1,\ldots,X'_n$ until round $2m+1$. 
    Note that $G[X_1\cup \ldots\cup X_n\cup X'_1\cup \ldots\cup X'_n]$ is bipartite, thus $R_t \subseteq X_1\cup \ldots \cup X_n$ when $t$ is odd and $R_t \subseteq X_1'\cup \ldots\cup X_n'$ when $t$ is even, until round $2m+2$. %the rabbit is restricted to vertices in $X_1,\ldots,X_n$ at odd time steps, and to vertices in $X'_1,\ldots,X'_n$ at even time steps, until round $2m+2$. 
    Furthermore, observe that when $W_t\supseteq X'_i$, the set $X_i\cup X'_i$ becomes decontaminated and remains so for all subsequent rounds (until round $2m+2$)} unless the rabbit moves to some vertex in $Y$, which is not possible.
   Finally, let the subset $S_j$, $j\in [m]$, contain numbers $a_p,a_q,$ and $a_r$. Then, in the round $2j+2$, hunters shoot at all vertices in $X'_p,X'_q,X'_r$, and hence decontaminate
    $X_p\cup X'_p\cup X_q\cup X'_q \cup X_r \cup X'_r$ { for all subsequent rounds (until round $2m+2$)}. Since $S_1,\ldots, S_m$ form a partition, after $2m+2$ rounds all vertices in $X_1,\ldots,X_n, X'_1,\ldots,X'_n$ will be decontaminated and as noted above, the rabbit is restricted to only these vertices for all these rounds. Hence, the rabbit gets shot. 
    
    \item The rabbit starts in $V_1$: {Observe that the induced graph $G[\bigcup_{i \in[2m+3]} V_{i}]$ is bipartite, therefore, since $R_1=V_1$ and $W_{2m+3} = V_{2m+3}$, we have $R_{2m+3} = \bigcup_{i \in[m+1]} V_{2i-1} $. }%after $2m+2$ rounds (in the $2m+3$rd time step), the rabbit will be restricted to the graph induced by vertices in $V_1,\ldots,V_{2m+3}$.
    %Since $W_{2m+3} = V_{2m+3}$ and the induced graph $G[\bigcup_{i \in[2m+3]} V_{i}]$ is bipartite, the rabbit territory at time $2m+3$ is given by $R_{2m+3} = \bigcup_{i \in[m+1]} V_{2i-1} $.
    Then at time $2m+4$, we have $W_{2m+4} = V_{2m+2}$ and  $R_{2m+4} = \bigcup_{i \in[m]} V_{2i}$. A simple induction on $t$ shows that, for $1\le t\le m+1$, $R_{2m + 2t} = \bigcup_{i \in[m-t+2]} V_{2i}$ and
    $R_{2m + 2t + 1} = \bigcup_{i \in[m-t+2]} V_{2i-1}$. Therefore, {$R_{4m+3} =V_1$,} $W_{4m+4} = V_2$ and thus $R_{4m+4} = \emptyset$.
\end{enumerate}

In the other direction, suppose $\mathcal{S}$ does not admit a \textsc{3-partition}. We will show that the hunters do not have a winning strategy using only $\beta$ hunters. Assume, by contradiction, that such a winning strategy exists. We begin by observing that once the rabbit reaches a vertex in $U$ {(i.e., $R_{t}\cap U \neq \emptyset$ for some $t>0$)}, then the rabbit can never be shot since the hunting number of $G[U]$ is greater than $\beta$. {Thus, to complete our proof, we only need to show that for any strategy of $k$ hunters the rabbit has a walk that ensures $R_{t}\cap U \neq \emptyset$ for some $t>0$. Furthermore, {observe that} all $\beta+2$ vertices of $U$ are twins and there are at most $\beta$ hunters, {thus} if $R_{t-1}\cap (Y \cup Z_3 \cup Z_1 \cup V_{2m+3}) \neq \emptyset$ for some $t>1$, then $R_{t}\cap U \neq \emptyset$.}
As  a consequence, we can safely assume that $W_1=Z_1$ and $W_2={Z_3}$ (otherwise the rabbit can reach $U$ in time step 2 and 3, respectively). Similarly, $W_3 = Y$, otherwise the rabbit has an escape strategy by starting on some $X_i$, moving to $X_i'$ in the second time step, then to some $v \in Y \setminus W_3$ in the third time step, and finally moving to $U\setminus W_4$ in the fourth time step.\\  
Let $T$ be the first time step such that $R_T \cap \bigcup_{i \in [n]} (X_i \cup X'_i) = \emptyset $.  For each odd time step $t$ between $3$ and $T$, $W_t = Y$ holds (otherwise the rabbit can reach $U$ in the next time step). Observe that $T$ is necessarily an even number. {Moreover, observe that since hunters are shooting at $Y$ in every odd time step, to decontaminate $X_i\cup X'_i$ the hunters must shoot at every vertex in $X'_i$ in some even round.} Furthermore, since hunters were able to decontaminate $\bigcup_{i \in [n]} (X_i \cup X'_i)$, they need to shoot at vertices in $\bigcup_{i \in [n]}  X'_i$ for at least $m+1$ (even) time steps. This is obviously due to the non-feasibility of the \textsc{3-partition} instance implying that it is not possible to cover all vertices of $\bigcup_{i \in [n]}  X'_i$ with only $m$ subsets of size $\beta$. For $ t \leq T$, let $l_t$ be the largest index such that $R_t \cap V_{l_t} \neq \emptyset$ (i.e., the index of the lowest $V_i$ in Figure \ref{fig:constrainedHardness} that is not decontaminated {at time $t$}). We clearly have $l_1 = 1$, $l_2 = 2$ and $l_3 = 3$.  Observe that if at some even time $t$ the hunters shoot at $V_{l_{t-1}+1}$, then $l_t = l_{t-1} - 1$. 
Otherwise, if the hunters shoot at other vertices ({i.e., $W_t \neq V_{l_{t-1}+1}$}), for example, in  $\bigcup_{i \in [n]}  X'_i$, then $l_t = l_{t-1} + 1$. If $t$ is odd,  $W_t = Y$ implying that $l_t = l_{t-1} + 1$.  Then, if we consider two consecutive time slots $t$ and $t+1$, we either have $l_{t+1} = l_{t-1}$ or $l_{t+1} = l_{t-1} + 2$.  
Since hunters have to shoot at $\bigcup_{i \in [n]}  X'_i$ for at least $m+1$  time steps, $l_{t+1}-l_{t-1}$ increases by $2$, at least $m+1$ times (for some odd $t$). Since $T$ is even and  hunters are shooting at (even partially)  $\bigcup_{i \in [n]}  X'_i$ at time $T$,  $l_{t+1}-l_{t-1}$ increases by $2$, at least $m$ times between $3$ and $T-2$. One can then write that {$l_{T-2} -l_2 = (l_{T-2} - l_{T-4}) + (l_{T-4}-l_{T-6})+\cdots + (l_{4} - l_2) \geq 2m$} implying that $l_{T-2} \geq 2m+2$ and $l_{T-1} \geq 2m+3$ (since $W_{T-1} = Y$). The rabbit can then reach $U$ at time $T$ implying that the hunter's strategy was not a winning one.  
\qed
\end{proof}

\begin{figure}[htbp]
\centering
\vspace{-6mm}
\includegraphics
[scale=0.25]{reduc2.pdf} \vspace{-5mm}
    \caption{Building $H$ from $G$ (Proposition \ref{pro:loop}): $1 \leq k \leq |S|$, $|A|=n-k$, $|B|=k$, $|C|=2k$, $H[A]=\overline{K}_{n-k}$, $H[B]=\overline{K}_{k}$ and $H[C]=\overline{K}_{2k}$ are complete graphs with loops.}
\label{fig:reduc2}
    \end{figure}

\begin{proposition}
It is NP-hard to compute $h(G)$ in a graph with loops.
\label{pro:loop}
\end{proposition}
\begin{proof}
We prove the result using a reduction from the problem of computing $h_S(G)$.  
Consider a graph $G=(V,E)$ and a subset  $S \subset V$ representing the possible initial positions of the rabbit. Let $n=|V|$ and  let $k$ be a number satisfying $1 \leq k \leq |S|$. We build a graph $H$ as follows. The graph $H$ contains $G$ as a subgraph in addition to $3$ complete subgraphs with loops: {$H[A]=\overline{K}_{n-k}$, $H[B]=\overline{K}_{k}$ and $H[C]=\overline{K}_{2k}$}. The set $A$ is fully connected to $B$, $C$ and $S$  while $B$ is fully connected to $V$ (and $A$) (see Figure \ref{fig:reduc2}). 
%The number of vertices of $H$ is then equal to $2n + 2k$.   
Observe that {$H[A \cup C]=\overline{K}_{n+k}$}  implying that $h(H) \geq n + k$. We claim that $h(H) = n+k$ if and only if $h_S(G) \leq k$. 
\begin{claim}
$h_S(G) \leq k$, if and only if $h(H) = n+k$.
\end{claim}
\begin{proof}
Assume that $h(H) = n +k$ and $h_S(G) > k \geq 1$. Let $W_1,...,W_T$ be a winning strategy in $H$ using $n+k$ hunters.\\ 
Let us first assume that there exists $t \geq 1$ such that $R_t \cap (A \cup B)  \neq \emptyset$.  Let $\delta$ be the last $t$ such that $R_t \cap (A \cup B)  \neq \emptyset$. 
Then, $W_{\delta + 1} \supset (A \cup B) $ holds, {implying $|W_{\delta + 1} \cap (V\cup C)| \leq k$}. \\
If $R_{\delta} \cap A \neq \emptyset$, then $R_{\delta+1} \cap C \neq \emptyset$ and $R_{\delta+1} \cap S \neq \emptyset$ since $|S| >k$ and $|C| >k$.  Using that $S$ (and even $V$) is connected to $B$ and $C$ is connected to $A$, we deduce that $W_{\delta + 2}$ should also contain $ A \cup B$ {and $R_{\delta+2} \cap C \neq \emptyset$}. By induction on $t \geq \delta + 1$, as long as $R_t \cap V \neq \emptyset$,  we  have $W_{t+1} \supset A \cup B$ and $R_{t+1} \cap C \neq \emptyset$.
Since the game is supposed to end,  
there exists some $T'$ between $\delta+1$ and $T$ such that $R_{T'} \cap V = \emptyset$. {Observe that between rounds $\delta+1$ and $T'$ at most $k$ hunters can shoot at vertices in $V$, while $R_{\delta}$ contains a vertex (of $A$) fully connected with $S$. Therefore, by considering the game played between rounds $\delta+1$ and $T'$ restricted to $V$, we obtain $h_S(G)\le k$, which is a contradiction.\\ }%This implies that} the hunters were able to  {decontaminate $V$} using at most $k$ hunters while the rabbit entered into $S$ (from $A$) at time $1+\delta$. This leads to contradiction since $k <h_S(G)$.\\
Assume now that $R_{\delta} \cap B \neq \emptyset$, then $R_{\delta + 1} \cap S \neq \emptyset$ implying that $W_{\delta + 2} \supset (A \cup B) $ (therefore,  $|W_{\delta + 2} \cap V| \leq k$). Since the rabbit {can enter} $V$ (from $B$) at time $1 + \delta$ and $k <h_S(G)$, we can deduce from Lemma \ref{nodeconaminationS} that $R_{\delta + 2} \cap S \neq \emptyset$.
{A simple} 
 induction on $t \geq \delta + 1$ {shows that we will always  have $W_{t+1} \supset A \cup B$ and $R_{t+1} \cap S \neq \emptyset$ implying that} the game will never end. \\
 {Note that a crucial fact in the proof of the first (resp. second) case is that the rabbit {can enter} $S$ (resp. $V$) at time $1+\delta$, and at most $k <h_S(G)$ hunters can {shoot at vertices of} $V$ from that time on.\\} Let us now assume that $\delta $ does not exist which is equivalent to say that $R_t \cap (A \cup B) = \emptyset$ $\forall t \geq 1$. We consequently have $W_1 \supset (A \cup B)$, $R_1 \cap V \neq \emptyset$ and $R_1 \cap {S \neq \emptyset}$. %C \neq \emptyset$. 
 The situation is then similar to the previous case where we had $R_{\delta} \cap B \neq \emptyset$ since the rabbit can be at any vertex of $V {\setminus W_1}$ at time $1$. The same induction on $t$ shows that $W_{t+1} \supset A \cup B$ and $R_{t+1} \cap S \neq \emptyset$ implying that the game is endless. \\
Finally, let us now show that $h(H) = n+k$ when $k \geq h_S(G)$. This  can be done by providing a winning strategy. A possible one starts with $W_1 = B \cup V$ leading to $R_1 = A \cup C$.  Assume that $W'_1, W'_2,..., W'_p$ is a winning strategy (for some time $p$) allowing to {shoot} the rabbit in $G$ if he starts at $S$. We can then take $W_2 = W'_1 \cup A \cup B$, $W_3 = W'_2 \cup A \cup B$, etc. So $W_{p+1} =W'_p \cup A \cup B$. At time $p+1$ we have $R_{p+1} = C$. Therefore, by setting   $W_{p+2} = A  \cup C$, the whole graph is decontaminated. 
\qed\end{proof}
The claim immediately proves the result since one can compute $h_S(G)$ by  computing the {hunting number} of $O(log(n))$ graphs of type $H$ for different values of $k$. 
\qed\end{proof}

We are now able to state the main complexity result. 

\begin{theorem}
It is NP-hard to compute $h(G)$ in a  bipartite graph.
\label{th:main}
\end{theorem}
\begin{proof}
This is a consequence of Lemma \ref{lem:bipartite} and Proposition  \ref{pro:loop}.
\qed\end{proof}

%\section{Inapproximability}

Let us focus now on the existence of polynomial-time approximation algorithms with additive guarantees. 
%Let $n = |V|$. 
An obvious $O(n)$ additive guarantee is given by the upper bound $n$ for the {hunting number}.  We prove that  it is not possible to do much better than $O(n)$.  
\begin{theorem}
It is NP-hard to additively approximate $h(G)$ within $O(n^{1 - \epsilon})$ for any constant $\epsilon >0$. \label{th:add}
\end{theorem}
\begin{proof}
Assume that we have a polynomial-time approximation algorithm with an $O(n^{1 - \epsilon})$ additive guarantee. Given any graph $G$, let us build the graph $C_G^p$ with $p= \lceil n^{2\frac{1-\epsilon}{\epsilon}}\rceil $. Using the approximation algorithm, we get an upper bound $u$ satisfying inequalities $h(C^p_G) \leq u \leq h(C^p_G) + O({(np)}^{1-\epsilon})$. From  Lemma \ref{lem:C}, we know that $h(C^p_G) = p \times h(G)$ implying that $ \frac{u}{p} -\frac{1}{p} O({(np)}^{1-\epsilon}) \leq h(G) \leq \frac{u}{p}$.  Using that $p= \lceil n^{2\frac{1-\epsilon}{\epsilon}} \rceil$ leads to $\frac{u}{p} + O(\frac{1}{n^{1-\epsilon}}) \leq   h(G)  \leq  \frac{u}{p}$. Since $h(G)$ is integer and $O(\frac{1}{n^{1-\epsilon}})$ is negligible,
%\ale{***Shall we say that $G$ should be big enough? \wal{that is obvious since we talk about big O}}
we get a polynomial-time algorithm to compute $h(G)$ (the construction is obviously polynomial for constant $\epsilon$). Theorem \ref{th:main} allows to conclude.
\qed\end{proof}

Since graphs for which $h(G) = 1$ or $h(G)= n$ are well characterized, one might expect polynomial-time algorithms for either small values of $h$ or large values of $h$. The next theorem states that this is not the case {when {either} $h$ {or} $n-h$ are upper bounded by a (small) power of $n$}.  

\begin{theorem}
It is NP-hard to compute $h(G)$ for simple graph instances where  $n - h(G) = O(n^{\epsilon})$ (resp. $h(G) = O(n^{\epsilon})$) for any constant $\epsilon > 0$.
\end{theorem}
\begin{proof}
Given a simple graph $G$ with $n = |V(G)|$, take $k$ so that $n \le k^{\epsilon}$ and  consider the graph $G \nabla {K}_k$.  Notice that $G \nabla {K}_k$ does not contain loops.  
From Lemma \ref{lem:comp}, we have $h(G)+k-1 \leq h(G \nabla {K}_k)  \leq  h(G) +k$ implying that {$h(G \nabla {K}_k) -k \leq  h(G) \leq  h(G \nabla {K}_k) -k+1$.} Furthermore $n(G \nabla {K}_k) - h(G \nabla {K}_k) \leq  (n+k)- h(G) -k+1\le n \le (n+k)^{\epsilon}$, therefore  $n(G \nabla {K}_k) - h(G \nabla {K}_k)$ is $O(n(G \nabla {K}_k)^{\epsilon})$.   A polynomial-time algorithm  computing the {hunting number} if  {$n-h=O(n^{\epsilon})$, applied on $G \nabla {K}_k$,} can  be used to get a lower bound and an upper bound for $h(G)$, whose difference   is $1$. In other words, we have a constant additive  approximation for $h(G)$ contradicting Theorem \ref{th:add}. \\
To prove NP-hardness for instances where $h = O(n^{\epsilon})$, one can start from a graph $G$ and add a stable set of size $\Omega(n^{\frac{1}{\epsilon}})$ to get a new graph $H$ for which $h(G) = h(H) = O(n(H)^{\epsilon})$.\qed\end{proof}



\section{Characterization of $1$-hunterwin graphs with loops}

\begin{figure}
    \centering
    \vspace{-5mm}
    \includegraphics[scale=0.25]{forbid.pdf}
    \caption{The $4$ forbidden graphs $H_1$, $H_2$, $H_3$ and $H_4$}
    \label{fig:forbid}
\end{figure}
It is possible to linearly characterize graphs for which the {hunting number} equals 1 using the characterization in \cite{journals/combinatorics/BritnellW13,HASLEGRAVE20141} and Lemma \ref{lem:bipartite}. Nonetheless we provide a characterization by means of forbidden subgraphs extending the one in \cite{journals/combinatorics/BritnellW13,HASLEGRAVE20141} for loopless graphs. {We will prove that  cycles, connected loops and the $4$ graphs $H_1$, $H_2$, $H_3$ and $H_4$ shown on Figure \ref{fig:forbid} are precisely the forbidden subgraphs in a 1-hunterwin graph with loops.  Notice that $H_1$ is the $3$-spider graph used in \cite{journals/combinatorics/BritnellW13,HASLEGRAVE20141} to characterize 1-hunterwin acyclic simple graphs.}

\begin{theorem}
A graph $G$ is $1$-hunterwin if and only if it does not contain cycles,  two connected loops, 
{ $H_1$,  $H_2$,  $H_3$ or $H_4$.}
%a path $c_1b_1a_1xa_2b_2c_2$ and (a  path $xa_3b_3c_3$ or a path $xa_3b_3$ with a loop on $b_3$ or a path $xa_3$ with a loop on $a_3$ or a loop on $x$).
\end{theorem}
\begin{proof}
First of all let us apply the construction of Lemma \ref{lem:bipartite} and obtain the graph $B_G$. We will prove that $G$ contains a forbidden subgraph among those listed above if and only if $B_G$ contains a cycle or a $3$-spider.\\
If $G$ contains a cycle $v_1v_2...v_kv_1$, then $B_G$ contains the cycle $v_1v_2'...v_k'v_1$ if $k$ is even  %($v_1v_2'...v_kv_1$ 
{and $v_1 v_2'...v_k v'_1 v_2 ... v'_k v_1$ }
if $k$ is odd.\\
If $G$ contains two loops on $v_1$ and $v_k$ and a path $v_1...v_k$, then $B_G$ contains the cycle $v_1v_2'...v_k'v_k...v_2v_1'v_1$ if $k$ is even and $v_1v_2'...v_kv_k'...v_2v_1'v_1$ if $k$ is odd.\\
If $G$ contains paths $c_1b_1a_1xa_2b_2c_2$ and $xa_3b_3c_3$ {(i.e., $H_1$),} then $B_G$ will contain the $3$-spider formed by $c'_1b_1a'_1xa'_2b_2c'_2$ and $xa'_3b_3c'_3$. If $G$ contains paths $c_1b_1a_1xa_2b_2c_2$ and $xa_3b_3$ with a loop on $b_3$ {(i.e., $H_2$),} then $B_G$ contains the $3$-spider formed by $c'_1b_1a'_1xa'_2b_2c'_2$ and 
%$xa'_3b_3b'_3a_3$
{$xa'_3b_3b'_3$}. If $G$ contains paths $c_1b_1a_1xa_2b_2c_2$ and $xa_3$ with a loop on $a_3$ {(i.e., $H_3$),} then $B_G$ contains the $3$-spider formed by $c'_1b_1a'_1xa'_2b_2c'_2$ and $xa'_3a_3x'$. If $G$ contains paths $c_1b_1a_1xa_2b_2c_2$ and a loop on $x$ {(i.e., $H_4$),} then $B_G$ contains the $3$-spider formed by $c'_1b_1a'_1xa'_2b_2c'_2$ and $xx'a_1b'_2$.\\
Now suppose $B_G$ has a cycle or a $3$-spider $S$. If $S$ contains two edges $aa',bb'$, then $S$ contains a path joining at least two of the endpoints of the above edges. Such a path projects into a walk joining $a$ and $b$ in $G$, implying that $G$ contains two connected loops. We can thus suppose that $S$ contains at most one edge of the form $aa'$. If $S$ does not contain two copies $v,v'$, then its projection on $G$ is a cycle (resp. $3$-spider), since $S$ is a cycle (resp. $3$-spider). If $S$ contains two copies of the same vertex, let $v,v'$ be a pair of such copies that are closest in $S$: if $vv'$ is not an edge of $S$, then a shortest path of $S$ joining $v$ and $v'$ has length at least 3 and does not contain two copies of the same vertex, thus it projects into a cycle of $G$. We can thus assume in what follows that $S$ contains an edge $vv'$ and no other edge of this form. \\
If $S$ is a cycle, let $S'$ be the subpath of $S$ joining $v$ and $v'$ having length at least 3: if $S'$ does not contain copies of the same vertex, then it projects into a cycle of $G$; if $S'$ contains copies of the same vertex, let $w,w'$ be a pair of such copies that are closest in 
%$S\\\\
{$S'$}: there must exist a path in 
%$S$ 
{$S'$}
joining $w$ and $w'$ of length at least 3 not containing copies of the same vertex, this path projects into a cycle of $G$.\\
If $S$ is a $3$-spider, let $P$ be the (possibly empty) path in $S$ joining {the center of the spider $x$ and the endpoint of the edge $vv'$ closest to $x$ (note that the edge $vv'$ does not belong to $P$)}. {Let $L_1,L_2$ be the two legs of the spider that do not contain $vv'$}. {If in $P\cup L_1\cup L_2$} there are no copies of the same vertex, then the projection of {$P\cup L_1\cup L_2\cup vv'$ into $G$ is} a path $c_1b_1a_1xa_2b_2c_2$ and a path $xa_3v$ with a loop on $v$ or a path $xv$ with a loop on $v$ or a loop on $x$ (when the length of $P$ is respectively $2,1$ or 0) corresponding respectively to $H_2$, $H_3$ and $H_4$. 
%\wal{*** notice that the sentence above is not so obvious*** the leg containing $vv'$ might not have $v$ as a leaf...while this is not a big problem,...but still need make it more clear
%\ale{***now we project only until $v$ (or $v'$)}}
{Otherwise, if there are copies of the same vertex in $P\cup L_1\cup L_2$, let $w,w'$ be the closest such pair}: there must exist a path in {$P\cup L_1\cup L_2$}, of length at least $3$, joining $w$ and $w'$ not containing copies of the same vertex, this path projects into a cycle of $G$. 
\qed\end{proof}


\section{Final remarks}
We know that deciding whether one hunter can win is polynomial, but the complexity of deciding whether the hunting number is less than some given constant $k$ is unknown for $k\ge 2$ even in the case of trees.\\ %The complexity of computing the hunting number of trees is open too. \\
Deciding whether $h(G)\le k$ can be seen as a special case of the integer matrix mortality problem where given $m$ binary square matrices of size $n$, one wants to determine whether any product obtained using these matrices results in the zero matrix; this is PSPACE-complete (see \cite{benameur2024complexityresultscopsrobber} for references). {This implies that our problem is in PSPACE and might well be PSPACE-complete, since we have no evidence that it belongs to NP.} %{Our problem might well be PSPACE-complete }too, since we have no evidence that it belongs to NP. 
In this direction, it would also be interesting to bound the number of rounds in a fastest winning hunter strategy.
%: if this can be bounded by a polynomial in the order of $G$, deciding whether $h(G)\le k$ would be NP-complete. \ale{***I am not sure anymore of this, maybe we just omit this last sentence} \\%Observe that the transposition of the number of rounds  into matrix mortality is the length of a product that equals the zero matrix, the shortest such length might be exponential in the size of the matrices \cite{10.1007/978-3-031-72621-7_8}. 
\\
%As a further link with matrix mortality, we observe that our characterization of graphs for which $h(G)=1$ implies that the integer matrix mortality problem is polynomial when the input matrices are obtained by setting to $0$ all the entries in a column of the same symmetric matrix (in other words they are of the form $M(I-e_ie_i^T)$ for a symmetric matrix of size $n$ and $1\le i \le n$). We can think of the integer matrix mortality problem as a generalized hunter and rabbit game played with $m$ graphs on the same $n$ vertices. First of all rabbit picks an initial vertex; then, at every round, hunter picks a graph $G_i$ and rabbit must move to a vertex adjacent with his position according to $G_i$. If rabbit can always move he wins, otherwise hunter wins. This game seems an interesting game to analyze  \\
\section*{Acknowledgments}
We would like  to thank Antoine Amarilli for preliminary discussions about the problem.
This research benefited from the support of the FMJH Program Gaspard Monge for optimization and operations research and their interactions with data science.

\bibliographystyle{abbrvnat}
\bibliography{sample}

\end{document}