\section{Weighted players}

\will{Add definition of cost to player $i$ under the weighted model, i.e., $c_i(a_C) = w_i \cdot f(n_{a_i}(a_C)$.}

A simple way to incorporate weighted players is to assume that each player $i \in N$ has weight $w_i \in \mathbb{N}_{>0}$, and to adapt the envy-free condition to one where no player within the coalition would prefer to swap their assign resource with another player within the coalition. In other words, all players must face the same expected `base weight', which is the total weight on the resource chosen by the player, less their own weight. We can write the base weight faced by some player $i$ in agreement $a_C$ as $\mathbb{E}_{\mu_{-C}}[f(n_{a_j}(a_C) - w_i)]$. While this envy-free definition is a natural generalisation of the unweighted case, we will see that it is a very strong condition. The definition of credibility need only be slightly extended to adapt to weighted players. To adapt our existing notation to weighted players, let $n_x(a)$ represent the total weight of players (rather than the number of players) that choose resource $x$ under the outcome $a$.

\begin{definition}[Credibility and envy-freeness for weighted players]
    Consider a subset of weighted players $C \subseteq N$ and a common prior $\mu_{-C}$ about the behaviour of players in $N \setminus C$. 
    
    An agreement $a_C \in M^{|C|}$ is \textit{credible} if no player can benefit from a unilateral deviation, i.e. for all $i \in C$ and $k \in M$,

    \[ \mathbb{E}_{\mu_{-C}}[f(n_{a_i}(a_C))] \leq \mathbb{E}_{\mu_{-C}}[f(n_k(a_C) + w_i)] \]

    An agreement $a_C \in M^{|C|}$ is \textit{envy-free} if no player would prefer to swap their assigned resources with another player. That is, for all $i, j \in N$ where $a_j \neq a_j$,

    \[ \mathbb{E}_{\mu_{-C}}[f(n_{a_i}(a_C)) - w_i] \leq \mathbb{E}_{\mu_{-C}}[f(n_{a_j}(a_C) - w_j)] \]
\end{definition}

It can be easily checked that Proposition \ref{prop:covering_agreements}, which established the relations between different properties of an agreement among unweighted players, applies equally to weighted players. 

% Notice that the envy-free condition is typically sufficient to ensure that an agreement is credible: if a player does not envy any other player, then there is no reason for it to deviate. However, there is an exception when some resource is not used by any player. There is no player to `envy' in this case, but any player sharing a resource with others would also want to deviate to the unused resource.

% \begin{lemma} \label{lem:envy-free and credible}
%     If an agreement $a_C$ within a coalition $C$ is such that no resource is unused, and the agreement is envy-free, then the agreement is credible. 
% \end{lemma}

% \begin{proof}
%     This can be clearly observed from the definitions of no-envy and credibility. When all resources are used, an agreement is envy-free if $\mathbb{E}_{\mu_{-C}}[f(n_{a_i}(a_C)) ] \geq \mathbb{E}_{\mu_{-C}}[f(n_{a_j}(a_C) - w_j + w_i)] > \mathbb{E}_{\mu_{-C}}[f(n_{a_j}(a_C) + w_i)]$, and thus the agreement is credible.
% \end{proof}

% \begin{definition}[Fairness]
%     Given a subset of players $C \subseteq N$ and a common prior $\mu_{-C}$ about the behaviour of players in $N \setminus C$, an agreement $a_C \in M^|C|$ is \textit{fair} if when resources are ranked from least-used to most-used (i.e., for $x, y \in M$, $x < y \implies n_x(a) \leq n_y(a)$), for any two players $i, j \in N$ such that $w_i < w_j$,

%     \[ \mathbb{E}_{\mu_{-C}}[f(n_{a_i}(a_C))] \leq \mathbb{E}_{\mu_{-C}}[f(n_j(a_C))] \]
% \end{definition}

\subsection{Socially optimal outcomes}

The cost incurred by a player of weight $w_i$, choosing a resource $a_i$ of total weight $n_{a_i}(a)$, is $c_i(a) \coloneqq w_i \cdot f(n_{a_i}(a))$. The social costs of an outcome $a \in M^n$, with respect to $\bar c$ and $\hat c$, are expressed as

\[ \bar c(a) = \sum_{i \in N} w_i \cdot f(n_{a_i}(a)) = \sum_{x \in M} n_x(a) \cdot f(n_x(a)) \]

and

\[ \hat c(a) = \max_{x \in M} f(n_x(a)) \]

Notice that $\hat c$ represents the unit cost of the most-used resource, rather than the highest cost incurred by any player. While these two concepts coincide for unweighted players, they are distinct for weighted players. The former is more appropriate in the congestion setting, particularly in the literature of identical machine scheduling, where the objective is to complete a number of weighted tasks (players) using identical and parallel machines (resources) in the shortest time (cost).

The problem of minimising $\bar c$ or $\hat c$ closely relates to the well-studied problem of multi-way number partitioning (MNP), first presented as the identical machine scheduling problem \cite{Graham69Identical_machine_scheduling}, where the goal is to partition a multi-set of natural numbers `as evenly as possible' into a given number of bins. We shall use `MNP' to denote such a partitioning problem, to distinguish it from the communication partition of this paper. The player weights correspond to the multi-set of natural numbers, and the resources correspond to the bins. Partitioning the numbers `as evenly as possible' typically means one of three distinct objectives (e.g., \cite{Korf2010MNP_Objectives}): minimise the size of the largest bin (`minimax'); maximise the size of the smallest bin (`maximin'); or minimise the gap between the largest and the smallest bin (`min-gap'). Finding the optimal partition is NP-complete for each of these objectives (see e.g., \cite{Garey1983NP-completeness}). We also consider a fourth objective that is less commonly discussed in the literature, which is to minimise the variance of the bin sizes (`min-var'). 

It is immediate clear that finding a $\hat c$-optimal outcome is equivalent to the minimax MNP: because the cost function in increasing, minimising the unit cost of the most-used resource is equivalent to minimise the size of the largest bin. 

Finding an equivalence for the $\bar c$-optimal outcome is a little more nuanced. When the cost function is linear (e.g., $f(x) = x$), the cost associated with a resource of total weight $x$ is $x^2$, and thus the $\bar c$ of an outcome is the sum of squares of the weights on each resource. Minimising this is equivalent to minimising the variance of the total weights on the resources, and so the problem is equivalent to the min-var MNP. When the cost function is very convex, e.g., $f(x) = e^x$, then the $\bar c$-optimal outcome tends to the solution of the minimax MNP (and among outcomes with the same minimax, the one with the lowest next-most-used resource, etc), as the cost associated with the most-used resource dominates the costs associated with all other resources.

\begin{example}[Linear cost and the min-var objective]
    Consider a game with weights $w = \{5, 3, 2, 2, 1\}$, $4$ identical resources, and two outcomes $a = [5|31|2|2]$ and $b = [5|3|21|2]$. Both outcomes are minimax, maximin, and min-gap, and yet $b$ is `more evenly distributed' than $a$. For a linear cost function $f(x) = x$, $b$ is indeed the lowest-cost, min-var outcome. 
\end{example}

% \begin{example}[$\bar c$-optimality]
%     Consider a game with player-weights $w = \{46, 39, 27, 26, 16, 13, 10\}$ and $3$ resources, taken from \cite{Walter13MNP_examples}. The optimal MNP of this game differs depending on the objective function: the minimax MNP is $((27, 26), (46, 16), (39, 13, 10))$, where the minimax is $62$; the maximin MNP is $((46, 10), (27, 16, 13), (39, 26))$, where the maximin is $53$; the min-gap MNP is $((39, 16), (46, 13), \\(27, 26, 10))$, where the min-gap is $8$.

%     If the cost function of a weighted singleton congestion game is linear, then the min-gap MNP and the min-var MNP coincide, and thus the min-gap MNP is $\bar c$-optimal. However, if the cost function is highly convex (e.g. $f(x) = e^x$), then the $\bar c$-optimal outcome coincides with that of the minimax MNP. 
% \end{example}

\subsection{Optimal partition}

We now consider whether a communication partition can induce social optimal outcomes with weighted players. We focus on $\hat c$-optimality, for which there is a known equivalent objective, the minimax, in the MNP problem, regardless of the specification of the cost function. Given the symmetry principle, we can again apply Proposition \ref{prop:subgame} and separately consider the subgame played by each coalition $C$ within a partition $\pi$ where all the players in $C$ form a grand coalition.

%It can be easily shown that a minimax outcome need not be envy-free. Consider a game where $w = \{2, 2, 1\}$ and $m = 2$: The minimax outcome is $[21|2]$, but both players choosing the first resource envy the player choosing the second. Envy-freeness is a very strong condition for weighted players, as we show below.

%Any player sharing a resource would envy another player that shares a different resource with players of lesser combined weights. To achieve envy-freeness, in any coalition, one of two conditions must be satisfied: $|C| \leq m$, so that each player can choose a different resource alone; or $|C| \mod m = 0$ and for all $i,j \in C$, $w_i = w_j$, that is, all players in $C$ have the same weight, and that the coalition contains multiples of $m$ of such players.

It turns out that envy-free is quite a strong condition in the augmented SCG with weighted players, and it imposes significant constraints on the agreements that can be reached. We again consider two cases separately: $|C| \leq m$ and $|C| > m$.

\begin{lemma}
    Under the symmetry principle, if a coalition $C$ with size $|C| \leq m$ reaches an agreement $a_C \in M^{|C|}$ that is credible, envy-free and Pareto-optimal, then all players in $C$ choose different resources, i.e., $a_i \neq a_j$ for all $i, j \in C$, $i \neq j$.
\end{lemma}

\begin{proof}
    %From Proposition \ref{prop:covering_agreements} an agreement is envy-free and credible if and only if each player choose a different resource, i.e. $a_i \neq a_j$ for all $i, j \in C$, $i \neq j$. ($\Leftarrow$) It is clear from the definitions that under such an agreement, both envy-free and credibility are trivially satisfied. ($\Rightarrow$) Consider an agreement where some players, $i, j$, share a resource. This is clearly not credible, as either player has the incentive to deviate to an unused resource. 
    This is an application of Proposition \ref{prop:covering_agreements}, which implies that such an agreement must be covering. %A covering agreement for a coalition of size $|C| \leq m$ is one where all players choose different resources.
\end{proof}

% \begin{lemma}
%     Under the symmetry principle, if an agreement $a_C \in M^{|C|}$ within some coalition $C$ is credible, envy-free and Pareto-optimal, then either $|C| \leq m$ and all players choose different resources ($a_i \neq a_j$ for all $i, j \in C$), or $|C| > m$ and there exists $b \in \mathbb{N}_{>0}$ such that $b$ is divisible by the weight of each player in $C$, and that for given weight $w_i$, the number of players with such weight is divisible by $\frac{b}{w_i} + 1$. The agreement is such each player $i$ which uses the resource $a_i$, shares it with $\frac{b}{w_i}$ other players of the same weight $w_i$.
% \end{lemma}

\begin{lemma} \label{lem:conditions_for_CrPE}
    Under the symmetry principle, if a coalition $C$ of size $|C| > m$ reaches an agreement $a_C \in M^{|C|}$ that is credible, envy-free and Pareto-optimal, then there exists some integer $b$ such that $b$ is divisible by the weight of each player in $C$, and that for a given weight $w_i$ where $i \in C$, the number of players with such weight is divisible by $\frac{b}{w_i} + 1$, and the agreement is such that each resource is shared by $\frac{b}{w_i} + 1$ players of the same weight $w_i$. 
\end{lemma}

\begin{proof}
    From Proposition \ref{prop:covering_agreements}, we only need to consider covering agreements that are envy-free. Consider two players $i, j$ with weights $w_i \leq w_j$, and two cases: $a_i = a_j$ (they share the same resource) and $a_i \neq a_j$ (they choose different resources). 

    If $a_i = a_j$, consider some player $k$ where $a_k \neq a_i$. For $i$ to not envy $k$, we require $n_{a_i} - w_i \leq n_{a_k} - w_k$, and for $k$ to not envy $j$, we require $n_{a_k} - w_k \leq n_{a_j} - w_j$. These inequalities imply $n_{a_i} - w_i \leq n_{a_j} - w_j$, or $w_i \geq w_j$ (since $a_i = a_j$). Because we assume $w_i \leq w_j$, it must be the case that $w_i = w_j$. Thus, the only way to satisfy envy-free is if all players that share the same resource to have the same weight. 

    If $a_i \neq a_j$, then envy-freeness between $i$ and $j$ implies that $n_{a_i} - w_i = n_{a_j} - w_j$, which we set to be the value $b$. Since all players choosing the same resource must have the same weight, $b$ must be divisible by both $w_i$ and $w_j$, and that the number of players choosing resources $a_i$ and $a_j$ are $\frac{b}{w_i} + 1$ and $\frac{b}{w_j} + 1$ respectively. Applying the same reasoning to all resources, we see that the number of players of any given weight $w$ must be a multiple of $\frac{b}{w} + 1$ (the number of players of such weight sharing one resource).
\end{proof}

Lemma \ref{lem:conditions_for_CrPE} restricts the types of coalition for which envy-free agreements can arise. Clearly, given a weighted game $G = \langle w, m, f \rangle$, there always exists a partition that induces an envy-free, credible, and Pareto-optimal strategy profile: we can simply construct one where all coalitions are of size $\leq m$. However, there is no guarantee that the induced partition equilibrium is $\hat c$-optimal.

Combining the above lemmas, we get the following proposition that describes the conditions for a partition to induce credible, envy-free, and Pareto-optimal agreements within each coalition.

\begin{proposition}
    Under the symmetry principle, if the partition $\pi$ induces credible, envy-free and Pareto-optimal agreements within each coalition, then for each $C \in \pi$ where $|C| \leq m$, $a_C$ is a covering agreement; and there exists some $b \in \mathbb{N}$ such that for each $C \in \pi$ where $|C| > m$, $b$ is divisible by the weight of each player in $C$, and that for a given weight $w_i$ where $i \in C$, the number of players with such weight is divisible by $\frac{b}{w_i} + 1$, and the agreement is such that each resource is shared by $\frac{b}{w_i} + 1$ players of the same weight $w_i$.
\end{proposition}

It is worth noting that for a partition to induce $\hat c$-optimal outcomes, at most one coalition in the partition can reach an agreement where the resources have different total weights. If there were more than one such coalition, then there is a non-zero probability that these coalitions independently agree to stack the most weight on the same resource.

Finally, to show that a partition which induces $\hat c$-optimal outcomes may not exist, we provide a counter-example.

\begin{example}
    Consider the game with weights $w = \{1, 2, 3\}$ and $m = 2$. The $\hat c$-optimal outcome is $[12|3]$ (numbers represent weights rather than player indices). Clearly, the only partition that can ensure such an outcome is the grand coalition, as any other partition where some players do not communicate has a positive probability of mis-coordination. However, the agreement $[12|3]$ is not envy-free, as both the players of weights $1$ and $2$ envy the allocated resource choice of the player with weight $3$. 
\end{example}

% \will{Explore iterative minimax. Under what conditions of the cost function are they optimal?}

% \will{Can separately explore $\bar c$-optimality and $\hat c$-optimality.}

% \begin{example} [Minimax yet not $\bar c$-optimal]
    
% \end{example}

%Whether an efficient partition exists is a computational problem. We first need to find the set of $\hat c$-optimal outcomes, then check if there exists a partition that induces only such optimal outcomes. We will study this problem in our future works.

%Because the resources remain symmetric, allowing players to have differential weights does not break the symmetry of the conjecture players outside of a coalition have about the behaviour of players within the coalition. What remains challenging is to ensure that each coalition has an envy-free and credible partial outcome that they would coordinate on. Similar to the case of unweighted players, this can be achieved if the choices within a coalition is as `flat' as possible. 

%This problem is trivial for any coalition of size $\leq m$: players can choose different resources to achieve a credible and envy-free agreement. For coalitions of size $|C| > m$, the problem becomes a little more complex. We show that this is equivalent to finding the maximin share (MMS, see e.g. (cite Budish)) of $n$ objects with values $w_i$ and $m$ bins.

%[Not true. Consider the distributions $\{4, 2, 2\}$ and $\{3, 3, 2\}$ by players with weights $\{3, 2, 2, 1\}$, which are $[31|2|2]$ and $[3|21|2]$ respectively. Both are maximin distributions, but the social cost of \{3, 3, 2\} is lower than that of \{4, 2, 2\} (difference being 4f(4) + 2f(2) - 6f(3) = 4[f(4) - f(3)] - 2[f(3) - f(2)]). 

%Due to the increasing and convex properties of the cost function, an `efficient' agreement should intuitive be as flat as possible, i.e. it should iteratively minimise the cost of the most-used resource. We use this intuition to define a flat agreement.

% \begin{definition}[Iterative minimax agreement]
%     An agreement $a_C \in M^{|C|}$ is a minimax agreement between a set of players $C \subseteq N$ if $\max_{j \in M} n_j(a_C) \leq n_j(a'_C)$ for all $a'_C \in M^{|C|}$. A reduction of an agreement is a part of the agreement reached by the remaining players after removing a most-used resource and players choosing that resource from the agreement. An agreement is flat if it is a minimax agreement under iterative reductions.
% \end{definition}

% \begin{example}
%     Consider a 5-player, 3-resource game where players have weights $w = \{3, 3, 2, 2, 1\}$. The agreement $[31|22|3]$ is a minimax agreement between all players (with 4 being the cost of the most-used resource). We then consider both possible reductions: by removing players with weights \{3, 1\} or \{2, 2\}. In the first case, we have a reduced agreement of $[22|3]$, which is a minimax agreement between players of weights \{3, 2, 2\} and two resources. In the second case, we have a reduced agreement of $[31|3]$, which is also a minimax agreement between players of weights \{3, 3, 1\} and two resources. Further reducing either case gets to a single players, single resource game, of which the reduced agreement $[3]$ is trivially a minimax agreement. 
% \end{example}

% \subsubsection*{EDIT NOTE} 

% Given a set of players, need some agreement that is 1) credible, 2) envy-free, and 3) efficient.

% This is a \href{https://en.wikipedia.org/wiki/Multiway_number_partitioning}{multiway number partitioning problem}: given a multiset $w$ of numbers, find a partition of $w$ into $m$ subsets, such that the $m$ sums are `as near as possible'. Where `as near as possible' could mean: 

% \begin{itemize}
%     \item Minimise the largest sum (minimax), which is referred to as the problem of \href{https://en.wikipedia.org/wiki/Identical-machines_scheduling}{idential-machines scheduling}. This is equivalent to $\hat c$-optimality.
%     \item Maximise the smallest sum (maximin). Equivalent to finding the maxmin share (MMS). Use to find the minimum `fair' share that one could expect.
%     \item Minimise the gap between the largest and smallest. This does not necessarily induce the $\bar c$-optimal outcome (see example below).
% \end{itemize}

% These three problems are distinct for $m \geq 3$, and each is NP-complete. Consider an example: a three-way partitioning of $(46, 39, 27, 26, 16, 13, 10)$, taken from \href{https://link.springer.com/article/10.1007/s10100-011-0217-4}{this paper}. The optimal partitions for the three measures are all distinct: 
% \begin{itemize}
%     \item $((46, 10), (27, 16, 13), (39, 26))$, or $(56,56,65)$, is the maximin outcome,
%     \item $((27, 26), (46, 16), (39, 13, 10))$, or $(53,62,62)$, is the minimax outcome, 
%     \item $((39, 16), (46, 13), (27, 26, 10))$, or $(55,59,63)$, is the min-gap outcome.
% \end{itemize}

% For linear cost $f(x) = x$, these distributions have the social costs that is the sum of squares of the weights in each resource. This equates to $10497$, $10497$ and $10475$, so the min-gap distribution is the most efficient. For highly convex costs, e.g. if $f(x) = x^{50}$ or $f(x) = e^{x}$, then minimising the max becomes the efficient outcome. 

% Given that there is no clear equivalence between any of these objectives and $\bar c$-optimality for a general increasing and convex cost function $f$, we focus on the more tangible problem of $\hat c$-optimality, which is equivalent to finding a minimax partition and used for solving the practical problem of identical machine scheduling.

% % Example: None of the three measures are equivalent to $\bar c$-optimality with linear cost $f(x) = x$. Consider $w = \{5, 3, 2, 2, 1\}$, four resources, and two outcomes $A = [5|3|21|2]$ and $B = [5|31|2|2]$. Both outcomes have the same maxmin, minmax, and min-gap, but $A$ has a lower cost (44) compared to $B$ (48). Thus none of the three objectives are equivalent to minimising social cost $\bar c$. In fact, with linear cost, a fourth objective, minimising the variance (min-var) is equivalent to minimising $\bar c$. This is because the variance is a linear function of the sum of squares of the weights on each resource, which is also $\bar c$ in this case.

% % \begin{align*}
% %     \sigma^2    &= \frac{\sum_{j \in M} (w_j - \bar w)^2 }{n}  \\
% %                 &= \frac{\sum_{j \in M} w_j^2}{n} - \bar w^2
% % \end{align*}

% What can we say about $\hat c$-optimality in this case then? 

% First consider the grand coalition, where players can coordinate on any outcome. Is a minimax outcome envy-free and credible? A minimax outcome may not be credible, as it doesn't ensure that the any resource that is not the most-used is well-balanced, and therefore some player may profitably switch between resources. A simple example here $w = \{10, 3, 2, 1\}$ and $m = 3$. The outcome $[10|32|1]$ is clearly minimax but not credible (deviation to $[10|3|21]$ or $[10|2|31]$). Nor is a credible outcome minimax: consider the maximin outcome above, which is credible, but is not minimax.

% On envy-freeness, clearly a minimax outcome need not be envy-free. Consider a game $w = \{2, 2, 1\}$, $m = 2$: The minimax outcome is $[21|2]$, but both players choosing the first resource envy the player choosing the second. Note that envy-freeness is a very strong condition: any player sharing a resource would envy another player that uses a resource alone. The only way to achieve envy-freeness, in any coalition, is to ensure that either 1) $|C| \leq m$ (each player choose a different resource), or 2) $|C| \mod m = 0$ and for all $i \in C$, $w_i = w$ (all players in $C$ have the same weight).

% What does this leave us in terms of cheap-talk partition? Clearly, want to avoid players with high weights from choosing the same resource. This can be avoided by putting them into the same coalition. 

% Result: for some weighted games, no partition can achieve envy-freeness, credibility, and $\hat c$-optimality.


% Literature: \href{https://www.ijcai.org/Proceedings/09/Papers/096.pdf}{Multiway number partitioning}, 


% The minimising the usage of the maximum resource (minimax) is one potential method. Given that utility is the negative of cost, this is equivalent to maximising the smallest utility associate with a resource (maximin). 

% Q: Can an agreement be higher max-cost resource be more $\bar c$-efficient than another with lower max-cost resource? E.g. [5|2|1] vs [4|2|2]. Answer is no. Given an agreement $a$, let $a^i$ denote the $i$th most-used resource in $a$ (tie-breaks are irrelevant). Suppose we can two agreements $a, b$, and there exists two indices $i, j$ where $i < j$, such that $n(a^i) > n(b^i) \geq n(b^j) > n(a^j)$. 

% For simplicity consider the case where $n(a^i) = n(b^i) + 1 = n(b^j) + 1 = n(a^j) + 2$. This is a single step (in a potentially multiple number of steps) to assess the difference in costs between the two agreements. 

% Cost associate with $i$ and $j$ for $a$ is $n(a^i) f(n(a^i)) + n(a^j) f(n(a^j))$, and that for $b$ is 
% $n(b^i) f(n(b^i)) + n(b^j) f(n(b^j))$. Thus the difference is 

% \[ n(b^i)[f(n(a^i)) - f(n(b^i))] - n(a^j)[f(n(b^j)) - f(n(a^j)] + f(n(a^i)) - f(n(b^j))  \]

% First term > second term (convexity), and second term > fourth term (increasing), so the difference is positive, and thus the change reduces cost.

% Counter example: $f(x) = x$, consider player weights $\{4, 4, 3, 3, 1, 1\}$ and two agreements with distributions $[44|3|3|11]$ and $[43|43|1|1]$, the total cost of the former is $8^2 + 3^2 + 3^2 + 2^2 = 86$, that of the latter is $7^2 + 7^2 + 1^2 + 1^2 = 100$. So the general rule that the agreement with higher leading cost is less efficient is not true. Note that the optimal distribution in this case is $[4|4|31|31]$.


% Algorithm:

% If $n \geq m$, then we're done: each player choosing a different resource is efficient, envy-free, and credible.

% Otherwise, iterative remove players whose weight $w_i \geq \lceil \frac{n}{m} \rceil$, and one resource for each such player removed. Left with a reduced game, where $w_i \leq \lfloor \frac{n}{m} \rfloor$ for all players.

%%%%%%%%%%%%%%%%%%%%%%%%%%%%%%%%%%%%%%%%%%%%%%%%%%%%%%%%%%%%%%%%%%%%%%%%
