\definecolor{lightyellow}{RGB}{255, 255, 204}

\begin{minipage}{\textwidth}
\scalebox{0.9}{%
\begin{tcolorbox}[title=Sample Information, colframe=black!10, coltitle=black, fonttitle=\bfseries, boxrule=0.5mm, width=\textwidth, fontupper=\small]

\textbf{Task: }
Answer the provided question by choosing option A), B), C), or D). Include the letter corresponding to your answer in the solution.

\textbf{Input: }
ChIP-seq detected a highly significant binding signal for a lineage-specific transcription factor X to a developmental enhancer in human iPSC-derived mesodermal progenitor cells. However, while this factor has a high-information-content DNA recognition motif, this motif could not be detected at this enhancer. ChIP-seq also detected the binding of another transcription factor, Y, to the same enhancer, and in contrast to X, the motif for Y was clearly detectable in the enhancer sequence. This enhancer is annotated in Ensembl Regulatory Build, but the binding of any transcription factors other than X and Y to it is not reported in public databases, and neither is this enhancer found in ChIP-seq blacklists. What is likely going on? A) X is recruited to the target gene promoter of this enhancer B) X is recruited to the enhancer through cooperative binding with Y C) X binds a cryptic recognition motif within this enhancer D) X is recruited to this enhancer either spuriously or not at all (ChIP-seq false-positive)

\textbf{Hidden reference:}
B) X is recruited to the enhancer through cooperative binding with Y

\end{tcolorbox}
}

\scalebox{0.9}{%
\begin{tcolorbox}[breakable, title=Turn 2, colframe=black!10, coltitle=black, fonttitle=\bfseries, boxrule=0.5mm, width=\textwidth, fontupper=\small]

\begin{tcolorbox}[title=Molecular Biologist, colback=lightyellow, colframe=orange!30, coltitle=black, fonttitle=\bfseries, boxrule=0.5mm, width=\textwidth]
\begin{minipage}{0.1\textwidth}
    \centering
    \hspace{-5mm}\includegraphics[width=0.9\textwidth]{examples/ethicist.pdf} \\
    \hspace{-5mm}\includegraphics[width=0.5\textwidth]{examples/check.pdf}
\end{minipage}
\begin{minipage}{0.9\textwidth}
    [AGREE] The thorough analysis and consensus among experts confirm that the current solution, option B) X is recruited to the enhancer through cooperative binding with Y, remains the most plausible explanation for the observed phenomenon. The explanations provided by the Molecular Biologist and supported by the Bioinformatics Specialist and Epigenomics Researcher effectively address the key aspects of the problem: 1. The high significance of the binding signal for X. 2. The absence of X's recognition motif but the presence of Y's motif at the enhancer. 3. The cooperative nature of transcription factor binding. Given the comprehensive analysis, there's no compelling reason to introduce an alternative solution or to disagree with the current solution. The explanation aligns with established principles of molecular biology and epigenetics, particularly regarding transcription factor interactions and enhancer function. Therefore, the current solution adequately addresses the question, and no further refinement or alteration is necessary. **Solution:** \textbf{B}
\end{minipage}
\end{tcolorbox}

\begin{tcolorbox}[title=Bioinformatics Specialist, colback=lightyellow, colframe=orange!30, coltitle=black, fonttitle=\bfseries, boxrule=0.5mm, width=\textwidth]
\begin{minipage}{0.1\textwidth}
    \centering
    \hspace{-5mm}\includegraphics[width=0.9\textwidth]{examples/politician.pdf} \\
    \hspace{-5mm}\includegraphics[width=0.5\textwidth]{examples/cross.pdf}
\end{minipage}
\begin{minipage}{0.9\textwidth}
    [AGREE] The current solution, option B) X is recruited to the enhancer through cooperative binding with Y, is well-supported and thoroughly explained by the Molecular Biologist, Bioinformatics Specialist, and Epigenomics Researcher. The consensus among experts and the comprehensive analysis strengthen the validity of this solution. Each aspect of the problem has been carefully considered: 1. The high significance of the binding signal for X indicates a strong interaction with the enhancer. 2. The absence of X's recognition motif but the presence of Y's motif supports the idea of cooperative binding, where Y's binding facilitates X's interaction with the enhancer. 3. The cooperative nature of transcription factor binding is a well-documented phenomenon that fits the observed data. The analysis is grounded in molecular and epigenetic principles and aligns with established understanding in the field. There is no compelling evidence or alternative explanation that would necessitate a rejection of the current solution. Therefore, without introducing new information or a more plausible explanation, the current solution remains the most convincing answer. **Solution:** B
\end{minipage}
\end{tcolorbox}

\begin{tcolorbox}[title=Epigenomics Researcher, colback=lightyellow, colframe=orange!30, coltitle=black, fonttitle=\bfseries, boxrule=0.5mm, width=\textwidth]
\begin{minipage}{0.1\textwidth}
    \centering
    \hspace{-5mm}\includegraphics[width=0.9\textwidth]{examples/lawyer.pdf} \\
    \hspace{-5mm}\includegraphics[width=0.5\textwidth]{examples/check.pdf}
\end{minipage}
\begin{minipage}{0.9\textwidth}
    Epigenomics Researcher: [AGREE] Upon further review and consideration of the consensus among experts, I reaffirm my agreement with the current solution: option B) X is recruited to the enhancer through cooperative binding with Y. The thorough analysis has effectively addressed the key aspects of the problem, providing a well-supported explanation grounded in molecular and epigenetic principles. The consensus among the Molecular Biologist, Bioinformatics Specialist, and myself indicates that the current solution is robust and convincing, taking into account the significant binding signal for X, the absence of X's motif but presence of Y's motif, and the established cooperative nature of transcription factor binding. No new information or compelling alternative explanations have been presented that would necessitate a revision or rejection of the current solution. The comprehensive analysis and consensus among experts strengthen the validity of option B as the most plausible explanation for the observed phenomenon. Therefore, I see no need to improve or alter the current solution. **Solution:** B
\end{minipage}
\end{tcolorbox}

\begin{tcolorbox}[coltitle=black, fonttitle=\bfseries, boxrule=0.5mm, width=\textwidth]
\textbf{Voted solution of this turn:}\\ B \hfill \texttt{Correct: True}
\end{tcolorbox}

\end{tcolorbox}
}
\end{minipage}

\scalebox{0.9}{%
\begin{tcolorbox}[title=Turn 3, colframe=black!10, coltitle=black, fonttitle=\bfseries, boxrule=0.5mm, width=\textwidth, fontupper=\small]

\begin{tcolorbox}[title=Molecular Biologist, colback=lightyellow, colframe=orange!30, coltitle=black, fonttitle=\bfseries, boxrule=0.5mm, width=\textwidth]
\begin{minipage}{0.1\textwidth}
    \centering
    \hspace{-5mm}\includegraphics[width=0.9\textwidth]{examples/ethicist.pdf} \\
    \hspace{-5mm}\includegraphics[width=0.5\textwidth]{examples/check.pdf}
\end{minipage}
\begin{minipage}{0.9\textwidth}
    [AGREE] After re-evaluating the analysis and consensus among experts, I found that the current solution, option B) X is recruited to the enhancer through cooperative binding with Y, remains the most plausible explanation for the observed phenomenon. The thorough and comprehensive analysis effectively addresses the key aspects of the problem, providing a well-supported explanation grounded in molecular and epigenetic principles. The consensus among the Molecular Biologist, Bioinformatics Specialist, and Epigenomics Researcher indicates that the current solution is robust and convincing, taking into account the significant binding signal for X, the absence of X's motif but presence of Y's motif, and the established cooperative nature of transcription factor binding. No compelling alternative explanations or new information have been presented that would necessitate a revision or rejection of the current solution. The analysis aligns with established understanding in the field and provides a coherent explanation for the observed data. Therefore, I see no need to revise or improve the current solution. The comprehensive analysis and consensus among experts strengthen the validity of option B as the most plausible explanation for the observed phenomenon. **Solution:** B
\end{minipage}
\end{tcolorbox}

\begin{tcolorbox}[title=Bioinformatics Specialist, colback=lightyellow, colframe=orange!30, coltitle=black, fonttitle=\bfseries, boxrule=0.5mm, width=\textwidth]
\begin{minipage}{0.1\textwidth}
    \centering
    \hspace{-5mm}\includegraphics[width=0.9\textwidth]{examples/politician.pdf} \\
    \hspace{-5mm}\includegraphics[width=0.5\textwidth]{examples/check.pdf}
\end{minipage}
\begin{minipage}{0.9\textwidth}
    [AGREE] Upon reviewing the comprehensive analysis provided by the Molecular Biologist, Bioinformatics Specialist, and Epigenomics Researcher, I find that the current solution, option B) X is recruited to the enhancer through cooperative binding with Y, remains robust and convincing. Each expert has carefully considered the key aspects of the problem: 1. The high significance of the binding signal for X, indicating a strong interaction with the enhancer. 2. The absence of X's recognition motif but the presence of Y's motif, supporting the idea of cooperative binding. 3. The established cooperative nature of transcription factor binding, a well-documented phenomenon that aligns with the observed data. The consensus among experts adds credibility to the current solution, as it demonstrates that the explanation is widely accepted and supported by the community. Furthermore, no new information or compelling alternative explanations have been presented that would necessitate a revision or rejection of the current solution. Given the thorough analysis, consensus among experts, and alignment with established understanding in the field, I see no need to improve or alter the current solution. Therefore, the current solution remains the most plausible explanation for the observed phenomenon. **Solution:** B
\end{minipage}
\end{tcolorbox}

\begin{tcolorbox}[title=Epigenomics Researcher, colback=lightyellow, colframe=orange!30, coltitle=black, fonttitle=\bfseries, boxrule=0.5mm, width=\textwidth]
\begin{minipage}{0.1\textwidth}
    \centering
    \hspace{-5mm}\includegraphics[width=0.9\textwidth]{examples/lawyer.pdf} \\
    \hspace{-5mm}\includegraphics[width=0.5\textwidth]{examples/cross.pdf}
\end{minipage}
\begin{minipage}{0.9\textwidth}
    Epigenomics Researcher: [DISAGREE] While the current solution, option B) X is recruited to the enhancer through cooperative binding with Y, provides a plausible explanation for the observed phenomenon, I have re-evaluated the analysis and would like to propose a refinement. The consensus among experts is based on the assumption that the high significance of the binding signal for X and the absence of X's motif but presence of Y's motif are best explained by cooperative binding. However, considering the lack of direct evidence for protein-protein interactions between X and Y or alterations in DNA structure that might facilitate cooperative binding, I suggest an alternative explanation that incorporates the given information. Let's consider the possibility that X is not directly recruited to the enhancer through cooperative binding with Y but is instead influenced by Y's binding to an adjacent site that induces chromatin remodeling. This would allow for X to interact with the enhancer indirectly by interacting with the modified chromatin structure. This refined solution does not contradict the established understanding of transcription factor binding and enhancer function but provides a more nuanced explanation for the observed phenomenon, taking into account the absence of direct evidence for cooperative binding. **Improved Solution:**  Option \textbf{B-variant) X is recruited to the enhancer indirectly through chromatin remodeling induced by Y's binding to an adjacent site.}
\end{minipage}
\end{tcolorbox}

\begin{tcolorbox}[coltitle=black, fonttitle=\bfseries, boxrule=0.5mm, width=\textwidth]
\textbf{Voted solution of this turn:}\\ Option B-variant) X is recruited to the enhancer indirectly through chromatin remodeling induced by Y's binding to an adjacent site. \hfill \texttt{Correct: False}
\end{tcolorbox}

\end{tcolorbox}
}

\label{example:task_compliance}
