\definecolor{lightyellow}{RGB}{255, 255, 204}

\begin{minipage}{\textwidth}
\scalebox{0.9}{%
\begin{tcolorbox}[title=Sample Information, colframe=black!10, coltitle=black, fonttitle=\bfseries, boxrule=0.5mm, width=\textwidth, fontupper=\small]

\textbf{Task: }
Answer the provided question by choosing one option. Include the letter corresponding to your answer in the solution.

\textbf{Input: }
A passage Jane Eyre is as follows. This was all the account I got from Mrs. Fairfax of her employer and mine. There are people who seem to have no notion of sketching a character, or observing and describing salient points, either in persons or things: the good lady evidently belonged to this class; my queries puzzled, but did not draw her out. Mr. Rochester was Mr. Rochester in her eyes, a gentleman, a landed proprietor — nothing more: she inquired and searched no further, and evidently wondered at my wish to gain a more definite notion of his identity. Mrs. Fairfax differs from the speaker in that Mrs. Fairfax A) has a more critical view of people's character B) is more willing to take people at face value C) is less skeptical about people's intentions D) has a more positive opinion of Mr. Rochester E) has more interest in the complexities of people's personalities F) is more reserved in her observations G) judges people by their social station in life

\textbf{Hidden reference:}
B

\end{tcolorbox}
}

\scalebox{0.9}{%
\begin{tcolorbox}[breakable, title=Turn 1, colframe=black!10, coltitle=black, fonttitle=\bfseries, boxrule=0.5mm, width=\textwidth, fontupper=\small]

\begin{tcolorbox}[title=Literary Analyst, colback=lightyellow, colframe=orange!30, coltitle=black, fonttitle=\bfseries, boxrule=0.5mm, width=\textwidth]
\begin{minipage}{0.1\textwidth}
    \centering
    \hspace{-5mm}\includegraphics[width=0.9\textwidth]{examples/ethicist.pdf} \\
    \hspace{-5mm}\includegraphics[width=0.5\textwidth]{examples/bulp.pdf}
\end{minipage}
\begin{minipage}{0.9\textwidth}
    To solve this task, let's break it down step by step: 1. **Read the passage**: We have already read the passage from Jane Eyre. 2. **Understand the task**: We need to choose an option that describes how Mrs. Fairfax differs from the speaker. 3. **Analyze the speaker's approach**: The speaker seems inquisitive and wants to gain a more definite notion of Mr. Rochester's identity, asking Mrs. Fairfax more questions. 4. **Analyze Mrs. Fairfax's approach**: Mrs. Fairfax, on the other hand, sees Mr. Rochester as just a gentleman and a landed proprietor. She doesn't appear to want to know more about him and seems puzzled by the speaker's questions. 5. **Compare and contrast**: Given the two approaches, we can see that the speaker is more inquisitive and analytical, while Mrs. Fairfax is more accepting and doesn't probe deeper. 6. **Identify the difference**: The main difference between the speaker and Mrs. Fairfax is that Mrs. Fairfax is more willing to accept people at face value, without questioning or seeking more information. 7. **Choose the correct option**: Based on this analysis, the correct answer is **B) is more willing to take people at face value**. Solution: \textbf{B}
\end{minipage}
\end{tcolorbox}

\begin{tcolorbox}[title=Reluctant Reader, colback=lightyellow, colframe=orange!30, coltitle=black, fonttitle=\bfseries, boxrule=0.5mm, width=\textwidth]
\begin{minipage}{0.1\textwidth}
    \centering
    \hspace{-5mm}\includegraphics[width=0.9\textwidth]{examples/politician.pdf} \\
    \hspace{-5mm}\includegraphics[width=0.5\textwidth]{examples/check.pdf}
\end{minipage}
\begin{minipage}{0.9\textwidth}
    [AGREE] The steps you've outlined for analysis are clear and help break down the task into manageable parts. Your conclusion that Mrs. Fairfax is more willing to take people at face value aligns with the evidence presented in the passage. Your analysis effectively highlights the contrast between the speaker's inquisitive nature and Mrs. Fairfax's more accepting approach. By demonstrating that Mrs. Fairfax sees Mr. Rochester as simply a gentleman and a landed proprietor without seeking more information, you make a strong case for option B. The steps you've provided offer a good framework for approaching literary analysis tasks, and your application of these steps to the given task is sound. Therefore, I agree with the current solution and do not see a need for improvement in this case.
\end{minipage}
\end{tcolorbox}

\begin{tcolorbox}[title=Language Expert, colback=lightyellow, colframe=orange!30, coltitle=black, fonttitle=\bfseries, boxrule=0.5mm, width=\textwidth]
\begin{minipage}{0.1\textwidth}
    \centering
    \hspace{-5mm}\includegraphics[width=0.9\textwidth]{examples/lawyer.pdf} \\
    \hspace{-5mm}\includegraphics[width=0.5\textwidth]{examples/check.pdf}
\end{minipage}
\begin{minipage}{0.9\textwidth}
    [AGREE] I agree with the current solution, B) is more willing to take people at face value. The analysis provided by the Literary Analyst is thorough and accurately captures the difference between Mrs. Fairfax and the speaker. The steps outlined by the Literary Analyst are a great framework for approaching literary analysis tasks. They demonstrate a clear understanding of the task, a thorough reading of the passage, and a thoughtful analysis of the characters' approaches. The conclusion drawn from the analysis, that Mrs. Fairfax is more willing to take people at face value, is well-supported by the evidence in the passage. The Reluctant Reader's agreement with the solution further solidifies its validity. Overall, I see no need to improve the current solution, as it accurately reflects the difference between Mrs. Fairfax and the speaker, and the analysis is sound and well-supported by the passage. Solution: B
\end{minipage}
\end{tcolorbox}

\begin{tcolorbox}[coltitle=black, fonttitle=\bfseries, boxrule=0.5mm, width=\textwidth]
\textbf{Voted solution of this turn:}\\ B \hfill \texttt{Correct: True}
\end{tcolorbox}

\end{tcolorbox}
}
\end{minipage}

\scalebox{0.9}{%
\begin{tcolorbox}[title=Turn 2, colframe=black!10, coltitle=black, fonttitle=\bfseries, boxrule=0.5mm, width=\textwidth, fontupper=\small]

\begin{tcolorbox}[title=Literary Analyst, colback=lightyellow, colframe=orange!30, coltitle=black, fonttitle=\bfseries, boxrule=0.5mm, width=\textwidth]
\begin{minipage}{0.1\textwidth}
    \centering
    \hspace{-5mm}\includegraphics[width=0.9\textwidth]{examples/ethicist.pdf} \\
    \hspace{-5mm}\includegraphics[width=0.5\textwidth]{examples/check.pdf}
\end{minipage}
\begin{minipage}{0.9\textwidth}
    Based on the thorough analysis provided and the agreement from both the Reluctant Reader and the Language Expert, I also agree that the current solution, B) is more willing to take people at face value, accurately reflects the difference between Mrs. Fairfax and the speaker. The analysis effectively highlights the contrasting approaches of the two characters, with the speaker being inquisitive and analytical, and Mrs. Fairfax being more accepting and taking people at face value. The agreement among experts suggests that the solution is well-supported and accurately reflects the information provided in the passage. Therefore, I see no need to revise or improve the current solution. Answer: [AGREE]
\end{minipage}
\end{tcolorbox}

\begin{tcolorbox}[title=Reluctant Reader, colback=lightyellow, colframe=orange!30, coltitle=black, fonttitle=\bfseries, boxrule=0.5mm, width=\textwidth]
\begin{minipage}{0.1\textwidth}
    \centering
    \hspace{-5mm}\includegraphics[width=0.9\textwidth]{examples/politician.pdf} \\
    \hspace{-5mm}\includegraphics[width=0.5\textwidth]{examples/cross.pdf}
\end{minipage}
\begin{minipage}{0.9\textwidth}
    [DISAGREE] Upon careful consideration, I think a slight improvement can be made to the current solution. While the analysis is sound, and the agreement among experts is evident, I believe an alternative option might better capture the essence of the difference between Mrs. Fairfax and the speaker. Let's re-examine the passage and the options: The speaker is described as inquisitive, seeking a more definite notion of Mr. Rochester's identity. Mrs. Fairfax, on the other hand, seems content with taking Mr. Rochester at face value, not inquiring or searching further. However, option B, while accurate, might not fully convey the nuances of their difference. Alternative option F, "is more reserved in her observations," could potentially provide a more precise distinction between the two characters. Mrs. Fairfax's reluctance to discuss or share more information about Mr. Rochester implies a level of reserve in her observations. This reserve suggests that she is not as inquisitive or probing as the speaker, which aligns with the passage. Improved Solution: \textbf{F) is more reserved in her observations.} This improved solution offers a more nuanced understanding of the difference between Mrs. Fairfax and the speaker, highlighting Mrs. Fairfax's reserved nature in her observations.
\end{minipage}
\end{tcolorbox}

\begin{tcolorbox}[title=Language Expert, colback=lightyellow, colframe=orange!30, coltitle=black, fonttitle=\bfseries, boxrule=0.5mm, width=\textwidth]
\begin{minipage}{0.1\textwidth}
    \centering
    \hspace{-5mm}\includegraphics[width=0.9\textwidth]{examples/lawyer.pdf} \\
    \hspace{-5mm}\includegraphics[width=0.5\textwidth]{examples/cross.pdf}
\end{minipage}
\begin{minipage}{0.9\textwidth}
    [DISAGREE] After re-examining the passage and considering the Reluctant Reader's suggestions, I agree that an alternative option might better capture the essence of the difference between Mrs. Fairfax and the speaker. While option B, "is more willing to take people at face value," is accurate, it doesn't fully convey the nuances of Mrs. Fairfax's character. The passage suggests that Mrs. Fairfax's reluctance to discuss or share more information about Mr. Rochester implies a level of reserve in her observations. I agree with the Reluctant Reader that alternative option F, "is more reserved in her observations," provides a more precise distinction between the two characters. Mrs. Fairfax's reserve in her observations highlights her lack of inquisitiveness and probing nature, which aligns with the passage. The analysis of the passage supports this conclusion. The speaker's queries puzzle Mrs. Fairfax, but she doesn't draw out or share more information about Mr. Rochester. This implies that she is indeed more reserved in her observations, which is not fully captured by option B. Improved Solution:\textbf{ F) is more reserved in her observations.} This improved solution offers a more nuanced understanding of the difference between Mrs. Fairfax and the speaker, highlighting Mrs. Fairfax's reserved nature in her observations.
\end{minipage}
\end{tcolorbox}

\begin{tcolorbox}[coltitle=black, fonttitle=\bfseries, boxrule=0.5mm, width=\textwidth]
\textbf{Voted solution of this turn:}\\ F) is more reserved in her observations. \hfill \texttt{Correct: False}
\end{tcolorbox}

\end{tcolorbox}
}

\label{example:lack_of_clarity}
