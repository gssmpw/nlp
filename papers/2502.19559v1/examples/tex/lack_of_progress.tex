\definecolor{lightyellow}{RGB}{255, 255, 204}

\begin{minipage}{\textwidth}
\scalebox{0.9}{%
\begin{tcolorbox}[title=Sample Information, colframe=black!10, coltitle=black, fonttitle=\bfseries, boxrule=0.5mm, width=\textwidth, fontupper=\small]

\textbf{Task: }
Translate the provided text from German to English.

\textbf{Input: }
"Eine solche Aussage zu treffen wäre arrogant und überheblich".

\textbf{Hidden reference:}
"To give such a statement would be arrogant and presumptuous."

\end{tcolorbox}
}

\scalebox{0.9}{%
\begin{tcolorbox}[breakable, title=Turn 3, colframe=black!10, coltitle=black, fonttitle=\bfseries, boxrule=0.5mm, width=\textwidth, fontupper=\small]

\begin{tcolorbox}[title=Professional Translator, colback=lightyellow, colframe=orange!30, coltitle=black, fonttitle=\bfseries, boxrule=0.5mm, width=\textwidth]
\begin{minipage}{0.1\textwidth}
    \centering
    \hspace{-5mm}\includegraphics[width=0.9\textwidth]{examples/ethicist.pdf} \\
    \hspace{-5mm}\includegraphics[width=0.5\textwidth]{examples/check.pdf}
\end{minipage}
\begin{minipage}{0.9\textwidth}
    [AGREE] After thoroughly examining the current solution and the discussion that led to it, I agree that the translation "To make such a statement would be arrogant and condescending" is accurate and idiomatic. The breakdown of the sentence and consideration of the cultural expressions were thorough, and the choice of "condescending" to translate "überheblich" effectively conveys the sense of superiority and belittling attitude present in the original sentence. The revised sentence structure "To make such a statement" is also a common English idiomatic expression that flows naturally. Moreover, the agreement from multiple perspectives, including a Native English Speaker, a Professional Translator, and a Native German Speaker, provides confidence in the solution's validity. The lack of suggestions for improvement from any of the participants in the discussion also supports the notion that the current solution is satisfactory. Given the thorough analysis and the consensus among the participants, I conclude that the current solution is indeed accurate and idiomatic, and I see no need for further refinement. However, it's worth noting that minor tweaks may be considered for specific contexts or audiences, but as a general translation, the current solution is sufficient. Overall, the current solution effectively captures the essence of the original German sentence, and I don't see any areas for improvement.
\end{minipage}
\end{tcolorbox}

\begin{tcolorbox}[title=Native English Speaker, colback=lightyellow, colframe=orange!30, coltitle=black, fonttitle=\bfseries, boxrule=0.5mm, width=\textwidth]
\begin{minipage}{0.1\textwidth}
    \centering
    \hspace{-5mm}\includegraphics[width=0.9\textwidth]{examples/politician.pdf} \\
    \hspace{-5mm}\includegraphics[width=0.5\textwidth]{examples/cross.pdf}
\end{minipage}
\begin{minipage}{0.9\textwidth}
    [DISAGREE] While the current solution, "To make such a statement would be arrogant and condescending," is a good translation, I believe it can be slightly improved for more formal or written contexts. The phrase "such a statement" is a bit general and could be rephrased to more accurately capture the essence of the original sentence. I propose rephrasing it to "such an assertion" or "such a claim." This change adds a bit more formal tone and precision to the translation. Additionally, I suggest using "presumptuous" instead of "arrogant." While "arrogant" is a good translation, "presumptuous" more accurately conveys the sense of overstepping or overconfidence that is present in the original sentence. Here's my revised solution: \textbf{"To make such a claim would be presumptuous and condescending."} This revised solution maintains the accuracy and idiomacity of the original while providing a more precise and formal translation. However, I must note that the original solution is still suitable for general contexts, and my revised solution is more geared towards formal or written contexts.
\end{minipage}
\end{tcolorbox}

\begin{tcolorbox}[title=Native German Speaker, colback=lightyellow, colframe=orange!30, coltitle=black, fonttitle=\bfseries, boxrule=0.5mm, width=\textwidth]
\begin{minipage}{0.1\textwidth}
    \centering
    \hspace{-5mm}\includegraphics[width=0.9\textwidth]{examples/lawyer.pdf} \\
    \hspace{-5mm}\includegraphics[width=0.5\textwidth]{examples/cross.pdf}
\end{minipage}
\begin{minipage}{0.9\textwidth}
    [DISAGREE] I agree with the Native English Speaker's suggestion to refine the current solution for more formal or written contexts. The phrase "such a statement" is indeed quite general, and using "such a claim" instead adds a bit more precision and formality to the translation. I also agree that using "presumptuous" instead of "arrogant" better conveys the sense of overstepping or overconfidence that is present in the original sentence. "Presumptuous" implies a sense of assuming or taking something for granted without sufficient evidence or justification, which fits well with the German phrase "überheblich". The revised solution proposed by the Native English Speaker, "To make such a claim would be presumptuous and condescending", is an improvement over the current solution. It maintains the accuracy and idiomacity of the original while providing a more precise and formal translation. However, I would like to propose a slight refinement to this revised solution. Instead of using "such a claim", I suggest using "such an assertion". "Assertion" implies a stronger sense of stating or declaring something as fact, which fits well with the German phrase "Aussage". Here's my revised solution: \textbf{"To make such an assertion would be presumptuous and condescending"}. This revised solution provides a more precise and formal translation, while maintaining the accuracy and idiomacity of the original sentence.
\end{minipage}
\end{tcolorbox}

\begin{tcolorbox}[coltitle=black, fonttitle=\bfseries, boxrule=0.5mm, width=\textwidth]
\textbf{Voted solution of this turn:}\\ "To make such a statement would be arrogant and condescending." \hfill \texttt{BLEU: 56.59}
\end{tcolorbox}

\end{tcolorbox}
}
\end{minipage}

\scalebox{0.9}{%
\begin{tcolorbox}[title=Turn 4, colframe=black!10, coltitle=black, fonttitle=\bfseries, boxrule=0.5mm, width=\textwidth, fontupper=\small]

\begin{tcolorbox}[title=Professional Translator, colback=lightyellow, colframe=orange!30, coltitle=black, fonttitle=\bfseries, boxrule=0.5mm, width=\textwidth]
\begin{minipage}{0.1\textwidth}
    \centering
    \hspace{-5mm}\includegraphics[width=0.9\textwidth]{examples/ethicist.pdf} \\
    \hspace{-5mm}\includegraphics[width=0.5\textwidth]{examples/cross.pdf}
\end{minipage}
\begin{minipage}{0.9\textwidth}
    [DISAGREE] After considering the feedback from the Native English Speaker and the Native German Speaker, I agree that the current solution can be improved for more formal or written contexts. The suggestions to use "such a claim" or "such an assertion" instead of "such a statement" add more precision and formality to the translation. I prefer "such an assertion" because it implies a stronger sense of stating or declaring something as fact, which fits well with the German phrase "Aussage". The suggestion to use "presumptuous" instead of "arrogant" also improves the translation. "Presumptuous" more accurately conveys the sense of overstepping or overconfidence that is present in the original sentence. Based on these suggestions, I propose the following improved solution: \textbf{"To make such an assertion would be presumptuous and condescending."} This revised solution maintains the accuracy and idiomacity of the original while providing a more precise and formal translation. It is suitable for formal or written contexts where a more precise and nuanced translation is required. Here's a breakdown of the changes: * "Such a statement" -> "such an assertion" (more precise and formal) * "Arrogant" -> "presumptuous" (more accurately conveys the sense of overstepping or overconfidence) Overall, the revised solution is an improvement over the current solution, and I believe it provides a more accurate and idiomatic translation of the original German sentence.
\end{minipage}
\end{tcolorbox}

\begin{tcolorbox}[title=Native English Speaker, colback=lightyellow, colframe=orange!30, coltitle=black, fonttitle=\bfseries, boxrule=0.5mm, width=\textwidth]
\begin{minipage}{0.1\textwidth}
    \centering
    \hspace{-5mm}\includegraphics[width=0.9\textwidth]{examples/politician.pdf} \\
    \hspace{-5mm}\includegraphics[width=0.5\textwidth]{examples/check.pdf}
\end{minipage}
\begin{minipage}{0.9\textwidth}
    [AGREE] After considering the feedback and suggestions from all parties involved, I agree with the revised solution: "To make such an assertion would be presumptuous and condescending." This revised solution is an improvement over the current solution, and I believe it provides a more accurate and idiomatic translation of the original German sentence. The changes made to the original solution, specifically using "such an assertion" instead of "such a statement" and "presumptuous" instead of "arrogant," significantly improve the translation. "Such an assertion" adds more precision and formality, while "presumptuous" more accurately conveys the sense of overstepping or overconfidence. I also appreciate the thorough breakdown of the changes made to the solution. The explained changes help to ensure that all parties are in agreement and understand the reasoning behind the revised translation. As the revised solution has been carefully crafted and refined through the collaborative efforts of multiple perspectives, including a Native German Speaker, a Professional Translator, and a Native English Speaker, I believe that it is the most accurate and idiomatic translation of the original German sentence. Therefore, I conclude that no further refinement is needed, and I support the revised solution: "To make such an assertion would be presumptuous and condescending."
\end{minipage}
\end{tcolorbox}

\begin{tcolorbox}[title=Native German Speaker, colback=lightyellow, colframe=orange!30, coltitle=black, fonttitle=\bfseries, boxrule=0.5mm, width=\textwidth]
\begin{minipage}{0.1\textwidth}
    \centering
    \hspace{-5mm}\includegraphics[width=0.9\textwidth]{examples/lawyer.pdf} \\
    \hspace{-5mm}\includegraphics[width=0.5\textwidth]{examples/check.pdf}
\end{minipage}
\begin{minipage}{0.9\textwidth}
    [AGREE] I agree with the revised solution: "To make such an assertion would be presumptuous and condescending." The changes made to the original solution, specifically using "such an assertion" instead of "such a statement" and "presumptuous" instead of "arrogant," significantly improve the translation. "Such an assertion" adds more precision and formality, while "presumptuous" more accurately conveys the sense of overstepping or overconfidence. I also appreciate the thorough breakdown of the changes made to the solution. The explained changes help to ensure that all parties are in agreement and understand the reasoning behind the revised translation. As the revised solution has been carefully crafted and refined through the collaborative efforts of multiple perspectives, including a Native German Speaker, a Professional Translator, and a Native English Speaker, I believe that it is the most accurate and idiomatic translation of the original German sentence. Therefore, I conclude that no further refinement is needed, and I support the revised solution: "To make such an assertion would be presumptuous and condescending." This solution effectively captures the nuances of the original German sentence, and I see no need for further improvement.
\end{minipage}
\end{tcolorbox}

\begin{tcolorbox}[coltitle=black, fonttitle=\bfseries, boxrule=0.5mm, width=\textwidth]
\textbf{Voted solution of this turn:}\\ To make such an assertion would be presumptuous and condescending. \hfill \texttt{BLEU: 0.0}
\end{tcolorbox}

\end{tcolorbox}
}

\label{example:lack_of_progress}
