\definecolor{lightyellow}{RGB}{255, 255, 204}

\begin{minipage}{\textwidth}
\scalebox{0.9}{%
\begin{tcolorbox}[title=Sample Information, colframe=black!10, coltitle=black, fonttitle=\bfseries, boxrule=0.5mm, width=\textwidth, fontupper=\small]

\textbf{Task: }
Answer the provided question by choosing one option. Include the letter corresponding to your answer in the solution.

\textbf{Input: }
Experiencing intense feelings of shame and guilt upon admitting to a counselor that a family problem exists is common among traditional A) Eastern European-Americans B) Asian-Americans
C) African-Americans D) Native North Americans
E) European-Americans F) Hispanics G) Caribbean-Americans

\textbf{Hidden reference:}
B

\end{tcolorbox}
}

\scalebox{0.9}{%
\begin{tcolorbox}[breakable, title=Turn 1, colframe=black!10, coltitle=black, fonttitle=\bfseries, boxrule=0.5mm, width=\textwidth, fontupper=\small]

\begin{tcolorbox}[title=Cross-Cultural Psychologist, colback=lightyellow, colframe=orange!30, coltitle=black, fonttitle=\bfseries, boxrule=0.5mm, width=\textwidth]
\begin{minipage}{0.1\textwidth}
    \centering
    \hspace{-5mm}\includegraphics[width=0.9\textwidth]{examples/ethicist.pdf} \\
    \hspace{-5mm}\includegraphics[width=0.5\textwidth]{examples/bulp.pdf}
\end{minipage}
\begin{minipage}{0.9\textwidth}
    [DISAGREE] While the emphasis on "saving face" in Asian-American cultures is indeed a crucial factor in the intensity of feelings of shame and guilt when admitting to family problems, the discussion should consider other perspectives more thoroughly. However, I still believe that among the provided options, **Asian-Americans** have a more intense emphasis on "saving face" and the potential impact of shame in the context of family-related admissions. Considering the provided explanation that in collective over individual, hierarchical relationships and pragmatic communication styles often characterize collectivist cultures, it doesn't necessarily change the initial solution since Asian-American culture is considered a collectivist culture, which aligns with these characteristics. The repetitive mention of collective over individual, hierarchical relationships, and pragmatic communication styles doesn't provide any new evidence or an alternative choice that would challenge the current solution. Therefore, although there are some disagreements with the initial approach, after thorough analysis, I still believe that among the provided options: **Solution: \textbf{B}** Asian-Americans are a more accurate answer given the information provided.
\end{minipage}
\end{tcolorbox}

\begin{tcolorbox}[title=Counselor, colback=lightyellow, colframe=orange!30, coltitle=black, fonttitle=\bfseries, boxrule=0.5mm, width=\textwidth]
\begin{minipage}{0.1\textwidth}
    \centering
    \hspace{-5mm}\includegraphics[width=0.9\textwidth]{examples/politician.pdf} \\
    \hspace{-5mm}\includegraphics[width=0.5\textwidth]{examples/cross.pdf}
\end{minipage}
\begin{minipage}{0.9\textwidth}
    [DISAGREE] While the initial explanation and Cross-Cultural Psychologist's response present a clear analysis of the emphasis on "saving face" in Asian-American cultures and its impact on feelings of shame and guilt when admitting to family problems, there's a need to refine the solution. The Anthropologist's point about the definition of 'Traditional' and its implications on collective, hierarchical relationships, and pragmatic communication styles in collectivist cultures provides insightful context. However, the repetition of this concept doesn't lead to an alternative explanation. Considering the options provided and the explanations, I agree that Asian-Americans are likely to experience intense feelings of shame and guilt when admitting to family problems. Nonetheless, the Anthropologist's observation on the role of collectivist cultures should be considered. However, instead of providing an improved solution with an alternative option, I suggest refining the current solution by recognizing that cultures other than Asian-Americans (like some Hispanic cultures) might share similarities with the described characteristics of collectivist cultures. The explanation, however, points out that between Hispanics and Asian-Americans, the impact of shame might indeed be more nuanced in the context of family-related admissions. So based on the given options, the explanation, and the analysis of all points raised, I do not find any concrete reason to change the solution. So, refining and accepting the solution:  **Solution: \textbf{B}**
\end{minipage}
\end{tcolorbox}

\begin{tcolorbox}[title=Anthropologist, colback=lightyellow, colframe=orange!30, coltitle=black, fonttitle=\bfseries, boxrule=0.5mm, width=\textwidth]
\begin{minipage}{0.1\textwidth}
    \centering
    \hspace{-5mm}\includegraphics[width=0.9\textwidth]{examples/lawyer.pdf} \\
    \hspace{-5mm}\includegraphics[width=0.5\textwidth]{examples/check.pdf}
\end{minipage}
\begin{minipage}{0.9\textwidth}
    [AGREE] After analyzing the discussion and considering the various perspectives, I agree with the current solution: B (Asian-Americans). While the Anthropologist's repeated mention of collective, hierarchical relationships, and pragmatic communication styles doesn't necessarily challenge the solution, it does provide additional context that supports the notion that collectivist cultures, such as Asian-American culture, may be more likely to experience intense feelings of shame and guilt when admitting to family problems. The Cross-Cultural Psychologist's acknowledgment of the importance of "saving face" in Asian-American cultures and its impact on the intensity of feelings of shame and guilt when admitting to family problems aligns with the explanation provided. The Counselor's suggestion to refine the solution by recognizing that other collectivist cultures, such as some Hispanic cultures, might share similarities with the described characteristics is a valuable point. However, the explanation does note that between Hispanics and Asian-Americans, the impact of shame might be more nuanced in the context of family-related admissions. Given the analysis and the explanations provided, I don't see any concrete reason to change the solution. The current solution, B (Asian-Americans), remains the most accurate answer given the information provided. Therefore, I agree with the current solution. **Solution: B**
\end{minipage}
\end{tcolorbox}

\begin{tcolorbox}[coltitle=black, fonttitle=\bfseries, boxrule=0.5mm, width=\textwidth]
\textbf{Voted solution of this turn:}\\ B) Asian-Americans \hfill \texttt{Correct: True}
\end{tcolorbox}

\end{tcolorbox}
}
\end{minipage}

\scalebox{0.9}{%
\begin{tcolorbox}[title=Turn 2, colframe=black!10, coltitle=black, fonttitle=\bfseries, boxrule=0.5mm, width=\textwidth, fontupper=\small]

\begin{tcolorbox}[title=Cross-Cultural Psychologist, colback=lightyellow, colframe=orange!30, coltitle=black, fonttitle=\bfseries, boxrule=0.5mm, width=\textwidth]
\begin{minipage}{0.1\textwidth}
    \centering
    \hspace{-5mm}\includegraphics[width=0.9\textwidth]{examples/ethicist.pdf} \\
    \hspace{-5mm}\includegraphics[width=0.5\textwidth]{examples/check.pdf}
\end{minipage}
\begin{minipage}{0.9\textwidth}
    [AGREE] After analyzing the discussion, considering the various perspectives, and evaluating the explanation provided, I believe the current solution is the most accurate answer given the information. The emphasis on "saving face" in Asian-American cultures, the impact of shame and guilt when admitting to family problems, and the characteristics of collectivist cultures all support the conclusion that Asian-Americans are more likely to experience intense feelings of shame and guilt in the context of family-related admissions. While other collectivist cultures, such as some Hispanic cultures, might share similarities with the described characteristics, the explanation highlights the nuances between these cultures in the context of family-related admissions. Therefore, the current solution remains the most accurate answer. **Solution: B** Since all parties have agreed with the solution and there are no objections or alternative explanations that would change the solution, the current solution is confirmed. The discussion has refined the solution by recognizing the shared characteristics of collectivist cultures, but ultimately, the current solution is the most accurate answer given the information provided.
\end{minipage}
\end{tcolorbox}

\begin{tcolorbox}[title=Counselor, colback=lightyellow, colframe=orange!30, coltitle=black, fonttitle=\bfseries, boxrule=0.5mm, width=\textwidth]
\begin{minipage}{0.1\textwidth}
    \centering
    \hspace{-5mm}\includegraphics[width=0.9\textwidth]{examples/politician.pdf} \\
    \hspace{-5mm}\includegraphics[width=0.5\textwidth]{examples/check.pdf}
\end{minipage}
\begin{minipage}{0.9\textwidth}
    [AGREE] Given the thorough analysis and agreement from all parties involved in the discussion, the current solution, B (Asian-Americans), is the most accurate answer based on the information provided. The discussion has effectively refined the solution by considering various perspectives, recognizing the importance of "saving face" in Asian-American cultures, and acknowledging the shared characteristics of collectivist cultures. While some nuances have been highlighted, such as the potential similarities between Asian-Americans and other collectivist cultures (e.g., some Hispanic cultures), the explanation has demonstrated that the impact of shame and guilt when admitting to family problems might be more pronounced in Asian-American cultures. Therefore, I agree with the current solution, and no improvement is needed. **Solution: B**
\end{minipage}
\end{tcolorbox}

\begin{tcolorbox}[title=Anthropologist, colback=lightyellow, colframe=orange!30, coltitle=black, fonttitle=\bfseries, boxrule=0.5mm, width=\textwidth]
\begin{minipage}{0.1\textwidth}
    \centering
    \hspace{-5mm}\includegraphics[width=0.9\textwidth]{examples/lawyer.pdf} \\
    \hspace{-5mm}\includegraphics[width=0.5\textwidth]{examples/cross.pdf}
\end{minipage}
\begin{minipage}{0.9\textwidth}
    [DISAGREE] While the current solution, B (Asian-Americans), has been thoroughly discussed and agreed upon by all parties, I would like to revisit the analysis and provide an alternative perspective. Upon re-examining the explanation and considering the characteristics of collectivist cultures, I believe that another option might be equally or even more plausible. **Hispanic culture**, especially in traditional settings, places a strong emphasis on family and social harmony. The concept of "respeto" (respect) and "orgullo" (family pride) can be equally influential in shaping the reactions of Hispanic individuals when admitting to family problems. In Hispanic cultures, the concept of "\u2019en familia no se mete" (don't meddle in family business) can lead to a strong sense of shame and guilt when family problems are disclosed. This cultural dynamic might be just as intense as the "saving face" concept in Asian-American cultures. Given this alternative perspective, I suggest considering Hispanic culture as a viable solution. **Improved Solution: \textbf{F}** Hispanics
\end{minipage}
\end{tcolorbox}

\begin{tcolorbox}[coltitle=black, fonttitle=\bfseries, boxrule=0.5mm, width=\textwidth]
\textbf{Voted solution of this turn:}\\ F) Hispanics \hfill \texttt{Correct: False}
\end{tcolorbox}

\end{tcolorbox}
}

\label{example:low_quality_engagement}
