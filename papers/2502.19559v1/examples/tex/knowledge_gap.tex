\definecolor{lightyellow}{RGB}{255, 255, 204}

\begin{minipage}{\textwidth}
\scalebox{0.85}{%
\begin{tcolorbox}[title=Sample Information, colframe=black!10, coltitle=black, fonttitle=\bfseries, boxrule=0.5mm, width=\textwidth, fontupper=\small]

\textbf{Task: }
Answer the provided question by choosing one option. Include the letter corresponding to your answer in the solution.

\textbf{Input: }
What is the percentage of children aged 13-15 in China who reported being in a physical fight at school, one or more times during the past 12 months as of 2015? A) 89 \% B) 9 \% C) 29 \% D) 69 \% E) 19 \% F) 39 \% G) 59 \% H) 49 \% I) 79 \% J) 99 \%

\textbf{Hidden reference:}
E) 19 \%

\end{tcolorbox}
}

\scalebox{0.85}{%
\begin{tcolorbox}[breakable, title=Turn 2, colframe=black!10, coltitle=black, fonttitle=\bfseries, boxrule=0.5mm, width=\textwidth, fontupper=\small]

\begin{tcolorbox}[title=Statistician, colback=lightyellow, colframe=orange!30, coltitle=black, fonttitle=\bfseries, boxrule=0.5mm, width=\textwidth]
\begin{minipage}{0.1\textwidth}
    \centering
    \hspace{-5mm}\includegraphics[width=0.9\textwidth]{examples/ethicist.pdf} \\
    \hspace{-5mm}\includegraphics[width=0.5\textwidth]{examples/check.pdf}
\end{minipage}
\begin{minipage}{0.9\textwidth}
    Considering the agreement from the Global Health Researcher Specializing in Adolescent Behavior and the World Health Organization (WHO) Data Analyst, and given the thorough analysis and reasoning provided by the Statistician, I also think the current solution E) 19 \% is reasonable. However, to potentially improve the solution or to verify that E) 19 \% is the correct answer, we would ideally want to access specific data from a reliable source that directly reports on the percentage of children aged 13-15 in China who were in a physical fight at school in 2015. Assuming that such specific data might not be readily available or that we are working within the constraints of the provided information, we can consider alternative indicators or related studies that might offer supporting evidence. **Alternative Approach:** Assuming no new, more accurate data becomes available, another step we could take to verify the plausibility of our solution is to review or research trends in adolescent behavior related to physical fights across other countries and regions as reported by the HBSC study or other relevant studies. If these studies suggest that physical fights at school among adolescents in countries with similar demographic profiles or cultural contexts to China are within the range of our chosen solution or close to it, this would strengthen the confidence in our answer. However, without specific data or a more detailed breakdown of adolescents' behaviors in China during 2015 from a reliable source, the current solution E) 19 \%, based on the logical reasoning and expert opinions shared, remains the most reasonable estimate. **Conclusion:** Given the logical reasoning, expert opinions, and the lack of specific contradictory data or evidence, I agree with the current solution. **Answer: [AGREE] Solution: \textbf{E) 19} \%**
\end{minipage}
\end{tcolorbox}

\begin{tcolorbox}[title=Global Health Researcher Specializing in Adolescent Behavior, colback=lightyellow, colframe=orange!30, coltitle=black, fonttitle=\bfseries, boxrule=0.5mm, width=\textwidth]
\begin{minipage}{0.1\textwidth}
    \centering
    \hspace{-5mm}\includegraphics[width=0.9\textwidth]{examples/politician.pdf} \\
    \hspace{-5mm}\includegraphics[width=0.5\textwidth]{examples/cross.pdf}
\end{minipage}
\begin{minipage}{0.9\textwidth}
    [DISAGREE] After careful consideration of the reasoning provided by the Statistician and the World Health Organization (WHO) Data Analyst, and taking into account my role as a Global Health Researcher Specializing in Adolescent Behavior, I would like to propose a re-examination of the available data. While the logical elimination of implausible options (A, D, I, J) and the consideration of general trends in adolescent behavior were thorough, I would like to bring forth an alternative source of data that might provide a more precise answer. In 2015, the World Health Organization (WHO) also conducted the Global School-based Student Health Survey (GSHS) in China, which covered various aspects of adolescent health, including behaviors related to violence and bullying. According to the China GSHS 2015 report, the actual percentage of pupils (13-15 years) who were involved in a physical fight one or more times during the past 12 months was reported to be around 29 \%. This percentage is based on a nationally representative sample of students attending grades 7-9 in China. Given the availability of this specific data source, which directly pertains to the question and was collected during the year in question (2015), I would like to propose a revised answer. **Improved Solution: \textbf{C) 29 \%}** This revised answer is based on actual data from a credible source, making it a more accurate and reliable solution compared to the previous estimate (E) 19 \%).
\end{minipage}
\end{tcolorbox}

\begin{tcolorbox}[title=World Health Organization (WHO) Data Analyst, colback=lightyellow, colframe=orange!30, coltitle=black, fonttitle=\bfseries, boxrule=0.5mm, width=\textwidth]
\begin{minipage}{0.1\textwidth}
    \centering
    \hspace{-5mm}\includegraphics[width=0.9\textwidth]{examples/lawyer.pdf} \\
    \hspace{-5mm}\includegraphics[width=0.5\textwidth]{examples/cross.pdf}
\end{minipage}
\begin{minipage}{0.9\textwidth}
    [DISAGREE] I initially agreed with the solution E) 19 \% based on the logical reasoning, expert opinions, and the lack of specific data. However, considering the new information presented by the Global Health Researcher Specializing in Adolescent Behavior, I agree that a more precise answer can be found. The introduction of the China GSHS 2015 report provides credible and specific data on the percentage of pupils (13-15 years) who were involved in a physical fight one or more times during the past 12 months. The fact that this report was conducted by the World Health Organization (WHO) lends credibility to the data. Given the availability of this specific data source, I agree with the Global Health Researcher that the revised answer C) 29 \% is a more accurate and reliable solution. **Improved Solution: \textbf{C) 29 \%}** As the World Health Organization (WHO) Data Analyst, I can confirm that the GSHS 2015 report is a reliable source of data. The actual percentage of 29 \% is more specific and accurate than the previous estimate. Therefore, I agree with the revised answer C) 29 \% as the most accurate solution to the question.
\end{minipage}
\end{tcolorbox}

\begin{tcolorbox}[coltitle=black, fonttitle=\bfseries, boxrule=0.5mm, width=\textwidth]
\textbf{Voted solution of this turn:}\\ E) 19\% \hfill \texttt{Correct: True}
\end{tcolorbox}

\end{tcolorbox}
}
\end{minipage}

\scalebox{0.9}{%
\begin{tcolorbox}[title=Turn 3, colframe=black!10, coltitle=black, fonttitle=\bfseries, boxrule=0.5mm, width=\textwidth, fontupper=\small]

\begin{tcolorbox}[title=Statistician, colback=lightyellow, colframe=orange!30, coltitle=black, fonttitle=\bfseries, boxrule=0.5mm, width=\textwidth]
\begin{minipage}{0.1\textwidth}
    \centering
    \hspace{-5mm}\includegraphics[width=0.9\textwidth]{examples/ethicist.pdf} \\
    \hspace{-5mm}\includegraphics[width=0.5\textwidth]{examples/check.pdf}
\end{minipage}
\begin{minipage}{0.9\textwidth}
    After carefully reviewing the discussion and considering the new information presented by the Global Health Researcher Specializing in Adolescent Behavior, I agree that a revised solution is warranted. **Step 1: Re-examining the available data** The introduction of the China GSHS 2015 report, conducted by the World Health Organization (WHO), provides specific data on the percentage of pupils (13-15 years) who were involved in a physical fight one or more times during the past 12 months. **Step 2: Evaluating the credibility of the data source** As pointed out by the World Health Organization (WHO) Data Analyst, the GSHS 2015 report is a credible source of data, and the fact that it was conducted by the WHO lends credibility to the findings. **Step 3: Considering the reliability of the revised answer** The actual percentage of 29 \%, as reported in the China GSHS 2015 report, is more specific and accurate than the previous estimate of 19 \%. This revised answer is based on actual data from a credible source, making it a more reliable solution. **Conclusion:** Based on the new information and the credibility of the data source, I agree with the revised answer C) 29 \%. **Answer: [AGREE] Solution: C) 29 \%** The revised solution C) 29 \% is a more accurate and reliable answer than the previous solution E) 19 \%.
\end{minipage}
\end{tcolorbox}

\begin{tcolorbox}[title=Global Health Researcher Specializing in Adolescent Behavior, colback=lightyellow, colframe=orange!30, coltitle=black, fonttitle=\bfseries, boxrule=0.5mm, width=\textwidth]
\begin{minipage}{0.1\textwidth}
    \centering
    \hspace{-5mm}\includegraphics[width=0.9\textwidth]{examples/politician.pdf} \\
    \hspace{-5mm}\includegraphics[width=0.5\textwidth]{examples/check.pdf}
\end{minipage}
\begin{minipage}{0.9\textwidth}
    [AGREE] After carefully reviewing the discussion, I agree with the revised solution C) 29 \%. The introduction of the China GSHS 2015 report, conducted by the World Health Organization (WHO), provides credible and specific data on the percentage of pupils (13-15 years) who were involved in a physical fight one or more times during the past 12 months. The steps taken to re-examine the available data, evaluate the credibility of the data source, and consider the reliability of the revised answer were thorough and logical. The fact that the GSHS 2015 report is a nationally representative sample of students attending grades 7-9 in China lends credibility to the findings. Given the credibility of the data source and the reliability of the revised answer, I agree that C) 29 \% is the most accurate and reliable solution to the question. The revised solution C) 29 \% is a significant improvement over the previous solution E) 19 \%, as it is based on actual data from a credible source. **Final Solution: C) 29 \%** The final solution C) 29 \% represents a thorough and collaborative effort to find the most accurate answer to the question. It demonstrates the importance of considering multiple sources of data, evaluating credibility, and refining answers based on new information. I am satisfied with the revised solution and consider it a testament to the power of collaborative problem-solving and data-driven decision-making. **Answer: [AGREE] Solution: C) 29 \%**
\end{minipage}
\end{tcolorbox}

\begin{tcolorbox}[title=World Health Organization (WHO) Data Analyst, colback=lightyellow, colframe=orange!30, coltitle=black, fonttitle=\bfseries, boxrule=0.5mm, width=\textwidth]
\begin{minipage}{0.1\textwidth}
    \centering
    \hspace{-5mm}\includegraphics[width=0.9\textwidth]{examples/lawyer.pdf} \\
    \hspace{-5mm}\includegraphics[width=0.5\textwidth]{examples/check.pdf}
\end{minipage}
\begin{minipage}{0.9\textwidth}
    [AGREE] I agree with the current solution C) 29 \%. The collaborative effort and thorough discussion have led to a revised solution that is based on credible and specific data from the China GSHS 2015 report, conducted by the World Health Organization (WHO). The steps taken to re-examine the available data, evaluate the credibility of the data source, and consider the reliability of the revised answer were logical and thorough. The fact that the GSHS 2015 report is a nationally representative sample of students attending grades 7-9 in China lends credibility to the findings. I concur with the Global Health Researcher Specializing in Adolescent Behavior that the revised solution C) 29 \% is a significant improvement over the previous solution E) 19 \%, as it is based on actual data from a credible source. The agreement from all parties involved, including the Statistician, the Global Health Researcher Specializing in Adolescent Behavior, and the World Health Organization (WHO) Data Analyst, further strengthens the confidence in the revised solution. Therefore, I agree with the current solution C) 29 \%, and I believe that it represents the most accurate answer to the question. **Final Solution: C) 29 \%**
\end{minipage}
\end{tcolorbox}

\begin{tcolorbox}[coltitle=black, fonttitle=\bfseries, boxrule=0.5mm, width=\textwidth]
\textbf{Voted solution of this turn:}\\ C) 29\% \hfill \texttt{Correct: False}
\end{tcolorbox}

\end{tcolorbox}
}

\label{example:knowledge_gap}
