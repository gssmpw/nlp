\section{Related Work}
Existing benchmarks \cite{ASDiv,GSM8K,SVAMP,mawps}  assess the reasoning capabilities of LLMs by posing queries and evaluating their responses. However, these benchmarks do not fully capture the scientific discovery potential of LLMs. For true scientific discovery, LLMs must integrate information gained from critical decisions, which in turn influences their future decision-making performance. 

Our benchmark is based on the principles of causal graph structure discovery, albeit with a few relaxations. While it shares similarities with existing methods in causal discovery \cite{jiralerspong2024efficient,long2023can,choi2022lmpriors,kiciman2023causal}, its distinctiveness emerges from the iterative updating of knowledge based on the model's decisions about the underlying graphs. Traditional LLM-based causal graph discovery methods typically rely on meta-data associated with variables, querying whether an edge exists between two nodes (A, B), sometimes evaluating all possible pairs and other times employing strategies to reduce the number of queries. 

Unlike these approaches that primarily focus on querying and response generation, our setup prompts models to suggest optimal interventions, enabling them to uncover hidden structures. This experimental framework is analogous to scientific discovery, where the model continuously updates its knowledge through a series of interventions (experiments) to reveal new insights (causal graph discovery).

%%%%%%%%% Chemistry Figure %%%%%%%%%
\begin{figure}[t]
\centering
{\includegraphics[width=0.95\linewidth]{Fig/Chemistry.pdf}
}\\
\caption{Illustration of the Chemistry setting. The brackets indicate (molecule index, molecule state). Figures (a) and (b) illustrate the change in state after an intervention on molecule 0. Figures (c) and (d) present a case where causal graph A and causal graph B result in the same observations.}
\label{Chemistry}
\end{figure}
%%%%%%%%%%%%%%%%%%%%%%%%%%%%%%%%%%%%