\section{Results}
\label{sec:results}

In this section we evaluate 
the results of 
morphological pretraining (Sect.~\ref{results-pretraining-performance}),
zero- and few shot evolution (using the pretrained model; Sect.~\ref{results-zero-and-few-shot-evolution}),
and simultaneous co-design from scratch (without pretraining; Sect.~\ref{results-simultaneous-co-design}).


\subsection{Pretraining performance}
\label{results-pretraining-performance}

Across three independent trials,
each using a distinct dataset of 
randomly-generated morphologies and environments,
pretraining 
exhibited stable learning trajectories 
with low variance across trials (Fig.~\ref{fig:results-performance}A), 
converging in approximately 1,400 learning steps (56 minutes of wall-clock time).
% 
Loss was defined as the ratio of final to initial distance from the target light source. 
At initialization with random controller weights, this ratio was 1.0, indicating robots remained stationary throughout simulation. 
After pretraining, the loss stabilized at approximately 0.3, representing a 70\% improvement. 
That is, in environments sampled from the training distribution, robots using the pretrained universal controller traversed an average of 70\% of their initial distance to the light source. 
Since each training batch used novel morphologies, we omitted model selection with a \mbox{validation set.}

% The pretrained model effectively controlled diverse, previously unseen robot morphologies in challenging tasks at the training distribution's edge (Appx.~\ref{appendix-dataset-random-robot-generation}). 
% Fig.~\ref{fig:results-zero-shot-transfer} shows the pretrained model's generalization performance across more than 24,000 distinct test morphologies. 
% Despite the test environments being more challenging than the training distribution, the mean performance of the test morphologies aligned with the final loss observed during pretraining. 

To visualize the breadth of morphological diversity handled by the pretrained controller, Fig.~\ref{fig:appendix-zero-shot-robot-grid} showcases a representative sample of successful robots. 
These examples were selected uniformly from the top-performing 50\% of the test morphology set. 
The selected bodies exhibit high variation in both scale and morphological characteristics demonstrating the non-trivial generalization of the universal controller.


%%%%%%%%%%%%%%%%%%%%%%%%%%%%%%%%%%%%%%%%%%%
\begin{figure}[t]
    \centering
    \includegraphics[width=\columnwidth]{figs/shape-stats.pdf}
    \vspace{-20pt}
    \caption{\textbf{Evolved populations.}
    Population performance, phenotype footprint size, and body mass for the initial (randomly generated) and evolved design populations.
    }
    \vspace{-18pt}
    \label{fig:appendix-robot-data-gen-stats}
\end{figure}
%%%%%%%%%%%%%%%%%%%%%%%%%%%%%%%%%%%%%%%%%%%

\subsection{Zero- and few shot evolution (with pretraining)}
\label{results-zero-and-few-shot-evolution}

A population of morphologies was evolved through random mutation and crossover operations, using the pretrained universal controller.
On the same challenging set of tasks used for evaluating pretrained controller generalization,
the population converges to near optimal performance
in 100 generations of evolution (17 minutes of wall-clock time)
without finetuning the controller (``zero shot evolution''; Sect.~\ref{methods-zero-shot-evolution}).
% 
Although zero-shot evolution shows rapid convergence in controlling thousands of distinct bodies, this success masks a key pattern: design population diversity decreases as performance improves. 
Fig.~\ref{fig:results-performance}C reveals this pattern---after a brief diversity spike at evolution's onset, the population gradually homogenizes. 
We term this phenomenon diversity collapse, measuring diversity as the population's mean, pairwise Hamming distance in (and normalized to) the genotype space $\mathcal{G}$ (defined in Sect.~\ref{methods-morphology-design-space}). This metric naturally reflects differences in morphology (body) as well as sensing and actuation masking in the universal controller (brain).

We found that generational finetuning of the universal controller for the current population (``few shot evolution'')
not only preserves diversity but in fact significantly increases diversity (Fig.~\ref{fig:results-performance}D).
This is a somewhat surprising result as there was no explicit selection pressure to maintain diversity.
The process of morphological evolution seems to intrinsically increase population diversity. 
However, in absence of generational finetuning, there is a tipping point at which it is easier to purge diversity, replacing the worst designs with slightly modified clones of the best, than to discover novel morphological innovations with superior performance.


\subsection{Simultaneous co-design (\textit{without} pretraining)}
\label{results-simultaneous-co-design}

Ablating pretraining (and funetuning),
and instead simultaneously optimizing morphology and universal control, together from scratch, 
results once again in rapid diversity collapse (Fig.~\ref{fig:results-performance}B).
Performance plateaus in well under 180 generations, corresponding to 360 controller learning steps and 109 minutes of wall-clock time.
% 
The extent of diversity collapse can be seen in Fig.~\ref{fig:results-morpho-grid}B, where we visualize morphologies from one of the three independent trials,
and in Fig.~\ref{fig:appendix-robot-data-gen-stats} where we plot morphological variance across evolved populations in terms of footprint size and body weight.

In all three co-design paradigms (zero shot, few shot, simultaneous), universal control enabled successful crossover (Fig.~\ref{fig:appendix-variation-operator-stats}).
In terms of offspring survival,
crossover was initially much more successful than mutation.
But in the case of simultaneous co-design, this was not an apples to apples comparison because each generation provided the controller with more time to learn how to control the population, and the randomly initialized controller was very bad at the task.
And so it was not clear if the success of offspring was due to changes in parent morphology or improvements to the universal controller.
The superior performance of pretraining across random morphologies, shows that the designs produced by crossover during simultaneous co-design were no better than random designs. 
In zero-shot and few-shot evolution, however, the pretrained controller is quite good at the very start, and in zero-shot the controller is not updated during evolution, providing clear evidence of successful crossover prior to diversity collapse.

