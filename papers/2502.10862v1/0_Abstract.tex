\begin{abstract}

The co-design of robot morphology and neural control typically requires using reinforcement learning to approximate a unique control policy gradient for each body plan, demanding massive amounts of training data to measure the performance of each design. Here we show that a universal, morphology-agnostic controller can be rapidly and directly obtained by gradient-based optimization through differentiable simulation. This process of morphological pretraining allows the designer to explore non-differentiable changes to a robot's physical layout (e.g.~adding, removing and recombining discrete body parts) and immediately determine which revisions are beneficial and which are deleterious using the pretrained model. We term this process ``zero-shot evolution'' and compare it with the simultaneous co-optimization of a universal controller alongside an evolving design population. We find the latter results in \textit{diversity collapse}, a previously unknown pathology whereby the population---and thus the controller's training data---converges to similar designs that are easier to steer with a shared universal controller. We show that zero-shot evolution with a pretrained controller quickly yields a diversity of highly performant designs, and by fine-tuning the pretrained controller on the current population throughout evolution, diversity is not only preserved but significantly increased as superior performance is achieved.

\end{abstract}


