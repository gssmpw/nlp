%%%%%%%% ICML 2025 EXAMPLE LATEX SUBMISSION FILE %%%%%%%%%%%%%%%%%

\documentclass{article}

% Recommended, but optional, packages for figures and better typesetting:
\usepackage{microtype}
\usepackage{graphicx}
% \pdfcompresslevel=0
% \pdfobjcompresslevel=0
\usepackage{subfigure}
\usepackage{booktabs} % for professional tables

% hyperref makes hyperlinks in the resulting PDF.
% If your build breaks (sometimes temporarily if a hyperlink spans a page)
% please comment out the following usepackage line and replace
% \usepackage{icml2025} with \usepackage[nohyperref]{icml2025} above.
\usepackage{hyperref}

% Attempt to make hyperref and algorithmic work together better:
\newcommand{\theHalgorithm}{\arabic{algorithm}}

% Use the following line for the initial blind version submitted for review:
% \usepackage{icml2025}


% If accepted, instead use the following line for the camera-ready submission:
\usepackage[accepted]{icml2025}

%%%%%%%%%%%% !!!! %%%%%%%%%%%%

%SK: removed ICML accepted footnote for preprint; switch to anonymous for openreview

%%%%%%%%%%%% !!!! %%%%%%%%%%%%

% For theorems and such
\usepackage{amsmath}
\usepackage{amssymb}
\usepackage{mathtools}
\usepackage{amsthm}

% if you use cleveref..
\usepackage[capitalize,noabbrev]{cleveref}

%%%%%%%%%%%%%%%%%%%%%%%%%%%%%%%%
% THEOREMS
%%%%%%%%%%%%%%%%%%%%%%%%%%%%%%%%
\theoremstyle{plain}
\newtheorem{theorem}{Theorem}[section]
\newtheorem{proposition}[theorem]{Proposition}
\newtheorem{lemma}[theorem]{Lemma}
\newtheorem{corollary}[theorem]{Corollary}
\theoremstyle{definition}
\newtheorem{definition}[theorem]{Definition}
\newtheorem{assumption}[theorem]{Assumption}
\theoremstyle{remark}
\newtheorem{remark}[theorem]{Remark}

% Todonotes is useful during development; simply uncomment the next line
%    and comment out the line below the next line to turn off comments
%\usepackage[disable,textsize=tiny]{todonotes}
\usepackage[textsize=tiny]{todonotes}


% The \icmltitle you define below is probably too long as a header.
% Therefore, a short form for the running title is supplied here:
\icmltitlerunning{Accelerated 
co-design of robots
through morphological pretraining}

\begin{document}

\twocolumn[
\icmltitle{Accelerated 
co-design of robots
through morphological pretraining}

% It is OKAY to include author information, even for blind
% submissions: the style file will automatically remove it for you
% unless you've provided the [accepted] option to the icml2025
% package.

% List of affiliations: The first argument should be a (short)
% identifier you will use later to specify author affiliations
% Academic affiliations should list Department, University, City, Region, Country


\begin{icmlauthorlist}
\icmlauthor{Luke Strgar}{n}
\icmlauthor{Sam Kriegman}{n}
\end{icmlauthorlist}

\icmlaffiliation{n}{Northwestern University, Evanston, IL, USA}

\icmlcorrespondingauthor{Luke Strgar}{lvs@u.northwestern.edu}
\icmlcorrespondingauthor{Sam Kriegman}{sam.kriegman@northwestern.edu}

% You may provide any keywords that you find helpful for describing your paper
% These are used to populate the "keywords" metadata in the PDF but will not be shown in the document
\icmlkeywords{Phototaxis, Soft robot}

\vskip 0.3in
]

\printAffiliationsAndNotice{}


\begin{figure*}[t]
    \centering
    % \includegraphics[width=\textwidth]{figs/dataflow-teaser.pdf}
    \includegraphics[width=\textwidth]{figs/teaser.jpg}
    \vspace{-20pt}
    \caption{\textbf{Universal control of differentiable robots.}
    Large-scale pretraining and finetuning
    of a universal controller was achieved by
    averaging simulation gradients across the robot's body, world, and goal.
    The controller is shared by an arbitrarily large and morphologically diverse population of robots as
    they undergo morphological evolution.
    The objective is to find designs that can move quickly across a previously-unseen terrain toward a randomly-positioned light source (glowing white spheres).
    }
    \vspace{-10pt}
    \label{fig:intro-robot-teaser}
\end{figure*}

The escalating challenges of managing vast sensor-generated data, particularly in audio applications, necessitate innovative solutions. Current systems face significant computational and storage demands, especially in real-time applications like gunshot detection systems (GSDS), and the proliferation of edge sensors exacerbates these issues. This paper proposes a groundbreaking approach with a near-sensor model tailored for intelligent audio-sensing frameworks. Utilizing a Fast Fourier Transform (FFT) module, convolutional neural network (CNN) layers, and HyperDimensional Computing (HDC), our model excels in low-energy, rapid inference, and online learning. It is highly adaptable for efficient ASIC design implementation, offering superior energy efficiency compared to conventional embedded CPUs or GPUs, and is compatible with the trend of shrinking microphone sensor sizes. Comprehensive evaluations at both software and hardware levels underscore the model's efficacy. Software assessments through detailed ROC curve analysis revealed a delicate balance between energy conservation and quality loss, achieving up to 82.1\% energy savings with only 1.39\% quality loss. Hardware evaluations highlight the model's commendable energy efficiency when implemented via ASIC design, especially with the Google Edge TPU, showcasing its superiority over prevalent embedded CPUs and GPUs.

%具身智能体在复杂场景下 manipulation 的 performance robustness 和泛化能力始终是一个广受关注的研究方向。其中,visuomotor imitation learning 是具身智能体 Policy 的主流范式之一,它允许 agent 从高维视觉观察和机器人本体感知中 effectively 学习 manipulation skills。
%然而,增加场景的复杂度和 visual distraction,会导致在简单场景下表现良好的决策模型性能下降。实际上,不仅是 simple imitation learning policy,先进的多模态 foundation models such as GPT-4o 或 vision language action models (VLA),也不能很好地关注一张语义丰富的图片中的特定的局部问题。对于 robot control or 多模态大模型,其往往侧重于 action prediction, observation mapping or 多模态 alignment,而缺少直观的视觉感知增强。模型需要隐性地或遵循 high-level text instruction 从相关的视觉区域中获得面向任务语义的定位知识。
%To tackle this challenge problem, we introduce Imit Diff, a diffusion transformer imitation learning framework with dual resolution enhancement guided by fine-grained semantics information。具体来说,our work 有三个关键组成部分。
%1) Semanstic Injection. Imit Diff 通过 vision language models (VLM) 和 vision foundation models 的 pretrain knowledge 将面向任务的语义信息和高层文本指导转化为显式的 pixel-level 视觉定位标签,注入到 environment observation中。
%2) Dual Res Fusion。 我们构建了双分辨率图像观测流,使用双分辨率视觉编码器分别提取全局和细粒度视觉特征。多尺度视觉信息随后在 attention block 中进行融合,在保证计算 effiency 的前提下,为全局视觉观测引入多尺度细粒度信息,提升场景理解能力。
%3) Consistency policy on diffusion transformer。Diffusion based imitation policies 通常受到 denoise times 的困扰。我们建立了基于 consistency policy 的 DiT action head。Policy 的决策层可以通过 single step denoise 实现系统高频响应。额外地,受益于较快的 inference time,我们引入 temperal ensemble 改善预测动作的平滑性。
%我们设计了四个在 manipulation 精细度上具有挑战性的现实世界任务来评估 Imit Diff,并通过增加场景复杂度和 visual distraction 来测试模型的场景理解能力。额外地,我们设计了 visual distraction 和 category generalization 的 zero shot 实验来验证模型是否受益于 dual res enhancement framework and fine-grained semantics injection。实验结果表明,Imit Diff outperforms 现有的 strong baselines。
%In summary, the contributions of our work are three-fold:
%1) We propose Imit Diff, a DiT architecture imitation learning framework with dual res enhancement guied by fine-grained semantics information.
%2) 我们构建了 open-set vision foundation models pipeline 来获得显式视觉遮罩。该方法能够有效处理机器人控制场景的运动模糊、遮挡、物体丢失情况。并将其作为 fine-grained 语义信息引导 policy decision。
%3) 我们在DiT上实现了consistency policy,显著减少了模型推理时间。通过异步控制框架,实现了 open-set vision foundation models 工作流下的实时控制。
%The code will be publicly available soon。

\section{Introduction}


\label{Intro}
The performance robustness and generalization capabilities of embodied agents in complex manipulation scenarios have long been a focus of significant research interest \citep{ju2025robo, yuan2024learning}. Visuomotor imitation learning is one of the mainstream paradigms of robot manipulation policy \citep{chi2023diffusion, shridhar2023perceiver, ze2023gnfactor, florence2022implicit, hansen2022pre}. This approach enables agents to derive state estimation and decision-making capabilities from expert demonstrations that incorporate high-dimensional visual observations and robot proprioception \citep{ze20243d}.

However, as scene complexity and visual distractions increase, the performance of decision models that excel in simpler environments tends to degrade \citep{zheng2024instruction, liurobustness}. Not only do simple imitation learning policies face challenges, but even advanced multimodal foundation models, such as GPT-4o \citep{hurst2024gpt} or vision language action models (VLA) \citep{liu2024rdt, brohan2022rt, brohan2023rt, o2023open, kim2024openvla, wen2024diffusion}, struggle to accurately focus on specific details within semantically complex images. In fact, in robot control and embodied multimodal foundation models, the focus is often on action prediction, observation mapping, or multimodal alignment. Therefore, intuitive visual perception enhancement is typically lacking. Models can only acquire task-oriented semantic localization knowledge from relevant visual regions either implicitly or when guided by high-level text instructions \citep{reuss2023multimodal}.

To tackle this challenge problem, we introduce \textbf{Imit Diff}, a diffusion transformer imitation learning framework with dual resolution enhancement guided by fine-grained semantics information. Specifically, our work has three key components:

\begin{enumerate}

\item \textbf{Semanstic injection.} Imit Diff transforms task-oriented semantic information and high-level textual guidance into explicit pixel-level visual localization labels through the pretrain knowledge of vision language models (VLM) and vision foundation models, and injects them into the policy observation.

\item \textbf{Dual resolution (dual res) fusion.} We develop a dual res image observation stream and employed a dual res vision encoder to extract global and fine-grained visual features. The extracted multi-scale visual information is subsequently fused within an attention block, integrating fine-grained details into the global visual feature. This approach enhances scene understanding while maintaining computational efficiency.

\item \textbf{Consistency policy on diffusion transformer (DiT).} Diffusion-based imitation policies often suffer from inefficiencies due to the required denoising steps. To address this, we design a DiT \citep{peebles2023scalable} action head incorporating a consistency policy \citep{song2023consistency}, enabling the decision layer to achieve high-frequency system responses through single-step denoising. Furthermore, leveraging faster inference times, we introduce temperal ensemble to enhance the smoothness of predicted actions.

\end{enumerate}

We design four real-world tasks with challenging manipulation precision to evaluate Imit Diff and test the model's scene understanding capabilities by introducing increased scene complexity and visual distractions. Additionally, we conducted zero-shot experiments on visual distraction and category generalization to assess the benefits of the dual res enhancement framework and fine-grained semantic injection. Experimental results demonstrate that Imit Diff significantly outperforms existing strong baselines. 

In summary, the contributions of our work are three-fold:

\begin{enumerate}

\item We propose Imit Diff, a DiT architecture imitation learning framework with dual res enhancement guied by fine-grained semantics information.

\item We developed an open-set vision foundation model pipeline to generate explicit visual masks. This approach effectively addresses challenges such as motion blur, occlusion, and object loss in robot control scenarios, leveraging the generated masks as fine-grained semantic information to guide policy decisions.

\item We implemented a consistency policy on DiT, which significantly reduced the model inference time. Through the asynchronous control framework, we achieved real-time control under the workflow of open-set vision foundation models.

\end{enumerate}

The code will be made publicly available soon.

\begin{figure}[!hb]
    \centering
    % \vspace{-8pt}
    \includegraphics[width=\columnwidth]{figs/dataflow-teaser-1col.pdf}
    \vspace{-16pt}
    \caption{\textbf{Overview of the proposed method.}
    End-to-end differentiable policy training across tens of millions of morphologically distinct robots---morphological pretraining---produces a universal controller, which was kept frozen throughout zero-shot evolution
    and finetuned for each generation of few-shot evolution.
    }
    \vspace{-8pt}
    \label{fig:intro-dataflow}
\end{figure}

\begin{figure*}[t]
    \centering
    \includegraphics[width=\textwidth]{figs/phylogenetic-tree.jpg}
    \vspace{-14pt}
    \caption{\textbf{Few-shot evolution.}
    % Co-design of body (morphology) and brain (control) using a pretrained universal controller.
    A population of 8192 
    initially random designs (a pair of which are shown in the top row)
    were randomly recombined and mutated to produce 8192 offspring, temporarily expanding the population to 16384 designs.
    All designs in the population were driven by the same universal controller, which was rapidly pretrained (before evolution)
    and finetuned for the current population (at every generation of evolution) using analytical gradients
    from differentiable simulation.
    % , which was lightly finetuned started from the pretrained model at each generation. 
    Deleting the worst performing designs and replacing them with the best offspring, and repeating this process for several generations,
    yields a diversity of increasingly performant designs, and ultimately a final population of 8192 winning designs (bottom row), each with their own unique evolutionary history (phylogeny).
    An example phylogenetic tree, colored by loss (decreasing from gray to cyan to pink), is shown for one of winning designs.
    }
    \vspace{-8pt}
    \label{fig:intro-phylogenetic-tree}
\end{figure*}

\begin{figure}[t]
    \centering
    \includegraphics[width=\columnwidth]{figs/gps.jpg}
    \vspace{-18pt}
    \caption{\textbf{Genotype to phenotype.}
    Designs are 
    encoded by voxel genotype (\textbf{A}), 
    which is expressed as a
    spring-mass phenotype (\textbf{B}),
    and evaluated in a
    differentiable environment (\textbf{C}).
    The springs (teal lines in B and C) and masses (small orange spheres) are motorized and sensorized, respectively.
    }
    \vspace{-12pt}
    \label{fig:methods-geno-pheno-sim}
\end{figure}

% \begin{figure}[b]
    \centering
    \vspace{-10pt}
    \includegraphics[width=\columnwidth]{figs/xover.pdf}
    \vspace{-18pt}
    \caption{\textbf{Recombination of substructures.}
    A pair of designs (parents; \textbf{A, B}) is combined via crossover to produce a new design  (offspring; \textbf{C}) that inherits components from both parents.
}
    \label{fig:results-xover}
    \vspace{-8pt}
\end{figure}  % inside methods
% \begin{figure*}[t]
    \centering
    \includegraphics[width=0.975\textwidth]{figs/com_trace.jpg}
    \vspace{-10pt}
    \caption{\textbf{Evolution of phototaxis.}
    The five worst designs in the population are depicted before (top row) and after (bottom row) zero-shot evolution.
    Each design (black dotted footprints) was placed in the center of a randomly generated map.
    Before evolution,
    not all of designs in the population could
    move (gray to white trajectories)
    across any terrain
    toward a light source (gold stars) using the pretrained controller.
    After evolution, they could.
    One of the design principles that evolution discovered is that larger footprints increase locomotion stability.
    }
    \vspace{-12pt}
    \label{fig:results-com-traces}
\end{figure*} % inside methods
% \begin{figure*}[t]
    \centering
    \includegraphics[width=0.95\textwidth]{figs/performance.pdf}
    \vspace{-8pt}
    \caption{\textbf{Performance and diversity.} 
    Morphological pretraining (\textbf{A}) converges with 70\% improvement from baseline. 
    Simultaneous co-design (from scratch without pretraining; \textbf{B}) achieves similar training loss; 
    but, population diversity (mean pairwise Hamming distance on genotypes) collapses as evolution converges to a single species of similar designs which simplifies shared control.
    Zero shot evolution (using the pretrained controller; \textbf{C}) rapidly improves test performance, but also suffers diversity collapse as evolution compiles slightly modified clones of the designs that are the most compatible with the pretrained model.
    Few-shot evolution (\textbf{D}) resets the pretrained controller at the start of each generation 
    % (spikes in the training curve) 
    and performs 60 finetuning steps per generation. 
    This significantly increases and sustains diversity as well as performance. 
    Solid lines represent 
    the batch (training; blue)
    or
    population (test; orange) means, 
    averaged across three independent trials.
    Shaded regions surrounding the mean curves show the minimum and maximum values across the three trials.
    }
    \vspace{-10pt}
    \label{fig:results-performance}
\end{figure*} % inside methods

\section{Methods}
\label{sec:methods} 

In this section we describe 
the morphological design space (the ``morphospace''; Sect.~\ref{methods-morphology-design-space}),
the simulated physical environment (Sect.~\ref{methods-simulator}),
the universal controller (Sect.~\ref{methods-universal-control}),
morphological pretraining (Sect.~\ref{methods-pretraining}),
zero shot evolution (Sect.~\ref{methods-zero-shot-evolution}),
few shot evolution with generational finetuning (Sect.~\ref{methods-few-shot-evolution}),
and 
simultaneous co-design (Sect.~\ref{methods-codesign}).


\subsection{Morphospace}
\label{methods-morphology-design-space}

Robot morphologies were genetically encoded as contiguous collections of voxels 
within a $6 \times 6 \times 4$ (Length $\times$ Width $\times$ Height) binary genotype workspace, $\mathcal{G}$. 
% Connectivity was defined through 6-neighbor adjacency (face-adjacent voxels), ensuring physically realizable robots without disconnected parts. 
Voxelized genotypes were then mapped to a phenotype space $\mathcal{P}$ comprising 
masses $\mathcal{M}$ and springs $\mathcal{S}$ 
arranged in a cubic lattice 
with 10 cm$^3$ unit cells 
(Fig.~\ref{fig:methods-geno-pheno-sim}).
% 
More specifically, a genotype voxel at position $(i,j,k)$ in $\mathcal{G}$ is expressed phenotypically by eight masses,
one in each corners of the corresponding cubic cell in $\mathcal{P}$ with coordinates $(0.1i+\delta_x,0.1j+\delta_y,0.1k+\delta_z)$ where $\delta_{x,y,z} \in \{0,0.1\}$. 
Springs are then connected to these masses in two patterns: (1) axial springs along cube edges, and (2) planar diagonal springs in each face. 
Adjacent genotype voxels share masses and springs at their interfaces, 
ensuring that contiguous structures in $\mathcal{G}$ mapped to cohesive mass-spring networks in $\mathcal{P}$.

The resultant $6 \times 6 \times 4$ workspace accommodated a maximum of $|\mathcal{M}|=245$ potential mass positions and $|\mathcal{S}|=1648$ potential springs. 
Each robot was centered in the x-y plane according to its center of mass and shifted to the bottom of the workspace to ensure ground contact prior to behavior. 
This procedure ensured stable initial conditions for locomotion while maintaining consistent relative positioning between robots of different morphologies.

To identify unique morphologies, we defined an equivalence relation on the genotype space that accounted for translations and symmetries. Two genotypes were considered identical if, after aligning their occupied voxels to the origin, one could be transformed into the other through any combination of: (1) 90° rotations about the z-axis, (2) reflection about the x-axis, or (3) reflection about the y-axis. Each unique design was represented by its lexicographically minimal form across all such transformations.

\subsection{Differentiable simulation}
\label{methods-simulator}

We here extend the differentiable 2D mass-springs simulators developed by \citet{hu2019difftaichi} and \cite{strgar2024evolution} to three dimensions and add exterception: perception of external stimuli outside the body, namely light.
% we simulated 3D robots using a Hookean mass-spring system with semi-implicit Euler integration. 
Masses on $\mathcal{M}$ hosting photoreceptors were connected by actuating springs on $\mathcal{S}$ (defined above in Sect.~\ref{methods-morphology-design-space}), which exerted forces on their endpoint masses to perform phototaxis: movement toward a light source. 

During simulation, 
% the universal controller (which detailed in the following section, Sect.~\ref{methods-universal-control})
spring rest lengths may be actuated continuously between $\pm 20\%$ of their initial values derived from $\mathcal{P}$ (see Sect.~\ref{methods-morphology-design-space}). 
Spring forces were computed according to Hooke's law $F = k(L - L_0)$, where $k=1.5 \times 10^4$ N/m is the spring stiffness coefficient, $L$ is the current spring length, and $L_0$ is the modulated rest length. 
Resulting impulses, as well as damping and gravitational forces, were used to update velocities for each mass, and in turn mass positions were updated using the new velocities.

The terrains along which robots behaved were modeled using randomly sampled height maps (see Appx.~\ref{appendix-dataset-random-environment-generation} for details). 
During simulation, terrain heights at arbitrary coordinates $(x, y)$ were computed through bilinear interpolation of the height map. 
For collision handling, we detected when a mass' updated $z$-coordinate fell below the interpolated terrain height at its $(x, y)$ position. Upon detection, we employed a three-phase resolution: (1) iterative bisection on the interval [0, $dt$] to estimate the time of impact and advance the mass to the contact point, (2) velocity projection onto the contact surface normal (estimated via central differences), and (3) constrained motion along the surface tangent for the remaining timestep. Following \citet{strgar2024evolution}, friction forces were computed by negating the tangential velocity component and clamping its magnitude to not exceed the magnitude of the normal velocity component.

Our simulator was implemented in the Taichi programming language \cite{hu2019difftaichi}, providing both GPU acceleration for parallel, multi-robot simulation and automatic differentiation capabilities. The simulator was directly integrated with a PyTorch-based universal controller (Sect.~\ref{methods-universal-control}), enabling end-to-end backpropagation through 1000 timesteps ($dt = 0.004$s) of physics simulation and neural control for gradient-based optimization of the controller parameters.


%%%%%%%%%%%%%%%%%%%%%%%%%%%%%%%%%%%%%%%%%%%%
\begin{figure}[b]
    \centering
    \vspace{-10pt}
    \includegraphics[width=\columnwidth]{figs/xover.pdf}
    \vspace{-18pt}
    \caption{\textbf{Recombination of substructures.}
    A pair of designs (parents; \textbf{A, B}) is combined via crossover to produce a new design  (offspring; \textbf{C}) that inherits components from both parents.
}
    \label{fig:results-xover}
    \vspace{-8pt}
\end{figure}
\begin{figure*}[t]
    \centering
    \includegraphics[width=0.975\textwidth]{figs/com_trace.jpg}
    \vspace{-10pt}
    \caption{\textbf{Evolution of phototaxis.}
    The five worst designs in the population are depicted before (top row) and after (bottom row) zero-shot evolution.
    Each design (black dotted footprints) was placed in the center of a randomly generated map.
    Before evolution,
    not all of designs in the population could
    move (gray to white trajectories)
    across any terrain
    toward a light source (gold stars) using the pretrained controller.
    After evolution, they could.
    One of the design principles that evolution discovered is that larger footprints increase locomotion stability.
    }
    \vspace{-12pt}
    \label{fig:results-com-traces}
\end{figure*}
\begin{figure*}[t]
    \centering
    \includegraphics[width=0.95\textwidth]{figs/performance.pdf}
    \vspace{-8pt}
    \caption{\textbf{Performance and diversity.} 
    Morphological pretraining (\textbf{A}) converges with 70\% improvement from baseline. 
    Simultaneous co-design (from scratch without pretraining; \textbf{B}) achieves similar training loss; 
    but, population diversity (mean pairwise Hamming distance on genotypes) collapses as evolution converges to a single species of similar designs which simplifies shared control.
    Zero shot evolution (using the pretrained controller; \textbf{C}) rapidly improves test performance, but also suffers diversity collapse as evolution compiles slightly modified clones of the designs that are the most compatible with the pretrained model.
    Few-shot evolution (\textbf{D}) resets the pretrained controller at the start of each generation 
    % (spikes in the training curve) 
    and performs 60 finetuning steps per generation. 
    This significantly increases and sustains diversity as well as performance. 
    Solid lines represent 
    the batch (training; blue)
    or
    population (test; orange) means, 
    averaged across three independent trials.
    Shaded regions surrounding the mean curves show the minimum and maximum values across the three trials.
    }
    \vspace{-10pt}
    \label{fig:results-performance}
\end{figure*}
%%%%%%%%%%%%%%%%%%%%%%%%%%%%%%%%%%%%%%%%%%%%


\subsection{The universal controller}
\label{methods-universal-control}

We employed a simple multi-layer perceptron (MLP) as a universal controller for adaptive phototaxis: guiding a population of thousands of morphologically diverse robots towards arbitrarily positioned light sources across randomly varying, rugged terrains.
The network mapped two input streams to spring actuation signals: photosensor readings from masses and central pattern generator (CPG) inputs. 
To accommodate all possible body plans in the morphospace (defined in Sect.~\ref{methods-morphology-design-space}), the network's input dimension was set to $|\mathcal{M}|$ (the maximum number of masses) and output dimension to $|\mathcal{S}|$ (the maximum number of springs). Sensors and actuators not present in the a specific robot's body had their corresponding signals masked to zero, providing an implicit morphological conditioning through observation and action space masking.

Each mass-bound photosensor measured light intensity following the inverse square law relative to the light source position. 
Sensor readings for each robot were normalized by subtracting the mean computed across that robot's active (unmasked) sensors, providing a zero-centered, embodied irradiance gradient. 

Following \citet{hu2019difftaichi} and \citet{strgar2024evolution},
CPG inputs consisted of five sinusoidal waves with angular frequency $\omega=10$ rad/s and phase offsets evenly spaced by $2\pi/5$ radians. Over the 4-second simulation period (1000 timesteps, $\Delta t=4e-3$s), these oscillators completed approximately six cycles.

The MLP architecture consisted of an input layer (dimension 250: $|\mathcal{M}|$ mass sensors plus 5 CPG inputs), three hidden layers (dimension 256 each), and an output layer (dimension 1648: $|\mathcal{S}|$ springs). Each hidden layer was followed by layer normalization and ReLU activation, while the output layer used a $\tanh$ activation. All layers included learnable biases. In total the model consisted of 620,912 learnable parameters. 

Network weights were initialized using a Xavier uniform distribution (gain=1.0) \cite{glorot2010understanding} with zero-initialized biases, and the network was optimized using Adam \cite{adamkingma} ($\beta_1=0.9$, $\beta_2=0.999$) with gradient norm clipping at 1.0. 
Learning rates were scheduled using variants of cosine annealing with restarts (detailed in Sects.~\ref{methods-pretraining},~\ref{methods-few-shot-evolution}, and~\ref{methods-codesign}).


\subsection{Morphological pretraining}
\label{methods-pretraining}

The universal controller was pretrained 
across a dataset of over 10 million distinct robot morphologies 
(see Appx.~\ref{appendix-dataset-random-robot-generation} for details). 
The controller was trained over 1400 learning steps to minimize the batch mean of $d_1/d_0$, where $d_1$ and $d_0$ represent each robot's final and initial distances from its target light source, respectively. 
This relative distance formulation ensured robots were not penalized for being initialized far from their targets and equally incentivized fine-grained control in robots initialized near their targets.

We used a batch size of 8192, distributed in equal partitions of 1024 across a single compute node consisting of eight H100 SXM GPUs. 
Each sample consisted of a randomly-generated robot morphology (see Appx.~\ref{appendix-dataset-random-robot-generation} for details) 
a randomly-generated terrain shape
and a randomly-positioned light source (see Appx.~\ref{appendix-dataset-random-environment-generation} for details), and was seen exactly once during training. 
Training used a cosine annealing with restarts schedule, with initial learning rate $1e^{-3}$, cycle length starting at 10 steps and doubling each restart, minimum learning rate $1e^{-5}$, and a decay rate of 0.7 applied to the starting learning rate at each cycle. 

\subsection{Zero-shot evolution}
\label{methods-zero-shot-evolution}

Here, we introduce a novel robot design paradigm that leverages a frozen, pretrained universal controller to rapidly evaluate non-differentiable changes to a given robot's body plan. 
By using a single, fixed controller for all body plans,
the design space may be efficiently explored without the computational burden of training a custom controller for each body plan. 
We refer to this method as ``zero-shot evolution''. 

We initialized a population of 8192 random robot morphologies (unseen during pretraining) and evaluated each on a fixed test set of terrain and light source position pairs (see Appx.~\ref{appendix-dataset-evaluation-environments} for details). A simple genetic algorithm was then applied iteratively: the population produced an equal number of offspring through two variation operators (described below), new offspring were evaluated once on the test set, and the top 50\% across parents and offspring (using cached evaluation scores for parents) were selected to form the next generation.

Robot offspring were produced through one of two variation operators: mutation and recombination. The population was partitioned into two distinct groups: a random 25\% of members were assigned to produce offspring through mutation, while the remaining 75\% were reserved for producing offspring through recombination (or crossover). Each member in the mutation group produced a single offspring through random bit flip mutations performed on their genotype. Flips occurred with probability $p = 1/N$ where $N = 6 \times 6 \times 4$, the total number of voxels in the robot's genotype. After mutation, genotypes were processed to ensure validity: only the largest connected component was retained, and the resulting structure was translated to the bottom center of the workspace. If a mutation produced a body that was either empty or identical to a previously seen body, the process was repeated with the mutation rate increased by 2.5\% until a valid, unique design was obtained.

From the recombination group (75\% of the population), pairs of distinct parents were randomly sampled to produce offspring through crossover (Fig.~\ref{fig:results-xover}). 
For each sampled pair, an offspring's genotype was created using a bitwise exclusive or (XOR) operation on the parent genotypes. As with mutation, post-processing retained only the largest connected component and centered it at the bottom of the workspace. If the resulting design duplicated a previous one, it was discarded. The sampling and generation process was repeated until the number of offspring equaled the size of the recombination group (75\% of the population).


\subsection{Few-shot evolution}
\label{methods-few-shot-evolution}

In this experiment we extend the zero-shot paradigm (described above in Sect.~\ref{methods-zero-shot-evolution}) by fine-tuning 
the pretrained universal controller 
to the current population
at every generation of morphological evolution.
We refer to this approach as ``few-shot evolution''.
% 
The experimental setup of few-shot evolution matched the zero-shot case, with one key difference: before evaluation, each generation received 60 fine-tuning steps (30 for parents, 30 for offspring).
The number of fine-tuning steps was
empirically chosen to balance controller adaptation against evolutionary search while maintaining comparable maximum wall-clock time across experiments. 
At the start of each generation, the controller's weights were reset to their pretrained values and the optimizer state was reinitialized. 
Fine-tuning used a cosine annealing learning rate schedule with initial and minimum rates of $3.5e^{-4}$ and $3.5e^{-5}$, respectively. The cycle length was set to 100; however each cycle was truncated to align each cycle with one generation's 60 fine-tuning steps resulting in an effective minimum learning rate of $1.5e^{-4}$. 
Since every generation re-initialized the pretrained weights, we did not decay the learning rate at the start of each cycle. 

\subsection{Simultaneous co-design from scratch}
\label{methods-codesign}

In our third and final experimental group, 
we remove morphological pretraining
and instead 
simultaneously 
evolve a population of robots
and
learn their universal controller, 
from scratch.
Unlike few-shot evolution, controller parameters and optimizer state are inherited across generations rather than being reset. The genetic algorithm operates as before, but we reduce the per-generation training to just 2 learning steps (1 for parents, 1 for offspring) to maintain parity with our pretraining experiments, where each training batch was unique.

Initially, we employed the same cosine annealing learning rate schedule used in morphological pretraining, but we found it was beneficial to reduce the start-of-cycle learning rate decay factor from 0.7 to 0.65 in order to stabilize learning across cycle restarts in this setting.

% \begin{figure}[t]
    \centering
    \includegraphics[width=\columnwidth]{figs/shape-stats.pdf}
    \vspace{-20pt}
    \caption{\textbf{Evolved populations.}
    Population performance, phenotype footprint size, and body mass for the initial (randomly generated) and evolved design populations.
    }
    \vspace{-18pt}
    \label{fig:appendix-robot-data-gen-stats}
\end{figure} % inside results

\section{Results}
\label{sec:results}

In this section we evaluate 
the results of 
morphological pretraining (Sect.~\ref{results-pretraining-performance}),
zero- and few shot evolution (using the pretrained model; Sect.~\ref{results-zero-and-few-shot-evolution}),
and simultaneous co-design from scratch (without pretraining; Sect.~\ref{results-simultaneous-co-design}).


\subsection{Pretraining performance}
\label{results-pretraining-performance}

Across three independent trials,
each using a distinct dataset of 
randomly-generated morphologies and environments,
pretraining 
exhibited stable learning trajectories 
with low variance across trials (Fig.~\ref{fig:results-performance}A), 
converging in approximately 1,400 learning steps (56 minutes of wall-clock time).
% 
Loss was defined as the ratio of final to initial distance from the target light source. 
At initialization with random controller weights, this ratio was 1.0, indicating robots remained stationary throughout simulation. 
After pretraining, the loss stabilized at approximately 0.3, representing a 70\% improvement. 
That is, in environments sampled from the training distribution, robots using the pretrained universal controller traversed an average of 70\% of their initial distance to the light source. 
Since each training batch used novel morphologies, we omitted model selection with a \mbox{validation set.}

% The pretrained model effectively controlled diverse, previously unseen robot morphologies in challenging tasks at the training distribution's edge (Appx.~\ref{appendix-dataset-random-robot-generation}). 
% Fig.~\ref{fig:results-zero-shot-transfer} shows the pretrained model's generalization performance across more than 24,000 distinct test morphologies. 
% Despite the test environments being more challenging than the training distribution, the mean performance of the test morphologies aligned with the final loss observed during pretraining. 

To visualize the breadth of morphological diversity handled by the pretrained controller, Fig.~\ref{fig:appendix-zero-shot-robot-grid} showcases a representative sample of successful robots. 
These examples were selected uniformly from the top-performing 50\% of the test morphology set. 
The selected bodies exhibit high variation in both scale and morphological characteristics demonstrating the non-trivial generalization of the universal controller.


%%%%%%%%%%%%%%%%%%%%%%%%%%%%%%%%%%%%%%%%%%%
\begin{figure}[t]
    \centering
    \includegraphics[width=\columnwidth]{figs/shape-stats.pdf}
    \vspace{-20pt}
    \caption{\textbf{Evolved populations.}
    Population performance, phenotype footprint size, and body mass for the initial (randomly generated) and evolved design populations.
    }
    \vspace{-18pt}
    \label{fig:appendix-robot-data-gen-stats}
\end{figure}
%%%%%%%%%%%%%%%%%%%%%%%%%%%%%%%%%%%%%%%%%%%

\subsection{Zero- and few shot evolution (with pretraining)}
\label{results-zero-and-few-shot-evolution}

A population of morphologies was evolved through random mutation and crossover operations, using the pretrained universal controller.
On the same challenging set of tasks used for evaluating pretrained controller generalization,
the population converges to near optimal performance
in 100 generations of evolution (17 minutes of wall-clock time)
without finetuning the controller (``zero shot evolution''; Sect.~\ref{methods-zero-shot-evolution}).
% 
Although zero-shot evolution shows rapid convergence in controlling thousands of distinct bodies, this success masks a key pattern: design population diversity decreases as performance improves. 
Fig.~\ref{fig:results-performance}C reveals this pattern---after a brief diversity spike at evolution's onset, the population gradually homogenizes. 
We term this phenomenon diversity collapse, measuring diversity as the population's mean, pairwise Hamming distance in (and normalized to) the genotype space $\mathcal{G}$ (defined in Sect.~\ref{methods-morphology-design-space}). This metric naturally reflects differences in morphology (body) as well as sensing and actuation masking in the universal controller (brain).

We found that generational finetuning of the universal controller for the current population (``few shot evolution'')
not only preserves diversity but in fact significantly increases diversity (Fig.~\ref{fig:results-performance}D).
This is a somewhat surprising result as there was no explicit selection pressure to maintain diversity.
The process of morphological evolution seems to intrinsically increase population diversity. 
However, in absence of generational finetuning, there is a tipping point at which it is easier to purge diversity, replacing the worst designs with slightly modified clones of the best, than to discover novel morphological innovations with superior performance.


\subsection{Simultaneous co-design (\textit{without} pretraining)}
\label{results-simultaneous-co-design}

Ablating pretraining (and funetuning),
and instead simultaneously optimizing morphology and universal control, together from scratch, 
results once again in rapid diversity collapse (Fig.~\ref{fig:results-performance}B).
Performance plateaus in well under 180 generations, corresponding to 360 controller learning steps and 109 minutes of wall-clock time.
% 
The extent of diversity collapse can be seen in Fig.~\ref{fig:results-morpho-grid}B, where we visualize morphologies from one of the three independent trials,
and in Fig.~\ref{fig:appendix-robot-data-gen-stats} where we plot morphological variance across evolved populations in terms of footprint size and body weight.

In all three co-design paradigms (zero shot, few shot, simultaneous), universal control enabled successful crossover (Fig.~\ref{fig:appendix-variation-operator-stats}).
In terms of offspring survival,
crossover was initially much more successful than mutation.
But in the case of simultaneous co-design, this was not an apples to apples comparison because each generation provided the controller with more time to learn how to control the population, and the randomly initialized controller was very bad at the task.
And so it was not clear if the success of offspring was due to changes in parent morphology or improvements to the universal controller.
The superior performance of pretraining across random morphologies, shows that the designs produced by crossover during simultaneous co-design were no better than random designs. 
In zero-shot and few-shot evolution, however, the pretrained controller is quite good at the very start, and in zero-shot the controller is not updated during evolution, providing clear evidence of successful crossover prior to diversity collapse.



\section{Discussion}
\label{sec:discussion}

 
In this paper, we introduced 
the large-scale pretraining and finetuning of a universal controller using differentiable simulation
and demonstrated how this approach
accelerates the 
design
of complex robots.
% 
The learned controller 
allows most 
randomly-generated morphologies (mass-spring networks)
to orient along a randomly-generated stimulus (light) vector in three dimensions, 
and to follow the vector to its source (phototaxis)
across challenging, randomly-generated environments (terrains)---more or less: 
some designs were much better
% (faster)
than others, and some outright fail (Fig.~\ref{fig:results-com-traces}).
Using the pretrained model as a prior, the designer 
can quickly
explore a diversity of changes---from subtle mutations to large recombinations---across
arbitrary numbers of
distinct designs in parallel
without destroying the functionality of working designs, and without constantly readapting the controller to support every morphological innovation.

We intentionally chose a vanilla evolutionary algorithm 
as ``the designer'' 
and a minimal neural architecture for the universal controller
to illustrate the power and potential of our approach.
We were particularly surprised by the effectiveness of a simple MLP in controlling 
such large numbers of morphologically complex robots 
across such challenging terrains.
Interestingly, the gaits generated by the universal controller were quite different from those 
tailored for individual body plans
in similar conditions 
\cite{strgar2024evolution};
instead of walking or ambling across the rugged terrain,
the universal controller
discovered patterns of saltation
(hopping) not unlike that of kangaroos, in which coordinated actuation of muscles is followed by an aerial phase.


It is important to note, however, that while this controller was universal across the robot's morphology and task environment, 
we only considered 
a single
material (soft),
percept (light),
actuator (linear),
and task (phototaxis).
Extending this approach to multiple tasks 
that demand 
more intricate, multi-material body plans with
multi-modal sensing
(e.g.~not just moving toward a single stimulus source, 
but reacting to various other stimuli, 
manipulating objects, 
and working with or against other robots...)
may require gradually complexifying the neural architecture.
This will likely also require replacing the direct genotype-to-phenotype mapping with
more a sophisticated (pleiotropic)
compression of phenotypes 
into a latent genome
\cite{li2025generating}.
Instead of presupposing voxel cells with
two dozen springs and eight masses,
latent genes could control the expression of more atomic building blocks,
such as individual masses and springs (or subatomic particles within them),
allowing other kinds of non-cubic cells \cite{hummer2024noncubic} to emerge.
If extended to self-reconfigurable robots, the latent genome or many such genomes may be expressed in myriad ways by a single robot with universal self control.
% 

We also identified in this paper a previously unknown yet inherent problem of co-designing morphology and universal control---diversity collapse---%
and showed how to solve this problem through
generational finetuning.
However, this first investigation of diversity collapse 
only considered a single measure of morphological diversity.
Other metrics at both the morphological and behavioral level could be formulated or derived from a latent genotype space.
Such metrics could then be incorporated into the design algorithm as a constraint or additional objective. 


Another important limitation of this work was that the simulated designs were not transferred to reality.
Doing so may require
higher resolution simulations (Fig.~\ref{fig:results-dog})
or
improvements to the simulator, e.g. it's model of contact, light, and light sensors.
Adding noise to these models can also ensure that the robot's behavior does not exploit inaccuracies of the simulation \cite{jakobi1995noise}.
Or the simulator could be augmented with a neural network that learns the residual physics that were not accounted for a priori \cite{gao2024sim}.
However, the universal controller itself might help reduce the simulation-reality gap since it is already by definition insensitive to a wide range of variation in the simulated robot's body and world.


Despite these limitations, the sheer scale and efficiency achieved by this work
opens a new frontier in robot co-design through automatic differentiation, 
suggesting the breadth of infrastructure and theory developed in fields of deep learning and neural networks may be leveraged by robot co-design in future work.


% \section*{Code}
\label{sec:code}

Code is temporarily withheld to maintain anonymity during review and will be made available upon publication.

  % SK: not necessary for preprint or camera ready (make site!)
% \section*{Impact Statement}
\label{sec:impact}

The goal of this paper is to advance the competence and diversity of intelligent robots.
There are numerous potential benefits
and societal consequences of robots,
% \cite{lin2014robot}, 
none which we feel must be specifically highlighted here.

% This work advances automated robot control and design through universal, robot-agnostic control. There are numerous potential benefits such as enabling rapid development of more capable robotic systems for beneficial applications. We acknowledge, however, that robot automation, and automation of robot design, poses risks such as 
% labor displacement and deployment safety concerns.

% Authors are \textbf{required} to include a statement of the potential 
% broader impact of their work, including its ethical aspects and future 
% societal consequences. This statement should be in an unnumbered 
% section at the end of the paper (co-located with Acknowledgements -- 
% the two may appear in either order, but both must be before References), 
% and does not count toward the paper page limit. In many cases, where 
% the ethical impacts and expected societal implications are those that 
% are well established when advancing the field of Machine Learning, 
% substantial discussion is not required, and a simple statement such 
% as the following will suffice:

% ``This paper presents work whose goal is to advance the field of 
% Machine Learning. There are many potential societal consequences 
% of our work, none which we feel must be specifically highlighted here.''

% The above statement can be used verbatim in such cases, but we 
% encourage authors to think about whether there is content which does 
% warrant further discussion, as this statement will be apparent if the 
% paper is later flagged for ethics review. % SK: not necessary for preprint
\section*{Acknowledgments}
This research was supported by
NSF award FRR-2331581
and
Schmidt Sciences AI2050 grant G-22-64506.
% and
% Templeton World Charity Foundation award no.~20650. 



\bibliography{main}
\bibliographystyle{icml2025}

\newpage
\appendix
\section{Appendix}
\subsection{Metric Optimization}  \label{app:pg}
We utilize the REINFORCE algorithm to optimize the performance metric. The detailed optimization process is proved in the following equations:
    \begin{equation}
        \small
        \begin{aligned}
            &\nabla_{\Lambda}\hat{l}(\Lambda)
            =\nabla_{\Lambda} \mathbb{E}_{s\sim \pi{(\mathcal{B},\cdot;{\Lambda})}} \mathcal{R}(\hat{\mathcal{D}}, f(\Theta^*(\Lambda)))\\
            &=\nabla_{\Lambda}\sum_{s\in[0,1]^{|\mathcal{B}|}} \mathcal{R}(\hat{\mathcal{D}}, f(\Theta^*(\Lambda))) \cdot \pi(\mathcal{B},s;{\Lambda}) \\
            &=\sum_{s\in[0,1]^{|\mathcal{B}|}} \mathcal{R}(\hat{\mathcal{D}}, f(\Theta^*(\Lambda))) \cdot 
            \frac{\nabla_{\Lambda}\pi(\mathcal{B},s;{\Lambda})}{\pi(\mathcal{B},s;{\Lambda})}\cdot \pi(\mathcal{B},s;{\Lambda})\\
            &= \sum_{s\in[0,1]^{|\mathcal{B}|}} \mathcal{R}(\hat{\mathcal{D}}, f(\Theta^*(\Lambda))) \cdot \nabla_{\Lambda}log(\pi(\mathcal{B},s;{\Lambda}))\cdot \pi(\mathcal{B},s;{\Lambda})\\
            &=\mathbb{E}_{s\sim \pi(\mathcal{B},\cdot;{\Lambda})}[\mathcal{R}(\hat{\mathcal{D}}, f(\Theta^*(\Lambda)))\cdot \nabla_{\Lambda}log(\pi(\mathcal{B},s;{\Lambda}))],
        \end{aligned}
    \end{equation}

\subsection{Learning Algorithm}  \label{app:learning_algorithm}
The detailed optimization steps of the proposed framework are given in Algorithm \ref{al:method}.

\subsection{Detail of Studied Methods} \label{app:studied method}
To show the compatibility of our method, we apply the DVR framework on four recommendation backbones, i.e., BRPMF~\cite{koren2009matrix}, NeuMF~\cite{he2017neural}, MGCF~\cite{wang2019neural}, and LightGCN~\cite{he2020lightgcn}. We select BPRMF due to its widespread adoption in recommendation systems and proven effectiveness in practical applications. NeuMF, an MLP-based approach, extends the capabilities of BPRMF by capturing intricate user-item relationships. We leverage GNN-based models, such as MGCF and LightGCN, known for their state-of-the-art performance and competitive outcomes across various techniques, to serve as the recommendation backbone. 

Based on these backbones, different versions of the DVR model are tailored to optimize diverse metrics. For simplicity, we designate models optimized for ranking accuracy as DVR-Loss, DVR-Recall, and DVR-NDCG. Likewise, models focused on diversity and fairness metrics are labeled as DVR-CC, DVR-ILD, and DVR-Gini. 


We compare our framework with various data valuation methods for recommendations. BPR~\cite{10.5555/1795114.1795167} uniformly samples negative items and treats all training data equally in constructing the training objective. TCE-BPR and RCE-BPR are extensions of the TCE and RCE techniques \cite{10.1145/3437963.3441800}, aimed at dynamically filtering out noisy positive interactions during training based on loss values. In our implementation, we replace the original point-wise loss with a pair-wise ranking loss objective to ensure a fair comparison with these methods. AOBPR \cite{10.1145/2556195.2556248} enhances the BPR algorithm by incorporating adaptive sampling techniques that prioritize popular negative items. WBPR \cite{gantner2012personalized} assumes that unexplored popular items by a user are more likely to be true negatives. PRIS \cite{10.1145/3366423.3380187} assigns higher weights to informative negative samples using importance sampling. TIL-UI and TIL-MI \cite{wu2022adapting} learn the data value of training triplets through two aggregation strategies by optimizing the BPR loss within the training batch.

\subsection{Implementation Details} \label{app:implenmentation}
We optimize all models using the Adam optimizer with Xavier initialization \cite{glorot2010understanding} and maintain a fixed embedding size of 64 across all methods. When constructing the ranking loss objective, every positive item is associated with one sampled negative item for an efficient computation. Grid search is applied to choose learning rate and weight decay from $\left\{1e^{-4}, 1e^{-3}, 1e^{-2}, 1e^{-1}\right\}$ and $\left\{1e^{-6}, 1e^{-5}, 1e^{-4}, 1e^{-3}\right\}$. The backbone models NeuMF, MGCCF, and LightGCN utilize the provided implementations, with MGCF and LightGCN featuring two graph convolution layers. The total number of training epochs is set to 2000 for all models with an early stopping design. Given the initial training stages' limited information, we pre-train the recommendation model without data valuator for 1000 epochs to get meaningful embeddings. We set the number of the pre-training epochs to 1000. All experiments are conducted on GPU machines (NVIDIA GeForce RTX 3090).

\begin{algorithm}[H]
    \caption{The Proposed Method}
    \label{al:method}
    
    \textbf{Input:} Learning rates $\alpha$ and $\beta$, outer mini-batch size $B_1$, inner mini-batch size $B_2$, outer iteration count $T_1$, inner iteration count $T_2$, moving average window $W$, training pairs $\mathcal{D}_{1}=\{(u,i)\}_{k=1}^{L_1}$, validation pairs $\mathcal{D}_{2}=\{(u,i)\}_{k=1}^{L_2}$
    
    \textbf{Initialize:} parameters $\Theta$ and $\Lambda$, moving average $\delta=0$
    
    \begin{algorithmic}[1]
    \FOR{outer iteration $t_1=1,2,...,T_1$}
        \STATE Sample a mini-batch from the entire training dataset: $\mathcal{\hat{B}}=(u,i)_{k=1}^{B_1}\sim \mathcal{D}_1$
        \STATE Uniformly sample negative items $(j)_{k=1}^{B_1}$ for training pairs $(u,i)_{k=1}^{B_1}$ to get training data $\mathcal{B}=(u,i,j)_{k=1}^{B_1}$ for the recommendation model
        \FOR{$(u,i,j) \in \mathcal{B}$}
        \STATE Calculate the Shapley value by Eq. (\ref{eq:svcal}) and assign it to $w_{uij}$
        \ENDFOR
        \STATE Normalize the Shapley value $w_{uij}$ within the batch $\mathcal{B}$ as $\hat{w}_{uij}$ 
        \STATE Compute sample selection vector $s_{uij}=\text{Ber}(\hat{w}_{uij})$ from Bernoulli distribution
        \STATE Update the data valuator model by \ref{eq:mse}
        \FOR{inner iteration $t_2=1,2,...,T_2$}
        \STATE Sample a mini-batch $(u,i,j)_{m=1}^{B_2}\sim \mathcal{B}$
        \STATE Update the recommendation model:
    $$\Theta \leftarrow \Theta-\frac{\alpha}{B_2} \sum_{m=1}^{B_2} s_{uij} \cdot \nabla_\Theta \mathcal{L}_{\text{BPR}}(u,i,j;\Theta)$$
        \ENDFOR
    \STATE Update the data valuator:
    $$ \begin{array}{r}
    \Lambda \longleftarrow \Lambda - \beta [\mathcal{R}(\mathcal{D}_2, f(\Theta^*(\Lambda)))-\delta ]\\ \cdot \nabla_{\Lambda}log(\pi(\mathcal{B},(s_{uij})_{k=1}^{B_1};{\Lambda})
    \end{array}
    $$
    \STATE Update the moving average reward:
    $$
    \delta \leftarrow \frac{W-1}{W} \delta+\frac{1}{W} \mathcal{R}(\mathcal{D}_2, f(\Theta^*(\Lambda)))
    $$                               
    \ENDFOR
    \end{algorithmic}
\end{algorithm}








\begin{figure*}[b]
    \centering
    \includegraphics[width=\textwidth]{figs/variation_operators.pdf}
    \vspace{-20pt}
    \caption{\textbf{Success of crossover vs.~mutation.}
    Early in evolution, random crossover is more successful than random mutation.
    But, after a few generations, mutations that more finely tune good designs were less likely to be deleterious than swapping large components between designs. 
    }
    \vspace{-8pt}
    \label{fig:appendix-variation-operator-stats}
\end{figure*} 


\begin{figure}[b]
    \centering
    % \vspace{-2pt}
    \includegraphics[width=0.5\textwidth]{figs/light-source-distribution.pdf}
    % \includegraphics[width=0.875\columnwidth]{figs/light-source-distribution.pdf}
    \vspace{-8pt}
    \caption{\textbf{Phototaxis training and testing.}
    During pretraining, simult.~co-design, and few-shot finetuning,
    training light source locations (gray circles) were sampled uniformly within 
    a circle centered on the robot's initial position (blue square).
    At every learning step,
    a batch of 8192 randomly positioned lights was sampled,
    and each was paired with a unique, random morphology and random terrain.
    Test light source locations (orange stars) 
    were identical across all methods for fair comparison.
    }
    \vspace{-8pt}
    \label{fig:methods-lightsource-dist}
\end{figure}

\begin{figure}[t]
    \centering
    \includegraphics[width=0.65\textwidth]{figs/dog.jpg}
    % \vspace{-4pt}
    \caption{\textbf{Scaling morphology.}
    The embarrassingly parallel nature of the co-design pipeline
    allows the compute required to simulate 1024 robots with up to 1648 springs 
    (i.e.~a single GPU)
    to be redistributed 
    for a single robot with 1,115,157 springs.
    }
    \vspace{-10pt}
    \label{fig:results-dog}
\end{figure}

\begin{figure*}[t]
    \centering
    \includegraphics[width=\textwidth,keepaspectratio=true]{figs/morpho-grid.jpg}
    \vspace{-16pt}
    \caption{\textbf{Morphological distinctiveness.}
    Robot designs shown are sampled uniformly from each generation's test performance distribution and arranged (left to right, top to bottom) by morphological distinctiveness, defined as the mean pairwise Hamming distance to its peer designs. Performance scores appear below each design. The initial population (\textbf{A}) exhibits diverse morphologies with broad performance variation, serving as the starting point for all methods. After 180 generations, simultaneous co-design (\textbf{B}) yields high-performing but morphologically homogeneous designs. In contrast, both zero-shot evolution at generation 31 (\textbf{C}) and few-shot evolution at generation 6 (\textbf{D}) achieve equal or superior performance while maintaining greater morphological diversity and complexity.
    } 
    \vspace{-8pt}
    \label{fig:results-morpho-grid}
\end{figure*}

\begin{figure*}[t]
    \centering
    \includegraphics[width=\textwidth]{figs/zero-shot-robot-grid.jpg}
    \vspace{-18pt}
    \caption{\textbf{Generalization of pretrained universal controller.} Randomly sampled morphologies from the top 50\% of performers in generation 0 of zero-shot evolution. The universal controller successfully controls these diverse, previously unseen robot designs, demonstrating effective generalization across morphologies.}
    \vspace{-8pt}
    \label{fig:appendix-zero-shot-robot-grid}
\end{figure*}


\end{document}

