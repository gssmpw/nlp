%%%%%%%% ICML 2025 EXAMPLE LATEX SUBMISSION FILE %%%%%%%%%%%%%%%%%

\documentclass{article}

% Recommended, but optional, packages for figures and better typesetting:
\usepackage{microtype}
\usepackage{graphicx}
% \pdfcompresslevel=0
% \pdfobjcompresslevel=0
\usepackage{subfigure}
\usepackage{booktabs} % for professional tables

% hyperref makes hyperlinks in the resulting PDF.
% If your build breaks (sometimes temporarily if a hyperlink spans a page)
% please comment out the following usepackage line and replace
% \usepackage{icml2025} with \usepackage[nohyperref]{icml2025} above.
\usepackage{hyperref}

% Attempt to make hyperref and algorithmic work together better:
\newcommand{\theHalgorithm}{\arabic{algorithm}}

% Use the following line for the initial blind version submitted for review:
% \usepackage{icml2025}


% If accepted, instead use the following line for the camera-ready submission:
\usepackage[accepted]{icml2025}

%%%%%%%%%%%% !!!! %%%%%%%%%%%%

%SK: removed ICML accepted footnote for preprint; switch to anonymous for openreview

%%%%%%%%%%%% !!!! %%%%%%%%%%%%

% For theorems and such
\usepackage{amsmath}
\usepackage{amssymb}
\usepackage{mathtools}
\usepackage{amsthm}

% if you use cleveref..
\usepackage[capitalize,noabbrev]{cleveref}

%%%%%%%%%%%%%%%%%%%%%%%%%%%%%%%%
% THEOREMS
%%%%%%%%%%%%%%%%%%%%%%%%%%%%%%%%
\theoremstyle{plain}
\newtheorem{theorem}{Theorem}[section]
\newtheorem{proposition}[theorem]{Proposition}
\newtheorem{lemma}[theorem]{Lemma}
\newtheorem{corollary}[theorem]{Corollary}
\theoremstyle{definition}
\newtheorem{definition}[theorem]{Definition}
\newtheorem{assumption}[theorem]{Assumption}
\theoremstyle{remark}
\newtheorem{remark}[theorem]{Remark}

% Todonotes is useful during development; simply uncomment the next line
%    and comment out the line below the next line to turn off comments
%\usepackage[disable,textsize=tiny]{todonotes}
\usepackage[textsize=tiny]{todonotes}


% The \icmltitle you define below is probably too long as a header.
% Therefore, a short form for the running title is supplied here:
\icmltitlerunning{Accelerated 
co-design of robots
through morphological pretraining}

\begin{document}

\twocolumn[
\icmltitle{Accelerated 
co-design of robots
through morphological pretraining}

% It is OKAY to include author information, even for blind
% submissions: the style file will automatically remove it for you
% unless you've provided the [accepted] option to the icml2025
% package.

% List of affiliations: The first argument should be a (short)
% identifier you will use later to specify author affiliations
% Academic affiliations should list Department, University, City, Region, Country


\begin{icmlauthorlist}
\icmlauthor{Luke Strgar}{n}
\icmlauthor{Sam Kriegman}{n}
\end{icmlauthorlist}

\icmlaffiliation{n}{Northwestern University, Evanston, IL, USA}

\icmlcorrespondingauthor{Luke Strgar}{lvs@u.northwestern.edu}
\icmlcorrespondingauthor{Sam Kriegman}{sam.kriegman@northwestern.edu}

% You may provide any keywords that you find helpful for describing your paper
% These are used to populate the "keywords" metadata in the PDF but will not be shown in the document
\icmlkeywords{Phototaxis, Soft robot}

\vskip 0.3in
]

\printAffiliationsAndNotice{}


\begin{figure*}[t]
    \centering
    % \includegraphics[width=\textwidth]{figs/dataflow-teaser.pdf}
    \includegraphics[width=\textwidth]{figs/teaser.jpg}
    \vspace{-20pt}
    \caption{\textbf{Universal control of differentiable robots.}
    Large-scale pretraining and finetuning
    of a universal controller was achieved by
    averaging simulation gradients across the robot's body, world, and goal.
    The controller is shared by an arbitrarily large and morphologically diverse population of robots as
    they undergo morphological evolution.
    The objective is to find designs that can move quickly across a previously-unseen terrain toward a randomly-positioned light source (glowing white spheres).
    }
    \vspace{-10pt}
    \label{fig:intro-robot-teaser}
\end{figure*}

\begin{abstract}

Hierarchical clustering is a powerful tool for exploratory data analysis, organizing data into a tree of clusterings from which a partition can be chosen. This paper generalizes these ideas by proving that, for any reasonable hierarchy, one can optimally solve any center-based clustering objective over it (such as $k$-means). Moreover, these solutions can be found exceedingly quickly and are \emph{themselves} necessarily hierarchical. 
%Thus, given a cluster tree, we show that one can quickly generate a myriad of \emph{new} hierarchies from it. 
Thus, given a cluster tree, we show that one can quickly access a plethora of new, equally meaningful hierarchies.
Just as in standard hierarchical clustering, one can then choose any desired partition from these new hierarchies. We conclude by verifying the utility of our proposed techniques across datasets, hierarchies, and partitioning schemes.


\end{abstract}


\section{Introduction}%

Decision-making is at the heart of artificial intelligence systems, enabling agents to navigate complex environments, achieve goals, and adapt to changing conditions. Traditional decision-making frameworks often rely on associations or statistical correlations between variables, which can lead to suboptimal outcomes when the underlying causal relationships are ignored \citep{pearl2009causal}. 
The rise of causal inference as a field has provided powerful frameworks and tools to address these challenges, such as structural causal models and potential outcomes frameworks \citep{rubin1978bayesian,pearl2000causality}. 
Unlike traditional methods, \textit{causal decision-making} focuses on identifying and leveraging cause-effect relationships, allowing agents to reason about the consequences of their actions, predict counterfactual scenarios, and optimize decisions in a principled way \citep{spirtes2000causation}. In recent years, numerous decision-making methods based on causal reasoning have been developed, finding applications in diverse fields such as recommender systems \citep{zhou2017large}, clinical trials \citep{durand2018contextual}, finance \citep{bai2024review}, and ride-sharing platforms \citep{wan2021pattern}. Despite these advancements, a fundamental question persists: 

\begin{center}
    \textit{When and why do we need causal modeling in decision-making?}
\end{center} 

% Numerous decision-making methods based on causal reasoning have been developed recently with wide applications 
% %Decision makings based on causal reasoning have been widely applied 
% in a variety of fields, including 
% recommender systems \citep{zhou2017large}, clinical trials \citep{durand2018contextual}, 
% finance \citep{bai2024review}, 
% ride-sharing platforms \citep{wan2021pattern}, and so on. 


 

% At the intersection of these fields, causal decision-making seeks to answer critical questions: How can agents make decisions when causal knowledge is incomplete? How do we integrate learning and reasoning about causality into real-world decision-making systems? What role do interventions, counterfactuals, and observational data play in guiding decisions? 

% Our review is structured as follows: 
 

This question is closely tied to the concept of counterfactual thinking—reasoning about what might have happened under alternative decisions or actions. Counterfactual analysis is crucial in domains where the outcomes of unchosen decisions are challenging, if not impossible, to observe. For instance, a business leader selecting one marketing strategy over another may never fully know the outcome of the unselected option \citep{rubin1974estimating, pearl2009causal}. Similarly, in econometrics, epidemiology, psychology, and social sciences, \textit{the inability to observe counterfactuals directly often necessitates causal approaches} \citep{morgan2015counterfactuals, imbens2015causal}. 
Conversely, non-causal analysis may suffice in scenarios where alternative outcomes are readily determinable. For example, a personal investor's actions may have negligible impact on stock market dynamics, enabling potential outcomes of alternate investment decisions to be inferred from existing stock price time series \citep{angrist2008mostly}. However, even in cases where counterfactual outcomes are theoretically calculable—such as in environments with known models like AlphaGo—exhaustively computing all possible outcomes is computationally infeasible \citep{silver2017mastering, silver2018general}. 
In such scenarios, causal modeling remains advantageous by offering \textit{structured ways to infer outcomes efficiently and make robust decisions}. 


%This perspective not only enhances the interpretability of decisions but also provides a principled framework for addressing uncertainty, guiding actions, and improving performance across a broad range of applications.

% Data-driven decision-making exists before the causal revolution. \textit{So when and why do we need causal modelling in decision-making?} 
% This is closely related to the presence of counterfactuals in many applications. 
% The counterfactual thinking involves considering what would have happened in an alternate scenario where a different decision or action was taken. 
% In many fields, including econometrics, epidemiology, psychology, and social sciences, accessing outcomes from unchosen decisions is often challenging if not impossible. 
% For example, a business leader who selects one marketing strategy over another may never know the outcome of the unselected option. 
% Conversely, non-causal analysis may be adequate in situations where potential outcomes of alternate actions are more readily determinable: for example, the investment of a personal investor may have minimal impact on the market, therefore her counterfactual investment decision's outcomes can still be calculated with the data of stock price time series. 
% However, it is important to note that even when counterfactuals are theoretically calculable, as in environments with known models like AlphaGo, computing all possible outcomes may not be feasible. 
% In such scenarios, a causal perspective  remains beneficial. 


 

% 1. significance of decision making
% 2. role of causal in decision making
% 3. refer to the https://jair.org/index.php/jair/article/view/13428/26917

% Decision makings based on causal reasoning have been widely applied in a variety of fields, including recommender systems \citep{zhou2017large}, clinical trials \citep{durand2018contextual}, 
% business economics scenarios \citep{shen2015portfolio}, 
% ride-sharing platforms \citep{wan2021pattern}, and so on. 
% However, most existing works primarily assume either sophisticated prior knowledge or strong causal models to conduct follow-up decision-making. To make effective and trustworthy decisions, it is critical to have a thorough understanding of the causal connections between actions, environments, and outcomes.

\begin{figure}[!t]
    \centering
    \includegraphics[width = .75\linewidth]{Figure/3Steps_V2.png}
    \caption{Workflow of the \acrlong{CDM}. $f_1$, $f_2$, and $f_3$ represent the impact sizes of the directed edges. Variables enclosed in solid circles are observed, while those in dashed circles are actionable.}\label{fig:cdm}
\end{figure}


Most existing works primarily assume either sophisticated prior knowledge or strong causal models to conduct follow-up decision-making. To make effective and trustworthy decisions, it is critical to have a thorough understanding of the causal relationships among actions, environments, and outcomes. This review synthesizes the current state of research in \acrfull{CDM}, providing an overview of foundational concepts, recent advancements, and practical applications. Specifically, this work discusses the connections of \textbf{three primary components of decision-making} through a causal lens: 1) discovering causal relationships through \textit{\acrfull{CSL}}, 2) understanding the impacts of these relationships through \textit{\acrfull{CEL}}, and 3) applying the knowledge gained from the first two aspects to decision making via \textit{\acrfull{CPL}}. 

Let $\boldsymbol{S}$ denote the state of the environment, which includes all relevant feature information about the environment the decision-makers interact with, $A$ the action taken, $\pi$ the action policy that determines which action to take, and $R$ the reward observed after taking action $A$. As illustrated in Figure \ref{fig:cdm}, \acrshort{CDM} typically begins with \acrshort{CSL}, which aims to uncover the unknown causal relationships among various variables of interest. Once the causal structure is established, \acrshort{CEL} is used to assess the impact of a specific action on the outcome rewards. To further explore more complex action policies and refine decision-making strategies, \acrshort{CPL} is employed to evaluate a given policy or identify an optimal policy. In practice, it is also common to move directly from \acrshort{CSL} to \acrshort{CPL} without conducting \acrshort{CEL}. Furthermore, \acrshort{CPL} has the potential to improve both \acrshort{CEL} and \acrshort{CSL} by facilitating the development of more effective experimental designs \citep{zhu2019causal,simchi2023multi} or adaptively refining causal structures \citep{sauter2024core}. %However, these are beyond the scope of this paper.

\begin{figure}[!t]
    \centering
    \includegraphics[width = .9\linewidth]{Figure/Table_of_Six_Scenarios_S.png}
    \caption{Common data dependence structures (paradigms) in \acrshort{CDM}. Detailed notations and explanations can be found in Section \ref{sec:paradigms}.}
    \label{Fig:paradigms}
\end{figure}
Building on this framework, decision-making problems discussed in the literature can be further categorized into \textbf{six paradigms}, as summarized in Figure \ref{Fig:paradigms}. These paradigms summarize the common assumptions about data dependencies frequently employed in practice. Paradigms 1-3 describe the data structures in offline learning settings, where data is collected according to an unknown and fixed behavior policy. In contrast, paradigms 4-6 capture the online learning settings, where policies dynamically adapt to newly collected data, enabling continuous policy improvement. These paradigms also reflect different assumptions about state dependencies. The simplest cases, paradigms 1 and 4, assume that all observations are independent, implying no long-term effects of actions on future observations. To account for sequental dependencies, the \acrfull{MDP} framework, summarized in paradigms 2 and 5, assumes Markovian state transition. Specifically, it assumes that given the current state-action pair $(S_t, A_t)$, the next state $S_{t+1}$ and reward $R_t$ are independent of all prior states $\{S_j\}_{j < t}$ and actions $\{A_j\}_{j < t}$. When such independence assumptions do not hold, paradigms 3 and 6 account for scenarios where all historical observations may impact state transitions and rewards. This includes but not limited to researches on \acrfull{POMDP} \citep{hausknecht2015deep, littman2009tutorial}, panel data analysis \citep{hsiao2007panel,hsiao2022analysis}, \acrfull{DTR} with finite stages \citep{chakraborty2014dynamic, chakraborty2013statistical}. 

Each \acrshort{CDM} task has been studied under different paradigms, with \acrshort{CSL} extensively explored within paradigm 1. \acrshort{CEL} and offline \acrshort{CPL} encompass paradigms 1-3, while online \acrshort{CPL} spans paradigms 4-6. By organizing the discussion around these three tasks and six paradigms, this review aims to provide a cohesive framework for understanding the field of \acrlong{CDM} across diverse tasks and data structures.

%Recognizing the importance of long-term effects in decision-making

%Further discussions on these paradigms and their connections to various causal decision-making problems are provided in Section \ref{sec:paradigms}.


\textbf{Contribution.} In this paper, we conduct a comprehensive survey of \acrshort{CDM}. 
Our contributions are as follows. 
\begin{itemize}
    \item We for the first time organize the related causal decision-making areas into three tasks and six paradigms, connecting previously disconnected areas (including economics, statistics, machine learning, and reinforcement learning) using a consistent language. For each paradigm and task, we provide a few taxonomies to establish a unified view of the recent literature.
    \item We provide a comprehensive overview of \acrshort{CDM}, covering all three major tasks and six classic problem structures, addressing gaps in existing reviews that either focus narrowly on specific tasks or paradigms or overlook the connection between decision-making and causality (detailed in Section \ref{sec::related_work}).
    %\item We outline three key challenges that emerge when utilizing CDM in practice. Moreover, we delve into a comprehensive discussion on the recent advancements and progress made in addressing these challenges. We also suggest six future directions for these problems.
    \item We provide real-world examples to illustrate the critical role of causality in decision-making and to reveal how \acrshort{CSL}, \acrshort{CEL} and \acrshort{CPL} are inherently interconnected in daily applications, often without explicit recognition.
    \item We are actively maintaining and expanding a GitHub repository and online book, providing detailed explanations of key methods reviewed in this paper, along with a code package and demos to support their implementation, with URL: \url{https://causaldm.github.io/Causal-Decision-Making}.
\end{itemize}
% Our review is structured as follows: 


%%%%%%%%%%%%%%%%%%%%%%%%%%%%%%%%%%
%  causal helps over "Correlational analysis"
%Correlational analysis, though widely used in various fields, has inherent limitations, particularly when it comes to decision-making. While it identifies relationships between variables, it fails to establish causality, often leading to misinterpretations and misguided decisions. For example, the positive correlation between ice cream sales and drowning incidents is a classic example of how correlational data can be misleading, as both are influenced by a third factor, temperature, rather than causing each other. Such spurious correlations, due to oversight of confounding variables, underscore the necessity of causal modeling in decision making. Causal models excel where correlational analysis falls short, offering predictive power and a deeper understanding of underlying mechanisms. They enable us to predict the outcomes of interventions, even under untested conditions, and provide insights into the processes leading to these outcomes, thereby informing more effective strategies. Moreover, causal models are good at generalizing findings across different contexts, a capability often limited in purely correlational studies. 

%  causal helps in causal RL 
%From another complementary angle, although causal concepts have traditionally not been explicitly incorporated in fields like online bandits \citep{lattimore2020bandit} and \acrfull{RL} \citep{sutton2018reinforcement}, much of the literature in these areas implicitly relies on basic assumptions outlined in Section \ref{sec:prelim_assump} to utilize observed data in place of potential outcomes in their analyses, and there is also a growing recognition of the significance of the causal perspective \citep{lattimore2016causal, zeng2023survey} in these areas. 
% \textbf{Read causal RL survey and summarize. } However, by integrating causal concepts and leverging existing methodologies, we open up possibilities for developing more robust models to remove spurious correlation and selection bias \citep{xu2023instrumental, forney2017counterfactual}, designing more sample-efficient \citep{sontakke2021causal, seitzer2021causal} and robust \citep{dimakopoulou2019balanced, ye2023doubly} algorithms, and improving the generalizability \citep{zhang2017transfer, eghbal2021learning}, explanability \citep{foerster2018counterfactual, herlau2022reinforcement}, and fairness \citep{zhang2018fairness,huang2022achieving,balakrishnan2022scales} of these methods. %, and safety \cite{hart2020counterfactual}

%


%\subsection{Paper Structure}
The remainder of this paper is organized as follows: Section \ref{sec::related_work} provides an overview of related survey papers. Section \ref{sec:preliminary} introduces the foundational concepts, assumptions, and notations that form the foundation for the subsequent discussions. In Section \ref{sec:3task6paradigm}, we offer a detailed introduction to the three key tasks and six learning paradigms in \acrshort{CDM}. Sections \ref{Sec:CSL} through \ref{sec:Online CPL} form the core of the paper, with each section dedicated to a specific topic within \acrshort{CDM}: \acrshort{CSL}, \acrshort{CEL}, Offline \acrshort{CPL}, and Online \acrshort{CPL}, respectively. Section \ref{sec:assump_violated} then explores extensions needed when standard causal assumptions are violated. To illustrate the practical application of the \acrshort{CDM} framework, Section \ref{sec:real_data} presents two real-world case studies. Finally, Section \ref{sec:conclusion} concludes the paper with a summary of our contributions and a discussion of additional research directions that are actively being explored.



\begin{figure}[!hb]
    \centering
    % \vspace{-8pt}
    \includegraphics[width=\columnwidth]{figs/dataflow-teaser-1col.pdf}
    \vspace{-16pt}
    \caption{\textbf{Overview of the proposed method.}
    End-to-end differentiable policy training across tens of millions of morphologically distinct robots---morphological pretraining---produces a universal controller, which was kept frozen throughout zero-shot evolution
    and finetuned for each generation of few-shot evolution.
    }
    \vspace{-8pt}
    \label{fig:intro-dataflow}
\end{figure}

\begin{figure*}[t]
    \centering
    \includegraphics[width=\textwidth]{figs/phylogenetic-tree.jpg}
    \vspace{-14pt}
    \caption{\textbf{Few-shot evolution.}
    % Co-design of body (morphology) and brain (control) using a pretrained universal controller.
    A population of 8192 
    initially random designs (a pair of which are shown in the top row)
    were randomly recombined and mutated to produce 8192 offspring, temporarily expanding the population to 16384 designs.
    All designs in the population were driven by the same universal controller, which was rapidly pretrained (before evolution)
    and finetuned for the current population (at every generation of evolution) using analytical gradients
    from differentiable simulation.
    % , which was lightly finetuned started from the pretrained model at each generation. 
    Deleting the worst performing designs and replacing them with the best offspring, and repeating this process for several generations,
    yields a diversity of increasingly performant designs, and ultimately a final population of 8192 winning designs (bottom row), each with their own unique evolutionary history (phylogeny).
    An example phylogenetic tree, colored by loss (decreasing from gray to cyan to pink), is shown for one of winning designs.
    }
    \vspace{-8pt}
    \label{fig:intro-phylogenetic-tree}
\end{figure*}

\begin{figure}[t]
    \centering
    \includegraphics[width=\columnwidth]{figs/gps.jpg}
    \vspace{-18pt}
    \caption{\textbf{Genotype to phenotype.}
    Designs are 
    encoded by voxel genotype (\textbf{A}), 
    which is expressed as a
    spring-mass phenotype (\textbf{B}),
    and evaluated in a
    differentiable environment (\textbf{C}).
    The springs (teal lines in B and C) and masses (small orange spheres) are motorized and sensorized, respectively.
    }
    \vspace{-12pt}
    \label{fig:methods-geno-pheno-sim}
\end{figure}

% \begin{figure}[b]
    \centering
    \vspace{-10pt}
    \includegraphics[width=\columnwidth]{figs/xover.pdf}
    \vspace{-18pt}
    \caption{\textbf{Recombination of substructures.}
    A pair of designs (parents; \textbf{A, B}) is combined via crossover to produce a new design  (offspring; \textbf{C}) that inherits components from both parents.
}
    \label{fig:results-xover}
    \vspace{-8pt}
\end{figure}  % inside methods
% \begin{figure*}[t]
    \centering
    \includegraphics[width=0.975\textwidth]{figs/com_trace.jpg}
    \vspace{-10pt}
    \caption{\textbf{Evolution of phototaxis.}
    The five worst designs in the population are depicted before (top row) and after (bottom row) zero-shot evolution.
    Each design (black dotted footprints) was placed in the center of a randomly generated map.
    Before evolution,
    not all of designs in the population could
    move (gray to white trajectories)
    across any terrain
    toward a light source (gold stars) using the pretrained controller.
    After evolution, they could.
    One of the design principles that evolution discovered is that larger footprints increase locomotion stability.
    }
    \vspace{-12pt}
    \label{fig:results-com-traces}
\end{figure*} % inside methods
% \begin{figure*}[t]
    \centering
    \includegraphics[width=0.95\textwidth]{figs/performance.pdf}
    \vspace{-8pt}
    \caption{\textbf{Performance and diversity.} 
    Morphological pretraining (\textbf{A}) converges with 70\% improvement from baseline. 
    Simultaneous co-design (from scratch without pretraining; \textbf{B}) achieves similar training loss; 
    but, population diversity (mean pairwise Hamming distance on genotypes) collapses as evolution converges to a single species of similar designs which simplifies shared control.
    Zero shot evolution (using the pretrained controller; \textbf{C}) rapidly improves test performance, but also suffers diversity collapse as evolution compiles slightly modified clones of the designs that are the most compatible with the pretrained model.
    Few-shot evolution (\textbf{D}) resets the pretrained controller at the start of each generation 
    % (spikes in the training curve) 
    and performs 60 finetuning steps per generation. 
    This significantly increases and sustains diversity as well as performance. 
    Solid lines represent 
    the batch (training; blue)
    or
    population (test; orange) means, 
    averaged across three independent trials.
    Shaded regions surrounding the mean curves show the minimum and maximum values across the three trials.
    }
    \vspace{-10pt}
    \label{fig:results-performance}
\end{figure*} % inside methods

\section{Methods}
\label{sec:methods} 

In this section we describe 
the morphological design space (the ``morphospace''; Sect.~\ref{methods-morphology-design-space}),
the simulated physical environment (Sect.~\ref{methods-simulator}),
the universal controller (Sect.~\ref{methods-universal-control}),
morphological pretraining (Sect.~\ref{methods-pretraining}),
zero shot evolution (Sect.~\ref{methods-zero-shot-evolution}),
few shot evolution with generational finetuning (Sect.~\ref{methods-few-shot-evolution}),
and 
simultaneous co-design (Sect.~\ref{methods-codesign}).


\subsection{Morphospace}
\label{methods-morphology-design-space}

Robot morphologies were genetically encoded as contiguous collections of voxels 
within a $6 \times 6 \times 4$ (Length $\times$ Width $\times$ Height) binary genotype workspace, $\mathcal{G}$. 
% Connectivity was defined through 6-neighbor adjacency (face-adjacent voxels), ensuring physically realizable robots without disconnected parts. 
Voxelized genotypes were then mapped to a phenotype space $\mathcal{P}$ comprising 
masses $\mathcal{M}$ and springs $\mathcal{S}$ 
arranged in a cubic lattice 
with 10 cm$^3$ unit cells 
(Fig.~\ref{fig:methods-geno-pheno-sim}).
% 
More specifically, a genotype voxel at position $(i,j,k)$ in $\mathcal{G}$ is expressed phenotypically by eight masses,
one in each corners of the corresponding cubic cell in $\mathcal{P}$ with coordinates $(0.1i+\delta_x,0.1j+\delta_y,0.1k+\delta_z)$ where $\delta_{x,y,z} \in \{0,0.1\}$. 
Springs are then connected to these masses in two patterns: (1) axial springs along cube edges, and (2) planar diagonal springs in each face. 
Adjacent genotype voxels share masses and springs at their interfaces, 
ensuring that contiguous structures in $\mathcal{G}$ mapped to cohesive mass-spring networks in $\mathcal{P}$.

The resultant $6 \times 6 \times 4$ workspace accommodated a maximum of $|\mathcal{M}|=245$ potential mass positions and $|\mathcal{S}|=1648$ potential springs. 
Each robot was centered in the x-y plane according to its center of mass and shifted to the bottom of the workspace to ensure ground contact prior to behavior. 
This procedure ensured stable initial conditions for locomotion while maintaining consistent relative positioning between robots of different morphologies.

To identify unique morphologies, we defined an equivalence relation on the genotype space that accounted for translations and symmetries. Two genotypes were considered identical if, after aligning their occupied voxels to the origin, one could be transformed into the other through any combination of: (1) 90° rotations about the z-axis, (2) reflection about the x-axis, or (3) reflection about the y-axis. Each unique design was represented by its lexicographically minimal form across all such transformations.

\subsection{Differentiable simulation}
\label{methods-simulator}

We here extend the differentiable 2D mass-springs simulators developed by \citet{hu2019difftaichi} and \cite{strgar2024evolution} to three dimensions and add exterception: perception of external stimuli outside the body, namely light.
% we simulated 3D robots using a Hookean mass-spring system with semi-implicit Euler integration. 
Masses on $\mathcal{M}$ hosting photoreceptors were connected by actuating springs on $\mathcal{S}$ (defined above in Sect.~\ref{methods-morphology-design-space}), which exerted forces on their endpoint masses to perform phototaxis: movement toward a light source. 

During simulation, 
% the universal controller (which detailed in the following section, Sect.~\ref{methods-universal-control})
spring rest lengths may be actuated continuously between $\pm 20\%$ of their initial values derived from $\mathcal{P}$ (see Sect.~\ref{methods-morphology-design-space}). 
Spring forces were computed according to Hooke's law $F = k(L - L_0)$, where $k=1.5 \times 10^4$ N/m is the spring stiffness coefficient, $L$ is the current spring length, and $L_0$ is the modulated rest length. 
Resulting impulses, as well as damping and gravitational forces, were used to update velocities for each mass, and in turn mass positions were updated using the new velocities.

The terrains along which robots behaved were modeled using randomly sampled height maps (see Appx.~\ref{appendix-dataset-random-environment-generation} for details). 
During simulation, terrain heights at arbitrary coordinates $(x, y)$ were computed through bilinear interpolation of the height map. 
For collision handling, we detected when a mass' updated $z$-coordinate fell below the interpolated terrain height at its $(x, y)$ position. Upon detection, we employed a three-phase resolution: (1) iterative bisection on the interval [0, $dt$] to estimate the time of impact and advance the mass to the contact point, (2) velocity projection onto the contact surface normal (estimated via central differences), and (3) constrained motion along the surface tangent for the remaining timestep. Following \citet{strgar2024evolution}, friction forces were computed by negating the tangential velocity component and clamping its magnitude to not exceed the magnitude of the normal velocity component.

Our simulator was implemented in the Taichi programming language \cite{hu2019difftaichi}, providing both GPU acceleration for parallel, multi-robot simulation and automatic differentiation capabilities. The simulator was directly integrated with a PyTorch-based universal controller (Sect.~\ref{methods-universal-control}), enabling end-to-end backpropagation through 1000 timesteps ($dt = 0.004$s) of physics simulation and neural control for gradient-based optimization of the controller parameters.


%%%%%%%%%%%%%%%%%%%%%%%%%%%%%%%%%%%%%%%%%%%%
\begin{figure}[b]
    \centering
    \vspace{-10pt}
    \includegraphics[width=\columnwidth]{figs/xover.pdf}
    \vspace{-18pt}
    \caption{\textbf{Recombination of substructures.}
    A pair of designs (parents; \textbf{A, B}) is combined via crossover to produce a new design  (offspring; \textbf{C}) that inherits components from both parents.
}
    \label{fig:results-xover}
    \vspace{-8pt}
\end{figure}
\begin{figure*}[t]
    \centering
    \includegraphics[width=0.975\textwidth]{figs/com_trace.jpg}
    \vspace{-10pt}
    \caption{\textbf{Evolution of phototaxis.}
    The five worst designs in the population are depicted before (top row) and after (bottom row) zero-shot evolution.
    Each design (black dotted footprints) was placed in the center of a randomly generated map.
    Before evolution,
    not all of designs in the population could
    move (gray to white trajectories)
    across any terrain
    toward a light source (gold stars) using the pretrained controller.
    After evolution, they could.
    One of the design principles that evolution discovered is that larger footprints increase locomotion stability.
    }
    \vspace{-12pt}
    \label{fig:results-com-traces}
\end{figure*}
\begin{figure*}[t]
    \centering
    \includegraphics[width=0.95\textwidth]{figs/performance.pdf}
    \vspace{-8pt}
    \caption{\textbf{Performance and diversity.} 
    Morphological pretraining (\textbf{A}) converges with 70\% improvement from baseline. 
    Simultaneous co-design (from scratch without pretraining; \textbf{B}) achieves similar training loss; 
    but, population diversity (mean pairwise Hamming distance on genotypes) collapses as evolution converges to a single species of similar designs which simplifies shared control.
    Zero shot evolution (using the pretrained controller; \textbf{C}) rapidly improves test performance, but also suffers diversity collapse as evolution compiles slightly modified clones of the designs that are the most compatible with the pretrained model.
    Few-shot evolution (\textbf{D}) resets the pretrained controller at the start of each generation 
    % (spikes in the training curve) 
    and performs 60 finetuning steps per generation. 
    This significantly increases and sustains diversity as well as performance. 
    Solid lines represent 
    the batch (training; blue)
    or
    population (test; orange) means, 
    averaged across three independent trials.
    Shaded regions surrounding the mean curves show the minimum and maximum values across the three trials.
    }
    \vspace{-10pt}
    \label{fig:results-performance}
\end{figure*}
%%%%%%%%%%%%%%%%%%%%%%%%%%%%%%%%%%%%%%%%%%%%


\subsection{The universal controller}
\label{methods-universal-control}

We employed a simple multi-layer perceptron (MLP) as a universal controller for adaptive phototaxis: guiding a population of thousands of morphologically diverse robots towards arbitrarily positioned light sources across randomly varying, rugged terrains.
The network mapped two input streams to spring actuation signals: photosensor readings from masses and central pattern generator (CPG) inputs. 
To accommodate all possible body plans in the morphospace (defined in Sect.~\ref{methods-morphology-design-space}), the network's input dimension was set to $|\mathcal{M}|$ (the maximum number of masses) and output dimension to $|\mathcal{S}|$ (the maximum number of springs). Sensors and actuators not present in the a specific robot's body had their corresponding signals masked to zero, providing an implicit morphological conditioning through observation and action space masking.

Each mass-bound photosensor measured light intensity following the inverse square law relative to the light source position. 
Sensor readings for each robot were normalized by subtracting the mean computed across that robot's active (unmasked) sensors, providing a zero-centered, embodied irradiance gradient. 

Following \citet{hu2019difftaichi} and \citet{strgar2024evolution},
CPG inputs consisted of five sinusoidal waves with angular frequency $\omega=10$ rad/s and phase offsets evenly spaced by $2\pi/5$ radians. Over the 4-second simulation period (1000 timesteps, $\Delta t=4e-3$s), these oscillators completed approximately six cycles.

The MLP architecture consisted of an input layer (dimension 250: $|\mathcal{M}|$ mass sensors plus 5 CPG inputs), three hidden layers (dimension 256 each), and an output layer (dimension 1648: $|\mathcal{S}|$ springs). Each hidden layer was followed by layer normalization and ReLU activation, while the output layer used a $\tanh$ activation. All layers included learnable biases. In total the model consisted of 620,912 learnable parameters. 

Network weights were initialized using a Xavier uniform distribution (gain=1.0) \cite{glorot2010understanding} with zero-initialized biases, and the network was optimized using Adam \cite{adamkingma} ($\beta_1=0.9$, $\beta_2=0.999$) with gradient norm clipping at 1.0. 
Learning rates were scheduled using variants of cosine annealing with restarts (detailed in Sects.~\ref{methods-pretraining},~\ref{methods-few-shot-evolution}, and~\ref{methods-codesign}).


\subsection{Morphological pretraining}
\label{methods-pretraining}

The universal controller was pretrained 
across a dataset of over 10 million distinct robot morphologies 
(see Appx.~\ref{appendix-dataset-random-robot-generation} for details). 
The controller was trained over 1400 learning steps to minimize the batch mean of $d_1/d_0$, where $d_1$ and $d_0$ represent each robot's final and initial distances from its target light source, respectively. 
This relative distance formulation ensured robots were not penalized for being initialized far from their targets and equally incentivized fine-grained control in robots initialized near their targets.

We used a batch size of 8192, distributed in equal partitions of 1024 across a single compute node consisting of eight H100 SXM GPUs. 
Each sample consisted of a randomly-generated robot morphology (see Appx.~\ref{appendix-dataset-random-robot-generation} for details) 
a randomly-generated terrain shape
and a randomly-positioned light source (see Appx.~\ref{appendix-dataset-random-environment-generation} for details), and was seen exactly once during training. 
Training used a cosine annealing with restarts schedule, with initial learning rate $1e^{-3}$, cycle length starting at 10 steps and doubling each restart, minimum learning rate $1e^{-5}$, and a decay rate of 0.7 applied to the starting learning rate at each cycle. 

\subsection{Zero-shot evolution}
\label{methods-zero-shot-evolution}

Here, we introduce a novel robot design paradigm that leverages a frozen, pretrained universal controller to rapidly evaluate non-differentiable changes to a given robot's body plan. 
By using a single, fixed controller for all body plans,
the design space may be efficiently explored without the computational burden of training a custom controller for each body plan. 
We refer to this method as ``zero-shot evolution''. 

We initialized a population of 8192 random robot morphologies (unseen during pretraining) and evaluated each on a fixed test set of terrain and light source position pairs (see Appx.~\ref{appendix-dataset-evaluation-environments} for details). A simple genetic algorithm was then applied iteratively: the population produced an equal number of offspring through two variation operators (described below), new offspring were evaluated once on the test set, and the top 50\% across parents and offspring (using cached evaluation scores for parents) were selected to form the next generation.

Robot offspring were produced through one of two variation operators: mutation and recombination. The population was partitioned into two distinct groups: a random 25\% of members were assigned to produce offspring through mutation, while the remaining 75\% were reserved for producing offspring through recombination (or crossover). Each member in the mutation group produced a single offspring through random bit flip mutations performed on their genotype. Flips occurred with probability $p = 1/N$ where $N = 6 \times 6 \times 4$, the total number of voxels in the robot's genotype. After mutation, genotypes were processed to ensure validity: only the largest connected component was retained, and the resulting structure was translated to the bottom center of the workspace. If a mutation produced a body that was either empty or identical to a previously seen body, the process was repeated with the mutation rate increased by 2.5\% until a valid, unique design was obtained.

From the recombination group (75\% of the population), pairs of distinct parents were randomly sampled to produce offspring through crossover (Fig.~\ref{fig:results-xover}). 
For each sampled pair, an offspring's genotype was created using a bitwise exclusive or (XOR) operation on the parent genotypes. As with mutation, post-processing retained only the largest connected component and centered it at the bottom of the workspace. If the resulting design duplicated a previous one, it was discarded. The sampling and generation process was repeated until the number of offspring equaled the size of the recombination group (75\% of the population).


\subsection{Few-shot evolution}
\label{methods-few-shot-evolution}

In this experiment we extend the zero-shot paradigm (described above in Sect.~\ref{methods-zero-shot-evolution}) by fine-tuning 
the pretrained universal controller 
to the current population
at every generation of morphological evolution.
We refer to this approach as ``few-shot evolution''.
% 
The experimental setup of few-shot evolution matched the zero-shot case, with one key difference: before evaluation, each generation received 60 fine-tuning steps (30 for parents, 30 for offspring).
The number of fine-tuning steps was
empirically chosen to balance controller adaptation against evolutionary search while maintaining comparable maximum wall-clock time across experiments. 
At the start of each generation, the controller's weights were reset to their pretrained values and the optimizer state was reinitialized. 
Fine-tuning used a cosine annealing learning rate schedule with initial and minimum rates of $3.5e^{-4}$ and $3.5e^{-5}$, respectively. The cycle length was set to 100; however each cycle was truncated to align each cycle with one generation's 60 fine-tuning steps resulting in an effective minimum learning rate of $1.5e^{-4}$. 
Since every generation re-initialized the pretrained weights, we did not decay the learning rate at the start of each cycle. 

\subsection{Simultaneous co-design from scratch}
\label{methods-codesign}

In our third and final experimental group, 
we remove morphological pretraining
and instead 
simultaneously 
evolve a population of robots
and
learn their universal controller, 
from scratch.
Unlike few-shot evolution, controller parameters and optimizer state are inherited across generations rather than being reset. The genetic algorithm operates as before, but we reduce the per-generation training to just 2 learning steps (1 for parents, 1 for offspring) to maintain parity with our pretraining experiments, where each training batch was unique.

Initially, we employed the same cosine annealing learning rate schedule used in morphological pretraining, but we found it was beneficial to reduce the start-of-cycle learning rate decay factor from 0.7 to 0.65 in order to stabilize learning across cycle restarts in this setting.

% \begin{figure}[t]
    \centering
    \includegraphics[width=\columnwidth]{figs/shape-stats.pdf}
    \vspace{-20pt}
    \caption{\textbf{Evolved populations.}
    Population performance, phenotype footprint size, and body mass for the initial (randomly generated) and evolved design populations.
    }
    \vspace{-18pt}
    \label{fig:appendix-robot-data-gen-stats}
\end{figure} % inside results

\section{Results}
\label{sec:results}

In this section we evaluate 
the results of 
morphological pretraining (Sect.~\ref{results-pretraining-performance}),
zero- and few shot evolution (using the pretrained model; Sect.~\ref{results-zero-and-few-shot-evolution}),
and simultaneous co-design from scratch (without pretraining; Sect.~\ref{results-simultaneous-co-design}).


\subsection{Pretraining performance}
\label{results-pretraining-performance}

Across three independent trials,
each using a distinct dataset of 
randomly-generated morphologies and environments,
pretraining 
exhibited stable learning trajectories 
with low variance across trials (Fig.~\ref{fig:results-performance}A), 
converging in approximately 1,400 learning steps (56 minutes of wall-clock time).
% 
Loss was defined as the ratio of final to initial distance from the target light source. 
At initialization with random controller weights, this ratio was 1.0, indicating robots remained stationary throughout simulation. 
After pretraining, the loss stabilized at approximately 0.3, representing a 70\% improvement. 
That is, in environments sampled from the training distribution, robots using the pretrained universal controller traversed an average of 70\% of their initial distance to the light source. 
Since each training batch used novel morphologies, we omitted model selection with a \mbox{validation set.}

% The pretrained model effectively controlled diverse, previously unseen robot morphologies in challenging tasks at the training distribution's edge (Appx.~\ref{appendix-dataset-random-robot-generation}). 
% Fig.~\ref{fig:results-zero-shot-transfer} shows the pretrained model's generalization performance across more than 24,000 distinct test morphologies. 
% Despite the test environments being more challenging than the training distribution, the mean performance of the test morphologies aligned with the final loss observed during pretraining. 

To visualize the breadth of morphological diversity handled by the pretrained controller, Fig.~\ref{fig:appendix-zero-shot-robot-grid} showcases a representative sample of successful robots. 
These examples were selected uniformly from the top-performing 50\% of the test morphology set. 
The selected bodies exhibit high variation in both scale and morphological characteristics demonstrating the non-trivial generalization of the universal controller.


%%%%%%%%%%%%%%%%%%%%%%%%%%%%%%%%%%%%%%%%%%%
\begin{figure}[t]
    \centering
    \includegraphics[width=\columnwidth]{figs/shape-stats.pdf}
    \vspace{-20pt}
    \caption{\textbf{Evolved populations.}
    Population performance, phenotype footprint size, and body mass for the initial (randomly generated) and evolved design populations.
    }
    \vspace{-18pt}
    \label{fig:appendix-robot-data-gen-stats}
\end{figure}
%%%%%%%%%%%%%%%%%%%%%%%%%%%%%%%%%%%%%%%%%%%

\subsection{Zero- and few shot evolution (with pretraining)}
\label{results-zero-and-few-shot-evolution}

A population of morphologies was evolved through random mutation and crossover operations, using the pretrained universal controller.
On the same challenging set of tasks used for evaluating pretrained controller generalization,
the population converges to near optimal performance
in 100 generations of evolution (17 minutes of wall-clock time)
without finetuning the controller (``zero shot evolution''; Sect.~\ref{methods-zero-shot-evolution}).
% 
Although zero-shot evolution shows rapid convergence in controlling thousands of distinct bodies, this success masks a key pattern: design population diversity decreases as performance improves. 
Fig.~\ref{fig:results-performance}C reveals this pattern---after a brief diversity spike at evolution's onset, the population gradually homogenizes. 
We term this phenomenon diversity collapse, measuring diversity as the population's mean, pairwise Hamming distance in (and normalized to) the genotype space $\mathcal{G}$ (defined in Sect.~\ref{methods-morphology-design-space}). This metric naturally reflects differences in morphology (body) as well as sensing and actuation masking in the universal controller (brain).

We found that generational finetuning of the universal controller for the current population (``few shot evolution'')
not only preserves diversity but in fact significantly increases diversity (Fig.~\ref{fig:results-performance}D).
This is a somewhat surprising result as there was no explicit selection pressure to maintain diversity.
The process of morphological evolution seems to intrinsically increase population diversity. 
However, in absence of generational finetuning, there is a tipping point at which it is easier to purge diversity, replacing the worst designs with slightly modified clones of the best, than to discover novel morphological innovations with superior performance.


\subsection{Simultaneous co-design (\textit{without} pretraining)}
\label{results-simultaneous-co-design}

Ablating pretraining (and funetuning),
and instead simultaneously optimizing morphology and universal control, together from scratch, 
results once again in rapid diversity collapse (Fig.~\ref{fig:results-performance}B).
Performance plateaus in well under 180 generations, corresponding to 360 controller learning steps and 109 minutes of wall-clock time.
% 
The extent of diversity collapse can be seen in Fig.~\ref{fig:results-morpho-grid}B, where we visualize morphologies from one of the three independent trials,
and in Fig.~\ref{fig:appendix-robot-data-gen-stats} where we plot morphological variance across evolved populations in terms of footprint size and body weight.

In all three co-design paradigms (zero shot, few shot, simultaneous), universal control enabled successful crossover (Fig.~\ref{fig:appendix-variation-operator-stats}).
In terms of offspring survival,
crossover was initially much more successful than mutation.
But in the case of simultaneous co-design, this was not an apples to apples comparison because each generation provided the controller with more time to learn how to control the population, and the randomly initialized controller was very bad at the task.
And so it was not clear if the success of offspring was due to changes in parent morphology or improvements to the universal controller.
The superior performance of pretraining across random morphologies, shows that the designs produced by crossover during simultaneous co-design were no better than random designs. 
In zero-shot and few-shot evolution, however, the pretrained controller is quite good at the very start, and in zero-shot the controller is not updated during evolution, providing clear evidence of successful crossover prior to diversity collapse.



\vspace{-6mm}
\section{Discussion}
\vspace{-2mm}
Our findings shed light on a wide range of human factors in designing personhood credentials.
%considered in PHC design, including user perceptions (RQ1), factors influencing preferences (RQ2) and design suggestions (RQ3). 
In this section, we discuss how the findings can contribute to current state of knowledge in designing user-centered and secure PHC design. 
\vspace{-2mm}
\subsection{Main Findings}
\vspace{-2mm}
%\fixme{need more work on explaining main findings and how these are unique and adding new knowledge to the literature. need critical mapping to literature. please read the discussion as reference: https://arxiv.org/pdf/2410.01817 ;  https://dl.acm.org/doi/abs/10.1145/3544548.3581498; https://www.usenix.org/system/files/usenixsecurity24-sharma.pdf} \ayae{updated}

\textbf{Nuanced Credential Preference}
Identity management has long been a focal point of user-facing systems, including social media platforms, gaming environments, and collaborative tools, etc~\cite{gorwa2020unpacking, cetinkaya2007verification, sharma2024future, sharma2024unpacking}. We have also observed technological and ideological shifts towards 
decentralized identity- commonly referred as --self-sovereign identity amidst the criticism of large technology companies' data handling practices~\cite{nytimesCambridgeAnalytica}.The most cited case is decentralized (DIDs), with emerging proposed systems, DECO~\cite{zhang2020deco}, Town-Crier~\cite{zhang2016town} -- where users authorizing the release of personal credentials from user devices to websites for proving certain characteristics about themselves. 
While initiatives such as the W3C’s Decentralized Identifier Working Group aim to develop standards~\cite{identityDecentralizedIdentity, w3cccgDecentralizedIdentifiers}, they largely fail to address the technical and usability goals.
Nonetheless, users' preference on managing (e.g. recovery, data handling, trust on issuer ecosystem) verification credentials remains largely unexplored. 

Our work sheds light on trade-offs people consider in onboarding and managing PHC. Our findings highlight some concerns towards PHCs, partly because of “unknown risk” vectors as a new technology compared to traditional verification. Despite these concerns, we find diverse level of adoption preferences
influenced by the “type of data required” for PHC credential issuance and verification as well as personal “security standards” for different services (e.g, finance, health, government
related). 
%the different factors end users consider if they were to onboard this emerging personhood credential systems. 
%We identified the relative preferences between various biometrics and other credentials as proof of personhood. Our study results support prior work\cite{Zimmermann2017-wr, mare2016study,De-Luca2015-mp,Bhagavatula2015-fi}, showing that fingerprint and iris recognitions are preferred with their security aspects. 
%\fixme{add main findings in 4 lines and mentioned what is unique about the findings comparing / citing literature}
We also expanded the findings to include nuanced preferences and underlying reasoning that explains why users have certain preferences, extending beyond what existing literature has limited to identifying preferences alone.
For instance, their preference for government-issued IDs is associated with their familiarity with a traditional method of verification. Also, their varied preference for biometrics is backed by subjective perceptions around efficacy and privacy sensitivity. For instance, they often considered facial recognition more resilient verification process than fingerprints.
%. 
%These results emphasized the need for clear explanations of the data requirements of PHC to ensure their efficacy in terms of privacy and security.

%Previous work has explored users’ preferences for different types of credentials\fixme{why authentication, this work is about verification} in identity management systems. For instance, studies such as \cite{mare2016study,De-Luca2015-mp,Bhagavatula2015-fi, Zimmermann2017-wr} have compared usability and user perceptions of various types of biometrics (e.g., iPhone fingerprint, Android Face) as a way of identity management in daily life. \fixme{compared for what? incomplete sentence, dones't add anything}. In contrast, this study investigates users' preferences of credential as proof of personhood to verify legitimacy and uniqueness.\fixme{is this the goal of this study? no far no mentioned of personhood credential at all.}. 



%In the government benefits scenario, government ID was preferred, whereas phone numbers were the most commonly selected credential in the LLM application scenario. For social media, biometrics such as face and fingerprints were highly preferred. This dynamic nature of credential preference depending on contexts emphasis the need of developing context-aware PHC systems. One future direction respecting users' context-dependent preference is decentralized architecture where users get PHCs from multiple PHC issuers and use them differently depending on contexts. 

\textbf{How to Build Trust \& Scale PHCs}
More recently, a cross-industry collaboration \cite{adler2024personhood} resurfaced the conversation of personhood credential 
\cite{borge2017proof, ford2020identity, de2024personhood, sharma2024experts} emphasizing the need of an ecosystem for PHC issuers to facilitate scalability. A fundamental challenge remains the "chicken-and-egg" dilemma: the absence of a broad ecosystem of PHC issuers hinders the adoption of systems that rely on PHCs, and conversely, the lack of such systems makes it difficult to incentivize the establishment of PHC issuers. Our findings highlight an interdependence and the importance of carefully considering both the issuing ecosystem (e.g., centralized versus decentralized models) and the type of issuer (e.g., government entities versus private organizations) as key factors influencing trust and perceptions of security among users to onboard.


%This is one of the fundamental limitations of current approaches to PHC creation to address the chicken-and-egg problem in PHC issuance. Without an ecosystem with a broad base of PHC issuers, systems that leverage PHCs will be slow to arise; conversely, without systems that leverage PHCs, it is hard to motivate the creation of PHC issuers. Our findings indicate issuing ecosystem (centralized vs. decentralized), and the issuer (government vs. private company) as the main factors impacting people's trust and security perception. To facilitate broader and equitable access, we suggest future work to involve stakeholders whom people trust as a PHC issuer. 

Our results highlighted the perceived benefits by users regarding PHCs, which is fairness in representation, aligning with that prior work's discussed proof of personhood is a key in fair online environments, especially when voting or accessing limited resources by eliminating AI-powered manipulations or duplications
\cite{de2024personhood}. 
However, our research also surfaced significant concerns that must be addressed to improve user trust and acceptance. Participants expressed apprehensions about the centralization of PHC issuance, fearing over-concentration of power and control. Ambiguities in regulations surrounding PHCs further compounded these concerns, reflecting broader uncertainties about governance and accountability. Additionally, users highlighted the need to ensure the authenticity of PHCs while addressing risks associated with the misuse of anonymity. These findings resonate with existing literature on privacy and security education, which underscores the importance of user understanding and clear communication in the adoption of new technologies \cite{acquisti2015privacy}. Future research should focus on increasing the explainability and transparency of PHC systems to address these concerns. 

%for scalability, contextual service-specific privacy/security design opportunities of PHC for better users interaction; however, to our knowledge, the preference of personhood credential has not been explored from the viewpoint of end users, particularly investigating factors that might be limiting equitable access to such technologies. Our study is the first study to investigate users' perspectives to understand the pain points and conceptualize designs to address those challenges. 


%\fixme{wording seems like result section. Rewrite as if its you who are interpreting the result and giving statement}
%When participants described their current practices with identity verification and preferences regarding PHCs, their explanations reflected that their decision-making was fundamentally rooted in the trust they had in stakeholders. 

%Users' trust is shaped by their own experiences and shared stories by family and friends, and the reputations they have encountered \fixme{what does it mean? again too much about just trust, is this the only interesting result? rewrite this paragraph highlighting the results from RQ2}. 
%Taking this into account, there are two potential directions for future development of PHCs: 1) involve trustworthy stakeholders: 2) build systems that reduce reliance on all third parties, including stakeholders. These directions are indicated in the participants' design sketches. Users favored government involvement in PHC systems due to trust with the government, while also showing interest in blockchain architectures where its decentralized and transparent nature reduces the cost of trust by minimizing the need for intermediaries and enabling verifiable interactions.  Some participants proposed hybrid approaches managing both centralized surveillance by the government and a decentralized verification system.



%\textbf{Perceived Benefits and Concerns of PHCs} Participants emphasized that PHCs could minimize personal information exposure and mitigate the deceptive activities by fake accounts and bots. These features address user concerns about privacy and efficiency in digital interactions. On the other hand, the primary concerns revolved around data security and trustworthiness of issuers. Participants feared that a single PHC issuer could become a single point of failure in the event of a data breach. They were also skeptical about private sector issuers and expected for transparency and accountability.
\vspace{-2mm}
\subsection{Design Implications}
\vspace{-2mm}
Drawing upon users' needs and preferences, we suggest actionable design implication for personhood credentials. 
%\tanusree{read 5.3 of this paper to better understand how to write design implication https://www.usenix.org/system/files/usenixsecurity23-sharma.pdf and 5.2 in this paper https://dl.acm.org/doi/pdf/10.1145/3544548.3581498. ideal way is to also investigate ccs paper proceeding on usability and internet measure track paper to see how this community usually present design implications or if there is anything unique. current draft read so high level to get any useful design direction for future}
\iffalse
\textbf{Transparency Interface}
We observed participants faced trade-offs between familiar
security guarantees associated with traditional verification
methods over the less clear assurances of emerging PHC. Moreover, they also anticipated the potential risk of the centralized power of the issuer and uncertain regulations. Such explainability and accountability issues can be mitigated by designing interfaces driven by transparency. As described in the sixth principle of Privacy by Design \cite{cavoukian2021privacy} - \textit{"The data subject is made fully aware of the personal data being collected, and for what purpose(s). All the component parts and operations remain visible and transparent, to users and providers alike."}, PHC issuers are expected to state explicitly the purpose, operations and policies regarding personal data collection practices. For example, they could leverage privacy nutrition label, which is the concept of standardized presentation of privacy policies ideated from food nutrition label \cite{kelley2009nutrition}. In practice, Apple has launched iOS privacy labels that embodied privacy nutrition label in their platforms \cite{li2022understanding}. By prototyping the information to be explained for end users through the iterative process proposed by prior work of transparency design \cite{eiband2018bringing}, they could clearly communicate the essential information while decrease cognitive load of users by standardizing the information according to the format of privacy nutrition labels.
\fi

\textbf{Interface Design to Facilitate Verification Choice.}
Our findings shows users' skepticism, partly because of \textit{unknown risk''} vectors of PHC as a new technology. Despite the unknowns, our findings also indicates diverse level of adoption/onboarding preferences towards PHC issuance, such as, type of data requirement to verify themselves which largely depends on their own \textit{``security standard''} developed for different types of services. With the preferences in mind, one possible way to first clearly add list of purpose, operation and policy of PHC leveraging design such as, privacy nutrition label\cite{kelley2009nutrition, li2022understanding} like Apple, to allow users to know the new tech.
As of credential issuance choice, a tiered system of PHCs with varying levels of verification strength
%an adaptive interface can be designed
to allow users to choose ground truth data (e.g., gov id, face, video, fingerprint, social graph, etc) based on their security standards. The interface could explicitly add a design of tiered verification options, each corresponding to a different level of security: Level 1 (Low Sensitivity): email or phone verification; Level 2 (Medium Sensitivity): physical id;  Level 3 (High Sensitivity): Biometric fusion (e.g., facial recognition + voice print);  Level 4 (Very High Sensitivity): Multi-factor PHC (e.g., biometrics + social graph verification+physical id), etc. We can implement modular architecture, allowing easy addition or modification of verification methods as technology evolves. 

\textbf{Portability of Personhood Credential.} Our result reveals varied preferences across different services, ranging from financial, health, government to social media. One expectation of users to have a stremlined approach so they don't need to onboard with multiple verification for services they use. A possible way is to design interoperable cryptographically verifiable credentials. Verifiable credential contains claim about the credential holder, issued by a trusted entity, and can be verified without contacting the issuer to prove themselves across various platforms. This is essentially leveraging emerging solutions, such as, Zero-Knowledge Proofs~\cite{tobin2016inevitable} with design principles and standards, such as, W3C verifiable credentials data model, DIF universal resolver~\cite{mazzocca2024survey}, etc. 

\textbf{Dynamic \& Multi-factor Personhood Verification}
One of the repeated concerns in our study whether PHC can be inadvertently shared/used by friends/family, hacked, stolen.  
%Although PHC verification itself does not contain identifiable information, there is a possibility that PHC will be misused if the user's local device is hacked or stolen. 
To address this concern, a potential design implication is to design a dynamic Multi-Factor Personhood Credential (PHC). First, when users verify themselves for the first time, the prompt can combine interactive biometric such as video call, with interaction knowledge question and sharing physical id to compare several ground truth data to issue a robust PHC. Second, to maintain security, a period biometrics verification with time-based trigger system can be designed to prompt for biometrics verification at random intervals or during high risk activities. 

%In this case, even on a platform where PHCs ensure one unique individual per account, malicious users could gain access to multiple accounts, resulting in significant damage to the platform's trustworthiness with sybil attack \cite{douceur2002sybil}. To ensure their PHC is used only by themselves, service providers could add another layer of security to the PHC authentication process. For instance, participants suggested multi-factor authentication (MFA) when interacting service providers to confirm PHC is used by the authentic user. 
%When users perform identity verification on various online services using PHC, combining PHC with biometric authentication on local devices, such as Apple Face ID, can reduce vulnerabilities. Having a 

\iffalse
\textbf{Dynamic Authentication}
One of the representative users’ concerns is the case when their PHC is hacked and used across different services by a third person. Although PHC itself does not contain identifiable information, there is a possibility that PHC will be misused if the user's local device is hacked or stolen. In this case, even on a platform where PHCs ensure one unique individual per account, malicious users could gain access to multiple accounts, resulting in significant damage to the platform's trustworthiness with sybil attack \cite{douceur2002sybil}. To ensure their PHC is used only by themselves, service providers could add another layer of security to the PHC authentication process. For instance, participants suggested multi-factor authentication (MFA) when interacting service providers to confirm PHC is used by the authentic user. When users perform identity verification on various online services using PHC, combining PHC with biometric authentication on local devices, such as Apple Face ID, can reduce vulnerabilities.
\fi

\textbf{Decentralized Standards for Industry-Government Issuance System.}
 Our work indicates issuance system and issuers (e.g. govt, private company; decentralized vs centralized) are one of the main factors leading to security and trust perception, thus the broader adoption. In the same line, fundamental limitation of current approaches to PHC creation is their signal failure to address the chicken-and-egg problem in PHC issuance. Without an ecosystem with a broad base of PHC issuers, systems that leverage PHCs will be slow to arise; conversely, without systems that leverage PHCs, it is hard to motivate the creation of PHC issuers. In the United States, for instance, despite industry coalitions pursuing decentralized identity credentials for at least seven years~\cite{mediumDecentralizedIdentity}, digitally signed state IDs are currently available only in California~\cite{caWalletPilot}. 
 To ensure global accessibility in supporting multiple stakeholders as issuers, the system would incorporate cross-chain interoperability protocols like Polkadot or Cosmos and utilize a permissioned blockchain network (e.g., Hyperledger Fabric) to create a distributed ledger for credential issuance and verification. Smart contracts~\cite{sharma2023mixed} would govern the issuance process, ensuring compliance with predefined standards set by both industry and government entities. 
%When it comes to online identity verification, users' trust in government is more apparent when considering who should be the stakeholders. They suggested that PHC architecture involves the government in various ways (e.g., serving as the PHC issuer and supervising PHC issuers). However, the government may not always be considered a trustworthy entity for users, as Hosking \cite{hosking2019decline} discussed the decline of trust in government due to structural uncertainty in society, such as financial crisis and economic inequality. Thus, government involvement should be framed as colocation as one of trusted entities, without centralizing authority within the government. We suggest that the government provides accreditation to PHC issuers to guarantee their reliability. For instance, U.S. Department of Education approves organizations that carried out accreditation to universities and colleges to assure the quality of institutions and programs \cite{eaton2015overview}. Users can assess the security and ethical standards of PHC issuers based on accreditation the government issued, and service providers can also refer to this accreditation when determining which PHCs they should support in their services.
%\tanusree{unoack how this type of choice can be provided to users and which way?} \ayae{updated}
\iffalse
\textbf{Blockchain-based Infrastructure}
Users’ concern against PHC issuers’ trustworthiness may be addressed by decentralizing infrastructure, for instance, blockchain-based issuer proposed by the participant. Blockchain technology drastically reduce the cost of trust given its distributed and transparent nature that the transactional data is open to the entire network and distributed consensus offers secure data exchanges \cite{casey2018blockchain, de2024personhood}. Leveraging this nature, blockchain-based identity management solutions has emerged in recent years \cite{liu2020blockchain}. For example, Sovrin achieves privacy-oriented identity management like identifying private customers under pseudonyms with zero-knowledge proof encryption \fixme{how come zkp is an encryption?} \cite{tobin2016inevitable}. Also, ShoCard elimiates the need for a third-party database by encrypting personal information on individuals' devices and storing and managing data derived from this information on the blockchain \cite{al2023enhancing}. We suggest PHC issuers store only verification metadata on-chain without storing any information about credentials or PHCs while users store PHCs and details information in local devices.  By doing so, users can manage their credentials and PHCs by themselves while achieving greater transparency in issuers' data practices.
%\tanusree{not just maybe, rather you need to connect the literature and provide specific design implication of such design of infrastructure} \ayae{updated}
\fi


%\tanusree{need to unpack, not clear how such system can be design/developed. need to be near-specific on the design} \ayae{updated}

%\textbf{Database Protection}

%\tanusree{unpack certain design for interface, be specific. in essence, this section is to show your design skill} \ayae{updated}

\vspace{-4mm}
\subsection{Limitations}
\vspace{-2mm}
%We were limited to recruiting participants covering a wider range of demographics via social media and recruitment platforms such as CloudResearch and Prolific.

Our interview study has several limitations. Our recruitment of participants resulted in limited diversity in educational backgrounds and age groups. This may restrict the generalizability of our findings, as individuals with a lower age range might exhibit more preferences toward new technology and can have stricter expectations compared to participants with other
educational backgrounds and age range. 

%which may lack representation from individuals with less formal educational opportunities and older adults.
%\fixme{what limitation in recruitment? exactly what aspect you consider as limitation?}. 
%Future work should explore additional recruitment strategies to reach a more diverse participant pool from a broader spectrum of ages and educational backgrounds. 
Second participants who didn't have prior knowledge of PHC responded based on an explanation or information video presented during the study. Their responses might differ if they were provided with hands-on interactive PHC system to better convey and understand the concept.
Building on the findings from this study, future work will focus on developing a functional prototype to provide participants with a more immersive and tangible experience.
%, facilitating a deeper understanding of their experience.
%Based on our design implications, future work also includes proposing potential components of PHC architecture and exploring an interface that reflects user needs, such as transparency and accountability, by enhancing the explainability of complex management of personal data in PHC.

\vspace{-2mm}
\section{Conclusion}
\vspace{-2mm}
%We interviewed 23 participants to explore users' perceptions and preferences regarding PHCs for verifying legitimacy and uniqueness in online interactions. 
Our study uncovered diverse user perceptions, including trade-offs between traditional verification methods and emerging approaches such as PHCs, as well as dilemmas between physical and digital verification. Furthermore, we highlighted nuanced preferences for each system design dimension: credentials, issuers, and architectures. Additionally, practical PHC functions such as limited credential validity, sensitivity-based selection, interactive human checks, and distributed issuance architectures were identified through design suggestions.
To our knowledge, this is the first user study to focus on PHCs. Our findings extend beyond PHCs, shedding light on key insights for identity verification.
%\vspace{-2mm}

\section{Ethics Considerations}
Our study design and procedures were reviewed and approved by the Institutional Review Board (IRB). We also considered the following ethical aspects:
\textbf{Disclosures}: All collected data (audio transcripts, sketches, and survey results) did not include personally identifiable information and were analyzed anonymously;
\textbf{Experiments with informed consent}: We ensured informed consent , and participants were informed that their participation was voluntary;
\textbf{Deception}: Participants were fully informed of all aspects of interview participation beforehand, clearly stating the scope of the study and its data collection.

\section{Open Science}
 To ensure transparency of this study, all details of our interviews, including the study procedures and survey contents, are available in the following link \url{https://anonymous.4open.science/r/PHC-user-study-14BB/}. We also ensure reproducibility by providing detailed documentation on how to proceed with our interview in Section \ref{sec:method}.
% \section*{Code}
\label{sec:code}

Code is temporarily withheld to maintain anonymity during review and will be made available upon publication.

  % SK: not necessary for preprint or camera ready (make site!)
% \section*{Impact Statement}
\label{sec:impact}

The goal of this paper is to advance the competence and diversity of intelligent robots.
There are numerous potential benefits
and societal consequences of robots,
% \cite{lin2014robot}, 
none which we feel must be specifically highlighted here.

% This work advances automated robot control and design through universal, robot-agnostic control. There are numerous potential benefits such as enabling rapid development of more capable robotic systems for beneficial applications. We acknowledge, however, that robot automation, and automation of robot design, poses risks such as 
% labor displacement and deployment safety concerns.

% Authors are \textbf{required} to include a statement of the potential 
% broader impact of their work, including its ethical aspects and future 
% societal consequences. This statement should be in an unnumbered 
% section at the end of the paper (co-located with Acknowledgements -- 
% the two may appear in either order, but both must be before References), 
% and does not count toward the paper page limit. In many cases, where 
% the ethical impacts and expected societal implications are those that 
% are well established when advancing the field of Machine Learning, 
% substantial discussion is not required, and a simple statement such 
% as the following will suffice:

% ``This paper presents work whose goal is to advance the field of 
% Machine Learning. There are many potential societal consequences 
% of our work, none which we feel must be specifically highlighted here.''

% The above statement can be used verbatim in such cases, but we 
% encourage authors to think about whether there is content which does 
% warrant further discussion, as this statement will be apparent if the 
% paper is later flagged for ethics review. % SK: not necessary for preprint
\section*{Acknowledgments}
This research was supported by
NSF award FRR-2331581
and
Schmidt Sciences AI2050 grant G-22-64506.
% and
% Templeton World Charity Foundation award no.~20650. 



\bibliography{main}
\bibliographystyle{icml2025}

\clearpage
\appendix
\onecolumn
\section{Implementation Details}
\subsection{Token-aware Preference Data Construction}
\label{sec:impl}
For all models that used for preference data construction, we adopt the following prompts presented in Figure \ref{fig: prompt-decom}, \ref{fig: prompt-selfinst}, \ref{fig: prompt-recomb}, \ref{fig: prompt-sub}, \ref{fig: prompt-neg} and \ref{fig: prompt-sub}. We set the temperate as 0.5 for all steps to ensure diversity. To ensure the data quality, we filter instructions with less than three constraints and more than ten constraints. We also filter preference pairs with the same chosen and rejected responses. 

For constraint dropout, we set the dropout ratio $\alpha$ to 0.3 to ensure that negative examples are sufficiently negative, meanwhile not deviate too much from the positive sample. We avoid dropout on the first constraint, as it often establishes the foundation for the task, and dropping the first one would make the recombined instruction overly biased.

\subsection{Token-aware Preference Optimization}
\label{sec:impl-dpo}
Our experiments are based on Llama-Factory \cite{zheng2024llamafactory}, and we trained all models on 8 A100-80GB SXM GPUs. The \texttt{per\_device\_train\_batch\_size} was set to 1, \texttt{gradient\_accumulation\_steps} to 8, leading to an overal batch size as 64, and we used bfloat16 precision. The learning rate is set as 1e-06 with cosine decay,and each model is trained with 2 epochs. We set $\beta$ to 0.2 for all DPO-based experiments, $\beta$ as 3.0 and $\gamma$ as 1.0 for all SimPO-based experiments, $\beta$ as 1.0 for all IPO-based methods referring to the settings of \citet{meng2024simpo}. All of the final loss includes 0.1x of the SFT loss.

\section{The Influence of Noising Scheme}
\label{app:noising}

Previous work has proposed various noising strategies in contrastive training \cite{lai-etal-2021-saliency-based}. While we leverage Constraint-Dropout for negative sample generation, to make a fair comparison with other strategies, we implement the following strategies: 1) Constraint-Negate: Leverage the model to generate an opposite constraint. 2) Constraint-Substitute: Substitute the constraint with an unrelated constraint.

\begin{figure}[h]
\centering
\includegraphics[width=0.6\linewidth]{figures/drop_ratio.png}
\caption{The variation of results on CFBench and AlpacaEval2 with different dropout ratios.}
\label{fig:drop_ratio}
\end{figure}

As shown in Table \ref{tab:detail-noising}, both the negation and substitution applied on the constraints would lead to performance degradation. After a thoroughly inspect of the derived data, we realize that instructions derived from both dropout and negation would lead to instructions too far from the positive instruction, therefore the derived negative response would also deviate too much from the original instruction. An effective negative sample should fulfill both negativity, consistency and contrastiveness, and constrait-dropout is a simple yet effective method to achieve this goal.

We also provide the variation of the results on CF-Bench and AlpacaEval2 with different constraint dropout ratios. As shown in Figure \ref{fig:drop_ratio}, with the dropout ratio increased from 0.1 to 0.5, the results on CF-Bench firstly increases and then slightly decreases. On the other hand, the results on AlpacaEval2 declines a lot with a higher dropout ratio. This denotes that a suboptimal droout ratio is essential for the balance between complex instruction and general instruction following abilities, with lower ratio may decrease the effectiveness of general instruction alignment, while higher ratio may be harmful for complex instruction alignment. Finally, we set the constraint dropout ratio as 0.3 in all experiments.

\begin{table*}[tt]
\centering
\resizebox{1.0\textwidth}{!}{
\begin{tabular}{cc|ccccc|ccccc}
\toprule
\multirow{3}{*}{\textbf{Scenario}} & \multirow{3}{*}{\textbf{Method}} & \multicolumn{5}{c|}{\textbf{Meta-LLaMA-3-8B-Instruct}}                                    & \multicolumn{5}{c}{\textbf{Qwen-2-7B-Instruct}}                                          \\
                                   &                                  & \multicolumn{3}{c}{\textbf{CF-Bench}}         & \multicolumn{2}{c|}{\textbf{AlpacaEval2}} & \multicolumn{3}{c}{\textbf{CF-Bench}}         & \multicolumn{2}{c}{\textbf{AlpacaEval2}} \\
                                   &                                  & \textbf{CSR}  & \textbf{ISR}  & \textbf{PSR}  & \textbf{LC\%}      & \textbf{Avg.Len}     & \textbf{CSR}  & \textbf{ISR}  & \textbf{PSR}  & \textbf{LC\%}      & \textbf{Avg.Len}    \\ \midrule
\multirow{6}{*}{PreInst}           & baseline                         & 0.64          & 0.24          & 0.34          & 21.07              & 1702                 & 0.74          & 0.36          & 0.49          & 15.53              & 1688                \\ \cline{2-12} 
                                   & Constraint-Drop               & \textbf{0.71} & \textbf{0.34} & \textbf{0.45} & \textbf{23.43}     & 1682           & \textbf{0.79} & \textbf{0.43}  & \textbf{0.54}          & \textbf{19.31}     & 1675                \\
                                   & Constraint-Negate             & 0.68          & 0.28          & 0.39          & 18.94              & 1688                 & 0.75          & 0.37          & 0.50          & 17.82              & 1663                \\
                                   & Constraint-Substitute             & 0.68          & 0.28          & 0.40          & 20.48              & 1706                 & 0.76          & 0.39          & 0.51          & 19.05              & 1709                \\ \bottomrule
\end{tabular}}
\caption{Experiment results of different noising strategies on instruction following benchmarks.}
\label{tab:detail-noising}
\end{table*}

\section{Mathematical Derivations}
\subsection{Preliminary: DPO in the Token Level Marcov Decision Process}
\label{app: prel}
% In the most classic RLHF methods, the optimization goal is typically expressed as an entropy bonus using the following KL-constrained:

% \begin{align}
% &
% \max_{\pi_\theta} \mathbb{E}_{a_t \sim \pi_\theta(\cdot | \mathbf{s}_t)} \sum_{t=0}^{T} [r(\mathbf{s}_t, \mathbf{a}_t) - \beta \mathcal{D}_{KL}[\pi_{\theta}(\mathbf{a}_t | \mathbf{s}_t)||\pi_{ref}(\mathbf{a}_t | \mathbf{s}_t)]]
% % \label{eq: rlhf_obj}
% \\
% &
% =\max_{\pi_\theta} \mathbb{E}_{a_t \sim \pi_\theta(\cdot | \mathbf{s}_t)} \sum_{t=0}^{T} [r(\mathbf{s}_t, \mathbf{a}_t) - \beta \log \frac{\pi_{\theta}(\mathbf{a}_t | \mathbf{s}_t)}{\pi_{ref}(\mathbf{a}_t | \mathbf{s}_t)}]
% % \nonumber
% \\
% &
% =\max_{\pi_\theta} \mathbb{E}_{a_t \sim \pi_\theta(\cdot | \mathbf{s}_t)} [ \sum_{t=0}^{T} ( r(\mathbf{s}_t, \mathbf{a}_t) + \beta \log \pi_{ref}(\mathbf{a}_t | \mathbf{s}_t) ) + \beta \mathcal{H}(\pi_\theta) | \mathbf{s}_0 \sim \rho(\mathbf{s}_0) ]
% % \nonumber
% \label{eq: rlhf_objective}
% \end{align}

As demonstrated in \citet{rafailov2024rqlanguagemodel}, the Bradley-Terry preference model in token-level Marcov Decision Process (MDP) is:

\begin{equation}
p^*\left(\tau^w \succeq \tau^l\right)=\frac{\exp \left(\sum_{i=1}^N r\left(\mathbf{s}_i^w, \mathbf{a}_i^w\right)\right)}{\exp \left(\sum_{i=1}^N r\left(\mathbf{s}_i^w, \mathbf{a}_i^w\right)\right)+\exp \left(\sum_{i=1}^M r\left(\mathbf{s}_i^l, \mathbf{a}_i^l\right)\right)}
\label{eq: tdpo_bt}
\end{equation}

\label{app: tdpo}
The formula using the $Q$-function to measure the relationship between the current timestep and future returns:

% From $r$ to $Q^*$
\begin{equation}
Q^*(s_t, a_t) =
\begin{cases} 
r(s_t, a_t) + \beta \log \pi_{ref}(a_t | s_t) + V^*(s_{t+1}), & \text{if } s_{t+1} \text{ is not terminal} \\
r(s_t, a_t) + \beta \log \pi_{ref}(a_t | s_t), & \text{if } s_{t+1} \text{ is terminal}
\end{cases}
\label{eq: t_return}
\end{equation}

Derive the total reward obtained along the entire trajectory based on the above definitions:
\begin{align}
& \sum_{t=0}^{T-1} r(s_t, a_t)
 = \sum_{t=0}^{T-1} ( Q^*(s_t, a_t) - \beta \log \pi_{\text{ref}}(a_t | s_t) - V^*(s_{t+1}) )
\label{eq: r_sum}
\end{align}

Combining this with the fixed point solution of the optimal policy \cite{Ziebart2010ModelingPA, Levine2018ReinforcementLA}, we can further derive:
\begin{align}
\sum_{t=0}^{T-1} r(s_t, a_t)
& = Q^*(s_0, a_0) - \beta \log \pi_{ref}(a_0 | s_0) 
+ \sum_{t=1}^{T-1} ( Q^*(s_t, a_t) - V^*(s_t) - \beta \log \pi_{\text{ref}}(a_t | s_t) )
\\
& = Q^*(s_0, a_0) - \beta \log \pi_{ref}(a_0 | s_0) + \sum_{t=1}^{T-1} \beta \log \frac{\pi^*(a_t | s_t)}{\pi_{\text{ref}}(a_t | s_t)}
% \nonumber
\\
& = V^*(s_0) + \sum_{t=0}^{T-1} \beta \log \frac{\pi^*(a_t | s_t)}{\pi_{\text{ref}}(a_t | s_t)}
% \nonumber
\end{align}

By substituting the above result into Eq. \ref{eq: tdpo_bt}, we can eliminate $V^*(S_0)$ in the same way as removing the partition function in DPO, obtaining the Token-level BT model that conforms to the MDP:
% By substituting the above result into equation \ref{eq: tdpo_bt}, we can obtain the Token-level BT model that conforms to the Markov Decision Process:

\begin{equation}
p_{\pi^*}\left(\tau^w \succeq \tau^l\right)=\sigma\left(\sum_{t=0}^{N-1} \beta \log \frac{\pi^*\left(\mathbf{a}_t^w \mid \mathbf{s}_t^w\right)}{\pi_{\mathrm{ref}}\left(\mathbf{a}_t^w \mid \mathbf{s}_t^w\right)}-\sum_{t=0}^{M-1} \beta \log \frac{\pi^*\left(\mathbf{a}_t^l \mid \mathbf{s}_t^l\right)}{\pi_{\mathrm{ref}}\left(\mathbf{a}_t^l \mid \mathbf{s}_t^l\right)}\right)
\end{equation}

Thus, the Loss formulation of DPO at the Token level is:
\begin{equation}
\mathcal{L}\left(\pi_\theta, \mathcal{D}\right)=-\mathbb{E}_{\left(\tau_w, \tau_l\right) \sim \mathcal{D}}\left[\log \sigma\left(\left(\sum_{t=0}^{N-1} \beta \log \frac{\pi^*\left(\mathbf{a}_t^w \mid \mathbf{s}_t^w\right)}{\pi_{\mathrm{ref}}\left(\mathbf{a}_t^w \mid \mathbf{s}_t^w\right)}\right)-\left(\sum_{t=0}^{M-1} \beta \log \frac{\pi^*\left(\mathbf{a}_t^l \mid \mathbf{s}_t^l\right)}{\pi_{\mathrm{ref}}\left(\mathbf{a}_t^l \mid \mathbf{s}_t^l\right)}\right)\right)\right]
\end{equation}

\subsection{Proof of Dynamic Token Weight in Token-level DPO}
\label{app: change_beta}

In classic RLHF methods, the optimization objective is typically formulated with an entropy bonus, expressed through a Kullback-Leibler (KL) divergence constraint as follows:

\begin{align}
&
\max_{\pi_\theta} \mathbb{E}_{a_t \sim \pi_\theta(\cdot | \mathbf{s}_t)} \sum_{t=0}^{T} [r(\mathbf{s}_t, \mathbf{a}_t) - \beta \mathcal{D}_{KL}[\pi_{\theta}(\mathbf{a}_t | \mathbf{s}_t)||\pi_{ref}(\mathbf{a}_t | \mathbf{s}_t)]]
% \label{eq: rlhf_obj}
\\
&
=\max_{\pi_\theta} \mathbb{E}_{a_t \sim \pi_\theta(\cdot | \mathbf{s}_t)} \sum_{t=0}^{T} [r(\mathbf{s}_t, \mathbf{a}_t) - \beta \log \frac{\pi_{\theta}(\mathbf{a}_t | \mathbf{s}_t)}{\pi_{ref}(\mathbf{a}_t | \mathbf{s}_t)}]
% \nonumber
\label{eq: rlhf_objective}
\end{align}

This can be further rewritten by separating the terms involving the reference policy and the entropy of the current policy:

$$\max_{\pi_\theta} \mathbb{E}_{a_t \sim \pi_\theta(\cdot | \mathbf{s}_t)} [ \sum_{t=0}^{T} ( r(\mathbf{s}_t, \mathbf{a}_t) + \beta \log \pi_{ref}(\mathbf{a}_t | \mathbf{s}_t) ) + \beta \mathcal{H}(\pi_\theta) | \mathbf{s}_0 \sim \rho(\mathbf{s}_0) ]$$

When the coefficient $\beta$ is treated as a variable that depends on the timestep $t$ \cite{li20242ddposcalingdirectpreference}, the objective transforms to:

\begin{align}
&
\max_{\pi_\theta} \mathbb{E}_{a_t \sim \pi_\theta(\cdot | \mathbf{s}_t)} \sum_{t=0}^{T} [( r(\mathbf{s}_t, \mathbf{a}_t) + \beta_t \log \pi_{ref}(\mathbf{a}_t | \mathbf{s}_t)) - \beta_t \log \pi_{\theta}(\mathbf{a}_t | \mathbf{s}_t)]
\end{align}

\noindent where $\beta_t$ depends solely on $\mathbf{a}_t$ and $\mathbf{s}_t$. Following the formulation by \citet{Levine2018ReinforcementLA}, the above expression can be recast to incorporate the KL divergence explicitly:

\begin{align}
&
\max_{\pi_\theta} \mathbb{E}_{a_t \sim \pi_\theta(\cdot | \mathbf{s}_t)} \sum_{t=0}^{T} [( r(\mathbf{s}_t, \mathbf{a}_t) + \beta_t \log \pi_{ref}(\mathbf{a}_t | \mathbf{s}_t)) - \beta_t \log \pi_{\theta}(\mathbf{a}_t | \mathbf{s}_t)]
\end{align}

\noindent where the value function  $V(\mathbf{s}_t)$ is defined as:

\begin{align}
V(\mathbf{s}_t) = \beta_t \log \int_{\mathcal{A}} [\exp\frac{r(\mathbf{s}_t, \mathbf{a}_t)}{\beta_t} \pi_{ref}(\mathbf{a}_t | \mathbf{s}_t)] \, d\mathbf{a}_t
\end{align}

When the KL divergence term is minimized—implying that the two distributions are identical—the expectation in Eq. \eqref{eq: rlhf_objective} reaches its maximum value. Therefore, the optimal policy satisfies:

\begin{align}
\pi_\theta(\mathbf{a}_t | \mathbf{s}_t) = \frac{1}{\exp(V(\mathbf{s}_t))} \exp\left(\frac{r(\mathbf{s}_t, \mathbf{a}_t) + \beta_t \log \pi_{ref}(\mathbf{a}_t | \mathbf{s}_t)}{\beta_t}\right)
\end{align}

Based on this relationship, we define the optimal Q-function as:

\begin{equation}
Q^*(s_t, a_t) =
\begin{cases} 
r(s_t, a_t) + \beta_t \log \pi_{ref}(a_t | s_t) + V^*(s_{t+1}), & \text{if } s_{t+1} \text{ is not terminal} \\
r(s_t, a_t) + \beta_t \log \pi_{ref}(a_t | s_t), & \text{if } s_{t+1} \text{ is terminal}
\end{cases}
\label{eq: t_return}
\end{equation}

Consequently, the optimal policy can be expressed as:
% $Q(\mathbf{s}_t, \mathbf{a}_t) = r(\mathbf{s}_t, \mathbf{a}_t) + \beta_t \log \pi_{\text{ref}}(\mathbf{a}_t | \mathbf{s}_t)$, thus we can obtain the solution for the optimal policy:
\begin{align}
\pi_\theta(\mathbf{a}_t | \mathbf{s}_t) = e^{(Q(\mathbf{s}_t, \mathbf{a}_t) - V(\mathbf{s}_t))/\beta_t}
\label{eq: fixed_point_2}
\end{align}

By taking the natural logarithm of both sides, we obtain a log-linear relationship for the optimal policy at the token level, which is expressed with the optimial Q-function:
\begin{align}
\beta_t \log \pi_\theta(\mathbf{a}_t \mid \mathbf{s}_t) = Q_\theta(\mathbf{s}_t, \mathbf{a}_t) - V_\theta(\mathbf{s}_t)
\end{align}


This equation establishes a direct relationship between the scaled log-ratio of the optimal policy to the reference policy and the reward function $r(\mathbf{s}_t, \mathbf{a}_t)$:

\begin{align}
\beta_t \log \frac{\pi^*(\mathbf{a}_t \mid \mathbf{s}_t)}{\pi_{\text{ref}}(\mathbf{a}_t \mid \mathbf{s}_t)} = r(\mathbf{s}_t, \mathbf{a}_t) + V^*(\mathbf{s}_{t+1}) - V^*(\mathbf{s}_t)
\end{align}

Furthermore, following the definition by \citet{rafailov2024rqlanguagemodel}'s definition, two reward functions $r(\mathbf{s}_t, \mathbf{a}_t)$ and $r'(\mathbf{s}_t, \mathbf{a}_t)$ are considered equivalent if there exists a potential function $\Phi(\mathbf{s})$, such that:

\begin{align}
r'(\mathbf{s}_t, \mathbf{a}_t) =r(\mathbf{s}_t, \mathbf{a}_t) + \Phi(\mathbf{s}_{t+1})  - \Phi(\mathbf{s}_{t})
\end{align}

This equivalence implies that the optimal advantage function remains invariant under such transformations of the reward function. Consequently, we derive why the coefficient $beta$ in direct preference optimization can be variable, depending on the state and action, thereby allowing for more flexible and adaptive policy optimization in RLHF frameworks.

% In the most classic RLHF methods, the optimization goal is typically expressed as an entropy bonus using the following KL-constrained:
% \begin{align}
% &
% \max_{\pi_\theta} \mathbb{E}_{a_t \sim \pi_\theta(\cdot | \mathbf{s}_t)} \sum_{t=0}^{T} [r(\mathbf{s}_t, \mathbf{a}_t) - \beta \mathcal{D}_{KL}[\pi_{\theta}(\mathbf{a}_t | \mathbf{s}_t)||\pi_{ref}(\mathbf{a}_t | \mathbf{s}_t)]]
% % \label{eq: rlhf_obj}
% \\
% &
% =\max_{\pi_\theta} \mathbb{E}_{a_t \sim \pi_\theta(\cdot | \mathbf{s}_t)} \sum_{t=0}^{T} [r(\mathbf{s}_t, \mathbf{a}_t) - \beta \log \frac{\pi_{\theta}(\mathbf{a}_t | \mathbf{s}_t)}{\pi_{ref}(\mathbf{a}_t | \mathbf{s}_t)}]
% % \nonumber
% \\
% &
% =\max_{\pi_\theta} \mathbb{E}_{a_t \sim \pi_\theta(\cdot | \mathbf{s}_t)} [ \sum_{t=0}^{T} ( r(\mathbf{s}_t, \mathbf{a}_t) + \beta \log \pi_{ref}(\mathbf{a}_t | \mathbf{s}_t) ) + \beta \mathcal{H}(\pi_\theta) | \mathbf{s}_0 \sim \rho(\mathbf{s}_0) ]
% % \nonumber
% \label{eq: rlhf_objective}
% \end{align}


% When $\beta$ is considered as a variable dependent on $t$, Eq. \ref{eq: rlhf_objective} is transformed into:
% \begin{align}
% &
% \max_{\pi_\theta} \mathbb{E}_{a_t \sim \pi_\theta(\cdot | \mathbf{s}_t)} \sum_{t=0}^{T} [( r(\mathbf{s}_t, \mathbf{a}_t) + \beta_t \log \pi_{ref}(\mathbf{a}_t | \mathbf{s}_t)) - \beta_t \log \pi_{\theta}(\mathbf{a}_t | \mathbf{s}_t)]
% \end{align}

% \noindent where $\beta_t$ depends solely on $\mathbf{a}_t$ and $\mathbf{s}_t$. Then, according to \citet{Levine2018ReinforcementLA}, the above formula can be rewritten in a form that includes the KL divergence:
% \begin{align}
% &
% =\mathbb{E}_{\mathbf{s}_t} [ -\beta_t D_{KL}\left( \pi_\theta(\mathbf{a}_t | \mathbf{s}_t) \bigg\| \frac{1}{\exp(V(\mathbf{s}_t))} \exp\left(\frac{r(\mathbf{s}_t, \mathbf{a}_t) + \beta_t \log \pi_{ref}(\mathbf{a}_t | \mathbf{s}_t)}{\beta_t}\right) \right) + V(\mathbf{s}_t) ]
% \label{eq: rlhf_objective_2}
% \end{align}

% \noindent where $V(\mathbf{s}_t) = \beta_t \log \int_{\mathcal{A}} [\exp\frac{r(\mathbf{s}_t, \mathbf{a}_t)}{\beta_t} \pi_{ref}(\mathbf{a}_t | \mathbf{s}_t)] \, d\mathbf{a}_t$. When the KL divergence term is minimized, meaning the two distributions are the same, the above expectation reaches its maximum value. That is:
% \begin{align}
% \pi_\theta(\mathbf{a}_t | \mathbf{s}_t) = \frac{1}{\exp(V(\mathbf{s}_t))} \exp\left(\frac{r(\mathbf{s}_t, \mathbf{a}_t) + \beta_t \log \pi_{ref}(\mathbf{a}_t | \mathbf{s}_t)}{\beta_t}\right)
% \end{align}

% Based on this, we define that:
% \begin{equation}
% Q^*(s_t, a_t) =
% \begin{cases} 
% r(s_t, a_t) + \beta_t \log \pi_{ref}(a_t | s_t) + V^*(s_{t+1}), & \text{if } s_{t+1} \text{ is not terminal} \\
% r(s_t, a_t) + \beta_t \log \pi_{ref}(a_t | s_t), & \text{if } s_{t+1} \text{ is terminal}
% \end{cases}
% \label{eq: t_return}
% \end{equation}

% Thus we can obtain the solution for the optimal policy:
% % $Q(\mathbf{s}_t, \mathbf{a}_t) = r(\mathbf{s}_t, \mathbf{a}_t) + \beta_t \log \pi_{\text{ref}}(\mathbf{a}_t | \mathbf{s}_t)$, thus we can obtain the solution for the optimal policy:
% \begin{align}
% \pi_\theta(\mathbf{a}_t | \mathbf{s}_t) = e^{(Q(\mathbf{s}_t, \mathbf{a}_t) - V(\mathbf{s}_t))/\beta_t}
% \label{eq: fixed_point_2}
% \end{align}

% By log-linearizing the fixed point solution of the optimal policy at the token level, we obtain:
% \begin{align}
% &
% \beta_t \log \pi_\theta(\mathbf{a}_t \mid \mathbf{s}_t) = Q_\theta(\mathbf{s}_t, \mathbf{a}_t) - V_\theta(\mathbf{s}_t)
% \end{align}

% Then, combining with Eq. \ref{eq: t_return}:
% \begin{align}
% \beta_t \log \frac{\pi^*(\mathbf{a}_t \mid \mathbf{s}_t)}{\pi_{\text{ref}}(\mathbf{a}_t \mid \mathbf{s}_t)} = r(\mathbf{s}_t, \mathbf{a}_t) + V^*(\mathbf{s}_{t+1}) - V^*(\mathbf{s}_t).
% \end{align}

% Thus, we can establish the relationship between $\beta_t \log \frac{\pi^*(\mathbf{a}_t \mid \mathbf{s}_t)}{\pi_{\text{ref}}(\mathbf{a}_t \mid \mathbf{s}_t)}$ and $r(\mathbf{s}_t, \mathbf{a}_t)$. 

% According to \citet{rafailov2024rqlanguagemodel}'s definition, two reward functions $r(\mathbf{s}_t, \mathbf{a}_t)$ and $r'(\mathbf{s}_t, \mathbf{a}_t)$ are equivalent if there exists a potential function $\Phi(\mathbf{s})$, such that $r'(\mathbf{s}_t, \mathbf{a}_t) =r(\mathbf{s}_t, \mathbf{a}_t) + \Phi(\mathbf{s}_{t+1})  - \Phi(\mathbf{s}_{t})$. We can conclude that the optimal advantage function is $\beta_t \log \frac{\pi^*(\mathbf{a}_t \mid \mathbf{s}_t)}{\pi_{\text{ref}}(\mathbf{a}_t \mid \mathbf{s}_t)}$.

\section{Detailed Experiment Results}
\label{sec:app-results}
In this section, we presented detailed experiment results which are omitted in the main body of this paper due to space limitation. The detailed experiment results of different methods on ComplexBench, FollowBench and AlpacaEval2 are presented in Table \ref{tab:complexbench}, \ref{tab:alpaca-eval} and \ref{tab:followbench}. The detailed results for the ablative studies of confidence metrics is presented in Table \ref{tab:detail-confidence}. The detailed results for the ablative studies of confidence metrics is presented in Table \ref{tab:detail-noising}. We also present a case study in Table \ref{tab:case-study}, which visualize the token-level weight derived from calibrated confidence score.


\begin{table*}[ht]
\centering
\resizebox{1.0\textwidth}{!}{
\begin{tabular}{cc|cccc|cccc}
\hline
\multirow{3}{*}{\textbf{Scenario}} & \multirow{3}{*}{\textbf{Method}} & \multicolumn{8}{c}{\textbf{ComplexBench}}                                                                                                         \\
                                   &                                  & \multicolumn{4}{c}{\textbf{Meta-Llama3-8B-Instruct}}                    & \multicolumn{4}{c}{\textbf{Qwen2-7B-Instruct}}                          \\
                                   &                                  & \textbf{Overall} & \textbf{And}   & \textbf{Chain} & \textbf{Selection} & \textbf{Overall} & \textbf{And}   & \textbf{Chain} & \textbf{Selection} \\ \hline
\multicolumn{2}{c|}{baseline}                          & 61.49            & 57.22          & 57.22          & 53.55              & 67.24            & 62.58          & 62.58          & 58.97              \\ \hline
\multirow{6}{*}{SelfInst}          & Self-Reward                      & 62.45            & 58.23          & 58.23          & 54.07              & 66.98            & 63.02          & 63.02          & 57.75              \\
                                   & w/ BSM                           & 64.13            & 58.01          & 58.01          & 56.62              & 67.02            & 62.37          & 62.37          & 57.85              \\
                                   & w/ GPT-4                         & 64.05            & 59.44          & 59.44          & 54.78              & —                & —              & —              & —                  \\ \cline{2-10} 
                                   & Self-Correct                     & 55.91            & 49.85          & 49.85          & 46.91              & 64.41            & 59.59          & 59.59          & 55.04              \\
                                   & ISHEEP                           & 62.67            & 57.79          & 57.79          & 54.63              & 67.32            & 61.95          & 61.95          & 59.64              \\ \cline{2-10} 
                                   & \textbf{MuSC}                    & \textbf{65.98}   & \textbf{63.45} & \textbf{63.45} & \textbf{55.96}     & \textbf{69.39}   & \textbf{65.45} & \textbf{65.45} & \textbf{59.79}     \\ \hline
\multirow{7}{*}{PreInst}           & Self-Reward                      & 62.03            & 56.94          & 56.94          & 53.09              & 66.45            & 61.37          & 61.37          & 57.64              \\
                                   & w/ BSM                           & 64.30            & 57.58          & 57.58          & 56.47              & 67.43            & 62.95          & 62.95          & 58.41              \\
                                   & w/ GPT-4                         & 63.52            & 59.08          & 59.08          & 53.91              & —                & —              & —              & —                  \\ \cline{2-10} 
                                   & Self-Correct                     & 60.79            & 55.65          & 55.65          & 52.02              & 64.32            & 60.16          & 60.16          & 54.63              \\
                                   & ISHEEP                           & 62.92            & 56.37          & 56.37          & 54.83              & 67.13            & 64.45          & 64.45          & 57.54              \\
                                   & SFT                              & 53.93            & 45.77          & 45.77          & 44.09              & 65.89            & 60.16          & 60.16          & 57.39              \\ \cline{2-10} 
                                   & \textbf{MuSC}                    & \textbf{64.73}   & \textbf{59.23} & \textbf{59.23} & \textbf{55.91}     & \textbf{70.00}   & \textbf{66.88} & \textbf{66.88} & \textbf{61.38}     \\ \hline
\end{tabular}}
\label{tab:complexbench}
\caption{Detailed experiment results of different methods on ComplexBench.}
\label{tab:complexbench}
\end{table*}

\begin{table*}[ht]
\centering
\resizebox{0.75\textwidth}{!}{
\begin{tabular}{cc|ccc|ccc}
\hline
\multirow{3}{*}{\textbf{Scenario}} & \multirow{3}{*}{\textbf{Method}} & \multicolumn{6}{c}{\textbf{FollowBench}}                                                               \\
                                   &                                  & \multicolumn{3}{c}{\textbf{Meta-Llama3-8B-Instruct}} & \multicolumn{3}{c}{\textbf{Qwen2-7B-Instruct}}  \\
                                   &                                  & \textbf{HSR}     & \textbf{SSR}     & \textbf{CSL}   & \textbf{HSR}   & \textbf{SSR}   & \textbf{CSL}  \\ \hline
\multicolumn{2}{c|}{baseline}                                         & 62.39            & 73.07            & 2.76           & 59.81          & 71.69          & 2.46          \\ \hline
\multirow{6}{*}{SelfInst}          & Self-Reward                      & 61.20            & 72.22            & 2.56           & 55.36          & 69.71          & 2.34          \\
                                   & w/ BSM                           & 64.30            & 73.84            & 2.80           & 57.83          & 70.53          & 2.41          \\
                                   & w/ GPT-4                         & 62.18            & 73.34            & 2.66           & —              & —              & —             \\ \cline{2-8} 
                                   & Self-Correct                     & 54.38            & 67.19            & 2.02           & 51.98          & 67.89          & 2.16          \\
                                   & ISHEEP                           & 62.77            & 72.86            & 2.52           & 57.01          & 69.88          & 2.36          \\ \cline{2-8} 
                                   & \textbf{MuSC}                    & \textbf{66.71}   & \textbf{74.84}   & \textbf{2.92}  & \textbf{62.60} & \textbf{72.57} & \textbf{2.82} \\ \hline
\multirow{7}{*}{PreInst}           & Self-Reward                      & 60.88            & 72.17            & 2.64           & 56.45          & 70.00          & 2.44          \\
                                   & w/ BSM                           & 63.96            & 73.78            & 2.66           & 58.02          & 70.62          & 2.42          \\
                                   & w/ GPT-4                         & 64.02            & 73.26            & 2.64           & —              & —              & —             \\ \cline{2-8} 
                                   & Self-Correct                     & 60.11            & 70.94            & 2.70           & 49.47          & 66.35          & 1.98          \\
                                   & ISHEEP                           & 63.54            & 73.21            & 2.64           & 55.52          & 69.62          & 2.28          \\
                                   & SFT                              & 50.06            & 66.48            & 2.04           & 47.36          & 64.67          & 1.96          \\ \cline{2-8} 
                                   & \textbf{MuSC}                    & \textbf{66.90}   & \textbf{75.11}   & \textbf{2.99}  & \textbf{62.73} & \textbf{73.09} & \textbf{2.86} \\ \hline
\end{tabular}}
\caption{Detailed experiment results of different methods on FollowBench.}
\label{tab:followbench}
\end{table*}

\begin{table*}[ht]
\centering
\resizebox{0.9\textwidth}{!}{
\begin{tabular}{cc|cccccc}
\hline
\multirow{3}{*}{\textbf{Scenario}} & \multirow{3}{*}{\textbf{Method}} & \multicolumn{6}{c}{\textbf{AlpacaEval2}}                                                                          \\
                                   &                                  & \multicolumn{3}{c}{\textbf{Meta-Llama3-8B-Instruct}}    & \multicolumn{3}{c}{\textbf{Qwen2-7B-Instruct}}          \\
                                   &                                  & \textbf{LC (\%)} & \textbf{WR (\%)} & \textbf{Avg. Len} & \textbf{LC (\%)} & \textbf{WR (\%)} & \textbf{Avg. Len} \\ \hline
\multicolumn{2}{c|}{baseline}                                         & 21.07            & 18.73            & 1702              & 15.53            & 13.70            & 1688              \\ \hline
\multirow{6}{*}{SelfInst}          & Self-Reward                      & 19.21            & 19.18            & 1824              & 16.81            & 15.66            & 1756              \\
                                   & w/ BSM                           & 19.03            & 18.34            & 1787              & 16.94            & 15.09            & 1710              \\
                                   & w/ GPT-4                         & 19.55            & 18.53            & 1767              & —                & —                & —                 \\ \cline{2-8} 
                                   & Self-Correct                     & 7.97             & 9.34             & 1919              & 14.01            & 10.92            & 1497              \\
                                   & ISHEEP                           & 22.00            & 19.50            & 1707              & 16.99            & 14.04            & 1619              \\ \cline{2-8} 
                                   & \textbf{MuSC}                    & \textbf{23.87}   & \textbf{20.91}   & \textbf{1708}     & \textbf{20.08}   & \textbf{15.67}   & \textbf{1595}     \\ \hline
\multirow{7}{*}{PreInst}           & Self-Reward                      & 19.93            & 19.04            & 1789              & 15.98            & 15.62            & 1796              \\
                                   & w/ BSM                           & 20.98            & 20.75            & 1829              & 17.17            & 16.21            & 1764              \\
                                   & w/ GPT-4                         & 18.02            & 17.74            & 1804              & —                & —                & —                 \\ \cline{2-8} 
                                   & Self-Correct                     & 6.20             & 5.81             & 1593              & 14.46            & 14.02            & 1737              \\
                                   & ISHEEP                           & 20.23            & 17.86            & 1703              & 16.52            & 13.36            & 1627              \\
                                   & SFT                              & 10.00            & 6.22             & 1079              & 9.52             & 5.25             & 979               \\ \cline{2-8} 
                                   & \textbf{MuSC}                    & \textbf{23.74}   & \textbf{19.53}   & \textbf{1631}     & \textbf{20.29}   & \textbf{15.91}   & \textbf{1613}     \\ \hline
\end{tabular}}
\caption{Detailed experiment results of different methods on AlpacaEval2.}
\label{tab:alpaca-eval}
\end{table*}

\begin{table}[ht]
\centering
\resizebox{0.95\textwidth}{!}{
\begin{tabular}{cc|ccccc|ccccc}
\toprule
\multirow{3}{*}{\textbf{Scenario}} & \multirow{3}{*}{\textbf{Method}} & \multicolumn{5}{c|}{\textbf{Meta-Llama-3-8B-Instruct}}                                    & \multicolumn{5}{c}{\textbf{Qwen-2-7B-Instruct}}                                          \\
                                   &                                  & \multicolumn{3}{c}{\textbf{CF-Bench}}         & \multicolumn{2}{c|}{\textbf{AlpacaEval2}} & \multicolumn{3}{c}{\textbf{CF-Bench}}         & \multicolumn{2}{c}{\textbf{AlpacaEval2}} \\
                                   &                                  & \textbf{CSR}  & \textbf{ISR}  & \textbf{PSR}  & \textbf{LC (\%)}   & \textbf{Avg. Len}       & \textbf{CSR}  & \textbf{ISR}  & \textbf{PSR}  & \textbf{LC (\%)}   & \textbf{Avg. Len}      \\ \midrule
\multirow{6}{*}{PreInst}           & Baseline                         & 0.64          & 0.24          & 0.34          & 21.07                & 1702               & 0.74          & 0.36          & 0.49          & 15.53                & 1688              \\ \cline{2-12} 
                                   % & MuSC w/o conf                  & 0.70          & 0.30          & 0.41          & 21.19                & 1703               & 0.79          & 0.44          & 0.56          & 18.91                & 1604              \\ \cline{2-12} 
                                   & w/ perplexity                    & 0.70          & 0.32          & 0.43          & 22.99                & 1744               & 0.79          & 0.43          & 0.54          & 19.31                & 1675              \\
                                   & w/ PMI                           & 0.69          & 0.29          & 0.41          & 21.92                & 1713               & 0.78          & 0.43          & 0.55          & 17.42                & 1651              \\
                                   & w/ KLDiv                         & 0.69          & 0.31          & 0.42          & 21.86                & 1686               & 0.78          & 0.42          & 0.54          & 18.73                & 1614              \\
                                   & \textbf{w/ entropy}                       & \textbf{0.71} & \textbf{0.34} & \textbf{0.44} & \textbf{23.74}       & 1631               & \textbf{0.80} & \textbf{0.46} & \textbf{0.58} & \textbf{20.38}       & 1683              \\ \bottomrule
\end{tabular}}
\caption{Detailed experiment results of different confidence metrics as the fine-grained weight.}
\label{tab:detail-confidence}
\end{table}

\setlength{\fboxsep}{0pt}

\begin{table}[t]
\centering
\resizebox{1.0\textwidth}{!}{
\begin{tabular}{l}
\hline
\textbf{Instruction 1}: Identify the incorrect word in the given sentence: "The group acquired the skill to ski\\ in a suitable manner." Rewrite the identified incorrect word correctly, ensuring the new sentence\\ adheres to proper capitalization, punctuation, and grammar rules. Include at least one synonym of\\ the incorrect word to test the model's understanding of context and vocabulary, and ensure the synon-\\ ym is used appropriately in the context of the sentence. \\\hline
\textbf{Chosen Response 1}: \colorbox{red!38}{\strut \:The}\colorbox{red!30}{\strut \:group}\colorbox{red!12}{\strut \:acquired}\colorbox{red!50}{\strut \:the}\colorbox{red!25}{\strut \:skill}\colorbox{red!50}{\strut \:to}\colorbox{red!38}{\strut \:ski}\colorbox{red!50}{\strut \:in}\colorbox{red!50}{\strut \:a}\colorbox{red!0}{\strut \:proficient}\colorbox{red!25}{\strut \:manner}\colorbox{red!50}{\strut \:.}\colorbox{red!35}{\strut \:The}\colorbox{red!12}{\strut \:incorrect}\colorbox{red!25}{\strut \:word}\\\colorbox{red!50}{\strut \:is}\colorbox{red!38}{\strut \:"suitable"}\colorbox{red!50}{\strut \:.}\colorbox{red!45}{\strut \:A}\colorbox{red!18}{\strut \:synonym}\colorbox{red!50}{\strut \:for}\colorbox{red!38}{\strut \:"suitable"}\colorbox{red!50}{\strut \:is}\colorbox{red!5}{\strut \:"appropriate"}\colorbox{red!50}{\strut \:,}\colorbox{red!40}{\strut \:but}\colorbox{red!50}{\strut \:it}\colorbox{red!25}{\strut \:doesn't}\colorbox{red!10}{\strut \:fit}\colorbox{red!50}{\strut \:the}\colorbox{red!32}{\strut \:context}\colorbox{red!50}{\strut \:of}\colorbox{red!50}{\strut \:the}\colorbox{red!28}{\strut \:sentence}\colorbox{red!50}{\strut \:.}\\\colorbox{red!8}{\strut \:"Proficient"}\colorbox{red!50}{\strut \:is}\colorbox{red!50}{\strut \:a}\colorbox{red!18}{\strut \:better}\colorbox{red!20}{\strut \:synonym}\colorbox{red!50}{\strut \:in}\colorbox{red!45}{\strut \:this}\colorbox{red!25}{\strut \:context}\colorbox{red!50}{\strut \:as}\colorbox{red!45}{\strut \:it}\colorbox{red!50}{\strut \:means}\colorbox{red!5}{\strut \:skilled}\colorbox{red!50}{\strut \:or}\colorbox{red!32}{\strut \:expert}\colorbox{red!50}{\strut \:.} \\\hline

\textbf{Rejected Response 1}: \colorbox{red!14}{\strut \:The}\colorbox{red!25}{\strut \:word}\colorbox{red!43}{\strut \:"acquired"}\colorbox{red!32}{\strut \:seems}\colorbox{red!36}{\strut \:incorrect}\colorbox{red!0}{\strut \:in}\colorbox{red!0}{\strut \:the}\colorbox{red!27}{\strut \:given}\colorbox{red!23}{\strut \:sentence}\colorbox{red!0}{\strut \:.}\colorbox{red!0}{\strut \:A}\colorbox{red!39}{\strut \:better}\colorbox{red!20}{\strut \:word}\colorbox{red!0}{\strut \:to}\\\colorbox{red!16}{\strut \:use}\colorbox{red!0}{\strut \:in}\colorbox{red!0}{\strut \:this}\colorbox{red!27}{\strut \:context}\colorbox{red!23}{\strut \:would}\colorbox{red!0}{\strut \:be} \colorbox{red!50}{\strut \:"learned"}\colorbox{red!0}{\strut \:.}\colorbox{red!18}{\strut \:Rewritten}\colorbox{red!20}{\strut \:sentence}\colorbox{red!0}{\strut \::}\colorbox{red!9}{\strut \:"The}\colorbox{red!23}{\strut \:group}\colorbox{red!30}{\strut \:learned}\colorbox{red!0}{\strut \:the}\colorbox{red!25}{\strut \:skill}\colorbox{red!0}{\strut \:to}\colorbox{red!9}{\strut \:ski}\colorbox{red!0}{\strut \:in}\colorbox{red!0}{\strut \:a}\\\colorbox{red!36}{\strut \:suitable}\colorbox{red!23}{\strut \:manner}\colorbox{red!0}{\strut \:."} \\\hline

% Create a compelling tagline that effectively highlights the superior efficiency and optimized performance of a product, service, or process. The tagline should communicate the key benefits of streamlined workflows and cost-effective solutions. It should be catchy and memorable, enhancing brand recognition and fostering positive associations with the company's commitment to innovation and excellence. Ensure that the tagline resonates with the target audience and is concise and clear, avoiding unnecessary complexity or jargon. \\\hline

% \colorbox{red!26}{\strut \:Effortlessly}\colorbox{red!32}{\strut \:Optimized}\colorbox{red!50}{\strut \::}\colorbox{red!0}{\strut \:Streamlining}\colorbox{red!15}{\strut \:Efficiency}\colorbox{red!50}{\strut \:,}\colorbox{red!9}{\strut \:Maximizing}\colorbox{red!21}{\strut \:Value} \\\hline

% \colorbox{red!50}{\strut \:Unleash}\colorbox{red!42}{\strut \:Unmatched}\colorbox{red!25}{\strut \:Efficiency}\colorbox{red!0}{\strut \::}\colorbox{red!44}{\strut \:Experience}\colorbox{red!0}{\strut \:the}\colorbox{red!39}{\strut \:Power}\colorbox{red!0}{\strut \:of}\colorbox{red!39}{\strut \:Streamlined}\colorbox{red!28}{\strut \:Solutions} \\\hline

\textbf{Instruction 2}: Rewrite the given sentence in C\# syntax, using an idiom or metaphor to express dis-\\ belief, and ensure the sentence maintains its original meaning while being grammatically correct in\\ C\#. The sentence to be rewritten is: "The idea that pigs can fly seems more plausible than his expla-\\nation for being late."\\\hline

\textbf{Chosen Response 2}:\colorbox{red!50}{\strut \:csharp\textbackslash n}\colorbox{red!50}{\strut \:using}\colorbox{red!50}{\strut \:System}\colorbox{red!50}{\strut \:;\textbackslash n}\colorbox{red!50}{\strut \:class}\colorbox{red!44}{\strut \:Program\textbackslash n}\colorbox{red!50}{\strut \:\{\textbackslash n}\colorbox{red!50}{\strut \:static}\colorbox{red!50}{\strut \:void}\colorbox{red!44}{\strut \:Main}\colorbox{red!50}{\strut \:(}\colorbox{red!50}{\strut \:string}\colorbox{red!39}{\strut \:[]}\colorbox{red!50}{\strut \:args}\\\colorbox{red!50}{\strut \:\textbackslash n)}\colorbox{red!44}{\strut \:\{\textbackslash n}\colorbox{red!50}{\strut \:Console.WriteLine}\colorbox{red!50}{\strut \:(}\colorbox{red!36}{\strut \:"His}\colorbox{red!19}{\strut \:explanation}\colorbox{red!50}{\strut \:for}\colorbox{red!50}{\strut \:being}\colorbox{red!36}{\strut \:late}\colorbox{red!50}{\strut \:is}\colorbox{red!50}{\strut \:so}\colorbox{red!22}{\strut \:baffling}\colorbox{red!50}{\strut \:that}\colorbox{red!31}{\strut \:it's}\colorbox{red!0}{\strut \:akin}\colorbox{red!50}{\strut \:to}\colorbox{red!31}{\strut \:saying}\colorbox{red!39}{\strut \:pigs}\\\colorbox{red!50}{\strut \:can}\colorbox{red!36}{\strut \:fly}\colorbox{red!50}{\strut \:,}\colorbox{red!19}{\strut \:making}\colorbox{red!50}{\strut \:it}\colorbox{red!33}{\strut \:more}\colorbox{red!8}{\strut \:believable}\colorbox{red!50}{\strut \:in}\colorbox{red!25}{\strut \:comparison}\colorbox{red!44}{\strut \:.}\colorbox{red!50}{\strut \:)}\colorbox{red!50}{\strut \:;\textbackslash n}\colorbox{red!50}{\strut \:\}\textbackslash n}\colorbox{red!50}{\strut \:\}} \\\hline
\textbf{Rejected Response 2}:\colorbox{red!50}{\strut \:Wow}\colorbox{red!0}{\strut \:,}\colorbox{red!15}{\strut \:it's}\colorbox{red!30}{\strut \:truly}\colorbox{red!39}{\strut \:hard}\colorbox{red!0}{\strut \:to}\colorbox{red!20}{\strut \:believe}\colorbox{red!0}{\strut \:that}\colorbox{red!9}{\strut \:pigs}\colorbox{red!33}{\strut \:flying}\colorbox{red!20}{\strut \:could}\colorbox{red!0}{\strut \:be}\colorbox{red!11}{\strut \:more}\colorbox{red!39}{\strut \:plausible}\colorbox{red!0}{\strut \:than}\\\colorbox{red!15}{\strut \:my}\colorbox{red!22}{\strut \:explanation}\colorbox{red!4}{\strut \:for}\colorbox{red!0}{\strut \:being}\colorbox{red!11}{\strut \:late}\colorbox{red!4}{\strut \:!}\\\hline

\end{tabular}}
\caption{Visualization of dynamic weights derived for chosen and rejected responses, based on our proposed calibrated entropy score. We select two samples from the datasets as an illustration.}
\label{tab:case-study}
\end{table}


\begin{figure}[h]
    \centering
    \includegraphics[width=0.8\linewidth]{figures/prompt-decom.png}
    \caption{The prompt template used for instruction decomposition.}
    \label{fig: prompt-decom}
    \vspace{-1mm}
\end{figure}

\begin{figure}[h]
    \centering
    \includegraphics[width=0.8\linewidth]{figures/prompt-recomb.png}
    \caption{The prompt template used for constraint recombination.}
    \label{fig: prompt-recomb}
    \vspace{-1mm}
\end{figure}

\begin{figure}[h]
    \centering
    \includegraphics[width=0.8\linewidth]{figures/prompt-selfinst.png}
    \caption{The prompt template used for self-instruct.}
    \label{fig: prompt-selfinst}
    \vspace{-1mm}
\end{figure}

\begin{figure}[h]
    \centering
    \includegraphics[width=0.8\linewidth]{figures/prompt-sub.png}
    \caption{The prompt template used for constraint substitution.}
    \label{fig: prompt-sub}
    \vspace{-1mm}
\end{figure}

\begin{figure}[h]
    \centering
    \includegraphics[width=0.8\linewidth]{figures/prompt-neg.png}
    \caption{The prompt template used for constraint negation.}
    \label{fig: prompt-neg}
    \vspace{-1mm}
\end{figure}


\begin{figure*}[b]
    \centering
    \includegraphics[width=\textwidth]{figs/variation_operators.pdf}
    \vspace{-20pt}
    \caption{\textbf{Success of crossover vs.~mutation.}
    Early in evolution, random crossover is more successful than random mutation.
    But, after a few generations, mutations that more finely tune good designs were less likely to be deleterious than swapping large components between designs. 
    }
    \vspace{-8pt}
    \label{fig:appendix-variation-operator-stats}
\end{figure*} 


\begin{figure}[b]
    \centering
    % \vspace{-2pt}
    \includegraphics[width=0.5\textwidth]{figs/light-source-distribution.pdf}
    % \includegraphics[width=0.875\columnwidth]{figs/light-source-distribution.pdf}
    \vspace{-8pt}
    \caption{\textbf{Phototaxis training and testing.}
    During pretraining, simult.~co-design, and few-shot finetuning,
    training light source locations (gray circles) were sampled uniformly within 
    a circle centered on the robot's initial position (blue square).
    At every learning step,
    a batch of 8192 randomly positioned lights was sampled,
    and each was paired with a unique, random morphology and random terrain.
    Test light source locations (orange stars) 
    were identical across all methods for fair comparison.
    }
    \vspace{-8pt}
    \label{fig:methods-lightsource-dist}
\end{figure}

\begin{figure}[t]
    \centering
    \includegraphics[width=0.65\textwidth]{figs/dog.jpg}
    % \vspace{-4pt}
    \caption{\textbf{Scaling morphology.}
    The embarrassingly parallel nature of the co-design pipeline
    allows the compute required to simulate 1024 robots with up to 1648 springs 
    (i.e.~a single GPU)
    to be redistributed 
    for a single robot with 1,115,157 springs.
    }
    \vspace{-10pt}
    \label{fig:results-dog}
\end{figure}

\begin{figure*}[t]
    \centering
    \includegraphics[width=\textwidth,keepaspectratio=true]{figs/morpho-grid.jpg}
    \vspace{-16pt}
    \caption{\textbf{Morphological distinctiveness.}
    Robot designs shown are sampled uniformly from each generation's test performance distribution and arranged (left to right, top to bottom) by morphological distinctiveness, defined as the mean pairwise Hamming distance to its peer designs. Performance scores appear below each design. The initial population (\textbf{A}) exhibits diverse morphologies with broad performance variation, serving as the starting point for all methods. After 180 generations, simultaneous co-design (\textbf{B}) yields high-performing but morphologically homogeneous designs. In contrast, both zero-shot evolution at generation 31 (\textbf{C}) and few-shot evolution at generation 6 (\textbf{D}) achieve equal or superior performance while maintaining greater morphological diversity and complexity.
    } 
    \vspace{-8pt}
    \label{fig:results-morpho-grid}
\end{figure*}

\begin{figure*}[t]
    \centering
    \includegraphics[width=\textwidth]{figs/zero-shot-robot-grid.jpg}
    \vspace{-18pt}
    \caption{\textbf{Generalization of pretrained universal controller.} Randomly sampled morphologies from the top 50\% of performers in generation 0 of zero-shot evolution. The universal controller successfully controls these diverse, previously unseen robot designs, demonstrating effective generalization across morphologies.}
    \vspace{-8pt}
    \label{fig:appendix-zero-shot-robot-grid}
\end{figure*}


\end{document}

