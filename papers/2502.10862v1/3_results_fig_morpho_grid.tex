\begin{figure*}[t]
    \centering
    \includegraphics[width=\textwidth,keepaspectratio=true]{figs/morpho-grid.jpg}
    \vspace{-16pt}
    \caption{\textbf{Morphological distinctiveness.}
    Robot designs shown are sampled uniformly from each generation's test performance distribution and arranged (left to right, top to bottom) by morphological distinctiveness, defined as the mean pairwise Hamming distance to its peer designs. Performance scores appear below each design. The initial population (\textbf{A}) exhibits diverse morphologies with broad performance variation, serving as the starting point for all methods. After 180 generations, simultaneous co-design (\textbf{B}) yields high-performing but morphologically homogeneous designs. In contrast, both zero-shot evolution at generation 31 (\textbf{C}) and few-shot evolution at generation 6 (\textbf{D}) achieve equal or superior performance while maintaining greater morphological diversity and complexity.
    } 
    \vspace{-8pt}
    \label{fig:results-morpho-grid}
\end{figure*}