\section{Unsupervised Setting} \label{sec:unlabeled}


\subsection{Approximate Volume Optimality}

Suppose $Y_1,\cdots,Y_n,Y_{n+1}$ are independently drawn from a distribution $P$ on $\mathbb{R}$. The goal is to predict $Y_{n+1}$ based on the first $n$ samples $Y_1,\cdots,Y_n$. To be specific, we would like to construct a data-dependent set $\widehat{C}=\widehat{C}(Y_1,\cdots,Y_n)$ such that 
\begin{equation}
\mathbb{P}(Y_{n+1}\in \widehat{C})\geq 1-\alpha. \label{eq:un-cov}
\end{equation}
Among all data-dependent set that satisfies (\ref{eq:un-cov}), our goal is to find the one with the smallest volume, quantified by the Lebesgue measure $\vol(\widehat{C})=\lambda(\widehat{C})$.
When the distribution $P$ is known, one can directly minimize $\lambda(C)$, subject to $P(C)\geq 1-\alpha$ without even using the data. In particular, when $P\ll \lambda$, an optimal solution is given by the density level set
$$C_{\rm opt}=\left\{\frac{dP}{d\lambda}>t\right\}\cup D,$$
for some $t>0$ and $D$ is some subset of $\{dP/d\lambda =t\}$.

In general, $P$ may not be absolutely continuous and the density need not exist. Nonetheless, we can still define the optimal volume by
$$\opt(P,1-\alpha)=\inf\{\vol(C):P(C)\geq 1-\alpha\}.$$
Note that without any assumption on $P$, the above optimization problem may not have a unique solution. Moreover, it is  possible that the infimum cannot be achieved by any measurable set. Therefore, a natural relaxation is to consider approximate volume optimality. For some $\epsilon\in (0,\alpha)$, a prediction set $\widehat{C}$ is called $\epsilon$-optimal if
\begin{equation}
\vol(\widehat{C})\leq \opt(P,1-\alpha+\epsilon), \label{eq:un-vol}
\end{equation}
either in expectation or with high probability.

The notion of volume optimality defined by (\ref{eq:un-vol}) is quite different from those considered in the literature. A popular quantity that has already been studied is the volume of set difference $\vol\left(\widehat{C}\Delta C_{\rm opt}\right)$ \citep{Lei2013DistributionFreePS,izbicki2020flexible,chernozhukov2021distributional}. However, this much stronger notion requires that the optimal solution $C_{\rm opt}$ must not only exist but also be unique. Usually additional assumptions need to be imposed in the neighborhood of the boundary of $C_{\rm opt}$ in order that the set difference vanishes in the large sample limit. In comparison, the definition (\ref{eq:un-vol}) only requires the volume to be controlled, which can be achieved even if $\widehat{C}$ is not close to $C_{\rm opt}$, or when $C_{\rm opt}$ does not even exist. Indeed, from a practical point of view, any set with coverage and volume control would serve the purpose of valid prediction. Insisting the closeness to a questionable target $C_{\rm opt}$ comes at the cost of unnecessary assumptions on the data generating process.

Another notion considered in the literature is close to our formulation (\ref{eq:un-vol}). Instead of relaxing the coverage probability level from $1-\alpha$ to $1-\alpha+\epsilon$, one can consider the following approximate volume optimality,
\begin{equation}
\vol(\widehat{C})\leq \opt(P,1-\alpha) + \epsilon. \label{eq:additive}
\end{equation}
Results of interval length optimality in the sense of (\ref{eq:additive}) have been studied by \citep{chernozhukov2021distributional,kiyani2024length}. However, the $\epsilon$ in (\ref{eq:additive}) is usually proportional to the scale of the distribution $P$, or may depend on $P$ in some other ways. In comparison, the $\epsilon$ in (\ref{eq:un-vol}) has the unit of probability, and as we will show later, can be made independent of the distribution $P$, which leads to more natural and cleaner theoretical results with fewer assumptions.


\subsection{Impossibility of Distribution-Free Volume Optimality}

It is known that conformal prediction achieves the coverage property (\ref{eq:un-cov}) in a distribution-free sense, meaning that (\ref{eq:un-cov}) holds uniformly for all distributions $P$. One naturally hopes that the approximate volume optimality (\ref{eq:un-vol}) can also be established in a distribution-free way. Perhaps not surprisingly, this goal is too ambitious. The theorem below rigorously proves the impossibility of the task.

\begin{theorem}\label{thm:impossibility}
Consider observations $Y_1$, $Y_2$, $\dots$, $Y_n$, $Y_{n+1}$ sampled $i.i.d.$ from a distribution $P$ on $\R$. 
Suppose $\widehat{C}=\widehat{C}(Y_1,\cdots,Y_n)$ satisfies $\mathbb{P}(Y_{n+1}\in \widehat{C})\geq 1-\alpha$
for all distribution $P$. 
Then, for any $\epsilon \in (0,\alpha)$, there exists some distribution $P$ on $\R$, such that the expected volume of the prediction set is at least
$$\E\vol(\widehat{C})\geq \opt(P,1-\alpha+\epsilon).$$
\end{theorem}

The above impossibility result can be regarded as a consequence of a nonparametric testing lower bound. Consider the following hypothesis testing problem,
\begin{eqnarray*}
H_0:&& P=P_0 \\
H_1:&& P\in\left\{P:\TV(P,P_0)>1-\delta\right\}.
\end{eqnarray*}
It is well known that a testing procedure with both vanishing Type-1 and Type-2 errors does not exist without further constraining the alternative hypothesis, even when $\delta$ is arbitrarily close to $0$ \citep{lecam1960necessary,barron1989uniformly}. In the setting of distribution-free inference with simultaneous coverage and volume guarantees, the coverage property involves the measure $P^{n+1}$, while the expected volume is defined by another measure $P^n\otimes\lambda$. When restricting the support of $P$ to the unit interval $[0,1]$, $\lambda$ becomes the uniform probability, and thus both $P^{n+1}$ and $P^n\otimes\lambda$ are probability distributions. It turns out achieving approximate volume optimality is related to hypothesis testing between $P^{n+1}$ and $P^n\otimes\lambda$ with total variation separation.






\subsection{Distribution-Free Restricted Volume Optimality}

The impossibility result implies a volume lower bound $\opt(P,1-\alpha+\epsilon)$, where the coverage level $1-\alpha+\epsilon$ can be arbitrarily close to $1$. This means that, at least in the worst case, the volume cannot be smaller than that of the support of $P$.

To avoid this triviality, in this section, we consider a weaker notion of volume optimality by only considering prediction sets that are unions of $k$ intervals. We use $\mathcal{C}_k$ to denote the collection of all sets that are unions of $k$ intervals. The restricted optimal volume with respect to the class $\mathcal{C}_k$ is defined by
\begin{equation}
% \opt_k(P,1-\alpha) = \inf\left\{\vol(C):P(C)\geq 1-\alpha, C \in \calC_k \right\}.\label{eq:res-opt}
\opt_k(P, 1-\alpha) = \inf_{C \in \mathcal{C}_k} \left\{\vol(C) : P(C) \geq 1-\alpha \right\}. \label{eq:res-opt}
\end{equation}

\begin{remark}\label{rem:noloss:smooth}
We remark that we are still in a distribution-free setting, since no assumption is imposed on $P$. Instead, the restriction only constrains the shape of the prediction set. 
From a practical point of view, it is reasonable to require that
$\widehat{C}\in \mathcal{C}_k$,
since a more complicated prediction set would be hard to interpret. Moreover, as long as $P$ admits a density function with at most $k$ modes, the two notions match,
$$\opt_k(P,1-\alpha)=\opt(P,1-\alpha).$$
More generally, it can be shown that
$$\opt_k(P,1-\alpha)\leq\opt(P,1-\alpha+\epsilon),$$
for some $\epsilon\in (0,\alpha)$, whenever $P$ can be approximated by a distribution with at most $k$ modes. This, in particular, includes the situation where the density of $P$ can be well estimated by a kernel density estimator. A rigorous statement will be given in Appendix \ref{sec:DPvsKDE}. 
% \anote{put above in remark environment that can be referenced?}
% %\label{rem:noloss:smooth}
\end{remark}

Given the observations $Y_1,\cdots,Y_n$, we define the empirical distribution $\mathbb{P}_n=\frac{1}{n}\sum_{i=1}^n\delta_{Y_i}$. To achieve restricted volume optimality, one can use
\begin{equation}
\widehat{C}=\argmin_{C \in \calC_k}\left\{\vol(C): \mathbb{P}_n(C)\geq 1-\alpha \right\}. \label{eq:erm}
\end{equation}
According to its definition, the prediction set (\ref{eq:erm}) satisfies both $\mathbb{P}_n(\widehat{C})\geq 1-\alpha$ and $\vol(\widehat{C})=\opt_k(\mathbb{P}_n,1-\alpha)$. The coverage and volume guarantees under $P$ can be obtained via
\begin{equation}
\sup_{C \in \calC_k}|\mathbb{P}_n(C)-P(C)| = O_P\left(\sqrt{{\rm VC}(\calC)/n}\right),\label{eq:un-con}
\end{equation}
with ${\rm VC}(\calC)=O(k)$. Therefore, approximate optimality can be achieved by (\ref{eq:erm}) whenever (\ref{eq:un-con}) holds.

A naive exhaustive search to find (\ref{eq:erm}) requires exponential computational time. We show that an efficient dynamic programming algorithm (Algorithm \ref{alg:dp}) can solve (\ref{eq:erm}) approximately with some additional slack $\gamma$, which determines the computational complexity. Theoretical guarantees of Algorithm \ref{alg:dp} are given in the following proposition.
%The dynamic programming table $DP(i,j,l)$ stores the minimum volume of $i$ intervals that cover $l \gamma n$ points in $Y_{(1)}, \dots, Y_{(j)}$ and the right endpoint of the rightmost interval is at $Y_{(j)}$, where $Y_{(1)}, \dots, Y_{(n)}$ are training data points $Y_1,\dots, Y_n$ sorted in non-decreasing order.
%For each state in DP table, we enumerate all possible left endpoint of the rightmost interval and the right endpoint of the previous interval (if it exists). 
%Then, we use the standard backtrack approach on the DP table to find the prediction set $\widehat C_{DP}$.

\documentclass[conference]{IEEEtran}
\usepackage{times}
\usepackage[T1]{fontenc}
\usepackage[utf8]{inputenc}

% numbers option provides compact numerical references in the text. 
\usepackage[numbers]{natbib}
\usepackage{multicol}
% \usepackage[bookmarks=true]{hyperref}
\usepackage[pagebackref=true,breaklinks=true,bookmarks=true,colorlinks]{hyperref}
% basic
%\usepackage{color,xcolor}
\usepackage{color}
\usepackage{epsfig}
\usepackage{graphicx}
\usepackage{algorithm,algorithmic}
% \usepackage{algpseudocode}
%\usepackage{ulem}

% figure and table
\usepackage{adjustbox}
\usepackage{array}
\usepackage{booktabs}
\usepackage{colortbl}
\usepackage{float,wrapfig}
\usepackage{framed}
\usepackage{hhline}
\usepackage{multirow}
% \usepackage{subcaption} % issues a warning with CVPR/ICCV format
% \usepackage[font=small]{caption}
\usepackage[percent]{overpic}
%\usepackage{tikz} % conflict with ECCV format

% font and character
\usepackage{amsmath,amsfonts,amssymb}
% \let\proof\relax      % for ECCV llncs class
% \let\endproof\relax   % for ECCV llncs class
\usepackage{amsthm} 
\usepackage{bm}
\usepackage{nicefrac}
\usepackage{microtype}
\usepackage{contour}
\usepackage{courier}
%\usepackage{palatino}
%\usepackage{times}

% layout
\usepackage{changepage}
\usepackage{extramarks}
\usepackage{fancyhdr}
\usepackage{lastpage}
\usepackage{setspace}
\usepackage{soul}
\usepackage{xspace}
\usepackage{cuted}
\usepackage{fancybox}
\usepackage{afterpage}
%\usepackage{enumitem} % conflict with IEEE format
%\usepackage{titlesec} % conflict with ECCV format

% ref
% commenting these two out for this submission so it looks the same as RSS example
% \usepackage[breaklinks=true,colorlinks,backref=True]{hyperref}
% \hypersetup{colorlinks,linkcolor={black},citecolor={MSBlue},urlcolor={magenta}}
\usepackage{url}
\usepackage{quoting}
\usepackage{epigraph}

% misc
\usepackage{enumerate}
\usepackage{paralist,tabularx}
\usepackage{comment}
\usepackage{pdfpages}
% \usepackage[draft]{todonotes} % conflict with CVPR/ICCV/ECCV format



% \usepackage{todonotes}
% \usepackage{caption}
% \usepackage{subcaption}

\usepackage{pifont}% http://ctan.org/pkg/pifont

% extra symbols
\usepackage{MnSymbol}



%
\setlength\unitlength{1mm}
\newcommand{\twodots}{\mathinner {\ldotp \ldotp}}
% bb font symbols
\newcommand{\Rho}{\mathrm{P}}
\newcommand{\Tau}{\mathrm{T}}

\newfont{\bbb}{msbm10 scaled 700}
\newcommand{\CCC}{\mbox{\bbb C}}

\newfont{\bb}{msbm10 scaled 1100}
\newcommand{\CC}{\mbox{\bb C}}
\newcommand{\PP}{\mbox{\bb P}}
\newcommand{\RR}{\mbox{\bb R}}
\newcommand{\QQ}{\mbox{\bb Q}}
\newcommand{\ZZ}{\mbox{\bb Z}}
\newcommand{\FF}{\mbox{\bb F}}
\newcommand{\GG}{\mbox{\bb G}}
\newcommand{\EE}{\mbox{\bb E}}
\newcommand{\NN}{\mbox{\bb N}}
\newcommand{\KK}{\mbox{\bb K}}
\newcommand{\HH}{\mbox{\bb H}}
\newcommand{\SSS}{\mbox{\bb S}}
\newcommand{\UU}{\mbox{\bb U}}
\newcommand{\VV}{\mbox{\bb V}}


\newcommand{\yy}{\mathbbm{y}}
\newcommand{\xx}{\mathbbm{x}}
\newcommand{\zz}{\mathbbm{z}}
\newcommand{\sss}{\mathbbm{s}}
\newcommand{\rr}{\mathbbm{r}}
\newcommand{\pp}{\mathbbm{p}}
\newcommand{\qq}{\mathbbm{q}}
\newcommand{\ww}{\mathbbm{w}}
\newcommand{\hh}{\mathbbm{h}}
\newcommand{\vvv}{\mathbbm{v}}

% Vectors

\newcommand{\av}{{\bf a}}
\newcommand{\bv}{{\bf b}}
\newcommand{\cv}{{\bf c}}
\newcommand{\dv}{{\bf d}}
\newcommand{\ev}{{\bf e}}
\newcommand{\fv}{{\bf f}}
\newcommand{\gv}{{\bf g}}
\newcommand{\hv}{{\bf h}}
\newcommand{\iv}{{\bf i}}
\newcommand{\jv}{{\bf j}}
\newcommand{\kv}{{\bf k}}
\newcommand{\lv}{{\bf l}}
\newcommand{\mv}{{\bf m}}
\newcommand{\nv}{{\bf n}}
\newcommand{\ov}{{\bf o}}
\newcommand{\pv}{{\bf p}}
\newcommand{\qv}{{\bf q}}
\newcommand{\rv}{{\bf r}}
\newcommand{\sv}{{\bf s}}
\newcommand{\tv}{{\bf t}}
\newcommand{\uv}{{\bf u}}
\newcommand{\wv}{{\bf w}}
\newcommand{\vv}{{\bf v}}
\newcommand{\xv}{{\bf x}}
\newcommand{\yv}{{\bf y}}
\newcommand{\zv}{{\bf z}}
\newcommand{\zerov}{{\bf 0}}
\newcommand{\onev}{{\bf 1}}

% Matrices

\newcommand{\Am}{{\bf A}}
\newcommand{\Bm}{{\bf B}}
\newcommand{\Cm}{{\bf C}}
\newcommand{\Dm}{{\bf D}}
\newcommand{\Em}{{\bf E}}
\newcommand{\Fm}{{\bf F}}
\newcommand{\Gm}{{\bf G}}
\newcommand{\Hm}{{\bf H}}
\newcommand{\Id}{{\bf I}}
\newcommand{\Jm}{{\bf J}}
\newcommand{\Km}{{\bf K}}
\newcommand{\Lm}{{\bf L}}
\newcommand{\Mm}{{\bf M}}
\newcommand{\Nm}{{\bf N}}
\newcommand{\Om}{{\bf O}}
\newcommand{\Pm}{{\bf P}}
\newcommand{\Qm}{{\bf Q}}
\newcommand{\Rm}{{\bf R}}
\newcommand{\Sm}{{\bf S}}
\newcommand{\Tm}{{\bf T}}
\newcommand{\Um}{{\bf U}}
\newcommand{\Wm}{{\bf W}}
\newcommand{\Vm}{{\bf V}}
\newcommand{\Xm}{{\bf X}}
\newcommand{\Ym}{{\bf Y}}
\newcommand{\Zm}{{\bf Z}}

% Calligraphic

\newcommand{\Ac}{{\cal A}}
\newcommand{\Bc}{{\cal B}}
\newcommand{\Cc}{{\cal C}}
\newcommand{\Dc}{{\cal D}}
\newcommand{\Ec}{{\cal E}}
\newcommand{\Fc}{{\cal F}}
\newcommand{\Gc}{{\cal G}}
\newcommand{\Hc}{{\cal H}}
\newcommand{\Ic}{{\cal I}}
\newcommand{\Jc}{{\cal J}}
\newcommand{\Kc}{{\cal K}}
\newcommand{\Lc}{{\cal L}}
\newcommand{\Mc}{{\cal M}}
\newcommand{\Nc}{{\cal N}}
\newcommand{\nc}{{\cal n}}
\newcommand{\Oc}{{\cal O}}
\newcommand{\Pc}{{\cal P}}
\newcommand{\Qc}{{\cal Q}}
\newcommand{\Rc}{{\cal R}}
\newcommand{\Sc}{{\cal S}}
\newcommand{\Tc}{{\cal T}}
\newcommand{\Uc}{{\cal U}}
\newcommand{\Wc}{{\cal W}}
\newcommand{\Vc}{{\cal V}}
\newcommand{\Xc}{{\cal X}}
\newcommand{\Yc}{{\cal Y}}
\newcommand{\Zc}{{\cal Z}}

% Bold greek letters

\newcommand{\alphav}{\hbox{\boldmath$\alpha$}}
\newcommand{\betav}{\hbox{\boldmath$\beta$}}
\newcommand{\gammav}{\hbox{\boldmath$\gamma$}}
\newcommand{\deltav}{\hbox{\boldmath$\delta$}}
\newcommand{\etav}{\hbox{\boldmath$\eta$}}
\newcommand{\lambdav}{\hbox{\boldmath$\lambda$}}
\newcommand{\epsilonv}{\hbox{\boldmath$\epsilon$}}
\newcommand{\nuv}{\hbox{\boldmath$\nu$}}
\newcommand{\muv}{\hbox{\boldmath$\mu$}}
\newcommand{\zetav}{\hbox{\boldmath$\zeta$}}
\newcommand{\phiv}{\hbox{\boldmath$\phi$}}
\newcommand{\psiv}{\hbox{\boldmath$\psi$}}
\newcommand{\thetav}{\hbox{\boldmath$\theta$}}
\newcommand{\tauv}{\hbox{\boldmath$\tau$}}
\newcommand{\omegav}{\hbox{\boldmath$\omega$}}
\newcommand{\xiv}{\hbox{\boldmath$\xi$}}
\newcommand{\sigmav}{\hbox{\boldmath$\sigma$}}
\newcommand{\piv}{\hbox{\boldmath$\pi$}}
\newcommand{\rhov}{\hbox{\boldmath$\rho$}}
\newcommand{\upsilonv}{\hbox{\boldmath$\upsilon$}}

\newcommand{\Gammam}{\hbox{\boldmath$\Gamma$}}
\newcommand{\Lambdam}{\hbox{\boldmath$\Lambda$}}
\newcommand{\Deltam}{\hbox{\boldmath$\Delta$}}
\newcommand{\Sigmam}{\hbox{\boldmath$\Sigma$}}
\newcommand{\Phim}{\hbox{\boldmath$\Phi$}}
\newcommand{\Pim}{\hbox{\boldmath$\Pi$}}
\newcommand{\Psim}{\hbox{\boldmath$\Psi$}}
\newcommand{\Thetam}{\hbox{\boldmath$\Theta$}}
\newcommand{\Omegam}{\hbox{\boldmath$\Omega$}}
\newcommand{\Xim}{\hbox{\boldmath$\Xi$}}


% Sans Serif small case

\newcommand{\Gsf}{{\sf G}}

\newcommand{\asf}{{\sf a}}
\newcommand{\bsf}{{\sf b}}
\newcommand{\csf}{{\sf c}}
\newcommand{\dsf}{{\sf d}}
\newcommand{\esf}{{\sf e}}
\newcommand{\fsf}{{\sf f}}
\newcommand{\gsf}{{\sf g}}
\newcommand{\hsf}{{\sf h}}
\newcommand{\isf}{{\sf i}}
\newcommand{\jsf}{{\sf j}}
\newcommand{\ksf}{{\sf k}}
\newcommand{\lsf}{{\sf l}}
\newcommand{\msf}{{\sf m}}
\newcommand{\nsf}{{\sf n}}
\newcommand{\osf}{{\sf o}}
\newcommand{\psf}{{\sf p}}
\newcommand{\qsf}{{\sf q}}
\newcommand{\rsf}{{\sf r}}
\newcommand{\ssf}{{\sf s}}
\newcommand{\tsf}{{\sf t}}
\newcommand{\usf}{{\sf u}}
\newcommand{\wsf}{{\sf w}}
\newcommand{\vsf}{{\sf v}}
\newcommand{\xsf}{{\sf x}}
\newcommand{\ysf}{{\sf y}}
\newcommand{\zsf}{{\sf z}}


% mixed symbols

\newcommand{\sinc}{{\hbox{sinc}}}
\newcommand{\diag}{{\hbox{diag}}}
\renewcommand{\det}{{\hbox{det}}}
\newcommand{\trace}{{\hbox{tr}}}
\newcommand{\sign}{{\hbox{sign}}}
\renewcommand{\arg}{{\hbox{arg}}}
\newcommand{\var}{{\hbox{var}}}
\newcommand{\cov}{{\hbox{cov}}}
\newcommand{\Ei}{{\rm E}_{\rm i}}
\renewcommand{\Re}{{\rm Re}}
\renewcommand{\Im}{{\rm Im}}
\newcommand{\eqdef}{\stackrel{\Delta}{=}}
\newcommand{\defines}{{\,\,\stackrel{\scriptscriptstyle \bigtriangleup}{=}\,\,}}
\newcommand{\<}{\left\langle}
\renewcommand{\>}{\right\rangle}
\newcommand{\herm}{{\sf H}}
\newcommand{\trasp}{{\sf T}}
\newcommand{\transp}{{\sf T}}
\renewcommand{\vec}{{\rm vec}}
\newcommand{\Psf}{{\sf P}}
\newcommand{\SINR}{{\sf SINR}}
\newcommand{\SNR}{{\sf SNR}}
\newcommand{\MMSE}{{\sf MMSE}}
\newcommand{\REF}{{\RED [REF]}}

% Markov chain
\usepackage{stmaryrd} % for \mkv 
\newcommand{\mkv}{-\!\!\!\!\minuso\!\!\!\!-}

% Colors

\newcommand{\RED}{\color[rgb]{1.00,0.10,0.10}}
\newcommand{\BLUE}{\color[rgb]{0,0,0.90}}
\newcommand{\GREEN}{\color[rgb]{0,0.80,0.20}}

%%%%%%%%%%%%%%%%%%%%%%%%%%%%%%%%%%%%%%%%%%
\usepackage{hyperref}
\hypersetup{
    bookmarks=true,         % show bookmarks bar?
    unicode=false,          % non-Latin characters in AcrobatÕs bookmarks
    pdftoolbar=true,        % show AcrobatÕs toolbar?
    pdfmenubar=true,        % show AcrobatÕs menu?
    pdffitwindow=false,     % window fit to page when opened
    pdfstartview={FitH},    % fits the width of the page to the window
%    pdftitle={My title},    % title
%    pdfauthor={Author},     % author
%    pdfsubject={Subject},   % subject of the document
%    pdfcreator={Creator},   % creator of the document
%    pdfproducer={Producer}, % producer of the document
%    pdfkeywords={keyword1} {key2} {key3}, % list of keywords
    pdfnewwindow=true,      % links in new window
    colorlinks=true,       % false: boxed links; true: colored links
    linkcolor=red,          % color of internal links (change box color with linkbordercolor)
    citecolor=green,        % color of links to bibliography
    filecolor=blue,      % color of file links
    urlcolor=blue           % color of external links
}
%%%%%%%%%%%%%%%%%%%%%%%%%%%%%%%%%%%%%%%%%%%



\pdfinfo{
   /Author (Homer Simpson)
   /Title  (Robots: Our new overlords)
   /CreationDate (D:20101201120000)
   /Subject (Robots)
   /Keywords (Robots;Overlords)
}

\begin{document}

% paper title
\title{Diffusion Policy}
\title{Diffusion Policy: \\ \LARGE{ Visuomotor Policy Learning via Action Diffusion}}
%\title{Diffusion Policy: \\ \LARGE{Learning Visuomotor Policies with Conditional Diffusion Process}}
% You will get a Paper-ID when submitting a pdf file to the conference system
% \author{Author Names Omitted for Anonymous Review. Paper-ID [6]}
\author{
Cheng Chi$^1$, Siyuan Feng$^2$, Yilun Du$^3$, Zhenjia Xu$^1$, Eric Cousineau$^2$, Benjamin Burchfiel$^2$, Shuran Song$^1$ \\ 
$^1$ Columbia University \quad\quad 
$^2$ Toyota Research Institute \quad\quad
$^3$ MIT
\\ \href{https://diffusion-policy.cs.columbia.edu}{https://diffusion-policy.cs.columbia.edu}
}

%\maketitle

% \begin{figure*}[t]
%     \centering
%     \includegraphics[width=\linewidth]{figure/DP_teaser.pdf}
%     \caption{\textbf{Different Form of Policy Representations.} a) Explicit policy with different types of action representations.  b) Implicit policy learns an energy function conditioned on both action and observation and optimizes for actions that minimize the energy landscape c) Diffusion policy learns a gradient field to refine actions. \todo{update the figure with the same example. }  }
%     \label{fig:policy_rep}
%     %https://docs.google.com/drawings/d/1SNd5_khk3RsYuE9JCwUVjmRED-eF3UrO78XnzbOwE4Y/edit?usp=sharing
% \end{figure*}

\twocolumn[{%
	\renewcommand\twocolumn[1][]{#1}%
	\maketitle
        \vspace{-5mm}
	\begin{center}
		\includegraphics[width=0.95\textwidth]{figure/DP_teaser.pdf}
		\captionof{figure}{\textbf{Policy Representations.} \label{fig:policy_rep} a) Explicit policy with different types of action representations.  b) Implicit policy learns an energy function conditioned on both action and observation and optimizes for actions that minimize the energy landscape c) Diffusion policy refines noise into actions via a learned gradient field. This formulation provides stable training, allows the learned policy to accurately model multimodal action distributions, and accommodates high-dimensional action sequences. 
        % \todo{update figure to make b and c consistent, both 2D or both 3D. Make it clear c is gradient of b change J (a) -> E (a)}
        } 
	%https://docs.google.com/drawings/d/1SNd5_khk3RsYuE9JCwUVjmRED-eF3UrO78XnzbOwE4Y/edit?usp=sharing
	\end{center}
}]


\begin{abstract}
%This consistent performance boost provides strong evidence of the Diffusion Policy's effectiveness, which is attributed to its inherited properties of the diffusion formulation. 
%By learning the gradient field of an implicit action score and performing Stochastic Langevin Dynamics sampling on this gradient field, Diffusion policy is able to accurately model the multimodal action distribution, scalable to high-dimensional action space, and achieve stable training.  
This paper introduces Diffusion Policy, a new way of generating robot behavior by representing a robot's visuomotor policy as a conditional denoising diffusion process. We benchmark Diffusion Policy across 12 different tasks from 4 different robot manipulation benchmarks and find that it consistently outperforms existing state-of-the-art robot learning methods with an average improvement of 46.9\%. 
Diffusion Policy learns the gradient of the action-distribution score function and iteratively optimizes with respect to this gradient field during inference via a series of stochastic Langevin dynamics steps.
We find that the diffusion formulation yields powerful advantages when used for robot policies, including gracefully handling multimodal action distributions, being suitable for high-dimensional action spaces, and exhibiting impressive training stability.
To fully unlock the potential of diffusion models for visuomotor policy learning on physical robots, this paper presents a set of key technical contributions including the incorporation of receding horizon control, visual conditioning, and the time-series diffusion transformer. 
We hope this work will help motivate a new generation of policy learning techniques that are able to leverage the powerful generative modeling capabilities of diffusion models. Code, data, and training details will be publicly available.

% that are often designed specifically for those benchmarks.  
%we introduce a new form of policy that represents a robot's visuomotor policy as a ``conditional diffusion denoising process on action'', Diffusion Policy. In this formulation, the policy learns to infer the gradient field of the action score for K denoising iterations conditioned on its visual observation. The output action sequence can be then inferred by performing Stochastic Langevin Dynamics sampling on this learned gradient field. 
% This formulation allows the learned policy to accurately model the multimodal action distribution, scalable to high-dimensional action space, and achieve stable training.
% We systematically evaluate the algorithm over 11 tasks from 4 different benchmarks that range from both simulated and real-world environments, single- and multi-task benchmarks, and fully- and under-actuated systems.
% The consistent performance boost against state-of-the-art methods across all benchmarks, tasks, and scenarios provides strong evidence of the effectiveness of the Diffusion Policy. We also provide detailed analysis to carefully examine the characteristics of the proposed algorithm and the impacts of the key design decisions. 
% The code, data, and training details will be publicly available.
\end{abstract}

\IEEEpeerreviewmaketitle



\section{Introduction}
\label{sec:introduction}
The business processes of organizations are experiencing ever-increasing complexity due to the large amount of data, high number of users, and high-tech devices involved \cite{martin2021pmopportunitieschallenges, beerepoot2023biggestbpmproblems}. This complexity may cause business processes to deviate from normal control flow due to unforeseen and disruptive anomalies \cite{adams2023proceddsriftdetection}. These control-flow anomalies manifest as unknown, skipped, and wrongly-ordered activities in the traces of event logs monitored from the execution of business processes \cite{ko2023adsystematicreview}. For the sake of clarity, let us consider an illustrative example of such anomalies. Figure \ref{FP_ANOMALIES} shows a so-called event log footprint, which captures the control flow relations of four activities of a hypothetical event log. In particular, this footprint captures the control-flow relations between activities \texttt{a}, \texttt{b}, \texttt{c} and \texttt{d}. These are the causal ($\rightarrow$) relation, concurrent ($\parallel$) relation, and other ($\#$) relations such as exclusivity or non-local dependency \cite{aalst2022pmhandbook}. In addition, on the right are six traces, of which five exhibit skipped, wrongly-ordered and unknown control-flow anomalies. For example, $\langle$\texttt{a b d}$\rangle$ has a skipped activity, which is \texttt{c}. Because of this skipped activity, the control-flow relation \texttt{b}$\,\#\,$\texttt{d} is violated, since \texttt{d} directly follows \texttt{b} in the anomalous trace.
\begin{figure}[!t]
\centering
\includegraphics[width=0.9\columnwidth]{images/FP_ANOMALIES.png}
\caption{An example event log footprint with six traces, of which five exhibit control-flow anomalies.}
\label{FP_ANOMALIES}
\end{figure}

\subsection{Control-flow anomaly detection}
Control-flow anomaly detection techniques aim to characterize the normal control flow from event logs and verify whether these deviations occur in new event logs \cite{ko2023adsystematicreview}. To develop control-flow anomaly detection techniques, \revision{process mining} has seen widespread adoption owing to process discovery and \revision{conformance checking}. On the one hand, process discovery is a set of algorithms that encode control-flow relations as a set of model elements and constraints according to a given modeling formalism \cite{aalst2022pmhandbook}; hereafter, we refer to the Petri net, a widespread modeling formalism. On the other hand, \revision{conformance checking} is an explainable set of algorithms that allows linking any deviations with the reference Petri net and providing the fitness measure, namely a measure of how much the Petri net fits the new event log \cite{aalst2022pmhandbook}. Many control-flow anomaly detection techniques based on \revision{conformance checking} (hereafter, \revision{conformance checking}-based techniques) use the fitness measure to determine whether an event log is anomalous \cite{bezerra2009pmad, bezerra2013adlogspais, myers2018icsadpm, pecchia2020applicationfailuresanalysispm}. 

The scientific literature also includes many \revision{conformance checking}-independent techniques for control-flow anomaly detection that combine specific types of trace encodings with machine/deep learning \cite{ko2023adsystematicreview, tavares2023pmtraceencoding}. Whereas these techniques are very effective, their explainability is challenging due to both the type of trace encoding employed and the machine/deep learning model used \cite{rawal2022trustworthyaiadvances,li2023explainablead}. Hence, in the following, we focus on the shortcomings of \revision{conformance checking}-based techniques to investigate whether it is possible to support the development of competitive control-flow anomaly detection techniques while maintaining the explainable nature of \revision{conformance checking}.
\begin{figure}[!t]
\centering
\includegraphics[width=\columnwidth]{images/HIGH_LEVEL_VIEW.png}
\caption{A high-level view of the proposed framework for combining \revision{process mining}-based feature extraction with dimensionality reduction for control-flow anomaly detection.}
\label{HIGH_LEVEL_VIEW}
\end{figure}

\subsection{Shortcomings of \revision{conformance checking}-based techniques}
Unfortunately, the detection effectiveness of \revision{conformance checking}-based techniques is affected by noisy data and low-quality Petri nets, which may be due to human errors in the modeling process or representational bias of process discovery algorithms \cite{bezerra2013adlogspais, pecchia2020applicationfailuresanalysispm, aalst2016pm}. Specifically, on the one hand, noisy data may introduce infrequent and deceptive control-flow relations that may result in inconsistent fitness measures, whereas, on the other hand, checking event logs against a low-quality Petri net could lead to an unreliable distribution of fitness measures. Nonetheless, such Petri nets can still be used as references to obtain insightful information for \revision{process mining}-based feature extraction, supporting the development of competitive and explainable \revision{conformance checking}-based techniques for control-flow anomaly detection despite the problems above. For example, a few works outline that token-based \revision{conformance checking} can be used for \revision{process mining}-based feature extraction to build tabular data and develop effective \revision{conformance checking}-based techniques for control-flow anomaly detection \cite{singh2022lapmsh, debenedictis2023dtadiiot}. However, to the best of our knowledge, the scientific literature lacks a structured proposal for \revision{process mining}-based feature extraction using the state-of-the-art \revision{conformance checking} variant, namely alignment-based \revision{conformance checking}.

\subsection{Contributions}
We propose a novel \revision{process mining}-based feature extraction approach with alignment-based \revision{conformance checking}. This variant aligns the deviating control flow with a reference Petri net; the resulting alignment can be inspected to extract additional statistics such as the number of times a given activity caused mismatches \cite{aalst2022pmhandbook}. We integrate this approach into a flexible and explainable framework for developing techniques for control-flow anomaly detection. The framework combines \revision{process mining}-based feature extraction and dimensionality reduction to handle high-dimensional feature sets, achieve detection effectiveness, and support explainability. Notably, in addition to our proposed \revision{process mining}-based feature extraction approach, the framework allows employing other approaches, enabling a fair comparison of multiple \revision{conformance checking}-based and \revision{conformance checking}-independent techniques for control-flow anomaly detection. Figure \ref{HIGH_LEVEL_VIEW} shows a high-level view of the framework. Business processes are monitored, and event logs obtained from the database of information systems. Subsequently, \revision{process mining}-based feature extraction is applied to these event logs and tabular data input to dimensionality reduction to identify control-flow anomalies. We apply several \revision{conformance checking}-based and \revision{conformance checking}-independent framework techniques to publicly available datasets, simulated data of a case study from railways, and real-world data of a case study from healthcare. We show that the framework techniques implementing our approach outperform the baseline \revision{conformance checking}-based techniques while maintaining the explainable nature of \revision{conformance checking}.

In summary, the contributions of this paper are as follows.
\begin{itemize}
    \item{
        A novel \revision{process mining}-based feature extraction approach to support the development of competitive and explainable \revision{conformance checking}-based techniques for control-flow anomaly detection.
    }
    \item{
        A flexible and explainable framework for developing techniques for control-flow anomaly detection using \revision{process mining}-based feature extraction and dimensionality reduction.
    }
    \item{
        Application to synthetic and real-world datasets of several \revision{conformance checking}-based and \revision{conformance checking}-independent framework techniques, evaluating their detection effectiveness and explainability.
    }
\end{itemize}

The rest of the paper is organized as follows.
\begin{itemize}
    \item Section \ref{sec:related_work} reviews the existing techniques for control-flow anomaly detection, categorizing them into \revision{conformance checking}-based and \revision{conformance checking}-independent techniques.
    \item Section \ref{sec:abccfe} provides the preliminaries of \revision{process mining} to establish the notation used throughout the paper, and delves into the details of the proposed \revision{process mining}-based feature extraction approach with alignment-based \revision{conformance checking}.
    \item Section \ref{sec:framework} describes the framework for developing \revision{conformance checking}-based and \revision{conformance checking}-independent techniques for control-flow anomaly detection that combine \revision{process mining}-based feature extraction and dimensionality reduction.
    \item Section \ref{sec:evaluation} presents the experiments conducted with multiple framework and baseline techniques using data from publicly available datasets and case studies.
    \item Section \ref{sec:conclusions} draws the conclusions and presents future work.
\end{itemize}
\section{Method}\label{sec:method}
\begin{figure}
    \centering
    \includegraphics[width=0.85\textwidth]{imgs/heatmap_acc.pdf}
    \caption{\textbf{Visualization of the proposed periodic Bayesian flow with mean parameter $\mu$ and accumulated accuracy parameter $c$ which corresponds to the entropy/uncertainty}. For $x = 0.3, \beta(1) = 1000$ and $\alpha_i$ defined in \cref{appd:bfn_cir}, this figure plots three colored stochastic parameter trajectories for receiver mean parameter $m$ and accumulated accuracy parameter $c$, superimposed on a log-scale heatmap of the Bayesian flow distribution $p_F(m|x,\senderacc)$ and $p_F(c|x,\senderacc)$. Note the \emph{non-monotonicity} and \emph{non-additive} property of $c$ which could inform the network the entropy of the mean parameter $m$ as a condition and the \emph{periodicity} of $m$. %\jj{Shrink the figures to save space}\hanlin{Do we need to make this figure one-column?}
    }
    \label{fig:vmbf_vis}
    \vskip -0.1in
\end{figure}
% \begin{wrapfigure}{r}{0.5\textwidth}
%     \centering
%     \includegraphics[width=0.49\textwidth]{imgs/heatmap_acc.pdf}
%     \caption{\textbf{Visualization of hyper-torus Bayesian flow based on von Mises Distribution}. For $x = 0.3, \beta(1) = 1000$ and $\alpha_i$ defined in \cref{appd:bfn_cir}, this figure plots three colored stochastic parameter trajectories for receiver mean parameter $m$ and accumulated accuracy parameter $c$, superimposed on a log-scale heatmap of the Bayesian flow distribution $p_F(m|x,\senderacc)$ and $p_F(c|x,\senderacc)$. Note the \emph{non-monotonicity} and \emph{non-additive} property of $c$. \jj{Shrink the figures to save space}}
%     \label{fig:vmbf_vis}
%     \vspace{-30pt}
% \end{wrapfigure}


In this section, we explain the detailed design of CrysBFN tackling theoretical and practical challenges. First, we describe how to derive our new formulation of Bayesian Flow Networks over hyper-torus $\mathbb{T}^{D}$ from scratch. Next, we illustrate the two key differences between \modelname and the original form of BFN: $1)$ a meticulously designed novel base distribution with different Bayesian update rules; and $2)$ different properties over the accuracy scheduling resulted from the periodicity and the new Bayesian update rules. Then, we present in detail the overall framework of \modelname over each manifold of the crystal space (\textit{i.e.} fractional coordinates, lattice vectors, atom types) respecting \textit{periodic E(3) invariance}. 

% In this section, we first demonstrate how to build Bayesian flow on hyper-torus $\mathbb{T}^{D}$ by overcoming theoretical and practical problems to provide a low-noise parameter-space approach to fractional atom coordinate generation. Next, we present how \modelname models each manifold of crystal space respecting \textit{periodic E(3) invariance}. 

\subsection{Periodic Bayesian Flow on Hyper-torus \texorpdfstring{$\mathbb{T}^{D}$}{}} 
For generative modeling of fractional coordinates in crystal, we first construct a periodic Bayesian flow on \texorpdfstring{$\mathbb{T}^{D}$}{} by designing every component of the totally new Bayesian update process which we demonstrate to be distinct from the original Bayesian flow (please see \cref{fig:non_add}). 
 %:) 
 
 The fractional atom coordinate system \citep{jiao2023crystal} inherently distributes over a hyper-torus support $\mathbb{T}^{3\times N}$. Hence, the normal distribution support on $\R$ used in the original \citep{bfn} is not suitable for this scenario. 
% The key problem of generative modeling for crystal is the periodicity of Cartesian atom coordinates $\vX$ requiring:
% \begin{equation}\label{eq:periodcity}
% p(\vA,\vL,\vX)=p(\vA,\vL,\vX+\vec{LK}),\text{where}~\vec{K}=\vec{k}\vec{1}_{1\times N},\forall\vec{k}\in\mathbb{Z}^{3\times1}
% \end{equation}
% However, there does not exist such a distribution supporting on $\R$ to model such property because the integration of such distribution over $\R$ will not be finite and equal to 1. Therefore, the normal distribution used in \citet{bfn} can not meet this condition.

To tackle this problem, the circular distribution~\citep{mardia2009directional} over the finite interval $[-\pi,\pi)$ is a natural choice as the base distribution for deriving the BFN on $\mathbb{T}^D$. 
% one natural choice is to 
% we would like to consider the circular distribution over the finite interval as the base 
% we find that circular distributions \citep{mardia2009directional} defined on a finite interval with lengths of $2\pi$ can be used as the instantiation of input distribution for the BFN on $\mathbb{T}^D$.
Specifically, circular distributions enjoy desirable periodic properties: $1)$ the integration over any interval length of $2\pi$ equals 1; $2)$ the probability distribution function is periodic with period $2\pi$.  Sharing the same intrinsic with fractional coordinates, such periodic property of circular distribution makes it suitable for the instantiation of BFN's input distribution, in parameterizing the belief towards ground truth $\x$ on $\mathbb{T}^D$. 
% \yuxuan{this is very complicated from my perspective.} \hanlin{But this property is exactly beautiful and perfectly fit into the BFN.}

\textbf{von Mises Distribution and its Bayesian Update} We choose von Mises distribution \citep{mardia2009directional} from various circular distributions as the form of input distribution, based on the appealing conjugacy property required in the derivation of the BFN framework.
% to leverage the Bayesian conjugacy property of von Mises distribution which is required by the BFN framework. 
That is, the posterior of a von Mises distribution parameterized likelihood is still in the family of von Mises distributions. The probability density function of von Mises distribution with mean direction parameter $m$ and concentration parameter $c$ (describing the entropy/uncertainty of $m$) is defined as: 
\begin{equation}
f(x|m,c)=vM(x|m,c)=\frac{\exp(c\cos(x-m))}{2\pi I_0(c)}
\end{equation}
where $I_0(c)$ is zeroth order modified Bessel function of the first kind as the normalizing constant. Given the last univariate belief parameterized by von Mises distribution with parameter $\theta_{i-1}=\{m_{i-1},\ c_{i-1}\}$ and the sample $y$ from sender distribution with unknown data sample $x$ and known accuracy $\alpha$ describing the entropy/uncertainty of $y$,  Bayesian update for the receiver is deducted as:
\begin{equation}
 h(\{m_{i-1},c_{i-1}\},y,\alpha)=\{m_i,c_i \}, \text{where}
\end{equation}
\begin{equation}\label{eq:h_m}
m_i=\text{atan2}(\alpha\sin y+c_{i-1}\sin m_{i-1}, {\alpha\cos y+c_{i-1}\cos m_{i-1}})
\end{equation}
\begin{equation}\label{eq:h_c}
c_i =\sqrt{\alpha^2+c_{i-1}^2+2\alpha c_{i-1}\cos(y-m_{i-1})}
\end{equation}
The proof of the above equations can be found in \cref{apdx:bayesian_update_function}. The atan2 function refers to  2-argument arctangent. Independently conducting  Bayesian update for each dimension, we can obtain the Bayesian update distribution by marginalizing $\y$:
\begin{equation}
p_U(\vtheta'|\vtheta,\bold{x};\alpha)=\mathbb{E}_{p_S(\bold{y}|\bold{x};\alpha)}\delta(\vtheta'-h(\vtheta,\bold{y},\alpha))=\mathbb{E}_{vM(\bold{y}|\bold{x},\alpha)}\delta(\vtheta'-h(\vtheta,\bold{y},\alpha))
\end{equation} 
\begin{figure}
    \centering
    \vskip -0.15in
    \includegraphics[width=0.95\linewidth]{imgs/non_add.pdf}
    \caption{An intuitive illustration of non-additive accuracy Bayesian update on the torus. The lengths of arrows represent the uncertainty/entropy of the belief (\emph{e.g.}~$1/\sigma^2$ for Gaussian and $c$ for von Mises). The directions of the arrows represent the believed location (\emph{e.g.}~ $\mu$ for Gaussian and $m$ for von Mises).}
    \label{fig:non_add}
    \vskip -0.15in
\end{figure}
\textbf{Non-additive Accuracy} 
The additive accuracy is a nice property held with the Gaussian-formed sender distribution of the original BFN expressed as:
\begin{align}
\label{eq:standard_id}
    \update(\parsn{}'' \mid \parsn{}, \x; \alpha_a+\alpha_b) = \E_{\update(\parsn{}' \mid \parsn{}, \x; \alpha_a)} \update(\parsn{}'' \mid \parsn{}', \x; \alpha_b)
\end{align}
Such property is mainly derived based on the standard identity of Gaussian variable:
\begin{equation}
X \sim \mathcal{N}\left(\mu_X, \sigma_X^2\right), Y \sim \mathcal{N}\left(\mu_Y, \sigma_Y^2\right) \Longrightarrow X+Y \sim \mathcal{N}\left(\mu_X+\mu_Y, \sigma_X^2+\sigma_Y^2\right)
\end{equation}
The additive accuracy property makes it feasible to derive the Bayesian flow distribution $
p_F(\boldsymbol{\theta} \mid \mathbf{x} ; i)=p_U\left(\boldsymbol{\theta} \mid \boldsymbol{\theta}_0, \mathbf{x}, \sum_{k=1}^{i} \alpha_i \right)
$ for the simulation-free training of \cref{eq:loss_n}.
It should be noted that the standard identity in \cref{eq:standard_id} does not hold in the von Mises distribution. Hence there exists an important difference between the original Bayesian flow defined on Euclidean space and the Bayesian flow of circular data on $\mathbb{T}^D$ based on von Mises distribution. With prior $\btheta = \{\bold{0},\bold{0}\}$, we could formally represent the non-additive accuracy issue as:
% The additive accuracy property implies the fact that the "confidence" for the data sample after observing a series of the noisy samples with accuracy ${\alpha_1, \cdots, \alpha_i}$ could be  as the accuracy sum  which could be  
% Here we 
% Here we emphasize the specific property of BFN based on von Mises distribution.
% Note that 
% \begin{equation}
% \update(\parsn'' \mid \parsn, \x; \alpha_a+\alpha_b) \ne \E_{\update(\parsn' \mid \parsn, \x; \alpha_a)} \update(\parsn'' \mid \parsn', \x; \alpha_b)
% \end{equation}
% \oyyw{please check whether the below equation is better}
% \yuxuan{I fill somehow confusing on what is the update distribution with $\alpha$. }
% \begin{equation}
% \update(\parsn{}'' \mid \parsn{}, \x; \alpha_a+\alpha_b) \ne \E_{\update(\parsn{}' \mid \parsn{}, \x; \alpha_a)} \update(\parsn{}'' \mid \parsn{}', \x; \alpha_b)
% \end{equation}
% We give an intuitive visualization of such difference in \cref{fig:non_add}. The untenability of this property can materialize by considering the following case: with prior $\btheta = \{\bold{0},\bold{0}\}$, check the two-step Bayesian update distribution with $\alpha_a,\alpha_b$ and one-step Bayesian update with $\alpha=\alpha_a+\alpha_b$:
\begin{align}
\label{eq:nonadd}
     &\update(c'' \mid \parsn, \x; \alpha_a+\alpha_b)  = \delta(c-\alpha_a-\alpha_b)
     \ne  \mathbb{E}_{p_U(\parsn' \mid \parsn, \x; \alpha_a)}\update(c'' \mid \parsn', \x; \alpha_b) \nonumber \\&= \mathbb{E}_{vM(\bold{y}_b|\bold{x},\alpha_a)}\mathbb{E}_{vM(\bold{y}_a|\bold{x},\alpha_b)}\delta(c-||[\alpha_a \cos\y_a+\alpha_b\cos \y_b,\alpha_a \sin\y_a+\alpha_b\sin \y_b]^T||_2)
\end{align}
A more intuitive visualization could be found in \cref{fig:non_add}. This fundamental difference between periodic Bayesian flow and that of \citet{bfn} presents both theoretical and practical challenges, which we will explain and address in the following contents.

% This makes constructing Bayesian flow based on von Mises distribution intrinsically different from previous Bayesian flows (\citet{bfn}).

% Thus, we must reformulate the framework of Bayesian flow networks  accordingly. % and do necessary reformulations of BFN. 

% \yuxuan{overall I feel this part is complicated by using the language of update distribution. I would like to suggest simply use bayesian update, to provide intuitive explantion.}\hanlin{See the illustration in \cref{fig:non_add}}

% That introduces a cascade of problems, and we investigate the following issues: $(1)$ Accuracies between sender and receiver are not synchronized and need to be differentiated. $(2)$ There is no tractable Bayesian flow distribution for a one-step sample conditioned on a given time step $i$, and naively simulating the Bayesian flow results in computational overhead. $(3)$ It is difficult to control the entropy of the Bayesian flow. $(4)$ Accuracy is no longer a function of $t$ and becomes a distribution conditioned on $t$, which can be different across dimensions.
%\jj{Edited till here}

\textbf{Entropy Conditioning} As a common practice in generative models~\citep{ddpm,flowmatching,bfn}, timestep $t$ is widely used to distinguish among generation states by feeding the timestep information into the networks. However, this paper shows that for periodic Bayesian flow, the accumulated accuracy $\vc_i$ is more effective than time-based conditioning by informing the network about the entropy and certainty of the states $\parsnt{i}$. This stems from the intrinsic non-additive accuracy which makes the receiver's accumulated accuracy $c$ not bijective function of $t$, but a distribution conditioned on accumulated accuracies $\vc_i$ instead. Therefore, the entropy parameter $\vc$ is taken logarithm and fed into the network to describe the entropy of the input corrupted structure. We verify this consideration in \cref{sec:exp_ablation}. 
% \yuxuan{implement variant. traditionally, the timestep is widely used to distinguish the different states by putting the timestep embedding into the networks. citation of FM, diffusion, BFN. However, we find that conditioned on time in periodic flow could not provide extra benefits. To further boost the performance, we introduce a simple yet effective modification term entropy conditional. This is based on that the accumulated accuracy which represents the current uncertainty or entropy could be a better indicator to distinguish different states. + Describe how you do this. }



\textbf{Reformulations of BFN}. Recall the original update function with Gaussian sender distribution, after receiving noisy samples $\y_1,\y_2,\dots,\y_i$ with accuracies $\senderacc$, the accumulated accuracies of the receiver side could be analytically obtained by the additive property and it is consistent with the sender side.
% Since observing sample $\y$ with $\alpha_i$ can not result in exact accuracy increment $\alpha_i$ for receiver, the accuracies between sender and receiver are not synchronized which need to be differentiated. 
However, as previously mentioned, this does not apply to periodic Bayesian flow, and some of the notations in original BFN~\citep{bfn} need to be adjusted accordingly. We maintain the notations of sender side's one-step accuracy $\alpha$ and added accuracy $\beta$, and alter the notation of receiver's accuracy parameter as $c$, which is needed to be simulated by cascade of Bayesian updates. We emphasize that the receiver's accumulated accuracy $c$ is no longer a function of $t$ (differently from the Gaussian case), and it becomes a distribution conditioned on received accuracies $\senderacc$ from the sender. Therefore, we represent the Bayesian flow distribution of von Mises distribution as $p_F(\btheta|\x;\alpha_1,\alpha_2,\dots,\alpha_i)$. And the original simulation-free training with Bayesian flow distribution is no longer applicable in this scenario.
% Different from previous BFNs where the accumulated accuracy $\rho$ is not explicitly modeled, the accumulated accuracy parameter $c$ (visualized in \cref{fig:vmbf_vis}) needs to be explicitly modeled by feeding it to the network to avoid information loss.
% the randomaccuracy parameter $c$ (visualized in \cref{fig:vmbf_vis}) implies that there exists information in $c$ from the sender just like $m$, meaning that $c$ also should be fed into the network to avoid information loss. 
% We ablate this consideration in  \cref{sec:exp_ablation}. 

\textbf{Fast Sampling from Equivalent Bayesian Flow Distribution} Based on the above reformulations, the Bayesian flow distribution of von Mises distribution is reframed as: 
\begin{equation}\label{eq:flow_frac}
p_F(\btheta_i|\x;\alpha_1,\alpha_2,\dots,\alpha_i)=\E_{\update(\parsnt{1} \mid \parsnt{0}, \x ; \alphat{1})}\dots\E_{\update(\parsn_{i-1} \mid \parsnt{i-2}, \x; \alphat{i-1})} \update(\parsnt{i} | \parsnt{i-1},\x;\alphat{i} )
\end{equation}
Naively sampling from \cref{eq:flow_frac} requires slow auto-regressive iterated simulation, making training unaffordable. Noticing the mathematical properties of \cref{eq:h_m,eq:h_c}, we  transform \cref{eq:flow_frac} to the equivalent form:
\begin{equation}\label{eq:cirflow_equiv}
p_F(\vec{m}_i|\x;\alpha_1,\alpha_2,\dots,\alpha_i)=\E_{vM(\y_1|\x,\alpha_1)\dots vM(\y_i|\x,\alpha_i)} \delta(\vec{m}_i-\text{atan2}(\sum_{j=1}^i \alpha_j \cos \y_j,\sum_{j=1}^i \alpha_j \sin \y_j))
\end{equation}
\begin{equation}\label{eq:cirflow_equiv2}
p_F(\vec{c}_i|\x;\alpha_1,\alpha_2,\dots,\alpha_i)=\E_{vM(\y_1|\x,\alpha_1)\dots vM(\y_i|\x,\alpha_i)}  \delta(\vec{c}_i-||[\sum_{j=1}^i \alpha_j \cos \y_j,\sum_{j=1}^i \alpha_j \sin \y_j]^T||_2)
\end{equation}
which bypasses the computation of intermediate variables and allows pure tensor operations, with negligible computational overhead.
\begin{restatable}{proposition}{cirflowequiv}
The probability density function of Bayesian flow distribution defined by \cref{eq:cirflow_equiv,eq:cirflow_equiv2} is equivalent to the original definition in \cref{eq:flow_frac}. 
\end{restatable}
\textbf{Numerical Determination of Linear Entropy Sender Accuracy Schedule} ~Original BFN designs the accuracy schedule $\beta(t)$ to make the entropy of input distribution linearly decrease. As for crystal generation task, to ensure information coherence between modalities, we choose a sender accuracy schedule $\senderacc$ that makes the receiver's belief entropy $H(t_i)=H(p_I(\cdot|\vtheta_i))=H(p_I(\cdot|\vc_i))$ linearly decrease \emph{w.r.t.} time $t_i$, given the initial and final accuracy parameter $c(0)$ and $c(1)$. Due to the intractability of \cref{eq:vm_entropy}, we first use numerical binary search in $[0,c(1)]$ to determine the receiver's $c(t_i)$ for $i=1,\dots, n$ by solving the equation $H(c(t_i))=(1-t_i)H(c(0))+tH(c(1))$. Next, with $c(t_i)$, we conduct numerical binary search for each $\alpha_i$ in $[0,c(1)]$ by solving the equations $\E_{y\sim vM(x,\alpha_i)}[\sqrt{\alpha_i^2+c_{i-1}^2+2\alpha_i c_{i-1}\cos(y-m_{i-1})}]=c(t_i)$ from $i=1$ to $i=n$ for arbitrarily selected $x\in[-\pi,\pi)$.

After tackling all those issues, we have now arrived at a new BFN architecture for effectively modeling crystals. Such BFN can also be adapted to other type of data located in hyper-torus $\mathbb{T}^{D}$.

\subsection{Equivariant Bayesian Flow for Crystal}
With the above Bayesian flow designed for generative modeling of fractional coordinate $\vF$, we are able to build equivariant Bayesian flow for each modality of crystal. In this section, we first give an overview of the general training and sampling algorithm of \modelname (visualized in \cref{fig:framework}). Then, we describe the details of the Bayesian flow of every modality. The training and sampling algorithm can be found in \cref{alg:train} and \cref{alg:sampling}.

\textbf{Overview} Operating in the parameter space $\bthetaM=\{\bthetaA,\bthetaL,\bthetaF\}$, \modelname generates high-fidelity crystals through a joint BFN sampling process on the parameter of  atom type $\bthetaA$, lattice parameter $\vec{\theta}^L=\{\bmuL,\brhoL\}$, and the parameter of fractional coordinate matrix $\bthetaF=\{\bmF,\bcF\}$. We index the $n$-steps of the generation process in a discrete manner $i$, and denote the corresponding continuous notation $t_i=i/n$ from prior parameter $\thetaM_0$ to a considerably low variance parameter $\thetaM_n$ (\emph{i.e.} large $\vrho^L,\bmF$, and centered $\bthetaA$).

At training time, \modelname samples time $i\sim U\{1,n\}$ and $\bthetaM_{i-1}$ from the Bayesian flow distribution of each modality, serving as the input to the network. The network $\net$ outputs $\net(\parsnt{i-1}^\mathcal{M},t_{i-1})=\net(\parsnt{i-1}^A,\parsnt{i-1}^F,\parsnt{i-1}^L,t_{i-1})$ and conducts gradient descents on loss function \cref{eq:loss_n} for each modality. After proper training, the sender distribution $p_S$ can be approximated by the receiver distribution $p_R$. 

At inference time, from predefined $\thetaM_0$, we conduct transitions from $\thetaM_{i-1}$ to $\thetaM_{i}$ by: $(1)$ sampling $\y_i\sim p_R(\bold{y}|\thetaM_{i-1};t_i,\alpha_i)$ according to network prediction $\predM{i-1}$; and $(2)$ performing Bayesian update $h(\thetaM_{i-1},\y^\calM_{i-1},\alpha_i)$ for each dimension. 

% Alternatively, we complete this transition using the flow-back technique by sampling 
% $\thetaM_{i}$ from Bayesian flow distribution $\flow(\btheta^M_{i}|\predM{i-1};t_{i-1})$. 

% The training objective of $\net$ is to minimize the KL divergence between sender distribution and receiver distribution for every modality as defined in \cref{eq:loss_n} which is equivalent to optimizing the negative variational lower bound $\calL^{VLB}$ as discussed in \cref{sec:preliminaries}. 

%In the following part, we will present the Bayesian flow of each modality in detail.

\textbf{Bayesian Flow of Fractional Coordinate $\vF$}~The distribution of the prior parameter $\bthetaF_0$ is defined as:
\begin{equation}\label{eq:prior_frac}
    p(\bthetaF_0) \defeq \{vM(\vm_0^F|\vec{0}_{3\times N},\vec{0}_{3\times N}),\delta(\vc_0^F-\vec{0}_{3\times N})\} = \{U(\vec{0},\vec{1}),\delta(\vc_0^F-\vec{0}_{3\times N})\}
\end{equation}
Note that this prior distribution of $\vm_0^F$ is uniform over $[\vec{0},\vec{1})$, ensuring the periodic translation invariance property in \cref{De:pi}. The training objective is minimizing the KL divergence between sender and receiver distribution (deduction can be found in \cref{appd:cir_loss}): 
%\oyyw{replace $\vF$ with $\x$?} \hanlin{notations follow Preliminary?}
\begin{align}\label{loss_frac}
\calL_F = n \E_{i \sim \ui{n}, \flow(\parsn{}^F \mid \vF ; \senderacc)} \alpha_i\frac{I_1(\alpha_i)}{I_0(\alpha_i)}(1-\cos(\vF-\predF{i-1}))
\end{align}
where $I_0(x)$ and $I_1(x)$ are the zeroth and the first order of modified Bessel functions. The transition from $\bthetaF_{i-1}$ to $\bthetaF_{i}$ is the Bayesian update distribution based on network prediction:
\begin{equation}\label{eq:transi_frac}
    p(\btheta^F_{i}|\parsnt{i-1}^\calM)=\mathbb{E}_{vM(\bold{y}|\predF{i-1},\alpha_i)}\delta(\btheta^F_{i}-h(\btheta^F_{i-1},\bold{y},\alpha_i))
\end{equation}
\begin{restatable}{proposition}{fracinv}
With $\net_{F}$ as a periodic translation equivariant function namely $\net_F(\parsnt{}^A,w(\parsnt{}^F+\vt),\parsnt{}^L,t)=w(\net_F(\parsnt{}^A,\parsnt{}^F,\parsnt{}^L,t)+\vt), \forall\vt\in\R^3$, the marginal distribution of $p(\vF_n)$ defined by \cref{eq:prior_frac,eq:transi_frac} is periodic translation invariant. 
\end{restatable}
\textbf{Bayesian Flow of Lattice Parameter \texorpdfstring{$\boldsymbol{L}$}{}}   
Noting the lattice parameter $\bm{L}$ located in Euclidean space, we set prior as the parameter of a isotropic multivariate normal distribution $\btheta^L_0\defeq\{\vmu_0^L,\vrho_0^L\}=\{\bm{0}_{3\times3},\bm{1}_{3\times3}\}$
% \begin{equation}\label{eq:lattice_prior}
% \btheta^L_0\defeq\{\vmu_0^L,\vrho_0^L\}=\{\bm{0}_{3\times3},\bm{1}_{3\times3}\}
% \end{equation}
such that the prior distribution of the Markov process on $\vmu^L$ is the Dirac distribution $\delta(\vec{\mu_0}-\vec{0})$ and $\delta(\vec{\rho_0}-\vec{1})$, 
% \begin{equation}
%     p_I^L(\boldsymbol{L}|\btheta_0^L)=\mathcal{N}(\bm{L}|\bm{0},\bm{I})
% \end{equation}
which ensures O(3)-invariance of prior distribution of $\vL$. By Eq. 77 from \citet{bfn}, the Bayesian flow distribution of the lattice parameter $\bm{L}$ is: 
\begin{align}% =p_U(\bmuL|\btheta_0^L,\bm{L},\beta(t))
p_F^L(\bmuL|\bm{L};t) &=\mathcal{N}(\bmuL|\gamma(t)\bm{L},\gamma(t)(1-\gamma(t))\bm{I}) 
\end{align}
where $\gamma(t) = 1 - \sigma_1^{2t}$ and $\sigma_1$ is the predefined hyper-parameter controlling the variance of input distribution at $t=1$ under linear entropy accuracy schedule. The variance parameter $\vrho$ does not need to be modeled and fed to the network, since it is deterministic given the accuracy schedule. After sampling $\bmuL_i$ from $p_F^L$, the training objective is defined as minimizing KL divergence between sender and receiver distribution (based on Eq. 96 in \citet{bfn}):
\begin{align}
\mathcal{L}_{L} = \frac{n}{2}\left(1-\sigma_1^{2/n}\right)\E_{i \sim \ui{n}}\E_{\flow(\bmuL_{i-1} |\vL ; t_{i-1})}  \frac{\left\|\vL -\predL{i-1}\right\|^2}{\sigma_1^{2i/n}},\label{eq:lattice_loss}
\end{align}
where the prediction term $\predL{i-1}$ is the lattice parameter part of network output. After training, the generation process is defined as the Bayesian update distribution given network prediction:
\begin{equation}\label{eq:lattice_sampling}
    p(\bmuL_{i}|\parsnt{i-1}^\calM)=\update^L(\bmuL_{i}|\predL{i-1},\bmuL_{i-1};t_{i-1})
\end{equation}
    

% The final prediction of the lattice parameter is given by $\bmuL_n = \predL{n-1}$.
% \begin{equation}\label{eq:final_lattice}
%     \bmuL_n = \predL{n-1}
% \end{equation}

\begin{restatable}{proposition}{latticeinv}\label{prop:latticeinv}
With $\net_{L}$ as  O(3)-equivariant function namely $\net_L(\parsnt{}^A,\parsnt{}^F,\vQ\parsnt{}^L,t)=\vQ\net_L(\parsnt{}^A,\parsnt{}^F,\parsnt{}^L,t),\forall\vQ^T\vQ=\vI$, the marginal distribution of $p(\bmuL_n)$ defined by \cref{eq:lattice_sampling} is O(3)-invariant. 
\end{restatable}


\textbf{Bayesian Flow of Atom Types \texorpdfstring{$\boldsymbol{A}$}{}} 
Given that atom types are discrete random variables located in a simplex $\calS^K$, the prior parameter of $\boldsymbol{A}$ is the discrete uniform distribution over the vocabulary $\parsnt{0}^A \defeq \frac{1}{K}\vec{1}_{1\times N}$. 
% \begin{align}\label{eq:disc_input_prior}
% \parsnt{0}^A \defeq \frac{1}{K}\vec{1}_{1\times N}
% \end{align}
% \begin{align}
%     (\oh{j}{K})_k \defeq \delta_{j k}, \text{where }\oh{j}{K}\in \R^{K},\oh{\vA}{KD} \defeq \left(\oh{a_1}{K},\dots,\oh{a_N}{K}\right) \in \R^{K\times N}
% \end{align}
With the notation of the projection from the class index $j$ to the length $K$ one-hot vector $ (\oh{j}{K})_k \defeq \delta_{j k}, \text{where }\oh{j}{K}\in \R^{K},\oh{\vA}{KD} \defeq \left(\oh{a_1}{K},\dots,\oh{a_N}{K}\right) \in \R^{K\times N}$, the Bayesian flow distribution of atom types $\vA$ is derived in \citet{bfn}:
\begin{align}
\flow^{A}(\parsn^A \mid \vA; t) &= \E_{\N{\y \mid \beta^A(t)\left(K \oh{\vA}{K\times N} - \vec{1}_{K\times N}\right)}{\beta^A(t) K \vec{I}_{K\times N \times N}}} \delta\left(\parsn^A - \frac{e^{\y}\parsnt{0}^A}{\sum_{k=1}^K e^{\y_k}(\parsnt{0})_{k}^A}\right).
\end{align}
where $\beta^A(t)$ is the predefined accuracy schedule for atom types. Sampling $\btheta_i^A$ from $p_F^A$ as the training signal, the training objective is the $n$-step discrete-time loss for discrete variable \citep{bfn}: 
% \oyyw{can we simplify the next equation? Such as remove $K \times N, K \times N \times N$}
% \begin{align}
% &\calL_A = n\E_{i \sim U\{1,n\},\flow^A(\parsn^A \mid \vA ; t_{i-1}),\N{\y \mid \alphat{i}\left(K \oh{\vA}{KD} - \vec{1}_{K\times N}\right)}{\alphat{i} K \vec{I}_{K\times N \times N}}} \ln \N{\y \mid \alphat{i}\left(K \oh{\vA}{K\times N} - \vec{1}_{K\times N}\right)}{\alphat{i} K \vec{I}_{K\times N \times N}}\nonumber\\
% &\qquad\qquad\qquad-\sum_{d=1}^N \ln \left(\sum_{k=1}^K \out^{(d)}(k \mid \parsn^A; t_{i-1}) \N{\ydd{d} \mid \alphat{i}\left(K\oh{k}{K}- \vec{1}_{K\times N}\right)}{\alphat{i} K \vec{I}_{K\times N \times N}}\right)\label{discdisc_t_loss_exp}
% \end{align}
\begin{align}
&\calL_A = n\E_{i \sim U\{1,n\},\flow^A(\parsn^A \mid \vA ; t_{i-1}),\N{\y \mid \alphat{i}\left(K \oh{\vA}{KD} - \vec{1}\right)}{\alphat{i} K \vec{I}}} \ln \N{\y \mid \alphat{i}\left(K \oh{\vA}{K\times N} - \vec{1}\right)}{\alphat{i} K \vec{I}}\nonumber\\
&\qquad\qquad\qquad-\sum_{d=1}^N \ln \left(\sum_{k=1}^K \out^{(d)}(k \mid \parsn^A; t_{i-1}) \N{\ydd{d} \mid \alphat{i}\left(K\oh{k}{K}- \vec{1}\right)}{\alphat{i} K \vec{I}}\right)\label{discdisc_t_loss_exp}
\end{align}
where $\vec{I}\in \R^{K\times N \times N}$ and $\vec{1}\in\R^{K\times D}$. When sampling, the transition from $\bthetaA_{i-1}$ to $\bthetaA_{i}$ is derived as:
\begin{equation}
    p(\btheta^A_{i}|\parsnt{i-1}^\calM)=\update^A(\btheta^A_{i}|\btheta^A_{i-1},\predA{i-1};t_{i-1})
\end{equation}

The detailed training and sampling algorithm could be found in \cref{alg:train} and \cref{alg:sampling}.




\input{text/evaluation_v2.tex}
\section{Related Work}
The landscape of large language model vulnerabilities has been extensively studied in recent literature \cite{crothers2023machinegeneratedtextcomprehensive,shayegani2023surveyvulnerabilitieslargelanguage,Yao_2024,Huang2023ASO}, that propose detailed taxonomies of threats. These works categorize LLM attacks into distinct types, such as adversarial attacks, data poisoning, and specific vulnerabilities related to prompt engineering. Among these, prompt injection attacks have emerged as a significant and distinct category, underscoring their relevance to LLM security.

The following high-level overview of the collected taxonomy of LLM vulnerabilities is defined in \cite{Yao_2024}:
\begin{itemize}
    \item Adversarial Attacks: Data Poisoning, Backdoor Attacks
    \item Inference Attacks: Attribute Inference, Membership Inferences
    \item Extraction Attacks
    \item Bias and Unfairness
Exploitation
    \item Instruction Tuning Attacks: Jailbreaking, Prompt Injection.
\end{itemize}
Prompt injection attacks are further classified in \cite{shayegani2023surveyvulnerabilitieslargelanguage} into the following: Goal hijacking and \textbf{Prompt leakage}.

The reviewed taxonomies underscore the need for comprehensive frameworks to evaluate LLM security. The agentic approach introduced in this paper builds on these insights, automating adversarial testing to address a wide range of scenarios, including those involving prompt leakage and role-specific vulnerabilities.

\subsection{Prompt Injection and Prompt Leakage}

Prompt injection attacks exploit the blending of instructional and data inputs, manipulating LLMs into deviating from their intended behavior. Prompt injection attacks encompass techniques that override initial instructions, expose private prompts, or generate malicious outputs \cite{Huang2023ASO}. A subset of these attacks, known as prompt leakage, aims specifically at extracting sensitive system prompts embedded within LLM configurations. In \cite{shayegani2023surveyvulnerabilitieslargelanguage}, authors differentiate between prompt leakage and related methods such as goal hijacking, further refining the taxonomy of LLM-specific vulnerabilities.

\subsection{Defense Mechanisms}

Various defense mechanisms have been proposed to address LLM vulnerabilities, particularly prompt injection and leakage \cite{shayegani2023surveyvulnerabilitieslargelanguage,Yao_2024}. We focused on cost-effective methods like instruction postprocessing and prompt engineering, which are viable for proprietary models that cannot be retrained. Instruction preprocessing sanitizes inputs, while postprocessing removes harmful outputs, forming a dual-layer defense. Preprocessing methods include perplexity-based filtering \cite{Jain2023BaselineDF,Xu2022ExploringTU} and token-level analysis \cite{Kumar2023CertifyingLS}. Postprocessing employs another set of techniques, such as censorship by LLMs \cite{Helbling2023LLMSD,Inan2023LlamaGL}, and use of canary tokens and pattern matching \cite{vigil-llm,rebuff}, although their fundamental limitations are noted \cite{Glukhov2023LLMCA}. Prompt engineering employs carefully designed instructions \cite{Schulhoff2024ThePR} and advanced techniques like spotlighting \cite{Hines2024DefendingAI} to mitigate vulnerabilities, though no method is foolproof \cite{schulhoff-etal-2023-ignore}. Adversarial training, by incorporating adversarial examples into the training process, strengthens models against attacks \cite{Bespalov2024TowardsBA,Shaham2015UnderstandingAT}.

\subsection{Security Testing for Prompt Injection Attacks}

Manual testing, such as red teaming \cite{ganguli2022redteaminglanguagemodels} and handcrafted "Ignore Previous Prompt" attacks \cite{Perez2022IgnorePP}, highlights vulnerabilities but is limited in scale. Automated approaches like PAIR \cite{chao2024jailbreakingblackboxlarge} and GPTFUZZER \cite{Yu2023GPTFUZZERRT} achieve higher success rates by refining prompts iteratively or via automated fuzzing. Red teaming with LLMs \cite{Perez2022RedTL} and reinforcement learning \cite{anonymous2024diverse} uncovers diverse vulnerabilities, including data leakage and offensive outputs. Indirect Prompt Injection (IPI) manipulates external data to compromise applications \cite{Greshake2023NotWY}, adapting techniques like SQL injection to LLMs \cite{Liu2023PromptIA}. Prompt secrecy remains fragile, with studies showing reliable prompt extraction \cite{Zhang2023EffectivePE}. Advanced frameworks like Token Space Projection \cite{Maus2023AdversarialPF} and Weak-to-Strong Jailbreaking Attacks \cite{zhao2024weaktostrongjailbreakinglargelanguage} exploit token-space relationships, achieving high success rates for prompt extraction and jailbreaking.

\subsection{Agentic Frameworks for Evaluating LLM Security}

The development of multi-agent systems leveraging large language models (LLMs) has shown promising results in enhancing task-solving capabilities \cite{Hong2023MetaGPTMP, Wang2023UnleashingTE, Talebirad2023MultiAgentCH, Wu2023AutoGenEN, Du2023ImprovingFA}. A key aspect across various frameworks is the specialization of roles among agents \cite{Hong2023MetaGPTMP, Wu2023AutoGenEN}, which mimics human collaboration and improves task decomposition.

Agentic frameworks and the multi-agent debate approach benefit from agent interaction, where agents engage in conversations or debates to refine outputs and correct errors \cite{Wu2023AutoGenEN}. For example, debate systems improve factual accuracy and reasoning by iteratively refining responses through collaborative reasoning \cite{Du2023ImprovingFA}, while AG2 allows agents to autonomously interact and execute tasks with minimal human input.

These frameworks highlight the viability of agentic systems, showing how specialized roles and collaborative mechanisms lead to improved performance, whether in factuality, reasoning, or task execution. By leveraging the strengths of diverse agents, these systems demonstrate a scalable approach to problem-solving.

Recent research on testing LLMs using other LLMs has shown that this approach can be highly effective \cite{chao2024jailbreakingblackboxlarge, Yu2023GPTFUZZERRT, Perez2022RedTL}. Although the papers do not explicitly employ agentic frameworks they inherently reflect a pattern similar to that of an "attacker" and a "judge". \cite{chao2024jailbreakingblackboxlarge}  This pattern became a focal point for our work, where we put the judge into a more direct dialogue, enabling it to generate attacks based on the tested agent response in an active conversation.

A particularly influential paper in shaping our approach is Jailbreaking Black Box Large Language Models in Twenty Queries \cite{chao2024jailbreakingblackboxlarge}. This paper not only introduced the attacker/judge architecture but also provided the initial system prompts used for a judge.
\section{Conclusion}
In this work, we propose a simple yet effective approach, called SMILE, for graph few-shot learning with fewer tasks. Specifically, we introduce a novel dual-level mixup strategy, including within-task and across-task mixup, for enriching the diversity of nodes within each task and the diversity of tasks. Also, we incorporate the degree-based prior information to learn expressive node embeddings. Theoretically, we prove that SMILE effectively enhances the model's generalization performance. Empirically, we conduct extensive experiments on multiple benchmarks and the results suggest that SMILE significantly outperforms other baselines, including both in-domain and cross-domain few-shot settings.


\bibliographystyle{plainnat}
\bibliography{references}

\appendix

%\title{Generating 3D \hl{Small} Binding Molecules Using Shape-Conditioned Diffusion Models with Guidance}
%\date{\vspace{-5ex}}

%\author{
%	Ziqi Chen\textsuperscript{\rm 1}, 
%	Bo Peng\textsuperscript{\rm 1}, 
%	Tianhua Zhai\textsuperscript{\rm 2},
%	Xia Ning\textsuperscript{\rm 1,3,4 \Letter}
%}
%\newcommand{\Address}{
%	\textsuperscript{\rm 1}Computer Science and Engineering, The Ohio Sate University, Columbus, OH 43210.
%	\textsuperscript{\rm 2}Perelman School of Medicine, University of Pennsylvania, Philadelphia, PA 19104.
%	\textsuperscript{\rm 3}Translational Data Analytics Institute, The Ohio Sate University, Columbus, OH 43210.
%	\textsuperscript{\rm 4}Biomedical Informatics, The Ohio Sate University, Columbus, OH 43210.
%	\textsuperscript{\Letter}ning.104@osu.edu
%}

%\newcommand\affiliation[1]{%
%	\begingroup
%	\renewcommand\thefootnote{}\footnote{#1}%
%	\addtocounter{footnote}{-1}%
%	\endgroup
%}



\setcounter{secnumdepth}{2} %May be changed to 1 or 2 if section numbers are desired.

\setcounter{section}{0}
\renewcommand{\thesection}{S\arabic{section}}

\setcounter{table}{0}
\renewcommand{\thetable}{S\arabic{table}}

\setcounter{figure}{0}
\renewcommand{\thefigure}{S\arabic{figure}}

\setcounter{algorithm}{0}
\renewcommand{\thealgorithm}{S\arabic{algorithm}}

\setcounter{equation}{0}
\renewcommand{\theequation}{S\arabic{equation}}


\begin{center}
	\begin{minipage}{0.95\linewidth}
		\centering
		\LARGE 
	Generating 3D Binding Molecules Using Shape-Conditioned Diffusion Models with Guidance (Supplementary Information)
	\end{minipage}
\end{center}
\vspace{10pt}

%%%%%%%%%%%%%%%%%%%%%%%%%%%%%%%%%%%%%%%%%%%%%
\section{Parameters for Reproducibility}
\label{supp:experiments:parameters}
%%%%%%%%%%%%%%%%%%%%%%%%%%%%%%%%%%%%%%%%%%%%%

We implemented both \SE and \methoddiff using Python-3.7.16, PyTorch-1.11.0, PyTorch-scatter-2.0.9, Numpy-1.21.5, Scikit-learn-1.0.2.
%
We trained the models using a Tesla V100 GPU with 32GB memory and a CPU with 80GB memory on Red Hat Enterprise 7.7.
%
%We released the code, data, and the trained model at Google Drive~\footnote{\url{https://drive.google.com/drive/folders/146cpjuwenKGTd6Zh4sYBy-Wv6BMfGwe4?usp=sharing}} (will release to the public on github once the manuscript is accepted).

%===================================================================
\subsection{Parameters of \SE}
%===================================================================


In \SE, we tuned the dimension of all the hidden layers including VN-DGCNN layers
(Eq.~\ref{eqn:shape_embed}), MLP layers (Eq.~\ref{eqn:se:decoder}) and
VN-In layer (Eq.~\ref{eqn:se:decoder}), and the dimension $d_p$ of generated shape latent embeddings $\shapehiddenmat$ with the grid-search algorithm in the 
parameter space presented in Table~\ref{tbl:hyper_se}.
%
We determined the optimal hyper-parameters according to the mean squared errors of the predictions of signed distances for 1,000 validation molecules that are selected as described in Section ``Data'' 
in the main manuscript.
%
The optimal dimension of all the hidden layers is 256, and the optimal dimension $d_p$ of shape latent embedding \shapehiddenmat is 128.
%
The optimal number of points $|\pc|$ in the point cloud \pc is 512.
%
We sampled 1,024 query points in $\mathcal{Z}$ for each molecule shape.
%
We constructed graphs from point clouds, which are employed to learn $\shapehiddenmat$ with VN-DGCNN layer (Eq.~\ref{eqn:shape_embed}), using the $k$-nearest neighbors based on Euclidean distance with $k=20$.
%
We set the number of VN-DGCNN layers as 4.
%
We set the number of MLP layers in the decoder (Eq.~\ref{eqn:se:decoder}) as 2.
%
We set the number of VN-In layers as 1.

%
We optimized the \SE model via Adam~\cite{adam} with its parameters (0.950, 0.999), %betas (0.95, 0.999), 
learning rate 0.001, and batch size 16.
%
We evaluated the validation loss every 2,000 training steps.
%
We scheduled to decay the learning rate with a factor of 0.6 and a minimum learning rate of 1e-6 if 
the validation loss does not decrease in 5 consecutive evaluations.
%
The optimal \SE model has 28.3K learnable parameters. 
%
We trained the \SE model %for at most 80 hours 
with $\sim$156,000 training steps.
%
The training took 80 hours with our GPUs.
%
The trained \SE model achieved the minimum validation loss at 152,000 steps.


\begin{table*}[!h]
  \centering
      \caption{{Hyper-Parameter Space for \SE Optimization}}
  \label{tbl:hyper_se}
  \begin{threeparttable}
 \begin{scriptsize}
      \begin{tabular}{
%	@{\hspace{2pt}}l@{\hspace{2pt}}
	@{\hspace{2pt}}l@{\hspace{5pt}} 
	@{\hspace{2pt}}r@{\hspace{2pt}}         
	}
        \toprule
        %Notation &
          Hyper-parameters &  Space\\
        \midrule
        %$t_a$    & 
         %hidden layer dimension         & \{16, 32, 64, 128\} \\
         %atom/node embedding dimension &  \{16, 32, 64, 128\} \\
         %$\latent^{\add}$/$\latent^{\delete}$ dimension        & \{8, 16, 32, 64\} \\
         hidden layer dimension            & \{128, 256\}\\
         dimension $d_p$ of \shapehiddenmat        &  \{64, 128\} \\
         \#points in \pc        & \{512, 1,024\} \\
         \#query points in $\mathcal{Z}$                & 1,024 \\%1024 \\%\bo{\{1024\}}\\
         \#nearest neighbors              & 20          \\
         \#VN-DGCNN layers (Eq~\ref{eqn:shape_embed})               & 4            \\
         \#MLP layers in Eq~\ref{eqn:se:decoder} & 4           \\
        \bottomrule
      \end{tabular}
%  	\begin{tablenotes}[normal,flushleft]
%  		\begin{footnotesize}
%  	
%  	\item In this table, hidden dimension represents the dimension of hidden layers and 
%  	atom/node embeddings; latent dimension represents the dimension of latent embedding \latent.
%  	\par
%  \end{footnotesize}
%  
%\end{tablenotes}
%      \begin{tablenotes}
%      \item 
%      \par
%      \end{tablenotes}
\end{scriptsize}
  \end{threeparttable}
\end{table*}

%
\begin{table*}[!h]
  \centering
      \caption{{Hyper-Parameter Space for \methoddiff Optimization}}
  \label{tbl:hyper_diff}
  \begin{threeparttable}
 \begin{scriptsize}
      \begin{tabular}{
%	@{\hspace{2pt}}l@{\hspace{2pt}}
	@{\hspace{2pt}}l@{\hspace{5pt}} 
	@{\hspace{2pt}}r@{\hspace{2pt}}         
	}
        \toprule
        %Notation &
          Hyper-parameters &  Space\\
        \midrule
        %$t_a$    & 
         %hidden layer dimension         & \{16, 32, 64, 128\} \\
         %atom/node embedding dimension &  \{16, 32, 64, 128\} \\
         %$\latent^{\add}$/$\latent^{\delete}$ dimension        & \{8, 16, 32, 64\} \\
         scalar hidden layer dimension         & 128 \\
         vector hidden layer dimension         & 32 \\
         weight of atom type loss $\xi$ (Eq.~\ref{eqn:loss})  & 100           \\
         threshold of step weight $\delta$ (Eq.~\ref{eqn:diff:obj:pos}) & 10 \\
         \#atom features $K$                   & 15 \\
         \#layers $L$ in \molpred             & 8 \\
         %\# \eqgnn/\invgnn layers     &  8 \\
         %\# heads {$n_h$} in $\text{MHA}^{\mathtt{x}}/\text{MHA}^{\mathtt{v}}$                               & 16 \\
         \#nearest neighbors {$N$}  (Eq.~\ref{eqn:geometric_embedding} and \ref{eqn:attention})            & 8          \\
         {\#diffusion steps $T$}                  & 1,000 \\
        \bottomrule
      \end{tabular}
%  	\begin{tablenotes}[normal,flushleft]
%  		\begin{footnotesize}
%  	
%  	\item In this table, hidden dimension represents the dimension of hidden layers and 
%  	atom/node embeddings; latent dimension represents the dimension of latent embedding \latent.
%  	\par
%  \end{footnotesize}
%  
%\end{tablenotes}
%      \begin{tablenotes}
%      \item 
%      \par
%      \end{tablenotes}
\end{scriptsize}
  \end{threeparttable}

\end{table*}


%===================================================================
\subsection{Parameters of \methoddiff}
%===================================================================

Table~\ref{tbl:hyper_diff} presents the parameters used to train \methoddiff.
%
In \methoddiff, we set the hidden dimensions of all the MLP layers and the scalar hidden layers in GVPs (Eq.~\ref{eqn:pred:gvp} and Eq.~\ref{eqn:mess:gvp}) as 128. %, including all the MLP layers in \methoddiff and the scalar dimension of GVP layers in Eq.~\ref{eqn:pred:gvp} and Eq.~\ref{eqn:mess:gvp}. %, and MLP layer (Eq.~\ref{eqn:diff:graph:atompred}) as 128.
%
We set the dimensions of all the vector hidden layers in GVPs as 32.
%
We set the number of layers $L$ in \molpred as 8.
%and the number of layers in graph neural networks as 8.
%
Both two GVP modules in Eq.~\ref{eqn:pred:gvp} and Eq.~\ref{eqn:mess:gvp} consist of three GVP layers. %, which consisa GVP modset the number of layer of GVP modules %is a multi-head attention layer ($\text{MHA}^{\mathtt{x}}$ or $\text{MHA}^{\mathtt{h}}$) with 16 heads.
% 
We set the number of VN-MLP layers in Eq.~\ref{eqn:shaper} as 1 and the number of MLP layers as 2 for all the involved MLP functions.
%

We constructed graphs from atoms in molecules, which are employed to learn the scalar embeddings and vector embeddings for atoms %predict atom coordinates and features  
(Eq.~\ref{eqn:geometric_embedding} and \ref{eqn:attention}), using the $N$-nearest neighbors based on Euclidean distance with $N=8$. 
%
We used $K=15$ atom features in total, indicating the atom types and its aromaticity.
%
These atom features include 10 non-aromatic atoms (i.e., ``H'', ``C'', ``N'', ``O'', ``F'', ``P'', ``S'', ``Cl'', ``Br'', ``I''), 
and 5 aromatic atoms (i.e., ``C'', ``N'', ``O'', ``P'', ``S'').
%
We set the number of diffusion steps $T$ as 1,000.
%
We set the weight $\xi$ of atom type loss (Eq.~\ref{eqn:loss}) as $100$ to balance the values of atom type loss and atom coordinate loss.
%
We set the threshold $\delta$ (Eq.~\ref{eqn:diff:obj:pos}) as 10.
%
The parameters $\beta_t^{\mathtt{x}}$ and $\beta_t^{\mathtt{v}}$ of variance scheduling in the forward diffusion process of \methoddiff are discussed in 
Supplementary Section~\ref{supp:forward:variance}.
%
%Please note that as in \squid, we did not perform extensive hyperparameter optimization for \methoddiff.
%
Following \squid, we did not perform extensive hyperparameter tunning for \methoddiff given that the used 
hyperparameters have enabled good performance.

%
We optimized the \methoddiff model via Adam~\cite{adam} with its parameters (0.950, 0.999), learning rate 0.001, and batch size 32.
%
We evaluated the validation loss every 2,000 training steps.
%
We scheduled to decay the learning rate with a factor of 0.6 and a minimum learning rate of 1e-5 if 
the validation loss does not decrease in 10 consecutive evaluations.
%
The \methoddiff model has 7.8M learnable parameters. 
%
We trained the \methoddiff model %for at most 60 hours 
with $\sim$770,000 training steps.
%
The training took 70 hours with our GPUs.
%
The trained \methoddiff achieved the minimum validation loss at 758,000 steps.

During inference, %the sampling, 
following Adams and Coley~\cite{adams2023equivariant}, we set the variance $\phi$ 
of atom-centered Gaussians as 0.049, which is used to build a set of points for shape guidance in Section ``\method with Shape Guidance'' 
in the main manuscript.
%
We determined the number of atoms in the generated molecule using the atom number distribution of training molecules that have surface shape sizes similar to the condition molecule.
%
The optimal distance threshold $\gamma$ is 0.2, and the optimal stop step $S$ for shape guidance is 300.
%
With shape guidance, each time we updated the atom position (Eq.~\ref{eqn:shape_guidance}), we randomly sampled the weight $\sigma$ from $[0.2, 0.8]$. %\bo{(XXX)}.
%
Moreover, when using pocket guidance as mentioned in Section ``\method with Pocket Guidance'' in the main manuscript, each time we updated the atom position (Eq.~\ref{eqn:pocket_guidance}), we randomly sampled the weight $\epsilon$ from $[0, 0.5]$. 
%
For each condition molecule, it took around 40 seconds on average to generate 50 molecule candidates with our GPUs.



%%%%%%%%%%%%%%%%%%%%%%%%%%%%%%%%%%%%%%%%%%%%%%
\section{Performance of \decompdiff with Protein Pocket Prior}
\label{supp:app:decompdiff}
%%%%%%%%%%%%%%%%%%%%%%%%%%%%%%%%%%%%%%%%%%%%%%

In this section, we demonstrate that \decompdiff with protein pocket prior, referred to as \decompdiffbeta, shows very limited performance in generating drug-like and synthesizable molecules compared to all the other methods, including \methodwithpguide and \methodwithsandpguide.
%
We evaluate the performance of \decompdiffbeta in terms of binding affinities, drug-likeness, and diversity.
%
We compare \decompdiffbeta with \methodwithpguide and \methodwithsandpguide and report the results in Table~\ref{tbl:comparison_results_decompdiff}.
%
Note that the results of \methodwithpguide and \methodwithsandpguide here are consistent with those in Table~\ref{tbl:overall_results_docking2} in the main manuscript.
%
As shown in Table~\ref{tbl:comparison_results_decompdiff}, while \decompdiffbeta achieves high binding affinities in Vina M and Vina D, it substantially underperforms \methodwithpguide and \methodwithsandpguide in QED and SA.
%
Particularly, \decompdiffbeta shows a QED score of 0.36, while \methodwithpguide substantially outperforms \decompdiffbeta in QED (0.77) with 113.9\% improvement.
%
\decompdiffbeta also substantially underperforms \methodwithpguide in terms of SA scores (0.55 vs 0.76).
%
These results demonstrate the limited capacity of \decompdiffbeta in generating drug-like and synthesizable molecules.
%
As a result, the generated molecules from \decompdiffbeta can have considerably lower utility compared to other methods.
%
Considering these limitations of \decompdiffbeta, we exclude it from the baselines for comparison.

\begin{table*}[!h]
	\centering
		\caption{Comparison on PMG among \methodwithpguide, \methodwithsandpguide and \decompdiffbeta}
	\label{tbl:comparison_results_decompdiff}
\begin{threeparttable}
	\begin{scriptsize}
	\begin{tabular}{
		@{\hspace{2pt}}l@{\hspace{2pt}}
		%
		%@{\hspace{2pt}}l@{\hspace{2pt}}
		%
		@{\hspace{2pt}}r@{\hspace{2pt}}
		@{\hspace{2pt}}r@{\hspace{2pt}}
		%
		@{\hspace{6pt}}r@{\hspace{6pt}}
		%
		@{\hspace{2pt}}r@{\hspace{2pt}}
		@{\hspace{2pt}}r@{\hspace{2pt}}
		%
		@{\hspace{5pt}}r@{\hspace{5pt}}
		%
		@{\hspace{2pt}}r@{\hspace{2pt}}
		@{\hspace{2pt}}r@{\hspace{2pt}}
		%
		@{\hspace{5pt}}r@{\hspace{5pt}}
		%
		@{\hspace{2pt}}r@{\hspace{2pt}}
	         @{\hspace{2pt}}r@{\hspace{2pt}}
		%
		@{\hspace{5pt}}r@{\hspace{5pt}}
		%
		@{\hspace{2pt}}r@{\hspace{2pt}}
		@{\hspace{2pt}}r@{\hspace{2pt}}
		%
		@{\hspace{5pt}}r@{\hspace{5pt}}
		%
		@{\hspace{2pt}}r@{\hspace{2pt}}
		@{\hspace{2pt}}r@{\hspace{2pt}}
		%
		@{\hspace{5pt}}r@{\hspace{5pt}}
		%
		@{\hspace{2pt}}r@{\hspace{2pt}}
		@{\hspace{2pt}}r@{\hspace{2pt}}
		%
		@{\hspace{5pt}}r@{\hspace{5pt}}
		%
		@{\hspace{2pt}}r@{\hspace{2pt}}
		%@{\hspace{2pt}}r@{\hspace{2pt}}
		%@{\hspace{2pt}}r@{\hspace{2pt}}
		}
		\toprule
		\multirow{2}{*}{method} & \multicolumn{2}{c}{Vina S$\downarrow$} & & \multicolumn{2}{c}{Vina M$\downarrow$} & & \multicolumn{2}{c}{Vina D$\downarrow$} & & \multicolumn{2}{c}{{HA\%$\uparrow$}}  & & \multicolumn{2}{c}{QED$\uparrow$} & & \multicolumn{2}{c}{SA$\uparrow$} & & \multicolumn{2}{c}{Div$\uparrow$} & %& \multirow{2}{*}{SR\%$\uparrow$} & 
		& \multirow{2}{*}{time$\downarrow$} \\
	    \cmidrule{2-3}\cmidrule{5-6} \cmidrule{8-9} \cmidrule{11-12} \cmidrule{14-15} \cmidrule{17-18} \cmidrule{20-21}
		& Avg. & Med. &  & Avg. & Med. &  & Avg. & Med. & & Avg. & Med.  & & Avg. & Med.  & & Avg. & Med.  & & Avg. & Med.  & & \\ %& & \\
		%\multirow{2}{*}{method} & \multirow{2}{*}{\#c\%} &  \multirow{2}{*}{\#u\%} &  \multirow{2}{*}{QED} & \multicolumn{3}{c}{$\nmax=50$} & & \multicolumn{2}{c}{$\nmax=1$}\\
		%\cmidrule(r){5-7} \cmidrule(r){8-10} 
		%& & & & \avgshapesim(std) & \avggraphsim(std  &  \diversity(std  & & \avgshapesim(std) & \avggraphsim(std \\
		\midrule
		%Reference                          & -5.32 & -5.66 & & -5.78 & -5.76 & & -6.63 & -6.67 & & - & - & & 0.53 & 0.49 & & 0.77 & 0.77 & & - & - & %& 23.1 & & - \\
		%\midrule
		%\multirow{4}{*}{PM} 
		%& \AR & -5.06 & -4.99 & &  -5.59 & -5.29 & &  -6.16 & -6.05 & &  37.69 & 31.00 & &  0.50 & 0.49 & &  0.66 & 0.65 & & - & - & %& 7.0 & 
		%& 7,789 \\
		%& \pockettwomol   & -4.50 & -4.21 & &  -5.70 & -5.27 & &  -6.43 & -6.25 & &  48.00 & 51.00 & &  0.58 & 0.58 & &  \textbf{0.77} & \textbf{0.78} & &  0.69 & 0.71 &  %& 24.9 & 
		%& 2,544 \\
		%& \targetdiff     & -4.88 & \underline{-5.82} & &  -6.20 & \underline{-6.36} & &  \textbf{-7.37} & \underline{-7.51} & &  57.57 & 58.27 & &  0.50 & 0.51 & &  0.60 & 0.59 & &  0.72 & 0.71 & % & 10.4 & 
		%& 1,252 \\
		 \decompdiffbeta             & -4.72 & -4.86 & & \textbf{-6.84} & \textbf{-6.91} & & \textbf{-8.85} & \textbf{-8.90} & &  {72.16} & {72.16} & &  0.36 & 0.36 & &  0.55 & 0.55 & & 0.59 & 0.59 & & 3,549 \\ 
		%-4.76 & -6.18 & &  \textbf{-6.86} & \textbf{-7.52} & &  \textbf{-8.85} & \textbf{-8.96} & &  \textbf{72.7} & \textbf{89.8} & &  0.36 & 0.34 & &  0.55 & 0.57 & & 0.59 & 0.59 & & 15.4 \\
		%& \decompdiffref  & -4.58 & -4.77 & &  -5.47 & -5.51 & &  -6.43 & -6.56 & &  47.76 & 48.66 & &  0.56 & 0.56 & &  0.70 & 0.69  & &  0.72 & 0.72 &  %& 15.2 & 
		%& 1,859 \\
		%\midrule
		%\multirow{2}{*}{PC}
		\methodwithpguide       &  \underline{-5.53} & \underline{-5.64} & & {-6.37} & -6.33 & &  \underline{-7.19} & \underline{-7.52} & &  \underline{78.75} & \textbf{94.00} & &  \textbf{0.77} & \textbf{0.80} & &  \textbf{0.76} & \textbf{0.76} & & 0.63 & 0.66 & & 462 \\
		\methodwithsandpguide   & \textbf{-5.81} & \textbf{-5.96} & &  \underline{-6.50} & \underline{-6.58} & & -7.16 & {-7.51} & &  \textbf{79.92} & \underline{93.00} & &  \underline{0.76} & \underline{0.79} & &  \underline{0.75} & \underline{0.74} & & 0.64 & 0.66 & & 561\\
		\bottomrule
	\end{tabular}%
	\begin{tablenotes}[normal,flushleft]
		\begin{footnotesize}
	\item 
\!\!Columns represent: {``Vina S'': the binding affinities between the initially generated poses of molecules and the protein pockets; 
		``Vina M'': the binding affinities between the poses after local structure minimization and the protein pockets;
		``Vina D'': the binding affinities between the poses determined by AutoDock Vina~\cite{Eberhardt2021} and the protein targets;
		``QED'': the drug-likeness score;
		``SA'': the synthesizability score;
		``Div'': the diversity among generated molecules;
		``time'': the time cost to generate molecules.}
		
		\par
		\par
		\end{footnotesize}
	\end{tablenotes}
	\end{scriptsize}
\end{threeparttable}
  \vspace{-10pt}    
\end{table*}



%===================================================================
\section{{Additional Experimental Results on SMG}}
\label{supp:app:results}
%===================================================================

%-------------------------------------------------------------------------------------------------------------------------------------
\subsection{Comparison on Shape and Graph Similarity}
\label{supp:app:results:overall_shape}
%-------------------------------------------------------------------------------------------------------------------------------------

%\ziqi{Outline for this section:
%	\begin{itemize}
%		\item \method can consistently generate molecules with novel structures (low graph similarity) and similar shapes (high shape similarity), such that these molecules have comparable binding capacity with the condition molecules, and potentially better properties as will be shown in Table~\ref{tbl:overall_results_quality_10}.
%	\end{itemize}
%}

\begin{table*}[!h]
	\centering
		\caption{Similarity Comparison on SMG}
	\label{tbl:overall_sim}
\begin{threeparttable}
	\begin{scriptsize}
	\begin{tabular}{
		@{\hspace{0pt}}l@{\hspace{8pt}}
		%
		@{\hspace{8pt}}l@{\hspace{8pt}}
		%
		@{\hspace{8pt}}c@{\hspace{8pt}}
		@{\hspace{8pt}}c@{\hspace{8pt}}
		%
	    	@{\hspace{0pt}}c@{\hspace{0pt}}
		%
		@{\hspace{8pt}}c@{\hspace{8pt}}
		@{\hspace{8pt}}c@{\hspace{8pt}}
		%
		%@{\hspace{8pt}}r@{\hspace{8pt}}
		}
		\toprule
		$\delta_g$  & method          & \avgshapesim$\uparrow$(std) & \avggraphsim$\downarrow$(std) & & \maxshapesim$\uparrow$(std) & \maxgraphsim$\downarrow$(std)       \\ %& \#n\%$\uparrow$  \\ 
		\midrule
		%\multirow{5}{0.079\linewidth}%{\hspace{0pt}0.1} & \dataset   & 0.0             & 0.628(0.139)          & 0.567(0.068)          & 0.078(0.010)          &  & 0.588(0.086)          & 0.081(0.013)          & 4.7              \\
		%&  \squid($\lambda$=0.3) & 0.0             & 0.320(0.000)          & 0.420(0.163)          & \textbf{0.056}(0.032) &  & 0.461(0.170)          & \textbf{0.065}(0.033) & 1.4              \\
		%& \squid($\lambda$=1.0) & 0.0             & 0.414(0.177)          & 0.483(0.184)          & \underline{0.064}(0.030)  &  & 0.531(0.182)          & \underline{0.073}(0.029)  & 2.4              \\
		%& \method               & \underline{1.6}     & \textbf{0.857}(0.034) & \underline{0.773}(0.045)  & 0.086(0.011)          &  & \underline{0.791}(0.053)  & 0.087(0.012)          & \underline{5.1}      \\
		%& \methodwithsguide      & \textbf{3.7}    & \underline{0.833}(0.062)  & \textbf{0.812}(0.037) & 0.088(0.009)          &  & \textbf{0.835}(0.047) & 0.089(0.010)          & \textbf{6.2}     \\ 
		%\cmidrule{2-10}
		%& improv\% & - & 36.5 & 43.2 & -53.6 &  & 42.0 & -33.8 & 31.9  \\
		%\midrule
		\multirow{6}{0.059\linewidth}{\hspace{0pt}0.3} & \dataset             & 0.745(0.037)          & \textbf{0.211}(0.026) &  & 0.815(0.039)          & \textbf{0.215}(0.047)      \\ %    & \textbf{100.0}   \\
			& \squid($\lambda$=0.3) & 0.709(0.076)          & 0.237(0.033)          &  & 0.841(0.070)          & 0.253(0.038)        \\ %  & 45.5             \\
		    & \squid($\lambda$=1.0) & 0.695(0.064)          & \underline{0.216}(0.034)  &  & 0.841(0.056)          & 0.231(0.047)        \\ %  & 84.3             \\
			& \method               & \underline{0.770}(0.039)  & 0.217(0.031)          &  & \underline{0.858}(0.038)  & \underline{0.220}(0.046)  \\ %& \underline{87.1}     \\
			& \methodwithsguide     & \textbf{0.823}(0.029) & 0.217(0.032)          &  & \textbf{0.900}(0.028) & 0.223(0.048)  \\ % & 86.0             \\ 
		%\cmidrule{2-7}
		%& improv\% & 10.5 & -2.8 &  & 7.0 & -2.3  \\ % & %-12.9  \\
		\midrule
		\multirow{6}{0.059\linewidth}{\hspace{0pt}0.5} & \dataset & 0.750(0.037)          & \textbf{0.225}(0.037) &  & 0.819(0.039)          & \textbf{0.236}(0.070)          \\ %& \textbf{100.0}   \\
			& \squid($\lambda$=0.3)  & 0.728(0.072)          & 0.301(0.054)          &  & \underline{0.888}(0.061)  & 0.355(0.088)          \\ %& 85.9             \\
			& \squid($\lambda$=1.0)  & 0.699(0.063)          & 0.233(0.043)          &  & 0.850(0.057)          & 0.263(0.080)          \\ %& \underline{99.5}     \\
			& \method               & \underline{0.771}(0.039)  & \underline{0.229}(0.043)  &  & 0.862(0.036)          & \textbf{0.236}(0.065) \\ %& 99.2             \\
			& \methodwithsguide    & \textbf{0.824}(0.029) & \underline{0.229}(0.044)  &  & \textbf{0.903}(0.027) & \underline{0.242}(0.069)  \\ %& 99.0             \\ 
		%\cmidrule{2-7}
		%& improv\% & 9.9 & -1.8 &  & 1.7 & 0.0 \\ %& -0.8  \\
		\midrule
		\multirow{6}{0.059\linewidth}{\hspace{0pt}0.7} 
		& \dataset &  0.750(0.037) & \textbf{0.226}(0.038) & & 0.819(0.039) & \underline{0.240}(0.081) \\ %& \textbf{100.0} \\
		%& \dataset & 12.3            & 0.736(0.076)          & 0.768(0.037)          & \textbf{0.228}(0.042) &  & 0.819(0.039)          & \underline{0.242}(0.085)  & \textbf{100.0}   \\
			& \squid($\lambda$=0.3) &  0.735(0.074)          & 0.328(0.070)          &  & \underline{0.900}(0.062)  & 0.435(0.143)          \\ %& 95.4             \\
			& \squid($\lambda$=1.0) &  0.699(0.064)          & 0.234(0.045)          &  & 0.851(0.057)          & 0.268(0.090)          \\ %& \underline{99.9}     \\
			& \method               &  \underline{0.771}(0.039)  & \underline{0.229}(0.043)  &  & 0.862(0.036)          & \textbf{0.237}(0.066) \\ %& 99.3             \\
			& \methodwithsguide     &  \textbf{0.824}(0.029) & 0.230(0.045)          &  & \textbf{0.903}(0.027) & 0.244(0.074)          \\ %& 99.2             \\ 
		%\cmidrule{2-7}
		%& improv\% & 9.9 & -1.3 &  & 0.3 & 1.3 \\%& -0.7  \\
		\midrule
		\multirow{6}{0.059\linewidth}{\hspace{0pt}1.0} 
		& \dataset & 0.750(0.037)          & \textbf{0.226}(0.038) &  & 0.819(0.039)          & \underline{0.242}(0.085)  \\%& \textbf{100.0}  \\
		& \squid($\lambda$=0.3) & 0.740(0.076)          & 0.349(0.088)          &  & \textbf{0.909}(0.065) & 0.547(0.245)       \\ %   & \textbf{100.0}  \\
		& \squid($\lambda$=1.0) & 0.699(0.064)          & 0.235(0.045)          &  & 0.851(0.057)          & 0.271(0.097)          \\ %& \textbf{100.0}   \\
		& \method               & \underline{0.771}(0.039)  & \underline{0.229}(0.043)  &  & 0.862(0.036)          & \textbf{0.237}(0.066) \\ %& \underline{99.3}  \\
		& \methodwithsguide      & \textbf{0.824}(0.029) & 0.230(0.045)          &  & \underline{0.903}(0.027)  & 0.244(0.076)          \\ %& 99.2            \\
		%\cmidrule{2-7}
		%& improv\% &  9.9               & -1.3              &  & -0.7              & -2.1           \\ %       & -0.7 \\
		\bottomrule
	\end{tabular}%
	\begin{tablenotes}[normal,flushleft]
		\begin{footnotesize}
	\item 
\!\!Columns represent: ``$\delta_g$'': the graph similarity constraint; 
%``\#d\%'': the percentage of molecules that satisfy the graph similarity constraint and are with high \shapesim ($\shapesim>=0.8$);
%``\diversity'': the diversity among the generated molecules;
``\avgshapesim/\avggraphsim'': the average of shape or graph similarities between the condition molecules and generated molecules with $\graphsim<=\delta_g$;
``\maxshapesim'': the maximum of shape similarities between the condition molecules and generated molecules with $\graphsim<=\delta_g$;
``\maxgraphsim'': the graph similarities between the condition molecules and the molecules with the maximum shape similarities and $\graphsim<=\delta_g$;
%``\#n\%'': the percentage of molecules that satisfy the graph similarity constraint ($\graphsim<=\delta_g$).
%
``$\uparrow$'' represents higher values are better, and ``$\downarrow$'' represents lower values are better.
%
 Best values are in \textbf{bold}, and second-best values are \underline{underlined}. 
\par
		\par
		\end{footnotesize}
	\end{tablenotes}
\end{scriptsize}
\end{threeparttable}
  \vspace{-10pt}    
\end{table*}
%\label{tbl:overall_sim}


{We evaluate the shape similarity \shapesim and graph similarity \graphsim of molecules generated from}
%Table~\ref{tbl:overall_sim} presents the comparison of shape-conditioned molecule generation among 
\dataset, \squid, \method and \methodwithsguide under different graph similarity constraints  ($\delta_g$=1.0, 0.7, 0.5, 0.3). 
%
%During the evaluation, for each molecule in the test set, all the methods are employed to generate or identify 50 molecules with similar shapes.
%
We calculate evaluation metrics using all the generated molecules satisfying the graph similarity constraints.
%
Particularly, when $\delta_g$=1.0, we do not filter out any molecules based on the constraints and directly calculate metrics on all the generated molecules.
%
When $\delta_g$=0.7, 0.5 or 0.3, we consider only generated molecules with similarities lower than $\delta_g$.
%
Based on \shapesim and \graphsim as described in Section ``Evaluation Metrics'' in the main manuscript,
we calculate the following metrics using the subset of molecules with \graphsim lower than $\delta_g$, from a set of 50 generated molecules for each test molecule and report the average of  these metrics across all test molecules:
%
(1) \avgshapesim\ measures the average \shapesim across each subset of generated molecules with $\graphsim$ lower than $\delta_g$; %per test molecule, with the overall average calculated across all test molecules; }%the 50 generated molecules for each test molecule, averaged across all test molecules;
(2) \avggraphsim\ calculates the average \graphsim for each set; %, with these means averaged across all test molecules}; %} 50 molecules, %\bo{@Ziqi rephrase}, with results averaged on the test set;\ziqi{with the average computed over the test set; }
(3) \maxshapesim\ determines the maximum \shapesim within each set; %, with these maxima averaged across all test molecules; }%\hl{among 50 molecules}, averaged across all test molecules;
(4) \maxgraphsim\ measures the \graphsim of the molecule with maximum \shapesim in each set. %, averaged across all test molecules; }%\hl{among 50 molecules}, averaged across all test molecules;

%
As shown in Table~\ref{tbl:overall_sim}, \method and \methodwithsguide demonstrate outstanding performance in terms of the average shape similarities (\avgshapesim) and the average graph similarities (\avggraphsim) among generated molecules.
%
%\ziqi{
%Table~\ref{tbl:overall} also shows that \method and \methodwithsguide consistently outperform all the baseline methods in average shape similarities (\avgshapesim) and only slightly underperform 
%the best baseline \dataset in average graph similarities (\avggraphsim).
%}
%
Specifically, when $\delta_g$=0.3, \methodwithsguide achieves a substantial 10.5\% improvement in \avgshapesim\ over the best baseline \dataset. 
%
In terms of \avggraphsim, \methodwithsguide also achieves highly comparable performance with \dataset (0.217 vs 0.211, in \avggraphsim, lower values indicate better performance).
%
%This trend remains consistent across various $\delta_g$ values.
This trend remains consistent when applying various similarity constraints (i.e., $\delta_g$) as shown in Table~\ref{tbl:overall_sim}.


Similarly, \method and \methodwithsguide demonstrate superior performance in terms of the average maximum shape similarity across generated molecules for all test molecules (\maxshapesim), as well as the average graph similarity of the molecules with the maximum shape similarities (\maxgraphsim). %maximum shape similarities of generated molecules (\maxshapesim) and the average graph similarities of molecules with the maximum shape similarities (\maxgraphsim). %\bo{\maxgraphsim is misleading... how about $\text{avgMSim}_\text{g}$}
%
%\bo{
%in terms of the maximum shape similarities (\maxshapesim) and the maximum graph similarities (\maxgraphsim) among all the generated molecules.
%@Ziqi are the metrics maximum values or the average of maximum values?
%}
%
Specifically, at \maxshapesim, Table~\ref{tbl:overall_sim} shows that \methodwithsguide outperforms the best baseline \squid ($\lambda$=0.3) when $\delta_g$=0.3, 0.5, and 0.7, and only underperforms
it by 0.7\% when $\delta$=1.0.
%
We also note that the molecules generated by {\methodwithsguide} with the maximum shape similarities have substantially lower graph similarities ({\maxgraphsim}) compared to those generated by {\squid} ({$\lambda$}=0.3).
%\hl{We also note that the molecules with the maximum shape similarities generated by {\methodwithsguide} are with significantly lower graph similarities ({\maxgraphsim}) than those generated by {\squid} ({$\lambda$}=0.3).}
%
%\bo{@Ziqi please rephrase the language}
%
%\bo{
%@Ziqi the conclusion is not obvious. You may want to remind the meaning of \maxshapesim and \maxgraphsim here, and based on what performance you say this.
%}
%
%\bo{\st{This also underscores the ability of {\methodwithsguide} in generating molecules with similar shapes to condition molecules and novel graph structures.}}
%
As evidenced by these results, \methodwithsguide features strong capacities of generating molecules with similar shapes yet novel graph structures compared to the condition molecule, facilitating the discovery of promising drug candidates.
%

\begin{comment}
\ziqi{replace \#n\% with the percentage of novel molecules that do not exist in the dataset and update the discussion accordingly}
%\ziqi{
Table~\ref{tbl:overall_sim} also presents \bo{\#n\%}, the percentage of molecules generated by each method %\st{(\#n\%)} 
with graph similarities lower than the constraint $\delta_g$. 
%
%\bo{
%Table~\ref{tbl:overall_sim} also presents \#n\%, the percentage of generated molecules with graph similarities lower than the constraint $\delta_g$, of different methods. 
%}
%
As shown in Table~\ref{tbl:overall_sim},  when a restricted constraint (i.e., $\delta_g$=0.3) is applied, \method and \methodwithsguide could still generate a sufficient number of molecules satisfying the constraint.
%
Particularly, when $\delta_g$=0.3, \method outperforms \squid with $\lambda$=0.3 by XXX and \squid with $\lambda$=1.0 by XXX.
% achieve the second and the third in \#n\% and only underperform the best baseline \dataset.
%
This demonstrates the ability of \method in generating molecules with novel structures. 
%
When $\delta_g$=0.5, 0.7 and 1.0, both methods generate over 99.0\% of molecules satisfying the similarity constraint $\delta_g$.
%
%Note that \dataset is guaranteed to identify at least 50 molecules satisfying the $\delta_g$ by searching within a training dataset of diverse molecules.
%
Note that \dataset is a search algorithm that always first identifies the molecules satisfying $\delta_g$ and then selects the top-50 molecules of the highest shape similarities among them. 
%
Due to the diverse molecules in %\hl{the subset} \bo{@Ziqi why do you want to stress subset?} of 
the training set, \dataset can always identify at least 50 molecules under different $\delta_g$ and thus achieve 100\% in \#n\%.
%
%\bo{
%Note that \dataset is a search algorithm that always generate molecules XXX
%@Ziqi
%We need to discuss here. For \dataset, \#n\% in this table does not look aligned with that in Fig 1 if the highlighted defination is correct...
%}
%
%Thus, \dataset achieves 100.0\% in \#n\% under different $\delta_g$.
%
It is also worth noting that when $\delta_g$=1.0, \#n\% reflects the validity among all the generated molecules. 
%
As shown in Table~\ref{tbl:overall_sim}, \method and \methodwithsguide are able to generate 99.3\% and 99.2\% valid molecules.
%
This demonstrates their ability to effectively capture the underlying chemical rules in a purely data-driven manner without relying on any prior knowledge (e.g., fragments) as \squid does.
%
%\bo{
%@Ziqi I feel this metric is redundant with the avg graph similarity when constraint is 1.0. Generally, if the avg similarity is small. You have more mols satisfying the requirement right?
%}
\end{comment}

Table~\ref{tbl:overall_sim} also shows that by incorporating shape guidance, \methodwithsguide
%\bo{
%@Ziqi where does this come from...
%}
substantially outperforms \method in both \avgshapesim and \maxshapesim, while maintaining comparable graph similarities (i.e., \avggraphsim\ and \maxgraphsim).
%
Particularly, when $\delta_g$=0.3, \methodwithsguide 
establishes a considerable improvement of 6.9\% and 4.9\%
%\bo{\st{achieves 6.9\% and 4.9\% improvements}} 
over \method in \avgshapesim and \maxshapesim, respectively. 
%
%\hl{In the meanwhile}, 
%\bo{@Ziqi it is not the right word...}
Meanwhile, \methodwithsguide achieves the same \avggraphsim with \method and only slightly underperforms \method in \maxgraphsim (0.223 vs 0.220).
%\bo{
%XXX also achieves XXX
%}
%it maintains the same \avggraphsim\ with \method and only slightly underperforms \method in \maxgraphsim (0.223 vs 0.220).
%
%Compared with \method, \methodwithsguide consistently generates molecules with higher shape similarities while maintaining comparable graph similarities.
%
%\bo{
%@Ziqi you may want to highlight the utility of "generating molecules with higher shape similarities while maintaining comparable graph similarities" in real drug discovery applications.
%
%
%\bo{
%@Ziqi You did not present the details of method yet...
%}
%
%\methodwithsguide leverages additional shape guidance to push the predicted atoms to the shape of condition molecules \bo{and XXX (@Ziqi boosts the shape similarities XXX)} , as will be discussed in Section ``\method with Shape Guidance'' later.
%
The superior performance of \methodwithsguide suggests that the incorporation of shape guidance effectively boosts the shape similarities of generated molecules without compromising graph similarities.
%
%This capability could be crucial in drug discovery, 
%\bo{@Ziqi it is a strong statement. Need citations here}, 
%as it enables the discovery of drug candidates that are both more potentially effective due to the improved shape similarities and novel induced by low graph similarities.
%as it could enable the identification of candidates with similar binding patterns %with the condition molecule (i.e., high shape similarities) 
%(i.e., high shape similarities) and graph structures distinct from the condition molecules (i.e., low graph similarities).
%\bo{\st{and enjoys novel structures (i.e., low graph similarities) with potentially better properties. } \ziqi{change enjoys}}
%\bo{
%and enjoys potentially better properties (i.e., low graph similarities). \ziqi{this looks weird to me... need to discuss}
%}
%\st{potentially better properties (i.e., low graph similarities).}}

%-------------------------------------------------------------------------------------------------------------------------------------
\subsection{Comparison on Validity and Novelty}
\label{supp:app:results:valid_novel}
%-------------------------------------------------------------------------------------------------------------------------------------

We evaluate the ability of \method and \squid to generate molecules with valid and novel 2D molecular graphs.
%
We calculate the percentages of the valid and novel molecules among all the generated molecules.
%
As shown in Table~\ref{tbl:validity_novelty}, both \method and \methodwithsguide outperform \squid with $\lambda$=0.3 and $\lambda$=1.0 in generating novel molecules.
%
Particularly, almost all valid molecules generated by \method and \methodwithsguide are novel (99.8\% and 99.9\% at \#n\%), while the best baseline \squid with $\lambda$=0.3 achieves 98.4\% in novelty.
%
In terms of the percentage of valid and novel molecules among all the generated ones (\#v\&n\%), \method and \methodwithsguide again outperform \squid with $\lambda$=0.3 and $\lambda$=1.0.
%
We also note that at \#v\%,  \method (99.1\%) and \methodwithsguide (99.2\%) slightly underperform \squid with $\lambda$=0.3 and $\lambda$=1.0 (100.0\%) in generating valid molecules.
%
\squid guarantees the validity of generated molecules by incorporating valence rules into the generation process and ensuring it to avoid fragments that violate these rules.
%
Conversely, \method and \methodwithsguide use a purely data-driven approach to learn the generation of valid molecules.
%
These results suggest that, even without integrating valence rules, \method and \methodwithsguide can still achieve a remarkably high percentage of valid and novel generated molecules.

\begin{table*}
	\centering
		\caption{Comparison on Validity and Novelty between \method and \squid}
	\label{tbl:validity_novelty}
	\begin{scriptsize}
\begin{threeparttable}
%	\setlength\tabcolsep{0pt}
	\begin{tabular}{
		@{\hspace{3pt}}l@{\hspace{10pt}}
		%
		@{\hspace{10pt}}r@{\hspace{10pt}}
		%
		@{\hspace{10pt}}r@{\hspace{10pt}}
		%
		@{\hspace{10pt}}r@{\hspace{3pt}}
		}
		\toprule
		method & \#v\% & \#n\% & \#v\&n\% \\
		\midrule
		\squid ($\lambda$=0.3) & \textbf{100.0} & 96.7 & 96.7 \\
		\squid ($\lambda$=1.0) & \textbf{100.0} & 98.4 & 98.4 \\
		\method & 99.1 & 99.8 & 98.9 \\
		\methodwithsguide & 99.2 & \textbf{99.9} & \textbf{99.1} \\
		\bottomrule
	\end{tabular}%
	%
	\begin{tablenotes}[normal,flushleft]
		\begin{footnotesize}
	\item 
\!\!Columns represent: ``\#v\%'': the percentage of generated molecules that are valid;
		``\#n\%'': the percentage of valid molecules that are novel;
		``\#v\&n\%'': the percentage of generated molecules that are valid and novel.
		Best values are in \textbf{bold}. 
		\par
		\end{footnotesize}
	\end{tablenotes}
\end{threeparttable}
\end{scriptsize}
\end{table*}


%-------------------------------------------------------------------------------------------------------------------------------------
\subsection{Additional Quality Comparison between Desirable Molecules Generated by \method and \squid}
\label{supp:app:results:quality_desirable}
%-------------------------------------------------------------------------------------------------------------------------------------

\begin{table*}[!h]
	\centering
		\caption{Comparison on Quality of Generated Desirable Molecules between \method and \squid ($\delta_g$=0.5)}
	\label{tbl:overall_results_quality_05}
	\begin{scriptsize}
\begin{threeparttable}
	\begin{tabular}{
		@{\hspace{0pt}}l@{\hspace{16pt}}
		@{\hspace{0pt}}l@{\hspace{2pt}}
		%
		@{\hspace{6pt}}c@{\hspace{6pt}}
		%
		%@{\hspace{3pt}}c@{\hspace{3pt}}
		@{\hspace{3pt}}c@{\hspace{3pt}}
		@{\hspace{3pt}}c@{\hspace{3pt}}
		@{\hspace{3pt}}c@{\hspace{3pt}}
		@{\hspace{3pt}}c@{\hspace{3pt}}
		%
		%
		}
		\toprule
		group & metric & 
        %& \dataset 
        & \squid ($\lambda$=0.3) & \squid ($\lambda$=1.0)  &  \method & \methodwithsguide  \\
		%\multirow{2}{*}{method} & \multirow{2}{*}{\#c\%} &  \multirow{2}{*}{\#u\%} &  \multirow{2}{*}{QED} & \multicolumn{3}{c}{$\nmax=50$} & & \multicolumn{2}{c}{$\nmax=1$}\\
		%\cmidrule(r){5-7} \cmidrule(r){8-10} 
		%& & & & \avgshapesim(std) & \avggraphsim(std  &  \diversity(std  & & \avgshapesim(std) & \avggraphsim(std \\
		\midrule
		\multirow{2}{*}{stability}
		& atom stability ($\uparrow$) & 
        %& 0.990 
        & \textbf{0.996} & 0.995 & 0.992 & 0.989     \\
		& mol stability ($\uparrow$) & 
        %& 0.875 
        & \textbf{0.948} & 0.947 & 0.886 & 0.839    \\
		%\midrule
		%\multirow{3}{*}{Drug-likeness} 
		%& QED ($\uparrow$) & 
        %& \textbf{0.805} 
        %& 0.766 & 0.760 & 0.755 & 0.751    \\
	%	& SA ($\uparrow$) & 
        %& \textbf{0.874} 
        %& 0.814 & 0.813 & 0.699 & 0.692    \\
	%	& Lipinski ($\uparrow$) & 
        %& \textbf{4.999} 
        %& 4.979 & 4.980 & 4.967 & 4.975    \\
		\midrule
		\multirow{4}{*}{3D structures} 
		& RMSD ($\downarrow$) & 
        %& \textbf{0.419} 
        & 0.907 & 0.906 & 0.897 & \textbf{0.881}    \\
		& JS. bond lengths ($\downarrow$) & 
        %& \textbf{0.286} 
        & 0.457 & 0.477 & 0.436 & \textbf{0.428}    \\
		& JS. bond angles ($\downarrow$) & 
        %& \textbf{0.078} 
        & 0.269 & 0.289 & \textbf{0.186} & 0.200    \\
		& JS. dihedral angles ($\downarrow$) & 
        %& \textbf{0.151} 
        & 0.199 & 0.209 & \textbf{0.168} & 0.170    \\
		\midrule
		\multirow{5}{*}{2D structures} 
		& JS. \#bonds per atoms ($\downarrow$) & 
        %& 0.325 
        & 0.291 & 0.331 & \textbf{0.176} & 0.181    \\
		& JS. basic bond types ($\downarrow$) & 
        %& \textbf{0.055} 
        & \textbf{0.071} & 0.083 & 0.181 & 0.191    \\
		%& JS. freq. bond types ($\downarrow$) & 
        %& \textbf{0.089} 
        %& 0.123 & 0.130 & 0.245 & 0.254    \\
		%& JS. freq. bond pairs ($\downarrow$) & 
        %& \textbf{0.078} 
        %& 0.085 & 0.089 & 0.209 & 0.221    \\
		%& JS. freq. bond triplets ($\downarrow$) & 
        %& \textbf{0.089} 
        %& 0.097 & 0.114 & 0.211 & 0.223    \\
		%\midrule
		%\multirow{3}{*}{Rings} 
		& JS. \#rings ($\downarrow$) & 
        %& 0.142 
        & 0.280 & 0.330 & \textbf{0.043} & 0.049    \\
		& JS. \#n-sized rings ($\downarrow$) & 
        %& \textbf{0.055} 
        & \textbf{0.077} & 0.091 & 0.099 & 0.112    \\
		& \#Intersecting rings ($\uparrow$) & 
        %& \textbf{6} 
        & \textbf{6} & 5 & 4 & 5    \\
		%\method (+bt)            & 100.0 & 98.0 & 100.0 & 0.742 & 0.772 (0.040) & 0.211 (0.033) & & 0.862 (0.036) & 0.211 (0.033) & 0.743 (0.043) \\
		%\methodwithguide (+bt)    & 99.8 & 98.0 & 100.0 & 0.736 & 0.814 (0.031) & 0.193 (0.042) & & 0.895 (0.029) & 0.193 (0.042) & 0.745 (0.045) \\
		%
		\bottomrule
	\end{tabular}%
	\begin{tablenotes}[normal,flushleft]
		\begin{footnotesize}
	\item 
\!\!Rows represent:  {``atom stability'': the proportion of stable atoms that have the correct valency; 
		``molecule stability'': the proportion of generated molecules with all atoms stable;
		%``QED'': the drug-likeness score;
		%``SA'': the synthesizability score;
		%``Lipinski'': the Lipinski 
		``RMSD'': the root mean square deviation (RMSD) between the generated 3D structures of molecules and their optimal conformations; % identified via energy minimization;
		``JS. bond lengths/bond angles/dihedral angles'': the Jensen-Shannon (JS) divergences of bond lengths, bond angles and dihedral angles;
		``JS. \#bonds per atom/basic bond types/\#rings/\#n-sized rings'': the JS divergences of bond counts per atom, basic bond types, counts of all rings, and counts of n-sized rings;
		%``JS. \#rings/\#n-sized rings'': the JS divergences of the total counts of rings and the counts of n-sized rings;
		``\#Intersecting rings'': the number of rings observed in the top-10 frequent rings of both generated and real molecules. } \par
		\par
		\end{footnotesize}
	\end{tablenotes}
\end{threeparttable}
\end{scriptsize}
\end{table*}

%\label{tbl:overall_quality05}

\begin{table*}[!h]
	\centering
		\caption{Comparison on Quality of Generated Desirable Molecules between \method and \squid ($\delta_g$=0.7)}
	\label{tbl:overall_results_quality_07}
	\begin{scriptsize}
\begin{threeparttable}
	\begin{tabular}{
		@{\hspace{0pt}}l@{\hspace{14pt}}
		@{\hspace{0pt}}l@{\hspace{2pt}}
		%
		@{\hspace{4pt}}c@{\hspace{4pt}}
		%
		%@{\hspace{3pt}}c@{\hspace{3pt}}
		@{\hspace{3pt}}c@{\hspace{3pt}}
		@{\hspace{3pt}}c@{\hspace{3pt}}
		@{\hspace{3pt}}c@{\hspace{3pt}}
		@{\hspace{3pt}}c@{\hspace{3pt}}
		%
		%
		}
		\toprule
		group & metric & 
        %& \dataset 
        & \squid ($\lambda$=0.3) & \squid ($\lambda$=1.0)  &  \method & \methodwithsguide  \\
		%\multirow{2}{*}{method} & \multirow{2}{*}{\#c\%} &  \multirow{2}{*}{\#u\%} &  \multirow{2}{*}{QED} & \multicolumn{3}{c}{$\nmax=50$} & & \multicolumn{2}{c}{$\nmax=1$}\\
		%\cmidrule(r){5-7} \cmidrule(r){8-10} 
		%& & & & \avgshapesim(std) & \avggraphsim(std  &  \diversity(std  & & \avgshapesim(std) & \avggraphsim(std \\
		\midrule
		\multirow{2}{*}{stability} 
		& atom stability ($\uparrow$) & 
        %&  0.990 
        & \textbf{0.995} & 0.995 & 0.992 & 0.988 \\
		& molecule stability ($\uparrow$) & 
        %& 0.876 
        & 0.944 & \textbf{0.947} & 0.885 & 0.839 \\
		\midrule
		%\multirow{3}{*}{Drug-likeness} 
		%& QED ($\uparrow$) & 
        %& \textbf{0.805} 
        %& 0.766 & 0.760 & 0.755 & 0.751    \\
	%	& SA ($\uparrow$) & 
        %& \textbf{0.874} 
        %& 0.814 & 0.813 & 0.699 & 0.692    \\
	%	& Lipinski ($\uparrow$) & 
        %& \textbf{4.999} 
        %& 4.979 & 4.980 & 4.967 & 4.975    \\
	%	\midrule
		\multirow{4}{*}{3D structures} 
		& RMSD ($\downarrow$) & 
        %& \textbf{0.420} 
        & 0.897 & 0.906 & 0.897 & \textbf{0.881}    \\
		& JS. bond lengths ($\downarrow$) & 
        %& \textbf{0.286} 
        & 0.457 & 0.477 & 0.436 & \textbf{0.428}    \\
		& JS. bond angles ($\downarrow$) & 
        %& \textbf{0.078} 
        & 0.269 & 0.289 & \textbf{0.186} & 0.200    \\
		& JS. dihedral angles ($\downarrow$) & 
        %& \textbf{0.151} 
        & 0.199 & 0.209 & \textbf{0.168} & 0.170    \\
		\midrule
		\multirow{5}{*}{2D structures} 
		& JS. \#bonds per atoms ($\downarrow$) & 
        %& 0.325 
        & 0.285 & 0.329 & \textbf{0.176} & 0.181    \\
		& JS. basic bond types ($\downarrow$) & 
        %& \textbf{0.055} 
        & \textbf{0.067} & 0.083 & 0.181 & 0.191    \\
	%	& JS. freq. bond types ($\downarrow$) & 
        %& \textbf{0.089} 
        %& 0.123 & 0.130 & 0.245 & 0.254    \\
	%	& JS. freq. bond pairs ($\downarrow$) & 
        %& \textbf{0.078} 
        %& 0.085 & 0.089 & 0.209 & 0.221    \\
	%	& JS. freq. bond triplets ($\downarrow$) & 
        %& \textbf{0.089} 
        %& 0.097 & 0.114 & 0.211 & 0.223    \\
	%	\midrule
	%	\multirow{3}{*}{Rings} 
		& JS. \#rings ($\downarrow$) & 
        %& 0.143 
        & 0.273 & 0.328 & \textbf{0.043} & 0.049    \\
		& JS. \#n-sized rings ($\downarrow$) & 
        %& \textbf{0.055} 
        & \textbf{0.076} & 0.091 & 0.099 & 0.112    \\
		& \#Intersecting rings ($\uparrow$) & 
        %& \textbf{6} 
        & \textbf{6} & 5 & 4 & 5    \\
		%\method (+bt)            & 100.0 & 98.0 & 100.0 & 0.742 & 0.772 (0.040) & 0.211 (0.033) & & 0.862 (0.036) & 0.211 (0.033) & 0.743 (0.043) \\
		%\methodwithguide (+bt)    & 99.8 & 98.0 & 100.0 & 0.736 & 0.814 (0.031) & 0.193 (0.042) & & 0.895 (0.029) & 0.193 (0.042) & 0.745 (0.045) \\
		%
		\bottomrule
	\end{tabular}%
	\begin{tablenotes}[normal,flushleft]
		\begin{footnotesize}
	\item 
\!\!Rows represent:  {``atom stability'': the proportion of stable atoms that have the correct valency; 
		``molecule stability'': the proportion of generated molecules with all atoms stable;
		%``QED'': the drug-likeness score;
		%``SA'': the synthesizability score;
		%``Lipinski'': the Lipinski 
		``RMSD'': the root mean square deviation (RMSD) between the generated 3D structures of molecules and their optimal conformations; % identified via energy minimization;
		``JS. bond lengths/bond angles/dihedral angles'': the Jensen-Shannon (JS) divergences of bond lengths, bond angles and dihedral angles;
		``JS. \#bonds per atom/basic bond types/\#rings/\#n-sized rings'': the JS divergences of bond counts per atom, basic bond types, counts of all rings, and counts of n-sized rings;
		%``JS. \#rings/\#n-sized rings'': the JS divergences of the total counts of rings and the counts of n-sized rings;
		``\#Intersecting rings'': the number of rings observed in the top-10 frequent rings of both generated and real molecules. } \par
		\par
		\end{footnotesize}
	\end{tablenotes}
\end{threeparttable}
\end{scriptsize}
\end{table*}

%\label{tbl:overall_quality07}

\begin{table*}[!h]
	\centering
		\caption{Comparison on Quality of Generated Desirable Molecules between \method and \squid ($\delta_g$=1.0)}
	\label{tbl:overall_results_quality_10}
	\begin{scriptsize}
\begin{threeparttable}
	\begin{tabular}{
		@{\hspace{0pt}}l@{\hspace{14pt}}
		@{\hspace{0pt}}l@{\hspace{2pt}}
		%
		@{\hspace{4pt}}c@{\hspace{4pt}}
		%
		%@{\hspace{3pt}}c@{\hspace{3pt}}
		@{\hspace{3pt}}c@{\hspace{3pt}}
		@{\hspace{3pt}}c@{\hspace{3pt}}
		@{\hspace{3pt}}c@{\hspace{3pt}}
		@{\hspace{3pt}}c@{\hspace{3pt}}
		%
		%
		}
		\toprule
		group & metric & 
        %& \dataset 
        & \squid ($\lambda$=0.3) & \squid ($\lambda$=1.0)  &  \method & \methodwithsguide \\
		%\multirow{2}{*}{method} & \multirow{2}{*}{\#c\%} &  \multirow{2}{*}{\#u\%} &  \multirow{2}{*}{QED} & \multicolumn{3}{c}{$\nmax=50$} & & \multicolumn{2}{c}{$\nmax=1$}\\
		%\cmidrule(r){5-7} \cmidrule(r){8-10} 
		%& & & & \avgshapesim(std) & \avggraphsim(std  &  \diversity(std  & & \avgshapesim(std) & \avggraphsim(std \\
		\midrule
		\multirow{2}{*}{stability}
		& atom stability ($\uparrow$) & 
        %& 0.990 
        & \textbf{0.995} & \textbf{0.995} & 0.992 & 0.988     \\
		& mol stability ($\uparrow$) & 
        %& 0.876 
        & 0.942 & \textbf{0.947} & 0.885 & 0.839    \\
		\midrule
	%	\multirow{3}{*}{Drug-likeness} 
	%	& QED ($\uparrow$) & 
        %& \textbf{0.805} 
        %& \textbf{0.766} & 0.760 & 0.755 & 0.751    \\
	%	& SA ($\uparrow$) & 
        %& \textbf{0.874} 
        %& \textbf{0.813} & \textbf{0.813} & 0.699 & 0.692    \\
	%	& Lipinski ($\uparrow$) & 
        %& \textbf{4.999} 
        %& 4.979 & \textbf{4.980} & 4.967 & 4.975    \\
	%	\midrule
		\multirow{4}{*}{3D structures} 
		& RMSD ($\downarrow$) & 
        %& \textbf{0.420} 
        & 0.898 & 0.906 & 0.897 & \textbf{0.881}    \\
		& JS. bond lengths ($\downarrow$) & 
        %& \textbf{0.286} 
        & 0.457 & 0.477 & 0.436 & \textbf{0.428}    \\
		& JS. bond angles ($\downarrow$) & 
        %& \textbf{0.078} 
        & 0.269 & 0.289 & \textbf{0.186} & 0.200   \\
		& JS. dihedral angles ($\downarrow$) & 
        %& \textbf{0.151} 
        & 0.199 & 0.209 & \textbf{0.168} & 0.170    \\
		\midrule
		\multirow{5}{*}{2D structures} 
		& JS. \#bonds per atoms ($\downarrow$) & 
        %& 0.325 
        & 0.280 & 0.330 & \textbf{0.176} & 0.181    \\
		& JS. basic bond types ($\downarrow$) & 
        %& \textbf{0.055} 
        & \textbf{0.066} & 0.083 & 0.181 & 0.191   \\
	%	& JS. freq. bond types ($\downarrow$) & 
        %& \textbf{0.089} 
        %& \textbf{0.123} & 0.130 & 0.245 & 0.254    \\
	%	& JS. freq. bond pairs ($\downarrow$) & 
        %& \textbf{0.078} 
        %& \textbf{0.085} & 0.089 & 0.209 & 0.221    \\
	%	& JS. freq. bond triplets ($\downarrow$) & 
        %& \textbf{0.089} 
        %& \textbf{0.097} & 0.114 & 0.211 & 0.223    \\
		%\midrule
		%\multirow{3}{*}{Rings} 
		& JS. \#rings ($\downarrow$) & 
        %& 0.143 
        & 0.269 & 0.328 & \textbf{0.043} & 0.049    \\
		& JS. \#n-sized rings ($\downarrow$) & 
        %& \textbf{0.055} 
        & \textbf{0.075} & 0.091 & 0.099 & 0.112    \\
		& \#Intersecting rings ($\uparrow$) & 
        %& \textbf{6} 
        & \textbf{6} & 5 & 4 & 5    \\
		%\method (+bt)            & 100.0 & 98.0 & 100.0 & 0.742 & 0.772 (0.040) & 0.211 (0.033) & & 0.862 (0.036) & 0.211 (0.033) & 0.743 (0.043) \\
		%\methodwithguide (+bt)    & 99.8 & 98.0 & 100.0 & 0.736 & 0.814 (0.031) & 0.193 (0.042) & & 0.895 (0.029) & 0.193 (0.042) & 0.745 (0.045) \\
		%
		\bottomrule
	\end{tabular}%
	\begin{tablenotes}[normal,flushleft]
		\begin{footnotesize}
	\item 
\!\!Rows represent:  {``atom stability'': the proportion of stable atoms that have the correct valency; 
		``molecule stability'': the proportion of generated molecules with all atoms stable;
		%``QED'': the drug-likeness score;
		%``SA'': the synthesizability score;
		%``Lipinski'': the Lipinski 
		``RMSD'': the root mean square deviation (RMSD) between the generated 3D structures of molecules and their optimal conformations; % identified via energy minimization;
		``JS. bond lengths/bond angles/dihedral angles'': the Jensen-Shannon (JS) divergences of bond lengths, bond angles and dihedral angles;
		``JS. \#bonds per atom/basic bond types/\#rings/\#n-sized rings'': the JS divergences of bond counts per atom, basic bond types, counts of all rings, and counts of n-sized rings;
		%``JS. \#rings/\#n-sized rings'': the JS divergences of the total counts of rings and the counts of n-sized rings;
		``\#Intersecting rings'': the number of rings observed in the top-10 frequent rings of both generated and real molecules. } \par
		\par
		\end{footnotesize}
	\end{tablenotes}
\end{threeparttable}
\end{scriptsize}
\end{table*}

%\label{tbl:overall_quality10}

Similar to Table~\ref{tbl:overall_results_quality_desired} in the main manuscript, we present the performance comparison on the quality of desirable molecules generated by different methods under different graph similarity constraints $\delta_g$=0.5, 0.7 and 1.0, as detailed in Table~\ref{tbl:overall_results_quality_05}, Table~\ref{tbl:overall_results_quality_07}, and Table~\ref{tbl:overall_results_quality_10}, respectively.
%
Overall, these tables show that under varying graph similarity constraints, \method and \methodwithsguide can always generate desirable molecules with comparable quality to baselines in terms of stability, 3D structures, and 2D structures.
%
These results demonstrate the strong effectiveness of \method and \methodwithsguide in generating high-quality desirable molecules with stable and realistic structures in both 2D and 3D.
%
This enables the high utility of \method and \methodwithsguide in discovering promising drug candidates.


\begin{comment}
The results across these tables demonstrate similar observations with those under $\delta_g$=0.3 in Table~\ref{tbl:overall_results_quality_desired}.
%
For stability, when $\delta_g$=0.5, 0.7 or 1.0, \method and \methodwithsguide achieve comparable performance or fall slightly behind \squid ($\lambda$=0.3) and \squid ($\lambda$=1.0) in atom stability and molecule stability.
%
For example, when $\delta_g$=0.5, as shown in Table~\ref{tbl:overall_results_quality_05}, \method achieves similar performance with the best baseline \squid ($\lambda$=0.3) in atom stability (0.992 for \method vs 0.996 for \squid with $\lambda$=0.3).
%
\method underperforms \squid ($\lambda$=0.3) in terms of molecule stability.
%
For 3D structures, \method and \methodwithsguide also consistently generate molecules with more realistic 3D structures compared to \squid.
%
Particularly, \methodwithsguide achieves the best performance in RMSD and JS of bond lengths across $\delta_g$=0.5, 0.7 and 1.0.
%
In JS of dihedral angles, \method achieves the best performance among all the methods.
%
\method and \methodwithsguide underperform \squid in JS of bond angles, primarily because \squid constrains the bond angles in the generated molecules.
%
For 2D structures, \method and \methodwithsguide again achieve the best performance 
\end{comment}

%===================================================================
\section{Additional Experimental Results on PMG}
\label{supp:app:results_PMG}
%===================================================================

%\label{tbl:comparison_results_decompdiff}


%-------------------------------------------------------------------------------------------------------------------------------------
%\subsection{{Additional Comparison for PMG}}
%\label{supp:app:results:docking}
%-------------------------------------------------------------------------------------------------------------------------------------

In this section, we present the results of \methodwithpguide and \methodwithsandpguide when generating 100 molecules. 
%
Please note that both \methodwithpguide and \methodwithsandpguide show remarkable efficiency over the PMG baselines.
%
\methodwithpguide and \methodwithsandpguide generate 100 molecules in 48 and 58 seconds on average, respectively, while the most efficient baseline \targetdiff requires 1,252 seconds.
%
We report the performance of \methodwithpguide and \methodwithsandpguide against state-of-the-art PMG baselines in Table~\ref{tbl:overall_results_docking_100}.


%
According to Table~\ref{tbl:overall_results_docking_100}, \methodwithpguide and \methodwithsandpguide achieve comparable performance with the PMG baselines in generating molecules with high binding affinities.
%
Particularly, in terms of Vina S, \methodwithsandpguide achieves very comparable performance (-4.56 kcal/mol) to the third-best baseline \decompdiff (-4.58 kcal/mol) in average Vina S; it also achieves the third-best performance (-4.82 kcal/mol) among all the methods and slightly underperforms the second-best baseline \AR (-4.99 kcal/mol) in median Vina S
%
\methodwithsandpguide also achieves very close average Vina M (-5.53 kcal/mol) with the third-best baseline \AR (-5.59 kcal/mol) and the third-best performance (-5.47 kcal/mol) in median Vina M.
%
Notably, for Vina D, \methodwithpguide and \methodwithsandpguide achieve the second and third performance among all the methods.
%
In terms of the average percentage of generated molecules with Vina D higher than those of known ligands (i.e., HA), \methodwithpguide (58.52\%) and \methodwithsandpguide (58.28\%) outperform the best baseline \targetdiff (57.57\%).
%
These results signify the high utility of \methodwithpguide and \methodwithsandpguide in generating molecules that effectively bind with protein targets and have better binding affinities than known ligands.

In addition to binding affinities, \methodwithpguide and \methodwithsandpguide also demonstrate similar performance compared to the baselines in metrics related to drug-likeness and diversity.
%
For drug-likeness, both \methodwithpguide and \methodwithsandpguide achieve the best (0.67) and the second-best (0.66) QED scores.
%
They also achieve the third and fourth performance in SA scores.
%
In terms of the diversity among generated molecules,  \methodwithpguide and \methodwithsandpguide slightly underperform the baselines, possibly due to the design that generates molecules with similar shapes to the ligands.
%
These results highlight the strong ability of \methodwithpguide and \methodwithsandpguide in efficiently generating effective binding molecules with favorable drug-likeness and diversity.
%
This ability enables them to potentially serve as promising tools to facilitate effective and efficient drug development.

\begin{table*}[!h]
	\centering
		\caption{Additional Comparison on PMG When All Methods Generate 100 Molecules}
	\label{tbl:overall_results_docking_100}
\begin{threeparttable}
	\begin{scriptsize}
	\begin{tabular}{
		@{\hspace{2pt}}l@{\hspace{2pt}}
		%
		@{\hspace{2pt}}r@{\hspace{2pt}}
		%
		@{\hspace{2pt}}r@{\hspace{2pt}}
		@{\hspace{2pt}}r@{\hspace{2pt}}
		%
		@{\hspace{6pt}}r@{\hspace{6pt}}
		%
		@{\hspace{2pt}}r@{\hspace{2pt}}
		@{\hspace{2pt}}r@{\hspace{2pt}}
		%
		@{\hspace{5pt}}r@{\hspace{5pt}}
		%
		@{\hspace{2pt}}r@{\hspace{2pt}}
		@{\hspace{2pt}}r@{\hspace{2pt}}
		%
		@{\hspace{5pt}}r@{\hspace{5pt}}
		%
		@{\hspace{2pt}}r@{\hspace{2pt}}
	         @{\hspace{2pt}}r@{\hspace{2pt}}
		%
		@{\hspace{5pt}}r@{\hspace{5pt}}
		%
		@{\hspace{2pt}}r@{\hspace{2pt}}
		@{\hspace{2pt}}r@{\hspace{2pt}}
		%
		@{\hspace{5pt}}r@{\hspace{5pt}}
		%
		@{\hspace{2pt}}r@{\hspace{2pt}}
		@{\hspace{2pt}}r@{\hspace{2pt}}
		%
		@{\hspace{5pt}}r@{\hspace{5pt}}
		%
		@{\hspace{2pt}}r@{\hspace{2pt}}
		@{\hspace{2pt}}r@{\hspace{2pt}}
		%
		@{\hspace{5pt}}r@{\hspace{5pt}}
		%
		@{\hspace{2pt}}r@{\hspace{2pt}}
		%@{\hspace{2pt}}r@{\hspace{2pt}}
		%@{\hspace{2pt}}r@{\hspace{2pt}}
		}
		\toprule
		\multirow{2}{*}{method} & \multicolumn{2}{c}{Vina S$\downarrow$} & & \multicolumn{2}{c}{Vina M$\downarrow$} & & \multicolumn{2}{c}{Vina D$\downarrow$} & & \multicolumn{2}{c}{{HA\%$\uparrow$}}  & & \multicolumn{2}{c}{QED$\uparrow$} & & \multicolumn{2}{c}{SA$\uparrow$} & & \multicolumn{2}{c}{Div$\uparrow$} & %& \multirow{2}{*}{SR\%$\uparrow$} & 
		& \multirow{2}{*}{time$\downarrow$} \\
	    \cmidrule{2-3}\cmidrule{5-6} \cmidrule{8-9} \cmidrule{11-12} \cmidrule{14-15} \cmidrule{17-18} \cmidrule{20-21}
		 & Avg. & Med. &  & Avg. & Med. &  & Avg. & Med. & & Avg. & Med.  & & Avg. & Med.  & & Avg. & Med.  & & Avg. & Med.  & & \\ %& & \\
		%\multirow{2}{*}{method} & \multirow{2}{*}{\#c\%} &  \multirow{2}{*}{\#u\%} &  \multirow{2}{*}{QED} & \multicolumn{3}{c}{$\nmax=50$} & & \multicolumn{2}{c}{$\nmax=1$}\\
		%\cmidrule(r){5-7} \cmidrule(r){8-10} 
		%& & & & \avgshapesim(std) & \avggraphsim(std  &  \diversity(std  & & \avgshapesim(std) & \avggraphsim(std \\
		\midrule
		Reference                          & -5.32 & -5.66 & & -5.78 & -5.76 & & -6.63 & -6.67 & & - & - & & 0.53 & 0.49 & & 0.77 & 0.77 & & - & - & %& 23.1 & 
		& - \\
		\midrule
		\AR & \textbf{-5.06} & -4.99 & &  -5.59 & -5.29 & &  -6.16 & -6.05 & &  37.69 & 31.00 & &  0.50 & 0.49 & &  0.66 & 0.65 & & 0.70 & 0.70 & %& 7.0 & 
		& 7,789 \\
		\pockettwomol   & -4.50 & -4.21 & &  -5.70 & -5.27 & &  -6.43 & -6.25 & &  48.00 & 51.00 & &  0.58 & 0.58 & &  \textbf{0.77} & \textbf{0.78} & &  0.69 & 0.71 &  %& 24.9 & 
		& 2,150 \\
		\targetdiff     & -4.88 & \textbf{-5.82} & &  \textbf{-6.20} & \textbf{-6.36} & &  \textbf{-7.37} & \textbf{-7.51} & &  57.57 & 58.27 & &  0.50 & 0.51 & &  0.60 & 0.59 & &  \textbf{0.72} & 0.71 & % & 10.4 & 
		& 1,252 \\
		%& \decompdiffbeta                    & 63.03 & %-4.72 & -4.86 & & \textbf{-6.84} & \textbf{-6.91} & & \textbf{-8.85} & \textbf{-8.90} & &  \textbf{72.16} & \textbf{72.16} & &  0.36 & 0.36 & &  0.55 & 0.55 & & 0.59 & 0.59 & & 14.9 \\ 
		%-4.76 & -6.18 & &  \textbf{-6.86} & \textbf{-7.52} & &  \textbf{-8.85} & \textbf{-8.96} & &  \textbf{72.7} & \textbf{89.8} & &  0.36 & 0.34 & &  0.55 & 0.57 & & 0.59 & 0.59 & & 15.4 \\
		\decompdiffref  & -4.58 & -4.77 & &  -5.47 & -5.51 & &  -6.43 & -6.56 & &  47.76 & 48.66 & &  0.56 & 0.56 & &  0.70 & 0.69  & &  \textbf{0.72} & \textbf{0.72} &  %& 15.2 & 
		& 1,859 \\
		\midrule
		%\method & 14.04 & 9.74 & &  -2.80 & -3.87 & &  -6.32 & -6.41 & &  42.37 & 40.40 & &  0.70 & 0.71 & &  0.73 & 0.72 & & 0.71 & 0.74 & & 42 \\
		%\methodwithsguide & 1.04 & -0.33 & &  -4.23 & -4.39 & &  -6.31 & -6.46 & &  46.18 & 44.00 & &  0.69 & 0.71 & &  0.72 & 0.71 & & 0.70 & 0.73 & 53 \\
		\methodwithpguide      & -4.15 & -4.59 & &  -5.41 & -5.34 & &  -6.49 & -6.74 & &  \textbf{58.52} & 59.00 & &  \textbf{0.67} & \textbf{0.69} & &  0.68 & 0.68 & & 0.67 & 0.70 & %& 28.0 & 
		& 48 \\
		\methodwithsandpguide  & -4.56 & -4.82 & &  -5.53 & -5.47 & &  -6.60 & -6.78 & &  58.28 & \textbf{60.00} & &  0.66 & 0.68 & &  0.67 & 0.66 & & 0.68 & 0.71 &
		& 58 \\
		\bottomrule
	\end{tabular}%
	\begin{tablenotes}[normal,flushleft]
		\begin{footnotesize}
	\item 
\!\!Columns represent: {``Vina S'': the binding affinities between the initially generated poses of molecules and the protein pockets; 
		``Vina M'': the binding affinities between the poses after local structure minimization and the protein pockets;
		``Vina D'': the binding affinities between the poses determined by AutoDock Vina~\cite{Eberhardt2021} and the protein pockets;
		``HA'': the percentage of generated molecules with Vina D higher than those of condition molecules;
		``QED'': the drug-likeness score;
		``SA'': the synthesizability score;
		``Div'': the diversity among generated molecules;
		``time'': the time cost to generate molecules.}
		\par
		\par
		\end{footnotesize}
	\end{tablenotes}
	\end{scriptsize}
\end{threeparttable}
\end{table*}


%\label{tbl:overall_results_docking_100}

%-------------------------------------------------------------------------------------------------------------------------------------
%\subsection{{Comparison of Pocket Guidance}}
%\label{supp:app:results:docking}
%-------------------------------------------------------------------------------------------------------------------------------------


\begin{comment}
%-------------------------------------------------------------------------------------------------------------------------------------
\subsection{\ziqi{Simiarity Comparison for Pocket-based Molecule Generation}}
%-------------------------------------------------------------------------------------------------------------------------------------


\begin{table*}[t!]
	\centering
	\caption{{Overall Comparison on Similarity for Pocket-based Molecule Generation}}
	\label{tbl:docking_results_similarity}
	\begin{small}
		\begin{threeparttable}
			\begin{tabular}{
					@{\hspace{0pt}}l@{\hspace{5pt}}
					%
					@{\hspace{3pt}}l@{\hspace{3pt}}
					%
					@{\hspace{3pt}}r@{\hspace{8pt}}
					@{\hspace{3pt}}c@{\hspace{3pt}}
					%
					@{\hspace{3pt}}c@{\hspace{3pt}}
					@{\hspace{3pt}}c@{\hspace{3pt}}
					%
					@{\hspace{0pt}}c@{\hspace{0pt}}
					%
					@{\hspace{3pt}}c@{\hspace{3pt}}
					@{\hspace{3pt}}c@{\hspace{3pt}}
					%
					@{\hspace{3pt}}r@{\hspace{3pt}}
				}
				\toprule
				$\delta_g$  & method          & \#d\%$\uparrow$ & $\diversity_d$$\uparrow$(std) & \avgshapesim$\uparrow$(std) & \avggraphsim$\downarrow$(std) & & \maxshapesim$\uparrow$(std) & \maxgraphsim$\downarrow$(std)       & \#n\%$\uparrow$  \\ 
				\midrule
				%\multirow{6}{0.059\linewidth}{\hspace{0pt}0.1} 
				%& \AR   & 4.4 & 0.781(0.076) & 0.511(0.197) & \textbf{0.056}(0.020) & & 0.619(0.222) & 0.074(0.024) & 21.4  \\
				%& \pockettwomol & 6.6 & 0.795(0.099) & 0.519(0.216) & 0.063(0.020) & & 0.608(0.236) & 0.076(0.022) & \textbf{24.1}  \\
				%& \targetdiff & 2.0 & 0.872(0.041) & 0.619(0.110) & 0.068(0.018) & & 0.721(0.146) & 0.075(0.023) & 17.7  \\
				%& \decompdiffbeta & 0.0 & - & 0.374(0.138) & 0.059(0.031) & & 0.414(0.141) & \textbf{0.058}(0.032) & 9.8  \\
				%& \decompdiffref & 8.5 & 0.805(0.096) & 0.810(0.070) & 0.076(0.018) & & 0.861(0.085) & 0.076(0.020) & 11.3  \\
				%& \methodwithpguide   &  9.9 & \textbf{0.876}(0.041) & 0.795(0.058) & 0.073(0.015) & & 0.869(0.073) & 0.076(0.020) & 17.7  \\
				%& \methodwithsandpguide & \textbf{11.9} & 0.872(0.036) & \textbf{0.813}(0.052) & 0.075(0.014) & & \textbf{0.874}(0.069) & 0.080(0.014) & 17.0  \\
				%\cmidrule{2-10}
				%& improv\% & 40.4$^*$ & 8.8$^*$ & 0.4 & -30.4$^*$ &  & 1.6 & -30.0$^*$ & -26.3$^*$  \\
				%\midrule
				\multirow{7}{0.059\linewidth}{\hspace{0pt}1.0} 
				& \AR & 14.6 & 0.681(0.163) & 0.644(0.119) & 0.236(0.123) & & 0.780(0.110) & 0.284(0.177) & 95.8  \\
				& \pockettwomol & 18.6 & 0.711(0.152) & 0.654(0.131) &   \textbf{0.217}(0.129) & & 0.778(0.121) &   \textbf{0.243}(0.137) &  \textbf{98.3}  \\
				& \targetdiff & 7.1 &  \textbf{0.785}(0.085) & 0.622(0.083) & 0.238(0.122) & & 0.790(0.102) & 0.274(0.158) & 90.4  \\
				%& \decompdiffbeta & 0.1 & 0.589(0.030) & 0.494(0.124) & 0.263(0.143) & & 0.567(0.143) & 0.275(0.162) & 67.7  \\
				& \decompdiffref & 37.3 & 0.721(0.108) & 0.770(0.087) & 0.282(0.130) & & \textbf{0.878}(0.059) & 0.343(0.174) & 83.7  \\
				& \methodwithpguide   &  27.4 & 0.757(0.134) & 0.747(0.078) & 0.265(0.165) & & 0.841(0.081) & 0.272(0.168) & 98.1  \\
				& \methodwithsandpguide &\textbf{45.2} & 0.724(0.142) &   \textbf{0.789}(0.063) & 0.265(0.162) & & 0.876(0.062) & 0.264(0.159) & 97.8  \\
				\cmidrule{2-10}
				& Improv\%  & 21.2$^*$ & -3.6 & 2.5$^*$ & -21.7$^*$ &  & -0.1 & -8.4$^*$ & -0.2  \\
				\midrule
				\multirow{7}{0.059\linewidth}{\hspace{0pt}0.7} 
				& \AR   & 14.5 & 0.692(0.151) & 0.644(0.119) & 0.233(0.116) & & 0.779(0.110) & 0.266(0.140) & 94.9  \\
				& \pockettwomol & 18.6 & 0.711(0.152) & 0.654(0.131) & \textbf{0.217}(0.129) & & 0.778(0.121) & \textbf{0.243}(0.137) & \textbf{98.2}  \\
				& \targetdiff & 7.1 & \textbf{0.786}(0.084) & 0.622(0.083) & 0.238(0.121) & & 0.790(0.101) & 0.270(0.151) & 90.3  \\
				%& \decompdiffbeta & 0.1 & 0.589(0.030) & 0.494(0.124) & 0.263(0.142) & &0.567(0.143) & 0.273(0.156) & 67.6  \\
				& \decompdiffref & 36.2 & 0.721(0.113) & 0.770(0.086) & 0.273(0.123) & & \textbf{0.876}(0.059) & 0.325(0.139) & 82.3  \\
				& \methodwithpguide   &  27.4 & 0.757(0.134) & 0.746(0.078) & 0.263(0.160) & & 0.841(0.081) & 0.271(0.164) & 96.8  \\
				& \methodwithsandpguide      & \textbf{45.0} & 0.732(0.129) & \textbf{0.789}(0.063) & 0.262(0.157) & & \textbf{0.876}(0.063) & 0.262(0.153) & 96.2  \\
				\cmidrule{2-10}
				& Improv\%  & 24.3$^*$ & -3.6 & 2.5$^*$ & -20.8$^*$ &  & 0.0 & -7.6$^*$ & -1.5  \\
				\midrule
				\multirow{7}{0.059\linewidth}{\hspace{0pt}0.5} 
				& \AR   & 14.1 & 0.687(0.160) & 0.639(0.124) & 0.218(0.097) & & 0.778(0.110) & 0.260(0.130) & 89.8  \\
				& \pockettwomol & 18.5 & 0.711(0.152) & 0.649(0.134) & \textbf{0.209}(0.114) & & 0.777(0.121) & \textbf{0.240}(0.131) & \textbf{93.2}  \\
				& \targetdiff & 7.1 & \textbf{0.786}(0.084) & 0.621(0.083) & 0.230(0.111) & & 0.788(0.105) & 0.254(0.127) & 86.5  \\
				%&\decompdiffbeta & 0.1 & 0.595(0.025) & 0.494(0.124) & 0.254(0.129) & & 0.565(0.142) & 0.259(0.138) & 63.9  \\
				& \decompdiffref & 34.7 & 0.730(0.105) & 0.769(0.086) & 0.261(0.109) & & 0.874(0.080) & 0.301(0.117) & 77.3   \\
				& \methodwithpguide  &  27.2 & 0.765(0.123) & 0.749(0.075) & 0.245(0.135) & & 0.840(0.082) & 0.252(0.137) & 88.6  \\
				& \methodwithsandpguide & \textbf{44.3} & 0.738(0.122) & \textbf{0.791}(0.059) & 0.247(0.132) &  & \textbf{0.875}(0.065) & 0.249(0.130) & 88.8  \\
				\cmidrule{2-10}
				& Improv\%   & 27.8$^*$ & -2.7 & 2.9$^*$ & -17.6$^*$ &  & 0.2 & -3.4 & -4.7$^*$  \\
				\midrule
				\multirow{7}{0.059\linewidth}{\hspace{0pt}0.3} 
				& \AR   & 12.2 & 0.704(0.146) & 0.614(0.146) & 0.164(0.059) & & 0.751(0.138) & 0.206(0.059) & 66.4  \\
				& \pockettwomol & 17.1 & 0.731(0.129) & 0.617(0.163) & \textbf{0.155}(0.056) & & 0.740(0.159) & \textbf{0.190}(0.076) & \textbf{71.0}  \\
				& \targetdiff & 6.2 & \textbf{0.809}(0.061) & 0.619(0.087) & 0.181(0.068) & & 0.768(0.119) & 0.196(0.076) & 61.7  \\				
                %& \decompdiffbeta & 0.0 & - & 0.489(0.124) & 0.195(0.080) & & 0.547(0.139) & 0.203(0.087) & 42.0  \\
				& \decompdiffref & 27.7 & 0.775(0.081) & 0.767(0.086) & 0.202(0.062) & & 0.854(0.093) & 0.216(0.068) & 52.6  \\
				& \methodwithpguide   &  24.4 & 0.805(0.084) & 0.763(0.066) & 0.180(0.074) & & 0.847(0.080) & \textbf{0.190}(0.059) & 61.4  \\
				& \methodwithsandpguide & \textbf{36.3} & 0.789(0.081) & \textbf{0.800}(0.056) & 0.181(0.071) & &\textbf{0.878}(0.067) & \textbf{0.190}(0.078) & 61.8  \\
				\cmidrule{2-10}
				& improv\% & 31.1$^*$ & 3.9$^*$ & 4.3$^*$ & -16.5$^*$ &  & 2.8$^*$ & 0.0 & -12.9$^*$  \\
				\bottomrule
			\end{tabular}%
			\begin{tablenotes}[normal,flushleft]
				\begin{footnotesize}
					\item 
					\!\!Columns represent: \ziqi{``$\delta_g$'': the graph similarity constraint; ``\#n\%'': the percentage of molecules that satisfy the graph similarity constraint ($\graphsim<=\delta_g$);
						``\#d\%'': the percentage of molecules that satisfy the graph similarity constraint and are with high \shapesim ($\shapesim>=0.8$);
						``\avgshapesim/\avggraphsim'': the average of shape or graph similarities between the condition molecules and generated molecules with $\graphsim<=\delta_g$;
						``\maxshapesim'': the maximum of shape similarities between the condition molecules and generated molecules with $\graphsim<=\delta_g$;
						``\maxgraphsim'': the graph similarities between the condition molecules and the molecules with the maximum shape similarities and $\graphsim<=\delta_g$;
						``\diversity'': the diversity among the generated molecules.
						%
						``$\uparrow$'' represents higher values are better, and ``$\downarrow$'' represents lower values are better.
						%
						Best values are in \textbf{bold}, and second-best values are \underline{underlined}. 
					} 
					%\todo{double-check the significance value}
					\par
					\par
				\end{footnotesize}
			\end{tablenotes}
		\end{threeparttable}
	\end{small}
	\vspace{-10pt}    
\end{table*}
%\label{tbl:docking_results_similarity}

\bo{@Ziqi you may want to check my edits for the discussion in Table 1 first.
%
If the pocket if known, do you still care about the shape similarity in real applications?
}

\ziqi{Table~\ref{tbl:docking_results_similarity} presents the overall comparison on similarity-based metrics between \methodwithpguide, \methodwithsandpguide and other baselines under different graph similarity constraints  ($\delta_g$=1.0, 0.7, 0.5, 0.3), similar to Table~\ref{tbl:overall}. 
%
As shown in Table~\ref{tbl:docking_results_similarity}, regarding desirable molecules,  \methodwithsandpguide consistently outperforms all the baseline methods in the likelihood of generating desirable molecules (i.e., $\#d\%$).
%
For example, when $\delta_g$=1.0, at $\#d\%$, \methodwithsandpguide (45.2\%) demonstrates significant improvement of $21.2\%$ compared to the best baseline \decompdiff (37.3\%).
%
In terms of $\diversity_d$, \methodwithpguide and \methodwithsandpguide also achieve the second and the third best performance. 
%
Note that the best baseline \targetdiff in $\diversity_d$ achieves the least percentage of desirable molecules (7.1\%), substantially lower than \methodwithpguide and \methodwithsandpguide.
%
This makes its diversity among desirable molecules incomparable with other methods. 
%
When $\delta_g$=0.7, 0.5, and 0.3, \methodwithsandpguide also establishes a significant improvement of 24.3\%, 27.8\%, and 31.1\% compared to the best baseline method \decompdiff.
%
It is also worth noting that the state-of-the-art baseline \decompdiff underperforms \methodwithpguide and \methodwithsandpguide in binding affinities as shown in Table~\ref{tbl:overall_results_docking}, even though it outperforms \methodwithpguide in \#d\%.
%
\methodwithpguide and \methodwithsandpguide also achieve the second and the third best performance in $\diversity_d$ at $\delta_g$=0.7, 0.5, and 0.3. 
%
The superior performance of \methodwithpguide and \methodwithsandpguide in $\#d\%$ at small $\delta_g$ indicates their strong capacity in generating desirable molecules of novel graph structures, thereby facilitating the discovery of novel drug candidates.
%
}

\ziqi{Apart from the desirable molecules, \methodwithpguide and \methodwithsandpguide also demonstrate outstanding performance in terms of the average shape similarities (\avgshapesim) and the average graph similarities (\avggraphsim).
%
Specifically, when $\delta_g$=1.0, \methodwithsandpguide achieves a significant 2.5\% improvement in \avgshapesim\ over the best baseline \decompdiff. 
%
In terms of \avggraphsim, \methodwithsandpguide also achieves higher performance than the baseline \decompdiff of the highest \avgshapesim (0.265 vs 0.282).
%
Please note that all the baseline methods except \decompdiff achieve substantially lower performance in \avgshapesim than \methodwithpguide and \methodwithsandpguide, even though these methods achieve higher \avggraphsim values.
%
This trend remains consistent when applying various similarity constraints (i.e., $\delta_g$) as shown in Table~\ref{tbl:overall_results_docking}.
}

\ziqi{Similarly, \methodwithpguide and \methodwithsandpguide also achieve superior performance in \maxshapesim and \maxgraphsim.
%
Specifically, when $\delta_g$=1.0, for \maxshapesim, \methodwithsandpguide achieves highly comparable performance in \maxshapesim\ compared to the best baseline \decompdiff (0.876 vs 0.878).
%
We also note that \methodwithsandpguide achieves lower \maxgraphsim\ than the \decompdiff with 23.0\% difference. 
%
When $\delta_g$ gets smaller from 0.7 to 0.3, \methodwithsandpguide maintains a high \maxshapesim value around 0.876, while the best baseline \decompdiff has \maxshapesim decreased from 0.878 to 0.854.
%
This demonstrates the superior ability of \methodwithsandpguide in generating molecules with similar shapes and novel structures.
%
}

\ziqi{
In terms of \#n\%, when $\delta_g$=1.0, the percentage of molecules with \graphsim below $\delta_g$ can be interpreted as the percentage of valid molecules among all the generated molecules. 
%
As shown in Table~\ref{tbl:docking_results_similarity}, \methodwithpguide and \methodwithsandpguide are able to generate 98.1\% and 97.8\% of valid molecules, slightly below the best baseline \pockettwomol (98.3\%). 
%
When $\delta_g$=0.7, 0.5, or 0.3, all the methods, including \methodwithpguide and \methodwithsandpguide, can consistently find a sufficient number of novel molecules that meet the graph similarity constraints.
%
The only exception is \decompdiff, which substantially underperforms all the other methods in \#n\%.
}
\end{comment}

%%%%%%%%%%%%%%%%%%%%%%%%%%%%%%%%%%%%%%%%%%%%%
\section{Properties of Molecules in Case Studies for Targets}
\label{supp:app:results:properties}
%%%%%%%%%%%%%%%%%%%%%%%%%%%%%%%%%%%%%%%%%%%%%

%-------------------------------------------------------------------------------------------------------------------------------------
\subsection{Drug Properties of Generated Molecules}
\label{supp:app:results:properties:drug}
%-------------------------------------------------------------------------------------------------------------------------------------

Table~\ref{tbl:drug_property} presents the drug properties of three generated molecules: NL-001, NL-002, and NL-003.
%
As shown in Table~\ref{tbl:drug_property}, each of these molecules has a favorable profile, making them promising drug candidates. 
%
{As discussed in Section ``Case Studies for Targets'' in the main manuscript, all three molecules have high binding affinities in terms of Vina S, Vina M and Vina D, and favorable QED and SA values.
%
In addition, all of them meet the Lipinski's rule of five criteria~\cite{Lipinski1997}.}
%
In terms of physicochemical properties, all these properties of NL-001, NL-002 and NL-003, including number of rotatable bonds, molecule weight, LogP value, number of hydrogen bond doners and acceptors, and molecule charges, fall within the desired range of drug molecules. 
%
This indicates that these molecules could potentially have good solubility and membrane permeability, essential qualities for effective drug absorption.

These generated molecules also demonstrate promising safety profiles based on the predictions from ICM~\cite{Neves2012}.
%
In terms of drug-induced liver injury prediction scores, all three molecules have low scores (0.188 to 0.376), indicating a minimal risk of hepatotoxicity. 
%
NL-001 and NL-002 fall under `Ambiguous/Less concern' for liver injury, while NL-003 is categorized under 'No concern' for liver injury. 
%
Moreover, all these molecules have low toxicity scores (0.000 to 0.236). 
%
NL-002 and NL-003 do not have any known toxicity-inducing functional groups. 
%
NL-001 and NL-003 are also predicted not to include any known bad groups that lead to inappropriate features.
%
These attributes highlight the potential of NL-001, NL-002, and NL-003 as promising treatments for cancers and Alzheimer’s disease.

%\begin{table*}
	\centering
		\caption{Drug Properties of Generated Molecules}
	\label{tbl:binding_drug_mols}
	\begin{scriptsize}
\begin{threeparttable}
	\begin{tabular}{
		@{\hspace{6pt}}r@{\hspace{6pt}}
		@{\hspace{6pt}}r@{\hspace{6pt}}
		@{\hspace{6pt}}r@{\hspace{6pt}}
		@{\hspace{6pt}}r@{\hspace{6pt}}
		@{\hspace{6pt}}r@{\hspace{6pt}}
		@{\hspace{6pt}}r@{\hspace{6pt}}
		@{\hspace{6pt}}r@{\hspace{6pt}}
		@{\hspace{6pt}}r@{\hspace{6pt}}
		@{\hspace{6pt}}r@{\hspace{6pt}}
		%
		}
		\toprule
Target & Molecule & Vina S & Vina M & Vina D & QED   & SA   & Logp  & Lipinski \\
\midrule
\multirow{3}{*}{CDK6} & NL-001 & -6.817      & -7.251    & -8.319     & 0.834 & 0.72 & 1.313 & 5        \\
& NL-002 & -6.970       & -7.605    & -8.986     & 0.851 & 0.74 & 3.196 & 5        \\
\cmidrule{2-9}
& 4AU & 0.736       & -5.939    & -7.592     & 0.773 & 0.79 & 2.104 & 5        \\
\midrule
\multirow{2}{*}{NEP} & NL-003 & -11.953     & -12.165   & -12.308    & 0.772 & 0.57 & 2.944 & 5        \\
\cmidrule{2-9}
& BIR & -9.399      & -9.505    & -9.561     & 0.463 & 0.73 & 2.677 & 5        \\
		\bottomrule
	\end{tabular}%
	\begin{tablenotes}[normal,flushleft]
		\begin{footnotesize}
	\item Columns represent: {``Target'': the names of protein targets;
		``Molecule'': the names of generated molecules and known ligands;
		``Vina S'': the binding affinities between the initially generated poses of molecules and the protein pockets; 
		``Vina M'': the binding affinities between the poses after local structure minimization and the protein pockets;
		``Vina D'': the binding affinities between the poses determined by AutoDock Vina~\cite{Eberhardt2021} and the protein pockets;
		``HA'': the percentage of generated molecules with Vina D higher than those of condition molecules;
		``QED'': the drug-likeness score;
		``SA'': the synthesizability score;
		``Div'': the diversity among generated molecules;
		``time'': the time cost to generate molecules.}
\!\! \par
		\par
		\end{footnotesize}
	\end{tablenotes}
\end{threeparttable}
\end{scriptsize}
  \vspace{-10pt}    
\end{table*}

%\label{tbl:binding_drug_mols}

\begin{table*}
	\centering
		\caption{Drug Properties of Generated Molecules}
	\label{tbl:drug_property}
	\begin{scriptsize}
\begin{threeparttable}
	\begin{tabular}{
		@{\hspace{0pt}}p{0.23\linewidth}@{\hspace{5pt}}
		%
		@{\hspace{1pt}}r@{\hspace{2pt}}
		@{\hspace{2pt}}r@{\hspace{6pt}}
		@{\hspace{6pt}}r@{\hspace{6pt}}
		%
		}
		\toprule
		Property Name & NL-001 & NL-002 & NL-003 \\
		\midrule
Vina S & -6.817 &  -6.970 & -11.953 \\
Vina M & -7.251 & -7.605 & -12.165 \\
Vina D & -8.319 & -8.986 & -12.308 \\
QED    & 0.834  & 0.851  & 0.772 \\
SA       & 0.72    & 0.74    & 0.57    \\
Lipinski & 5 & 5 & 5 \\
%bbbScore          & 3.386                                                                                        & 4.240                                                                                        & 3.892      \\
%drugLikeness      & -0.081                                                                                       & -0.442                                                                                       & -0.325     \\
%molLogP1          & 1.698                                                                                        & 2.685                                                                                        & 2.382      \\
\#rotatable bonds          & 3                                                                                        & 2                                                                                        & 2      \\
molecule weight         & 267.112                                                                                      & 270.117                                                                                      & 390.206    \\
molecule LogP           & 1.698                                                                                        & 2.685                                                                                        & 2.382     \\
\#hydrogen bond doners           & 1                                                                                        & 1                                                                                        & 2      \\
\#hydrogen bond acceptors           & 5                                                                                       & 3                                                                                        & 5      \\
\#molecule charges   & 1                                                                                        & 0                                                                                        & 0      \\
drug-induced liver injury predScore    & 0.227                                                                                        & 0.376                                                                                        & 0.188      \\
drug-induced liver injury predConcern  & Ambiguous/Less concern                                                                       & Ambiguous/Less concern                                                                       & No concern \\
drug-induced liver injury predLabel    & Warnings/Precautions/Adverse reactions & Warnings/Precautions/Adverse reactions & No match   \\
drug-induced liver injury predSeverity & 2                                                                                        & 3                                                                                        & 2      \\
%molSynth1         & 0.254                                                                                        & 0.220                                                                                        & 0.201      \\
%toxicity class         & 0.480                                                                                        & 0.480                                                                                        & 0.450      \\
toxicity names         & hydrazone                                                                                    &   -                                                                                           &   -         \\
toxicity score         & 0.236                                                                                        & 0.000                                                                                        & 0.000      \\
bad groups         & -                                                                                             & Tetrahydroisoquinoline:   allergies                                                          &   -         \\
%MolCovalent       &                                                                                              &                                                                                              &            \\
%MolProdrug        &                                                                                              &                                                                                              &            \\
		\bottomrule
	\end{tabular}%
	\begin{tablenotes}[normal,flushleft]
		\begin{footnotesize}
	\item ``-'': no results found by algorithms
\!\! \par
		\par
		\end{footnotesize}
	\end{tablenotes}
\end{threeparttable}
\end{scriptsize}
  \vspace{-10pt}    
\end{table*}

%\label{tbl:drug_property}

%-------------------------------------------------------------------------------------------------------------------------------------
\subsection{Comparison on ADMET Profiles between Generated Molecules and Approved Drugs}
\label{supp:app:results:properties:admet}
%-------------------------------------------------------------------------------------------------------------------------------------

\paragraph{Generated Molecules for CDK6}
%
Table~\ref{tbl:admet_cdk6} presents the comparison on ADMET profiles between two generated molecules for CDK6 and the approved CDK6 inhibitors, including Abemaciclib~\cite{Patnaik2016}, Palbociclib~\cite{Lu2015}, and Ribociclib~\cite{Tripathy2017}.
%
As shown in Table~\ref{tbl:admet_cdk6}, the generated molecules, NL-001 and NL-002, exhibit comparable ADMET profiles with those of the approved CDK6 inhibitors. 
%
Importantly, both molecules demonstrate good potential in most crucial properties, including Ames mutagenesis, favorable oral toxicity, carcinogenicity, estrogen receptor binding, high intestinal absorption and favorable oral bioavailability.
%
Although the generated molecules are predicted as positive in hepatotoxicity and mitochondrial toxicity, all the approved drugs are also predicted as positive in these two toxicity.
%
This result suggests that these issues might stem from the limited prediction accuracy rather than being specific to our generated molecules.
%
Notably, NL-001 displays a potentially better plasma protein binding score compared to other molecules, which may improve its distribution within the body. 
%
Overall, these results indicate that NL-001 and NL-002 could be promising candidates for further drug development.


\begin{table*}
	\centering
		\caption{Comparison on ADMET Profiles among Generated Molecules and Approved Drugs Targeting CDK6}
	\label{tbl:admet_cdk6}
	\begin{scriptsize}
\begin{threeparttable}
	\begin{tabular}{
		%@{\hspace{0pt}}p{0.23\linewidth}@{\hspace{5pt}}
		%
		@{\hspace{6pt}}l@{\hspace{5pt}}
		@{\hspace{6pt}}r@{\hspace{6pt}}
		@{\hspace{6pt}}r@{\hspace{6pt}}
		@{\hspace{6pt}}r@{\hspace{6pt}}
		@{\hspace{6pt}}r@{\hspace{6pt}}
		@{\hspace{6pt}}r@{\hspace{6pt}}
		%
		%
		@{\hspace{6pt}}r@{\hspace{6pt}}
		%@{\hspace{6pt}}r@{\hspace{6pt}}
		%
		}
		\toprule
		\multirow{2}{*}{Property name} & \multicolumn{2}{c}{Generated molecules} & & \multicolumn{3}{c}{FDA-approved drugs} \\
		\cmidrule{2-3}\cmidrule{5-7}
		 & NL--001 & NL--002 & & Abemaciclib & Palbociclib & Ribociclib \\
		\midrule
\rowcolor[HTML]{D2EAD9}Ames   mutagenesis                             & --   &  --  & & + &  --  & --  \\
\rowcolor[HTML]{D2EAD9}Acute oral toxicity (c)                           & III & III & &  III          & III          & III         \\
Androgen receptor binding                         & +                          & +            &              & +            & +            & +             \\
Aromatase binding                                 & +                          & +            &              & +            & +            & +            \\
Avian toxicity                                    & --                          & --          &                & --            & --            & --            \\
Blood brain barrier                               & +                          & +            &              & +            & +            & +            \\
BRCP inhibitior                                   & --                          & --          &                & --            & --            & --            \\
Biodegradation                                    & --                          & --          &                & --            & --            & --           \\
BSEP inhibitior            & +                          & +            &              & +            & +            & +        \\
Caco-2                                            & +                          & +            &              & --            & --            & --            \\
\rowcolor[HTML]{D2EAD9}Carcinogenicity (binary)                          & --                          & --             &             & --            & --            & --          \\
\rowcolor[HTML]{D2EAD9}Carcinogenicity (trinary)                         & Non-required               & Non-required   &            & Non-required & Non-required & Non-required  \\
Crustacea aquatic toxicity & --                          & --            &              & --            & --            & --            \\
 CYP1A2 inhibition                                 & +                          & +            &              & --            & --            & +             \\
CYP2C19 inhibition                                & --                          & +            &              & +            & --            & +            \\
CYP2C8 inhibition                                 & --                          & --           &               & +            & +            & +            \\
CYP2C9 inhibition                                 & --                          & --           &               & --            & --            & +             \\
CYP2C9 substrate                                  & --                          & --           &               & --            & --            & --            \\
CYP2D6 inhibition                                 & --                          & --           &               & --            & --            & --            \\
CYP2D6 substrate                                  & --                          & --           &               & --            & --            & --            \\
CYP3A4 inhibition                                 & --                          & +            &              & --            & --            & --            \\
CYP3A4 substrate                                  & +                          & --            &              & +            & +            & +            \\
\rowcolor[HTML]{D2EAD9}CYP inhibitory promiscuity                        & +                          & +                    &      & +            & --            & +            \\
Eye corrosion                                     & --                          & --           &               & --            & --            & --            \\
Eye irritation                                    & --                          & --           &               & --            & --            & --             \\
\rowcolor[HTML]{D8E7FF}Estrogen receptor binding                         & +                          & +                    &      & +            & +            & +            \\
Fish aquatic toxicity                             & --                          & +            &              & +            & --            & --            \\
Glucocorticoid receptor   binding                 & +                          & +             &             & +            & +            & +            \\
Honey bee toxicity                                & --                          & --           &               & --            & --            & --            \\
\rowcolor[HTML]{D2EAD9}Hepatotoxicity                                    & +                          & +            &              & +            & +            & +             \\
Human ether-a-go-go-related gene inhibition     & +                          & +               &           & +            & --            & --           \\
\rowcolor[HTML]{D8E7FF}Human intestinal absorption                       & +                          & +             &             & +            & +            & +    \\
\rowcolor[HTML]{D8E7FF}Human oral bioavailability                        & +                          & +              &            & +            & +            & +     \\
\rowcolor[HTML]{D2EAD9}MATE1 inhibitior                                  & --                          & --              &            & --            & --            & --    \\
\rowcolor[HTML]{D2EAD9}Mitochondrial toxicity                            & +                          & +                &          & +            & +            & +    \\
Micronuclear                                      & +                          & +                          & +            & +            & +           \\
\rowcolor[HTML]{D2EAD9}Nephrotoxicity                                    & --                          & --             &             & --            & --            & --             \\
Acute oral toxicity                               & 2.325                      & 1.874    &     & 1.870        & 3.072        & 3.138        \\
\rowcolor[HTML]{D8E7FF}OATP1B1 inhibitior                                & +                          & +              &            & +            & +            & +             \\
\rowcolor[HTML]{D8E7FF}OATP1B3 inhibitior                                & +                          & +              &            & +            & +            & +             \\
\rowcolor[HTML]{D2EAD9}OATP2B1 inhibitior                                & --                          & --             &             & --            & --            & --             \\
OCT1 inhibitior                                   & --                          & --        &                  & +            & --            & +             \\
OCT2 inhibitior                                   & --                          & --        &                  & --            & --            & +             \\
P-glycoprotein inhibitior                         & --                          & --        &                  & +            & +            & +     \\
P-glycoprotein substrate                          & --                          & --        &                  & +            & +            & +     \\
PPAR gamma                                        & +                          & +          &                & +            & +            & +      \\
\rowcolor[HTML]{D8E7FF}Plasma protein binding                            & 0.359        & 0.745     &    & 0.865        & 0.872        & 0.636       \\
Reproductive toxicity                             & +                          & +          &                & +            & +            & +           \\
Respiratory toxicity                              & +                          & +          &                & +            & +            & +         \\
Skin corrosion                                    & --                          & --        &                  & --            & --            & --           \\
Skin irritation                                   & --                          & --        &                  & --            & --            & --         \\
Skin sensitisation                                & --                          & --        &                  & --            & --            & --          \\
Subcellular localzation                           & Mitochondria               & Mitochondria  &             & Lysosomes    & Mitochondria & Mitochondria \\
Tetrahymena pyriformis                            & 0.398                      & 0.903         &             & 1.033        & 1.958        & 1.606         \\
Thyroid receptor binding                          & +                          & +             &             & +            & +            & +           \\
UGT catelyzed                                     & --                          & --           &               & --            & --            & --           \\
\rowcolor[HTML]{D8E7FF}Water solubility                                  & -3.050                     & -3.078              &       & -3.942       & -3.288       & -2.673     \\
		\bottomrule
	\end{tabular}%
	\begin{tablenotes}[normal,flushleft]
		\begin{footnotesize}
	\item Blue cells highlight crucial properties where a negative outcome (``--'') is desired; for acute oral toxicity (c), a higher category (e.g., ``III'') is desired; and for carcinogenicity (trinary), ``Non-required'' is desired.
	%
	Green cells highlight crucial properties where a positive result (``+'') is desired; for plasma protein binding, a lower value is desired; and for water solubility, values higher than -4 are desired~\cite{logs}.
\!\! \par
		\par
		\end{footnotesize}
	\end{tablenotes}
\end{threeparttable}
\end{scriptsize}
  \vspace{--10pt}    
\end{table*}

%\label{tbl:admet_cdk6}

\paragraph{Generated Molecule for NEP}
%
Table~\ref{tbl:admet_nep} presents the comparison on ADMET profiles between a generated molecule for NEP targeting Alzheimer's disease and three approved drugs, Donepezil, Galantamine, and Rivastigmine, for Alzheimer's disease~\cite{Hansen2008}.
%
Overall, NL-003 exhibits a comparable ADMET profile with the three approved drugs.  
%
Notably, same as other approved drugs, NL-003 is predicted to be able to penetrate the blood brain barrier, a crucial property for Alzheimer's disease.
%  
In addition, it demonstrates a promising safety profile in terms of Ames mutagenesis, favorable oral toxicity, carcinogenicity, estrogen receptor binding, high intestinal absorption, nephrotoxicity and so on.
%
These results suggest that NL-003 could be promising candidates for the drug development of Alzheimer's disease.

\begin{table*}
	\centering
		\caption{Comparison on ADMET Profiles among Generated Molecule Targeting NEP and Approved Drugs for Alzhimer's Disease}
	\label{tbl:admet_nep}
	\begin{scriptsize}
\begin{threeparttable}
	\begin{tabular}{
		@{\hspace{6pt}}l@{\hspace{5pt}}
		%
		@{\hspace{6pt}}r@{\hspace{6pt}}
		@{\hspace{6pt}}r@{\hspace{6pt}}
		@{\hspace{6pt}}r@{\hspace{6pt}}
		@{\hspace{6pt}}r@{\hspace{6pt}}
		@{\hspace{6pt}}r@{\hspace{6pt}}
		%
		%
		%@{\hspace{6pt}}r@{\hspace{6pt}}
		%
		}
		\toprule
		\multirow{2}{*}{Property name} & Generated molecule & & \multicolumn{3}{c}{FDA-approved drugs} \\
\cmidrule{2-2}\cmidrule{4-6}
			& NL--003 & & Donepezil	& Galantamine & Rivastigmine \\
		\midrule
\rowcolor[HTML]{D2EAD9} 
Ames   mutagenesis                            & --                      &              & --                                    & --                                 & --                     \\
\rowcolor[HTML]{D2EAD9}Acute oral toxicity (c)                       & III           &                       & III                                  & III                               & II                      \\
Androgen receptor binding                     & +      &      & +            & --         & --         \\
Aromatase binding                             & --     &       & +            & --         & --        \\
Avian toxicity                                & --     &                               & --                                    & --                                 & --                        \\
\rowcolor[HTML]{D8E7FF} 
Blood brain barrier                           & +      &                              & +                                    & +                                 & +                        \\
BRCP inhibitior                               & --     &       & --            & --         & --         \\
Biodegradation                                & --     &                               & --                                    & --                                 & --                        \\
BSEP inhibitior                               & +      &      & +            & --         & --         \\
Caco-2                                        & +      &      & +            & +         & +         \\
\rowcolor[HTML]{D2EAD9} 
Carcinogenicity (binary)                      & --     &                               & --                                    & --                                 & --                        \\
\rowcolor[HTML]{D2EAD9} 
Carcinogenicity (trinary)                     & Non-required    &                     & Non-required                         & Non-required                      & Non-required             \\
Crustacea aquatic toxicity                    & +               &                     & +                                    & +                                 & --                        \\
CYP1A2 inhibition                             & +               &                     & +                                    & --                                 & --                        \\
CYP2C19 inhibition                            & +               &                     & --                                    & --                                 & --                        \\
CYP2C8 inhibition                             & +               &                     & --                                    & --                                 & --                        \\
CYP2C9 inhibition                             & --              &                      & --                                    & --                                 & --                        \\
CYP2C9 substrate                              & --              &                      & --                                    & --                                 & --                        \\
CYP2D6 inhibition                             & --              &                      & +                                    & --                                 & --                        \\
CYP2D6 substrate                              & --              &                      & +                                    & +                                 & +                        \\
CYP3A4 inhibition                             & --              &                      & --                                    & --                                 & --                        \\
CYP3A4 substrate                              & +               &                     & +                                    & +                                 & --                        \\
\rowcolor[HTML]{D2EAD9} 
CYP inhibitory promiscuity                    & +               &                     & +                                    & --                                 & --                        \\
Eye corrosion                                 & --     &       & --            & --         & --         \\
Eye irritation                                & --     &       & --            & --         & --         \\
Estrogen receptor binding                     & +      &      & +            & --         & --         \\
Fish aquatic toxicity                         & --     &                               & +                                    & +                                 & +                        \\
Glucocorticoid receptor binding             & --      &      & +            & --         & --         \\
Honey bee toxicity                            & --    &                                & --                                    & --                                 & --                        \\
\rowcolor[HTML]{D2EAD9} 
Hepatotoxicity                                & +     &                               & +                                    & --                                 & --                        \\
Human ether-a-go-go-related gene inhibition & +       &     & +            & --         & --         \\
\rowcolor[HTML]{D8E7FF} 
Human intestinal absorption                   & +     &                               & +                                    & +                                 & +                        \\
\rowcolor[HTML]{D8E7FF} 
Human oral bioavailability                    & --    &                                & +                                    & +                                 & +                        \\
\rowcolor[HTML]{D2EAD9} 
MATE1 inhibitior                              & --    &                                & --                                    & --                                 & --                        \\
\rowcolor[HTML]{D2EAD9} 
Mitochondrial toxicity                        & +     &                               & +                                    & +                                 & +                        \\
Micronuclear                                  & +     &       & --            & --         & +         \\
\rowcolor[HTML]{D2EAD9} 
Nephrotoxicity                                & --    &                                & --                                    & --                                 & --                        \\
Acute oral toxicity                           & 2.704  &      & 2.098        & 2.767     & 2.726     \\
\rowcolor[HTML]{D8E7FF} 
OATP1B1 inhibitior                            & +      &                              & +                                    & +                                 & +                        \\
\rowcolor[HTML]{D8E7FF} 
OATP1B3 inhibitior                            & +      &                              & +                                    & +                                 & +                        \\
\rowcolor[HTML]{D2EAD9} 
OATP2B1 inhibitior                            & --     &                               & --                                    & --                                 & --                        \\
OCT1 inhibitior                               & +      &      & +            & --         & --         \\
OCT2 inhibitior                               & --     &       & +            & --         & --         \\
P-glycoprotein inhibitior                     & +      &      & +            & --         & --         \\
\rowcolor[HTML]{D8E7FF} 
P-glycoprotein substrate                      & +      &                              & +                                    & +                                 & --                        \\
PPAR gamma                                    & +      &      & --            & --         & --         \\
\rowcolor[HTML]{D8E7FF} 
Plasma protein binding                        & 0.227   &                             & 0.883                                & 0.230                             & 0.606                    \\
Reproductive toxicity                         & +       &     & +            & +         & +         \\
Respiratory toxicity                          & +       &     & +            & +         & +         \\
Skin corrosion                                & --      &      & --            & --         & --         \\
Skin irritation                               & --      &      & --            & --         & --         \\
Skin sensitisation                            & --      &      & --            & --         & --         \\
Subcellular localzation                       & Mitochondria & &Mitochondria & Lysosomes & Mitochondria  \\
Tetrahymena pyriformis                        & 0.053           &                     & 0.979                                & 0.563                             & 0.702                        \\
Thyroid receptor binding                      & +       &     & +            & +         & --             \\
UGT catelyzed                                 & --      &      & --            & +         & --             \\
\rowcolor[HTML]{D8E7FF} 
Water solubility                              & -3.586   &                            & -2.425                               & -2.530                            & -3.062                       \\
		\bottomrule
	\end{tabular}%
	\begin{tablenotes}[normal,flushleft]
		\begin{footnotesize}
	\item Blue cells highlight crucial properties where a negative outcome (``--'') is desired; for acute oral toxicity (c), a higher category (e.g., ``III'') is desired; and for carcinogenicity (trinary), ``Non-required'' is desired.
	%
	Green cells highlight crucial properties where a positive result (``+'') is desired; for plasma protein binding, a lower value is desired; and for water solubility, values higher than -4 are desired~\cite{logs}.
\!\! \par
		\par
		\end{footnotesize}
	\end{tablenotes}
\end{threeparttable}
\end{scriptsize}
  \vspace{--10pt}    
\end{table*}

%\label{tbl:admet_nep}

\clearpage
%%%%%%%%%%%%%%%%%%%%%%%%%%%%%%%%%%%%%%%%%%%%%
\section{Algorithms}
\label{supp:algorithms}
%%%%%%%%%%%%%%%%%%%%%%%%%%%%%%%%%%%%%%%%%%%%%

Algorithm~\ref{alg:shapemol} describes the molecule generation process of \method.
%
Given a known ligand \molx, \method generates a novel molecule \moly that has a similar shape to \molx and thus potentially similar binding activity.
%
\method can also take the protein pocket \pocket as input and adjust the atoms of generated molecules for optimal fit and improved binding affinities.
%
Specifically, \method first calculates the shape embedding \shapehiddenmat for \molx using the shape encoder \SEE described in Algorithm~\ref{alg:see_shaperep}.
%
Based on \shapehiddenmat, \method then generates a novel molecule with a similar shape to \molx using the diffusion-based generative model \methoddiff as in Algorithm~\ref{alg:diffgen}.
%
During generation, \method can use shape guidance to directly modify the shape of \moly to closely resemble the shape of \molx.
%
When the protein pocket \pocket is available, \method can also use pocket guidance to ensure that \moly is specifically tailored to closely fit within \pocket.

\begin{algorithm}[!h]
    \caption{\method}
    \label{alg:shapemol}
         %\hspace*{\algorithmicindent} 
	\textbf{Required Input: $\molx$} \\
 	%\hspace*{\algorithmicindent} 
	\textbf{Optional Input: $\pocket$} 
    \begin{algorithmic}[1]
        \FullLineComment{calculate a shape embedding with Algorithm~\ref{alg:see_shaperep}}
        \State $\shapehiddenmat$, $\pc$ = $\SEE(\molx)$
        \FullLineComment{generate a molecule conditioned on the shape embedding with Algorithm~\ref{alg:diffgen}}
         \If{\pocket is not available}
        \State $\moly = \diffgenerative(\shapehiddenmat, \molx)$
        \Else
        \State $\moly = \diffgenerative(\shapehiddenmat, \molx, \pocket)$
        \EndIf
        \State \Return \moly
    \end{algorithmic}
\end{algorithm}
%\label{alg:shapemol}

\begin{algorithm}[!h]
    \caption{\SEE for shape embedding calculation}
    \label{alg:see_shaperep}
    \textbf{Required Input: $\molx$}
    \begin{algorithmic}[1]
        %\Require $\molx$
        \FullLineComment{sample a point cloud over the molecule surface shape}
        \State $\pc$ = $\text{samplePointCloud}(\molx)$
        \FullLineComment{encode the point cloud into a latent embedding (Equation~\ref{eqn:shape_embed})}
        \State $\shapehiddenmat = \SEE(\pc)$
        \FullLineComment{move the center of \pc to zero}
        \State $\pc = \pc - \text{center}(\pc)$
        \State \Return \shapehiddenmat, \pc
    \end{algorithmic}
\end{algorithm}
%\label{alg:see_shaperep}

\begin{algorithm}[!h]
    \caption{\diffgenerative for molecule generation}
    \label{alg:diffgen}
    	\textbf{Required Input: $\molx$, \shapehiddenmat} \\
 	%\hspace*{\algorithmicindent} 
	\textbf{Optional Input: $\pocket$} 
    \begin{algorithmic}[1]
        \FullLineComment{sample the number of atoms in the generated molecule}
        \State $n = \text{sampleAtomNum}(\molx)$
        \FullLineComment{sample initial positions and types of $n$ atoms}
        \State $\{\pos_T\}^n = \mathcal{N}(0, I)$
        \State $\{\atomfeat_T\}^n = C(K, \frac{1}{K})$
        \FullLineComment{generate a molecule by denoising $\{(\pos_T, \atomfeat_T)\}^n$ to $\{(\pos_0, \atomfeat_0)\}^n$}
        \For{$t = T$ to $1$}
            \IndentLineComment{predict the molecule without noise using the shape-conditioned molecule prediction module \molpred}{1.5}
            \State $(\tilde{\pos}_{0,t}, \tilde{\atomfeat}_{0,t}) = \molpred(\pos_t, \atomfeat_t, \shapehiddenmat)$
            \If{use shape guidance and $t > s$}
                \State $\tilde{\pos}_{0,t} = \shapeguide(\tilde{\pos}_{0,t}, \molx)$
                %\State $\tilde{\pos}_{0,t} = \pos^*_{0,t}$
            \EndIf
            \IndentLineComment{sample $(\pos_{t-1}, \atomfeat_{t-1})$ from $(\pos_t, \atomfeat_t)$ and $(\tilde{\pos}_{0,t}, \tilde{\atomfeat}_{0,t})$}{1.5}
            \State $\pos_{t-1} = P(\pos_{t-1}|\pos_t, \tilde{\pos}_{o,t})$
            \State $\atomfeat_{t-1} = P(\atomfeat_{t-1}|\atomfeat_t, \tilde{\atomfeat}_{o,t})$
            \If{use pocket guidance and $\pocket$ is available}
                \State $\pos_{t-1} = \pocketguide(\pos_{t-1}, \pocket)$
                %\State $\pos_{t-1} = \pos_{t-1}^*$
            \EndIf  
        \EndFor
        \State \Return $\moly = (\pos_0, \atomfeat_0)$
    \end{algorithmic}
\end{algorithm}
%\label{alg:diffgen}

%\input{algorithms/train_SE}
%\label{alg:train_se}

%\begin{algorithm}[!h]
    \caption{Training Procedure of \methoddiff}
    \label{alg:diffgen}
    \begin{algorithmic}[1]
        \Require $\shapehiddenmat, \molx, \pocket$
        \FullLineComment{sample the number of atoms in the generated molecule}
    \end{algorithmic}
\end{algorithm}
%\label{alg:train_diff}

%---------------------------------------------------------------------------------------------------------------------
\section{{Equivariance and Invariance}}
\label{supp:ei}
%---------------------------------------------------------------------------------------------------------------------

%.................................................................................................
\subsection{Equivariance}
\label{supp:ei:equivariance}
%.................................................................................................

{Equivariance refers to the property of a function $f(\pos)$ %\bo{is it the property of the function or embedding (x)?} 
that any translation and rotation transformation from the special Euclidean group SE(3)~\cite{Atz2021} applied to a geometric object
$\pos\in\mathbb{R}^3$ is mirrored in the output of $f(\pos)$, accordingly.
%
This property ensures $f(\pos)$ to learn a consistent representation of an object's geometric information, regardless of its orientation or location in 3D space.
%
%As a result, it provides $f(\pos)$ better generalization capabilities~\cite{Jonas20a}.
%
Formally, given any translation transformation $\mathbf{t}\in\mathbb{R}^3$ and rotation transformation $\mathbf{R}\in\mathbb{R}^{3\times3}$ ($\mathbf{R}^{\mathsf{T}}\mathbf{R}=\mathbb{I}$), %\xia{change the font types for $^{\mathsf{T}}$ and $\mathbb{I}$ in the entire manuscript}), 
$f(\pos)$ is equivariant with respect to these transformations %$g$ (\bo{where is $g$...})
if it satisfies
\begin{equation}
f(\mathbf{R}\pos+\mathbf{t}) = \mathbf{R}f(\pos) + \mathbf{t}. %\ \text{where}\ \hiddenpos = f(\pos).
\end{equation}
%
%where $\hiddenpos=f(\pos)$ is the output of $\pos$. 
%
In \method, both \SE and \methoddiff are developed to guarantee equivariance in capturing the geometric features of objects regardless of any translation or rotation transformations, as will be detailed in the following sections.
}

%.................................................................................................
\subsection{Invariance}
\label{supp:ei:invariance}
%.................................................................................................

%In contrast to equivariance, 
Invariance refers to the property of a function that its output {$f(\pos)$} remains constant under any translation and rotation transformations of the input $\pos$. %a geometric object's feature $\pos$.
%
This property enables $f(\pos)$ to accurately capture %a geometric object's 
the inherent features (e.g., atom features for 3D molecules) that are invariant of its orientation or position in 3D space.
%
Formally, $f(\pos)$ is invariant under any translation $\mathbf{t}$ and  rotation $\mathbf{R}$ if it satisfies
%
\begin{equation}
f(\mathbf{R}\pos+\mathbf{t}) = f(\pos).
\end{equation}
%
In \method, both \SE and \methoddiff capture the inherent features of objects in an invariant way, regardless of any translation or rotation transformations, as will be detailed in the following sections.

%%%%%%%%%%%%%%%%%%%%%%%%%%%%%%%%%%%%%%%%%%%%%
\section{Point Cloud Construction}
\label{supp:point_clouds}
%%%%%%%%%%%%%%%%%%%%%%%%%%%%%%%%%%%%%%%%%%%%%

In \method, we represented molecular surface shapes using point clouds (\pc).
%
$\pc$
serves as input to \SE, from which we derive shape latent embeddings.
%
To generate $\pc$, %\bo{\st{create this}}, \bo{generate $\pc$}
we initially generated a molecular surface mesh using the algorithm from the Open Drug Discovery Toolkit~\cite{Wjcikowski2015oddt}.
%
Following this, we uniformly sampled points on the mesh surface with probability proportional to the face area, %\xia{how to uniformly?}, ensuring the sampling is done proportionally to the face area, with
using the algorithm from PyTorch3D~\cite{ravi2020pytorch3d}.
%
This point cloud $\pc$ is then centralized by setting the center of its points to zero.
%
%

%%%%%%%%%%%%%%%%%%%%%%%%%%%%%%%%%%%%%%%%%%%%%
\section{Query Point Sampling}
\label{supp:training:shapeemb}
%%%%%%%%%%%%%%%%%%%%%%%%%%%%%%%%%%%%%%%%%%%%%

As described in Section ``Shape Decoder (\SED)'', the signed distances of query points $z_q$ to molecule surface shape $\pc$ are used to optimize \SE.
%
In this section, we present how to sample these points $z_q$ in 3D space.
%
Particularly, we first determined the bounding box around the molecular surface shape, using the maximum and minimum \mbox{($x$, $y$, $z$)-axis} coordinates for points in our point cloud \pc,
denoted as $(x_\text{min}, y_\text{min}, z_\text{min})$ and $(x_\text{max}, y_\text{max}, z_\text{max})$.
%
We extended this box slightly by defining its corners as \mbox{$(x_\text{min}-1, y_\text{min}-1, z_\text{min}-1)$} and \mbox{$(x_\text{max}+1, y_\text{max}+1, z_\text{max}+1)$}.
%
For sampling $|\mathcal{Z}|$ query points, we wanted an even distribution of points inside and outside the molecule surface shape.
%
%\ziqi{Typically, within this bounding box, molecules occupy only a small portion of volume, which makes it more likely to sample
%points outside the molecule surface shape.}
%
When a bounding box is defined around the molecule surface shape, there could be a lot of empty spaces within the box that the molecule does not occupy due to 
its complex and irregular shape.
%
This could lead to that fewer points within the molecule surface shape could be sampled within the box.
%
Therefore, we started by randomly sampling $3k$ points within our bounding box to ensure that there are sufficient points within the surface.
%
We then determined whether each point lies within the molecular surface, using an algorithm from Trimesh~\footnote{https://trimsh.org/} based on the molecule surface mesh.
%
If there are $n_w$ points found within the surface, we selected $n=\min(n_w, k/2)$ points from these points, 
and randomly choose the remaining 
%\bo{what do you mean by remaining? If all the 3k sampled points are inside the surface, you get no points left.} 
$k-n$ points 
from those outside the surface.
%
For each query point, we determined its signed distance to the molecule surface by its closest distance to points in \pc with a sign indicating whether it is inside the surface.

%%%%%%%%%%%%%%%%%%%%%%%%%%%%%%%%%%%%%%%%%%%%%
\section{Forward Diffusion (\diffnoise)}
\label{supp:forward}
%%%%%%%%%%%%%%%%%%%%%%%%%%%%%%%%%%%%%%%%%%%%%

%===================================================================
\subsection{{Forward Process}}
\label{supp:forward:forward}
%===================================================================

Formally, for atom positions, the probability of $\pos_t$ sampled given $\pos_{t-1}$, denoted as $q(\pos_t|\pos_{t-1})$, is defined as follows,
%\xia{revise the representation, should be $\beta^x_t$ -- note the space} as follows,
%
\begin{equation}
q(\pos_t|\pos_{t-1}) = \mathcal{N}(\pos_t|\sqrt{1-\beta^{\mathtt{x}}_t}\pos_{t-1}, \beta^{\mathtt{x}}_t\mathbb{I}), 
\label{eqn:noiseposinter}
\end{equation}
%
%\xia{should be a comma after the equation. you also missed it. }
%\st{in which} 
where %\hl{$\pos_0$ denotes the original atom position;} \xia{no $\pos_0$ in the equation...}
%$\mathbf{I}$ denotes the identity matrix;
$\mathcal{N}(\cdot)$ is a Gaussian distribution of $\pos_t$ with mean $\sqrt{1-\beta_t^{\mathtt{x}}}\pos_{t-1}$ and covariance $\beta_t^{\mathtt{x}}\mathbf{I}$.
%\xia{what is $\mathcal{N}$? what is $q$? you abused $q$. need to be crystal clear... }
%\bo{Should be $\sim$ not $=$ in the equation}
%
Following Hoogeboom \etal~\cite{hoogeboom2021catdiff}, 
%the forward process for the discrete atom feature $\atomfeat_t\in\mathbb{R}^K$ adds 
%categorical noise into $\atomfeat_{t-1}$ according to a variance schedule $\beta_t^v\in (0, 1)$. %as follows, %\hl{$\betav_t\in (0, 1)$} as follows,
%\xia{presentation...check across the entire manuscript... } as follows,
%
%\ziqi{Formally, 
for atom features, the probability of $\atomfeat_t$ across $K$ classes given $\atomfeat_{t-1}$ is defined as follows,
%
\begin{equation}
q(\atomfeat_t|\atomfeat_{t-1}) = \mathcal{C}(\atomfeat_t|(1-\beta^{\mathtt{v}}_t) \atomfeat_{t-1}+\beta^{\mathtt{v}}_t\mathbf{1}/K),
\label{eqn:noisetypeinter}
\end{equation}
%
where %\hl{$\atomfeat_0$ denotes the original atom positions}; 
$\mathcal{C}$ is a categorical distribution of $\atomfeat_t$ derived from the %by 
noising $\atomfeat_{t-1}$ with a uniform noise $\beta^{\mathtt{v}}_t\mathbf{1}/K$ across $K$ classes.
%adding an uniform noise $\beta^v_t$ to $\atomfeat_{t-1}$ across K classes.
%\xia{there is always a comma or period after the equations. Equations are part of a sentence. you always missed it. }
%\xia{what is $\mathcal{C}$? what does $q$ mean? it is abused. }

Since the above distributions form Markov chains, %} \xia{grammar!}, 
the probability of any $\pos_t$ or $\atomfeat_t$ can be derived from $\pos_0$ or $\atomfeat_0$:
%samples $\mol_0$ as follows,
%
\begin{eqnarray}
%\begin{aligned}
& q(\pos_t|\pos_{0}) & = \mathcal{N}(\pos_t|\sqrt{\cumalpha^{\mathtt{x}}_t}\pos_0, (1-\cumalpha^{\mathtt{x}}_t)\mathbb{I}), \label{eqn:noisepos}\\
& q(\atomfeat_t|\atomfeat_0)  & = \mathcal{C}(\atomfeat_t|\cumalpha^{\mathtt{v}}_t\atomfeat_0 + (1-\cumalpha^{\mathtt{v}}_t)\mathbf{1}/K), \label{eqn:noisetype}\\
& \text{where }\cumalpha^{\mathtt{u}}_t & = \prod\nolimits_{\tau=1}^{t}\alpha^{\mathtt{u}}_\tau, \ \alpha^{\mathtt{u}}_\tau=1 - \beta^{\mathtt{u}}_\tau, \ {\mathtt{u}}={\mathtt{x}} \text{ or } {\mathtt{v}}.\;\;\;\label{eqn:noiseschedule}
%\end{aligned}
\label{eqn:pos_prior}
\end{eqnarray}
%\xia{always punctuations after equations!!! also use ``eqnarray" instead of ``equation" + ``aligned" for multiple equations, each
%with a separate reference numbering...}
%\st{in which}, 
%where \ziqi{$\cumalpha^u_t = \prod_{\tau=1}^{t}\alpha^u_\tau$ and $\alpha^u_\tau=1 - \beta^u_\tau$ ($u$=$x$ or $v$)}.
%\xia{no such notations in the above equations; also subscript $s$ is abused with shape};
%$K$ is the number of categories for atom features.
%
%The details about noise schedules $\beta^x_t$ and $\beta^v_t$ are available in Supplementary Section \ref{XXX}. \ziqi{add trend}
%
Note that $\bar{\alpha}^{\mathtt{u}}_t$ ($\mathtt{u}={\mathtt{x}}\text{ or }{\mathtt{v}}$)
%($u$=$x$ or $v$) 
is monotonically decreasing from 1 to 0 over $t=[1,T]$. %\xia{=???}. 
%
As $t\rightarrow 1$, $\cumalpha^{\mathtt{x}}_t$ and $\cumalpha^{\mathtt{v}}_t$ are close to 1, leading to that $\pos_t$ or $\atomfeat_t$ approximates 
%the original data 
$\pos_0$ or $\atomfeat_0$.
%
Conversely, as  $t\rightarrow T$, $\cumalpha^{\mathtt{x}}_t$ and $\cumalpha^{\mathtt{v}}_t$ are close to 0,
leading to that $q(\pos_T|\pos_{0})$ %\st{$\rightarrow \mathcal{N}(\mathbf{0}, \mathbf{I})$} 
resembles  {$\mathcal{N}(\mathbf{0}, \mathbb{I})$} 
and $q(\atomfeat_T|\atomfeat_0)$ %\st{$\rightarrow \mathcal{C}(\mathbf{I}/K)$} 
resembles {$\mathcal{C}(\mathbf{1}/K)$}.

Using Bayes theorem, the ground-truth Normal posterior of atom positions $p(\pos_{t-1}|\pos_t, \pos_0)$ can be calculated in a
closed form~\cite{ho2020ddpm} as below,
%
\begin{eqnarray}
& p(\pos_{t-1}|\pos_t, \pos_0) = \mathcal{N}(\pos_{t-1}|\mu(\pos_t, \pos_0), \tilde{\beta}^\mathtt{x}_t\mathbb{I}), \label{eqn:gt_pos_posterior_1}\\
&\!\!\!\!\!\!\!\!\!\!\!\mu(\pos_t, \pos_0)\!=\!\frac{\sqrt{\bar{\alpha}^{\mathtt{x}}_{t-1}}\beta^{\mathtt{x}}_t}{1-\bar{\alpha}^{\mathtt{x}}_t}\pos_0\!+\!\frac{\sqrt{\alpha^{\mathtt{x}}_t}(1-\bar{\alpha}^{\mathtt{x}}_{t-1})}{1-\bar{\alpha}^{\mathtt{x}}_t}\pos_t, 
\tilde{\beta}^\mathtt{x}_t\!=\!\frac{1-\bar{\alpha}^{\mathtt{x}}_{t-1}}{1-\bar{\alpha}^{\mathtt{x}}_{t}}\beta^{\mathtt{x}}_t.\;\;\;
\end{eqnarray}
%
%\xia{Ziqi, please double check the above two equations!}
Similarly, the ground-truth categorical posterior of atom features $p(\atomfeat_{t-1}|\atomfeat_{t}, \atomfeat_0)$ can be calculated~\cite{hoogeboom2021catdiff} as below,
%
\begin{eqnarray}
& p(\atomfeat_{t-1}|\atomfeat_{t}, \atomfeat_0) = \mathcal{C}(\atomfeat_{t-1}|\mathbf{c}(\atomfeat_t, \atomfeat_0)), \label{eqn:gt_atomfeat_posterior_1}\\
& \mathbf{c}(\atomfeat_t, \atomfeat_0) = \tilde{\mathbf{c}}/{\sum_{k=1}^K \tilde{c}_k}, \label{eqn:gt_atomfeat_posterior_2} \\
& \tilde{\mathbf{c}} = [\alpha^{\mathtt{v}}_t\atomfeat_t + \frac{1 - \alpha^{\mathtt{v}}_t}{K}]\odot[\bar{\alpha}^{\mathtt{v}}_{t-1}\atomfeat_{0}+\frac{1-\bar{\alpha}^{\mathtt{v}}_{t-1}}{K}], 
\label{eqn:gt_atomfeat_posterior_3}
%\label{eqn:atomfeat_posterior}
\end{eqnarray}
%
%\xia{Ziqi: please double check the above equations!}
%
where $\tilde{c}_k$ denotes the likelihood of $k$-th class across $K$ classes in $\tilde{\mathbf{c}}$; 
$\odot$ denotes the element-wise product operation;
$\tilde{\mathbf{c}}$ is calculated using $\atomfeat_t$ and $\atomfeat_{0}$ and normalized into $\mathbf{c}(\atomfeat_t, \atomfeat_0)$ so as to represent
probabilities. %\xia{is this correct? is $\tilde{c}_k$ always greater than 0?}
%\xia{how is it calculated?}.
%\ziqi{the likelihood distribution $\tilde{c}$ is calculated by $p(\atomfeat_t|\atomfeat_{t-1})p(\atomfeat_{t-1}|\atomfeat_0)$, according to 
%Equation~\ref{eqn:noisetypeinter} and \ref{eqn:noisetype}.
%\xia{need to write the key idea of the above calculation...}
%
The proof of the above equations is available in Supplementary Section~\ref{supp:forward:proof}.

%===================================================================
\subsection{Variance Scheduling in \diffnoise}
\label{supp:forward:variance}
%===================================================================

Following Guan \etal~\cite{guan2023targetdiff}, we used a sigmoid $\beta$ schedule for the variance schedule $\beta_t^{\mathtt{x}}$ of atom coordinates as below,

\begin{equation}
\beta_t^{\mathtt{x}} = \text{sigmoid}(w_1(2 t / T - 1)) (w_2 - w_3) + w_3
\end{equation}
in which $w_i$($i$=1,2, or 3) are hyperparameters; $T$ is the maximum step.
%
We set $w_1=6$, $w_2=1.e-7$ and $w_3=0.01$.
%
For atom types, we used a cosine $\beta$ schedule~\cite{nichol2021} for $\beta_t^{\mathtt{v}}$ as below,

\begin{equation}
\begin{aligned}
& \bar{\alpha}_t^{\mathtt{v}} = \frac{f(t)}{f(0)}, f(t) = \cos(\frac{t/T+s}{1+s} \cdot \frac{\pi}{2})^2\\
& \beta_t^{\mathtt{v}} = 1 - \alpha_t^{\mathtt{v}} = 1 - \frac{\bar{\alpha}_t^{\mathtt{v}} }{\bar{\alpha}_{t-1}^{\mathtt{v}} }
\end{aligned}
\end{equation}
in which $s$ is a hyperparameter and set as 0.01.

As shown in Section ``Forward Diffusion Process'', the values of $\beta_t^{\mathtt{x}}$ and $\beta_t^{\mathtt{v}}$ should be 
sufficiently small to ensure the smoothness of forward diffusion process. In the meanwhile, their corresponding $\bar{\alpha}_t$
values should decrease from 1 to 0 over $t=[1,T]$.
%
Figure~\ref{fig:schedule} shows the values of $\beta_t$ and $\bar{\alpha}_t$ for atom coordinates and atom types with our hyperparameters.
%
Please note that the value of $\beta_{t}^{\mathtt{x}}$ is less than 0.1 for 990 out of 1,000 steps. %\bo{\st{, though it increases when $t$ is close to 1,000}}.
%
This guarantees the smoothness of the forward diffusion process.
%\bo{add $\beta_t^{\mathtt{x}}$ and $\beta_t^{\mathtt{v}}$ in the legend of the figure...}
%\bo{$\beta_t^{\mathtt{v}}$ does not look small when $t$ is close to 1000...}

\begin{figure}
	\begin{subfigure}[t]{.45\linewidth}
		\centering
		\includegraphics[width=.7\linewidth]{figures/var_schedule_beta.pdf}
	\end{subfigure}
	%
	\hfill
	\begin{subfigure}[t]{.45\linewidth}
		\centering
		\includegraphics[width=.7\linewidth]{figures/var_schedule_alpha.pdf}
	\end{subfigure}
	\caption{Schedule}
	\label{fig:schedule}
\end{figure}

%===================================================================
\subsection{Derivation of Forward Diffusion Process}
\label{supp:forward:proof}
%===================================================================

In \method, a Gaussian noise and a categorical noise are added to continuous atom position and discrete atom features, respectively.
%
Here, we briefly describe the derivation of posterior equations (i.e., Eq.~\ref{eqn:gt_pos_posterior_1}, and   \ref{eqn:gt_atomfeat_posterior_1}) for atom positions and atom types in our work.
%
We refer readers to Ho \etal~\cite{ho2020ddpm} and Kong \etal~\cite{kong2021diffwave} %\bo{add XXX~\etal here...} \cite{ho2020ddpm,kong2021diffwave} 	
for a detailed description of diffusion process for continuous variables and Hoogeboom \etal~\cite{hoogeboom2021catdiff} for
%\bo{add XXX~\etal here...} \cite{hoogeboom2021catdiff} for
the description of diffusion process for discrete variables.

For continuous atom positions, as shown in Kong \etal~\cite{kong2021diffwave}, according to Bayes theorem, given $q(\pos_t|\pos_{t-1})$ defined in Eq.~\ref{eqn:noiseposinter} and 
$q(\pos_t|\pos_0)$ defined in Eq.~\ref{eqn:noisepos}, the posterior $q(\pos_{t-1}|\pos_{t}, \pos_0)$ is derived as below (superscript $\mathtt{x}$ is omitted for brevity),

\begin{equation}
\begin{aligned}
& q(\pos_{t-1}|\pos_{t}, \pos_0)  = \frac{q(\pos_t|\pos_{t-1}, \pos_0)q(\pos_{t-1}|\pos_0)}{q(\pos_t|\pos_0)} \\
& =  \frac{\mathcal{N}(\pos_t|\sqrt{1-\beta_t}\pos_{t-1}, \beta_{t}\mathbf{I}) \mathcal{N}(\pos_{t-1}|\sqrt{\bar{\alpha}_{t-1}}\pos_{0}, (1-\bar{\alpha}_{t-1})\mathbf{I}) }{ \mathcal{N}(\pos_{t}|\sqrt{\bar{\alpha}_t}\pos_{0}, (1-\bar{\alpha}_t)\mathbf{I})}\\
& =  (2\pi{\beta_t})^{-\frac{3}{2}} (2\pi{(1-\bar{\alpha}_{t-1})})^{-\frac{3}{2}} (2\pi(1-\bar{\alpha}_t))^{\frac{3}{2}} \times \exp( \\
& -\frac{\|\pos_t - \sqrt{\alpha}_t\pos_{t-1}\|^2}{2\beta_t} -\frac{\|\pos_{t-1} - \sqrt{\bar{\alpha}}_{t-1}\pos_{0} \|^2}{2(1-\bar{\alpha}_{t-1})} \\
& + \frac{\|\pos_t - \sqrt{\bar{\alpha}_t}\pos_0\|^2}{2(1-\bar{\alpha}_t)}) \\
& = (2\pi\tilde{\beta}_t)^{-\frac{3}{2}} \exp(-\frac{1}{2\tilde{\beta}_t}\|\pos_{t-1}-\frac{\sqrt{\bar{\alpha}_{t-1}}\beta_t}{1-\bar{\alpha}_t}\pos_0 \\
& - \frac{\sqrt{\alpha_t}(1-\bar{\alpha}_{t-1})}{1-\bar{\alpha}_t}\pos_{t}\|^2) \\
& \text{where}\ \tilde{\beta}_t = \frac{1-\bar{\alpha}_{t-1}}{1-\bar{\alpha}_t}\beta_t.
\end{aligned}
\end{equation}
%\bo{marked part does not look right to me.}
%\bo{How to you derive from the second equation to the third one?}

Therefore, the posterior of atom positions is derived as below,

\begin{equation}
q(\pos_{t-1}|\pos_{t}, \pos_0)\!\!=\!\!\mathcal{N}(\pos_{t-1}|\frac{\sqrt{\bar{\alpha}_{t-1}}\beta_t}{1-\bar{\alpha}_t}\pos_0 + \frac{\sqrt{\alpha_t}(1-\bar{\alpha}_{t-1})}{1-\bar{\alpha}_t}\pos_{t}, \tilde{\beta}_t\mathbf{I}).
\end{equation}

For discrete atom features, as shown in Hoogeboom \etal~\cite{hoogeboom2021catdiff} and Guan \etal~\cite{guan2023targetdiff},
according to Bayes theorem, the posterior $q(\atomfeat_{t-1}|\atomfeat_{t}, \atomfeat_0)$ is derived as below (supperscript $\mathtt{v}$ is omitted for brevity),

\begin{equation}
\begin{aligned}
& q(\atomfeat_{t-1}|\atomfeat_{t}, \atomfeat_0) =  \frac{q(\atomfeat_t|\atomfeat_{t-1}, \atomfeat_0)q(\atomfeat_{t-1}|\atomfeat_0)}{\sum_{\scriptsize{\atomfeat}_{t-1}}q(\atomfeat_t|\atomfeat_{t-1}, \atomfeat_0)q(\atomfeat_{t-1}|\atomfeat_0)} \\
%& = \frac{\mathcal{C}(\atomfeat_t|(1-\beta_t)\atomfeat_{t-1} + \beta_t\frac{\mathbf{1}}{K}) \mathcal{C}(\atomfeat_{t-1}|\bar{\alpha}_{t-1}\atomfeat_0+(1-\bar{\alpha}_{t-1})\frac{\mathbf{1}}{K})} \\
\end{aligned}
\end{equation}

For $q(\atomfeat_t|\atomfeat_{t-1}, \atomfeat_0)$, we have % $\atomfeat_t=\atomfeat_{t-1}$ with probability $1-\beta_t+\beta_t / K$, and $\atomfeat_t \neq \atomfeat_{t-1}$
%with probability $\beta_t / K$.
%
%Therefore, this function can be symmetric, that is, 
%
\begin{equation}
\begin{aligned}
q(\atomfeat_t|\atomfeat_{t-1}, \atomfeat_0) & = \mathcal{C}(\atomfeat_t|(1-\beta_t)\atomfeat_{t-1} + \beta_t/{K})\\
& = \begin{cases}
1-\beta_t+\beta_t/K,\!&\text{when}\ \atomfeat_{t} = \atomfeat_{t-1},\\
\beta_t / K,\! &\text{when}\ \atomfeat_{t} \neq \atomfeat_{t-1},
\end{cases}\\
& = \mathcal{C}(\atomfeat_{t-1}|(1-\beta_t)\atomfeat_{t} + \beta_t/{K}).
\end{aligned}
%\mathcal{C}(\atomfeat_{t-1}|(1-\beta_{t})\atomfeat_{t} + \beta_t\frac{\mathbf{1}}{K}).
\end{equation}
%
Therefore, we have
%\bo{why it can be symmetric}
%
\begin{equation}
\begin{aligned}
& q(\atomfeat_t|\atomfeat_{t-1}, \atomfeat_0)q(\atomfeat_{t-1}|\atomfeat_0) \\
& = \mathcal{C}(\atomfeat_{t-1}|(1-\beta_t)\atomfeat_{t} + \beta_t\frac{\mathbf{1}}{K}) \mathcal{C}(\atomfeat_{t-1}|\bar{\alpha}_{t-1}\atomfeat_0+(1-\bar{\alpha}_{t-1})\frac{\mathbf{1}}{K}) \\
& = [\alpha_t\atomfeat_t + \frac{1 - \alpha_t}{K}]\odot[\bar{\alpha}_{t-1}\atomfeat_{0}+\frac{1-\bar{\alpha}_{t-1}}{K}].
\end{aligned}
\end{equation}
%
%\bo{what is $\tilde{\mathbf{c}}$...}
Therefore, with $q(\atomfeat_t|\atomfeat_{t-1}, \atomfeat_0)q(\atomfeat_{t-1}|\atomfeat_0) = \tilde{\mathbf{c}}$, the posterior is as below,

\begin{equation}
q(\atomfeat_{t-1}|\atomfeat_{t}, \atomfeat_0) = \mathcal{C}(\atomfeat_{t-1}|\mathbf{c}(\atomfeat_t, \atomfeat_0)) = \frac{\tilde{\mathbf{c}}}{\sum_{k}^K\tilde{c}_k}.
\end{equation}

%%%%%%%%%%%%%%%%%%%%%%%%%%%%%%%%%%%%%%%%%%%%%
\section{{Backward Generative Process} (\diffgenerative)}
\label{supp:backward}
%%%%%%%%%%%%%%%%%%%%%%%%%%%%%%%%%%%%%%%%%%%%%

Following Ho \etal~\cite{ho2020ddpm}, with $\tilde{\pos}_{0,t}$, the probability of $\pos_{t-1}$ denoised from $\pos_t$, denoted as $p(\pos_{t-1}|\pos_t)$,
can be estimated %\hl{parameterized} \xia{???} 
by the approximated posterior $p_{\boldsymbol{\Theta}}(\pos_{t-1}|\pos_t, \tilde{\pos}_{0,t})$ as below,
%
\begin{equation}
\begin{aligned}
p(\pos_{t-1}|\pos_t) & \approx p_{\boldsymbol{\Theta}}(\pos_{t-1}|\pos_t, \tilde{\pos}_{0,t}) \\
& = \mathcal{N}(\pos_{t-1}|\mu_{\boldsymbol{\Theta}}(\pos_t, \tilde{\pos}_{0,t}),\tilde{\beta}_t^{\mathtt{x}}\mathbb{I}),
\end{aligned}
\label{eqn:aprox_pos_posterior}
\end{equation}
%
where ${\boldsymbol{\Theta}}$ is the learnable parameter; $\mu_{\boldsymbol{\Theta}}(\pos_t, \tilde{\pos}_{0,t})$ is an estimate %estimation
of $\mu(\pos_t, \pos_{0})$ by replacing $\pos_0$ with its estimate $\tilde{\pos}_{0,t}$ 
in Equation~{\ref{eqn:gt_pos_posterior_1}}.
%
Similarly, with $\tilde{\atomfeat}_{0,t}$, the probability of $\atomfeat_{t-1}$ denoised from $\atomfeat_t$, denoted as $p(\atomfeat_{t-1}|\atomfeat_t)$, 
can be estimated %\hl{parameterized} 
by the approximated posterior $p_{\boldsymbol{\Theta}}(\atomfeat_{t-1}|\atomfeat_t, \tilde{\atomfeat}_{0,t})$ as below,
%
\begin{equation}
\begin{aligned}
p(\atomfeat_{t-1}|\atomfeat_t)\approx p_{\boldsymbol{\Theta}}(\atomfeat_{t-1}|\atomfeat_{t}, \tilde{\atomfeat}_{0,t}) 
=\mathcal{C}(\atomfeat_{t-1}|\mathbf{c}_{\boldsymbol{\Theta}}(\atomfeat_t, \tilde{\atomfeat}_{0,t})),\!\!\!\!
\end{aligned}
\label{eqn:aprox_atomfeat_posterior}
\end{equation}
%
where $\mathbf{c}_{\boldsymbol{\Theta}}(\atomfeat_t, \tilde{\atomfeat}_{0,t})$ is an estimate of $\mathbf{c}(\atomfeat_t, \atomfeat_0)$
by replacing $\atomfeat_0$  
with its estimate $\tilde{\atomfeat}_{0,t}$ in Equation~\ref{eqn:gt_atomfeat_posterior_1}.



%===================================================================
\section{\method Loss Function Derivation}
\label{supp:training:loss}
%===================================================================

In this section, we demonstrate that a step weight $w_t^{\mathtt{x}}$ based on the signal-to-noise ratio $\lambda_t$ should be 
included into the atom position loss (Eq.~\ref{eqn:diff:obj:pos}).
%
In the diffusion process for continuous variables, the optimization problem is defined 
as below~\cite{ho2020ddpm},
%
\begin{equation*}
\begin{aligned}
& \arg\min_{\boldsymbol{\Theta}}KL(q(\pos_{t-1}|\pos_t, \pos_0)|p_{\boldsymbol{\Theta}}(\pos_{t-1}|\pos_t, \tilde{\pos}_{0,t})) \\
& = \arg\min_{\boldsymbol{\Theta}} \frac{\bar{\alpha}_{t-1}(1-\alpha_t)}{2(1-\bar{\alpha}_{t-1})(1-\bar{\alpha}_{t})}\|\tilde{\pos}_{0, t}-\pos_0\|^2 \\
& = \arg\min_{\boldsymbol{\Theta}} \frac{1-\alpha_t}{2(1-\bar{\alpha}_{t-1})\alpha_{t}} \|\tilde{\boldsymbol{\epsilon}}_{0,t}-\boldsymbol{\epsilon}_0\|^2,
\end{aligned}
\end{equation*}
where $\boldsymbol{\epsilon}_0 = \frac{\pos_t - \sqrt{\bar{\alpha}_t}\pos_0}{\sqrt{1-\bar{\alpha}_t}}$ is the ground-truth noise variable sampled from $\mathcal{N}(\mathbf{0}, \mathbf{1})$ and is used to sample $\pos_t$ from $\mathcal{N}(\pos_t|\sqrt{\cumalpha_t}\pos_0, (1-\cumalpha_t)\mathbf{I})$ in Eq.~\ref{eqn:noisetype};
$\tilde{\boldsymbol{\epsilon}}_0 = \frac{\pos_t - \sqrt{\bar{\alpha}_t}\tilde{\pos}_{0, t}}{\sqrt{1-\bar{\alpha}_t}}$ is the predicted noise variable. 

%A simplified training objective is proposed by Ho \etal~\cite{ho2020ddpm} as below,
Ho \etal~\cite{ho2020ddpm} further simplified the above objective as below and
demonstrated that the simplified one can achieve better performance:
%
\begin{equation}
\begin{aligned}
& \arg\min_{\boldsymbol{\Theta}} \|\tilde{\boldsymbol{\epsilon}}_{0,t}-\boldsymbol{\epsilon}_0\|^2 \\
& = \arg\min_{\boldsymbol{\Theta}} \frac{\bar{\alpha}_t}{1-\bar{\alpha}_t}\|\tilde{\pos}_{0,t}-\pos_0\|^2,
\end{aligned}
\label{eqn:supp:losspos}
\end{equation}
%
where $\lambda_t=\frac{\bar{\alpha}_t}{1-\bar{\alpha}_t}$ is the signal-to-noise ratio.
%
While previous work~\cite{guan2023targetdiff} applies uniform step weights across
different steps, we demonstrate that a step weight should be included into the atom position loss according to Eq.~\ref{eqn:supp:losspos}.
%
However, the value of $\lambda_t$ could be very large when $\bar{\alpha}_t$ is close to 1 as $t$ approaches 1.
%
Therefore, we clip the value of $\lambda_t$ with threshold $\delta$ in Eq.~\ref{eqn:diff:obj:pos}.




\end{document}




\begin{proposition}\label{thm:DP}
For any $\gamma\in(0,\alpha)$, Algorithm \ref{alg:dp} computes a prediction set $\widehat{C}_{\rm DP}\in \calC_k$ by dynamic programming with time complexity $O(n^3k/\gamma)$ such that 
    \begin{enumerate}
        \item $\mathbb{P}_n(\widehat{C}_{\rm DP})\geq 1-\alpha$;
        \item $\vol(\widehat{C}_{\rm DP})\leq \opt_k(\mathbb{P}_n,1-\alpha+\gamma)$.
    \end{enumerate}
\end{proposition}

Together with (\ref{eq:un-con}), the coverage and volume guarantees of the dynamic programming can also be generalized from $\mathbb{P}_n$ to $P$.

\subsection{Conformalizing Dynamic Programming}\label{sec:cdp}

Having understood the generalization ability of dynamic programming, we are ready to conformalize the procedure to achieve a finite-sample coverage property. For simplicity, we will adopt the framework of split conformal prediction, though in principle full conformal prediction can also be applied here.

In the split conformal predicition framework, the data set is split into two halves. The first half is used to compute a conformity score, and the second half determines the quantile level. For convenience of notation, let us assume, from now on, that the sample size is $2n$. The split conformal procedure is outlined below.
\begin{enumerate}
\item Compute a score function $q(\cdot)$ using $Y_1,\cdots,Y_n$.
\item Evaluate $q(Y_{n+1}),\cdots, q(Y_{2n})$, and order them as $q_1\leq \cdots\leq q_n$.
\item Output the prediction set
\begin{equation}
\widehat{C}=\left\{y: q(y)\geq q_{\lfloor(n+1)\alpha\rfloor}\right\}. \label{eq:pred-set-un}
\end{equation}
\end{enumerate}
By the exchangeability of $Y_1,\cdots,Y_{2n},Y_{2n+1}$, the prediction set $\widehat{C}$ satisfies
$$\mathbb{P}\left(Y_{2n+1}\in \widehat{C}\right)\geq 1-\alpha,$$
where the above probability is over the randomness of $(Y_1,\cdots,Y_n)$, that of $(Y_{n+1},\cdots,Y_{2n})$, and that of $Y_{2n+1}$.

To conformalize the dynamic programming that approximately computes (\ref{eq:erm}), we will first compute a nested system $S_1\subset\cdots\subset S_m\subset \mathbb{R}$ using the data $Y_1,\cdots,Y_n$. The nested system is required to satisfy the following assumption.

\begin{assumption}\label{as:ne-un}
The sets $S_1\subset\cdots\subset S_m\subset\mathbb{R}$ are measurable with respect to the $\sigma$-field generated by $Y_1,\cdots,Y_n$. Moreover, for some positive integer $k$, some $\alpha\in(0,1)$ and some $\delta,\gamma$ such that $3\delta+\gamma+n^{-1}\leq \alpha$, we have
\begin{enumerate}
\item $\mathbb{P}_n(S_j)=\frac{j}{m}$ and $S_j\in\calC_k$ for all $j\in[m]$.
\item There exists some $j^*\in[m]$, such that\\ $\mathbb{P}_n(S_{j^*})\geq 1-\alpha+n^{-1}+3\delta$ and \\$\vol(S_{j^*})\leq \opt_k(\mathbb{P}_n,1-\alpha+\frac{1}{n}+3\delta+\gamma)$.
\end{enumerate}
Here, $\mathbb{P}_n$ denotes the empirical distribution $\frac{1}{n}\sum_{i=1}^n\delta_{Y_i}$ of the first half of the data.
\end{assumption}
To construct a nested system $\{S_j\}_{j\in[m]}$ that satisfies the above assumption, one only needs to make sure that there exists one subset $S_{j^*}$ in the system that is computed by the dynamic programming (Algorithm \ref{alg:dp}) with confidence level $1-\alpha+n^{-1}+3\delta$ and slack parameter $\gamma$. The rest of the sets in the system can be constructed just to satisfy $\mathbb{P}_n(S_j)=\frac{j}{m}$. In Section \ref{sec:con-nest-m}, we will present a greedy expansion/contraction algorithm that satisfies Assumption \ref{as:ne-un}.

%Though the exact way of construction will not affect the theory, it does have an impact on the practical performance. More discussion on the impact and a recommendation of a specific construction will be given in Appendix \ref{sec:con-nest}.  In particular, we can interpret previous methods as constructing these nested sets in a way that is not volume-aware.  Our construction, on the other hand, takes the volume of the nested sets into account, leading to improved volume efficiency in practice.  
%\vnote{Just added two sentences here about the nested sets.  I think that we should be selling this, since it actually provides a huge improvement over DCP-QR*, but phrasing can be changed.}
With a nested system $\{S_j\}_{j\in[m]}$ satisfying Assumption \ref{as:ne-un}, we can define the conformity score as
\begin{equation}
q(y) = \sum_{j=1}^m\mathbb{I}\{y\in S_j\}. \label{eq:score-un}
\end{equation}
The equivalence between nested system and conformity score was advocated by \cite{gupta2022nested}.
Intuitively, $q(y)$ quantifies the depth of the location $y$. A higher score implies that $y$ is covered by more sets in the nested system, and thus the location should be more likely to be included in the prediction set. Applying the standard split conformal framework, our prediction set based on conformalized dynamic programming is defined by (\ref{eq:pred-set-un}) with the conformity score (\ref{eq:score-un}).

\begin{theorem}\label{thm:unsupervised}
Consider i.i.d. observations $Y_1,\cdots,Y_{2n},Y_{2n+1}$ generated by some distribution $P$ on $\mathbb{R}$. Let $\widehat{C}_{\rm CP-DP}$ be the split conformal prediction set defined by the score (\ref{eq:score-un}) based on a nested system $\{S_j\}_{j\in[m]}$ satisfying Assumption \ref{as:ne-un}. Suppose the parameter $\delta$ in Assumption \ref{as:ne-un} satisfies $\delta\gg \sqrt{\frac{k+\log n}{n}}$. Then the following properties hold.
\begin{enumerate}
\item Coverage: $\mathbb{P}\left(Y_{2n+1}\in\widehat{C}_{\rm CP-DP}\right)\geq 1-\alpha$.
\item Restricted volume optimality: \\$\vol(\widehat{C}_{\rm CP-DP})\leq \opt_k\left(P,1-\alpha+\frac{1}{n}+4\delta+\gamma\right)$ with probability at least $1-2\delta$.
\end{enumerate}
\end{theorem}

We emphasize that Theorem \ref{thm:unsupervised} guarantees both distribution-free coverage and distribution-free volume optimality properties. In practice, $k$ is usually chosen to be a constant for prediction interpretability. By setting $\gamma=O\left(\sqrt{\frac{\log n}{n}}\right)$, the volume sub-optimality is at most $\frac{1}{n}+4\delta+\gamma=O\left(\sqrt{\frac{\log n}{n}}\right)$.


\iffalse
\begin{remark}
A careful reader may notice that the restricted volume optimality is established with high probability. It will be also interesting to obtain a bound for the expected volume $\mathbb{E}\vol(\widehat{C}_{\rm CP-DP})$. In fact, a careful scrutiny of the proof leads to the tail bound
$$\mathbb{P}\left(\vol(\widehat{C}_{\rm CP-DP})> \opt_k\left(P,1-\alpha+\gamma+O\left(\sqrt{\frac{k+t}{n}}\right)\right)\right)\leq 2e^{-t},$$
for all $t>0$. Then, expected volume can be bounded by integrating out the tail probability
$$\mathbb{E}\vol(\widehat{C}_{\rm CP-DP})=\int_0^{\infty}\mathbb{P}\left(\vol(\widehat{C}_{\rm CP-DP})>x\right)dx.$$
Under some weak regularity assumption on $P$, the above integral will lead to $(1+\epsilon')\opt_k(P,1-\alpha+\epsilon)$ for some $\epsilon,\epsilon'>0$, which then can further be bounded by $\opt_k(P,1-\alpha+\epsilon'')$ for some $\epsilon''$. Such a strategy certainly covers distribution classes such as Gaussian mixtures, but in general it should be done case by case.
\end{remark}
\fi

\iffalse
\begin{theorem}\label{thm:unsupervised}
    Given $n$ independent observations $Y_1, \dots, Y_n$ drawn from $P$ and parameters $\alpha, \gamma > 0$, the dynamic programming outputs a prediction set $\widehat{C}(Y_1,\cdots,Y_n) = S_1 \cup \dots \cup S_k$ which is the union of $k$ intervals such that for $Y_{n+1}$ drawn from $P$,
    %$$P(\widehat{C}(Y_1,\cdots,Y_n))\geq 1-\alpha,$$
    $$\Pr(Y_{n+1}\in \widehat{C}(Y_1,\cdots,Y_n))\geq 1-\alpha,$$
    and with high probability,
    $$\vol(\widehat{C}(Y_1,\cdots,Y_n))\leq \opt_k\left(P,1-\alpha+\gamma+\delta_{n,k}\right),$$
    where $\delta_{n,k} = O(\sqrt{k/n} + \sqrt{\log n/n})$.
\end{theorem}

We now show the construction of the prediction set using dynamic programming. First, we consider a set of $n$ data points $X \subset \R$. We aim to find $k$ intervals to cover at least $1-\alpha$ fraction of points in $X$. 
Let $\opt_k(X, 1-\alpha) = \inf \{\vol(C): C \in \calC_k, |C \cap X| \geq (1-\alpha)n \}$ be the optimal coverage volume of $k$ intervals that cover $1-\alpha$ fraction of $X$. We show that the dynamic programming procedure finds $k$ intervals that cover $1-\alpha$ fraction of points and have a near-optimal volume. 



We next show how to use this dynamic programming to construct the prediction set that satisfies the coverage condition and the volume optimality in Theorem~\ref{thm:unsupervised}.

\begin{proof}[Proof of Theorem~\ref{thm:unsupervised}]
Let $P_n$ be the empirical distribution of the given data points $Y_1, \dots, Y_n$. 
Note that the VC dimension of the class of $k$ intervals $\calC_k$ is $2k$. 
By the Talagrand's inequality, we have with probability at least $1- 1/\mathrm{poly}(n)$,
$$\sup_{C\in\mathcal{C}_k}|P_n(C)-P(C)| \leq \delta_{n,k},$$
where $\delta_{n,k} = O(\sqrt{k/n} + \sqrt{\log n/n})$.
 
%For example, one could take $\delta_{n,k}\asymp \frac{k}{\sqrt{n}}$, or perhaps it can be improved to $\sqrt{\frac{k}{n}}$. 
By Lemma~\ref{thm:DP}, the dynamic programming with parameters $\alpha, \gamma$, and $k$ outputs $k$ intervals that satisfies %$\widehat{C}(Y_1,\cdots,Y_n)$ satisfying
$$P_n(\widehat{C}(Y_1,\cdots,Y_n))\geq 1-\alpha,$$
and
$$\vol(\widehat{C}(Y_1,\cdots,Y_n)) \leq \opt_k(P_n,1-\alpha+\gamma).$$
Therefore, we have with probability at least $1-1/\mathrm{poly}(n)$,
$$P(\widehat{C}(Y_1,\cdots,Y_n))\geq 1-\alpha-\delta_{n,k},$$
and 
\begin{eqnarray*}
\vol(\widehat{C}(Y_1,\cdots,Y_n)) &\leq& \opt_k(P_n,1-\alpha+\gamma) \\
&=& \inf\{\vol(C):P_n(C)\geq 1-\alpha+\gamma, C\in\mathcal{C}_k\} \\
&\leq& \inf\{\vol(C):P(C)\geq 1-\alpha+\gamma+\delta_{n,k}, C\in\mathcal{C}_k\} \\
&=& \opt_k(P,1-\alpha+\gamma+\delta_{n,k}).
\end{eqnarray*}

Finally, we adjust the parameters of the dynamic programming to be $\alpha' = \alpha - \delta_{n,k} - 1/\mathrm{poly}(n)$. Then, we have the prediction set outputs by the adjusted dynamic programming satisfies
$$\Pr(Y_{n+1}\in \widehat{C}(Y_1,\cdots,Y_n))\geq 1-\alpha,$$
and with probability at least $1-1/\mathrm{poly}(n)$,
$$\vol(\widehat{C}(Y_1,\cdots,Y_n))\leq \opt_k(P,1-\alpha+\gamma+\delta'_{n,k}),$$
where $\delta'_{n,k} = 2\delta_{n,k} + 1/\mathrm{poly}(n).$
\end{proof}
\fi

% In fact, one could adjust the parameter of the dynamic programming, so that with high probability, we have
% $$P(\widehat{C}(Y_1,\cdots,Y_n))\geq 1-\alpha,$$
% and
% $$\vol(\widehat{C}(Y_1,\cdots,Y_n))\leq \opt_k(P,1-\alpha+\gamma+2\delta_{n,k}),$$
% with high probability. One could integrate out the probability tail to get in-expectation bounds (this requires additional details to be checked).

