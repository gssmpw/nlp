\section{Numerical Experiments}\label{sec:numerical_experiments}

We complement our theoretical guarantees with an evaluation of our methods for both the unsupervised setting of Section~\ref{sec:unlabeled} and the supervised setting of Section~\ref{sec:labeled}. 

\subsection{Construction of Nested Systems}\label{sec:con-nest-m}

We first describe a procedure that generates a nested system $\{S_j\}_{j\in[m]}$ that satisfies Assumption \ref{as:ne-un}. The construction involves the following steps:
\begin{enumerate}
\item \textit{Generate $S_{j^*}$ by Dynamic Programming.} For $j^* = \lceil(1-\alpha + n^{-1} + 3\delta) m \rceil$, we generate $S_{j^*}$ by applying Algorithm \ref{alg:dp} with coverage level $1-j^*/m$ and slack $\gamma=1/m$.
\item \textit{Generate $S_{j^*+1},\cdots,S_m$ by Greedy Expansion.} For each $j>j^*$, we iteratively identify the closest uncovered data point to the boundary of the current $k$ intervals and expand the nearest interval to cover it. Once the intervals cover $\lceil jn/m \rceil$ data points, we define the union as $S_j$ and move on to the construction of $S_{j+1}$.
\item \textit{Generate $S_1,\cdots, S_{j^*-1}$ by Greedy Contraction.} For each $j<j^*$, we iteratively remove a boundary point of the current $k$ intervals that results in the maximum volume reduction. Once the intervals after contraction cover exactly $\lceil jn/m \rceil$ data points, we define the union as $S_j$ and move on to the construction of $S_{j-1}$.
\end{enumerate}

In the supervised setting, the above procedure will be applied to quantiles $Y_1(x),\cdots,Y_L(x)$ computed from $\widehat{F}(\cdot\mid x)$ with $L=m$ for all $x\in\{X_{n+1},\cdots,X_{2n+1}\}$.


\subsection{Comparison in Unsupervised Settings}

The algorithm in Section~\ref{sec:unlabeled} is compared against the method based on kernel density estimation due to \citet{Lei2013DistributionFreePS} and evaluated on several different  distributions. 
Though the original conformalized KDE was proposed in the full conformal framework, we will consider its split conformal version for a direct comparison. We believe the comparison between the full conformal versions of the two methods will lead to the same conclusion.
For the conformalized DP method, the conformity score is constructed based on the nested system described in Section \ref{sec:con-nest-m} with $m=50$ and $\delta = \sqrt{(k + \log n)/n}$. 
The conformalized KDE is also in the form of (\ref{eq:pred-set-un}), with the conformity score given by
$q_{\rm KDE}(x)=\frac{1}{n\rho}\sum_{i=1}^nK\left(\frac{y-Y_i}{\rho}\right)$,
where $K(\cdot)$ is the standard Gaussian kernel and $\rho$ is the bandwidth parameter.
Both methods involve a single tuning parameter, $k$ for conformalized DP and $\rho$ for conformalized KDE.
\begin{figure}[htbp]
    \centering
    \begin{subfigure}[b]{0.48\columnwidth}
        \includegraphics[width=\textwidth]{figures/mix_gaussian3_DP_k3.png}
        \caption{The histogram of the dataset and the prediction set given by conformalized DP with $k=3$ intervals (The first interval is at $[-6.03,-5.97]$.). The volume of the prediction sets is $3.1438$.}
    \end{subfigure}
    \hfill
    \begin{subfigure}[b]{0.48\columnwidth}
        \includegraphics[width=\textwidth]{figures/mix_gaussian3_KDE_bandwidth0.5.eps}
        \caption{The kernel density estimation with bandwidth $\rho=0.5$ and the prediction set given by conformalized KDE \citep{Lei2013DistributionFreePS}. The volume of the prediction sets is $4.4775$.}
    \end{subfigure}
    \hfill
    \begin{subfigure}[b]{0.48\columnwidth}
        \includegraphics[width=\textwidth]{figures/mix_gaussian3_DP.eps}
        \caption{Volumes of prediction sets by conformalized DP is not sensitive to the choice of $k$.}
    \end{subfigure}
    \hfill
    \begin{subfigure}[b]{0.48\columnwidth}
        \includegraphics[width=\textwidth]{figures/mix_gaussian3_KDE.eps}
        \caption{Volumes of prediction sets by conformalized KDE is highly densitive to the choice of $\rho$.}
    \end{subfigure}        
    \caption{Conformal prediction sets on the mixture of Gaussians data from $P = \frac{1}{3}N(-6,0.0001)+\frac{1}{3}N(0,1)+\frac{1}{3}N(8,0.25)$. The coverage probability is $80\%$.
    %The left plot shows the histogram of the dataset and the prediction set given by our proposed conformalized DP with $k=3$ intervals. (The first interval is at $[-6.03,-5.97]$.) The right plot shows the kernel density estimation with bandwidth $\rho=0.5$ and the prediction set by conformalized KDE.  The volume of the prediction sets by conformalized DP and conformalized KDE are $3.1438$ (left) and $4.4775$ (right). 
    The theoretically optimal volume is $3.0178$.}
    \label{fig:intro:unlabeled}
\end{figure}
Figure~\ref{fig:intro:unlabeled} summarizes the results using data generated from a mixture of Gaussians $\frac{1}{3}N(-6,0.0001)+\frac{1}{3}N(0,1)+\frac{1}{3}N(8,0.25)$. Additional experiments on other distributions including standard Gaussian, censored Gaussian and ReLU-transformed Gaussian will be presented in Appendix~\ref{sec:numerical}.


\subsection{Comparison in Supervised Settings}

The algorithms for the supervised setting are compared against conformalized quantile regression (CQR)  \citep{romano2019conformalized},  distributional conformal prediction methods (DCP-QR and DCP-QR*) of \citet{chernozhukov2021distributional}, and CD-Split and HPD-Split methods~\citep{izbicki2022cd} against benchmark simulated datasets in \citet{romano2019conformalized, izbicki2020flexible} (Figures \ref{fig:intro:labeled} and~\ref{fig:exp:bimodal}). The implementation details of all methods are given in Appendix~\ref{sec:numerical}. 
\begin{figure}[H]
    \centering
    \begin{subfigure}{0.48\columnwidth}
        \includegraphics[width=\textwidth]{figures/dcp-qrstar-synthetic.png}
    \end{subfigure}
    \hfill
    \begin{subfigure}{0.48\columnwidth}
        \includegraphics[width=\textwidth]{figures/conformalizedDP-k5-synthetic.png}
    \end{subfigure}
    \caption{Results in the supervised setting on a synthetic data from \citet{romano2019conformalized} for target coverage 0.7.  The left plot shows the output of DCP-QR*, the state of the art method by \citet{chernozhukov2021distributional}, which outputs prediction sets with average volume 1.29.  The right plot shows the output of our method with \(k = 5\) intervals, which achieves a significantly improved average volume of 0.45. }
    \label{fig:intro:labeled}
\end{figure}

\begin{figure}[H]
    \centering
    \begin{subfigure}{0.48\columnwidth}
        \includegraphics[width=\textwidth]{figures/bimodal_hpd_split.png}
    \end{subfigure}
    \hfill
    \begin{subfigure}{0.48\columnwidth}
        \includegraphics[width=\textwidth]{figures/bimodal_dp_k2_c70.png}
    \end{subfigure}
    \caption{Results in the supervised setting on a synthetic data with $20$ dimensional feature from \citet{izbicki2020flexible} for target coverage 0.7. The left plot shows the output of HPD-Split method by~\citet{izbicki2022cd}, with average volume $3.60$. The right plot shows the output of our method with $k=2$ intervals, which has an average volume $3.55$.}
    \label{fig:exp:bimodal}
\end{figure}

\begin{table}[H]
    \centering
    \caption{Comparison on simulated data in~\citet{romano2019conformalized}.}
    \label{tab:conformal_results1}
    \begin{tabular}{p{5cm} p{5cm} p{5cm}}
        \toprule
        \textbf{Method} & \textbf{Average Volume} & \textbf{Empirical Coverage (\%)} \\
        \midrule
        CQR & 1.42 & 70.62 \\
        DCP-QR & 1.48 & 71.60 \\
        DCP-QR* & 1.29 & 71.06 \\
        CD-split & 1.83 & 69.94 \\
        HPD-split & 1.75 & 69.44 \\
        Conformalized-DP ($k$=1) & 1.14 & 74.04 \\
        Conformalized-DP ($k$=5) & \textbf{0.45} & 72.36 \\
        \bottomrule
    \end{tabular}
\end{table}

\begin{table}[H]
    \centering
    \caption{Comparison on simulated data in~\citet{izbicki2020flexible}.}
    \label{tab:conformal_results2}
    \begin{tabular}{p{5cm} p{5cm} p{5cm}}
        \toprule
        \textbf{Method} & \textbf{Average Volume} & \textbf{Empirical Coverage (\%)} \\
        \midrule
        CQR & 4.10 & 71.54 \\
        DCP-QR & 4.04 & 70.85 \\
        DCP-QR* & 4.05 & 69.66 \\
        CD-split & 3.69 & 69.86 \\
        HPD-split & 3.60 & 69.64 \\
        Conformalized-DP ($k$=1) & 4.00 & 68.98 \\
        Conformalized-DP ($k$=2) & \textbf{3.55} & 69.42 \\
        \bottomrule
    \end{tabular}
\end{table}

As shown in Tables~\ref{tab:conformal_results1} and~\ref{tab:conformal_results2}, our method  outperforms previous methods by outputting prediction sets that are unions of intervals. Among all other methods, the CD-split and HDP-split \citep{izbicki2020flexible} are also able to produce unions of intervals. However, since these methods rely on consistent estimation of the conditional density, our conformalized DP still produces prediction sets with smaller volumes. The comparison is more pronounced on the first dataset (see Figure \ref{fig:intro:labeled} and Table~\ref{tab:conformal_results1}), where it would be inappropriate to assume a smooth conditional density, but our method is based on conformalizing the estimation of conditional CDF, and thus works in much more general settings.















These experiments together demonstrate that our methods give a clear advantage over competitive procedures under these different conditions. More crucially, our methods also come with theoretical guarantees for both coverage and volume-optimality with respect to unions of $k$ intervals. 

Finally, we remark that our focus on the family of structured prediction sets conceptually differs from the predominant approach of designing new conformity score functions in existing approaches. 
The prediction sets that are output e.g., unions of intervals, are often more interpretable.
Apart from being more interpretable, the shift in focus to structured prediction sets, which is in the same spirit of \cite{gupta2022nested}, is what allows us to overcome any sample efficiency concerns, to obtain {\em distribution-free volume-optimality}, through {computationally efficient algorithms}. 
