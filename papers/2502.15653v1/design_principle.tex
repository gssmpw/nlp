\section{BBTM Design Principle and Architecture} %System Architecture of BBRM} 
\label{sa} 
%\vspace{-0.5mm} 
In this section, we present the BBTM architecture and design principles by explaining the authorities and describing the advantages of applying blockchain to the SCMS. We also explain the rationale for using two blockchains to separate the distinct functionalities of the GCCF and the GPF, as opposed to using a single blockchain. The acronyms and their explanations used in our BBTM scheme are provided in Table \ref{table:acronyms}.


%In this section, we present the BBTM architecture and design principles by explaining authorities and by describing the advantages of applying blockchain for SCMS and the rationale for using two blockchains to separate two distinct functionalities of GCCF and GPF as opposed to using one blockchain. We present the acronyms with explanations for the terms we use in our BBTM scheme in Table \ref{table:acronyms}.  


\subsection{Advantages of Using Blockchain for Trust Management}

BBTM applies blockchains for two functionalities: i) generating the GCCF from aggregating certificates to add or to revoke the certificate for each authority following the policy to replace the smart contract to make the SCMS simpler and rid the corresponding vulnerabilities, e.g., each authority being the single point of failure and ii) controlling the GPF to replace the smart contract to protect rule of authority and update new rule for authority.
Blockchain is appropriate for such purposes since the transactions within the ledger are immutable, transparent, automatic, and distributed. The immutability (after confirmation and the associated processing delay) enhances trust in transactions within the GCCF and GPF. This means it makes the transactions of generating GCCF through adding and revoking certificates more trustworthy, and it increases the reliability of transactions for generating GPF by adding or deleting authority policies. The transparency provides BBTM with better accountability for each smart contract issuing certificates from upper authorities such as Electors, RCA, ICA, etc and managing GPF by SCMS manager via PG. Blockchain protocol synchronizes the GCCF and GPF automatically and additionally. Finally, the blockchain concentrates on synchronizing data in the ledger through a distributed consensus protocol for GCCF and GPF. Note that SCMS is based on a distributed and decentralized network. BBTM replaces SCMS network divided into two functional authorities to decentralize the SCMS by avoiding the single authority  processing GCCF and GPF.





%\vspace{-2mm}

\subsection{Advantage of Using Two Blockchains Instead of One Blockchain}
%\vspace{-0.5mm} 
BBTM provides two functionalities in SCMS (GCCF and GPF). We use two separate blockchains as opposed to one blockchain because different data formats are stored in each blockchain. Using two blockchains with two functionalities enables synchronization and causality control without extra mechanisms. For instance, If each authority who is in the process of each functionality sends a transaction in each separate blockchain, it can be more trustworthy than using one blockchain. In contrast, BBTM using two blockchains enables a clearer ordering service through the Ordering Service Provider in the permissioned blockchain, as it prevents forks from occurring.




%\vspace{-2mm}
\subsection{BBTM Structure and Entities}
%\vspace{-0.5mm} 

\begin{figure}[t]
\centering
{\includegraphics[width=0.48\textwidth]{graphics/BBTMStructure}}
% ,height=7cm 
\caption{Blockchain-Based Trust Management (BBTM) %\syc{Omit the "Legend" on the top right corner} \as{omitted}
}
\label{fig:BbTM}
%\vspace{-4mm}
\end{figure}

In this section, we describe the structure of BBTM and the authorities considered in BBTM for our proposed approach management. Our BBTM aims at two functions of SCMS manager to generate GCCF to aggregate all certificate processes in SCMS and to manage GPF. Fig.~\ref{fig:BbTM} shows an overview of the proposed BBTM architecture as well as the authorities considered in BBTM. The following authorities are part of the BBTM design: 

\begin{table}[t]
\small
\centering
\begin{tabular} { | m{1.2cm}| m{6cm} | }
\hline
\textbf{Notation} & \textbf{Description } \\\hline 
$S$ & Set of Entities in BBTM Blockchain \\\hline 
$PG$ & Policy Generator \\\hline 
$OSP$ & Ordering Service Provider \\\hline
$GCCF$ & Global Certificate Chain File \\\hline
$GPF$ & Global Policy File \\\hline
$CERT_{i}$ & Certificate of the Entity $i \in S$ \\\hline
$SIGN$ & Signed certificate algorithm \\\hline 
\end{tabular}
\caption{Notations and Variables}
\label{table:notations}
\vspace{-5mm}
\end{table}

\textbf{Elector:} The core element of BBTM, responsible for processing votes to add or revoke Root Certificates (RC) and Elector Certificates (EC) according to the GPF.

\textbf{Root Certificate Authority (RCA):} Another core component of BBTM, the RCA manages the addition and revocation of certificates for authorities such as ICA, MA, and PG, following the GPF.

\textbf{Policy Generator (PG):} Acts as the central manager within SCMS, controlling the GCCF and GPF, and aggregating certificates from all authorities.   

\textbf{Intermediate Certificate Authority (ICA):} A secondary Certificate Authority issued by the RCA, responsible for signing certificates for PCA, RA, ECA, and LA.  

\textbf{Pseudonym Certificate Authority (PCA):} Issues short-term pseudonym identification certificates for vehicles, with the ICA as its issuer.

\textbf{Register Authority (RA):} Validates and processes requests from vehicles through a certificate validation smart contract, issued by the ICA.

\textbf{Enrollment Certificate Authority (ECA):} Issues enrollment certificates for vehicles, containing real identification information such as owner, manufacturer, and color.   

\textbf{End-entity (EE):} Represents vehicles with read-only access to the blockchain, holding an Enrollment Certificate (EC) and a Pseudonym Certificate (PC). The addition and revocation of EE certificates are not covered in this paper, as they are not stored on the blockchain.

\textbf{Ordering Service Provider:} Manages transaction ordering for smart contracts related to GCCF and GPF, and broadcasts transactions to other BBTM authorities.  

Note that the GCCF structure in SCMS includes versioning, endorsements from EBRM, and certificates from entities such as Elector, RCA, ICA, LA, and PCA~\cite{GCCFBenedikt}, while the GPF structure consists of entities, rules, and statuses (alive or dead).  

All components in the Blockchain-Based Trust Management (BBTM) network communicate through a secure channel, as shown by the solid line in Fig.~\ref{fig:BbTM}. In our design, all SCMS components utilize a distributed ledger to represent both the GCCF and GPF, containing historical transactions related to certificate management and policy actions. Blocks aggregate transactions before being added to the blockchains, enabling tracking of certificate generation and revocation.

The GCCF blockchain supports two roles, allowing authorized SCMS members to either read or write via smart contracts. Authorized members receive new transaction blocks from the Ordering Service Provider, while smart contracts manage functions like adding, revoking, and validating certificates. The GPF blockchain restricts access to the PG for adding or deleting policies, with other authorities having read-only access. Consensus algorithms facilitate agreement among authorities on transaction content and order within the BBTM blockchain. We use a distributed consensus protocol designed for consortium blockchains, avoiding less efficient protocols intended for permissionless blockchains.

\captionsetup[table]{skip=12pt}
\begin{table*}[hbt!]
\small
\centering 
{\begin{tabular}{ |l|m{14.5cm}| } 
\hline
\textbf{Name} & \textbf{Description} \\\hline
Version & This attribute contains the version number of the certificate and helps to keep track of the change in the certificate format.  \\\hline
Serial Number & This attribute serves as a unique identifier to distinguish among certificates. \\\hline 
Subject Name* & This field provides information about the entities for which certificate addition or revocation is required, e.g., elector, ICA, MA, and PG, etc. \\\hline 
Issuer Name & This attribute contains the information of the elector who creates and signs the certificate. \\\hline 
Subject Public Key* & This field provides information about the public key of the entities for which certificate addition or revocation is required, e.g., elector, ICA, MA, and PG, etc. \\\hline 
Subject Unique ID* & This attribute contains the unique identifier value of the entities for which certificate addition or revocation is required, e.g., elector, ICA, MA, and PG, etc. \\\hline
Issuer Unique ID & This field contains the unique identifier value of the elector, who creates and signs the certificate.  \\\hline
Validity Period & This attribute holds two values: 1) valid from (determines the date when the certificate becomes valid) and 2) valid to (determines the date after which the certificate is expired). \\\hline
Digital Signature & This attribute contains the digital signature of the elector who creates and signs the certificate. \\\hline 
Algorithm & This field determines the name of the hashing algorithm and the signing algorithm used in the certificate.\\\hline
Function Type & We have two \textit{certificate} functions: i) \textit{AddCert}, and ii) \textit{RevokeCert}, This field contains the name of the specific function. \\\hline
 
\end{tabular}} 

\caption{Certificate Format using blockchain. The Asterisks (*) are the fields that are dependent on the function type and, more specifically, on whether it is for SCMS-authority certificate generation/revocation}

\label{fig:certificateformat}
\vspace{-6mm}
\end{table*}



%\vspace{-3.5mm}




 
\begin{comment}
In this section, we describe the EBRM components using consortium blockchain.  The line communicating between components in the diagram is a secure and access control relationship line. An implementation of the system may not only work with multiple logical roles in SCMS, but also follows the EBRM logical roles. An initial set of electors and the self-signed certificate of the electors are created as a part of process during the launch of the SCMS infrastructure, and then the SCMS manager defines the policy. A blockchain network represents EBRM including logical roles (e.g., adding and revoking EC, and RC). Fig. \ref{fig:ebrt} also shows the functions in EBRM blockchain. There are three types of function in the EBRM using consortium blockchain:



%  The EBRM  was originally designed for V2X use cases \cite{brecht2019security}. However, in SCMS, it is suggested to use different cryptography system between $SCMS$ $Specific$ and $Standard$ $PKI$ $hierarchy$. The EBRM using consortium blockchain is a different cryptography system in $SCMS$ $Specific$ and provides more benefits than the EBRM in the SCMS, such as decentralization, transparency, and availability. We present the acronyms with explanation for the terms we use in our EBRM blockchain scheme in Table \ref{table:acronyms}. 

\subsection{Blockchain Background: Consortium Blockchain and Smart Contract}
\wjf{I suggest to get rid of this section 4.1}

%\section{Blockchain Preliminaries}
%\label{sec:BlockchainPreliminaries}

%\syc{Change and revise this section} as{changed}

Blockchain \cite{nakamoto2008bitcoin} is an append-only ledger where information is recorded in a distributed manner using cryptographic method that gives guarantee of only adding a transaction on ledger but not modifying it. Consortium blockchain are permissioned blockchain where the consortium membership are controlled with identity-based control (e.g., registration, credential, and certificate) and  enabling authentication. By contrast, permissionless blockchain such as those used in cryptocurrencies, i.e. Bitcoin, Ethereum for anonymous and censorless financial transactions, the nodes are self-certified and are encouraged to continuously use new identities (similar to the Sybil attack but not for malicious purpose), making it difficult to implement authentication. In permissionless and consortium blockchain, the block header contains the metadata about the block (e.g. block number, current block hash, previous block header hash etc.). But, in consortium blockchain there is also an additional section, block metadata which contains the certificate and signature of the block creator. %Fig. \ref{fig:blockchainstructure} shows an overview of consortium blockchain structure. 
%\syc{Like the following:} Because consortium blockchains are permissioned and control the consortium membership with identity-based control (e.g., registration, credential, and certificate), it enables authentication. 
%By contrast, permissionless blockchains, such as those used in cryptocurrencies for anonymous and censorless financial transactions, the nodes are self-certified and are encouraged to continuously use new identities (similar to the Sybil attack but not for malicious purpose), making it difficult to implement authentication. 

% \subsection{Smart Contract}
% \label{SmartContract}
%\syc{Move to the Implementation section?} \as{I need to discuss on this, since this a background section, will it be okay to put in the design section where we discussed about the design of our project}
Smart contract is a computerized transaction protocol \cite{szabo1996smart} stored on blockchain-based platform to execute an agreement or, an protocol. Smart contract is integrated with the blockchain at first In Ethereum \cite{buterin2014next}. Smart contracts are shared by all nodes and then transactions are broadcasted to the blockchain network through consensus algorithm. Smart Contract is implemented using logical rules.   

For example, say, there are two ICAs, ICA1 and ICA2. ICA1 is authorized in SCMS while ICA2 is not authorized in SCMS. In this case, the attacker compromising RCA can provide the RCA certificate to ICA2 while keeping the subject name as of ICA1. Then, ICA2 can provide its certificate to their choice of subordinate components keeping the subject name as of the authorized components and this process can continue to the components till bottom level.
In this way, the whole SCMS system can be compromised by the attacker.

\wjf{this paragraph with the Figure 3 used a number of variables, but none of them was defined beforehand.\shb{addressed}}
%Definition\syc{Define consortium blockchain - include that they are also called permissioned and what that means}

Fig \ref{fig:Authentication} shows authentication phase which represents the logical rules and execution process of authentication. Each entity generates verifying information of the entity including $ID_{i}$, $CERT_{i}$, and $K_{i}$ and then encrypts it using the asymmetric encryption $enc$. Channel can decrypt it from each entity and then they will store it in authenticated list. When the entity  transaction $Tx_{i}$  with $enc(info_{i})$, the channel verifies $dec(enc(info_{i}))$ through authenticated list, as well as processes transaction. 

\begin{figure*}[hbt!]
\centering
{\includegraphics[width=0.8\textwidth,height=3.5cm
]{graphics/Authenticatephase.pdf}}
\caption{Authentication Phase }
\label{fig:Authentication}
%\vspace{-4mm}
\end{figure*}


%%We evaluate the approach of using blockchain in EBRM and find some advantages over the EBRM system used in SCMS. The blockchain based EBRM provides a method 
%BBRM, the blockchain-driven EBRM replaces the RCA with the secure ledger in order to make the SCMS simpler and to rid the vulnerabilities of RCA, e.g., single point of failure. In the state of the art SCMS, RCA provides certificates to the SCMS authorities, i.e. ICA, PG and MA (commonly termed as RC). However, in our approach, the RC is not stored in RCA, rather it is stored in a distributed manner using blockchain and the respective components in SCMS get the RC from blockchain, not from RCA. In our case, there is no RCA and the RC is different from the RC in SCMS as the RC is constructed in a distributed manner by electors. So, an attacker cannot compromise the RCA. Another important aspect is that the elector voting management is handled by SCMS manager in SCMS. But, in our case the management role is replaced by blockchain which makes the elector voting management a distributed management replacing the centralized management in SCMS. Blockchain is an append only log where past transactions cannot be deleted, so, once the voting for an EC or RC is reflected in the blockchain, that result cannot be altered. The result can be changed only after another voting round for the same. But, this does not delete the previous voting result from blockchain. This means any member in the blockchain can track down all the history of records from the beginning. But, in SCMS, the certificates can be deleted from the certificate store. Blockchain also provides certificate transparency since the verifiers in the blockchain can view the transaction log and verify the certificates. The availability of the voting result is immediate in the blockchain. The verifiers of the blockchain, i.e. ICA, PG and MA can get the voting result for EC or RC selection or revocation immediately after it is widely available in the blockchain.      
  


%In this subsection, we provide the reason behind using one blockchain instead of two blockchains for two voting cases - EC selection or revocation and RC selection or revocation. If we use two blockchains, one blockchain, say EC blockchain can be used for elector selection or revocation and the second one, say RC blockchain can be used for RC selection or revocation. But, in that case, there arises a synchronization issue meaning that the result or effect, i.e. revoking an EC in EC blockchain should also reflect in the RC blockchain immediately after the members reach to an agreement about the effect. Otherwise, a revoked elector can still participate in the voting process in RC blockchain while it is revoked in the EC blockchain. Using one blockchain can reflect all the effect for EC and RC selection or revocation and make available to all the components in the blockchain immediately after the effect is finalized by the components participating in the blockchain. Another problem is about the management role to synchronize between these two blockchains. One option can be using a dedicated member to manage all the related effect in EC blockchain to RC blockchain. But, in that case, the member can be vulnerable to single point of compromise. Another option can be broadcasting transactions in both blockchain which causes redundancy. Moreover, there is a resource utilization issue where setting up two blockchains in each member can consume more resource than one blockchain. Considering the above factors, it is better to use one blockchain in our case. 


%On the other hands, the electors-driven blockchain replaces the root CA (RCA)\wjf{does that mean the electors create a distributed PKI?}, meaning that instead of the root CA self-certifying itself, the electors construct the root certificate in a distributed manner using blockchain, that will get distributed to the components in SCMS. Each of the votes is in the forms of certificates/transactions (built on X.509) where greater than n (majority votes) certificates/transactions supporting the same root certificate payload (the same subject-public-key binding in the payload but with n+1 distinct issuers) mean that the corresponding root certificate becomes effective. In other words, the electors issue certificates where there are n+1 or more certificates needed for it to either become a valid root certificate or revoke an existing root certificate. In the voting process, there is only positives votes means when an elector supports a root certificate then only it provides vote for the certificate. Adding negative votes has additional complexities such as increasing the blockchain size, more complex counting process whereas adding positive votes makes it more simpler.


%SCMS Manager is the entity such as the governmental agency of DoT or NHTSA, governing and overseeing the SCMS operations \cite{brecht2018security}. The SCMS Manager has management role in EBRM, but our research contribution replaces that role using a blockchain and thus decreases the load on the vehicle-relevant governmental agencies. In BBRM, SCMS manager has the role of policy making and governing the policies. Since the SCMS Manager is not directly involved in the blockchain, it is not included in Fig. \ref{fig:ebrt}. 

\end{comment}
