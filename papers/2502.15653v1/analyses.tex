\section{Experimental Result}
\label{analy}
In this section, we show the analyses of BBTM and our system to show that it is appropriate for the SCMS. We analyze the transaction scalability in relation to the SCMS, blockchain size, resource consumption, throughput, and transaction time, as well as communication latency between vehicles and RA on 5G and LTE networks.    

\captionsetup[table]{skip=12pt}
\begin{table*}[hbt!]
\small
\centering 
{\begin{tabular}{ |p{2cm}|p{1.6cm}|p{1.6cm}|p{2cm}|p{2.5cm}|p{2.5cm}|p{2cm}| } 
\hline
 & \textbf{RAM (\%)} & \textbf{CPU (\%)}& \textbf{BLockchain size (kB)}& \textbf{Thoughput(Tx/s)}& \textbf{Thoughput(kB/s)}& \textbf{Transaction latency (s)}\\\hline
BBRM~\cite{sarker2021blockchain} & 31.80 & 4.2 & 62.8772 & 2.7926421 & 15.140614757501 & 358.658 \\\hline
BBTM(GCCF) & 18.80 & 5.8 & 84.3912 & 2.142140 & 13.58842377 & 624.714 \\\hline
BBTM(GPF) & 18.80 & 5.8 & 81.131 & 2.55033 & 14.2740025815 & 621.341 \\\hline
\end{tabular}} 
\caption{Blockchain Performance}
\label{tlb:Block}
%\vspace{-6mm}
\end{table*}

\subsection{Transaction Scalability and Blockchain size}
BBTM is for GCCF and GPF management within SCMS, meaning that vehicle certificate is out of our scope. Therefore, the application does not require high transaction scalability. SCMS is designed and introduces the update of authorities' certificates 1 to 12 years~\cite{TimelineBenedikt}.
In this section, we study the transaction scalability and the blockchain size rate in kB per event.
The event defines each function of GCCF and GPF. BBTM, with its emphasis on the trust management of SCMS, exhibits transaction scalability comparable to or below that of most other blockchain implementations and applications. Its scalability requirements align with the default capabilities provided by Hyperledger Fabric. It's noteworthy that the performance of the Hyperledger Fabric blockchain is contingent on various factors, including block size, transaction size, network size, and hardware configuration, which Hyperledger fabric support ledger growth rate of 20,000 transactions per second with 2.9 kB per transaction, corresponding 58,000 kB/s or 1.829 $\times$ $10^{12}$ kB/year. In Table~\ref{tlb:Block}, for each event of BBTM(GCCF) and BBTM(GPF), the blockchain size growth rate is 1.3 $\times$ $10^{-7}$ and 14 $\times$ $10^{-7}$ kB/year respectively. The per-event scalability for BBTM is considerably lower, approximately ten orders of magnitude smaller, than the capacity supported by Hyperledger Fabric. 

\subsection{Transaction Throughput}
Transaction throughput serves as a metric for gauging the speed of a blockchain system. In the context of blockchain, it represents the rate at which valid transactions are successfully committed by the blockchain network over a specified time period. In our scenario, we assess the transaction throughput specifically for five functions in BBTM, as transactions sent through this function have the potential to alter the blockchain state. The calculation of transaction throughput is expressed in both transactions per second (Tx/s) and kilobytes per second (kB/s). The calculation of transaction latency involves deriving the average from five iterations, each comprising the generation of 1000 transactions. From Table~\ref{tlb:Block}, it can be seen that the transaction throughput for BBTM (GCCF) and BBTM (GPF) are  $\approx$ 2.1 and 2.5 Tx/s and 13 and 14 kB/s respectively. Each event of BBTM (GCCF) and BBTM (GPF) has a different transaction throughput because the blockchain size of GCCF and GPF is different. The blockchain size in BBRM is smaller than our blockchain size in BBTM (GGCF) and BBTM (GPF). Therefore, it only takes a few seconds to complete one event.      

\subsection{Resource Consumption}
We have also calculated the resource consumption (CPU usage, RAM) for running the blockchain for two different scenarios. All authorities can host various applications, each responsible for executing all the functions associated with a BBTM (GCCF) and BBTM (GPF), thereby consuming the required resources.  This blockchain application will indeed consume resources, and as a result, the measurement can provide insights into the configuration requirements for running all authorities. We have 4 GB memory for each of Google Cloud servers. We have analyzed the percentage of RAM utilized in each machine for two distinct scenarios, such as BBTM (GCCF) and BBTM (GPF). It is noticeable that the percentage of RAM usage has also decreased by more than 2x for BBRM, as it constitutes 31.8\% of the total 2048 GB RAM, while BBTM utilizes 18.8\% of the 4056 GB RAM. However, the percentage of CPU did not change in all scenarios. The results are shown in Table \ref{tlb:Block}.

\subsection{Transaction and Communication Latency}
Within blockchain systems, transaction latency signifies the duration required for a transaction to gain acceptance across the network. This encompasses the timeframe from a client initiating a transaction to the moment it achieves widespread availability in the network. To elaborate, transaction latency encompasses the time taken for a client to submit a transaction to peers, the execution of the transaction by these peers, the transmission of responses from peers to the client, the dissemination of the transaction and responses to the Ordering Service Provider (OSP) by the client, and the subsequent delivery of the transaction and responses to all peers by the OSP. The calculation of transaction latency involves deriving the average from five iterations, each comprising the generation of 1000 transactions. From Table~\ref{tlb:Block}, it can be seen that the transaction latency for BBTM (GCCF) and BBTM (GPF) are 624.714s and 621.341s respectively. It can be derived from the analysis of cloud-based services that are increasing Transaction latency than BBRM.  

In our case, communication latency is defined as the total time of verifying the certificate from EE to RA. We evaluate the communication latency using various statistical measures: the median, representing the average value of total communication latency; the Interquartile Range (IQR), indicating the stability of connectivity; whiskers, illustrating less stable connectivity; and crossed-outlier points, highlighting unstable connectivity. Additionally, a combination of the median and IQR is used to depict the skewness of latency. In Fig.~\ref{fig:lat}, the medians for communication latency on 5G and LTE are 0.159 and 0.173 ms, respectively. The box on 5G and LTE ranges from 0.149 to 0.164 ms and from 0.16 to 0.193 ms, respectively. Out-of-range latency on 5G is 0.125 to 0.188 ms, and on LTE is 0.114 to 0.243 ms, respectively.



% The standard average size of each TCP packet is $\approx$ (20-60) bytes~\cite{udpvstcp}. Thus it needs only 1-1.5 packets to send in RTT but in our case it needs $\approx$ (29-98) TCP packets. Though the number of packet increases from 29 to 98 packets, the latency increases only 2-2.5 times compared with RTT according to the Fig.~\ref{fig:rttvsas5g} and Fig.~\ref{fig:rttvsaslte}. The total latency of our application-specific protocol in 5G and LTE ranges from 111.48 ms to 568.44 ms and 162.86 ms to 640.18 ms respectively. Though the total number of rounds to achieve the required performance varies in federated learning, if we assume the total number of rounds, for example, 10,000 to achieve the required performance in federated learning then the total latency for the verification in our case ranges from 1114800 ms (1114.8 seconds/ 18.58 minutes) to 5684400 ms (5684.4 seconds/94.74 minutes) in 5G and 1628600 ms (1628.6 seconds/ 27.1433333 minutes) to 6401800 ms (6401.8 seconds/ 106.69 minutes) in LTE. We also show the computing latency and network latency comparison in Fig.~\ref{fig:lat} for 5G and LTE where network latency is the dominant factor by the factor of 65 and 76 for 5G and LTE respectively. \textbf{need to plot graphs on this part}


% \begin{figure}[t]
% \centering
% {\includegraphics [width=0.5\textwidth]{graphics/rtthops1}}
% \caption{ Round-trip time and hops between vehicle and FLS
% }
% \label{fig:rtthops}
% \end{figure}



% \begin{figure}[t]
% \centering
% {\includegraphics [width=0.5\textwidth]{graphics/totallatency5Gbar}}
% \caption{Total Latency in 5G between vehicle and FLS considering different locations of FLS
% }
% \label{fig:5g}
% \end{figure}




% \begin{figure}[t]
% \centering
% {\includegraphics [width=0.5\textwidth]{graphics/totallatencyltebar}}
% \caption{Total Latency in LTE between vehicle and FLS considering different locations of FLS
% }
% \label{fig:lte}
% \end{figure}



\begin{figure}[t]
\centering
{\includegraphics [width=0.5\textwidth]{graphics/BBTMcomlatency}}
\caption{Communication latency between vehicle and RA, executed ever cellular networks (5G and LTE) during driving 
}
\label{fig:lat}
\end{figure}





% \begin{figure*}[h]
% \centering
% \subcaptionbox{Round-trip time and hops between vehicle and servers 
% \label{fig:rtthops}}{\includegraphics[width=.45\textwidth]{graphics/rtthops}}\hfill
% \subcaptionbox{Driving Route and Base Station Location
% \label{fig:drbs}}{\includegraphics[width=.45\textwidth, height=45mm]{graphics/mapsfinal}}\\
% \caption{Our Experimental Results} 
% \label{fig:oer}
% \end{figure*}


% \begin{figure*}[h]
% \centering
% \subcaptionbox{Round-trip time and hops between vehicle and servers 
% \label{fig:rtthops}}{\includegraphics[width=.5\textwidth]{graphics/rtthops}}\hfill
% \subcaptionbox{Networking Latency between vehicle and FL Server, SCMS PKI and FL PKI
% \label{fig:drbs}}{\includegraphics[width=.5\textwidth]{graphics/alllatency}}\\
% \caption{Experimental Results on RTT, Hops and Networking Latency} 
% \label{fig:lat}
% \end{figure*}


% \begin{figure}[t]
% \centering
% {\includegraphics [width=0.5\textwidth]{graphics/rtthops}}
% \caption{ Round-trip time and hops between vehicle and servers
% }
% \label{fig:rtthops}
% \end{figure}


% \begin{figure}[t]
% \centering
% {\includegraphics [width=0.5\textwidth]{graphics/alllatency}}
% \caption{Networking Latency between vehicle and FL Server, SCMS PKI and FL PKI
% }
% \label{fig:lat}
% \end{figure}

