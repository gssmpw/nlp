\section{Introduction}
\label{intro}

The vehicular networking technology has the potential to significantly improve the connected vehicle with the cellular network that accesses the cloud services and shares the information. Typically, it is significant for Vehicle-to-Everything (V2X) and Vehicle-to-Vehicle (V2V) communication between vehicles, devices, and infrastructures to provide secure network and privacy because the continuous broadcast of Basic Safety Messages (BSMs) potentially reduce unimpaired vehicle accidents by 80\%~\cite{brecht2019security}. In this regard, the US Department of Transportation (USDOT) proposed a project called Security Credential Management System (SCMS) for V2X communication to support these security requirements of vehicular networking. Vehicles broadcast BSMs to support V2X safety applications such as autonomous vehicles or connected autonomous vehicle. The backbone of the autonomous vehicle system is the onboard intelligent processing capabilities using the real-time data collected through V2X communication and sensors such as cameras, RADAR, LIDAR, and mechanical control units. This data includes BSMs (sender’s time, position, and speed etc.), traffic flow, road, and network conditions, etc. The sending vehicle signs each BSM and the receiving vehicle verifies the signed BSM to prevent an attacker from injecting false messages.
 

%\vspace{-0.5mm}
In SCMS, a Public Key Infrastructure (PKI) is proposed to issue pseudonym certificates to vehicles and infrastructure devices for maintaining reliable communication between them by dividing the generation and provisioning process of those certificates among multiple organizations. However, a recent project has shown many limitations of SCMS such as short-lived pseudonym CA, non-guaranteed quality of service (QoS), short-lived infrastructure-to-vehicle (I2V), and unbounded channel access delay~\cite{furtado2018threat}~\cite{qayyum2020securing}. SCMS provides Trust Management (TM) via sophisticated processes, such as the Elector-based Root Management (EBRM) and pseudonym Certificate Authority (PCA). TM introduces multiple distributed sets of authorities to generate and govern the certificate chain and policy generation files (required to generate and revoke the certificates for other authorities). Our work focuses on the TM providing the Chain of Trust for the PKI and managing the policy for authorities. While TM is designed to be resilient against authority compromise and provides distributed management using a sophisticated structure~\cite{brecht2019security}, our work using blockchain provides better resiliency and avoids the single point of failure in SCMS.          



\textbf{Contribution.}
Our work enhances the SCMS PKI based on distributed PKI management by introducing a blockchain for Blockchain-Based Trust Management (BBTM). BBTM replaces the Policy Generator with a single blockchain divided into functional smart contracts: 1) Global Certificate Chain File (GCCF) smart contract, which manages certificate addition and revocation, and securely shares GCCF among SCMS authorities; 2) Global Policy File (GPF) smart contract, which manages SCMS authorities' operations efficiently and securely. BBTM offers several security advantages, including improved decentralization in PKI, greater resilience against single points of failure, and enhanced transparency in certificate, policy, and CRL transactions. Additionally, BBTM simplifies the SCMS architecture by replacing the Policy Generator and reducing the SCMS manager's role in managing GCCF, policies, and CRLs through the blockchain protocol. 


\textbf{Paper Organization.}
Section \ref{sec:background} describes the state of the art SCMS for vehicular networking PKI, blockchain-Based Root Management, and blockchain. Our work is based on the SCMS PKI system. Section \ref{ps} defines the contribution scope and threat model of our approach. Section \ref{sa} describes the design principle and system architecture of the proposed approach while Section \ref{sec:bbtmdesign} provides the actual design including transaction, functions of GCCF and GPF, and network setup in BBTM. Section~\ref{secur} and Section \ref{impl} presents the security analysis and the implementation details of the proposed design and the experimental result. Section \ref{rw} reviews the related work and Section \ref{conclusion} concludes the paper.




% \textbf{• Agreement of Root Authority (RA) of trust between SCMS and FL} FL and SCMS build agreement of sharing chain of certificate in SCMS and FL to correspond between SCMS PKI and FL PKI. Therefore, FL PKI is able to verify vehicular PC.     

% \textbf{• Providing users' privacy through PC} FL can provide data privacy. However, it requires to verify users' identities. However, vehicle can use PC in SCMS, which provides user' privacy. 


% During Fourth industrial revolution, the connectivity between Internet of Vehicle (IoV) has potential to significantly improve the Connected Vehicle (CV) with the 5G technologies that access the cloud services and share information between vehicles and infrastructures \cite{han2019exploiting} \cite{pokhrel2020improving}. Typically, it is significant for Vehicle-to-Vehicle (V2V) and Vehicle-to-Everything (V2X) communication between vehicles, devices, and infrastructures to provide secure network and privacy because the continuous broadcast of Basic Safety Messages (BSMs) potentially reduce unimpaired vehicle accidents by 80\% \cite{brecht2019security}. In this regard, the US Department of Transportation (USDOT) proposed a project called Security Credential Management System (SCMS) for V2X communication. Vehicles broadcast BSMs to support V2X safety applications such as autonomous vehicle or a Connected Autonomous Vehicle (CAV). The backbone of autonomous vehicles system is the onboard intelligent processing capabilities using the real-time data collected through V2X communication and sensors such as cameras, RADAR, LIDAR, and mechanical control units. This data includes BSMs (sender’s time, position, and speed etc.), traffic flow, road, and network conditions etc. The self-driving vehicles evaluate each receiving message, verify the signature since an incorrect/malicious message can even jeopardize the CAV system and lead it to make wrong decision. Each BSM is signed by the sending vehicle and the signed BSM is verified by the receiving vehicles to prevent an attacker from injecting false messages. In SCMS, a Public Key Infrastructure (PKI) is proposed to issue pseudonym certificates to vehicles and infrastructure devices for maintaining reliable communication between them by dividing the generation and provisioning process of those certificates among multiple organizations. However, a recent project has shown many limitations of SCMS such as short-lived pseudonym CA, non-guaranteed quality of service (QoS), short-lived infrastructure-to-vehicle (I2V), and unbounded channel access delay \cite{furtado2018threat} \cite{qayyum2020securing} .
