\section{Evaluation}
\label{impl}

\captionsetup[table]{skip=12pt}
\begin{table*}[hbt!]
\small
\centering 
{\begin{tabular}{ |p{2cm}|p{1.6cm}|p{1.6cm}|p{2cm}|p{2.5cm}|p{2.5cm}|p{2cm}| } 
\hline
 & \textbf{RAM (\%)} & \textbf{CPU (\%)}& \textbf{BLockchain size (kB)}& \textbf{Thoughput(Tx/s)}& \textbf{Thoughput(kB/s)}& \textbf{Transaction latency (s)}\\\hline
BBRM~\cite{sarker2021blockchain} & 31.80 & 4.2 & 62.8772 & 2.7926421 & 15.140614757501 & 358.658 \\\hline
BBTM(GCCF) & 18.80 & 5.8 & 84.3912 & 2.142140 & 13.58842377 & 624.714 \\\hline
BBTM(GPF) & 18.80 & 5.8 & 81.131 & 2.55033 & 14.2740025815 & 621.341 \\\hline
\end{tabular}} 
\caption{Blockchain Performance}
\label{tlb:Block}
%\vspace{-6mm}
\end{table*}

\subsection{Experimental Setup}
In this section, we outline the implementation details for BBTM. We developed a BBTM prototype using Hyperledger Fabric (version 2.0) to perform transactions on Google Cloud servers. This Proof-of-Concept (PoC) is essential for enabling vehicle communication with RA via cellular networks while driving. The experiment is conducted on a Google Cloud servers running Ubuntu, with a laptop positioned in a vehicle. We utilize LTE and 5G networks from a major mobile operator in the USA to facilitate communication between the vehicle and RA.  
\subsection{Experimental Results}
In this section, we show the analyses of BBTM and our system to show that it is appropriate for the SCMS. We analyze the transaction scalability in relation to the SCMS, blockchain size, resource consumption, throughput, and transaction time, as well as communication latency between vehicles and RA on 5G and LTE networks.

\subsubsection{Transaction Scalability and Blockchain size}
BBTM is for GCCF and GPF management within SCMS, meaning that vehicle certificate is out of our scope. Therefore, the application does not require high transaction scalability. SCMS is designed to manage the lifecycle of authorities' certificates, which can range from 1 to 12 years~\cite{TimelineBenedikt}.

In this section, we examine transaction scalability and the blockchain size growth rate in kB per event. Each event corresponds to a function of the Global Certificate Chain File (GCCF) and Global Policy File (GPF). BBTM emphasizes trust management in SCMS and demonstrates transaction scalability that is comparable to or lower than most other blockchain implementations. Its scalability aligns with the default capabilities of Hyperledger Fabric, which supports a ledger growth rate of 20,000 transactions per second at 2.9 kB per transaction, equating to 58,000 kB/s or approximately 1.829××10121012 B/year. As shown in Fig.~\ref{fig:bsize}, the blockchain size growth rates for BBTM(GCCF) and BBTM(GPF) are 1.3 $\times$ $10^{-7}$ and 14 $\times$ $10^{-7}$ kB/year respectively. Notably, the per-event scalability of BBTM is about ten orders of magnitude smaller than the capacity supported by Hyperledger Fabric.

\subsubsection{Transaction Throughput}

Transaction throughput is a key metric for evaluating the speed of a blockchain system, representing the rate at which valid transactions are committed by the network over a specific time period. In our study, we assess transaction throughput for five functions in BBTM, as these transactions can alter the blockchain state. Throughput is measured in transactions per second (Tx/s) and kilobytes per second (kB/s). We calculate transaction latency by averaging results from five iterations, each generating 1,000 transactions. As shown in Table~\ref{tlb:Block}, the transaction throughput for BBTM (GCCF) and BBTM (GPF) is approximately 2.1 and 2.5 Tx/s, and 13 and 14 kB/s, respectively. The differences in transaction throughput between BBTM (GCCF) and BBTM (GPF) are due to variations in blockchain size, with BBTM having a smaller size than GCCF and GPF, allowing events to complete in just a few seconds.


\begin{figure}[t]
\centering
{\includegraphics [width=0.41\textwidth]{graphics/BBTMcomlatency}}
\caption{Communication latency between vehicle and RA, executed ever cellular networks (5G and LTE) during driving 
}
\label{fig:lat}
\end{figure}

\subsubsection{Resource Consumption}
We have also calculated the resource consumption (CPU usage, RAM) for running the blockchain for two different scenarios. All authorities can host various applications, each responsible for executing all the functions associated with a BBTM (GCCF) and BBTM (GPF), thereby consuming the required resources.  This blockchain application will indeed consume resources, and as a result, the measurement can provide insights into the configuration requirements for running all authorities. We have 4 GB memory for each of Google Cloud servers. We have analyzed the percentage of RAM utilized in each machine for two distinct scenarios, such as BBTM (GCCF) and BBTM (GPF). It is noticeable that the percentage of RAM usage has also decreased by more than 2x for BBRM, as it constitutes 31.8\% of the total 2048 GB RAM, while BBTM utilizes 18.8\% of the 4056 GB RAM. However, the percentage of CPU did not change in all scenarios. The results are shown in Fig.~\ref{fig:per}.

\begin{figure}[t]
\centering
{\includegraphics [width=0.4\textwidth]{graphics/performance.pdf}}
\vspace{-6mm}
\caption{Blockchain Performance 
}
\label{fig:per}
\end{figure}

\subsubsection{Transaction and Communication Latency}
"In blockchain systems, transaction latency refers to the time required for a transaction to be accepted across the network. This includes the period from when a client initiates a transaction to its widespread availability. Specifically, transaction latency encompasses the time taken for a client to submit a transaction to peers, the execution of that transaction, the return of responses to the client, and the subsequent dissemination of the transaction and responses to the Ordering Service Provider (OSP), which then delivers them to all peers. We calculate transaction latency by averaging results from five iterations, each involving the generation of 1,000 transactions. As shown in Table~\ref{tlb:Block}, the transaction latencies for BBTM (GCCF) and BBTM (GPF) are 624.714s and 621.341s, respectively. Our analysis indicates that cloud-based services contribute to increased transaction latency compared to BBRM.

In our study, communication latency is defined as the total time required to verify the certificate from the EE to the RA. We evaluate communication latency using several statistical measures: the median represents the average latency; the Interquartile Range (IQR) indicates connectivity stability; whiskers show less stable connectivity; and crossed-outlier points highlight instability. The combination of the median and IQR also illustrates latency skewness. As shown in Fig.~\ref{fig:lat}, the medians for communication latency are 0.159 ms for 5G and 0.173 ms for LTE. The IQR for 5G ranges from 0.149 to 0.164 ms, while for LTE, it ranges from 0.160 to 0.193 ms. Out-of-range latencies are 0.125 to 0.188 ms for 5G and 0.114 to 0.243 ms for LTE.