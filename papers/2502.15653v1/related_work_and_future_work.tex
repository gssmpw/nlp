\section{Related Work}
\label{rw}
In this section, we describe the related work of blockchain for decentralized PKI and the application of certificate management for vehicular networks.

\textbf{Blockchain for Decentralized PKI:} Blockchain is used to create a decentralized Public Key Infrastructure (PKI), improving resilience against central authority failures. In papers ~\cite{plessing2020revisiting, axon2016pb}, a blockchain-based PKI (BB-PKI) implemented via a CertCoin fork enhances anonymity by separating user identity from certificates. The paper~\cite{sermpinis2021detract} introduces DeTract, which enables web servers to independently generate X.509 certificates using uPort, an Ethereum-based identity platform for managing certificate transactions. Additionally, this paper~\cite{toorani2021decentralized} presents a blockchain-based Web of Trust (WoT) system that eliminates the need for a centralized Certificate Revocation List (CRL), while this paper~\cite{adja2021blockchain} improves X.509 validity mechanisms based on distribution points to reduce time delays. Finally, this paper~\cite{turan2024semi} presents SemiDec-PKI, a blockchain-based infrastructure that enhances fault tolerance and security for various certificate types by integrating a Web of Trust with a centralized model.



%  that was defined in PKI Request for Comments (RFC) 5217. The objective of the framework is to provide operational requirements between PKI authorities to establish trust relationships. In which, RFC 5217 defines the relationship between certification authority of independent PKI as ‘cross-certification.’ Cross-certification can be direct where there are only two independent PKI domains or if there are more than two, then RFC 5217 suggests the relationship to be either unilateral or bilateral; our work can be related closely to bilateral. As for the architecture of cross-certification, RFC 5217 suggested using a neutral CA as a bridge between the independent PKI. Our work differs from those architectures suggested in RFC 5217 because we establish trust at the highest level of authority between two PKIs, where in this context it would be between SCMS and FL PKI managers. The complexity of PKI-like SCMS where the RCA is also connected to multiple authorities within its organization, makes it important to consider the security implications if it’s compromised by a third party. Therefore, we propose a design where other collaborators of PKIs only are connected to authorities that are essential according to the agreement between the PKI managers. This would hence limit the risk of data and privacy being compromised within an organization.
\textbf{Application of certificate management for Vehicular Networking:} Application of certificate management is applicable for blockchain to enhance security aspect. In this paper\cite{khan2020accountable}, the Accountable Credential Management System (ACMS) enhances vehicular communication security by transparently managing certificates and ensuring trustworthiness without relying on traditional certificate revocation lists. This paper~\cite{byunx2023privacy} presents the challenge of establishing trust between these two separate PKIs for vehicular networks. In this paper~\cite{byun2024secure}, there is multi-authority management between the federated learning server and SCMS for the vehicular network. In \cite{lei2020blockchain}, the author describes the growing importance of security and privacy in Vehicular Communication Systems (VCS) within the Intelligent Transportation Systems (ITS).      

% \section{Discussion and Future work}
% \label{dfw}
% \textbf{Applying for Other Vehicular Application using SCMS}



% \subsection{Multi-Domain PKI Application}
% An example of the application of the RFC 5217 multi-domain PKI would be in the context of electrical grid infrastructure. \cite{vaidya_multi-domain_2015} illustrates the usage of multi-domain PKI through the electric vehicle network (EV) or vehicle-to-grid (V2G). The current V2G network consists of various electric power services that are independent organizations by themselves that provide energy and services to electric vehicles at charging stations. The overall V2G PKI model, however, is based on a hierarchical trust model. Hence, trust between inter-domain PKI is not supported where the author claims that it will increase the memory space to store root certificates and increases the time for validation between CAs. Therefore, the paper proposed a PKI model that addresses those shortcomings through peer-to-peer (P2P) trust relationships in their cross-certification. While this approach is novel, the architecture of this P2P is still using root CA as the main component of cross-certification. As we have highlighted our contribution in earlier sections and the differences in our concept architecture from RFC 5217’s multi-domain PKI, the P2P V2G network architecture also deviates from the suggested architecture of multi-domain PKI.

% VANET has strict requirements when it comes to security and privacy. This is to ensure the safety of road users and to prevent catastrophic events caused by bad actors if security is compromised. Therefore, there is multiple research  over the years in the domain of authentication schemes within VANET. Within the context of our paper, regardless of the categories, the purpose of privacy preserving authentication schemes is to be able to authenticate legitimate users and have a specific protocol in identifying malicious attackers (both insider and outsider). However, with future development and research in the intelligent transportation system (ITS) \cite{kaffash_big_2021} \cite{gautam_analysis_2021} that would provide a variety services that includes infotainment\cite{oche_vanets_2020}, federated learning \cite{posner_federated_2021} and etc, it is important to address the authentication schemes beyond the context of vehicular networking and into the trust relationship between organizations.


% earlier sections we have covered the overview of FL; it is crucial that we acknowledge the potential threat on data privacy where FL can be vulnerable to inference attacks. \cite{truex_hybrid_2019} \cite{fereidooni_safelearn_2021} This vulnerability would expose data information as well as the source of the trained model through inference. \cite{noauthor_practical_nodate} mentioned about the need of PKI to prevent impersonation of clients. With that, we believe by implementing PKI in FL and defining the trust relationship in that multi domain PKI concept would give additional layer of protection of data privacy between independent PKI domains.

% Although the main scope of this paper focuses on the interaction between SCMS-PKI and FL PKI, it is important to reiterate that our contribution lies within the context of the interaction between independent PKI to establish trust while taking in consideration of the properties and components of the PKI domain. It is crucial that the architecture includes the requirements of organization in terms of privacy to be the building block on the design of the interaction.

% \textbf{Multi-Domain PKI}


