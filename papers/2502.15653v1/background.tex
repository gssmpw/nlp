\section{Preliminaries} 
\label{sec:background}

We provide an overview of the SCMS in Section~\ref{sec:scms} and then focus on TM, management of authorities in the SCMS, since our research focuses on BBTM to advance the SCMS design.

\begin{table}[t]
\small
\centering
\begin{tabular} { |p{2cm} | p{5.5cm} | }
\hline 
\textbf{Acronym} & \textbf{Explanation}  \\\hline 
RCA & Root Certificate Authority \\\hline 
ICA & Intermediate Certificate Authority \\\hline
ECA & Enrollment Certificate Authority \\\hline
PG & Policy Generator \\\hline 
MA & Misbehaviour Authority \\\hline
RA & Registration Authority \\\hline
PCA & Pseudonym Certificate Authority \\\hline 
DCM & Device Configuration Manager \\\hline
OSP & Ordering Service Provider \\\hline 
GCCF & Global Certificate Chain File \\\hline
GPF & Global Policy File \\\hline
BBRM & Blockchain-Based Root Management \\\hline  
EBRM & Elector-Based Root Management \\\hline 
BBTM & Blockchain-Based Trust Management \\\hline 
SCMS & Security Credential Management System \\\hline 
\end{tabular}
\caption{Acronyms}
\label{table:acronyms}
%\vspace{-10mm}
\end{table} 

\subsection{Security Credential Management System} %(SCMS)}
\label{sec:scms}
%\vspace{-0.5mm}
The Security Credential Management System (SCMS)~\cite{brecht2018security}~\cite{whyte2013security} is a Public Key Infrastructure (PKI) that is used for issuing digital certificates to vehicles, infrastructure, and pedestrian/mobile devices involved in vehicle-to-everything (V2X) communications. The purpose of this PKI system is to enable secure and private communications between these different entities participating in the V2X network. The SCMS adopts a robust threat model to protect the privacy of vehicles. Beyond considering a standard adversary who may access the V2X communication channels when the digital certificates are being used, the SCMS also considers the possibility of an honest-but-curious adversary. This type of adversary would compromise the authority entities involved in the SCMS operations during the process of generating the digital certificates for vehicles and other entities. The SCMS is designed to safeguard against this stronger adversary model, not just the basic eavesdropping threat~\cite{chen2017protecting,brorsson2018guarding}. The strong threat model adopted by the SCMS motivates the use of a decentralized PKI (Public Key Infrastructure) architecture. This is to ensure a separation of the different SCMS authority entities so that a compromise of any single authority does not lead to a breach of vehicular privacy. Therefore, the SCMS provides a distributed PKI system involving multiple authorities, rather than a traditional centralized PKI which has a single Certificate Authority responsible for all certificate management tasks like signing, generation, and issuance. This decentralized, multi-authority approach is a key design choice of the SCMS to better protect the privacy of vehicles and other entities participating in the V2X communication network, even in the face of an honest-but-curious adversary compromising one of the authorities. Fig. ~\ref{fig:scmsarc} illustrates the different authority entities involved in the SCMS and the interactions between them. The lines connecting the entities represent the various interactions and operations that take place as part of the SCMS processes.

\begin{figure}[t]
\centering
\includegraphics[width=0.45\textwidth]{graphics/scms_structure}
\vspace{-2mm}
\caption{SCMS Authorities and Interactions} 
\label{fig:scmsarc}
\vspace{-5mm}
\end{figure}


%\vspace{-2mm}
\subsection{Blockchain-based Root Management}
\label{ebrm}
%\vspace{-0.5mm} 
In  SCMS, devices such as Registration Certificate Authority (RCA), Intermediate Certificate Authority (ICA), Pseudonym Certificate (PG), Management Authority (MA), Enrollment Certificate Authority (ECA), Root Authority (RA), Pseudonym Certificate Authority (PCA), Linkage Authority 1 (LA1), Linkage Authority 2 (LA2), and End Entity (EE) create a standard Public Key Infrastructure (PKI) hierarchy. This hierarchy maintains a chain of trust, which is an ordered list of certificates that allows the receiver to verify all certificates up to the top of the chain, among the authorities in SCMS and EE devices. Elector-based Root Management (EBRM)~\cite{sarker2021blockchain} and standard PKI hierarchy involve a group of electors who are responsible for overseeing and making decisions about the RCA. The electors use their self-signed certificates and follow a democratic voting process to add or remove RCA certificates or elector certificates. 
In BBRM, electors vote for adding or revoking the certificate of RCA or elector by signing an endorsement and then all endorsements are aggregated into the ballot ledger in the blockchain. A ballot ledger can be trusted by other authorities in the blockchain system where is a quorum of valid elector endorsements. This voting rule is defined in the GPF. There are four types of endorsement; adding RCA certificate, revoking RCA certificate, adding an elector certificate, and revoking an elector certificate. BBRM can track the history of the certificate because blockchain can record all of transactions of endorsement. In SCMS, SCMS manager conducts this voting process. 


%\subsection{Consortium Blockchain} %Consortium Blockchain and Smart Contract}
%\label{Consortiumblockchain}

%Blockchain \cite{nakamoto2008bitcoin} is essentially an append-only ledger where information is recorded in a distributed manner using cryptographic method that gives a guarantee of only adding a transaction on ledger but not modifying it. Consortium blockchain are permissioned blockchain where the consortium membership are controlled with identity-based control (e.g., registration, credential, and certificate) and  enabling authentication. By contrast, permissionless blockchain such as those used in cryptocurrencies, i.e. Bitcoin, Ethereum for anonymous and censorless financial transactions, the nodes are self-certified and are encouraged to continuously use new identities (similar to the Sybil attack but not for malicious purpose), making it difficult to implement authentication. In permissionless and consortium blockchain, the block header contains the metadata about the block (e.g. block number, current block hash, previous block header hash etc.). But, in consortium blockchain there is also an additional section, block metadata which contains the certificate and signature of the block creator. Fig. \ref{fig:blockchainstructure} shows an overview of consortium blockchain structure.

%\subsection{Smart Contract}
%\label{SmartContract}
%Smart contract is a computerized transaction protocol \cite{szabo1996smart} stored on blockchain-based platform to execute an agreement or, an protocol. Smart contract is integrated with the blockchain at first In Ethereum \cite{buterin2014next}. Smart contracts are shared by all nodes and then transactions are broadcasted to the blockchain network through consensus algorithm. Smart Contract is implemented using logical rules.   





% We suppose a standard federated learning setting with a server and clients $C$ with local training dataset $D_{C}$; the datasets may or may not be independently and identically distributed (iid). Specifically, the server generates the global model parameter $\theta^r$ in global model through pre-trained process in round $r$. Server selects chosen clients, a subset of clients $c \subset C$, and sends it to chosen clients $c$. $n$ chosen clients $c_n$ deploy the local model parameter using current global model parameter $\theta^r$. The clients aim to solve the optimization problem through $stochastic$ $gradients$ $descent$ which is the most popular method. All chosen clients compute $\nabla_{r,c_n}$=$\frac{\partial F(\theta^r,B_{c_n})}{\partial \theta^r}$, where $F(\theta^r,B_{c_n})$ is to compute the loss using $B_{c_n}$ which is a randomly sampled minibatch from the local trained dataset $D_{c_n}$. Each chosen client $c_n$ runs $stochastic$ $gradients$ $descent$ in multiple epoch and then sends it to a server. Finally, the server aggregates the local models from the chosen clients $c_n$ using the Federated Average (FedAvg) that is one of popular aggregation rules. Note that the server is possible to select other aggregation rules (e.g., Adaptive FedAvg, Krum, Multiple-Krum, bulyan, and Trimmed-mean, Median). Formally, we have $\theta^{r+1}$ = $\frac{1}{n}$ $\sum_{i=1}^{n}$ $\nabla_{r,c_i}$. The server broadcasts new global model $\theta^{r+1}$ to randomly new $n$ chosen clients $c_n \subset C$. This process is repeated until the global model parameter $\theta^r$ reach low loss $F(D_{val},\theta)$ from validation dataset $D_{val}$.  





% It also consists of 5 processes. In the secure aggregation protocol setting, all clients $C$ gain key pairs ($d_u^{pk}$,$d_u^{sk}$) to receive from the trusted third party and the security parameter $k$, which provides the success probability of attackers. The security parameter $k$ generates public parameter $pp$ from Key Agreement, which is key pair generated algorithm. In advertise key process, an user $u$ generates the key pairs ($c_u^{pk}$, $c_u^{sk}$) by key pairs ($s_u^{pk}$,$s_u^{sk}$) and $pp$, which generated by $pp$. All users generate $\sigma_{u}$ from computing ($d_u^{sk},c_u^{pk}\lVert s_u^{pk}$) through Signature scheme and then they send $\sigma_{u}$ to server. The server generates all users list \{($v,c_v^{pk},s_v^{pk}$,$\sigma_{v}$)$\}_{v\in G_{1}}$, where $G$ is group and then they broadcast it. In Share keys process,All users receive list \{($v,c_v^{pk},s_v^{pk}$,$\sigma_{v}$)$\}_{v\in G_{1}}$ and generates group share key pair ($v,s_{u,v}^{sk},b_{u,v}$) from Secret Sharing. They send $e_{u,v}$ computed by AE.enc(KA.agree($c_u^{sk},c_v^{pk}$),$u\lVert v \lVert s_{u,v}^{sk}\lVert b_{u,v}$) to server. The server generates lists of ciphertests $\{e_{u,v}$\}$_{v \in G_2}$ and then send to each user $u \subseteq G_2$, denote with $G_2 \subseteq G_1$. Next process is Masked Input Collection. Users receive the list of ciphertexts $\{e_{u,v}$\}$_{v \in G_2}$ from the server. each user computes $s_{u,v}$ from KA.agree($s_u^{sk},s_v^{pk}$) and generate masked input vector $y_u$ through computing  Security Credential Management System (SCMS) ~\cite{brecht2018security}~\cite{whyte2013security} is a PKI design for issuing digital certificates to vehicles, infrastructures, and pedestrians/mobile nodes for V2X communications. SCMS adopts a strong threat model against the vehicular privacy. In addition to the standard adversary accessing the V2X channels at the time of the certificate use, SCMS considers an honest-but-curious adversary compromising the authority entities involved in the SCMS operations during the digital certificate generation~\cite{chen2017protecting,brorsson2018guarding}. Such threat model motivates a decentralized PKI for the separation of the SCMS authority entities so that no single authority compromise breaches vehicular privacy. SCMS therefore provides a distributed PKI involving multiple authorities, as opposed to the traditional centralized PKI (such as that used for public Internet) having one certificate authority for the certificate management including the certificate signing, generation, and issuance. Fig. ~\ref{fig:scmsarc} depicts the authority entities and their interactions where the line connections indicate interactions for the SCMS operations. Our work advances the operations of the electors and the RCA, which are highlighted with dotted box in Fig. ~\ref{fig:scmsarc} and described in greater details in Section~\ref{ebrm}. 
