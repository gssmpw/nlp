\section{Security Analysis}
\label{secur}
In this section, we analyze the feasibility and viability of the threats and attacks outlined in our threat model.

\begin{figure}[t]
\centering
{\includegraphics [width=0.4\textwidth]{graphics/bsize.pdf}}
\caption{Blockchain Size 
}
\label{fig:bsize}
\end{figure}


\textbf{Privacy End-entity Preservation:}
In our proposed approach, the objective of encrypting the response message with the public key of a pseudonym certificate is to ensure that no unauthorized entities can access sensitive information. This encryption process enhances data confidentiality and protects the privacy of the communication, safeguarding it from potential threats and ensuring that only intended recipients can decrypt and access the information.

\textbf{Malicious or Compromised End-entity or authority:} The proposed architecture is designed to withstand the following attacks mentioned in our threat model:

Blockchain Attacks: Blockchain networks are generally regarded as highly secure and resistant to unauthorized manipulation because no single entity has control over all the authorization processes. To successfully launch an attack, an attacker would need to control a significant portion of the blockchain network, which is typically very difficult to achieve.

Man-in-the-Middle Attacks: The Trust Management platform, combined with a consortium blockchain, can securely store the trust certificate chain. This makes it extremely difficult for attackers to steal or alter these certificates, thereby protecting the integrity of the communication.

Denial-of-Service (DoS) and Distributed Denial-of-Service (DDoS): While PKI and TLS do not inherently prevent DoS or DDoS attacks, the shared blockchain ledger enables other nodes to maintain the BBTM system even in the event of an attack. If any nodes are compromised, they can recover data from their peers upon restoration, ensuring both data integrity and availability. 




% this section, we analyze the feasibility and viability of the threats and attacks in our threat model. The security measures in each scenario are highlighted using boldface. 

%In this section, we analyze the feasibility and viability of the threats and attacks outlined in our threat model. The security measures implemented in each scenario are emphasized using boldface.
%This analysis aims to provide a comprehensive understanding of potential vulnerabilities and the corresponding protective strategies employed to mitigate them.
