\title{Blockchain-based Trust Management in Security Credential Management System for Vehicular Network \\
%{\footnotesize \textsuperscript{*}Note: Sub-titles are not %captured in Xplore and
%should not be used}
%\thanks{ISBN 978-3-903176-28-7 © 2020 IFIP}
%Managing Trust for Vehicular Communication Outside of the Vehicular Network
}

\author{\IEEEauthorblockN{SangHyun Byun, Arijet Sarker\IEEEauthorrefmark{3}, Sang-Yoon Chang, Jugal Kalita}
\IEEEauthorblockA{%\textit{Department of Computer Science}, %\\
\textit{University of Colorado Colorado Springs, Florida Polytechnic University\IEEEauthorrefmark{3}}\\ %, USA}\\
\{sbyun,schang2,jkalita\}@uccs.edu, asarker@floridapoly.edu\IEEEauthorrefmark{3}}
}

\maketitle
\begingroup\renewcommand\thefootnote{\textsection}

\endgroup

\begin{abstract}
Cellular networking is advancing as a wireless technology to support diverse applications in vehicular communication, enabling vehicles to interact with various applications to enhance the driving experience, even when managed by different authorities. Security Credential Management System (SCMS) is the Public Key Infrastructure (PKI) for vehicular networking and the state-of-the-art distributed PKI to protect the privacy-preserving vehicular networking against an honest-but-curious authority using multiple authorities and to decentralize the trust management. We build a Blockchain-Based Trust Management (BBTM) to provide even greater decentralization and security. Specifically, BBTM uses the blockchain to 1) replace the existing Policy Generator (PG), 2) manage the policy of each authority in SCMS, 3) aggregate the Global Certificate Chain File (GCCF), and 4) provide greater accountability and transparency on the aforementioned functionalities. We implement BBTM on Hyperledger Fabric using a smart contract for experimentation and analyses. Our experiments show that BBTM is lightweight in processing, efficient management in the certificate chain and ledger size, supports a bandwidth of multiple transactions per second, and provides validated end-entities.

% Cellular networking is evolving as a wireless technology to support the wide range of applications in vehicular communication. With the advent of cellular technologies, vehicles are projected to communicate with many applications to improve driving experience. These applications do not necessarily need to be under the administration of the same authority. Secure Credential Management System (SCMS) is a Public Key Infrastructure (PKI) which provides certificates to vehicles to preserve vehicular privacy and supports many vehicular applications such as basic safety message (BSM), misbehavior reporting etc. On the other hand, applications like federated learning (used to enable data-driven machine learning for better self-driving experience while protecting the vehicular data privacy) are outside of this SCMS-managed vehicular network and can have separate PKI structure managed by different authority. Since the trust management and PKI for vehicular vs. federated learning have been studied and developed separately, there is a need to establish trust between these two PKIs for secure communication between vehicles and external applications (outside of SCMS). In this work, we demonstrate and analyze how these two orthogonal PKIs can establish trust with each other and thus the end entities (vehicles and federated learning servers) can securely communicate with each other in an efficient manner maintaining vehicular privacy (utilizing SCMS). We also analyze our application-specific latency in comparison with round-trip time (RTT) using 5G and LTE to demonstrate the latency requirement for our proposed approach.
% \syc{The following are about the first two sentences. 1) I would avoid "supposed to" and replace it with "will", "planned to", or "projected to". 2) How is the cellular related to vehicular? (This can be brief here but you would want to be explicit about it; the description/explanation can get expanded in the Intro.) 3) I would emphasize the relevance of federated learning in vehicular networking, e.g., use federated learning to enable data-driven machine learning while protecting the vehicular privacy.} \as{addressed} \syc{"Since the trust management and PKi for vehicular vs. federated learning have been studied and developed separately,"} \as{addressed} 

\end{abstract}

\vspace{0.1in}
\begin{IEEEkeywords}
Public Key Infrastructure, Security Credential Management System, Permission Blockchain, Vehicular Networking
\end{IEEEkeywords}