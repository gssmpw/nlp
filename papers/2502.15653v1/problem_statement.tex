\section{Problem Statement}
\label{ps}

We describe the problem statement and the scope of the contribution in Section~\ref{subsec:problem_scope} and the threat model in Section~\ref{subsec:tm}. 
%\vspace{-2mm}

\subsection{Contribution Scope}
\label{subsec:problem_scope}
%\vspace{-0.5mm}  

Although there are multiple authorities for SCMS, our focus is on TM consisting of whole authorities in SCMS. Securing the trust of the Chain is critical for PKI because it provides the trust relationship and security for SCMS, which in turn provides the certificates and the trust in keys to the vehicles for vehicular networking. Therefore, our work defends against the single authority in SCMS compromise by having a blockchain to replace the SCMS Manager divided into PG functions (aggregating the GCCF and managing the GPF) and securely to control the authorities through GPF. Our work also focuses on the blockchain to replace the SCMS Manager's role (whose primary role is to set and administer the SCMS policy) and reduce the authorities in SCMS since the smart contracts replace several functions in the SCMS that have previously been executed by the SCMS Manager. 


\subsection{Threat Model}
\label{subsec:tm} 
In our threat model, we assume that adversaries cannot compromise the SCMS Manager, preventing them from controlling authority policies or manipulating the GCCF and GPF. We also assume that adversaries cannot break blockchain certificates using cryptographic methods like forging digital signatures or finding hash collisions. Within our BBTM Framework, we recognize that adversaries may monitor GCCF and GPF transactions as they are broadcast to SCMS authorities. We expect SCMS authorities to adhere to the GPF, which includes root management, addressing misbehavior in the MA, and managing the Certificate Revocation List (CRL), thus implementing accountability. Each authority has different access to smart contracts via the GPF. The distributed nature of the blockchain protects the integrity of stored data such as GCCF and GPF. We identify potential attacks, including blockchain tampering, information disclosure, privilege escalation, Denial of Service, Distributed Denial of Service, and Man-in-the-Middle attacks. Our approach effectively manages smart contract accessibility related to the addition or revocation of certificates and authority policy control.        

