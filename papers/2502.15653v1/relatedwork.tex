\section{Related Work}
\label{rw}
In this section, we describe the related work of blockchain for decentralized PKI and the application of certificate management for vehicular networks.

\textbf{Blockchain for Decentralized PKI:} Blockchain is used to create a decentralized Public Key Infrastructure (PKI), improving resilience against central authority failures. In papers ~\cite{plessing2020revisiting, axon2016pb}, a blockchain-based PKI (BB-PKI) implemented via a CertCoin fork enhances anonymity by separating user identity from certificates. The paper~\cite{sermpinis2021detract} introduces DeTract, which enables web servers to independently generate X.509 certificates using uPort, an Ethereum-based identity platform for managing certificate transactions. Additionally, this paper~\cite{toorani2021decentralized} presents a blockchain-based Web of Trust (WoT) system that eliminates the need for a centralized Certificate Revocation List (CRL), while this paper~\cite{adja2021blockchain} improves X.509 validity mechanisms based on distribution points to reduce time delays. Finally, this paper~\cite{turan2024semi} presents SemiDec-PKI, a blockchain-based infrastructure that enhances fault tolerance and security for various certificate types by integrating a Web of Trust with a centralized model.



%  that was defined in PKI Request for Comments (RFC) 5217. The objective of the framework is to provide operational requirements between PKI authorities to establish trust relationships. In which, RFC 5217 defines the relationship between certification authority of independent PKI as ‘cross-certification.’ Cross-certification can be direct where there are only two independent PKI domains or if there are more than two, then RFC 5217 suggests the relationship to be either unilateral or bilateral; our work can be related closely to bilateral. As for the architecture of cross-certification, RFC 5217 suggested using a neutral CA as a bridge between the independent PKI. Our work differs from those architectures suggested in RFC 5217 because we establish trust at the highest level of authority between two PKIs, where in this context it would be between SCMS and FL PKI managers. The complexity of PKI-like SCMS where the RCA is also connected to multiple authorities within its organization, makes it important to consider the security implications if it’s compromised by a third party. Therefore, we propose a design where other collaborators of PKIs only are connected to authorities that are essential according to the agreement between the PKI managers. This would hence limit the risk of data and privacy being compromised within an organization.
\textbf{Application of certificate management for Vehicular Networking:} Application of certificate management is applicable for blockchain to enhance security aspect. In this paper\cite{khan2020accountable}, the Accountable Credential Management System (ACMS) enhances vehicular communication security by transparently managing certificates and ensuring trustworthiness without relying on traditional certificate revocation lists. This paper~\cite{byunx2023privacy} presents the challenge of establishing trust between these two separate PKIs for vehicular networks. In this paper~\cite{byun2024secure}, there is multi-authority management between the federated learning server and SCMS for the vehicular network. In \cite{lei2020blockchain}, the author describes the growing importance of security and privacy in Vehicular Communication Systems (VCS) within the Intelligent Transportation Systems (ITS).      

%