\section{BBTM Design Principle }
\label{sec:bbtmdesign}
In this section, we describe the design of the network setup, functionalities, and transactions of GCCFL and GPF used in BBTM. We also describe the decentralized trust management for SCMS in BBTM.  

\subsection{Notations and Variables}
Notations and variables are explicitly defined in Table \ref{table:notations}. $S$ is the set of entities used in BBTM. The set of the entities involved in BBTM is $S$ =\{all authorities in SCMS\}. For any entity is $i$, where $i$ $\in$ $S$, $CERT_{i}$ defines the certificate of the entity $i$. We define Policy Generator, Order Service Provider, Global Chain Certificate File, and Global Policy File as $PG$, $OSP$, $GCCF$, and $GPF$ respectively. $SIGN$ defines the certificate sign algorithm.  




\subsection{Network Setup} 
In this subsection, we outline the design of the BBTM network setup, establishing two channels: 1) the System Channel dedicated to storing consortium configuration, and 2) the Application Channel designed for sharing ledger and chaincode among channel members. At this stage, the initial set of all authorities is created, and credentials for them are generated on their respective machines.  

\begin{algorithm}[t]
\caption{Adding and Revoke Certificate}\label{al:addcert}
\begin{flushleft}
\hspace*{\algorithmicindent} \textbf{Input:} $CERT_i$, $CERT_i[Arguments]$, $CERT_{PG}$ \\
\hspace*{\algorithmicindent} \textbf{Output:} $GCCF$  
\end{flushleft}
\begin{algorithmic}[1]
%\State $\textit{LEN}$ $\gets$ $COUNT.$ $CERT_i[ARGS]$
\If {$Adding \& Verifying$ $CERT_i$ and $CERT_i[Arguments]$}
\State $CERT_i$ sign $CERT_i[Arguments]$
\State $GCCF$ $\gets$ signed $CERT_i[Arguments]$
\State $\textbf{return}$  $Updated GCCF.Function type$
\Else 
\State $\textbf{return}$ $Not Adding Verify$
\EndIf
\If {$Revoke \& Verifying$ $CERT_{PG}$}
\State $GCCF$ $\gets$ signed $CERT_i[Arguments]$
\State $\textbf{return}$ $Updated GCCF.Function type$
\Else 
\State $\textbf{return}$ $Not Revoking Verify$
\EndIf
\end{algorithmic}
\end{algorithm} 

\begin{algorithm}[t]
\caption{Validate Certificate}\label{al:valcert}
\begin{flushleft}
\hspace*{\algorithmicindent} \textbf{Input:} $CERT_{S}$,\\
\hspace*{\algorithmicindent} \textbf{Output:} $BOOL$  
\end{flushleft}
\begin{algorithmic}[1]
\If {$Verifying$ $CERT_S$ through $GCCF$}
\State $\textbf{return}$ $Success$
\Else 
\State $\textbf{return}$ $Not Verify$ 
\EndIf
\end{algorithmic}
\end{algorithm} 
\setlength{\textfloatsep}{1mm}
\setlength{\floatsep}{1mm}

\textbf{1) System Channel:} In this step, the genesis block of the system channel is generated through the Order Service Provider (OSP) registration and consortium Member Registration procedures. In the OSP Registration procedure, the configuration of the ordering service is established by incorporating all authorities and consensus algorithms, where is used by the permissioned consensus algorithm. The Consortium Member Registration procedure defines a consortium member by incorporating the certificate of the authorities. 

\textbf{2) Application Channel:} The Application channel is established through both Consortium Member Policy and Consortium Member Setting. In Consortium Member Policy, the guidelines for consortium members, such as the privileges for reading and writing in the consortium blockchain, are outlined. In Consortium Member Setting, information about consortium members, specifically in the context of BBTM, is included all authorities. The genesis block of application channel (GCCF and GPF) is generated using them and then it is broadcasted to all authorities from OSP. 

\begin{algorithm}[t]
\caption{Adding and Revoke Policy}\label{al:addpolicy}
\begin{flushleft}
\hspace*{\algorithmicindent} \textbf{Input:} $CERT_{PG}$, $Policy{live}$ \\
\hspace*{\algorithmicindent} \textbf{Output:} $GPF$  
\end{flushleft}
\begin{algorithmic}[1]
\If {$Adding \& Verifying$ $CERT_{PG}$}
\State $Policy.status = alive$
\State $\textbf{return}$ $Updated GPF$
\EndIf
\If {$Revoke \& Verifying$ $CERT_{PG}$}
\State $Policy.status=death$
\State $\textbf{return}$ $Updated GPF$
\EndIf
\end{algorithmic}
\end{algorithm} 

\subsection{Functionalities of GCCF and GPF} 
There are total five functions; 1) AddCert, 2) RevokeCert, 3) ValidateCert, 4) AddPolicy, and 5) RevokePolicy. The functions (AddCert, RevokeCert, and ValidateCert) within GCCF can collectively be referred to as PKI-signed, revoked, and validated functions. The AddCert belongs to the signed certificate process from a high-level authority in Algorithm~. For instance, RCA signs ICA, MA, and PG. The RevokeCert belongs to the revocation certificate process that the SCMS manager updates the message of a revoked certificate in Algorithm~\ref{al:addcert}. Transactions are submitted through two functions (AddCert and RevokeCert) utilizing the same payload format, indicating the same certificate format. However, the classification is divided into two types to differentiate between the distinct objectives of the function, specifying whether it is intended for adding or revoking a certificate. The certificate format adheres to the X.509 standard, incorporating an extended field labeled \emph{Function Type}. The description of the certificate field is defined in Table ~\ref{fig:certificateformat}. The ValidateCert function is used to validate certificates through reading transactions in blockchain. The pseudocode of ValidateCert function is shown in Algorithm~\ref{al:valcert}. Each transaction is associated with a key value, which serves as the index entry for that transaction. To retrieve the certificate information of a specific transaction, a search is conducted in the ledger using the transaction's index entry. If it cannot verify certificate, the error message is displayed; otherwise, a verified success message is displayed.    

The functions (AddPolicy and RevokePolicy) within GPF can collectively be referred to as managing policy file of all authorities in SCMS. The AddPolicy function is governed by the PG under the SCMS manager, while RevokePolicy also corresponds to the PG rules set by the SCMS manager. Transactions are submitted using two functions that contain the same payload format meaning the same policy file format including entities, rules, and statutes (alive and death). The pseudocode of AddPolicy and RevokePolicy are shown in Algorithm~\ref{al:addpolicy}. Note that rules in GPF are global configuration information~\cite{brecht2018security}.   


\begin{comment}
\begin{algorithm}[t]
\caption{Revoke Certificate}\label{al:revokecert}
\begin{flushleft}
\hspace*{\algorithmicindent} \textbf{Input:} $CERT_{PG}$, $CERT_i[Arguments[Key]]$ \\
\hspace*{\algorithmicindent} \textbf{Output:} $GCCF$  
\end{flushleft}
\begin{algorithmic}[1]
\If {$Verifying$ $CERT_{PG}$}
\State $GCCF$ $\gets$ signed $CERT_i[Arguments]$
\State $\textbf{return}$ $Updated GCCF$
\Else 
\State $Error$ 
\EndIf
\end{algorithmic}
\end{algorithm} 
%\vspace{-3.5mm}
\end{comment}