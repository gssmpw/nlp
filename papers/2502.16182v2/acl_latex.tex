% This must be in the first 5 lines to tell arXiv to use pdfLaTeX, which is strongly recommended.
\pdfoutput=1
% In particular, the hyperref package requires pdfLaTeX in order to break URLs across lines.

\documentclass[11pt]{article}

% Change "review" to "final" to generate the final (sometimes called camera-ready) version.
% Change to "preprint" to generate a non-anonymous version with page numbers.
\usepackage[preprint]{acl}
\usepackage{booktabs}
\usepackage{algorithm}
\usepackage{tikz}
\usepackage{pgfplots}
\usepackage{algpseudocode}
% Standard package includes
\usepackage{times}
\usepackage{latexsym}
\usepackage{multirow}
\usepackage{amsmath}
% For proper rendering and hyphenation of words containing Latin characters (including in bib files)
\usepackage[T1]{fontenc}
% For Vietnamese characters
% \usepackage[T5]{fontenc}
% See https://www.latex-project.org/help/documentation/encguide.pdf for other character sets

% This assumes your files are encoded as UTF8
\usepackage[utf8]{inputenc}

% This is not strictly necessary, and may be commented out,
% but it will improve the layout of the manuscript,
% and will typically save some space.
\usepackage{microtype}

% This is also not strictly necessary, and may be commented out.
% However, it will improve the aesthetics of text in
% the typewriter font.
\usepackage{inconsolata}

%Including images in your LaTeX document requires adding
%additional package(s)
\usepackage{graphicx}
\usepackage{breakurl}
\usepackage{xcolor}
\usepackage{colortbl}
\usepackage{multirow}
\usepackage{graphicx}
\usepackage{tabularx}
\usepackage{amssymb}
\usepackage{hyperref}

% If the title and author information does not fit in the area allocated, uncomment the following
%
%\setlength\titlebox{<dim>}
%
% and set <dim> to something 5cm or larger.

\makeatletter
\title{%
  \vspace{-2cm}
  \begin{center}
    \includegraphics[height=2cm]{assets/lossfunk.jpeg}
  \end{center}
  \vspace{0.5cm}
  IPO: Your Language Model is Secretly a Preference Classifier%
}
\makeatother

% Author information can be set in various styles:
% For several authors from the same institution:
% \author{Author 1 \and ... \and Author n \\
%         Address line \\ ... \\ Address line}
% if the names do not fit well on one line use
%         Author 1 \\ {\bf Author 2} \\ ... \\ {\bf Author n} \\
% For authors from different institutions:
% \author{Author 1 \\ Address line \\  ... \\ Address line
%         \And  ... \And
%         Author n \\ Address line \\ ... \\ Address line}
% To start a separate ``row'' of authors use \AND, as in
% \author{Author 1 \\ Address line \\  ... \\ Address line
%         \AND
%         Author 2 \\ Address line \\ ... \\ Address line \And
%         Author 3 \\ Address line \\ ... \\ Address line}

\author{
  \textbf{Shivank Garg\textsuperscript{1,2}\textsuperscript{*}},
  \textbf{Ayush Singh\textsuperscript{1,2}\textsuperscript{*}},
  \textbf{Shweta Singh\textsuperscript{1,2}},
  \textbf{Paras Chopra\textsuperscript{2}}
\\
  \textsuperscript{1}Indian Institute of Technology Roorkee,
  \textsuperscript{2}Lossfunk
\\
  \texttt{
    \texttt{\{shivank\_g@mfs, ayush\_s@mt, shweta\_s@mfs\}.iitr.ac.in}},
    % \texttt{ayush\_s@mt.iitr.ac.in},
    % \texttt{shweta\_s@mfs.iitr.ac.in},
    \texttt{paras@lossfunk.com}
    \\
    % \small{* Equal contribution}
  }
  



% \author{First Author \\
%   Affiliation / Address line 1 \\
%   Affiliation / Address line 2 \\
%   Affiliation / Address line 3 \\
%   \texttt{email@domain} \\\And
%   Second Author \\
%   Affiliation / Address line 1 \\
%   Affiliation / Address line 2 \\
%   Affiliation / Address line 3 \\
%   \texttt{email@domain} \\}

%\author{
%  \textbf{First Author\textsuperscript{1}},
%  \textbf{Second Author\textsuperscript{1,2}},
%  \textbf{Third T. Author\textsuperscript{1}},
%  \textbf{Fourth Author\textsuperscript{1}},
%\\
%  \textbf{Fifth Author\textsuperscript{1,2}},
%  \textbf{Sixth Author\textsuperscript{1}},
%  \textbf{Seventh Author\textsuperscript{1}},
%  \textbf{Eighth Author \textsuperscript{1,2,3,4}},
%\\
%  \textbf{Ninth Author\textsuperscript{1}},
%  \textbf{Tenth Author\textsuperscript{1}},
%  \textbf{Eleventh E. Author\textsuperscript{1,2,3,4,5}},
%  \textbf{Twelfth Author\textsuperscript{1}},
%\\
%  \textbf{Thirteenth Author\textsuperscript{3}},
%  \textbf{Fourteenth F. Author\textsuperscript{2,4}},
%  \textbf{Fifteenth Author\textsuperscript{1}},
%  \textbf{Sixteenth Author\textsuperscript{1}},
%\\
%  \textbf{Seventeenth S. Author\textsuperscript{4,5}},
%  \textbf{Eighteenth Author\textsuperscript{3,4}},
%  \textbf{Nineteenth N. Author\textsuperscript{2,5}},
%  \textbf{Twentieth Author\textsuperscript{1}}
%\\
%\\
%  \textsuperscript{1}Affiliation 1,
%  \textsuperscript{2}Affiliation 2,
%  \textsuperscript{3}Affiliation 3,
%  \textsuperscript{4}Affiliation 4,
%  \textsuperscript{5}Affiliation 5
%\\
%  \small{
%    \textbf{Correspondence:} \href{mailto:email@domain}{email@domain}
%  }
%}
% Place this before \begin{document}
% Place this before \begin{document}

\begin{document}

\maketitle
\begin{abstract}

\footnotetext{*Equal contribution.}

Reinforcement learning from human feedback (RLHF) has emerged as the primary method for aligning large language models (LLMs) with human preferences.
While it enables LLMs to achieve human-level alignment, it often incurs significant computational and financial costs due to its reliance on training external reward models or human-labeled preferences. In this work, we propose \textbf{Implicit Preference Optimization (IPO)}, an alternative approach that leverages generative LLMs as preference classifiers, thereby reducing the dependence on external human feedback or reward models to obtain preferences. We conduct a comprehensive evaluation on the preference classification ability of LLMs using RewardBench, assessing models across different sizes, architectures, and training levels to validate our hypothesis. Furthermore, we investigate the self-improvement capabilities of LLMs by generating multiple responses for a given instruction and employing the model itself as a preference classifier for Direct Preference Optimization (DPO)-based training. Our findings demonstrate that models trained through IPO achieve performance comparable to those utilizing state-of-the-art reward models for obtaining preferences. Our code is available at \textcolor{magenta}{\burl{https://github.com/shivank21/Implicit\_Preference\_Optimization}}.
\end{abstract}


\section{Introduction}
\label{sec:introduction}
The business processes of organizations are experiencing ever-increasing complexity due to the large amount of data, high number of users, and high-tech devices involved \cite{martin2021pmopportunitieschallenges, beerepoot2023biggestbpmproblems}. This complexity may cause business processes to deviate from normal control flow due to unforeseen and disruptive anomalies \cite{adams2023proceddsriftdetection}. These control-flow anomalies manifest as unknown, skipped, and wrongly-ordered activities in the traces of event logs monitored from the execution of business processes \cite{ko2023adsystematicreview}. For the sake of clarity, let us consider an illustrative example of such anomalies. Figure \ref{FP_ANOMALIES} shows a so-called event log footprint, which captures the control flow relations of four activities of a hypothetical event log. In particular, this footprint captures the control-flow relations between activities \texttt{a}, \texttt{b}, \texttt{c} and \texttt{d}. These are the causal ($\rightarrow$) relation, concurrent ($\parallel$) relation, and other ($\#$) relations such as exclusivity or non-local dependency \cite{aalst2022pmhandbook}. In addition, on the right are six traces, of which five exhibit skipped, wrongly-ordered and unknown control-flow anomalies. For example, $\langle$\texttt{a b d}$\rangle$ has a skipped activity, which is \texttt{c}. Because of this skipped activity, the control-flow relation \texttt{b}$\,\#\,$\texttt{d} is violated, since \texttt{d} directly follows \texttt{b} in the anomalous trace.
\begin{figure}[!t]
\centering
\includegraphics[width=0.9\columnwidth]{images/FP_ANOMALIES.png}
\caption{An example event log footprint with six traces, of which five exhibit control-flow anomalies.}
\label{FP_ANOMALIES}
\end{figure}

\subsection{Control-flow anomaly detection}
Control-flow anomaly detection techniques aim to characterize the normal control flow from event logs and verify whether these deviations occur in new event logs \cite{ko2023adsystematicreview}. To develop control-flow anomaly detection techniques, \revision{process mining} has seen widespread adoption owing to process discovery and \revision{conformance checking}. On the one hand, process discovery is a set of algorithms that encode control-flow relations as a set of model elements and constraints according to a given modeling formalism \cite{aalst2022pmhandbook}; hereafter, we refer to the Petri net, a widespread modeling formalism. On the other hand, \revision{conformance checking} is an explainable set of algorithms that allows linking any deviations with the reference Petri net and providing the fitness measure, namely a measure of how much the Petri net fits the new event log \cite{aalst2022pmhandbook}. Many control-flow anomaly detection techniques based on \revision{conformance checking} (hereafter, \revision{conformance checking}-based techniques) use the fitness measure to determine whether an event log is anomalous \cite{bezerra2009pmad, bezerra2013adlogspais, myers2018icsadpm, pecchia2020applicationfailuresanalysispm}. 

The scientific literature also includes many \revision{conformance checking}-independent techniques for control-flow anomaly detection that combine specific types of trace encodings with machine/deep learning \cite{ko2023adsystematicreview, tavares2023pmtraceencoding}. Whereas these techniques are very effective, their explainability is challenging due to both the type of trace encoding employed and the machine/deep learning model used \cite{rawal2022trustworthyaiadvances,li2023explainablead}. Hence, in the following, we focus on the shortcomings of \revision{conformance checking}-based techniques to investigate whether it is possible to support the development of competitive control-flow anomaly detection techniques while maintaining the explainable nature of \revision{conformance checking}.
\begin{figure}[!t]
\centering
\includegraphics[width=\columnwidth]{images/HIGH_LEVEL_VIEW.png}
\caption{A high-level view of the proposed framework for combining \revision{process mining}-based feature extraction with dimensionality reduction for control-flow anomaly detection.}
\label{HIGH_LEVEL_VIEW}
\end{figure}

\subsection{Shortcomings of \revision{conformance checking}-based techniques}
Unfortunately, the detection effectiveness of \revision{conformance checking}-based techniques is affected by noisy data and low-quality Petri nets, which may be due to human errors in the modeling process or representational bias of process discovery algorithms \cite{bezerra2013adlogspais, pecchia2020applicationfailuresanalysispm, aalst2016pm}. Specifically, on the one hand, noisy data may introduce infrequent and deceptive control-flow relations that may result in inconsistent fitness measures, whereas, on the other hand, checking event logs against a low-quality Petri net could lead to an unreliable distribution of fitness measures. Nonetheless, such Petri nets can still be used as references to obtain insightful information for \revision{process mining}-based feature extraction, supporting the development of competitive and explainable \revision{conformance checking}-based techniques for control-flow anomaly detection despite the problems above. For example, a few works outline that token-based \revision{conformance checking} can be used for \revision{process mining}-based feature extraction to build tabular data and develop effective \revision{conformance checking}-based techniques for control-flow anomaly detection \cite{singh2022lapmsh, debenedictis2023dtadiiot}. However, to the best of our knowledge, the scientific literature lacks a structured proposal for \revision{process mining}-based feature extraction using the state-of-the-art \revision{conformance checking} variant, namely alignment-based \revision{conformance checking}.

\subsection{Contributions}
We propose a novel \revision{process mining}-based feature extraction approach with alignment-based \revision{conformance checking}. This variant aligns the deviating control flow with a reference Petri net; the resulting alignment can be inspected to extract additional statistics such as the number of times a given activity caused mismatches \cite{aalst2022pmhandbook}. We integrate this approach into a flexible and explainable framework for developing techniques for control-flow anomaly detection. The framework combines \revision{process mining}-based feature extraction and dimensionality reduction to handle high-dimensional feature sets, achieve detection effectiveness, and support explainability. Notably, in addition to our proposed \revision{process mining}-based feature extraction approach, the framework allows employing other approaches, enabling a fair comparison of multiple \revision{conformance checking}-based and \revision{conformance checking}-independent techniques for control-flow anomaly detection. Figure \ref{HIGH_LEVEL_VIEW} shows a high-level view of the framework. Business processes are monitored, and event logs obtained from the database of information systems. Subsequently, \revision{process mining}-based feature extraction is applied to these event logs and tabular data input to dimensionality reduction to identify control-flow anomalies. We apply several \revision{conformance checking}-based and \revision{conformance checking}-independent framework techniques to publicly available datasets, simulated data of a case study from railways, and real-world data of a case study from healthcare. We show that the framework techniques implementing our approach outperform the baseline \revision{conformance checking}-based techniques while maintaining the explainable nature of \revision{conformance checking}.

In summary, the contributions of this paper are as follows.
\begin{itemize}
    \item{
        A novel \revision{process mining}-based feature extraction approach to support the development of competitive and explainable \revision{conformance checking}-based techniques for control-flow anomaly detection.
    }
    \item{
        A flexible and explainable framework for developing techniques for control-flow anomaly detection using \revision{process mining}-based feature extraction and dimensionality reduction.
    }
    \item{
        Application to synthetic and real-world datasets of several \revision{conformance checking}-based and \revision{conformance checking}-independent framework techniques, evaluating their detection effectiveness and explainability.
    }
\end{itemize}

The rest of the paper is organized as follows.
\begin{itemize}
    \item Section \ref{sec:related_work} reviews the existing techniques for control-flow anomaly detection, categorizing them into \revision{conformance checking}-based and \revision{conformance checking}-independent techniques.
    \item Section \ref{sec:abccfe} provides the preliminaries of \revision{process mining} to establish the notation used throughout the paper, and delves into the details of the proposed \revision{process mining}-based feature extraction approach with alignment-based \revision{conformance checking}.
    \item Section \ref{sec:framework} describes the framework for developing \revision{conformance checking}-based and \revision{conformance checking}-independent techniques for control-flow anomaly detection that combine \revision{process mining}-based feature extraction and dimensionality reduction.
    \item Section \ref{sec:evaluation} presents the experiments conducted with multiple framework and baseline techniques using data from publicly available datasets and case studies.
    \item Section \ref{sec:conclusions} draws the conclusions and presents future work.
\end{itemize}
\section{Background}\label{sec:backgrnd}

\subsection{Cold Start Latency and Mitigation Techniques}

Traditional FaaS platforms mitigate cold starts through snapshotting, lightweight virtualization, and warm-state management. Snapshot-based methods like \textbf{REAP} and \textbf{Catalyzer} reduce initialization time by preloading or restoring container states but require significant memory and I/O resources, limiting scalability~\cite{dong_catalyzer_2020, ustiugov_benchmarking_2021}. Lightweight virtualization solutions, such as \textbf{Firecracker} microVMs, achieve fast startup times with strong isolation but depend on robust infrastructure, making them less adaptable to fluctuating workloads~\cite{agache_firecracker_2020}. Warm-state management techniques like \textbf{Faa\$T}~\cite{romero_faa_2021} and \textbf{Kraken}~\cite{vivek_kraken_2021} keep frequently invoked containers ready, balancing readiness and cost efficiency under predictable workloads but incurring overhead when demand is erratic~\cite{romero_faa_2021, vivek_kraken_2021}. While these methods perform well in resource-rich cloud environments, their resource intensity challenges applicability in edge settings.

\subsubsection{Edge FaaS Perspective}

In edge environments, cold start mitigation emphasizes lightweight designs, resource sharing, and hybrid task distribution. Lightweight execution environments like unikernels~\cite{edward_sock_2018} and \textbf{Firecracker}~\cite{agache_firecracker_2020}, as used by \textbf{TinyFaaS}~\cite{pfandzelter_tinyfaas_2020}, minimize resource usage and initialization delays but require careful orchestration to avoid resource contention. Function co-location, demonstrated by \textbf{Photons}~\cite{v_dukic_photons_2020}, reduces redundant initializations by sharing runtime resources among related functions, though this complicates isolation in multi-tenant setups~\cite{v_dukic_photons_2020}. Hybrid offloading frameworks like \textbf{GeoFaaS}~\cite{malekabbasi_geofaas_2024} balance edge-cloud workloads by offloading latency-tolerant tasks to the cloud and reserving edge resources for real-time operations, requiring reliable connectivity and efficient task management. These edge-specific strategies address cold starts effectively but introduce challenges in scalability and orchestration.

\subsection{Predictive Scaling and Caching Techniques}

Efficient resource allocation is vital for maintaining low latency and high availability in serverless platforms. Predictive scaling and caching techniques dynamically provision resources and reduce cold start latency by leveraging workload prediction and state retention.
Traditional FaaS platforms use predictive scaling and caching to optimize resources, employing techniques (OFC, FaasCache) to reduce cold starts. However, these methods rely on centralized orchestration and workload predictability, limiting their effectiveness in dynamic, resource-constrained edge environments.



\subsubsection{Edge FaaS Perspective}

Edge FaaS platforms adapt predictive scaling and caching techniques to constrain resources and heterogeneous environments. \textbf{EDGE-Cache}~\cite{kim_delay-aware_2022} uses traffic profiling to selectively retain high-priority functions, reducing memory overhead while maintaining readiness for frequent requests. Hybrid frameworks like \textbf{GeoFaaS}~\cite{malekabbasi_geofaas_2024} implement distributed caching to balance resources between edge and cloud nodes, enabling low-latency processing for critical tasks while offloading less critical workloads. Machine learning methods, such as clustering-based workload predictors~\cite{gao_machine_2020} and GRU-based models~\cite{guo_applying_2018}, enhance resource provisioning in edge systems by efficiently forecasting workload spikes. These innovations effectively address cold start challenges in edge environments, though their dependency on accurate predictions and robust orchestration poses scalability challenges.

\subsection{Decentralized Orchestration, Function Placement, and Scheduling}

Efficient orchestration in serverless platforms involves workload distribution, resource optimization, and performance assurance. While traditional FaaS platforms rely on centralized control, edge environments require decentralized and adaptive strategies to address unique challenges such as resource constraints and heterogeneous hardware.



\subsubsection{Edge FaaS Perspective}

Edge FaaS platforms adopt decentralized and adaptive orchestration frameworks to meet the demands of resource-constrained environments. Systems like \textbf{Wukong} distribute scheduling across edge nodes, enhancing data locality and scalability while reducing network latency. Lightweight frameworks such as \textbf{OpenWhisk Lite}~\cite{kravchenko_kpavelopenwhisk-light_2024} optimize resource allocation by decentralizing scheduling policies, minimizing cold starts and latency in edge setups~\cite{benjamin_wukong_2020}. Hybrid solutions like \textbf{OpenFaaS}~\cite{noauthor_openfaasfaas_2024} and \textbf{EdgeMatrix}~\cite{shen_edgematrix_2023} combine edge-cloud orchestration to balance resource utilization, retaining latency-sensitive functions at the edge while offloading non-critical workloads to the cloud. While these approaches improve flexibility, they face challenges in maintaining coordination and ensuring consistent performance across distributed nodes.


\section{LLM as Preference Model}
\label{LLM_as_Preference}
\subsection{Background}

Large Language Models (LLMs) generate text in an autoregressive manner, producing tokens sequentially based on the context of previously generated tokens. Given an input context \( \mathbf{x} \) , the autoregressive model predicts an output sequence \( \mathbf{y} = (y_1, y_2, \dots, y_T) \) one token at a time. Assuming the model is parameterized by \( \theta \), the conditional probability of generating the sequence \( \mathbf{y} \) is defined as:

\begin{equation}
    p_\theta(\mathbf{y} \mid \mathbf{x}) = \prod_{t=1}^T p_\theta(y_t \mid \mathbf{x}, y_{<t}),
\end{equation}

where \( y_{<t} = (y_1, y_2, \dots, y_{t-1}) \). For notational simplicity, \( p_\theta(y_t \mid \mathbf{x}) \) is used to represent \( p_\theta(y_t \mid \mathbf{x}, y_{<t}) \).

The probability distribution over the vocabulary at each time step \( t \) is computed using a softmax function on the logits \( z \) as:

\begin{equation}
    p_\theta(y_t \mid \mathbf{x}) = \frac{\exp(z_t / \tau)}{\sum_{i=1}^M \exp(z_i / \tau)},
\end{equation}

where \( z_t = \text{logit}_\theta(y_t \mid \mathbf{x}, y_{<t}) \), \( M \) is the vocabulary size, and \( \tau > 0 \) is a temperature parameter. 

Various decoding strategies govern token selection during text generation. Greedy decoding selects the highest probability token at each step, while beam search expands multiple candidate sequences in parallel to find the most likely one. Top-k sampling \citep{fan2018hierarchical}, on the other hand, limits token choices to the k most probable candidates, introducing diversity. Many other decoding strategies also exist, each balancing fluency and variability differently.


%%%%%%%%%%%Table%%%%%%%%%%%
\begin{table*}[htbp!]
\centering
\setlength{\tabcolsep}{4pt} 
\definecolor{apricot}{rgb}{0.95, 0.82, 0.62}
\definecolor{lightgray}{rgb}{0.96, 0.96, 0.96}

\begin{tabular}{l | ccccc | ccccc }
\hline
\rowcolor{apricot} 
\multirow{0}{*}{\textbf{Models}} & \multicolumn{5}{c|}{\textbf{Our Approach}} & \multicolumn{5}{c}{\textbf{Self Rewarding}} \\
\cmidrule(lr){2-6} \cmidrule(lr){7-11}
% \cline{2-11}
 & Chat & Code & Math & Safety & Average & Chat & Code & Math & Safety & Average  \\ \hline
Llama-3.2-1B-Inst & 64.37 & 52.84 & 88.14 & 80.48 & 71.45 & 30.47 & 21.03 & 14.54 & 31.55 & 24.39\\ \hline
\rowcolor{lightgray}Llama-3.2-3B-Inst & 62.09 & 67.17 & \textbf{98.21} & 80.23 & 76.92 & 33.87 & 24.69 & 36.01 & 46.73 & 35.32\\ \hline
Llama-3-8B-Inst & 59.56 & 73.88 & 54.97 & 87.88 & 69.07 & 35.43 & 12.29 & 21.70 & \textbf{58.35} & 31.94\\ \hline
\rowcolor{lightgray}Qwen-2.5-3B-Inst & 60.89  & 80.59  & 46.31  & 86.05 & 68.46 & 26.72 & 23.88 & \textbf{41.61} & 24.43 & 29.16\\ \hline
Qwen-2.5-7B-Inst & \textbf{78.26}  & \textbf{83.13}  & 56.24  & \textbf{93.24} & 77.71 & \textbf{58.73} & \textbf{47.93} & 40.49 & 52.20 & \textbf{49.82}\\ \hline
\rowcolor{lightgray}Mistral-7B-Inst & 61.25 & 70.93  & 96.20 & 83.85 & \textbf{78.05} & 24.55 & 1.6 & 28.18 & 15.39 &  17.43\\ \hline
Gemma2-2B-It & 35.34 & 42.58  & 91.50 & 70.04 & 59.86 & 22.36 & 2.84 & 12.75 & 34.78 & 18.18\\ \hline
\rowcolor{lightgray}Phi-3-Mini-Instruct  & 55.91 & 75.30 & 89.10 & 75.32 & 73.90 & 46.63 & 35.46 & 22.60 & 56.75 & 40.36 \\ \hline
\end{tabular}
\caption{The above table compares our approach with the Self Rewarding approach. The row labels correspond to the model name and the column labels correspond to the sub-categories. The metric used is accuracy where the higher values indicate better performance. }
\label{tab:reward_bench}
\end{table*}



% \begin{table*}[h!]
% \centering

% \begin{tabular}{|l|ccccc|ccccc|}
% \hline
% \multirow{}{}{Models} & \multicolumn{5}{c|}{Our Approach} & \multicolumn{5}{c|}{Self Rewarding} \\
% \cline{2-11}
%  & Chat & Code & Math & Safety & Random & Chat & Code & Math & Safety & Random \\ \hline\hline
% Llama-1B-Inst & 64.37 & 52.84 & 88.14 & 80.48 & 50 & 0 & 0 & 0 & 0 & 50 \\ \hline
% Llama-3B-Inst & 62.09 & 67.17 & \textbf{98.21} & 80.23 & 50 & 0 & 0 & 0 & 0 & 50 \\ \hline
% Llama-8B-Inst & 59.56 & 73.88 & 44.97 & 87.88 & 50 & 0 & 0 & 0 & 0 & 50 \\ \hline
% Qwen-3B-Inst & 60.89  & 80.59  & 46.31  & 86.05 & 50 & 0 & 0 & 0 & 0 & 50 \\ \hline
% Qwen-7B-Inst & \textbf{78.26}  & \textbf{83.13}  & 36.24  & \textbf{93.24}  & 50 & 0 & 0 & 0 & 0 & 50 \\ \hline
% Pythia-1.4B  & 19.17 & 4.47 & 51.90 & 23.61 & 50 & 0 & 0 & 0 & 0 & 50 \\ \hline
% Pythia-6.9B  & 23.67 & 34.55 & 55.26 & 35.60 & 50 & 0 & 0 & 0 & 0 & 50 \\ \hline
% Phi-3-Mini  & 55.91 & 75.30 & 99.10 & 75.32 & 50 & 0 & 0 & 0 & 0 & 50 \\ \hline
% Mistral-7B-Inst & 61.25 & 70.93  & 96.20 & 83.85 & 50 & 0 & 0 & 0 & 0 & 50 \\ \hline
% Gemma2-2B-It & 35.34 & 42.58  & 91.50 & 70.04 & 50 & 0 & 0 & 0 & 0 & 50 \\ \hline
% GPT4-o-Mini & 83.72 & 95.32  & 59.50 & 91.74 & 50 & 0 & 0 & 0 & 0 & 50 \\ \hline
% \end{tabular}
% \caption{The above table compares our approach with the Score Based Self Rewarding approach. The row labels correspond to the model name and the column labels correspond to the sub-categories. The metric used is accuracy where the higher values indicate better performance. }
% \label{tab:model_comparison}
% \end{table*}

% \begin{tabular}{|l||ccccc||ccccc|}
% \hline
% \multirow{}{}{Models} & \multicolumn{5}{c||}{Our Approach} & \multicolumn{5}{c|}{Self Rewarding} \\
% \cline{2-11}
%  & Chat & Code & Math & Safety & Random & Chat & Code & Math & Safety & Random \\ \hline\hline
% Llama-1B-Inst & 64.37 & 52.84 & 88.14 & 80.48 & X1 & 0 & 0 & 0 & 0 & X2 \\ \hline
% Llama-3B-Inst & 62.09 & 67.17 & \textbf{98.21} & 80.23 & X3 & 0 & 0 & 0 & 0 & X4 \\ \hline
% Llama-8B-Inst & 59.56 & 73.88 & 44.97 & 87.88 & X5 & 0 & 0 & 0 & 0 & X6 \\ \hline\hline
% Qwen-3B-Inst & 60.89  & 80.59  & 46.31  & 86.05 & X7 & 0 & 0 & 0 & 0 & X8 \\ \hline
% Qwen-7B-Inst & \textbf{78.26}  & \textbf{83.13}  & 36.24  & \textbf{93.24}  & X9 & 0 & 0 & 0 & 0 & X10 \\ \hline\hline
% Pythia-1.4B  & 0 & 0 & 0 & 0 & X11 & 0 & 0 & 0 & 0 & X12 \\ \hline
% Pythia-6.9B  & 0 & 0 & 0 & 0 & X13 & 0 & 0 & 0 & 0 & X14 \\ \hline\hline
% Phi-3-Mini(3.8B)  & 0 & 0 & 0 & 0 & X15 & 0 & 0 & 0 & 0 & X16 \\ \hline
% Phi-3-Small(7.4B)  & 0 & 0 & 0 & 0 & X17 & 0 & 0 & 0 & 0 & X18 \\ \hline \hline
% Mistral-7B-Inst & 61.25 & 70.93  & 96.20 & 83.85 & X19 & 0 & 0 & 0 & 0 & X20 \\ \hline
% \end{tabular}
% \caption{Reward Bench}
% \label{tab:model_comparison}
% \end{table*}


% \begin{table*}[h!]
% \centering
% \begin{tabular}{l|cccc|cccc}
% \hline
% \multirow{}{}{Models} & \multicolumn{4}{c|}{Our Approach} & \multicolumn{4}{c}{Self Rewarding} \\
% \cline{2-9}
%  & Chat & Code & Math & Safety & Chat & Code & Math & Safety \\
% \hline
% Llama-1B-Inst & 57.62 & 51.46 & 47.57 & 70.24 & 0 & 0 & 0 & 0 \\ \hline
% Llama-3b-Inst & 51.42 & 51.90 & 48.52 & 76.89 & 0 & 0 & 0 & 0 \\ \hline
% Qwen-3B-Inst & 60.98 & 53.07 & 56.33 & 68.96 & 0 & 0 & 0 & 0 \\ \hline
% Qwen-7B-Inst & \textbf{68.73} & \textbf{57.30} & \textbf{58.15} & \textbf{89.49} & 0 & 0 & 0 & 0 \\ \hline
% Mistral-7b-Inst & 54.00 & 52.48 & 52.04 & 71.52 & 0 & 0 & 0 & 0 \\ \hline
% Llama-8B-Inst & 50.64 & 54.97 & 48.70 & 68.25 & 0 & 0 & 0 & 0 \\ \hline

% \end{tabular}
% \caption{RM Bench}
% \label{tab:model_comparison}
% \end{table*}
%%%%%%%%%%%%%%%%%%%%%%%%%%%

\subsection{Methodology}
Our approach leverages a language model as a preference model, evaluating response appropriateness through binary classification. The model determines whether a response is suitable by generating either "Yes" or "No." To guide this assessment, we employ category-specific prompts, which are detailed in Appendix \ref{Prompts_table}. The logits corresponding to the output tokens of "Yes" and "No" are extracted from the first output token and scaled to compute their respective probabilities. The response with the highest "Yes" probability is selected as the accepted response, while the one with the lowest is classified as rejected. We hypothesize that higher-quality responses will have a greater likelihood of receiving a "Yes."

%\subsubsection{Prompting}

%In our experiments, we observed that guiding the language model to initiate its responses with "Yes" or "No" was essential, particularly for smaller models whose outputs are highly sensitive to prompt phrasing. We developed broad, category-specific prompts tailored to different query types to ensure consistency and reliability. Following prior research \citep{lambert2024rewardbenchevaluatingrewardmodels, liu2024rmbenchbenchmarkingrewardmodels}, we classify prompts into four overarching categories: Code, Math, Chat and Safety. Additional details about prompts are provided in Appendix \ref{Prompts_table}. An example of the prompts is shown in Figure \ref{Sample_Prompts}.

% \begin{itemize}
%   \setlength{\itemsep}{0.05em} % Adjust the space between items
%   \item \textbf{Code:} Questions that involve generating or interpreting code.
%   \item \textbf{Math:} Questions related to mathematical computations or reasoning.
%   \item \textbf{Chat:} General conversational questions, often open-ended or casual.
%   \item \textbf{Safety:} Questions focused on evaluating or ensuring safety in responses.
%   For our evaluation on the Reward Bench dataset we further subcategorize it into Safety General and Safety Refusal. 
% \end{itemize}



\begin{figure}[h]
    \centering
    \includegraphics[width=\columnwidth]{assets/prompt.pdf}
    \caption{Example outputs from Reward Bench using our approach.}
    \label{Sample_Prompts}
\end{figure}



\subsubsection{Preference Classification}
\label{Preference Modeling}

In our experiments, we observed that guiding the language model to initiate its responses with "Yes" or "No" was essential, particularly for smaller models whose outputs are highly sensitive to prompt phrasing. We developed broad, category-specific prompts tailored to different query types to ensure consistency and reliability. Following prior research \citep{lambert2024rewardbenchevaluatingrewardmodels, liu2024rmbenchbenchmarkingrewardmodels}, we classify prompts into four overarching categories: Code, Math, Chat and Safety. Additional details about prompts are provided in Appendix \ref{Prompts_table}. An example prompt is shown in Figure \ref{Sample_Prompts}.

To quantify preferences, we extract the output token probabilities for "Yes" and "No" from the response. The detailed approach is outlined below:

Given an input token sequence \( \mathbf{x} = (x_1, x_2, \dots, x_T) \), a language model \( f(\cdot) \) generates a probability distribution over the vocabulary \( \mathcal{V} \) for the next token. Specifically, the model outputs a logit vector \( \mathbf{z} \in \mathbb{R}^{|\mathcal{V}|} \), where  

\begin{equation}
\mathbf{z} = f(\mathbf{x}).
\end{equation}

To derive probabilities, we apply the softmax function over the logits:  

\begin{equation}
p_i = \frac{\exp(z_i)}{\sum_{j \in \mathcal{V}} \exp(z_j)}, \quad \forall i \in \mathcal{V},
\end{equation}

where \( p_i \) represents the probability assigned to token \( i \). Thus we define probability of "Yes" token as $p_{\text{yes}}$ and "No" token as $p_{\text{no}}$. Then we normalize the probabilities to ensure a fair comparison:

% We define the sets of token indices corresponding to "Yes" and "No" as \( V_{\text{yes}} \subset \mathcal{V} \) and \( V_{\text{no}} \subset \mathcal{V} \), respectively. The cumulative probabilities of these responses are computed as:  

% \begin{equation}
% p_{\text{yes}} = \sum_{i \in V_{\text{yes}}} p_i, \quad p_{\text{no}} = \sum_{i \in V_{\text{no}}} p_i.
% \end{equation}


\begin{equation}
p_{\text{yes}}' = \frac{p_{\text{yes}}}{p_{\text{yes}} + p_{\text{no}}}, \quad 
p_{\text{no}}' = \frac{p_{\text{no}}}{p_{\text{yes}} + p_{\text{no}}}.
\end{equation}

The final values \( (p_{\text{yes}}', p_{\text{no}}') \) represent the normalized likelihoods of the model predicting "Yes" or "No" .

\subsection{Experiments}
\subsubsection{Benchmarking Our Approach}

To evaluate our approach, we conducted experiments using LLMs of varying sizes and architectures. We compared instruction-tuned models with their base counterparts. Additionally, we analyzed the effect of fine-tuning on a specialized task like code/math problems on preference classification by including models fine-tuned for these tasks. For comparisons involving a reward model we use the Skywork Reward Llama 8B model \cite{liu2024skywork} as the baseline.
The detailed results for all the comparisons are available in Appendix \ref{All Results}.

In particular, we tested the following models:

\begin{itemize}
  \setlength{\itemsep}{0.05em} % Adjust the space between items
  \item \textbf{LLaMA Family \citep{dubey2024llama}:} LLaMA-3.2-1B, LLaMA-3.2-1B-Instruct, LLaMA-3.2-3B, LLaMA-3.2-3B-Instruct, Meta LLaMA 3-8B, Meta LLaMA 3-8B-Instruct.
  \item \textbf{Mistral Family \citep{jiang2023mistral}:} Mistral 7B, Mistral 7B-Instruct.
  \item \textbf{Qwen Family \citep{yang2024qwen2}:} Qwen2.5-3B, Qwen2.5-3B-Instruct, Qwen2.5-7B, Qwen2.5-7B-Instruct.
  \item \textbf{Code Generation Models:} Starcoder2-7B \citep{lozhkov2024starcoder}, CodeGemma-7B-It \citep{team2024codegemma}, Qwen-Coder-7B-Inst \citep{hui2024qwen2}, Qwen-Coder-3B-Inst.
  \item \textbf{Math Generation Models:} Qwen-Math-7B-Inst, Qwen-Math-1.5B-Instruct \citep{yang2024qwen2}, Deepseek-Math-7B \citep{shao2024deepseekmathpushinglimitsmathematical}, Llemma-7B \citep{azerbayev2024llemmaopenlanguagemodel}.
  \item \textbf{Other Models:} Phi-3-mini-128k-Instruct \citep{abdin2024phi}, Gemma 2B-Instruct \citep{team2024gemma}, GPT-4o Mini \citep{openai2024gpt4technicalreport}.

\end{itemize}

To evaluate model performance, we selected Reward Bench due to its high-quality and diversity. Reward Bench consists of 23 question categories, which are grouped into four broad types: Chat, Code, Math, and Safety. We also benchmark our approach on RM-Bench, results of which can be found in Table \ref{tab:rm_bench_levels}.

We define accuracy as the proportion of cases where the model assigns a higher probability to the preferred response \( y^w \) over the less preferred response \( y^l \):

\[
\text{Acc} = \frac{1}{N} \sum_{i=1}^{N} \mathbb{I} \left[ p_{\text{yes}}(x_i, y^w_i) > p_{\text{yes}}(x_i, y^l_i) \right]
\]

where \( \mathbb{I} [\cdot] \) is the indicator function, returning 1 if the condition holds and 0 otherwise and $N$ is the number of data points.

To ensure optimal model performance, we developed an automated pipeline for selecting the most effective category-specific prompts. Further details on prompt selection can be found in Appendix \ref{Prompts_table}.

\subsubsection{Comparision against Self Rewarding Approach}
We benchmarked our approach against the preference classification approach used in the Self-Rewarding Language Model\footnote{The Self-Rewarding approach performs very poorly on Base Models, so we tested their method on only Instruct models.}. Their approach involves scoring responses using a numerical reward of up to 5 \citep{yuan2024selfrewarding,li2024selfalignment}. Each response is evaluated based on its relevance, completeness, clarity, and informativeness. The comparitive results are shown in Table \ref{tab:reward_bench}.
%%%%%%%%%%%%%%%%%%%%%%%%%%%%%%%%%%%%
\begin{figure*}[h]
    \centering
    \includegraphics[width=\textwidth]{assets/math_code.pdf}
    \caption{\textbf{Left}: Our approach on Code Specific Model where the dashed line is a reward model. \textbf{Right}: Our approach on 4 different math-specific models where the striped bar is the reward model.}
    \label{Code results}
\end{figure*}
%%%%%%%%%%%%%%%%%%%%%%%%%%%%%%%%%%%%
\subsection{Findings}
Our approach demonstrated robust and consistent performance across all subcategories of the Reward Bench, particularly when compared to the self-rewarding approach. This performance gap was particularly pronounced in smaller models, where our approach significantly outperformed the self-rewarding approach. The self-rewarding approach assigns discrete rewards ranging from 1 to 5 for each response, making it challenging to differentiate between them, often rating both the chosen and the rejected response as the same.

Another insight was that most models perform well on safety, indicating safety tuning across all the models during training. Chat performance remains relatively consistent across models, suggesting a similar level of optimization for conversational abilities. However, performance on code and math varies significantly, largely depending on the type of training data used \citep{gunasekar2023textbooks,petty2024does,aryabumi2024code}. For example, the Qwen family excels in coding tasks, while Llama 3.2, Mistral, Gemma, and Phi models demonstrate strong mathematical capabilities.

Another finding was that larger models consistently outperformed smaller models, as shown in Table \ref{tab:reward_bench} and that instruction-tuned models consistently outperformed their base counterparts, reinforcing the effectiveness of instruction-based fine-tuning even in acting as preference classifiers. Additional results of our approach on RM-Bench can be found in \ref{All Results}.

On proprietary models, such as GPT, our approach remained competitive. Results using our approach on GPT-4o-Mini on Reward Bench can be found in Appendix \ref{GPT_Results}. 

\subsection{Performance of Math and Code Specific Models}

To better understand the applicability of our approach in mathematical and coding tasks, we evaluated four models fine-tuned for code completion and four models optimized for mathematical problem-solving. These models were benchmarked against Skywork-Llama8B-Reward Model, which serves as a strong baseline for preference modeling.

Among the code-specific models, Qwen consistently achieved the highest performance across all evaluated categories, performing as well as the Reward Model. 

In contrast, all math-specific models underperformed compared to both the general instruct-tuned version and the Reward Model. We hypothesize that this underperformance stems from the training objective of math-specific models, which prioritize generating chain-of-thought reasoning \citep{yang2024qwen2,shao2024deepseekmathpushinglimitsmathematical,gao2024designing,zhou2024dual} rather than adhering to strict instruction-following behavior required for binary Yes/No classification.
\section{\dpo}

\begin{figure*}
    \centering
    \includegraphics[width=\linewidth]{imgs/dpo.pdf}
    \caption{\textbf{Overview of the \dpo framework,} The dynamic reward scaling mechanism adjusts the update strength based on the reward margin, improving optimization stability and robustness.}
    \label{fig:dpo_alg}
\end{figure*}

In this section, we propose \dpo, an extension of the traditional DPO framework. \dpo introduces Dynamic Reward Scaling, which dynamically adjusts the update strength based on the confidence of training pairs, ensuring effective utilization of high-quality samples while mitigating the impact of noisy or low-confidence data.

\subsection{Background: Direct Preference Optimization}
The DPO framework is a preference-based learning method that optimizes model parameters $\theta$ by aligning model outputs with human preferences. Given a query $\mathbf{x}$ and corresponding responses $y_w$ (positive) and $y_l$ (negative), the DPO loss is defined as:
\begin{equation}
\ell_{\text{DPO}}(\theta) = 
\mathbb{E}_{\mathbf{x}, y_w, y_l} 
\Big[ 
    - \log \sigma \Big( 
        \beta \Big( 
            \log \frac{\pi_\theta(y_w | \mathbf{x})}{\pi_{\text{ref}}(y_w | \mathbf{x})} 
            - 
            \log \frac{\pi_\theta(y_l | \mathbf{x})}{\pi_{\text{ref}}(y_l | \mathbf{x})} 
        \Big)
    \Big)
\Big],
\end{equation}
where $\pi_\theta$ is the model's predicted probability distribution, $\pi_{\text{ref}}$ is a reference policy, $\beta$ is a scaling factor, and $\sigma(\cdot)$ is the sigmoid function. Traditional DPO treats all training pairs equally, regardless of their quality differences. This uniform scaling fails to prioritize high-quality pairs with clear preference distinctions, leading to inefficient use of informative samples and suboptimal optimization.

\subsection{\dpo: Key Contributions and Improvements}
\paragraph{Training on all possible comparison pairs instead of the hardest pairs}.  
Unlike many recent MLLM alignment approaches that prioritize training on the hardest comparison pairs, \dpo incorporates all possible comparison pairs for a single query into the training process. Specifically, for any query with multiple responses, every response pair with differing ranks is treated as a valid comparison pair. This comprehensive approach captures more nuanced ranking information, allowing the model to learn from a broader set of preferences. However, this strategy also introduces a challenge: pairs involving responses with similar ranks (e.g., rank 3 and rank 4) often have lower reward margins compared to pairs with more distinct rankings (e.g., rank 1 and rank 4). Treating all pairs equally, as in traditional DPO, exacerbates the issue of uniform scaling and underutilizes the high-confidence information contained in larger reward margins. To address this, \dpo introduces Dynamic Reward Scaling, which dynamically adjusts the update strength based on the reward margin to prioritize high-confidence training pairs.

\begin{wrapfigure}{r}{0.34\linewidth}
\vspace{-0.7cm}
  \begin{center}
    \includegraphics[width=\linewidth]{imgs/beta_curve.pdf}
\vspace{-0.4cm}
\caption{Effect of $k$ on $1 - e^{-k \delta}$.}
\label{fig:beta_func}
\end{center}
\vspace{-0.4cm}
\end{wrapfigure}
\paragraph{Definition of dynamic reward scaling}. Reward models can naturally provide a pairwise reward margin, which serves as a straightforward signal for scaling. However, two critical aspects must be addressed: (1) ensuring the signal quality is sufficiently high, and (2) bounding the signal to prevent overly aggressive updates that might destabilize training.


Regarding the first aspect, our experiments reveal that publicly available models, such as GPT-4o and LLaVA-Critic, perform inadequately in scoring our dataset. Conversely, our \abbr-Reward-7B model surpasses several publicly available 72B models, offering a reliable and robust reward signal. We use this model to compute the reward margin: 
 $\delta = r(y_w) - r(y_l),$
where $r(y_w)$ and $r(y_l)$ are the scores assigned to the positive and negative samples.

For the second factor, we control the scaling factor $\beta(\delta)$ using the following formulation:
\[
\beta(\delta) = \beta_{\text{ori}} \Big( 1 + w \big( 1 - e^{-k \delta} \big) \Big),
\]

where $\beta_{\text{ori}}$ is the initial default scaling factor, $w$ is a parameter balancing the dynamic component's contribution, and $k$ is a tunable hyperparameter that adjusts $\beta(\delta)$'s sensitivity to changes in $\delta$. The function $1 - e^{-k \delta}$ is bounded between $[0, 1]$, {as illustrated in Figure~\ref{fig:beta_func}}. A smaller $k$ value keeps most $\beta(\delta)$ values near $\beta_{\text{ori}}$, with slow growth as $\delta$ increases. In contrast, a larger $k$ makes $\beta(\delta)$ highly responsive to changes in $\delta$, quickly reaching its maximum. To avoid overly aggressive updates, we constrain $\beta(\delta)$ within $[\beta_{\text{ori}}, (1 + w) \beta_{\text{ori}}]$. Overall, Dynamic Reward Scaling significantly enhances \dpo by leveraging high-quality reward signals and tailoring optimization steps to the confidence level of training pairs. This results in improved robustness, efficiency, and overall effectiveness of the framework. We discuss the similarities and differing perspectives between our approach and existing methods in Appendix~\ref{sec:app_com_beta}.

\section{Conclusion}
In this work, we propose a simple yet effective approach, called SMILE, for graph few-shot learning with fewer tasks. Specifically, we introduce a novel dual-level mixup strategy, including within-task and across-task mixup, for enriching the diversity of nodes within each task and the diversity of tasks. Also, we incorporate the degree-based prior information to learn expressive node embeddings. Theoretically, we prove that SMILE effectively enhances the model's generalization performance. Empirically, we conduct extensive experiments on multiple benchmarks and the results suggest that SMILE significantly outperforms other baselines, including both in-domain and cross-domain few-shot settings. 
\section{Limitation}
The use of 3D-printed PLA for structural components improves improving ease of assembly and reduces weight and cost, yet it causes deformation under heavy load, which can diminish end-effector precision. Using metal, such as aluminum, would remedy this problem. Additionally, \robot relies on integrated joint relative encoders, requiring manual initialization in a fixed joint configuration each time the system is powered on. Using absolute joint encoders could significantly improve accuracy and ease of use, although it would increase the overall cost. 

%Reliance on commercially available actuators simplifies integration but imposes constraints on control frequency and customization, further limiting the potential for tailored performance improvements.

% The 6 DoF configuration provides sufficient mobility for most tasks; however, certain bimanual operations could benefit from an additional degree of freedom to handle complex joint constraints more effectively. Furthermore, the limited torque density of commercially available proprioceptive actuators restricts the payload and torque output, making the system less suitability for handling heavier loads or high-torque applications. 

The 6 DoF configuration of the arm provides sufficient mobility for single-arm manipulation tasks, yet it shows a limitation in certain bimanual manipulation problems. Specifically, when \robot holds onto a rigid object with both hands, each arm loses 1 DoF because the hands are fixed to the object during grasping. This leads to an underactuated kinematic chain which has a limited mobility in 3D space. We can achieve more mobility by letting the object slip inside the grippers, yet this renders the grasp less robust and simulation difficult. Therefore, we anticipate that designing a lightweight 3 DoF wrist in place of the current 2 DoF wrist allows a more diverse repertoire of manipulation in bimanual tasks.

Finally, the limited torque density of commercially available proprioceptive actuators restricts the performance. Currently, all of our actuators feature a 1:10 gear ratio, so \robot can handle up to 2.5 kg of payload. To handle a heavier object and manipulate it with higher torque, we expect the actuator to have 1:20$\sim$30 gear ratio, but it is difficult to find an off-the-shelf product that meets our requirements. Customizing the actuator to increase the torque density while minimizing the weight will enable \robot to move faster and handle more diverse objects.

%These constraints highlight opportunities for improvement in future iterations, including alternative materials for enhanced rigidity, custom actuator designs for higher control precision and torque density, the adoption of absolute joint encoders, and optimized configurations to balance dexterity and weight.


\section*{Acknowledgment}
This work was supported by the National Natural Science Foundation of China (62441239,~U23A20319,~62172056,~62472394,~62192784, \\ U22B2038) as well as the 8th Young Elite Scientists Sponsorship Program by CAST (2022QNRC001).



\bibliography{acl}

\appendix
\begin{table}[h]
    \centering
    \begin{tabular}{lc}
    \toprule
        \textbf{Parameter} & \textbf{value} \\ \midrule
        num\_epochs & 3 \\
        save\_steps & 100 \\
        eval\_steps & 100 \\
        logging\_steps & 100 \\
        batch\_size & 64 \\
        gradient\_accumulation & 1 \\
        weight\_decay & 0.01 \\
        bf16 & True \\
        \bottomrule
    \end{tabular}
    \caption{Training hyperparameters}
    \label{tab:training-hyperparameter}
\end{table}
\section{Dataset}
\label{sec:dataset}

\subsection{Data Collection}

To analyze political discussions on Discord, we followed the methodology in \cite{singh2024Cross-Platform}, collecting messages from politically-oriented public servers in compliance with Discord's platform policies.

Using Discord's Discovery feature, we employed a web scraper to extract server invitation links, names, and descriptions, focusing on public servers accessible without participation. Invitation links were used to access data via the Discord API. To ensure relevance, we filtered servers using keywords related to the 2024 U.S. elections (e.g., Trump, Kamala, MAGA), as outlined in \cite{balasubramanian2024publicdatasettrackingsocial}. This resulted in 302 server links, further narrowed to 81 English-speaking, politics-focused servers based on their names and descriptions.

Public messages were retrieved from these servers using the Discord API, collecting metadata such as \textit{content}, \textit{user ID}, \textit{username}, \textit{timestamp}, \textit{bot flag}, \textit{mentions}, and \textit{interactions}. Through this process, we gathered \textbf{33,373,229 messages} from \textbf{82,109 users} across \textbf{81 servers}, including \textbf{1,912,750 messages} from \textbf{633 bots}. Data collection occurred between November 13th and 15th, covering messages sent from January 1st to November 12th, just after the 2024 U.S. election.

\subsection{Characterizing the Political Spectrum}
\label{sec:timeline}

A key aspect of our research is distinguishing between Republican- and Democratic-aligned Discord servers. To categorize their political alignment, we relied on server names and self-descriptions, which often include rules, community guidelines, and references to key ideologies or figures. Each server's name and description were manually reviewed based on predefined, objective criteria, focusing on explicit political themes or mentions of prominent figures. This process allowed us to classify servers into three categories, ensuring a systematic and unbiased alignment determination.

\begin{itemize}
    \item \textbf{Republican-aligned}: Servers referencing Republican and right-wing and ideologies, movements, or figures (e.g., MAGA, Conservative, Traditional, Trump).  
    \item \textbf{Democratic-aligned}: Servers mentioning Democratic and left-wing ideologies, movements, or figures (e.g., Progressive, Liberal, Socialist, Biden, Kamala).  
    \item \textbf{Unaligned}: Servers with no defined spectrum and ideologies or opened to general political debate from all orientations.
\end{itemize}

To ensure the reliability and consistency of our classification, three independent reviewers assessed the classification following the specified set of criteria. The inter-rater agreement of their classifications was evaluated using Fleiss' Kappa \cite{fleiss1971measuring}, with a resulting Kappa value of \( 0.8191 \), indicating an almost perfect agreement among the reviewers. Disagreements were resolved by adopting the majority classification, as there were no instances where a server received different classifications from all three reviewers. This process guaranteed the consistency and accuracy of the final categorization.

Through this process, we identified \textbf{7 Republican-aligned servers}, \textbf{9 Democratic-aligned servers}, and \textbf{65 unaligned servers}.

Table \ref{tab:statistics} shows the statistics of the collected data. Notably, while Democratic- and Republican-aligned servers had a comparable number of user messages, users in the latter servers were significantly more active, posting more than double the number of messages per user compared to their Democratic counterparts. 
This suggests that, in our sample, Democratic-aligned servers attract more users, but these users were less engaged in text-based discussions. Additionally, around 10\% of the messages across all server categories were posted by bots. 

\subsection{Temporal Data} 

Throughout this paper, we refer to the election candidates using the names adopted by their respective campaigns: \textit{Kamala}, \textit{Biden}, and \textit{Trump}. To examine how the content of text messages evolves based on the political alignment of servers, we divided the 2024 election year into three periods: \textbf{Biden vs Trump} (January 1 to July 21), \textbf{Kamala vs Trump} (July 21 to September 20), and the \textbf{Voting Period} (after September 20). These periods reflect key phases of the election: the early campaign dominated by Biden and Trump, the shift in dynamics with Kamala Harris replacing Joe Biden as the Democratic candidate, and the final voting stage focused on electoral outcomes and their implications. This segmentation enables an analysis of how discourse responds to pivotal electoral moments.

Figure \ref{fig:line-plot} illustrates the distribution of messages over time, highlighting trends in total messages volume and mentions of each candidate. Prior to Biden's withdrawal on July 21, mentions of Biden and Trump were relatively balanced. However, following Kamala's entry into the race, mentions of Trump surged significantly, a trend further amplified by an assassination attempt on him, solidifying his dominance in the discourse. The only instance where Trump’s mentions were exceeded occurred during the first debate, as concerns about Biden’s age and cognitive abilities temporarily shifted the focus. In the final stages of the election, mentions of all three candidates rose, with Trump’s mentions peaking as he emerged as the victor.
\begin{table*}[!ht]
\centering
\resizebox{0.70\textwidth}{!}
{
\begin{tabular}{l|ccc|ccc|ccc}
\hline
& \multicolumn{6}{c|}{\textbf{Span Identification}} & \multicolumn{3}{c}{\textbf{Incongruence Alignment}} \\
\cline{2-7}

\textbf{Models} & \multicolumn{3}{c|}{\textbf{Testimony $T1$}} & \multicolumn{3}{c|}{\textbf{Testimony $T2$}} & \multicolumn{3}{c}{\textbf{between $T1$ \& $T2$}} \\
\cline{2-10}
 & \textbf{Pre} & \textbf{Rec} & \textbf{F1}  & \textbf{Pre} & \textbf{Rec} & \textbf{F1}  & \textbf{Pre} & \textbf{Rec} & \textbf{F1} \\


\hline

\multirow{1}{*}{\rotatebox{0}{\bf LLAMA-3 [8B] }} &  0.496 &  0.583 &  0.490  &  0.515 &  0.579 &  0.498  &  0.519 &  0.621 &  0.512 \\

\hline

\multirow{1}{*}{\rotatebox{0}{\bf GPT 4o mini }} &  0.505 & 0.574 & 0.491 & 0.528 & 0.566 & 0.505 & 0.524 & 0.614 & 0.513 \\

\hline

\end{tabular}}
% \label{Performance:Span}
\vspace{-3mm}
\caption{Performance evaluation of Incongruence span identification on \dataset\ using GPT 4o mini.}
\label{tab:GPT-results}
\end{table*}
\section{Prompts}
\label{Prompts_table}
Based on the predefined categories, a pool of N prompts were generated using GPT. A small sample of 50 data points was selected from each category in Reward Bench, and the prompts were tested on this subset to determine the most effective one. The model used for testing was LLaMA-1B-Instruct, and the prompt with the highest accuracy in each category was used for evaluation on Reward Bench. The final set of selected prompts for Reward Bench evaluations are shown in Table \ref{Prompts} and those for DPO are shown in \ref{dpo_prompts} .

\textbf{Note:} For DPO-based training, instructions were classified into four categories

Categories of Prompts: 
\begin{itemize}
  \setlength{\itemsep}{0.05em} % Adjust the space between items
  \item \textbf{Code:} Questions that involve generating or interpreting code.
  \item \textbf{Math:} Questions related to mathematical computations or reasoning.
  \item \textbf{Chat:} General conversational questions, often open-ended or casual.
  \item \textbf{Safety:} Questions focused on evaluating or ensuring safety in responses.
  For our evaluation on the Reward Bench dataset we further subcategorize it into Safety General and Safety Refusal. 
\end{itemize}

% \begin{table*}[h]
%     \centering
%     % \resizebox{\columnwidth}{!}{%
%     \begin{tabular}{l|p}
%         \toprule
%         \textbf{Task} & \textbf{Prompt} \\
%         \midrule
%         Chat & Below is a question and its response. Assess if the response is helpful and correct. Answer just Yes/No.\\
%         \hline
%         Code & Below is a coding-related question along with a response containing code. Your task is to verify whether the provided code is correct, functional, and aligned with the requirements of the question. Answer with just Yes/No. \\
%         \hline
%         Maths & Below is a math question along with a response containing a solution. Your task is to verify whether the solution is correct, the reasoning is logical, and the final answer is accurate. Answer with just Yes/No. \\
%         \hline
%         Safety(General) & Below is a potentially unsafe or inappropriate question, along with a refusal response. Verify whether the refusal response is appropriate. Answer Yes/No. \\
%         \hline
%         Safety(Refusal) & Below is a question and a refusal response. Ensure the refusal response aligns with ethical and safety guidelines while maintaining professional clarity. Answer Yes/No. \\
%         \bottomrule
%     \end{tabular}%
%     % }
% \end{table*}
\begin{table*}[h]
    \label{Prompts}
    \centering
    \renewcommand{\arraystretch}{1.2} % Improves row spacing
    \setlength{\tabcolsep}{8pt} % Adjusts column spacing
    \begin{tabular}{l|p{12cm}}
        \toprule
        \textbf{Task} & \textbf{Prompt} \\
     \midrule
            \textbf{Chat}  & Below is a question and its response. Assess if the response is helpful and correct. Answer just Yes/No.\\
            
            \textbf{Code} & Below is a coding-related question along with a response containing code. Your task is to verify whether the provided code is correct, functional, and aligned with the requirements of the question. Answer with just Yes/No. \\
            
            \textbf{Maths} & Below is a math question along with a response containing a solution. Your task is to verify whether the solution is correct, the reasoning is logical, and the final answer is accurate. Answer with just Yes/No. \\
            
            \textbf{Safety(General)} & Below is a potentially unsafe or inappropriate question, along with a refusal response. Verify whether the refusal response is appropriate. Answer Yes/No. \\
           
            \textbf{Safety(Refusal)} & Below is a question and a refusal response. Ensure the refusal response aligns with ethical and safety guidelines while maintaining professional clarity. Answer Yes/No. \\
            \bottomrule
    \end{tabular}
        \caption{Final prompts used for Evaluation on Reward and RM Bench}
    \label{Prompts}
\end{table*}

\begin{table*}[h]
    \centering
    \renewcommand{\arraystretch}{1.2}
    \begin{tabular}{l|p{14cm}}
        \hline
        \textbf{Category} & \textbf{Prompt} \\ 
        \hline
        \textbf{Code} & Below is a coding-related question along with a response containing code. Your task is to verify whether the provided code is correct, functional, and aligned with the requirements of the question. Answer with just Yes/No. \\ 
        \textbf{Chat} & Below is a question, along with a response to that question. Your task is to verify whether the response is helpful, factually correct, logically sound, and contextually appropriate. Answer with just Yes/No. \\ 
        \textbf{Safety} & Inspect the response to ensure it is safe and appropriate for the question below. Answer Yes/No. \\ 
        \textbf{Math} & Below is a math question along with a response containing a solution. Your task is to verify whether the solution is correct, the reasoning is logical, and the final answer is accurate. Answer with just Yes/No. \\ 
        \hline
    \end{tabular}
    \caption{Final set of prompts used for DPO.}
    \label{dpo_prompts}
\end{table*}


\begin{tabular}{lllll|rrrrrr|rrrrrr}
\toprule
LLM & Temp & Feature & \textit{all-layers} & \textit{per-layer} & \multicolumn{6}{c}{Train AUROC} & \multicolumn{6}{c}{Test AUROC} \\
\cmidrule(lr){6-11} \cmidrule(lr){12-17}
 &  &  &  &  & CoQA & HaluevalQA & NQOpen & SQuADv2 & TriviaQA & TruthfulQA & CoQA & HaluevalQA & NQOpen & SQuADv2 & TriviaQA & TruthfulQA \\
\midrule
\textcolor{gray}{Llama3.1-8B} & \textcolor{gray}{0.1} & \textcolor{gray}{$\attnscore$} &  & \textcolor{gray}{\checkmark} & \textcolor{gray}{0.509} & \textcolor{gray}{0.667} & \textcolor{gray}{0.607} & \textcolor{gray}{0.556} & \textcolor{gray}{0.567} & \textcolor{gray}{0.563} & \textcolor{gray}{0.541} & \textcolor{gray}{0.653} & \textcolor{gray}{0.631} & \textcolor{gray}{0.575} & \textcolor{gray}{0.571} & \textcolor{gray}{0.650} \\
\textcolor{gray}{Llama3.1-8B }& \textcolor{gray}{0.1} & \textcolor{gray}{$\attnscore$} & \textcolor{gray}{\checkmark} &  & \textcolor{gray}{0.494} & \textcolor{gray}{0.614} & \textcolor{gray}{0.568} & \textcolor{gray}{0.522} & \textcolor{gray}{0.522} & \textcolor{gray}{0.489} & \textcolor{gray}{0.504} & \textcolor{gray}{0.587} & \textcolor{gray}{0.558} & \textcolor{gray}{0.521} & \textcolor{gray}{0.511} & \textcolor{gray}{0.537} \\
Llama3.1-8B & 0.1 & $\attnlogdet$ &  & \checkmark & 0.574 & 0.776 & 0.702 & 0.688 & 0.739 & 0.709 & 0.606 & 0.770 & 0.713 & 0.708 & 0.741 & 0.777 \\
Llama3.1-8B & 0.1 & $\attnlogdet$ & \checkmark &  & 0.843 & 0.884 & 0.851 & 0.839 & 0.861 & 0.913 & 0.770 & 0.837 & 0.768 & 0.758 & 0.827 & 0.820 \\
Llama3.1-8B & 0.1 & $\attneig$ &  & \checkmark & 0.764 & 0.828 & 0.713 & 0.742 & 0.793 & 0.680 & 0.729 & 0.799 & 0.728 & 0.749 & 0.773 & 0.790 \\
Llama3.1-8B & 0.1 & $\attneig$ & \checkmark &  & 0.861 & 0.895 & 0.878 & 0.858 & 0.867 & 0.979 & 0.776 & 0.838 & 0.755 & 0.781 & 0.822 & 0.819 \\
Llama3.1-8B & 0.1 & $\lapeig$ &  & \checkmark & 0.758 & 0.817 & 0.698 & 0.707 & 0.781 & 0.708 & 0.757 & 0.793 & 0.711 & 0.733 & 0.780 & 0.764 \\
Llama3.1-8B & 0.1 & $\lapeig$ & \checkmark &  & 0.869 & 0.901 & 0.864 & 0.855 & 0.896 & 0.903 & \textbf{0.836} & \textbf{0.867} & \textbf{0.793} & \textbf{0.782} & \textbf{0.872} & \textbf{0.822} \\
\midrule
\textcolor{gray}{Llama3.1-8B} & \textcolor{gray}{1.0} & \textcolor{gray}{$\attnscore$} &  & \textcolor{gray}{\checkmark} & \textcolor{gray}{0.514} & \textcolor{gray}{0.640} & \textcolor{gray}{0.607} & \textcolor{gray}{0.558} & \textcolor{gray}{0.578} & \textcolor{gray}{0.533} & \textcolor{gray}{0.525} & \textcolor{gray}{0.642} & \textcolor{gray}{0.607} & \textcolor{gray}{0.572} & \textcolor{gray}{0.602} & \textcolor{gray}{0.629} \\
\textcolor{gray}{Llama3.1-8B} & \textcolor{gray}{1.0} & \textcolor{gray}{$\attnscore$} & \textcolor{gray}{\checkmark} &  & \textcolor{gray}{0.507} & \textcolor{gray}{0.602} & \textcolor{gray}{0.580} & \textcolor{gray}{0.534} & \textcolor{gray}{0.535} & \textcolor{gray}{0.546} & \textcolor{gray}{0.493} & \textcolor{gray}{0.589} & \textcolor{gray}{0.556} & \textcolor{gray}{0.538} & \textcolor{gray}{0.532} & \textcolor{gray}{0.541} \\
Llama3.1-8B & 1.0 & $\attnlogdet$ &  & \checkmark & 0.596 & 0.755 & 0.704 & 0.697 & 0.750 & 0.757 & 0.597 & 0.763 & 0.757 & 0.686 & 0.754 & 0.771 \\
Llama3.1-8B & 1.0 & $\attnlogdet$ & \checkmark &  & 0.848 & 0.882 & 0.856 & 0.846 & 0.867 & 0.930 & 0.769 & 0.827 & 0.793 & 0.748 & 0.842 & 0.814 \\
Llama3.1-8B & 1.0 & $\attneig$ &  & \checkmark & 0.762 & 0.820 & 0.758 & 0.754 & 0.800 & 0.796 & 0.723 & 0.784 & 0.732 & 0.728 & 0.796 & 0.770 \\
Llama3.1-8B & 1.0 & $\attneig$ & \checkmark &  & 0.867 & 0.889 & 0.873 & 0.867 & 0.876 & 0.972 & 0.782 & 0.819 & 0.790 & 0.768 & 0.843 & \textbf{0.833} \\
Llama3.1-8B & 1.0 & $\lapeig$ &  & \checkmark & 0.760 & 0.803 & 0.732 & 0.722 & 0.795 & 0.751 & 0.743 & 0.789 & 0.725 & 0.724 & 0.794 & 0.764 \\
Llama3.1-8B & 1.0 & $\lapeig$ & \checkmark &  & 0.879 & 0.896 & 0.866 & 0.857 & 0.901 & 0.918 & \textbf{0.830} & \textbf{0.874} & \textbf{0.827} & \textbf{0.791} & \textbf{0.889} & 0.829 \\
\midrule
\textcolor{gray}{Llama3.2-3B} & \textcolor{gray}{0.1} & \textcolor{gray}{$\attnscore$} &  & \textcolor{gray}{\checkmark} & \textcolor{gray}{0.526} & \textcolor{gray}{0.697} & \textcolor{gray}{0.592} & \textcolor{gray}{0.570} & \textcolor{gray}{0.570} & \textcolor{gray}{0.569} & \textcolor{gray}{0.547} & \textcolor{gray}{0.714} & \textcolor{gray}{0.643} & \textcolor{gray}{0.582} & \textcolor{gray}{0.551} & \textcolor{gray}{0.564} \\
\textcolor{gray}{Llama3.2-3B} & \textcolor{gray}{0.1} & \textcolor{gray}{$\attnscore$} & \textcolor{gray}{\checkmark} &  & \textcolor{gray}{0.506} & \textcolor{gray}{0.635} & \textcolor{gray}{0.523} & \textcolor{gray}{0.515} & \textcolor{gray}{0.534} & \textcolor{gray}{0.473} & \textcolor{gray}{0.519} & \textcolor{gray}{0.644} & \textcolor{gray}{0.573} & \textcolor{gray}{0.561} & \textcolor{gray}{0.510} & \textcolor{gray}{0.489} \\
Llama3.2-3B & 0.1 & $\attnlogdet$ &  & \checkmark & 0.573 & 0.762 & 0.692 & 0.682 & 0.719 & 0.725 & 0.579 & 0.774 & 0.735 & 0.698 & 0.711 & 0.674 \\
Llama3.2-3B & 0.1 & $\attnlogdet$ & \checkmark &  & 0.782 & 0.868 & 0.845 & 0.827 & 0.824 & 0.918 & 0.695 & 0.843 & 0.763 & \textbf{0.749} & 0.796 & 0.678 \\
Llama3.2-3B & 0.1 & $\attneig$ &  & \checkmark & 0.675 & 0.782 & 0.750 & 0.725 & 0.755 & 0.727 & 0.626 & 0.792 & 0.734 & 0.695 & 0.724 & 0.720 \\
Llama3.2-3B & 0.1 & $\attneig$ & \checkmark &  & 0.814 & 0.873 & 0.872 & 0.852 & 0.842 & 0.963 & 0.723 & 0.844 & 0.772 & 0.744 & 0.788 & 0.688 \\
Llama3.2-3B & 0.1 & $\lapeig$ &  & \checkmark & 0.681 & 0.774 & 0.733 & 0.708 & 0.733 & 0.722 & 0.676 & 0.781 & 0.736 & 0.697 & 0.732 & 0.690 \\
Llama3.2-3B & 0.1 & $\lapeig$ & \checkmark &  & 0.831 & 0.875 & 0.837 & 0.832 & 0.852 & 0.895 & \textbf{0.801} & \textbf{0.857} & \textbf{0.779} & 0.736 & \textbf{0.826} & \textbf{0.743} \\
\midrule
\textcolor{gray}{Llama3.2-3B} & \textcolor{gray}{1.0} & \textcolor{gray}{$\attnscore$} &  & \textcolor{gray}{\checkmark} & \textcolor{gray}{0.532} & \textcolor{gray}{0.668} & \textcolor{gray}{0.588} & \textcolor{gray}{0.578} & \textcolor{gray}{0.553} & \textcolor{gray}{0.555} & \textcolor{gray}{0.557} & \textcolor{gray}{0.637} & \textcolor{gray}{0.592} & \textcolor{gray}{0.593} & \textcolor{gray}{0.558} & \textcolor{gray}{0.675} \\
\textcolor{gray}{Llama3.2-3B} & \textcolor{gray}{1.0} & \textcolor{gray}{$\attnscore$} & \textcolor{gray}{\checkmark} &  & \textcolor{gray}{0.512} & \textcolor{gray}{0.606} & \textcolor{gray}{0.554} & \textcolor{gray}{0.529} & \textcolor{gray}{0.517} & \textcolor{gray}{0.484} & \textcolor{gray}{0.509} & \textcolor{gray}{0.588} & \textcolor{gray}{0.546} & \textcolor{gray}{0.530} & \textcolor{gray}{0.515} & \textcolor{gray}{0.581} \\
Llama3.2-3B & 1.0 & $\attnlogdet$ &  & \checkmark & 0.578 & 0.738 & 0.677 & 0.720 & 0.716 & 0.739 & 0.597 & 0.724 & 0.678 & 0.707 & 0.711 & 0.742 \\
Llama3.2-3B & 1.0 & $\attnlogdet$ & \checkmark &  & 0.784 & 0.869 & 0.816 & 0.839 & 0.831 & 0.924 & 0.700 & 0.801 & 0.690 & 0.734 & 0.789 & \textbf{0.795} \\
Llama3.2-3B & 1.0 & $\attneig$ &  & \checkmark & 0.642 & 0.777 & 0.716 & 0.747 & 0.763 & 0.735 & 0.641 & 0.756 & 0.696 & 0.703 & 0.746 & 0.748 \\
Llama3.2-3B & 1.0 & $\attneig$ & \checkmark &  & 0.819 & 0.878 & 0.847 & 0.876 & 0.847 & 0.978 & 0.724 & 0.819 & \textbf{0.694} & 0.749 & 0.804 & 0.723 \\
Llama3.2-3B & 1.0 & $\lapeig$ &  & \checkmark & 0.695 & 0.764 & 0.683 & 0.719 & 0.727 & 0.682 & 0.715 & 0.754 & 0.671 & 0.711 & 0.738 & 0.767 \\
Llama3.2-3B & 1.0 & $\lapeig$ & \checkmark &  & 0.842 & 0.885 & 0.803 & 0.850 & 0.863 & 0.911 & \textbf{0.812} & \textbf{0.828} & 0.693 & \textbf{0.757} & \textbf{0.832} & 0.787 \\
\midrule
\textcolor{gray}{Phi3.5} & \textcolor{gray}{0.1} & \textcolor{gray}{$\attnscore$} &  & \textcolor{gray}{\checkmark} & \textcolor{gray}{0.517} & \textcolor{gray}{0.559} & \textcolor{gray}{0.565} & \textcolor{gray}{0.606} & \textcolor{gray}{0.625} & \textcolor{gray}{0.601} & \textcolor{gray}{0.528} & \textcolor{gray}{0.551} & \textcolor{gray}{0.637} & \textcolor{gray}{0.621} & \textcolor{gray}{0.628} & \textcolor{gray}{0.637} \\
\textcolor{gray}{Phi3.5} & \textcolor{gray}{0.1} & \textcolor{gray}{$\attnscore$} & \textcolor{gray}{\checkmark} &  & \textcolor{gray}{0.499} & \textcolor{gray}{0.538} & \textcolor{gray}{0.532} & \textcolor{gray}{0.473} & \textcolor{gray}{0.539} & \textcolor{gray}{0.522} & \textcolor{gray}{0.505} & \textcolor{gray}{0.511} & \textcolor{gray}{0.578} & \textcolor{gray}{0.458} & \textcolor{gray}{0.534} & \textcolor{gray}{0.554} \\
Phi3.5 & 0.1 & $\attnlogdet$ &  & \checkmark & 0.583 & 0.732 & 0.741 & 0.711 & 0.757 & 0.720 & 0.585 & 0.726 & 0.785 & 0.726 & 0.772 & 0.765 \\
Phi3.5 & 0.1 & $\attnlogdet$ & \checkmark &  & 0.845 & 0.863 & 0.905 & 0.852 & 0.875 & 0.981 & 0.723 & 0.802 & 0.802 & 0.759 & 0.842 & 0.716 \\
Phi3.5 & 0.1 & $\attneig$ &  & \checkmark & 0.760 & 0.781 & 0.793 & 0.745 & 0.802 & 0.854 & 0.678 & 0.764 & 0.790 & 0.747 & 0.791 & \textbf{0.774} \\
Phi3.5 & 0.1 & $\attneig$ & \checkmark &  & 0.862 & 0.867 & 0.904 & 0.861 & 0.881 & 0.999 & 0.728 & 0.802 & 0.787 & 0.740 & 0.838 & 0.761 \\
Phi3.5 & 0.1 & $\lapeig$ &  & \checkmark & 0.734 & 0.758 & 0.737 & 0.704 & 0.775 & 0.759 & 0.716 & 0.757 & 0.761 & 0.732 & 0.768 & 0.741 \\
Phi3.5 & 0.1 & $\lapeig$ & \checkmark &  & 0.856 & 0.860 & 0.897 & 0.841 & 0.884 & 0.965 & \textbf{0.810} & \textbf{0.819} & \textbf{0.815} & \textbf{0.791} & \textbf{0.858} & 0.717 \\
\midrule
\textcolor{gray}{Phi3.5} & \textcolor{gray}{1.0} & \textcolor{gray}{$\attnscore$} &  & \textcolor{gray}{\checkmark} & \textcolor{gray}{0.499} & \textcolor{gray}{0.567} & \textcolor{gray}{0.615} & \textcolor{gray}{0.626} & \textcolor{gray}{0.637} & \textcolor{gray}{0.618} & \textcolor{gray}{0.533} & \textcolor{gray}{0.581} & \textcolor{gray}{0.630} & \textcolor{gray}{0.645} & \textcolor{gray}{0.642} & \textcolor{gray}{0.626} \\
\textcolor{gray}{Phi3.5} & \textcolor{gray}{1.0} & \textcolor{gray}{$\attnscore$} & \textcolor{gray}{\checkmark} &  & \textcolor{gray}{0.489} & \textcolor{gray}{0.540} & \textcolor{gray}{0.566} & \textcolor{gray}{0.469} & \textcolor{gray}{0.553} & \textcolor{gray}{0.541} & \textcolor{gray}{0.520} & \textcolor{gray}{0.541} & \textcolor{gray}{0.594} & \textcolor{gray}{0.504} & \textcolor{gray}{0.540} & \textcolor{gray}{0.554} \\
Phi3.5 & 1.0 & $\attnlogdet$ &  & \checkmark & 0.587 & 0.733 & 0.773 & 0.722 & 0.766 & 0.753 & 0.557 & 0.762 & 0.784 & 0.736 & 0.772 & 0.763 \\
Phi3.5 & 1.0 & $\attnlogdet$ & \checkmark &  & 0.842 & 0.868 & 0.921 & 0.859 & 0.879 & 0.971 & 0.745 & 0.818 & 0.815 & 0.769 & 0.848 & 0.755 \\
Phi3.5 & 1.0 & $\attneig$ &  & \checkmark & 0.755 & 0.794 & 0.820 & 0.790 & 0.809 & 0.864 & 0.710 & 0.795 & 0.787 & 0.752 & 0.799 & 0.747 \\
Phi3.5 & 1.0 & $\attneig$ & \checkmark &  & 0.858 & 0.871 & 0.924 & 0.876 & 0.887 & 0.998 & 0.771 & 0.829 & 0.798 & 0.782 & 0.850 & \textbf{0.802} \\
Phi3.5 & 1.0 & $\lapeig$ &  & \checkmark & 0.733 & 0.755 & 0.755 & 0.718 & 0.779 & 0.713 & 0.723 & 0.769 & 0.755 & 0.732 & 0.792 & 0.732 \\
Phi3.5 & 1.0 & $\lapeig$ & \checkmark &  & 0.856 & 0.863 & 0.911 & 0.849 & 0.889 & 0.961 & \textbf{0.821} & \textbf{0.836} & \textbf{0.826} & \textbf{0.795} & \textbf{0.872} & 0.777 \\
\bottomrule

\end{tabular}

% \begin{table}[h]
%     \centering
%     \renewcommand{\arraystretch}{1.3}
%     \begin{tabular}{l|c}
%         \hline
%         \textbf{Model} & \textbf{Accuracy (\%)} \\ 
%         \hline
%         llemma\_7b & 74.50 \\ 
%         \hline
%         deepseek-math-7b-instruct & 83.89 \\ 
%         \hline
%         Qwen2.5-Math-1.5B-Instruct & 48.99 \\ 
%         \hline
%         Qwen2.5-Math-7B-Instruct & 66.67 \\ 
%         \hline
%     \end{tabular}
%     \caption{Accuracy of Math-Specific Models}
%     \label{tab:math_models}
% \end{table}

% \begin{table*}[h]
%     \centering
%     \renewcommand{\arraystretch}{1.3}
%     \begin{tabular}{l|cccccc}
%         \hline
%         \textbf{Model} & \textbf{hep-python} & \textbf{hep-java} & \textbf{hep-cpp} & \textbf{hep-js} & \textbf{hep-go} & \textbf{hep-rust} \\ 
%         \hline
%         starcoder2-7b & 67.07 & 55.49 & 53.66 & 48.78 & 50.00 & 59.76 \\ 
%         \hline
%         codegemma-7b-it & 31.71 & 12.20 & 14.63 & 20.12 & 14.02 & 12.20 \\ 
%         \hline
%         Qwen2.5-Coder-3B-Instruct & 84.15 & 85.98 & 86.59 & 85.98 & 86.59 & 84.15 \\ 
%         \hline
%         Qwen2.5-Coder-7B-Instruct & 96.34 & 97.56 & 94.51 & 92.07 & 93.90 & 90.85 \\ 
%         \hline
%     \end{tabular}
%     \caption{Accuracy for Code Specific Models}
%     \label{tab:hep_accuracy}
% \end{table*}
\begin{table*}[htbp!]
\centering
\setlength{\tabcolsep}{4pt} 
\definecolor{apricot}{rgb}{0.95, 0.82, 0.62}
\definecolor{lightgray}{rgb}{0.96, 0.96, 0.96}

\begin{tabular}{l | ccccc | ccccc }
\hline
\rowcolor{apricot} 
\multirow{0}{*}{\textbf{Models}} & \multicolumn{5}{c|}{\textbf{Our Approach}} & \multicolumn{5}{c}{\textbf{Self Rewarding}} \\
\cmidrule(lr){2-6} \cmidrule(lr){7-11}
% \cline{2-11}
 & Chat & Code & Math & Safety & Average & Chat & Code & Math & Safety & Average  \\ \hline
Llama-3.2-1B-Inst & 64.37 & 52.84 & 88.14 & 80.48 & 71.45 & 30.47 & 21.03 & 14.54 & 31.55 & 24.39\\ \hline
\rowcolor{lightgray}Llama-3.2-3B-Inst & 62.09 & 67.17 & \textbf{98.21} & 80.23 & 76.92 & 33.87 & 24.69 & 36.01 & 46.73 & 35.32\\ \hline
Llama-3-8B-Inst & 59.56 & 73.88 & 54.97 & 87.88 & 69.07 & 35.43 & 12.29 & 21.70 & \textbf{58.35} & 31.94\\ \hline
\rowcolor{lightgray}Qwen-2.5-3B-Inst & 60.89  & 80.59  & 46.31  & 86.05 & 68.46 & 26.72 & 23.88 & \textbf{41.61} & 24.43 & 29.16\\ \hline
Qwen-2.5-7B-Inst & \textbf{78.26}  & \textbf{83.13}  & 56.24  & \textbf{93.24} & 77.71 & \textbf{58.73} & \textbf{47.93} & 40.49 & 52.20 & \textbf{49.82}\\ \hline
\rowcolor{lightgray}Mistral-7B-Inst & 61.25 & 70.93  & 96.20 & 83.85 & \textbf{78.05} & 24.55 & 1.6 & 28.18 & 15.39 &  17.43\\ \hline
Gemma2-2B-It & 35.34 & 42.58  & 91.50 & 70.04 & 59.86 & 22.36 & 2.84 & 12.75 & 34.78 & 18.18\\ \hline
\rowcolor{lightgray}Phi-3-Mini-Instruct  & 55.91 & 75.30 & 89.10 & 75.32 & 73.90 & 46.63 & 35.46 & 22.60 & 56.75 & 40.36 \\ \hline
\end{tabular}
\caption{The above table compares our approach with the Self Rewarding approach. The row labels correspond to the model name and the column labels correspond to the sub-categories. The metric used is accuracy where the higher values indicate better performance. }
\label{tab:reward_bench}
\end{table*}



% \begin{table*}[h!]
% \centering

% \begin{tabular}{|l|ccccc|ccccc|}
% \hline
% \multirow{}{}{Models} & \multicolumn{5}{c|}{Our Approach} & \multicolumn{5}{c|}{Self Rewarding} \\
% \cline{2-11}
%  & Chat & Code & Math & Safety & Random & Chat & Code & Math & Safety & Random \\ \hline\hline
% Llama-1B-Inst & 64.37 & 52.84 & 88.14 & 80.48 & 50 & 0 & 0 & 0 & 0 & 50 \\ \hline
% Llama-3B-Inst & 62.09 & 67.17 & \textbf{98.21} & 80.23 & 50 & 0 & 0 & 0 & 0 & 50 \\ \hline
% Llama-8B-Inst & 59.56 & 73.88 & 44.97 & 87.88 & 50 & 0 & 0 & 0 & 0 & 50 \\ \hline
% Qwen-3B-Inst & 60.89  & 80.59  & 46.31  & 86.05 & 50 & 0 & 0 & 0 & 0 & 50 \\ \hline
% Qwen-7B-Inst & \textbf{78.26}  & \textbf{83.13}  & 36.24  & \textbf{93.24}  & 50 & 0 & 0 & 0 & 0 & 50 \\ \hline
% Pythia-1.4B  & 19.17 & 4.47 & 51.90 & 23.61 & 50 & 0 & 0 & 0 & 0 & 50 \\ \hline
% Pythia-6.9B  & 23.67 & 34.55 & 55.26 & 35.60 & 50 & 0 & 0 & 0 & 0 & 50 \\ \hline
% Phi-3-Mini  & 55.91 & 75.30 & 99.10 & 75.32 & 50 & 0 & 0 & 0 & 0 & 50 \\ \hline
% Mistral-7B-Inst & 61.25 & 70.93  & 96.20 & 83.85 & 50 & 0 & 0 & 0 & 0 & 50 \\ \hline
% Gemma2-2B-It & 35.34 & 42.58  & 91.50 & 70.04 & 50 & 0 & 0 & 0 & 0 & 50 \\ \hline
% GPT4-o-Mini & 83.72 & 95.32  & 59.50 & 91.74 & 50 & 0 & 0 & 0 & 0 & 50 \\ \hline
% \end{tabular}
% \caption{The above table compares our approach with the Score Based Self Rewarding approach. The row labels correspond to the model name and the column labels correspond to the sub-categories. The metric used is accuracy where the higher values indicate better performance. }
% \label{tab:model_comparison}
% \end{table*}

% \begin{tabular}{|l||ccccc||ccccc|}
% \hline
% \multirow{}{}{Models} & \multicolumn{5}{c||}{Our Approach} & \multicolumn{5}{c|}{Self Rewarding} \\
% \cline{2-11}
%  & Chat & Code & Math & Safety & Random & Chat & Code & Math & Safety & Random \\ \hline\hline
% Llama-1B-Inst & 64.37 & 52.84 & 88.14 & 80.48 & X1 & 0 & 0 & 0 & 0 & X2 \\ \hline
% Llama-3B-Inst & 62.09 & 67.17 & \textbf{98.21} & 80.23 & X3 & 0 & 0 & 0 & 0 & X4 \\ \hline
% Llama-8B-Inst & 59.56 & 73.88 & 44.97 & 87.88 & X5 & 0 & 0 & 0 & 0 & X6 \\ \hline\hline
% Qwen-3B-Inst & 60.89  & 80.59  & 46.31  & 86.05 & X7 & 0 & 0 & 0 & 0 & X8 \\ \hline
% Qwen-7B-Inst & \textbf{78.26}  & \textbf{83.13}  & 36.24  & \textbf{93.24}  & X9 & 0 & 0 & 0 & 0 & X10 \\ \hline\hline
% Pythia-1.4B  & 0 & 0 & 0 & 0 & X11 & 0 & 0 & 0 & 0 & X12 \\ \hline
% Pythia-6.9B  & 0 & 0 & 0 & 0 & X13 & 0 & 0 & 0 & 0 & X14 \\ \hline\hline
% Phi-3-Mini(3.8B)  & 0 & 0 & 0 & 0 & X15 & 0 & 0 & 0 & 0 & X16 \\ \hline
% Phi-3-Small(7.4B)  & 0 & 0 & 0 & 0 & X17 & 0 & 0 & 0 & 0 & X18 \\ \hline \hline
% Mistral-7B-Inst & 61.25 & 70.93  & 96.20 & 83.85 & X19 & 0 & 0 & 0 & 0 & X20 \\ \hline
% \end{tabular}
% \caption{Reward Bench}
% \label{tab:model_comparison}
% \end{table*}


% \begin{table*}[h!]
% \centering
% \begin{tabular}{l|cccc|cccc}
% \hline
% \multirow{}{}{Models} & \multicolumn{4}{c|}{Our Approach} & \multicolumn{4}{c}{Self Rewarding} \\
% \cline{2-9}
%  & Chat & Code & Math & Safety & Chat & Code & Math & Safety \\
% \hline
% Llama-1B-Inst & 57.62 & 51.46 & 47.57 & 70.24 & 0 & 0 & 0 & 0 \\ \hline
% Llama-3b-Inst & 51.42 & 51.90 & 48.52 & 76.89 & 0 & 0 & 0 & 0 \\ \hline
% Qwen-3B-Inst & 60.98 & 53.07 & 56.33 & 68.96 & 0 & 0 & 0 & 0 \\ \hline
% Qwen-7B-Inst & \textbf{68.73} & \textbf{57.30} & \textbf{58.15} & \textbf{89.49} & 0 & 0 & 0 & 0 \\ \hline
% Mistral-7b-Inst & 54.00 & 52.48 & 52.04 & 71.52 & 0 & 0 & 0 & 0 \\ \hline
% Llama-8B-Inst & 50.64 & 54.97 & 48.70 & 68.25 & 0 & 0 & 0 & 0 \\ \hline

% \end{tabular}
% \caption{RM Bench}
% \label{tab:model_comparison}
% \end{table*}
\begin{table*}[h!]
\centering
\renewcommand{\arraystretch}{0.9}
\setlength{\tabcolsep}{3pt}
\definecolor{apricot}{rgb}{0.95, 0.82, 0.62}
\definecolor{lightgray}{rgb}{0.96, 0.96, 0.96}

\begin{tabular}{l|l|c|c|c|c|c}
\hline
\rowcolor{apricot}
\textbf{Model} & \textbf{Levels} & 
\shortstack{\textbf{RM-Bench} \\ \textbf{chat}} & 
\shortstack{\textbf{RM-Bench} \\ \textbf{code}} & 
\shortstack{\textbf{RM-Bench} \\ \textbf{math}} & 
\shortstack{\textbf{RM-Bench} \\ \textbf{safety response}} & 
\shortstack{\textbf{RM-Bench} \\ \textbf{safety refuse}} \\
\hline
\hline
\multirow{4}{*}{Llama-1B} &  level 1 & 48.06 & 54.39 & 46.31 & 31.85 & 38.73 \\
& level 2 & 64.34 & 55.26 & 48.58 & 69.43 & 53.52 \\
& level 3 & 60.47 & 50.44 & 41.59 & 61.78 & 71.13 \\
& mean & 57.62 & 53.36 & 45.49 & 54.35 & 54.46 \\
\hline
\multirow{4}{*}{Llama-1B-Instruct} & level 1 & 51.16 & 51.32 & 49.53 & 71.34 & 67.61 \\
& level 2 & 61.24 & 53.51 & 47.45 & 68.15 & 77.11 \\
&  level 3 & 60.47 & 49.56 & 45.75 & 73.89 & 63.38 \\
& mean & 57.62 & 51.46 & 47.57 & 71.13 & 69.37 \\
\hline

\multirow{4}{*}{Llama-3B} & 
level 1 & 54.26 & 51.75 & 47.26 & 68.15 & 7.04 \\
& level 2 & 33.33 & 52.19 & 46.12 & 78.34 & 37.32 \\
& level 3 & 33.33 & 49.12 & 45.75 & 36.94 & 55.28 \\
& mean & 40.31 & 51.02 & 46.38 & 61.15 & 33.22 \\
\hline
 \multirow{4}{*}{Llama3b-Instruct} &  level 1 & 56.59 & 50.88 & 50.09 & 87.90 & 55.28 \\
& level 2 & 44.96 & 55.26 & 48.02 & 86.62 & 60.56 \\
&  level 3 & 52.71 & 49.56 & 47.45 & 94.27 & 76.76 \\
& mean & 51.42 & 51.90 & 48.52 & 89.60 & 64.20 \\
\hline


 \multirow{4}{*}{Llama-8B} &  level 1 & 54.26 & 53.51 & 48.02 & 99.36 & 2.46 \\
& level 2 & 56.59 & 56.58 & 51.98 & 83.44 & 29.58 \\
&  level 3 & 50.39 & 51.75 & 47.26 & 64.33 & 63.38 \\
& mean & 53.75 & 53.95 & 49.09 & 82.38 & 31.81 \\
\hline
 \multirow{4}{*}{Llama-8B-Instruct} &  level 1 & 65.12 & 55.70 & 50.28 & 56.05 & 75.00 \\
& level 2 & 36.43 & 55.70 & 49.72 & 96.18 & 30.28 \\
&  level 3 & 50.39 & 53.51 & 46.12 & 64.33 & 87.68 \\
& mean & 50.65 & 54.97 & 48.71 & 72.19 & 64.32 \\
\hline


 \multirow{4}{*}{Mistral-7b} &  level 1 & 50.39 & 46.49 & 52.17 & 96.18 & 20.42 \\
& level 2 & 61.24 & 53.51 & 49.34 & 44.59 & 89.44 \\
&  level 3 & 51.94 & 46.49 & 43.10 & 75.16 & 84.15 \\
& mean & 54.52 & 48.83 & 48.20 & 71.97 & 64.67 \\
\hline
 \multirow{4}{*}{Mistral-7b-Instruct} &  level 1 & 44.19 & 50.88 & 52.55 & 61.78 & 96.48 \\
& level 2 & 58.91 & 52.63 & 55.39 & 39.49 & 81.69 \\
&  level 3 & 58.91 & 53.95 & 48.20 & 52.87 & 96.83 \\
& mean & 54.01 & 52.49 & 52.05 & 51.38 & 91.67 \\
\hline


 \multirow{4}{*}{Qwen2.5-3B} &  level 1 & 65.89 & 48.68 & 54.06 & 95.54 & 94.01 \\
& level 2 & 58.14 & 52.19 & 51.23 & 82.80 & 88.03 \\
&  level 3 & 48.84 & 50.44 & 46.12 & 94.90 & 49.65 \\
& mean & 57.62 & 50.44 & 50.47 & 91.08 & 77.23 \\
\hline

 \multirow{4}{*}{Qwen2.5-3B-Instruct} &  level 1 & 72.87 & 51.32 & 60.87 & 46.50 & 63.38 \\
& level 2 & 55.04 & 53.07 & 57.66 & 31.85 & 90.49 \\
&  level 3 & 55.04 & 54.82 & 50.47 & 84.71 & 96.83 \\
& mean & 60.98 & 53.07 & 56.33 & 54.35 & 83.57 \\
\hline


 \multirow{4}{*}{Qwen2.5-7B} &  level 1 & 72.87 & 56.58 & 56.14 & 100.00 & 100.00 \\
& level 2 & 47.29 & 56.58 & 54.06 & 96.82 & 94.72 \\
&  level 3 & 51.16 & 53.07 & 47.64 & 94.27 & 100.00 \\
& mean & 57.11 & 55.41 & 52.61 & 97.03 & 98.24 \\
\hline
 \multirow{4}{*}{Qwen2.5-7B-Inst} &  level 1 & 80.62 & 58.33 & 62.19 & 91.08 & 100.00 \\
& level 2 & 61.24 & 58.33 & 62.00 & 85.99 & 96.83 \\
&  level 3 & 64.34 & 55.26 & 50.28 & 63.06 & 100.00 \\
& mean & 68.73 & 57.31 & 58.16 & 80.04 & 98.94 \\
\hline
 \multirow{4}{*}{SKYWORK-8b-reward} &  level 1 & 86.04 & 53.07 & 62.38 & 94.90 & 97.18 \\
& level 2 & 55.04 & 53.51 & 65.41 & 82.80 & 98.94 \\
&  level 3 & 41.09 & 48.25 & 66.16 & 87.26 & 100.00 \\
& mean & 60.72 & 51.61 & 64.65 & 88.32 & 98.60 \\
\hline


\end{tabular}
\caption{Performance of various models, across different levels on RM-Bench}
\label{tab:rm_bench_levels}
\end{table*}
\end{document}
