% This must be in the first 5 lines to tell arXiv to use pdfLaTeX, which is strongly recommended.
\pdfoutput=1
% In particular, the hyperref package requires pdfLaTeX in order to break URLs across lines.

\documentclass[11pt]{article}

% Change "review" to "final" to generate the final (sometimes called camera-ready) version.
% Change to "preprint" to generate a non-anonymous version with page numbers.
\usepackage[preprint]{acl}
\usepackage{booktabs}
\usepackage{algorithm}
\usepackage{tikz}
\usepackage{pgfplots}
\usepackage{algpseudocode}
% Standard package includes
\usepackage{times}
\usepackage{latexsym}
\usepackage{multirow}
\usepackage{amsmath}
% For proper rendering and hyphenation of words containing Latin characters (including in bib files)
\usepackage[T1]{fontenc}
% For Vietnamese characters
% \usepackage[T5]{fontenc}
% See https://www.latex-project.org/help/documentation/encguide.pdf for other character sets

% This assumes your files are encoded as UTF8
\usepackage[utf8]{inputenc}

% This is not strictly necessary, and may be commented out,
% but it will improve the layout of the manuscript,
% and will typically save some space.
\usepackage{microtype}

% This is also not strictly necessary, and may be commented out.
% However, it will improve the aesthetics of text in
% the typewriter font.
\usepackage{inconsolata}

%Including images in your LaTeX document requires adding
%additional package(s)
\usepackage{graphicx}
\usepackage{breakurl}
\usepackage{xcolor}
\usepackage{colortbl}
\usepackage{multirow}
\usepackage{graphicx}
\usepackage{tabularx}
\usepackage{amssymb}
\usepackage{hyperref}

% If the title and author information does not fit in the area allocated, uncomment the following
%
%\setlength\titlebox{<dim>}
%
% and set <dim> to something 5cm or larger.

\makeatletter
\title{%
  \vspace{-2cm}
  \begin{center}
    \includegraphics[height=2cm]{assets/lossfunk.jpeg}
  \end{center}
  \vspace{0.5cm}
  IPO: Your Language Model is Secretly a Preference Classifier%
}
\makeatother

% Author information can be set in various styles:
% For several authors from the same institution:
% \author{Author 1 \and ... \and Author n \\
%         Address line \\ ... \\ Address line}
% if the names do not fit well on one line use
%         Author 1 \\ {\bf Author 2} \\ ... \\ {\bf Author n} \\
% For authors from different institutions:
% \author{Author 1 \\ Address line \\  ... \\ Address line
%         \And  ... \And
%         Author n \\ Address line \\ ... \\ Address line}
% To start a separate ``row'' of authors use \AND, as in
% \author{Author 1 \\ Address line \\  ... \\ Address line
%         \AND
%         Author 2 \\ Address line \\ ... \\ Address line \And
%         Author 3 \\ Address line \\ ... \\ Address line}

\author{
  \textbf{Shivank Garg\textsuperscript{1,2}\textsuperscript{*}},
  \textbf{Ayush Singh\textsuperscript{1,2}\textsuperscript{*}},
  \textbf{Shweta Singh\textsuperscript{1,2}},
  \textbf{Paras Chopra\textsuperscript{2}}
\\
  \textsuperscript{1}Indian Institute of Technology Roorkee,
  \textsuperscript{2}Lossfunk
\\
  \texttt{
    \texttt{\{shivank\_g@mfs, ayush\_s@mt, shweta\_s@mfs\}.iitr.ac.in}},
    % \texttt{ayush\_s@mt.iitr.ac.in},
    % \texttt{shweta\_s@mfs.iitr.ac.in},
    \texttt{paras@lossfunk.com}
    \\
    % \small{* Equal contribution}
  }
  



% \author{First Author \\
%   Affiliation / Address line 1 \\
%   Affiliation / Address line 2 \\
%   Affiliation / Address line 3 \\
%   \texttt{email@domain} \\\And
%   Second Author \\
%   Affiliation / Address line 1 \\
%   Affiliation / Address line 2 \\
%   Affiliation / Address line 3 \\
%   \texttt{email@domain} \\}

%\author{
%  \textbf{First Author\textsuperscript{1}},
%  \textbf{Second Author\textsuperscript{1,2}},
%  \textbf{Third T. Author\textsuperscript{1}},
%  \textbf{Fourth Author\textsuperscript{1}},
%\\
%  \textbf{Fifth Author\textsuperscript{1,2}},
%  \textbf{Sixth Author\textsuperscript{1}},
%  \textbf{Seventh Author\textsuperscript{1}},
%  \textbf{Eighth Author \textsuperscript{1,2,3,4}},
%\\
%  \textbf{Ninth Author\textsuperscript{1}},
%  \textbf{Tenth Author\textsuperscript{1}},
%  \textbf{Eleventh E. Author\textsuperscript{1,2,3,4,5}},
%  \textbf{Twelfth Author\textsuperscript{1}},
%\\
%  \textbf{Thirteenth Author\textsuperscript{3}},
%  \textbf{Fourteenth F. Author\textsuperscript{2,4}},
%  \textbf{Fifteenth Author\textsuperscript{1}},
%  \textbf{Sixteenth Author\textsuperscript{1}},
%\\
%  \textbf{Seventeenth S. Author\textsuperscript{4,5}},
%  \textbf{Eighteenth Author\textsuperscript{3,4}},
%  \textbf{Nineteenth N. Author\textsuperscript{2,5}},
%  \textbf{Twentieth Author\textsuperscript{1}}
%\\
%\\
%  \textsuperscript{1}Affiliation 1,
%  \textsuperscript{2}Affiliation 2,
%  \textsuperscript{3}Affiliation 3,
%  \textsuperscript{4}Affiliation 4,
%  \textsuperscript{5}Affiliation 5
%\\
%  \small{
%    \textbf{Correspondence:} \href{mailto:email@domain}{email@domain}
%  }
%}
% Place this before \begin{document}
% Place this before \begin{document}

\begin{document}

\maketitle
\begin{abstract}

\footnotetext{*Equal contribution.}

Reinforcement learning from human feedback (RLHF) has emerged as the primary method for aligning large language models (LLMs) with human preferences.
While it enables LLMs to achieve human-level alignment, it often incurs significant computational and financial costs due to its reliance on training external reward models or human-labeled preferences. In this work, we propose \textbf{Implicit Preference Optimization (IPO)}, an alternative approach that leverages generative LLMs as preference classifiers, thereby reducing the dependence on external human feedback or reward models to obtain preferences. We conduct a comprehensive evaluation on the preference classification ability of LLMs using RewardBench, assessing models across different sizes, architectures, and training levels to validate our hypothesis. Furthermore, we investigate the self-improvement capabilities of LLMs by generating multiple responses for a given instruction and employing the model itself as a preference classifier for Direct Preference Optimization (DPO)-based training. Our findings demonstrate that models trained through IPO achieve performance comparable to those utilizing state-of-the-art reward models for obtaining preferences. Our code is available at \textcolor{magenta}{\burl{https://github.com/shivank21/Implicit\_Preference\_Optimization}}.
\end{abstract}


\documentclass[../main.tex]{subfiles}
\graphicspath{{../images/}}
\makeatletter
\def\input@path{{../images/}}
\makeatother
\begin{document}
\section{Introduction}
\begin{figure}
\centering
\begin{tikzpicture}
\node[inner sep=0pt] (ws) at (0, 0) {
\includegraphics[height=.4\textwidth, trim={10cm 0 10cm 0},clip]{world_space.png}};
\node[inner sep=0pt] (cs) at (6,0) {\includegraphics[height=.4\textwidth, trim={10cm 1cm 10cm 4cm},clip]{conf_space.png}};
\end{tikzpicture}
\vspace{-5pt}
\label{fig:pbrm_intro}
\caption{\textbf{Left}: Shows world space obstacles as grey spheres. Robots start and goal configuration is colored red and green, respectively. Configurations along the computed path are colored transparent blue. \textbf{Right:} Mapped world space scenario to configuration space. Obstacle region is the grey mesh. Red spheres are collision-free regions computed by the neural SCDF. The optimized shortest path in the convex corridor is the blue curve.}
\vspace{-25pt}
\end{figure}
Motion planning is the problem of finding a collision-free trajectory that connects a given start and goal configuration. The planning takes place in the configuration space of the robot. For single body robots, like mobile robots or drones, the configuration space and the world space are usually the same. This simplifies the planning, since explicit obstacle representations are available which enables geometrical tools like separating hyperplanes, smallest distance to obstacles etc., to be used when designing motion planning algorithms. For multi-body robots like manipulators, the situation is completely different. The world space obstacles are usually mapped to non-convex regions, and to make the problem even harder, the mapping is usually not known. Forming explicit representations of the obstacle region in the configuration space is usually too expensive or intractable. Despite all of this, sampling based planners are used with great success, which mainly is due to their use of implicit representations of the obstacle region. The basic idea is to construct a graph in the configuration space that covers and connects the collision-free region. From this graph, a path can be extracted that connects a given start and goal configuration. The approach is computationally expensive, since the graph is constructed with the smallest geometrical building block available, points, which represents a collision-check. Furthermore, the extracted paths from the graph are non-smooth and jagged due to the stochastic nature of the approach. This adds an additional post-processing step to the process, where the paths are shortcutted and smoothened, before the path can be used for tracking. Clearly a lot of time is invested to form this graph and produce smooth paths. Thus, if the obstacles start to move, then all of this work is done in no use, since all points that make up this graph need to be re-verified, which is simply too time consuming to be done in real time.
\\\\
In this work, we want to address the existing drawbacks of the sampling based planners. Our main contribution is an improved motion planner where each vertex in the graph covers a collision-free region in the form of a sphere instead of a point and where the edges are formed with neighboring intersecting spheres. This representation has the advantage of instead of returning piecewise linear paths, returning a sequence of overlapping spheres, i.e. a convex corridor, that connects a given start and goal configuration, illustrated in Figure \ref{fig:pbrm_intro}. This convex corridor allows us to use convex optimization to produce smooth trajectories, instead of computationally expensive post-processing methods. The representation further allows us to estimate the coverage of the collision-free space, which gives us awareness and feedback in the offline roadmap construction phase. Finally, our representation is simple to adapt to moving obstacles, simply requery for the new radii and recheck for intersections. 
\\\\
The spherical collision-free regions are formed using a signed distance function (SDF), which is a function that returns the smallest distance from an arbitrary point to the boundary of an obstacle. As the name implies, the distance is signed, thus if the point is inside the obstacle it is negative otherwise positive. If the distance is positive, a sphere with radius equal to the distance is guaranteed to cover a collision-free region. Using an SDF in motion planning is not new, but what is novel about our approach is that we express the distance in the configuration space instead of the world space and by doing so allows us to form these convex collision-free regions. We refer to the resulting SDF as a signed configuration distance function (SCDF). Computing an SCDF analytically is non-trivial, our approach is therefore to parameterize the SCDF with a deep neural network and learn the mapping by supervised learning. Our resulting neural SCDF can compute distances for different parameter values of obstacle shapes and we also show how multiple distances can be combined, thus making our approach flexible.
\section{Related work}
Motion planning algorithms can roughly be divided into three families, grid-based, sampling based and optimization based methods. Grid-based methods (GBM) discretize the planning space from which a graph is then compiled. A standard search method is A$^\star$ \citep{a_star}, which is classified as an \textit{informed} search method, since it employs a heuristic function to speed up the search. A$^\star$ guarantees to return an optimal path at the level of discretization used. GBMs usually discretize the planning space by a regular lattice and this limits the GBMs to problems with low dimensionality due to the curse of dimensionality. Thus, GBMs are usually limited to single-body robots where the degrees of freedom (DOF) are low. To overcome the inherent scaling problem with the GBMs, stochastic methods are usually used for multi-body robots. These methods are termed as sampling-based methods (SBM) and core members within this family are the rapidly-exploring random trees (RRT) \citep{rrt} and the probabilistic roadmap (PRM) \citep{prm}. RRT grows a tree from the start configuration and explores the collision-free region in a rapid way until it is able to connect to the goal region. RRT is usually improved by bi-directional planning \citep{rrt_connect}, i.e. an additional tree is grown from the goal configuration and the trees are tested for connection after any tree has been expanded. RRT is a single-query method, thus it searches for a path from scratch each time it is queried. Contrary to this, PRM is a multi-query method, which solves for multiple queries without starting from scratch. PRM does this by creating a roadmap (graph) that covers the collision-free space as an offline step. The graph is then used to solve for multiple queries. PRMs are used in cases where the environment does not change since the extra offline step is too computationally costly and needs to be re-done if the environment is changed. In our work, we address this inherent issue by using a different roadmap representation. Our vertices in the graph cover a collision-free region in the form of spheres and we form the edges by checking for intersecting spheres. If something in the environment changes, we recompute the spheres radii and recheck the intersections, without relying on collision detection. We use a trained neural network to compute the sphere radius, therefore querying for the radius can be done fast, hence our representation enables the PRM for dynamic environments.
\\\\
In the recent decades, optimization based methods (OBM) \citep{chomp, schulman, itomp, stomp} have been introduced as an alternative to SBM for multi-body robots. Like the SBM, the OBMs scale well to higher dimensional problems and produce smoother motion. It is common to use a SDF in the optimization since it is a smooth function, thus enabling gradient-based methods. However, the standard way of expressing the SDF is in world space. The distance therefore needs to be mapped to the configuration space by the forward kinematics. This mapping makes the optimization problem a non-linear program (NLP), which is computationally expensive to solve. Recently, a different approach has been proposed. In \cite{mp_gcs} motion planning is formulated as a convex optimization problem by using the graph of convex sets framework \citep{gcs}. The underlying idea is to decompose the collision-free space into intersecting convex sets from which a convex optimization problem is formulated. In cases where an explicit representation of the obstacles in the configuration space exists, like for single-body robots, creating collision-free convex regions can be done fast \citep{iris}. For multi-body robots, this is non-trivial. Existing work does this successfully \citep{iris_nlp, iris_c} by an optimization based approach, but the methods are still too time consuming to be used in the presence of moving obstacles. Our approach is instead to use deep learning to learn an SDF expressed in the configuration space. With this, we can query for shortest distances to the collision boundary, which allows us to expand spherical regions which are collision-free. Our approach is fast and therefore enables our suggested roadmap planner to be used in dynamic environments.
\\\\
Recent research has focused on learning collision detection \citep{fk_kernel_distance, diffco, graphdistnet} by predicting the signed distance between the robot links and the surrounding obstacles in the world space. The learned SDF is used in trajectory optimization but since the distance is expressed in the world space, the problem becomes an NLP and therefore takes a long time to solve. We take a novel approach and suggest to instead express the signed distance in the configuration space. This allows us to improve the PRM at the same time as it enables convex optimization for trajectory optimization, which runs faster and is more reliable than NLP solvers. In \cite{cspf} a learned signed distance function in the configuration space is proposed similar to our approach. However, their approach is restricted to point cloud representations, while we propose to represent the obstacles as parameterized geometric shapes, e.g. spheres. Furthermore, we also show how to use our learned SCDF to improve an existing roadmap planner.
\section{Problem formulation}
A robot is located in the world space, $\W \subset \R^3 $. The unique location of the robot is given by its configuration $\q \in \C$, where $\C$ is the configuration space. The set of points covered by the robots bodies at a certain configuration is expressed as $\B(\q) \subset \W$. The robot is surrounded by $\NrObst$ obstacles $\O = \bigcup_{i=1}^{\NrObst} \O_i$, where  $\O_i \subset \W$. The representation of the obstacle in the configuration space is the set $\C\O_i = \{\q \in \C \: |\: \B(\q) \cap \O_i \neq \emptyset \}$. The obstacle space is formed as $\Co = \bigcup_{i=1}^{\NrObst} \C \O_i$. The complement is referred to as the free space, $\Cf = \C \setminus \Co$. The path planning problem is a tuple, ($\Cf$, $\qStart$, $\qGoal$), where we want to connect a query pair, consisting of a start, $\qStart$, and goal configuration, $\qGoal$, with a geometric path, $\q(s): [0, 1] \mapsto \Cf$, such that $\q(0)=\qStart$ and $\q(1)=\qGoal$, or report correctly when such a path does not exist.
\end{document}

\section{Basic Background: Supervised Learning and the PAC Model}
\label{sec:background}

At this point almost everyone has heard of machine learning (ML). Anyone likely to stumble upon this article will have also heard of its most influential special case, supervised learning, and those theoretically inclined will also be familiar with the PAC model. Nonetheless, I will set the stage by  recapping the basics.

\subsection{Basics of Supervised Learning}%Let's set the stage in any case

\emph{Supervised Learning} is the task of ``coming up'' with a function $f: \X \to \Y$ to ``explain'' or ``fit'' a sequence of input/output examples   $(x_1,y_1), \ldots, (x_n,y_n)$, with $x_i \in \X$ and $y_i \in \Y$.  Here $\X$ is a \emph{data domain} consisting of \emph{datapoints} $x \in \X$, $\Y$ is a \emph{label set} consisting of \emph{labels} $y \in \Y$, and the sequence $(x_1,y_1),\ldots,(x_n,y_n)$ is the \emph{training data} consisting of \emph{labeled examples (a.k.a. samples)}~$(x_i,y_i)$.  I~will refer to the chosen function $f$ as a \emph{predictor}, and to $n$ as the \emph{sample size}. A \emph{learning algorithm} takes as input training data, and outputs (some representation of) a predictor $f \in \Y^\X$.\footnote{Note that this describes the usual \emph{batch}, a.k.a.~\emph{offline}, setting of supervised learning. I do not discuss other paradigms such as online or active learning in this article.} 



Success in supervised learning is defined as \emph{generalization} to  future examples: For a typical \emph{test example}  $(x_{\tst},y_{\tst})$, the predicted label $y'_{\tst}=f(x_{\tst})$ should ``equal'' $y_{\tst}$, perhaps approximately. We usually assume the test example is drawn from the same  ``source'' as the training data  --- commonly, i.i.d.~from the same distribution. The quality of the prediction is quantified by $\ell(y'_{\tst},y_{\tst})$, where $\ell:~\Y~\times~\Y \to \RR_{\geq 0}$ is a \emph{loss function} chosen as part of the problem definition. Common loss functions include the 0-1 loss $\ell_{0-1}(y',y) = [y' \neq y]$ for \emph{classification} problems,\footnote{The notation $[P]$ denotes $1$ when predicate $P$ is true, and denotes $0$ when $P$ is false.} as well as the absolute loss $|y'-y|$ or squared loss $(y'-y)^2$ for \emph{regression problems} featuring $\Y  \sse \RR$.

Nontrivial generalization properties are typically only possible if one assumes something about the data.\footnote{The need for such an assumption is formalized by the  \emph{no free lunch theorems} of supervised learning \cite{wolpert_connection_1992,wolpert_lack_1996,schaffer_conservation_1994}.} The Bayesian approach to  machine learning, common in many applications, assumes some parametric form for the distribution generating the data, and postulates a prior on the parameters. This is not the approach I will take in this article. Instead, I will focus on the frequentist --- and some would say ``worst-case'' or ``adversarial'' ---  approach that is common in the computational learning theory community, embodied by the PAC model. Here we assume that the (training and test) data can be explained, perhaps approximately, by a function in some ``simple enough to learn'' class of functions $\H \sse \Y^\X$, often called the \emph{hypotheses}. Equivalently, we  seek a predictor which explains the unseen data roughly  as well as the best hypothesis $h^* \in \H$, whether or not we assume that $h^*$ itself provides a perfect explanation.



 \paragraph{Common Algorithmic Templates.} Perhaps the best known general-purpose supervised learning algorithm is \emph{empirical risk minimization (ERM)}, which chooses as its predictor a hypothesis $f \in \H$ minimizing $\frac{1}{n} \sum_{i=1}^n \ell(f(x_i),y_i)$ --- a quantity called the \emph{training error}, \emph{empirical error}, or \emph{empirical risk} of $f$. %\footnote{When multiple hypotheses minimize the empirical risk, we assume ERM breaks ties arbitrarily.}
A common template for generalizing ERM involves adding a \emph{regularization term} $\psi(f)$ to the  objective function, typically chosen to measure some notion of ``hypothesis complexity.'' An algorithm instantiating this template is known as a \emph{structural risk minimizer (SRM)}, and chooses as its predictor the hypothesis $f \in \H$ minimizing the \emph{structural risk} $\frac{1}{n} \sum_{i=1}^n \ell(f(x_i),y_i) + \psi(f)$. Other well-known algorithms, such as gradient descent and its variations,  can frequently be interpreted as approximate implementations of ERM or SRM.


\paragraph{Proper vs Improper Learning.} A learning algorithm is said to be \emph{proper} if its predictor $f$ is always chosen from the hypothesis class, i.e., $f \in \H$, otherwise it is said to be \emph{improper}. ERM  is an example of a proper learning algorithm, as are SRM algorithms of the form described above.  In the \emph{proper regime} of learning, algorithms are required to be proper. This article will be concerned with the more flexible \emph{improper regime} (a.k.a \emph{representation-independent learning}), where no such constraint is placed on the learner. In other words, all we care about is predictive power at test time, rather than any insights derived from the functional form or representation of the predictor~itself.


\subsection{The PAC Model}
A standard mathematical setup for evaluation of supervised learning algorithms, at least in the theoretical computer science community, is Valiant's \emph{Probably Approximately Correct (PAC) model} of learning (see e.g.~\cite{kearns_introduction_1994,mohri_foundations_2018}). Here, we assume there is an unknown distribution $\D$ on $\X \times \Y$ from which training and test data are  drawn.  Specifically, the labeled datapoints of the training set  $(x_1,y_1), \ldots, (x_n,y_n)$, as well as the test data  $(x_\tst,y_\tst)$, are i.i.d.~from $\D$. Often it is assumed that $\D$ lies in some class of distributions of interest. The \emph{true expected loss}, or simply \emph{loss}, of a predictor $f: \X \to \Y$ is the expected loss it incurs on draws from $\D$, written $L_\D(f) = \Ex_{(x,y) \sim \D} \ell(f(x),y)$.


There are two main ``settings'' in PAC learning. The  \emph{realizable setting} only requires that the data be perfectly explained by some hypothesis in $\H$. More generally, the \emph{agnostic setting} makes no assumption relating the data to the hypotheses, but shifts the goalposts as necessary to allow nontrivial guarantees: the expected loss at test time is evaluated only ``relative'' to that of the best hypothesis $h^* \in \H$. There are other settings which make more nuanced assumptions, such as $\D$ being of a particular parametric form or its support living in some (unknown) lower-dimensional space, etc. I will mostly discuss the realizable and agnostic settings in this article, those being the simplest and most studied from a theoretical perspective. %TODO:We will briefly discuss other settings in Section ??

The PAC model demands high probability guarantees of learners, in the worst case over distributions of interest. Consider first the realizable setting, where $\D$ is such that $\min_{h \in \H} L_{\D}(h) = 0$. A PAC learner has \emph{error} $\epsilon=\epsilon(n)$ and \emph{confidence} $\delta=\delta(n)$ if, when training data consists of $n$ i.i.d~samples from a realizable distribution $\D$, it produces a predictor $f$  satisfying $L_\D(f) \leq \epsilon$ with probability at least $1-\delta$. In the agnostic setting, where $\D$ can be arbitrary, we require $L_\D(f) - \min_{h \in \H} L_\D(h) \leq \epsilon$ with probability $1-\delta$.

In both the realizable and agnostic settings, we look for PAC learners with small $\epsilon$ and $\delta$ as a function of the sample size $n$. An equivalent perspective looks at the sample complexity $m(\epsilon,\delta)$, which is the minimum sample size which guarantees error  at most $\epsilon$ with probability at least $1-\delta$. We say a problem is \emph{PAC learnable} if its PAC sample complexity is finite whenever $\epsilon,\delta > 0$.

For most PAC learning problems, learnability and sample complexity are characterized in terms of a  ``dimension'' of the hypothesis class. Most prominently this is the \emph{VC dimension} for binary classification, the \emph{fat shattering dimension} for agnostic regression, and the \emph{DS dimension} for multiclass classification (see \cite{anthony_neural_1999,daniely_optimal_2014,brukhim_characterization_2022}). Treatment of these is beyond the scope of this article. The unfamiliar reader need not worry, however,  as dimensions will feature only tangentially in our~discussion.




%\paragraph{Learning settings: Realizable, Agnostic, etc.} In learning theory, evaluating a supervised learning algorithm requires specifying a data model and an objective. We will leave the details of the data model flexible for now, to allow for both the PAC model and the adversarial transductive model. Nonetheless we will describe two variations, which we call ``settings'', which cut across different models. The  \emph{realizable setting}  requires only that the data be perfectly explained by some hypothesis $h \in \H$ --- i.e., there exists a hypothesis which is guaranteed to suffer a loss of $0$ on training and test data. The performance of the learning algorithm is its expected loss at test time for some ``worst case'' realizable instance. More generally, the \emph{agnostic setting} makes no assumption relating the data to the hypotheses, but shifts the goalposts as necessary to allow nontrivial guarantees: the expected loss at test time is evaluated only ``relative'' to that of the best hypothesis $h^* \in \H$, again for some ``worst case'' instance. There are other settings which make more nuanced assumptions about the data, such as it is drawn from a distribution of a particular parametric form, or that it lives in some (unknown) lower-dimensional space, etc. We will mostly discuss the realizable and agnostic settings, those being the simplest and most studied from a theoretical perspective.




%%% Local Variables:
%%% mode: latex
%%% TeX-master: "learning_matching"
%%% End:

\section{LLM as Preference Model}
\label{LLM_as_Preference}
\subsection{Background}

Large Language Models (LLMs) generate text in an autoregressive manner, producing tokens sequentially based on the context of previously generated tokens. Given an input context \( \mathbf{x} \) , the autoregressive model predicts an output sequence \( \mathbf{y} = (y_1, y_2, \dots, y_T) \) one token at a time. Assuming the model is parameterized by \( \theta \), the conditional probability of generating the sequence \( \mathbf{y} \) is defined as:

\begin{equation}
    p_\theta(\mathbf{y} \mid \mathbf{x}) = \prod_{t=1}^T p_\theta(y_t \mid \mathbf{x}, y_{<t}),
\end{equation}

where \( y_{<t} = (y_1, y_2, \dots, y_{t-1}) \). For notational simplicity, \( p_\theta(y_t \mid \mathbf{x}) \) is used to represent \( p_\theta(y_t \mid \mathbf{x}, y_{<t}) \).

The probability distribution over the vocabulary at each time step \( t \) is computed using a softmax function on the logits \( z \) as:

\begin{equation}
    p_\theta(y_t \mid \mathbf{x}) = \frac{\exp(z_t / \tau)}{\sum_{i=1}^M \exp(z_i / \tau)},
\end{equation}

where \( z_t = \text{logit}_\theta(y_t \mid \mathbf{x}, y_{<t}) \), \( M \) is the vocabulary size, and \( \tau > 0 \) is a temperature parameter. 

Various decoding strategies govern token selection during text generation. Greedy decoding selects the highest probability token at each step, while beam search expands multiple candidate sequences in parallel to find the most likely one. Top-k sampling \citep{fan2018hierarchical}, on the other hand, limits token choices to the k most probable candidates, introducing diversity. Many other decoding strategies also exist, each balancing fluency and variability differently.


%%%%%%%%%%%Table%%%%%%%%%%%
\begin{table*}[htbp!]
\centering
\setlength{\tabcolsep}{4pt} 
\definecolor{apricot}{rgb}{0.95, 0.82, 0.62}
\definecolor{lightgray}{rgb}{0.96, 0.96, 0.96}

\begin{tabular}{l | ccccc | ccccc }
\hline
\rowcolor{apricot} 
\multirow{0}{*}{\textbf{Models}} & \multicolumn{5}{c|}{\textbf{Our Approach}} & \multicolumn{5}{c}{\textbf{Self Rewarding}} \\
\cmidrule(lr){2-6} \cmidrule(lr){7-11}
% \cline{2-11}
 & Chat & Code & Math & Safety & Average & Chat & Code & Math & Safety & Average  \\ \hline
Llama-3.2-1B-Inst & 64.37 & 52.84 & 88.14 & 80.48 & 71.45 & 30.47 & 21.03 & 14.54 & 31.55 & 24.39\\ \hline
\rowcolor{lightgray}Llama-3.2-3B-Inst & 62.09 & 67.17 & \textbf{98.21} & 80.23 & 76.92 & 33.87 & 24.69 & 36.01 & 46.73 & 35.32\\ \hline
Llama-3-8B-Inst & 59.56 & 73.88 & 54.97 & 87.88 & 69.07 & 35.43 & 12.29 & 21.70 & \textbf{58.35} & 31.94\\ \hline
\rowcolor{lightgray}Qwen-2.5-3B-Inst & 60.89  & 80.59  & 46.31  & 86.05 & 68.46 & 26.72 & 23.88 & \textbf{41.61} & 24.43 & 29.16\\ \hline
Qwen-2.5-7B-Inst & \textbf{78.26}  & \textbf{83.13}  & 56.24  & \textbf{93.24} & 77.71 & \textbf{58.73} & \textbf{47.93} & 40.49 & 52.20 & \textbf{49.82}\\ \hline
\rowcolor{lightgray}Mistral-7B-Inst & 61.25 & 70.93  & 96.20 & 83.85 & \textbf{78.05} & 24.55 & 1.6 & 28.18 & 15.39 &  17.43\\ \hline
Gemma2-2B-It & 35.34 & 42.58  & 91.50 & 70.04 & 59.86 & 22.36 & 2.84 & 12.75 & 34.78 & 18.18\\ \hline
\rowcolor{lightgray}Phi-3-Mini-Instruct  & 55.91 & 75.30 & 89.10 & 75.32 & 73.90 & 46.63 & 35.46 & 22.60 & 56.75 & 40.36 \\ \hline
\end{tabular}
\caption{The above table compares our approach with the Self Rewarding approach. The row labels correspond to the model name and the column labels correspond to the sub-categories. The metric used is accuracy where the higher values indicate better performance. }
\label{tab:reward_bench}
\end{table*}



% \begin{table*}[h!]
% \centering

% \begin{tabular}{|l|ccccc|ccccc|}
% \hline
% \multirow{}{}{Models} & \multicolumn{5}{c|}{Our Approach} & \multicolumn{5}{c|}{Self Rewarding} \\
% \cline{2-11}
%  & Chat & Code & Math & Safety & Random & Chat & Code & Math & Safety & Random \\ \hline\hline
% Llama-1B-Inst & 64.37 & 52.84 & 88.14 & 80.48 & 50 & 0 & 0 & 0 & 0 & 50 \\ \hline
% Llama-3B-Inst & 62.09 & 67.17 & \textbf{98.21} & 80.23 & 50 & 0 & 0 & 0 & 0 & 50 \\ \hline
% Llama-8B-Inst & 59.56 & 73.88 & 44.97 & 87.88 & 50 & 0 & 0 & 0 & 0 & 50 \\ \hline
% Qwen-3B-Inst & 60.89  & 80.59  & 46.31  & 86.05 & 50 & 0 & 0 & 0 & 0 & 50 \\ \hline
% Qwen-7B-Inst & \textbf{78.26}  & \textbf{83.13}  & 36.24  & \textbf{93.24}  & 50 & 0 & 0 & 0 & 0 & 50 \\ \hline
% Pythia-1.4B  & 19.17 & 4.47 & 51.90 & 23.61 & 50 & 0 & 0 & 0 & 0 & 50 \\ \hline
% Pythia-6.9B  & 23.67 & 34.55 & 55.26 & 35.60 & 50 & 0 & 0 & 0 & 0 & 50 \\ \hline
% Phi-3-Mini  & 55.91 & 75.30 & 99.10 & 75.32 & 50 & 0 & 0 & 0 & 0 & 50 \\ \hline
% Mistral-7B-Inst & 61.25 & 70.93  & 96.20 & 83.85 & 50 & 0 & 0 & 0 & 0 & 50 \\ \hline
% Gemma2-2B-It & 35.34 & 42.58  & 91.50 & 70.04 & 50 & 0 & 0 & 0 & 0 & 50 \\ \hline
% GPT4-o-Mini & 83.72 & 95.32  & 59.50 & 91.74 & 50 & 0 & 0 & 0 & 0 & 50 \\ \hline
% \end{tabular}
% \caption{The above table compares our approach with the Score Based Self Rewarding approach. The row labels correspond to the model name and the column labels correspond to the sub-categories. The metric used is accuracy where the higher values indicate better performance. }
% \label{tab:model_comparison}
% \end{table*}

% \begin{tabular}{|l||ccccc||ccccc|}
% \hline
% \multirow{}{}{Models} & \multicolumn{5}{c||}{Our Approach} & \multicolumn{5}{c|}{Self Rewarding} \\
% \cline{2-11}
%  & Chat & Code & Math & Safety & Random & Chat & Code & Math & Safety & Random \\ \hline\hline
% Llama-1B-Inst & 64.37 & 52.84 & 88.14 & 80.48 & X1 & 0 & 0 & 0 & 0 & X2 \\ \hline
% Llama-3B-Inst & 62.09 & 67.17 & \textbf{98.21} & 80.23 & X3 & 0 & 0 & 0 & 0 & X4 \\ \hline
% Llama-8B-Inst & 59.56 & 73.88 & 44.97 & 87.88 & X5 & 0 & 0 & 0 & 0 & X6 \\ \hline\hline
% Qwen-3B-Inst & 60.89  & 80.59  & 46.31  & 86.05 & X7 & 0 & 0 & 0 & 0 & X8 \\ \hline
% Qwen-7B-Inst & \textbf{78.26}  & \textbf{83.13}  & 36.24  & \textbf{93.24}  & X9 & 0 & 0 & 0 & 0 & X10 \\ \hline\hline
% Pythia-1.4B  & 0 & 0 & 0 & 0 & X11 & 0 & 0 & 0 & 0 & X12 \\ \hline
% Pythia-6.9B  & 0 & 0 & 0 & 0 & X13 & 0 & 0 & 0 & 0 & X14 \\ \hline\hline
% Phi-3-Mini(3.8B)  & 0 & 0 & 0 & 0 & X15 & 0 & 0 & 0 & 0 & X16 \\ \hline
% Phi-3-Small(7.4B)  & 0 & 0 & 0 & 0 & X17 & 0 & 0 & 0 & 0 & X18 \\ \hline \hline
% Mistral-7B-Inst & 61.25 & 70.93  & 96.20 & 83.85 & X19 & 0 & 0 & 0 & 0 & X20 \\ \hline
% \end{tabular}
% \caption{Reward Bench}
% \label{tab:model_comparison}
% \end{table*}


% \begin{table*}[h!]
% \centering
% \begin{tabular}{l|cccc|cccc}
% \hline
% \multirow{}{}{Models} & \multicolumn{4}{c|}{Our Approach} & \multicolumn{4}{c}{Self Rewarding} \\
% \cline{2-9}
%  & Chat & Code & Math & Safety & Chat & Code & Math & Safety \\
% \hline
% Llama-1B-Inst & 57.62 & 51.46 & 47.57 & 70.24 & 0 & 0 & 0 & 0 \\ \hline
% Llama-3b-Inst & 51.42 & 51.90 & 48.52 & 76.89 & 0 & 0 & 0 & 0 \\ \hline
% Qwen-3B-Inst & 60.98 & 53.07 & 56.33 & 68.96 & 0 & 0 & 0 & 0 \\ \hline
% Qwen-7B-Inst & \textbf{68.73} & \textbf{57.30} & \textbf{58.15} & \textbf{89.49} & 0 & 0 & 0 & 0 \\ \hline
% Mistral-7b-Inst & 54.00 & 52.48 & 52.04 & 71.52 & 0 & 0 & 0 & 0 \\ \hline
% Llama-8B-Inst & 50.64 & 54.97 & 48.70 & 68.25 & 0 & 0 & 0 & 0 \\ \hline

% \end{tabular}
% \caption{RM Bench}
% \label{tab:model_comparison}
% \end{table*}
%%%%%%%%%%%%%%%%%%%%%%%%%%%

\subsection{Methodology}
Our approach leverages a language model as a preference model, evaluating response appropriateness through binary classification. The model determines whether a response is suitable by generating either "Yes" or "No." To guide this assessment, we employ category-specific prompts, which are detailed in Appendix \ref{Prompts_table}. The logits corresponding to the output tokens of "Yes" and "No" are extracted from the first output token and scaled to compute their respective probabilities. The response with the highest "Yes" probability is selected as the accepted response, while the one with the lowest is classified as rejected. We hypothesize that higher-quality responses will have a greater likelihood of receiving a "Yes."

%\subsubsection{Prompting}

%In our experiments, we observed that guiding the language model to initiate its responses with "Yes" or "No" was essential, particularly for smaller models whose outputs are highly sensitive to prompt phrasing. We developed broad, category-specific prompts tailored to different query types to ensure consistency and reliability. Following prior research \citep{lambert2024rewardbenchevaluatingrewardmodels, liu2024rmbenchbenchmarkingrewardmodels}, we classify prompts into four overarching categories: Code, Math, Chat and Safety. Additional details about prompts are provided in Appendix \ref{Prompts_table}. An example of the prompts is shown in Figure \ref{Sample_Prompts}.

% \begin{itemize}
%   \setlength{\itemsep}{0.05em} % Adjust the space between items
%   \item \textbf{Code:} Questions that involve generating or interpreting code.
%   \item \textbf{Math:} Questions related to mathematical computations or reasoning.
%   \item \textbf{Chat:} General conversational questions, often open-ended or casual.
%   \item \textbf{Safety:} Questions focused on evaluating or ensuring safety in responses.
%   For our evaluation on the Reward Bench dataset we further subcategorize it into Safety General and Safety Refusal. 
% \end{itemize}



\begin{figure}[h]
    \centering
    \includegraphics[width=\columnwidth]{assets/prompt.pdf}
    \caption{Example outputs from Reward Bench using our approach.}
    \label{Sample_Prompts}
\end{figure}



\subsubsection{Preference Classification}
\label{Preference Modeling}

In our experiments, we observed that guiding the language model to initiate its responses with "Yes" or "No" was essential, particularly for smaller models whose outputs are highly sensitive to prompt phrasing. We developed broad, category-specific prompts tailored to different query types to ensure consistency and reliability. Following prior research \citep{lambert2024rewardbenchevaluatingrewardmodels, liu2024rmbenchbenchmarkingrewardmodels}, we classify prompts into four overarching categories: Code, Math, Chat and Safety. Additional details about prompts are provided in Appendix \ref{Prompts_table}. An example prompt is shown in Figure \ref{Sample_Prompts}.

To quantify preferences, we extract the output token probabilities for "Yes" and "No" from the response. The detailed approach is outlined below:

Given an input token sequence \( \mathbf{x} = (x_1, x_2, \dots, x_T) \), a language model \( f(\cdot) \) generates a probability distribution over the vocabulary \( \mathcal{V} \) for the next token. Specifically, the model outputs a logit vector \( \mathbf{z} \in \mathbb{R}^{|\mathcal{V}|} \), where  

\begin{equation}
\mathbf{z} = f(\mathbf{x}).
\end{equation}

To derive probabilities, we apply the softmax function over the logits:  

\begin{equation}
p_i = \frac{\exp(z_i)}{\sum_{j \in \mathcal{V}} \exp(z_j)}, \quad \forall i \in \mathcal{V},
\end{equation}

where \( p_i \) represents the probability assigned to token \( i \). Thus we define probability of "Yes" token as $p_{\text{yes}}$ and "No" token as $p_{\text{no}}$. Then we normalize the probabilities to ensure a fair comparison:

% We define the sets of token indices corresponding to "Yes" and "No" as \( V_{\text{yes}} \subset \mathcal{V} \) and \( V_{\text{no}} \subset \mathcal{V} \), respectively. The cumulative probabilities of these responses are computed as:  

% \begin{equation}
% p_{\text{yes}} = \sum_{i \in V_{\text{yes}}} p_i, \quad p_{\text{no}} = \sum_{i \in V_{\text{no}}} p_i.
% \end{equation}


\begin{equation}
p_{\text{yes}}' = \frac{p_{\text{yes}}}{p_{\text{yes}} + p_{\text{no}}}, \quad 
p_{\text{no}}' = \frac{p_{\text{no}}}{p_{\text{yes}} + p_{\text{no}}}.
\end{equation}

The final values \( (p_{\text{yes}}', p_{\text{no}}') \) represent the normalized likelihoods of the model predicting "Yes" or "No" .

\subsection{Experiments}
\subsubsection{Benchmarking Our Approach}

To evaluate our approach, we conducted experiments using LLMs of varying sizes and architectures. We compared instruction-tuned models with their base counterparts. Additionally, we analyzed the effect of fine-tuning on a specialized task like code/math problems on preference classification by including models fine-tuned for these tasks. For comparisons involving a reward model we use the Skywork Reward Llama 8B model \cite{liu2024skywork} as the baseline.
The detailed results for all the comparisons are available in Appendix \ref{All Results}.

In particular, we tested the following models:

\begin{itemize}
  \setlength{\itemsep}{0.05em} % Adjust the space between items
  \item \textbf{LLaMA Family \citep{dubey2024llama}:} LLaMA-3.2-1B, LLaMA-3.2-1B-Instruct, LLaMA-3.2-3B, LLaMA-3.2-3B-Instruct, Meta LLaMA 3-8B, Meta LLaMA 3-8B-Instruct.
  \item \textbf{Mistral Family \citep{jiang2023mistral}:} Mistral 7B, Mistral 7B-Instruct.
  \item \textbf{Qwen Family \citep{yang2024qwen2}:} Qwen2.5-3B, Qwen2.5-3B-Instruct, Qwen2.5-7B, Qwen2.5-7B-Instruct.
  \item \textbf{Code Generation Models:} Starcoder2-7B \citep{lozhkov2024starcoder}, CodeGemma-7B-It \citep{team2024codegemma}, Qwen-Coder-7B-Inst \citep{hui2024qwen2}, Qwen-Coder-3B-Inst.
  \item \textbf{Math Generation Models:} Qwen-Math-7B-Inst, Qwen-Math-1.5B-Instruct \citep{yang2024qwen2}, Deepseek-Math-7B \citep{shao2024deepseekmathpushinglimitsmathematical}, Llemma-7B \citep{azerbayev2024llemmaopenlanguagemodel}.
  \item \textbf{Other Models:} Phi-3-mini-128k-Instruct \citep{abdin2024phi}, Gemma 2B-Instruct \citep{team2024gemma}, GPT-4o Mini \citep{openai2024gpt4technicalreport}.

\end{itemize}

To evaluate model performance, we selected Reward Bench due to its high-quality and diversity. Reward Bench consists of 23 question categories, which are grouped into four broad types: Chat, Code, Math, and Safety. We also benchmark our approach on RM-Bench, results of which can be found in Table \ref{tab:rm_bench_levels}.

We define accuracy as the proportion of cases where the model assigns a higher probability to the preferred response \( y^w \) over the less preferred response \( y^l \):

\[
\text{Acc} = \frac{1}{N} \sum_{i=1}^{N} \mathbb{I} \left[ p_{\text{yes}}(x_i, y^w_i) > p_{\text{yes}}(x_i, y^l_i) \right]
\]

where \( \mathbb{I} [\cdot] \) is the indicator function, returning 1 if the condition holds and 0 otherwise and $N$ is the number of data points.

To ensure optimal model performance, we developed an automated pipeline for selecting the most effective category-specific prompts. Further details on prompt selection can be found in Appendix \ref{Prompts_table}.

\subsubsection{Comparision against Self Rewarding Approach}
We benchmarked our approach against the preference classification approach used in the Self-Rewarding Language Model\footnote{The Self-Rewarding approach performs very poorly on Base Models, so we tested their method on only Instruct models.}. Their approach involves scoring responses using a numerical reward of up to 5 \citep{yuan2024selfrewarding,li2024selfalignment}. Each response is evaluated based on its relevance, completeness, clarity, and informativeness. The comparitive results are shown in Table \ref{tab:reward_bench}.
%%%%%%%%%%%%%%%%%%%%%%%%%%%%%%%%%%%%
\begin{figure*}[h]
    \centering
    \includegraphics[width=\textwidth]{assets/math_code.pdf}
    \caption{\textbf{Left}: Our approach on Code Specific Model where the dashed line is a reward model. \textbf{Right}: Our approach on 4 different math-specific models where the striped bar is the reward model.}
    \label{Code results}
\end{figure*}
%%%%%%%%%%%%%%%%%%%%%%%%%%%%%%%%%%%%
\subsection{Findings}
Our approach demonstrated robust and consistent performance across all subcategories of the Reward Bench, particularly when compared to the self-rewarding approach. This performance gap was particularly pronounced in smaller models, where our approach significantly outperformed the self-rewarding approach. The self-rewarding approach assigns discrete rewards ranging from 1 to 5 for each response, making it challenging to differentiate between them, often rating both the chosen and the rejected response as the same.

Another insight was that most models perform well on safety, indicating safety tuning across all the models during training. Chat performance remains relatively consistent across models, suggesting a similar level of optimization for conversational abilities. However, performance on code and math varies significantly, largely depending on the type of training data used \citep{gunasekar2023textbooks,petty2024does,aryabumi2024code}. For example, the Qwen family excels in coding tasks, while Llama 3.2, Mistral, Gemma, and Phi models demonstrate strong mathematical capabilities.

Another finding was that larger models consistently outperformed smaller models, as shown in Table \ref{tab:reward_bench} and that instruction-tuned models consistently outperformed their base counterparts, reinforcing the effectiveness of instruction-based fine-tuning even in acting as preference classifiers. Additional results of our approach on RM-Bench can be found in \ref{All Results}.

On proprietary models, such as GPT, our approach remained competitive. Results using our approach on GPT-4o-Mini on Reward Bench can be found in Appendix \ref{GPT_Results}. 

\subsection{Performance of Math and Code Specific Models}

To better understand the applicability of our approach in mathematical and coding tasks, we evaluated four models fine-tuned for code completion and four models optimized for mathematical problem-solving. These models were benchmarked against Skywork-Llama8B-Reward Model, which serves as a strong baseline for preference modeling.

Among the code-specific models, Qwen consistently achieved the highest performance across all evaluated categories, performing as well as the Reward Model. 

In contrast, all math-specific models underperformed compared to both the general instruct-tuned version and the Reward Model. We hypothesize that this underperformance stems from the training objective of math-specific models, which prioritize generating chain-of-thought reasoning \citep{yang2024qwen2,shao2024deepseekmathpushinglimitsmathematical,gao2024designing,zhou2024dual} rather than adhering to strict instruction-following behavior required for binary Yes/No classification.
\begin{table}[t]
    \centering
    \resizebox{.48\textwidth}{!}{%
    \tablestyle{4pt}{1.2}
    \begin{tabular}{l|l|ccc|ccccc}
    \Xhline{0.11em}
        \textbf{Model} & \textbf{Stage} &\textbf{MM-AlignBench}  &\textbf{WildVision}\\
        \hline
        LLaVANext$^I$ & SFT            &9.5 / -69.2 &30.4 / -34.2 \\
        \rowcolor{gray!20}
        LLaVANext$^I$ & SFT+DPO        &11.1 / -64.5 & 35.5 / -23.4 \\
        \hline
        LLaVANext$^I$-OA & SFT       &57.1 / +11.1 &29.6 / -31.3   \\
        \rowcolor{gray!20}
        % LLaVANext$^I$-OA & SFT+DPO   &65.5 / +23.2 &35.2 / -22.8  \\
        LLaVANext$^I$-OA & SFT+DPO   &64.3 / +22.4 &41.8 / -10.1  \\
        \hline
        InternVL2-8B   & SFT                &31.4 / -21.8 & 48.6 / +1.4\\
        \rowcolor{gray!20}
        InternVL2-8B   & SFT+DPO            &64.7 / +19.4 & 51.4 / +1.9\\
    \Xhline{0.11em}
    \end{tabular}
    }%
    \caption{\textbf{Performance of applying DPO with OmniAlign-V-DPO}. For models finetuned with human-aligned data, by employing DPO training, the model's alignment with human perference further improved.}
    \label{tab: dpo}
    \vspace{-10pt}
\end{table}
\section*{Conclusion}
This paper aims to enhance our understanding of the computational complexity of computing various Shapley value variants. We found that for various ML models --- including decision trees, regression tree ensembles, weighted automata, and linear regression --- both local and global interventional and baseline SHAP can be computed in polynomial time under HMM modeled distributions. This extends popular algorithms, such as TreeSHAP, beyond their empirical distributional scope. We also establish strict complexity gaps between the various SHAP variants (baseline, interventional, and conditional) and prove the intractability of computing SHAP for tree ensembles and neural networks in simplified scenarios. Overall, we present SHAP as a versatile framework whose complexity depends on four key factors: \begin{inparaenum}[(i)] \item model type, \item SHAP variant, \item distribution modeling approach, \item and local vs. global explanations\end{inparaenum}. We believe this perspective provides deeper insight into the computational complexity of SHAP, paving the way for future work.




%We believe that our framework provides a more intricate understanding of SHAP computation complexity across different models, distributions, and variants, paving the way for further research.

Our work opens promising directions for future research. First, expanding our computational analysis to other SHAP-related metrics, such as asymmetric SHAP~\citep{frye20} and SAGE~\citep{covert2020understanding}, would be valuable. Additionally, we aim to explore more expressive distribution classes and relaxed assumptions beyond those in Section \ref{sec:tractable} while maintaining tractable SHAP computation. Finally, when exact computation is intractable (Section \ref{sec:intractable}), investigating the approximability of SHAP metrics through approximation and parameterized complexity theory~\citep{downey2012parameterized} is an important direction.

%Our work opens several promising avenues for future research on the computational properties of explainable AI methods, with a particular focus on SHAP. First, it would be interesting to broaden the computational analysis conducted in this work to include other popular SHAP-related metrics in the literature, such as asymmetric SHAP \cite{frye20} and SAGE \cite{covert2020understanding}. Also, in the future, we aim to explore more expressive distribution classes and relaxed distributional assumptions—extending beyond those examined in Section \ref{sec:tractable} —that still yield tractable SHAP computation. Finally, when exact computation proves intractable (Section \ref{sec:intractable}), it is worthwhile to theoretically investigate the question of the approximability of computing the SHAP metrics across various configurations, through the lens of approximation and parametrized complexity theory \cite{arora2009computational}.

%This paper aims to deepen our understanding of the computational complexity involved in obtaining different Shapley value variants. We found that for a variety of ML models, including decision trees, tree ensembles for regression, weighted automata, and linear regression models — computing both local and global interventional and baseline SHAP can be done in polynomial time when distributions are modeled by HMMs. This extends the distributional scope of popular algorithms like TreeSHAP, which is limited to empirical distributions. Additionally, we demonstrate a strict complexity gap between SHAP variants, showing that interventional and baseline SHAP can be strictly easier to compute than conditional SHAP. Despite these positive results, we uncovered intractability for various SHAP variants in neural networks and tree ensembles. Finally, we provided generalized complexity relations across SHAP variants. We believe that our framework offers a deeper understanding of the complexity involved in computing SHAP across various variants, models, distributions, as well as in both local and global computations, laying the groundwork for future research. 
\section{Limitations} 

In this work, we compared the effectiveness and interplay of SFT and RL-based methods, under fixed data constraints. In particular, we chose offline methods like DPO and KTO as the baseline implementation of the RL method because it eliminates the need for reward modeling or iterative finetuning. This means that the process of development is limited to collecting an offline dataset and fientuning it - making it the most fair comparable to SFT in terms of implementation effort, compute costs and annotation efforts. Since this baseline RL method shows optimal performance over SFT, we hope that this motivates future work to study more complex RL-based methods and their interplay with SFT. In addition, we used GPT4o annotation for synthetic data generation, and also for evaluating Summarization and Helpfulness, which could include potential biases inherited from the model. 

In addition, we limited the size of the model to under 10 Billion parameters, to keep the finetuning cost low enough to ignore as compared to the data annotation costs. In addition, it would be extremely compute resource intensive to run thousands of finetuning runs with larger model sizes like 70B parameters. We hope that future work would study the scaling trends of RL-based methods against different model sizes, and also study the compute-data trade-off in-depth.

\section{Acknowledgement}
The authors are grateful to  Afonso S. Bandeira, Anastasia Kireeva, Alexander S. Wein, Samuel B. Hopkins, and Tim Kunisky for helpful discussions. 




\bibliography{acl}

\appendix
\begin{table}[h]
    \centering
    \begin{tabular}{lc}
    \toprule
        \textbf{Parameter} & \textbf{value} \\ \midrule
        num\_epochs & 3 \\
        save\_steps & 100 \\
        eval\_steps & 100 \\
        logging\_steps & 100 \\
        batch\_size & 64 \\
        gradient\_accumulation & 1 \\
        weight\_decay & 0.01 \\
        bf16 & True \\
        \bottomrule
    \end{tabular}
    \caption{Training hyperparameters}
    \label{tab:training-hyperparameter}
\end{table}
\section{Dataset Generation}
\label{sec:dataset}
\revise{
To train the proposed GNN, we constructed a dataset of building structures and a subset of these structures were subjected to fire simulations using FEA. The dataset generation process is illustrated in \figref{fig:dataset_generation_procedure}. Initially, a total of 33,000 building structures with geometrical details, material properties, and gravity loads were created. Due to randomness in generating these structures, a filter is applied to remove unreasonable data after gravity load simulation, which included 15,377 structures. A trade-off between computational feasibility and model performance is made among the remaining 17,623 structures. As further labeling structures with MIDR requires resource-intensive fire simulations via OpenSeesRT, a large proportion of 16,050 structures is selected as unlabeled dataset. On the other hand, each of the other 1,573 structures was further subjected to 30 different fire simulations, forming the labeled dataset containing $1,573\times 30 = 47,190$ fire cases.} This section details the step-by-step process for generating the dataset, including geometry creation, material property assignment, and simulations due to gravity loads and fire scenarios. 
% To train the proposed neural network, we constructed a dataset comprising building structure data and a subset of fire scenario data. The dataset generation process is illustrated in \figref{fig:dataset_generation_procedure}. 
% A total of 33,000 building structures with geometric details, material properties, and gravity loads were initially created. Out of these, 3,000 structures were selected as labeled data, and the remaining 30,000 were designated as unlabeled data. Further, about half of them filtered out due to instability under gravity loads only. 
\begin{figure*}[h!]
    \centering
    \includegraphics[width=0.8\linewidth]{figures/dataset_filter_procedure.pdf}
    \caption{Workflow for dataset generation (geometry, material property, gravity loads, and fire scenarios).}
    \label{fig:dataset_generation_procedure}
\end{figure*}

\subsection{Geometry Generation}
\label{subsec:geometry_generation}
The geometry of the building structures forms the foundation of the dataset. Regular 
\revise{3D structures} resembling multi-story parking structures or shopping malls were generated, with parameters such as building floor dimensions and story heights selected randomly. Each building structure is composed of multiple rooms, which serve as the basic unit in this study. A room herein is a cuboid space defined by specific length, width, and height. Within a structure, rooms of the same dimensions are uniformly arranged along the length, width, and height, corresponding to the $x$-, $y$-, and $z$-axes, respectively. Structures vary in room size and number of rooms along each axis. Specifically, the room length, width, and height are independently sampled from a uniform distribution within the interval $[2, 5]$ meters along the three directions of the structure. Similarly, the room number along each axis is uniformly sampled independently as an integer within the interval $[2, 7]$, i.e., the maximum number of stories of the buildings simulated in this study is 7.

To introduce variability and simulate real-world scenarios, approximately $8\%$ of structural elements (beams or columns) are randomly removed after initial geometry creation. 
\revise{Such removal is not fire-induced damage, but reflects functional diversity often observed in real buildings, such as open spaces designed for activities in shopping malls, e.g., ice skating rinks. Examples of the generated geometries are illustrated in \figref{fig:example_generated_geometry}, showcasing the diversity and realism of the dataset. This element removal does not affect the definition of room's geometry in the structure and nor does it affect the number of considered fire scenarios.} 

\revise{A range of coefficient of variation values ($3.3\%$ to $17.5\%$) was derived from prior studies that investigated the statistics of geometrical and material properties of structural components of buildings (e.g., \cite{mirza1979variations, lee2004probabilistic}). These studies provide empirical data on the natural variability in parameters such as Young's modulus, yield strength, and dimensions of structural elements due to manufacturing tolerances and material inconsistencies. By selecting $8\%$ for the removal of structural elements in our database, we aimed to maintain a level of variability that is representative of real-world uncertainties while ensuring computational feasibility. This choice ensures that the database captures realistic deviations without introducing extreme cases that may not be commonly encountered in practice.}

\begin{figure*}[h!]
    \centering
    \includegraphics[width=\linewidth]{figures/example_generated_geometry.pdf}
    \caption{Examples of generated structural geometry of different sizes (all dimensions in meters).}
    \label{fig:example_generated_geometry} 
\end{figure*}

{\blockRevise

In this study, we opted for a deterministic square, dimension of $0.1$ m, solid cross-sectional steel elements due to their simplicity in modeling and analysis. Square sections exhibit uniform geometrical properties in all directions, simplifying the computation of structural responses and avoiding complications associated with more complex shapes, such as wide-flange sections, facilitating the computational efficiency and scalability to generate a large dataset. This choice also helps to mitigate issues related to stress concentrations and facilitates a more straightforward representation of structural behavior under thermal loads. 

\textit{Remark:} The selected cross-section provides a comparable flexural rigidity to a $W 130 \times 130 \times 28.1$ wide-flange section (metric units), albeit with significantly higher axial rigidity. This cross-section is acceptable for gravity-load-designed frames under service loading conditions where the models assume fully rigid, moment-resisting beam-column connections for the evaluation of the IDR under thermal loading. This assumption is reasonable in this computational study where the primary interest is to understand the global deformation response of frames under fire conditions. The selection of uniform square cross-sections for both beams and columns, rather than adherence to standard capacity design principles, was made here primarily for computational efficiency and to reduce design parameters in the database generation process. This choice allows for simplified and scalable approach to analyze the fire-induced response of generic steel frames without the need for large section variations, where this study mainly focuses on the fire vulnerability assessment using ML-based predictions. However, if additional loading conditions, e.g., seismic or wind loads, were to be considered, larger sections, strong-column/weak-beam principle, and ductile detailing would be required in the generated buildings for realistic structural behavior under combined loading conditions. Future studies may also consider investigating the influence of variable cross-sectional dimensions and semi-rigid connections on the structural performance under fire conditions. 
} % blockRevise

\subsection{Material Properties}
Steel is chosen as the material for the structures. To reflect real-world variations, we randomly assign one of five slightly different steel material types to each structural element. \revise{
The ranges of material properties are provided in \tabref{tab:material_property_ranges} and the properties are sampled from uniform distributions of the corresponding ranges. These variations simulate differences arising from manufacturing batches or regional material properties. That these properties are at ambient temperature and change when the temperature rises due to a fire. The selection of materials with varying properties is aimed at increasing the diversity of the data. Our goal is to represent as wide a range of data as possible with a limited amount of building structure data, thereby enhancing the generalization ability of the GNN. Our assumed material property ranges are expected to be wider than the real-world conditions based on findings in \cite{mirza1979variations, lee2004probabilistic}. Therefore, we are essentially tackling a more challenging and general task. If we can solve this problem, we are confident that our method will perform equally well or even better in real-world scenarios.
}
\begin{table}[h!]
    \centering
    \caption{Material properties ranges for considered steel structures.}
    \begin{tabular}{lc}
        \toprule
        Property & Range \\
        \midrule
        Young's modulus & [168, 252] GPa \\
        Yield strength & [220, 330] MPa \\
        Strain-hardening ratio & [0.8, 1.2] \% \\
        \bottomrule
    \end{tabular}
    \label{tab:material_property_ranges}
\end{table}

\subsection{Gravity Loads}
Gravity loads are applied to columns and beams based on their \revise{influence (tributary) areas as typically conducted in structural analysis. The considered ``service'' load conditions include the column self-weight and the additional loads directly supported on the beams from their self-weight and weights of the reinforced concrete slabs, people as live load, and building content. An edge beam typically carries approximately half the gravity load supported by a parallel interior beam}. The ranges of gravity loads are listed in \tabref{tab:gravity_load_ranges}. \revise{The loads are sampled from uniform distributions of the corresponding ranges.} Structures that failed to meet an MIDR threshold of $1\%$ under gravity loads were deemed unacceptable designs and filtered out, as such configurations of randomly chosen geometry, material, and gravity load combinations were considered unrealistic from a regulatory and practicality points of view.
\begin{table}[h!]
    \centering
    \caption{Gravity load ranges for considered beams and columns.}
    \begin{tabular}{lc}
        \toprule
        Element & Range (kN/m)  \\
        \midrule
        Column & [0.5, 1.0]  \\
        Edge beam & [1.5, 4.5]  \\
        Interior beam & [3.0, 7.5]  \\
        \bottomrule
    \end{tabular}
    \label{tab:gravity_load_ranges}
\end{table} 

\subsection{Rule-based Thermal Load Generation}
\label{subsec:thermal_load_generation}
To evaluate a building's structural response during a fire event, we employed a simplified rule-based approach for thermal load generation. 
% Previous studies \cite{nan_structuralfire_2023} have demonstrated that steel structures rapidly equilibrate with surrounding gases temperatures due to efficient heat exchange. Consequently, gas temperatures can be directly used as inputs for FEA tools, e.g., OpenSees, simplifying the process of modeling thermal loads. 
% Accurately simulating temperature fields in fire scenarios poses significant challenges. Advanced thermodynamic simulations, such as those performed using Fire Dynamics Simulator (FDS) \cite{mcgrattan_fire_2000}, provide precise temperature predictions. However, these methods are hindered by high computational costs, prolonging execution times, and limited scalability, making them impractical for generating large datasets. Additionally, real-world fire loads often display substantial spatial variability across different rooms \cite{dundar_fire_2023}, resulting in scenario-specific temperature fields with limited generalizability. For example, studies on bridge fires \cite{he_study_2024} have demonstrated that environmental factors, such as wind speeds, can significantly influence temperature distributions. Furthermore, even within identical scenarios, variations in fire modeling methodologies can produce distinctly different temperature fields \cite{zhang_temperature_2020, du_new_2012}. These challenges emphasize the need for efficient and adaptable methods to generate fire temperature data.
% To address these issues, we adopted a rule-based approach to model temperature variations. 
According to \cite{spearpoint_fire_2008}, a typical fire development follows a predictable pattern. During the {\em{growth stage}}, the temperature rises slowly and approximately linearly after ignition. This is followed by the {\em{flashover stage}}, where temperatures increase rapidly to peak values. After reaching the peak, the temperature either stabilizes or continues to rise slowly until the {\em{decay stage}} begins. Inspired by this fire development pattern, we describe the temperature evolution in time, $t$, prior to the decay stage in two distinct stages:
\begin{enumerate}
    \item {\bf{Initial linear increase stage}}: For $t \in [0, t_1)$, temperature increases gradually and linearly as the fire spreads through the building. This stage represents the time before the fire directly affects a structural element.  
    \item {\bf{ISO 834 fire curve stage}}: For $t \in [t_1, t_{\thre}]$, temperature rises rapidly following the ISO 834 curve \cite{ISO834}, modeling the direct impact of the fire on the structural element. 
\end{enumerate}
The slope of the linear temperature increase, $c$, and the transition time, $t_1$, are influenced by the spatial relationship between the fire source and the structural element. For the second stage of temperature evolution, we utilize the ISO 834 curve, a widely accepted standard for fire resistance testing. This standardized fire curve describes the temperature rise over time, enabling rapid and consistent thermal fields across various scenarios. The duration of fire simulation in this study is set to $t_{\thre}=60$ minutes. This value represents the upper limit for the temperature evolution of each structural element, providing a consistent basis for analyzing the structural response to fire.

Let $(x, y, z)$ represents the midpoint of a structural element and $(x_{\subfire}, y_{\subfire}, z_{\subfire})$ the fire source point. \revise{Integer parameters $h$ and $h_{\subfire}$ correspond to the respective floor levels of the element and the fire source}. The temperature evolution for each element is expressed as follows:
\begin{enumerate}
    \item Linear increase stage ($0 < t < t_1$):
    \begin{equation}
    T(t) = c \cdot t,
    \end{equation}
    where $c$, the rate of temperature increase ($^\circ\mathrm{C}/\mathrm{min}$), depends on the height difference between the element, $h$, and the fire source, $h_{\subfire}$:
    \begin{equation}
        c = 
        \begin{cases} 
        5\left/\left(h - h_{\subfire} + 1\right)\right., & h \geq h_{\subfire}, \\
        2\left/\left(h_{\subfire} - h\right)\right., & h < h_{\subfire}.
        \end{cases}
    \end{equation}
     \item ISO 834 stage ($t \geq t_1$):
\begin{equation}
    T(t) = c \cdot t_1 + 345 \log_{10} \left(8 \left(t - t_1\right) + 1\right).
\end{equation}
\end{enumerate}

The transition (arrival) time $t_1$, marking the end of the linear stage, depends on the spatial distance between the fire source and the element. We define the following two Euclidean distances $L_p$ in the $xy$ plane and $L_s$ in the $xyz$ space:
\begin{eqnarray}
L_p & \triangleq & \sqrt{(x - x_{\subfire})^2 + (y - y_{\subfire})^2}, \\
\label{eq:Lp}
L_s & \triangleq & \sqrt{(x - x_{\subfire})^2 + (y - y_{\subfire})^2 + (z - z_{\subfire})^2}.
\label{eq:Ls}
\end{eqnarray}
Accordingly, the transition time, $t_1$, is expressed as follows:
\begin{equation}
    t_1 = 
    \begin{cases}
    \beta_{1} \cdot \left(1 - \exp\left\{- L_s\left/\alpha_{1}\right.\right\}\right), & h > h_{\subfire}, \\
    \beta_{2} \cdot \left(1 - \exp\left\{- L_p\left/\alpha_{2}\right.\right\}\right), & h = h_{\subfire}, \\
    \beta_{3} \cdot \left(1 - \exp\left\{- L_s\left/\alpha_{3}\right.\right\}\right), & h < h_{\subfire} .
    \end{cases}
    \label{eq:t1}
\end{equation}
The parameters $\beta_i$ and $\alpha_i$ for determining $t_1$ are summarized in Table~\ref{tab:fire_spread_parameters}. In this study, we take $r_{\mathrm{up}}=0.95$ and $r_{\mathrm{down}}=0.97$.
\begin{table}[ht]
    \centering
    \caption{Fire spread parameters for $t_1$ calculations.}
    \begin{tabular}{lcc}
        \toprule
        Case  & $\beta_i$ & $\alpha_i$  \\
        \midrule
        $i=1$, Upward spread & $16 \left.\left(1-r_{\mathrm{up}}^{\left|h-h_{\subfire}\right|}\right)\right/\left(1-r_{\mathrm{up}}\right)$ & $10$  \\
        $i=2$, Horizontal spread & $18$ & $18$  \\
        $i=3$, Downward spread & $30 \left.\left(1-r_{\mathrm{down}}^{\left|h-h_{\subfire}\right|}\right)\right/\left(1-r_{\mathrm{down}}\right)$ & $5$  \\
        \bottomrule
    \end{tabular}
    \label{tab:fire_spread_parameters}
\end{table}

\figref{fig:t1_curve} illustrates the $t_1$ curves for various fire scenarios: (1) fire originating on the lower floor, $h-h_{\subfire}=1$ with rapid upward spread, (2) fire on the same floor, $h=h_{\subfire}$ with the fastest spread, and (3) fire on the upper floor, $h_{\subfire}-h=1$ with slow downward spread. The exponential decay in $t_1$ reflects the accelerating fire propagation speed as the distance increases. \figref{fig:t1_curve} also indicates that the employed simplified model is consistent with the Markov chain-based dynamic model given by \cite{cheng_dynamic_2011}, where the rooms at the same floor of the fire point start flashover slightly before the corresponding upper floors. Additionally, $\beta_{1}$ and $\beta_{3}$ are the summation of a geometric sequence, where story level $h$ is the index. The common ratios $r_{\mathrm{up}}<1$ in $\beta_{1}$ and $r_{\mathrm{down}}<1$ in $\beta_{3}$ indicate that the fire speeds up to spread through the next story, which is consistent with the real-world fire spread mechanism given in \cite{hokugo_mechanism_2000}. The temperature profile within the range $t \in [0, t_{\thre}]$ is subsequently used as the thermal load in OpenSeesRT simulations to compute displacements at each structural node at time $t_{\thre}$.
\begin{figure}[h!]
    \centering
    \includegraphics[width=0.8\linewidth]{figures/m204_t1_curve.pdf}
    \caption{Three examples for the $t_1$ curve.}
    \label{fig:t1_curve}
\end{figure}

\revise{
\textit{Remark:} The effects of structural elements, such as concrete floor slabs and partitions, are not explicitly modeled in our approach. Instead, their influence is implicitly captured through the careful selection of the parameters $ \alpha, \beta, r_\mathrm{up} $, and $ r_\mathrm{down} $. This parameterization provides a unified framework for generating temperature fields. Indeed, fire propagation is governed by a multitude of factors and remains an open research question. For instance, if the fire resistance of a floor slab is enhanced by fire protective coating, the corresponding model can account for this by decreasing $\alpha_1$ \& $\alpha_3$, increasing $\beta_1$ \& $\beta_3$, and adopting larger values for $r_\mathrm{up}$ \& $r_\mathrm{down}$, which collectively slow down the vertical spread of fire. Conversely, scenarios involving higher amounts of combustible materials would warrant the opposite adjustments. This flexible and integrated approach avoids the need to design separate models for different fire propagation scenarios while still capturing the essential effects.
}

\revise{
In conclusion, our rule-based approach is a computationally efficient method for approximating fire temperature fields, enabling large-scale dataset generation to train predictive models. By combining ISO 834 fire curves with spatial considerations and embedding structural effects through parameter calibration, the method achieves a balanced trade-off between accuracy and scalability, making it a practical solution for thermal load modeling in fire scenarios. After generating the temperature of each beam or column according to the middle point, the temperature is applied as uniform thermal load to the elements of the structure in question using OpenSeesRT. 
}

% In conclusion, this rule-based approach is a computationally efficient method to approximate fire temperature fields, enabling large-scale dataset generation to train predictive models. By combining ISO 834 fire curves with spatial considerations, the method balances accuracy and scalability, making it a practical solution for thermal load modeling in fire scenarios.

% \subsection{Interstory Drift Ratio}
\subsection{OpenSeesRT Simulation}
\label{subsec:opensees_simulation}

The thermal and mechanical responses of 3D frame structures under combined fire and gravity loads are simulated using OpenSeesRT \cite{perez2024openseesrt}. \revise{In the simulation, the IDR of each node at $t_{\thre}$ is computed using the computed nodal displacements. Each structural model features six degrees of freedom per node (3 translational  and 3 rotational), with linear geometrical transformations (\texttt{geomTransf: Linear}) defining how the element local coordinate systems are mapped to the global coordinate system and assuming small displacements and rotations. Although OpenSeesRT allows a variety of options for modeling finite deformations, in the present simulations and mainly for simplicity, we did not consider large deformations. All bottom nodes (nodes on the ground) are fully constrained in all six degrees of freedom, while degrees of freedom os all other nodes are free.} Material behavior is temperature-dependent and modeled with \texttt{Steel01Thermal}, while fiber-based sections (\texttt{FiberThermal}) capture nonlinear interactions between thermal and mechanical responses at the cross-section level. \revise{Structural elements are represented as displacement-based Euler-Bernoulli beam-columns (\texttt{dispBeamColumnThermal}). This element  formulation accounts for thermal strains (temperature gradients) in the section, which is discretized into fibers. Numerical integration is used along the length of each element using three integration (Gauss) points, one at each end and the third in the middle of the element.}

{\revise{Thermal expansion of steel members plays a crucial role in IDR development. In reality, reinforced concrete floor slabs heat at a different rate than steel members due to their higher thermal mass and lower thermal conductivity. This differential heating can lead to restrained thermal expansion, introducing axial compression in beams and affecting the overall structural response. In this study, explicit {\em{composite action}} between steel members and concrete slabs is not modeled. Instead, our approach focuses on isolating the response of the steel structural frame, which is often the critical load-bearing component in fire scenarios. This assumption aligns with prior studies \cite{Possidente_2024} demonstrating that steel structures reach thermal equilibrium with surrounding gases quickly, allowing the use of uniform thermal loading in fire analysis. Future work could enhance this framework by incorporating slab-beam interaction effects, through a refined FEA for an extended dataset where constraints imposed by floor slabs are explicitly considered.}

The analysis begins with the application of gravity loads, followed by incremental thermal loads simulating the fire exposure. A static nonlinear solver using  \texttt{ExpressNewton} algorithm ensures convergence, while the \texttt{NormDispIncr} test maintains accuracy. An incremental \texttt{LoadControl} scheme with small step sizes is employed to guarantee numerical stability, using 10\% for gravity loads and 1\% for thermal loads. 

\revise{
In the thermal load analysis, uniform thermal load is applied to each beam or column, i.e., the temperature of each element is set to be that at the middle point, according to \secref{subsec:thermal_load_generation}. The \texttt{Steel01Thermal} material allows the properties (e.g., Young's modulus and yield strength) to be adjusted at increasing temperatures according to \cite{EN1993} using its Table 3.1: Reduction factors for the stress-strain relationship of carbon steel at elevated temperatures. For example, if the Young’s modulus at ambient temperature is $E_0$, then as the temperature ($T$) increases, the modulus changes as $E(T) = \eta (T) \times E_0$. \cite{EN1993} directly provides the values of $\eta(T) \in \left[0,1\right] $ at every $100 ^\circ\mathrm{C}$ interval and recommends using linear interpolation to obtain $\eta(T)$ for intermediate values of $T$.
} OpenSeesRT documentation \cite{OpenSeesThermalExamples} provides several examples of thermal analyses.

This modeling framework accommodates variations in material properties, cross-sectional geometries, and temperature profiles, providing robust simulations of structural behavior under fire conditions. The primary settings and configurations for the OpenSeesRT simulations are summarized in \tabref{tab:ops_detail}.
\begin{table}[h!]
    \centering
        \caption{Key settings of OpenSeesRT simulations.}
    \begin{tabular}{l|>{\raggedright\arraybackslash}p{0.6\linewidth}} %
    \toprule
    Modeling Aspect     & Details \\
    \midrule
    Geometry            & 3D models; 6 degrees of freedom per node \\
    Transformation      & geomTransf: Linear \\ 
    Material            & Steel01Thermal \\
    Section             & FiberThermal; Cross-section: $0.1$ m $\times$ $0.1$ m \\ 
    Element type        & {dispBeamColumnThermal} \\ 
    Loading             & Gravity loads: {beamUniform}; Thermal loads: {beamThermal} \\
    Integration scheme  & Incremental {LoadControl}; Step size: $10\%$ (gravity analysis), $1\%$ (thermal analysis) \\
    Nonlinear solver    & {ExpressNewton} algorithm; {UmfPack} solver; Convergence test: {NormDispIncr} tolerance: $10^{-8}$; Maximum \# iterations per step: $1000$. \\ 
    \bottomrule
    \end{tabular}
    \label{tab:ops_detail}
\end{table}

For each structure in the labeled dataset, 30 fire points are selected using a dual-granularity approach, \revise{i.e., two-stage sampling strategy,} to ensure they are well-distributed. Specifically, rooms are sequentially selected, with one fire point randomly chosen within each selected room. If a building is large and contains more than 30 rooms, we randomly select 30 rooms without replacement, i.e., ensuring that no more than one fire point is located in the same room. Conversely, if the building is small and has fewer than 30 rooms, all rooms are initially selected, with one fire point randomly assigned to each room. Additionally, rooms are then selected with replacement until a total of 30 fire points are assigned. \revise{The room-level sampling prioritizes selecting distinct rooms to avoid spatial clustering of fire points, while the point-level sampling ensures intra-room variability. This approach aligns with stratified sampling principles commonly used for efficient spatial representation, where multi-stage sampling strategies optimize coverage and variability, e.g., \cite{arunachalam_generalized_2023}, and enables a more comprehensive characterizing of how the structures respond under fire conditions.}
% This selection method prevents fire points from clustering too closely while maintaining an element of randomness. By distributing fire points in this manner, the 30 fire scenarios are effectively utilized, enabling a more comprehensive characterizing of how the structures respond under fire conditions.

\subsection{Summary of the Dataset Generation}
As discussed in this section and related to  \figref{fig:dataset_generation_procedure}, three key steps were considered in the development of the dataset: 
\begin{enumerate}
    \item {\bf{Filtering process}}: Structures with MIDR exceeding $1\%$ under gravity loads were excluded,  resulting in $1,573$ labeled structures retained for fire simulation and $16,050$ unlabeled structures for training the MFSP predictor.
    \item {\bf{Fire simulations}}: For each retained labeled structure, 30 fire scenarios were simulated using OpenSeesRT, yielding $47,190$ fire cases.
    \item {\bf{Data distribution check}}: MIDR distributions for labeled and unlabeled data under gravity loads were highly similar, because both datasets were generated using the same method. Under fire conditions, the MIDR distribution shifted, reflecting significant structural deformation with values reaching a maximum of about 6\%, an average of 1.70\%, and a standard deviation of 1.12\%. This step ensured a diverse and comprehensive dataset for the proposed predictive framework.
\end{enumerate}
The statistical distribution histograms for MIDR (after applying the $1\%$ filtering threshold \revise{for gravity load responses}) under different loading conditions are plotted in \figref{fig:histogram_mdr}. Figures \ref{fig:histogram_mdr}(a) and \ref{fig:histogram_mdr}(b) show the MIDR distributions of the labeled and unlabeled data, respectively, under gravity loads only. \figref{fig:histogram_mdr}(c) shows the MIDR distribution of the labeled data under the combined effects of gravity and fire loads. Fire load causes the structures to significantly deform, leading to a noticeably \revise{right-skewed} MIDR distribution.

\begin{figure*}[h!]
    \centering
    \includegraphics[width=\linewidth]{figures/histogram_mdr.pdf}
    \caption{Histograms of MIDR for labeled and unlabeled structures with gravity loads and fire cases.}
    \label{fig:histogram_mdr}
\end{figure*}

\revise{
This dataset provides the basis for training and testing the performance of the GNN-based framework. Although we employed a simplified rule-based thermal load generation method compared with conventional CFD-based simulations, the temperature field, the changes of the material properties, and the response of the structures, are all still highly nonlinear and complex. Therefore, it is still a challenging task for the NN to predict the MIDRs based on this dataset.
}
\section{Results on GPT}
\label{GPT_Results}
We also evaluated our approach on proprietary models like GPT-4o-Mini and found that it significantly outperformed both the Self-Rewarding approach and the Binary Approach. In the Binary Approach, the model is given both the chosen and rejected responses along with the prompt and is asked to select the better one. To mitigate positional bias—where LLMs tend to favor the first response—a random shuffle is applied to ensure that neither the chosen nor the rejected response receivs a systematic advantage. The results for Binary Eval were taken directely from Reward Bench\footnote{\url{https://huggingface.co/spaces/allenai/reward-bench}}.  The results for the same are shown in Table \ref{accuracy_gpt}.
\section{Prompts}
\label{Prompts_table}
Based on the predefined categories, a pool of N prompts were generated using GPT. A small sample of 50 data points was selected from each category in Reward Bench, and the prompts were tested on this subset to determine the most effective one. The model used for testing was LLaMA-1B-Instruct, and the prompt with the highest accuracy in each category was used for evaluation on Reward Bench. The final set of selected prompts for Reward Bench evaluations are shown in Table \ref{Prompts} and those for DPO are shown in \ref{dpo_prompts} .

\textbf{Note:} For DPO-based training, instructions were classified into four categories

Categories of Prompts: 
\begin{itemize}
  \setlength{\itemsep}{0.05em} % Adjust the space between items
  \item \textbf{Code:} Questions that involve generating or interpreting code.
  \item \textbf{Math:} Questions related to mathematical computations or reasoning.
  \item \textbf{Chat:} General conversational questions, often open-ended or casual.
  \item \textbf{Safety:} Questions focused on evaluating or ensuring safety in responses.
  For our evaluation on the Reward Bench dataset we further subcategorize it into Safety General and Safety Refusal. 
\end{itemize}

% \begin{table*}[h]
%     \centering
%     % \resizebox{\columnwidth}{!}{%
%     \begin{tabular}{l|p}
%         \toprule
%         \textbf{Task} & \textbf{Prompt} \\
%         \midrule
%         Chat & Below is a question and its response. Assess if the response is helpful and correct. Answer just Yes/No.\\
%         \hline
%         Code & Below is a coding-related question along with a response containing code. Your task is to verify whether the provided code is correct, functional, and aligned with the requirements of the question. Answer with just Yes/No. \\
%         \hline
%         Maths & Below is a math question along with a response containing a solution. Your task is to verify whether the solution is correct, the reasoning is logical, and the final answer is accurate. Answer with just Yes/No. \\
%         \hline
%         Safety(General) & Below is a potentially unsafe or inappropriate question, along with a refusal response. Verify whether the refusal response is appropriate. Answer Yes/No. \\
%         \hline
%         Safety(Refusal) & Below is a question and a refusal response. Ensure the refusal response aligns with ethical and safety guidelines while maintaining professional clarity. Answer Yes/No. \\
%         \bottomrule
%     \end{tabular}%
%     % }
% \end{table*}
\begin{table*}[h]
    \label{Prompts}
    \centering
    \renewcommand{\arraystretch}{1.2} % Improves row spacing
    \setlength{\tabcolsep}{8pt} % Adjusts column spacing
    \begin{tabular}{l|p{12cm}}
        \toprule
        \textbf{Task} & \textbf{Prompt} \\
     \midrule
            \textbf{Chat}  & Below is a question and its response. Assess if the response is helpful and correct. Answer just Yes/No.\\
            
            \textbf{Code} & Below is a coding-related question along with a response containing code. Your task is to verify whether the provided code is correct, functional, and aligned with the requirements of the question. Answer with just Yes/No. \\
            
            \textbf{Maths} & Below is a math question along with a response containing a solution. Your task is to verify whether the solution is correct, the reasoning is logical, and the final answer is accurate. Answer with just Yes/No. \\
            
            \textbf{Safety(General)} & Below is a potentially unsafe or inappropriate question, along with a refusal response. Verify whether the refusal response is appropriate. Answer Yes/No. \\
           
            \textbf{Safety(Refusal)} & Below is a question and a refusal response. Ensure the refusal response aligns with ethical and safety guidelines while maintaining professional clarity. Answer Yes/No. \\
            \bottomrule
    \end{tabular}
        \caption{Final prompts used for Evaluation on Reward and RM Bench}
    \label{Prompts}
\end{table*}

\begin{table*}[h]
    \centering
    \renewcommand{\arraystretch}{1.2}
    \begin{tabular}{l|p{14cm}}
        \hline
        \textbf{Category} & \textbf{Prompt} \\ 
        \hline
        \textbf{Code} & Below is a coding-related question along with a response containing code. Your task is to verify whether the provided code is correct, functional, and aligned with the requirements of the question. Answer with just Yes/No. \\ 
        \textbf{Chat} & Below is a question, along with a response to that question. Your task is to verify whether the response is helpful, factually correct, logically sound, and contextually appropriate. Answer with just Yes/No. \\ 
        \textbf{Safety} & Inspect the response to ensure it is safe and appropriate for the question below. Answer Yes/No. \\ 
        \textbf{Math} & Below is a math question along with a response containing a solution. Your task is to verify whether the solution is correct, the reasoning is logical, and the final answer is accurate. Answer with just Yes/No. \\ 
        \hline
    \end{tabular}
    \caption{Final set of prompts used for DPO.}
    \label{dpo_prompts}
\end{table*}


\section{More Comprehensive Analysis of Baseline Performances}
\label{appendix: all results}
In addition to the models discussed in the \autoref{sec:results},  
 \autoref{performance1_all} and \autoref{performance2_all} provide a more comprehensive analysis of baseline performances, including evaluations of models from the Yi-1.5 (6B, 9B), OLMo-2 (7B), Mistral (Small), granite-3.1 (2B), Gemma-2 (2B, 9B), Llama-3.1 (8B), Mixtral (8x7B), and K2~\citep{liu2025llm360k2building65b}.
\definecolor{color11}{rgb}{1, 0.8, 0.8}  % 红色20%混合
\definecolor{color12}{rgb}{1, 0.9, 0.9}  % 红色10%混合
\definecolor{color21}{RGB}{255, 224, 127}  % 更亮的黄色调,温暖
\definecolor{color22}{RGB}{255, 239, 179}  % 浅黄色调,亮度增加
\definecolor{color31}{RGB}{198, 230, 195}  % 更亮的绿色,柔和清新
\definecolor{color32}{RGB}{224, 239, 225}  % 更亮的浅绿色,清新且明亮

{
\linespread{1}
\begin{table}[H]
\scriptsize
\centering
\resizebox{\textwidth}{!}{%
\begin{tabular}{p{4cm}<{\raggedright\arraybackslash}*{7}{p{2cm}<{\centering\arraybackslash}}}
\toprule
\textbf{Model} & \textbf{Overall} & \textbf{Overall} & \textbf{Overall} & \textbf{Overall} & \textbf{Easy} & \textbf{Middle} & \textbf{Hard}\\
& \textbf{(sample)} & \textbf{(subfield)} & \textbf{(field)} & \textbf{(discipline)} & \textbf{(sample)} & \textbf{(sample)} & \textbf{(sample)}\\
\midrule
\rowcolor{color11}
\multicolumn{8}{c}{\textbf{\textit{Reasoning Models}}}\\
\midrule
\rowcolor{color12}
DeepSeek-R1 &\boxed{61.82} & \boxed{62.61} & \boxed{61.23} & \textbf{59.95} & \underline{63.59} & \boxed{63.63} &\boxed{56.87} \\
\rowcolor{color12}
o1-2024-12-17 &\textbf{60.24} & \underline{61.25} & \underline{59.94} & \underline{59.44} & \textbf{64.40} & \underline{61.44} &\underline{53.67} \\
\rowcolor{color12}
DeepSeek-R1-Zero &\textbf{60.24} & \textbf{61.62} & \textbf{60.95} & \boxed{60.99} & \boxed{65.06} & \textbf{62.61} &50.99 \\
\rowcolor{color12}
o3-mini-2025-01-31-high &\underline{55.22} & 54.94 & 52.11 & 48.32 & 53.05 & 56.09 &\textbf{56.16} \\
\rowcolor{color12}
o3-mini-2025-01-31-medium &52.69 & 52.66 & 49.95 & 46.07 & 51.30 & 53.79 &52.37 \\
\rowcolor{color12}
o3-mini-2025-01-31-low &48.03 & 48.51 & 45.89 & 42.63 & 48.80 & 50.21 &43.53 \\
\rowcolor{color12}
o1-mini-2024-09-12 &45.22 & 45.46 & 42.53 & 39.33 & 46.77 & 47.34 &40.00 \\
\rowcolor{color12}
QwQ &43.59 & 44.40 & 43.19 & 41.63 & 46.46 & 47.40 &34.07 \\
\midrule
\rowcolor{color21}
\multicolumn{8}{c}{\textbf{\textit{Chat Models}}}\\
\midrule
\rowcolor{color22}
Doubao-1.5-pro-32k-250115 &\boxed{55.09} & \boxed{56.55} & \boxed{55.62} & \boxed{54.39} & \underline{57.70} & \boxed{60.15} &\boxed{43.80} \\
\rowcolor{color22}
Doubao-1.5-pro-32k-241225 &\textbf{50.93} & \underline{52.41} & \underline{51.76} & 51.24 & 53.54 & \textbf{56.56} &\underline{38.70} \\
\rowcolor{color22}
qwen-max-2025-01-25 &\underline{50.08} & \textbf{52.75} & \textbf{52.47} & \underline{51.65} & \textbf{58.16} & \underline{54.95} &33.09 \\
\rowcolor{color22}
claude-3-5-sonnet-20241022 &48.16 & 51.38 & 51.23 & \textbf{53.15} & \boxed{59.04} & 51.91 &29.99 \\
\rowcolor{color22}
gemini-2.0-flash &47.73 & 48.70 & 47.80 & 46.10 & 53.06 & 49.56 &\textbf{38.84} \\
\rowcolor{color22}
DeepSeek-V3 &47.40 & 49.10 & 48.31 & 47.35 & 55.63 & 50.11 &33.86 \\
\rowcolor{color22}
MiniMax-Text-01 &45.11 & 47.46 & 46.97 & 47.06 & 54.51 & 48.60 &28.98 \\
\rowcolor{color22}
gpt-4o-2024-11-20 &44.40 & 47.62 & 47.50 & 48.84 & 56.84 & 48.75 &23.50 \\
\rowcolor{color22}
Llama-3.1-405B-Instruct &43.14 & 46.43 & 45.83 & 47.35 & 56.06 & 46.31 &23.70 \\
\rowcolor{color22}
gpt-4o-2024-08-06 &41.64 & 44.79 & 44.91 & 46.29 & 55.22 & 45.11 &20.98 \\
\rowcolor{color22}
Qwen2.5-72B-Instruct &40.75 & 43.66 & 43.32 & 42.10 & 48.84 & 45.42 &24.10 \\
\rowcolor{color22}
Mistral-Large-Instruct-2411 &40.65 & 43.38 & 43.13 & 43.37 & 52.92 & 43.28 &22.81 \\
\rowcolor{color22}
qwen-max-2024-09-19 &39.96 & 42.93 & 42.16 & 41.62 & 50.23 & 43.63 &22.60 \\
\rowcolor{color22}
gpt-4o-2024-05-13 &39.76 & 43.19 & 43.13 & 45.23 & 53.37 & 42.38 &20.45 \\
\rowcolor{color22}
Qwen2.5-32B-Instruct &38.76 & 41.18 & 40.40 & 39.43 & 47.42 & 43.05 &22.13 \\
\rowcolor{color22}
Llama-3.3-70B-Instruct &37.69 & 40.56 & 40.15 & 41.12 & 49.68 & 40.68 &19.55 \\
\rowcolor{color22}
phi-4 &37.65 & 39.59 & 38.61 & 37.66 & 45.43 & 40.91 &23.69 \\
\rowcolor{color22}
Qwen2.5-14B-Instruct &35.15 & 37.72 & 37.41 & 36.07 & 44.82 & 37.90 &19.97 \\
\rowcolor{color22}
Llama-3.1-70B-Instruct &34.86 & 38.94 & 39.18 & 40.57 & 48.22 & 37.85 &15.22 \\
\rowcolor{color22}
Yi-Lightning &33.42 & 36.57 & 36.45 & 36.92 & 43.38 & 35.32 &19.35 \\
\rowcolor{color22}
Mixtral-8x22B-Instruct-v0.1 &29.23 & 32.14 & 32.28 & 32.82 & 42.52 & 29.73 &13.82 \\
\rowcolor{color22}
Qwen2.5-7B-Instruct &28.78 & 30.78 & 30.37 & 30.63 & 37.77 & 30.98 &15.23 \\
\rowcolor{color22}
gemma-2-27b-it &27.43 & 30.50 & 30.42 & 31.30 & 40.90 & 27.45 &12.64 \\
\rowcolor{color22}
Yi-1.5-34B-Chat &26.03 & 28.81 & 28.84 & 29.08 & 36.99 & 26.74 &12.81 \\
\rowcolor{color22}
Mistral-Small-Instruct-2409 &25.89 & 28.46 & 28.69 & 28.59 & 37.93 & 25.89 &12.70 \\
\rowcolor{color22}
gemma-2-9b-it &24.04 & 26.89 & 27.06 & 27.74 & 37.81 & 23.05 &10.60 \\
\rowcolor{color22}
Qwen2.5-3B-Instruct &23.31 & 25.45 & 25.86 & 25.57 & 33.10 & 23.50 &12.24 \\
\rowcolor{color22}
Yi-1.5-9B-Chat &23.17 & 25.32 & 25.65 & 26.07 & 32.75 & 24.05 &11.19 \\
\rowcolor{color22}
K2-Chat &22.47 & 24.59 & 24.61 & 24.61 & 30.58 & 21.95 &14.44 \\
\rowcolor{color22}
Mixtral-8x7B-Instruct-v0.1 &22.10 & 24.76 & 24.73 & 26.36 & 34.19 & 20.52 &11.49 \\
\rowcolor{color22}
granite-3.1-8b-instruct &20.83 & 22.85 & 22.92 & 22.26 & 29.48 & 19.79 &13.09 \\
\rowcolor{color22}
Llama-3.1-8B-Instruct &20.50 & 24.07 & 24.52 & 26.12 & 32.82 & 20.37 &7.22 \\
\rowcolor{color22}
Yi-1.5-6B-Chat &19.24 & 21.32 & 21.39 & 22.21 & 27.90 & 18.83 &10.44 \\
\rowcolor{color22}
Qwen2.5-1.5B-Instruct &18.82 & 20.91 & 20.75 & 22.11 & 27.41 & 18.19 &10.45 \\
\rowcolor{color22}
OLMo-2-1124-13B-Instruct &18.66 & 20.46 & 20.60 & 21.80 & 27.10 & 17.85 &10.74 \\
\rowcolor{color22}
gemma-2-2b-it &18.61 & 19.91 & 19.97 & 20.50 & 26.40 & 16.95 &12.85 \\
\rowcolor{color22}
granite-3.1-2b-instruct &17.92 & 19.02 & 19.11 & 19.58 & 23.94 & 17.58 &11.87 \\
\rowcolor{color22}
Mistral-7B-Instruct-v0.3 &17.82 & 19.64 & 19.65 & 20.09 & 26.37 & 16.64 &10.41 \\
\rowcolor{color22}
MAP-Neo-7B-Instruct-v0.1 &17.05 & 18.52 & 18.42 & 18.70 & 23.26 & 16.62 &10.95 \\
\rowcolor{color22}
OLMo-2-1124-7B-Instruct &16.81 & 18.08 & 18.57 & 18.85 & 22.80 & 15.82 &11.90 \\
\rowcolor{color22}
Qwen2.5-0.5B-Instruct &10.77 & 11.92 & 12.47 & 13.47 & 14.90 & 10.88 &6.07 \\
\midrule
\rowcolor{color31}
\multicolumn{8}{c}{\textbf{\textit{Base Models}}}\\
\midrule
\rowcolor{color32}
Qwen2.5-72B & \boxed{34.33} & \boxed{38.08} & \boxed{38.70} & \boxed{39.54} & \boxed{46.20} & \boxed{38.12} &\textbf{15.01} \\
\rowcolor{color32}
Qwen2.5-32B & \textbf{33.16} & \textbf{36.52} & \textbf{37.33} & \textbf{38.29} & \textbf{45.12} & \textbf{36.58} &14.34 \\
\rowcolor{color32}
DeepSeek-V3-Base & \underline{32.14} & \underline{34.79} & \underline{34.58} & \underline{34.71} & 41.28 & \underline{34.50} &\boxed{18.20} \\
\rowcolor{color32}
Qwen2.5-14B & 30.19 & 33.33 & 34.14 & 34.54 & \underline{42.27} & 31.44 &\underline{14.85} \\
\rowcolor{color32}
Yi-1.5-34B & 27.62 & 30.78 & 31.03 & 32.55 & 39.68 & 27.95 &13.86 \\
\rowcolor{color32}
Llama-3.1-70B & 27.22 & 30.52 & 31.28 & 32.55 & 40.78 & 26.95 &12.78 \\
\rowcolor{color32}
Qwen2.5-7B & 25.36 & 28.19 & 28.73 & 29.60 & 36.58 & 25.94 &12.10 \\
\rowcolor{color32}
Llama-3.1-405B & 25.23 & 28.09 & 28.33 & 30.15 & 37.58 & 25.12 &11.86 \\
\rowcolor{color32}
gemma-2-27b & 24.49 & 27.35 & 27.96 & 28.58 & 36.26 & 24.07 &12.27 \\
\rowcolor{color32}
Yi-1.5-9B & 23.10 & 25.40 & 25.69 & 26.17 & 32.52 & 22.98 &12.96 \\
\rowcolor{color32}
gemma-2-9b & 22.56 & 25.19 & 25.47 & 26.26 & 33.88 & 21.34 &12.20 \\
\rowcolor{color32}
Mixtral-8x22B-v0.1 & 22.41 & 24.71 & 25.04 & 25.02 & 32.78 & 21.67 &12.26 \\
\rowcolor{color32}
Mixtral-8x7B-v0.1 & 21.76 & 24.92 & 25.45 & 27.36 & 33.66 & 20.32 &11.13 \\
\rowcolor{color32}
K2 & 20.92 & 23.62 & 23.88 & 24.65 & 30.97 & 20.01 &11.40 \\
\rowcolor{color32}
Yi-1.5-6B & 20.20 & 22.10 & 22.56 & 23.44 & 28.37 & 19.64 &12.20 \\
\rowcolor{color32}
Qwen2.5-3B & 20.14 & 22.81 & 23.30 & 24.42 & 30.42 & 19.81 &9.40 \\
\rowcolor{color32}
Llama-3.1-8B & 19.93 & 22.42 & 22.77 & 23.87 & 30.33 & 18.94 &10.16 \\
\rowcolor{color32}
Mistral-7B-v0.3 & 19.48 & 21.50 & 21.81 & 22.27 & 27.62 & 18.65 &11.96 \\
\rowcolor{color32}
Qwen2.5-1.5B & 17.17 & 19.31 & 19.80 & 21.35 & 24.52 & 16.79 &9.74 \\
\rowcolor{color32}
granite-3.1-2b-base & 16.18 & 17.90 & 17.91 & 18.09 & 23.13 & 14.91 &10.67 \\
\rowcolor{color32}
OLMo-2-1124-13B & 16.07 & 18.75 & 19.82 & 21.37 & 27.24 & 14.41 &6.57 \\
\rowcolor{color32}
MAP-Neo-7B & 15.76 & 17.48 & 18.26 & 19.54 & 22.86 & 14.64 &9.83 \\
\rowcolor{color32}
granite-3.1-8b-base & 15.69 & 16.98 & 16.79 & 16.65 & 20.40 & 15.65 &10.60 \\
\rowcolor{color32}
OLMo-2-1124-7B & 15.15 & 17.62 & 18.30 & 19.60 & 24.43 & 13.83 &7.15 \\
\rowcolor{color32}
gemma-2-2b & 13.57 & 14.98 & 15.43 & 16.21 & 19.12 & 13.47 &7.63 \\
\rowcolor{color32}
Qwen2.5-0.5B & 10.74 & 11.88 & 12.09 & 13.25 & 14.41 & 10.78 &6.65 \\

\bottomrule
\end{tabular}
}
\captionsetup{font=footnotesize}
\caption{\textbf{Detailed Performance Overview on \benchmark - Pivot Table 1.} 
LLMs are scored sample-wise, subfield-wise, field-wise, and discipline-wise levels to ensure fair assessment despite imbalanced question counts. 
The columns Easy(sample), Middle(sample), and Hard(sample) represent average scores according to difficulty. 
The highest score in each column is indicated with a \boxed{box}; the second-best score is in \textbf{bold}, and the third-best score is \underline{underlined}.}
\label{performance1_all}
\end{table}
}

\input{table/result/performance_2}
% \begin{table}[h]
%     \centering
%     \renewcommand{\arraystretch}{1.3}
%     \begin{tabular}{l|c}
%         \hline
%         \textbf{Model} & \textbf{Accuracy (\%)} \\ 
%         \hline
%         llemma\_7b & 74.50 \\ 
%         \hline
%         deepseek-math-7b-instruct & 83.89 \\ 
%         \hline
%         Qwen2.5-Math-1.5B-Instruct & 48.99 \\ 
%         \hline
%         Qwen2.5-Math-7B-Instruct & 66.67 \\ 
%         \hline
%     \end{tabular}
%     \caption{Accuracy of Math-Specific Models}
%     \label{tab:math_models}
% \end{table}

% \begin{table*}[h]
%     \centering
%     \renewcommand{\arraystretch}{1.3}
%     \begin{tabular}{l|cccccc}
%         \hline
%         \textbf{Model} & \textbf{hep-python} & \textbf{hep-java} & \textbf{hep-cpp} & \textbf{hep-js} & \textbf{hep-go} & \textbf{hep-rust} \\ 
%         \hline
%         starcoder2-7b & 67.07 & 55.49 & 53.66 & 48.78 & 50.00 & 59.76 \\ 
%         \hline
%         codegemma-7b-it & 31.71 & 12.20 & 14.63 & 20.12 & 14.02 & 12.20 \\ 
%         \hline
%         Qwen2.5-Coder-3B-Instruct & 84.15 & 85.98 & 86.59 & 85.98 & 86.59 & 84.15 \\ 
%         \hline
%         Qwen2.5-Coder-7B-Instruct & 96.34 & 97.56 & 94.51 & 92.07 & 93.90 & 90.85 \\ 
%         \hline
%     \end{tabular}
%     \caption{Accuracy for Code Specific Models}
%     \label{tab:hep_accuracy}
% \end{table*}
\begin{table*}[htbp!]
\centering
\setlength{\tabcolsep}{4pt} 
\definecolor{apricot}{rgb}{0.95, 0.82, 0.62}
\definecolor{lightgray}{rgb}{0.96, 0.96, 0.96}

\begin{tabular}{l | ccccc | ccccc }
\hline
\rowcolor{apricot} 
\multirow{0}{*}{\textbf{Models}} & \multicolumn{5}{c|}{\textbf{Our Approach}} & \multicolumn{5}{c}{\textbf{Self Rewarding}} \\
\cmidrule(lr){2-6} \cmidrule(lr){7-11}
% \cline{2-11}
 & Chat & Code & Math & Safety & Average & Chat & Code & Math & Safety & Average  \\ \hline
Llama-3.2-1B-Inst & 64.37 & 52.84 & 88.14 & 80.48 & 71.45 & 30.47 & 21.03 & 14.54 & 31.55 & 24.39\\ \hline
\rowcolor{lightgray}Llama-3.2-3B-Inst & 62.09 & 67.17 & \textbf{98.21} & 80.23 & 76.92 & 33.87 & 24.69 & 36.01 & 46.73 & 35.32\\ \hline
Llama-3-8B-Inst & 59.56 & 73.88 & 54.97 & 87.88 & 69.07 & 35.43 & 12.29 & 21.70 & \textbf{58.35} & 31.94\\ \hline
\rowcolor{lightgray}Qwen-2.5-3B-Inst & 60.89  & 80.59  & 46.31  & 86.05 & 68.46 & 26.72 & 23.88 & \textbf{41.61} & 24.43 & 29.16\\ \hline
Qwen-2.5-7B-Inst & \textbf{78.26}  & \textbf{83.13}  & 56.24  & \textbf{93.24} & 77.71 & \textbf{58.73} & \textbf{47.93} & 40.49 & 52.20 & \textbf{49.82}\\ \hline
\rowcolor{lightgray}Mistral-7B-Inst & 61.25 & 70.93  & 96.20 & 83.85 & \textbf{78.05} & 24.55 & 1.6 & 28.18 & 15.39 &  17.43\\ \hline
Gemma2-2B-It & 35.34 & 42.58  & 91.50 & 70.04 & 59.86 & 22.36 & 2.84 & 12.75 & 34.78 & 18.18\\ \hline
\rowcolor{lightgray}Phi-3-Mini-Instruct  & 55.91 & 75.30 & 89.10 & 75.32 & 73.90 & 46.63 & 35.46 & 22.60 & 56.75 & 40.36 \\ \hline
\end{tabular}
\caption{The above table compares our approach with the Self Rewarding approach. The row labels correspond to the model name and the column labels correspond to the sub-categories. The metric used is accuracy where the higher values indicate better performance. }
\label{tab:reward_bench}
\end{table*}



% \begin{table*}[h!]
% \centering

% \begin{tabular}{|l|ccccc|ccccc|}
% \hline
% \multirow{}{}{Models} & \multicolumn{5}{c|}{Our Approach} & \multicolumn{5}{c|}{Self Rewarding} \\
% \cline{2-11}
%  & Chat & Code & Math & Safety & Random & Chat & Code & Math & Safety & Random \\ \hline\hline
% Llama-1B-Inst & 64.37 & 52.84 & 88.14 & 80.48 & 50 & 0 & 0 & 0 & 0 & 50 \\ \hline
% Llama-3B-Inst & 62.09 & 67.17 & \textbf{98.21} & 80.23 & 50 & 0 & 0 & 0 & 0 & 50 \\ \hline
% Llama-8B-Inst & 59.56 & 73.88 & 44.97 & 87.88 & 50 & 0 & 0 & 0 & 0 & 50 \\ \hline
% Qwen-3B-Inst & 60.89  & 80.59  & 46.31  & 86.05 & 50 & 0 & 0 & 0 & 0 & 50 \\ \hline
% Qwen-7B-Inst & \textbf{78.26}  & \textbf{83.13}  & 36.24  & \textbf{93.24}  & 50 & 0 & 0 & 0 & 0 & 50 \\ \hline
% Pythia-1.4B  & 19.17 & 4.47 & 51.90 & 23.61 & 50 & 0 & 0 & 0 & 0 & 50 \\ \hline
% Pythia-6.9B  & 23.67 & 34.55 & 55.26 & 35.60 & 50 & 0 & 0 & 0 & 0 & 50 \\ \hline
% Phi-3-Mini  & 55.91 & 75.30 & 99.10 & 75.32 & 50 & 0 & 0 & 0 & 0 & 50 \\ \hline
% Mistral-7B-Inst & 61.25 & 70.93  & 96.20 & 83.85 & 50 & 0 & 0 & 0 & 0 & 50 \\ \hline
% Gemma2-2B-It & 35.34 & 42.58  & 91.50 & 70.04 & 50 & 0 & 0 & 0 & 0 & 50 \\ \hline
% GPT4-o-Mini & 83.72 & 95.32  & 59.50 & 91.74 & 50 & 0 & 0 & 0 & 0 & 50 \\ \hline
% \end{tabular}
% \caption{The above table compares our approach with the Score Based Self Rewarding approach. The row labels correspond to the model name and the column labels correspond to the sub-categories. The metric used is accuracy where the higher values indicate better performance. }
% \label{tab:model_comparison}
% \end{table*}

% \begin{tabular}{|l||ccccc||ccccc|}
% \hline
% \multirow{}{}{Models} & \multicolumn{5}{c||}{Our Approach} & \multicolumn{5}{c|}{Self Rewarding} \\
% \cline{2-11}
%  & Chat & Code & Math & Safety & Random & Chat & Code & Math & Safety & Random \\ \hline\hline
% Llama-1B-Inst & 64.37 & 52.84 & 88.14 & 80.48 & X1 & 0 & 0 & 0 & 0 & X2 \\ \hline
% Llama-3B-Inst & 62.09 & 67.17 & \textbf{98.21} & 80.23 & X3 & 0 & 0 & 0 & 0 & X4 \\ \hline
% Llama-8B-Inst & 59.56 & 73.88 & 44.97 & 87.88 & X5 & 0 & 0 & 0 & 0 & X6 \\ \hline\hline
% Qwen-3B-Inst & 60.89  & 80.59  & 46.31  & 86.05 & X7 & 0 & 0 & 0 & 0 & X8 \\ \hline
% Qwen-7B-Inst & \textbf{78.26}  & \textbf{83.13}  & 36.24  & \textbf{93.24}  & X9 & 0 & 0 & 0 & 0 & X10 \\ \hline\hline
% Pythia-1.4B  & 0 & 0 & 0 & 0 & X11 & 0 & 0 & 0 & 0 & X12 \\ \hline
% Pythia-6.9B  & 0 & 0 & 0 & 0 & X13 & 0 & 0 & 0 & 0 & X14 \\ \hline\hline
% Phi-3-Mini(3.8B)  & 0 & 0 & 0 & 0 & X15 & 0 & 0 & 0 & 0 & X16 \\ \hline
% Phi-3-Small(7.4B)  & 0 & 0 & 0 & 0 & X17 & 0 & 0 & 0 & 0 & X18 \\ \hline \hline
% Mistral-7B-Inst & 61.25 & 70.93  & 96.20 & 83.85 & X19 & 0 & 0 & 0 & 0 & X20 \\ \hline
% \end{tabular}
% \caption{Reward Bench}
% \label{tab:model_comparison}
% \end{table*}


% \begin{table*}[h!]
% \centering
% \begin{tabular}{l|cccc|cccc}
% \hline
% \multirow{}{}{Models} & \multicolumn{4}{c|}{Our Approach} & \multicolumn{4}{c}{Self Rewarding} \\
% \cline{2-9}
%  & Chat & Code & Math & Safety & Chat & Code & Math & Safety \\
% \hline
% Llama-1B-Inst & 57.62 & 51.46 & 47.57 & 70.24 & 0 & 0 & 0 & 0 \\ \hline
% Llama-3b-Inst & 51.42 & 51.90 & 48.52 & 76.89 & 0 & 0 & 0 & 0 \\ \hline
% Qwen-3B-Inst & 60.98 & 53.07 & 56.33 & 68.96 & 0 & 0 & 0 & 0 \\ \hline
% Qwen-7B-Inst & \textbf{68.73} & \textbf{57.30} & \textbf{58.15} & \textbf{89.49} & 0 & 0 & 0 & 0 \\ \hline
% Mistral-7b-Inst & 54.00 & 52.48 & 52.04 & 71.52 & 0 & 0 & 0 & 0 \\ \hline
% Llama-8B-Inst & 50.64 & 54.97 & 48.70 & 68.25 & 0 & 0 & 0 & 0 \\ \hline

% \end{tabular}
% \caption{RM Bench}
% \label{tab:model_comparison}
% \end{table*}
\begin{table*}[h!]
\centering
\renewcommand{\arraystretch}{0.9}
\setlength{\tabcolsep}{3pt}
\definecolor{apricot}{rgb}{0.95, 0.82, 0.62}
\definecolor{lightgray}{rgb}{0.96, 0.96, 0.96}

\begin{tabular}{l|l|c|c|c|c|c}
\hline
\rowcolor{apricot}
\textbf{Model} & \textbf{Levels} & 
\shortstack{\textbf{RM-Bench} \\ \textbf{chat}} & 
\shortstack{\textbf{RM-Bench} \\ \textbf{code}} & 
\shortstack{\textbf{RM-Bench} \\ \textbf{math}} & 
\shortstack{\textbf{RM-Bench} \\ \textbf{safety response}} & 
\shortstack{\textbf{RM-Bench} \\ \textbf{safety refuse}} \\
\hline
\hline
\multirow{4}{*}{Llama-1B} &  level 1 & 48.06 & 54.39 & 46.31 & 31.85 & 38.73 \\
& level 2 & 64.34 & 55.26 & 48.58 & 69.43 & 53.52 \\
& level 3 & 60.47 & 50.44 & 41.59 & 61.78 & 71.13 \\
& mean & 57.62 & 53.36 & 45.49 & 54.35 & 54.46 \\
\hline
\multirow{4}{*}{Llama-1B-Instruct} & level 1 & 51.16 & 51.32 & 49.53 & 71.34 & 67.61 \\
& level 2 & 61.24 & 53.51 & 47.45 & 68.15 & 77.11 \\
&  level 3 & 60.47 & 49.56 & 45.75 & 73.89 & 63.38 \\
& mean & 57.62 & 51.46 & 47.57 & 71.13 & 69.37 \\
\hline

\multirow{4}{*}{Llama-3B} & 
level 1 & 54.26 & 51.75 & 47.26 & 68.15 & 7.04 \\
& level 2 & 33.33 & 52.19 & 46.12 & 78.34 & 37.32 \\
& level 3 & 33.33 & 49.12 & 45.75 & 36.94 & 55.28 \\
& mean & 40.31 & 51.02 & 46.38 & 61.15 & 33.22 \\
\hline
 \multirow{4}{*}{Llama3b-Instruct} &  level 1 & 56.59 & 50.88 & 50.09 & 87.90 & 55.28 \\
& level 2 & 44.96 & 55.26 & 48.02 & 86.62 & 60.56 \\
&  level 3 & 52.71 & 49.56 & 47.45 & 94.27 & 76.76 \\
& mean & 51.42 & 51.90 & 48.52 & 89.60 & 64.20 \\
\hline


 \multirow{4}{*}{Llama-8B} &  level 1 & 54.26 & 53.51 & 48.02 & 99.36 & 2.46 \\
& level 2 & 56.59 & 56.58 & 51.98 & 83.44 & 29.58 \\
&  level 3 & 50.39 & 51.75 & 47.26 & 64.33 & 63.38 \\
& mean & 53.75 & 53.95 & 49.09 & 82.38 & 31.81 \\
\hline
 \multirow{4}{*}{Llama-8B-Instruct} &  level 1 & 65.12 & 55.70 & 50.28 & 56.05 & 75.00 \\
& level 2 & 36.43 & 55.70 & 49.72 & 96.18 & 30.28 \\
&  level 3 & 50.39 & 53.51 & 46.12 & 64.33 & 87.68 \\
& mean & 50.65 & 54.97 & 48.71 & 72.19 & 64.32 \\
\hline


 \multirow{4}{*}{Mistral-7b} &  level 1 & 50.39 & 46.49 & 52.17 & 96.18 & 20.42 \\
& level 2 & 61.24 & 53.51 & 49.34 & 44.59 & 89.44 \\
&  level 3 & 51.94 & 46.49 & 43.10 & 75.16 & 84.15 \\
& mean & 54.52 & 48.83 & 48.20 & 71.97 & 64.67 \\
\hline
 \multirow{4}{*}{Mistral-7b-Instruct} &  level 1 & 44.19 & 50.88 & 52.55 & 61.78 & 96.48 \\
& level 2 & 58.91 & 52.63 & 55.39 & 39.49 & 81.69 \\
&  level 3 & 58.91 & 53.95 & 48.20 & 52.87 & 96.83 \\
& mean & 54.01 & 52.49 & 52.05 & 51.38 & 91.67 \\
\hline


 \multirow{4}{*}{Qwen2.5-3B} &  level 1 & 65.89 & 48.68 & 54.06 & 95.54 & 94.01 \\
& level 2 & 58.14 & 52.19 & 51.23 & 82.80 & 88.03 \\
&  level 3 & 48.84 & 50.44 & 46.12 & 94.90 & 49.65 \\
& mean & 57.62 & 50.44 & 50.47 & 91.08 & 77.23 \\
\hline

 \multirow{4}{*}{Qwen2.5-3B-Instruct} &  level 1 & 72.87 & 51.32 & 60.87 & 46.50 & 63.38 \\
& level 2 & 55.04 & 53.07 & 57.66 & 31.85 & 90.49 \\
&  level 3 & 55.04 & 54.82 & 50.47 & 84.71 & 96.83 \\
& mean & 60.98 & 53.07 & 56.33 & 54.35 & 83.57 \\
\hline


 \multirow{4}{*}{Qwen2.5-7B} &  level 1 & 72.87 & 56.58 & 56.14 & 100.00 & 100.00 \\
& level 2 & 47.29 & 56.58 & 54.06 & 96.82 & 94.72 \\
&  level 3 & 51.16 & 53.07 & 47.64 & 94.27 & 100.00 \\
& mean & 57.11 & 55.41 & 52.61 & 97.03 & 98.24 \\
\hline
 \multirow{4}{*}{Qwen2.5-7B-Inst} &  level 1 & 80.62 & 58.33 & 62.19 & 91.08 & 100.00 \\
& level 2 & 61.24 & 58.33 & 62.00 & 85.99 & 96.83 \\
&  level 3 & 64.34 & 55.26 & 50.28 & 63.06 & 100.00 \\
& mean & 68.73 & 57.31 & 58.16 & 80.04 & 98.94 \\
\hline
 \multirow{4}{*}{SKYWORK-8b-reward} &  level 1 & 86.04 & 53.07 & 62.38 & 94.90 & 97.18 \\
& level 2 & 55.04 & 53.51 & 65.41 & 82.80 & 98.94 \\
&  level 3 & 41.09 & 48.25 & 66.16 & 87.26 & 100.00 \\
& mean & 60.72 & 51.61 & 64.65 & 88.32 & 98.60 \\
\hline


\end{tabular}
\caption{Performance of various models, across different levels on RM-Bench}
\label{tab:rm_bench_levels}
\end{table*}
\end{document}
