\section{Related Works}
TCP, as a connection-oriented reliable transport protocol, is used by the vast majority of service applications today. Congestion control algorithms are key to ensuring high efficiency of TCP transmission. CUBIC is the default algorithm for the Linux kernel, which treats packet loss as a signal of network congestion. CUBIC uses a cubic function to control the growth of the congestion window, and its congestion window growth is independent of RTT, which allows for good RTT fairness between TCP connections sharing multiple bottleneck links. Unlike CUBIC, delay-based congestion control algorithms treat delay growth as a congestion signal in the network. CUBIC-FIT\cite{a21:J} extends the CUBIC algorithm framework using a delay-based approach. It simulates multiple CUBIC flows in a single TCP connection and adjusts the simulated CUBIC flows using end-to-end queuing delay to improve CUBIC's throughput performance over wireless links. Copa\cite{a23:V} optimizes throughput and delay using a Markov model and detects if the link buffer is full by observing changes in delay. Copa+\cite{a24:W} enhances Copa with a parameter adaptation mechanism and an optimized competitive mode entrance criterion. It can adaptively clear bottleneck buffer occupancy to correctly estimate the base RTT. However, it is too simplistic to judge the network congestion based solely on a single indicator such as packet loss or delay, which may lead to misjudgment of the network conditions. BBR estimates link capacity based on the observed bottleneck bandwidth and round-trip propagation time, and adjusts its data transmission rate to maximize bandwidth utilization without increasing the buffer queue. TCP-Enewreno\cite{a54:M} is designed for 5G network scenarios, where the sender-side estimates the network bandwidth and dynamically adjusts the congestion window based on the estimated bandwidth value. The paper\cite{a57:K} proposes the mmS-TCP algorithm, which modifies the window adjustment mechanism of S-TCP\cite{a58:K}, enhancing the average total throughput and fairness within the protocol. Additionally, by incorporating CoDel\cite{a59:K}, it ensures that the TCP protocol can maintain its performance in millimeter-wave networks. D-TCP\cite{a60:M} estimates the available channel bandwidth, utilizes bandwidth information to derive congestion control factors, and adaptively increases or decreases the congestion window to address fluctuations in millimeter-wave channel conditions. PCC\cite{a37:M} points out that even with extensive modifications, it is difficult to achieve sustained high performance for congestion control algorithms based on TCP's congestion control framework. Therefore, PCC proposes a performance-oriented congestion control framework where the sender-side adjusts its transmission strategy based on observed performance metrics.

The congestion control algorithm mentioned above, based on the sender-side, can only perceive the network condition indirectly. The sender-side adjusts the sending window based on the ACK feedback sent by the receiver-side to understand the degree of network congestion. In wireless networks, especially in high-speed networks like 5G, the sender-side faces difficulty in accurately and promptly perceiving the congestion status of the network. In comparison to sender-side, receiver-side can directly perceive the network and obtain real feedback from the network. Therefore, there are now many congestion control schemes that address issues with TCP transmission by implementing improvements at receiver-side. In DRWA\cite{a25:H}, a countermeasure implemented at receiver-side is designed to solve the buffer problem in resource-constrained environments such as Wifi. Receiver-side increases its receive window to make the RTT closer to its minimum RTT. In DFCSD\cite{a26:P}, AQM (Active Queue Management) is integrated into a loss-based congestion algorithm by controlling the RTAC\cite{a27:H} of the cellular network and implemented at the receiver-side to meet application requirements for high throughput and low latency. HRCC\cite{a37:B} proposes a hybrid receiver-side congestion control framework for WebRTC (Web Real-time Communication). The framework leverages reinforcement learning to observe network links and periodically adjust the estimated bandwidth using heuristic schemes. RBBR\cite{a38:H} present a receiver-driven BBR algorithm, which differs from the original BBR algorithm by estimating the sending rate at receiver-side. The paper\cite{a64:Y} proposes a receiver-side traffic control algorithm that adjusts the sender's upstream buffer by monitoring available upload capacity and dynamically adjusting the receive window. Homa\cite{a39:B} is a receiver-driven low-latency transport protocol for data center scenarios. It uses priority queues in the network to ensure low latency for short messages, and receiver-side dynamically manages priority allocation. AMRT\cite{a40:J} proposes a receiver-driven transmission scheme that uses anti-ECN(Explicit Congestion Notification) marking to increase the transmission rate in situations where link utilization is insufficient. RPO\cite{a41:J} retains the advantage of receiver-driven transmission while making reasonable use of low-priority opportunity packets and ECN marking to improve network utilization.

With the diversification of network scenarios and the complexity of network environments, manual summarization of different network scenario characteristics and the targeted design of corresponding heuristic congestion control algorithms are no longer sufficient to solve the problems of network transmission. Therefore, learning-based congestion control algorithms have begun to receive more and more attention, as RL agents can provide optimal decisions through interaction with the network environment and autonomous learning to further optimize network transmission. Remy\cite{a61:K} is an early attempt to model congestion control, treating it as a partially observable Markov decision process problem, adjusting the transmission window differently for various network states. Aurora\cite{a28:N} is the first to use deep reinforcement learning to drive an agent to automatically explore network congestion control strategies, mapping observed network statistics (such as latency and throughput) to rate selection. Eagle\cite{a62:S} is an online learning algorithm that combines expert knowledge with deep reinforcement learning, enabling the algorithm to adapt to new network conditions. QTCP\cite{a63:W} combines TCP with Q-learning algorithm, allowing the sender-side to learn the optimal congestion control strategy online. Orca\cite{a29:S} combines classical congestion control strategies with advanced modern deep reinforcement learning (DRL) techniques for congestion control, using CUBIC as the underlying logic and deep reinforcement learning for coarse-grained regulation. AUTO\cite{a30:X} proposes a multi-objective reinforcement learning-based adaptive congestion control method, where the policy agent can generate optimal policies for all possible network states and preferences. Libra\cite{a31:Z} proposes a unified congestion control framework that enhances flexibility, adaptability, and practicality by combining the wisdom of classical and reinforcement learning-based congestion control algorithms. RayNet\cite{a32:L} points out that learning-based congestion control protocols are still in the early stages, and proposes a scalable and adaptable simulation framework for developing learning-based network protocols to promote the development of learning-based network protocols. 

Additionally, the diversification of applications has posed challenges for network transmission. Compared to other applications, video streaming requires larger bandwidth and lower latency to ensure a satisfactory user experience. For video streaming, client-side video players typically use Adaptive Bitrate (ABR) algorithms to optimize Quality of Experience (QoE) for users. Pensieve\cite{a42:H} proposes an ABR algorithm based on reinforcement learning, which trains a neural network based on observation results collected by the client-side video player to select appropriate bitrates and resolutions for future video chunks. ABRaider\cite{a43:W} proposes a multi-stage reinforcement learning consisting of offline and online phases. In the offline phase, ABRaider integrates the advantages of various ABR algorithms and formulates strategies suitable for different environments, while in the online phase, it focuses on learning the individual user's environment. However, to improve the quality of video services, it is necessary to consider how to reduce end-to-end congestion delay. The GCC\cite{a44:G} algorithm proposes for video conferencing applications mainly consists of sender-based congestion control based on packet loss and receiver-based congestion control based on delay. Sender-side synthesizes the results of both control algorithms to obtain a final sending rate. Iris\cite{a45:T} utilizes a congestion control model based on statistical learning to control the sending rate and enhances the algorithm's adaptability to the environment by updating the adjustment step size through online learning. Reference\cite{a46:A} explores how to achieve optimal streaming in a 5G environment and provides guidelines for implementing it in 5G networks using the Self-Clocked Rate Adaptation for Multimedia (SCReAM)\cite{a47:I} congestion control algorithm as an example.

To maintain optimal network data transmission performance, congestion control algorithms should accurately identify changes in network bandwidth and rapidly and appropriately respond to these changes. Additionally, the adaptability of congestion control algorithms to the application is crucial. When designing congestion control algorithms, the characteristics and requirements of the upper-layer applications should be taken into consideration.