\section{Related Works}
\label{Sec:Related_Works}
\vspace{-8pt}
Compared to small purely VTOL rotorcraft like quadrotors, FW-UAVs provide several advantages, such as longer flight endurance, lower energy consumption, and the capacity to carry heavier payloads. However, these advantages come at the cost of increased state-coupled dynamics complexities affected as well by unknown aerodynamics effects~\cite{Dobrokhodov2020}. 
Therefore, quickly generating and tracking dynamically feasible trajectories present a significant challenge. Most existing trajectory generation algorithms for FW-UAVs simplify the problem by focusing on kinematic models, bypassing the intricate flight dynamics. In \cite{Chitsaz} an extension of the Dubins path is used to compute a time-optimal trajectory with curvature constraints. A similar approach is found in \cite{ryu2023path}, where Dubins-based motion primitives are modified to incorporate smoother transitions between segments. In \cite{Wang},  the trajectory generation is treated as a kinematic planning problem, connecting lines and arcs of constant curvature. However, when tested in simulation, this method leads to instantaneous accelerations at segment connections, causing dynamic instabilities during trajectory tracking.
Similarly, in \cite{Frazzoli}, a set of dynamically feasible "trim primitives" are concatenated to create a complex motion plan. However, these methods, primarily based on Dubins paths, fail to provide $\mathcal{C}^3$ continuity at the segment junctions \cite{Gros, Johannes}.
Other methods compute short trajectories directly in the system’s states~\cite{Barry, Levin}, imposing high computational costs due to nonlinear flight dynamics. In trajectory generation, differential flatness~\cite{Nieuwstadt} enables transformation from flat output space to state and control input space~\cite{Martin}, widely applied in quadrotors~\cite{Mellinger} for aerobatic maneuvers.

For fixed-wing aircraft, \cite{hauser1997aggressive} introduces differential flatness under coordinated flight, later utilized in~\cite{Bry2015AggressiveFO, Liu}. However, unlike \cite{Liu, Guozheng}, which use offline trajectory generation with MINCO~\cite{wang2022geometrically} and test only in simulation, we propose a fast, online method optimizing Bernstein polynomials~\cite{kielas2019bebot} up to the third derivative. Unlike SP-line~\cite{Johannes_book, Tal} or nonlinear Bezier-based methods~\cite{Celestini}, our approach exploits Bernstein polynomial properties~\cite{kielas2019bebot} for efficient online trajectory generation via quadratic optimization~\cite{GertzOSQP}, ensuring smooth, continuous tracking.

As in~\cite{hauser1997aggressive, Bry2015AggressiveFO}, our trajectories are parameterized in space to maintain constant cruising velocity. Unlike~\cite{Bry2015AggressiveFO, Liu}, we validate in challenging outdoor conditions with small FW aircraft and deploy in real-time. This work bridges theory and practice, providing a clear formulation that can be implemented on real robots.


%Other approaches compute short trajectories directly in the system’s states~\cite{Barry, Levin} resulting in a high computational burden due to the need to account for highly nonlinear flight dynamics. 
%In the realm of trajectory generation, differential flatness \cite{Nieuwstadt}
%allows trajectories to be transformed from flat output space to the state and control input space \cite{Martin}.
%This technique has been widely applied in quadrotors \cite{Mellinger}, demonstrating that position-based trajectories can be effectively tracked to perform aerobatic maneuvers.
%In \cite{hauser1997aggressive}, the differential flatness property for a fixed-wing aircraft flying in coordinated conditions is introduced. This concept has been applied in more recent works like \cite{Bry2015AggressiveFO,Liu}. However, unlike \cite{Liu} and \cite{Guozheng}, where trajectories are generated offline using MINCO \cite{wang2022geometrically} and tested solely in simulation, we propose an online fast trajectory generation algorithm, based on the optimization of Bernestein polynomials \cite{kielas2019bebot} between multiple waypoints up to the third order derivative. Differently from solutions based on SP-lines \cite{Johannes_book, Tal} or simulation based non linear optimization of Bezier Curves starting from interfered fluid dynamical system motion primitives~\cite{Celestini}, our method leverages the mathematical properties of Bernstein polynomials \cite{kielas2019bebot}, and is capable of generating fast and efficient trajectories online through quadratic optimization~\cite{GertzOSQP}, ensuring the continuity in positions and higher derivatives resulting in smooth trajectory tracking. As proposed in \cite{hauser1997aggressive, Bry2015AggressiveFO}, the generated trajectory is parameterized in space, to maintain a constant cruising velocity to keep the aircraft safely aloft. Unlike \cite{Bry2015AggressiveFO, Liu}, we validate our approach in challenging outdoor conditions for small FW and deploy it in real-time onboard. Our work bridges the gap between theory and practice, offering a clear formulation that can be implemented on real robots.
% Additionally, inspired by \cite{Mao}, which applies linearized nonlinear constraints locally.