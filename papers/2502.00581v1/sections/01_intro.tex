\section{Introduction}


In recent years, the deployment of small Fixed-Wing Unmanned Aerial Vehicles (FW-UAVs) has significantly increased across various applications, including environmental monitoring \cite{GREEN2019465}, low-altitude surveillance \cite{Jaimes}, and support for first responders in search and rescue operations \cite{Lyu}. Their popularity is primarily due to their superior endurance, extended operational range, and lower energy consumption compared to traditional Vertical Take-Off and Landing (VTOL) platforms like quadrotors.
Since FW-UAVs cannot hover in place or execute sharp turns and must maintain continuous motion to remain airborne, accurate trajectory planning and precise tracking are essential for their safe operations.  
Given the intrinsic nonlinear and coupled translational and rotational dynamics of FW-UAVs, shown by their non-holonomic nature and the requirements of applying a rolling motion to change their heading,  the generated trajectory must exhibit 
$\mathcal{C}^3$ continuity and be planned to keep the system's states within a safe operating envelope \cite{Tekles, Johannes}.  \begin{figure}[t!]
\includegraphics[width=\columnwidth,  trim=0cm 5cm 0cm 0cm, clip]{figs/Fig1_intro.pdf}
  \caption{Continuous, online trajectory replanning between multiple waypoints during a sample real-world flight.
}
  \label{fig:fig1}
\vspace{-20pt}
\end{figure}

Several solutions often bypass the complexity of flight dynamics and instead focus solely on kinematic models relaxing the $C^3$ continuity or compute short trajectories directly in the system's states, resulting in a high computational burden \cite{Barry, Levin}. In case the coordinated flight condition is respected \cite{hauser1997aggressive}, FW-UAVs can be identified as a differentially flat system similar to multicopters \cite{Mellinger}.  Leveraging the system's differential flatness property, the planning problem can be simplified by directly mapping the flat output variables and their higher derivatives in the system's state space, obtaining the desired input values required to follow the trajectory  \cite{Bry2015AggressiveFO}.


In this work, we propose a novel strategy for real time dynamically feasible trajectory planning and control for FW-UAV. 
Differently from previous works, which rely on offline, computationally inefficient suboptimal optimization~\cite{Guozheng} or optimization on top of primitive Dubins-polynomial curves \cite{LugoCrdenas2014DubinsPG}, our approach is entirely based on Bernstein polynomials~\cite{kielas2022bernstein}, determining the corresponding coefficients via convex quadratic optimization, offering a more efficient and scalable method for real-time trajectory planning. The contribution of the paper can be summarized as
\begin{itemize}
\item We propose a novel trajectory planning and control approach that leverages the differential flatness of FW-UAVs. Our method effectively bridges the gap between theory and practice, providing a clear formulation that can be implemented on real robots.

\item We demonstrate continuous trajectory replanning, which we show is helpful to dynamically adjust the curvature constraint as the UAV advances along its path.

\item We present simulations and real-world results that showcase the effectiveness of our approach, even in challenging conditions for small FW such as wind disturbances, enabling efficient real-time onboard computation of trajectories spanning hundreds of meters.


\end{itemize}
%Moreover, the , assuring its dynamical feasibility.
%Compared existing solutions, our approach has been widely tested in  

%Taking inspiration from the mathematical formulation of the differential flatness provided in \cite{hauser1997aggressive}, we developed a full stack for online planning and control, generating real time dynamical feasible trajectories in the robot flat output space. In comparison with  \cite{Bry2015AggressiveFO}, which builds on top of Dubins-polynomial curves \cite{LugoCrdenas2014DubinsPG}, we proposed a trajectory modeling method based on linear convex quadratic optimization of Bernstein polynomials \cite{Jensen} which support actively the linearization of non linear constraints for dynamically feasibility of the trajectory. 
%Compared to the current state of the art, our method computes online trajectories directly on board relying soleling on the perception information provided by the onboard sensors. 
%The hardware and the software developed in this paper are completely open sourced. 

%The rest of the paper is organized as follow: In Section~\ref{Sec:Related_Works} an overview of the relevant literature is proposed, Section~\ref{sec:System_modeling} presents an overview of the system modeling and the proposed differential flatness euqation for coordinated flight conditio while in Section~\ref{sec:Planning} the planning approach and trajectory generation is discussed. Finally in Section~\ref{sec:Experimental_Results}  the experimental results are presented to the reader while in Section~\ref{sec:Conclusion}  the conclusion and future work are discussed. 

%trim={1cm 7cm 0 3cm}


%Having lunch then going to lab and movung forward here



%In summary, in this paper, we present a strategy for controlling the maneuvers of a fixed-wing unmanned aerial vehicle by exploiting its Differential Flatness property. This approach enables efficient online motion planning for different types of trajectories and precise waypoint navigation. Additionally, we develop a non-linear cost function for generating optimized trajectories that integrate kinematic constraints while passing through specified waypoints. Our approach is validated through extensive testing in the fixed-wing simulator, which we are releasing as an open-source tool for the research community. Additionally, we validated it through real-world experiments to demonstrate its practical applicability and robustness.



% will explain more details about Bernstein in the trajectory section
% As expressed in \cite{Bernstein paper}, Bernstein basis provides numerical stability [30], as well as useful geometric properties and computationally efficient algorithms that can be used to derive and implement efficient trajectory planning algorithms for the computation of trajectory bounds, trajectory extrema, minimum temporal and spatial separation between two trajectories. Bernstein polynomials also allow the representation of continuous time trajectories using low-order approximations between multiple waypoints. 


%Before you need to introduce Bernstein before speaking of non linear constraints 
% Leveraging Bernstein Polynomials,
% However, the non-holonomic nature of the Fixed Wing system,  present unique challenges when related to trajectory generation and tracking.
%Differently from quadrotors, Fixed Wing UAVs use aerodynamical surfaces and continuous motion to stay aloft and control their movements in the air, constrained to generate always enough vertical lift to fly, control algorithms needs to precisely keeps the states of the system always in a safe operational envelope \cite{Tekles}. 