\begin{figure}[t!]
%\includegraphics[width=0.7\columnwidth, scale=1.0, trim={2cm 0.7cm 2cm 2cm}, keepaspectratio]{figs/Fig2_framing.pdf}
\includegraphics[width=0.8\columnwidth, trim={1cm 1cm 3cm 0.5cm}, keepaspectratio]{figs/Fig2_framing.pdf}
  \caption{Frames' visualization and convention. The Velocity frame $\mathcal{V}$ differs from the Body frame $\mathcal{B}$ by the angles $\alpha$ and $\beta$, whereas $\mathcal{I}$ defines the Inertial fixed frame. 
}
 \label{fig:fig2}
\vspace{-10pt}
\end{figure}

\begin{figure*}[!t]
  \centering
  \includegraphics[width=1\textwidth]{figs/Fig_3_software_architecture_reduced.pdf}
  \caption{Architecture overview. The trajectory generated by our planner is forwarded to the Differential Flatness Block, which computes the desired inputs to control the attitude and the longitudinal thrust of the simulated or real robot. %The mission is monitored through the Ground Station. 
  \label{fig:software_architecture}}
  \vspace{-20pt}
\end{figure*}


\section{System Modeling}
\label{sec:System_modeling}

In this section, we outline a simple aircraft model that captures the important features of the coordinated flight condition adopted in the rest of this paper.
As shown in Fig.~\ref{fig:fig2}, we use the following frame convention: the inertial frame is denoted by $\mathcal{I}$ while $\mathcal{B}$ denotes the vehicle’s rigid body frame that it is aligned with the FW's longitudinal, lateral, and vertical axes. The velocity frame $\mathcal{V}$, centered at $\mathcal{B}$, is rotated with respect to the $\mathcal{B}$ by the sideslip angle $\beta$ and the attack angle $\alpha$, therefore always keeping the corresponding $\mathcal{V}_x$ axis aligned with the direction of the aircraft velocity. The purpose of introducing the velocity frame $\mathcal{V}$ is due to the fact that the FW can be subject to lateral winds that may deviate its nose from the desired direction of motion. 
The state of the FW system is defined as $\mathbf{X} = \{\mathbf{x}, \mathbf{\dot{x}}, \mathbf{R}, \bm{\omega}_v\}$, where $\mathbf{x} \in \mathbb{R}^3$ represents the position of FW in inertial frame $\mathcal{I}$,  
%The rotation between $\mathcal{B}$ and  $\mathcal{V}$, is represented by $R_{\mathcal{V}}^{\mathcal{B}}(\alpha, \beta)$, 
while $\mathbf{R} \in SO(3)$ with $\mathbf{R}= \mathbf{R}_{\mathcal{B}}^{\mathcal{I}}\mathbf{R}_{\mathcal{V}}^{\mathcal{B}}$ denotes the rotation of $\mathcal{V}$ with respect to $\mathcal{I}$. This can also be parameterized using Euler angles roll ($\phi$), pitch ($\theta$), and yaw ($\psi$). The velocity of fixed-wing is represented by the velocity vector $\mathbf{\dot{x}} = [\dot{x}_{x}~\dot{x}_{y}~\dot{x}_{z}]^\top$, with a zero lateral velocity component ($\dot{x}_{y} = 0$) to satisfy the coordinated flight condition. This condition is general to different aircraft configurations, assuming that the motion of the plane is aligned with the relative wind direction and executes curves keeping the lift vector always aligned with $\mathcal{V}$ vertical axis. 

%The orientation dynamics can be described by equation $\dot{R} = R\hat{\mathbf{\omega}}_{b}$, where $\hat{\mathbf{\omega}}_{b}$ is the skew-symmetric matrix of the instantaneous angular velocity vector $\boldsymbol{\omega}_b = [\omega_{b_x}, \omega_{b_y}, \omega_{b_z}]^T $ expressed in the body frame. 

The system dynamics can be described as
\begin{equation}
    \begin{split}
    \mathbf{\dot{x}} = V\mathbf{R}\mathbf{e}_1,&~\mathbf{\ddot{x}} = \mathbf{g} + \mathbf{R}\mathbf{a}_v,\\
    \mathbf{\dot{R}} = \mathbf{R}\hat{\bm\omega}_v,~\dot{\bm{\omega}}_v &= \mathbf{J}^{-1}(-\bm{\omega}_v \times \mathbf{J}\bm{\omega}_v + \bm{\tau}),\label{eq:system_dynamics4}
\end{split}
\end{equation}
where $V = \lVert \dot{\mathbf{x}} \rVert$, $\mathbf{g} = [0~0~-g]^{\top}$ is the gravity vector in $\mathcal{I}$, $\mathbf{e}_1 = [1~0~0]^{\top}$ is the versor aligned with the $x$ direction of $\mathcal{V}$, $\mathbf{a_v} = [a_{v_x}~0~a_{v_z}]^\top$ contains the axial and normal accelerations, while $\boldsymbol{\omega}_v = [\omega_{v_x}~\omega_{v_y}~ \omega_{v_z}]^\top$ represents the angular velocity of $\mathcal{V}$ wrt. $\mathcal{I}$, expressed in $\mathcal{V}$ and $\hat{\bm\omega}_v$ its corresponding skew-symmetric matrix. Finally, $\mathbf{J} \in \mathbb{R}^{3 \times 3}$ describes the inertia acting on each direction of the body frame $\mathcal{B}$, while $\bm{\tau}$ expresses the torque applied on the system due to the action of control surfaces, like ailerons, elevators, and rudder. 
Moreover, as described in \cite{hauser1997aggressive}, to maintain coordinated flight conditions, the second and third components of the angular velocity $\boldsymbol{\omega}_v$, are constrained to be 
\begin{equation}
    \omega_{v_y} = -(a_{v_z} + g_{v_z})/V,~\omega_{v_z} = g_{v_y}/V,\label{eq:system_dynamics5}
\end{equation}
where $\mathbf{g}_{v} = \mathbf{R}^{\top} \mathbf{g}$.
Therefore, the coordinated flight conditions do not impose any constraints on $\omega_{v_x}$ of the FW. 

\subsection{Aerodynamics and Propulsion Model}
\label{sec:Aerodynamics_and_prop}
%The coordinated flight model is a so powerful mathematical notation, that despite being very concise, can describe the translational dynamics and attitude kinematics of a fixed wing aircraft, leveraging normal and axial accelerations, respectively $a_{v_x}$ and $a_{v_z}$ and the roll velocity $\omega_1$. 

More realistic attitude dynamics can be obtained by also including the acceleration due to lift, drag, and thrust, respectively $a_L$, $a_D$, and $a_T$, which are generally modeled as a function of the altitude $x_z$, airspeed $V_a$, and angle of attack $\alpha$. In this paper, we consider $a_L$, $a_D$, and $a_T$  to be~\cite{tseng1988calculation}

\begin{align}
a_L &= \frac{\sigma(x_z)V_a^2SC_L}{2m} + a_{L,0}, \\
a_D = &\frac{\sigma(x_z) V_a^2 S C_{d}}{2m},~a_T = T/m + a_D,
\end{align}
where $\sigma$, $S$, $T$, and $m$ are the air density, wing surface area, motor thrust and mass of FW. The lift coefficient $C_L$, initial lift acceleration $a_{L,0}$, and drag coefficient $C_D$ depend on the aerodynamic properties of the aircraft, including its shape and angle of attack. 
Therefore, the axial and normal acceleration inputs to the system are given by
\begin{equation}
    a_{v_x} = a_T \cos{\alpha} - a_D,~a_{v_z} = -a_T \sin{\alpha} - a_L.
\end{equation}


\subsection{Differential Flatness}
\label{sec:differential_flat}
%Adding here a general introduction to diff flat and how to find R des given the versors
%Mathematically, this can be expressed as:
%\begin{align}
%\dot{\mathbf{X}} &= f(\mathbf{X}, \mathbf{u}) \\
%(\mathbf{X}, \mathbf{u}) &= \Psi(z, \dot{z}, ..., z^i) 
%\end{align}
%where $\Psi(\dots)$ denotes the mapping function between the flat outputs $z$ and the system states $\mathbf{X}$ and the inputs $\mathbf{u}$, with $\text{dim}(\mathbf{z}) = \text{dim}(\mathbf{u})$.

This section provides an overview of the Differential Flatness and Feedback Trajectory Tracking blocks shown in Fig. \ref{fig:software_architecture}. A system is considered differentially flat if there exists a set of of flat outputs, such that the system's state and input can be fully described in terms of these outputs and their derivatives.
In the case of an FW system operating under the coordinated flight equations introduced in Section~\ref{sec:System_modeling}, it becomes a feedback linearizable system~\cite{hauser1997aggressive}, where the flat output is represented by the position $\mathbf{x}$, while the inputs to the model are $\mathbf{u} = [\dot{a}_{v_x}~\omega_{v_x}~\dot{a}_{v_z}]^\top$. 
Following \cite{hauser1997aggressive} and differentiating the acceleration expression $\ddot{\mathbf{x}}$ in eq. \eqref{eq:system_dynamics4}, we obtain $\mathbf{x}^{(3)} = \mathbf{R}(\bm{\omega}_v  \times \mathbf{a}_v + \dot{\mathbf{a}}_v)$, which is equivalent to
\begin{align}
\mathbf{x}^{(3)} &= 
\begin{bmatrix}
\omega_{v_y} a_{v_z} \\
\omega_{v_z} a_{v_x} \\
-\omega_{v_y} a_{v_x}
\end{bmatrix}
+ \mathbf{R}
\begin{bmatrix}
\dot{a}_{v_x} \\
-\dot{a}_{v_z}\omega_{v_x} \\
-\dot{a}_{v_z} 
\end{bmatrix}.
\end{align}

Inverting the following expression directly lead to the final differential flatness equation

%Diff flat equation 71 from IJRR paper 
\begin{equation}
\begin{aligned}
\begin{bmatrix}
\dot{a}_{v_x} \\
\omega_{v_x} \\
\dot{a}_{v_z}
\end{bmatrix} = 
\begin{bmatrix}
-\omega_{v_y}a_{v_z} \\
\omega_{v_z}a_{v_x}/a_{v_z} \\
\omega_{v_y}a_{v_x}
\end{bmatrix} + 
\begin{bmatrix}
1 & 0 & 0 \\
0 & -1/a_{v_z} & 0 \\
0 & 0 & 1
\end{bmatrix} \mathbf{R}^\top
\mathbf{x}^{(3)},
\end{aligned}
\label{eq:differential_flatness}
\end{equation}
where $\mathbf{R}=[\mathbf{r}_{x}~\mathbf{r}_{y}~\mathbf{r}_{z}]$ with $\mathbf{r}_{x} = \dot{\mathbf{x}}/\lVert{\dot{\mathbf{x}}}\rVert$, 
$\mathbf{r}_{z} = \mathbf{a}_{n} / a_{v_z}$, and $\mathbf{r}_{y} = \mathbf{r}_{z} \times \mathbf{r}_{x}$. 
Therefore, $a_{v_z} = - \lVert{\mathbf{a}_{n}}\rVert $ where $\mathbf{a}_{n}$ is found by the projection of $\ddot{\mathbf{x}}$ in the normal plane as $\mathbf{a}_{n} = (\ddot{\mathbf{x}}- \mathbf{g} - a_{v_x} \mathbf{r}_{v_x})$, where $a_{v_x} = \mathbf{r}_{x}^\top (\ddot{\mathbf{x}} - \mathbf{g})$. To respect the coordinated flight condition $a_{v_y} = 0$. The differential flatness equation only holds if the flatness constraints, namely $\mathbf{\dot{x}} \neq 0$ and $a_{v_z} \neq 0$, are satisfied. This is intuitive, as the aircraft's lack of hovering capability and the inability to control the system when the aircraft is perpendicular to the desired trajectory direction make these constraints necessary.
%makes possible to find the axial acceleration $a_{v_x} = \mathbf{r}_{v_x}^T (\ddot{\mathbf{x}} - \mathbf{g}) $. The normal acceleration $a_{v_z}$ is found by the projection of $a_{v_x}$ in the normal plane as  $a_{v_z} = -\lVert(\ddot{\mathbf{x}\rVert - g - a_{v_1} \mathbf{r}_{v_x}^T)}$. Thus the full matrix $R$, describing the desired attitude of $\mathcal{V}$ given the input $\mathbf{z}^{(3)}$ can be retrieve knowing that $\mathbf{r}_{v_z} = a_{v_z} / \lVert a_{v_z}\rVert$ and $\mathbf{r}_{v_y} = \mathbf{r}_{v_z} \times \mathbf{r}_{v_x}$, since no accelerations are wanted on $\mathcal{V}_y$.

We define the system's control input sent to the inner attitude controller~\cite{REINHARDT202191,Coates} the desired orientation matrix $\mathbf{R}_{c}$, expressed through Euler angles $\theta_c$, $\phi_c$, and $\psi_c$, along with angular velocities $\omega_{v_x}$, $\omega_{v_y}$, and axial acceleration $a_{v_x}$ represented as thrust $a_T$. This forms the commanded control input $\mathbf{u}_{c} = [\theta_c~\phi_c~\omega_{v_x}~\omega_{v_y}~a_T]^\top$. Specifically, $\mathbf{u}_{c}$ is derived by first calculating $\mathbf{R}_c$ from the previous $\mathbf{R}$ expression. Subsequently, we consider the following cascade PID loop to compute the commanded jerk
\begin{equation}
    \mathbf{x}_{c}^{(3)} = \mathbf{x}^{(3)}_{r} + k_2 \ddot{\mathbf{e}} + k_1 \dot{\mathbf{e}} + k_0 \mathbf{e},
\end{equation}
where $\mathbf{e} = \mathbf{x}_{r}(t) - \mathbf{x}(t)$, and $k_2, k_1, k_0$ the feedback gains. Finally, based on the differential flatness model in eq. \eqref{eq:differential_flatness},  we derive $\omega_{v_x}~\text{and~}\omega_{v_y}$ considering $\mathbf{x}_{c}^{(3)}$ and $\mathbf{R}_c$ in place of $\mathbf{x}^{(3)}$ and $\mathbf{R}$ respectively. This allows to achieve a trajectory tracking given the state feedback $\mathbf{x}(t)$.


\subsection{Trajectory Time Parametrization}
Despite the strength of the differential flatness approach, the  desired tangential acceleration along the trajectory can vary depending on how the trajectory is formulated with respect to time. Due to the natural minimization of the tracking error $\mathbf{e}$ towards the reference trajectory tracking point, an abrupt change of the desired thrust may happen if the trajectory presents variations in the reference velocities $\dot{\mathbf{x}}_{r}$ and acceleration $\ddot{\mathbf{x}}_{r}$.
In this condition, the FW can slow down below a safe cruising airspeed, producing a loss of airflow and control of the aerodynamic surfaces. 

To prevent such a scenario, we introduce a path parameterization variable $s(t)$ that defines how the desired trajectory values are allocated along the path $\mathbf{x}_{r}:= \mathbf{x}_{r}(s(t))$ for $t \geq 0$, where $s$ represents the distance along the desired path. This parameterization enables dynamic inversion of trajectory $\mathbf{x}_{r}(t) $ based on the distance travelled while maintaining a constant cruising velocity. Therefore, eq.~\eqref{eq:differential_flatness} is modified as 


%ensures that the axial acceleration $a_{v_x}$ remains constant and predetermined for the entirety of the trajectory; however, it also means that $a_{v_x}$ is no longer available as control input in equation \ref{eq:differential_flatness}. Thus, apath parameterization $s(t)$ variable is introduced, which follows the desired trajectory $\rho_s(t):= \rho(s(t))$ for $t \geq 0$, where $s$ is the distance along the desired path. This path parameterization enables dynamic inversion of trajectory $\rho(t)$ based on the distance travelled while maintaining a constant cruising velocity. It also provides an alternative state representation where $\dot{a}_{v_x}$ can be expressed as $s^{(3)}$, allowing us to modify the differential flatness equation \ref{eq:differential_flatness} in \cite{hauser1997aggressive} as shown below:

\begin{equation}
\begin{split}
& \mathbf{M}
\begin{bmatrix}
s^{(3)} \\
\omega_{v_x} \\
\dot{a}_{v_z}
\end{bmatrix} = 
\begin{bmatrix}
a_{v_z}\omega_{v_y} + \dot{a}_{v_z}\\
a_{v_x}\omega_{v_z} \\
-a_{v_x}\omega_{v_y}
\end{bmatrix} \\
 &- 
\mathbf{R}^\top \left[3\frac{\delta^2\mathbf{x}_{r}}{\delta s^2}\ddot{s}\dot{s} + \frac{\delta^3\mathbf{x}_{r}}{\delta s^3}\dot{s}^3 + k_2\mathbf{\dddot{e}} + k_1\mathbf{\dot{e}} + k_0\mathbf{e}\right],
\end{split}
\label{eq:time_param_differential_flatness}
\end{equation}
where $\mathbf{M}$ is the decoupling matrix represented as
\begin{equation}
\mathbf{M}
 = 
\begin{bmatrix}
\vdots & 0 & 0\\
\mathbf{R}^\top\frac{\delta\mathbf{x}_{r}}{\delta s}& a_{v_z} & 0 \\
\vdots & 0 & -1
\end{bmatrix}. 
\label{eq:time_param_M}
\end{equation} 

%The primary limitation of this method is to avoid flying perpendicular to the desired path.
%If the plane flies perpendicular, the first column of $R$ (roll) will become orthogonal to $\rho^{'}$, causing $\mathbf{M}$ to become singular and non-invertible. In practice, this situation is unlikely to occur unless the controller is not properly initialized. 

%trackin as just an equation 

%trajectory parametrization 


% ----------------------------------------------------------
% Extra

 %Here stating describing diff flat


%This section outlines the conventions used for defining reference frames in fixed-wing systems, their trajectory, and related control systems. It also introduces the dynamic model of a fixed-wing, which describes its motion, state, and desired control input, which will be used for its coordinated flights and differential flatness formulation.

% Across the paper, we will refer to derivatives of position in $W$ using the notation $\mathbf{\dot{\Delta}, \ddot{\Delta}, \dddot{\Delta}}$. 

%At the end, visualization of the proposed system architecture is depicted in Fig \ref{fig:software_architecture}, which highlights the adaptability and flexibility of the software stack that has been rigorously tested in both simulation and real-world scenarios. The hardware and software configurations for real-world experiments are explained in details in the experimental results section. 


%\subsection{Coordinate System:}

%As shown in the figure, we follow the standard fixed-wing aerodynamic coordinate system convention. The inertial frame $I$ is defined by three axes in the Forward-Left-Up (FLU) convention, while $I_{enu}$ refers to the inertial frame in the East-North-Up (ENU) convention. The fixed-wing rigid body frame $B$ is represented by three axes $[b_x, b_y, b_z]^T$, which are aligned with the aircraft’s longitudinal, lateral, and vertical axes, respectively. The position of body frame $B$ is represented by $\mathbf{x} = [x, y, z]^T$ in the inertial frame $I$ as a translation vector. An intermediate vehicle frame $V$ is introduced, aligned with the inertial frame, and used to express the relative orientation of the body frame $B$ with respect to inertial frame $I$. The transformation between $I$ and $B$ is represented by the transformation matrix $\mathbf{T} = [\mathbf{R}, \mathbf{x}^T]$, where $\mathbf{R}$ is the rigid body rotation matrix between that defines the orientation of frame $B$ relative to $I$.  


%\subsection{System Dynamics:}

% writing draft and will make better once content is written (currently working on this)

%The state of the fixed-wing system $\mathbf{X}$ is defined by several key components: the aircraft position $\mathbf{x} = [x, y, z]^T \in \mathbb{R}^3$; its velocity $\mathbf{v}_{b} = [v_{b_x}, v_{b_y}, v_{b_z}]^T$  represented in the body frame $B$; orientation $\mathbf{R}$ described by Euler angles roll ($\phi$), pitch ($\theta$), yaw ($\psi$) in the intermediate vehicle frame $V$, and the angular velocity $\mathbf{\omega}=[\omega_{b_x}, \omega_{b_y}, \omega_{b_z}]^T$ around each body axis, represented in the body frame $B$. 

%The whole body of a fixed wing is considered a single rigid body and modeled using the Runge-Kutta 4 (RK4) method for its translation and rotational dynamics. The system dynamic model is as follows:


% \begin{equation}
% \begin{subequations}
%     \begin{align}
         
%         \label{eq:1a} \\
%         d - e &= f 
%         \label{eq:1b} \\
%         g \times h &= i 
%         \label{eq:1c}
%     \end{align}
% \end{subequations}
% \end{equation}

%The motion of the system is defined by the normal and axial acceleration ($a_{v_z}$ and $a_{v_x})$ and the roll rate $\omega_x$. Our system is designed to follow the coordinated flights. The state is defined by the position $\mathbf{x} \in \mathbb{R}^3$ of the aircraft represented in $I$ frame, velocity $\mathbf{v}$, and orientation $\mathbf{R}$. The evaluation of orientation $R$ can be expressed by the instantaneous angular velocity $\mathbf{\omega}=[\omega_x, \omega_y, \omega_z]^T$ expressed in the intermediate velocity frame. 

%\subsection{Fixed Wing Coordinated Flight Condition}

%The planning and control developed and described in this paper respect the rule of coordinated flight, which is defined as a condition where the body velocity of the vehicle is contained on the longitudinal plane, thus defining $v^B_y = v^B_z = 0$. The coordinated flight condition is expressed in the velocity frame $\mathcal{V}$, which is different from $B$ by the angle of attack of the vehicle, and it can be described by the rotation $\mathbf{R}_v$ from the body frame $B$. Consequently, velocities and accelerations can be mapped back in the world frame as $\dot{\Delta} = \mathbf{R}_v \mathbf{v}_v$ and $\ddot{\Delta} = \mathbf{g} + \mathbf{R}_v \mathbf{a}_v$, where $\mathbf{g}$ is the gravity vector. To keep the system in condition of coordinated flight the pitch and yaw rates are respectively constrained to be: $\omega_{v_y} = -(a_{v_z} + g_{v_z})/V$ and $\omega_{v_z} = g_{v_y}/V$, where 

%$V = \abs{\dot{\Delta}}$ and $g_{v_y}$ 


%and $g_{v_z}$ represents the component of the gravity vector projected on $\mathbf{v}_{y}$ and $\mathbf{v}_{z}$.

%In the same expression, components $a_{v_x}$ and $a_{v_z}$ represent the axial and normal acceleration of the vehicle, and their derivatives will be the output of the differential flat model presented in the next section

%sent to an inner attitude controller using a cascade PID loop where angular velocities $\omega_{v_x}$ and $\omega_{v_y}$ and the axial acceleration $a_{v_x}$, expressed in terms of thrust $T$ .

%Considering the differential flatness model just derived, the final input to our system consists in the desired orientation matrix $\mathbf{R}$, decomposed in the eulerian angles $\theta$, $\phi$ and $\psi$, the angular velocities $\omega_{v_x}$ and $\omega_{v_y}$ and the axial acceleration $a_{v_x}$, expressed in terms of thrust $T$, defining the vector of the desired controller inputs $\mathbf{u}_{des} = [\theta, \phi, \omega_{v_x}, \omega_{v_y}, T]$, forwarded to an inner attitude controller based on a cascade PID loop. 

%However, as shown in eq. \ref{eq:differential_flatness}, this expression alone stabilize the system on a desired trajectory $\rho(t)$, without accounting for deviation from the desired path $\mathbf{e}(t)$. To integrate the trajectory deviation feedback into eq. \ref{eq:differential_flatness}, the desired trajectory jerk i.e., flat output derivative $\mathbf{z}^{(3)}$ can be modified to include the closed loop error $\mathbf{e} = \rho(t) - \mathbf{x}(t)$, where where $\mathbf{x}(t)$ is the current position of system and $\rho(t)$ is the reference position at given time $t$. In particular, as visualized in \cite{Bry2015AggressiveFO} final equation for $\mathbf{z}^{(3)}$ can be expressed as: