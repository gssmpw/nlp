%%%%%%%%%%%%%%%%%%%%%%%%%%%%%%%%%%%%%%%%%%%%%%%%%%%%%%%%%%%%%%%%%%%%%%%%%%%%%%%%
%2345678901234567890123456789012345678901234567890123456789012345678901234567890
%        1         2         3         4         5         6         7         8

\documentclass[letterpaper, 10 pt, conference]{ieeeconf}  % Comment this line out if you need a4paper

%\documentclass[a4paper, 10pt, conference]{ieeeconf}      % Use this line for a4 paper

\IEEEoverridecommandlockouts                              % This command is only needed if 
                                                          % you want to use the \thanks command

\overrideIEEEmargins                                      % Needed to meet printer 
\usepackage[english]{babel}
\usepackage{amsmath}
\usepackage{amssymb}
\usepackage{booktabs}
\usepackage{siunitx}
\usepackage[dvips]{graphicx}
\usepackage{multirow}
\usepackage{amsfonts}
\usepackage{enumerate}
\usepackage{tabularx}
\usepackage{algorithm,algorithmic}
\usepackage{bm}
\usepackage{enumerate}
\usepackage{adjustbox}
\usepackage{xcolor}
\usepackage{siunitx}
\usepackage{booktabs}
\usepackage{mdframed}
\usepackage{adjustbox}
\usepackage{authblk}
\usepackage{color}
\usepackage{xurl}
\usepackage{subcaption}
\urlstyle{rm}
\usepackage{cite}
\makeatletter
\let\NAT@parse\undefined
\makeatother
\usepackage{hyperref}
\usepackage{relsize}
\usepackage{float}
\usepackage{pifont}


\title{\LARGE \bf
Trajectory Planning and Control for Differentially Flat\\ Fixed-Wing Aerial Systems 
}


\author{Luca Morando$^{1}$$^*$, Sanket A. Salunkhe$^{1}$$^*$, Nishanth Bobbili$^{1}$, Jeffrey Mao$^{1}$, Luca Masci$^{1}$, Hung Nguyen$^{2}$,\\ Cristino de Souza$^{2}$, and Giuseppe Loianno$^{1}$% <-this % stops a space

\thanks{$^*$Equal contribution and authors listed in alphabetical order.}
\thanks{$^1$The authors are with the New York University, Tandon School of Engineering, Brooklyn, NY 11201, USA. {\tt\footnotesize email: \{luca.morando, sas9908, nb3553, jm7752, lm5175, loiannog\}@nyu.edu}.}%
\thanks{$^2$The authors are with the Autonomous Robotics Research Center-Technology Innovation Institute, Abu Dhabi, UAE. {\tt\footnotesize email:  \{hung.tuan,cristino.dsouza\}@tii.ae}.}
\thanks{This work was supported by the Technology Innovation Institute, the NSF CAREER Award 2145277, and the DARPA YFA Grant D22AP00156-00, Qualcomm Research, Nokia, and NYU Wireless. Giuseppe Loianno serves as consultant for the Technology Innovation Institute. This arrangement has been reviewed and approved by the New York University in accordance with its policy on objectivity in research.}
}


\begin{document}



\maketitle
\thispagestyle{empty}
\pagestyle{empty}


%%%%%%%%%%%%%%%%%%%%%%%%%%%%%%%%%%%%%%%%%%%%%%%%%%%%%%%%%%%%%%%%%%%%%%%%%%%%%%%%
\begin{abstract}

Efficient real-time trajectory planning and control for fixed-wing unmanned aerial vehicles is challenging due to their non-holonomic nature, complex dynamics, and the additional uncertainties introduced by unknown aerodynamic effects. 
In this paper, we present a fast and efficient real-time trajectory planning and control approach for fixed-wing unmanned aerial vehicles, leveraging the differential flatness property of fixed-wing aircraft in coordinated flight conditions to generate dynamically feasible trajectories. The approach provides the ability to continuously replan trajectories, which we show is useful to dynamically account for the curvature constraint as the aircraft advances along its path. 
Extensive simulations and real-world experiments validate our approach, showcasing its effectiveness in generating trajectories even in challenging conditions for small FW such as wind disturbances.

% In this paper, we present an efficient real-time trajectory planning and generation framework for fixed-wing aircraft. This framework ensures dynamic feasibility, which is critical for non-holonomic systems like fixed-wings. Due to fixed-wings' continuous forward motion, we require a smooth transition between trajectories at waypoints while maintaining coupled dynamics and constant speed. In our approach, we represent our trajectory using a Bernstein polynomial with continuous replanning to respect non-linear coupled dynamics.  


\end{abstract}


%%%%%%%%%%%%%%%%%%%%%%%%%%%%%%%%%%%%%%%%%%%%%%%%%%%%%%%%%%%%%%%%%%%%%%%%%%%%%%%%

\section{Introduction}\label{sec:Intro} 


Novel view synthesis offers a fundamental approach to visualizing complex scenes by generating new perspectives from existing imagery. 
This has many potential applications, including virtual reality, movie production and architectural visualization \cite{Tewari2022NeuRendSTAR}. 
An emerging alternative to the common RGB sensors are event cameras, which are  
 bio-inspired visual sensors recording events, i.e.~asynchronous per-pixel signals of changes in brightness or color intensity. 

Event streams have very high temporal resolution and are inherently sparse, as they only happen when changes in the scene are observed. 
Due to their working principle, event cameras bring several advantages, especially in challenging cases: they excel at handling high-speed motions 
and have a substantially higher dynamic range of the supported signal measurements than conventional RGB cameras. 
Moreover, they have lower power consumption and require varied storage volumes for captured data that are often smaller than those required for synchronous RGB cameras \cite{Millerdurai_3DV2024, Gallego2022}. 

The ability to handle high-speed motions is crucial in static scenes as well,  particularly with handheld moving cameras, as it helps avoid the common problem of motion blur. It is, therefore, not surprising that event-based novel view synthesis has gained attention, although color values are not directly observed.
Notably, because of the substantial difference between the formats, RGB- and event-based approaches require fundamentally different design choices. %

The first solutions to event-based novel view synthesis introduced in the literature demonstrate promising results \cite{eventnerf, enerf} and outperform non-event-based alternatives for novel view synthesis in many challenging scenarios. 
Among them, EventNeRF \cite{eventnerf} enables novel-view synthesis in the RGB space by assuming events associated with three color channels as inputs. 
Due to its NeRF-based architecture \cite{nerf}, it can handle single objects with complete observations from roughly equal distances to the camera. 
It furthermore has limitations in training and rendering speed: 
the MLP used to represent the scene requires long training time and can only handle very limited scene extents or otherwise rendering quality will deteriorate. 
Hence, the quality of synthesized novel views will degrade for larger scenes. %

We present Event-3DGS (E-3DGS), i.e.,~a new method for novel-view synthesis from event streams using 3D Gaussians~\cite{3dgs} 
demonstrating fast reconstruction and rendering as well as handling of unbounded scenes. 
The technical contributions of this paper are as follows: 
\begin{itemize}
\item With E-3DGS, we introduce the first approach for novel view synthesis from a color event camera that combines 3D Gaussians with event-based supervision. 
\item We present frustum-based initialization, adaptive event windows, isotropic 3D Gaussian regularization and 3D camera pose refinement, and demonstrate that high-quality results can be obtained. %

\item Finally, we introduce new synthetic and real event datasets for large scenes to the community to study novel view synthesis in this new problem setting. 
\end{itemize}
Our experiments demonstrate systematically superior results compared to EventNeRF \cite{eventnerf} and other baselines. 
The source code and dataset of E-3DGS are released\footnote{\url{https://4dqv.mpi-inf.mpg.de/E3DGS/}}. 





\section{Related Work}
\label{lit_review}

\begin{highlight}
{

Our research builds upon {\em (i)} Assessing Web Accessibility, {\em (ii)} End-User Accessibility Repair, and {\em (iii)} Developer Tools for Accessibility.

\subsection{Assessing Web Accessibility}
From the earliest attempts to set standards and guidelines, web accessibility has been shaped by a complex interplay of technical challenges, legal imperatives, and educational campaigns. Over the past 25 years, stakeholders have sought to improve digital inclusion by establishing foundational standards~\cite{chisholm2001web, caldwell2008web}, enforcing legal obligations~\cite{sierkowski2002achieving, yesilada2012understanding}, and promoting a broader culture of accessibility awareness among developers~\cite{sloan2006contextual, martin2022landscape, pandey2023blending}. 
Despite these longstanding efforts, systemic accessibility issues persist. According to the 2024 WebAIM Million report~\cite{webaim2024}, 95.9\% of the top one million home pages contained detectable WCAG violations, averaging nearly 57 errors per page. 
These errors take many forms: low color contrast makes the interface difficult for individuals with color deficiency or low vision to read text; missing alternative text leaves users relying on screen readers without crucial visual context; and unlabeled form inputs or empty links and buttons hinder people who navigate with assistive technologies from completing basic tasks. 
Together, these accessibility issues not only limit user access to critical online resources such as healthcare, education, and employment but also result in significant legal risks and lost opportunities for businesses to engage diverse audiences. Addressing these pervasive issues requires systematic methods to identify, measure, and prioritize accessibility barriers, which is the first step toward achieving meaningful improvements.

Prior research has introduced methods blending automation and human evaluation to assess web accessibility. Hybrid approaches like SAMBA combine automated tools with expert reviews to measure the severity and impact of barriers, enhancing evaluation reliability~\cite{brajnik2007samba}. Quantitative metrics, such as Failure Rate and Unified Web Evaluation Methodology, support large-scale monitoring and comparative analysis, enabling cost-effective insights~\cite{vigo2007quantitative, martins2024large}. However, automated tools alone often detect less than half of WCAG violations and generate false positives, emphasizing the need for human interpretation~\cite{freire2008evaluation, vigo2013benchmarking}. Recent progress with large pretrained models like Large Language Models (LLMs)~\cite{dubey2024llama,bai2023qwen} and Large Multimodal Models (LMMs)~\cite{liu2024visual, bai2023qwenvl} offers a promising step forward, automating complex checks like non-text content evaluation and link purposes, achieving higher detection rates than traditional tools~\cite{lopez2024turning, delnevo2024interaction}. Yet, these large models face challenges, including dependence on training data, limited contextual judgment, and the inability to simulate real user experiences. These limitations underscore the necessity of combining models with human oversight for reliable, user-centered evaluations~\cite{brajnik2007samba, vigo2013benchmarking, delnevo2024interaction}. 

Our work builds on these prior efforts and recent advancements by leveraging the capabilities of large pretrained models while addressing their limitations through a developer-centric approach. CodeA11y integrates LLM-powered accessibility assessments, tailored accessibility-aware system prompts, and a dedicated accessibility checker directly into GitHub Copilot---one of the most widely used coding assistants. Unlike standalone evaluation tools, CodeA11y actively supports developers throughout the coding process by reinforcing accessibility best practices, prompting critical manual validations, and embedding accessibility considerations into existing workflows.
% This pervasive shortfall reflects the difficulty of scaling traditional approaches---such as manual audits and automated tools---that either demand immense human effort or lack the nuanced understanding needed to capture real-world user experiences. 
%
% In response, a new wave of AI-driven methods, many powered by large language models (LLMs), is emerging to bridge these accessibility detection and assessment gaps. Early explorations, such as those by Morillo et al.~\cite{morillo2020system}, introduced AI-assisted recommendations capable of automatic corrections, illustrating how computational intelligence can tackle the repetitive, common errors that plague large swaths of the web. Building on this foundation, Huang et al.~\cite{huang2024access} proposed ACCESS, a prompt-engineering framework that streamlines the identification and remediation of accessibility violations, while López-Gil et al.~\cite{lopez2024turning} demonstrated how LLMs can help apply WCAG success criteria more consistently---reducing the reliance on manual effort. Beyond these direct interventions, recent work has also begun integrating user experiences more seamlessly into the evaluation process. For example, Huq et al.~\cite{huq2024automated} translate user transcripts and corresponding issues into actionable test reports, ensuring that accessibility improvements align more closely with authentic user needs.
% However, as these AI-driven solutions evolve, researchers caution against uncritical adoption. Othman et al.~\cite{othman2023fostering} highlight that while LLMs can accelerate remediation, they may also introduce biases or encourage over-reliance on automated processes. Similarly, Delnevo et al.~\cite{delnevo2024interaction} emphasize the importance of contextual understanding and adaptability, pointing to the current limitations of LLM-based systems in serving the full spectrum of user needs. 
% In contrast to this backdrop, our work introduces and evaluates CodeA11y, an LLM-augmented extension for GitHub Copilot that not only mitigates these challenges by providing more consistent guidance and manual validation prompts, but also aligns AI-driven assistance with developers’ workflows, ultimately contributing toward more sustainable propulsion for building accessible web.

% Broader implications of inaccessibility—legal compliance, ethical concerns, and user experience
% A Historical Review of Web Accessibility Using WAVE
% "I tend to view ads almost like a pestilence": On the Accessibility Implications of Mobile Ads for Blind Users

% In the research domain, several methods have been developed to assess and enhance web accessibility. These include incorporating feedback into developer tools~\cite{adesigner, takagi2003accessibility, bigham2010accessibility} and automating the creation of accessibility tests and reports for user interfaces~\cite{swearngin2024towards, taeb2024axnav}. 

% Prior work has also studied accessibility scanners as another avenue of AI to improve web development practices~\cite{}.
% However, a persistent challenge is that developers need to be aware of these tools to utilize them effectively. With recent advancements in LLMs, developers might now build accessible websites with less effort using AI assistants. However, the impact of these assistants on the accessibility of their generated code remains unclear. This study aims to investigate these effects.

\subsection{End-user Accessibility Repair}
In addition to detecting accessibility errors and measuring web accessibility, significant research has focused on fixing these problems.
Since end-users are often the first to notice accessibility problems and have a strong incentive to address them, systems have been developed to help them report or fix these problems.

Collaborative, or social accessibility~\cite{takagi2009collaborative,sato2010social}, enabled these end-user contributions to be scaled through crowd-sourcing.
AccessMonkey~\cite{bigham2007accessmonkey} and Accessibility Commons~\cite{kawanaka2008accessibility} were two examples of repositories that store accessibility-related scripts and metadata, respectively.
Other work has developed browser extensions that leverage crowd-sourced databases to automatically correct reading order, alt-text, color contrast, and interaction-related issues~\cite{sato2009s,huang2015can}.

One drawback of collaborative accessibility approaches is that they cannot fix problems for an ``unseen'' web page on-demand, so many projects aim to automatically detect and improve interfaces without the need for an external source of fixes.
A large body of research has focused on making specific web media (e.g., images~\cite{gleason2019making,guinness2018caption, twitterally, gleason2020making, lee2021image}, design~\cite{potluri2019ai,li2019editing, peng2022diffscriber, peng2023slide}, and videos~\cite{pavel2020rescribe,peng2021say,peng2021slidecho,huh2023avscript}) accessible through a combination of machine learning (ML) and user-provided fixes.
Other work has focused on applying more general fixes across all websites.

Opportunity accessibility addressed a common accessibility problem of most websites: by default, content is often hard to see for people with visual impairments, and many users, especially older adults, do not know how to adjust or enable content zooming~\cite{bigham2014making}.
To this end, a browser script (\texttt{oppaccess.js}) was developed that automatically adjusted the browser's content zoom to maximally enlarge content without introducing adverse side-effects (\textit{e.g.,} content overlap).
While \texttt{oppaccess.js} primarily targeted zoom-related accessibility, recent work aimed to enable larger types of changes, by using LLMs to modify the source code of web pages based on user questions or directives~\cite{li2023using}.

Several efforts have been focused on improving access to desktop and mobile applications, which present additional challenges due to the unavailability of app source code (\textit{e.g.,} HTML).
Prefab is an approach that allows graphical UIs to be modified at runtime by detecting existing UI widgets, then replacing them~\cite{dixon2010prefab}.
Interaction Proxies used these runtime modification strategies to ``repair'' Android apps by replacing inaccessible widgets with improved alternatives~\cite{zhang2017interaction, zhang2018robust}.
The widget detection strategies used by these systems previously relied on a combination of heuristics and system metadata (\textit{e.g.,} the view hierarchy), which are incomplete or missing in the accessible apps.
To this end, ML has been employed to better localize~\cite{chen2020object} and repair UI elements~\cite{chen2020unblind,zhang2021screen,wu2023webui,peng2025dreamstruct}.

In general, end-user solutions to repairing application accessibility are limited due to the lack of underlying code and knowledge of the semantics of the intended content.

\subsection{Developer Tools for Accessibility}
Ultimately, the best solution for ensuring an accessible experience lies with front-end developers. Many efforts have focused on building adequate tooling and support to help developers with ensuring that their UI code complies with accessibility standards.

Numerous automated accessibility testing tools have been created to help developers identify accessibility issues in their code: i) static analysis tools, such as IBM Equal Access Accessibility Checker~\cite{ibm2024toolkit} or Microsoft Accessibility Insights~\cite{accessibilityinsights2024}, scan the UI code's compliance with predefined rules derived from accessibility guidelines; and ii) dynamic or runtime accessibility scanners, such as Chrome Devtools~\cite{chromedevtools2024} or axe-Core Accessibility Engine~\cite{deque2024axe}, perform real-time testing on user interfaces to detect interaction issues not identifiable from the code structure. While these tools greatly reduce the manual effort required for accessibility testing, they are often criticized for their limited coverage. Thus, experts often recommend manually testing with assistive technologies to uncover more complex interaction issues. Prior studies have created accessibility crawlers that either assist in developer testing~\cite{swearngin2024towards,taeb2024axnav} or simulate how assistive technologies interact with UIs~\cite{10.1145/3411764.3445455, 10.1145/3551349.3556905, 10.1145/3544548.3580679}.

Similar to end-user accessibility repair, research has focused on generating fixes to remediate accessibility issues in the UI source code. Initial attempts developed heuristic-based algorithms for fixing specific issues, for instance, by replacing text or background color attributes~\cite{10.1145/3611643.3616329}. More recent work has suggested that the code-understanding capabilities of LLMs allow them to suggest more targeted fixes.
For example, a study demonstrated that prompting ChatGPT to fix identified WCAG compliance issues in source code could automatically resolve a significant number of them~\cite{othman2023fostering}. Researchers have sought to leverage this capability by employing a multi-agent LLM architecture to automatically identify and localize issues in source code and suggest potential code fixes~\cite{mehralian2024automated}.

While the approaches mentioned above focus on assessing UI accessibility of already-authored code (\textit{i.e.,} fixing existing code), there is potential for more proactive approaches.
For example, LLMs are often used by developers to generate UI source code from natural language descriptions or tab completions~\cite{chen2021evaluating,GitHubCopilot,lozhkov2024starcoder,hui2024qwen2,roziere2023code,zheng2023codegeex}, but LLMs frequently produce inaccessible code by default~\cite{10.1145/3677846.3677854,mowar2024tab}, leading to inaccessible output when used by developers without sufficient awareness of accessibility knowledge.
The primary focus of this paper is to design a more accessibility-aware coding assistant that both produces more accessible code without manual intervention (\textit{e.g.,} specific user prompting) and gradually enables developers to implement and improve accessibility of automatically-generated code through IDE UI modifications (\textit{e.g.}, reminder notifications).

}
\end{highlight}



% Work related to this paper includes {\em (i)} Web Accessibility and {\em (ii)} Developer Practices in AI-Assisted Programming.

% \ipstart{Web Accessibility: Practice, Evaluation, and Improvements} Substantial efforts have been made to set accessibility standards~\cite{chisholm2001web, caldwell2008web}, establish legal requirements~\cite{sierkowski2002achieving, yesilada2012understanding}, and promote education and advocacy among developers~\cite{sloan2006contextual, martin2022landscape, pandey2023blending}. In the research domain, several methods have been developed to assess and enhance web accessibility. These include incorporating feedback into developer tools~\cite{adesigner, takagi2003accessibility, bigham2010accessibility} and automating the creation of accessibility tests and reports for user interfaces~\cite{swearngin2024towards, taeb2024axnav}. 
% % Prior work has also studied accessibility scanners as another avenue of AI to improve web development practices~\cite{}.
% However, a persistent challenge is that developers need to be aware of these tools to utilize them effectively. With recent advancements in LLMs, developers might now build accessible websites with less effort using AI assistants. However, the impact of these assistants on the accessibility of their generated code remains unclear. This study aims to investigate these effects.

% \ipstart{Developer Practices in AI-Assisted Programming}
% Recent usability research on AI-assisted development has examined the interaction strategies of developers while using AI coding assistants~\cite{barke2023grounded}.
% They observed developers interacted with these assistants in two modes -- 1) \textit{acceleration mode}: associated with shorter completions and 2) \textit{exploration mode}: associated with long completions.
% Liang {\em et al.} \cite{liang2024large} found that developers are driven to use AI assistants to reduce their keystrokes, finish tasks faster, and recall the syntax of programming languages. On the other hand, developers' reason for rejecting autocomplete suggestions was the need for more consideration of appropriate software requirements. This is because primary research on code generation models has mainly focused on functional correctness while often sidelining non-functional requirements such as latency, maintainability, and security~\cite{singhal2024nofuneval}. Consequently, there have been increasing concerns about the security implications of AI-generated code~\cite{sandoval2023lost}. Similarly, this study focuses on the effectiveness and uptake of code suggestions among developers in mitigating accessibility-related vulnerabilities. 


% ============================= additional rw ============================================
% - Paulina Morillo, Diego Chicaiza-Herrera, and Diego Vallejo-Huanga. 2020. System of Recommendation and Automatic Correction of Web Accessibility Using Artificial Intelligence. In Advances in Usability and User Experience, Tareq Ahram and Christianne Falcão (Eds.). Springer International Publishing, Cham, 479–489
% - Juan-Miguel López-Gil and Juanan Pereira. 2024. Turning manual web accessibility success criteria into automatic: an LLM-based approach. Universal Access in the Information Society (2024). https://doi.org/10.1007/s10209-024-01108-z
% - s
% - Calista Huang, Alyssa Ma, Suchir Vyasamudri, Eugenie Puype, Sayem Kamal, Juan Belza Garcia, Salar Cheema, and Michael Lutz. 2024. ACCESS: Prompt Engineering for Automated Web Accessibility Violation Corrections. arXiv:2401.16450 [cs.HC] https://arxiv.org/abs/2401.16450
% - Syed Fatiul Huq, Mahan Tafreshipour, Kate Kalcevich, and Sam Malek. 2025. Automated Generation of Accessibility Test Reports from Recorded User Transcripts. In Proceedings of the 47th International Conference on Software Engineering (ICSE) (Ottawa, Ontario, Canada). IEEE. https://ics.uci.edu/~seal/publications/2025_ICSE_reca11.pdf To appear in IEEE Xplore
% - Achraf Othman, Amira Dhouib, and Aljazi Nasser Al Jabor. 2023. Fostering websites accessibility: A case study on the use of the Large Language Models ChatGPT for automatic remediation. In Proceedings of the 16th International Conference on PErvasive Technologies Related to Assistive Environments (Corfu, Greece) (PETRA ’23). Association for Computing Machinery, New York, NY, USA, 707–713. https://doi.org/10.1145/3594806.3596542
% - Zsuzsanna B. Palmer and Sushil K. Oswal. 0. Constructing Websites with Generative AI Tools: The Accessibility of Their Workflows and Products for Users With Disabilities. Journal of Business and Technical Communication 0, 0 (0), 10506519241280644. https://doi.org/10.1177/10506519241280644
% ============================= additional rw ============================================
\begin{figure}[t!]
%\includegraphics[width=0.7\columnwidth, scale=1.0, trim={2cm 0.7cm 2cm 2cm}, keepaspectratio]{figs/Fig2_framing.pdf}
\includegraphics[width=0.8\columnwidth, trim={1cm 1cm 3cm 0.5cm}, keepaspectratio]{figs/Fig2_framing.pdf}
  \caption{Frames' visualization and convention. The Velocity frame $\mathcal{V}$ differs from the Body frame $\mathcal{B}$ by the angles $\alpha$ and $\beta$, whereas $\mathcal{I}$ defines the Inertial fixed frame. 
}
 \label{fig:fig2}
\vspace{-10pt}
\end{figure}

\begin{figure*}[!t]
  \centering
  \includegraphics[width=1\textwidth]{figs/Fig_3_software_architecture_reduced.pdf}
  \caption{Architecture overview. The trajectory generated by our planner is forwarded to the Differential Flatness Block, which computes the desired inputs to control the attitude and the longitudinal thrust of the simulated or real robot. %The mission is monitored through the Ground Station. 
  \label{fig:software_architecture}}
  \vspace{-20pt}
\end{figure*}


\section{System Modeling}
\label{sec:System_modeling}

In this section, we outline a simple aircraft model that captures the important features of the coordinated flight condition adopted in the rest of this paper.
As shown in Fig.~\ref{fig:fig2}, we use the following frame convention: the inertial frame is denoted by $\mathcal{I}$ while $\mathcal{B}$ denotes the vehicle’s rigid body frame that it is aligned with the FW's longitudinal, lateral, and vertical axes. The velocity frame $\mathcal{V}$, centered at $\mathcal{B}$, is rotated with respect to the $\mathcal{B}$ by the sideslip angle $\beta$ and the attack angle $\alpha$, therefore always keeping the corresponding $\mathcal{V}_x$ axis aligned with the direction of the aircraft velocity. The purpose of introducing the velocity frame $\mathcal{V}$ is due to the fact that the FW can be subject to lateral winds that may deviate its nose from the desired direction of motion. 
The state of the FW system is defined as $\mathbf{X} = \{\mathbf{x}, \mathbf{\dot{x}}, \mathbf{R}, \bm{\omega}_v\}$, where $\mathbf{x} \in \mathbb{R}^3$ represents the position of FW in inertial frame $\mathcal{I}$,  
%The rotation between $\mathcal{B}$ and  $\mathcal{V}$, is represented by $R_{\mathcal{V}}^{\mathcal{B}}(\alpha, \beta)$, 
while $\mathbf{R} \in SO(3)$ with $\mathbf{R}= \mathbf{R}_{\mathcal{B}}^{\mathcal{I}}\mathbf{R}_{\mathcal{V}}^{\mathcal{B}}$ denotes the rotation of $\mathcal{V}$ with respect to $\mathcal{I}$. This can also be parameterized using Euler angles roll ($\phi$), pitch ($\theta$), and yaw ($\psi$). The velocity of fixed-wing is represented by the velocity vector $\mathbf{\dot{x}} = [\dot{x}_{x}~\dot{x}_{y}~\dot{x}_{z}]^\top$, with a zero lateral velocity component ($\dot{x}_{y} = 0$) to satisfy the coordinated flight condition. This condition is general to different aircraft configurations, assuming that the motion of the plane is aligned with the relative wind direction and executes curves keeping the lift vector always aligned with $\mathcal{V}$ vertical axis. 

%The orientation dynamics can be described by equation $\dot{R} = R\hat{\mathbf{\omega}}_{b}$, where $\hat{\mathbf{\omega}}_{b}$ is the skew-symmetric matrix of the instantaneous angular velocity vector $\boldsymbol{\omega}_b = [\omega_{b_x}, \omega_{b_y}, \omega_{b_z}]^T $ expressed in the body frame. 

The system dynamics can be described as
\begin{equation}
    \begin{split}
    \mathbf{\dot{x}} = V\mathbf{R}\mathbf{e}_1,&~\mathbf{\ddot{x}} = \mathbf{g} + \mathbf{R}\mathbf{a}_v,\\
    \mathbf{\dot{R}} = \mathbf{R}\hat{\bm\omega}_v,~\dot{\bm{\omega}}_v &= \mathbf{J}^{-1}(-\bm{\omega}_v \times \mathbf{J}\bm{\omega}_v + \bm{\tau}),\label{eq:system_dynamics4}
\end{split}
\end{equation}
where $V = \lVert \dot{\mathbf{x}} \rVert$, $\mathbf{g} = [0~0~-g]^{\top}$ is the gravity vector in $\mathcal{I}$, $\mathbf{e}_1 = [1~0~0]^{\top}$ is the versor aligned with the $x$ direction of $\mathcal{V}$, $\mathbf{a_v} = [a_{v_x}~0~a_{v_z}]^\top$ contains the axial and normal accelerations, while $\boldsymbol{\omega}_v = [\omega_{v_x}~\omega_{v_y}~ \omega_{v_z}]^\top$ represents the angular velocity of $\mathcal{V}$ wrt. $\mathcal{I}$, expressed in $\mathcal{V}$ and $\hat{\bm\omega}_v$ its corresponding skew-symmetric matrix. Finally, $\mathbf{J} \in \mathbb{R}^{3 \times 3}$ describes the inertia acting on each direction of the body frame $\mathcal{B}$, while $\bm{\tau}$ expresses the torque applied on the system due to the action of control surfaces, like ailerons, elevators, and rudder. 
Moreover, as described in \cite{hauser1997aggressive}, to maintain coordinated flight conditions, the second and third components of the angular velocity $\boldsymbol{\omega}_v$, are constrained to be 
\begin{equation}
    \omega_{v_y} = -(a_{v_z} + g_{v_z})/V,~\omega_{v_z} = g_{v_y}/V,\label{eq:system_dynamics5}
\end{equation}
where $\mathbf{g}_{v} = \mathbf{R}^{\top} \mathbf{g}$.
Therefore, the coordinated flight conditions do not impose any constraints on $\omega_{v_x}$ of the FW. 

\subsection{Aerodynamics and Propulsion Model}
\label{sec:Aerodynamics_and_prop}
%The coordinated flight model is a so powerful mathematical notation, that despite being very concise, can describe the translational dynamics and attitude kinematics of a fixed wing aircraft, leveraging normal and axial accelerations, respectively $a_{v_x}$ and $a_{v_z}$ and the roll velocity $\omega_1$. 

More realistic attitude dynamics can be obtained by also including the acceleration due to lift, drag, and thrust, respectively $a_L$, $a_D$, and $a_T$, which are generally modeled as a function of the altitude $x_z$, airspeed $V_a$, and angle of attack $\alpha$. In this paper, we consider $a_L$, $a_D$, and $a_T$  to be~\cite{tseng1988calculation}

\begin{align}
a_L &= \frac{\sigma(x_z)V_a^2SC_L}{2m} + a_{L,0}, \\
a_D = &\frac{\sigma(x_z) V_a^2 S C_{d}}{2m},~a_T = T/m + a_D,
\end{align}
where $\sigma$, $S$, $T$, and $m$ are the air density, wing surface area, motor thrust and mass of FW. The lift coefficient $C_L$, initial lift acceleration $a_{L,0}$, and drag coefficient $C_D$ depend on the aerodynamic properties of the aircraft, including its shape and angle of attack. 
Therefore, the axial and normal acceleration inputs to the system are given by
\begin{equation}
    a_{v_x} = a_T \cos{\alpha} - a_D,~a_{v_z} = -a_T \sin{\alpha} - a_L.
\end{equation}


\subsection{Differential Flatness}
\label{sec:differential_flat}
%Adding here a general introduction to diff flat and how to find R des given the versors
%Mathematically, this can be expressed as:
%\begin{align}
%\dot{\mathbf{X}} &= f(\mathbf{X}, \mathbf{u}) \\
%(\mathbf{X}, \mathbf{u}) &= \Psi(z, \dot{z}, ..., z^i) 
%\end{align}
%where $\Psi(\dots)$ denotes the mapping function between the flat outputs $z$ and the system states $\mathbf{X}$ and the inputs $\mathbf{u}$, with $\text{dim}(\mathbf{z}) = \text{dim}(\mathbf{u})$.

This section provides an overview of the Differential Flatness and Feedback Trajectory Tracking blocks shown in Fig. \ref{fig:software_architecture}. A system is considered differentially flat if there exists a set of of flat outputs, such that the system's state and input can be fully described in terms of these outputs and their derivatives.
In the case of an FW system operating under the coordinated flight equations introduced in Section~\ref{sec:System_modeling}, it becomes a feedback linearizable system~\cite{hauser1997aggressive}, where the flat output is represented by the position $\mathbf{x}$, while the inputs to the model are $\mathbf{u} = [\dot{a}_{v_x}~\omega_{v_x}~\dot{a}_{v_z}]^\top$. 
Following \cite{hauser1997aggressive} and differentiating the acceleration expression $\ddot{\mathbf{x}}$ in eq. \eqref{eq:system_dynamics4}, we obtain $\mathbf{x}^{(3)} = \mathbf{R}(\bm{\omega}_v  \times \mathbf{a}_v + \dot{\mathbf{a}}_v)$, which is equivalent to
\begin{align}
\mathbf{x}^{(3)} &= 
\begin{bmatrix}
\omega_{v_y} a_{v_z} \\
\omega_{v_z} a_{v_x} \\
-\omega_{v_y} a_{v_x}
\end{bmatrix}
+ \mathbf{R}
\begin{bmatrix}
\dot{a}_{v_x} \\
-\dot{a}_{v_z}\omega_{v_x} \\
-\dot{a}_{v_z} 
\end{bmatrix}.
\end{align}

Inverting the following expression directly lead to the final differential flatness equation

%Diff flat equation 71 from IJRR paper 
\begin{equation}
\begin{aligned}
\begin{bmatrix}
\dot{a}_{v_x} \\
\omega_{v_x} \\
\dot{a}_{v_z}
\end{bmatrix} = 
\begin{bmatrix}
-\omega_{v_y}a_{v_z} \\
\omega_{v_z}a_{v_x}/a_{v_z} \\
\omega_{v_y}a_{v_x}
\end{bmatrix} + 
\begin{bmatrix}
1 & 0 & 0 \\
0 & -1/a_{v_z} & 0 \\
0 & 0 & 1
\end{bmatrix} \mathbf{R}^\top
\mathbf{x}^{(3)},
\end{aligned}
\label{eq:differential_flatness}
\end{equation}
where $\mathbf{R}=[\mathbf{r}_{x}~\mathbf{r}_{y}~\mathbf{r}_{z}]$ with $\mathbf{r}_{x} = \dot{\mathbf{x}}/\lVert{\dot{\mathbf{x}}}\rVert$, 
$\mathbf{r}_{z} = \mathbf{a}_{n} / a_{v_z}$, and $\mathbf{r}_{y} = \mathbf{r}_{z} \times \mathbf{r}_{x}$. 
Therefore, $a_{v_z} = - \lVert{\mathbf{a}_{n}}\rVert $ where $\mathbf{a}_{n}$ is found by the projection of $\ddot{\mathbf{x}}$ in the normal plane as $\mathbf{a}_{n} = (\ddot{\mathbf{x}}- \mathbf{g} - a_{v_x} \mathbf{r}_{v_x})$, where $a_{v_x} = \mathbf{r}_{x}^\top (\ddot{\mathbf{x}} - \mathbf{g})$. To respect the coordinated flight condition $a_{v_y} = 0$. The differential flatness equation only holds if the flatness constraints, namely $\mathbf{\dot{x}} \neq 0$ and $a_{v_z} \neq 0$, are satisfied. This is intuitive, as the aircraft's lack of hovering capability and the inability to control the system when the aircraft is perpendicular to the desired trajectory direction make these constraints necessary.
%makes possible to find the axial acceleration $a_{v_x} = \mathbf{r}_{v_x}^T (\ddot{\mathbf{x}} - \mathbf{g}) $. The normal acceleration $a_{v_z}$ is found by the projection of $a_{v_x}$ in the normal plane as  $a_{v_z} = -\lVert(\ddot{\mathbf{x}\rVert - g - a_{v_1} \mathbf{r}_{v_x}^T)}$. Thus the full matrix $R$, describing the desired attitude of $\mathcal{V}$ given the input $\mathbf{z}^{(3)}$ can be retrieve knowing that $\mathbf{r}_{v_z} = a_{v_z} / \lVert a_{v_z}\rVert$ and $\mathbf{r}_{v_y} = \mathbf{r}_{v_z} \times \mathbf{r}_{v_x}$, since no accelerations are wanted on $\mathcal{V}_y$.

We define the system's control input sent to the inner attitude controller~\cite{REINHARDT202191,Coates} the desired orientation matrix $\mathbf{R}_{c}$, expressed through Euler angles $\theta_c$, $\phi_c$, and $\psi_c$, along with angular velocities $\omega_{v_x}$, $\omega_{v_y}$, and axial acceleration $a_{v_x}$ represented as thrust $a_T$. This forms the commanded control input $\mathbf{u}_{c} = [\theta_c~\phi_c~\omega_{v_x}~\omega_{v_y}~a_T]^\top$. Specifically, $\mathbf{u}_{c}$ is derived by first calculating $\mathbf{R}_c$ from the previous $\mathbf{R}$ expression. Subsequently, we consider the following cascade PID loop to compute the commanded jerk
\begin{equation}
    \mathbf{x}_{c}^{(3)} = \mathbf{x}^{(3)}_{r} + k_2 \ddot{\mathbf{e}} + k_1 \dot{\mathbf{e}} + k_0 \mathbf{e},
\end{equation}
where $\mathbf{e} = \mathbf{x}_{r}(t) - \mathbf{x}(t)$, and $k_2, k_1, k_0$ the feedback gains. Finally, based on the differential flatness model in eq. \eqref{eq:differential_flatness},  we derive $\omega_{v_x}~\text{and~}\omega_{v_y}$ considering $\mathbf{x}_{c}^{(3)}$ and $\mathbf{R}_c$ in place of $\mathbf{x}^{(3)}$ and $\mathbf{R}$ respectively. This allows to achieve a trajectory tracking given the state feedback $\mathbf{x}(t)$.


\subsection{Trajectory Time Parametrization}
Despite the strength of the differential flatness approach, the  desired tangential acceleration along the trajectory can vary depending on how the trajectory is formulated with respect to time. Due to the natural minimization of the tracking error $\mathbf{e}$ towards the reference trajectory tracking point, an abrupt change of the desired thrust may happen if the trajectory presents variations in the reference velocities $\dot{\mathbf{x}}_{r}$ and acceleration $\ddot{\mathbf{x}}_{r}$.
In this condition, the FW can slow down below a safe cruising airspeed, producing a loss of airflow and control of the aerodynamic surfaces. 

To prevent such a scenario, we introduce a path parameterization variable $s(t)$ that defines how the desired trajectory values are allocated along the path $\mathbf{x}_{r}:= \mathbf{x}_{r}(s(t))$ for $t \geq 0$, where $s$ represents the distance along the desired path. This parameterization enables dynamic inversion of trajectory $\mathbf{x}_{r}(t) $ based on the distance travelled while maintaining a constant cruising velocity. Therefore, eq.~\eqref{eq:differential_flatness} is modified as 


%ensures that the axial acceleration $a_{v_x}$ remains constant and predetermined for the entirety of the trajectory; however, it also means that $a_{v_x}$ is no longer available as control input in equation \ref{eq:differential_flatness}. Thus, apath parameterization $s(t)$ variable is introduced, which follows the desired trajectory $\rho_s(t):= \rho(s(t))$ for $t \geq 0$, where $s$ is the distance along the desired path. This path parameterization enables dynamic inversion of trajectory $\rho(t)$ based on the distance travelled while maintaining a constant cruising velocity. It also provides an alternative state representation where $\dot{a}_{v_x}$ can be expressed as $s^{(3)}$, allowing us to modify the differential flatness equation \ref{eq:differential_flatness} in \cite{hauser1997aggressive} as shown below:

\begin{equation}
\begin{split}
& \mathbf{M}
\begin{bmatrix}
s^{(3)} \\
\omega_{v_x} \\
\dot{a}_{v_z}
\end{bmatrix} = 
\begin{bmatrix}
a_{v_z}\omega_{v_y} + \dot{a}_{v_z}\\
a_{v_x}\omega_{v_z} \\
-a_{v_x}\omega_{v_y}
\end{bmatrix} \\
 &- 
\mathbf{R}^\top \left[3\frac{\delta^2\mathbf{x}_{r}}{\delta s^2}\ddot{s}\dot{s} + \frac{\delta^3\mathbf{x}_{r}}{\delta s^3}\dot{s}^3 + k_2\mathbf{\dddot{e}} + k_1\mathbf{\dot{e}} + k_0\mathbf{e}\right],
\end{split}
\label{eq:time_param_differential_flatness}
\end{equation}
where $\mathbf{M}$ is the decoupling matrix represented as
\begin{equation}
\mathbf{M}
 = 
\begin{bmatrix}
\vdots & 0 & 0\\
\mathbf{R}^\top\frac{\delta\mathbf{x}_{r}}{\delta s}& a_{v_z} & 0 \\
\vdots & 0 & -1
\end{bmatrix}. 
\label{eq:time_param_M}
\end{equation} 

%The primary limitation of this method is to avoid flying perpendicular to the desired path.
%If the plane flies perpendicular, the first column of $R$ (roll) will become orthogonal to $\rho^{'}$, causing $\mathbf{M}$ to become singular and non-invertible. In practice, this situation is unlikely to occur unless the controller is not properly initialized. 

%trackin as just an equation 

%trajectory parametrization 


% ----------------------------------------------------------
% Extra

 %Here stating describing diff flat


%This section outlines the conventions used for defining reference frames in fixed-wing systems, their trajectory, and related control systems. It also introduces the dynamic model of a fixed-wing, which describes its motion, state, and desired control input, which will be used for its coordinated flights and differential flatness formulation.

% Across the paper, we will refer to derivatives of position in $W$ using the notation $\mathbf{\dot{\Delta}, \ddot{\Delta}, \dddot{\Delta}}$. 

%At the end, visualization of the proposed system architecture is depicted in Fig \ref{fig:software_architecture}, which highlights the adaptability and flexibility of the software stack that has been rigorously tested in both simulation and real-world scenarios. The hardware and software configurations for real-world experiments are explained in details in the experimental results section. 


%\subsection{Coordinate System:}

%As shown in the figure, we follow the standard fixed-wing aerodynamic coordinate system convention. The inertial frame $I$ is defined by three axes in the Forward-Left-Up (FLU) convention, while $I_{enu}$ refers to the inertial frame in the East-North-Up (ENU) convention. The fixed-wing rigid body frame $B$ is represented by three axes $[b_x, b_y, b_z]^T$, which are aligned with the aircraft’s longitudinal, lateral, and vertical axes, respectively. The position of body frame $B$ is represented by $\mathbf{x} = [x, y, z]^T$ in the inertial frame $I$ as a translation vector. An intermediate vehicle frame $V$ is introduced, aligned with the inertial frame, and used to express the relative orientation of the body frame $B$ with respect to inertial frame $I$. The transformation between $I$ and $B$ is represented by the transformation matrix $\mathbf{T} = [\mathbf{R}, \mathbf{x}^T]$, where $\mathbf{R}$ is the rigid body rotation matrix between that defines the orientation of frame $B$ relative to $I$.  


%\subsection{System Dynamics:}

% writing draft and will make better once content is written (currently working on this)

%The state of the fixed-wing system $\mathbf{X}$ is defined by several key components: the aircraft position $\mathbf{x} = [x, y, z]^T \in \mathbb{R}^3$; its velocity $\mathbf{v}_{b} = [v_{b_x}, v_{b_y}, v_{b_z}]^T$  represented in the body frame $B$; orientation $\mathbf{R}$ described by Euler angles roll ($\phi$), pitch ($\theta$), yaw ($\psi$) in the intermediate vehicle frame $V$, and the angular velocity $\mathbf{\omega}=[\omega_{b_x}, \omega_{b_y}, \omega_{b_z}]^T$ around each body axis, represented in the body frame $B$. 

%The whole body of a fixed wing is considered a single rigid body and modeled using the Runge-Kutta 4 (RK4) method for its translation and rotational dynamics. The system dynamic model is as follows:


% \begin{equation}
% \begin{subequations}
%     \begin{align}
         
%         \label{eq:1a} \\
%         d - e &= f 
%         \label{eq:1b} \\
%         g \times h &= i 
%         \label{eq:1c}
%     \end{align}
% \end{subequations}
% \end{equation}

%The motion of the system is defined by the normal and axial acceleration ($a_{v_z}$ and $a_{v_x})$ and the roll rate $\omega_x$. Our system is designed to follow the coordinated flights. The state is defined by the position $\mathbf{x} \in \mathbb{R}^3$ of the aircraft represented in $I$ frame, velocity $\mathbf{v}$, and orientation $\mathbf{R}$. The evaluation of orientation $R$ can be expressed by the instantaneous angular velocity $\mathbf{\omega}=[\omega_x, \omega_y, \omega_z]^T$ expressed in the intermediate velocity frame. 

%\subsection{Fixed Wing Coordinated Flight Condition}

%The planning and control developed and described in this paper respect the rule of coordinated flight, which is defined as a condition where the body velocity of the vehicle is contained on the longitudinal plane, thus defining $v^B_y = v^B_z = 0$. The coordinated flight condition is expressed in the velocity frame $\mathcal{V}$, which is different from $B$ by the angle of attack of the vehicle, and it can be described by the rotation $\mathbf{R}_v$ from the body frame $B$. Consequently, velocities and accelerations can be mapped back in the world frame as $\dot{\Delta} = \mathbf{R}_v \mathbf{v}_v$ and $\ddot{\Delta} = \mathbf{g} + \mathbf{R}_v \mathbf{a}_v$, where $\mathbf{g}$ is the gravity vector. To keep the system in condition of coordinated flight the pitch and yaw rates are respectively constrained to be: $\omega_{v_y} = -(a_{v_z} + g_{v_z})/V$ and $\omega_{v_z} = g_{v_y}/V$, where 

%$V = \abs{\dot{\Delta}}$ and $g_{v_y}$ 


%and $g_{v_z}$ represents the component of the gravity vector projected on $\mathbf{v}_{y}$ and $\mathbf{v}_{z}$.

%In the same expression, components $a_{v_x}$ and $a_{v_z}$ represent the axial and normal acceleration of the vehicle, and their derivatives will be the output of the differential flat model presented in the next section

%sent to an inner attitude controller using a cascade PID loop where angular velocities $\omega_{v_x}$ and $\omega_{v_y}$ and the axial acceleration $a_{v_x}$, expressed in terms of thrust $T$ .

%Considering the differential flatness model just derived, the final input to our system consists in the desired orientation matrix $\mathbf{R}$, decomposed in the eulerian angles $\theta$, $\phi$ and $\psi$, the angular velocities $\omega_{v_x}$ and $\omega_{v_y}$ and the axial acceleration $a_{v_x}$, expressed in terms of thrust $T$, defining the vector of the desired controller inputs $\mathbf{u}_{des} = [\theta, \phi, \omega_{v_x}, \omega_{v_y}, T]$, forwarded to an inner attitude controller based on a cascade PID loop. 

%However, as shown in eq. \ref{eq:differential_flatness}, this expression alone stabilize the system on a desired trajectory $\rho(t)$, without accounting for deviation from the desired path $\mathbf{e}(t)$. To integrate the trajectory deviation feedback into eq. \ref{eq:differential_flatness}, the desired trajectory jerk i.e., flat output derivative $\mathbf{z}^{(3)}$ can be modified to include the closed loop error $\mathbf{e} = \rho(t) - \mathbf{x}(t)$, where where $\mathbf{x}(t)$ is the current position of system and $\rho(t)$ is the reference position at given time $t$. In particular, as visualized in \cite{Bry2015AggressiveFO} final equation for $\mathbf{z}^{(3)}$ can be expressed as:
\section{Trajectory Planning}
\label{sec:Planning}

% problem definition for motion planning 
%In the previous section, we introduced the differentially flat model for FW constrained to a coordinated flying condition. 
We focus on designing an optimal, dynamically feasible trajectory for a FW that leverages the differential flatness property and employs Bernstein polynomials adhering to the following conditions

 %Using this concise yet powerful representation, it is possible to obtain the desired input $\mathbf{u}$ to the vehicle directly from the third derivative of the desired trajectory $\rho(t)$, which has to be continuous and differentiable, in order to guarantee smooth roll rate control. 
 
\begin{itemize}
    \item The axial velocity of the plane $\dot{x}_{x} \neq 0$.
    \item The trajectory should satisfy that $a_{v_z} \neq 0 $.
    \item Bounding the maximum curvature $\kappa$ of the trajectory.
\end{itemize}
We formulate a convex quadratic optimization of Bernstein polynomials to minimize the trajectory jerk $\mathbf{x}_{r}^{(3)}(t)$ (input in eq. \eqref{eq:differential_flatness}) while keeping velocity, acceleration and curvature $\kappa$ constraints within specified bounds. %Since the FW aircraft are constrained to continuous forward motion.%, each Bernstein trajectory is followed by a circular loitering motion with a predetermined radius when no new trajectory is defined. This loitering phase serves as the fixed-wing equivalent of the hovering phase seen in quadrotors or tail-sitters, as described in \cite{lu2024trajectory}.

%that will span  entire flight is significantly challenging compared to multi-rotor platforms \cite{loianno2016estimation}. Unlike quadrotors or tail-sitters \cite{lu2024trajectory}, which can hover in place, fixed-wing must continue its forward motion. This constraint forces the fixed wing either to immediately transition into a loiter motion or begin a new trajectory as soon as it completes its current one. In addition to a smooth transition between trajectories, it is important to respect the kino-dynamic constraint of a fixed-wing while generating these trajectories, particularly its minimum turning radius, which directly influences the required roll angle during flight. Respecting these kino-dynamic constraints and ensuring smooth transitions between trajectories is important to maintain stability and control throughout the entire flight. 

% solution discussion about a seamless transition
%To ensure the seamless transition between trajectories, we generate continuous and dynamically feasible trajectories within the vehicle's flat output space, defined by the vector $\mathbf{x}$. Throughout the entire flight span, the vehicle's trajectory is composed of three distinct types of polynomial trajectories: i) line trajectory, ii) loiter trajectory, and iii) Bernstein Trajectory. Each of these trajectories is defined by a sequence of flat outputs, which correspond to the desired positions and their higher derivatives up to the jerk at each time $t$ from the initial time $t_0$ to a final time $t_f$ for the respective trajectory.




\subsection{Bernstein Trajectory}
\label{sec:planner}

A Bernstein polynomial shows interesting properties in terms of smoothness and ability to impose global spatial constraints compared to time-based polynomials~\cite{kielas2019bebot,kielas2022bernstein}. For a given $m_j$  trajectory, it can be described by the following form of degree $n$
\begin{equation}
C_{n,m_j}(t) = \sum_{i=0}^{n}\mathbf{p}_{i,n}^{m_j}\beta^n_i(t),   \quad t\in [t_0, t_f]
\label{eq:bernstein_equation}
\end{equation}
where $\mathbf{p}_{i,n}^{m_j}$ are the Bernstein coefficient or control points of  size $n$ control, and $\beta^n_i(t)$ is the Bernstein basis. The $k^{\text{th}}$ derivative of the polynomial can be obtained as
\begin{equation}
\frac{d^{k}}{dt^{k}}C_{n,m_j}(t) = \frac{n!}{(n-k)!(t_f - t_0)^k} \sum_{i=0}^{n-k} {{\mathbf{p}}^{{m_j}^{'}}_{i,n-k}}\beta_{i}^{n-k}(t),
\label{eq:bernstein_derivative}
\end{equation}
with ${\mathbf{p}}^{{m_j}^{'}}_{i,n-k} = \mathbf{p}_{i,n}^{m_j}\mathbf{D}_k$ and $\mathbf{D}_k = \text{diag}(\mathbf{c}\circledast^k, \mathbf{c}\circledast^k, \cdots, \mathbf{c}\circledast^k)$ is the Differential matrix with $ \mathbf{c} = [-1, 1]$ convoluted $k$ times. Considering $M+1$ waypoints, a full trajectory $\mathbf{x}_{r}(t)$ can be modeled by stacking together $M$ Bernstein polynomials connected at the extremal points as

%A trajectory $\rho(t)$ can be efficiently modeled as a piece-wise stacking of consecutive Bernstein polynomial segments passing through multiple waypoints, combined into a single optimized Bernstein polynomial and connected at the extremes in order to be continuous-time in the flat output space $\mathbf{x}=\{x, y, z\}$ and its higher derivatives. 

%These trajectories can be efficiently optimized and evaluated under various constraints, including velocity, acceleration, and, especially in our case, roll rate and turning radius. By representing continuous-time trajectories as piecewise Bézier curves passing through multiple waypoints, these can be combined into a single Bernstein polynomial after optimization, ensuring both computational efficiency and respect the imposed constraints.

 %The general equation of Bernstein polynomial trajectory for a single dimension is represented as \cite{kielas2022bernstein}
%\begin{equation}
%C_n(t) = \sum_{i=0}^{n}\mathbf{p}_{i,n}\beta^n_i(t),   \quad t\in [t_0, t_f]
%\label{eq:bernstein_equation}
%\end{equation}

%where $\mathbf{p}_{i,n}$ are the Bernstein coefficient or control points, with each segment having a total of $n$ control points, and $\beta^n_i(t)$ is the Bernstein basis. To represent the complete trajectory passing through $M+1$ waypoints, we employ a set of $M$ Bernstein polynomials, where $M$ is the total number of trajectory segments:

\begin{equation}
\mathbf{x}_{r}(t)  = 
\begin{cases} 
    \sum_{i=0}^{n}\mathbf{p}_{i,n}^{m_1}\beta_i^n(T_1 - t) \quad \text{for} \ t\in [0, T_1]\\
    \sum_{i=0}^{n}\mathbf{p}_{i,n}^{m_2}\beta_i^n(T_2 - t) \quad \text{for} \ t\in [T_1, T_2] \\
    \vdots \\ 
    \sum_{i=0}^{n}\mathbf{p}_{i,n}^{M}\beta_i^n(T_{M}-t) \quad \text{for} \ t\in [T_{M-1}, T_M]
\end{cases}
\label{eq:piecewise_bernstein_equation}
\end{equation}
where $\mathbf{p}_{i,n}^{m_j}$ is the $i^{th}$ control point of the $m_j$ sub trajectory, with $j \in [1, M]$, and the time instants $T_1, T_2, \dots, T_M$ represent the allocated time for each of sub trajectory. 

%\begin{itemize}
%\item $P^{'}_{i,n-m} = P_{i,n}\mathbf{D}_m$
%\item Differential Matrix: $\mathbf{D}_m = \begin{bmatrix}\mathbf{s}\circledast^m  & 0 & \cdots  & 0 \\ 0 & \mathbf{s}\circledast^m & \cdots  & 0 \\\vdots  & \vdots  & \ddots  & \vdots  \\0 & 0 & \cdots  & \mathbf{s}\circledast^m\end{bmatrix}$
%\item $\mathbf{s} = [-1, 1] \quad \text{and} \quad \underbrace{s \circledast s \circledast \cdots \circledast s}_{m}=\mathbf{s}\circledast^m$
%\end{itemize}

To find the Bernstein Coefficients $\mathbf{p}$ we formalize  a Convex Quadratic Programming (QP) problem \cite{mao2023robust}
\begin{equation}
\begin{aligned}
\text{min} \quad & \mathbf{p}_d^T\mathbf{Q} \mathbf{p}_d\\
\text{s.t.} \quad & \mathbf{A}_{eq}\mathbf{p}_d = \mathbf{b}_{eq} \\
& \mathbf{A}_{ineq}\mathbf{p}_d \le  \mathbf{b}_{ineq}
\end{aligned}
\end{equation}
where $\mathbf{Q} = \text{diag}(Q_1, \hdots, Q_M)$ with $Q_i \in \mathbb{R}^{n \times n}$ representing the Hessian semi-definite matrix of the objective function, related to the $n$ number of Bernstein Coefficients each sub trajectory. The vector $\mathbf{p}_d$, with dimension $M \times n$, contains the Bernstein coefficients to be optimized for each spatial dimension $d$. To ensure continuity in position and higher derivatives between the segments, the optimization problem is subject to various equality and inequality constraints, which are represented by the matrices $\mathbf{A}_{eq}, \mathbf{A}_{ineq}$, and vectors $\mathbf{b}_{eq}, \mathbf{b}_{ineq}$


% Matrix $\mathbf{A}_{eq}, \mathbf{A}_{ineq}$, and vector $\mathbf{b}_{eq}, \mathbf{b}_{ineq}$ are derived from the equality and inequality constraints imposed by user for each dimension $d$. 
%The optimization problem is solved using an off-the-shelf OOQP \cite{gertz2003object} convex solver. 

% The vectors $\mathbf{b}_{id}$ consists of all number $i$ of constraints imposed by the user for each dimension $d$. Finally the matrix $\mathbf{A}  = \text{diag}(A_1, \hdots, A_j, \hdots, A_m)$ is composed by submatrix $A_j$ which one with dimension $A_j \in \mathbb{R}^{i \times n}$ and it is stacked for the $d$ dimesions of the polynomial.

\begin{enumerate}
    \renewcommand{\labelenumi}{\roman{enumi}.}
    \item \textit{Endpoint constraint:}
    Considering a starting time $t_0$ and an ending time $t_{f}$, we constrain $\mathbf{x}_{r}$ at the reference waypoints position $\mathbf{x}_{r}$, velocity $\dot{\mathbf{x}}_{r}$, and acceleration $\ddot{\mathbf{x}}_{r}$
     \begin{equation}\begin{aligned}
        C_{n,0}^{(k)}(t_0) = \mathbf{x}^{(k)}(t_0), \qquad C_{n,M}^{(k)}(t_f) = \mathbf{x}^{(k)}(t_f) 
    \end{aligned}.\end{equation}

    \item \textit{Continuity Constraints:}
    %Given a set $S_{\mathcal{B}}$ of $m_{i+1}$ waypoints, defined by a starting and ending time $t_{{m}_0}$ $t_{{m}_f}$,
    The goal is to ensure the continuity in position and higher derivatives of the trajectory $\mathbf{x}_r(t)$ at the junction of the $M$ sub trajectories as

    % \begin{equation}\begin{aligned}
    % \mathbf{C_m}(t_f) = \mathbf{C_{m+1}}(0) \\
    % \left\|  \mathbf{\dot{C}_m}(t_f) \right\| = \left\| \mathbf{\dot{C}_{m+1}}(0) \right\| \\
    % \left\|  \mathbf{\ddot{C}_m}(t_f) \right\| = \left\| \mathbf{\ddot{C}_{m+1}}(0) \right\|
    % \end{aligned}\end{equation}

    \begin{equation}\begin{aligned}
        C_{n,m}(t_{f}) = C_{n,m+1}(t_{{0}}). \\
    \end{aligned}\end{equation}



    
    \item \textit{Dynamic feasibility Constraints:}
    Given the FW dynamics, the curvature $\kappa = f(\dot{\mathbf{x}}_{r_x}, \dot{\mathbf{x}}_{r_y}, \ddot{\mathbf{x}}_{r_x}, \ddot{\mathbf{x}}_{r_y})$ evaluated from $t_0$ to $t_f$ of a given trajectory, needs to be constrained for its entire duration within the range  $\kappa_{min} \leq \kappa \leq \kappa_{max}$ to be considered feasible in order to avoid exceeding the maximum roll angle of the aircraft. Due to the non-linear nature of the curvature function $\kappa$, we apply a Taylor expansion around the equilibrium point to linearize the constraint, allowing us to maintain the original convex optimization problem formulation. The constraint  $k$ on the lineared curve is
    
    %The curvature $\kappa$ is a nonlinear function of $v_x, v_y, a_x$ and $a_y$ as expressed in the following equation:

    %\begin{equation}\begin{aligned}
   % k = f(v_x, v_y, a_x, a_y) = \frac{v_xa_y - a_xv_y}{(v_x^2 + v_y^2)^{3/2}}
    %\label{eq:curvature_k}
    %\end{aligned}\end{equation}

    \begin{equation}
    \begin{split}
        % &\phantom{=} k_{min} \leq k \leq k_{max} \\
        & \kappa_{min} \leq 
f(\dot{\mathbf{x}}_{r_x}, \dot{\mathbf{x}}_{r_y}, \ddot{\mathbf{x}}_{r_x}, \ddot{\mathbf{x}}_{r_y}) + \\
        &\begin{bmatrix} 
            \frac{\partial f}{\partial  \dot{\mathbf{x}}_{r_x}} &  \frac{\partial f}{\partial  \dot{\mathbf{x}}_{r_y}} & 
            \frac{\partial f}{\partial  \ddot{\mathbf{x}}_{r_x}} & \frac{\partial f}{\partial  \ddot{\mathbf{x}}_{r_y}}
        \end{bmatrix}
        \begin{bmatrix} 
            \dot{\mathbf{x}}_{r_x} - \dot{\mathbf{x}}_{r_x}(t_{rp}) \\\dot{\mathbf{x}}_{r_y} - \dot{\mathbf{x}}_{r_y}(t_{rp}) \\ \ddot{\mathbf{x}}_{r_x} - \ddot{\mathbf{x}}_{r_x}(t_{rp})  \\ \ddot{\mathbf{x}}_{r_y} - \ddot{\mathbf{x}}_{r_y}(t_{rp}) 
        \end{bmatrix} 
        \leq \kappa_{max}.
    \end{split}
    \label{eq:curvature}
    \end{equation}
    where $t_{rp} \in [t_0, t_f]$ represents the time instant where the linearization is applied.
    In particular, for a continuous linearization of the entire trajectory around a local point, a replanning strategy visible in Fig. \ref{fig:planned_mission} (top right) is applied at constant intervals. To avoid discontinuities between the current trajectory $\mathbf{x}_{r, j-1}$ and new replanned trajectory $\mathbf{x}_{r, j}$, we account for the optimization time $t_{opt}$ such that $\mathbf{x}_{r, j}(t_0) = \mathbf{x}_{r, j-1}(t + t_{opt})$.
    
    %The Taylor series approximation of the nonlinear constraint $k$ is accurate near the equilibrium point but insufficient for the entire trajectory. To address this problem, we implemented periodic replanning to maintain the $k$-constaint throughout the trajectory. As shown in Figure \ref{fig:planned_mission} (upper right), the replanning strategy is applied at constant intervals. In order  This transition point ensures continuity, allowing the aircraft to seamlessly switch to the new replanned trajectory when it reaches the $\rho(s(t))_j$. The real-world implementation and effectiveness of this strategy are discussed in the experimental results section. 
   
    
\end{enumerate}




\section{Numerical Experiments}\label{sec:experiments}



\begin{figure}[t]
    \centering
    \includegraphics[width=0.8\textwidth]{img/regret_vs_iter.pdf}
    \caption{
    Regret scaling for Tsallis-INF and two other bandit algorithms. Each configuration $(T)$ is run for 512 trials. The interval between the 10th and 90th percentile is overlaid. The thicker dashed line represents a linear fit on the $T\geq 10^5$ subset of the log-log data.}
    \label{fig:regret-comparison}
\end{figure}


To validate our theoretical results,
we conduct a few numerical experiments.

The first experiment compares
Tsallis-INF against two baselines in terms of the regret: 
the classical UCB1~\citep{auer2002finite} and Exp3~\citep{auer2002nonstochastic} algorithms, 
which are known to have $O(T)$ and $\tilde{O}(\sqrt{T})$ regret bounds respectively in the adversarial setting.
We compare them on the game associated with $A$ defined in \eqref{eq:example-2x2-game-matrix-simplified},
with varying $T$ and $\eps=T^{-1/3}$,
where feedback $r_t$ follows a Bernoulli distribution over $\{ -1, 1 \}$ such that $ \E[r_t \mid i_t, j_t] = A(i_t, j_t)$.
As discussed, Theorem~\ref{thm:general-bound-together} predicts a regret of $\tilde{O}(T^{1/3})$ for Tsallis-INF.
The result of the experiment agrees with all these bounds in Figure~\ref{fig:regret-comparison},
where the asymptotic slope in the log-log plot (shown with a linear fit on the $T\geq 10^5$ region) is close to the theoretical prediction.


\begin{figure}[t]
    \centering
    \includegraphics[width=0.8\textwidth]{img/identify_P_vs_iterations_by_H1.pdf}
    \caption{
        Experimental validation of Tsallis-INF's PSNE identification capability.
        The plot shows the algorithm's success rate in correctly identifying PSNE
        against the number of itrations.
        We use a hard instance of a $256\times 256$ matrix and $\Delta_1=0.1$,
        running 512 trials for each $\Delta_{\min}$ values
        over a horizon of $128\OPT$ iterations,
        where $\OPT$ is the theoretical lower bound for PSNE identification.
        The $x$-axis is scaled by $1/\OPT$.
        }
    \label{fig:PSNE-id-rate}
\end{figure}

We have discussed in Section~\ref{sec:PSNE_complexity} that Tsallis-INF needs $\frac{\omegar+\omegac}{\Delta_{\min}}$ iterations to identify the PSNE of a game. To validate our theoretical bounds, we conduct our second experiment using the following hard instance  introduced by \citet{maiti2024midsearch}:
\begin{equation}
    A=\begin{bNiceArray}{ccccc}[nullify-dots, margin, custom-line = {letter=I, tikz=dashed}, cell-space-limits = 4pt]
        0 & 2{\Delta_{\min}} & \Block{1-3}{} 2{\Delta_1} &\Cdots& 2{\Delta_1} \\
        -2{\Delta_{\min}} & \Block{4-4}{} 
                       0      & 1      & \Cdots & 1      \\
        \Block{3-1}{}
        -2{\Delta_1} & -1     & \Ddots & \Ddots & \Vdots \\
        \Vdots       & \Vdots & \Ddots & \Ddots & 1      \\
        -2{\Delta_1} & -1     & \Cdots & -1     & 0      \\
    \end{bNiceArray},
    \label{eq:psne-experiment-array}
\end{equation}
where the top-left entry is the PSNE. We set the number of actions $n=m=256$ and the gap $\Delta_1=0.1$, and vary the value of $\Delta_{\min}$.
Let $\OPT$ represent the theoretical optimal bound for identifying PSNE (ignoring log terms), defined as 
$\OPT=
\sum_{i \in [m] \setminus \{ \istar \}} \frac{1}{{\Deltar}^2_i}
+
\sum_{j \in [n] \setminus \{ \jstar \}} \frac{1}{{\Deltac}^2_j}
$,
which simplifies to $\frac{1}{2\Delta_{\min}^2}+\frac{m-2}{2\Delta_1^2}$ in this experiment.
\citepos{maiti2024midsearch} achieve the optimal $\tilde{O}(\OPT)$ sample complexity,
and their Figure~2 suggests that the sample complexity of Tsallis-INF divided by $\OPT$ is unbounded as $\Delta_{\min}$ decreases,
but our analysis in Section~\ref{sec:PSNE_complexity} disagrees with this trend.
As shown in Figure~\ref{fig:PSNE-id-rate},
the number of iterations needed to identify the PSNE divided by $\OPT$
decreases and then increases
as $\Delta_{\min}$ varies.
Lemma~\ref{lem:sqrtk-ratio} predicts the minimum ratio occurs when $\frac{\Delta_{\min}}{\Delta_1}=\frac{1}{\sqrt{m}+1}=1/17$,
and among the values we tested,
the minimum is reached when $\frac{\Delta_{\min}}{\Delta_1}=0.005/0.1=1/20$,
closely matching the prediction.
This supports our derived bound of $\tilde{O}\rbrm[\big]{\sqrt{m}\cdot \OPT}$.

The code for reproducing the experiments is available on 
\url{https://github.com/EtaoinWu/instance-dependent-game-learning}.


\section{Conclusion}
\label{sec:Conclusion}
In this paper, we proposed a complete real-time planning and control approach for continuous, reliable, and fast online generation of dynamically feasible Bernstein trajectories and control for FW aircrafts. The generated trajectories span kilometers, navigating through multiple waypoints. By leveraging differential flatness equations for coordinated flight, we ensure precise trajectory tracking. Our approach guarantees smooth transitions from simulation to real-world applications, enabling timely field deployment. The system also features a user-friendly mission planning interface. Continuous replanning  maintains the rajectory curvature 
$\kappa$ within limits, preventing abrupt roll changes.

Future works will include the ability to add  a higher-level kinodynamic path planner to optimize waypoint spatial allocation and improve replanning success, and enhancing the trajectory-tracking algorithm by refining the aerodynamic coefficient estimation. 



%\addtolength{\textheight}{-9cm}   % This command serves to balance the column lengths
                                  % on the last page of the document manually. It shortens
                                  % the textheight of the last page by a suitable amount.
                                  % This command does not take effect until the next page
                                  % so it should come on the page before the last. Make
                                  % sure that you do not shorten the textheight too much.




%%%%%%%%%%%%%%%%%%%%%%%%%%%%%%%%%%%%%%%%%%%%%%%%%%%%%%%%%%%%%%%%%%%%%%%%%%%%%%%%

\bibliographystyle{IEEEtran}
\bibliography{reference}




\end{document}
