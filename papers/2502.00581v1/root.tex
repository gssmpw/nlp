%%%%%%%%%%%%%%%%%%%%%%%%%%%%%%%%%%%%%%%%%%%%%%%%%%%%%%%%%%%%%%%%%%%%%%%%%%%%%%%%
%2345678901234567890123456789012345678901234567890123456789012345678901234567890
%        1         2         3         4         5         6         7         8

\documentclass[letterpaper, 10 pt, conference]{ieeeconf}  % Comment this line out if you need a4paper

%\documentclass[a4paper, 10pt, conference]{ieeeconf}      % Use this line for a4 paper

\IEEEoverridecommandlockouts                              % This command is only needed if 
                                                          % you want to use the \thanks command

\overrideIEEEmargins                                      % Needed to meet printer 
\usepackage[english]{babel}
\usepackage{amsmath}
\usepackage{amssymb}
\usepackage{booktabs}
\usepackage{siunitx}
\usepackage[dvips]{graphicx}
\usepackage{multirow}
\usepackage{amsfonts}
\usepackage{enumerate}
\usepackage{tabularx}
\usepackage{algorithm,algorithmic}
\usepackage{bm}
\usepackage{enumerate}
\usepackage{adjustbox}
\usepackage{xcolor}
\usepackage{siunitx}
\usepackage{booktabs}
\usepackage{mdframed}
\usepackage{adjustbox}
\usepackage{authblk}
\usepackage{color}
\usepackage{xurl}
\usepackage{subcaption}
\urlstyle{rm}
\usepackage{cite}
\makeatletter
\let\NAT@parse\undefined
\makeatother
\usepackage{hyperref}
\usepackage{relsize}
\usepackage{float}
\usepackage{pifont}


\title{\LARGE \bf
Trajectory Planning and Control for Differentially Flat\\ Fixed-Wing Aerial Systems 
}


\author{Luca Morando$^{1}$$^*$, Sanket A. Salunkhe$^{1}$$^*$, Nishanth Bobbili$^{1}$, Jeffrey Mao$^{1}$, Luca Masci$^{1}$, Hung Nguyen$^{2}$,\\ Cristino de Souza$^{2}$, and Giuseppe Loianno$^{1}$% <-this % stops a space

\thanks{$^*$Equal contribution and authors listed in alphabetical order.}
\thanks{$^1$The authors are with the New York University, Tandon School of Engineering, Brooklyn, NY 11201, USA. {\tt\footnotesize email: \{luca.morando, sas9908, nb3553, jm7752, lm5175, loiannog\}@nyu.edu}.}%
\thanks{$^2$The authors are with the Autonomous Robotics Research Center-Technology Innovation Institute, Abu Dhabi, UAE. {\tt\footnotesize email:  \{hung.tuan,cristino.dsouza\}@tii.ae}.}
\thanks{This work was supported by the Technology Innovation Institute, the NSF CAREER Award 2145277, and the DARPA YFA Grant D22AP00156-00, Qualcomm Research, Nokia, and NYU Wireless. Giuseppe Loianno serves as consultant for the Technology Innovation Institute. This arrangement has been reviewed and approved by the New York University in accordance with its policy on objectivity in research.}
}


\begin{document}



\maketitle
\thispagestyle{empty}
\pagestyle{empty}


%%%%%%%%%%%%%%%%%%%%%%%%%%%%%%%%%%%%%%%%%%%%%%%%%%%%%%%%%%%%%%%%%%%%%%%%%%%%%%%%
\begin{abstract}

Efficient real-time trajectory planning and control for fixed-wing unmanned aerial vehicles is challenging due to their non-holonomic nature, complex dynamics, and the additional uncertainties introduced by unknown aerodynamic effects. 
In this paper, we present a fast and efficient real-time trajectory planning and control approach for fixed-wing unmanned aerial vehicles, leveraging the differential flatness property of fixed-wing aircraft in coordinated flight conditions to generate dynamically feasible trajectories. The approach provides the ability to continuously replan trajectories, which we show is useful to dynamically account for the curvature constraint as the aircraft advances along its path. 
Extensive simulations and real-world experiments validate our approach, showcasing its effectiveness in generating trajectories even in challenging conditions for small FW such as wind disturbances.

% In this paper, we present an efficient real-time trajectory planning and generation framework for fixed-wing aircraft. This framework ensures dynamic feasibility, which is critical for non-holonomic systems like fixed-wings. Due to fixed-wings' continuous forward motion, we require a smooth transition between trajectories at waypoints while maintaining coupled dynamics and constant speed. In our approach, we represent our trajectory using a Bernstein polynomial with continuous replanning to respect non-linear coupled dynamics.  


\end{abstract}


%%%%%%%%%%%%%%%%%%%%%%%%%%%%%%%%%%%%%%%%%%%%%%%%%%%%%%%%%%%%%%%%%%%%%%%%%%%%%%%%

\section{Introduction}
\label{sec:intro}


\ps{Challenges of technology scaling}

The growing demand for computing performance has always been met by increasing the number of transistors per chip, which is only possible due to CMOS technology scaling.
However, as we keep pushing the boundaries of technology scaling, we encounter multiple challenges.
Firstly, whenever we transition to a more advanced technology node, the non-recurring cost due to physical design, verification, software, mask sets, and prototyping almost doubles \cite{cost-tech-node}.
As a result, designing a chip in an advanced technology node is only economically viable if the chip is manufactured in vast quantities.
Secondly, many chip components such as I/O drivers, analog circuits, or \gls{srams} have reached their scaling limit.
This means that we cannot shrink these components further, even if we use a more advanced technology with a smaller feature size.
Thirdly, advanced technology nodes suffer from high defect rates, diminishing the yield and inflating the recurring cost.
To tackle these challenges, new chip-design paradigms have been developed.

\ps{Why 2.5D integration?}

One of these new paradigms is 2.5D integration, where multiple silicon dies called chiplets are integrated into the same package.
Once designed, a single chiplet can be reused in multiple 2.5D stacked chips, which increases the ratio of production volume to non-recurring cost.
Another advantage is that multiple chiplets - fabricated in different technologies - can be integrated into the same package.
This means that only components that can take full advantage of technology scaling are built in bleeding-edge technologies.
Components that have reached their scaling limit are fabricated in more mature and hence less costly technology nodes.
Furthermore, chiplets are smaller than monolithic chips.
Therefore, manufacturing chiplets results in less silicon area loss due to fabrication defects and hence a higher yield.
Due to these economic advantages, chip vendors such as AMD \cite{amd-chiplet} and NVIDIA \cite{chiplet-book} have adopted the 2.5D integration paradigm.  

\ps{Challenges of 2.5D integration}

An important challenge when designing 2.5D stacked chips is the construction of a low-latency and high-throughput \gls{ici}. 
To build an \gls{ici}, we connect different chiplets using \gls{d2d} links.
These links are fabricated in an organic package substrate, silicon bridge, or silicon interposer, and they are connected to the chiplets using \gls{c4} bumps or microbumps.
The number of bumps per chiplet is limited, and so is the bandwidth of \gls{d2d} links.
In addition to having lower bandwidth than links in monolithic chips, \gls{d2d} links also have higher latency.
This latency is caused by wire delay and by \gls{phys} that are necessary in both the sending and the receiving chiplet.
\gls{phys} are needed to convert between protocols, voltage levels, and frequencies, which are usually different between on-chiplet links and \gls{d2d} links.
Due to these limitations, the \gls{ici} can quickly become a bottleneck.

\ps{How we solve these challenges differently than the related work does.}

Existing approaches to maximize the performance of the \gls{ici} either optimize the placement of chiplets (with potentially heterogeneous shapes) for a predetermined \gls{ici} topology 
\cite{ho,liu,seemuth,eris,osmolovskyi,tap25d,chiou}, select one topology out of a set of candidates \cite{coskun-1, coskun-2}, or they optimize the \gls{ici} topology for a 2D grid of homogeneously shaped chiplets on an active interposer \cite{butterdonut, cluscross, kite}.
To the best of our knowledge, there is no prior work on \gls{ici} topologies for chips with heterogeneously shaped chiplets or with passive silicon interposers or silicon bridges.
To fill this gap, we propose \name, a novel optimization methodology to jointly optimize the chiplet placement and \gls{ici} topology of such architectures.
\ifnb
\else
\newpage
\fi

\ps{Details on \name~and the key idea}

The key idea is as follows: 
We optimize the chiplet placement without a predetermined topology.
For each placement generated by an optimization algorithm, we infer a placement-based \gls{ici} topology by connecting chiplets that are in close proximity in that specific placement.
We then compute the latency and throughput of this combination of placement and topology for different traffic types.
These latencies and throughputs together with the total chip area are used to compute a user-defined quality-score of the placement, which is returned to the optimization algorithm.
Based on this quality score, the algorithm can further optimize the placement.
By following this iterative process, we jointly optimize the chiplet placement and the \gls{ici} topology.

\ps{Short evaluation-summary}

We provide our open-source framework implementing the proposed placement and topology co-optimization methodology, which we evaluate using both synthetic traffic and traffic traces.
A 2D grid of chiplets with a mesh topology is used as a baseline since many proposals for 2.5D stacked chips \cite{dataflow_accel_dnn, cifher, simba, hecaton, dojo} use such an architecture.
We reduce the latency of synthetic L1-to-L2 and L2-to-memory traffic, the two most important traffic types for cache coherency traffic, by up to 28\% and 62\% respectively.
For real traffic traces, we reduce the average packet latency for almost all traces and architectures considered (reduced by an 8\% or 18\% on average depending on the configuration of \gls{phys} within a chiplet).


\section{Related work}


Recent advances in single-image animatable head avatar generation can be categorized into mainly 2D-based and 3D-based approaches. 

\paragraph{\bf Image to 2D Animatable Avatar.}
2D-based methods, leveraging the power of convolutional neural networks (CNNs)~\cite{DBLP:conf/cvpr/KarrasLAHLA20,DBLP:conf/cvpr/IsolaZZE17,DBLP:conf/nips/GoodfellowPMXWOCB14}, often employ generative adversarial networks (GANs)~\cite{DBLP:conf/cvpr/StyleGAN} for direct image synthesis. Early approaches~\cite{DBLP:conf/cvpr/WangDYSW23,DBLP:conf/cvpr/BurkovPGL20,DBLP:conf/iccv/ZakharovSBL19} focus on injecting expression and pose features into the generator network, often utilizing architectures like U-Net or StyleGAN~\cite{DBLP:conf/cvpr/StyleGAN}.
Some other 2D methods~\cite{DBLP:journals/corr/abs-2407-03168,DBLP:conf/cvpr/ZhangQZZW0CW023,DBLP:conf/cvpr/HongZS022,DBLP:conf/mm/DrobyshevCKILZ22,DBLP:conf/cvpr/BurkovPGL20,DBLP:conf/nips/SiarohinLT0S19} represent expressions and poses as warping fields applied to the source image. 
Benefiting from advances in image and video diffusion networks, more recent 2D-based works~\cite{DBLP:journals/corr/abs-2410-07718,DBLP:journals/corr/abs-2406-08801,DBLP:conf/eccv/TianWZB24} get improved results with diffusion techniques. 
However, these methods still face challenges related to long generation times and significant computational resource demands. Audio-driven 2D control methods~\cite{DBLP:conf/cvpr/ZhangCWZSGSW23,DBLP:journals/corr/abs-2211-12368,DBLP:conf/iccv/GuoCLLBZ21} are easy to use but cannot explicitly control facial expressions and poses. 2D-based techniques often struggle with large pose or expression variations due to the lack of an explicit 3D structure, sometimes producing unrealistic distortions or identity changes. While some 2D methods~\cite{SadTalker,StyleHEAT,Pirenderer,DBLP:conf/cvpr/WangM021,MegaPortraits} incorporate 3D Morphable Models (3DMMs)~\cite{DBLP:conf/fgr/GerigMBELSV18,DBLP:journals/tog/LiBBL017,DBLP:conf/avss/PaysanKARV09,DBLP:conf/siggraph/BlanzV99} to mitigate these issues, they typically cannot achieve free-viewpoint rendering. 

\vspace{-0.1in}

\begin{figure*}[h]
    \centering
    \includegraphics[width=0.9\linewidth]{images/framework.pdf}
    \caption{\textbf{Overall Framework.} Our framework utilizes learnable query features attached to FLAME vertices to perform cross-attention with the extracted multi-level image features. The extracted features are then decoded to reconstruct the Gaussian avatar in the canonical space, which can be animated utilizing standard linear blend skinning (LBS) and corrective blendshapes as the FLAME model did and rendered in real-time on various platforms.}
    \label{fig:framework}
\end{figure*}

\paragraph{\bf Image to 3D Animatable Avatar.}
3D-aware methods offer improved geometric consistency and free-viewpoint rendering capabilities. Early 3D approaches~\cite{DBLP:conf/eccv/KhakhulinSLZ22,DBLP:conf/cvpr/XuYCWDJT20} utilize 3DMMs for head avatar reconstruction. With the advent of Neural Radiance Fields (NeRFs)~\cite{DBLP:conf/eccv/MildenhallSTBRN20}, many recent methods~\cite{DBLP:conf/siggraph/YuFZWYBCSWSW23,DBLP:conf/cvpr/MaZQLZ23,DBLP:conf/cvpr/LiZWZ0CZWB023,GPAvatar,ye2024real3d,deng2024portrait4d,deng2024portrait4d2,DBLP:conf/eccv/KiMC24,DBLP:conf/cvpr/BaiFWZSYS23,PointAvatar,Nerfies,INSTA} have adopted this representation for higher fidelity, particularly in modeling fine details like hair. However, NeRF-based~\cite{DBLP:conf/cvpr/ZhangZLHLWGCL024,HAvatar,DBLP:conf/cvpr/BaiTHSTQMDDOPTB23,AD-NeRF,DBLP:journals/tog/GaoZXHGZ22,DBLP:journals/tog/ParkSHBBGMS21,DBLP:conf/cvpr/AtharXSSS22,DBLP:journals/corr/abs-2112-05637,DBLP:conf/iccv/TretschkTGZLT21,DBLP:conf/cvpr/GafniTZN21,DBLP:conf/eccv/KiMC24,DBLP:conf/cvpr/BaiFWZSYS23,PointAvatar,Nerfies,DBLP:conf/siggraph/YuFZWYBCSWSW23,DBLP:conf/cvpr/MaZQLZ23,DBLP:conf/cvpr/LiZWZ0CZWB023} approaches often require extensive training data, including multi-view or single-view videos, raising privacy concerns and limiting generalization to unseen identities. Some methods~\cite{DBLP:conf/cvpr/SunWWLZZL23,DBLP:conf/3dim/ZhuangMKS22,DBLP:journals/pami/SunWZHWL24,DBLP:journals/tvcg/TangZYZCMW24,DBLP:conf/iclr/XuZLZBFS23} bypass this data requirement by training generators with random noise and then inverting them for identity-specific reconstruction, but inversion accuracy remains a challenge. Test-time optimization offers another alternative, but its computational cost limits practical applications. Several recent works~\cite{goha2023,hidenerf2023,gpavatar2024,ye2024real3d,ma2024cvthead,deng2024portrait4d,deng2024portrait4d2,GGHead} have explored one-shot 3D head reconstruction to address the limitations of data requirements and computational cost. These methods employ various techniques, such as tri-plane features, deformation fields, point-based expression fields, and vertex-feature transformers. Despite these advancements, NeRF-based methods often struggle with real-time rendering. 
Recently, 3D Gaussian Splatting~\cite{GaussianSplatting} has emerged as a promising alternative, offering both high-quality results and fast rendering speeds. However, existing Gaussian Splatting methods~\cite{GaussianAvatar,DBLP:conf/cvpr/XuCL00ZL24} typically rely on video data for training for each person, limiting their ability to generalize to new identities. Instead, the most recent work, GAGAvatar~\cite{GAGAvatar}, proposes a one-shot 3D Gaussian-based head avatar generation method. However, it still relies heavily on complex 2D neural post-processing to achieve optimal animation outcomes, thus it is not a pure 3D solution and the extra neural network hinders its application on various platforms. In contrast, our work generates Gaussian heads that are immediately animatable and renderable without additional networks or post-processing steps, enabling seamless integration into existing rendering pipelines for real-time animation and rendering across a wide range of platforms, including mobile phones. 
\begin{figure}[t!]
%\includegraphics[width=0.7\columnwidth, scale=1.0, trim={2cm 0.7cm 2cm 2cm}, keepaspectratio]{figs/Fig2_framing.pdf}
\includegraphics[width=0.8\columnwidth, trim={1cm 1cm 3cm 0.5cm}, keepaspectratio]{figs/Fig2_framing.pdf}
  \caption{Frames' visualization and convention. The Velocity frame $\mathcal{V}$ differs from the Body frame $\mathcal{B}$ by the angles $\alpha$ and $\beta$, whereas $\mathcal{I}$ defines the Inertial fixed frame. 
}
 \label{fig:fig2}
\vspace{-10pt}
\end{figure}

\begin{figure*}[!t]
  \centering
  \includegraphics[width=1\textwidth]{figs/Fig_3_software_architecture_reduced.pdf}
  \caption{Architecture overview. The trajectory generated by our planner is forwarded to the Differential Flatness Block, which computes the desired inputs to control the attitude and the longitudinal thrust of the simulated or real robot. %The mission is monitored through the Ground Station. 
  \label{fig:software_architecture}}
  \vspace{-20pt}
\end{figure*}


\section{System Modeling}
\label{sec:System_modeling}

In this section, we outline a simple aircraft model that captures the important features of the coordinated flight condition adopted in the rest of this paper.
As shown in Fig.~\ref{fig:fig2}, we use the following frame convention: the inertial frame is denoted by $\mathcal{I}$ while $\mathcal{B}$ denotes the vehicle’s rigid body frame that it is aligned with the FW's longitudinal, lateral, and vertical axes. The velocity frame $\mathcal{V}$, centered at $\mathcal{B}$, is rotated with respect to the $\mathcal{B}$ by the sideslip angle $\beta$ and the attack angle $\alpha$, therefore always keeping the corresponding $\mathcal{V}_x$ axis aligned with the direction of the aircraft velocity. The purpose of introducing the velocity frame $\mathcal{V}$ is due to the fact that the FW can be subject to lateral winds that may deviate its nose from the desired direction of motion. 
The state of the FW system is defined as $\mathbf{X} = \{\mathbf{x}, \mathbf{\dot{x}}, \mathbf{R}, \bm{\omega}_v\}$, where $\mathbf{x} \in \mathbb{R}^3$ represents the position of FW in inertial frame $\mathcal{I}$,  
%The rotation between $\mathcal{B}$ and  $\mathcal{V}$, is represented by $R_{\mathcal{V}}^{\mathcal{B}}(\alpha, \beta)$, 
while $\mathbf{R} \in SO(3)$ with $\mathbf{R}= \mathbf{R}_{\mathcal{B}}^{\mathcal{I}}\mathbf{R}_{\mathcal{V}}^{\mathcal{B}}$ denotes the rotation of $\mathcal{V}$ with respect to $\mathcal{I}$. This can also be parameterized using Euler angles roll ($\phi$), pitch ($\theta$), and yaw ($\psi$). The velocity of fixed-wing is represented by the velocity vector $\mathbf{\dot{x}} = [\dot{x}_{x}~\dot{x}_{y}~\dot{x}_{z}]^\top$, with a zero lateral velocity component ($\dot{x}_{y} = 0$) to satisfy the coordinated flight condition. This condition is general to different aircraft configurations, assuming that the motion of the plane is aligned with the relative wind direction and executes curves keeping the lift vector always aligned with $\mathcal{V}$ vertical axis. 

%The orientation dynamics can be described by equation $\dot{R} = R\hat{\mathbf{\omega}}_{b}$, where $\hat{\mathbf{\omega}}_{b}$ is the skew-symmetric matrix of the instantaneous angular velocity vector $\boldsymbol{\omega}_b = [\omega_{b_x}, \omega_{b_y}, \omega_{b_z}]^T $ expressed in the body frame. 

The system dynamics can be described as
\begin{equation}
    \begin{split}
    \mathbf{\dot{x}} = V\mathbf{R}\mathbf{e}_1,&~\mathbf{\ddot{x}} = \mathbf{g} + \mathbf{R}\mathbf{a}_v,\\
    \mathbf{\dot{R}} = \mathbf{R}\hat{\bm\omega}_v,~\dot{\bm{\omega}}_v &= \mathbf{J}^{-1}(-\bm{\omega}_v \times \mathbf{J}\bm{\omega}_v + \bm{\tau}),\label{eq:system_dynamics4}
\end{split}
\end{equation}
where $V = \lVert \dot{\mathbf{x}} \rVert$, $\mathbf{g} = [0~0~-g]^{\top}$ is the gravity vector in $\mathcal{I}$, $\mathbf{e}_1 = [1~0~0]^{\top}$ is the versor aligned with the $x$ direction of $\mathcal{V}$, $\mathbf{a_v} = [a_{v_x}~0~a_{v_z}]^\top$ contains the axial and normal accelerations, while $\boldsymbol{\omega}_v = [\omega_{v_x}~\omega_{v_y}~ \omega_{v_z}]^\top$ represents the angular velocity of $\mathcal{V}$ wrt. $\mathcal{I}$, expressed in $\mathcal{V}$ and $\hat{\bm\omega}_v$ its corresponding skew-symmetric matrix. Finally, $\mathbf{J} \in \mathbb{R}^{3 \times 3}$ describes the inertia acting on each direction of the body frame $\mathcal{B}$, while $\bm{\tau}$ expresses the torque applied on the system due to the action of control surfaces, like ailerons, elevators, and rudder. 
Moreover, as described in \cite{hauser1997aggressive}, to maintain coordinated flight conditions, the second and third components of the angular velocity $\boldsymbol{\omega}_v$, are constrained to be 
\begin{equation}
    \omega_{v_y} = -(a_{v_z} + g_{v_z})/V,~\omega_{v_z} = g_{v_y}/V,\label{eq:system_dynamics5}
\end{equation}
where $\mathbf{g}_{v} = \mathbf{R}^{\top} \mathbf{g}$.
Therefore, the coordinated flight conditions do not impose any constraints on $\omega_{v_x}$ of the FW. 

\subsection{Aerodynamics and Propulsion Model}
\label{sec:Aerodynamics_and_prop}
%The coordinated flight model is a so powerful mathematical notation, that despite being very concise, can describe the translational dynamics and attitude kinematics of a fixed wing aircraft, leveraging normal and axial accelerations, respectively $a_{v_x}$ and $a_{v_z}$ and the roll velocity $\omega_1$. 

More realistic attitude dynamics can be obtained by also including the acceleration due to lift, drag, and thrust, respectively $a_L$, $a_D$, and $a_T$, which are generally modeled as a function of the altitude $x_z$, airspeed $V_a$, and angle of attack $\alpha$. In this paper, we consider $a_L$, $a_D$, and $a_T$  to be~\cite{tseng1988calculation}

\begin{align}
a_L &= \frac{\sigma(x_z)V_a^2SC_L}{2m} + a_{L,0}, \\
a_D = &\frac{\sigma(x_z) V_a^2 S C_{d}}{2m},~a_T = T/m + a_D,
\end{align}
where $\sigma$, $S$, $T$, and $m$ are the air density, wing surface area, motor thrust and mass of FW. The lift coefficient $C_L$, initial lift acceleration $a_{L,0}$, and drag coefficient $C_D$ depend on the aerodynamic properties of the aircraft, including its shape and angle of attack. 
Therefore, the axial and normal acceleration inputs to the system are given by
\begin{equation}
    a_{v_x} = a_T \cos{\alpha} - a_D,~a_{v_z} = -a_T \sin{\alpha} - a_L.
\end{equation}


\subsection{Differential Flatness}
\label{sec:differential_flat}
%Adding here a general introduction to diff flat and how to find R des given the versors
%Mathematically, this can be expressed as:
%\begin{align}
%\dot{\mathbf{X}} &= f(\mathbf{X}, \mathbf{u}) \\
%(\mathbf{X}, \mathbf{u}) &= \Psi(z, \dot{z}, ..., z^i) 
%\end{align}
%where $\Psi(\dots)$ denotes the mapping function between the flat outputs $z$ and the system states $\mathbf{X}$ and the inputs $\mathbf{u}$, with $\text{dim}(\mathbf{z}) = \text{dim}(\mathbf{u})$.

This section provides an overview of the Differential Flatness and Feedback Trajectory Tracking blocks shown in Fig. \ref{fig:software_architecture}. A system is considered differentially flat if there exists a set of of flat outputs, such that the system's state and input can be fully described in terms of these outputs and their derivatives.
In the case of an FW system operating under the coordinated flight equations introduced in Section~\ref{sec:System_modeling}, it becomes a feedback linearizable system~\cite{hauser1997aggressive}, where the flat output is represented by the position $\mathbf{x}$, while the inputs to the model are $\mathbf{u} = [\dot{a}_{v_x}~\omega_{v_x}~\dot{a}_{v_z}]^\top$. 
Following \cite{hauser1997aggressive} and differentiating the acceleration expression $\ddot{\mathbf{x}}$ in eq. \eqref{eq:system_dynamics4}, we obtain $\mathbf{x}^{(3)} = \mathbf{R}(\bm{\omega}_v  \times \mathbf{a}_v + \dot{\mathbf{a}}_v)$, which is equivalent to
\begin{align}
\mathbf{x}^{(3)} &= 
\begin{bmatrix}
\omega_{v_y} a_{v_z} \\
\omega_{v_z} a_{v_x} \\
-\omega_{v_y} a_{v_x}
\end{bmatrix}
+ \mathbf{R}
\begin{bmatrix}
\dot{a}_{v_x} \\
-\dot{a}_{v_z}\omega_{v_x} \\
-\dot{a}_{v_z} 
\end{bmatrix}.
\end{align}

Inverting the following expression directly lead to the final differential flatness equation

%Diff flat equation 71 from IJRR paper 
\begin{equation}
\begin{aligned}
\begin{bmatrix}
\dot{a}_{v_x} \\
\omega_{v_x} \\
\dot{a}_{v_z}
\end{bmatrix} = 
\begin{bmatrix}
-\omega_{v_y}a_{v_z} \\
\omega_{v_z}a_{v_x}/a_{v_z} \\
\omega_{v_y}a_{v_x}
\end{bmatrix} + 
\begin{bmatrix}
1 & 0 & 0 \\
0 & -1/a_{v_z} & 0 \\
0 & 0 & 1
\end{bmatrix} \mathbf{R}^\top
\mathbf{x}^{(3)},
\end{aligned}
\label{eq:differential_flatness}
\end{equation}
where $\mathbf{R}=[\mathbf{r}_{x}~\mathbf{r}_{y}~\mathbf{r}_{z}]$ with $\mathbf{r}_{x} = \dot{\mathbf{x}}/\lVert{\dot{\mathbf{x}}}\rVert$, 
$\mathbf{r}_{z} = \mathbf{a}_{n} / a_{v_z}$, and $\mathbf{r}_{y} = \mathbf{r}_{z} \times \mathbf{r}_{x}$. 
Therefore, $a_{v_z} = - \lVert{\mathbf{a}_{n}}\rVert $ where $\mathbf{a}_{n}$ is found by the projection of $\ddot{\mathbf{x}}$ in the normal plane as $\mathbf{a}_{n} = (\ddot{\mathbf{x}}- \mathbf{g} - a_{v_x} \mathbf{r}_{v_x})$, where $a_{v_x} = \mathbf{r}_{x}^\top (\ddot{\mathbf{x}} - \mathbf{g})$. To respect the coordinated flight condition $a_{v_y} = 0$. The differential flatness equation only holds if the flatness constraints, namely $\mathbf{\dot{x}} \neq 0$ and $a_{v_z} \neq 0$, are satisfied. This is intuitive, as the aircraft's lack of hovering capability and the inability to control the system when the aircraft is perpendicular to the desired trajectory direction make these constraints necessary.
%makes possible to find the axial acceleration $a_{v_x} = \mathbf{r}_{v_x}^T (\ddot{\mathbf{x}} - \mathbf{g}) $. The normal acceleration $a_{v_z}$ is found by the projection of $a_{v_x}$ in the normal plane as  $a_{v_z} = -\lVert(\ddot{\mathbf{x}\rVert - g - a_{v_1} \mathbf{r}_{v_x}^T)}$. Thus the full matrix $R$, describing the desired attitude of $\mathcal{V}$ given the input $\mathbf{z}^{(3)}$ can be retrieve knowing that $\mathbf{r}_{v_z} = a_{v_z} / \lVert a_{v_z}\rVert$ and $\mathbf{r}_{v_y} = \mathbf{r}_{v_z} \times \mathbf{r}_{v_x}$, since no accelerations are wanted on $\mathcal{V}_y$.

We define the system's control input sent to the inner attitude controller~\cite{REINHARDT202191,Coates} the desired orientation matrix $\mathbf{R}_{c}$, expressed through Euler angles $\theta_c$, $\phi_c$, and $\psi_c$, along with angular velocities $\omega_{v_x}$, $\omega_{v_y}$, and axial acceleration $a_{v_x}$ represented as thrust $a_T$. This forms the commanded control input $\mathbf{u}_{c} = [\theta_c~\phi_c~\omega_{v_x}~\omega_{v_y}~a_T]^\top$. Specifically, $\mathbf{u}_{c}$ is derived by first calculating $\mathbf{R}_c$ from the previous $\mathbf{R}$ expression. Subsequently, we consider the following cascade PID loop to compute the commanded jerk
\begin{equation}
    \mathbf{x}_{c}^{(3)} = \mathbf{x}^{(3)}_{r} + k_2 \ddot{\mathbf{e}} + k_1 \dot{\mathbf{e}} + k_0 \mathbf{e},
\end{equation}
where $\mathbf{e} = \mathbf{x}_{r}(t) - \mathbf{x}(t)$, and $k_2, k_1, k_0$ the feedback gains. Finally, based on the differential flatness model in eq. \eqref{eq:differential_flatness},  we derive $\omega_{v_x}~\text{and~}\omega_{v_y}$ considering $\mathbf{x}_{c}^{(3)}$ and $\mathbf{R}_c$ in place of $\mathbf{x}^{(3)}$ and $\mathbf{R}$ respectively. This allows to achieve a trajectory tracking given the state feedback $\mathbf{x}(t)$.


\subsection{Trajectory Time Parametrization}
Despite the strength of the differential flatness approach, the  desired tangential acceleration along the trajectory can vary depending on how the trajectory is formulated with respect to time. Due to the natural minimization of the tracking error $\mathbf{e}$ towards the reference trajectory tracking point, an abrupt change of the desired thrust may happen if the trajectory presents variations in the reference velocities $\dot{\mathbf{x}}_{r}$ and acceleration $\ddot{\mathbf{x}}_{r}$.
In this condition, the FW can slow down below a safe cruising airspeed, producing a loss of airflow and control of the aerodynamic surfaces. 

To prevent such a scenario, we introduce a path parameterization variable $s(t)$ that defines how the desired trajectory values are allocated along the path $\mathbf{x}_{r}:= \mathbf{x}_{r}(s(t))$ for $t \geq 0$, where $s$ represents the distance along the desired path. This parameterization enables dynamic inversion of trajectory $\mathbf{x}_{r}(t) $ based on the distance travelled while maintaining a constant cruising velocity. Therefore, eq.~\eqref{eq:differential_flatness} is modified as 


%ensures that the axial acceleration $a_{v_x}$ remains constant and predetermined for the entirety of the trajectory; however, it also means that $a_{v_x}$ is no longer available as control input in equation \ref{eq:differential_flatness}. Thus, apath parameterization $s(t)$ variable is introduced, which follows the desired trajectory $\rho_s(t):= \rho(s(t))$ for $t \geq 0$, where $s$ is the distance along the desired path. This path parameterization enables dynamic inversion of trajectory $\rho(t)$ based on the distance travelled while maintaining a constant cruising velocity. It also provides an alternative state representation where $\dot{a}_{v_x}$ can be expressed as $s^{(3)}$, allowing us to modify the differential flatness equation \ref{eq:differential_flatness} in \cite{hauser1997aggressive} as shown below:

\begin{equation}
\begin{split}
& \mathbf{M}
\begin{bmatrix}
s^{(3)} \\
\omega_{v_x} \\
\dot{a}_{v_z}
\end{bmatrix} = 
\begin{bmatrix}
a_{v_z}\omega_{v_y} + \dot{a}_{v_z}\\
a_{v_x}\omega_{v_z} \\
-a_{v_x}\omega_{v_y}
\end{bmatrix} \\
 &- 
\mathbf{R}^\top \left[3\frac{\delta^2\mathbf{x}_{r}}{\delta s^2}\ddot{s}\dot{s} + \frac{\delta^3\mathbf{x}_{r}}{\delta s^3}\dot{s}^3 + k_2\mathbf{\dddot{e}} + k_1\mathbf{\dot{e}} + k_0\mathbf{e}\right],
\end{split}
\label{eq:time_param_differential_flatness}
\end{equation}
where $\mathbf{M}$ is the decoupling matrix represented as
\begin{equation}
\mathbf{M}
 = 
\begin{bmatrix}
\vdots & 0 & 0\\
\mathbf{R}^\top\frac{\delta\mathbf{x}_{r}}{\delta s}& a_{v_z} & 0 \\
\vdots & 0 & -1
\end{bmatrix}. 
\label{eq:time_param_M}
\end{equation} 

%The primary limitation of this method is to avoid flying perpendicular to the desired path.
%If the plane flies perpendicular, the first column of $R$ (roll) will become orthogonal to $\rho^{'}$, causing $\mathbf{M}$ to become singular and non-invertible. In practice, this situation is unlikely to occur unless the controller is not properly initialized. 

%trackin as just an equation 

%trajectory parametrization 


% ----------------------------------------------------------
% Extra

 %Here stating describing diff flat


%This section outlines the conventions used for defining reference frames in fixed-wing systems, their trajectory, and related control systems. It also introduces the dynamic model of a fixed-wing, which describes its motion, state, and desired control input, which will be used for its coordinated flights and differential flatness formulation.

% Across the paper, we will refer to derivatives of position in $W$ using the notation $\mathbf{\dot{\Delta}, \ddot{\Delta}, \dddot{\Delta}}$. 

%At the end, visualization of the proposed system architecture is depicted in Fig \ref{fig:software_architecture}, which highlights the adaptability and flexibility of the software stack that has been rigorously tested in both simulation and real-world scenarios. The hardware and software configurations for real-world experiments are explained in details in the experimental results section. 


%\subsection{Coordinate System:}

%As shown in the figure, we follow the standard fixed-wing aerodynamic coordinate system convention. The inertial frame $I$ is defined by three axes in the Forward-Left-Up (FLU) convention, while $I_{enu}$ refers to the inertial frame in the East-North-Up (ENU) convention. The fixed-wing rigid body frame $B$ is represented by three axes $[b_x, b_y, b_z]^T$, which are aligned with the aircraft’s longitudinal, lateral, and vertical axes, respectively. The position of body frame $B$ is represented by $\mathbf{x} = [x, y, z]^T$ in the inertial frame $I$ as a translation vector. An intermediate vehicle frame $V$ is introduced, aligned with the inertial frame, and used to express the relative orientation of the body frame $B$ with respect to inertial frame $I$. The transformation between $I$ and $B$ is represented by the transformation matrix $\mathbf{T} = [\mathbf{R}, \mathbf{x}^T]$, where $\mathbf{R}$ is the rigid body rotation matrix between that defines the orientation of frame $B$ relative to $I$.  


%\subsection{System Dynamics:}

% writing draft and will make better once content is written (currently working on this)

%The state of the fixed-wing system $\mathbf{X}$ is defined by several key components: the aircraft position $\mathbf{x} = [x, y, z]^T \in \mathbb{R}^3$; its velocity $\mathbf{v}_{b} = [v_{b_x}, v_{b_y}, v_{b_z}]^T$  represented in the body frame $B$; orientation $\mathbf{R}$ described by Euler angles roll ($\phi$), pitch ($\theta$), yaw ($\psi$) in the intermediate vehicle frame $V$, and the angular velocity $\mathbf{\omega}=[\omega_{b_x}, \omega_{b_y}, \omega_{b_z}]^T$ around each body axis, represented in the body frame $B$. 

%The whole body of a fixed wing is considered a single rigid body and modeled using the Runge-Kutta 4 (RK4) method for its translation and rotational dynamics. The system dynamic model is as follows:


% \begin{equation}
% \begin{subequations}
%     \begin{align}
         
%         \label{eq:1a} \\
%         d - e &= f 
%         \label{eq:1b} \\
%         g \times h &= i 
%         \label{eq:1c}
%     \end{align}
% \end{subequations}
% \end{equation}

%The motion of the system is defined by the normal and axial acceleration ($a_{v_z}$ and $a_{v_x})$ and the roll rate $\omega_x$. Our system is designed to follow the coordinated flights. The state is defined by the position $\mathbf{x} \in \mathbb{R}^3$ of the aircraft represented in $I$ frame, velocity $\mathbf{v}$, and orientation $\mathbf{R}$. The evaluation of orientation $R$ can be expressed by the instantaneous angular velocity $\mathbf{\omega}=[\omega_x, \omega_y, \omega_z]^T$ expressed in the intermediate velocity frame. 

%\subsection{Fixed Wing Coordinated Flight Condition}

%The planning and control developed and described in this paper respect the rule of coordinated flight, which is defined as a condition where the body velocity of the vehicle is contained on the longitudinal plane, thus defining $v^B_y = v^B_z = 0$. The coordinated flight condition is expressed in the velocity frame $\mathcal{V}$, which is different from $B$ by the angle of attack of the vehicle, and it can be described by the rotation $\mathbf{R}_v$ from the body frame $B$. Consequently, velocities and accelerations can be mapped back in the world frame as $\dot{\Delta} = \mathbf{R}_v \mathbf{v}_v$ and $\ddot{\Delta} = \mathbf{g} + \mathbf{R}_v \mathbf{a}_v$, where $\mathbf{g}$ is the gravity vector. To keep the system in condition of coordinated flight the pitch and yaw rates are respectively constrained to be: $\omega_{v_y} = -(a_{v_z} + g_{v_z})/V$ and $\omega_{v_z} = g_{v_y}/V$, where 

%$V = \abs{\dot{\Delta}}$ and $g_{v_y}$ 


%and $g_{v_z}$ represents the component of the gravity vector projected on $\mathbf{v}_{y}$ and $\mathbf{v}_{z}$.

%In the same expression, components $a_{v_x}$ and $a_{v_z}$ represent the axial and normal acceleration of the vehicle, and their derivatives will be the output of the differential flat model presented in the next section

%sent to an inner attitude controller using a cascade PID loop where angular velocities $\omega_{v_x}$ and $\omega_{v_y}$ and the axial acceleration $a_{v_x}$, expressed in terms of thrust $T$ .

%Considering the differential flatness model just derived, the final input to our system consists in the desired orientation matrix $\mathbf{R}$, decomposed in the eulerian angles $\theta$, $\phi$ and $\psi$, the angular velocities $\omega_{v_x}$ and $\omega_{v_y}$ and the axial acceleration $a_{v_x}$, expressed in terms of thrust $T$, defining the vector of the desired controller inputs $\mathbf{u}_{des} = [\theta, \phi, \omega_{v_x}, \omega_{v_y}, T]$, forwarded to an inner attitude controller based on a cascade PID loop. 

%However, as shown in eq. \ref{eq:differential_flatness}, this expression alone stabilize the system on a desired trajectory $\rho(t)$, without accounting for deviation from the desired path $\mathbf{e}(t)$. To integrate the trajectory deviation feedback into eq. \ref{eq:differential_flatness}, the desired trajectory jerk i.e., flat output derivative $\mathbf{z}^{(3)}$ can be modified to include the closed loop error $\mathbf{e} = \rho(t) - \mathbf{x}(t)$, where where $\mathbf{x}(t)$ is the current position of system and $\rho(t)$ is the reference position at given time $t$. In particular, as visualized in \cite{Bry2015AggressiveFO} final equation for $\mathbf{z}^{(3)}$ can be expressed as:
\section{Trajectory Planning}
\label{sec:Planning}

% problem definition for motion planning 
%In the previous section, we introduced the differentially flat model for FW constrained to a coordinated flying condition. 
We focus on designing an optimal, dynamically feasible trajectory for a FW that leverages the differential flatness property and employs Bernstein polynomials adhering to the following conditions

 %Using this concise yet powerful representation, it is possible to obtain the desired input $\mathbf{u}$ to the vehicle directly from the third derivative of the desired trajectory $\rho(t)$, which has to be continuous and differentiable, in order to guarantee smooth roll rate control. 
 
\begin{itemize}
    \item The axial velocity of the plane $\dot{x}_{x} \neq 0$.
    \item The trajectory should satisfy that $a_{v_z} \neq 0 $.
    \item Bounding the maximum curvature $\kappa$ of the trajectory.
\end{itemize}
We formulate a convex quadratic optimization of Bernstein polynomials to minimize the trajectory jerk $\mathbf{x}_{r}^{(3)}(t)$ (input in eq. \eqref{eq:differential_flatness}) while keeping velocity, acceleration and curvature $\kappa$ constraints within specified bounds. %Since the FW aircraft are constrained to continuous forward motion.%, each Bernstein trajectory is followed by a circular loitering motion with a predetermined radius when no new trajectory is defined. This loitering phase serves as the fixed-wing equivalent of the hovering phase seen in quadrotors or tail-sitters, as described in \cite{lu2024trajectory}.

%that will span  entire flight is significantly challenging compared to multi-rotor platforms \cite{loianno2016estimation}. Unlike quadrotors or tail-sitters \cite{lu2024trajectory}, which can hover in place, fixed-wing must continue its forward motion. This constraint forces the fixed wing either to immediately transition into a loiter motion or begin a new trajectory as soon as it completes its current one. In addition to a smooth transition between trajectories, it is important to respect the kino-dynamic constraint of a fixed-wing while generating these trajectories, particularly its minimum turning radius, which directly influences the required roll angle during flight. Respecting these kino-dynamic constraints and ensuring smooth transitions between trajectories is important to maintain stability and control throughout the entire flight. 

% solution discussion about a seamless transition
%To ensure the seamless transition between trajectories, we generate continuous and dynamically feasible trajectories within the vehicle's flat output space, defined by the vector $\mathbf{x}$. Throughout the entire flight span, the vehicle's trajectory is composed of three distinct types of polynomial trajectories: i) line trajectory, ii) loiter trajectory, and iii) Bernstein Trajectory. Each of these trajectories is defined by a sequence of flat outputs, which correspond to the desired positions and their higher derivatives up to the jerk at each time $t$ from the initial time $t_0$ to a final time $t_f$ for the respective trajectory.




\subsection{Bernstein Trajectory}
\label{sec:planner}

A Bernstein polynomial shows interesting properties in terms of smoothness and ability to impose global spatial constraints compared to time-based polynomials~\cite{kielas2019bebot,kielas2022bernstein}. For a given $m_j$  trajectory, it can be described by the following form of degree $n$
\begin{equation}
C_{n,m_j}(t) = \sum_{i=0}^{n}\mathbf{p}_{i,n}^{m_j}\beta^n_i(t),   \quad t\in [t_0, t_f]
\label{eq:bernstein_equation}
\end{equation}
where $\mathbf{p}_{i,n}^{m_j}$ are the Bernstein coefficient or control points of  size $n$ control, and $\beta^n_i(t)$ is the Bernstein basis. The $k^{\text{th}}$ derivative of the polynomial can be obtained as
\begin{equation}
\frac{d^{k}}{dt^{k}}C_{n,m_j}(t) = \frac{n!}{(n-k)!(t_f - t_0)^k} \sum_{i=0}^{n-k} {{\mathbf{p}}^{{m_j}^{'}}_{i,n-k}}\beta_{i}^{n-k}(t),
\label{eq:bernstein_derivative}
\end{equation}
with ${\mathbf{p}}^{{m_j}^{'}}_{i,n-k} = \mathbf{p}_{i,n}^{m_j}\mathbf{D}_k$ and $\mathbf{D}_k = \text{diag}(\mathbf{c}\circledast^k, \mathbf{c}\circledast^k, \cdots, \mathbf{c}\circledast^k)$ is the Differential matrix with $ \mathbf{c} = [-1, 1]$ convoluted $k$ times. Considering $M+1$ waypoints, a full trajectory $\mathbf{x}_{r}(t)$ can be modeled by stacking together $M$ Bernstein polynomials connected at the extremal points as

%A trajectory $\rho(t)$ can be efficiently modeled as a piece-wise stacking of consecutive Bernstein polynomial segments passing through multiple waypoints, combined into a single optimized Bernstein polynomial and connected at the extremes in order to be continuous-time in the flat output space $\mathbf{x}=\{x, y, z\}$ and its higher derivatives. 

%These trajectories can be efficiently optimized and evaluated under various constraints, including velocity, acceleration, and, especially in our case, roll rate and turning radius. By representing continuous-time trajectories as piecewise Bézier curves passing through multiple waypoints, these can be combined into a single Bernstein polynomial after optimization, ensuring both computational efficiency and respect the imposed constraints.

 %The general equation of Bernstein polynomial trajectory for a single dimension is represented as \cite{kielas2022bernstein}
%\begin{equation}
%C_n(t) = \sum_{i=0}^{n}\mathbf{p}_{i,n}\beta^n_i(t),   \quad t\in [t_0, t_f]
%\label{eq:bernstein_equation}
%\end{equation}

%where $\mathbf{p}_{i,n}$ are the Bernstein coefficient or control points, with each segment having a total of $n$ control points, and $\beta^n_i(t)$ is the Bernstein basis. To represent the complete trajectory passing through $M+1$ waypoints, we employ a set of $M$ Bernstein polynomials, where $M$ is the total number of trajectory segments:

\begin{equation}
\mathbf{x}_{r}(t)  = 
\begin{cases} 
    \sum_{i=0}^{n}\mathbf{p}_{i,n}^{m_1}\beta_i^n(T_1 - t) \quad \text{for} \ t\in [0, T_1]\\
    \sum_{i=0}^{n}\mathbf{p}_{i,n}^{m_2}\beta_i^n(T_2 - t) \quad \text{for} \ t\in [T_1, T_2] \\
    \vdots \\ 
    \sum_{i=0}^{n}\mathbf{p}_{i,n}^{M}\beta_i^n(T_{M}-t) \quad \text{for} \ t\in [T_{M-1}, T_M]
\end{cases}
\label{eq:piecewise_bernstein_equation}
\end{equation}
where $\mathbf{p}_{i,n}^{m_j}$ is the $i^{th}$ control point of the $m_j$ sub trajectory, with $j \in [1, M]$, and the time instants $T_1, T_2, \dots, T_M$ represent the allocated time for each of sub trajectory. 

%\begin{itemize}
%\item $P^{'}_{i,n-m} = P_{i,n}\mathbf{D}_m$
%\item Differential Matrix: $\mathbf{D}_m = \begin{bmatrix}\mathbf{s}\circledast^m  & 0 & \cdots  & 0 \\ 0 & \mathbf{s}\circledast^m & \cdots  & 0 \\\vdots  & \vdots  & \ddots  & \vdots  \\0 & 0 & \cdots  & \mathbf{s}\circledast^m\end{bmatrix}$
%\item $\mathbf{s} = [-1, 1] \quad \text{and} \quad \underbrace{s \circledast s \circledast \cdots \circledast s}_{m}=\mathbf{s}\circledast^m$
%\end{itemize}

To find the Bernstein Coefficients $\mathbf{p}$ we formalize  a Convex Quadratic Programming (QP) problem \cite{mao2023robust}
\begin{equation}
\begin{aligned}
\text{min} \quad & \mathbf{p}_d^T\mathbf{Q} \mathbf{p}_d\\
\text{s.t.} \quad & \mathbf{A}_{eq}\mathbf{p}_d = \mathbf{b}_{eq} \\
& \mathbf{A}_{ineq}\mathbf{p}_d \le  \mathbf{b}_{ineq}
\end{aligned}
\end{equation}
where $\mathbf{Q} = \text{diag}(Q_1, \hdots, Q_M)$ with $Q_i \in \mathbb{R}^{n \times n}$ representing the Hessian semi-definite matrix of the objective function, related to the $n$ number of Bernstein Coefficients each sub trajectory. The vector $\mathbf{p}_d$, with dimension $M \times n$, contains the Bernstein coefficients to be optimized for each spatial dimension $d$. To ensure continuity in position and higher derivatives between the segments, the optimization problem is subject to various equality and inequality constraints, which are represented by the matrices $\mathbf{A}_{eq}, \mathbf{A}_{ineq}$, and vectors $\mathbf{b}_{eq}, \mathbf{b}_{ineq}$


% Matrix $\mathbf{A}_{eq}, \mathbf{A}_{ineq}$, and vector $\mathbf{b}_{eq}, \mathbf{b}_{ineq}$ are derived from the equality and inequality constraints imposed by user for each dimension $d$. 
%The optimization problem is solved using an off-the-shelf OOQP \cite{gertz2003object} convex solver. 

% The vectors $\mathbf{b}_{id}$ consists of all number $i$ of constraints imposed by the user for each dimension $d$. Finally the matrix $\mathbf{A}  = \text{diag}(A_1, \hdots, A_j, \hdots, A_m)$ is composed by submatrix $A_j$ which one with dimension $A_j \in \mathbb{R}^{i \times n}$ and it is stacked for the $d$ dimesions of the polynomial.

\begin{enumerate}
    \renewcommand{\labelenumi}{\roman{enumi}.}
    \item \textit{Endpoint constraint:}
    Considering a starting time $t_0$ and an ending time $t_{f}$, we constrain $\mathbf{x}_{r}$ at the reference waypoints position $\mathbf{x}_{r}$, velocity $\dot{\mathbf{x}}_{r}$, and acceleration $\ddot{\mathbf{x}}_{r}$
     \begin{equation}\begin{aligned}
        C_{n,0}^{(k)}(t_0) = \mathbf{x}^{(k)}(t_0), \qquad C_{n,M}^{(k)}(t_f) = \mathbf{x}^{(k)}(t_f) 
    \end{aligned}.\end{equation}

    \item \textit{Continuity Constraints:}
    %Given a set $S_{\mathcal{B}}$ of $m_{i+1}$ waypoints, defined by a starting and ending time $t_{{m}_0}$ $t_{{m}_f}$,
    The goal is to ensure the continuity in position and higher derivatives of the trajectory $\mathbf{x}_r(t)$ at the junction of the $M$ sub trajectories as

    % \begin{equation}\begin{aligned}
    % \mathbf{C_m}(t_f) = \mathbf{C_{m+1}}(0) \\
    % \left\|  \mathbf{\dot{C}_m}(t_f) \right\| = \left\| \mathbf{\dot{C}_{m+1}}(0) \right\| \\
    % \left\|  \mathbf{\ddot{C}_m}(t_f) \right\| = \left\| \mathbf{\ddot{C}_{m+1}}(0) \right\|
    % \end{aligned}\end{equation}

    \begin{equation}\begin{aligned}
        C_{n,m}(t_{f}) = C_{n,m+1}(t_{{0}}). \\
    \end{aligned}\end{equation}



    
    \item \textit{Dynamic feasibility Constraints:}
    Given the FW dynamics, the curvature $\kappa = f(\dot{\mathbf{x}}_{r_x}, \dot{\mathbf{x}}_{r_y}, \ddot{\mathbf{x}}_{r_x}, \ddot{\mathbf{x}}_{r_y})$ evaluated from $t_0$ to $t_f$ of a given trajectory, needs to be constrained for its entire duration within the range  $\kappa_{min} \leq \kappa \leq \kappa_{max}$ to be considered feasible in order to avoid exceeding the maximum roll angle of the aircraft. Due to the non-linear nature of the curvature function $\kappa$, we apply a Taylor expansion around the equilibrium point to linearize the constraint, allowing us to maintain the original convex optimization problem formulation. The constraint  $k$ on the lineared curve is
    
    %The curvature $\kappa$ is a nonlinear function of $v_x, v_y, a_x$ and $a_y$ as expressed in the following equation:

    %\begin{equation}\begin{aligned}
   % k = f(v_x, v_y, a_x, a_y) = \frac{v_xa_y - a_xv_y}{(v_x^2 + v_y^2)^{3/2}}
    %\label{eq:curvature_k}
    %\end{aligned}\end{equation}

    \begin{equation}
    \begin{split}
        % &\phantom{=} k_{min} \leq k \leq k_{max} \\
        & \kappa_{min} \leq 
f(\dot{\mathbf{x}}_{r_x}, \dot{\mathbf{x}}_{r_y}, \ddot{\mathbf{x}}_{r_x}, \ddot{\mathbf{x}}_{r_y}) + \\
        &\begin{bmatrix} 
            \frac{\partial f}{\partial  \dot{\mathbf{x}}_{r_x}} &  \frac{\partial f}{\partial  \dot{\mathbf{x}}_{r_y}} & 
            \frac{\partial f}{\partial  \ddot{\mathbf{x}}_{r_x}} & \frac{\partial f}{\partial  \ddot{\mathbf{x}}_{r_y}}
        \end{bmatrix}
        \begin{bmatrix} 
            \dot{\mathbf{x}}_{r_x} - \dot{\mathbf{x}}_{r_x}(t_{rp}) \\\dot{\mathbf{x}}_{r_y} - \dot{\mathbf{x}}_{r_y}(t_{rp}) \\ \ddot{\mathbf{x}}_{r_x} - \ddot{\mathbf{x}}_{r_x}(t_{rp})  \\ \ddot{\mathbf{x}}_{r_y} - \ddot{\mathbf{x}}_{r_y}(t_{rp}) 
        \end{bmatrix} 
        \leq \kappa_{max}.
    \end{split}
    \label{eq:curvature}
    \end{equation}
    where $t_{rp} \in [t_0, t_f]$ represents the time instant where the linearization is applied.
    In particular, for a continuous linearization of the entire trajectory around a local point, a replanning strategy visible in Fig. \ref{fig:planned_mission} (top right) is applied at constant intervals. To avoid discontinuities between the current trajectory $\mathbf{x}_{r, j-1}$ and new replanned trajectory $\mathbf{x}_{r, j}$, we account for the optimization time $t_{opt}$ such that $\mathbf{x}_{r, j}(t_0) = \mathbf{x}_{r, j-1}(t + t_{opt})$.
    
    %The Taylor series approximation of the nonlinear constraint $k$ is accurate near the equilibrium point but insufficient for the entire trajectory. To address this problem, we implemented periodic replanning to maintain the $k$-constaint throughout the trajectory. As shown in Figure \ref{fig:planned_mission} (upper right), the replanning strategy is applied at constant intervals. In order  This transition point ensures continuity, allowing the aircraft to seamlessly switch to the new replanned trajectory when it reaches the $\rho(s(t))_j$. The real-world implementation and effectiveness of this strategy are discussed in the experimental results section. 
   
    
\end{enumerate}




\section{Experiments}
\label{sec:experiment}

\subsection{Experimental Setup}
\label{sec:exp_setup}
The experiments are mainly conducted on SD1.5 \cite{sd1} and SDXL \cite{sdxl} without refiner. The LRM is first trained on Pick-a-Pic and then used to fine-tune diffusion models through LPO. Unless otherwise specified, we employ \textit{homogeneous optimization}.

\textbf{LRM Training.} We denote the LRM based on SD1.5 and SDXL as LRM-1.5 and LRM-XL, respectively. They are trained on the filtered Pick-a-Pic v1 \cite{pickscore} as clarified in Sec.\;\ref{sec:lrm_train}. The $gs$ in the VFE module is set to 7.5. 
More details are in \cref{sec:experimental_detail}.

\textbf{LPO Training.} The same 4k prompts in SPO are used for the LPO training, randomly sampled from the training set of Pick-a-Pic v1. The DDIM scheduler \cite{ddim} with 20 inference steps is employed. We use all steps for sampling and training, \ie $t\in[0,50,...,900,950]$. The dynamic threshold range $[th_{min}, th_{max}]$ is set to $[0.35, 0.5]$ for SD1.5 and $[0.45, 0.6]$ for SDXL. The $\beta$ in Eqn.\;(\ref{eq:spo_loss}) is set to 500 and the $K$ in the sampling process is set to 4. Further details can be found in \cref{sec:experimental_detail}.

\begin{table}[t]
    \centering
    \vspace{-2.5mm}
    \caption{General and aesthetic preference scores on Pick-a-Pic validation unique set. $^*$ denotes the metrics are copied from \cite{spo}. Others are evaluated using the official model.}
    \vskip 0.05in
    \label{tab:preferenece_eval}
    \scriptsize
    \setlength{\tabcolsep}{1.0mm}{
    \scalebox{1.1}{
    \begin{tabular}{l c c c c c}
         \toprule
         Method & PickScore & ImageReward & HPSv2 & HPSv2.1 & Aesthetic \\
         \midrule
         \textcolor{gray}{SD1.5} & & & & & \\
         \hspace{1pt} Original & 20.56 & 0.0076 & 26.46 & 24.05 & 5.468 \\
         \hspace{1pt} $^*$DDPO & 21.06 & 0.0817 & - & 24.91 & 5.591 \\
         \hspace{1pt} $^*$D3PO & 20.76 & -0.1235 & - & 23.97 & 5.527 \\
         \hspace{1pt} Diff.-DPO & 20.99 & 0.3020 & 27.03 & 25.54 & 5.595 \\
         \hspace{1pt} SPO & 21.22 & 0.1678 & 26.73 & 25.83 & 5.927 \\
         \rowcolor{cyan!15}\hspace{1pt} LPO & \textbf{21.69} & \textbf{0.6588} & \textbf{27.64} & \textbf{27.86} & \textbf{5.945} \\
         \midrule
         \textcolor{gray}{SDXL} & & & & & \\
         \hspace{1pt} Original & 21.65 & 0.4780 & 27.06 & 26.05 & 5.920 \\
         \hspace{1pt} Diff.-DPO & 22.22 & 0.8527 & 28.10 & 28.47 & 5.939 \\
         \hspace{1pt} MaPO & 21.89 & 0.7660 & 27.61 & 27.44 & 6.095 \\
         \hspace{1pt} SPO & 22.70 & 0.9951 & 28.42 & 31.15 & 6.343 \\
         \rowcolor{cyan!15}\hspace{1pt} LPO & \textbf{22.86} & \textbf{1.2166} & \textbf{28.96} & \textbf{31.89} & \textbf{6.360} \\
         \bottomrule
    \end{tabular}}}
    % \vspace{-3mm}
    \vskip -0.15in
\end{table}


\begin{table*}[t]
    \vspace{-2.5mm}
    \caption{Quantitative results on T2I-CompBench++ \cite{t2i_compbench}.}
    \vskip 0.05in
    \label{tab:t2i_eval}
    \centering
    \scriptsize
    \setlength{\tabcolsep}{1.8mm}{
    \scalebox{1.1}{
    \begin{tabular}{c l c c c c c c c c}
         \toprule
         Model & Method & Color & Shape & Texture & 2D-Spatial & 3D-Spatial & Numeracy & Non-Spatial & Complex \\
         \midrule
         \multirow{4}{*}{SD1.5} & Original \cite{sd1} & 0.3783 & 0.3616 & 0.4172 & 0.1230 & 0.2967 & 0.4485 & 0.3104 & 0.2999 \\
         & Diff.-DPO \cite{diffusion_dpo} & 0.4090 & 0.3664 & 0.4253 & 0.1336 & 0.3124 & 0.4543 & \textbf{0.3115} & 0.3042 \\
         & SPO \cite{spo} & 0.4112 & 0.4019 & 0.4044 & 0.1301 & 0.2909 & 0.4372 & 0.3008 & 0.2988 \\
         & \cellcolor{cyan!15}LPO & 
         \cellcolor{cyan!15}\textbf{0.5042} &
         \cellcolor{cyan!15}\textbf{0.4522} & 
         \cellcolor{cyan!15}\textbf{0.5259} & 
         \cellcolor{cyan!15}\textbf{0.1928} & 
         \cellcolor{cyan!15}\textbf{0.3562} & 
         \cellcolor{cyan!15}\textbf{0.4845} & 
         \cellcolor{cyan!15}0.3110 &
         \cellcolor{cyan!15}\textbf{0.3308}\\
         \midrule
         \multirow{5}{*}{SDXL} & Original \cite{sdxl} & 0.5833 & 0.4782 & 0.5211 & 0.1936 & 0.3319 & 0.4874 & 0.3137 & 0.3327 \\
         & Diff.-DPO \cite{diffusion_dpo} & 0.6941 & 0.5311 & 0.6127 & 0.2153 & 0.3686 & 0.5304 & \textbf{0.3178} & 0.3525 \\
         & MaPO \cite{mapo} & 0.6090 & 0.5043 & 0.5485 & 0.1964 & 0.3473 & 0.5015 & 0.3154 & 0.3229 \\
         & SPO \cite{spo} & 0.6410 & 0.4999 & 0.5551 & 0.2096 & 0.3629 & 0.4931 & 0.3098 & 0.3467 \\
         & \cellcolor{cyan!15}LPO & 
         \cellcolor{cyan!15}\textbf{0.7351} & 
         \cellcolor{cyan!15}\textbf{0.5463} & \cellcolor{cyan!15}\textbf{0.6606} &
         \cellcolor{cyan!15}\textbf{0.2414} &
         \cellcolor{cyan!15}\textbf{0.4075} &
         \cellcolor{cyan!15}\textbf{0.5493} &
         \cellcolor{cyan!15}0.3152 &
         \cellcolor{cyan!15}\textbf{0.3801}\\
         \bottomrule
    \end{tabular}}}
    \vspace{-2mm}
    % \vskip -0.1in
\end{table*}


\begin{table*}[t]
    \begin{minipage}{0.63\linewidth}
        \vspace{-2mm}
        \caption{Quantitative results on GenEval \cite{geneval}.}
        \vskip 0.05in
        \label{tab:geneval}
        \centering
        \scriptsize
        \setlength{\tabcolsep}{1.1mm}{
        \scalebox{1.1}{
        \begin{tabular}{l l c c c c c c c}
             \toprule
             Model & Method & \makecell[c]{Single \\ Object} & \makecell[c]{Two \\ Object} & Counting & Colors & Position & \makecell[c]{Color \\ Attribution} & Overall \\
             \midrule
             \multirow{4}{*}{SD1.5} & Original & 97.50 & 37.12 & 34.69 & 75.53 & 3.75 & 6.75 & 42.56 \\
             & Diff.-DPO & \textbf{98.44} & 38.38 & 36.25 & 77.93 & 4.50 & 7.25 & 43.79 \\
             & SPO & 95.00 & 33.84 & 32.50 & 69.95 & 4.25 & 7.25 & 40.46 \\
             & \cellcolor{cyan!15}LPO & \cellcolor{cyan!15}97.81 &
             \cellcolor{cyan!15}\textbf{54.80}&
             \cellcolor{cyan!15}\textbf{40.94}&
             \cellcolor{cyan!15}\textbf{79.52}&
             \cellcolor{cyan!15}\textbf{7.00}& 
             \cellcolor{cyan!15}\textbf{10.25}&
             \cellcolor{cyan!15}\textbf{48.39}\\
             \midrule
             \multirow{5}{*}{SDXL} & Original & 93.75 & 63.38 & 30.94 & 80.05 & 9.25 & 19.00 & 49.40  \\
             & Diff.-DPO & 99.06 & 76.52 & \textbf{45.00} & 88.83 & 11.50 & 25.75 & 57.78 \\
             & MaPO & 95.63 & 68.94 & 32.19 & 83.51 & 11.50 & 17.75 & 51.59 \\
             & SPO & 94.38 & 69.44 & 31.88 & 81.65 & 10.25 & 15.50 & 50.52  \\
             & \cellcolor{cyan!15}LPO & \cellcolor{cyan!15}\textbf{99.69} &
             \cellcolor{cyan!15}\textbf{81.57} &
             \cellcolor{cyan!15}43.75 &
             \cellcolor{cyan!15}\textbf{89.10} &
             \cellcolor{cyan!15}\textbf{14.00} &
             \cellcolor{cyan!15}\textbf{27.50} & 
             \cellcolor{cyan!15}\textbf{59.27}\\
             \bottomrule
        \end{tabular}}}
        \vskip -0.1in
    \end{minipage}
    \hfill
    \begin{minipage}{0.35\linewidth}
        \vspace{-2mm}
        \caption{Comparisons of training speed.}
        \vskip 0.05in
        \label{tab:speed}
        \centering
        % \footnotesize
        \scriptsize
        \setlength{\tabcolsep}{1.1mm}{
        \scalebox{1.1}{
        \begin{tabular}{l c c c}
             \toprule
             Method & \makecell[c]{Reward \\ Modeling} & \makecell[c]{Preference \\ Optimization} & \makecell[c]{Total $\downarrow$ \\ (A100 h)} \\
             \midrule
             \textcolor{gray}{SD1.5} \\
             \hspace{1pt} Diff.-DPO & 0 & 240 & 240 \\
             \hspace{1pt} SPO & 32 & 48 & 80 \\
             \hspace{1pt} \cellcolor{cyan!15}LPO & \cellcolor{cyan!15}\textbf{15} & \cellcolor{cyan!15}\textbf{8} & \cellcolor{cyan!15}\textbf{23} \\
             \midrule
             \textcolor{gray}{SDXL} \\
             \hspace{1pt} Diff.-DPO & 0 & 2,560 & 2,560 \\
             \hspace{1pt} SPO & 116 & 118 & 234 \\
             \hspace{1pt} \cellcolor{cyan!15}LPO & \cellcolor{cyan!15}\textbf{52} & \cellcolor{cyan!15}\textbf{40} & \cellcolor{cyan!15}\textbf{92} \\
             \bottomrule
        \end{tabular}}}
        \vskip -0.1in
    \end{minipage}
    \vspace{-0.8mm}
\end{table*}

\begin{table}[t]
    \centering
    \vspace{-2mm}
    \caption{Heterogeneous optimization based on LRM-SD1.5. P-S and I-R denote the PickScore and ImageReward metrics.}
    \vskip 0.05in
    \label{tab:sd15_for_sd21}
    \scriptsize
    \setlength{\tabcolsep}{1.0mm}{
    \scalebox{1.0}{
    \begin{tabular}{c c c c c c c c}
         \toprule
         Model & Method & Aesthetic & GenEval & P-S & I-R & HPSv2 & HPSv2.1\\
         \midrule
         SD2.1 & Original & 5.673 & 48.59 & 20.92 & 0.3063 & 27.05 & 25.49 \\
         \tiny(Same VAE) & \cellcolor{cyan!15}LPO & \cellcolor{cyan!15}\textbf{5.969} & \cellcolor{cyan!15}\textbf{56.01}  & \cellcolor{cyan!15}\textbf{21.76} & \cellcolor{cyan!15}\textbf{0.7978} & \cellcolor{cyan!15}\textbf{28.05} & \cellcolor{cyan!15}\textbf{28.61} \\
         \midrule
         SDXL & Original & 5.920 & \textbf{49.40} & \textbf{21.65} & \textbf{0.4780} & 27.06 & 26.05\\
         \tiny(Diff. VAE) & \cellcolor{cyan!15}LPO & \cellcolor{cyan!15}\textbf{5.953} & \cellcolor{cyan!15}40.85 & \cellcolor{cyan!15}20.82 & \cellcolor{cyan!15}0.3919 & \cellcolor{cyan!15}\textbf{27.10} & \cellcolor{cyan!15}\textbf{26.69} \\
         \bottomrule
    \end{tabular}}}
    % \vspace{-2mm}
    \vskip -0.15in
\end{table}


\textbf{Baseline Methods.} We compare LPO with DDPO \cite{ddpo}, D3PO \cite{d3po}, Diffusion-DPO \cite{diffusion_dpo}, MaPO \cite{mapo}, and SPO \cite{spo}. These methods are trained on similar datasets, such as Pick-a-Pic v1 and v2, to ensure a fair comparison. Details are provided in \cref{sec:experimental_detail}.


\textbf{Evaluation Protocol.} We evaluate various diffusion models across three dimensions: general preference, aesthetic preference, and text-image alignment. The PickScore \cite{pickscore}, HPSv2 \cite{hpsv2}, HPSv2.1 \cite{hpsv2}, and ImageReward \cite{imagereward} are utilized to assess the general preference. The aesthetic preference is evaluated using the Aesthetic Score \cite{aesthetic}. Consistent with \cite{spo}, both general and aesthetic preferences are assessed on the validation unique split of Pick-a-Pic v1, which has 500 different prompts. For text-image alignment, we employ the GenEval \cite{geneval} and T2I-CompBench++ \cite{t2i_compbench} metrics. All images are generated using the DDIM scheduler with 20 steps. Additionally, to assess the correlations between the LRM and aesthetics as well as text-image alignment, we propose two corresponding metrics. Specifically, we calculate the score gaps $G_*,*\in\{A,C,L\}$ between winning and losing images, where $A$, $C$, $L$ represent Aesthetic, CLIP, and LRM. For LRM, the score is taken at $t=0$. Then the Pearson Correlation Coefficient \cite{pearson} between $G_L$ and $G_A$ is referred to as \textit{Aes-Corr} while that between $G_L$ and $G_C$ is termed \textit{CLIP-Corr}. They are evaluated on the validation unique and test unique splits of Pick-a-Pic v1.

\subsection{Main Results}


\textbf{Quantitative Comparison.} As indicated in Tab.\;\ref{tab:preferenece_eval}, Tab.\;\ref{tab:t2i_eval}, and Tab.\;\ref{tab:geneval}, Diffusion-DPO excels in enhancing the text-image alignment, while SPO focuses more on aesthetics. LPO outperforms both methods across three dimensions, achieving higher Aesthetic Scores and superior performance on T2I-CompBench++ and GenEval metrics, leading to improved general preference scores. The user study results indicate similar findings, as discussed in \cref{sec:add_exp}. Notably, the LPO-optimized SD1.5 even exhibits performance comparable to the original SDXL model across various metrics.  We further validate the effectiveness of \textit{heterogeneous optimization} in Tab.\;\ref{tab:sd15_for_sd21}. SD1.5 and SD2.1 \cite{sd1} share the same VAE encoder, but SD1.5 has a smaller text encoder. Remarkably, fine-tuning SD2.1 using LRM-1.5 still yields significant improvements across various aspects, demonstrating that a smaller and inferior diffusion model can effectively fine-tune a larger and more advanced model as long as they share the same VAE encoder. In contrast, applying LRM-1.5 for the LPO of SDXL is ineffective due to the distribution mismatch in their latent spaces, which arises from differences in their VAE encoders.

\textbf{Qualitative Comparison.} The qualitative comparisons of various methods are illustrated in Fig.\;\ref{fig:main_comparison} and Fig.\;\ref{fig:vis_15_1}-Fig.\;\ref{fig:vis_xl_4}. The images generated by Diffusion-DPO exhibit deficiencies in color and detail, whereas those produced by SPO demonstrate lower semantic relevance. Additionally, SPO's excessive focus on aesthetics may lead to an overabundance of details in some images, making them appear cluttered. In contrast, the images produced by LPO achieve a strong balance between text-image alignment and aesthetic quality, delivering a higher overall image quality.


\textbf{Training Efficiency Comparison.} LPO achieves significantly faster training speed. As shown in Tab.\;\ref{tab:speed}, considering the time required for both reward modeling and preference optimization, LPO requires only 23 A100 hours for SD1.5---just 1/10 of the training time needed for Diffusion-DPO and 1/3.5 of that for SPO. For SDXL, LPO's training time is reduced to 1/28 and 1/2.5 of that for Diffusion-DPO and SPO, respectively. This efficiency is primarily due to LPO performing reward modeling and preference optimization directly in the latent space, avoiding the additional computational overhead of converting to pixel space.

\subsection{Ablation Studies}
\label{sec:ablation_study}
If not specified, ablation experiments are conducted on SD1.5. Due to space limitations, we only use PickScore to reflect general preference in Tab.\;\ref{tab:ablation_data} and Tab.\;\ref{tab:ablation_lrm}.


\textbf{MPCF.} As shown in Tab.\;\ref{tab:ablation_data}, MPCF plays a critical role in LRM training. As discussed in Sec.\;\ref{sec:lrm_train}, the inconsistent preference issue makes training on the full dataset (wo MPCF) ineffective, since it hinders the LRM from adequately focusing on aesthetics or text-image alignment, resulting in inferior LPO performance. On the other hand, different filtering strategies can profoundly impact the preference patterns of both the LRM and LPO-optimized models. The first filtering strategy strictly requires that winning images score higher than losing images across all aspects. However, since the diffusion model lacks explicit text-image alignment pre-training like CLIP, it is prone to overfitting to the visual features of the images, as indicated by a higher Aes-Corr. This overfitting results in reduced attention to alignment, as reflected by lower CLIP-Corr and GenEval scores. The second and third strategies relax the aesthetic constraints to varying degrees. However, excessively lenient constraints (the 3rd strategy) may cause LRM to focus solely on text-image alignment while neglecting image quality, resulting in a negative Aes-Corr. In contrast, the second strategy balances these two aspects better, leading to the highest general preference scores.


\begin{table}[t]
    \centering
    \vspace{-2.5mm}
    \caption{Ablation results on MPCF of LRM's training data. The second strategy balances aesthetics and alignment better.}
    \vskip 0.05in
    \label{tab:ablation_data}
    \scriptsize
    \setlength{\tabcolsep}{1.0mm}{
    \scalebox{1.1}{
    \begin{tabular}{c c c c c c}
         \toprule
         \multirow{2}{*}{Strategy} & \multicolumn{2}{c}{LRM} & \multicolumn{3}{c}{LPO} \\
         \cmidrule(lr){2-3} \cmidrule(lr){4-6}
          & Aes-Corr & CLIP-Corr & Aesthetic & GenEval & PickScore \\
         \midrule
         wo MPCF & 0.1342 & 0.2274 & 5.772 & 45.66 & 21.49 \\
         1 & \textbf{0.4860} & 0.1011 & \textbf{6.390} & 45.77 & \underline{21.61} \\
         \rowcolor{cyan!15}2 & 0.1136 & 0.3588 & \underline{5.945} & \underline{48.39} & \textbf{21.69} \\
         3 & -0.1152 & \textbf{0.4480} & 5.750 & \textbf{48.62} & 21.47 \\
         \bottomrule
    \end{tabular}}}
    % \vspace{-2mm}
    \vskip -0.1in
\end{table}


\begin{table}[t]
    \centering
    \vspace{-2mm}
    \caption{Ablation results on the VFE module of LRM. Introducing VFE leads to better alignment and general preferences.}
    \vskip 0.05in
    \label{tab:ablation_lrm}
    \scriptsize
    \setlength{\tabcolsep}{1.0mm}{
    \scalebox{1.1}{
    \begin{tabular}{c c c c c c c }
         \toprule
         \multirow{2}{*}{VFE} & \multirow{2}{*}{$gs$} & \multicolumn{2}{c}{LRM} & \multicolumn{3}{c}{LPO} \\
         \cmidrule(lr){3-4} \cmidrule(lr){5-7}
          &  & Aes-Corr & CLIP-Corr & Aesthetic & GenEval & PickScore\\
         \midrule
         \xmark & 1.0 & \textbf{0.1712} & 0.3211 & \textbf{6.053} & 46.60 & 21.51  \\
         \cmark & 3.0 & 0.1233 & 0.3441 & 5.923 & 47.35 & 21.53 \\
         \rowcolor{cyan!15}\cmark & 7.5 & 0.1136 & 0.3588 & \underline{5.945} & \textbf{48.39} & \textbf{21.69}\\
         \cmark & 10.0 & 0.1063 & \textbf{0.3592} & 5.937 & \underline{48.13} & \underline{21.56}\\
         \bottomrule
    \end{tabular}}}
    % \vspace{-2mm}
    \vskip -0.1in
\end{table}


\begin{table}[t]
    \centering
    \vspace{-2.5mm}
    \caption{Ablation results on optimization timestep ranges in LPO.}
    \vskip 0.05in
    \label{tab:ablation_timestep}
    \scriptsize
    \setlength{\tabcolsep}{1.0mm}{
    \scalebox{1.1}{
    \begin{tabular}{c c c c c c c}
         \toprule
         Range of $t$ & Aesthetic & GenEval & P-S & I-R & HPSv2 & HPSv2.1 \\
         \midrule
         \texttt{[}0, 200\texttt{]} & 5.434 & 40.11 & 20.46 & -0.0987 & 26.25 & 23.61 \\
         \texttt{[}250, 450\texttt{]} & 5.527 & 43.00 & 20.76 & 0.1430 & 26.90 & 25.37 \\
         \texttt{[}500, 700\texttt{]} & 5.742 & 44.44 & 20.95 & 0.1591 & 26.71 & 25.16\\
         \texttt{[}750, 950\texttt{]} & \underline{5.853} & \underline{48.28} & 
         \underline{21.54} & \underline{0.6337} & \underline{27.47} & \underline{27.64} \\
         \midrule
         \texttt{[}0, 450\texttt{]} & 5.573 & 42.71 & 20.63 & 0.0204 & 26.69 & 24.88 \\
         \texttt{[}0, 700\texttt{]} & 5.765 & 44.93 & 21.02 & 0.3087 &  27.10 & 26.25\\
         \rowcolor{cyan!15}\texttt{[}0, 950\texttt{]} & \textbf{5.945} & \textbf{48.39} & \textbf{21.69} & \textbf{0.6588} & \textbf{27.64} & \textbf{27.86} \\
         \bottomrule
    \end{tabular}}}
    % \vspace{-2mm}
    \vskip -0.1in
\end{table}


\begin{table}[t]
    \centering
    \vspace{-2mm}
    \caption{Ablation results on different threshold strategies.}
    \vskip 0.05in
    \label{tab:ablation_threshold}
    \scriptsize
    \setlength{\tabcolsep}{1.0mm}{
    \scalebox{1.1}{
    \begin{tabular}{c c c c c c c }
         \toprule
          Threshold & Aesthetic & GenEval & P-S & I-R & HPSv2 & HPSv2.1\\
         \midrule
         0.3 & 5.853 & 46.75 & 21.22 & 0.5112  & 27.30 & 27.12 \\ 
         0.4 & 5.832 & 48.32 & 21.32 & 0.4789 & 27.08 & 26.37 \\
         0.5 & 5.900 & 48.39 & 21.57 & 0.6088 & 27.54 & \underline{27.42} \\
         0.6 & 5.877 & 47.97 & 21.35 & 0.5510 & 27.25 & 26.73 \\
         \midrule
         \texttt{[}0.3, 0.45\texttt{]} & \underline{5.916} & \textbf{49.43} & \underline{21.58} & \underline{0.6405} & \underline{27.55} & 27.33\\
         \rowcolor{cyan!15}\texttt{[}0.35, 0.5\texttt{]} & \textbf{5.945} & 48.39 & \textbf{21.69} & \textbf{0.6588} & \textbf{27.64} & \textbf{27.86} \\
         \texttt{[}0.4, 0.55\texttt{]} & 5.882 & \underline{48.77} & 21.48 & 0.4791 & 27.30 & 27.13\\
         \bottomrule
    \end{tabular}}}
    % \vspace{-2mm}
    \vskip -0.1in
\end{table}


\textbf{Structure of LRM.} As illustrated in Tab.\;\ref{tab:ablation_lrm}, the introduction of VFE ($gs>1$) leads to lower Aes-Corr values but higher CLIP-Corr values, indicating an enhanced emphasis on text-image alignment. This results in improvements in both the GenEval score and PickScore, with only a minor decline in the Aesthetic Score. As $gs$ increases, the LRM's correlation with alignment steadily improves, while its correlation with aesthetics decreases. When $gs$ is set to 7.5, the model achieves the best overall performance.

\textbf{Optimization Timesteps.} Tab.\;\ref{tab:ablation_timestep} ablates different optimization timestep ranges, indicating that larger timesteps lead to better performance. The results achieved within the range of $[750, 950]$ are nearly comparable to those achieved through optimization across the entire denoising process, \ie $[0,950]$. We suggest this is because diffusion models focus on low-frequency information, such as image layout and style, during larger timesteps, while emphasizing high-frequency texture details during smaller timesteps. The low-frequency components formed in higher timesteps play a decisive role in determining the overall quality of the generated images. This observation also demonstrates the effectiveness of LRM, even in very large timesteps. The qualitative comparison of different ranges is shown in Fig.\;\ref{fig:vis_timestep}.

\textbf{Dynamic Sampling Threshold.} The standard deviation $\sigma_t$ of samples at smaller timesteps is relatively small according to the DDPM scheduling \cite{ddpm}, making the constant threshold insufficient to accommodate all timesteps. As indicated in Tab.\;\ref{tab:ablation_threshold}, the dynamic threshold strategy generally outperforms the constant threshold across different intervals, effectively alleviating this problem. We further explore other dynamic strategies in \cref{sec:add_exp}.


\section{Conclusion and future work}
In this study, we examined the ability of LLMs to produce self-generated counterfactual explanations (SCEs).
We design a prompt-based setup for evaluating the efficacy of \SCEs.
Our results show that LLMs consistently struggle with generating valid \SCEs. In many cases model prediction on a \SCE does not yield the same target prediction for which the model crafted the \SCE.
Surprisingly, we find that LLMs put significant emphasis on the context---the prediction on \SCE is significantly impacted by the presence of original prediction and instructions for generating the \SCE.
Based on this empirical evidence, we argue that LLMs are still far from being able to explain their own predictions counterfactually.
Our findings add to similar insights from recent studies on other forms of self-explanations~\cite{lanham2023measuring,tanneru2024quantifying}.



Our work opens several avenues for future work. Inspired by counterfactual data augmentation~\cite{sachdeva2023catfood}, one could include the counterfactual explanation capabilities a part of the LLM training process. This inclusion may enhance the counterfactual reasoning capabilities of the LLM. Follow ups should also explore the effect of prompt tuning, specifically, model-tailored prompts for generating \SCEs. These approaches might lead to better quality \SCEs.


We limited our investigation to open source models of upto 70B parameters. Extending our analysis to larger and more recent models, \eg, DeepSeek R1 671B, and closed source models like OpenAI o3 would be an interesting avenue for future work.

Finally, our experiments were limited to relatively simple tasks: classification and mathematics problems where the solution is an integer. This limitation was mainly due to the fact that it is difficult to automatically judge validity of answers for more open-ended language generation tasks like search and information retrieval. Scaling our analysis to such tasks would require significant human-annotation resources, and is an important direction for future investigations.



%\addtolength{\textheight}{-9cm}   % This command serves to balance the column lengths
                                  % on the last page of the document manually. It shortens
                                  % the textheight of the last page by a suitable amount.
                                  % This command does not take effect until the next page
                                  % so it should come on the page before the last. Make
                                  % sure that you do not shorten the textheight too much.




%%%%%%%%%%%%%%%%%%%%%%%%%%%%%%%%%%%%%%%%%%%%%%%%%%%%%%%%%%%%%%%%%%%%%%%%%%%%%%%%

\bibliographystyle{IEEEtran}
\bibliography{reference}




\end{document}
