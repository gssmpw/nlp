\section{More Related Works}
\label{sec:app_more_related_works}
\textbf{LLMs for Data Augmentation} annotate pseudo-labels via their advanced zero-shot text classification performance. \textit{E.g.}, LLM-GNN~\cite{chen2023label}, Cella~\cite{zhang2024cost} and ~\cite{hu2024low} propose heuristics to actively select and annotate pseudo-labels for supervised GNN training. ~\cite{pan2024distilling} performs knowledge distillation with LLMs as teachers. ~\cite{yu2023empower,li2024enhancing} generate synthetic node text with LLMs. The performance of these methods depend on the capability of LLM, and may still require relatively high annotating and training cost.

\textbf{LLMs for Graph Property Reasoning} focus on reason graph structure properties (e.g., shortest path, node degree, etc)~\cite{tang2024grapharena, dai2024large, yuan2024gracore, ouyang2024gundam}. Representative works include~\cite{perozzi2024let, chen2024graphwiz, zhang2024can, cao2024graphinsight, wei2024gita}.

\textbf{Tuning LMs/GNNs towards Better Task-Specific Performance} aims to push the limits of task-specific performance on TAGs other than generalization. These methods develop novel techniques to optimize LMs or GNNs for pushing the limits of in-domain performance~\cite{chien2021node, duan2023simteg, he2023harnessing, zhao2022learning, zhu2021textgnn, li2021adsgnn, yang2021graphformers, bi2021leveraging, pang2022improving, zolnai2024stage, yang2021graphformers}.

\textbf{Text embeddings} Generating unified text embeddings is a critical research area with broad applications, including web search and question answering. Numerous text encoders~\cite{reimers2019sentence, liu2019roberta, song2020mpnet} based on pre-trained language models have served as the foundation for various embedding models. Recently, decoder-only LLMs have been widely adopted for text embedding tasks~\cite{li2023towards, moreira2024nv} achieving remarkable performance on the Massive Text Embedding Benchmark (MTEB)~\cite{muennighoff2022mteb}. This progress stems from LLM2Vec~\cite{behnamghader2024llm2vec}, which introduces a novel unsupervised approach to transforming decoder-only LLMs into embedding models, including modifications to enable bidirectional attention. Recent findings~\cite{li2024making} suggest that retaining the unidirectional attention mechanism enhances LLM2Vec’s empirical performance.



\section{Experiment Details}

\subsection{Dataset Details}
\label{sec:app_datasets}
\textbf{Meta-Data}
In Table.~\ref{tab:dataset_meta_data}, we show the meta-data of all the eleven datasets used in our experiments.

\begin{table}[h]
\centering
\begin{tabular}{@{}ccccc@{}}
\toprule
 & \begin{tabular}[c]{@{}c@{}}Number\\ of Nodes\end{tabular} & \begin{tabular}[c]{@{}c@{}}Number \\ of  Edges\end{tabular} & \begin{tabular}[c]{@{}c@{}}Number \\ of  Classes\end{tabular} & \begin{tabular}[c]{@{}c@{}}Ground Truth\\ Homophily Ratio\end{tabular} \\ \midrule
Cora & 2708 & 10556 & 7 & 0.809 \\
Citeseer & 3186 & 8450 & 6 & 0.764 \\
Pubmed & 19717 & 88648 & 3 & 0.792 \\
History & 41551 & 503180 & 12 & 0.662 \\
Children & 76875 & 2325044 & 24 & 0.464 \\
Sportsfit & 173055 & 3020134 & 13 & 0.9 \\
Wikics & 11701 & 431726 & 10 & 0.678 \\
Cornell & 191 & 292 & 5 & 0.115 \\
Texas & 187 & 310 & 5 & 0.067 \\
Wisconsin & 265 & 510 & 5 & 0.152 \\
Washington & 229 & 394 & 5 & 0.149 \\ \bottomrule
\end{tabular}
\vspace{-0.cm}
\caption{Meta data of the datasets in this study.}
\label{tab:dataset_meta_data}
\end{table}


\textbf{Dataset Split} For the datasets (all the homophily graphs) that have been used for study in TSGFM~\cite{chen2024text}, we follow their implementation to perform data pre-processing, obtain raw texts and do data split, the introduction to data source can be found at Appendix.D.2 in their original paper, the code can be found at the link \footnote{\url{https://github.com/CurryTang/TSGFM/tree/master?tab=readme-ov-file}}.

As to the heterophily graphs, the four datasets are originally from~\cite{craven1998learning}. We obtain the raw texts from~\cite{yan2023comprehensive}, which can be found from\footnote{\url{https://github.com/sktsherlock/TAG-Benchmark/tree/master}}. As to data split, for zero-shot inference, all the nodes are marked as test data; for few-shot setting, $k$ labeled nodes are randomly sampled per class and the rests are marked as test data.
To the best of our knowledge, the four heterophily graph datasets used in this study are the only graphs that provide raw texts feature.





\subsection{LLM-BP Implementation Details}
\label{sec:app_llm_bp_implementation_details}
\textbf{Infrastructure and Seeds}
All the local experiments run on a server with AMD EPYC 7763 64-Core Processor and eight NVIDIA RTX 6000 Ada GPU cards, methods are mainly implemented with PyTorch~\cite{paszke2019pytorch}, Torch-Geometric~\cite{fey2019fast} and Huggingface Transformers~\cite{wolf2019huggingface}.
To obtain the embeddings, all the encoders that run locally on the server without API calling in this study run with the random seed $42$.


\textbf{Class Embedding}

$\bullet$ \textbf{Zero-Shot Setting:} We uniformly randomly sample $20 c$ nodes per graph, where $c$ denotes the number of classes, we employ GPT-4o~\cite{hurst2024gpt} to infer their labels. With the predictions from LLMs, the sampled nodes form distinct clusters. For each cluster, we take the top-$k$ (10 in the experiments) nodes whose embedding are closest with the cluster center and calculate their average embedding as the class embedding.

We notice that some works directly feed text descriptions into encoders as class embeddings~\cite{yang2024gl, chen2024text}, we find that different encoders can be highly sensitive to variations in text description. Therefore, we adopt the above method to ensure fairness among different encoders.


$\bullet$\textbf{Few-Shot Setting:} We directly take the class embedding as the averaged embeddings of labeled nodes per class.


\textbf{The Task-Adaptive Encoder:} We directly adopt the pre-trained LLM2Vec encoder released by~\cite{li2024making}, which is based on Mistral7B-v0.1~\cite{jiang2023mistral}. We check the pre-training data used in the original paper for aligning LLM decoders with the embedding space, the datasets are mainly for text-retrieval and therefore do not overlap with the TAG datasets adopted in our study. For detailed introduction of the datasets for LLM2Vec pre-training, see Section 4.1 training data in the original paper.
\label{sec:app_task_adaptive_implementation}
The task-adaptive prompting follows the format as:

\textit{
\text{$\langle$ Instruct $\rangle$}\\
``Given the \text{\{task description\}}, classify it into one of the following $k$ classes: \\
\text{\{class labels\}}\\
\text{$\langle$ query$\rangle$}\\
\text{\{raw node texts\}}.''\\
\text{$\langle$ response $\rangle$}\\}


, where the \{task descriptions\} prompts for each dataset is the same as that used for vanilla LLMs, see Table.~\ref{tab:vanilla_llm_task_description} for details. 


\textbf{Hyper-Parameters for BP algorithm}
For LLM-BP, we adopt $5$ message-passing layers, for its linear approximation form, we use a single layer.
The temperature hyper-parameter $\tau$ in computing node potential initialization in Eq.~\eqref{eq:node_potential} is set as $0.025$ for LLM-BP and $0.01$ for LLM-BP (appr.) across all the datasets. Attached in Table.~\ref{tab:pred_h} is the homophily ratio $r$ we used (predicted by GPT-4o-mini~\cite{hurst2024gpt}.
\begin{table}[h]
\centering
\resizebox{\textwidth}{!}{\begin{tabular}{@{}cccccccccccc@{}}
\toprule
 & Cora & Citeseer & Pubmed & History & Children & Sportsfit & Wikics & Cornell & Texas & Wisconsin & Washington \\ \midrule
\begin{tabular}[c]{@{}c@{}}Ground Truth \\ Homophony Ratio\end{tabular} & 0.81 & 0.76 & 0.79 & 0.66 & 0.46 & 0.90 & 0.67 & 0.11 & 0.06 & 0.15 & 0.19 \\
\begin{tabular}[c]{@{}c@{}}$r$ predicted by \\ GPT-4o-mini\end{tabular} & 0.70 & 0.81 & 0.81 & 0.73 & 0.35 & 0.81 & 0.52 & 0.05 & 0.04 & 0.06 & 0.02 \\ \bottomrule
\end{tabular}}
\vspace{-0.cm}
\caption{$r$ predicted by GPT-4o-mini, that is used in all the experiments in this study.}
\label{tab:pred_h}
\end{table}



\subsection{Baseline Implementation Details}
\label{sec:app_implementation_details_baseline}
$\bullet$ \textbf{Vanilla Encoders} Vanilla encoders like SBert~\cite{reimers2019sentence}, Roberta~\cite{liu2019roberta} and text-embedding-3-large~\cite{openai2024textembedding} directly encode the raw text of the nodes. LLM2Vec uses the prompts:

\begin{align}
\label{eq:vanilla_llm2vec}
\resizebox{0.43 \textwidth}{!}{$\langle\text{Instruct}\rangle \{\text{task\_description}\} \langle\text{query}\rangle {X_i} \langle\text{response}\rangle$}.
\end{align}, where the $\{\text{task\_description}\}$ for each dataset is provided in Appendix.~\ref{sec:app_prompt_llm2vec_task_description}.

$\bullet$ \textbf{Vanilla LLMs} Prompts for GPT-4o and GPT-3.5-turbo adopts the format as follows:

\label{sec:app_vanilla_LLM_implementation}
\textit{
``role'': ``system''\\
``content'': ``You are a chatbot who is an expert in text classification''\\
``role'': ``user''\\
``content'': ``We have \text{\{task description\}} from the following $k$ categories: \text{\{class labels\}}\\
The text is as follows:\\
\text{\{raw node text\}}\\
Please tell which category the text belongs to:''
}


The \{task description\} for the vanilla LLMs for each class is provided in Appendix.~\ref{sec:app_prompt_vanilla_llm}.


$\bullet$ \textbf{Tuning LM/GNNs}
We adopt the pre-trained UniGLM~\cite{fang2024uniglm} released by the official implementation, which adopts Bert as the encoder, for direct inference.
For ZeroG~\cite{li2024zerog}, we re-implement the method and train it on ogbn-arxiv~\cite{hu2020open} for fair comparison with other baselines.

As to GNNs tuning methods, we pre-train GraphMAE~\cite{hou2022graphmae} and DGI~\cite{velivckovic2018deep} on ogbn-arxiv~\cite{hu2020open}, where the input for both models are from SBert~\cite{reimers2019sentence}, and we follow in implementation in TSGFM~\cite{chen2024text} benchmark.

$\bullet$ \textbf{Multi-Task GFMs} OFA~\cite{liu2023one} is trained on ogbn-arxiv~\cite{hu2020open}. As to GOFA, we directly adopt the model after pre-training~\cite{hu2021ogb, ding2023enhancing} and instruct fine-tuning on ogbn-arxiv~\cite{hu2020open} provided by the authors due to the huge pre-training cost, the zero-shot inference scheme also follows their original implementation.


$\bullet$ \textbf{LLMs with Graph Adapters} Both LLaGA~\cite{chen2024llaga} and GraphGPT~\cite{tang2024graphgpt} are trained on ogbn-arxiv~\cite{hu2020open}, we follow the hyper-parameter setting in their original implementation.


\section{Detailed Derivations}
\label{sec:app_detailed_derivation}
% \subsection{Derivation for Eq.~\eqref{eq:message_passing}}
% Here we present the derivation towards the message passing form in Eq.~\ref{eq:message_passing}. In node classification task, given node $i$, the objective is to minimize the mean-square error (MSE) of the node label prediction:
% \begin{align}
%     \min \text{MSE} (\hat{y_i}) = \mathbb{E}[(y_y - \hat{y_i})^2 | \XX],
% \end{align}, with the optimal solution $\hat{y_i}$ as:
% \begin{align}
%     \hat{y_i} = \sum_{y_i} y_i p(y_i |X),
% \end{align}where $p(y_i |X) = \sum_{Y \setminus i} \mathbb{P}(Y | X)$ is the posterior marginal. 
% Given the factorized form of the posterior distribution under the Markov Random Field (MRF) modeling $\mathbb{P}_\mathcal{G}(Y \mid X) \propto \prod_{i \in \mathcal{V}} \varphi_{X_i}(y_i) 
% \prod_{(i,j) \in \mathcal{E}} \psi_{ij}(y_i, y_j)$, where $\varphi_{X_i}(y_i) = \varphi_{y_i}(X_i)\phi_i(y_i)$, one can approximate $p(y_i |X)$ by running the loopy belief propagation:

% \begin{align}
%     m_{j\rightarrow i}^{(k)}(y_i) \cong \sum_{y_j} \psi_{ij}(y_i, y_j) \frac{p_j^{(k-1)}(y_j)}{m_{i\rightarrow j}^{(k-1)}(y_j)} \\
%     p_i^{(k)}(y_i) \approx p_i^{(0)}(y_i) \prod_{j \in \mathcal{N}(i)} m_{j\rightarrow i}^{(t)}(y_i),
% \end{align}$m^{(0)}_{i \rightarrow j}$ is initialized as $1 / c$ for all $i,j \in \mathcal{V}$.

% Considering numerical stability, the belief-propagation is updated in log-space, which leads to the message passing formulation in Eq.~\eqref{eq:message_passing}:
% \begin{align}
%  \log m_{j \to i}^{(k)}(y_i) \cong &\, \text{LSE}_{y_j}[\log \psi_{ij}(y_i,y_j) + \\ &
% \log p_j^{(k-1)}(y_j) - \log m_{i \to j}^{(k-1)}(y_j)], \nonumber \\
%  \log p_i^{(k)}(y_i) \cong &\log p_i^{(0)}(y_i) + \sum_{j \in \mathcal{N}(i)} \log m_{j \to i}^{(k)}(y_i),  \nonumber
% \end{align}where LSE stands for the log-sum-exp function.

% \subsection{Derivation for Eq.~\eqref{eq:message_passing}}
% In a node classification task, given node \(i\), we aim to minimize the mean-square error (MSE) of the node label prediction under observations \(\XX\):
% \begin{align}
%     \min \mathrm{MSE}\bigl(\hat{y}_i\bigr) 
%     \;=\; 
%     \mathbb{E}\Bigl[\bigl(y_i - \hat{y}_i\bigr)^2 \,\Bigm|\;\XX\Bigr].
% \end{align}
% The optimal solution \(\hat{y}_i\) is then
% \begin{align}
%     \hat{y}_i 
%     \;=\; 
%     \sum_{y_i}\,y_i\,p\bigl(y_i \,\bigm|\;\XX\bigr),
% \end{align}
% where \(p\bigl(y_i \,\bigm|\;\XX\bigr) = \sum_{Y \setminus i}\,\mathbb{P}\bigl(Y \,\bigm|\;\XX\bigr)\) is the posterior marginal.

% Under a Markov Random Field (MRF) assumption with graph \(\mathcal{G}=(\mathcal{V},\mathcal{E})\), the posterior factors as:
% \begin{align}
%     \mathbb{P}_\mathcal{G}(Y \mid \XX) 
%     \;\propto\; 
%     \prod_{i \in \mathcal{V}}\varphi_{X_i}\bigl(y_i\bigr)
%     \,\prod_{(i,j)\in \mathcal{E}}\psi_{ij}\bigl(y_i,y_j\bigr),
% \end{align}
% where \(\varphi_{X_i}\bigl(y_i\bigr)=\varphi_{y_i}\bigl(X_i\bigr)\phi_i\bigl(y_i\bigr)\).  
% To approximate each \(p(y_i\mid \XX)\), one can run \emph{loopy belief propagation} (LBP).  
% Let
% \begin{align}
%     p_j^{(k-1)}\bigl(y_j\bigr) 
%     \;=\; 
%     p_j^{(0)}\bigl(y_j\bigr)\,
%     \prod_{\ell\in \mathcal{N}(j)} m_{\ell \to j}^{(k-1)}\bigl(y_j\bigr),
%     \label{eq:p_j_definition}
% \end{align}
% where \(p_j^{(0)}\bigl(y_j\bigr)\) is the initial node potential, and \(m_{\ell\to j}^{(k-1)}\) are incoming messages from neighbors \(\ell\in \mathcal{N}(j)\).  
% Then the new message from node \(j\) to node \(i\) at iteration \(k\) takes the form
% \begin{align}
%     m_{j \to i}^{(k)}\bigl(y_i\bigr)
%     \;=\; 
%     \sum_{y_j}\,\psi_{ij}\bigl(y_i,y_j\bigr)\,
%     \frac{p_j^{(k-1)}\bigl(y_j\bigr)}{m_{i \to j}^{(k-1)}\bigl(y_j\bigr)},
%     \label{eq:message_update}
% \end{align}
% where dividing by \(m_{i \to j}^{(k-1)}\bigl(y_j\bigr)\) removes the previous contribution from node \(i\) to \(j\) (which is already included in \(p_j^{(k-1)}(y_j)\)) to avoid double counting.  
% Once all messages are updated, the \emph{node belief} can be written as
% \begin{align}
%     p_i^{(k)}\bigl(y_i\bigr) 
%     \;=\; 
%     p_i^{(0)}\bigl(y_i\bigr)\,
%     \prod_{j \in \mathcal{N}(i)} m_{j \to i}^{(k)}\bigl(y_i\bigr).
% \end{align}
% For numerical stability, we apply the same process in log-space (i.e., \(\log\!\sum\exp\) rather than \(\sum\) and products), yielding the message passing equations in Eq.~\eqref{eq:message_passing}.

% \begin{align}
%     \log m_{j \to i}^{(k)}\bigl(y_i\bigr)
%     &\;\cong\;
%     \mathrm{LSE}_{y_j}
%     \Bigl[\,
%         \log \psi_{ij}\bigl(y_i,y_j\bigr)
%         \;+\;
%         \log p_j^{(k-1)}\bigl(y_j\bigr)
%         \;-\;
%         \log m_{i \to j}^{(k-1)}\bigl(y_j\bigr)
%     \Bigr],
%     \nonumber\\
%     \log p_i^{(k)}\bigl(y_i\bigr)
%     &\;\cong\;
%     \log p_i^{(0)}\bigl(y_i\bigr)
%     \;+\;
%     \sum_{j\in \mathcal{N}(i)}
%     \log m_{j \to i}^{(k)}\bigl(y_i\bigr),
%     \label{eq:message_passing_log}
% \end{align}
% where \(\mathrm{LSE}\{\cdot\}\) denotes the log-sum-exp operator, and \(m_{i\to j}^{(0)}\) is initialized as a constant (e.g., \(1/c\)).
\subsection{Derivation for Eq.~\eqref{eq:message_passing}}
% In a node classification task, given node \(i\), we aim to minimize the mean-square error (MSE) of the node label prediction under observations \(\XX\):
% \begin{align}
%     \min \mathrm{MSE}\bigl(\hat{y}_i\bigr) 
%     \;=\; 
%     \mathbb{E}\Bigl[\bigl(y_i - \hat{y}_i\bigr)^2 \,\Bigm|\;\XX\Bigr].
% \end{align}
% The optimal solution \(\hat{y}_i\) then follows from
% \begin{align}
%     \hat{y}_i 
%     \;=\; 
%     \sum_{y_i}\,y_i\,p\bigl(y_i \,\bigm|\;\XX\bigr),
% \end{align}
% where \(p\bigl(y_i \,\bigm|\;\XX\bigr) = \sum_{Y\setminus i}\,\mathbb{P}\bigl(Y \,\bigm|\;\XX\bigr)\) is the posterior marginal.

% \paragraph{Factorized Posterior under MRF.}
% Assume a Markov Random Field (MRF) with graph \(\mathcal{G}=(\mathcal{V},\mathcal{E})\).  Then the posterior distribution factors as
% \begin{align}
%     \mathbb{P}_\mathcal{G}(Y \mid \XX) 
%     \;\propto\; 
%     \prod_{i \in \mathcal{V}}\varphi_{X_i}\bigl(y_i\bigr)
%     \,\prod_{(i,j)\in \mathcal{E}}\psi_{ij}\bigl(y_i,y_j\bigr),
% \end{align}
% where \(\varphi_{X_i}\bigl(y_i\bigr)=\varphi_{y_i}\bigl(X_i\bigr)\phi_i\bigl(y_i\bigr)\).  
% To approximate each \(p\bigl(y_i\mid \XX\bigr)\), one can employ \emph{loopy belief propagation} (LBP).  

% \paragraph{Node Belief and Messages.}
% Define the (intermediate) node belief at iteration \((k-1)\) as
% \begin{align}
%     p_j^{(k-1)}\bigl(y_j\bigr) 
%     \;=\; 
%     p_j^{(0)}\bigl(y_j\bigr)\,
%     \prod_{\ell\in \mathcal{N}(j)} m_{\ell \to j}^{(k-1)}\bigl(y_j\bigr),
%     \label{eq:p_j_definition}
% \end{align}
% where \(p_j^{(0)}\bigl(y_j\bigr)\) is the initial potential for node \(j\), and \(m_{\ell\to j}^{(k-1)}\) denotes the incoming message from neighbor \(\ell\).  
% The new message from node \(j\) to node \(i\) at iteration \(k\) then reads
% \begin{align}
%     m_{j \to i}^{(k)}\bigl(y_i\bigr)
%     \;=\; 
%     \sum_{y_j}\,\psi_{ij}\bigl(y_i,y_j\bigr)\,
%     \frac{p_j^{(k-1)}\bigl(y_j\bigr)}{m_{i \to j}^{(k-1)}\bigl(y_j\bigr)},
%     \label{eq:message_update}
% \end{align}
% where dividing by \(m_{i \to j}^{(k-1)}\bigl(y_j\bigr)\) cancels out the old contribution from \(i\) to \(j\) (already included in \(p_j^{(k-1)}\bigl(y_j\bigr)\)), thus preventing double-counting.

% \paragraph{Recovering the Reduced Product.}
% Meanwhile, the node belief for node \(i\) at iteration \(k\) is
% \begin{align}
%     p_i^{(k)}\bigl(y_i\bigr) 
%     \;=\; 
%     p_i^{(0)}\bigl(y_i\bigr)\,
%     \prod_{\ell \in \mathcal{N}(i)} m_{\ell \to i}^{(k)}\bigl(y_i\bigr).
%     \label{eq:node_belief}
% \end{align}
% From \eqref{eq:node_belief}, one observes
% \begin{align}
%     \varphi_{X_i}\bigl(y_i\bigr) 
%     \,\prod_{\ell \in \mathcal{N}(i)\setminus \{j\}} m_{\ell \to i}^{(k)}\bigl(y_i\bigr)
%     \;=\;
%     \frac{
%       \varphi_{X_i}\bigl(y_i\bigr)
%       \,\prod_{\ell \in \mathcal{N}(i)} m_{\ell \to i}^{(k)}\bigl(y_i\bigr)
%     }{
%       m_{j \to i}^{(k)}\bigl(y_i\bigr)
%     }
%     \;=\;
%     \frac{
%       p_i^{(k)}\bigl(y_i\bigr)
%     }{
%       m_{j \to i}^{(k)}\bigl(y_i\bigr)
%     },
%     \label{eq:factor_rewrite}
% \end{align}
% which is precisely the reduced product over all neighbors except \(j\).

% \paragraph{Log-Space Update.}
% For numerical stability, we often take the logarithm of both sides in \eqref{eq:message_update} and \eqref{eq:node_belief}, i.e., using \(\log\sum\exp(\cdot)\) rather than direct sums and products. This procedure yields the message passing equations in Eq.~\eqref{eq:message_passing}, typically written as:
% \begin{align}
%     \log m_{j \to i}^{(k)}\bigl(y_i\bigr)
%     &\;\cong\;
%     \mathrm{LSE}_{y_j}
%     \Bigl[\,
%         \log \psi_{ij}\bigl(y_i,y_j\bigr)
%         \;+\;
%         \log p_j^{(k-1)}\bigl(y_j\bigr)
%         \;-\;
%         \log m_{i \to j}^{(k-1)}\bigl(y_j\bigr)
%     \Bigr],
%     \nonumber\\
%     \log p_i^{(k)}\bigl(y_i\bigr)
%     &\;\cong\;
%     \log p_i^{(0)}\bigl(y_i\bigr)
%     \;+\;
%     \sum_{j\in \mathcal{N}(i)}
%     \log m_{j \to i}^{(k)}\bigl(y_i\bigr),
%     \label{eq:message_passing_log}
% \end{align}
% where \(\mathrm{LSE}\{\cdot\} \equiv \log \sum \exp(\cdot)\) and \(m_{i\to j}^{(0)}\) is initialized as a constant (e.g., \(1/c\)). By iterating these updates, LBP approximates each node’s posterior marginal \(p(y_i\mid \XX)\), which in turn provides an estimate \(\hat{y}_i = \sum_{y_i} y_i \, p_i^{(k)}(y_i)\) for the MMSE criterion.
In a node classification task, given a node \(i\), our goal is to minimize the mean-square error (MSE) in predicting the node label under the observations \(\XX\):
\begin{align}
    \min \mathrm{MSE}\bigl(\hat{y}_i\bigr) 
    \;=\; 
    \mathbb{E}\Bigl[\bigl(y_i - \hat{y}_i\bigr)^2 \,\Bigm|\;\XX\Bigr].
\end{align}
The optimal solution \(\hat{y}_i\) is then given by:
\begin{align}
    \hat{y}_i 
    \;=\; 
    \sum_{y_i} y_i \, p\bigl(y_i \mid \XX\bigr),
\end{align}
where the posterior marginal \(p\bigl(y_i \mid \XX\bigr)\) is computed as:
\begin{align}
    p\bigl(y_i \mid \XX\bigr) = \sum_{Y \setminus i} \mathbb{P}\bigl(Y \mid \XX\bigr).
\end{align}

\paragraph{Factorized Posterior under an MRF.}
Assuming a Markov Random Field (MRF) over a graph \(\mathcal{G}=(\mathcal{V}, \mathcal{E})\), the posterior distribution factors as:
\begin{align}
    \mathbb{P}_\mathcal{G}(Y \mid \XX) 
    \;\propto\; 
    \prod_{i \in \mathcal{V}} \varphi_{X_i}\bigl(y_i\bigr)
    \prod_{(i,j) \in \mathcal{E}} \psi_{ij}\bigl(y_i, y_j\bigr),
\end{align}
where the node potential is defined as \(\varphi_{X_i}\bigl(y_i\bigr) = \varphi_{y_i}\bigl(X_i\bigr)\phi_i\bigl(y_i\bigr)\).

\paragraph{General Message-Passing Framework.}
To compute the marginal \(p(y_i \mid \XX)\), the loopy belief propagation (LBP) algorithm iteratively updates messages between nodes. The general message update rule from node \(i\) to node \(j\) at iteration \(k\) is:
\begin{align}
    m_{i \to j}^{(k)}\bigl(y_j\bigr) 
    \;=\; 
    \alpha_{i \to j} \sum_{y_i} \Bigg[\varphi_{X_i}\bigl(y_i\bigr) \psi_{ij}\bigl(y_i, y_j\bigr)
    \prod_{\ell \in \mathcal{N}(i) \setminus j} m_{\ell \to i}^{(k-1)}\bigl(y_i\bigr)\Bigg],
    \label{eq:general_message}
\end{align}
where \(\alpha_{i \to j}\) is a normalization constant ensuring the message sums to 1.

\paragraph{Node Belief Updates.}
The node belief \(p_i^{(k)}\bigl(y_i\bigr)\) at iteration \(k\) is obtained by combining the node potential with incoming messages from all neighbors:
\begin{align}
    p_i^{(k)}\bigl(y_i\bigr) 
    \;=\; 
    \varphi_{X_i}\bigl(y_i\bigr) 
    \prod_{\ell \in \mathcal{N}(i)} m_{\ell \to i}^{(k)}\bigl(y_i\bigr).
    \label{eq:node_belief_update}
\end{align}

\paragraph{Reformulating the Messages.}
Substituting Eq.~\eqref{eq:node_belief_update} into Eq.~\eqref{eq:general_message} simplifies the message-passing equation. The message from node \(i\) to node \(j\) at iteration \(k\) can be rewritten as:
\begin{align}
    m_{i \to j}^{(k)}\bigl(y_j\bigr) 
    \;=\; 
    \alpha_{i \to j} \sum_{y_i} \psi_{ij}\bigl(y_i, y_j\bigr) 
    \frac{p_i^{(k)}\bigl(y_i\bigr)}{m_{j \to i}^{(k-1)}\bigl(y_i\bigr)}.
    \label{eq:message_reformulation}
\end{align}
This reformulation prevents double-counting the contribution of node \(j\) to node \(i\) in the previous iteration.

\paragraph{Log-Space Stability.}
To avoid numerical underflow, the log-space version of the message update is commonly used:
\begin{align}
    \log m_{i \to j}^{(k)}\bigl(y_j\bigr) 
    &\;=\; 
    \mathrm{LSE}_{y_i} \Bigl[\log \psi_{ij}\bigl(y_i, y_j\bigr) + \log p_i^{(k)}\bigl(y_i\bigr) - \log m_{j \to i}^{(k-1)}\bigl(y_i\bigr)\Bigr],
\end{align}
where \(\mathrm{LSE}(\cdot) \equiv \log \sum \exp(\cdot)\).

\paragraph{Summary.}
By iteratively applying these message updates and node belief calculations, LBP provides an approximation for the posterior marginal \(p(y_i \mid \XX)\). The final prediction \(\hat{y}_i\) under the MMSE criterion is:
\begin{align}
    \hat{y}_i 
    \;=\; 
    \sum_{y_i} y_i \, p_i^{(k)}\bigl(y_i\bigr).
\end{align}
This completes the derivation of the message-passing update in Eq.~\eqref{eq:message_passing}.



\section{More Experiment Results}
\label{sec:app_more_experiment_results}


\subsection{Significance Test of Effectiveness of Task-Adaptive Encoding}
\begin{figure}[h]
    \centering
    \includegraphics[width=0.32\linewidth]{images/confidence/mmlu.pdf} \includegraphics[width=0.32\linewidth]{images/confidence/unintended_jigsaw.pdf}
    \includegraphics[width=0.32\linewidth]{images/confidence/commonsenseqa.pdf}
    \caption{Increasing the confidence threshold has no effect until a point, after which conformal consistency increases while the coverage tends to zero}
    \label{fig:confidence}
\end{figure}

\label{sec:app_confidence}

We conduct significance test on the improvment of task-adaptive encoding over vanilla LLm2Vec~\cite{li2024making} and Text-Embedding-3-Large~\cite{openai2024textembedding} under the zero-shot setting, with results shown in Table.~\ref{tab:confidence}. We replicate experiment for $100$ times with random seeds from $42$ to $141$ and obtain classification accuracy of each method. To check normality, we first apply Shapiro-Wilk test~\cite{SHAPIRO1965}. If the data follows a normal distribution, we perform a Paired-t test~\cite{student1908probable}; otherwise, we use Wilcoxon Signed-Rank test~\cite{wilcoxon1992individual}, with packages from SciPy~\cite{2020SciPy-NMeth}.
The lower and upper bounds under $90\%$ confidence interval are estimated with bootstrap algorithm~\cite{tibshirani1993introduction} to sample $10,000$ times.
Task-adaptive encoding show statistically significant improvement over vanilla LLM2Vec in $9$ out of $11$ dataset and outperforms Text-Embedding-3-Large in $8$ out of $11$ datasets (bolded in the table).

\subsection{LLM Agents' Prediction on Homophily Ratio $r$}
\begin{figure*}[h]
    \centering
    \hfill
    \begin{minipage}{0.23\textwidth}
    \captionsetup{labelformat=empty}
        \centering
        \includegraphics[width=\textwidth]{Arxiv/figures/pred_h/pred_h_GPT-4o.pdf}
        \vspace{-0.8cm}
        \caption*{\scriptsize GPT-4o.}
    \end{minipage}
    \hfill
    \begin{minipage}{0.23\textwidth}
    \captionsetup{labelformat=empty}
        \centering
        \includegraphics[width=\textwidth]{Arxiv/figures/pred_h/pred_h_GPT-4o-mini.pdf}
        \vspace{-0.8cm}
        \caption*{\scriptsize GPT-4o-mini.}
    \end{minipage}
    \begin{minipage}{0.23\textwidth}
    \captionsetup{labelformat=empty}
        \centering
        \includegraphics[width=\textwidth]{Arxiv/figures/pred_h/pred_h_GPT-3.5-turbo.pdf}
        \vspace{-0.8cm}
        \caption*{\scriptsize GPT-3.5-turbo.}
    \end{minipage}
    \begin{minipage}{0.23\textwidth}
    \captionsetup{labelformat=empty}
        \centering
        \includegraphics[width=\textwidth]{Arxiv/figures/pred_h/pred_h_Mistral7B-Instruct-v0.3.pdf}
        \vspace{-0.8cm}
        \caption*{\scriptsize Mistral7B-Instruct-v0.3.}
    \end{minipage}
    \vspace{-0.cm}
    \caption{LLM agents' performance on predicting the homophily constant $r$.}
    \label{fig:predict_h_more}
\end{figure*}
More prediction performance of GPT-4o, GPT-3.5-turbo and Mistral7b-Instruct-v3 are shown in Fig.~\ref{fig:predict_h_more}.

%\subsection{Effectiveness of Class Information on Text-Embedding-3-Large}
%The effectiveness of class information on text-embedding-3-large is shown in Fig.~\ref{fig:class_condition_text-embedding-3-large}.
%\input{Arxiv/figures/class_condition_text-embedding-3-large}

\subsection{Zero-Shot Comparison with LLM-GNN~\cite{chen2023label} and TEA-GLM~\cite{wang2024llms}}
\label{sec:app_more_baselines}
\begin{table}[h]
\centering
\begin{tabular}{ccccc}
\hline
 & Cora & Citeseer & Pubmed & Wikics \\ \hline
DA-AGE-W & 74.96 & 58.41 & 65.85 & 59.13 \\
DA-RIM-W & 74.73 & 60.80 & 77.94 & 68.22 \\
DA-GraphPart-W & 68.61 & 68.82 & 79.89 & 67.13 \\ \hline
LLM-BP & 72.59 & 69.51 & 75.55 & 67.75 \\
LLM-BP (app.) & 71.41 & 68.66 & 76.81 & 67.96 \\ \hline
\end{tabular}
\caption{Accuracy compared with LLM-GNN, where `DA' denotes the `C-Density' methods proposed in ~\cite{chen2023label}, `-W' refers to the weighted cross-entropy loss function used for training,
AGE~\cite{cai2017active}, RIM~\cite{zhang2021rim}, GraphPart~\cite{ma2022partition} are different graph active learning baselines used in the original paper.}
\label{tab:compare_llm_gnn}
\end{table}
\begin{table}[h]
\centering
\begin{tabular}{ccccc}
\hline
 & Cora & Pubmed & History & Children \\ \hline
TEA-GLM & 20.2 & 84.8 & 52.8 & 27.1 \\ \hline
LLM-BP & 72.59 & 75.55 & 59.86 & 24.81 \\
LLM-BP (app.) & 71.41 & 76.81 & 59.49 & 29.4 \\ \hline
\end{tabular}
\caption{Accuracy compared with TEA-GLM~\cite{wang2024llms}.}
\label{tab:compare_tea-glm}
\end{table}
Here we present the comparison with LLM-GNN~\cite{chen2023label} in Table.~\ref{tab:compare_llm_gnn}. We compare with three different graph active learning heuristics from their original paper. Our training-free methods, LLM-BP and LLM-BP (appr.) achieves top performance on Citeseer and Wikics, while performs comparably with the baselines in Cora and Pubmed. Note that the results of LLM-GNN are from Table. 2 in the original paper.


The comparison with TEA-GLM is shown in Table.~\ref{tab:compare_tea-glm}. Results of TEA-GLM are from Table.1 in their original paper.

\subsection{Experiment Results in Few-Shot Setting}
\label{sec:app_few_shot}
We use $10$ different random seeds from $42$ to $52$ to sample the $k$-shot labeled nodes from training dataset, and report the average accuracy and macro $F1$ score with standard variance. Results are shown in Table.~\ref{tab:few_shot}. Across all the $k$s, our LLM-BP achieves the top ranking performance across all the eleven datasets, exhibiting similar insights with the zero-shot setting.

\begin{table*}[!ht]
\setlength{\tabcolsep}{2pt}
\captionsetup{font=small}
  \centering
  \caption{Few-shot learning on 10\% training data. We use the same protocol in Table~\ref{tab:long_term_results}. All results are averaged from four different forecasting horizons: $H \in \{96, 192, 336, 720\}$. \boldres{Red}: best, \secondres{Blue}: second best. Our full results are in Appendix~\ref{appx:few_shot_details}.}
\scalebox{0.95}{
    \begin{tabular}{c|cc|cc|cc|cc|cc|cc|cc|cc|cc|cc}
    \toprule
    \multirow{2}{*}{Methods} & \multicolumn{2}{c|}{LDM4TS} & \multicolumn{2}{c|}{CSDI} & \multicolumn{2}{c|}{ScoreGrad} & \multicolumn{2}{c|}{Autoformer} & \multicolumn{2}{c|}{FEDformer} & \multicolumn{2}{c|}{DLinear} & \multicolumn{2}{c|}{Informer} & \multicolumn{2}{c|}{TimesNet} & \multicolumn{2}{c|}{LightTS} & \multicolumn{2}{c}{Reformer} \\
    & \multicolumn{2}{c|}{\textbf{(Ours)}} & \multicolumn{2}{c|}{\citeyearpar{tashiro2021csdi}} & \multicolumn{2}{c|}{\citeyearpar{song2020score}} & \multicolumn{2}{c|}{\citeyearpar{wu2021autoformer}} & \multicolumn{2}{c|}{\citeyearpar{zhou2022fedformer}} & \multicolumn{2}{c|}{\citeyearpar{zeng2023transformers}} & \multicolumn{2}{c|}{\citeyearpar{zhou2021informer}} & \multicolumn{2}{c|}{\citeyearpar{wu2022timesnet}} & \multicolumn{2}{c|}{\citeyearpar{campos2023lightts}} & \multicolumn{2}{c}{\citeyearpar{kitaev2020reformer}}\\
    \midrule
    Metric & MSE & MAE & MSE & MAE & MSE & MAE & MSE & MAE & MSE & MAE & MSE & MAE & MSE & MAE & MSE & MAE & MSE & MAE & MSE & MAE \\
    \midrule
    \textit{ETTh1} & \textcolor{red}{\textbf{0.471}} & \textcolor{red}{\textbf{0.468}} & 0.849 & 0.665 & 1.031 & 0.709 & 0.701 & 0.596 & \textcolor{blue}{\underline{0.638}} & \textcolor{blue}{\underline{0.561}} & 0.691 & 0.599 & 1.199 & 0.808 & 0.869 & 0.628 & 1.375 & 0.877 & 1.249 & 0.833\\
    \textit{ETTh2} & \textcolor{red}{\textbf{0.452}} & \textcolor{red}{\textbf{0.460}} & 0.527 & 0.523 & 0.512 & 0.505 & 0.488 & 0.499 & \textcolor{blue}{\underline{0.466}} & 0.475 & 0.608 & 0.538 & 3.871 & 1.512 & 0.479 & \textcolor{blue}{\underline{0.465}} & 2.655 & 1.159 & 3.485 & 1.486\\
    \textit{ETTm1} & \textcolor{red}{\textbf{0.371}} & \textcolor{red}{\textbf{0.393}} & 0.784 & 0.606 & 1.015 & 0.678 & 0.802 & 0.628 & 0.721 & 0.605 & \textcolor{blue}{\underline{0.411}} & \textcolor{blue}{\underline{0.429}} & 1.192 & 0.820 & 0.479 & 0.465 & 0.970 & 0.704 & 1.426 & 0.856 \\
    \textit{ETTm2} & 0.336 & 0.373 & 0.334 & 0.385 & 0.446 & 0.447 & 1.341 & 0.930 & 0.463 & 0.488 & \textcolor{red}{\textbf{0.316}} & \textcolor{blue}{\underline{0.368}} & 3.369 & 1.439 & \textcolor{blue}{\underline{0.319}} & \textcolor{red}{\textbf{0.353}} & 0.987 & 0.755 & 3.978 & 1.587 \\
    
    \textit{Weather} & \textcolor{red}{\textbf{0.229}} & \textcolor{red}{\textbf{0.276}} & 0.295 & 0.333 & 0.347 & 0.351 & 0.300 & 0.342 & 0.284 & \textcolor{blue}{\underline{0.283}} & \textcolor{blue}{\underline{0.241}} & \textcolor{blue}{\underline{0.283}} & 0.597 & 0.494 & 0.279 & 0.301 & 0.289 & 0.322 & 0.526 & 0.469\\
    
    \textit{ECL} & \textcolor{red}{\textbf{0.172}} & \textcolor{red}{\textbf{0.275}} & 0.909 & 0.785 & 1.258 & 0.884 & 0.431 & 0.478 & 0.346 & 0.428 & \textcolor{blue}{\underline{0.180}} & \textcolor{blue}{\underline{0.280}} & 1.194 & 0.891 & 0.323 & 0.392 & 0.441 & 0.488 & 0.980 & 0.769 \\
    
    \textit{Traffic} & \textcolor{red}{\textbf{0.621}} & \textcolor{red}{\textbf{0.357}} & 1.744 & 0.871 & 2.100 & 1.020 & 0.749 & 0.446 & \textcolor{blue}{\underline{0.663}} & \textcolor{blue}{\underline{0.425}} & 0.945 & 0.570 & 1.534 & 0.811 & 0.951 & 0.535 & 1.247 & 0.684 & 1.551 & 0.821\\
\midrule
    1st Count & 6 & 6 & 0 & 0 & 0 & 0 & 0 & 0 & 0 & 0 & 1 & 0 & 0 & 0 & 0 & 1 & 0 & 0 & 0 & 0\\
    \bottomrule
    \end{tabular}%
}
  \label{tab:few_shot}%
\end{table*}%
\subsection{Zero-Shot Link Prediction Results}
\begin{table}[h]
\centering
\setlength{\tabcolsep}{2pt}
\resizebox{1.0\textwidth}{!}{\begin{tabular}{c|ccc|cccc}
\hline
 & \multicolumn{3}{c|}{Citation Graph} & \multicolumn{4}{c}{E-Commerce \& Knowledge Graph} \\
 & Cora & Citeseer & Pubmed & History & Children & Sportsfit & Wikics \\ \hline
OFA & 0.492 & -- & 0.481 & 0.431 & 0.484 & 0.517 & -- \\
LLaGA & 0.527 & -- & 0.543 & 0.515 & 0.500 & 0.502 & -- \\
GraphGPT & 0.520 & -- & 0.569 & 0.449 & 0.422 & 0.597 & -- \\
TEA-GLM & 0.586 & -- & 0.689 & 0.579 & 0.571 & 0.553 & -- \\ \hline
SBert & 0.979±0.033 & 0.990±0.001 & 0.979±0.003 & 0.985±0.002 & 0.972±0.030 & 0.975±0.003 & 0.972±0.003 \\
Text-Embedding-3-Large & 0.975±0.003 & 0.989±0.002 & 0.979±0.003 & 0.985±0.001 & 0.980±0.002 & 0.987±0.001 & 0.977±0.002 \\
LLM2Vec & 0.966±0.004 & 0.982±0.002 & 0.970±0.003 & 0.971±0.002 & 0.973±0.003 & 0.975±0.002 & 0.978±0.003 \\ \hline
\end{tabular}}
\caption{Performance on zero-shot link prediction tasks (AUC). Results of baselines are from~\cite{wang2024llms}.}
\label{tab:link_prediction}
\end{table}
\label{sec:app_link_prediction}

For each dataset, We randomly sample \& remove $1000$ edges and $1000$ node pairs from the graph as testing data. A straightforward approach is to compare the cosine similarity between node embeddings to determine the presence of a link. Specifically, we aggregate embeddings for $3$ layers on the incomplete graph and compute the cosine similarity between node representations, achieving better zero-shot performance than LLMs-with-Graph-Adapters methods~\cite{wang2024llms, chen2024llaga, tang2024graphgpt}, as shown in Table.~\ref{tab:link_prediction}. Note that the performance in the table refers to LLM-with-Graph-Adapters that have only been trained on other tasks and never on link prediction tasks.

We leave the design of task-adaptive embeddings and generalized graph structural utilization for link prediction as future work, including task-adaptive encoding prompts.







\section{Prompts}
\subsection{Task Description for Vanilla LLM2Vec without Class Information}
\label{sec:app_prompt_llm2vec_task_description}
Table.~\ref{tab:llm2vec_task_description} shows the task description for vanilla LLM2Vec encoder across all the datasets.
\begin{table}[h]
\small
\centering
\begin{tabular}{@{}cc@{}}
\toprule
Dataset & Task Description \\ \midrule
Cora & Encode the text of machine learning papers: \\
Citeseer & Encode the description or opening text of scientific publications: \\
Pubmed & Encode the title and abstract of scientific publications: \\
History & Encode the description or title of the book: \\
Children & Encode the description or title of the child literature: \\
Sportsfit & Encode the title of a good in sports \& fitness: \\
Wikics & Encode the entry and content of wikipedia: \\
Cornell & Encode the webpage text: \\
Texas & Encode the webpage text: \\
Wisconsin & Encode the webpage text: \\
Washington & Encode the webpage text: \\ \bottomrule
\end{tabular}
\caption{\{Task description\} for vanilla LLM2Vec~\cite{li2024making} encoder. See Eq.~\ref{eq:vanilla_llm2vec} for detailed prompting format.}
\label{tab:llm2vec_task_description}
\end{table}

\subsection{Prompts for Vanilla LLMs}
\label{sec:app_prompt_vanilla_llm}
Table.~\ref{tab:vanilla_llm_task_description} shows tha task description for vanilla LLM decoders.
\begin{table}[h]
\small
\centering
\begin{tabular}{@{}cc@{}}
\toprule
Dataset & Task Description \\ \midrule
Cora & opening text of machine learning papers \\
Citeseer & description or opening text of scientific publications \\
Pubmed & title and abstract of scientific publications \\
History & description or title of the book \\
Children & description or title of the child literature \\
Sportsfit & the title of a good in sports \& fitness \\
Wikics & entry and content of wikipedia \\
Cornell & webpage text \\
Texas & webpage text \\
Wisconsin & webpage text \\
Washington & webpage text \\ \bottomrule
\end{tabular}
\caption{\{Task description\} in the prompts for both vanilla LLM decoders (See Section.~\ref{sec:app_vanilla_LLM_implementation}) and task-adaptive encoder (See Section.~\ref{sec:app_task_adaptive_implementation}).}
\label{tab:vanilla_llm_task_description}
\end{table}

%\subsection{Class Description}
%\label{sec:app_class_description}

\newpage


%\begin{table}[t]
\scriptsize
\begin{tabular}{|p{1.2cm}|p{15cm}|}
\toprule
Class & \multicolumn{1}{c}{Description} \\ \midrule
\begin{tabular}[c]{@{}c@{}}Case \\ Based\end{tabular} & Case-based research solves problems by retrieving and adapting past cases, emphasizing analogy-based reasoning. It integrates memory structures for incremental learning and leverages domain-specific knowledge for interpretability and adaptability, excelling in fields like medical diagnosis and legal reasoning. \\ \midrule
\begin{tabular}[c]{@{}c@{}}Genetic \\ Algorithms\end{tabular} & Genetic Algorithms (GA), inspired by evolutionary biology, use selection, crossover, and mutation to iteratively optimize solutions. Their stochastic, population-based search excels in large or rugged spaces.\\ \midrule
\begin{tabular}[c]{@{}c@{}}Neural \\ Networks\end{tabular} & Neural Networks (NNs) are models to learn representation from data. Paper of  deep neural networks explores tasks on large-scale datasets. Novel architectures, combining multiple convolutional layers, pooling operations, and fully connected layers, are proposed to efficiently extract hierarchical features. Additionally, techniques such as dropout regularization and data augmentation are integrated to enhance generalization. \\ \midrule
\begin{tabular}[c]{@{}c@{}}Probabilistic \\ Methods\end{tabular} & Probabilistic methods model uncertainty and reason under incomplete information using Bayes' theorem. Key components include prior distributions, likelihood functions, and posterior distributions, enabling systems to adapt as new data. Advanced techniques such as Markov Chain Monte Carlo (MCMC), variational inference, and Bayesian optimization can approximate complex posterior distributions. Gaussian processes, Bayesian networks, and hierarchical models excel in tasks requiring uncertainty quantification, small-data generalization, and interpretable probabilistic predictions. Applications include model calibration, decision-making under ambiguity, and domains such as medical diagnosis, sensor fusion, and automated experimentation. \\ \midrule
\begin{tabular}[c]{@{}c@{}}Reinforcement \\ Learning\end{tabular} & Reinforcement Learning (RL) trains agents to optimize decisions by maximizing cumulative rewards in a Markov Decision Process (MDP). Key concepts include value functions, Q-learning, policy gradients, and the exploration-exploitation trade-off. Algorithms like DQN, Actor-Critic, and PPO solve high-dimensional decision problems. RL tackles challenges such as reward sparsity, credit assignment, and uncertainty, with applications in game playing, robotics, resource allocation, and autonomous systems. \\ \midrule
\begin{tabular}[c]{@{}c@{}}Rule \\ Learning\end{tabular} & Rule Learning involves discovering interpretable if-then rules from data to represent patterns or decision processes. Its uniqueness is its ability to produce human-readable and explainable models. Algorithms such as RIPPER and CN2 emphasize learning concise and accurate rule sets. \\ \midrule
Theory & Machine learning theory establishes the mathematical foundations and guarantees of learning algorithms, ensuring reliability, efficiency, and generalization. Key areas include statistical learning theory (model generalization) and computational learning theory (learning feasibility). Core concepts include VC dimension, PAC learning, generalization bounds, and convergence guarantees. Research analyzes overfitting, underfitting, bias-variance trade-offs, and sample complexity. \\ \bottomrule
\end{tabular}
\vspace{-0.3cm}
\caption{Class descriptions for Cora~\cite{mccallum2000automating}.}
\label{tab:class_description_cora}
\end{table}

%\begin{table}[h]
\scriptsize
\begin{tabular}{|p{1.5cm}|p{14.5cm}|}
\toprule
Class & Description \\ \midrule
Agents & The study focus on computational agents capable of rational behavior. Modeling agents requires to provide a structured approach to decision-making based on an agent's beliefs, desires, and intentions. Agents are grounded in both quantitative decision-theoretic models and symbolic reasoning approaches, enabling them to operate effectively in complex dynamic environments.     \\ \midrule
\begin{tabular}[c]{@{}l@{}}machine \\ learning(ML)\end{tabular} & Machine learning includes lazy learning, probabilistic learning, and case-based reasoning. Machine learning is studied in cognitive science as a model of human reasoning and in inductive learning for similarity-based generalization. key challenges lie in defining effective measures, evolving from rigid metrics to adaptive, context-aware approaches, such as non-symmetric similarity in structured learning. \\ \midrule
\begin{tabular}[c]{@{}l@{}}information \\ retrieval (IR)\end{tabular} & is the process of retrieving relevant information from large datasets, typically text collections. Methods include centroid-based classification, Naïve Bayes, k-NN, and decision trees by leveraging similarity measures to adjust for class density and term dependencies. Beyond supervised classification, unsupervised categorization aids automated retrieval, enabling dynamic query generation and document filtering. \\ \midrule
database (DB) & Databases manage structured data efficiently, especially in distributed environments, with semantic query caches that improve query performance by storing results and optimizing resource use. Database integration enables access to autonomous, heterogeneous sources through a unified schema. \\ \midrule
\begin{tabular}[c]{@{}l@{}}human-computer \\ interaction (HCI)\end{tabular} & Human-Computer Interaction (HCI) creates intuitive, adaptive interfaces for real-time collaboration and multimodal interaction. Customization supports tailored workflows in group collaboration, while modern systems enhance adaptation and learning through shared workspaces for real-time component importation. \\ \midrule
\begin{tabular}[c]{@{}l@{}}artificial \\ intelligence (AI)\end{tabular} & AI enables autonomous decision making to operate in dynamic environments with monitored execution. AI-driven learning and decision making aim to produce human-like performance by detecting and correcting discrepancies between internal models and real-world conditions.  \\ \bottomrule
\end{tabular}
\vspace{-0.3cm}
\caption{Class Description for Citeseer~\cite{giles1998citeseer}.}
\label{tab:class_description_citeseer}
\end{table}

%\begin{table}[h]
\scriptsize
\begin{tabular}{|p{2cm}|p{14.5cm}|}
\toprule
Class & Description \\ \midrule
\begin{tabular}[c]{@{}l@{}}Diabetes Mellitus \\ Experimental\end{tabular} & Experimental studies on  diabetic rats reveal molecular mechanisms of diabetes-related complications. Research focuses on ion transport regulation, including isoform- and tissue-specific changes, increased vascular permeability, assessed via  albumin permeation, marks diabetic damage in eyes, kidneys, and nerves. Key interventions include aldose reductase inhibitors (sorbinil, tolrestat) and castration.  \\ \midrule
\begin{tabular}[c]{@{}l@{}}Diabetes Mellitus \\ Type 1\end{tabular} & Type 1 Diabetes Mellitus (T1DM) is an autoimmune disease characterized by the destruction of insulin-producing $\beta$-cells, leading to insulin deficiency and lifelong dependence on exogenous insulin therapy. Genetic predisposition, particularly related with HLA-DR, plays a crucial role in disease susceptibility and is related with the presence of insulin autoantibodies (IAAs). Key methodologies included intravenous glucose tolerance tests (IVGTT), serological tests for ICAs and IAAs, and HLA-DQ genotyping, which collectively helped identify predictive markers for the progression to Type 1 diabetes.  \\ \midrule
\begin{tabular}[c]{@{}l@{}}Diabetes Mellitus \\ Type 2\end{tabular} & Type 2 Diabetes Mellitus (T2DM) is a complex metabolic disorder characterized by insulin resistance, impaired insulin secretion, and progressive glucose intolerance. The progression of normal glucose tolerance to impaired is mainly induced by insulin resistance,  and can also be contributed by $\beta$-cell dysfunction, and ultimately lead to diabetes. Individuals exhibiting low insulin sensitivity (M-value) and diminished acute insulin response (AIR) face a significantly higher risk of developing T2DM, underscoring the need for targeted interventions at different disease stages. T2DM progression is associated with changes in adipokine levels, lipid metabolism, and advanced glycation end-products (AGEs). \\ \bottomrule
\end{tabular}
\caption{Class decription for Pubmed~\cite{sen2008collective}.}
\label{tab:class_description_pubmed}
\end{table}
%\begin{table}[h]
\small
\begin{tabular}{|p{2cm}|p{14.5cm}|}
\toprule
Class & Description \\ \midrule
Other Sports & Explore a diverse range of sporting activities and equipment beyond the mainstream. From archery to skateboarding, roller skating to juggling, this category caters to niche and unconventional sports enthusiasts. \\ \midrule
Golf & Improve your swing and elevate your golf game with top-tier golf equipment and accessories. Explore clubs, balls, bags, carts, and apparel designed for maximum performance and enjoyment on the course. \\ \midrule
\begin{tabular}[c]{@{}l@{}}Hunting \\ \& Fishing\end{tabular} & Gear up for your next outdoor adventure with top-quality hunting and fishing equipment. Discover firearms, bows, fishing rods, reels, lures, and all the essential accessories for a successful and enjoyable experience. \\ \midrule
\begin{tabular}[c]{@{}l@{}}Exercise \\ \& Fitness\end{tabular} & Stay motivated and achieve your fitness goals with a wide selection of exercise and fitness equipment. Find treadmills, ellipticals, weights, yoga mats, and more to support your active lifestyle and wellness journey. \\ \midrule
Team Sports & Unite with your teammates and showcase your skills with top-quality team sports equipment. Find balls, goals, protective gear, and apparel for sports like basketball, football, soccer, baseball, and more. \\ \midrule
Accessories & Enhance your sports and fitness experience with a vast array of accessories. From hydration packs and wearable tech to protective gear, storage solutions, and more, these products ensure you're fully equipped for any activity. \\ \midrule
Swimming & Make a splash with swimming gear designed for comfort, performance, and safety. Explore swimsuits, goggles, kickboards, pool accessories, and more to enhance your swimming routine and aquatic workouts. \\ \midrule
\begin{tabular}[c]{@{}l@{}}Leisure Sports \\ \& Game Room\end{tabular} & Bring the fun and excitement of leisure sports into your home with game room equipment. Explore pool tables, air hockey, foosball, darts, and other recreational games for endless entertainment and quality family time. \\ \midrule
\begin{tabular}[c]{@{}l@{}}Airsoft \\ \& Paintball\end{tabular} & Experience the thrill of simulated combat with airsoft and paintball gear. Discover replica firearms, protective gear, ammunition, and accessories for intense and adrenaline-fueled games and competitions. \\ \midrule
\begin{tabular}[c]{@{}l@{}}Boating \\ \& Sailing\end{tabular} & Set sail for thrilling adventures on the water with boating and sailing gear. Discover boats, kayaks, canoes, paddleboards, life jackets, and essential accessories for safe and enjoyable aquatic experiences. \\ \midrule
Sports Medicine & Prioritize your health and recovery with sports medicine equipment and supplies. Explore braces, supports, therapy tools, and rehabilitation gear to prevent injuries, manage pain, and enhance your overall athletic performance. \\ \midrule
\begin{tabular}[c]{@{}l@{}}Tennis \\ \& Racquet Sports\end{tabular} & Serve up your best game with high-quality tennis and racquet sports equipment. Find rackets, balls, court accessories, and apparel for sports like tennis, badminton, squash, and more. \\ \midrule
Clothing & Stay comfortable and stylish while engaging in your favorite sports and activities with performance-driven clothing. Find athletic apparel, shoes, and accessories designed for optimal mobility and breathability. \\ \bottomrule
\end{tabular}
\caption{Class Description for Sportsfit~\cite{ni2019justifying}.}
\label{tab:class_description_sportsfit}
\end{table}
%\begin{table}[h]
\small
\begin{tabular}{|p{1cm}|p{15cm}|}
\toprule
Class & Description \\ \midrule
student & encompasses individuals actively enrolled in educational programs, ranging from undergraduate to graduate levels, across diverse disciplines. These individuals engage in academic activities, such as attending lectures, completing assignments, and participating in research or extracurricular projects, contributing significantly to the institution's learning environment. \\ \midrule
project & structured academic or extracurricular initiatives undertaken to achieve specific learning or research goals. Projects may vary in scope, involving individual or group work, and often integrate practical applications of theoretical concepts, culminating in presentations, reports, or tangible outcomes \\ \midrule
course & represents the structured curriculum designed to deliver specific knowledge and skills within a discipline. Courses typically include lectures, discussions, assignments, and assessments, providing students with the foundational and advanced understanding needed to excel in their fields of study. \\ \midrule
staff & non-academic personnel who support the institution's daily operations, such as administrative tasks, facilities management, and student services. Their contributions ensure a well-functioning environment conducive to education and research. \\ \midrule
faculty & academic professionals responsible for teaching, mentoring, and conducting research. Faculty members play a critical role in shaping the educational experience, fostering intellectual growth, and advancing knowledge within their respective disciplines. \\ \bottomrule
\end{tabular}
\caption{Class Description for Cornell, Texas, Wisconsin and Washington~\cite{craven1998learning}.}
\label{tab:class_description_cornell_texas_wisconsin_washington}
\end{table}
%\begin{table}[h]
\scriptsize
\begin{tabular}{|p{2cm}|p{14.0cm}|}
\toprule
Class & Description \\ \midrule
World & Provides rich insights into the historical, theological, and cultural dimensions of religions and their societal impacts and their shared and distinct traditions through primary and secondary sources, fostering comparative understanding. Together, these works illuminate global religious and historical knowledge. \\ \midrule
Americas & explore North America that comprises distinct cultural nations, each with unique historical roots, analyze how regions maintain their distinct identities and values, influencing political and social dynamics, including electoral outcomes and legislative behavior and pivotal moments in American history. Such narratives delve into the lives of ordinary people impacted by these developments, revealing the paradox of progress and the drive to assert the United States' role in shaping the modern world. \\ \midrule
Asia & Explores Asian history and military wars, focusing on military strategy, soldier experiences, and atrocities. Analyzes expansion and war's impact like Chosin Reservoir or the 1937 Battle for Nanjing,on society and global geopolitics. \\ \midrule
Military & analyze pivotal battles, campaigns, and military technologies. They cover air campaign strategies, ground operations in tough terrains, and airborne missions, detailing coordination and challenges. Combining historical rigor with visual and analytical insights, they illuminate key aspects of military history. \\ \midrule
Europe & explore European battles, key figures, and cultural developments, analyze soldier experiences at Agincourt, Waterloo, Somme, etc offering insights into warfare and leadership. Eyewitness accounts provide vivid perspectives on pivotal events that have shaped Europe. \\ \midrule
Russia & examine Russia’s tumultuous history, covering from the Civil War’s human toll, to Soviet rule’s establishment, analyze the Russian Revolution, and insights into Russia’s political evolution, ideological struggles, and global impact across different historical periods. \\ \midrule
Africa & examine Africa’s military, cultural, and historical dynamics, covering Cold War-era conflicts, global influence, and firsthand accounts. They explore the rise and fall of iconic tribes, blending history with cultural perspectives, and analyze African history through archaeology, revealing ancient states’ enduring legacies. \\ \midrule
\begin{tabular}[c]{@{}l@{}}Ancient \\ Civilizations\end{tabular} & explore early human societies, their cultures, innovations, and key figures. They examine empires like Ancient Egypt, ancient civilizations from Mesopotamia to Rome, and the roles of religion, law, and military power. Additionally, they detail travel, trade, and tourism, highlighting connections between distant cultures. \\ \midrule
Middle East & explore the Middle East’s historical, political, and cultural complexities, covering identity, conflict, and transformation, analyze national myths in politics, provide geopolitical insights with maps and data, and examine the roots of extremism, ideology, and violence, offering a comprehensive view of the region. \\ \midrule
\begin{tabular}[c]{@{}l@{}}Historical Study \\ \& \\ Educational Resources\end{tabular} & provide insights into historical methods, narratives, and interpretations across eras and region, including firsthand accounts of key events, analyses of historiography, and studies of specific periods like WWII, combining research with accessible narratives. Valuable for students and historians, they enhance historical understanding and teaching. \\ \midrule
Australia \& Oceania & explore Australia and Oceania’s history, cultures, and environment, from WWII campaigns in New Guinea, eto strategic battles in the Solomon Islands, offering detailed perspectives on pivotal events shaping the region. \\ \midrule
Arctic \& Antarctica & explore Arctic and Antarctic challenges, historic expeditions, and human resilience. They recount triumphs and tragedies of polar exploration, the race to the South Pole, and lost Arctic journeys, using archaeological and historical insights to reveal survival, discovery, and endurance in extreme conditions. \\ \bottomrule
\end{tabular}
\caption{Class Description for History~\cite{ni2019justifying}.}
\label{tab:class_description_bookhis}
\end{table}
%\begin{table}[h]
\scriptsize
\begin{tabular}{|p{2.5cm}|p{13.5cm}|}
\toprule
Class & Description \\ \midrule
\begin{tabular}[c]{@{}c@{}}Literature \\ \& Fiction\end{tabular} & Explore the wonderful world of children's literature and fiction books. From heartwarming tales to imaginative adventures, these books spark creativity and foster a love for reading. \\ \midrule
Animals & Learn about the fascinating world of animals through engaging stories and vibrant illustrations. Discover different species, habitats, and the importance of caring for our furry and feathered friends. \\ \midrule
\begin{tabular}[c]{@{}c@{}}Growing Up \\ \& Facts of Life\end{tabular} & Navigate the journey of growing up with age-appropriate books that address physical and emotional changes, relationships, and important life lessons with sensitivity and wisdom. \\ \midrule
Humor & Unleash the power of laughter with hilarious books that tickle the funny bone. Filled with puns, jokes, and silly scenarios, these reads are sure to bring endless giggles. \\ \midrule
\begin{tabular}[c]{@{}c@{}}Cars Trains \\ \& Things That Go\end{tabular} & Vroom! Choo-choo! Explore the exciting world of transportation with engaging books about cars, trains, planes, and more. Perfect for little ones fascinated by wheels and motion. \\ \midrule
\begin{tabular}[c]{@{}c@{}}Fairy Tales Folk Tales \\ \& Myths\end{tabular} & Dive into the enchanting realm of fairy tales, folk tales, and myths from around the world. These timeless stories ignite imagination and teach valuable life lessons. \\ \midrule
\begin{tabular}[c]{@{}c@{}}Activities Crafts \\ \& Games\end{tabular} & Unleash creativity and have fun with activity books, crafts projects, and games. From coloring to origami, these books provide endless entertainment and learning opportunities. \\ \midrule
\begin{tabular}[c]{@{}c@{}}Science Fiction \\ \& Fantasy\end{tabular} & Embark on epic adventures and explore imaginative worlds in captivating science fiction and fantasy tales. These books inspire curiosity and transport readers to realms beyond our own. \\ \midrule
Classics & Discover beloved literary treasures that have stood the test of time. These classic children's books are cherished for their timeless themes, memorable characters, and enduring lessons. \\ \midrule
\begin{tabular}[c]{@{}c@{}}Mysteries \\ \& Detectives\end{tabular} & Sharpen your problem-solving skills with thrilling mysteries and detective stories. Follow the clues, unravel the puzzles, and experience the excitement of solving cases. \\ \midrule
\begin{tabular}[c]{@{}c@{}}Action \\ \& Adventure\end{tabular} & Get your adrenaline pumping with heart-pounding action and adventure tales. These page-turners take readers on daring quests, thrilling escapades, and courageous journeys. \\ \midrule
\begin{tabular}[c]{@{}c@{}}Geography \\ \& Cultures\end{tabular} & Explore the rich tapestry of our world through fascinating books on geography and cultures. Discover new lands, traditions, and gain a deeper appreciation for diversity. \\ \midrule
\begin{tabular}[c]{@{}c@{}}Education \\ \& Reference\end{tabular} & Enhance learning and knowledge with educational and reference books. From encyclopedias to study guides, these resources support academic growth and intellectual curiosity. \\ \midrule
\begin{tabular}[c]{@{}c@{}}Arts Music \\ \& Photography\end{tabular} & Ignite creativity and appreciation for the arts with engaging books on music, visual arts, and photography. These books nurture artistic expression and cultural awareness. \\ \midrule
\begin{tabular}[c]{@{}c@{}}Holidays \\ \& Celebrations\end{tabular} & Embrace the joy and traditions of holidays and celebrations from around the world. These festive books provide insights into customs, histories, and the spirit of special occasions. \\ \midrule
\begin{tabular}[c]{@{}c@{}}Science Nature \\ \& How It Works\end{tabular} & Discover the wonders of science, nature, and how things work through captivating explanations and vibrant visuals. These books foster curiosity and a love for learning. \\ \midrule
Early Learning & Lay a strong foundation for learning with engaging books designed to support early childhood development. From ABCs to numbers, these resources promote essential skills. \\ \midrule
Biographies & Explore the remarkable lives of inspiring individuals through engaging biographies. These books celebrate achievements, perseverance, and the contributions of notable figures. \\ \midrule
History & Step back in time and learn about the events and people that shaped our world. These history books bring the past to life and nurture a deeper understanding of our heritage. \\ \midrule
Children's Cookbooks & Unleash your inner chef with fun and educational cookbooks designed specifically for young culinary enthusiasts. Learn cooking basics and create delicious dishes. \\ \midrule
Religions & Foster understanding and respect for diverse religious beliefs and traditions through insightful books on world religions, faiths, and spiritual practices. \\ \midrule
\begin{tabular}[c]{@{}c@{}}Sports \\ \& Outdoors\end{tabular} & Celebrate the spirit of sportsmanship and adventure with exciting books on sports, outdoor activities, and athletic pursuits. These reads inspire an active and healthy lifestyle. \\ \midrule
\begin{tabular}[c]{@{}c@{}}Comics \\ \& Graphic Novels\end{tabular} & Immerse yourself in the dynamic world of comics and graphic novels. These visually stunning books blend art and storytelling, captivating readers of all ages. \\ \midrule
\begin{tabular}[c]{@{}c@{}}Computers \\ \& Technology\end{tabular} & Discover the fascinating world of computers and technology through engaging books that explain concepts, coding, and the role of technology in our lives. \\ \bottomrule
\end{tabular}
\caption{Class Descriptions for Children~\cite{ni2019justifying}.}
\label{tab:class_description_bookchild}
\end{table}
%\begin{table}[h]
\scriptsize
\begin{tabular}{|p{2cm}|p{14cm}|}
\toprule
Class & Description \\ \midrule
\begin{tabular}[c]{@{}l@{}}computational \\ linguistics\end{tabular} & study of language using computational methods, focusing on the process of grouping together inflected forms of a word and identifying its dictionary or base form, known as the lemma. Computing linguistics simplify words by removing affixes, or ensure accurate base word selection based on grammatical and semantic context.   \\ \midrule
databases & structured system for storing, managing, and retrieving data, often accessed through a database server, such as those using DBMSs like MySQL or Oracle, to handle query processing, data analysis, and storage. Databases are designed for networked environments and often implement master-slave models for data redundancy and load balancing.  \\ \midrule
\begin{tabular}[c]{@{}l@{}}operating \\ systems\end{tabular} & Operating systems serve as the interface between hardware and software, encapsulate resources as objects, enabling inheritance and polymorphism to simplify system design and extend functionality. A file or device driver can be represented as an object, abstracting the implementation details while allowing access through defined methods. Each object can enforce access control and limited privileges based on user roles. Modern operating systems incorporate designs in components like kernels and user interfaces, leveraging these principles for improved modularity, maintainability, and resource protection.  \\ \midrule
\begin{tabular}[c]{@{}l@{}}computer \\ architecture\end{tabular} & Computer architecture defines the organization and design of a system's hardware, optimizing performance, efficiency, and compatibility. Earlier 32-bit architectures with the x86 instruction set featured 32-bit general-purpose registers(e.g., EAX, EBX) and supported 32-bit integer arithmetic, logical operations, and memory addressing. This was later succeeded by the era of 64-bit architectures, such as x86-64, offering enhanced processing capabilities. \\ \midrule
\begin{tabular}[c]{@{}l@{}}computer \\ security\end{tabular} & practice of protecting computing devices, networks, and data from unauthorized access, modification, or destruction. Core aspects is authentication, which verifies users' identities, and access control, which restricts unauthorized access to resources. Techniques such as encryption and secure data transmission safeguard sensitive information from interception.  \\ \midrule
\begin{tabular}[c]{@{}l@{}}internet \\ protocols\end{tabular} & a set of rules and conventions enabling communication between devices in a network, consisting of Internet Layer in the Internet Protocol Suite, which facilitates the internetworking of devices across network boundaries. The Internet Layer uses protocols like IPv4 and IPv6 to handle packet transmission, addressing, and routing, ensuring packets reach their intended destinations with functionalities like packet fragmentation, path MTU discovery, and error reporting via internet control message protocol. \\ \midrule
\begin{tabular}[c]{@{}l@{}}computer \\ file systems\end{tabular} & organizes, stores, and retrieves data on storage devices, providing the structure necessary to access and manage files. File System enables seamless interaction with local, network, or remote file systems without requiring applications to know the underlying file system type. The abstraction is achieved through an interface contract between the operating system's kernel and concrete file systems, simplifying the addition of support for new file system types.  \\ \midrule
\begin{tabular}[c]{@{}l@{}}distributed \\ computing \\ architecture\end{tabular} & Distributed computing coordinates multiple independent systems using a middleware layer for communication, integration, and management between distributed applications and operating systems. It abstracts data exchange, resource allocation, and process synchronization, enabling client-server or peer-to-peer communication. Technologies like web servers, application servers, and ESBs support seamless interaction, while ODBC and JDBCensure data integration across heterogeneous systems. Widely used in enterprise, telecom, and aerospace, it bridges modern and legacy systems, enabling transaction management and message-oriented communication while enhancing scalability, reliability, and efficiency. \\ \midrule
\begin{tabular}[c]{@{}l@{}}web \\ technology\end{tabular} & enables the creation and delivery of content and services over the World Wide Web, with web servers playing a central role. Web servers support static files and dynamic content generation using server-side scripting languages like PHP or ASP, enabling integration with databases for real-time data retrieval. Widely used web servers like Apache, Nginx, and Microsoft IIS dominate the market, with each optimized for specific use cases, such as handling high traffic or supporting embedded systems in networked devices. \\ \midrule
\begin{tabular}[c]{@{}l@{}}programming \\ language topics\end{tabular} & enables developers to write instructions for computers, which are executed through a programming language implementation. Implementations translate high-level code into a form the computer can execute, primarily through compilers or interpreters. Modern implementations often blend both approaches, such as bytecode compilation followed by interpretation or just-in-time (JIT) compilation, as seen in languages like Java or Python.  \\ \bottomrule
\end{tabular}
\caption{Class Description for Wikics~\cite{mernyei2020wiki}.}
\label{tab:class_description_sportsfit}
\end{table}
