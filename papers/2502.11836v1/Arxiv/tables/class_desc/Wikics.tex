\begin{table}[h]
\scriptsize
\begin{tabular}{|p{2cm}|p{14cm}|}
\toprule
Class & Description \\ \midrule
\begin{tabular}[c]{@{}l@{}}computational \\ linguistics\end{tabular} & study of language using computational methods, focusing on the process of grouping together inflected forms of a word and identifying its dictionary or base form, known as the lemma. Computing linguistics simplify words by removing affixes, or ensure accurate base word selection based on grammatical and semantic context.   \\ \midrule
databases & structured system for storing, managing, and retrieving data, often accessed through a database server, such as those using DBMSs like MySQL or Oracle, to handle query processing, data analysis, and storage. Databases are designed for networked environments and often implement master-slave models for data redundancy and load balancing.  \\ \midrule
\begin{tabular}[c]{@{}l@{}}operating \\ systems\end{tabular} & Operating systems serve as the interface between hardware and software, encapsulate resources as objects, enabling inheritance and polymorphism to simplify system design and extend functionality. A file or device driver can be represented as an object, abstracting the implementation details while allowing access through defined methods. Each object can enforce access control and limited privileges based on user roles. Modern operating systems incorporate designs in components like kernels and user interfaces, leveraging these principles for improved modularity, maintainability, and resource protection.  \\ \midrule
\begin{tabular}[c]{@{}l@{}}computer \\ architecture\end{tabular} & Computer architecture defines the organization and design of a system's hardware, optimizing performance, efficiency, and compatibility. Earlier 32-bit architectures with the x86 instruction set featured 32-bit general-purpose registers(e.g., EAX, EBX) and supported 32-bit integer arithmetic, logical operations, and memory addressing. This was later succeeded by the era of 64-bit architectures, such as x86-64, offering enhanced processing capabilities. \\ \midrule
\begin{tabular}[c]{@{}l@{}}computer \\ security\end{tabular} & practice of protecting computing devices, networks, and data from unauthorized access, modification, or destruction. Core aspects is authentication, which verifies users' identities, and access control, which restricts unauthorized access to resources. Techniques such as encryption and secure data transmission safeguard sensitive information from interception.  \\ \midrule
\begin{tabular}[c]{@{}l@{}}internet \\ protocols\end{tabular} & a set of rules and conventions enabling communication between devices in a network, consisting of Internet Layer in the Internet Protocol Suite, which facilitates the internetworking of devices across network boundaries. The Internet Layer uses protocols like IPv4 and IPv6 to handle packet transmission, addressing, and routing, ensuring packets reach their intended destinations with functionalities like packet fragmentation, path MTU discovery, and error reporting via internet control message protocol. \\ \midrule
\begin{tabular}[c]{@{}l@{}}computer \\ file systems\end{tabular} & organizes, stores, and retrieves data on storage devices, providing the structure necessary to access and manage files. File System enables seamless interaction with local, network, or remote file systems without requiring applications to know the underlying file system type. The abstraction is achieved through an interface contract between the operating system's kernel and concrete file systems, simplifying the addition of support for new file system types.  \\ \midrule
\begin{tabular}[c]{@{}l@{}}distributed \\ computing \\ architecture\end{tabular} & Distributed computing coordinates multiple independent systems using a middleware layer for communication, integration, and management between distributed applications and operating systems. It abstracts data exchange, resource allocation, and process synchronization, enabling client-server or peer-to-peer communication. Technologies like web servers, application servers, and ESBs support seamless interaction, while ODBC and JDBCensure data integration across heterogeneous systems. Widely used in enterprise, telecom, and aerospace, it bridges modern and legacy systems, enabling transaction management and message-oriented communication while enhancing scalability, reliability, and efficiency. \\ \midrule
\begin{tabular}[c]{@{}l@{}}web \\ technology\end{tabular} & enables the creation and delivery of content and services over the World Wide Web, with web servers playing a central role. Web servers support static files and dynamic content generation using server-side scripting languages like PHP or ASP, enabling integration with databases for real-time data retrieval. Widely used web servers like Apache, Nginx, and Microsoft IIS dominate the market, with each optimized for specific use cases, such as handling high traffic or supporting embedded systems in networked devices. \\ \midrule
\begin{tabular}[c]{@{}l@{}}programming \\ language topics\end{tabular} & enables developers to write instructions for computers, which are executed through a programming language implementation. Implementations translate high-level code into a form the computer can execute, primarily through compilers or interpreters. Modern implementations often blend both approaches, such as bytecode compilation followed by interpretation or just-in-time (JIT) compilation, as seen in languages like Java or Python.  \\ \bottomrule
\end{tabular}
\caption{Class Description for Wikics~\cite{mernyei2020wiki}.}
\label{tab:class_description_sportsfit}
\end{table}