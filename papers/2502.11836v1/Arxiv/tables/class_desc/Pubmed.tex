\begin{table}[h]
\scriptsize
\begin{tabular}{|p{2cm}|p{14.5cm}|}
\toprule
Class & Description \\ \midrule
\begin{tabular}[c]{@{}l@{}}Diabetes Mellitus \\ Experimental\end{tabular} & Experimental studies on  diabetic rats reveal molecular mechanisms of diabetes-related complications. Research focuses on ion transport regulation, including isoform- and tissue-specific changes, increased vascular permeability, assessed via  albumin permeation, marks diabetic damage in eyes, kidneys, and nerves. Key interventions include aldose reductase inhibitors (sorbinil, tolrestat) and castration.  \\ \midrule
\begin{tabular}[c]{@{}l@{}}Diabetes Mellitus \\ Type 1\end{tabular} & Type 1 Diabetes Mellitus (T1DM) is an autoimmune disease characterized by the destruction of insulin-producing $\beta$-cells, leading to insulin deficiency and lifelong dependence on exogenous insulin therapy. Genetic predisposition, particularly related with HLA-DR, plays a crucial role in disease susceptibility and is related with the presence of insulin autoantibodies (IAAs). Key methodologies included intravenous glucose tolerance tests (IVGTT), serological tests for ICAs and IAAs, and HLA-DQ genotyping, which collectively helped identify predictive markers for the progression to Type 1 diabetes.  \\ \midrule
\begin{tabular}[c]{@{}l@{}}Diabetes Mellitus \\ Type 2\end{tabular} & Type 2 Diabetes Mellitus (T2DM) is a complex metabolic disorder characterized by insulin resistance, impaired insulin secretion, and progressive glucose intolerance. The progression of normal glucose tolerance to impaired is mainly induced by insulin resistance,  and can also be contributed by $\beta$-cell dysfunction, and ultimately lead to diabetes. Individuals exhibiting low insulin sensitivity (M-value) and diminished acute insulin response (AIR) face a significantly higher risk of developing T2DM, underscoring the need for targeted interventions at different disease stages. T2DM progression is associated with changes in adipokine levels, lipid metabolism, and advanced glycation end-products (AGEs). \\ \bottomrule
\end{tabular}
\caption{Class decription for Pubmed~\cite{sen2008collective}.}
\label{tab:class_description_pubmed}
\end{table}