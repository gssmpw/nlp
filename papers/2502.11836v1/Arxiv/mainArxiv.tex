\pdfoutput=1
\documentclass{article}


\usepackage{PRIMEarxiv}

\usepackage[utf8]{inputenc} % allow utf-8 input
\usepackage[T1]{fontenc}    % use 8-bit T1 fonts
\usepackage{hyperref}       % hyperlinks
\usepackage{url}            % simple URL typesetting
\usepackage{booktabs}       % professional-quality tables
\usepackage{amsfonts}       % blackboard math symbols
\usepackage{nicefrac}       % compact symbols for 1/2, etc.
\usepackage{microtype}      % microtypography
\usepackage{lipsum}
\usepackage{fancyhdr}       % header
\usepackage{graphicx}       % graphics
\graphicspath{{media/}}     % organize your images and other figures under media/ folder


\usepackage{amsmath}
\usepackage{amssymb}
\usepackage{mathtools}
\usepackage{amsthm}
\usepackage{bbm}

\usepackage{multirow}
\usepackage{booktabs}
\usepackage{graphicx}   
\usepackage{subcaption} 
\usepackage{xspace}
% if you use cleveref..
\usepackage[capitalize,noabbrev]{cleveref}
\usepackage{bm}
\usepackage{algorithm}
\usepackage{algorithmic}
\usepackage{wrapfig}
\usepackage[hang]{footmisc}

\newcommand{\proj}{LLM-BP\xspace}
\newcommand{\GG}{\mathcal{G}}
\newcommand{\VV}{\mathcal{V}}
\newcommand{\EE}{\mathcal{E}}
\newcommand{\TT}{\mathcal{T}}
\newcommand{\CC}{\mathcal{C}}
\newcommand{\XX}{\boldsymbol{X}}
\newcommand{\YY}{\boldsymbol{Y}}
\newcommand{\xx}{\boldsymbol{x}}
\newcommand{\HH}{\boldsymbol{H}}
\newcommand{\hh}{\boldsymbol{h}}


\theoremstyle{plain}
\newtheorem{theorem}{Theorem}[section]
\newtheorem{proposition}[theorem]{Proposition}
\newtheorem{lemma}[theorem]{Lemma}
\newtheorem{corollary}[theorem]{Corollary}
\theoremstyle{definition}
\newtheorem{definition}[theorem]{Definition}
\newtheorem{assumption}[theorem]{Assumption}
\theoremstyle{remark}
\newtheorem{remark}[theorem]{Remark}
%\newcommand{\theHalgorithm}{\arabic{algorithm}}
%Header
\pagestyle{fancy}
\thispagestyle{empty}
\rhead{ \textit{ }} 

% Update your Headers here
\fancyhead[LO]{Model Generalization on Text Attribute Graphs: Principles with Large Language Models}
% \fancyhead[RE]{Firstauthor and Secondauthor} % Firstauthor et al. if more than 2 - must use \documentclass[twoside]{article}



  
%% Title
\title{Model Generalization on Text Attribute Graphs: \\
Principles with Large Language Models
%%%% Cite as
%%%% Update your official citation here when published 
%\thanks{\textit{\underline{Citation}}: 
%\textbf{Authors. Title. Pages.... DOI:000000/11111.}} 
}

\author{
  Haoyu Wang\\
  Department of ECE \\
Georgia Tech \\
  \texttt{haoyu.wang@gatech.edu} \\
  \And
  Shikun Liu\\
  Department of ECE \\
Georgia Tech \\
  \texttt{shikun.liu@gatech.edu} \\
  \And
  Rongzhe Wei\\
  Department of ECE \\
Georgia Tech \\
  \texttt{rongzhe.wei@gatech.edu} \\
  \And
  Pan Li\\
  Department of ECE \\
Georgia Tech \\
  \texttt{panli@gatech.edu} \\
  %% examples of more authors
  %% \AND
  %% Coauthor \\
  %% Affiliation \\
  %% Address \\
  %% \texttt{email} \\
  %% \And
  %% Coauthor \\
  %% Affiliation \\
  %% Address \\
  %% \texttt{email} \\
  %% \And
  %% Coauthor \\
  %% Affiliation \\
  %% Address \\
  %% \texttt{email} \\
}


\begin{document}
\maketitle


\begin{abstract}
Large language models (LLMs) have recently been introduced to graph learning, aiming to extend their zero-shot generalization success to tasks where labeled graph data is scarce. Among these applications, inference over text-attributed graphs (TAGs) presents unique challenges: existing methods struggle with LLMs' limited context length for processing large node neighborhoods and the misalignment between node embeddings and the LLM token space. To address these issues, we establish two key principles for ensuring generalization and derive the framework \proj accordingly: (1) \textbf{Unifying the attribute space with task-adaptive embeddings}, where we leverage LLM-based encoders and task-aware prompting to enhance generalization of the text attribute embeddings; (2) \textbf{Developing a generalizable graph information aggregation mechanism}, for which we adopt belief propagation with LLM-estimated parameters that adapt across graphs. Evaluations on 11 real-world TAG benchmarks demonstrate that \proj significantly outperforms existing approaches, achieving 8.10\% improvement with task-conditional embeddings and an additional 1.71\% gain from adaptive aggregation. Task-adaptive embeddings and codes are publicly available at\footnote{\url{https://github.com/Graph-COM/LLM_BP}, \\ \url{https://huggingface.co/datasets/Graph-COM/Text-Attributed-Graphs}}.
\end{abstract}


\section{Introduction} \label{sec:intro}

Inspired by the remarkable generalization capabilities of foundation models for text and image data~\cite{achiam2023gpt, liu2021swin, radford2021learning}, researchers have recently explored extending these successes to graph data~\cite{liu2023towards, mao2024graph, zhao2023gimlet, fan2024graph, he2023harnessing}, aiming to develop models that generalize to new or unseen graphs and thereby reduce reliance on costly human annotation~\cite{li2024glbench, chen2024text, feng2024taglas, li2024teg}. Among various types of graph data, \emph{text-attributed graphs} (TAGs) have found a wide range of applications. These graphs combine both topological relationships and textual attributes associated with each node, which naturally arises in recommendation systems (where user and item nodes may have textual descriptions or reviews)~\cite{bobadilla2013recommender}, academic graphs (where publications include extensive textual metadata)~\cite{mccallum2000automating, giles1998citeseer}, and financial networks (where transactions and accounts come with textual records)~\cite{kumar2016edge, kumar2018rev2}. Given the labeling challenges posed by cold-start problems in recommendation systems or fraud detection in financial networks, methods that can operate with limited labeled data are crucial. In particular, robust zero-shot node labeling across unseen TAGs has become an area of great interest.

Numerous studies have been dedicated to inference tasks on TAGs. Early efforts have primarily focused on adapting pre-trained language model (LM) encoders~\cite{li2024zerog, fang2024uniglm}, sometimes in combination with graph neural networks (GNNs)~\cite{hou2022graphmae, velivckovic2018deep}, to incorporate structural information. However, these approaches often struggle to achieve strong generalization performance, largely due to the limited capacity of the underlying models. With the advent of large language models (LLMs)~\cite{kaplan2020scaling,huang2022towards}, researchers have proposed two main strategies for integrating LLMs into TAG inference: 1) \textbf{Direct Node-Text Input.} Here, raw node texts are directly fed into LLMs. This method demonstrates reasonably good zero-shot performance on TAGs when text attributes are highly informative for node labels~\cite{chen2024exploring, li2024similarity}. However, when the textual attributes are insufficient, it becomes necessary to aggregate information from a larger neighborhood in the graph, while this is constrained by the limited context length LLMs can digest and reason over. 2) \textbf{Embedding-Based Integration.} In this approach, node texts and their neighboring structural information are first encoded into compressed embeddings, which are then processed by LLMs~\cite{chen2024llaga,tang2024graphgpt,luo2024enhance,wang2024llms,zhang2024graphtranslator}. Because LLMs are not inherently trained on arbitrary embedding spaces, aligning these embeddings with the LLM's token space is essential - an idea partly inspired by how vision-language models align multimodal data~\cite{radford2021learning, zhai2023sigmoid}. However, unlike the vision-language domain, where large-scale text–image pairs~\cite{schuhmann2022laion} are abundant, the graph domain typically lacks comparable datasets. This scarcity reduces the model’s generalization in practice.

\begin{figure*}[t]
    \centering
    \includegraphics[width=0.99\textwidth]{Arxiv/figures/pipeline/pipeline.pdf}
    \vspace{-0.2cm}
    \caption{The two generalization principles and the framework of LLM-BP.}
    \vspace{-0.4cm}
    \label{fig:pipe}
\end{figure*}



In contrast to prior heuristic approaches, this work aims to design a method from first principles for robust zero-shot generalization on TAGs. Because TAGs are inherently multimodal, the proposed method must simultaneously address potential distribution shifts in both textual attributes and graph structure. Specifically, text attributes can vary widely, for example from scientific papers~\cite{mccallum2000automating} to e-commerce product reviews~\cite{ni2019justifying}. Edge connection patterns can range from homophilic graphs such as citation networks, where papers on similar themes are linked~\cite{giles1998citeseer}, to heterophilic graphs such as webpages, which connect nodes with distinct topics~\cite{mernyei2020wiki}. Moreover, the labeling task itself can shift which requires a task-adaptive approach to process both textual features and network structure. Consequently, the core insight behind our model design is grounded in the following two key principles.

\textbf{Principle I: Unifying the text space and obtaining task-adaptive embeddings.} LLMs  offer powerful text-understanding capabilities that naturally unify the textual feature space. However, to handle the large-scale graph aggregation discussed later, we require these capabilities to extend beyond raw text to an embedding space. Hence, we propose to adopt LLM-based encoder models such as LLM2Vec~\cite{behnamghader2024llm2vec, li2024making} for text embedding. Although this approach might appear to be a naive extension of smaller LM-based embedding methods (e.g., those relying on SBERT~\cite{reimers2019sentence} or RoBERTa~\cite{liu2019roberta}), we argue that leveraging the decoder-induced encoder structure of LLMs is essential for achieving task-adaptive embeddings. In particular, we introduce a novel prompting strategy that encodes text attributes conditioned on inference-task descriptions, enabling significantly improved zero-shot inference - an ability not readily achieved by smaller LM-based embeddings.

\textbf{Principle II: Developing a generalizable adaptive graph information aggregation mechanism.} Graph structure determines the node neighboring relationships and thus the information aggregation from which nodes may benefit the inference. Inspired by the belief propagation (BP) algorithm~\cite{murphy2013loopy} that gives the optimal statistical inference over pairwise correlated random variables, we propose to regard the graph as a Markov Random Field (MRF), each node as a random variable, and mimic BP to aggregate information for node label inference. Because BP is rooted in basic mathematical principles, this approach is widely generalizable. Algorithmic adaptivity across different TAGs hinges on estimating the coupling coefficients in the graphs, which can be done by having LLMs analyze the attributes of sampled pairs of connected nodes. Moreover, this BP-inspired approach naturally adapts to varying levels of text attribute quality: nodes with higher-quality text attributes present greater influence on their neighbors, and vice versa.


By applying the two principles outlined above, we propose our new strategy, \proj, for zero-shot inference over TAGs. \proj does not require any training or fine-tuning. We evaluate \proj on 11 real-world TAG datasets from various domains, including citation networks~\cite{mccallum2000automating, giles1998citeseer, sen2008collective}, e-commerce~\cite{ni2019justifying}, knowledge graphs~\cite{mernyei2020wiki}, and webpage networks~\cite{craven1998learning}, covering both homophilic and heterophilic graph structures. 

Experimental results demonstrate the effectiveness of \proj. Notably, our task-conditional embeddings (Principle I) improve performance by $8.10\%$ on average compared to the best LM-based encoders. In addition, our BP-inspired aggregation mechanism (Principle II) provides an extra $1.71\%$ performance gain with our embeddings, demonstrating strong generalization across both homophilic and heterophilic TAGs.  Our experiments also reveal that current methods aligning graph-aggregated embeddings to LLM token spaces significantly underperform approaches that simply use smaller LM encoders without even incorporating graph structures. This outcome indicates that the primary source of generalization in these methods is the smaller LM’s text embeddings, rather than LLM-based reasoning on embeddings. It reinforces our earlier argument that limited training data hinders effective alignment in this context, urging caution for future work considering this strategy. %, casting doubt on the viability of such strategies within the graph learning domain.

%structure utlization, leveraging LLM agents, exhibits strong generalization capabilities across both homophilic and heterophilic TAGs, providing an additional $3.15\%$ performance improvement on average when applied to our encoding method.

%Unfortunately, our experiments also show that the Embedding-Based Integration methods that align graph-aggregated embeddings with the LLM token space significantly underperform those that only adopt smaller LM encoders even without incorporating graph structures. This suggests that the primary, if not the sole, source of generalization in these approaches stems from the smaller LM-encoded text embeddings rather than the reasoning capabilities of LLMs over these embeddings, which matches our previous argument on the lack of training data for achieving such alignment and poses a concern to this strategy in graph learning field. 

%Additionally, an intriguing phenomenon emerges from the experiments: for approaches that employ a projector to align latent graph embeddings with the LLM token space and utilize LLMs for reasoning~\cite{chen2024llaga, wang2024llms, tang2024graphgpt}, we find that merely using pre-trained LM encoders even without incorporating graph structures, significantly outperform these methods. This suggests that the primary, if not the sole, source of generalization in these approaches stems from the generalizable node representations rather than the reasoning capabilities of LLMs over latent graph embeddings.
%\pany{This work strongly rely on empirical performance, I suggest not to just say achieving SOTA or outperform baselines. Instead, argue how much gain can be brought by each technique: Conditional node text encoding? BP-based graph structure usage? }.  Meanwhile, our unified structure aggregation, . \hy{critic LLaGA or not here?} \pany{The current last paragraph is too high-level. Yes, indeed, we should argue against the approach that adopts alignment.}

%\pany{The last paragraph, still needs you to further work on it.}
%Interestingly, our study uncovers that relying solely on LLMs for reasoning doesn't necessarily enhance generalization. Instead, using text embeddings alone—such as those from Sentence-BERT~\cite{reimers2019sentence}—already delivers exceptional performance. These findings underscore the importance of understanding the principles that support the generalization of TAG models.




\section{Related Works}
%\pan{based on what I read: these works do not use llm, and also need model training?}
%\pan{Efforts to improve the model's zero-shot capability can be divided as follows}:
Here, we briefly review existing methods by examining how they enable model generalization across TAGs. 

\textbf{Tuning Smaller LM Encoders.} 
These methods typically rely on a source-domain graph for training. Notable works include ZeroG~\cite{li2024zerog} that tunes SBert~\cite{reimers2019sentence} on source datasets to align class-description embeddings with node text, thereby enhancing zero-shot performance on target datasets. Another approach, UniGLM~\cite{fang2024uniglm}, fine-tunes BERT~\cite{kenton2019bert} using contrastive learning on source datasets to yield a more generalizable encoder. GNNs trained with UniGLM embeddings in a supervised manner outperform models that directly adopt LM embeddings.

%These work node representation has recently attracted much research interest. 


\textbf{Training GNNs for Generalization.} These methods focus on leveraging graph structure in a generalizable manner. Among them, graph self-supervised learning~\cite{liu2022graph} is particularly common for producing representations without labeled data, often employing contrastive learning or masked modeling~\cite{velivckovic2018deep, hou2022graphmae, zhao2024all}. GraphMOE is a more recent technique inspired by the success of mixture-of-experts~\cite{shazeer2017outrageously}, pre-training parallel graph experts targeting different structures or domains~\cite{hou2024graphalign, liu2024one, xia2024anygraph, qin2023disentangled}. 
Others also consider LM-GNN co-training including \cite{he2024unigraph,zhu2024graphclip} that also follow a constrastive learning idea. %where \cite{he2024unigraph} proposes a novel structure-aware node text reconstruction scheme based on contrastive learning, and~\cite{zhu2024graphclip} introduces a contrastive graph-summary pretraining method combined with invariant learning. 
Note that, however, all these methods still require training.  %Recently a `Na\"{i}ve' method that simply aggregate the averaged neighbor embeddings exhibits good zero-shot performance on homophilic graphs~\cite{li2024glbench, chen2024text}. 
%\pany{this is too rough? can you elaborate on the technical aspects how they do so} 

%Works tuning small LM encoders or GNNs demonstrate promising zero-shot generalization capabilities. But their effectiveness can be heavily constrained by the limited complexity of the models. And insufficient training may even lead to negative transfer, particularly in cases of substantial node text domain shifts or graph structural transitions (e.g., from homophilic to heterophilic). 
In contrast to the above effort that adopts smaller LM encoders, works that involve the use of LLMs are reviewed in the following and may achieve better generalization. More related works including LLM-based data augmentations for GNN training for generalization and LLMs for other graph reasoning tasks can be found in Appendix.~\ref{sec:app_more_related_works}.

%\pany{why call llm as reasoners? Why not just finetuning (small) LM encoders; Train GNNs; }

%\pany{The name of each category should be properly given and match those later used in the exp. Also, it is unclear that the first two categories do not use llm and the rest use llms. This point should be clarified. moreover, how do you define/distinguish lm v.s. llm?}

\textbf{LLMs with Node-Text Input.} LLMs being directly fed with raw node texts demonstrates strong zero-shot ability on TAGs~\cite{chen2024exploring, huang2024can, li2024similarity}. However, they suffer from the limitation of not being able to incorporate graph structural information. % due to the limited context window size. That said, later we will show that, more capable LLMs, such as GPT-4o~\cite{hurst2024gpt}, serve as important baselines for zero-shot tasks on TAGs.

\textbf{LLMs with Graph-Embedding Input.} With smaller LM-encoded node embeddings, various strategies integrate graph structure by aggregating these embeddings, such as neighborhood-tree traversal or concatenating the averaged embeddings from different hops~\cite{chen2024llaga, tang2024graphgpt, luo2024enhance}, or via pre-trained GNNs~\cite{zhang2024graphtranslator, wang2024llms}. As mentioned earlier, these methods rely on aligning embeddings with the LLMs' token space. For instance, LLaGA~\cite{chen2024llaga} trains a simple MLP on citation networks and~\cite{wang2024llms} employs a linear projector on the ogbn-Arxiv~\cite{hu2020open} dataset, both using the next-token prediction loss, while \cite{tang2024graphgpt} adopts self-supervised structure-aware graph matching as the training objective. However, due to limited TAG-domain data, the space alignment in these methods often remains undertrained, leading to degraded performance. 

%LLMs are fed with latent graph representations instead of raw texts to integrate graph structures. These methods first adopt pre-trained LM encoders~\cite{reimers2019sentence} for node embeddings, and then encode graph structure either with handcrafted summarization (e.g., traversing the neighborhood tree or concatenating multi-hop embeddings)~\cite{chen2024llaga, tang2024graphgpt, luo2024enhance}, or with pre-trained GNNs~\cite{zhang2024graphtranslator, wang2024llms}. \pan{the sentence order should be shifted} 
%\hy{how to implement graph adapter}

%\pan{Given the intro has mentioned this, here, we just need to be brief. Perhaps, connect to the experiments: A projector is learned while just as the intro said, it suffers from xxx.} 
%A projector is always essential to align the graph representation space with the token space of LLMs, where LLaGA~\cite{chen2024llaga} trains a simple MLP on several citation networks, while ~\cite{wang2024llms} employs a linear projector trained on the ogbn-Arxiv~\cite{hu2020open} graph. Both methods formulate tasks on TAGs as next-token prediction tasks and use instruction fine-tuning to train the projector. Similarly, ~\cite{tang2024graphgpt} aligns a lightweight projector through self-supervised structure-aware graph matching, followed by instruction tuning.
%As mentioned in the introduction, existing methods train projectors on limited data, which significantly hinders their generalization ability. In addition, unlike CV~\cite{schuhmann2022laion}, currently there is even no large-scale and diverse graph data available for training such projectors with strong generalization ability, which further limits the effectiveness of these methods.

\textbf{Multi-Task Graph Foundation Models.} More ambitious studies aim to generalize across various graph-related tasks within a single framework. Notable approaches include graph prompting~\cite{liu2023graphprompt}, which introduces ``prompting nodes'' to transform diverse graph tasks into a unified format. These frameworks then train GNNs to address the tasks~\cite{li2024zerog, liu2023one, liu2024one} or further integrate LLMs~\cite{yang2024gl, kong2024gofa}. Although these works are impressive, they still fail to achieve zero-shot performance comparable to those methods that focus on specific graph data domains. %show promising generalization results and are selected as important baselines.\hy{don't know how to position}
% and training a unified model to solve them. 
%represent an interesting direction aimed at 
%\pany{I suggest to put some foundation model works in the main text, if space is allowed.}


%\shikun{not sure if this is a concern, but the related work section currently sounds like a longer and detailed version of intro paragraph 2}


%Supervised Fine-tuning LMs on a specific task achieves state-of-the-art performance through supervised fine-tuning. This demonstrates that the quality of node attributes can significantly influence the model's performance.
%\textbf{Task-Specific LM Fine-Tuning} via supervised learning demonstrates remarkable performance improvements~\cite{chien2021node,he2023harnessing,duan2023simteg,zhu2024efficient,zhu2024efficient,zhu2024efficient,zhao2022learning}.



%\pan{introduce the principles for generalization: Principle 1, Principle 1+3 (method), Principle 1+3 (result); Principle 2, principle 2+3 (method), principle 1+2+3 result.  }

%\pan{Discuss and limitation.}

\section{Generalization Principles for \proj}
%Here we first provide the notation, followed by the three key principles for model generalization on TAG data. 
%\pany{why not put notation and problem formulation separately. Also, when only focusing on node classifcation, you should give the explanation again why only focusing on this task, more detailed than intro.}

\subsection{Notations and Problem Formulation} 

Let $(\GG, X, Y)$ represent a TAG of interest, where $\GG(\VV, \EE)$ denotes the graph structure, $\VV$ is the node set of size $n$, and $\EE$ is the edge set. The node textual attributes are represented as $X = \{X_1, ..., X_n\}$, and each node belongs to one of $c$ classes, with labels given by $Y=\{y_1,y_2,...,y_n\} \in [c]^n$. 

The objective is to infer the labels of nodes in TAGs based on the node attributes and graph structure. This study primarily focuses on the \textbf{zero-shot} setting, where no labeled data are assumed to be available in advance. Additionally, a \textbf{few-shot} setting is considered, where $k$ labeled nodes are known for each class. Due to space limitations, results for the few-shot setting are provided in Appendix~\ref{sec:app_few_shot}.  % are randomly sampled from the training dataset.
% by first mapping the natural language texts into latent embeddings, and further designing functions to incorporate the graph structure information into the latent node representations. 

\subsection{Motivation and the Overall Framework}
 LLMs are commonly used as decoders for next-token prediction. While LLMs excel at processing natural language inputs, they are not inherently compatible with graph data. Recently, some studies have explored methods to integrate graph data into LLMs, primarily for reasoning tasks~\cite{perozzi2024let, zhang2024can, tang2024grapharena}. 

In the context of TAGs, accurate node label inference relies on effectively combining the attributes of multiple nodes, especially when a node's individual attributes are insufficient to determine its label. However, as noted earlier, LLMs are constrained by limited context windows, making it challenging to process all attributes from the potentially large set of connected nodes. Traditional approaches to compressing graph structural information involve creating embeddings, such as using GNNs to aggregate information from the target node's neighbors. While effective, these embedding methods do not seamlessly integrate with LLM inputs and often require non-trivial training effort to align the LLMs' token space with the node embedding space~\cite{chen2024llaga, wang2024llms, tang2024graphgpt}. 

Our approach, \proj, does not confine LLMs to their traditional usage. We first leverage their capabilities to generate task-adaptive node embeddings. Then, instead of requiring LLMs to directly process these embeddings, LLMs are further employed to analyze graph data and provide generalizable guidance in aggregating these embeddings. These two steps are to match the two generalization principles proposed in Sec.~\ref{sec:intro}. Classification is ultimately performed by computing the cosine similarity between the final node embeddings $\mathbf{h}^X=[h_1^X,...,h_n^X]$ and candidate class embeddings $\mathbf{q}^C=[q_1^C,...,q_c^C]$. In the zero-shot setting, class embeddings are generated as follows: we randomly sample $l \ll n$ nodes and employ LLMs to infer their labels. The embeddings of sampled nodes form distinct clusters based on LLMs' prediction. We compute the average embedding of the embeddings closest to the cluster center to obtain the class embedding. 
In the few-shot setting, class embeddings are obtained by averaging the embeddings of labeled nodes within each class. See Appendix.~\ref{sec:app_llm_bp_implementation_details} for details.




% expect the LLMs to take all info to make the inference 


% obtained via e.g. graph neural n 


% their connects 
% %And node neighboring information in the graph structure determines  represents the correlation between these nodes. 

% Given LLMs suffer from either limited context window to digest multiple nodes and their connects 
% Our framework follows the d



% The key idea behind zero-shot inference is to compute the cosine similarity between the summarized node representations and the class label embeddings, with the latter being the embeddings of descriptions for the classes. For example, the description of \emph{``student''} class in the Cornell~\cite{craven1998learning} dataset is: \textit{``encompasses individuals actively enrolled in educational programs, ranging from undergraduate to graduate levels, across diverse disciplines. These individuals engage in academic activities, such as attending lectures, completing assignments, and participating in research or extracurricular projects, contributing significantly to the institution's learning environment.''} The class descriptions are represented by $\CC= \{C_1, ..., C_{c}\}$. A detailed description of each class for all the datasets used in the study is provided in Appendix.~\ref{sec:app_class_description}. 

% \textbf{Few-shot} is also considered in this study. We adopt the $c$-way-$k$-shot approach, where $k$ labeled nodes per class are randomly sampled from the training dataset. The average embedding of these nodes within each class is then used as the class label embedding, replacing the embeddings of class descriptions $\CC= \{C_1, ..., C_{c}\}$ used in the zero-shot setting.

% In the following sections, we present two key principles identified for designing a method for TAG generalization.

\subsection{Principle I: Task-Adaptive Node Embeddings $\mathbf{h}^X$}
%\pan{the contribution from LM encoder to LLM encoder is limited. } 

Creating generalizable text embeddings is no longer a significant challenge. Even smaller LM encoders, such as SBert~\cite{reimers2019sentence}, are capable of achieving this. Indeed, most existing works utilize these encoders to generate initial node embeddings for TAGs~\cite{chen2024text,chen2024llaga, tang2024graphgpt, wang2024llms}. However, for these embeddings to be directly usable for label prediction without the need for additional transformation models, it is crucial to incorporate task-specific information. In other words, the embeddings must be tailored to the specific task, resulting in what we term task-adaptive embeddings.

Achieving task adaptivity, however, presents a notable challenge. Smaller LM encoders lack the expressive power necessary to encode nuanced task-specific information. This limitation motivates our adoption of LLM-induced encoders, driven by the emergent capabilities of LLMs in contextual learning~\cite{sahoo2024systematic, chen2023unleashing}.

There have been recent advancements in extending LLMs to generate text embeddings~\cite{behnamghader2024llm2vec, muennighoff2022mteb}. In our approach, we utilize a form of LLM2Vec~\cite{behnamghader2024llm2vec}, which transforms LLM decoders into encoders via retaining the unidirectional attention mechanism. Following the methodology 
 in~\cite{li2024making}, we extract the output embedding of $\langle\text{response}\rangle$ - the token positioned immediately after the inputs - as the text embedding for the input node attributes.

To embed task-specific information into node embeddings, we propose a prompting strategy structured with the following template:
\begin{align}
\label{eq:task_adaptive_llm2vec}
\resizebox{0.43 \textwidth}{!}{$\langle\text{Instruct}\rangle \{\text{task\_description}\} \{\text{class\_info}\} \langle\text{query}\rangle {X_i} \langle\text{response}\rangle$}.
\end{align}

Here, $\langle \cdot \rangle$ encloses specific tokens. The task details are described in $\{\text{task\_description}\}$, and $\{\text{class\_info}\}$ contains the basic information of each class. An example is given in Fig.~\ref{fig:pipe}.
The class information serves as a crucial contextual enhancement, enabling LLMs to generate embeddings in a conditioned manner. %, based on their learned knowledge and understanding from the pre-training process. 
For more detailed on the class-conditional prompt for each dataset used in this study, refer to Appendix.~\ref{sec:app_llm_bp_implementation_details} and ~\ref{sec:app_prompt_llm2vec_task_description}.


% There has been much effort recently on extending LLMs to conducting effective text embeddings~\cite{}

% To generate embeddings, we following~\cite{li2024making} 
% convert LLM decoders into encoders by retaining their unidirectional attention mechanism and using the output embedding of the token $\langle\text{response}\rangle$, positioned immediately after the input tokens, as the vector representation of text.

% LLMs hold extensive power to understand text information associated with the TAG. However, the challenge is to incorporate the task information when to 

% The consistency of node attributes is the dominant factor in generalization on TAGs.
% As evidenced by the scaling laws~\cite{kaplan2020scaling} and the success of LLM2Vec~\cite{behnamghader2024llm2vec}, embeddings derived from larger LLM decoders pre-trained on large-scale, diverse datasets (e.g., Mistral-7B~\cite{jiang2023mistral}) exhibit inherently superior generalization~\cite{muennighoff2022mteb} compared to those from small LM encoders (e.g., SBert~\cite{reimers2019sentence}) with limited model complexity. Following~\cite{li2024making}, we
% %\pan{this is not that ``we propose''. We simply borrow..}
% convert LLM decoders into encoders by retaining their unidirectional attention mechanism and using the output embedding of the token $\langle\text{response}\rangle$, positioned immediately after the input tokens, as the vector representation of text.

% Furthermore, as illustrated by the toy example in Fig.~\ref{fig:pipe}, node embeddings should be task-adaptive, ensuring that their distance are conditioned on the classes. Motivated by LLMs' emergent capability in contextual learning~\cite{sahoo2024systematic, chen2023unleashing}, we introduce a novel prompting strategy for encoders converted from LLM decoders, class-conditional encoding, with the following template: 
% %\panq{why duplicate (1) and (2), just need one. }
% %For any text to encode, the template is as follows:
% %\begin{align}
% %\label{eq:llm2vec}
% %\resizebox{0.45 \textwidth}{!}{$\langle\text{Instruct}\rangle \{\text{task\_description}\}  \langle\text{query}\rangle \{q_i\} \langle\text{response}\rangle$},
% %\end{align}where $\langle \rangle$ contains the specific tokens. In $\{\text{task\_description}\}$, an example of encoding book descriptions on e-commerce networks could be: 
% %\pan{note that you have already considered to introduce conditional encoding, which means principle I and III should be merged}
% %\textit{``Encode the description or title of a book."}
% %The prompt is then followed by $q_i$ representing the raw text to be encoded. For details of the LLM encoder pre-training, instruction fine-tuning, see Appendix.~\ref{sec:app_implementation_details_llm2vec}.
% %Having established the unified text space with LLM2Vec encoder, we now introduce the technique that adapt this embedding space for specific tasks. In particular, we aim to generate class-conditioned embeddings from generic embedding space through prompting the encoder with downstream task label description.
% %\textbf{Principle I with Adaption to Node Classification:}
% \begin{align}
% \label{eq:task_adaptive_llm2vec}
% \resizebox{0.43 \textwidth}{!}{$\langle\text{Instruct}\rangle \{\text{task\_description}\} \{\text{class\_info}\} \langle\text{query}\rangle {X_i} \langle\text{response}\rangle$}.
% \end{align}Here, $\langle \rangle$ encloses specific tokens. The task details are described in $\{\text{task\_description}\}$, and $\{\text{class\_info}\}$ contains the basic information of each class. For instance, the prompt for encoding book titles or descriptions in the History~\cite{ni2019justifying} e-commerce network can be structured as:

% \textit{``Given the description or title of the book, classify it into one of the following 12 classes: \\
% World,  \\
% Americans, \\
% Military, \\
% ...,\\
% Arctic \& Antarctica."}

% The class information serves as a crucial contextual enhancement, enabling LLMs to generate embeddings in a conditioned manner, based on their learned knowledge and understanding from the pre-training process. For encoding class descriptions for zero-shot inference, the prompt for $\{\text{task\_description}\}$ is simply:
% ``\textit{Encode the text:}''. For detailed class-conditional prompting templates for each dataset used in this study, refer to Appendix.~\ref{sec:app_class_conditional_encoding_prompt}.


%For the text description for each class, one can still adopt the template in Eq.~\ref{eq:llm2vec}. 


%\shikun{maybe first add add an example to showcase the usage. Then after that, use some sentences describing the results. For instance, suppose we have the downstream tasks with ground truth label being the categories of products xxx, we prompt the LLM encoder with . As a result, we find the resulting embeddings align closer to the task context and become more distinguishable as shown in fig~\ref{fig:tsne}.}


%Each node $v_i$ can then be classified according to the maximum cosine similarity with the node embeddings: 
%\begin{align}
%\label{eq:cosine_classification}
%    \hat{y_i} = \arg \max_{i} cos(\XX^{\TT}_i, \XX^{\CC}_i).
%\end{align}
%\shikun{why here we want to mention how to classify the node embeddings}

%\rw{In this section, I outlined the general pairwise MRF formulation and deferred the discussion on leveraging LLMs to obtain node likelihoods to the next section, where the algorithms will be introduced. Also, I didn't cite Jia et al here, you can consider cite it where we introduce the assumption for edge coupling potentials.}

\subsection{Principle II: Generalizable Graph  Aggregation} \label{sec:principle2}
Graph structures can provide essential correlated information for node label inference by characterizing the relationships between node attributes and labels~\cite{zhu2003semi,kipf2016semi,velivckovic2017graph,hamilton2017inductive,zhu2020beyond,wei2022understanding}.
% \pan{add more citations, especially foundational works in this area}
 Specifically, we may consider each node's label and attributes as random variables, and each edge as a coupling between them for connected node pairs. The fundamental BP algorithm enables principled statistical inference over this set of correlated random variables~\cite{murphy2013loopy}. Since BP is inherently agnostic to the application domain of the TAG, emulating BP offers a mechanism to aggregate correlation information encoded in the graph structure across domains.

\textbf{Markov Random Field Modeling} We consider the joint probability distribution $\mathbb{P}_\mathcal{G}(Y, X)$ over the graph where $Y$ and $X$ denotes the random variables of node labels and attributes, respectively. %Given a graph $\mathcal{G}$ with node set $\mathcal{V}$ and edge set $\mathcal{E}$, 
In $\mathbb{P}_\mathcal{G}(Y, X)$, the distribution over the node labels given the graph structure is denoted as
\begin{align}
\mathbb{P}_\mathcal{G}(Y)
= \frac{1}{Z_{\mathbf{Y}}} \prod_{i \in \mathcal{V}} \phi_i(y_i) \prod_{(i,j) \in \mathcal{E}} \psi_{ij}(y_i, y_j).
\end{align}
Here $\phi_i(y_i)$ denotes the unary potential for node $i$, $\psi_{ij}(y_i, y_j)$ is the edge potential capturing the correlation between labels $y_i$ and $y_j$ of adjacent nodes, and $Z_{Y}$ is the normalization constant.  For node attributes, MRF modeling assumes that each node’s attributes are conditionally independent of others given the node labels, which can be characterized by the distribution:  
\begin{align}
\mathbb{P}_\mathcal{G}(X \mid Y) 
= \prod_{i \in \mathcal{V}} \mathbb{P}_\mathcal{G}(X_i \mid y_i) 
= \prod_{i \in \mathcal{V}} \varphi_{y_i}(X_i)
\end{align}
where $\varphi_{y_i}(X_i)$ captures the likelihood of having node $i$’s attributes $X_i$ given its label $y_i$. % to the probability of observing the corresponding feature $x_i$.

The proposed modeling approach is highly adaptive, as it can capture the varying graph connectivity patterns across different TAGs through interpretable edge potentials. For instance, $\psi_{ij}(y_i, y_j)$ represents the unnormalized likelihood that nodes with labels $y_i$ and $y_j$ are connected. This formulation naturally incorporates the modeling of graph homophily and heterophily: $\psi_{ij}(y_i, y_i) > \psi_{ij}(y_i, y_j)$ indicates homophily, while $\psi_{ij}(y_i, y_i) < \psi_{ij}(y_i, y_j)$ reflects heterophily. Furthermore, $\varphi_{y_i}(X_i)$ enables the model to account for variations in the quality of text attributes (w.r.t. their indicative power for the labels) across different TAGs, further enhancing its adaptivity. For node classification, we can infer $\mathbb{P}_\mathcal{G}(Y \mid X) \propto \prod_{i \in \mathcal{V}} \varphi_{X_i}(y_i) 
\prod_{(i,j) \in \mathcal{E}} \psi_{ij}(y_i, y_j)$ where $\varphi_{X_i}(y_i) = \varphi_{y_i}(X_i)\phi_i(y_i)$.

\begin{algorithm}[t]
\caption{LLM-BP}
\label{alg:llm_bp}
\begin{algorithmic}[1]
\INPUT TAG $(\mathcal{G}, \boldsymbol{X})$
\OUTPUT Class label prediction $\{\hat{y}_i\}_{i\in[n]}$
\STATE $\hh^{X}$ $\leftarrow$ Task-adaptive encoding following Eq.~\eqref{eq:task_adaptive_llm2vec}
\IF{zero-shot}
\STATE Sample $l \ll n$ nodes, infer labels with LLMs,
\STATE Nodes clusters based on LLM prediction,
\STATE $\mathbf{q}^{C}$ $\gets$ Average embedding of samples near center,
\ELSIF{few-shot}
\STATE $\mathbf{q}^{C}$ $\gets$ Average embedding of $k$ samples per class,
\ENDIF
\STATE Estimate $\psi_{ij}(y_i, y_j)$ by employing the LLM to analyze the graph data (e.g., using Eq.~\eqref{eq:bp} based on the estimated homophily level $r$.)
\STATE Initialize $p^{(0)}(y_i)$ $\gets$ Eq.~\eqref{eq:node_potential} and $m^{(0)}_{i \to j} (y_j) = 1$
\STATE Run LLM-BP (Eq.~\eqref{eq:message_passing}) for $L$ iterations or its approximation (Eq.~\eqref{eq:bp_appr}) for single iteration
% \IF{use \textbf{BP}}
% \STATE $\psi_{ij}(y_i, y_j)$ $\gets$ Eq.~\eqref{eq:bp}
% \ELSIF{use \textbf{BP (appr.)}}
% \STATE $\psi_{ij}(y_i, y_j)$  $\gets$ Eq.~\eqref{eq:bp_appr}
% \ENDIF
% \STATE Self-potential $p^{(0)}(y_i;x_i)$ $\gets$ Eq.~\eqref{eq:node_potential}.
% \STATE Initialize $m^{(0)}_{i \to j} (y_j) = 1$
% \STATE $\log p_i^{(k)}(y_i;x_i)$  $\gets$ Message passing in Eq.~\eqref{eq:message_passing} for  $k$ iterations.
\STATE $\hat{y_i}$ $\gets$ $\arg \max_{y_i} \log p_i^{(k)}(y_i;x_i)$

\end{algorithmic}
\end{algorithm}

\textbf{Belief Propagation} %The BP algorithm is to make such an inference. 
 Exact inference for $\mathbb{P}_{\GG}(Y|X)$ is intractable in large-scale graphs with cycles~\cite{koller2009probabilistic}. In practice, loopy belief propagation (LBP) is often used to conduct an approximate inference~\cite{murphy2013loopy}, which follows: Initialize the distributions $p_j^{(0)}(y_j)\propto\varphi_{X_i}(y_i)$ and $m_{i \to j}^{(0)}(y_j) = 1/c$ for all $i,j\in\VV$. For $k=1,2,...,L$, we do 
 \begin{align}
 \label{eq:message_passing}
 \log m_{j \to i}^{(k)}(y_i) \cong &\, \text{LSE}_{y_j}[\log \psi_{ij}(y_i,y_j) + \\ &
\log p_j^{(k-1)}(y_j) - \log m_{i \to j}^{(k-1)}(y_j)], \nonumber \\
 \log p_i^{(k)}(y_i) \cong &\log p_i^{(0)}(y_i) + \sum_{j \in \mathcal{N}(i)} \log m_{j \to i}^{(k)}(y_i),  \nonumber
\end{align} where $\cong$ denotes the equality with difference up-to a constant. LSE stands for the log-sum-exp function: 
$\text{LSE}_{y_j} [f(y_i, y_j)] = \log \left[ \sum_{y_j} \exp (f(y_i, y_j)) \right]$. The final $\arg\max_{y_i} p_i^{(k)}(y_i)$ gives the label prediction. 
Detailed derivation can be found in Appendix.~\ref{sec:app_detailed_derivation}. %\pan{fixed notation in appendix accordingly}

\textbf{\proj} To execute the above LBP algorithm, we need to specify several components based on the TAG. First, $p_i^{(0)}(y_i)$ represents the distribution of the label $y_i$ given the observed attributes $X_i$ alone, which can be estimated using normalized cosine similarities: % between node and class label embeddings: applying the Softmax function~\cite{bridle1990probabilistic} to normalize the cosine similarity between node and class label embeddings:
\begin{align}
\label{eq:node_potential}
p_i^{(0)}(y_i) = \text{softmax}(\{\cos(h^{X}_i, q^{C}_k)/\tau\}_{k\in[c]})
\end{align}







%which within a unified probabilistic framework enables the MRF formulation to encompass both homophily and heterophily among connected nodes by encoding preferences for label similarity or difference.

%For node classification, we can infer $\mathbb{P}_\mathcal{G}(\mathbf{Y} \mid \mathbf{X})$. % for each node $i$. %, given the deterministic graph structure $\mathcal{G}$ and the observed attributes $\mathbf{X}$. 
%During inference, we achieve this by maximizing the posterior distribution of the labels:
%\begin{align}
%\arg\max_{\mathbf{Y}} \,& \mathbb{P}_\mathcal{G}(\mathbf{Y} \mid \mathbf{X}) 
%~\propto~ \\
%&\prod_{i \in \mathcal{V}} \bigl[\phi_i(y_i)\,\varphi_{x_i}(y_i)\bigr] 
%\prod_{(i,j) \in \mathcal{E}} \psi_{ij}(y_i, y_j).
%\end{align}
% First, we initialize $ p_i^{(0)}(y_i)$ by applying the Softmax function~\cite{bridle1990probabilistic} to normalize the cosine similarity between node and class label embeddings:
% \begin{align}
% %\label{eq:node_potential}
% p_i^{(0)}(y_i=c_j;x_i) = \text{softmax}(cos(\hh^{\XX}_i, \hh^{\CC}_j)/\tau)
% \end{align}
%\frac{\exp  \frac{1}{\tau} cos(\hh^{\XX}_i, \hh^{\CC}_j) }{\sum_j^c \exp  \frac{1}{\tau} cos( \hh^{\XX}_i, \hh^{\CC}_j )},
where $h_i^{X}$ and $h_k^{C}$ denote node $i$'s class-conditional embedding and class $k$'s embedding given by the LLM encoder as discussed in the previous section. %derived as described in the previous section, 
%and $\hh^{\CC}$ refers to the class embeddings, obtained from class descriptions in the zero-shot setting and from the average labeled node embeddings in the few-shot setting. 
$\cos(\cdot)$ denotes cosine similarity and $\tau$ is the temperature hyper-parameter. 

Second, we characterize the edge potentials $\psi_{ij}(y_i, y_j)$. We employ an LLM agent to assess the homophily level of the TAG. Specifically, we uniformly at random sample $T$ connected node pairs ($T\ll |\EE|$), and for each pair, we prompt the LLM to determine whether the two nodes belong to the same class based on their attributes, as illustrated in Fig.~\ref{fig:pipe}. The ratio of ``Yes'' responses, denoted by $r$ is used to set 
\begin{align}
\label{eq:bp}
\ \psi_{ij}(y_i, y_j) =
\begin{cases}
r, & \text{if } y_i = y_j \\
1-r, & \text{if } y_i \neq y_j
\end{cases},
\end{align}
Note that a more complex $\psi_{ij}(y_i, y_j)$ can be adopted by estimating the edge probabilities between any two classes. However, we choose the homophily level as a proof of concept. LLMs can provide a reasonably accurate estimation of the homophily level, as pairwise comparisons are typically much simpler tasks compared to full-scale classification.


% the graph structure and estimate $H$ by sampling a subset of node pairs, significantly smaller than the total number of edges in the graph.


% As a proof of concept, we assume 
% %, we adopt a widely-used assumption in graph statistical models, which posits that the coupling potential assigns one constant potential to edges between nodes of the same class and another constant potential for edges connecting nodes of different classes~\cite{holland1983stochastic,deshpande2018contextual}, specifically, we define the edge potential $\psi_{ij}(y_i, y_j)$ as:
% \begin{align}
% \label{eq:bp}
% \textbf{BP:} \ \psi_{ij}(y_i, y_j) =
% \begin{cases}
% H, & \text{if } y_i = y_j \\
% 1 - H, & \text{if } y_i \neq y_j
% \end{cases},
% \end{align}
% where $H \in [0,1]$ denotes the homophily ratio, indicating the proportion of edges that connect nodes belonging to the same class. We employ an LLM agent to analyze the graph structure and estimate $H$ by sampling a subset of node pairs, significantly smaller than the total number of edges in the graph.
% %using the following prompting template:
% %\textit{
% %``We have two $\{\textnormal{node\_type}\}$ from the following $\{\textnormal{\# class}\}$ categories:\\
% %$\{\textnormal{class\_info}\}$\\
% %The texts are as follows:\\
% %$\{\textnormal{Text 1}\}$\\
% %$\{\textnormal{Text 2}\}$\\
% %Please tell whether they belong to the same category or not by answering Yes or No after reasoning step by step.''
% %}
% An example is given in Fig.~\ref{fig:pipe}. Note that employing LLMs for such statistical analysis is much easier than directly instructing LLMs for text classification, due to the inherent fault tolerance provided by sampling.\hy{how to say}

\cite{wei2022understanding} demonstrated that linear propagation can approximate a single iteration of LBP when feature quality is limited. Based on this insight, we adopt the following approximate LBP formulation (denoted as BP appr.):
\begin{align}
 \log p_i^{(1)}(y_i) \cong &\log p_i^{(0)}(y_i)+
 \label{eq:bp_appr}\\
 &\text{sgn}(\log \frac{r}{1-r})\sum_{j \in \mathcal{N}(i)} \log p_j^{(0)}(y_i),  \nonumber    
\end{align}
where the homophily level $r$ influences the sign of the log-likelihood aggregation from neighboring nodes. We summarize the overall pipeline in Algorithm.~\ref{alg:llm_bp}

% Inspired by the empirical success of linear likelihood propagation as mentioned in~\cite{wei2022understanding}, we further propose a simplified linear approximation of the BP algorithm, whose edge potential is defined as:
% \begin{align}
% \label{eq:bp_appr}
% &\textbf{BP (appr.): }  \psi_{ij}(y_i, y_j) =\alpha \mathbbm{1}_{(H>t)} + \beta \mathbbm{1}_{(H \leq t)},
% \end{align}
% %\begin{align}
% %\label{eq:bp_appr}
% %&\textbf{BP (appr.): }  \psi_{ij}(y_i, y_j) = 
% %&\begin{cases}
% %    m, \ \text{if} \ y_i=y_j \ \text{and} \  H \geq t\\
% %    n, \ \text{if} \ y_i=y_j \ \text{and} \ H < t\\
% %    0, \ \text{if} \ y_i \neq y_j\\
% %\end{cases},
% %\end{align}
% where $t$ is the hyper-parameter serving as the threshold to determine homophily or heterophily of the graph structure, $\alpha, \beta$ are the two coefficients hyper-parameters for neighborhood node likelihood aggregation.
% \textcolor{red}{Original}
% Inspired by ~\cite{jia2021graph} \pany{We do not need to cite Jia. The key knowledge we use here is mrf and bp. We should cite them, these more foundamental concepts. Citing Jia et al will only distract people}, we model a TAG graph as a pair-wise Markov Random Field (MRF) \pany{it is not just model. We need to motivate why we consider this model and why we consider simulating BP.} and use its probabilistic framework to guide graph structure utilization. The probabilistic framework remains unchanged regardless the variance of the graph structures, thereby leading to unified structural interpretation .

% \textbf{Data Modeling.} In pair-wise MRF, we consider the generative process for the joint distribution of node labels and attributes. The joint probability distribution $p(\yy, \XX)$ \pany{using the capital $\mathbb{P}$} \pany{have we defined the notation $Y,\yy$. Also, why is there capital letters and little ones?}   is decomposed into two components: (1) the generation of the label distribution $p(\yy)$ \pany{This is not the common meaning of label distribution. Instead, this distribution is the distribution of y given the graph structure. So, the graph structure should be put in the notation as well. } and (2) the conditional distribution of attributes given labels $p(\XX \mid \yy)$. More formally, for the label distribution $p(\yy)$ in a pairwise Markov Random Field, it can be expressed as:
% % \begin{align*}
% %     p(\yy) = \frac{\psi(\yy)}{\sum_{\yy'} \psi(\yy')}, \psi(\yy) = \prod_{i \in \VV} \phi_i(y_i) \prod_{(i,j) \in \EE} \omega_{\mathbbm{1}(y_i = y_j)}
% % \end{align*}
% \begin{align}
%     p(\yy) = \frac{1}{Z_{\yy}} \prod_{i \in \VV} \phi_i(y_i) \prod_{(i,j) \in \EE} \psi_{ij}(y_i, y_j)
% \end{align}
% where $\phi_i(y_i)$ represents the unary potential for node $i$, $\psi_{ij}(y_i, y_j)$ denote the edge potential capturing the interactions between labels $y_i$ and $y_j$ of neighboring nodes, and $Z_{\yy}$ denote the normalization constant.

% For the attribute generation $p(\XX \mid \yy)$, the process assumes that the features of each node are conditionally independent given its corresponding label. Under this assumption, the distribution can be expressed in a factorized form as:
% \begin{align*}
%     p(\XX \mid \yy) = \prod_{i \in \VV} p(x_i \mid y_i) = \prod_{i \in \VV} \varphi_{x_i}(y_i).
% \end{align*}
% where $\varphi_{x_i}(\cdot)$ represents the likelihood function, which maps a given label $y_i$ to the probability of observing the corresponding feature $x_i$.

% \textbf{Principle II with Adaption to Node Classification:} \pany{why write this as an adaption? If the following is adaption, what do you mean by the above y.}

% In the context of a node classification task, the label inference involves estimating the conditional probability $p(y_i \mid \XX)$ for each node $i$ over deterministic graph structure \pany{graph structure should be highlighted here too} and attribute observations.
% % \textbf{Data Modeling.} In the MRF, we consider the generative process for the joint distribution of node labels and attributes. Specifically, the joint probability distribution $p(\yy, \XX)$ is decomposed into two components: (1) the generation of the label distribution $p(\yy)$ and (2) the conditional distribution of attributes given labels $p(\XX \mid \yy)$. More formally, for the label distribution $p(\yy)$ in a pairwise Markov Random Field, it can be expressed as:
% % % \begin{align*}
% % %     p(\yy) = \frac{\psi(\yy)}{\sum_{\yy'} \psi(\yy')}, \psi(\yy) = \prod_{i \in \VV} \phi_i(y_i) \prod_{(i,j) \in \EE} \omega_{\mathbbm{1}(y_i = y_j)}
% % % \end{align*}
% % \begin{align}
% %     p(\yy) = \frac{1}{Z_{\yy}} \prod_{i \in \VV} \phi_i(y_i) \prod_{(i,j) \in \EE} \psi_{ij}(y_i, y_j)
% % \end{align}
% % where $\phi_i(y_i)$ represents the unary potential for node $i$, and $\omega_{\mathbbm{1}(y_i = y_j)}$ denote the edge potential capturing the interactions between labels $y_i$ and $y_j$ of neighboring nodes. In this work, we adopt a widely-used assumption in graph statistical models, which posits that the coupling potential $\omega_{\mathbbm{1}(y_i = y_j)}$ assigns one constant potential to edges between nodes of the same class and another constant potential for edges connecting nodes of different classes~\cite{holland1983stochastic,deshpande2018contextual}.
% % For the attribute generation $p(\XX \mid \yy)$, the process assumes that the features of each node are conditionally independent given its corresponding label. Under this assumption, the distribution can be expressed in a factorized form as:
% % \begin{align*}
% %     p(\XX \mid \yy) = \prod_{i \in \VV} p(x_i \mid y_i) = \prod_{i \in \VV} \varphi_{x_i}(y_i).
% % \end{align*}
% % where $\varphi_{x_i}(\cdot)$ represents the likelihood function, which maps a given label $y_i$ to the probability of observing the corresponding feature $x_i$.
% During the inference stage, this is achieved by maximizing the posterior distribution of the observed attributes, i.e. \pany{just wrote $\max_{\yy} p(\yy \mid \XX) \propto p(\yy) p(\XX \mid \yy) = xxx$}
% \begin{align}
%     p(\yy \mid \XX) & = \frac{p(\XX, \yy)}{p(\XX)} \cong  \frac{p(\yy) p(\XX \mid \yy)}{\sum_{\yy'} p(\yy') p(\XX \mid \yy')} \\
%     & \cong \prod_{i \in \VV} \phi_i(y_i)\varphi_{x_i}(y_i) \prod_{(i,j) \in \EE} \psi_{ij}(y_i, y_j) 
%     % & \cong \prod_{i \in \VV} \phi_i(y_i)\varphi_{x_i}(y_i) \prod_{(i,j) \in \EE} \omega_{\mathbbm{1}(y_i = y_j)}
% \end{align}
% where 
% % the formulation represents a conditional random field~\cite{lafferty2001conditional} and 
% $\cong$ denotes “equality up to a normalization" \pany{no need this if use $\propto$}. To further characterize the edge potentials, we adopt a widely-used assumption in graph statistical models, which posits that the coupling potential assigns one constant potential to edges between nodes of the same class and another constant potential for edges connecting nodes of different classes~\cite{holland1983stochastic,deshpande2018contextual}, and we denote this potential as $\psi_{\mathbbm{1}(y_i = y_j)}$, and therefore \pany{no need to write the follow eq. Just denote $\psi_{ij}(y_i, y_j) = \psi_{\mathbbm{1}(y_i = y_j)}$}
% \begin{align} 
%     p(\yy \mid \XX) \cong \prod_{i \in \VV} \phi_i(y_i)\varphi_{x_i}(y_i) \prod_{(i,j) \in \EE} \psi_{\mathbbm{1}(y_i = y_j)}
% \end{align}
% \pany{Here explain how we define $\psi_{\mathbbm{1}(y_i = y_j)}$ in our case.} 
% % The assumption of data generation aligns with 2.1 in the original paper:

% % \begin{equation}
% %     p(\XX|y) = \prod_{i \in V} p(\XX_i|y_i) = \prod_{i \in V} f(y_i; \XX_i)
% % \end{equation}
% % During the inference stage, we have to assign labels according to the marginal likelihood $p(y_i | \XX)$. Since exact inference for the probability is intractable in large-scale graphs with cycles~\cite{}, we leverage the approximate inference algorithm via loopy belief propagation (LBP)~\cite{murphy2013loopy} and the belief-propagation process can be described as follows:
% % \begin{align}
% %     &\textbf{Initial: } p_i^{(0)}(y_i) = \text{\textcolor{red}{Introduce the initial condition}} \\
% %     &\textbf{Run $K$ Iterations: } \\
% %     &p_i^{(k)}(y_i) \cong p_i^{(0)}(y_i)\prod_{i \in \mathcal{N}(i)} m_{j \to i}^{(k)}(y_i), \\
% %     &m_{j \to i}^{(k)}(y_i) \cong \sum_{y_j} \psi_{\mathbbm{1}_{(y_i = y_j)}} p_j^{(k-1)} / m_{i \to j}^{(k-1)}(y_j)
% %     % &\log m_{ji}^{(t)}(y_i) \approx \mathrm{LSE}_{y_j} \left[ \log H_{ji}(y_j, y_i) \right. \\
% %     % & \quad + \log p_j^{(t-1)}(y_j) - \log m_{ij}^{(t-1)}(y_j) \left. \right]
% % \end{align}

% % In practice, people usually consider belief-propagation update in the log-space for numerical stability.
% During the inference stage, we assign labels by computing the marginal likelihood $p(y_i \mid \XX)$ \pany{as missing graph structure, this notation is confusing with those without graph structure}. Since exact inference for this probability is intractable in large-scale graphs with cycles~\cite{koller2009probabilistic}, we utilize the approximate inference algorithm via loopy belief propagation (LBP)~\cite{murphy2013loopy}. The belief propagation process is described as follows: \pany{do we really write so much for BP. I think this is more in the appendix. Here, our focus is more on how our alg implement of one iter BP. And, how LLMs help to estiamte the terms in the alg.}
% \begin{align}
% \label{eq:bp}
%     &\textbf{Initialization: } p_i^{(0)}(y_i) = \phi_i(y_i)\varphi_{x_i}(y_i), \label{eq:initialization} \\
%     &\textbf{Iterative Updates (for $K$ iterations):} \nonumber \\
%     &\quad p_i^{(k)}(y_i) \cong p_i^{(0)}(y_i) \prod_{j \in \mathcal{N}(i)} m_{j \to i}^{(k)}(y_i), \label{eq:message_update1} \\
%     &\quad m_{j \to i}^{(k)}(y_i) \cong \sum_{y_j} \psi_{\mathbbm{1}(y_i = y_j)} \frac{p_j^{(k-1)}(y_j)}{m_{i \to j}^{(k-1)}(y_j)}, \label{eq:message_update2}
% \end{align}
% where \( p_i^{(0)}(y_i) \) defines the prior beliefs about node \( i \), \( \mathcal{N}(i) \) denotes the set of neighbors of node \( i \), and \( \psi_{\mathbbm{1}(y_i = y_j)} \) represents the compatibility potential between nodes \( i \) and \( j \). The process iteratively updates the messages \( m_{j \to i} \) and beliefs \( p_i \) until convergence or a predefined maximum number of iterations \( K \) is reached. In practice, belief propagation updates are typically performed in log-space to enhance numerical stability.

% \begin{align}
%     &\log p_i^{(k)}(y_i) \cong \log p_i^{(0)}(y_i) + \sum_{j \in \mathcal{N}(i)} \log m_{j \to i}^{(k)}(y_i), \\
%     &\log m_{j \to i}^{(t)}(y_i) \cong \text{LSE}_{y_j} [\log \psi_{\mathbbm{1}_{(y_i = y_j)}} + \log p_j^{(k-1)}(y_j) - \log m_{i \to j}^{(k-1)}(y-j)]
% \end{align}

% where $\mathrm{LSE}_{y_j} \left[ f(y_i, y_j) \right] = \log \left[ \sum_{y_j} \exp \left( f(y_i, y_j) \right) \right]$.

% \begin{align}
% \label{eq:bp}
%     \text{BP:} equation
% \end{align}

% Inspired by ~\cite{wei2022understanding}, linear propagation of the likelihood may also lead to goo empirical performance. Therefore, we here propose another as alternative:

% \begin{align}
% \label{eq:bp_appr}
%     \text{BP. appr.}  equation
% \end{align}
%\begin{algorithm}[t]
\caption{LLM-BP}
\label{alg:llm_bp}
\begin{algorithmic}[1]
\INPUT TAG $(\mathcal{G}, \boldsymbol{X})$
\OUTPUT Class label prediction $\{\hat{y}_i\}_{i\in[n]}$
\STATE $\hh^{X}$ $\leftarrow$ Task-adaptive encoding following Eq.~\eqref{eq:task_adaptive_llm2vec}
\IF{zero-shot}
\STATE Sample $l \ll n$ nodes, infer labels with LLMs,
\STATE Nodes clusters based on LLM prediction,
\STATE $\mathbf{q}^{C}$ $\gets$ Average embedding of samples near center,
\ELSIF{few-shot}
\STATE $\mathbf{q}^{C}$ $\gets$ Average embedding of $k$ samples per class,
\ENDIF
\STATE Estimate $\psi_{ij}(y_i, y_j)$ by employing the LLM to analyze the graph data (e.g., using Eq.~\eqref{eq:bp} based on the estimated homophily level $r$.)
\STATE Initialize $p^{(0)}(y_i)$ $\gets$ Eq.~\eqref{eq:node_potential} and $m^{(0)}_{i \to j} (y_j) = 1$
\STATE Run LLM-BP (Eq.~\eqref{eq:message_passing}) for $L$ iterations or its approximation (Eq.~\eqref{eq:bp_appr}) for single iteration
% \IF{use \textbf{BP}}
% \STATE $\psi_{ij}(y_i, y_j)$ $\gets$ Eq.~\eqref{eq:bp}
% \ELSIF{use \textbf{BP (appr.)}}
% \STATE $\psi_{ij}(y_i, y_j)$  $\gets$ Eq.~\eqref{eq:bp_appr}
% \ENDIF
% \STATE Self-potential $p^{(0)}(y_i;x_i)$ $\gets$ Eq.~\eqref{eq:node_potential}.
% \STATE Initialize $m^{(0)}_{i \to j} (y_j) = 1$
% \STATE $\log p_i^{(k)}(y_i;x_i)$  $\gets$ Message passing in Eq.~\eqref{eq:message_passing} for  $k$ iterations.
\STATE $\hat{y_i}$ $\gets$ $\arg \max_{y_i} \log p_i^{(k)}(y_i;x_i)$

\end{algorithmic}
\end{algorithm}
%Then the overall algorithm can be defined as in Algorithm.~\ref{alg:llm_bp}.







\section{Experiments} 
%\pany{My expectation is to reserve at least 3 pages for exp. }
In this section, we evaluate \proj based on its two design principles, with a primary focus on zero-shot node classification tasks. Evaluations of few-shot node classification and link prediction tasks are provided in Appendix.~\ref{sec:app_few_shot}~\ref{sec:app_link_prediction}. First, we demonstrate the effectiveness of task-adaptive encoding and identify issues with existing methods that rely on aligning node embeddings with the LLM token space. Second, we validate the effectiveness of the proposed BP algorithm. Finally, we present the end-to-end performance of \proj, comparing it to state-of-the-art baselines. We first introduce the datasets and baselines used in the study:

\textbf{Datasets} As listed in Table~\ref{tab:datasets}, we selected eleven real-world TAG datasets that encompass a variety of text domain shifts, including citation networks, e-commerce data, knowledge graphs, and webpage networks, which cover both homophily and heterophily structures. For more details, see Appendix~\ref{sec:app_datasets}. 

\begin{table*}[ht]
    \footnotesize
    \centering
    \renewcommand{\arraystretch}{1.1} % Adjusts the row spacing
    \resizebox{16cm}{!} 
    { 
    \begin{tblr}{hline{1,2,Z} = 0.8pt, hline{3-Y} = 0.2pt,
                 colspec = {Q[l,m, 13em] Q[l,m, 6em] Q[c,m, 8em] Q[c,m, 5em] Q[l,m, 14em]},
                 colsep  = 4pt,
                 row{1}  = {0.4cm, font=\bfseries, bg=gray!30},
                 row{2-Z} = {0.2cm},
                 }
\textbf{Dataset}       & \textbf{Table Source} & \textbf{\# Tables / Statements} & \textbf{\# Words / Statement} & \textbf{Explicit Control}\\ 
\SetCell[c=5]{c} \textit{Single-sentence Table-to-Text}\\
ToTTo \cite{parikh2020tottocontrolledtabletotextgeneration}   & Wikipedia        & 83,141 / 83,141                  & 17.4                          & Table region      \\
LOGICNLG \cite{chen2020logicalnaturallanguagegeneration} & Wikipedia        & 7,392 / 36,960                  & 14.2                          & Table regions      \\ 
HiTab \cite{cheng-etal-2022-hitab}   & Statistics web   & 3,597 / 10,672                  & 16.4                          & Table regions \& reasoning operator \\ 
\SetCell[c=5]{c} \textit{Generic Table Summarization}\\
ROTOWIRE \cite{wiseman2017challengesdatatodocumentgeneration} & NBA games      & 4,953 / 4,953                   & 337.1                         & \textbf{\textit{X}}                   \\
SciGen \cite{moosavi2021scigen} & Sci-Paper      & 1,338 / 1,338                   & 116.0                         & \textbf{\textit{X}}                   \\
NumericNLG \cite{suadaa-etal-2021-towards} & Sci-Paper   & 1,355 / 1,355                   & 94.2                          & \textbf{\textit{X}}                    \\
\SetCell[c=5]{c} \textit{Table Question Answering}\\
FeTaQA \cite{nan2021fetaqafreeformtablequestion}     & Wikipedia      & 10,330 / 10,330                 & 18.9                          & Queries rewritten from ToTTo \\
\SetCell[c=5]{c} \textit{Query-Focused Table Summarization}\\
QTSumm \cite{zhao2023qtsummqueryfocusedsummarizationtabular}                        & Wikipedia      & 2,934 / 7,111                   & 68.0                          & Queries from real-world scenarios\\ 
\textbf{eC-Tab2Text} (\textit{ours})                           & e-Commerce products      & 1,452 / 3,354                   & 56.61                          & Queries from e-commerce products\\
    \end{tblr}
    }
\caption{Comparison between \textbf{eC-Tab2Text} (\textit{ours}) and existing table-to-text generation datasets. Statements and queries are used interchangeably. Our dataset specifically comprises tables from the e-commerce domain.}
\label{tab:datasets}
\end{table*}

%\subsection{Baselines and Settings}
\textbf{Baselines:} We select representative baselines from all existing categories for model generalization on TAGs:

%\pany{these categories are better aligned with related works}
$\bullet$ \textit{Vanilla LM / LLM Encoders}: including Sentence-BERT (SBert)~\cite{reimers2019sentence}, RoBERTa~\cite{liu2019roberta}, text-embedding-3-large~\cite{openai2024textembedding}, and bge-en-icl~\cite{li2024making}, a state-of-the-art LLM2Vec encoder.

$\bullet$ \textit{Vanilla LLMs}: including GPT-3.5-turbo~\cite{achiam2023gpt} and GPT-4o~\cite{hurst2024gpt}, the latter being among the most advanced LLMs in reasoning. They process raw node texts without incorporating graph structures.

$\bullet$ \textit{Tuning LM Encoder / GNNs}: including ZeroG~\cite{li2024zerog}, UniGLM~\cite{fang2024uniglm} that tune LM encoders, ZeroG is specifically proposed for zero-shot node classification. DGI~\cite{velivckovic2018deep}, GraphMAE~\cite{hou2022graphmae} that perform Graph-SSL are also compared.
%\pany{try not duplicate much with related work. The related work should focus on the highlevel idea of these methods and the potential issues of their generalization. The exp just mentions that using these models from this category, maybe with a brief recap of their pipeline}

$\bullet$ \textit{LLMs with Graph Adapters}: including LLaGA~\cite{chen2024llaga}, TEA-GLM~\cite{wang2024llms}, and GraphGPT~\cite{tang2024graphgpt}, which are the three representative works adopting LLMs with projectors to align compressed node representations with the token space.%all of them are implemented with embeddings from SBert~\cite{reimers2019sentence}.

$\bullet$ \textit{Multi-Task Graph Foundation Models}: Consisting of OFA~\cite{liu2023one} and GOFA~\cite{kong2024gofa}, which are the state-of-the-art multi-task foundation models.

$\bullet$ \textit{LLMs for Data Augmentation}: referring to LLM-GNN~\cite{chen2023label}, specifically designed for zero-shot node classification, which utilizes LLMs as annotators for pseudo-labels and further train GNNs for inference.

$\bullet$ \textit{Neighborhood Aggregation (NA)}: referring to the training-free method proposed in ~\cite{yang2024gl}, which injects graph structural information into node representations by directly aggregating the averaged neighborhood embeddings.

\textbf{Settings:} Unlike LLM-BP which is training-free, most of the baselines--except from vanilla encoders, LLMs or NA--require pre-training. Methods of vanilla encoders and LLM-BP that require sampling nodes to obtain class embeddings under zero-shot settings are repeated 30 times with seed 42 to 71, and the average performance is reported in the following experiment sections. Implementation details for baselines and LLM-BP can be found in Appendix.~\ref{sec:app_implementation_details_baseline}~\ref{sec:app_llm_bp_implementation_details}.

\subsection{Evaluation for Task-Adaptive Node Embedding}
%
\begin{wrapfigure}{r}{0.5\textwidth}
    \centering
    \includegraphics[width=0.7\linewidth]{Arxiv/figures/principle_1/plm_acc.pdf}
    \caption{Zero-Shot Accuracy of vanilla encoders vs. LLMs-with-Graph-Adapters. All the encoder-based methods do not leverage graph structure information.}
    \label{fig:principle_1}
    \vspace{-0.3cm}
\end{wrapfigure}
\begin{figure}[t]
    \centering 
    \begin{minipage}{0.46\textwidth}  % 右侧单独的一张大图
        \centering
        \includegraphics[width=\textwidth]{Arxiv/figures/principle_1/plm_acc.pdf}
        
        \caption{Zero-Shot Accuracy of vanilla encoders vs. LLMs-with-Graph-Adapters. All the encoder-based methods do not leverage graph structure information.}
        \label{fig:principle_1}
    \end{minipage}
    \begin{minipage}{0.46\textwidth}  % 左侧大图(由 4 小图组成)
        \centering
        \begin{minipage}{0.49\textwidth}
            \centering
            \includegraphics[width=\textwidth]{Arxiv/figures/tsne/citeseer_sbert_False.pdf}
            \caption*{\scriptsize SBert.}
        \end{minipage}
        \begin{minipage}{0.49\textwidth}
            \centering
            \includegraphics[width=\textwidth]{Arxiv/figures/tsne/citeseer_llmgpt_text-embedding-3-large_False.pdf}
            \caption*{\scriptsize Text-Embedding-3-Large.}
        \end{minipage}
        
        \begin{minipage}{0.47\textwidth}
            \centering
            \includegraphics[width=\textwidth]{Arxiv/figures/tsne/citeseer_llmicl_primary_False.pdf}
            \caption*{\scriptsize LLM2Vec.}
        \end{minipage}
        \begin{minipage}{0.47\textwidth}
            \centering
            \includegraphics[width=\textwidth]{Arxiv/figures/tsne/citeseer_llmicl_class_aware_False.pdf}
            \caption*{\scriptsize Task-Adaptive Encoder (Ours).}
        \end{minipage}
        \caption{t-SNE visualization of encoders on Citeseer.}
        \label{fig:tsne}
    \end{minipage}
    \hfill
   
\end{figure}
 
%Experiments with encoders only in this part does not consider graph structure usage and directly classify nodes according to their maximum cosine similarity with the class description embeddings. \pan{This should be merged into Exp 1 and Exp 2 separately.}

$\bullet$ \textbf{Exp.1: Ineffectiveness of LLMs w/ Graph Adapters}  
%\pan{Also, I think the title would better say the ineffectiveness of Graph Adaptor.} 
%\pan{Explain that even without using graph structures xxx} 
Figure~\ref{fig:principle_1} illustrates the accuracy of encoder-based methods alongside two representative LLMs-with-graph-adapters methods across each dataset. Notably, using text embeddings generated by SBert~\cite{reimers2019sentence} without incorporating graph structural information significantly outperforms both LLaGA~\cite{chen2024llaga} and GraphGPT~\cite{tang2024graphgpt}. These two methods align node representations that combine SBert embeddings with graph information to the LLMs' token space via a projector. This finding suggests that the generalization capabilities of these approaches primarily stem from the pre-trained language model encoders rather than the LLMs' inherent understanding of TAG data. Consequently, future works should exercise caution when adopting this strategy. % Moreover, incorporating LLMs with the embeddings from LM encoders without sufficient training for the projector, may even hinder generalization.



$\bullet$ \textbf{Exp.2: Effectiveness of The Task-Adaptive Encoder}  

\begin{wrapfigure}{r}{0.55\textwidth}
    \centering
    \vspace{-0.2cm}
    \begin{minipage}{0.17\textwidth}
    \captionsetup{labelformat=empty}
        \centering
        \includegraphics[width=\textwidth]{Arxiv/figures/class_condition/class_condition_acc_sbert.pdf}
        \vspace{-0.5cm}
        \caption*{\scriptsize SBert (Encoder).}
    \end{minipage}
    %\hfill
    \begin{minipage}{0.17\textwidth}
    \captionsetup{labelformat=empty}
        \centering
        \includegraphics[width=\textwidth]{Arxiv/figures/class_condition/class_condition_acc_roberta.pdf}
        \vspace{-0.5cm}
        \caption*{\scriptsize Roberta (Encoder).}
    \end{minipage}
    \begin{minipage}{0.17\textwidth}
    \captionsetup{labelformat=empty}
        \centering
        \includegraphics[width=\textwidth]{Arxiv/figures/class_condition/class_condition_acc_llmgpt_text-embedding-3-large.pdf}
        \vspace{-0.5cm}
        \caption*{\scriptsize  Text-Embedidng-3-Large.}
    \end{minipage}
    \vspace{-0.2cm}
    \caption{Class information fed into different encoders.}
    \label{fig:class_condition}
    \vspace{-0.6cm}
\end{wrapfigure}
According to Figure~\ref{fig:principle_1}, the task-adaptive encoder achieves the best performance on most of the datasets, enhancing the vanilla LLM2Vec on average by $2.3\%$, highlighting the importance of incorporating task-specific information during encoding. 
To further illustrate this, we use the Citeseer~\cite{giles1998citeseer} dataset as an example and perform t-SNE visualization~\cite{van2008visualizing} on the embeddings derived from the encoders. As shown in Fig.~\ref{fig:tsne}, when provided with class information, the task-adaptive encoder generates embeddings that exhibit tighter clustering for the same class compared to other baselines. The significance test of improvement from task-adatove encoding is provided in Table.~\ref{tab:confidence} in Appendix.~\ref{sec:app_confidence}.
%\begin{figure*}[t]
\centering
\includegraphics[width=0.49\linewidth]{figures/tsne-small.png}
\includegraphics[width=0.49\linewidth]{figures/tsne-large.png}
\caption{LW/non-LW pairs visualized across all models of the Llama family. Larger models tend to use their representation space more efficiently, leading to greater subword token representation similarity.}
\label{fig:tsne}
\end{figure*}


Note that the benefits of class information are observed only in encoders derived from LLM decoders potentially due to their strong contextual learning capabilities. As illustrated in Fig.~\ref{fig:class_condition}, incorporating class information into smaller LM encoders, such as SBert~\cite{reimers2019sentence} or RoBERTa~\cite{liu2019roberta}, may even degrade performance.  
Regarding Text-embedding-3-large~\cite{openai2024textembedding}, the impact of class information remains inconclusive due to the unknown internal mechanisms of the black-box encoder.







\subsection{Generalizable Graph Aggregation}
\begin{wrapfigure}{r}{0.46\textwidth}
    \centering
    \vspace{-0.9cm}
    \begin{minipage}{0.17\textwidth}
    \captionsetup{labelformat=empty}
        \centering
        %\includegraphics[width=\textwidth]{Arxiv/figures/pred_h/pred_h_GPT-4o.pdf}
        \resizebox{\textwidth}{!}{\begin{tabular}{@{}cc@{}}
        \toprule
        Graph Type & \# Sampled Edges \\ \midrule
        Citation & 100 \\
        E-Commerce & 100 \\
        Knowledge Graph & 100 \\
        Web Page & 50 \\ \bottomrule
        \end{tabular}}
        \vspace{-0.0cm}
        %\caption*{\scriptsize \# Sampled Edges.}
    \end{minipage}
    %\hfill
    \begin{minipage}{0.23\textwidth}
    \captionsetup{labelformat=empty}
        \centering
        \includegraphics[width=\textwidth]{Arxiv/figures/pred_h/pred_h_GPT-4o-mini.pdf}
        \vspace{-0.8cm}
        %\caption*{\scriptsize GPT-4o-mini.}
    \end{minipage}
    \vspace{-0.1cm}
    \caption{\small Left: Number of edges sampled per dataset. Right: GPT-4o-mini's prediction of the homophily level $r$.}
    \label{fig:predict_h}
    \vspace{-0.6cm}
\end{wrapfigure}

 %\pan{just explicitly say out principle II}
\begin{table*}[t]
\setlength{\tabcolsep}{2pt}
\renewcommand{\arraystretch}{1.2}
\resizebox{\textwidth}{!}{
\begin{tabular}{c|cccccccccccccc|cccccccc|cc}
\hline
 & \multicolumn{14}{c|}{Homophilic} & \multicolumn{8}{c|}{Heterophilic} & \multicolumn{2}{c}{Avg Rank} \\ \hline
 & \multicolumn{6}{c|}{Citation Graph} & \multicolumn{8}{c|}{E-Commerce \& Knowledge Graph} & \multicolumn{8}{c|}{Schools} & \multicolumn{2}{c}{} \\ \hline
Method & \multicolumn{2}{c}{Cora} & \multicolumn{2}{c}{Citeseer} & \multicolumn{2}{c|}{Pubmed} & \multicolumn{2}{c}{History} & \multicolumn{2}{c}{Children} & \multicolumn{2}{c}{Sportsfit} & \multicolumn{2}{c|}{Wikics} & \multicolumn{2}{c}{Cornell} & \multicolumn{2}{c}{Texas} & \multicolumn{2}{c}{Wisconsin} & \multicolumn{2}{c|}{Washington} &  &  \\
 & Acc & F1 & Acc & F1 & Acc & \multicolumn{1}{c|}{F1} & Acc & F1 & Acc & F1 & Acc & F1 & Acc & F1 & Acc & F1 & Acc & F1 & Acc & F1 & Acc & F1 & Acc & F1 \\ \hline
Sbert~\cite{reimers2019sentence} & 69.75 & 67.21 & 66.69 & 63.31 & 70.57 & \multicolumn{1}{c|}{71.38} & 53.53 & 20.45 & 22.59 & 20.13 & 43.79 & 38.26 & 59.06 & 56.19 & 63.66 & 54.39 & 64.58 & 49.79 & 62.10 & 52.07 & 63.52 & 48.00 & 7.27 & 7.09 \\
Roberta~\cite{liu2019roberta} & 70.71 & 68.47 & 66.95 & 63.57 & 69.54 & \multicolumn{1}{c|}{70.31} & 55.39 & 21.84 & 24.25 & 22.41 & 41.51 & 36.09 & 59.08 & 56.49 & 61.68 & 51.84 & 62.25 & 49.26 & 60.33 & 49.08 & 60.60 & 45.34 & 7.18 & 7.18 \\
Text-Embedding-3-Large~\cite{openai2024textembedding} & 71.90 & 69.87 & 66.24 & 63.30 & 75.96 & \multicolumn{1}{c|}{75.75} & 50.15 & 19.21 & 24.68 & 24.10 & 58.39 & 53.03 & 61.78 & 58.82 & 81.50 & 70.11 & 75.42 & 63.17 & 73.14 & 63.02 & 66.35 & 57.69 & 5.36 & 4.45 \\
LLM2Vec~\cite{behnamghader2024llm2vec} & 67.34 & 65.92 & 67.13 & 64.37 & 74.57 & \multicolumn{1}{c|}{74.65} & 53.14 & 19.06 & 25.56 & 24.31 & 57.00 & 52.29 & 62.34 & 58.32 & 81.26 & 69.08 & 76.68 & 63.12 & 73.36 & 62.50 & 65.92 & 53.34 & 5.64 & 5.36 \\ \hline
SBert + NA~\cite{yang2024gl} & 72.49 & 69.90 & 68.66 & 64.75 & 71.26 & \multicolumn{1}{c|}{71.87} & 57.86 & 21.98 & 25.28 & 22.74 & 46.84 & 40.85 & 66.26 & 63.57 & 54.21 & 44.66 & 56.04 & 41.09 & 54.23 & 46.11 & 58.88 & 43.05 & 5.82 & 6.00 \\ \hline
GPT-3.5-turbo~\cite{achiam2023gpt} & 70.11 & 52.11 & 66.83 & 47.58 & 89.75 & \multicolumn{1}{c|}{66.16} & 55.07 & 30.36 & 29.73 & 26.13 & \textbf{67.21} & 54.45 & 65.53 & 51.19 & 45.54 & 39.30 & 56.14 & 32.53 & 58.86 & 46.84 & 51.09 & 35.68 & 5.64 & 8.18 \\
GPT-4o~\cite{hurst2024gpt} & 70.29 & 62.95 & 64.77 & 47.78 & \textbf{89.85} & \multicolumn{1}{c|}{67.39} & 53.30 & 31.68 & \textbf{30.76} & \textbf{29.20} & 66.35 & 56.22 & 66.10 & 56.04 & 45.54 & 41.92 & 63.10 & 50.51 & 56.60 & 52.54 & 48.90 & 42.54 & 5.91 & 6.36 \\ \hline
UniGLM~\cite{fang2024uniglm} & 45.57 & 43.25 & 52.26 & 48.41 & 70.33 & \multicolumn{1}{c|}{69.78} & 44.24 & 24.84 & 21.48 & 19.17 & 33.46 & 32.99 & 55.05 & 52.08 & 23.03 & 22.06 & 21.39 & 18.90 & 27.16 & 26.45 & 24.01 & 23.08 & 11.36 & 9.91 \\
ZeroG~\cite{li2024zerog} & 60.4 & 56.02 & 50.35 & 45.15 & 74.68 & \multicolumn{1}{c|}{71.75} & 36.55 & 16.84 & 12.72 & 12.61 & 14.27 & 5.33 & 46.74 & 40.86 & 10.47 & 6.46 & 53.48 & 15.95 & 12.66 & 5.02 & 8.3 & 3.07 & 12.27 & 12.73 \\ \hline
DGI~\cite{velivckovic2018deep} & 16.79 & 12.77 & 15.24 & 15.04 & 25.10 & \multicolumn{1}{c|}{19.18} & 20.98 & 3.89 & 2.22 & 1.04 & 7.48 & 3.47 & 14.98 & 4.24 & 14.66 & 10.02 & 11.23 & 9.42 & 12.08 & 6.95 & 20.96 & 14.15 & 13.91 & 14.73 \\
GraphMAE~\cite{hou2022graphmae} & 15.13 & 7.10 & 8.11 & 7.67 & 36.56 & \multicolumn{1}{c|}{34.29} & 36.36 & 5.75 & 7.24 & 1.97 & 30.50 & 6.99 & 8.91 & 4.03 & 23.04 & 14.95 & 17.65 & 11.67 & 23.02 & 11.87 & 24.89 & 13.34 & 15.18 & 15.45 \\ \hline
OFA~\cite{liu2023one} & 20.36 & 16.57 & 41.31 & 33.37 & 28.18 & \multicolumn{1}{c|}{26.62} & 8.25 & 3.48 & 3.05 & 2.29 & 15.18 & 4.7 & 30.77 & 25.22 & 29.84 & 12.62 & 11.77 & 5.87 & 4.8 & 3.44 & 6.04 & 4.28 & 13.91 & 14.73 \\
GOFA~\cite{kong2024gofa} & 71.06 & 70.21 & 65.72 & 64.18 & 74.76 & \multicolumn{1}{c|}{73.00} & 56.25 & 31.57 & 12.15 & 7.73 & 37.87 & 33.19 & \textbf{68.62} & 62.93 & 39.50 & 35.47 & 38.37 & 29.54 & 32.51 & 25.12 & 31.02 & 21.24 & 8.00 & 7.45 \\ \hline
GraphGPT~\cite{tang2024graphgpt} & 17.48 & 12.68 & 13.93 & 12.78 & 42.94 & \multicolumn{1}{c|}{25.68} & 12.31 & 9.15 & 9.94 & 4.24 & 4.53 & 2.44 & 33.59 & 30.21 & 10.18 & 14.71 & 18.48 & 9.85 & 12.35 & 6.32 & 20.64 & 15.79 & 14.55 & 14.64 \\
LLAGA~\cite{chen2024llaga} & 11.62 & 14.42 & 19.52 & 23.34 & 7.56 & \multicolumn{1}{c|}{13.42} & 7.95 & 8.89 & 10.09 & 5.02 & 1.84 & 2.66 & 10.98 & 16.73 & 12.57 & 20.1 & 15.51 & 22.97 & 15.09 & 20.85 & 10.48 & 18.98 & 15.36 & 13.64 \\ \hline
LLM-BP & \textbf{72.59} & \textbf{71.10} & \textbf{69.51} & \textbf{66.29} & 75.55 & \multicolumn{1}{c|}{75.32} & \textbf{59.86} & 22.66 & 24.81 & 22.66 & 61.92 & \textbf{57.51} & 67.75 & 63.53 & 83.28 & 71.80 & \textbf{81.66} & \textbf{65.41} & \textbf{77.75} & \textbf{63.70} & \textbf{73.14} & \textbf{57.33} & \textbf{2.27} & 2.55 \\
LLM-BP (appr.) & 71.41 & 70.11 & 68.66 & 65.62 & 76.81 & \multicolumn{1}{c|}{\textbf{76.81}} & 59.49 & \textbf{23.02} & 29.40 & 28.45 & 61.51 & 57.09 & 67.96 & \textbf{64.27} & \textbf{84.92} & \textbf{74.19} & 79.39 & 64.63 & 75.65 & 62.53 & 70.04 & 55.53 & 2.45 & \textbf{2.27} \\ \hline
\end{tabular}}
\vspace{-0.2cm}
\caption{Zero-Shot End-to-End Evaluation. `NA' refers to neighborhood embedding aggregation.}
\label{tab:zero_shot}
\end{table*}
\begin{figure*}[t]
    \centering
    \begin{minipage}{0.22\textwidth}
    \captionsetup{labelformat=empty}
        \centering
        \includegraphics[width=\textwidth]{Arxiv/figures/principle_2/principle_2_acc_sbert.pdf}
        \vspace{-0.5cm}
        \caption*{\scriptsize SBert.}
    \end{minipage}
    %\hfill
    \begin{minipage}{0.22\textwidth}
    \captionsetup{labelformat=empty}
        \centering
        \includegraphics[width=\textwidth]{Arxiv/figures/principle_2/principle_2_acc_llmgpt_text-embedding-3-large.pdf}
        \vspace{-0.5cm}
        \caption*{\scriptsize Text-Embedding-3-Large.}
    \end{minipage}
    \begin{minipage}{0.22\textwidth}
    \captionsetup{labelformat=empty}
        \centering
        \includegraphics[width=\textwidth]{Arxiv/figures/principle_2/principle_2_acc_llmicl_primary.pdf}
        \vspace{-0.5cm}
        \caption*{\scriptsize LLM2Vec.}
    \end{minipage}
    \begin{minipage}{0.22\textwidth}
    \captionsetup{labelformat=empty}
        \centering
        \includegraphics[width=\textwidth]{Arxiv/figures/principle_2/principle_2_acc_llmicl_class_aware.pdf}
        \vspace{-0.5cm}
        \caption*{\scriptsize Class-Conditional Encoding (Ours).}
    \end{minipage}
    \vspace{-0.2cm}
    \caption{Experiments on graph information aggregation. `Raw' refers no graph structure usage, `w/ NA' refers to the neighborhood embedding aggregation (NA) proposed in~\cite{yang2024gl}, `w/ BP' refers to the belief propagation following Eq.~\ref{eq:bp}, `w/ BP (appr.)' refers to its simplified linear form that follows Eq.~\ref{eq:bp_appr}.}
    \label{fig:principle_2}
    \vspace{-0.3cm}
\end{figure*}
$\bullet$ \textbf{Exp.3: LLM Agents for Homophily Level $r$ Estimation} As shown in Fig.\ref{fig:predict_h} (Left), we randomly sample $k$ edges ($k=100$ for large graphs and $k=50$ for small ones), incorporating them into prompts (Fig.\ref{fig:pipe}) for LLM-based estimation of the homophily level $r$ (Sec.\ref{sec:principle2}). We evaluate four LLMs: GPT-4o, GPT-4o-mini\cite{hurst2024gpt}, GPT-3.5-Turbo~\cite{achiam2023gpt}, and Mistral-7B-Instruct v0.3~\cite{jiang2023mistral}. Each model responds to each node pair over five trials, with the final estimate determined by majority voting. Full results are provided in Fig.\ref{fig:predict_h_more} in appendix, demonstrating that GPT-4o-mini and GPT-4o effectively estimate $r$, GPT-3.5-Turbo performs reasonably well, while Mistral-7B-Instruct-v0.3 fails. Balancing accuracy and cost efficiency, we select GPT-4o-mini's estimation (Fig.~\ref{fig:predict_h} Right) for subsequent studies. % and adopt its estimation throughout our study. 

$\bullet$ \textbf{Exp.4: Effectiveness of the BP Algorithm}
Experimental results are presented in Fig.~\ref{fig:principle_2}, where we evaluate the four approaches over various graph structures. Specifically, We compare the BP algorithm (Eq.~\ref{eq:bp}) and its linear approximation (Eq.~\ref{eq:bp_appr}) with vanilla encoders that do not utilize structure (Raw) and the NA baseline. For all the four encoders across all the datasets, the proposed BP algorithm slightly outperforms its linear approximation, and they consistently outperform Raw. Moreover, in most datasets, they also surpass the NA baseline, particularly on heteophilic graphs, where direct neighborhood embedding aggregation negatively affects performance. These results highlight the generalizability of our data modeling approach and the effectiveness of the key-parameter estimation design in BP.


\subsection{End-to-End Evaluation}

$\bullet$ \textbf{Exp.5: Main Results in the Zero-Shot Setting}
The main experimental results are presented in Table~\ref{tab:zero_shot}. Among the baselines, vanilla encoders and LLMs demonstrate strong zero-shot generalization. GPT-3.5-Turbo~\cite{achiam2023gpt} ranks first on the Sportsfit dataset, while GPT-4o~\cite{hurst2024gpt} achieves the best performance on Pubmed and Children.

UniGLM~\cite{fang2024uniglm} and ZeroG~\cite{li2024zerog} perform well in domains aligned with their pre-training, such as citation networks (e.g., ZeroG enhances SBert's performance on Cora, Pubmed, and Wikics). However, both struggle on TAGs with unseen text distributions (e.g., Sportsfit) or novel graph structures (e.g., webpage networks), suggesting that fine-tuned LM encoders may suffer performance degradation on out-of-domain TAGs.
Similarly, graph-SSL methods (DGI~\cite{velivckovic2018deep}, GraphMAE~\cite{hou2022graphmae}) show limited generalization across structural shifts. 

Among multi-task graph foundation models, GOFA achieves strong performance, likely benefiting from a larger pre-training corpus for graph-text alignment~\cite{hu2021ogb, ding2023enhancing} compared to GraphGPT~\cite{tang2024graphgpt} and LLaGA~\cite{chen2024llaga}, which are trained solely on ogbn-arxiv. However, GOFA still requires broader pre-training and instruction fine-tuning to improve generalization under text domain shifts, and its reliance on GNNs may limit effectiveness on heterophilic data.

Notably, LLM-BP and LLM-BP (appr.) achieve the highest average ranking across all datasets. Another interesting observation is that when we randomly sample $20c$ nodes to obtain the class embeddings with the help of LLMs following Algorithm.~\ref{alg:llm_bp}, the zero-shot performance of the encoders in this setting is comparable to their performance between $5$-$10$-shot setting as shown in Table.~\ref{tab:few_shot} in the Appendix.
Further comparisons with LLM-GNN~\cite{chen2023label} and TEA-GLM~\cite{wang2024llms} are provided in Appendix~\ref{sec:app_more_baselines}.


$\bullet$ \textbf{Exp.6: Main Results under Few-shot Setting} We conduct the evaluation in $k=1,3,5,10$-shot settings. %with $c$-way settings, where $c$ remains the same as the total number of classes in each dataset. 
Using 10 different random seeds, we sample the shots from the training set and repeat the experiments 10 times. The experimental results are presented in Table~\ref{tab:few_shot} in Appendix~\ref{sec:app_more_experiment_results}. Across all 
$k$-shot settings, LLM-BP and LLM-BP (appr.) outperform the baseline models. %such as SBERT~\cite{reimers2019sentence}, Text-Embedding-3-Large~\cite{openai2024textembedding}, and LLM2Vec~\cite{li2024making}.

%$\bullet$ \textbf{Exp.7: Zero-Shot Link Prediction}
%We also extend our framework into the zero-shot link prediction task to validate the generalizability of our principles. Results are shown in Table.~\re in Appendix.~\ref{sec:app_link_prediction}.




%\subsection{Ablation Study}

%\textbf{Number of Aggregation Layers}

%A Table.

\section{Discussion and Limitations}
%Identifying the challenges in adopting LLMs to graphs, we propose two generalization principles for model generalization over TAG data. Following these principles, we introduce LLM-BP that integrates task-adaptive encoding for a unified node attribute space with belief propagation for generalized graph structural usage. Key algorithmic parameters are estimated using an LLM agent. Experimental results on node classification tasks demonstrate the effectiveness as well as generalizability of LLM-BP. 

Graph learning tasks often face substantial data constraints compared to other domains, underscoring the importance of establishing fundamental principles that foster model generalization. Our approach exemplifies this by leveraging LLMs to analyze graph data and determine suitable inference strategies, particularly via homophily estimation for belief propagation. While \proj achieves notable success on TAGs for node classification and extends partially to link prediction, it remains a step away from a fully comprehensive graph foundation model that addresses a wider range of graph learning tasks. Nonetheless, the core idea of leveraging LLM-driven graph analysis to guide algorithmic decisions aligned with task-specific inductive biases holds broad potential for future applications.




\section*{Acknowledgments}
%We thank Yuhan Li, Dr. Jia Li from HKUST and Zhikai Chen, Dr. Jiliang Tang from Michigan State university for their valuable explorations and benchmarks on text-attributed graphs~\cite{li2024glbench, chen2024text}. Additionally, we acknowledge the implementation of LLaGA~\cite{chen2024llaga} by Runjin Chen, Dr. Zhangyang Wang from UT Austin. Our dataset pre-processing are built upon these works. We thank Dr. Junteng Jia and Dr. Austin R. Benson for their work~\cite{jia2021graph}, which has provided us with valuable inspiration in the generalized use of  graph structures. 
H. Wang, S. Liu, R. Wei and Dr. P. Li are partially supported by NSF awards IIS-2239565, CCF-2402816, IIS-2428777, PHY-2117997; DOE award DE-FOA-0002785; JPMC faculty awards; Microsoft Azure Research Credits for Generative AI; and Meta research award.

We extend our sincere gratitude to Dr. Hongbin Pei, Dr. Zhen Wang, and Jingcheng Cen for their valuable assistance in identifying the raw node text of the heterophilic graphs used in this study.





%Bibliography
\bibliographystyle{unsrt}  
\bibliography{references}  

\newpage

\section{More Related Works}
\label{sec:app_more_related_works}
\textbf{LLMs for Data Augmentation} annotate pseudo-labels via their advanced zero-shot text classification performance. \textit{E.g.}, LLM-GNN~\cite{chen2023label}, Cella~\cite{zhang2024cost} and ~\cite{hu2024low} propose heuristics to actively select and annotate pseudo-labels for supervised GNN training. ~\cite{pan2024distilling} performs knowledge distillation with LLMs as teachers. ~\cite{yu2023empower,li2024enhancing} generate synthetic node text with LLMs. The performance of these methods depend on the capability of LLM, and may still require relatively high annotating and training cost.

\textbf{LLMs for Graph Property Reasoning} focus on reason graph structure properties (e.g., shortest path, node degree, etc)~\cite{tang2024grapharena, dai2024large, yuan2024gracore, ouyang2024gundam}. Representative works include~\cite{perozzi2024let, chen2024graphwiz, zhang2024can, cao2024graphinsight, wei2024gita}.

\textbf{Tuning LMs/GNNs towards Better Task-Specific Performance} aims to push the limits of task-specific performance on TAGs other than generalization. These methods develop novel techniques to optimize LMs or GNNs for pushing the limits of in-domain performance~\cite{chien2021node, duan2023simteg, he2023harnessing, zhao2022learning, zhu2021textgnn, li2021adsgnn, yang2021graphformers, bi2021leveraging, pang2022improving, zolnai2024stage, yang2021graphformers}.

\textbf{Text embeddings} Generating unified text embeddings is a critical research area with broad applications, including web search and question answering. Numerous text encoders~\cite{reimers2019sentence, liu2019roberta, song2020mpnet} based on pre-trained language models have served as the foundation for various embedding models. Recently, decoder-only LLMs have been widely adopted for text embedding tasks~\cite{li2023towards, moreira2024nv} achieving remarkable performance on the Massive Text Embedding Benchmark (MTEB)~\cite{muennighoff2022mteb}. This progress stems from LLM2Vec~\cite{behnamghader2024llm2vec}, which introduces a novel unsupervised approach to transforming decoder-only LLMs into embedding models, including modifications to enable bidirectional attention. Recent findings~\cite{li2024making} suggest that retaining the unidirectional attention mechanism enhances LLM2Vec’s empirical performance.



\section{Experiment Details}

\subsection{Dataset Details}
\label{sec:app_datasets}
\textbf{Meta-Data}
In Table.~\ref{tab:dataset_meta_data}, we show the meta-data of all the eleven datasets used in our experiments.

\begin{table}[h]
\centering
\begin{tabular}{@{}ccccc@{}}
\toprule
 & \begin{tabular}[c]{@{}c@{}}Number\\ of Nodes\end{tabular} & \begin{tabular}[c]{@{}c@{}}Number \\ of  Edges\end{tabular} & \begin{tabular}[c]{@{}c@{}}Number \\ of  Classes\end{tabular} & \begin{tabular}[c]{@{}c@{}}Ground Truth\\ Homophily Ratio\end{tabular} \\ \midrule
Cora & 2708 & 10556 & 7 & 0.809 \\
Citeseer & 3186 & 8450 & 6 & 0.764 \\
Pubmed & 19717 & 88648 & 3 & 0.792 \\
History & 41551 & 503180 & 12 & 0.662 \\
Children & 76875 & 2325044 & 24 & 0.464 \\
Sportsfit & 173055 & 3020134 & 13 & 0.9 \\
Wikics & 11701 & 431726 & 10 & 0.678 \\
Cornell & 191 & 292 & 5 & 0.115 \\
Texas & 187 & 310 & 5 & 0.067 \\
Wisconsin & 265 & 510 & 5 & 0.152 \\
Washington & 229 & 394 & 5 & 0.149 \\ \bottomrule
\end{tabular}
\vspace{-0.cm}
\caption{Meta data of the datasets in this study.}
\label{tab:dataset_meta_data}
\end{table}


\textbf{Dataset Split} For the datasets (all the homophily graphs) that have been used for study in TSGFM~\cite{chen2024text}, we follow their implementation to perform data pre-processing, obtain raw texts and do data split, the introduction to data source can be found at Appendix.D.2 in their original paper, the code can be found at the link \footnote{\url{https://github.com/CurryTang/TSGFM/tree/master?tab=readme-ov-file}}.

As to the heterophily graphs, the four datasets are originally from~\cite{craven1998learning}. We obtain the raw texts from~\cite{yan2023comprehensive}, which can be found from\footnote{\url{https://github.com/sktsherlock/TAG-Benchmark/tree/master}}. As to data split, for zero-shot inference, all the nodes are marked as test data; for few-shot setting, $k$ labeled nodes are randomly sampled per class and the rests are marked as test data.
To the best of our knowledge, the four heterophily graph datasets used in this study are the only graphs that provide raw texts feature.





\subsection{LLM-BP Implementation Details}
\label{sec:app_llm_bp_implementation_details}
\textbf{Infrastructure and Seeds}
All the local experiments run on a server with AMD EPYC 7763 64-Core Processor and eight NVIDIA RTX 6000 Ada GPU cards, methods are mainly implemented with PyTorch~\cite{paszke2019pytorch}, Torch-Geometric~\cite{fey2019fast} and Huggingface Transformers~\cite{wolf2019huggingface}.
To obtain the embeddings, all the encoders that run locally on the server without API calling in this study run with the random seed $42$.


\textbf{Class Embedding}

$\bullet$ \textbf{Zero-Shot Setting:} We uniformly randomly sample $20 c$ nodes per graph, where $c$ denotes the number of classes, we employ GPT-4o~\cite{hurst2024gpt} to infer their labels. With the predictions from LLMs, the sampled nodes form distinct clusters. For each cluster, we take the top-$k$ (10 in the experiments) nodes whose embedding are closest with the cluster center and calculate their average embedding as the class embedding.

We notice that some works directly feed text descriptions into encoders as class embeddings~\cite{yang2024gl, chen2024text}, we find that different encoders can be highly sensitive to variations in text description. Therefore, we adopt the above method to ensure fairness among different encoders.


$\bullet$\textbf{Few-Shot Setting:} We directly take the class embedding as the averaged embeddings of labeled nodes per class.


\textbf{The Task-Adaptive Encoder:} We directly adopt the pre-trained LLM2Vec encoder released by~\cite{li2024making}, which is based on Mistral7B-v0.1~\cite{jiang2023mistral}. We check the pre-training data used in the original paper for aligning LLM decoders with the embedding space, the datasets are mainly for text-retrieval and therefore do not overlap with the TAG datasets adopted in our study. For detailed introduction of the datasets for LLM2Vec pre-training, see Section 4.1 training data in the original paper.
\label{sec:app_task_adaptive_implementation}
The task-adaptive prompting follows the format as:

\textit{
\text{$\langle$ Instruct $\rangle$}\\
``Given the \text{\{task description\}}, classify it into one of the following $k$ classes: \\
\text{\{class labels\}}\\
\text{$\langle$ query$\rangle$}\\
\text{\{raw node texts\}}.''\\
\text{$\langle$ response $\rangle$}\\}


, where the \{task descriptions\} prompts for each dataset is the same as that used for vanilla LLMs, see Table.~\ref{tab:vanilla_llm_task_description} for details. 


\textbf{Hyper-Parameters for BP algorithm}
For LLM-BP, we adopt $5$ message-passing layers, for its linear approximation form, we use a single layer.
The temperature hyper-parameter $\tau$ in computing node potential initialization in Eq.~\eqref{eq:node_potential} is set as $0.025$ for LLM-BP and $0.01$ for LLM-BP (appr.) across all the datasets. Attached in Table.~\ref{tab:pred_h} is the homophily ratio $r$ we used (predicted by GPT-4o-mini~\cite{hurst2024gpt}.
\begin{table}[h]
\centering
\resizebox{\textwidth}{!}{\begin{tabular}{@{}cccccccccccc@{}}
\toprule
 & Cora & Citeseer & Pubmed & History & Children & Sportsfit & Wikics & Cornell & Texas & Wisconsin & Washington \\ \midrule
\begin{tabular}[c]{@{}c@{}}Ground Truth \\ Homophony Ratio\end{tabular} & 0.81 & 0.76 & 0.79 & 0.66 & 0.46 & 0.90 & 0.67 & 0.11 & 0.06 & 0.15 & 0.19 \\
\begin{tabular}[c]{@{}c@{}}$r$ predicted by \\ GPT-4o-mini\end{tabular} & 0.70 & 0.81 & 0.81 & 0.73 & 0.35 & 0.81 & 0.52 & 0.05 & 0.04 & 0.06 & 0.02 \\ \bottomrule
\end{tabular}}
\vspace{-0.cm}
\caption{$r$ predicted by GPT-4o-mini, that is used in all the experiments in this study.}
\label{tab:pred_h}
\end{table}



\subsection{Baseline Implementation Details}
\label{sec:app_implementation_details_baseline}
$\bullet$ \textbf{Vanilla Encoders} Vanilla encoders like SBert~\cite{reimers2019sentence}, Roberta~\cite{liu2019roberta} and text-embedding-3-large~\cite{openai2024textembedding} directly encode the raw text of the nodes. LLM2Vec uses the prompts:

\begin{align}
\label{eq:vanilla_llm2vec}
\resizebox{0.43 \textwidth}{!}{$\langle\text{Instruct}\rangle \{\text{task\_description}\} \langle\text{query}\rangle {X_i} \langle\text{response}\rangle$}.
\end{align}, where the $\{\text{task\_description}\}$ for each dataset is provided in Appendix.~\ref{sec:app_prompt_llm2vec_task_description}.

$\bullet$ \textbf{Vanilla LLMs} Prompts for GPT-4o and GPT-3.5-turbo adopts the format as follows:

\label{sec:app_vanilla_LLM_implementation}
\textit{
``role'': ``system''\\
``content'': ``You are a chatbot who is an expert in text classification''\\
``role'': ``user''\\
``content'': ``We have \text{\{task description\}} from the following $k$ categories: \text{\{class labels\}}\\
The text is as follows:\\
\text{\{raw node text\}}\\
Please tell which category the text belongs to:''
}


The \{task description\} for the vanilla LLMs for each class is provided in Appendix.~\ref{sec:app_prompt_vanilla_llm}.


$\bullet$ \textbf{Tuning LM/GNNs}
We adopt the pre-trained UniGLM~\cite{fang2024uniglm} released by the official implementation, which adopts Bert as the encoder, for direct inference.
For ZeroG~\cite{li2024zerog}, we re-implement the method and train it on ogbn-arxiv~\cite{hu2020open} for fair comparison with other baselines.

As to GNNs tuning methods, we pre-train GraphMAE~\cite{hou2022graphmae} and DGI~\cite{velivckovic2018deep} on ogbn-arxiv~\cite{hu2020open}, where the input for both models are from SBert~\cite{reimers2019sentence}, and we follow in implementation in TSGFM~\cite{chen2024text} benchmark.

$\bullet$ \textbf{Multi-Task GFMs} OFA~\cite{liu2023one} is trained on ogbn-arxiv~\cite{hu2020open}. As to GOFA, we directly adopt the model after pre-training~\cite{hu2021ogb, ding2023enhancing} and instruct fine-tuning on ogbn-arxiv~\cite{hu2020open} provided by the authors due to the huge pre-training cost, the zero-shot inference scheme also follows their original implementation.


$\bullet$ \textbf{LLMs with Graph Adapters} Both LLaGA~\cite{chen2024llaga} and GraphGPT~\cite{tang2024graphgpt} are trained on ogbn-arxiv~\cite{hu2020open}, we follow the hyper-parameter setting in their original implementation.


\section{Detailed Derivations}
\label{sec:app_detailed_derivation}
% \subsection{Derivation for Eq.~\eqref{eq:message_passing}}
% Here we present the derivation towards the message passing form in Eq.~\ref{eq:message_passing}. In node classification task, given node $i$, the objective is to minimize the mean-square error (MSE) of the node label prediction:
% \begin{align}
%     \min \text{MSE} (\hat{y_i}) = \mathbb{E}[(y_y - \hat{y_i})^2 | \XX],
% \end{align}, with the optimal solution $\hat{y_i}$ as:
% \begin{align}
%     \hat{y_i} = \sum_{y_i} y_i p(y_i |X),
% \end{align}where $p(y_i |X) = \sum_{Y \setminus i} \mathbb{P}(Y | X)$ is the posterior marginal. 
% Given the factorized form of the posterior distribution under the Markov Random Field (MRF) modeling $\mathbb{P}_\mathcal{G}(Y \mid X) \propto \prod_{i \in \mathcal{V}} \varphi_{X_i}(y_i) 
% \prod_{(i,j) \in \mathcal{E}} \psi_{ij}(y_i, y_j)$, where $\varphi_{X_i}(y_i) = \varphi_{y_i}(X_i)\phi_i(y_i)$, one can approximate $p(y_i |X)$ by running the loopy belief propagation:

% \begin{align}
%     m_{j\rightarrow i}^{(k)}(y_i) \cong \sum_{y_j} \psi_{ij}(y_i, y_j) \frac{p_j^{(k-1)}(y_j)}{m_{i\rightarrow j}^{(k-1)}(y_j)} \\
%     p_i^{(k)}(y_i) \approx p_i^{(0)}(y_i) \prod_{j \in \mathcal{N}(i)} m_{j\rightarrow i}^{(t)}(y_i),
% \end{align}$m^{(0)}_{i \rightarrow j}$ is initialized as $1 / c$ for all $i,j \in \mathcal{V}$.

% Considering numerical stability, the belief-propagation is updated in log-space, which leads to the message passing formulation in Eq.~\eqref{eq:message_passing}:
% \begin{align}
%  \log m_{j \to i}^{(k)}(y_i) \cong &\, \text{LSE}_{y_j}[\log \psi_{ij}(y_i,y_j) + \\ &
% \log p_j^{(k-1)}(y_j) - \log m_{i \to j}^{(k-1)}(y_j)], \nonumber \\
%  \log p_i^{(k)}(y_i) \cong &\log p_i^{(0)}(y_i) + \sum_{j \in \mathcal{N}(i)} \log m_{j \to i}^{(k)}(y_i),  \nonumber
% \end{align}where LSE stands for the log-sum-exp function.

% \subsection{Derivation for Eq.~\eqref{eq:message_passing}}
% In a node classification task, given node \(i\), we aim to minimize the mean-square error (MSE) of the node label prediction under observations \(\XX\):
% \begin{align}
%     \min \mathrm{MSE}\bigl(\hat{y}_i\bigr) 
%     \;=\; 
%     \mathbb{E}\Bigl[\bigl(y_i - \hat{y}_i\bigr)^2 \,\Bigm|\;\XX\Bigr].
% \end{align}
% The optimal solution \(\hat{y}_i\) is then
% \begin{align}
%     \hat{y}_i 
%     \;=\; 
%     \sum_{y_i}\,y_i\,p\bigl(y_i \,\bigm|\;\XX\bigr),
% \end{align}
% where \(p\bigl(y_i \,\bigm|\;\XX\bigr) = \sum_{Y \setminus i}\,\mathbb{P}\bigl(Y \,\bigm|\;\XX\bigr)\) is the posterior marginal.

% Under a Markov Random Field (MRF) assumption with graph \(\mathcal{G}=(\mathcal{V},\mathcal{E})\), the posterior factors as:
% \begin{align}
%     \mathbb{P}_\mathcal{G}(Y \mid \XX) 
%     \;\propto\; 
%     \prod_{i \in \mathcal{V}}\varphi_{X_i}\bigl(y_i\bigr)
%     \,\prod_{(i,j)\in \mathcal{E}}\psi_{ij}\bigl(y_i,y_j\bigr),
% \end{align}
% where \(\varphi_{X_i}\bigl(y_i\bigr)=\varphi_{y_i}\bigl(X_i\bigr)\phi_i\bigl(y_i\bigr)\).  
% To approximate each \(p(y_i\mid \XX)\), one can run \emph{loopy belief propagation} (LBP).  
% Let
% \begin{align}
%     p_j^{(k-1)}\bigl(y_j\bigr) 
%     \;=\; 
%     p_j^{(0)}\bigl(y_j\bigr)\,
%     \prod_{\ell\in \mathcal{N}(j)} m_{\ell \to j}^{(k-1)}\bigl(y_j\bigr),
%     \label{eq:p_j_definition}
% \end{align}
% where \(p_j^{(0)}\bigl(y_j\bigr)\) is the initial node potential, and \(m_{\ell\to j}^{(k-1)}\) are incoming messages from neighbors \(\ell\in \mathcal{N}(j)\).  
% Then the new message from node \(j\) to node \(i\) at iteration \(k\) takes the form
% \begin{align}
%     m_{j \to i}^{(k)}\bigl(y_i\bigr)
%     \;=\; 
%     \sum_{y_j}\,\psi_{ij}\bigl(y_i,y_j\bigr)\,
%     \frac{p_j^{(k-1)}\bigl(y_j\bigr)}{m_{i \to j}^{(k-1)}\bigl(y_j\bigr)},
%     \label{eq:message_update}
% \end{align}
% where dividing by \(m_{i \to j}^{(k-1)}\bigl(y_j\bigr)\) removes the previous contribution from node \(i\) to \(j\) (which is already included in \(p_j^{(k-1)}(y_j)\)) to avoid double counting.  
% Once all messages are updated, the \emph{node belief} can be written as
% \begin{align}
%     p_i^{(k)}\bigl(y_i\bigr) 
%     \;=\; 
%     p_i^{(0)}\bigl(y_i\bigr)\,
%     \prod_{j \in \mathcal{N}(i)} m_{j \to i}^{(k)}\bigl(y_i\bigr).
% \end{align}
% For numerical stability, we apply the same process in log-space (i.e., \(\log\!\sum\exp\) rather than \(\sum\) and products), yielding the message passing equations in Eq.~\eqref{eq:message_passing}.

% \begin{align}
%     \log m_{j \to i}^{(k)}\bigl(y_i\bigr)
%     &\;\cong\;
%     \mathrm{LSE}_{y_j}
%     \Bigl[\,
%         \log \psi_{ij}\bigl(y_i,y_j\bigr)
%         \;+\;
%         \log p_j^{(k-1)}\bigl(y_j\bigr)
%         \;-\;
%         \log m_{i \to j}^{(k-1)}\bigl(y_j\bigr)
%     \Bigr],
%     \nonumber\\
%     \log p_i^{(k)}\bigl(y_i\bigr)
%     &\;\cong\;
%     \log p_i^{(0)}\bigl(y_i\bigr)
%     \;+\;
%     \sum_{j\in \mathcal{N}(i)}
%     \log m_{j \to i}^{(k)}\bigl(y_i\bigr),
%     \label{eq:message_passing_log}
% \end{align}
% where \(\mathrm{LSE}\{\cdot\}\) denotes the log-sum-exp operator, and \(m_{i\to j}^{(0)}\) is initialized as a constant (e.g., \(1/c\)).
\subsection{Derivation for Eq.~\eqref{eq:message_passing}}
% In a node classification task, given node \(i\), we aim to minimize the mean-square error (MSE) of the node label prediction under observations \(\XX\):
% \begin{align}
%     \min \mathrm{MSE}\bigl(\hat{y}_i\bigr) 
%     \;=\; 
%     \mathbb{E}\Bigl[\bigl(y_i - \hat{y}_i\bigr)^2 \,\Bigm|\;\XX\Bigr].
% \end{align}
% The optimal solution \(\hat{y}_i\) then follows from
% \begin{align}
%     \hat{y}_i 
%     \;=\; 
%     \sum_{y_i}\,y_i\,p\bigl(y_i \,\bigm|\;\XX\bigr),
% \end{align}
% where \(p\bigl(y_i \,\bigm|\;\XX\bigr) = \sum_{Y\setminus i}\,\mathbb{P}\bigl(Y \,\bigm|\;\XX\bigr)\) is the posterior marginal.

% \paragraph{Factorized Posterior under MRF.}
% Assume a Markov Random Field (MRF) with graph \(\mathcal{G}=(\mathcal{V},\mathcal{E})\).  Then the posterior distribution factors as
% \begin{align}
%     \mathbb{P}_\mathcal{G}(Y \mid \XX) 
%     \;\propto\; 
%     \prod_{i \in \mathcal{V}}\varphi_{X_i}\bigl(y_i\bigr)
%     \,\prod_{(i,j)\in \mathcal{E}}\psi_{ij}\bigl(y_i,y_j\bigr),
% \end{align}
% where \(\varphi_{X_i}\bigl(y_i\bigr)=\varphi_{y_i}\bigl(X_i\bigr)\phi_i\bigl(y_i\bigr)\).  
% To approximate each \(p\bigl(y_i\mid \XX\bigr)\), one can employ \emph{loopy belief propagation} (LBP).  

% \paragraph{Node Belief and Messages.}
% Define the (intermediate) node belief at iteration \((k-1)\) as
% \begin{align}
%     p_j^{(k-1)}\bigl(y_j\bigr) 
%     \;=\; 
%     p_j^{(0)}\bigl(y_j\bigr)\,
%     \prod_{\ell\in \mathcal{N}(j)} m_{\ell \to j}^{(k-1)}\bigl(y_j\bigr),
%     \label{eq:p_j_definition}
% \end{align}
% where \(p_j^{(0)}\bigl(y_j\bigr)\) is the initial potential for node \(j\), and \(m_{\ell\to j}^{(k-1)}\) denotes the incoming message from neighbor \(\ell\).  
% The new message from node \(j\) to node \(i\) at iteration \(k\) then reads
% \begin{align}
%     m_{j \to i}^{(k)}\bigl(y_i\bigr)
%     \;=\; 
%     \sum_{y_j}\,\psi_{ij}\bigl(y_i,y_j\bigr)\,
%     \frac{p_j^{(k-1)}\bigl(y_j\bigr)}{m_{i \to j}^{(k-1)}\bigl(y_j\bigr)},
%     \label{eq:message_update}
% \end{align}
% where dividing by \(m_{i \to j}^{(k-1)}\bigl(y_j\bigr)\) cancels out the old contribution from \(i\) to \(j\) (already included in \(p_j^{(k-1)}\bigl(y_j\bigr)\)), thus preventing double-counting.

% \paragraph{Recovering the Reduced Product.}
% Meanwhile, the node belief for node \(i\) at iteration \(k\) is
% \begin{align}
%     p_i^{(k)}\bigl(y_i\bigr) 
%     \;=\; 
%     p_i^{(0)}\bigl(y_i\bigr)\,
%     \prod_{\ell \in \mathcal{N}(i)} m_{\ell \to i}^{(k)}\bigl(y_i\bigr).
%     \label{eq:node_belief}
% \end{align}
% From \eqref{eq:node_belief}, one observes
% \begin{align}
%     \varphi_{X_i}\bigl(y_i\bigr) 
%     \,\prod_{\ell \in \mathcal{N}(i)\setminus \{j\}} m_{\ell \to i}^{(k)}\bigl(y_i\bigr)
%     \;=\;
%     \frac{
%       \varphi_{X_i}\bigl(y_i\bigr)
%       \,\prod_{\ell \in \mathcal{N}(i)} m_{\ell \to i}^{(k)}\bigl(y_i\bigr)
%     }{
%       m_{j \to i}^{(k)}\bigl(y_i\bigr)
%     }
%     \;=\;
%     \frac{
%       p_i^{(k)}\bigl(y_i\bigr)
%     }{
%       m_{j \to i}^{(k)}\bigl(y_i\bigr)
%     },
%     \label{eq:factor_rewrite}
% \end{align}
% which is precisely the reduced product over all neighbors except \(j\).

% \paragraph{Log-Space Update.}
% For numerical stability, we often take the logarithm of both sides in \eqref{eq:message_update} and \eqref{eq:node_belief}, i.e., using \(\log\sum\exp(\cdot)\) rather than direct sums and products. This procedure yields the message passing equations in Eq.~\eqref{eq:message_passing}, typically written as:
% \begin{align}
%     \log m_{j \to i}^{(k)}\bigl(y_i\bigr)
%     &\;\cong\;
%     \mathrm{LSE}_{y_j}
%     \Bigl[\,
%         \log \psi_{ij}\bigl(y_i,y_j\bigr)
%         \;+\;
%         \log p_j^{(k-1)}\bigl(y_j\bigr)
%         \;-\;
%         \log m_{i \to j}^{(k-1)}\bigl(y_j\bigr)
%     \Bigr],
%     \nonumber\\
%     \log p_i^{(k)}\bigl(y_i\bigr)
%     &\;\cong\;
%     \log p_i^{(0)}\bigl(y_i\bigr)
%     \;+\;
%     \sum_{j\in \mathcal{N}(i)}
%     \log m_{j \to i}^{(k)}\bigl(y_i\bigr),
%     \label{eq:message_passing_log}
% \end{align}
% where \(\mathrm{LSE}\{\cdot\} \equiv \log \sum \exp(\cdot)\) and \(m_{i\to j}^{(0)}\) is initialized as a constant (e.g., \(1/c\)). By iterating these updates, LBP approximates each node’s posterior marginal \(p(y_i\mid \XX)\), which in turn provides an estimate \(\hat{y}_i = \sum_{y_i} y_i \, p_i^{(k)}(y_i)\) for the MMSE criterion.
In a node classification task, given a node \(i\), our goal is to minimize the mean-square error (MSE) in predicting the node label under the observations \(\XX\):
\begin{align}
    \min \mathrm{MSE}\bigl(\hat{y}_i\bigr) 
    \;=\; 
    \mathbb{E}\Bigl[\bigl(y_i - \hat{y}_i\bigr)^2 \,\Bigm|\;\XX\Bigr].
\end{align}
The optimal solution \(\hat{y}_i\) is then given by:
\begin{align}
    \hat{y}_i 
    \;=\; 
    \sum_{y_i} y_i \, p\bigl(y_i \mid \XX\bigr),
\end{align}
where the posterior marginal \(p\bigl(y_i \mid \XX\bigr)\) is computed as:
\begin{align}
    p\bigl(y_i \mid \XX\bigr) = \sum_{Y \setminus i} \mathbb{P}\bigl(Y \mid \XX\bigr).
\end{align}

\paragraph{Factorized Posterior under an MRF.}
Assuming a Markov Random Field (MRF) over a graph \(\mathcal{G}=(\mathcal{V}, \mathcal{E})\), the posterior distribution factors as:
\begin{align}
    \mathbb{P}_\mathcal{G}(Y \mid \XX) 
    \;\propto\; 
    \prod_{i \in \mathcal{V}} \varphi_{X_i}\bigl(y_i\bigr)
    \prod_{(i,j) \in \mathcal{E}} \psi_{ij}\bigl(y_i, y_j\bigr),
\end{align}
where the node potential is defined as \(\varphi_{X_i}\bigl(y_i\bigr) = \varphi_{y_i}\bigl(X_i\bigr)\phi_i\bigl(y_i\bigr)\).

\paragraph{General Message-Passing Framework.}
To compute the marginal \(p(y_i \mid \XX)\), the loopy belief propagation (LBP) algorithm iteratively updates messages between nodes. The general message update rule from node \(i\) to node \(j\) at iteration \(k\) is:
\begin{align}
    m_{i \to j}^{(k)}\bigl(y_j\bigr) 
    \;=\; 
    \alpha_{i \to j} \sum_{y_i} \Bigg[\varphi_{X_i}\bigl(y_i\bigr) \psi_{ij}\bigl(y_i, y_j\bigr)
    \prod_{\ell \in \mathcal{N}(i) \setminus j} m_{\ell \to i}^{(k-1)}\bigl(y_i\bigr)\Bigg],
    \label{eq:general_message}
\end{align}
where \(\alpha_{i \to j}\) is a normalization constant ensuring the message sums to 1.

\paragraph{Node Belief Updates.}
The node belief \(p_i^{(k)}\bigl(y_i\bigr)\) at iteration \(k\) is obtained by combining the node potential with incoming messages from all neighbors:
\begin{align}
    p_i^{(k)}\bigl(y_i\bigr) 
    \;=\; 
    \varphi_{X_i}\bigl(y_i\bigr) 
    \prod_{\ell \in \mathcal{N}(i)} m_{\ell \to i}^{(k)}\bigl(y_i\bigr).
    \label{eq:node_belief_update}
\end{align}

\paragraph{Reformulating the Messages.}
Substituting Eq.~\eqref{eq:node_belief_update} into Eq.~\eqref{eq:general_message} simplifies the message-passing equation. The message from node \(i\) to node \(j\) at iteration \(k\) can be rewritten as:
\begin{align}
    m_{i \to j}^{(k)}\bigl(y_j\bigr) 
    \;=\; 
    \alpha_{i \to j} \sum_{y_i} \psi_{ij}\bigl(y_i, y_j\bigr) 
    \frac{p_i^{(k)}\bigl(y_i\bigr)}{m_{j \to i}^{(k-1)}\bigl(y_i\bigr)}.
    \label{eq:message_reformulation}
\end{align}
This reformulation prevents double-counting the contribution of node \(j\) to node \(i\) in the previous iteration.

\paragraph{Log-Space Stability.}
To avoid numerical underflow, the log-space version of the message update is commonly used:
\begin{align}
    \log m_{i \to j}^{(k)}\bigl(y_j\bigr) 
    &\;=\; 
    \mathrm{LSE}_{y_i} \Bigl[\log \psi_{ij}\bigl(y_i, y_j\bigr) + \log p_i^{(k)}\bigl(y_i\bigr) - \log m_{j \to i}^{(k-1)}\bigl(y_i\bigr)\Bigr],
\end{align}
where \(\mathrm{LSE}(\cdot) \equiv \log \sum \exp(\cdot)\).

\paragraph{Summary.}
By iteratively applying these message updates and node belief calculations, LBP provides an approximation for the posterior marginal \(p(y_i \mid \XX)\). The final prediction \(\hat{y}_i\) under the MMSE criterion is:
\begin{align}
    \hat{y}_i 
    \;=\; 
    \sum_{y_i} y_i \, p_i^{(k)}\bigl(y_i\bigr).
\end{align}
This completes the derivation of the message-passing update in Eq.~\eqref{eq:message_passing}.



\section{More Experiment Results}
\label{sec:app_more_experiment_results}


\subsection{Significance Test of Effectiveness of Task-Adaptive Encoding}
\begin{figure}[h]
    \centering
    \includegraphics[width=0.32\linewidth]{images/confidence/mmlu.pdf} \includegraphics[width=0.32\linewidth]{images/confidence/unintended_jigsaw.pdf}
    \includegraphics[width=0.32\linewidth]{images/confidence/commonsenseqa.pdf}
    \caption{Increasing the confidence threshold has no effect until a point, after which conformal consistency increases while the coverage tends to zero}
    \label{fig:confidence}
\end{figure}

\label{sec:app_confidence}

We conduct significance test on the improvment of task-adaptive encoding over vanilla LLm2Vec~\cite{li2024making} and Text-Embedding-3-Large~\cite{openai2024textembedding} under the zero-shot setting, with results shown in Table.~\ref{tab:confidence}. We replicate experiment for $100$ times with random seeds from $42$ to $141$ and obtain classification accuracy of each method. To check normality, we first apply Shapiro-Wilk test~\cite{SHAPIRO1965}. If the data follows a normal distribution, we perform a Paired-t test~\cite{student1908probable}; otherwise, we use Wilcoxon Signed-Rank test~\cite{wilcoxon1992individual}, with packages from SciPy~\cite{2020SciPy-NMeth}.
The lower and upper bounds under $90\%$ confidence interval are estimated with bootstrap algorithm~\cite{tibshirani1993introduction} to sample $10,000$ times.
Task-adaptive encoding show statistically significant improvement over vanilla LLM2Vec in $9$ out of $11$ dataset and outperforms Text-Embedding-3-Large in $8$ out of $11$ datasets (bolded in the table).

\subsection{LLM Agents' Prediction on Homophily Ratio $r$}
\begin{figure*}[h]
    \centering
    \hfill
    \begin{minipage}{0.23\textwidth}
    \captionsetup{labelformat=empty}
        \centering
        \includegraphics[width=\textwidth]{Arxiv/figures/pred_h/pred_h_GPT-4o.pdf}
        \vspace{-0.8cm}
        \caption*{\scriptsize GPT-4o.}
    \end{minipage}
    \hfill
    \begin{minipage}{0.23\textwidth}
    \captionsetup{labelformat=empty}
        \centering
        \includegraphics[width=\textwidth]{Arxiv/figures/pred_h/pred_h_GPT-4o-mini.pdf}
        \vspace{-0.8cm}
        \caption*{\scriptsize GPT-4o-mini.}
    \end{minipage}
    \begin{minipage}{0.23\textwidth}
    \captionsetup{labelformat=empty}
        \centering
        \includegraphics[width=\textwidth]{Arxiv/figures/pred_h/pred_h_GPT-3.5-turbo.pdf}
        \vspace{-0.8cm}
        \caption*{\scriptsize GPT-3.5-turbo.}
    \end{minipage}
    \begin{minipage}{0.23\textwidth}
    \captionsetup{labelformat=empty}
        \centering
        \includegraphics[width=\textwidth]{Arxiv/figures/pred_h/pred_h_Mistral7B-Instruct-v0.3.pdf}
        \vspace{-0.8cm}
        \caption*{\scriptsize Mistral7B-Instruct-v0.3.}
    \end{minipage}
    \vspace{-0.cm}
    \caption{LLM agents' performance on predicting the homophily constant $r$.}
    \label{fig:predict_h_more}
\end{figure*}
More prediction performance of GPT-4o, GPT-3.5-turbo and Mistral7b-Instruct-v3 are shown in Fig.~\ref{fig:predict_h_more}.

%\subsection{Effectiveness of Class Information on Text-Embedding-3-Large}
%The effectiveness of class information on text-embedding-3-large is shown in Fig.~\ref{fig:class_condition_text-embedding-3-large}.
%\input{Arxiv/figures/class_condition_text-embedding-3-large}

\subsection{Zero-Shot Comparison with LLM-GNN~\cite{chen2023label} and TEA-GLM~\cite{wang2024llms}}
\label{sec:app_more_baselines}
\begin{table}[h]
\centering
\begin{tabular}{ccccc}
\hline
 & Cora & Citeseer & Pubmed & Wikics \\ \hline
DA-AGE-W & 74.96 & 58.41 & 65.85 & 59.13 \\
DA-RIM-W & 74.73 & 60.80 & 77.94 & 68.22 \\
DA-GraphPart-W & 68.61 & 68.82 & 79.89 & 67.13 \\ \hline
LLM-BP & 72.59 & 69.51 & 75.55 & 67.75 \\
LLM-BP (app.) & 71.41 & 68.66 & 76.81 & 67.96 \\ \hline
\end{tabular}
\caption{Accuracy compared with LLM-GNN, where `DA' denotes the `C-Density' methods proposed in ~\cite{chen2023label}, `-W' refers to the weighted cross-entropy loss function used for training,
AGE~\cite{cai2017active}, RIM~\cite{zhang2021rim}, GraphPart~\cite{ma2022partition} are different graph active learning baselines used in the original paper.}
\label{tab:compare_llm_gnn}
\end{table}
\begin{table}[h]
\centering
\begin{tabular}{ccccc}
\hline
 & Cora & Pubmed & History & Children \\ \hline
TEA-GLM & 20.2 & 84.8 & 52.8 & 27.1 \\ \hline
LLM-BP & 72.59 & 75.55 & 59.86 & 24.81 \\
LLM-BP (app.) & 71.41 & 76.81 & 59.49 & 29.4 \\ \hline
\end{tabular}
\caption{Accuracy compared with TEA-GLM~\cite{wang2024llms}.}
\label{tab:compare_tea-glm}
\end{table}
Here we present the comparison with LLM-GNN~\cite{chen2023label} in Table.~\ref{tab:compare_llm_gnn}. We compare with three different graph active learning heuristics from their original paper. Our training-free methods, LLM-BP and LLM-BP (appr.) achieves top performance on Citeseer and Wikics, while performs comparably with the baselines in Cora and Pubmed. Note that the results of LLM-GNN are from Table. 2 in the original paper.


The comparison with TEA-GLM is shown in Table.~\ref{tab:compare_tea-glm}. Results of TEA-GLM are from Table.1 in their original paper.

\subsection{Experiment Results in Few-Shot Setting}
\label{sec:app_few_shot}
We use $10$ different random seeds from $42$ to $52$ to sample the $k$-shot labeled nodes from training dataset, and report the average accuracy and macro $F1$ score with standard variance. Results are shown in Table.~\ref{tab:few_shot}. Across all the $k$s, our LLM-BP achieves the top ranking performance across all the eleven datasets, exhibiting similar insights with the zero-shot setting.

\begin{table*}[!ht]
\setlength{\tabcolsep}{2pt}
\captionsetup{font=small}
  \centering
  \caption{Few-shot learning on 10\% training data. We use the same protocol in Table~\ref{tab:long_term_results}. All results are averaged from four different forecasting horizons: $H \in \{96, 192, 336, 720\}$. \boldres{Red}: best, \secondres{Blue}: second best. Our full results are in Appendix~\ref{appx:few_shot_details}.}
\scalebox{0.95}{
    \begin{tabular}{c|cc|cc|cc|cc|cc|cc|cc|cc|cc|cc}
    \toprule
    \multirow{2}{*}{Methods} & \multicolumn{2}{c|}{LDM4TS} & \multicolumn{2}{c|}{CSDI} & \multicolumn{2}{c|}{ScoreGrad} & \multicolumn{2}{c|}{Autoformer} & \multicolumn{2}{c|}{FEDformer} & \multicolumn{2}{c|}{DLinear} & \multicolumn{2}{c|}{Informer} & \multicolumn{2}{c|}{TimesNet} & \multicolumn{2}{c|}{LightTS} & \multicolumn{2}{c}{Reformer} \\
    & \multicolumn{2}{c|}{\textbf{(Ours)}} & \multicolumn{2}{c|}{\citeyearpar{tashiro2021csdi}} & \multicolumn{2}{c|}{\citeyearpar{song2020score}} & \multicolumn{2}{c|}{\citeyearpar{wu2021autoformer}} & \multicolumn{2}{c|}{\citeyearpar{zhou2022fedformer}} & \multicolumn{2}{c|}{\citeyearpar{zeng2023transformers}} & \multicolumn{2}{c|}{\citeyearpar{zhou2021informer}} & \multicolumn{2}{c|}{\citeyearpar{wu2022timesnet}} & \multicolumn{2}{c|}{\citeyearpar{campos2023lightts}} & \multicolumn{2}{c}{\citeyearpar{kitaev2020reformer}}\\
    \midrule
    Metric & MSE & MAE & MSE & MAE & MSE & MAE & MSE & MAE & MSE & MAE & MSE & MAE & MSE & MAE & MSE & MAE & MSE & MAE & MSE & MAE \\
    \midrule
    \textit{ETTh1} & \textcolor{red}{\textbf{0.471}} & \textcolor{red}{\textbf{0.468}} & 0.849 & 0.665 & 1.031 & 0.709 & 0.701 & 0.596 & \textcolor{blue}{\underline{0.638}} & \textcolor{blue}{\underline{0.561}} & 0.691 & 0.599 & 1.199 & 0.808 & 0.869 & 0.628 & 1.375 & 0.877 & 1.249 & 0.833\\
    \textit{ETTh2} & \textcolor{red}{\textbf{0.452}} & \textcolor{red}{\textbf{0.460}} & 0.527 & 0.523 & 0.512 & 0.505 & 0.488 & 0.499 & \textcolor{blue}{\underline{0.466}} & 0.475 & 0.608 & 0.538 & 3.871 & 1.512 & 0.479 & \textcolor{blue}{\underline{0.465}} & 2.655 & 1.159 & 3.485 & 1.486\\
    \textit{ETTm1} & \textcolor{red}{\textbf{0.371}} & \textcolor{red}{\textbf{0.393}} & 0.784 & 0.606 & 1.015 & 0.678 & 0.802 & 0.628 & 0.721 & 0.605 & \textcolor{blue}{\underline{0.411}} & \textcolor{blue}{\underline{0.429}} & 1.192 & 0.820 & 0.479 & 0.465 & 0.970 & 0.704 & 1.426 & 0.856 \\
    \textit{ETTm2} & 0.336 & 0.373 & 0.334 & 0.385 & 0.446 & 0.447 & 1.341 & 0.930 & 0.463 & 0.488 & \textcolor{red}{\textbf{0.316}} & \textcolor{blue}{\underline{0.368}} & 3.369 & 1.439 & \textcolor{blue}{\underline{0.319}} & \textcolor{red}{\textbf{0.353}} & 0.987 & 0.755 & 3.978 & 1.587 \\
    
    \textit{Weather} & \textcolor{red}{\textbf{0.229}} & \textcolor{red}{\textbf{0.276}} & 0.295 & 0.333 & 0.347 & 0.351 & 0.300 & 0.342 & 0.284 & \textcolor{blue}{\underline{0.283}} & \textcolor{blue}{\underline{0.241}} & \textcolor{blue}{\underline{0.283}} & 0.597 & 0.494 & 0.279 & 0.301 & 0.289 & 0.322 & 0.526 & 0.469\\
    
    \textit{ECL} & \textcolor{red}{\textbf{0.172}} & \textcolor{red}{\textbf{0.275}} & 0.909 & 0.785 & 1.258 & 0.884 & 0.431 & 0.478 & 0.346 & 0.428 & \textcolor{blue}{\underline{0.180}} & \textcolor{blue}{\underline{0.280}} & 1.194 & 0.891 & 0.323 & 0.392 & 0.441 & 0.488 & 0.980 & 0.769 \\
    
    \textit{Traffic} & \textcolor{red}{\textbf{0.621}} & \textcolor{red}{\textbf{0.357}} & 1.744 & 0.871 & 2.100 & 1.020 & 0.749 & 0.446 & \textcolor{blue}{\underline{0.663}} & \textcolor{blue}{\underline{0.425}} & 0.945 & 0.570 & 1.534 & 0.811 & 0.951 & 0.535 & 1.247 & 0.684 & 1.551 & 0.821\\
\midrule
    1st Count & 6 & 6 & 0 & 0 & 0 & 0 & 0 & 0 & 0 & 0 & 1 & 0 & 0 & 0 & 0 & 1 & 0 & 0 & 0 & 0\\
    \bottomrule
    \end{tabular}%
}
  \label{tab:few_shot}%
\end{table*}%
\subsection{Zero-Shot Link Prediction Results}
\begin{table}[h]
\centering
\setlength{\tabcolsep}{2pt}
\resizebox{1.0\textwidth}{!}{\begin{tabular}{c|ccc|cccc}
\hline
 & \multicolumn{3}{c|}{Citation Graph} & \multicolumn{4}{c}{E-Commerce \& Knowledge Graph} \\
 & Cora & Citeseer & Pubmed & History & Children & Sportsfit & Wikics \\ \hline
OFA & 0.492 & -- & 0.481 & 0.431 & 0.484 & 0.517 & -- \\
LLaGA & 0.527 & -- & 0.543 & 0.515 & 0.500 & 0.502 & -- \\
GraphGPT & 0.520 & -- & 0.569 & 0.449 & 0.422 & 0.597 & -- \\
TEA-GLM & 0.586 & -- & 0.689 & 0.579 & 0.571 & 0.553 & -- \\ \hline
SBert & 0.979±0.033 & 0.990±0.001 & 0.979±0.003 & 0.985±0.002 & 0.972±0.030 & 0.975±0.003 & 0.972±0.003 \\
Text-Embedding-3-Large & 0.975±0.003 & 0.989±0.002 & 0.979±0.003 & 0.985±0.001 & 0.980±0.002 & 0.987±0.001 & 0.977±0.002 \\
LLM2Vec & 0.966±0.004 & 0.982±0.002 & 0.970±0.003 & 0.971±0.002 & 0.973±0.003 & 0.975±0.002 & 0.978±0.003 \\ \hline
\end{tabular}}
\caption{Performance on zero-shot link prediction tasks (AUC). Results of baselines are from~\cite{wang2024llms}.}
\label{tab:link_prediction}
\end{table}
\label{sec:app_link_prediction}

For each dataset, We randomly sample \& remove $1000$ edges and $1000$ node pairs from the graph as testing data. A straightforward approach is to compare the cosine similarity between node embeddings to determine the presence of a link. Specifically, we aggregate embeddings for $3$ layers on the incomplete graph and compute the cosine similarity between node representations, achieving better zero-shot performance than LLMs-with-Graph-Adapters methods~\cite{wang2024llms, chen2024llaga, tang2024graphgpt}, as shown in Table.~\ref{tab:link_prediction}. Note that the performance in the table refers to LLM-with-Graph-Adapters that have only been trained on other tasks and never on link prediction tasks.

We leave the design of task-adaptive embeddings and generalized graph structural utilization for link prediction as future work, including task-adaptive encoding prompts.







\section{Prompts}
\subsection{Task Description for Vanilla LLM2Vec without Class Information}
\label{sec:app_prompt_llm2vec_task_description}
Table.~\ref{tab:llm2vec_task_description} shows the task description for vanilla LLM2Vec encoder across all the datasets.
\begin{table}[h]
\small
\centering
\begin{tabular}{@{}cc@{}}
\toprule
Dataset & Task Description \\ \midrule
Cora & Encode the text of machine learning papers: \\
Citeseer & Encode the description or opening text of scientific publications: \\
Pubmed & Encode the title and abstract of scientific publications: \\
History & Encode the description or title of the book: \\
Children & Encode the description or title of the child literature: \\
Sportsfit & Encode the title of a good in sports \& fitness: \\
Wikics & Encode the entry and content of wikipedia: \\
Cornell & Encode the webpage text: \\
Texas & Encode the webpage text: \\
Wisconsin & Encode the webpage text: \\
Washington & Encode the webpage text: \\ \bottomrule
\end{tabular}
\caption{\{Task description\} for vanilla LLM2Vec~\cite{li2024making} encoder. See Eq.~\ref{eq:vanilla_llm2vec} for detailed prompting format.}
\label{tab:llm2vec_task_description}
\end{table}

\subsection{Prompts for Vanilla LLMs}
\label{sec:app_prompt_vanilla_llm}
Table.~\ref{tab:vanilla_llm_task_description} shows tha task description for vanilla LLM decoders.
\begin{table}[h]
\small
\centering
\begin{tabular}{@{}cc@{}}
\toprule
Dataset & Task Description \\ \midrule
Cora & opening text of machine learning papers \\
Citeseer & description or opening text of scientific publications \\
Pubmed & title and abstract of scientific publications \\
History & description or title of the book \\
Children & description or title of the child literature \\
Sportsfit & the title of a good in sports \& fitness \\
Wikics & entry and content of wikipedia \\
Cornell & webpage text \\
Texas & webpage text \\
Wisconsin & webpage text \\
Washington & webpage text \\ \bottomrule
\end{tabular}
\caption{\{Task description\} in the prompts for both vanilla LLM decoders (See Section.~\ref{sec:app_vanilla_LLM_implementation}) and task-adaptive encoder (See Section.~\ref{sec:app_task_adaptive_implementation}).}
\label{tab:vanilla_llm_task_description}
\end{table}

%\subsection{Class Description}
%\label{sec:app_class_description}

\newpage


%\begin{table}[t]
\scriptsize
\begin{tabular}{|p{1.2cm}|p{15cm}|}
\toprule
Class & \multicolumn{1}{c}{Description} \\ \midrule
\begin{tabular}[c]{@{}c@{}}Case \\ Based\end{tabular} & Case-based research solves problems by retrieving and adapting past cases, emphasizing analogy-based reasoning. It integrates memory structures for incremental learning and leverages domain-specific knowledge for interpretability and adaptability, excelling in fields like medical diagnosis and legal reasoning. \\ \midrule
\begin{tabular}[c]{@{}c@{}}Genetic \\ Algorithms\end{tabular} & Genetic Algorithms (GA), inspired by evolutionary biology, use selection, crossover, and mutation to iteratively optimize solutions. Their stochastic, population-based search excels in large or rugged spaces.\\ \midrule
\begin{tabular}[c]{@{}c@{}}Neural \\ Networks\end{tabular} & Neural Networks (NNs) are models to learn representation from data. Paper of  deep neural networks explores tasks on large-scale datasets. Novel architectures, combining multiple convolutional layers, pooling operations, and fully connected layers, are proposed to efficiently extract hierarchical features. Additionally, techniques such as dropout regularization and data augmentation are integrated to enhance generalization. \\ \midrule
\begin{tabular}[c]{@{}c@{}}Probabilistic \\ Methods\end{tabular} & Probabilistic methods model uncertainty and reason under incomplete information using Bayes' theorem. Key components include prior distributions, likelihood functions, and posterior distributions, enabling systems to adapt as new data. Advanced techniques such as Markov Chain Monte Carlo (MCMC), variational inference, and Bayesian optimization can approximate complex posterior distributions. Gaussian processes, Bayesian networks, and hierarchical models excel in tasks requiring uncertainty quantification, small-data generalization, and interpretable probabilistic predictions. Applications include model calibration, decision-making under ambiguity, and domains such as medical diagnosis, sensor fusion, and automated experimentation. \\ \midrule
\begin{tabular}[c]{@{}c@{}}Reinforcement \\ Learning\end{tabular} & Reinforcement Learning (RL) trains agents to optimize decisions by maximizing cumulative rewards in a Markov Decision Process (MDP). Key concepts include value functions, Q-learning, policy gradients, and the exploration-exploitation trade-off. Algorithms like DQN, Actor-Critic, and PPO solve high-dimensional decision problems. RL tackles challenges such as reward sparsity, credit assignment, and uncertainty, with applications in game playing, robotics, resource allocation, and autonomous systems. \\ \midrule
\begin{tabular}[c]{@{}c@{}}Rule \\ Learning\end{tabular} & Rule Learning involves discovering interpretable if-then rules from data to represent patterns or decision processes. Its uniqueness is its ability to produce human-readable and explainable models. Algorithms such as RIPPER and CN2 emphasize learning concise and accurate rule sets. \\ \midrule
Theory & Machine learning theory establishes the mathematical foundations and guarantees of learning algorithms, ensuring reliability, efficiency, and generalization. Key areas include statistical learning theory (model generalization) and computational learning theory (learning feasibility). Core concepts include VC dimension, PAC learning, generalization bounds, and convergence guarantees. Research analyzes overfitting, underfitting, bias-variance trade-offs, and sample complexity. \\ \bottomrule
\end{tabular}
\vspace{-0.3cm}
\caption{Class descriptions for Cora~\cite{mccallum2000automating}.}
\label{tab:class_description_cora}
\end{table}

%\begin{table}[h]
\scriptsize
\begin{tabular}{|p{1.5cm}|p{14.5cm}|}
\toprule
Class & Description \\ \midrule
Agents & The study focus on computational agents capable of rational behavior. Modeling agents requires to provide a structured approach to decision-making based on an agent's beliefs, desires, and intentions. Agents are grounded in both quantitative decision-theoretic models and symbolic reasoning approaches, enabling them to operate effectively in complex dynamic environments.     \\ \midrule
\begin{tabular}[c]{@{}l@{}}machine \\ learning(ML)\end{tabular} & Machine learning includes lazy learning, probabilistic learning, and case-based reasoning. Machine learning is studied in cognitive science as a model of human reasoning and in inductive learning for similarity-based generalization. key challenges lie in defining effective measures, evolving from rigid metrics to adaptive, context-aware approaches, such as non-symmetric similarity in structured learning. \\ \midrule
\begin{tabular}[c]{@{}l@{}}information \\ retrieval (IR)\end{tabular} & is the process of retrieving relevant information from large datasets, typically text collections. Methods include centroid-based classification, Naïve Bayes, k-NN, and decision trees by leveraging similarity measures to adjust for class density and term dependencies. Beyond supervised classification, unsupervised categorization aids automated retrieval, enabling dynamic query generation and document filtering. \\ \midrule
database (DB) & Databases manage structured data efficiently, especially in distributed environments, with semantic query caches that improve query performance by storing results and optimizing resource use. Database integration enables access to autonomous, heterogeneous sources through a unified schema. \\ \midrule
\begin{tabular}[c]{@{}l@{}}human-computer \\ interaction (HCI)\end{tabular} & Human-Computer Interaction (HCI) creates intuitive, adaptive interfaces for real-time collaboration and multimodal interaction. Customization supports tailored workflows in group collaboration, while modern systems enhance adaptation and learning through shared workspaces for real-time component importation. \\ \midrule
\begin{tabular}[c]{@{}l@{}}artificial \\ intelligence (AI)\end{tabular} & AI enables autonomous decision making to operate in dynamic environments with monitored execution. AI-driven learning and decision making aim to produce human-like performance by detecting and correcting discrepancies between internal models and real-world conditions.  \\ \bottomrule
\end{tabular}
\vspace{-0.3cm}
\caption{Class Description for Citeseer~\cite{giles1998citeseer}.}
\label{tab:class_description_citeseer}
\end{table}

%\begin{table}[h]
\scriptsize
\begin{tabular}{|p{2cm}|p{14.5cm}|}
\toprule
Class & Description \\ \midrule
\begin{tabular}[c]{@{}l@{}}Diabetes Mellitus \\ Experimental\end{tabular} & Experimental studies on  diabetic rats reveal molecular mechanisms of diabetes-related complications. Research focuses on ion transport regulation, including isoform- and tissue-specific changes, increased vascular permeability, assessed via  albumin permeation, marks diabetic damage in eyes, kidneys, and nerves. Key interventions include aldose reductase inhibitors (sorbinil, tolrestat) and castration.  \\ \midrule
\begin{tabular}[c]{@{}l@{}}Diabetes Mellitus \\ Type 1\end{tabular} & Type 1 Diabetes Mellitus (T1DM) is an autoimmune disease characterized by the destruction of insulin-producing $\beta$-cells, leading to insulin deficiency and lifelong dependence on exogenous insulin therapy. Genetic predisposition, particularly related with HLA-DR, plays a crucial role in disease susceptibility and is related with the presence of insulin autoantibodies (IAAs). Key methodologies included intravenous glucose tolerance tests (IVGTT), serological tests for ICAs and IAAs, and HLA-DQ genotyping, which collectively helped identify predictive markers for the progression to Type 1 diabetes.  \\ \midrule
\begin{tabular}[c]{@{}l@{}}Diabetes Mellitus \\ Type 2\end{tabular} & Type 2 Diabetes Mellitus (T2DM) is a complex metabolic disorder characterized by insulin resistance, impaired insulin secretion, and progressive glucose intolerance. The progression of normal glucose tolerance to impaired is mainly induced by insulin resistance,  and can also be contributed by $\beta$-cell dysfunction, and ultimately lead to diabetes. Individuals exhibiting low insulin sensitivity (M-value) and diminished acute insulin response (AIR) face a significantly higher risk of developing T2DM, underscoring the need for targeted interventions at different disease stages. T2DM progression is associated with changes in adipokine levels, lipid metabolism, and advanced glycation end-products (AGEs). \\ \bottomrule
\end{tabular}
\caption{Class decription for Pubmed~\cite{sen2008collective}.}
\label{tab:class_description_pubmed}
\end{table}
%\begin{table}[h]
\small
\begin{tabular}{|p{2cm}|p{14.5cm}|}
\toprule
Class & Description \\ \midrule
Other Sports & Explore a diverse range of sporting activities and equipment beyond the mainstream. From archery to skateboarding, roller skating to juggling, this category caters to niche and unconventional sports enthusiasts. \\ \midrule
Golf & Improve your swing and elevate your golf game with top-tier golf equipment and accessories. Explore clubs, balls, bags, carts, and apparel designed for maximum performance and enjoyment on the course. \\ \midrule
\begin{tabular}[c]{@{}l@{}}Hunting \\ \& Fishing\end{tabular} & Gear up for your next outdoor adventure with top-quality hunting and fishing equipment. Discover firearms, bows, fishing rods, reels, lures, and all the essential accessories for a successful and enjoyable experience. \\ \midrule
\begin{tabular}[c]{@{}l@{}}Exercise \\ \& Fitness\end{tabular} & Stay motivated and achieve your fitness goals with a wide selection of exercise and fitness equipment. Find treadmills, ellipticals, weights, yoga mats, and more to support your active lifestyle and wellness journey. \\ \midrule
Team Sports & Unite with your teammates and showcase your skills with top-quality team sports equipment. Find balls, goals, protective gear, and apparel for sports like basketball, football, soccer, baseball, and more. \\ \midrule
Accessories & Enhance your sports and fitness experience with a vast array of accessories. From hydration packs and wearable tech to protective gear, storage solutions, and more, these products ensure you're fully equipped for any activity. \\ \midrule
Swimming & Make a splash with swimming gear designed for comfort, performance, and safety. Explore swimsuits, goggles, kickboards, pool accessories, and more to enhance your swimming routine and aquatic workouts. \\ \midrule
\begin{tabular}[c]{@{}l@{}}Leisure Sports \\ \& Game Room\end{tabular} & Bring the fun and excitement of leisure sports into your home with game room equipment. Explore pool tables, air hockey, foosball, darts, and other recreational games for endless entertainment and quality family time. \\ \midrule
\begin{tabular}[c]{@{}l@{}}Airsoft \\ \& Paintball\end{tabular} & Experience the thrill of simulated combat with airsoft and paintball gear. Discover replica firearms, protective gear, ammunition, and accessories for intense and adrenaline-fueled games and competitions. \\ \midrule
\begin{tabular}[c]{@{}l@{}}Boating \\ \& Sailing\end{tabular} & Set sail for thrilling adventures on the water with boating and sailing gear. Discover boats, kayaks, canoes, paddleboards, life jackets, and essential accessories for safe and enjoyable aquatic experiences. \\ \midrule
Sports Medicine & Prioritize your health and recovery with sports medicine equipment and supplies. Explore braces, supports, therapy tools, and rehabilitation gear to prevent injuries, manage pain, and enhance your overall athletic performance. \\ \midrule
\begin{tabular}[c]{@{}l@{}}Tennis \\ \& Racquet Sports\end{tabular} & Serve up your best game with high-quality tennis and racquet sports equipment. Find rackets, balls, court accessories, and apparel for sports like tennis, badminton, squash, and more. \\ \midrule
Clothing & Stay comfortable and stylish while engaging in your favorite sports and activities with performance-driven clothing. Find athletic apparel, shoes, and accessories designed for optimal mobility and breathability. \\ \bottomrule
\end{tabular}
\caption{Class Description for Sportsfit~\cite{ni2019justifying}.}
\label{tab:class_description_sportsfit}
\end{table}
%\begin{table}[h]
\small
\begin{tabular}{|p{1cm}|p{15cm}|}
\toprule
Class & Description \\ \midrule
student & encompasses individuals actively enrolled in educational programs, ranging from undergraduate to graduate levels, across diverse disciplines. These individuals engage in academic activities, such as attending lectures, completing assignments, and participating in research or extracurricular projects, contributing significantly to the institution's learning environment. \\ \midrule
project & structured academic or extracurricular initiatives undertaken to achieve specific learning or research goals. Projects may vary in scope, involving individual or group work, and often integrate practical applications of theoretical concepts, culminating in presentations, reports, or tangible outcomes \\ \midrule
course & represents the structured curriculum designed to deliver specific knowledge and skills within a discipline. Courses typically include lectures, discussions, assignments, and assessments, providing students with the foundational and advanced understanding needed to excel in their fields of study. \\ \midrule
staff & non-academic personnel who support the institution's daily operations, such as administrative tasks, facilities management, and student services. Their contributions ensure a well-functioning environment conducive to education and research. \\ \midrule
faculty & academic professionals responsible for teaching, mentoring, and conducting research. Faculty members play a critical role in shaping the educational experience, fostering intellectual growth, and advancing knowledge within their respective disciplines. \\ \bottomrule
\end{tabular}
\caption{Class Description for Cornell, Texas, Wisconsin and Washington~\cite{craven1998learning}.}
\label{tab:class_description_cornell_texas_wisconsin_washington}
\end{table}
%\begin{table}[h]
\scriptsize
\begin{tabular}{|p{2cm}|p{14.0cm}|}
\toprule
Class & Description \\ \midrule
World & Provides rich insights into the historical, theological, and cultural dimensions of religions and their societal impacts and their shared and distinct traditions through primary and secondary sources, fostering comparative understanding. Together, these works illuminate global religious and historical knowledge. \\ \midrule
Americas & explore North America that comprises distinct cultural nations, each with unique historical roots, analyze how regions maintain their distinct identities and values, influencing political and social dynamics, including electoral outcomes and legislative behavior and pivotal moments in American history. Such narratives delve into the lives of ordinary people impacted by these developments, revealing the paradox of progress and the drive to assert the United States' role in shaping the modern world. \\ \midrule
Asia & Explores Asian history and military wars, focusing on military strategy, soldier experiences, and atrocities. Analyzes expansion and war's impact like Chosin Reservoir or the 1937 Battle for Nanjing,on society and global geopolitics. \\ \midrule
Military & analyze pivotal battles, campaigns, and military technologies. They cover air campaign strategies, ground operations in tough terrains, and airborne missions, detailing coordination and challenges. Combining historical rigor with visual and analytical insights, they illuminate key aspects of military history. \\ \midrule
Europe & explore European battles, key figures, and cultural developments, analyze soldier experiences at Agincourt, Waterloo, Somme, etc offering insights into warfare and leadership. Eyewitness accounts provide vivid perspectives on pivotal events that have shaped Europe. \\ \midrule
Russia & examine Russia’s tumultuous history, covering from the Civil War’s human toll, to Soviet rule’s establishment, analyze the Russian Revolution, and insights into Russia’s political evolution, ideological struggles, and global impact across different historical periods. \\ \midrule
Africa & examine Africa’s military, cultural, and historical dynamics, covering Cold War-era conflicts, global influence, and firsthand accounts. They explore the rise and fall of iconic tribes, blending history with cultural perspectives, and analyze African history through archaeology, revealing ancient states’ enduring legacies. \\ \midrule
\begin{tabular}[c]{@{}l@{}}Ancient \\ Civilizations\end{tabular} & explore early human societies, their cultures, innovations, and key figures. They examine empires like Ancient Egypt, ancient civilizations from Mesopotamia to Rome, and the roles of religion, law, and military power. Additionally, they detail travel, trade, and tourism, highlighting connections between distant cultures. \\ \midrule
Middle East & explore the Middle East’s historical, political, and cultural complexities, covering identity, conflict, and transformation, analyze national myths in politics, provide geopolitical insights with maps and data, and examine the roots of extremism, ideology, and violence, offering a comprehensive view of the region. \\ \midrule
\begin{tabular}[c]{@{}l@{}}Historical Study \\ \& \\ Educational Resources\end{tabular} & provide insights into historical methods, narratives, and interpretations across eras and region, including firsthand accounts of key events, analyses of historiography, and studies of specific periods like WWII, combining research with accessible narratives. Valuable for students and historians, they enhance historical understanding and teaching. \\ \midrule
Australia \& Oceania & explore Australia and Oceania’s history, cultures, and environment, from WWII campaigns in New Guinea, eto strategic battles in the Solomon Islands, offering detailed perspectives on pivotal events shaping the region. \\ \midrule
Arctic \& Antarctica & explore Arctic and Antarctic challenges, historic expeditions, and human resilience. They recount triumphs and tragedies of polar exploration, the race to the South Pole, and lost Arctic journeys, using archaeological and historical insights to reveal survival, discovery, and endurance in extreme conditions. \\ \bottomrule
\end{tabular}
\caption{Class Description for History~\cite{ni2019justifying}.}
\label{tab:class_description_bookhis}
\end{table}
%\begin{table}[h]
\scriptsize
\begin{tabular}{|p{2.5cm}|p{13.5cm}|}
\toprule
Class & Description \\ \midrule
\begin{tabular}[c]{@{}c@{}}Literature \\ \& Fiction\end{tabular} & Explore the wonderful world of children's literature and fiction books. From heartwarming tales to imaginative adventures, these books spark creativity and foster a love for reading. \\ \midrule
Animals & Learn about the fascinating world of animals through engaging stories and vibrant illustrations. Discover different species, habitats, and the importance of caring for our furry and feathered friends. \\ \midrule
\begin{tabular}[c]{@{}c@{}}Growing Up \\ \& Facts of Life\end{tabular} & Navigate the journey of growing up with age-appropriate books that address physical and emotional changes, relationships, and important life lessons with sensitivity and wisdom. \\ \midrule
Humor & Unleash the power of laughter with hilarious books that tickle the funny bone. Filled with puns, jokes, and silly scenarios, these reads are sure to bring endless giggles. \\ \midrule
\begin{tabular}[c]{@{}c@{}}Cars Trains \\ \& Things That Go\end{tabular} & Vroom! Choo-choo! Explore the exciting world of transportation with engaging books about cars, trains, planes, and more. Perfect for little ones fascinated by wheels and motion. \\ \midrule
\begin{tabular}[c]{@{}c@{}}Fairy Tales Folk Tales \\ \& Myths\end{tabular} & Dive into the enchanting realm of fairy tales, folk tales, and myths from around the world. These timeless stories ignite imagination and teach valuable life lessons. \\ \midrule
\begin{tabular}[c]{@{}c@{}}Activities Crafts \\ \& Games\end{tabular} & Unleash creativity and have fun with activity books, crafts projects, and games. From coloring to origami, these books provide endless entertainment and learning opportunities. \\ \midrule
\begin{tabular}[c]{@{}c@{}}Science Fiction \\ \& Fantasy\end{tabular} & Embark on epic adventures and explore imaginative worlds in captivating science fiction and fantasy tales. These books inspire curiosity and transport readers to realms beyond our own. \\ \midrule
Classics & Discover beloved literary treasures that have stood the test of time. These classic children's books are cherished for their timeless themes, memorable characters, and enduring lessons. \\ \midrule
\begin{tabular}[c]{@{}c@{}}Mysteries \\ \& Detectives\end{tabular} & Sharpen your problem-solving skills with thrilling mysteries and detective stories. Follow the clues, unravel the puzzles, and experience the excitement of solving cases. \\ \midrule
\begin{tabular}[c]{@{}c@{}}Action \\ \& Adventure\end{tabular} & Get your adrenaline pumping with heart-pounding action and adventure tales. These page-turners take readers on daring quests, thrilling escapades, and courageous journeys. \\ \midrule
\begin{tabular}[c]{@{}c@{}}Geography \\ \& Cultures\end{tabular} & Explore the rich tapestry of our world through fascinating books on geography and cultures. Discover new lands, traditions, and gain a deeper appreciation for diversity. \\ \midrule
\begin{tabular}[c]{@{}c@{}}Education \\ \& Reference\end{tabular} & Enhance learning and knowledge with educational and reference books. From encyclopedias to study guides, these resources support academic growth and intellectual curiosity. \\ \midrule
\begin{tabular}[c]{@{}c@{}}Arts Music \\ \& Photography\end{tabular} & Ignite creativity and appreciation for the arts with engaging books on music, visual arts, and photography. These books nurture artistic expression and cultural awareness. \\ \midrule
\begin{tabular}[c]{@{}c@{}}Holidays \\ \& Celebrations\end{tabular} & Embrace the joy and traditions of holidays and celebrations from around the world. These festive books provide insights into customs, histories, and the spirit of special occasions. \\ \midrule
\begin{tabular}[c]{@{}c@{}}Science Nature \\ \& How It Works\end{tabular} & Discover the wonders of science, nature, and how things work through captivating explanations and vibrant visuals. These books foster curiosity and a love for learning. \\ \midrule
Early Learning & Lay a strong foundation for learning with engaging books designed to support early childhood development. From ABCs to numbers, these resources promote essential skills. \\ \midrule
Biographies & Explore the remarkable lives of inspiring individuals through engaging biographies. These books celebrate achievements, perseverance, and the contributions of notable figures. \\ \midrule
History & Step back in time and learn about the events and people that shaped our world. These history books bring the past to life and nurture a deeper understanding of our heritage. \\ \midrule
Children's Cookbooks & Unleash your inner chef with fun and educational cookbooks designed specifically for young culinary enthusiasts. Learn cooking basics and create delicious dishes. \\ \midrule
Religions & Foster understanding and respect for diverse religious beliefs and traditions through insightful books on world religions, faiths, and spiritual practices. \\ \midrule
\begin{tabular}[c]{@{}c@{}}Sports \\ \& Outdoors\end{tabular} & Celebrate the spirit of sportsmanship and adventure with exciting books on sports, outdoor activities, and athletic pursuits. These reads inspire an active and healthy lifestyle. \\ \midrule
\begin{tabular}[c]{@{}c@{}}Comics \\ \& Graphic Novels\end{tabular} & Immerse yourself in the dynamic world of comics and graphic novels. These visually stunning books blend art and storytelling, captivating readers of all ages. \\ \midrule
\begin{tabular}[c]{@{}c@{}}Computers \\ \& Technology\end{tabular} & Discover the fascinating world of computers and technology through engaging books that explain concepts, coding, and the role of technology in our lives. \\ \bottomrule
\end{tabular}
\caption{Class Descriptions for Children~\cite{ni2019justifying}.}
\label{tab:class_description_bookchild}
\end{table}
%\begin{table}[h]
\scriptsize
\begin{tabular}{|p{2cm}|p{14cm}|}
\toprule
Class & Description \\ \midrule
\begin{tabular}[c]{@{}l@{}}computational \\ linguistics\end{tabular} & study of language using computational methods, focusing on the process of grouping together inflected forms of a word and identifying its dictionary or base form, known as the lemma. Computing linguistics simplify words by removing affixes, or ensure accurate base word selection based on grammatical and semantic context.   \\ \midrule
databases & structured system for storing, managing, and retrieving data, often accessed through a database server, such as those using DBMSs like MySQL or Oracle, to handle query processing, data analysis, and storage. Databases are designed for networked environments and often implement master-slave models for data redundancy and load balancing.  \\ \midrule
\begin{tabular}[c]{@{}l@{}}operating \\ systems\end{tabular} & Operating systems serve as the interface between hardware and software, encapsulate resources as objects, enabling inheritance and polymorphism to simplify system design and extend functionality. A file or device driver can be represented as an object, abstracting the implementation details while allowing access through defined methods. Each object can enforce access control and limited privileges based on user roles. Modern operating systems incorporate designs in components like kernels and user interfaces, leveraging these principles for improved modularity, maintainability, and resource protection.  \\ \midrule
\begin{tabular}[c]{@{}l@{}}computer \\ architecture\end{tabular} & Computer architecture defines the organization and design of a system's hardware, optimizing performance, efficiency, and compatibility. Earlier 32-bit architectures with the x86 instruction set featured 32-bit general-purpose registers(e.g., EAX, EBX) and supported 32-bit integer arithmetic, logical operations, and memory addressing. This was later succeeded by the era of 64-bit architectures, such as x86-64, offering enhanced processing capabilities. \\ \midrule
\begin{tabular}[c]{@{}l@{}}computer \\ security\end{tabular} & practice of protecting computing devices, networks, and data from unauthorized access, modification, or destruction. Core aspects is authentication, which verifies users' identities, and access control, which restricts unauthorized access to resources. Techniques such as encryption and secure data transmission safeguard sensitive information from interception.  \\ \midrule
\begin{tabular}[c]{@{}l@{}}internet \\ protocols\end{tabular} & a set of rules and conventions enabling communication between devices in a network, consisting of Internet Layer in the Internet Protocol Suite, which facilitates the internetworking of devices across network boundaries. The Internet Layer uses protocols like IPv4 and IPv6 to handle packet transmission, addressing, and routing, ensuring packets reach their intended destinations with functionalities like packet fragmentation, path MTU discovery, and error reporting via internet control message protocol. \\ \midrule
\begin{tabular}[c]{@{}l@{}}computer \\ file systems\end{tabular} & organizes, stores, and retrieves data on storage devices, providing the structure necessary to access and manage files. File System enables seamless interaction with local, network, or remote file systems without requiring applications to know the underlying file system type. The abstraction is achieved through an interface contract between the operating system's kernel and concrete file systems, simplifying the addition of support for new file system types.  \\ \midrule
\begin{tabular}[c]{@{}l@{}}distributed \\ computing \\ architecture\end{tabular} & Distributed computing coordinates multiple independent systems using a middleware layer for communication, integration, and management between distributed applications and operating systems. It abstracts data exchange, resource allocation, and process synchronization, enabling client-server or peer-to-peer communication. Technologies like web servers, application servers, and ESBs support seamless interaction, while ODBC and JDBCensure data integration across heterogeneous systems. Widely used in enterprise, telecom, and aerospace, it bridges modern and legacy systems, enabling transaction management and message-oriented communication while enhancing scalability, reliability, and efficiency. \\ \midrule
\begin{tabular}[c]{@{}l@{}}web \\ technology\end{tabular} & enables the creation and delivery of content and services over the World Wide Web, with web servers playing a central role. Web servers support static files and dynamic content generation using server-side scripting languages like PHP or ASP, enabling integration with databases for real-time data retrieval. Widely used web servers like Apache, Nginx, and Microsoft IIS dominate the market, with each optimized for specific use cases, such as handling high traffic or supporting embedded systems in networked devices. \\ \midrule
\begin{tabular}[c]{@{}l@{}}programming \\ language topics\end{tabular} & enables developers to write instructions for computers, which are executed through a programming language implementation. Implementations translate high-level code into a form the computer can execute, primarily through compilers or interpreters. Modern implementations often blend both approaches, such as bytecode compilation followed by interpretation or just-in-time (JIT) compilation, as seen in languages like Java or Python.  \\ \bottomrule
\end{tabular}
\caption{Class Description for Wikics~\cite{mernyei2020wiki}.}
\label{tab:class_description_sportsfit}
\end{table}



\end{document}
