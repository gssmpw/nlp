We conduct ablation studies on batch size to investigate its effect on task performance. Results verify previous work \cite{SimCLR} and theory \cite{SogCLR} that suggests larger batch sizes yield better performance in the contrastive setting. As shown in Table 3, increasing the batch size from 256 to 4096 leads to noticeable and consistent improvements in both individual metrics and the overall average performance. The best results are achieved at the largest batch size of 4096 (Myna-Base), indicating that larger batch sizes are beneficial for achieving optimal performance.

\begin{table*}[h]
\centering
\begin{tabular}{lcccccccc}
\toprule
\textbf{Approach} & \multicolumn{2}{c}{\textbf{Tags}} & \textbf{Genre} & \textbf{Key} & \multicolumn{2}{c}{\textbf{Emotion}} & \textbf{Average} \\
                  & \textbf{MTT\textsubscript{AUC}} & \textbf{MTT\textsubscript{AP}} & \textbf{GTZAN} & \textbf{GS} & \textbf{Emo\textsubscript{A}} & \textbf{Emo\textsubscript{V}} &  \\
\midrule
\midrule
Batch size 256 & 90.1 & 38.0 & 75.2 & 60.4 & 68.3 & 52.5 & 65.0 \\
Batch size 512 & 90.3 & 38.3 & 74.5 & 60.7 & 72.4 & 54.5 & 65.7 \\
Batch size 1024 & 90.4 & 38.8 & 74.1 & 61.8 & 69.9 & 56.3 & 65.9 \\
Batch size 2048 & 90.7 & 39.2 & 77.6 & 63.3 & 70.1 & 54.2 & 67.0 \\
Myna-Base (4096) & 90.8 & 39.5 & 78.3 & 63.5 & 73.5 & 55.8 & 67.9 \\

\bottomrule
\end{tabular}
\caption{Performance metrics across various tasks with increasing batch sizes for Myna-Base ($16 \times 16$ patches).}
\end{table*}