% ICCV 2025 Paper Template

\documentclass[10pt,twocolumn,letterpaper]{article}

%%%%%%%%% PAPER TYPE  - PLEASE UPDATE FOR FINAL VERSION
% \usepackage{iccv}              % To produce the CAMERA-READY version
% \usepackage[review]{iccv}      % To produce the REVIEW version
\usepackage[pagenumbers]{iccv} % To force page numbers, e.g. for an arXiv version

% Import additional packages in the preamble file, before hyperref
%
% --- inline annotations
%
\newcommand{\red}[1]{{\color{red}#1}}
\newcommand{\todo}[1]{{\color{red}#1}}
\newcommand{\TODO}[1]{\textbf{\color{red}[TODO: #1]}}
% --- disable by uncommenting  
% \renewcommand{\TODO}[1]{}
% \renewcommand{\todo}[1]{#1}



\newcommand{\VLM}{LVLM\xspace} 
\newcommand{\ours}{PeKit\xspace}
\newcommand{\yollava}{Yo’LLaVA\xspace}

\newcommand{\thisismy}{This-Is-My-Img\xspace}
\newcommand{\myparagraph}[1]{\noindent\textbf{#1}}
\newcommand{\vdoro}[1]{{\color[rgb]{0.4, 0.18, 0.78} {[V] #1}}}
% --- disable by uncommenting  
% \renewcommand{\TODO}[1]{}
% \renewcommand{\todo}[1]{#1}
\usepackage{slashbox}
% Vectors
\newcommand{\bB}{\mathcal{B}}
\newcommand{\bw}{\mathbf{w}}
\newcommand{\bs}{\mathbf{s}}
\newcommand{\bo}{\mathbf{o}}
\newcommand{\bn}{\mathbf{n}}
\newcommand{\bc}{\mathbf{c}}
\newcommand{\bp}{\mathbf{p}}
\newcommand{\bS}{\mathbf{S}}
\newcommand{\bk}{\mathbf{k}}
\newcommand{\bmu}{\boldsymbol{\mu}}
\newcommand{\bx}{\mathbf{x}}
\newcommand{\bg}{\mathbf{g}}
\newcommand{\be}{\mathbf{e}}
\newcommand{\bX}{\mathbf{X}}
\newcommand{\by}{\mathbf{y}}
\newcommand{\bv}{\mathbf{v}}
\newcommand{\bz}{\mathbf{z}}
\newcommand{\bq}{\mathbf{q}}
\newcommand{\bff}{\mathbf{f}}
\newcommand{\bu}{\mathbf{u}}
\newcommand{\bh}{\mathbf{h}}
\newcommand{\bb}{\mathbf{b}}

\newcommand{\rone}{\textcolor{green}{R1}}
\newcommand{\rtwo}{\textcolor{orange}{R2}}
\newcommand{\rthree}{\textcolor{red}{R3}}
\usepackage{amsmath}
%\usepackage{arydshln}
\DeclareMathOperator{\similarity}{sim}
\DeclareMathOperator{\AvgPool}{AvgPool}

\newcommand{\argmax}{\mathop{\mathrm{argmax}}}     



% It is strongly recommended to use hyperref, especially for the review version.
% hyperref with option pagebackref eases the reviewers' job.
% Please disable hyperref *only* if you encounter grave issues, 
% e.g. with the file validation for the camera-ready version.
%
% If you comment hyperref and then uncomment it, you should delete *.aux before re-running LaTeX.
% (Or just hit 'q' on the first LaTeX run, let it finish, and you should be clear).
\definecolor{iccvblue}{rgb}{0.21,0.49,0.74}
\usepackage[pagebackref,breaklinks,colorlinks,allcolors=iccvblue]{hyperref}

%%%%%%%%% PAPER ID  - PLEASE UPDATE
\def\paperID{7494} % *** Enter the Paper ID here
\def\confName{ICCV}
\def\confYear{2025}

%%%%%%%%% TITLE - PLEASE UPDATE
%\title{No Parameters, No Problem: \\3D Gaussian Splatting without Camera Intrinsics and Extrinsics}
\title{TrackGS: Optimizing COLMAP-Free 3D Gaussian Splatting \\with Global Track Constraints}

%%%%%%%%% AUTHORS - PLEASE UPDATE
\author{
	Dongbo Shi$^1$, Shen Cao$^2$, Lubin Fan$^2$, Bojian Wu$^2$, Jinhui Guo$^2$, Renjie Chen$^1$, Ligang Liu$^1$, Jieping Ye$^2$ \\ $^1$University of Science and Technology of China, $^2$Independent Researcher \\
}


\begin{document}

\twocolumn[{%
\renewcommand\twocolumn[1][]{#1}%
\maketitle
\vspace{-2em}
\begin{center}
    % \includegraphics[width=1\textwidth]{figures/scgs-teaser-v3.png}
    \includegraphics[width=1\textwidth]{figures/scgs-teaser-v3-lowRes.png}
    \vspace{-2em}
    \captionof{figure}{\textbf{Comparisons on novel view synthesis and camera poses.} We propose a novel model of 3D Gaussian Splatting without any known camera parameters by leveraging global track information. Compared with state-of-the-art methods, we provide not only higher rendering quality in novel view synthesis, but also more accurate estimation of camera poses on benchmark datasets, including the \textit{challenging} real-world indoor and outdoor or synthetic scenes with \textit{complicated} camera movements (the right column).}\vspace{0.2em}
    \label{fig:teaser}
\end{center}
}]

\begin{abstract}


The choice of representation for geographic location significantly impacts the accuracy of models for a broad range of geospatial tasks, including fine-grained species classification, population density estimation, and biome classification. Recent works like SatCLIP and GeoCLIP learn such representations by contrastively aligning geolocation with co-located images. While these methods work exceptionally well, in this paper, we posit that the current training strategies fail to fully capture the important visual features. We provide an information theoretic perspective on why the resulting embeddings from these methods discard crucial visual information that is important for many downstream tasks. To solve this problem, we propose a novel retrieval-augmented strategy called RANGE. We build our method on the intuition that the visual features of a location can be estimated by combining the visual features from multiple similar-looking locations. We evaluate our method across a wide variety of tasks. Our results show that RANGE outperforms the existing state-of-the-art models with significant margins in most tasks. We show gains of up to 13.1\% on classification tasks and 0.145 $R^2$ on regression tasks. All our code and models will be made available at: \href{https://github.com/mvrl/RANGE}{https://github.com/mvrl/RANGE}.

\end{abstract}

    
\section{Introduction}
Backdoor attacks pose a concealed yet profound security risk to machine learning (ML) models, for which the adversaries can inject a stealth backdoor into the model during training, enabling them to illicitly control the model's output upon encountering predefined inputs. These attacks can even occur without the knowledge of developers or end-users, thereby undermining the trust in ML systems. As ML becomes more deeply embedded in critical sectors like finance, healthcare, and autonomous driving \citep{he2016deep, liu2020computing, tournier2019mrtrix3, adjabi2020past}, the potential damage from backdoor attacks grows, underscoring the emergency for developing robust defense mechanisms against backdoor attacks.

To address the threat of backdoor attacks, researchers have developed a variety of strategies \cite{liu2018fine,wu2021adversarial,wang2019neural,zeng2022adversarial,zhu2023neural,Zhu_2023_ICCV, wei2024shared,wei2024d3}, aimed at purifying backdoors within victim models. These methods are designed to integrate with current deployment workflows seamlessly and have demonstrated significant success in mitigating the effects of backdoor triggers \cite{wubackdoorbench, wu2023defenses, wu2024backdoorbench,dunnett2024countering}.  However, most state-of-the-art (SOTA) backdoor purification methods operate under the assumption that a small clean dataset, often referred to as \textbf{auxiliary dataset}, is available for purification. Such an assumption poses practical challenges, especially in scenarios where data is scarce. To tackle this challenge, efforts have been made to reduce the size of the required auxiliary dataset~\cite{chai2022oneshot,li2023reconstructive, Zhu_2023_ICCV} and even explore dataset-free purification techniques~\cite{zheng2022data,hong2023revisiting,lin2024fusing}. Although these approaches offer some improvements, recent evaluations \cite{dunnett2024countering, wu2024backdoorbench} continue to highlight the importance of sufficient auxiliary data for achieving robust defenses against backdoor attacks.

While significant progress has been made in reducing the size of auxiliary datasets, an equally critical yet underexplored question remains: \emph{how does the nature of the auxiliary dataset affect purification effectiveness?} In  real-world  applications, auxiliary datasets can vary widely, encompassing in-distribution data, synthetic data, or external data from different sources. Understanding how each type of auxiliary dataset influences the purification effectiveness is vital for selecting or constructing the most suitable auxiliary dataset and the corresponding technique. For instance, when multiple datasets are available, understanding how different datasets contribute to purification can guide defenders in selecting or crafting the most appropriate dataset. Conversely, when only limited auxiliary data is accessible, knowing which purification technique works best under those constraints is critical. Therefore, there is an urgent need for a thorough investigation into the impact of auxiliary datasets on purification effectiveness to guide defenders in  enhancing the security of ML systems. 

In this paper, we systematically investigate the critical role of auxiliary datasets in backdoor purification, aiming to bridge the gap between idealized and practical purification scenarios.  Specifically, we first construct a diverse set of auxiliary datasets to emulate real-world conditions, as summarized in Table~\ref{overall}. These datasets include in-distribution data, synthetic data, and external data from other sources. Through an evaluation of SOTA backdoor purification methods across these datasets, we uncover several critical insights: \textbf{1)} In-distribution datasets, particularly those carefully filtered from the original training data of the victim model, effectively preserve the model’s utility for its intended tasks but may fall short in eliminating backdoors. \textbf{2)} Incorporating OOD datasets can help the model forget backdoors but also bring the risk of forgetting critical learned knowledge, significantly degrading its overall performance. Building on these findings, we propose Guided Input Calibration (GIC), a novel technique that enhances backdoor purification by adaptively transforming auxiliary data to better align with the victim model’s learned representations. By leveraging the victim model itself to guide this transformation, GIC optimizes the purification process, striking a balance between preserving model utility and mitigating backdoor threats. Extensive experiments demonstrate that GIC significantly improves the effectiveness of backdoor purification across diverse auxiliary datasets, providing a practical and robust defense solution.

Our main contributions are threefold:
\textbf{1) Impact analysis of auxiliary datasets:} We take the \textbf{first step}  in systematically investigating how different types of auxiliary datasets influence backdoor purification effectiveness. Our findings provide novel insights and serve as a foundation for future research on optimizing dataset selection and construction for enhanced backdoor defense.
%
\textbf{2) Compilation and evaluation of diverse auxiliary datasets:}  We have compiled and rigorously evaluated a diverse set of auxiliary datasets using SOTA purification methods, making our datasets and code publicly available to facilitate and support future research on practical backdoor defense strategies.
%
\textbf{3) Introduction of GIC:} We introduce GIC, the \textbf{first} dedicated solution designed to align auxiliary datasets with the model’s learned representations, significantly enhancing backdoor mitigation across various dataset types. Our approach sets a new benchmark for practical and effective backdoor defense.



\section{Related Work}
\label{sec:related-works}
\subsection{Novel View Synthesis}
Novel view synthesis is a foundational task in the computer vision and graphics, which aims to generate unseen views of a scene from a given set of images.
% Many methods have been designed to solve this problem by posing it as 3D geometry based rendering, where point clouds~\cite{point_differentiable,point_nfs}, mesh~\cite{worldsheet,FVS,SVS}, planes~\cite{automatci_photo_pop_up,tour_into_the_picture} and multi-plane images~\cite{MINE,single_view_mpi,stereo_magnification}, \etal
Numerous methods have been developed to address this problem by approaching it as 3D geometry-based rendering, such as using meshes~\cite{worldsheet,FVS,SVS}, MPI~\cite{MINE,single_view_mpi,stereo_magnification}, point clouds~\cite{point_differentiable,point_nfs}, etc.
% planes~\cite{automatci_photo_pop_up,tour_into_the_picture}, 


\begin{figure*}[!t]
    \centering
    \includegraphics[width=1.0\linewidth]{figures/overview-v7.png}
    %\caption{\textbf{Overview.} Given a set of images, our method obtains both camera intrinsics and extrinsics, as well as a 3DGS model. First, we obtain the initial camera parameters, global track points from image correspondences and monodepth with reprojection loss. Then we incorporate the global track information and select Gaussian kernels associated with track points. We jointly optimize the parameters $K$, $T_{cw}$, 3DGS through multi-view geometric consistency $L_{t2d}$, $L_{t3d}$, $L_{scale}$ and photometric consistency $L_1$, $L_{D-SSIM}$.}
    \caption{\textbf{Overview.} Given a set of images, our method obtains both camera intrinsics and extrinsics, as well as a 3DGS model. During the initialization, we extract the global tracks, and initialize camera parameters and Gaussians from image correspondences and monodepth with reprojection loss. We determine Gaussian kernels with recovered 3D track points, and then jointly optimize the parameters $K$, $T_{cw}$, 3DGS through the proposed global track constraints (i.e., $L_{t2d}$, $L_{t3d}$, and $L_{scale}$) and original photometric losses (i.e., $L_1$ and $L_{D-SSIM}$).}
    \label{fig:overview}
\end{figure*}

Recently, Neural Radiance Fields (NeRF)~\cite{2020NeRF} provide a novel solution to this problem by representing scenes as implicit radiance fields using neural networks, achieving photo-realistic rendering quality. Although having some works in improving efficiency~\cite{instant_nerf2022, lin2022enerf}, the time-consuming training and rendering still limit its practicality.
Alternatively, 3D Gaussian Splatting (3DGS)~\cite{3DGS2023} models the scene as explicit Gaussian kernels, with differentiable splatting for rendering. Its improved real-time rendering performance, lower storage and efficiency, quickly attract more attentions.
% Different from NeRF-based methods which need MLPs to model the scene and huge computational cost for rendering, 3DGS has stronger real-time performance, higher storage and computational efficiency, benefits from its explicit representation and gradient backpropagation.

\subsection{Optimizing Camera Poses in NeRFs and 3DGS}
Although NeRF and 3DGS can provide impressive scene representation, these methods all need accurate camera parameters (both intrinsic and extrinsic) as additional inputs, which are mostly obtained by COLMAP~\cite{colmap2016}.
% This strong reliance on COLMAP significantly limits their use in real-world applications, so optimizing the camera parameters during the scene training becomes crucial.
When the prior is inaccurate or unknown, accurately estimating camera parameters and scene representations becomes crucial.

% In early works, only photometric constraints are used for scene training and camera pose estimation. 
% iNeRF~\cite{iNerf2021} optimizes the camera poses based on a pre-trained NeRF model.
% NeRFmm~\cite{wang2021nerfmm} introduce a joint optimization process, which estimates the camera poses and trains NeRF model jointly.
% BARF~\cite{barf2021} and GARF~\cite{2022GARF} provide new positional encoding strategy to handle with the gradient inconsistency issue of positional embedding and yield promising results.
% However, they achieve satisfactory optimization results when only the pose initialization is quite closed to the ground-truth, as the photometric constrains can only improve the quality of camera estimation within a small range.
% Later, more prior information of geometry and correspondence, \ie monocular depth and feature matching, are introduced into joint optimisation to enhance the capability of camera poses estimation.
% SC-NeRF~\cite{SCNeRF2021} minimizes a projected ray distance loss based on correspondence of adjacent frames.
% NoPe-NeRF~\cite{bian2022nopenerf} chooses monocular depth maps as geometric priors, and defines undistorted depth loss and relative pose constraints for joint optimization.
In earlier studies, scene training and camera pose estimation relied solely on photometric constraints. iNeRF~\cite{iNerf2021} refines the camera poses using a pre-trained NeRF model. NeRFmm~\cite{wang2021nerfmm} introduces a joint optimization approach that simultaneously estimates camera poses and trains the NeRF model. BARF~\cite{barf2021} and GARF~\cite{2022GARF} propose a new positional encoding strategy to address the gradient inconsistency issues in positional embedding, achieving promising results. However, these methods only yield satisfactory optimization when the initial pose is very close to the ground truth, as photometric constraints alone can only enhance camera estimation quality within a limited range. Subsequently, 
% additional prior information on geometry and correspondence, such as monocular depth and feature matching, has been incorporated into joint optimization to improve the accuracy of camera pose estimation. 
SC-NeRF~\cite{SCNeRF2021} minimizes a projected ray distance loss based on correspondence between adjacent frames. NoPe-NeRF~\cite{bian2022nopenerf} utilizes monocular depth maps as geometric priors and defines undistorted depth loss and relative pose constraints.

% With regard to 3D Gaussian Splatting, CF-3DGS~\cite{CF-3DGS-2024} also leverages mono-depth information to constrain the optimization of local 3DGS for relative pose estimation and later learn a global 3DGS progressively in a sequential manner.
% InstantSplat~\cite{fan2024instantsplat} focus on sparse view scenes, first use DUSt3R~\cite{dust3r2024cvpr} to generate a set of densely covered and pixel-aligned points for 3D Gaussian initialization, then introduce a parallel grid partitioning strategy in joint optimization to speed up.
% % Jiang et al.~\cite{Jiang_2024sig} proposed to build the scene continuously and progressively, to next unregistered frame, they use registration and adjustment to adjust the previous registered camera poses and align unregistered monocular depths, later refine the joint model by matching detected correspondences in screen-space coordinates.
% \gjh{Jiang et al.~\cite{Jiang_2024sig} also implemented an incremental approach for reconstructing camera poses and scenes. Initially, they perform feature matching between the current image and the image rendered by a differentiable surface renderer. They then construct matching point errors, depth errors, and photometric errors to achieve the registration and adjustment of the current image. Finally, based on the depth map, the pixels of the current image are projected as new 3D Gaussians. However, this method still exhibits limitations when dealing with complex scenes and unordered images.}
% % CG-3DGS~\cite{sun2024correspondenceguidedsfmfree3dgaussian} follows CF-3DGS, first construct a coarse point cloud from mono-depth maps to train a 3DGS model, then progressively estimate camera poses based on this pre-trained model by constraining the correspondences between rendering view and ground-truth.
% \gjh{Similarly, CG-3DGS~\cite{sun2024correspondenceguidedsfmfree3dgaussian} first utilizes monocular depth estimation and the camera parameters from the first frame to initialize a set of 3D Gaussians. It then progressively estimates camera poses based on this pre-trained model by constraining the correspondences between the rendered views and the ground truth.}
% % Free-SurGS~\cite{freesurgs2024} matches the projection flow derived from 3D Gaussians with optical flow to estimate the poses, to compensate for the limitations of photometric loss.
% \gjh{Free-SurGS~\cite{freesurgs2024} introduces the first SfM-free 3DGS approach for surgical scene reconstruction. Due to the challenges posed by weak textures and photometric inconsistencies in surgical scenes, Free-SurGS achieves pose estimation by minimizing the flow loss between the projection flow and the optical flow. Subsequently, it keeps the camera pose fixed and optimizes the scene representation by minimizing the photometric loss, depth loss and flow loss.}
% \gjh{However, most current works assume camera intrinsics are known and primarily focus on optimizing camera poses. Additionally, these methods typically rely on sequentially ordered image inputs and incrementally optimize camera parameters and scene representation. This inevitably leads to drift errors, preventing the achievement of globally consistent results. Our work aims to address these issues.}

Regarding 3D Gaussian Splatting, CF-3DGS~\cite{CF-3DGS-2024} utilizes mono-depth information to refine the optimization of local 3DGS for relative pose estimation and subsequently learns a global 3DGS in a sequential manner. InstantSplat~\cite{fan2024instantsplat} targets sparse view scenes, initially employing DUSt3R~\cite{dust3r2024cvpr} to create a densely covered, pixel-aligned point set for initializing 3D Gaussian models, and then implements a parallel grid partitioning strategy to accelerate joint optimization. Jiang \etal~\cite{Jiang_2024sig} develops an incremental method for reconstructing camera poses and scenes, but it struggles with complex scenes and unordered images. 
% Similarly, CG-3DGS~\cite{sun2024correspondenceguidedsfmfree3dgaussian} progressively estimates camera poses using a pre-trained model by aligning the correspondences between rendered views and actual scenes. Free-SurGS~\cite{freesurgs2024} pioneers an SfM-free 3DGS method for reconstructing surgical scenes, overcoming challenges such as weak textures and photometric inconsistencies by minimizing the discrepancy between projection flow and optical flow.
%\pb{SF-3DGS-HT~\cite{ji2024sfmfree3dgaussiansplatting} introduced VFI into training as additional photometric constraints. They separated the whole scene into several local 3DGS models and then merged them hierarchically, which leads to a significant improvement on simple and dense view scenes.}
HT-3DGS~\cite{ji2024sfmfree3dgaussiansplatting} interpolates frames for training and splits the scene into local clips, using a hierarchical strategy to build 3DGS model. It works well for simple scenes, but fails with dramatic motions due to unstable interpolation and low efficiency.
% {While effective for simple scenes, it struggles with dramatic motion due to unstable view interpolation and suffers from low computational efficiency.}

However, most existing methods generally depend on sequentially ordered image inputs and incrementally optimize camera parameters and 3DGS, which often leads to drift errors and hinders achieving globally consistent results. Our work seeks to overcome these limitations.

\section{Methodology}

\subsection{Problem Definition}

Given a multivariate time series input $X \in \mathbb{R}^{C  \times T}$, multivariate time series forecasting tasks are designed to predict its future $F$ time steps $\hat{Y}\in \mathbb{R}^{C \times F}$ using past $T$ steps. $C $ is the number of variates or channels.

\subsection{Preliminary Analysis}

This section presents why RevIN~\citep{Kim_revin,liu2022non}, High-pass, and Low-pass filters fail to address the Mid-Frequency Spectrum Gap. Let the input univariate time series be $ x(t) $ with length $ T $ and target $ y(t) $ with length $ F $. 

\begin{definition}[Frequency Spectral Energy]\label{def:energy}
The Fourier transform of $x(t)$, $X(f)$, and its spectral energy $E_X(f)$ is given by:
\vspace{-0.2cm}
\begin{align}
X(f) = \sum_{t=0}^{T-1} x(t) e^{-i 2 \pi f t / {T-1}}, \quad &f = 0, 1, \dots, T-1\notag\\
E_X(f) = |X(f)|^2.
\end{align}
\vspace{-0.2cm}
\end{definition}

\textbf{Impact of RevIN on Frequency Spectrum \quad}
\begin{definition}[Reversible Instance Normalization]\label{def:RevIN}
Given a \textbf{forecast model} $ f: \mathbb{R}^T \rightarrow \mathbb{R}^F $ that generates a forecast $ \hat{y}(t) $ from a given input $x(t)$, RevIN is defined as:
\vspace{-0.2cm}
\begin{align}
&\hat{x}(t) = \frac{x(t) - \mu}{\sigma},\quad t = 0, 1, \dots, T-1\notag\\
&\hat{y}(t) = f(\hat{x}(t)), \quad \hat{y}(t)_{rev}= \hat{y}(t) \cdot \sigma + \mu,\notag\\
&\mu = \frac{1}{T} \sum_{t=0}^{T-1} x(t), \quad \sigma = \sqrt{\frac{1}{T} \sum_{t=0}^{T-1} (x(t) - \mu)^2}.
\end{align}
\vspace{-0.2cm}
\end{definition}

\begin{theorem} [Frequency Spectrum after RevIN] \label{theorem:RevIN}
\vspace{-0.2cm}
The spectral energy of $\hat{x}(t)$ (transformed using RevIN):
\begin{align}
E_{\hat{X}}(0)=0,& \quad f=0, \notag\\
E_{\hat{X}}(f) = \left( \frac{1}{\sigma} \right)^2 |X(f)|^2,&\quad f = 1,2,\dots, T-1 . 
\end{align}
\vspace{-0.2cm}
\end{theorem}
The proof is in Appendix~\ref{app:RevIN}. Theorem~\ref{theorem:RevIN} suggests that RevIN scales the absolute spectral energy by $ \sigma^2 $ but does not affect its relative distribution except $E_{\hat{X}}(0)=0$. Thus, RevIN preserves the relative spectral energy distribution and leaves the Mid-Frequency Spectrum Gap unresolved. \textit{However, our experiments still employ RevIN to ensure a fair comparison with other baselines.}
\begin{figure*}[h]
  \centering
  \includegraphics[width=1.\linewidth]{Faker/source/assets/jpg/ReFocus.jpg}
  \caption{General structure of \textbf{ReFocus}. `Adaptive Mid-Frequency Energy Optimizer (AMEO)' enhances mid-frequency components modeling, and `Energy-based Key-Frequency Picking Block' (EKPB) effectively captures shared Key-Frequency across channels}
  \label{fig:refocus}
\end{figure*}

\begin{figure*}[h]
  \centering
  \includegraphics[width=0.7\linewidth]{Faker/source/assets/jpg/ket.jpg}
  \caption{General process of the \textbf{Key-Frequency Enhanced Training strategy (KET)}, where spectral information from other channels is randomly introduced into each channel, to enhance the extraction of the shared Key-Frequency.}
  \label{fig:reshuffle}
\end{figure*}
\textbf{Impact of High- and Low-pass filter \quad}
We still define $\hat{x}(t)$ to be the filtered (processed) signal, obtained by applying a filter $H(f)$ (High/Low-pass filter). The filter $ H(f) $ is 1 in the passband (High/Low frequency) and 0 in the stopband (Middle frequency). So $E_{\hat{X}}(f)=0,\quad E_{\hat{X}}\leq E_X(f)$ for middle frequencies, which creates even larger gap.

\subsection{Overall Structure of The Proposed ReFocus}

In this section, we elucidate the overall architecture of \textbf{ReFocus}, depicted in Figure \ref{fig:refocus}. We define frequency domain projection as $D1\rightarrow D2$ representing a projection from dimension $D1$ to $D2$ in the frequency domain~\citep{xu2024fits}. Initially, we apply \textbf{AMEO} to the input $X \in \mathbb{R}^{C \times T}$, yielding the processed spectrum $ X_{am} \in \mathbb{R}^{C  \times T} $. Next, we use a projection $T\rightarrow D$ to transform $ X_{am}$ into the Variate Embedding $ X_{em} \in \mathbb{R}^{C  \times D}$~\citep{LiuiTransformer}. Then, $X_{em}$ go through $N$ \textbf{EKPB} to generate representation $H_{N+1}$, which is projected to obtain final prediction $\hat{Y}$. 

\textbf{Adaptive Mid-Frequency Energy Optimizer \quad}
Building upon the \textbf{Preliminary Analysis}, we propose a convolution- and residual learning-based solution to address the Mid-Frequency Spectrum Gap, which we denoted as AMEO. 
\begin{definition}[Adaptive Mid-Frequency Energy Optimizer]\label{def:AMEO}
AMEO is defined as:
\begin{align}
&\hat{x}(t) = x(t)-\frac{\beta}{K}\sum_{k=0}^{K-1} \tilde{x}(t+K-1-k),\notag\\
&\tilde{x}(t) =\notag\\
&\begin{cases}
x(t-(\frac{K}{2}+1)), \quad \text{if } \frac{K}{2}+1 \leq t < T+\frac{K}{2}+1, \\
0,  \quad\text{if } 0 \leq t < \frac{K}{2}+1 \text{ or } T+\frac{K}{2}+1 \leq t < T+K.
\end{cases}
\end{align}
\vspace{-0.2cm}
\end{definition}

It is equivalent to $x=x-\beta \cdot Conv(x)$. $Conv$ is a 1D convolution (Zero-padding at both ends, stride $s=1$, kernel size $K$, with values initialized as $ \frac{1}{K} $). $\beta \in \mathbb{R}^{1}$ is a hyperparameter.

\begin{theorem} [Frequency Spectrum after AMEO] \label{theorem:AMEO}
The spectral energy of $\hat{x}(t)$ obtained using AMEO:
\begin{align}
E_{\hat{X}}(f) =|X(f)|^2 \left\{1 - \beta \cdot \underbrace{\frac{1}{K} \sum_{k=0}^{K-1} e^{i 2 \pi f (\frac{3K}{2}-k -2) / {T-1}}}_{G(f)}\right\}^2
\end{align}
\vspace{-0.2cm}
\end{theorem}

The proof is in Appendix~\ref{app:AMEO}. We have $E_{\hat{X}}(f) =|X(f)|^2(1-\beta  \cdot G(f))^2$. Generally, $ G(f) $ behaves as a decay function, gradually reducing its value from \textbf{One} to \textbf{Zero}. Such \textbf{decay behavior} makes AMEO relatively enhances mid-frequency components, thus addressing the Mid-Frequency Spectrum Gap.

\textbf{Energy-based Key-Frequency Picking Block \quad} In each \textbf{EKPB}, the input $ H_i \in \mathbb{R}^{C  \times D} (H_1=X_{em}) $ is first processed through an MLP to generate $ H_i^k \in \mathbb{R}^{C  \times Q}$. Then, FFT is applied to get $ H_i^f \in \mathbb{R}^{C  \times (Q/2+1)}$. For $ H_i^f$, we calculate its energy, denoted as $ H_i^e \in \mathbb{R}^{C  \times (Q/2+1)}$. A cross-channel softmax is then applied to $ H_i^e$ per frequency to obtain a probability distribution $ H_i^{soft} \in \mathbb{R}^{C  \times (Q/2+1)}$. Using $H_i^{soft}$, we select values from $ H_i^f$ across channels for each frequency, resulting in $K^f_i \in \mathbb{R}^{1  \times (Q/2+1)}$, which represents the Shared Key-Frequency across all channels. Then iFFT is performed on $K^f_i$ to get $K_i\in \mathbb{R}^{1  \times Q}$, followed by projection $Q\rightarrow D$ and repeating (C times) to get $\hat{K}_i \in \mathbb{R}^{C  \times D}$. This $\hat{K}_i$ is point-wisely added to $\hat{H_i}\in \mathbb{R}^{C  \times D}$ , which is the projection of $ H_i$ using projection $D\rightarrow D$. Then, an MLP and $Add\&Norm$ is applied to the result $HK\in \mathbb{R}^{C  \times D}$ to fuse inter-series dependencies information, and another MLP and $Add\&Norm$ is used to capture intra-series variations~\citep{LiuiTransformer}. The output of each \textbf{EKPB} is $\hat{O_i} \in \mathbb{R}^{C  \times D}$, where $H_{i+1}=\hat{O_i}$.

\subsection{Key-Frequency Enhanced Training strategy}

In real-world time series, certain channels often exhibit spectral dependencies, which may not be fully captured in the training set, and the specific channels with such dependencies are also unknown~\citep{geweke1984freqchannel,Zhao2024freqchannel}. So this work borrows insight from recent advancement of mix-up in time series~\citep{zhou2023mixup,ansari2024mixup}, randomly introducing spectral information from other channels into each channel, to enhance the extraction of the shared Key-Frequency, as in Figure~\ref{fig:reshuffle}. Given a multivariate time series input $X \in \mathbb{R}^{C \times T}$ and its ground-truth $Y \in \mathbb{R}^{C \times F}$, we generate a pseudo sample pair: 

\begin{align}
X' = iFFT(FFT(X) +\alpha \cdot FFT(X[\text{perm},:]))&,  \notag\\ 
Y' = iFFT(FFT(Y) +\alpha \cdot FFT(Y[\text{perm},:]))&.
\end{align}

$\alpha \in \mathbb{R}^{C \times 1}$ is a weight vector sampled from a normal distribution, $\text{perm}$ is a reshuffled channel index. Since $FFT$ and $iFFT$ are linear operations, this mix-up process can be equivalently simplified in the \textbf{Time Domain}:
\begin{align}
X' = X +\alpha \cdot X[\text{perm},:]&,  \notag\\
Y' = Y +\alpha \cdot Y[\text{perm},:]&
 \end{align}
We alternate training between real and synthetic data to preserve the spectral dependencies in real samples. This combines the advantages of data augmentation, such as improved generalization, while mitigating potential drawbacks like over-smoothing and training instability~\citep{ryu2024tf,alkhalifah2022tf}.












\section{Experimental Results}
In this section, we present the main results in~\secref{sec:main}, followed by ablation studies on key design choices in~\secref{sec:ablation}.

\begin{table*}[t]
\renewcommand\arraystretch{1.05}
\centering
\setlength{\tabcolsep}{2.5mm}{}
\begin{tabular}{l|l|c|cc|cc}
type & model     & \#params      & FID$\downarrow$ & IS$\uparrow$ & Precision$\uparrow$ & Recall$\uparrow$ \\
\shline
GAN& BigGAN~\cite{biggan} & 112M & 6.95  & 224.5       & 0.89 & 0.38     \\
GAN& GigaGAN~\cite{gigagan}  & 569M      & 3.45  & 225.5       & 0.84 & 0.61\\  
GAN& StyleGan-XL~\cite{stylegan-xl} & 166M & 2.30  & 265.1       & 0.78 & 0.53  \\
\hline
Diffusion& ADM~\cite{adm}    & 554M      & 10.94 & 101.0        & 0.69 & 0.63\\
Diffusion& LDM-4-G~\cite{ldm}   & 400M  & 3.60  & 247.7       & -  & -     \\
Diffusion & Simple-Diffusion~\cite{diff1} & 2B & 2.44 & 256.3 & - & - \\
Diffusion& DiT-XL/2~\cite{dit} & 675M     & 2.27  & 278.2       & 0.83 & 0.57     \\
Diffusion&L-DiT-3B~\cite{dit-github}  & 3.0B    & 2.10  & 304.4       & 0.82 & 0.60    \\
Diffusion&DiMR-G/2R~\cite{liu2024alleviating} &1.1B& 1.63& 292.5& 0.79 &0.63 \\
Diffusion & MDTv2-XL/2~\cite{gao2023mdtv2} & 676M & 1.58 & 314.7 & 0.79 & 0.65\\
Diffusion & CausalFusion-H$^\dag$~\cite{deng2024causal} & 1B & 1.57 & - & - & - \\
\hline
Flow-Matching & SiT-XL/2~\cite{sit} & 675M & 2.06 & 277.5 & 0.83 & 0.59 \\
Flow-Matching&REPA~\cite{yu2024representation} &675M& 1.80 & 284.0 &0.81 &0.61\\    
Flow-Matching&REPA$^\dag$~\cite{yu2024representation}& 675M& 1.42&  305.7& 0.80& 0.65 \\
\hline
Mask.& MaskGIT~\cite{maskgit}  & 227M   & 6.18  & 182.1        & 0.80 & 0.51 \\
Mask. & TiTok-S-128~\cite{yu2024image} & 287M & 1.97 & 281.8 & - & - \\
Mask. & MAGVIT-v2~\cite{yu2024language} & 307M & 1.78 & 319.4 & - & - \\ 
Mask. & MaskBit~\cite{weber2024maskbit} & 305M & 1.52 & 328.6 & - & - \\
\hline
AR& VQVAE-2~\cite{vqvae2} & 13.5B    & 31.11           & $\sim$45     & 0.36           & 0.57          \\
AR& VQGAN~\cite{vqgan}& 227M  & 18.65 & 80.4         & 0.78 & 0.26   \\
AR& VQGAN~\cite{vqgan}   & 1.4B     & 15.78 & 74.3   & -  & -     \\
AR&RQTran.~\cite{rq}     & 3.8B    & 7.55  & 134.0  & -  & -    \\
AR& ViTVQ~\cite{vit-vqgan} & 1.7B  & 4.17  & 175.1  & -  & -    \\
AR & DART-AR~\cite{gu2025dart} & 812M & 3.98 & 256.8 & - & - \\
AR & MonoFormer~\cite{zhao2024monoformer} & 1.1B & 2.57 & 272.6 & 0.84 & 0.56\\
AR & Open-MAGVIT2-XL~\cite{luo2024open} & 1.5B & 2.33 & 271.8 & 0.84 & 0.54\\
AR&LlamaGen-3B~\cite{llamagen}  &3.1B& 2.18& 263.3 &0.81& 0.58\\
AR & FlowAR-H~\cite{flowar} & 1.9B & 1.65 & 296.5 & 0.83 & 0.60\\
AR & RAR-XXL~\cite{yu2024randomized} & 1.5B & 1.48 & 326.0 & 0.80 & 0.63 \\
\hline
MAR & MAR-B~\cite{mar} & 208M & 2.31 &281.7 &0.82 &0.57 \\
MAR & MAR-L~\cite{mar} &479M& 1.78 &296.0& 0.81& 0.60 \\
MAR & MAR-H~\cite{mar} & 943M&1.55& 303.7& 0.81 &0.62 \\
\hline
VAR&VAR-$d16$~\cite{var}   & 310M  & 3.30& 274.4& 0.84& 0.51    \\
VAR&VAR-$d20$~\cite{var}   &600M & 2.57& 302.6& 0.83& 0.56     \\
VAR&VAR-$d30$~\cite{var}   & 2.0B      & 1.97  & 323.1 & 0.82 & 0.59      \\
\hline
\modelname& \modelname-B    &172M   &1.72&280.4&0.82&0.59 \\
\modelname& \modelname-L   & 608M   & 1.28& 292.5&0.82&0.62\\
\modelname& \modelname-H    & 1.1B    & 1.24 &301.6&0.83&0.64\\
\end{tabular}
\caption{
\textbf{Generation Results on ImageNet-256.}
Metrics include Fréchet Inception Distance (FID), Inception Score (IS), Precision, and Recall. $^\dag$ denotes the use of guidance interval sampling~\cite{guidance}. The proposed \modelname-H achieves a state-of-the-art 1.24 FID on the ImageNet-256 benchmark without relying on vision foundation models (\eg, DINOv2~\cite{dinov2}) or guidance interval sampling~\cite{guidance}, as used in REPA~\cite{yu2024representation}.
}\label{tab:256}
\end{table*}

\subsection{Main Results}
\label{sec:main}
We conduct experiments on ImageNet~\cite{deng2009imagenet} at 256$\times$256 and 512$\times$512 resolutions. Following prior works~\cite{dit,mar}, we evaluate model performance using FID~\cite{fid}, Inception Score (IS)~\cite{is}, Precision, and Recall. \modelname is trained with the same hyper-parameters as~\cite{mar,dit} (\eg, 800 training epochs), with model sizes ranging from 172M to 1.1B parameters. See Appendix~\secref{sec:sup_hyper} for hyper-parameter details.





\begin{table}[t]
    \centering
    \begin{tabular}{c|c|c|c}
      model    &  \#params & FID$\downarrow$ & IS$\uparrow$ \\
      \shline
      VQGAN~\cite{vqgan}&227M &26.52& 66.8\\
      BigGAN~\cite{biggan}& 158M&8.43 &177.9\\
      MaskGiT~\cite{maskgit}& 227M&7.32& 156.0\\
      DiT-XL/2~\cite{dit} &675M &3.04& 240.8 \\
     DiMR-XL/3R~\cite{liu2024alleviating}& 525M&2.89 &289.8 \\
     VAR-d36~\cite{var}  & 2.3B& 2.63 & 303.2\\
     REPA$^\ddagger$~\cite{yu2024representation}&675M &2.08& 274.6 \\
     \hline
     \modelname-L & 608M&1.70& 281.5 \\
    \end{tabular}
    \caption{
    \textbf{Generation Results on ImageNet-512.} $^\ddagger$ denotes the use of DINOv2~\cite{dinov2}.
    }
    \label{tab:512}
\end{table}

\noindent\textbf{ImageNet-256.}
In~\tabref{tab:256}, we compare \modelname with previous state-of-the-art generative models.
Out best variant, \modelname-H, achieves a new state-of-the-art-performance of 1.24 FID, outperforming the GAN-based StyleGAN-XL~\cite{stylegan-xl} by 1.06 FID, masked-prediction-based MaskBit~\cite{maskgit} by 0.28 FID, AR-based RAR~\cite{yu2024randomized} by 0.24 FID, VAR~\cite{var} by 0.73 FID, MAR~\cite{mar} by 0.31 FID, and flow-matching-based REPA~\cite{yu2024representation} by 0.18 FID.
Notably, \modelname does not rely on vision foundation models~\cite{dinov2} or guidance interval sampling~\cite{guidance}, both of which were used in REPA~\cite{yu2024representation}, the previous best-performing model.
Additionally, our lightweight \modelname-B (172M), surpasses DiT-XL (675M)~\cite{dit} by 0.55 FID while achieving an inference speed of 9.8 images per second—20$\times$ faster than DiT-XL (0.5 images per second). Detailed speed comparison can be found in Appendix \ref{sec:speed}.



\noindent\textbf{ImageNet-512.}
In~\tabref{tab:512}, we report the performance of \modelname on ImageNet-512.
Similarly, \modelname-L sets a new state-of-the-art FID of 1.70, outperforming the diffusion based DiT-XL/2~\cite{dit} and DiMR-XL/3R~\cite{liu2024alleviating} by a large margin of 1.34 and 1.19 FID, respectively.
Additionally, \modelname-L also surpasses the previous best autoregressive model VAR-d36~\cite{var} and flow-matching-based REPA~\cite{yu2024representation} by 0.93 and 0.38 FID, respectively.




\noindent\textbf{Qualitative Results.}
\figref{fig:qualitative} presents samples generated by \modelname (trained on ImageNet) at 512$\times$512 and 256$\times$256 resolutions. These results highlight \modelname's ability to produce high-fidelity images with exceptional visual quality.

\begin{figure*}
    \centering
    \vspace{-6pt}
    \includegraphics[width=1\linewidth]{figures/qualitative.pdf}
    \caption{\textbf{Generated Samples.} \modelname generates high-quality images at resolutions of 512$\times$512 (1st row) and 256$\times$256 (2nd and 3rd row).
    }
    \label{fig:qualitative}
\end{figure*}

\subsection{Ablation Studies}
\label{sec:ablation}
In this section, we conduct ablation studies using \modelname-B, trained for 400 epochs to efficiently iterate on model design.

\noindent\textbf{Prediction Entity X.}
The proposed \modelname extends next-token prediction to next-X prediction. In~\tabref{tab:X}, we evaluate different designs for the prediction entity X, including an individual patch token, a cell (a group of surrounding tokens), a subsample (a non-local grouping), a scale (coarse-to-fine resolution), and an entire image.

Among these variants, cell-based \modelname achieves the best performance, with an FID of 2.48, outperforming the token-based \modelname by 1.03 FID and surpassing the second best design (scale-based \modelname) by 0.42 FID. Furthermore, even when using standard prediction entities such as tokens, subsamples, images, or scales, \modelname consistently outperforms existing methods while requiring significantly fewer parameters. These results highlight the efficiency and effectiveness of \modelname across diverse prediction entities.






\begin{table}[]
    \centering
    \scalebox{0.92}{
    \begin{tabular}{c|c|c|c|c}
        model & \makecell[c]{prediction\\entity} & \#params & FID$\downarrow$ & IS$\uparrow$\\
        \shline
        LlamaGen-L~\cite{llamagen} & \multirow{2}{*}{token} & 343M & 3.80 &248.3\\
        \modelname-B& & 172M&3.51&251.4\\
        \hline
        PAR-L~\cite{par} & \multirow{2}{*}{subsample}& 343M & 3.76 & 218.9\\
        \modelname-B&  &172M& 3.58&231.5\\
        \hline
        DiT-L/2~\cite{dit}& \multirow{2}{*}{image}& 458M&5.02&167.2 \\
         \modelname-B& & 172M&3.13&253.4 \\
        \hline
        VAR-$d16$~\cite{var} & \multirow{2}{*}{scale} & 310M&3.30 &274.4\\
        \modelname-B& &172M&2.90&262.8\\
        \hline
        \baseline{\modelname-B}& \baseline{cell} & \baseline{172M}&\baseline{2.48}&\baseline{269.2} \\
    \end{tabular}
    }
    \caption{\textbf{Ablation on Prediction Entity X.} Using cells as the prediction entity outperforms alternatives such as tokens or entire images. Additionally, under the same prediction entity, \modelname surpasses previous methods, demonstrating its effectiveness across different prediction granularities. }%
    \label{tab:X}
\end{table}

\noindent\textbf{Cell Size.}
A prediction entity cell is formed by grouping spatially adjacent $k\times k$ tokens, where a larger cell size incorporates more tokens and thus captures a broader context within a single prediction step.
For a $256\times256$ input image, the encoded continuous latent representation has a spatial resolution of $16\times16$. Given this, the image can be partitioned into an $m\times m$ grid, where each cell consists of $k\times k$ neighboring tokens. As shown in~\tabref{tab:cell}, we evaluate different cell sizes with $k \in \{1,2,4,8,16\}$, where $k=1$ represents a single token and $k=16$ corresponds to the entire image as a single entity. We observe that performance improves as $k$ increases, peaking at an FID of 2.48 when using cell size $8\times8$ (\ie, $k=8$). Beyond this, performance declines, reaching an FID of 3.13 when the entire image is treated as a single entity.
These results suggest that using cells rather than the entire image as the prediction unit allows the model to condition on previously generated context, improving confidence in predictions while maintaining both rich semantics and local details.





\begin{table}[t]
    \centering
    \scalebox{0.98}{
    \begin{tabular}{c|c|c|c}
    cell size ($k\times k$ tokens) & $m\times m$ grid & FID$\downarrow$ & IS$\uparrow$ \\
       \shline
       $1\times1$ & $16\times16$ &3.51&251.4 \\
       $2\times2$ & $8\times8$ & 3.04& 253.5\\
       $4\times4$ & $4\times4$ & 2.61&258.2 \\
       \baseline{$8\times8$} & \baseline{$2\times2$} & \baseline{2.48} & \baseline{269.2}\\
       $16\times16$ & $1\times1$ & 3.13&253.4  \\
    \end{tabular}
    }
    \caption{\textbf{Ablation on the cell size.}
    In this study, a $16\times16$ continuous latent representation is partitioned into an $m\times m$ grid, where each cell consits of $k\times k$ neighboring tokens.
    A cell size of $8\times8$ achieves the best performance, striking an optimal balance between local structure and global context.
    }
    \label{tab:cell}
\end{table}



\begin{table}[t]
    \centering
    \scalebox{0.95}{
    \begin{tabular}{c|c|c|c}
      previous cell & noise time step &  FID$\downarrow$ & IS$\uparrow$ \\
       \shline
       clean & $t_i=0, \forall i<n$& 3.45& 243.5\\
       increasing noise & $t_1<t_2<\cdots<t_{n-1}$& 2.95&258.8 \\
       decreasing noise & $t_1>t_2>\cdots>t_{n-1}$&2.78 &262.1 \\
      \baseline{random noise}  & \baseline{no constraint} &\baseline{2.48} & \baseline{269.2}\\
    \end{tabular}
    }
    \caption{
    \textbf{Ablation on Noisy Context Learning.}
    This study examines the impact of noise time steps ($t_1, \cdots, t_{n-1} \subset [0, 1]$) in previous entities ($t=0$ represents pure Gaussian noise).
    Conditioning on all clean entities (the ``clean'' variant) results in suboptimal performance.
    Imposing an order on noise time steps, either ``increasing noise'' or ``decreasing noise'', also leads to inferior results. The best performance is achieved with the "random noise" setting, where no constraints are imposed on noise time steps.
    }
    \label{tab:ncl}
\end{table}


\noindent\textbf{Noisy Context Learning.}
During training, \modelname employs Noisy Context Learning (NCL), predicting $X_n$ by conditioning on all previous noisy entities, unlike Teacher Forcing.
The noise intensity of previous entities is contorlled by noise time steps $\{t_1, \dots, t_{n-1}\} \subset [0, 1]$, where $t=0$ corresponds to pure Gaussian noise.
We analyze the impact of NCL in~\tabref{tab:ncl}.
When conditioning on all clean entities (\ie, the ``clean'' variant, where $t_i=0, \forall i<n$), which is equivalent to vanilla AR (\ie, Teacher Forcing), the suboptimal performance is obtained.
We also evaluate two constrained noise schedules: the ``increasing noise'' variant, where noise time steps increase over AR steps ($t_1<t_2< \cdots < t_{n-1}$), and the `` decreasing noise'' variant, where noise time steps decrease ($t_1>t_2> \cdots > t_{n-1}$).
While both settings improve over the ``clean'' variant, they remain inferior to our final ``random noise'' setting, where no constraints are imposed on noise time steps, leading to the best performance.




        

\section{Conclusion}
We introduce a novel approach, \algo, to reduce human feedback requirements in preference-based reinforcement learning by leveraging vision-language models. While VLMs encode rich world knowledge, their direct application as reward models is hindered by alignment issues and noisy predictions. To address this, we develop a synergistic framework where limited human feedback is used to adapt VLMs, improving their reliability in preference labeling. Further, we incorporate a selective sampling strategy to mitigate noise and prioritize informative human annotations.

Our experiments demonstrate that this method significantly improves feedback efficiency, achieving comparable or superior task performance with up to 50\% fewer human annotations. Moreover, we show that an adapted VLM can generalize across similar tasks, further reducing the need for new human feedback by 75\%. These results highlight the potential of integrating VLMs into preference-based RL, offering a scalable solution to reducing human supervision while maintaining high task success rates. 

\section*{Impact Statement}
This work advances embodied AI by significantly reducing the human feedback required for training agents. This reduction is particularly valuable in robotic applications where obtaining human demonstrations and feedback is challenging or impractical, such as assistive robotic arms for individuals with mobility impairments. By minimizing the feedback requirements, our approach enables users to more efficiently customize and teach new skills to robotic agents based on their specific needs and preferences. The broader impact of this work extends to healthcare, assistive technology, and human-robot interaction. One possible risk is that the bias from human feedback can propagate to the VLM and subsequently to the policy. This can be mitigated by personalization of agents in case of household application or standardization of feedback for industrial applications. 

{
    \small
    \bibliographystyle{ieeenat_fullname}
    \bibliography{main}
}

\end{document}
