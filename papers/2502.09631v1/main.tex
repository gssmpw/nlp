% CVPR 2024 Paper Template; see https://github.com/cvpr-org/author-kit

\documentclass[10pt,twocolumn,letterpaper]{article}

%%%%%%%%% PAPER TYPE  - PLEASE UPDATE FOR FINAL VERSION
% \usepackage{cvpr}              % To produce the CAMERA-READY version
% \usepackage[review]{cvpr}      % To produce the REVIEW version
\usepackage[pagenumbers]{cvpr} % To force page numbers, e.g. for an arXiv version

% Import additional packages in the preamble file, before hyperref
\newcommand{\CG}{\mathcal{G}\xspace}
\newcommand{\CV}{\mathcal{V}\xspace}
\newcommand{\CE}{\mathcal{E}\xspace}
\newcommand{\CA}{\mathcal{A}\xspace}
\newcommand{\CF}{\mathcal{F}\xspace}
\newcommand{\CR}{\mathcal{R}\xspace}
\newcommand{\CB}{\mathcal{B}\xspace}
\newcommand{\CX}{\mathcal{X}\xspace}
\newcommand{\CK}{\mathcal{K}\xspace}
\newcommand{\CM}{\mathcal{M}\xspace}
\newcommand{\CC}{\mathcal{C}\xspace}
\newcommand{\CL}{\mathcal{L}\xspace}
\newcommand{\CI}{\mathcal{I}\xspace}
\newcommand{\CQ}{\mathcal{Q}\xspace}
\newcommand{\CO}{\mathcal{O}\xspace}
\newcommand{\CP}{\mathcal{P}\xspace}
\newcommand{\CS}{\mathcal{S}\xspace}
\newcommand{\CT}{\mathcal{T}\xspace}
\newcommand{\CJ}{\mathcal{J}\xspace}
\usepackage[para]{footmisc}
\usepackage{subfig}
% \usepackage{subcaption}
% \usepackage{array}
% \usepackage{colortbl}



% It is strongly recommended to use hyperref, especially for the review version.
% hyperref with option pagebackref eases the reviewers' job.
% Please disable hyperref *only* if you encounter grave issues, 
% e.g. with the file validation for the camera-ready version.
%
% If you comment hyperref and then uncomment it, you should delete *.aux before re-running LaTeX.
% (Or just hit 'q' on the first LaTeX run, let it finish, and you should be clear).
\definecolor{cvprblue}{rgb}{0.21,0.49,0.74}
\usepackage[pagebackref,breaklinks,colorlinks,citecolor=cvprblue]{hyperref}

\usepackage{booktabs} % For formal tables
\usepackage{amsmath}
\usepackage{colortbl}
\usepackage{amsthm}
% \usepackage{cleveref}
\usepackage{gensymb}
\usepackage{multirow}
\usepackage{enumitem} % Remove itemize indent
\usepackage{mathtools}

\usepackage{pifont}% http://ctan.org/pkg/pifont
\newcommand{\cmark}{\textcolor{green}{\ding{51}}}%
\newcommand{\xmark}{\textcolor{red}{\ding{55}}}%

\newcommand{\norm}[1]{\left\lVert#1\right\rVert}
\def\MethodName{VNCA}
%%%%%%%%% PAPER ID  - PLEASE UPDATE

%%%%%%%%% TITLE - PLEASE UPDATE
\title{Volumetric Temporal Texture Synthesis for Smoke Stylization \\ using Neural Cellular Automata}

%%%%%%%%% AUTHORS - PLEASE UPDATE
\author{Dongqing Wang \quad Ehsan Pajouheshgar \quad Yitao Xu \quad Tong Zhang \quad Sabine Süsstrunk\\
School of Computer and Communication Sciences, EPFL \\
{\tt\small \{dongqing.wang, tong.zhang, alaa.abboud, sabine.susstrunk\}@epfl.ch}
} 

\begin{document}

\centering

\includegraphics[width=0.85\textwidth]{images/teaser.pdf}
\caption{
IP-Composer enables compositional generation from a set of visual concepts. These are portrayed through a set of input images, along with a prompt describing the desired concept to be extracted from each.
}

\twocolumn[{%
\maketitle%
%
\figfirstpagefigure%
}]

\begin{abstract}
Out-of-distribution (OOD) detection and OOD generalization are widely studied in Deep Neural Networks (DNNs), yet their relationship remains poorly understood. We empirically show that the degree of Neural Collapse (NC) in a network layer is inversely related with these objectives: stronger NC improves OOD detection but degrades generalization, while weaker NC enhances generalization at the cost of detection. This trade-off suggests that a single feature space cannot simultaneously achieve both tasks. To address this, we develop a theoretical framework linking NC to OOD detection and generalization. We show that entropy regularization mitigates NC to improve generalization, while a fixed Simplex Equiangular Tight Frame (ETF) projector enforces NC for better detection. Based on these insights, we propose a method to control NC at different DNN layers. In experiments, our method excels at both tasks across OOD datasets and DNN architectures. 

\begin{comment}   

Out-of-distribution (OOD) detection and OOD generalization are critical for deploying machine learning models in real-world scenarios. While substantial progress has been made in addressing these problems independently, few works have attempted to tackle them jointly. However, existing methods often rely on auxiliary OOD training data and primarily focus on covariate-shifted OOD data that share labels with in-distribution (ID) data. In contrast, we tackle the more realistic and challenging task of jointly detecting and generalizing to semantic OOD data with disjoint labels from the ID data, without auxiliary OOD training data.
Achieving both objectives simultaneously is inherently difficult due to a fundamental conflict — OOD generalization requires enhanced transferability, while OOD detection necessitates the inhibition of transfer.
To address this, we leverage insights from neural collapse (NC) — a phenomenon in deep networks where top-layer representations suppress feature variability and adopt a Simplex Equiangular Tight Frame (ETF) structure, impairing transferability. By controlling NC, we unify OOD detection and generalization: preventing NC enhances OOD transfer while inducing NC improves OOD detection.
Our proposed method excels at both tasks across various OOD datasets and architectures. 

\end{comment}


\end{abstract}
\section{Introduction}
\label{sec:intro}

Foundational models (FMs)~\cite{zhang2024data, zhou2023comprehensive} have shown remarkable progress in the healthcare domain, enabling professional-like assessment of disease diagnosis, treatment decision-making, and monitoring~\cite{zhang2023text, wang2022medclip, lu2023mi-zero}. 
Examples include LLaVA-Med~\cite{li2023llava}, Med-PaLM Multimodal~\cite{tu2024towards}, and Med-Flamingo~\cite{moor2023med}, have demonstrated their capacity on question answering, medical image analysis, and report generation.
These studies follow a predominant top-down model development strategy that requires upstream developers to collect data and train models for downstream tasks. 
Consequently, the developed model capabilities are heavily dependent on the training data, limiting their generalization performance in diverse clinical scenarios. 
For instance, Med-Gemini~\cite{yang2024advancing} reveals promising general capabilities in report generation while it lags behind state-of-the-art (SoTA) models on classification tasks, especially for out-of-domain applications. 
This indicates that while the generalizability of the foundation model is promising, more solutions are expected to meet the various specialized clinical needs.

To address these challenges, multi-center data centralization becomes essential to enhance model capacity and robustness across varied clinical scenarios~\cite{rajpurkar2022ai}. 
Centralizing distributed data can significantly improve model training and inference performance.
However, the process of medical data storage, transfer, and aggregation among centers requires extra efforts to ensure data security and system interoperability~\cite{bradford2020international}.
Moreover, a growing concern for patient privacy makes large-scale multi-center data sharing particularly challenging. 
While efforts like federated learning~\cite{wen2023survey, li2020review} can achieve good model performance on local data, the need for synchronized system coordination presents significant challenges, as clients are unable to update asynchronously. This limitation greatly restricts the practical capability of such approaches.
As a result, without a flexible collaboration, medical community still struggles to fully utilize the isolated data and local computation resources for comprehensive medical AI model development. 
To address this dilemma, open-source platforms encourage public data sharing and knowledge integration~\cite{markiewicz2021openneuro, zenodo}.
However, these platforms focus solely on raw data sharing while seldom providing collaborative model training or cooperation between different institutions.
Recently, collaborative learning has emerged as a viable approach for enhancing multi-model robustness~\cite{boulemtafes2020review}. 
For instance, software-like model development~\cite{raffel2023building} mimics software engineering practices by introducing structured workflows, enabling merging, version control, and continuous model integration.
Under this design, model ability can be strengthened with incremental knowledge updates similar to the version updating in software development. 

Although collaborative learning provides a multi-model collaboration, two key challenges remain in the leakage of raw data during collaboration~\cite{huang2023lorahub} and the synchronization of multiple collaborators~\cite{mcmahan2017communication} in the medical AI community. It is still challenging to integrate decentralized, privacy-sensitive data across institutions, leading to under-utilized insights and fragmented knowledge sharing~\cite{kaissis2020secure, rajpurkar2022ai, abdullah2021ethics}.
 To address these challenges, inspired by the collaborative software development, we propose \textbf{Med}ical \textbf{Fo}undation Models Me\textbf{rg}ing (\textbf{MedForge}), a cooperative workflow enabling continuously community-driven foundation model (FM) development.
MedForge enables a lightweight manner for individual centers to share their knowledge among multiple centers, minimizing the burden of data transmission and integration while enhancing model robustness.
Meanwhile, MedForge facilitates asynchronous and flexible collaboration, allowing individual centers to continuously update and improve medical FMs without the need for real-time synchronization.
Similar to open-source software development, MedForge incrementally updates medical knowledge and follows a sustainable model development scheme. 
This key design emphasizes a bottom-up construction of a multi-task medical FM, allowing downstream users to collaboratively build, refine, and update the upstream model according to their local resources. Our major contributions of MedForge are as below: 
\begin{enumerate}
    \item[$\bullet$] We introduce a collaborative workflow to promote the merging scheme of open-source software development. Our proposed MedForge allows distributed clinical centers to asynchronously contribute to comprehensive medical model construction while reducing transmitting costs among centers and avoiding the leakage of raw data, thus enhancing the utilization of private resources in the healthcare system. 
    \item[$\bullet$] We propose two effective knowledge-merging strategies for the asynchronous branch contribution. The MedForge-Fusion strategy updates the plugin module parameters of the main model during the merging phase, whereas the MedForge-Mixture strategy integrates the output of the plugin module by memorizing each contributor's coefficient. These strategies make MedForge more flexible and versatile. MedForge-Fusion is friendly to implement, while the MedForge-Mixture offers better performance and robustness.
    \item[$\bullet$]  We comprehensively evaluate model merging strategies to accumulate medical knowledge among multiple branch plugin modules. MedForge yields superior performance on medical classification tasks compared to other collaborative baselines across multiple datasets. We demonstrate the robustness of MedForge by shuffling the task order and evaluating various configurations of plugin modules and dataset distillation methods.
\end{enumerate}



\section{Related Work}
\label{sec:related}
\subsection{Collaborative Systems}
In the era of rapid growth in medical foundational models~\cite{huang2023visual,wang2022medclip, zhang2024data}, the top-down model development paradigm limits model capabilities by heavily relying on the resources available to the model builders. 
Such paradigm often restricts the potential of these models, as they cannot effectively utilize the diverse, private, and decentralized resources that exist within the broader medical community.
In contrast, collaborative systems present a promising alternative, offering a more flexible approach to model development.

Collaborative systems enable institutions to share knowledge by allowing distributed collaborators to contribute to a common goal~\cite{boulemtafes2020review}. 
To further protect patient privacy, federated learning (FL)~\cite{mcmahan2017communication} was proposed to alleviate such privacy concerns as server aggregating parameter updates from multiple clients without sharing their local data. 
While subsequent optimizations, such as aggregation algorithms~\cite{mcmahan2017communication, zhao2018federated, li2020federated}, secure learning~\cite{hardy2017private, xie2021crfl}, fairness improvements~\cite{sharma2022federated, zhao2022dynamic} and its application in medicine~\cite{kumar2024privacy}, have enhanced the capacity and applicability of FL, its real-world flexibility remains limited. This is primarily due to the need for synchronous updates, which require the server and clients to stay in sync, or model updates will be blocked.
This synchrony issue can be mitigated by open-source software platforms (e.g., GitHub~\cite{github}), allowing independent contributions from individual developers asynchronously. Such an asynchronous scheme enables faster iteration and the integration of specialized expertise, thus offering a more flexible and scalable approach.

Unlike synchronous collaboration, asynchronous collaboration does not require collaborators to work simultaneously and collaborators can individually complete their updates.
While the concept of asynchronous collaboration has been widely used in software development, its machine-learning applications remain limited~\cite{kandpal2023git, raffel2023building}. 
With the rise of global collaboration, large models~\cite{sahajBERT, le2023bloom} are usually co-developed by collaborators given various levels of data availability. However, this collaborative scheme requires the aggregation of local data and online synchronous cooperation of developers.
Software-like model update system~\cite{raffel2023building} alleviates the synchronous problem, where models are updated incrementally, similar to software development, by introducing merging and version control to model development.
However, the existing collaborative version control system~\cite{kandpal2023git} fails to address the complexities of medical scenarios because of the heavy dependency on plain parameter averaging across the full model without accounting for the varying requirements of different tasks.
To respond, we propose MedForge, which enables an asynchronous collaborative system and ensures strong robustness toward a continuous, community-driven enhancement of medical models while overcoming potential data leakage.

\begin{figure*}[t]
\begin{center}
\includegraphics[width=.85\linewidth]{fig_overview_v3.pdf}
\end{center}
\caption{
FastAtlas Overview: In each frame, we compute charts spanning fully or partially visible triangles (a), determine texture space bounding boxes for the visible portions of the view-space projections of each chart, and tightly pack these boxes into atlases (b, here $2K \times 2K$). We simultaneously bijectively parameterize and shade the charts into their atlas boxes, obtaining high quality texture space shading (c), and use this shading to render the shaded frames (d).}
\label{fig:overview}
\label{fig:alg_overview}
\end{figure*}

\section{Overview}
\label{sec:overview}
Our work has two core contributions: a real-time, GPU-based algorithm for tight packing of general parameterized charts into compact atlases; and a real-time TSS method that
utilizes this packing.  

\paragraph*{FastAtlas Packing.}
FastAtlas runs entirely on the GPU as a series of compute shaders. It takes the bounding boxes of parameterized charts as input, and packs them into an atlas (Fig~\ref{fig:overview}b, Sec.~\ref{sec:pack}). As such, the only input it requires are the dimensions of the bounding boxes.
Its outputs are deterministic; identical input charts are packed into identical atlases. This is critical for TSS and similar applications, as it ensures that consecutive frames taken from the same camera view have the same shading. Even minute shading differences across such frames can cause sampling jitter, leading to undesirable flicker \cite{baker2012rock}. 
While prior methods such as \cite{mueller2018shading,hladky2019tessellated,hladky2021snakebinning,Neff2022MSA} cap the dimensions of the charts that can be packed as-is for a given atlas size, and scale down all charts that exceed these dimensions, we scale all charts by the same factor, and do so only when strictly necessary to achieve packing success (Figs~\ref{fig:atlas},~\ref{fig:sas_issues}). 

\paragraph*{TSS using FastAtlas.}
Our end-to-end TSS atlas generation method combines the packing method above with a novel approach for computing seamless per-frame charts. 
We define our charts as the connected components of the visible surfaces in each frame (Fig.~\ref{fig:overview}a), and efficiently compute them using a parallel union-find algorithm (Sec.~\ref{sec:visible}). Since the boundaries of these charts coincide with the contours of the rendered surface, they are {\em invisible} to the viewer. This approach 
eliminates the artifacts caused by shading discontinuities along visible seams (Fig.~\ref{fig:seams}). 

\begin{parWithWrapFigure}
\begin{wrapfigure}{l}{.27\columnwidth}%
\includegraphics[width=\linewidth]{fig_inset_view_plane.pdf}%
\end{wrapfigure}
We bijectively parametrize the {\em visible portions} of our charts by projecting them to view space (inset). This maps a constant number of texels to each pixel in the final rendered output, evenly distributing residual undersampling error across all image pixels. While conceptually straightforward, efficiently parameterizing charts containing partially visible triangles using viewspace projection is non-trivial, as the visible portions may no longer be triangular (e.g. green triangle in the inset); applying naive projection to triangles with vertices behind the camera may produce ill-posed results. Clipping triangles before projection is both computationally expensive and significantly complicates downstream operations. We avoid explicit clipping by observing that all that is required for atlas packing is the dimensions of, potentially conservative, bounding boxes of these projected visible portions. We compute such bounding boxes without explicit chart clipping by adapting a conservative screen coverage estimator \shortcite{Blinn:CalculatingScreenCoverage} (Sec.~\ref{sec:box}). We then pack the computed boxes using FastAtlas. 
\end{parWithWrapFigure}

Finally, we shade the visible portion of each chart into its corresponding atlas bounding box (Fig~\ref{fig:overview}c). 
The resulting texture is then used during rasterization as a standard texture map (Fig. ~\ref{fig:overview}d). 
Our framework is compatible with all existing approaches for texture space shading, including forward shading, raytraced illumination, or deferred shading in texture space \cite{baker:2016}. In the examples shown, we use the standard forward shading based rendering pipeline included in the G3D Innovation Engine \cite{G3D17}, a commercial grade renderer.


\subsection{Model Merging}
In collaborative systems, proper model merging becomes increasingly vital for improving model knowledge integration from multiple sources in a resource-limited environment~\cite{li2023deep, yang2024model, goddard2024arcee}. Conceptually, model merging strategies can be categorized into entire model merging and partial model merging.

Entire model merging involves combining multiple model parameters to participate in the merging process by several means. Entire model merging can be viewed as an optimization problem~\cite{Matena_Raffel_2021, jin2022dataless, mavromatis2024packllm} or an alignment problem~\cite{ainsworth2022git, jordan2022repair, xu2024training, ainsworth2022git}, each offering unique advantages depending on the task at hand.
In the optimization-based approach, the goal is to find the best combination of multiple models to enhance performance and efficiency. For instance, using Fisher information approximation~\cite{Matena_Raffel_2021}, the optimization-based model merging can be interpreted as selecting parameters that maximize the joint likelihood of the models' posterior distributions. The optimization of model merging can also be guided by minimizing the prediction differences between the merged model and individual models~\cite{jin2022dataless}. 
With the development of large language models (LLM), optimization-based method is used to fuse multiple LLMs at test-time by minimizing perplexity over the input prompt~\cite{mavromatis2024packllm}.
To highlight, optimization-based methods are beneficial for scenarios requiring enhanced model performance and efficiency to integrate model parameters, while alignment-based methods~\cite{ainsworth2022git, jordan2022repair} are better suited for maintaining consistency and interpretability, facilitating critical information sharing across models.
For example, a training-free model merging strategy aligns relevant models by using a similarity matrix of their representations in both activation and weight spaces~\cite{xu2024training}.
Further, the alignment between the independently trained model and a reference model not only works for models with the same architecture but also for arbitrary model architectures~\cite{ainsworth2022git}.
In summary, the entire model merging methods can effectively integrate existing models into a merged model with enhanced functionality. However, they could lead to increased computational complexity and reduced flexibility, making them less scalable and harder to implement across diverse tasks.

Partial model merging refers to combining only specific components or layers of models to improve model merging efficiency and decrease the computational cost. 
Such specific components can come from the same network~\cite{kingetsu2021neural}, where the original network is divided into subnetworks for different purposes, and these subnetworks can then be recombined for new tasks.
Additionally, modules can originate from different functional networks and be merged using various strategies. For instance, arithmetic operations are applied in \cite{zhang2023composing} to fuse parameter-efficient modules.
While merging modules from different networks provides flexibility, the process also requires a selection strategy to ensure the resulting model aligns with the specific needs of the inference stage. 
The selection strategies are commonly designed based on the similarity of task~\cite{lv2023parameter} and domain clustering performance~\cite{chronopoulou2023adaptersoup}. Alternatively, the mixture-of-experts methods use a routing strategy to select appropriate component modules~\cite{ponti2023combining}. However, these strategies often require significant time and computational resources to filter through a large model pool. 
In contrast, LoRAHub~\cite{huang2023lorahub} offers a more lightweight approach, combining various LoRA modules for different tasks with minimal model training. Nevertheless, LoRAHub lacks flexibility for incremental updates, especially when handling unseen tasks.

Although the existing model merging approaches effectively combine the capabilities of individual models, these approaches often rely on raw data, leading to potential privacy risks. Our proposed MedForge emphasizes the prevention of raw data usage, which is particularly crucial in medical scenarios. Additionally, MedForge offers an extensible capability for incremental learning, enabling continuous model improvement.



\vspace{-5pt}
\section{Method}
\label{sec:method}
\begin{figure*}[t]
\begin{center}
\includegraphics[width=.85\linewidth]{fig_overview_v3.pdf}
\end{center}
\caption{
FastAtlas Overview: In each frame, we compute charts spanning fully or partially visible triangles (a), determine texture space bounding boxes for the visible portions of the view-space projections of each chart, and tightly pack these boxes into atlases (b, here $2K \times 2K$). We simultaneously bijectively parameterize and shade the charts into their atlas boxes, obtaining high quality texture space shading (c), and use this shading to render the shaded frames (d).}
\label{fig:overview}
\label{fig:alg_overview}
\end{figure*}

\section{Overview}
\label{sec:overview}
Our work has two core contributions: a real-time, GPU-based algorithm for tight packing of general parameterized charts into compact atlases; and a real-time TSS method that
utilizes this packing.  

\paragraph*{FastAtlas Packing.}
FastAtlas runs entirely on the GPU as a series of compute shaders. It takes the bounding boxes of parameterized charts as input, and packs them into an atlas (Fig~\ref{fig:overview}b, Sec.~\ref{sec:pack}). As such, the only input it requires are the dimensions of the bounding boxes.
Its outputs are deterministic; identical input charts are packed into identical atlases. This is critical for TSS and similar applications, as it ensures that consecutive frames taken from the same camera view have the same shading. Even minute shading differences across such frames can cause sampling jitter, leading to undesirable flicker \cite{baker2012rock}. 
While prior methods such as \cite{mueller2018shading,hladky2019tessellated,hladky2021snakebinning,Neff2022MSA} cap the dimensions of the charts that can be packed as-is for a given atlas size, and scale down all charts that exceed these dimensions, we scale all charts by the same factor, and do so only when strictly necessary to achieve packing success (Figs~\ref{fig:atlas},~\ref{fig:sas_issues}). 

\paragraph*{TSS using FastAtlas.}
Our end-to-end TSS atlas generation method combines the packing method above with a novel approach for computing seamless per-frame charts. 
We define our charts as the connected components of the visible surfaces in each frame (Fig.~\ref{fig:overview}a), and efficiently compute them using a parallel union-find algorithm (Sec.~\ref{sec:visible}). Since the boundaries of these charts coincide with the contours of the rendered surface, they are {\em invisible} to the viewer. This approach 
eliminates the artifacts caused by shading discontinuities along visible seams (Fig.~\ref{fig:seams}). 

\begin{parWithWrapFigure}
\begin{wrapfigure}{l}{.27\columnwidth}%
\includegraphics[width=\linewidth]{fig_inset_view_plane.pdf}%
\end{wrapfigure}
We bijectively parametrize the {\em visible portions} of our charts by projecting them to view space (inset). This maps a constant number of texels to each pixel in the final rendered output, evenly distributing residual undersampling error across all image pixels. While conceptually straightforward, efficiently parameterizing charts containing partially visible triangles using viewspace projection is non-trivial, as the visible portions may no longer be triangular (e.g. green triangle in the inset); applying naive projection to triangles with vertices behind the camera may produce ill-posed results. Clipping triangles before projection is both computationally expensive and significantly complicates downstream operations. We avoid explicit clipping by observing that all that is required for atlas packing is the dimensions of, potentially conservative, bounding boxes of these projected visible portions. We compute such bounding boxes without explicit chart clipping by adapting a conservative screen coverage estimator \shortcite{Blinn:CalculatingScreenCoverage} (Sec.~\ref{sec:box}). We then pack the computed boxes using FastAtlas. 
\end{parWithWrapFigure}

Finally, we shade the visible portion of each chart into its corresponding atlas bounding box (Fig~\ref{fig:overview}c). 
The resulting texture is then used during rasterization as a standard texture map (Fig. ~\ref{fig:overview}d). 
Our framework is compatible with all existing approaches for texture space shading, including forward shading, raytraced illumination, or deferred shading in texture space \cite{baker:2016}. In the examples shown, we use the standard forward shading based rendering pipeline included in the G3D Innovation Engine \cite{G3D17}, a commercial grade renderer.


Our goal is to increase the robustness of T2I models, particularly with rare or unseen concepts, which they struggle to generate. To do so, we investigate a retrieval-augmented generation approach, through which we dynamically select images that can provide the model with missing visual cues. Importantly, we focus on models that were not trained for RAG, and show that existing image conditioning tools can be leveraged to support RAG post-hoc.
As depicted in \cref{fig:overview}, given a text prompt and a T2I generative model, we start by generating an image with the given prompt. Then, we query a VLM with the image, and ask it to decide if the image matches the prompt. If it does not, we aim to retrieve images representing the concepts that are missing from the image, and provide them as additional context to the model to guide it toward better alignment with the prompt.
In the following sections, we describe our method by answering key questions:
(1) How do we know which images to retrieve? 
(2) How can we retrieve the required images? 
and (3) How can we use the retrieved images for unknown concept generation?
By answering these questions, we achieve our goal of generating new concepts that the model struggles to generate on its own.

\vspace{-3pt}
\subsection{Which images to retrieve?}
The amount of images we can pass to a model is limited, hence we need to decide which images to pass as references to guide the generation of a base model. As T2I models are already capable of generating many concepts successfully, an efficient strategy would be passing only concepts they struggle to generate as references, and not all the concepts in a prompt.
To find the challenging concepts,
we utilize a VLM and apply a step-by-step method, as depicted in the bottom part of \cref{fig:overview}. First, we generate an initial image with a T2I model. Then, we provide the VLM with the initial prompt and image, and ask it if they match. If not, we ask the VLM to identify missing concepts and
focus on content and style, since these are easy to convey through visual cues.
As demonstrated in \cref{tab:ablations}, empirical experiments show that image retrieval from detailed image captions yields better results than retrieval from brief, generic concept descriptions.
Therefore, after identifying the missing concepts, we ask the VLM to suggest detailed image captions for images that describe each of the concepts. 

\vspace{-4pt}
\subsubsection{Error Handling}
\label{subsec:err_hand}

The VLM may sometimes fail to identify the missing concepts in an image, and will respond that it is ``unable to respond''. In these rare cases, we allow up to 3 query repetitions, while increasing the query temperature in each repetition. Increasing the temperature allows for more diverse responses by encouraging the model to sample less probable words.
In most cases, using our suggested step-by-step method yields better results than retrieving images directly from the given prompt (see 
\cref{subsec:ablations}).
However, if the VLM still fails to identify the missing concepts after multiple attempts, we fall back to retrieving images directly from the prompt, as it usually means the VLM does not know what is the meaning of the prompt.

The used prompts can be found in \cref{app:prompts}.
Next, we turn to retrieve images based on the acquired image captions.

\vspace{-3pt}
\subsection{How to retrieve the required images?}

Given $n$ image captions, our goal is to retrieve the images that are most similar to these captions from a dataset. 
To retrieve images matching a given image caption, we compare the caption to all the images in the dataset using a text-image similarity metric and retrieve the top $k$ most similar images.
Text-to-image retrieval is an active research field~\cite{radford2021learning, zhai2023sigmoid, ray2024cola, vendrowinquire}, where no single method is perfect.
Retrieval is especially hard when the dataset does not contain an exact match to the query \cite{biswas2024efficient} or when the task is fine-grained retrieval, that depends on subtle details~\cite{wei2022fine}.
Hence, a common retrieval workflow is to first retrieve image candidates using pre-computed embeddings, and then re-rank the retrieved candidates using a different, often more expensive but accurate, method \cite{vendrowinquire}.
Following this workflow, we experimented with cosine similarity over different embeddings, and with multiple re-ranking methods of reference candidates.
Although re-ranking sometimes yields better results compared to simply using cosine similarity between CLIP~\cite{radford2021learning} embeddings, the difference was not significant in most of our experiments. Therefore, for simplicity, we use cosine similarity between CLIP embeddings as our similarity metric (see \cref{tab:sim_metrics}, \cref{subsec:ablations} for more details about our experiments with different similarity metrics).

\vspace{-3pt}
\subsection{How to use the retrieved images?}
Putting it all together, after retrieving relevant images, all that is left to do is to use them as context so they are beneficial for the model.
We experimented with two types of models; models that are trained to receive images as input in addition to text and have ICL capabilities (e.g., OmniGen~\cite{xiao2024omnigen}), and T2I models augmented with an image encoder in post-training (e.g., SDXL~\cite{podellsdxl} with IP-adapter~\cite{ye2023ip}).
As the first model type has ICL capabilities, we can supply the retrieved images as examples that it can learn from, by adjusting the original prompt.
Although the second model type lacks true ICL capabilities, it offers image-based control functionalities, which we can leverage for applying RAG over it with our method.
Hence, for both model types, we augment the input prompt to contain a reference of the retrieved images as examples.
Formally, given a prompt $p$, $n$ concepts, and $k$ compatible images for each concept, we use the following template to create a new prompt:
``According to these examples of 
$\mathord{<}c_1\mathord{>:<}img_{1,1}\mathord{>}, ... , \mathord{<}img_{1,k}\mathord{>}, ... , \mathord{<}c_n\mathord{>:<}img_{n,1}\mathord{>}, ... , $
$\mathord{<}img_{n,k}\mathord{>}$,
generate $\mathord{<}p\mathord{>}$'', 
where $c_i$ for $i\in{[1,n]}$ is a compatible image caption of the image $\mathord{<}img_{i,j}\mathord{>},  j\in{[1,k]}$. 

This prompt allows models to learn missing concepts from the images, guiding them to generate the required result. 

\textbf{Personalized Generation}: 
For models that support multiple input images, we can apply our method for personalized generation as well, to generate rare concept combinations with personal concepts. In this case, we use one image for personal content, and 1+ other reference images for missing concepts. For example, given an image of a specific cat, we can generate diverse images of it, ranging from a mug featuring the cat to a lego of it or atypical situations like the cat writing code or teaching a classroom of dogs (\cref{fig:personalization}).
\vspace{-2pt}
\begin{figure}[htp]
  \centering
   \includegraphics[width=\linewidth]{Assets/personalization.pdf}
   \caption{\textbf{Personalized generation example.}
   \emph{ImageRAG} can work in parallel with personalization methods and enhance their capabilities. For example, although OmniGen can generate images of a subject based on an image, it struggles to generate some concepts. Using references retrieved by our method, it can generate the required result.
}
   \label{fig:personalization}\vspace{-10pt}
\end{figure}
\begin{table*}[t!]
\centering
% \vspace{5pt}
\begin{small}
\begin{tabular}{l|c|c|c|c|c|c|c}
\toprule
\textbf{Method} & \textbf{Type} & \textbf{ToMi} & \textbf{BigToM} & \textbf{MMToM-QA} & \textbf{MuMA-ToM} & \textbf{Hi-ToM} & \textbf{All} \\
\midrule
SymbolicToM & Specific & \textbf{98.60} & - &  - & - & - & - \\
TimeToM & Specific & 87.80 & - &   - & - & - & - \\
% \textbf{96.00$^*$}
PercepToM & Specific & 82.90 & - & - & - & - & - \\
BIP-ALM & Specific & - & - & 76.70 & 33.90 & - & - \\
LIMP & Specific & - & - & - & 76.60 & - & - \\
\ours w/ Model Spec. & Specific & 88.80 & \textbf{86.75} & \textbf{79.83} & \textbf{84.00} & \textbf{74.00} & \textbf{82.68} \\
\midrule
Llama 3.1 70B & General & 72.00 & 77.83 & 43.83 & 55.78 & 35.00 & 47.41 \\
Gemini 2.0 Flash & General & 66.70 & 82.00 & 48.00 & 55.33 & 52.50 & 60.91\\
Gemini 2.0 Pro & General & 71.90 & 86.33 & 50.84 &  62.22 & 57.50 & 65.76 \\ 
GPT-4o & General & 77.00 & 82.42 & 44.00 & 63.55 & 50.00 & 63.39 \\
SimToM & General & 79.90 & 77.50 & 51.00 & 47.63 & 71.00 & 65.41\\ 
\ours & General & \textbf{88.30} & \textbf{86.92} & \textbf{75.50} & \textbf{81.44} & \textbf{72.50} & \textbf{80.93} \\
\bottomrule
\end{tabular}
\end{small}
\caption{Results of \ours and baselines on all benchmarks. There are two groups of methods: methods that require domain-specific knowledge (e.g., AutoToM w/ Model Spec.) or implementations (e.g., SymbolicToM) and methods that can be generally applied to any domain. ``-'' indicates that the domain-specific method is not applicable to the benchmark. The best results for each method type are highlighted in bold.}
\label{tab:results}
\vspace{-10pt}
\end{table*}



\section{Experiments}
\subsection{Experimental Settings}



We evaluated our method on multiple Theory of Mind benchmarks, including ToMi \citep{le2019revisiting}, BigToM \citep{gandhi2024understanding}, MMToM-QA \cite{jin2024mmtom}, MuMA-ToM \citep{shi2024muma}, and Hi-ToM \cite{he2023hi}. The diversity and complexity of these benchmarks pose significant reasoning challenges. For instance, MMToM-QA and MuMA-ToM incorporate both visual and textual input, while MuMA-ToM and Hi-ToM require higher-order inference. Additionally, MMToM-QA features exceptionally long contexts, and BigToM presents open-ended scenarios.



Besides the full \ours method, we additionally evaluated \ours given manually specified models (AutoToM w/ Model Spec.). 

We compared \ours against state-of-the-art baselines:
    \textbf{LLMs:} Llama 3.1 70B \citep{dubey2024llama}, Gemini 2.0 Flash, Gemini 2.0 Pro \cite{team2023gemini} and GPT-4o \cite{achiam2023gpt};
    
     \textbf{ToM prompting for LLMs:} SymbolicToM \cite{sclar2023minding}, SimToM \cite{wilf2023think}, TimeToM \cite{hou2024timetom}, and PercepToM \citep{jung2024perceptions};
 
  \textbf{Model-based inference:} BIP-ALM \cite{jin2024mmtom} and LIMP \cite{shi2024muma}.


For multimodal benchmarks, MMToM-QA and MuMA-ToM, we adopt the information fusion methods proposed by \citet{jin2024mmtom} and \citet{shi2024muma} to fuse information from visual and text inputs respectively. The fused information is in text form. We ensure that all methods use the same fused information as their input.


We use GPT-4o as the LLM backend for \ours and all ToM prompting and model-based inference baselines to ensure a fair comparison—except for TimeToM, which relies on GPT-4 and is not open-sourced.


\subsection{Results}
The main results are summarized in Table~\ref{tab:results}. Unlike \ours, many recent ToM baselines can only be applied to specific benchmarks. Among general methods, \ours achieves state-of-the-art results across all benchmarks. In particular, it outperforms its LLM backend, GPT-4o, by a large margin. This is because Bayesian inverse planning is more robust for inferring mental states given long contexts with complex environments and agent behavior. It is also more adept at recursive reasoning which is key to higher-order inference. Notably, \ours performs comparably to manually specified models, showing that automatic model discovery without domain knowledge is as effective as human-provided models. We provide additional results and qualitative examples in Appendix~\ref{sec:more_results}.


\subsection{Ablated Study}



\begin{figure}[t!]
  \centering
  \includegraphics[width=0.8\linewidth]{figures/comparison.pdf}
    \vspace{-10pt}
  \caption{Averaged performance and compute of the full \ours method (star) and the ablated methods (circles) on all benchmarks.}
  \label{fig:ablation}
  \vspace{-10pt}
\end{figure}


We evaluated the following variants of \ours for an ablation study: no hypothesis reduction (\textbf{w/o hypo. reduction}); always using POMDP (\textbf{w/ POMDP}); always using the initial model proposal without variable adjustment (\textbf{w/o variable adj.}); only considering the last timestep (\textbf{w/ last timestep}); and considering all timesteps without timestep adjustment (\textbf{w/ all timesteps}).

The results in Figure~\ref{fig:ablation} show that the full \ours method constructs a suitable BToM model, enabling rich ToM inferences while reducing compute. We analyze key model components below:

\textbf{Hypothesis reduction.}
Compared to the full method, \ours w/o hypo. reduction has a similar accuracy but consumes 53\% more tokens on average, demonstrating that hypothesis reduction optimizes efficiency without sacrificing performance.

\textbf{Variable adjustment.}
\ours dynamically identifies relevant variables for ToM inference, generalizing domain-specific BIP approaches to open-ended scenarios. Compared to its variant without variable adjustment, \ours improves performance with minimal additional compute. The variant that always uses POMDP performs well in scenarios aligned with the POMDP assumption (e.g., MMToM-QA) but generalizes poorly elsewhere and incurs much higher computational costs. %, leading to an 8.5% performance deficit.

\textbf{Timestep adjustment.}
By selecting relevant steps for inference, timestep adjustment enhances performance by focusing on essential information. In contrast, the variant using only the last timestep misses crucial details, significantly lowering performance. The variant incorporating all timesteps suffers from higher computational costs and reduced accuracy due to conditioning on unnecessary, potentially distracting information.



Full ablation results are provided in Appendix~\ref{sec:more_results_ablation}.

\section{Application to Mesh Texturing}
\label{sec:mesh_texturing}
VNCA can easily be extended for mesh texturing given a reference style following the line of research in Solid Texture Synthesis \cite{perlin_noise, kopf2007solid, on-demand-solid-texture}.
\Cref{fig:mesh} shows the stylized meshes by our method in comparison with the results from A. \citet{on-demand-solid-texture} and B. \citet{kopf2007solid} as two representative solid texturing methods for five different style images. 



\section{Conclusion and Discussion}

In this paper, we have presented VNCA, a novel volumetric temporal texture for stylizing volumetric smoke sequences with 2D exemplars. Our method achieves 1. high-quality appearance stylization with multiview and across-frame coherence w.r.t a reference image, 2. visually plausible motion as the original smoke data, and 3. speedup training time.
Our model can be extended to mesh texturing with a texture volume to be applied on various meshes.


\subsection{Limitations. }
Since we use the NCA framework, we inherit its shortcomings in that the homogenous update rule for NCA limits the synthesized texture from achieving high heterogeneity. 
In other words, VNCA can extract repeated patterns in the style image as features for stylization, but may not be able to synthesize stylizing features with a clear foreground and background.

{
    \small
    \bibliographystyle{ieeenat_fullname}
    \bibliography{main}
}



\begin{figure*}
  \centering
  \includegraphics[width=0.8\textwidth]{img/ours.png}
  \caption{\textbf{Stylization on Smoke Simulations. } VNCA can both synthesize RGB color and modify density distribution for a stylization consistent with the input style image.  } 
  \label{fig:res}
\end{figure*}


\begin{figure*}
  \centering
  \includegraphics[width=0.85\textwidth]{img/mesh.pdf}
  \caption{\textbf{Stylization on Meshes. } VNCA can be generalized and applied to mesh texturing. A. \citet{on-demand-solid-texture}, B. \citet{kopf2007solid}. Our method synthesizes mesh stylization with better consistency with the style image and fine-level details. }
  \label{fig:mesh}
\end{figure*}



\end{document}
