%% 
%% Copyright 2007-2024 Elsevier Ltd
%% 
%% This file is part of the 'Elsarticle Bundle'.
%% ---------------------------------------------
%% 
%% It may be distributed under the conditions of the LaTeX Project Public
%% License, either version 1.3 of this license or (at your option) any
%% later version.  The latest version of this license is in
%%    http://www.latex-project.org/lppl.txt
%% and version 1.3 or later is part of all distributions of LaTeX
%% version 1999/12/01 or later.
%% 
%% The list of all files belonging to the 'Elsarticle Bundle' is
%% given in the file `manifest.txt'.
%% 
%% Template article for Elsevier's document class `elsarticle'
%% with numbered style bibliographic references
%% SP 2008/03/01
%% $Id: elsarticle-template-num.tex 249 2024-04-06 10:51:24Z rishi $
%%
\documentclass[preprint,12pt]{elsarticle}


%% Use the option review to obtain double line spacing
%% \documentclass[authoryear,preprint,review,12pt]{elsarticle}

%% Use the options 1p,twocolumn; 3p; 3p,twocolumn; 5p; or 5p,twocolumn
%% for a journal layout:
%% \documentclass[final,1p,times]{elsarticle}
%% \documentclass[final,1p,times,twocolumn]{elsarticle}
%% \documentclass[final,3p,times]{elsarticle}
%% \documentclass[final,3p,times,twocolumn]{elsarticle}
%% \documentclass[final,5p,times]{elsarticle}
%% \documentclass[final,5p,times,twocolumn]{elsarticle}

%% For including figures, graphicx.sty has been loaded in
%% elsarticle.cls. If you prefer to use the old commands
%% please give \usepackage{epsfig}

%% The amsthm package provides extended theorem environments
%% \usepackage{amsthm}
\usepackage{graphicx} %用于图片
\usepackage{amsmath} %用于公式
% \usepackage[colorlinks,linkcolor=black,anchorcolor=black,citecolor=black]
\usepackage{flushend} %用于平衡最后一版
\usepackage[ruled,linesnumbered]{algorithm2e} %用于伪代码,含有竖线
\usepackage{amssymb}
\usepackage{booktabs}
\usepackage{float}
\usepackage{array}
\usepackage{subfigure}
\usepackage{multirow} % Required for multirows 表格的包
\usepackage{tabularx} %  表格的包
\usepackage{makecell}  %换行的包
\setcounter{secnumdepth}{4}  %4级标题
\setcounter{tocdepth}{4}  %4级标题
%% The lineno packages adds line numbers. Start line numbering with
%% \begin{linenumbers}, end it with \end{linenumbers}. Or switch it on
%% for the whole article with \linenumbers.
%% \usepackage{lineno}

\journal{Nuclear Physics B}

\begin{document}

\begin{frontmatter}

%% Title, authors and addresses

%% use the tnoteref command within \title for footnotes;
%% use the tnotetext command for theassociated footnote;
%% use the fnref command within \author or \affiliation for footnotes;
%% use the fntext command for theassociated footnote;
%% use the corref command within \author for corresponding author footnotes;
%% use the cortext command for theassociated footnote;
%% use the ead command for the email address,
%% and the form \ead[url] for the home page:
%% \title{Title\tnoteref{label1}}
%% \tnotetext[label1]{}
%% \author{Name\corref{cor1}\fnref{label2}}
%% \ead{email address}
%% \ead[url]{home page}
%% \fntext[label2]{}
%% \cortext[cor1]{}
%% \affiliation{organization={},
%%             addressline={},
%%             city={},
%%             postcode={},
%%             state={},
%%             country={}}
%% \fntext[label3]{}

\title{Multi-Perspective Attention Mechanism for Bias-Aware Sequential Recommendation}

%% use optional labels to link authors explicitly to addresses:
%% \author[label1,label2]{}
%% \affiliation[label1]{organization={},
%%             addressline={},
%%             city={},
%%             postcode={},
%%             state={},
%%             country={}}
%%
%% \affiliation[label2]{organization={},
%%             addressline={},
%%             city={},
%%             postcode={},
%%             state={},
%%             country={}}

\author[label1]{Mingjian Fu}\ead{sinceway@fzu.edu.cn} %% Author name
\author[label1]{Hengsheng Chen}\ead{231027050@fzu.edu.cn} %% Author name
\author[label1]{Dongchun Jiang}\ead{948234053@qq.com} %% Author name
%%\author[label1]{Fenglin Ni}\ead{877209078@qq.com}
\author[label1]{Yanchao Tan\corref{cor}}\ead{yctan@fzu.edu.cn}
%% Author affiliation
\cortext[cor]{Corresponding author.}
\affiliation[label1]{organization={College of Computer and Data Science},%Department and Organization
            addressline={Fuzhou University}, 
            city={Fuzhou},
            postcode={350108}, 
            country={China}}
        
%% Abstract
\begin{abstract}
In the era of advancing information technology, recommender systems have emerged as crucial tools for dealing with information overload. However, traditional recommender systems still have limitations in capturing the dynamic evolution of user behavior. To better understand and predict user behavior, especially taking into account the complexity of temporal evolution, sequential recommender systems have gradually become the focus of research. Currently, many sequential recommendation algorithms ignore the amplification effects of prevalent biases, which leads to recommendation results being susceptible to the Matthew Effect. Additionally, it will impose limitations on the recommender system's ability to deeply perceive and capture the dynamic shifts in user preferences, thereby diminishing the extent of its recommendation reach. To address this issue effectively, we propose a recommendation system based on sequential information and attention mechanism called Multi-Perspective Attention Bias Sequential Recommendation (MABSRec). Firstly, we reconstruct user sequences into three short types and utilize graph neural networks for item weighting. Subsequently, an adaptive multi-bias perspective attention module is proposed to enhance the accuracy of recommendations. Experimental results show that the MABSRec model exhibits significant advantages in all evaluation metrics, demonstrating its excellent performance in the sequence recommendation task. 
\end{abstract}

% %%Graphical abstract
% \begin{graphicalabstract}
% %\includegraphics{grabs}
% \end{graphicalabstract}

% %%Research highlights
% \begin{highlights}
% \item Research highlight 1
% \item Research highlight 2
% \end{highlights}

%% Keywords
\begin{keyword}
%% keywords here, in the form: keyword \sep keyword
Sequential recommendation, attention mechanism, graph neural network, bias handling.
%% PACS codes here, in the form: \PACS code \sep code

%% MSC codes here, in the form: \MSC code \sep code
%% or \MSC[2008] code \sep code (2000 is the default)

\end{keyword}

\end{frontmatter}

%% Add \usepackage{lineno} before \begin{document} and uncomment 
%% following line to enable line numbers
%% \linenumbers

%% main text
%%

%% Use \section commands to start a section
\section{Introduction}
The rapid advancement of information technology has led to significant progress in the production, storage, dissemination, and acquisition of information. However, this trend has also resulted in an unprecedented explosion of information that far exceeds the processing capabilities of individuals and organizations, thereby diminishing the utility of information \cite{10.1111/jcc4.12178}, \cite{doi:10.1287/orsc.1100.0634}. Meanwhile, this growth in information also poses serious challenges to the design and management of information systems \cite{10.3389/fpsyg.2023.1122200}. In this context, recommender systems have emerged as a powerful tool to address the challenge of information overload effectively. These systems provide highly personalized recommendations based on users' historical behavior and interests. Moreover, by leveraging implicit user feedback, such as clicks and purchases, recommender systems can perform deep learning and deliver personalized recommendations without relying solely on explicit search keywords \cite{4781121}. Therefore, in the context of information overload, recommendation systems have significantly optimized the experience of information retrieval, providing users with more personalized and diversified information services.

The application of recommender systems is extensive, spanning various fields such as e-commerce \cite{10.1145/3411564.3411572}, social media \cite{10.1145/1458082.1458205}, and tourism \cite{Braunhofer2014TechniquesFC}. Traditional recommender systems are generally classified into three main categories: collaborative filtering-based \cite{10.1145/3460231.3478854}, \cite{FANG2022109044}, \cite{10.1145/3459637.3482354}, content-based \cite{HUANG2022108596}, \cite{DELCARMENRODRIGUEZHERNANDEZ2021106740}, \cite{9773925}, and hybrid recommender systems \cite{10.1145/3460231.3474272}, \cite{10.1145/3511808.3557354}, \cite{9723533}. Collaborative filtering-based recommender systems utilize cosine similarity, Pearson's correlation coefficient, and other computational methods to measure the similarity between users or items to recommend items of similar interest to users. Content-based recommender systems focus on analyzing the characteristic attributes of items to achieve personalized recommendations by gaining insight into the user's interest in these characteristics. Hybrid recommender systems combine collaborative filtering-based and content-based recommender systems, aiming to overcome the shortcomings of a single recommendation algorithm and provide more accurate and diverse personalized recommendations. Although traditional systems have achieved notable success in meeting user needs, they struggle to capture the dynamic evolution of user behavior over time. To this end, the researchers carried out a study on sequence recommendation. Sequential recommender systems have been developed to emphasize temporal information, enabling better understanding and predicting user behavior, particularly its temporal evolution \cite{10.1145/3511808.3557268}. Bao et al. \cite{10.1145/3604915.3608857} proposed TALLRec to efficiently train Large Language Models (LLMs) by transforming recommendation data into instructions using two fine-tuning techniques, Alpaca Tuning and Rec-Tuning. Ren et al. \cite{ijcai2023p254} proposed a review-based recommendation method that leverages a self-supervised graph decomposition network. This network learns separate representations of users and items on latent factors through a graph decomposition learning module. Additionally, they introduced an intent-aware contrastive learning task to alleviate data sparsity and enhance the separation of user and item representations. Du et al. \cite{10.1145/3539618.3591679} proposed a framework called EMKD to improve the accuracy of sequence recommendation by training multiple parallel networks and performing comparative knowledge distillation. 
% Silva et al. \cite{pmlr-v235-silva24b} revealed the unexpected effectiveness of reinforcement learning in sequential recommender systems. Afterward, the authors proposed simpler auxiliary objective functions to replace reinforcement learning, achieving similar performance gains.

\begin{figure*}[h]
    \centering
    \includegraphics[width=1\linewidth]{Figure/Matthew_Effect.png}
    \caption{Matthew Effect in Recommendation}
    \label{fig:Matthew}
\end{figure*}

Although existing research on sequential recommender systems has demonstrated commendable performance, studies on the impact of various bias factors present in user data remain insufficient. Notably, prevalent biases such as popularity bias and amplified subjective bias are frequently observed in user-item interactions \cite{Guo2024}. Failure to account for these prevalent biases in recommender systems often leads to the manifestation of the Matthew Effect \cite{9778074}. To solve this problem, we propose a novel sequence recommendation model, called Multi-Perspective Attention Bias Sequential Recommendation (MABSRec). The model constructs multiple bias views from multiple bias perspectives, explicitly incorporating both popular bias and amplified subjective bias in user data. Subsequently, a graph convolution operation is used to enrich the representation of each item within each bias view. Finally, an attention fusion network is utilized to weigh the impact of various biases on users and output the predicted results. Experimental evaluations conducted on three real-world datasets indicate that the MABSRec model has significant advantages in all evaluation indexes, showing its excellent performance in the sequence recommendation task. The principal contributions of this study are delineated as follows:
\begin{itemize}
    \item We consider the prevalent bias in the user data and the amplified subjective bias, and divide the user sequence data into three biased short sequences: popularity-biased short sequences, subjectivity-biased short sequences, and debiased short sequences. Then, by constructing a graph convolutional neural network, the three kinds of biased sequences are constructed into the corresponding sequence graph information, so that the sequence recommendation model can more deeply mine the biased information of other users. 
    \item We adopt the multi-head self-attention mechanism with shared weights to capture the key information in the user sequence to capture the dynamic interest changes of users better. In addition, we introduce an adaptive multi-perspective attention bias module to fuse the three bias information to comprehensively consider the impact of different biases on user behavior.
    \item We comprehensively evaluate the proposed MABSRec model across multiple dimensions using three real-world datasets. Extensive experimental results substantiate the effectiveness and superiority of MABSRec. Additionally, ablation studies confirm the rationality and efficacy of the key components.
\end{itemize} 

The rest of this article is structured as follows: Section \ref{related work} provides a review of related work. Section \ref{method} introduces the problem definition and details of the proposed MABSRec model. Section \ref{experimental} evaluates the performance of the proposed MABSRec model and verifies the effectiveness of key components. Section \ref{conclusion} summarizes this work.

\section{Related Work}
\label{related work}
Sequence recommendation is one of the crucial tasks in recommender systems, and its core goal is to accurately predict the possible future sequence of items of interest based on the user's historical behavior sequence. At the early stage of the research on sequence recommendation, scholars drew on methods such as Markov chains to deal with the recommendation problem of sequence information \cite{10.1145/1772690.1772773}, \cite{10.1145/2766462.2767694}. However, with the rapid development of deep neural networks, deep learning models such as Convolutional Neural Network (CNN) and Recurrent Neural Network (RNN) have been introduced to sequence recommendation tasks. Although these traditional methods have achieved some success, they generally suffer from the long-tail effect, slow training speed, and limited ability to extract temporal features in long sequence recommendation scenarios.

In recent years, the attention mechanism, Graph Neural Network (GNN), and Contrastive Learning (CL) have been applied to sequence recommendation models due to their powerful comprehension capabilities, and have become mainstream research directions. In this section, we review the sequence recommendation models using these techniques.

\subsection{Attention Mechanism}
The attentional mechanism, an approach inspired by the human visual and cognitive systems, allows neural networks to selectively focus on important information within the input data during processing. This mechanism effectively improves the performance and generalization of the model. Zhou et al. \cite{10.1609/aaai.v33i01.33015941} introduced the attention mechanism into the Gated Recurrent Unit (GRU) model and designed the AUGRU module to cope with the problem of user interest drift. This method enables the model to better adapt to changes in user interests, however, it significantly reduces the computational efficiency of the network due to the high complexity of the GRU module in the computation process. Kang et al. \cite{8594844} proposed the SASRec recommendation model, an adaptation of the Transformer architecture \cite{10.5555/3295222.3295349}. SASRec introduces trainable positional embeddings, enabling the model to differentiate items based on their positional context. Additionally, it employs a multi-head self-attention mechanism to capture intricate relationships within sequences, thereby enhancing the model's ability to represent inter-item dependencies. Shin et al. \cite{Shin_Choi_Wi_Park_2024} proposed the BSARec model to balances the strengths of both approaches and mitigates the over-smoothing problem, which combines the Fourier transform and the self-attention mechanism. Liu et al. \cite{10.1145/3539618.3591717} proposed a novel linear attention mechanism in long sequence recommendation systems, named LinRec. LinRec reduces the complexity of the Transformer by changing the dot product order, $L_2$-normalization, and ELU activation, while maintaining the accuracy.
% Since the front-to-back order in a sequence is not necessarily strictly ordered, the left-to-right unidirectional modeling approach of RNN and SASRec can limit the ability of hidden representations in sequences of user behaviors. To address this problem, BERT4Rec \cite{10.1145/3357384.3357895} sequence recommendation model is proposed. BERT4Rec utilizes a bi-directional multi-head attention mechanism to capture sequence information and then learns this bi-directional representation model by predicting random masking terms in a sequence through joint conditioning on the left and right contexts of the random masking terms in the sequence. 

% In addition, some recommendation models combine the attention mechanism with Long Short-Term Memory (LSTM) to learn both long-term and short-term interest preferences in user sequence data \cite{10.1145/3357384.3357818}. The model utilizes a multi-head self-attention mechanism to capture the multiple interest preferences present in a sequential session on the one hand and uses LSTM to learn long-term interest features on the other. Finally, the long and short-term user interests are fused by a simple gating network to synthesize the long-term and short-term interests of the users, thus improving the performance of the recommender system. Lin et al. \cite{10.1145/3511808.3557095} proposed an approach to fuse attention mechanisms with memory networks. This method uses a target item and a memory vector as a dual query to retrieve information in a long sequence, and then the retrieved sequence information is used to update the memory vector. Then, the updated memory vectors are used to query the sequence to model the long-term dependencies within the sequence, thus solving the difficulties faced in modeling very long sequences.

\subsection{Graph Neural Network}
GNN is a class of deep learning models specifically designed for processing graph-structured data, among which the Graph Convolutional Network (GCN) \cite{kipf2017semisupervised} is one of the most representative models. GCN extracts feature information by performing convolutional operations on graphs to achieve deep learning and analysis of graph data. In recent years, many researchers have applied graph neural networks to the recommendation domain, achieving significant experimental results. Wang et al. proposed \cite{10.1145/3331184.3331267} an innovative recommendation method, called NGCF, which considers both users and items as nodes in a graph structure and constructs an information network using user-item interaction records. By representing users and items as nodes and capturing their interactions, NGCF effectively extends user and item representations. LightGCN \cite{10.1145/3397271.3401063} simplifies the model structure by using a simple aggregation weighting method to enhance the efficiency of model training and implementation. Specifically, LightGCN utilizes the user-item interaction matrix to construct the graph structure and updates node features through weighted aggregation. Chang et al. \cite{10.1145/3404835.3462968} proposed a metric learning-based approach for building user interest graph structures. This method first constructs a dynamic graph by measuring node similarity as a key metric and connecting two nodes only when their similarity exceeds a predefined threshold. Subsequently, the importance of nodes surrounding the target node is modeled by computing attention scores between the target node and its neighbors, as well as the attention coefficients of neighboring nodes relative to the target node. TransGNN \cite{10.1145/3626772.3657721} combines the strengths of the Transformer and GNN to help the GNN expand its receptive field. TransGNN uses three types of positional encodings to capture graph structural information and then alternates between the Transformer and GNN layers to focus each node on the most relevant samples. 

% In the research of applying graph neural networks to sequence recommendation, researchers usually construct the historical user-item interaction sequences into a sequence graph structure and then use graph neural networks to aggregate the neighbor information, to achieve the updating of node features and representation optimization. SR-GNN \cite{Wu_Tang_Zhu_Wang_Xie_Tan_2019} is a recommendation method based on graph neural networks, which is specifically used to deal with the information of user's session sequences. It first constructs the user behavior sequence as a directed graph structure and iteratively updates the feature information of each node using a gated graph neural network (GGNN). In this process, GGNN can effectively capture the transfer patterns and dependencies between nodes, thus reflecting the dynamics and complexity of user behavior. Finally, with the help of attention network, SR-GGNN weights and fuses the node features in the graph to obtain a representation of the whole session, which is used to predict the user's next behavior.  Ma et al. \cite{Ma_Ma_Zhang_Sun_Liu_Coates_2020} proposed MA-GNN that integrates the long and short-term interests of users to efficiently model user sequences. For users' short-term interests, MA-GNN performs graph modeling of recent session sequences through a sliding window strategy. For the long-term interests of users, MA-GNN uses a memory network to assist in modeling. This memory network records the potential long-term interest characteristics of all users, providing the system with global long-term interest information. Finally, short-term interest, long-term interest, and other related features are fused through the gating network to achieve a comprehensive multi-dimensional consideration of user interest. This approach not only fully considers the user's long-term and short-term interests, but also realizes the effective fusion of multiple interest features through the design of the gating network, thus improving the accuracy and efficiency of interest modeling.

\subsection{Contrastive Learning}
The core goal of CL is to achieve effective data representation learning by minimizing the distance between positive samples and a given anchor while maximizing the distance between negative samples and the same anchor. In sequence recommendation, CL can address the challenges posed by sparse interaction data and enhance the learning of effective representations \cite{lee2023hierarchicalcontrastivelearningmultiple}, \cite{10.1145/3539618.3591692}. CL4SRec \cite{9835621} is the first to introduce CL into the sequential recommendation domain. CL4SRec combines the traditional sequential prediction objective with a CL objective to improve the accuracy and versatility of recommender systems. Specifically, CL4SRec constructs user sequences from different perspectives and utilizes a contrastive loss function to learn more accurate user representations. DuoRec \cite{10.1145/3488560.3498433} improves the distribution of sequence representations and item embeddings by introducing CL regularization while utilizing both unsupervised and supervised contrastive samples. DCRec \cite{10.1145/3543507.3583361} improves the accuracy and diversity of recommender systems by unifying sequential pattern encoding with global synergistic relationship modeling through adaptive consistency-aware augmentation. In addition, DCRec utilizes CL for self-supervised signal extraction across different views, effectively capturing intra-sequence item transition patterns and inter-sequence user dependencies.


\section{The Proposed Method}
\label{method}
In this section, we first define the sequence recommendation problem. Then, we illustrate the novel framework MABSRec
in detail. The framework is shown in Figure \ref{fig: framework}. Finally, we present the definition of the loss function.
\begin{figure*}
    \centering
    \includegraphics[width=1\linewidth]{Figure//method/6.png}
    \caption{Framework diagram of the MABSRec model}
    \label{fig: framework}
\end{figure*}

\subsection{Problem Definition}
For the sequence recommendation problem, let $U$ and $I$ denote a set of users and a set of items, respectively. For each user $u \in U$, their historical interaction sequence can be defined as $S_u=\left(s_u^1, s_u^2, \cdots, s_u^t\right)$, where $s_u^j$ denotes the user's interaction with item $i \in I$ at the $j$ time step. $t$ denotes the length of the historical interaction sequences of different users, and $j$ denotes the position of item $i$ in the historical interaction sequence $S_u$ of user $u$. The goal of sequential recommendation is to maximize the likelihood of inferring from a collection of items the items that a user is most likely to interact with next, by giving a history of interactions. The objective function is as follows:
\begin{equation}
\arg \max _{i \in I} P\left(s_u^{t+1}=i \mid S_u\right)
\end{equation}

Since different users may have different sequence lengths, to maintain consistency, we perform complementary zero padding for sequence lengths less than the preset length and intercept sequence lengths that are more than the preset length.

% \subsection{Model Framework}
% The MABSRec model is based on user data and aims to overcome the effects of popularity bias and amplified subjective bias. 
% % The core idea lies in finely slicing the sequence information on which the deviation graph structure is constructed. After that, neighbor messages within various deviation graphs are passed through graph convolutional neural networks to enrich the sparse data. Subsequently, the Transformer module is utilized to dynamically capture the changes in sequence information and output the results of three different deviation types. Finally, the final prediction results are formed by fusing multiple deviations using an adaptive multi-view deviation attention network.
% The overall framework of MABSRec is shown in Figure \ref{fig: framework}, initially the model receives each user sequence as input, where circles represent items. Then, the model analyzes and processes the user's preference information to better understand the user's personalized needs and preference tendencies. Specifically, the model divides the user sequence data into three types of short sequence information: popularity-biased, subjectivity-biased, and debiased. $\mathcal{G}_{\mathcal{P}}$, $\mathcal{G}_{\mathcal{A}}$, and $\mathcal{G}_{\mathcal{D}}$ are the sequence graph structures constructed based on the above three types of short deviation sequences. These graph structures are formed by considering items as nodes and relationships between items in the sequence as edges. In the graph structures, each node represents an item, while the edges represent the sequence relationships between items. Subsequently, for each of these three different graph structures, the model performs a graph convolution operation to pass the neighbor information of each node to enrich the representation of each item in each deviant graph perspective. Next, the Transformer module is fed for capturing the dependencies and importance between different parts of the sequence to get the user's attention under different deviation perspectives. Through the Transformer module, the model can more accurately understand the user's preferences and behavioral patterns under different deviant perspectives. Finally, the model fuses the information about the users' attention points under each perspective to get the final prediction results. This fused result will be used as the final prediction output to provide users with more targeted and personalized recommendation services. 

\subsection{Multi-bias Processing in Data}
In recommender systems, data is often subject to popularity bias and amplified subjective bias. To visualize the popularity that each item receives in the dataset, MABSRec uses the number of times an item appears in the dataset to represent its popularity, denoted as $S_p$. The more times an item appears, the more popular it is. In addition, for the personalized preferences in the user sequences, the user's personalized weights for the items are defined as $S_c$. The weights of the items are calculated as follows:

\begin{equation}
\left\{\begin{array}{l}
S_{p}(u, i) = count(i) \\
S_{c}(u, i) = \frac{c_{i} \cdot c_{u}}{\left|c_{i}\right|}
\end{array}\right.
\end{equation}

where $count(i)$ denotes the count of the number of occurrences of each item $i$ in the dataset, $c_i$ is the one-hot representation of the category of item $i$ in a sequence. $\left|c_i\right|$ denotes the number of categories corresponding to the item. $c_u$ denotes the vector of the number of occurrences of all the categories in the sequence. By considering $S_p$ and $S_c$, MABSRec can more accurately understand the prevalence bias and the user's personalized preference in the recommender system. As shown in Figure \ref{fig: calculation}, 
\begin{figure*}[ht]
    \centering
    \includegraphics[width=1\linewidth]{Figure//method/7.png}
    \caption{Calculation of the degree of bias}
    \label{fig: calculation}
\end{figure*}all items in the sequence have seven category attributes: "Sports $\&$ Outdoors", "Other Sports", "Dance", "Paintball $\&$ Airsoft", "Boating $\&$ Water Sports", "Fan Shop", and "Home $\&$ Kitchen". For item $i_{1}$, it contains three categories: "Sports $\&$ Outdoors", "Other Sports", and "Dance". Therefore, the category one-hot corresponding to the item $i_{1}$ is denoted as $[1,1,1,0,0,0,0]$, and $\left|c_{i}\right|$ is 3, which denotes the number of categories of item $i_{1}$. $c_{u}$ is the vector of the number of occurrences of all the sequence categories, denoted as $[3,1,3,2,1,3,3]$. In addition, the orange numbers indicate the number of times each item appears in the dataset, i.e., $S_{p}$, and the green numbers indicate the weight, i.e., $S_{c}$, computed based on the category or other attributes of each item in the sequence.

Then, we filter and reorganize the items in the user sequences from three biased perspectives. We categorize those items that are popular among the entire user community but account for less in terms of personalized preferences as popular bias perspective items, denoted as $I_\mathcal{P}^\mathcal{S}$, and computed by the following formula:
\begin{align}
   I_{\mathcal{A}}^{S}=&\left\{i\mid rank\left(S_{_p}(u,i)>k_{\mathcal{P}}\cdot|I_{\mathcal{S}}|\right), \notag\right. \\ &\left. rank\left(S_{_c}(u,i)<k_{\mathcal{A}}\cdot|I_{\mathcal{S}}|\right),i\in I_{\mathcal{S}}\right\}
\end{align}
where, $I_{\mathcal{S}}$ denotes the set of all items in an interaction sequence, and $k_{\mathcal{P}}$, $k_{\mathcal{A}}$ are the degree of popularity bias and amplified subjective bias, respectively. 

In addition, we take those items with low audience degree but in line with the user's personality as amplified subjective bias perspective items, denoted as $I_{\mathcal{A}}^\mathcal{S}$, which are calculated as follows:
\begin{align}
    I_{\mathcal{A}}^{S}=&\left\{i\mid rank\left(S_{_p}(u,i)>k_{\mathcal{P}}\cdot|I_{\mathcal{S}}|\right), \notag\right. \\ &\left. rank\left(S_{_c}(u,i)<k_{\mathcal{A}}\cdot|I_{\mathcal{S}}|\right),i\in I_{\mathcal{S}}\right\}
\end{align}

All remaining items are debiased perspective items, denoted as $I_{\mathcal{D}}^\mathcal{S}$:
\begin{equation}
    I_{\mathcal{D}}^{\mathcal{S}}=I_{\mathcal{S}}-I_{\mathcal{P}}^{\mathcal{S}}-I_{\mathcal{A}}^{\mathcal{S}}
\end{equation}

It is worth noting that the forward and backward order of the items in the sequence does not change after the items are filtered and reorganized. Figure \ref{fig: recombination} shows the process of screening and reorganization of sequence data where the values of $k_{\mathcal{P}}$, $k_{\mathcal{A}}$ are both 0.5. 
\begin{figure*}[ht]
    \centering
    \includegraphics[width=1\linewidth]{Figure//method/8.png}
    \caption{Short sequence recombination}
    \label{fig: recombination}
\end{figure*}

\subsection{Project Diagram Construction}
% By filtering and reorganizing the deviations in the data, short sequences of the three bias were obtained.
Due to the short length of the sequences, it is difficult to provide sufficient contextual signals for the neural sequence encoder, thus limiting the performance of the model. To solve this problem, we utilize the construction method of item graph structure. For three short sequences with different biases, the same construction method is used to obtain graph structures ((i.e., $\mathcal{G}_{\mathcal{P}}$, $\mathcal{G}_{\mathcal{A}}$, and $\mathcal{G}_{\mathcal{D}}$) under three different bias perspectives. This construction of graph structures can learn cross-sequence dependencies between individual users under the same bias, which can help the neural network to better understand the users' behavioral patterns under different bias perspectives. 
% Such an approach also helps to overcome the problem of insufficient information brought about by short sequences, which in turn motivates the neural sequence encoder to learn the user's preferences more accurately. 

Specifically, the neighboring item pairs in each sequence $S$ are treated as edges of the item graph $\mathcal{G}$, and this view is represented using an adjacency matrix. The adjacency matrix $A_{\mathcal{G}}\in\mathbb{R}^{|I|\times|I|}$ is generated according to the following formula: 
\begin{equation}
    A_{\mathcal{G}}^u(i_p,i_q){=}\begin{cases}1,|p-q|=1\\0,otherwise&\end{cases};A_{\mathcal{G}}=\sum_{u\operatorname{=}1}^{|U|}A_{\mathcal{G}}^u
\end{equation}
where $A_{\mathcal{G}}^u$ denotes the neighboring item transitions of the specified user in the sequence $S_u$. $p$, $q$ denote the index position of the item $i$ in the sequence, and the corresponding adjacency matrix element is 1 if item $_p$ and item $i_q$ are adjacent in the sequence, and 0 otherwise. The complete adjacency matrix $A_{\mathcal{G}}$ is obtained by superimposing all the user's $A_{\mathcal{G}}^u$. It is important to emphasize that each value in the adjacency matrix $A_{\mathcal{G}}$ represents the number of transitions between neighboring items, and thus corresponds to the weights of the upper edges of the view $\mathcal{G}$. 
% This design emphasizes fairness even more, as it fully considers the number of conversions between different users and can reflect the relationship between items more comprehensively. By considering the conversion behavior between users, it is possible to understand more objectively the interaction patterns and mutual influences between projects. In addition, this design better reflects the degree of association between items because not only the direct switching relationships between items are considered, but also the degree of user involvement in these switches. 

After the above steps, the adjacency matrices of the graphs in the three bias perspectives are obtained as $A_{\mathcal{G}_{p}}$,$A_{\mathcal{G}_{A}}$, and $A_{\mathcal{G}_{D}}$. To emphasize the importance of each project node, the unit diagonal matrices of the three adjacency matrices are summed up to obtain new adjacency matrices $\tilde{A}_{\mathcal{G}_{p}}$, $\tilde{A}_{\mathcal{G}_{A}}$, and $\tilde{A}_{\mathcal{G}_{p}}$. Then, the rows of the adjacency matrices are summed up to obtain the degree matrices $D_{\mathcal{G}_{p}}$,$D_{\mathcal{G}_{A}}$, and $D_{\mathcal{G}_{D}}$. The degree matrices reflect the degree of each node, i.e., the number of edges connected to it. 
% Such an operation helps to better understand the connection relationship between the nodes in the project graph structure and provides the basis for the subsequent graph convolution operation. 
% The specific operation steps are shown in Figure \ref{fig: matric}.

% \begin{figure*}
%     \centering
%     \includegraphics[width=1\linewidth]{Figure//method/9.png}
%     \caption{The process of graph matrix computation}
%     \label{fig: matric}
% \end{figure*}
\begin{figure*}[ht]
    \centering
    \includegraphics[width=0.5\linewidth]{Figure//method/30.png}
    \caption{graph structure message passing aggregation }
    \label{fig: graph}
\end{figure*}
To simplify the model and improve its efficiency and interpretability, the model employs a strategy of culling the nonlinear activation operations and redundant transformation operations in the Graph Convolutional Network (GCN). 
% This simplification makes the whole message passing process more efficient and easier to understand and regulate. 
The formula of the whole message-passing process is represented as follows:
\begin{equation}
    X_{Type}^{(l+1)}=\left(D_{Type}^{-1/2}A_{{\mathcal{G}_{Type}}}D_{Type}^{-1/2}\right)X_{Type}^{(l)}
\end{equation}
where $l$ denotes the level index, and $Type$ denotes the type of bias. $X_{Type}^{(l)}\in\mathbb{R}^{|I|\times d}$ denotes the feature matrix after $l$ layers of message passing under some bias perspective, where $d$ denotes the dimension size of each embedding vector. As shown in Figure \ref{fig: graph}, in the initial state, the feature matrix $X_{Type}^{(0)}$ consists of the embedding matrices of all items. Then, the embedding feature information constructed based on the graph can be obtained by averaging the feature matrices of each layer by superposition. The embedded feature information $M_{type}$ constructed based on the graph is calculated as follows: 
\begin{equation}
    M_{type}=\frac{\sum_{l=1}^{L}X_{Type}^{(l)}}{L}
\end{equation}
where $L$ denotes the number of total layers of message delivery.

\subsection{Sequence Information Encoding}
% To facilitate the input of sequence information into the network model, it is then necessary to normalize the different user sequences. Specifically, we perform complementary zero padding for sequence lengths less than the preset length and intercept sequence lengths that are more than the preset length. Then, we use the Transformer module to encode the sequence information in multi-biased perspectives.
% the length of each sequence is fixed to $L$. For user interaction sequences with a length of more than $L$, the $L$ interaction items at the end of the sequence are intercepted as the user's interaction sequences; while sequences with a length of less than $L$, are filled with zeros to make the length of all the user sequences $L$. This ensures that the model receives sequences of the same length, which facilitates the processing and training of the network. The sequence length conversion formula is as follows:
% \begin{equation}
%     S_u=\left\{\begin{array}{ll}(s_u^{l-L+1},s_u^{l-L+2},\cdots,s_u^l)&,l>L\\(s_u^1,s_u^2,\cdots,s_u^l)&,l=L\\(0,0,\cdots,0,s_u^1,s_u^2,\cdots,s_u^l)&,l<L\end{array}\right.
% \end{equation}
% where $S_{u}=(s_{u}^{1},s_{u}^{2},\cdots,s_{u}^{l})$ denotes the user interaction sequence and $l$ is the length of the original sequence.
% Transformer has become one of the main ways to encode sequences in recommender systems. It can efficiently map temporally ordered tokens from sequences of items of different types into the latent representation space. With the Transformer module, the model can better understand and capture the interactions and dependencies between different items in a sequence of user behaviors. Therefore, 
The three biased feature matrices degree (i.e., $A_{\mathcal{G}_{p}}$,$A_{\mathcal{G}_{A}}$, and $A_{\mathcal{G}_{D}}$) obtained after graph message-passing process contain the potential vector information of the items in three biased perspectives. To record the location information of items at different locations in the sequence, we introduce a learnable location matrix $P\in\mathbb{R}^{L\times d}$. Then the potential vector representation $E\in\mathbb{R}^{L\times d}$ after location information embedding is defined as follows:
\begin{equation}
    E=\begin{bmatrix}M_{s_1}+P_1\\M_{s_2}+P_2\\...\\M_{s_L}+P_L\end{bmatrix}
\end{equation}
where $M_{s_{i}}$ denotes the potential vector representation of the $i$-th item in the sequence $s$. In this way, the item representation at each position contains its potential features and the embedding of the positional information, which enables the model to better understand and utilize the information of the items at different positions in the sequence. 

For the model to better understand and capture the interactions and dependencies between different items in a sequence of user behaviors, MABSRec introduces the Transformer module to encode the sequence information under multiple bias perspectives. Figure \ref{fig: transformer} illustrates the framework of the module.
\begin{figure*}[ht]
    \centering
    \includegraphics[width=1\linewidth]{Figure//method/29.png}
    \caption{Transformer framework}
    \label{fig: transformer}
\end{figure*}

Firstly, this module performs a Dropout operation on the obtained latent vector representation after location information embedding $E$ and the feature matrix after the Dropout operation is noted as $\tilde{E}$. 
% The Dropout operation randomly sets the outputs of some of the neurons to zero, which reduces the model's dependence on the training data and helps to prevent the overfitting phenomenon from occurring. 
After that, the multi-head self-attention mechanism is employed to capture the contextual information of the sequence, which operates as follows: 
\begin{equation}
     \text{MultiHead(H)}=\text{Concat}(head_{1},head_{2},\cdots,head_{h})\cdot W_{o}
\end{equation}
\begin{equation}  head_i=\text{Attention}\left(\tilde{E}W_{Q_i},\tilde{E}W_{K_i},\tilde{E}W_{V_i}\right)
\end{equation}
\begin{equation}
    \mathrm{Attention}(Q,K,V)=\mathrm{Dropout}\left(\mathrm{softmax}\left(\frac{QK^T}{\sqrt{d/h}}\right)\right)V
\end{equation}
where $W_Q\:\text{,}\:W_K\:\text{,}\:W_V\in\mathbb{R}^{d\times\frac dh}$ correspond to the head-specific mapping matrices for queries, keys, and values, respectively. $W_o$ is the parameter output matrix with dimension $d \times d$, and $h$ is the number of attention heads. In the multi-head self-attention mechanism, the linear transformations of the query, key, and value are first computed separately through $h$ heads. 
% Since it is a self-attention mechanism, this side of the query, key, and value are all positionally encoded input sequences $\tilde{E}$. 
During the computation of the model, for the results after the foregoing transformations, the scaled dot product attention mechanism is used to perform further processing. 
% The main rationale for choosing to use the scaled dot product attention mechanism is that it can effectively mitigate the Matthew effect that may occur when applying the traditional dot product attention, i.e., the larger attention scores are further amplified and the smaller scores are further compressed under the Softmax function, resulting in a polarized distribution of the scores, which not only makes the model's attention focus on a few key information points while ignoring other potentially useful information, it may also cause the gradient vanishing problem during backpropagation, thus hindering the learning process of the model. Scaling the dot product attention Scaling the result of the dot product by introducing a scaling factor helps to control the gradient of the Softmax function, thus avoiding extreme scoring values. This approach not only makes the distribution of the attention scores closer to the normal distribution and reduces the negative impact of extreme values on the model performance, but also facilitates the stable transfer of the gradient and improves the training efficiency and stability of the model. 
Finally, the outputs of all attention heads are concatenated and linearly transformed through a weight matrix $W_o$ to obtain the final output $x\in\mathbb{R}^d$ of the multi-head self-attention mechanism. To enhance the gradient flow of the model and facilitate the learning of sequence representations during training, we apply normalization and residual connections to the output of the multi-head self-attention mechanism, denoting the result as $\tilde{x}$.
% Finally, the outputs of all the attention heads are spliced together and linearly transformed by the weight matrix $W_o$ to obtain the final output of the multi-head self-attention mechanism $x\in\mathbb{R}^d$. Next, the outputs of the multi-head self-attention mechanism are normalized as well as residuals are concatenated to achieve better training results:
% \begin{equation}
%     \tilde{x}=\mathrm{LayerNorm}(x+\mathrm{Sublayer}(x))
% \end{equation}
% where $\mathrm{Sublayer}(x)$ denotes the output of the sublayer. This operation first sums the original input pair with the output of the sublayer through residual concatenation, after which the output is normalized using $\mathrm{LayerNorm}$ function. 
% This approach enhances the gradient flow of the model and makes it easier for the model to learn the representation of the sequence during training while avoiding the problem of vanishing or exploding gradients. 
In addition, a feed-forward neural network is introduced to transform the representation to make the model learn the effect of non-linearity:

\begin{equation}
    \mathrm{FFN}(\tilde{x})=\mathrm{GELU}(\tilde{x}W_1+b_1)W_2+b_2
\end{equation}
where $W_1\text{,}W_2\in\mathbb{R}^{d\times d}\:\text{,}\:b_1\:\text{,}\:b_2\in\mathbb{R}^d$ are the weight and bias matrices, respectively. $\mathrm{GELU}(\cdot)$ is the nonlinear activation function. 

It is worth noting that for all three biased feature inputs, the model chooses the same Transformer module to receive them. This approach enables the model to share parameters when processing sequence information under different biases, reducing the model's complexity and the number of parameters. By sharing the Transformer module, the model can better learn the sequence features under different biases and generalize to new datasets, improving the model's generalization ability. 
% At the same time, the shared module also enhances the robustness of the model, making the model better adaptable to changes in the input data.

\subsection{Adaptive Multi-bias Perspective Attention Module}
% Past research in recommender systems has tended to ignore the effects of multiple biases present in items or to focus on only one of them. However, in reality, the interaction between users and items is often affected by multiple biases, such as popularity bias, personalization preference bias, and user-specific preferences for items. To address this issue, the model processes all items from three perspectives separately, obtaining three types of item information: popularity-biased, subjectivity-biased, and debiased. The user feature outputs (i.e., $x_{\mathcal{P}}\text{,}x_{\mathcal{A}}\text{,}x_{\mathcal{D}}\in\mathbb{R}^d$) for each of the three bias perspectives are obtained through the same sequence coding layer, and these feature vectors capture the user's behavioral patterns and preference characteristics under different bias perspectives. 
To better fuse the three biased sequence information for recommendation, we propose adaptive multi-perspective attention learning. Specifically, a vector addition operation is first adopted to fuse these bias features so that the model can not only capture the impact of each single bias but also reveal the possible mutual enhancement or mutual inhibition among different biases. As follows, we define the feature vectors that fuse the three bias information: 
\begin{align}
     o_{u}=\mathrm{Concat}[x_{\mathcal{P}_{*}},x_{\mathcal{A}_{*}},x_{\mathcal{D}_{*}},x_{\mathcal{P}_{*}}\oplus x_{\mathcal{A}_{*}},\\ \notag x_{\mathcal{P}_{*}}\oplus x_{\mathcal{D}_{*}},x_{\mathcal{D}_{*}}\oplus x_{\mathcal{D}_{*}}\oplus x_{\mathcal{D}_{*}}]
\end{align}where $\oplus$ denotes vector summation. Then, for each user sequence $s_u$ the corresponding feature vector $o_u$ will be fed into a two-layer feed-forward neural network, and then its long dimensional information will be nonlinearly processed by the ReLU function, which enables the model to better focus on the learning of key features. Next, the Sigmoid function is utilized to output the scores $scores_u$ corresponding to the feature information under the three bias perspectives. These scores can indicate the degree of user preference for items under different bias perspectives, thus providing critical information for the recommendation model. The user's score $scores_u$ for the three bias perspectives is calculated as follows:
\begin{equation}
    scores_{u}=\sigma(W_{2}\mathrm{ReLU}(W_{1}o_{u}+b_{1})+b_{2})
\end{equation}where $W_{1}\in\mathbb{R}^{{d_{o}\times d}}\text{,}W_{2}\in\mathbb{R}^{d\times3}\text{,}b_{1}\in\mathbb{R}^{d}\text{,}b_{2}\in\mathbb{R}^{3}$ are all learnable parameters shared across sequence terms, $\sigma $ refers to the Sigmoid activation function, and $o_u$ refers to the dimension of $d_o$. Finally, the vector representation of the predicted items $e_{u}^{pred}$ is obtained by matrix multiplication of $scores_u$ with the corresponding eigenvectors $x_{\mathcal{P}_u}\text{,}x_{\mathcal{A}_u}\text{,}x_{\mathcal{D}_u}$ under the three bias perspectives.
\begin{equation}
    e_{u}^{pred}=\mathrm{matmul}\left(scores_{u},[x_{{\mathcal{P}_{{_{u}}}}}\|x_{{\mathcal{A}_{{\mathcal{u}}}}}\|x_{{\mathcal{D}_{{_{u}}}}}]\right)
\end{equation}
where $\mathrm{matmul}$ denotes the matrix multiplication, such that the resulting $e_{u}^{pred}$ represents the predicted preference of user $u$ for items with different bias perspectives.

\subsection{Loss Function}
In the training phase, the model is trained by using the last item in the user's interaction sequence as a label. To predict the next interaction item of the user, a recommendation score vector $\hat{y}_{u,i}$ is obtained by transposing and multiplying the predicted item vector $e_{u}^{pred}$ with the initial embedding representation $E\in\mathbb{R}^{|I|\times d}$ of all items as follows:
\begin{equation}
    \hat{y}_{u,i}=e_u^{pred}\cdot E^T
\end{equation}

The vector $\hat{y}_{u,i}$ represents the user's preference for all items under the current bias perspective. For each user's true next interaction item $i_{t+1}$, we define the recommendation error $\mathcal{L}_{rec}$:
\begin{equation}
    \mathcal{L}_{rec}=-\log\left(\frac{\exp\left(\hat{y}_{{u,i_{t+1}}}\right)}{\sum_{j\in I}\exp\left(\hat{y}_{u,j}\right)}\right)
\label{loss}
\end{equation}

This loss function measures the extent to which the probability distribution of items predicted by the model deviates from the true interaction items through the cross-entropy loss function. In Equation (\ref{loss}), the negative log-likelihood loss will be used to minimize the gap between the predicted probabilities and the true labels, where the numerator part $\exp(\hat{y}_{u,i_{t+1}})$ denotes the predicted probability of the next item, and the denominator part is the softmax normalization of the probabilities of all items. Optimizing the model parameters by minimizing the recommendation error loss makes it more accurate in predicting the user's behavior. 
% Such a loss function not only helps the model to better understand the user's behavioral patterns, but also allows the model to continuously adjust its parameters during the training process to adapt to changes in the user's behavior and the evolution of preferences.

\section{Experimental Results and Analysis}
\label{experimental}
In this section, we evaluate our algorithm comprehensively through experiments.
\subsection{Experiment Settings}
\subsubsection{Dataset}
To validate the effectiveness of the model, we evaluated it on three real-world benchmark datasets, i.e., Amazon Beauty, Amazon Sports, and MovieLens-20M datasets. The statistics for various datasets are provided in Table \ref{tab: dataset}. 
\begin{itemize}
    \item The \textbf{Amazon Beauty} dataset is obtained from the Amazon website and contains detailed behavioral records and product characteristic information of users when purchasing beauty products.
    \item The \textbf{Amazon Sports} dataset is a dataset for the Amazon Outdoor Sports product segment, similar to the Amazon Beauty dataset, which also contains user purchase and review information.
    \item The \textbf{MovieLens-20M} dataset is a classic movie rating dataset that contains rating information for over 12,000 movies from over 50,000 users.
\end{itemize}   
To ensure the reliability of the data, interactions with sequence lengths less than 5 were eliminated from the experiments. Also for the MovieLens-20M dataset, interactions with sequence lengths more than 50 were removed. 
\begin{table}[t]
    \centering
    \caption{Statistics of the Datasets}
    \resizebox{1\linewidth}{!}{
    \begin{tabular}{c|c|c|c|c|c} \cline{1-6}
         Datasets   &\#users   &\#items   &\#Interactions   &\#Avg.leng &\#Density  \\ \cline{1-6}
         Beauty   &52374   &121290   &469771  &8.97 &7.4e-5\\ \cline{1-6}
         Sports  &84368  &194714  &717464  &8.50 &4.4e-5 \\ \cline{1-6}
         ML-20M  &54437  &12360   &1727055   &31.73  &2.6e-3 \\ \cline{1-6}
    \end{tabular}
    }
    \label{tab: dataset}
\end{table}

\subsubsection{Baselines}
To validate the effectiveness of our model, we conducted comparative experiments with several classical baseline recommendation models and excellent recommendation models that have recently received much attention. The compared baselines are described as follows:
\begin{itemize}
    \item \textbf{Caser \cite{10.1145/3159652.3159656}:} This is a convolutional neural network-based recommendation model that focuses on learning local feature information of sequence data. 
    \item \textbf{GRU4Rec: \cite{hidasi2016sessionbasedrecommendationsrecurrentneural}} A recommender system based on session modeling techniques, the core idea of which is to deeply learn the user's behavioral sequences through a multilayered GRU layer and a fully-connected layer to capture the temporal dynamics of the user's behavior and the potential interest evolution.
    \item \textbf{SASRec: \cite{8594844}} This model introduces the Transformer model to the field of recommender systems, mainly by introducing positional embedding and self-attention mechanisms to deal with sequential data.
    \item \textbf{CL4SRec \cite{9835621}:} It innovatively performs censoring, masking, and disruption operations on the items of user sequences to generate enhanced data. 
    \item \textbf{DuoRec \cite{10.1145/3488560.3498433}:} It utilizes different Dropout techniques to augment the data while training sequences with the same target as positive samples. 
    \item \textbf{MAERec \cite{10.1145/3539618.3591692}:} Unlike traditional randomly augmented data, MAERec can dynamically select nodes to be masked based on the connectivity and importance of the nodes in the graph, thus generating more representative training samples. 
    \item \textbf{DCRec \cite{10.1145/3543507.3583361}:} Build a cross-view comparative learning framework that aims to learn the degree of user follower and normalize the follower distribution of all users to a normal distribution through KL dispersion to achieve the effect of debiasing.
\end{itemize}
% All of these models have been widely used and researched in the field of recommender systems, and each of them has specific advantages and applicable scenarios. By comparing with these baseline models, the performance and effectiveness of the MABSRec model can be more comprehensively evaluated, as well as its advantages in solving the biased recommendation problem.

\subsubsection{Evaluation Indicators}
During the model evaluation process, this chapter uses a Top-N correlation recommendation list and employs two main evaluation metrics: Recall Rate @N (Recall@N)  and Normalized Discounted Cumulative Gain @N (NDCG@N), where N $\in \{1, 5, 10\}$. 

Recall@N measures the model's ability to capture a user's true interest given a recommendation list of length N. The Recall@N of the entire dataset is the average of all users' Recall@N, providing a global view of the model's performance. Recall@N is calculated as follows:
% \begin{equation}
% \text { Recall@N }=\frac{|\mathrm{P}\cap \mathrm{T}|}{|\mathrm{T}|}
% \end{equation}where P denotes the set of results predicted by the model and T is the set of recommended items in the real test set.

NDCG@N considers the positional factors of the items in the recommendation list and measures the quality of the model's ranking in the recommendation list by calculating the Discounted Cumulative Gain (DCG). 
% DCG and NDCG@N are calculated as follows:
% \begin{equation}
% \mathrm{DCG}=rel_i+\sum_{i=2}^p \frac{rel_i}{\log_2 i}
% \end{equation}

% \begin{equation}
% \mathrm{NDCG}=\frac{\mathrm{DCG}}{\mathrm{IDCG}}
% \end{equation}where $rel_i$ denotes whether or not the hit is made at the ith position, and the result of the hit is 1, otherwise it is 0. IDCG denotes the DCG in the ideal state. As the predicted results are closer to the true results, the closer the value of NDCG is to 1, reflecting that the model better takes into account the ranking information of the recommendation list.

\subsection{Parameter Settings}
The batch size of all datasets is set as 512 and the sequence length is fixed to 50. To ensure the fairness of the experiments, an Adam optimizer with a learning rate of 0.001 is used for all the training models and an early-stopping strategy is set up: when the model fails to show any better metrics within 10 epochs, the training is stopped. In the MABSRec model, the parameters are set differently for different dataset characteristics. On both the Amazon Beauty and Amazon Sports datasets, the dropout rate is 0.4, and the number of Transformer layers and graph convolution layers is set as 2. On the MovieLens-20m dataset, due to the different characteristics of the dataset, the dropout rate is set as 0.1, and the number of Transformer layers and graph convolution layers is increased to 4. In addition, the number of multi-head self-attention heads is set as 1 on both the Amazon Beauty and Amazon Sports datasets, while it is increased to 8 on the MovieLens-20M dataset. The graph convolution dropout rates are 0.4, 0.5, and 0.3, respectively. Table \ref{tab: parameter} demonstrates the parameter settings in this experiment.

% The above parameter settings are intended to ensure the experiment is stable and reproducible and to make the experimental results on different datasets comparable and fair. These parameters are carefully considered and experimentally verified to ensure the model can perform well on each dataset.

\begin{table}[t]
    \centering
    \caption{PARAMETER SETTINGS}
    \resizebox{1\linewidth}{!}{
    \begin{tabular}{c|c|c|c} \cline{1-4}
        ~ &Beauty  &Sports  &ML-20M \\ \cline{1-4}
         batch size&512   &512   &512  \\ \cline{1-4}
         Adam learning rate&0.001   &0.001   &0.001  \\ \cline{1-4}
         maximum sequence length&50   &50   &50  \\ \cline{1-4}
         dropout rate&0.4  &0.4  &0.1 \\ \cline{1-4}
         number of Transformer layers&2   &2   &4 \\ \cline{1-4}
         \makecell[c]{number of multi-head \\self-attention heads}&1  & 1 & 8\\ \cline{1-4}
         number of graph convolution layers& 2 & 2& 4\\ \cline{1-4}
         graph convolution dropout rate&0.4  & 0.5 &0.3 \\ \cline{1-4}
    \end{tabular}
    }
    \label{tab: parameter}
\end{table}

\subsection{Comparative Experiment}
In this section, we conduct comparative experiments of the proposed method with the baseline model and select relevant evaluation metrics for assessment.

% \subsubsection{Effectiveness Comparison}
% For the performance of various models, we experimented MABSRec with other baseline models on different datasets and evaluated them according to different evaluation metrics (Recall@1, Recall@5, Recall@10, NDCG@5, NDCG@10). The experiment results are shown in table \ref{tab: compared}. Based on the results, we can make the following observations:



% \begin{itemize}
%     \item Among deep recommendation algorithms that do not employ graph structure or comparison learning, SASRec demonstrates optimal results across datasets and metrics. Particularly noteworthy is that on the MovieLens-20m dataset, SASRec achieves significant improvements over both Caser and GRU4Rec. In addition, on the Amazon Sports dataset, GRU4Rec slightly improves on some metrics. And Caser shows a relatively strong advantage on the MovieLens-20m dataset. These findings suggest that SASRec based on the self-attention mechanism has high performance and generalization in recommendation algorithms, and shows significant effect advantages on multiple datasets.

%     \item When contrast learning is introduced on top of the self-attention based mechanism, the performance of the overall models are significantly improved. In the case of the Amazon Sports dataset, CL4SRec and DuoRec show significant improvement compared to SASRec. On the MovieLens-20m dataset, the overall improvement of both CL4SRec and DuoRec is around 6\%. In addition, between these two recommendation models based on comparative learning, DuoRec slightly improves with respect to CL4SRec, especially on the Amazon Beauty dataset where it achieves improvement on all metrics. These results show that the introduction of comparative learning has a significant impact on the performance improvement of recommender systems.

%     \item For models that introduce graph structure, MAERec and DCRec each have their own characteristics. On the sparser Amazon Sports and Amazon Beauty datasets, MAERec shows a clear advantage. However, on the MovieLens-20m dataset, which has a smaller sparsity, DCRec performs well and is the most effective model besides MABSRec. Compared to SASRec without graph structure, DCRec improves at least 5.6 percentage points in all metrics. In addition, MAERec's performance on the MovieLens-20m dataset is very bad, which indicates that different recommendation models show different advantages under different sparsity conditions of the dataset, and MAERec is superior in the case of larger sparsity, while DCRec excels in the case of smaller sparsity.

%     \item The MABSRec model shows significant improvement in all datasets and evaluation metrics. This indicates that when the MABSRec model introduces a multi-bias view to deal with the sequence problem, it can more comprehensively consider multi-bias factors in the sequence of user behaviors, such as prevalence bias, subjective bias, and de-biasing. With this comprehensive bias perspective, MABSRec can capture the user's interests and behavioral patterns more accurately, thus improving the accuracy and personalization of recommendations. In addition, MABSRec is designed with full consideration of dataset characteristics and sparsity, so that it can be applied to datasets of different types and densities, and exhibits better generalization ability.

% \end{itemize}

\subsubsection{Performance Comparison}
For the performance of various models, we experimented MABSRec with other baseline models on different datasets and evaluated them according to different evaluation metrics (Recall@1, Recall@5, Recall@10, NDCG@5, NDCG@10). The experiment results are shown in table \ref{tab: compared}.
\begin{table*}[ht]
    \centering
    \caption{Experimental results for each model on three real datasets }
    \resizebox{1\linewidth}{!}{
    \begin{tabular}{c|c|c|c|c|c|c|c|c|c|c} \hline
         \makecell{Dataset}&Metric  &Caser   &GRU4Rec   &SASRec   &CL4SRec   &DuoRec   &MAERec   &DCRec  &\textbf{MABSRec}  &\#improve \\ \hline
         \multirow{5}{*}{\begin{tabular}[c]{@{}l@{}}Beauty\end{tabular}}
         &Recall@1   &0.0098  &0.0101  &0.0107  &0.0145  &\underline{0.0147}  &0.0138  &0.0118  &\textbf{0.0149}  &\textbf{1.36\%} \\
         &Recall@5   &0.0226  &0.0240  &0.0251  &0.0326  &\underline{0.0328}  &0.0327  &0.0272  &\textbf{0.0336}  &\textbf{2.44\%}\\
         &Recall@10   &0.0336  &0.0341  &0.0377  &0.0437  &0.0441  &\underline{0.0445}  &0.0362  &\textbf{0.0453}  &\textbf{1.80\%}\\
         &NDCG@5  &0.0175  &0.0183  &0.0202  &0.0237  &\underline{0.0238} &0.0232  &0.0196  &\textbf{0.0242}  &\textbf{1.68\%}\\
         &NDCG@10  &0.0209  &0.0210  &0.0234  &0.0272  &0.0275    &0.0272  &0.0225  &\textbf{0.0280}  &\textbf{1.82\%}\\ \hline
         \multirow{5}{*}{\begin{tabular}[c]{@{}l@{}}Sports\end{tabular}}
         &Recall@1   &0.0045  &0.0046  &0.0050  &0.0065  &\textbf{\underline{0.0071}}  &0.0060  &0.0053  &\textbf{0.0071}  &\textbf{0.00}\%\\
         &Recall@5   &0.0117  &0.0120  &0.0129  &\underline{0.0168}  &0.0165  &0.0163  &0.0129  &\textbf{0.0176} &\textbf{4.76\%} \\
         &Recall@10   &0.0163  &0.0169  &0.0186  &\underline{0.0243}  &0.0237  &0.0236  &0.0183  &\textbf{0.0251}  &\textbf{3.29\%} \\
         &NDCG@5  &0.0083  &0.0086  &0.0090  &0.0117  &\underline{0.0118}  &0.0112  &0.0091  &\textbf{0.0123}  &\textbf{4.24\%} \\
         &NDCG@10  &0.0098  &0.0101  &0.0108  &0.0141  &\underline{0.0142}  &0.0135  &0.0109  &\textbf{0.0147}  &\textbf{3.52\%} \\ \hline
         \multirow{5}{*}{\begin{tabular}[c]{@{}l@{}}ML-20M\end{tabular}}
         &Recall@1   &0.0908   &0.0899   &0.0954   &0.1032   &0.1031  &0.0673   &\underline{0.1064}  & \textbf{0.1100}  &\textbf{3.38\%} \\
         &Recall@5   &0.2182   &0.2134  & 0.2259 &0.2364  &0.2367   & 0.1816 & \underline{0.2415}  & \textbf{0.2463}  &\textbf{1.99\%} \\
         &Recall@10   &0.2974  & 0.2875  &0.3053   &0.3203   &0.3204   &0.2589  &\underline{0.3224}   &\textbf{0.3289}  &\textbf{2.02\%} \\
         &NDCG@5  &0.1532   &0.1488  &0.1624   & 0.1716  & 0.1719  & 0.1257 &\underline{0.1759}  & \textbf{0.1798} &\textbf{2.22\%} \\
         &NDCG@10  & 0.1825  &0.1797   &0.1880   &0.1986   &0.1989   &0.1506  & \underline{0.2020} &\textbf{0.2065}  &\textbf{2.23\%} 
\\ \hline
    \end{tabular}
    }
    \label{tab: compared}
\end{table*}

Based on the comparison results shown in table \ref{tab: compared}, the recommendation models with attention mechanisms significantly outperform traditional models based on RNN or CNN. Therefore, we focus on two evaluation metrics, Recall@10 and NDCG@10, to conduct in-depth performance analysis and exploration of recommendation models (e.g., SASRec, CL4sRec, Duorec, MAERec, DCRec, and MABSRec) that incorporate the attention mechanism, and to further reveal the strengths and characteristics of these models in recommender systems.

Figure \ref{fig: epoch} 
presents the training progress of the six recommendation models with attention modules on three different datasets. Overall, the MABSRec model shows excellent performance at the early stage of training on each dataset, and it can achieve high results after a relatively small number of Epochs, demonstrating an efficient training speed. As the training progresses, the performance of the MABSRec model continues to improve and shows high stability in the later stages of training, which is not prone to overfitting or performance degradation. This indicates that the model can fully mine the feature information of sequence data and effectively learn the user's preferences and interests. Therefore, from the comprehensive performance evaluation, the MABSRec model has important application prospects and research value in the field of recommender systems.

\begin{figure*}[htbp]
	\centering
	\begin{minipage}{0.32\linewidth}
		\centering
		\includegraphics[width=0.9\linewidth]{Figure/epoch-Beauty/Recall_10.png}
	\end{minipage}
	\begin{minipage}{0.32\linewidth}
		\centering
		\includegraphics[width=0.9\linewidth]{Figure/epoch-Sports/Recall_10.png}
	\end{minipage}
    \begin{minipage}{0.32\linewidth}
		\centering
		\includegraphics[width=0.9\linewidth]{Figure/epoch-Movie/Recall_10.png}
	\end{minipage}
    
	%\qquad
	%让图片换行,

    \begin{minipage}{0.32\linewidth}
		\centering
		\includegraphics[width=0.9\linewidth]{Figure/epoch-Beauty/NDCG_10.png}
	\end{minipage}
	\begin{minipage}{0.32\linewidth}
		\centering
		\includegraphics[width=0.9\linewidth]{Figure/epoch-Sports/NDCG_10.png}
	\end{minipage}
    \begin{minipage}{0.32\linewidth}
		\centering
		\includegraphics[width=0.9\linewidth]{Figure/epoch-Movie/NDCG_10.png}
	\end{minipage}
    \caption{Performance Trends on Three Datasets}
    \label{fig: epoch}
\end{figure*}

% \begin{figure*}[ht]
%  \centering
%  \subfigure{
%  \includegraphics[scale=0.3]{Figure/epoch-Beauty/Recall@10.png}
%  }
%  \quad
%  \subfigure{
%  \includegraphics[scale=0.3]{Figure/epoch-Beauty/NDCG@10.png}
%  }
 
%   \subfigure{
%  \includegraphics[scale=0.3]{Figure/epoch-Sports/Recall@10.png}
%  }
%  \quad
%  \subfigure{
%  \includegraphics[scale=0.3]{Figure/epoch-Sports/NDCG@10.png}
%  }

%   \subfigure{
%  \includegraphics[scale=0.3]{Figure/epoch-Movie/Recall@10.png}
%  }
%  \quad
%  \subfigure{
%  \includegraphics[scale=0.3]{Figure/epoch-Movie/NDCG@10.png}
%  }
%  %
%     \caption{Performance Trends}
%     \label{fig: epoch}
% \end{figure*}

To more accurately assess the performance of each model, we compare the training time required to train an epoch on each dataset. The results presented in Figures \ref{fig: time} demonstrate a significant difference in the training time required by different recommendation models within a single Epoch. Notably, the SASRec model performs the best in terms of training time. This phenomenon is mainly attributed to the simplicity of the SASRec model, which does not involve complex comparison learning or graph structure processing operations. In contrast, the MABSRec model has a slightly longer training time. Although it introduces graph structure information, its main time consumption is still concentrated in the Transformer model part because it does not need to perform time-consuming operations such as graph sampling or random wandering. So compared to other graph structure-related recommendation models, the MABSRec model still shows relatively efficient training speed.

\begin{figure*}[htbp]
	\centering
    \begin{minipage}{0.32\linewidth}
		\centering
		\includegraphics[width=0.9\linewidth]{Figure/Per-epoch-time/Beauty.png}
	\end{minipage}
	\begin{minipage}{0.32\linewidth}
		\centering
		\includegraphics[width=0.9\linewidth]{Figure/Per-epoch-time/Sports.png}
	\end{minipage}
    \begin{minipage}{0.32\linewidth}
		\centering
		\includegraphics[width=0.9\linewidth]{Figure/Per-epoch-time/movie.png}
	\end{minipage}
    \caption{Performance Trends on Three Datasets}
    \label{fig: time}
\end{figure*}

% \begin{figure*}[ht]
%  \centering
%  \subfigure{
%  \includegraphics[scale=0.2]{Figure/Per-epoch-time/Beauty.png}
%  }
%  \quad
%  \subfigure{
%  \includegraphics[scale=0.2]{Figure/Per-epoch-time/Sports.png}
%  }
 
%  \subfigure{
%  \includegraphics[scale=0.2]{Figure/Per-epoch-time/movie.png}
%  }
%  %
%     \caption{Performance Trends on the MovieLens-20m Dataset}
%     \label{fig: time}
% \end{figure*}

\subsubsection{Performance in Different Sequence Lengths}
Sequence length is an important parameter in recommender systems, which reflects the length of the interaction sequence between users and items. Sequences of different lengths may have an impact on the performance of the model. Therefore, this section investigates the performance of the MABSRec model with other recommendation models that incorporate the attention mechanism under different sequence length conditions. 

Figure \ref{fig: lengths}  
\begin{figure*}[ht]
	\centering
	\begin{minipage}{0.32\linewidth}
		\centering
		\includegraphics[width=0.9\linewidth]{Figure/length-Beauty/Recall_10.png}
	\end{minipage}
	\begin{minipage}{0.32\linewidth}
		\centering
		\includegraphics[width=0.9\linewidth]{Figure/length-Sports/Recall_10.png}
	\end{minipage}
    \begin{minipage}{0.32\linewidth}
		\centering
		\includegraphics[width=0.79\linewidth]{Figure/length-Movie/Recall_10.png}
	\end{minipage}
    
	%\qquad
	%让图片换行,

    \begin{minipage}{0.32\linewidth}
		\centering
		\includegraphics[width=0.9\linewidth]{Figure/length-Beauty/NDCG_10.png}
	\end{minipage}
	\begin{minipage}{0.32\linewidth}
		\centering
		\includegraphics[width=0.9\linewidth]{Figure/length-Sports/NDCG_10.png}
	\end{minipage}
    \begin{minipage}{0.32\linewidth}
		\centering
		\includegraphics[width=0.79\linewidth]{Figure/length-Movie/NDCG_10.png}
	\end{minipage}
    \caption{Performance of Different Sequence Lengths}
    \label{fig: lengths}
\end{figure*}
shows the performance comparison graphs of each model under different datasets with different sequence lengths. The experimental results show that as the sequence length increases, the MABSRec model improves in various metrics and all of them show excellent performance, which is clearly ahead of all other models. This finding reflects the strong adaptability of the MABSRec model and shows that it is able to effectively capture user interests and behavioral patterns when dealing with different sequence lengths. 



% \begin{figure*}[ht]
%  \centering
%  \subfigure{
%  \includegraphics[scale=0.2]{Figure/length-Beauty/Recall_10.png}
%  }
%  \quad
%  \subfigure{
%  \includegraphics[scale=0.2]{Figure/length-Beauty/NDCG_10.png}
%  }
 
%  \subfigure{
%  \includegraphics[scale=0.2]{Figure/length-Sports/Recall_10.png}
%  }
%  \quad
%  \subfigure{
%  \includegraphics[scale=0.2]{Figure/length-Sports/NDCG_10.png}
%  }
 
%  \subfigure{
%  \includegraphics[scale=0.2]{Figure/length-Movie/Recall_10.png}
%  }
%  \quad
%  \subfigure{
%  \includegraphics[scale=0.2]{Figure/length-Movie/NDCG_10.png}
%  }
%  %
%     \caption{Performance of Different Sequence Lengths}
%     \label{fig: lengths}
% \end{figure*}

\subsubsection{Comparative Analysis of Bias Perspectives}
To more clearly assess the necessity of multi-bias perspectives in the sequence recommendation problem, we compare MABSRec with the DuoRec and DCRec models that perform best on the MovieLens-20m dataset. The main reason for choosing these two models for comparison is that DuoRec is the best-performing model that does not consider the effect of any bias on the data, while DCRec fully considers the effect of popularity bias on the users. The experiment result is shown in Figure \ref{fig: bias}.
\begin{figure}[ht]
    \centering
    \includegraphics[width=1\linewidth]{Figure/multi-bias/Comparison_of_deviated_viewpoints.png}
    \caption{Comparison of Bias Perspectives}
    \label{fig: bias}
\end{figure}
From the experimental results, compared with the DuoRec and DCRec models, the folds of the prediction results of the MABSRec model are significantly closer to the folds of the real test data. This indicates that the MABSRec model has higher accuracy and closeness in predicting item clicking behavior. This finding emphasizes the importance of multi-bias perspectives for recommender systems and provides useful guidance for further exploration of recommender models.



\subsubsection{Ablation Study}
To investigate the effect of key components on the model performance, ablation studies were performed on three variants of the MABSRec model: \textbf{w/o G}, \textbf{w/o A}, and \textbf{w/o D}. The results are presented in Table \ref{tab: ablation}.

\begin{table*}[ht]
    \centering
    \caption{Results of ablation experiments}
    \resizebox{1\linewidth}{!}{
    \begin{tabular}{ccccccc} \hline
         \multirow{2}{0.07\textwidth}{\centering Ablation Setting}&\multicolumn{2}{c}{Beauty}  &\multicolumn{2}{c}{Sports}  &\multicolumn{2}{c}{ML-20M}\\ \cline{2-7}
         &NDCG@5  &NDCG@10  &NDCG@5  &NDCG@10  &NDCG@5  &NDCG@10\\ \hline
         MABSRec  &0.0242  &0.0280  &0.0123  &0.0147  &0.1798  &0.2065\\ \hline
         w/o G  &0.0173 &0.0208 &0.0087 &0.0107 &0.1366 &0.1624\\ \hline
         w/o A  &0.0235 &0.0274 &0.0117 &0.0138 &0.1758 &0.2021\\ \hline
         w/o D  &0.0164 &0.0198 &0.0083 &0.0105 &0.1272 &0.1518\\ \hline
    \end{tabular}
    } 
    \label{tab: ablation}
\end{table*}

\textbf{Remove Graph Information (w/o G):} The first variant, \textbf{w/o G}, removes the graph information from the model consisting of biased short sequences. It trains the model directly using three short sequences as inputs to the model. The results show that graph information is crucial for capturing important structures and features in the sequences, and its absence leads to impaired model performance in the recommendation task.

\textbf{Remove Adaptive Multi-bias Perspective Attention Module (w/o A):} The Second variant, \textbf{w/o A}, replaces the process of learning information about the adaptive multi-bias perspective attention module with a simple average pooling operation. The results indicate that The adaptive multi-bias perspective attention module can effectively handle multiple types of bias information and improve the model's performance.

\textbf{Remove Both of the Above (w/o D):} The third variant, \textbf{w/o D}, removes both the graph information as well as the adaptive multi-bias perspective attention module from the model. The results show that the model performance degradation with these two components removed is more significant. This further validates the importance of the graph information and adaptive multi-bias perspective attention module on model performance.

\section{Conclusion}
\label{conclusion}
Current research on sequential recommender systems has received much attention. However, there is a lack of research on the impact of multi-bias factors in user data. To address this problem, we propose a novel sequence recommendation model, called MABSRec. The model first considers the prevalent popular and amplified subjective bias in user data and constructs a multi-bias view. Afterward, the degree of influence of various biases on the user is weighed by an attention fusion network to deeply mine the information in the user data and improve the overall performance of the recommender system. We validate the excellent performance of the MABSRec model by analyzing its metric scores on three real-world datasets compared to those of some excellent sequential recommendation models. The model proposed in this work can provide new ideas and methods for solving the biased problem in sequential recommender systems and also provides useful insights and directions for subsequent research on recommender systems.

% \begin{thebibliography}{00}

% %% For authoryear reference style
% %% \bibitem[Author(year)]{label}
% %% Text of bibliographic item

% \bibitem[Lamport(1994)]{lamport94}
%   Leslie Lamport,
%   \textit{\LaTeX: a document preparation system},
%   Addison Wesley, Massachusetts,
%   2nd edition,
%   1994.

% \end{thebibliography}

\bibliographystyle{IEEEtran}
\bibliography{main}

\end{document}

\endinput
%%
%% End of file `elsarticle-template-num.tex'.
