\documentclass{article}

% needs to be imported first with Tex Live 2023
\usepackage[table]{xcolor}

\usepackage[accepted]{main}

% ########## custom ##########
% basic
%\usepackage{color,xcolor}
\usepackage{color}
\usepackage{epsfig}
\usepackage{graphicx}
\usepackage{algorithm,algorithmic}
% \usepackage{algpseudocode}
%\usepackage{ulem}

% figure and table
\usepackage{adjustbox}
\usepackage{array}
\usepackage{booktabs}
\usepackage{colortbl}
\usepackage{float,wrapfig}
\usepackage{framed}
\usepackage{hhline}
\usepackage{multirow}
% \usepackage{subcaption} % issues a warning with CVPR/ICCV format
% \usepackage[font=small]{caption}
\usepackage[percent]{overpic}
%\usepackage{tikz} % conflict with ECCV format

% font and character
\usepackage{amsmath,amsfonts,amssymb}
% \let\proof\relax      % for ECCV llncs class
% \let\endproof\relax   % for ECCV llncs class
\usepackage{amsthm} 
\usepackage{bm}
\usepackage{nicefrac}
\usepackage{microtype}
\usepackage{contour}
\usepackage{courier}
%\usepackage{palatino}
%\usepackage{times}

% layout
\usepackage{changepage}
\usepackage{extramarks}
\usepackage{fancyhdr}
\usepackage{lastpage}
\usepackage{setspace}
\usepackage{soul}
\usepackage{xspace}
\usepackage{cuted}
\usepackage{fancybox}
\usepackage{afterpage}
%\usepackage{enumitem} % conflict with IEEE format
%\usepackage{titlesec} % conflict with ECCV format

% ref
% commenting these two out for this submission so it looks the same as RSS example
% \usepackage[breaklinks=true,colorlinks,backref=True]{hyperref}
% \hypersetup{colorlinks,linkcolor={black},citecolor={MSBlue},urlcolor={magenta}}
\usepackage{url}
\usepackage{quoting}
\usepackage{epigraph}

% misc
\usepackage{enumerate}
\usepackage{paralist,tabularx}
\usepackage{comment}
\usepackage{pdfpages}
% \usepackage[draft]{todonotes} % conflict with CVPR/ICCV/ECCV format



% \usepackage{todonotes}
% \usepackage{caption}
% \usepackage{subcaption}

\usepackage{pifont}% http://ctan.org/pkg/pifont

% extra symbols
\usepackage{MnSymbol}


\section{Problem Studied}\label{sec:def}
We first present Fixed-Radius Near Neighbor (FRNN) queries and then formalize Aggregation Queries over Nearest Neighbors (AQNNs) that build on them. We then state our problem.

\subsection{Nearest Neighbor Queries}\label{subsec:FRNN}
We build on generalized Fixed-Radius Near Neighbor (FRNN) queries \cite{FRNNSurvey}. Given a dataset \( D \), a query object \( q \), a radius \( r \), and a distance function \( dist \), a generalized FRNN query retrieves all nearest neighbors of \( q \) within radius \( r \). More formally:
\[
NN_D(q, r) = \{x \in D \mid dist(x, q) \leq r\},
\]
where \(x\) is any data point in \(D\) and \(dist(x, q)\) denotes the distance between them. We use \(|NN_D(q,r)|\) to denote the neighborhood size of \(q\). As shown in Fig. \ref{fig:framework}, given a radius \(r\) and a target patient \(q\), patients in the dotted circle are nearest neighbors, and the neighborhood size is 6.

\subsection{Aggregation Queries over Nearest Neighbors}\label{subsec:AQNN} 
Given an FRNN query object \(q\) in dataset \(D\), a radius \(r\), and an attribute \(\texttt{attr}\), an Aggregation Query over Nearest Neighbors (AQNN) is defined as:
\[ \text{agg}(NN_D(q,r)[\texttt{attr}]) \]
where agg is an aggregation function, such as $\mathtt{AVG}$, $\mathtt{SUM}$, and $\mathtt{PCT}$, and \(NN_D(q,r)[\texttt{attr}]\) denotes the bag of values of attribute \texttt{attr} of all FRNN results of \(q\) within radius \(r\). 
% \end{definition}

An AQNN expresses aggregation operations to capture key insights about the neighborhood of a query object. For example, \(\mathtt{AVG}\) can be used to reflect the average heart rate or systolic blood pressure of patients in the neighborhood, providing a measure of typical health conditions. \(\mathtt{SUM}\) is useful for assessing cumulative effects, such as the total cost of treatments in the neighborhood that instructs public policy in terms of health. Similarly, $\mathtt{PCT}$ can be used to find the proportion of patients in the neighborhood of a patient of interest, relative to the population in the dataset.
%\laks{Why is finding the total \#meds to NNs or the total treatment cost of everyone in the NN interesting?}

% \texttt{MIN} and \texttt{MAX} are not included in the aggregation functions because they only capture extreme values, which may not represent the typical characteristics of the nearest neighbors and are more sensitive to outliers. 
% \laks{AVG is also sensitive to outliers, but we still allow it. isn't the real reason we don't consider MIN/MAX because they are amenable to estimation via sampling?} We choose \texttt{PCT} instead of \texttt{COUNT} in order to provide a normalized measure that remains comparable across different neighborhood sizes. It allows for more consistent interpretation of relative popularity \cite{moore1989introduction}.


Fig. \ref{fig:framework} illustrates an example of an AQNN: ``\textit{Find the average systolic blood pressure of patients similar to an insomnia patient \(q\)}''. The aggregation function is \(\mathtt{AVG}\) and the target attribute of interest is systolic blood pressure. Exact query evaluation requires consulting physicians (or predicting embeddings by an expensive machine learning model) for all 500 patients in \(D\) and calculate \(q\)'s nearest neighbors wrt \(r\) \cite{DBLP:journals/isci/RodriguesGSBA21}. We refer to such highly accurate but computationally expensive models as \textit{oracle models}, denoted as \(O\), including deep learning models trained on domain-specific data or human expert annotations \cite{DBLP:conf/sigmod/LuCKC18}. Using oracle models is very expensive \cite{sze2017efficient, DujianPQA, DBLP:journals/pvldb/KangGBHZ20}. To address that, we seek an approximate solution by \textit{proxy models}, denoted as \(P\), that are at least one order of magnitude cheaper than oracle models. In the example, if consulting physicians for one patient incurs one cost unit, calling a cheap machine learning model instead incurs at most \(0.1\) cost unit. Once the similar patients are identified, their systolic blood pressure values are averaged and returned as  output. The use of a proxy model may reduce the accuracy of the neighborhood prediction and hence, we should judiciously call oracle and proxy models to minimize the error of aggregate results.

Note that the values of the target attribute \texttt{attr} are \textit{not} predicted but are instead known quantities.

\subsection{Problem Statement}
Given an AQNN, our goal is to return an approximate aggregate result by leveraging both oracle and proxy models while reducing error and cost.


% ############################

% Attempt to make hyperref and algorithmic work together better:
\newcommand{\theHalgorithm}{\arabic{algorithm}}

%%%%%%%%%%%%%%%%%%%%%%%%%%%%%%%%
% THEOREMS
%%%%%%%%%%%%%%%%%%%%%%%%%%%%%%%%
\theoremstyle{plain}
\newtheorem{theorem}{Theorem}[section]
\newtheorem{proposition}[theorem]{Proposition}
\newtheorem{lemma}[theorem]{Lemma}
\newtheorem{corollary}[theorem]{Corollary}
\theoremstyle{definition}
\newtheorem{definition}[theorem]{Definition}
\newtheorem{assumption}[theorem]{Assumption}
\theoremstyle{remark}
\newtheorem{remark}[theorem]{Remark}

% Therefore, a short form for the running title is supplied here:
\icmltitlerunning{\adapts: Adapting Univariate Foundation Models to Probabilistic Multivariate Time Series Forecasting}

\allowdisplaybreaks

\newcommand\rebuttal[1]{#1}

\begin{document}

\addtocontents{toc}{\protect\setcounter{tocdepth}{0}}

\twocolumn[
\icmltitle{\adapts: Adapting Univariate Foundation Models to Probabilistic Multivariate Time Series Forecasting}

\begin{icmlauthorlist}
\icmlauthor{Abdelhakim Benechehab}{huawei,eurecom}
\icmlauthor{Vasilii Feofanov}{huawei}
\icmlauthor{Giuseppe Paolo}{huawei}
\icmlauthor{Albert Thomas}{huawei}
\icmlauthor{Maurizio Filippone}{kaust}
\icmlauthor{Bal\'{a}zs K\'{e}gl}{huawei}
\end{icmlauthorlist}

\icmlaffiliation{huawei}{Huawei Noah's Ark Lab, Paris, France}
\icmlaffiliation{eurecom}{Department of Data Science, EURECOM}
\icmlaffiliation{kaust}{Statistics Program, KAUST}

\icmlcorrespondingauthor{Abdelhakim Benechehab}{abdelhakim.benechehab@gmail.com}

\vskip 0.3in
]

\printAffiliationsAndNotice{} 

\begin{abstract}
Pre-trained foundation models (FMs) have shown exceptional performance in univariate time series forecasting tasks. However, several practical challenges persist, including managing intricate dependencies among features and quantifying uncertainty in predictions. This study aims to tackle these critical limitations by introducing \textbf{adapters}—feature-space transformations that facilitate the effective use of pre-trained univariate time series FMs for multivariate tasks. Adapters operate by projecting multivariate inputs into a suitable latent space and applying the FM independently to each dimension. Inspired by the literature on representation learning and partially stochastic Bayesian neural networks, we present a range of adapters and optimization/inference strategies. Experiments conducted on both synthetic and real-world datasets confirm the efficacy of adapters, demonstrating substantial enhancements in forecasting accuracy and uncertainty quantification compared to baseline methods. Our framework, \textbf{AdaPTS}, positions adapters as a modular, scalable, and effective solution for leveraging time series FMs in multivariate contexts, thereby promoting their wider adoption in real-world applications. We release the code at \href{https://github.com/abenechehab/AdaPTS}{https://github.com/abenechehab/AdaPTS}.
\end{abstract}

\section{Introduction}
\label{sec:introduction}
The business processes of organizations are experiencing ever-increasing complexity due to the large amount of data, high number of users, and high-tech devices involved \cite{martin2021pmopportunitieschallenges, beerepoot2023biggestbpmproblems}. This complexity may cause business processes to deviate from normal control flow due to unforeseen and disruptive anomalies \cite{adams2023proceddsriftdetection}. These control-flow anomalies manifest as unknown, skipped, and wrongly-ordered activities in the traces of event logs monitored from the execution of business processes \cite{ko2023adsystematicreview}. For the sake of clarity, let us consider an illustrative example of such anomalies. Figure \ref{FP_ANOMALIES} shows a so-called event log footprint, which captures the control flow relations of four activities of a hypothetical event log. In particular, this footprint captures the control-flow relations between activities \texttt{a}, \texttt{b}, \texttt{c} and \texttt{d}. These are the causal ($\rightarrow$) relation, concurrent ($\parallel$) relation, and other ($\#$) relations such as exclusivity or non-local dependency \cite{aalst2022pmhandbook}. In addition, on the right are six traces, of which five exhibit skipped, wrongly-ordered and unknown control-flow anomalies. For example, $\langle$\texttt{a b d}$\rangle$ has a skipped activity, which is \texttt{c}. Because of this skipped activity, the control-flow relation \texttt{b}$\,\#\,$\texttt{d} is violated, since \texttt{d} directly follows \texttt{b} in the anomalous trace.
\begin{figure}[!t]
\centering
\includegraphics[width=0.9\columnwidth]{images/FP_ANOMALIES.png}
\caption{An example event log footprint with six traces, of which five exhibit control-flow anomalies.}
\label{FP_ANOMALIES}
\end{figure}

\subsection{Control-flow anomaly detection}
Control-flow anomaly detection techniques aim to characterize the normal control flow from event logs and verify whether these deviations occur in new event logs \cite{ko2023adsystematicreview}. To develop control-flow anomaly detection techniques, \revision{process mining} has seen widespread adoption owing to process discovery and \revision{conformance checking}. On the one hand, process discovery is a set of algorithms that encode control-flow relations as a set of model elements and constraints according to a given modeling formalism \cite{aalst2022pmhandbook}; hereafter, we refer to the Petri net, a widespread modeling formalism. On the other hand, \revision{conformance checking} is an explainable set of algorithms that allows linking any deviations with the reference Petri net and providing the fitness measure, namely a measure of how much the Petri net fits the new event log \cite{aalst2022pmhandbook}. Many control-flow anomaly detection techniques based on \revision{conformance checking} (hereafter, \revision{conformance checking}-based techniques) use the fitness measure to determine whether an event log is anomalous \cite{bezerra2009pmad, bezerra2013adlogspais, myers2018icsadpm, pecchia2020applicationfailuresanalysispm}. 

The scientific literature also includes many \revision{conformance checking}-independent techniques for control-flow anomaly detection that combine specific types of trace encodings with machine/deep learning \cite{ko2023adsystematicreview, tavares2023pmtraceencoding}. Whereas these techniques are very effective, their explainability is challenging due to both the type of trace encoding employed and the machine/deep learning model used \cite{rawal2022trustworthyaiadvances,li2023explainablead}. Hence, in the following, we focus on the shortcomings of \revision{conformance checking}-based techniques to investigate whether it is possible to support the development of competitive control-flow anomaly detection techniques while maintaining the explainable nature of \revision{conformance checking}.
\begin{figure}[!t]
\centering
\includegraphics[width=\columnwidth]{images/HIGH_LEVEL_VIEW.png}
\caption{A high-level view of the proposed framework for combining \revision{process mining}-based feature extraction with dimensionality reduction for control-flow anomaly detection.}
\label{HIGH_LEVEL_VIEW}
\end{figure}

\subsection{Shortcomings of \revision{conformance checking}-based techniques}
Unfortunately, the detection effectiveness of \revision{conformance checking}-based techniques is affected by noisy data and low-quality Petri nets, which may be due to human errors in the modeling process or representational bias of process discovery algorithms \cite{bezerra2013adlogspais, pecchia2020applicationfailuresanalysispm, aalst2016pm}. Specifically, on the one hand, noisy data may introduce infrequent and deceptive control-flow relations that may result in inconsistent fitness measures, whereas, on the other hand, checking event logs against a low-quality Petri net could lead to an unreliable distribution of fitness measures. Nonetheless, such Petri nets can still be used as references to obtain insightful information for \revision{process mining}-based feature extraction, supporting the development of competitive and explainable \revision{conformance checking}-based techniques for control-flow anomaly detection despite the problems above. For example, a few works outline that token-based \revision{conformance checking} can be used for \revision{process mining}-based feature extraction to build tabular data and develop effective \revision{conformance checking}-based techniques for control-flow anomaly detection \cite{singh2022lapmsh, debenedictis2023dtadiiot}. However, to the best of our knowledge, the scientific literature lacks a structured proposal for \revision{process mining}-based feature extraction using the state-of-the-art \revision{conformance checking} variant, namely alignment-based \revision{conformance checking}.

\subsection{Contributions}
We propose a novel \revision{process mining}-based feature extraction approach with alignment-based \revision{conformance checking}. This variant aligns the deviating control flow with a reference Petri net; the resulting alignment can be inspected to extract additional statistics such as the number of times a given activity caused mismatches \cite{aalst2022pmhandbook}. We integrate this approach into a flexible and explainable framework for developing techniques for control-flow anomaly detection. The framework combines \revision{process mining}-based feature extraction and dimensionality reduction to handle high-dimensional feature sets, achieve detection effectiveness, and support explainability. Notably, in addition to our proposed \revision{process mining}-based feature extraction approach, the framework allows employing other approaches, enabling a fair comparison of multiple \revision{conformance checking}-based and \revision{conformance checking}-independent techniques for control-flow anomaly detection. Figure \ref{HIGH_LEVEL_VIEW} shows a high-level view of the framework. Business processes are monitored, and event logs obtained from the database of information systems. Subsequently, \revision{process mining}-based feature extraction is applied to these event logs and tabular data input to dimensionality reduction to identify control-flow anomalies. We apply several \revision{conformance checking}-based and \revision{conformance checking}-independent framework techniques to publicly available datasets, simulated data of a case study from railways, and real-world data of a case study from healthcare. We show that the framework techniques implementing our approach outperform the baseline \revision{conformance checking}-based techniques while maintaining the explainable nature of \revision{conformance checking}.

In summary, the contributions of this paper are as follows.
\begin{itemize}
    \item{
        A novel \revision{process mining}-based feature extraction approach to support the development of competitive and explainable \revision{conformance checking}-based techniques for control-flow anomaly detection.
    }
    \item{
        A flexible and explainable framework for developing techniques for control-flow anomaly detection using \revision{process mining}-based feature extraction and dimensionality reduction.
    }
    \item{
        Application to synthetic and real-world datasets of several \revision{conformance checking}-based and \revision{conformance checking}-independent framework techniques, evaluating their detection effectiveness and explainability.
    }
\end{itemize}

The rest of the paper is organized as follows.
\begin{itemize}
    \item Section \ref{sec:related_work} reviews the existing techniques for control-flow anomaly detection, categorizing them into \revision{conformance checking}-based and \revision{conformance checking}-independent techniques.
    \item Section \ref{sec:abccfe} provides the preliminaries of \revision{process mining} to establish the notation used throughout the paper, and delves into the details of the proposed \revision{process mining}-based feature extraction approach with alignment-based \revision{conformance checking}.
    \item Section \ref{sec:framework} describes the framework for developing \revision{conformance checking}-based and \revision{conformance checking}-independent techniques for control-flow anomaly detection that combine \revision{process mining}-based feature extraction and dimensionality reduction.
    \item Section \ref{sec:evaluation} presents the experiments conducted with multiple framework and baseline techniques using data from publicly available datasets and case studies.
    \item Section \ref{sec:conclusions} draws the conclusions and presents future work.
\end{itemize}

\section{RELATED WORK}
\label{sec:relatedwork}
In this section, we describe the previous works related to our proposal, which are divided into two parts. In Section~\ref{sec:relatedwork_exoplanet}, we present a review of approaches based on machine learning techniques for the detection of planetary transit signals. Section~\ref{sec:relatedwork_attention} provides an account of the approaches based on attention mechanisms applied in Astronomy.\par

\subsection{Exoplanet detection}
\label{sec:relatedwork_exoplanet}
Machine learning methods have achieved great performance for the automatic selection of exoplanet transit signals. One of the earliest applications of machine learning is a model named Autovetter \citep{MCcauliff}, which is a random forest (RF) model based on characteristics derived from Kepler pipeline statistics to classify exoplanet and false positive signals. Then, other studies emerged that also used supervised learning. \cite{mislis2016sidra} also used a RF, but unlike the work by \citet{MCcauliff}, they used simulated light curves and a box least square \citep[BLS;][]{kovacs2002box}-based periodogram to search for transiting exoplanets. \citet{thompson2015machine} proposed a k-nearest neighbors model for Kepler data to determine if a given signal has similarity to known transits. Unsupervised learning techniques were also applied, such as self-organizing maps (SOM), proposed \citet{armstrong2016transit}; which implements an architecture to segment similar light curves. In the same way, \citet{armstrong2018automatic} developed a combination of supervised and unsupervised learning, including RF and SOM models. In general, these approaches require a previous phase of feature engineering for each light curve. \par

%DL is a modern data-driven technology that automatically extracts characteristics, and that has been successful in classification problems from a variety of application domains. The architecture relies on several layers of NNs of simple interconnected units and uses layers to build increasingly complex and useful features by means of linear and non-linear transformation. This family of models is capable of generating increasingly high-level representations \citep{lecun2015deep}.

The application of DL for exoplanetary signal detection has evolved rapidly in recent years and has become very popular in planetary science.  \citet{pearson2018} and \citet{zucker2018shallow} developed CNN-based algorithms that learn from synthetic data to search for exoplanets. Perhaps one of the most successful applications of the DL models in transit detection was that of \citet{Shallue_2018}; who, in collaboration with Google, proposed a CNN named AstroNet that recognizes exoplanet signals in real data from Kepler. AstroNet uses the training set of labelled TCEs from the Autovetter planet candidate catalog of Q1–Q17 data release 24 (DR24) of the Kepler mission \citep{catanzarite2015autovetter}. AstroNet analyses the data in two views: a ``global view'', and ``local view'' \citep{Shallue_2018}. \par


% The global view shows the characteristics of the light curve over an orbital period, and a local view shows the moment at occurring the transit in detail

%different = space-based

Based on AstroNet, researchers have modified the original AstroNet model to rank candidates from different surveys, specifically for Kepler and TESS missions. \citet{ansdell2018scientific} developed a CNN trained on Kepler data, and included for the first time the information on the centroids, showing that the model improves performance considerably. Then, \citet{osborn2020rapid} and \citet{yu2019identifying} also included the centroids information, but in addition, \citet{osborn2020rapid} included information of the stellar and transit parameters. Finally, \citet{rao2021nigraha} proposed a pipeline that includes a new ``half-phase'' view of the transit signal. This half-phase view represents a transit view with a different time and phase. The purpose of this view is to recover any possible secondary eclipse (the object hiding behind the disk of the primary star).


%last pipeline applies a procedure after the prediction of the model to obtain new candidates, this process is carried out through a series of steps that include the evaluation with Discovery and Validation of Exoplanets (DAVE) \citet{kostov2019discovery} that was adapted for the TESS telescope.\par
%



\subsection{Attention mechanisms in astronomy}
\label{sec:relatedwork_attention}
Despite the remarkable success of attention mechanisms in sequential data, few papers have exploited their advantages in astronomy. In particular, there are no models based on attention mechanisms for detecting planets. Below we present a summary of the main applications of this modeling approach to astronomy, based on two points of view; performance and interpretability of the model.\par
%Attention mechanisms have not yet been explored in all sub-areas of astronomy. However, recent works show a successful application of the mechanism.
%performance

The application of attention mechanisms has shown improvements in the performance of some regression and classification tasks compared to previous approaches. One of the first implementations of the attention mechanism was to find gravitational lenses proposed by \citet{thuruthipilly2021finding}. They designed 21 self-attention-based encoder models, where each model was trained separately with 18,000 simulated images, demonstrating that the model based on the Transformer has a better performance and uses fewer trainable parameters compared to CNN. A novel application was proposed by \citet{lin2021galaxy} for the morphological classification of galaxies, who used an architecture derived from the Transformer, named Vision Transformer (VIT) \citep{dosovitskiy2020image}. \citet{lin2021galaxy} demonstrated competitive results compared to CNNs. Another application with successful results was proposed by \citet{zerveas2021transformer}; which first proposed a transformer-based framework for learning unsupervised representations of multivariate time series. Their methodology takes advantage of unlabeled data to train an encoder and extract dense vector representations of time series. Subsequently, they evaluate the model for regression and classification tasks, demonstrating better performance than other state-of-the-art supervised methods, even with data sets with limited samples.

%interpretation
Regarding the interpretability of the model, a recent contribution that analyses the attention maps was presented by \citet{bowles20212}, which explored the use of group-equivariant self-attention for radio astronomy classification. Compared to other approaches, this model analysed the attention maps of the predictions and showed that the mechanism extracts the brightest spots and jets of the radio source more clearly. This indicates that attention maps for prediction interpretation could help experts see patterns that the human eye often misses. \par

In the field of variable stars, \citet{allam2021paying} employed the mechanism for classifying multivariate time series in variable stars. And additionally, \citet{allam2021paying} showed that the activation weights are accommodated according to the variation in brightness of the star, achieving a more interpretable model. And finally, related to the TESS telescope, \citet{morvan2022don} proposed a model that removes the noise from the light curves through the distribution of attention weights. \citet{morvan2022don} showed that the use of the attention mechanism is excellent for removing noise and outliers in time series datasets compared with other approaches. In addition, the use of attention maps allowed them to show the representations learned from the model. \par

Recent attention mechanism approaches in astronomy demonstrate comparable results with earlier approaches, such as CNNs. At the same time, they offer interpretability of their results, which allows a post-prediction analysis. \par



% This must be in the first 5 lines to tell arXiv to use pdfLaTeX, which is strongly recommended.
\pdfoutput=1
% In particular, the hyperref package requires pdfLaTeX in order to break URLs across lines.

\documentclass[11pt]{article}

% Change "review" to "final" to generate the final (sometimes called camera-ready) version.
% Change to "preprint" to generate a non-anonymous version with page numbers.
\usepackage{acl}

% Standard package includes
\usepackage{times}
\usepackage{latexsym}

% Draw tables
\usepackage{booktabs}
\usepackage{multirow}
\usepackage{xcolor}
\usepackage{colortbl}
\usepackage{array} 
\usepackage{amsmath}

\newcolumntype{C}{>{\centering\arraybackslash}p{0.07\textwidth}}
% For proper rendering and hyphenation of words containing Latin characters (including in bib files)
\usepackage[T1]{fontenc}
% For Vietnamese characters
% \usepackage[T5]{fontenc}
% See https://www.latex-project.org/help/documentation/encguide.pdf for other character sets
% This assumes your files are encoded as UTF8
\usepackage[utf8]{inputenc}

% This is not strictly necessary, and may be commented out,
% but it will improve the layout of the manuscript,
% and will typically save some space.
\usepackage{microtype}
\DeclareMathOperator*{\argmax}{arg\,max}
% This is also not strictly necessary, and may be commented out.
% However, it will improve the aesthetics of text in
% the typewriter font.
\usepackage{inconsolata}

%Including images in your LaTeX document requires adding
%additional package(s)
\usepackage{graphicx}
% If the title and author information does not fit in the area allocated, uncomment the following
%
%\setlength\titlebox{<dim>}
%
% and set <dim> to something 5cm or larger.

\title{Wi-Chat: Large Language Model Powered Wi-Fi Sensing}

% Author information can be set in various styles:
% For several authors from the same institution:
% \author{Author 1 \and ... \and Author n \\
%         Address line \\ ... \\ Address line}
% if the names do not fit well on one line use
%         Author 1 \\ {\bf Author 2} \\ ... \\ {\bf Author n} \\
% For authors from different institutions:
% \author{Author 1 \\ Address line \\  ... \\ Address line
%         \And  ... \And
%         Author n \\ Address line \\ ... \\ Address line}
% To start a separate ``row'' of authors use \AND, as in
% \author{Author 1 \\ Address line \\  ... \\ Address line
%         \AND
%         Author 2 \\ Address line \\ ... \\ Address line \And
%         Author 3 \\ Address line \\ ... \\ Address line}

% \author{First Author \\
%   Affiliation / Address line 1 \\
%   Affiliation / Address line 2 \\
%   Affiliation / Address line 3 \\
%   \texttt{email@domain} \\\And
%   Second Author \\
%   Affiliation / Address line 1 \\
%   Affiliation / Address line 2 \\
%   Affiliation / Address line 3 \\
%   \texttt{email@domain} \\}
% \author{Haohan Yuan \qquad Haopeng Zhang\thanks{corresponding author} \\ 
%   ALOHA Lab, University of Hawaii at Manoa \\
%   % Affiliation / Address line 2 \\
%   % Affiliation / Address line 3 \\
%   \texttt{\{haohany,haopengz\}@hawaii.edu}}
  
\author{
{Haopeng Zhang$\dag$\thanks{These authors contributed equally to this work.}, Yili Ren$\ddagger$\footnotemark[1], Haohan Yuan$\dag$, Jingzhe Zhang$\ddagger$, Yitong Shen$\ddagger$} \\
ALOHA Lab, University of Hawaii at Manoa$\dag$, University of South Florida$\ddagger$ \\
\{haopengz, haohany\}@hawaii.edu\\
\{yiliren, jingzhe, shen202\}@usf.edu\\}



  
%\author{
%  \textbf{First Author\textsuperscript{1}},
%  \textbf{Second Author\textsuperscript{1,2}},
%  \textbf{Third T. Author\textsuperscript{1}},
%  \textbf{Fourth Author\textsuperscript{1}},
%\\
%  \textbf{Fifth Author\textsuperscript{1,2}},
%  \textbf{Sixth Author\textsuperscript{1}},
%  \textbf{Seventh Author\textsuperscript{1}},
%  \textbf{Eighth Author \textsuperscript{1,2,3,4}},
%\\
%  \textbf{Ninth Author\textsuperscript{1}},
%  \textbf{Tenth Author\textsuperscript{1}},
%  \textbf{Eleventh E. Author\textsuperscript{1,2,3,4,5}},
%  \textbf{Twelfth Author\textsuperscript{1}},
%\\
%  \textbf{Thirteenth Author\textsuperscript{3}},
%  \textbf{Fourteenth F. Author\textsuperscript{2,4}},
%  \textbf{Fifteenth Author\textsuperscript{1}},
%  \textbf{Sixteenth Author\textsuperscript{1}},
%\\
%  \textbf{Seventeenth S. Author\textsuperscript{4,5}},
%  \textbf{Eighteenth Author\textsuperscript{3,4}},
%  \textbf{Nineteenth N. Author\textsuperscript{2,5}},
%  \textbf{Twentieth Author\textsuperscript{1}}
%\\
%\\
%  \textsuperscript{1}Affiliation 1,
%  \textsuperscript{2}Affiliation 2,
%  \textsuperscript{3}Affiliation 3,
%  \textsuperscript{4}Affiliation 4,
%  \textsuperscript{5}Affiliation 5
%\\
%  \small{
%    \textbf{Correspondence:} \href{mailto:email@domain}{email@domain}
%  }
%}

\begin{document}
\maketitle
\begin{abstract}
Recent advancements in Large Language Models (LLMs) have demonstrated remarkable capabilities across diverse tasks. However, their potential to integrate physical model knowledge for real-world signal interpretation remains largely unexplored. In this work, we introduce Wi-Chat, the first LLM-powered Wi-Fi-based human activity recognition system. We demonstrate that LLMs can process raw Wi-Fi signals and infer human activities by incorporating Wi-Fi sensing principles into prompts. Our approach leverages physical model insights to guide LLMs in interpreting Channel State Information (CSI) data without traditional signal processing techniques. Through experiments on real-world Wi-Fi datasets, we show that LLMs exhibit strong reasoning capabilities, achieving zero-shot activity recognition. These findings highlight a new paradigm for Wi-Fi sensing, expanding LLM applications beyond conventional language tasks and enhancing the accessibility of wireless sensing for real-world deployments.
\end{abstract}

\section{Introduction}

In today’s rapidly evolving digital landscape, the transformative power of web technologies has redefined not only how services are delivered but also how complex tasks are approached. Web-based systems have become increasingly prevalent in risk control across various domains. This widespread adoption is due their accessibility, scalability, and ability to remotely connect various types of users. For example, these systems are used for process safety management in industry~\cite{kannan2016web}, safety risk early warning in urban construction~\cite{ding2013development}, and safe monitoring of infrastructural systems~\cite{repetto2018web}. Within these web-based risk management systems, the source search problem presents a huge challenge. Source search refers to the task of identifying the origin of a risky event, such as a gas leak and the emission point of toxic substances. This source search capability is crucial for effective risk management and decision-making.

Traditional approaches to implementing source search capabilities into the web systems often rely on solely algorithmic solutions~\cite{ristic2016study}. These methods, while relatively straightforward to implement, often struggle to achieve acceptable performances due to algorithmic local optima and complex unknown environments~\cite{zhao2020searching}. More recently, web crowdsourcing has emerged as a promising alternative for tackling the source search problem by incorporating human efforts in these web systems on-the-fly~\cite{zhao2024user}. This approach outsources the task of addressing issues encountered during the source search process to human workers, leveraging their capabilities to enhance system performance.

These solutions often employ a human-AI collaborative way~\cite{zhao2023leveraging} where algorithms handle exploration-exploitation and report the encountered problems while human workers resolve complex decision-making bottlenecks to help the algorithms getting rid of local deadlocks~\cite{zhao2022crowd}. Although effective, this paradigm suffers from two inherent limitations: increased operational costs from continuous human intervention, and slow response times of human workers due to sequential decision-making. These challenges motivate our investigation into developing autonomous systems that preserve human-like reasoning capabilities while reducing dependency on massive crowdsourced labor.

Furthermore, recent advancements in large language models (LLMs)~\cite{chang2024survey} and multi-modal LLMs (MLLMs)~\cite{huang2023chatgpt} have unveiled promising avenues for addressing these challenges. One clear opportunity involves the seamless integration of visual understanding and linguistic reasoning for robust decision-making in search tasks. However, whether large models-assisted source search is really effective and efficient for improving the current source search algorithms~\cite{ji2022source} remains unknown. \textit{To address the research gap, we are particularly interested in answering the following two research questions in this work:}

\textbf{\textit{RQ1: }}How can source search capabilities be integrated into web-based systems to support decision-making in time-sensitive risk management scenarios? 
% \sq{I mention ``time-sensitive'' here because I feel like we shall say something about the response time -- LLM has to be faster than humans}

\textbf{\textit{RQ2: }}How can MLLMs and LLMs enhance the effectiveness and efficiency of existing source search algorithms? 

% \textit{\textbf{RQ2:}} To what extent does the performance of large models-assisted search align with or approach the effectiveness of human-AI collaborative search? 

To answer the research questions, we propose a novel framework called Auto-\
S$^2$earch (\textbf{Auto}nomous \textbf{S}ource \textbf{Search}) and implement a prototype system that leverages advanced web technologies to simulate real-world conditions for zero-shot source search. Unlike traditional methods that rely on pre-defined heuristics or extensive human intervention, AutoS$^2$earch employs a carefully designed prompt that encapsulates human rationales, thereby guiding the MLLM to generate coherent and accurate scene descriptions from visual inputs about four directional choices. Based on these language-based descriptions, the LLM is enabled to determine the optimal directional choice through chain-of-thought (CoT) reasoning. Comprehensive empirical validation demonstrates that AutoS$^2$-\ 
earch achieves a success rate of 95–98\%, closely approaching the performance of human-AI collaborative search across 20 benchmark scenarios~\cite{zhao2023leveraging}. 

Our work indicates that the role of humans in future web crowdsourcing tasks may evolve from executors to validators or supervisors. Furthermore, incorporating explanations of LLM decisions into web-based system interfaces has the potential to help humans enhance task performance in risk control.






\section{Related Work}
\label{sec:relatedworks}

% \begin{table*}[t]
% \centering 
% \renewcommand\arraystretch{0.98}
% \fontsize{8}{10}\selectfont \setlength{\tabcolsep}{0.4em}
% \begin{tabular}{@{}lc|cc|cc|cc@{}}
% \toprule
% \textbf{Methods}           & \begin{tabular}[c]{@{}c@{}}\textbf{Training}\\ \textbf{Paradigm}\end{tabular} & \begin{tabular}[c]{@{}c@{}}\textbf{$\#$ PT Data}\\ \textbf{(Tokens)}\end{tabular} & \begin{tabular}[c]{@{}c@{}}\textbf{$\#$ IFT Data}\\ \textbf{(Samples)}\end{tabular} & \textbf{Code}  & \begin{tabular}[c]{@{}c@{}}\textbf{Natural}\\ \textbf{Language}\end{tabular} & \begin{tabular}[c]{@{}c@{}}\textbf{Action}\\ \textbf{Trajectories}\end{tabular} & \begin{tabular}[c]{@{}c@{}}\textbf{API}\\ \textbf{Documentation}\end{tabular}\\ \midrule 
% NexusRaven~\citep{srinivasan2023nexusraven} & IFT & - & - & \textcolor{green}{\CheckmarkBold} & \textcolor{green}{\CheckmarkBold} &\textcolor{red}{\XSolidBrush}&\textcolor{red}{\XSolidBrush}\\
% AgentInstruct~\citep{zeng2023agenttuning} & IFT & - & 2k & \textcolor{green}{\CheckmarkBold} & \textcolor{green}{\CheckmarkBold} &\textcolor{red}{\XSolidBrush}&\textcolor{red}{\XSolidBrush} \\
% AgentEvol~\citep{xi2024agentgym} & IFT & - & 14.5k & \textcolor{green}{\CheckmarkBold} & \textcolor{green}{\CheckmarkBold} &\textcolor{green}{\CheckmarkBold}&\textcolor{red}{\XSolidBrush} \\
% Gorilla~\citep{patil2023gorilla}& IFT & - & 16k & \textcolor{green}{\CheckmarkBold} & \textcolor{green}{\CheckmarkBold} &\textcolor{red}{\XSolidBrush}&\textcolor{green}{\CheckmarkBold}\\
% OpenFunctions-v2~\citep{patil2023gorilla} & IFT & - & 65k & \textcolor{green}{\CheckmarkBold} & \textcolor{green}{\CheckmarkBold} &\textcolor{red}{\XSolidBrush}&\textcolor{green}{\CheckmarkBold}\\
% LAM~\citep{zhang2024agentohana} & IFT & - & 42.6k & \textcolor{green}{\CheckmarkBold} & \textcolor{green}{\CheckmarkBold} &\textcolor{green}{\CheckmarkBold}&\textcolor{red}{\XSolidBrush} \\
% xLAM~\citep{liu2024apigen} & IFT & - & 60k & \textcolor{green}{\CheckmarkBold} & \textcolor{green}{\CheckmarkBold} &\textcolor{green}{\CheckmarkBold}&\textcolor{red}{\XSolidBrush} \\\midrule
% LEMUR~\citep{xu2024lemur} & PT & 90B & 300k & \textcolor{green}{\CheckmarkBold} & \textcolor{green}{\CheckmarkBold} &\textcolor{green}{\CheckmarkBold}&\textcolor{red}{\XSolidBrush}\\
% \rowcolor{teal!12} \method & PT & 103B & 95k & \textcolor{green}{\CheckmarkBold} & \textcolor{green}{\CheckmarkBold} & \textcolor{green}{\CheckmarkBold} & \textcolor{green}{\CheckmarkBold} \\
% \bottomrule
% \end{tabular}
% \caption{Summary of existing tuning- and pretraining-based LLM agents with their training sample sizes. "PT" and "IFT" denote "Pre-Training" and "Instruction Fine-Tuning", respectively. }
% \label{tab:related}
% \end{table*}

\begin{table*}[ht]
\begin{threeparttable}
\centering 
\renewcommand\arraystretch{0.98}
\fontsize{7}{9}\selectfont \setlength{\tabcolsep}{0.2em}
\begin{tabular}{@{}l|c|c|ccc|cc|cc|cccc@{}}
\toprule
\textbf{Methods} & \textbf{Datasets}           & \begin{tabular}[c]{@{}c@{}}\textbf{Training}\\ \textbf{Paradigm}\end{tabular} & \begin{tabular}[c]{@{}c@{}}\textbf{\# PT Data}\\ \textbf{(Tokens)}\end{tabular} & \begin{tabular}[c]{@{}c@{}}\textbf{\# IFT Data}\\ \textbf{(Samples)}\end{tabular} & \textbf{\# APIs} & \textbf{Code}  & \begin{tabular}[c]{@{}c@{}}\textbf{Nat.}\\ \textbf{Lang.}\end{tabular} & \begin{tabular}[c]{@{}c@{}}\textbf{Action}\\ \textbf{Traj.}\end{tabular} & \begin{tabular}[c]{@{}c@{}}\textbf{API}\\ \textbf{Doc.}\end{tabular} & \begin{tabular}[c]{@{}c@{}}\textbf{Func.}\\ \textbf{Call}\end{tabular} & \begin{tabular}[c]{@{}c@{}}\textbf{Multi.}\\ \textbf{Step}\end{tabular}  & \begin{tabular}[c]{@{}c@{}}\textbf{Plan}\\ \textbf{Refine}\end{tabular}  & \begin{tabular}[c]{@{}c@{}}\textbf{Multi.}\\ \textbf{Turn}\end{tabular}\\ \midrule 
\multicolumn{13}{l}{\emph{Instruction Finetuning-based LLM Agents for Intrinsic Reasoning}}  \\ \midrule
FireAct~\cite{chen2023fireact} & FireAct & IFT & - & 2.1K & 10 & \textcolor{red}{\XSolidBrush} &\textcolor{green}{\CheckmarkBold} &\textcolor{green}{\CheckmarkBold}  & \textcolor{red}{\XSolidBrush} &\textcolor{green}{\CheckmarkBold} & \textcolor{red}{\XSolidBrush} &\textcolor{green}{\CheckmarkBold} & \textcolor{red}{\XSolidBrush} \\
ToolAlpaca~\cite{tang2023toolalpaca} & ToolAlpaca & IFT & - & 4.0K & 400 & \textcolor{red}{\XSolidBrush} &\textcolor{green}{\CheckmarkBold} &\textcolor{green}{\CheckmarkBold} & \textcolor{red}{\XSolidBrush} &\textcolor{green}{\CheckmarkBold} & \textcolor{red}{\XSolidBrush}  &\textcolor{green}{\CheckmarkBold} & \textcolor{red}{\XSolidBrush}  \\
ToolLLaMA~\cite{qin2023toolllm} & ToolBench & IFT & - & 12.7K & 16,464 & \textcolor{red}{\XSolidBrush} &\textcolor{green}{\CheckmarkBold} &\textcolor{green}{\CheckmarkBold} &\textcolor{red}{\XSolidBrush} &\textcolor{green}{\CheckmarkBold}&\textcolor{green}{\CheckmarkBold}&\textcolor{green}{\CheckmarkBold} &\textcolor{green}{\CheckmarkBold}\\
AgentEvol~\citep{xi2024agentgym} & AgentTraj-L & IFT & - & 14.5K & 24 &\textcolor{red}{\XSolidBrush} & \textcolor{green}{\CheckmarkBold} &\textcolor{green}{\CheckmarkBold}&\textcolor{red}{\XSolidBrush} &\textcolor{green}{\CheckmarkBold}&\textcolor{red}{\XSolidBrush} &\textcolor{red}{\XSolidBrush} &\textcolor{green}{\CheckmarkBold}\\
Lumos~\cite{yin2024agent} & Lumos & IFT  & - & 20.0K & 16 &\textcolor{red}{\XSolidBrush} & \textcolor{green}{\CheckmarkBold} & \textcolor{green}{\CheckmarkBold} &\textcolor{red}{\XSolidBrush} & \textcolor{green}{\CheckmarkBold} & \textcolor{green}{\CheckmarkBold} &\textcolor{red}{\XSolidBrush} & \textcolor{green}{\CheckmarkBold}\\
Agent-FLAN~\cite{chen2024agent} & Agent-FLAN & IFT & - & 24.7K & 20 &\textcolor{red}{\XSolidBrush} & \textcolor{green}{\CheckmarkBold} & \textcolor{green}{\CheckmarkBold} &\textcolor{red}{\XSolidBrush} & \textcolor{green}{\CheckmarkBold}& \textcolor{green}{\CheckmarkBold}&\textcolor{red}{\XSolidBrush} & \textcolor{green}{\CheckmarkBold}\\
AgentTuning~\citep{zeng2023agenttuning} & AgentInstruct & IFT & - & 35.0K & - &\textcolor{red}{\XSolidBrush} & \textcolor{green}{\CheckmarkBold} & \textcolor{green}{\CheckmarkBold} &\textcolor{red}{\XSolidBrush} & \textcolor{green}{\CheckmarkBold} &\textcolor{red}{\XSolidBrush} &\textcolor{red}{\XSolidBrush} & \textcolor{green}{\CheckmarkBold}\\\midrule
\multicolumn{13}{l}{\emph{Instruction Finetuning-based LLM Agents for Function Calling}} \\\midrule
NexusRaven~\citep{srinivasan2023nexusraven} & NexusRaven & IFT & - & - & 116 & \textcolor{green}{\CheckmarkBold} & \textcolor{green}{\CheckmarkBold}  & \textcolor{green}{\CheckmarkBold} &\textcolor{red}{\XSolidBrush} & \textcolor{green}{\CheckmarkBold} &\textcolor{red}{\XSolidBrush} &\textcolor{red}{\XSolidBrush}&\textcolor{red}{\XSolidBrush}\\
Gorilla~\citep{patil2023gorilla} & Gorilla & IFT & - & 16.0K & 1,645 & \textcolor{green}{\CheckmarkBold} &\textcolor{red}{\XSolidBrush} &\textcolor{red}{\XSolidBrush}&\textcolor{green}{\CheckmarkBold} &\textcolor{green}{\CheckmarkBold} &\textcolor{red}{\XSolidBrush} &\textcolor{red}{\XSolidBrush} &\textcolor{red}{\XSolidBrush}\\
OpenFunctions-v2~\citep{patil2023gorilla} & OpenFunctions-v2 & IFT & - & 65.0K & - & \textcolor{green}{\CheckmarkBold} & \textcolor{green}{\CheckmarkBold} &\textcolor{red}{\XSolidBrush} &\textcolor{green}{\CheckmarkBold} &\textcolor{green}{\CheckmarkBold} &\textcolor{red}{\XSolidBrush} &\textcolor{red}{\XSolidBrush} &\textcolor{red}{\XSolidBrush}\\
API Pack~\cite{guo2024api} & API Pack & IFT & - & 1.1M & 11,213 &\textcolor{green}{\CheckmarkBold} &\textcolor{red}{\XSolidBrush} &\textcolor{green}{\CheckmarkBold} &\textcolor{red}{\XSolidBrush} &\textcolor{green}{\CheckmarkBold} &\textcolor{red}{\XSolidBrush}&\textcolor{red}{\XSolidBrush}&\textcolor{red}{\XSolidBrush}\\ 
LAM~\citep{zhang2024agentohana} & AgentOhana & IFT & - & 42.6K & - & \textcolor{green}{\CheckmarkBold} & \textcolor{green}{\CheckmarkBold} &\textcolor{green}{\CheckmarkBold}&\textcolor{red}{\XSolidBrush} &\textcolor{green}{\CheckmarkBold}&\textcolor{red}{\XSolidBrush}&\textcolor{green}{\CheckmarkBold}&\textcolor{green}{\CheckmarkBold}\\
xLAM~\citep{liu2024apigen} & APIGen & IFT & - & 60.0K & 3,673 & \textcolor{green}{\CheckmarkBold} & \textcolor{green}{\CheckmarkBold} &\textcolor{green}{\CheckmarkBold}&\textcolor{red}{\XSolidBrush} &\textcolor{green}{\CheckmarkBold}&\textcolor{red}{\XSolidBrush}&\textcolor{green}{\CheckmarkBold}&\textcolor{green}{\CheckmarkBold}\\\midrule
\multicolumn{13}{l}{\emph{Pretraining-based LLM Agents}}  \\\midrule
% LEMUR~\citep{xu2024lemur} & PT & 90B & 300.0K & - & \textcolor{green}{\CheckmarkBold} & \textcolor{green}{\CheckmarkBold} &\textcolor{green}{\CheckmarkBold}&\textcolor{red}{\XSolidBrush} & \textcolor{red}{\XSolidBrush} &\textcolor{green}{\CheckmarkBold} &\textcolor{red}{\XSolidBrush}&\textcolor{red}{\XSolidBrush}\\
\rowcolor{teal!12} \method & \dataset & PT & 103B & 95.0K  & 76,537  & \textcolor{green}{\CheckmarkBold} & \textcolor{green}{\CheckmarkBold} & \textcolor{green}{\CheckmarkBold} & \textcolor{green}{\CheckmarkBold} & \textcolor{green}{\CheckmarkBold} & \textcolor{green}{\CheckmarkBold} & \textcolor{green}{\CheckmarkBold} & \textcolor{green}{\CheckmarkBold}\\
\bottomrule
\end{tabular}
% \begin{tablenotes}
%     \item $^*$ In addition, the StarCoder-API can offer 4.77M more APIs.
% \end{tablenotes}
\caption{Summary of existing instruction finetuning-based LLM agents for intrinsic reasoning and function calling, along with their training resources and sample sizes. "PT" and "IFT" denote "Pre-Training" and "Instruction Fine-Tuning", respectively.}
\vspace{-2ex}
\label{tab:related}
\end{threeparttable}
\end{table*}

\noindent \textbf{Prompting-based LLM Agents.} Due to the lack of agent-specific pre-training corpus, existing LLM agents rely on either prompt engineering~\cite{hsieh2023tool,lu2024chameleon,yao2022react,wang2023voyager} or instruction fine-tuning~\cite{chen2023fireact,zeng2023agenttuning} to understand human instructions, decompose high-level tasks, generate grounded plans, and execute multi-step actions. 
However, prompting-based methods mainly depend on the capabilities of backbone LLMs (usually commercial LLMs), failing to introduce new knowledge and struggling to generalize to unseen tasks~\cite{sun2024adaplanner,zhuang2023toolchain}. 

\noindent \textbf{Instruction Finetuning-based LLM Agents.} Considering the extensive diversity of APIs and the complexity of multi-tool instructions, tool learning inherently presents greater challenges than natural language tasks, such as text generation~\cite{qin2023toolllm}.
Post-training techniques focus more on instruction following and aligning output with specific formats~\cite{patil2023gorilla,hao2024toolkengpt,qin2023toolllm,schick2024toolformer}, rather than fundamentally improving model knowledge or capabilities. 
Moreover, heavy fine-tuning can hinder generalization or even degrade performance in non-agent use cases, potentially suppressing the original base model capabilities~\cite{ghosh2024a}.

\noindent \textbf{Pretraining-based LLM Agents.} While pre-training serves as an essential alternative, prior works~\cite{nijkamp2023codegen,roziere2023code,xu2024lemur,patil2023gorilla} have primarily focused on improving task-specific capabilities (\eg, code generation) instead of general-domain LLM agents, due to single-source, uni-type, small-scale, and poor-quality pre-training data. 
Existing tool documentation data for agent training either lacks diverse real-world APIs~\cite{patil2023gorilla, tang2023toolalpaca} or is constrained to single-tool or single-round tool execution. 
Furthermore, trajectory data mostly imitate expert behavior or follow function-calling rules with inferior planning and reasoning, failing to fully elicit LLMs' capabilities and handle complex instructions~\cite{qin2023toolllm}. 
Given a wide range of candidate API functions, each comprising various function names and parameters available at every planning step, identifying globally optimal solutions and generalizing across tasks remains highly challenging.



\section{Preliminaries}
\label{Preliminaries}
\begin{figure*}[t]
    \centering
    \includegraphics[width=0.95\linewidth]{fig/HealthGPT_Framework.png}
    \caption{The \ourmethod{} architecture integrates hierarchical visual perception and H-LoRA, employing a task-specific hard router to select visual features and H-LoRA plugins, ultimately generating outputs with an autoregressive manner.}
    \label{fig:architecture}
\end{figure*}
\noindent\textbf{Large Vision-Language Models.} 
The input to a LVLM typically consists of an image $x^{\text{img}}$ and a discrete text sequence $x^{\text{txt}}$. The visual encoder $\mathcal{E}^{\text{img}}$ converts the input image $x^{\text{img}}$ into a sequence of visual tokens $\mathcal{V} = [v_i]_{i=1}^{N_v}$, while the text sequence $x^{\text{txt}}$ is mapped into a sequence of text tokens $\mathcal{T} = [t_i]_{i=1}^{N_t}$ using an embedding function $\mathcal{E}^{\text{txt}}$. The LLM $\mathcal{M_\text{LLM}}(\cdot|\theta)$ models the joint probability of the token sequence $\mathcal{U} = \{\mathcal{V},\mathcal{T}\}$, which is expressed as:
\begin{equation}
    P_\theta(R | \mathcal{U}) = \prod_{i=1}^{N_r} P_\theta(r_i | \{\mathcal{U}, r_{<i}\}),
\end{equation}
where $R = [r_i]_{i=1}^{N_r}$ is the text response sequence. The LVLM iteratively generates the next token $r_i$ based on $r_{<i}$. The optimization objective is to minimize the cross-entropy loss of the response $\mathcal{R}$.
% \begin{equation}
%     \mathcal{L}_{\text{VLM}} = \mathbb{E}_{R|\mathcal{U}}\left[-\log P_\theta(R | \mathcal{U})\right]
% \end{equation}
It is worth noting that most LVLMs adopt a design paradigm based on ViT, alignment adapters, and pre-trained LLMs\cite{liu2023llava,liu2024improved}, enabling quick adaptation to downstream tasks.


\noindent\textbf{VQGAN.}
VQGAN~\cite{esser2021taming} employs latent space compression and indexing mechanisms to effectively learn a complete discrete representation of images. VQGAN first maps the input image $x^{\text{img}}$ to a latent representation $z = \mathcal{E}(x)$ through a encoder $\mathcal{E}$. Then, the latent representation is quantized using a codebook $\mathcal{Z} = \{z_k\}_{k=1}^K$, generating a discrete index sequence $\mathcal{I} = [i_m]_{m=1}^N$, where $i_m \in \mathcal{Z}$ represents the quantized code index:
\begin{equation}
    \mathcal{I} = \text{Quantize}(z|\mathcal{Z}) = \arg\min_{z_k \in \mathcal{Z}} \| z - z_k \|_2.
\end{equation}
In our approach, the discrete index sequence $\mathcal{I}$ serves as a supervisory signal for the generation task, enabling the model to predict the index sequence $\hat{\mathcal{I}}$ from input conditions such as text or other modality signals.  
Finally, the predicted index sequence $\hat{\mathcal{I}}$ is upsampled by the VQGAN decoder $G$, generating the high-quality image $\hat{x}^\text{img} = G(\hat{\mathcal{I}})$.



\noindent\textbf{Low Rank Adaptation.} 
LoRA\cite{hu2021lora} effectively captures the characteristics of downstream tasks by introducing low-rank adapters. The core idea is to decompose the bypass weight matrix $\Delta W\in\mathbb{R}^{d^{\text{in}} \times d^{\text{out}}}$ into two low-rank matrices $ \{A \in \mathbb{R}^{d^{\text{in}} \times r}, B \in \mathbb{R}^{r \times d^{\text{out}}} \}$, where $ r \ll \min\{d^{\text{in}}, d^{\text{out}}\} $, significantly reducing learnable parameters. The output with the LoRA adapter for the input $x$ is then given by:
\begin{equation}
    h = x W_0 + \alpha x \Delta W/r = x W_0 + \alpha xAB/r,
\end{equation}
where matrix $ A $ is initialized with a Gaussian distribution, while the matrix $ B $ is initialized as a zero matrix. The scaling factor $ \alpha/r $ controls the impact of $ \Delta W $ on the model.

\section{HealthGPT}
\label{Method}


\subsection{Unified Autoregressive Generation.}  
% As shown in Figure~\ref{fig:architecture}, 
\ourmethod{} (Figure~\ref{fig:architecture}) utilizes a discrete token representation that covers both text and visual outputs, unifying visual comprehension and generation as an autoregressive task. 
For comprehension, $\mathcal{M}_\text{llm}$ receives the input joint sequence $\mathcal{U}$ and outputs a series of text token $\mathcal{R} = [r_1, r_2, \dots, r_{N_r}]$, where $r_i \in \mathcal{V}_{\text{txt}}$, and $\mathcal{V}_{\text{txt}}$ represents the LLM's vocabulary:
\begin{equation}
    P_\theta(\mathcal{R} \mid \mathcal{U}) = \prod_{i=1}^{N_r} P_\theta(r_i \mid \mathcal{U}, r_{<i}).
\end{equation}
For generation, $\mathcal{M}_\text{llm}$ first receives a special start token $\langle \text{START\_IMG} \rangle$, then generates a series of tokens corresponding to the VQGAN indices $\mathcal{I} = [i_1, i_2, \dots, i_{N_i}]$, where $i_j \in \mathcal{V}_{\text{vq}}$, and $\mathcal{V}_{\text{vq}}$ represents the index range of VQGAN. Upon completion of generation, the LLM outputs an end token $\langle \text{END\_IMG} \rangle$:
\begin{equation}
    P_\theta(\mathcal{I} \mid \mathcal{U}) = \prod_{j=1}^{N_i} P_\theta(i_j \mid \mathcal{U}, i_{<j}).
\end{equation}
Finally, the generated index sequence $\mathcal{I}$ is fed into the decoder $G$, which reconstructs the target image $\hat{x}^{\text{img}} = G(\mathcal{I})$.

\subsection{Hierarchical Visual Perception}  
Given the differences in visual perception between comprehension and generation tasks—where the former focuses on abstract semantics and the latter emphasizes complete semantics—we employ ViT to compress the image into discrete visual tokens at multiple hierarchical levels.
Specifically, the image is converted into a series of features $\{f_1, f_2, \dots, f_L\}$ as it passes through $L$ ViT blocks.

To address the needs of various tasks, the hidden states are divided into two types: (i) \textit{Concrete-grained features} $\mathcal{F}^{\text{Con}} = \{f_1, f_2, \dots, f_k\}, k < L$, derived from the shallower layers of ViT, containing sufficient global features, suitable for generation tasks; 
(ii) \textit{Abstract-grained features} $\mathcal{F}^{\text{Abs}} = \{f_{k+1}, f_{k+2}, \dots, f_L\}$, derived from the deeper layers of ViT, which contain abstract semantic information closer to the text space, suitable for comprehension tasks.

The task type $T$ (comprehension or generation) determines which set of features is selected as the input for the downstream large language model:
\begin{equation}
    \mathcal{F}^{\text{img}}_T =
    \begin{cases}
        \mathcal{F}^{\text{Con}}, & \text{if } T = \text{generation task} \\
        \mathcal{F}^{\text{Abs}}, & \text{if } T = \text{comprehension task}
    \end{cases}
\end{equation}
We integrate the image features $\mathcal{F}^{\text{img}}_T$ and text features $\mathcal{T}$ into a joint sequence through simple concatenation, which is then fed into the LLM $\mathcal{M}_{\text{llm}}$ for autoregressive generation.
% :
% \begin{equation}
%     \mathcal{R} = \mathcal{M}_{\text{llm}}(\mathcal{U}|\theta), \quad \mathcal{U} = [\mathcal{F}^{\text{img}}_T; \mathcal{T}]
% \end{equation}
\subsection{Heterogeneous Knowledge Adaptation}
We devise H-LoRA, which stores heterogeneous knowledge from comprehension and generation tasks in separate modules and dynamically routes to extract task-relevant knowledge from these modules. 
At the task level, for each task type $ T $, we dynamically assign a dedicated H-LoRA submodule $ \theta^T $, which is expressed as:
\begin{equation}
    \mathcal{R} = \mathcal{M}_\text{LLM}(\mathcal{U}|\theta, \theta^T), \quad \theta^T = \{A^T, B^T, \mathcal{R}^T_\text{outer}\}.
\end{equation}
At the feature level for a single task, H-LoRA integrates the idea of Mixture of Experts (MoE)~\cite{masoudnia2014mixture} and designs an efficient matrix merging and routing weight allocation mechanism, thus avoiding the significant computational delay introduced by matrix splitting in existing MoELoRA~\cite{luo2024moelora}. Specifically, we first merge the low-rank matrices (rank = r) of $ k $ LoRA experts into a unified matrix:
\begin{equation}
    \mathbf{A}^{\text{merged}}, \mathbf{B}^{\text{merged}} = \text{Concat}(\{A_i\}_1^k), \text{Concat}(\{B_i\}_1^k),
\end{equation}
where $ \mathbf{A}^{\text{merged}} \in \mathbb{R}^{d^\text{in} \times rk} $ and $ \mathbf{B}^{\text{merged}} \in \mathbb{R}^{rk \times d^\text{out}} $. The $k$-dimension routing layer generates expert weights $ \mathcal{W} \in \mathbb{R}^{\text{token\_num} \times k} $ based on the input hidden state $ x $, and these are expanded to $ \mathbb{R}^{\text{token\_num} \times rk} $ as follows:
\begin{equation}
    \mathcal{W}^\text{expanded} = \alpha k \mathcal{W} / r \otimes \mathbf{1}_r,
\end{equation}
where $ \otimes $ denotes the replication operation.
The overall output of H-LoRA is computed as:
\begin{equation}
    \mathcal{O}^\text{H-LoRA} = (x \mathbf{A}^{\text{merged}} \odot \mathcal{W}^\text{expanded}) \mathbf{B}^{\text{merged}},
\end{equation}
where $ \odot $ represents element-wise multiplication. Finally, the output of H-LoRA is added to the frozen pre-trained weights to produce the final output:
\begin{equation}
    \mathcal{O} = x W_0 + \mathcal{O}^\text{H-LoRA}.
\end{equation}
% In summary, H-LoRA is a task-based dynamic PEFT method that achieves high efficiency in single-task fine-tuning.

\subsection{Training Pipeline}

\begin{figure}[t]
    \centering
    \hspace{-4mm}
    \includegraphics[width=0.94\linewidth]{fig/data.pdf}
    \caption{Data statistics of \texttt{VL-Health}. }
    \label{fig:data}
\end{figure}
\noindent \textbf{1st Stage: Multi-modal Alignment.} 
In the first stage, we design separate visual adapters and H-LoRA submodules for medical unified tasks. For the medical comprehension task, we train abstract-grained visual adapters using high-quality image-text pairs to align visual embeddings with textual embeddings, thereby enabling the model to accurately describe medical visual content. During this process, the pre-trained LLM and its corresponding H-LoRA submodules remain frozen. In contrast, the medical generation task requires training concrete-grained adapters and H-LoRA submodules while keeping the LLM frozen. Meanwhile, we extend the textual vocabulary to include multimodal tokens, enabling the support of additional VQGAN vector quantization indices. The model trains on image-VQ pairs, endowing the pre-trained LLM with the capability for image reconstruction. This design ensures pixel-level consistency of pre- and post-LVLM. The processes establish the initial alignment between the LLM’s outputs and the visual inputs.

\noindent \textbf{2nd Stage: Heterogeneous H-LoRA Plugin Adaptation.}  
The submodules of H-LoRA share the word embedding layer and output head but may encounter issues such as bias and scale inconsistencies during training across different tasks. To ensure that the multiple H-LoRA plugins seamlessly interface with the LLMs and form a unified base, we fine-tune the word embedding layer and output head using a small amount of mixed data to maintain consistency in the model weights. Specifically, during this stage, all H-LoRA submodules for different tasks are kept frozen, with only the word embedding layer and output head being optimized. Through this stage, the model accumulates foundational knowledge for unified tasks by adapting H-LoRA plugins.

\begin{table*}[!t]
\centering
\caption{Comparison of \ourmethod{} with other LVLMs and unified multi-modal models on medical visual comprehension tasks. \textbf{Bold} and \underline{underlined} text indicates the best performance and second-best performance, respectively.}
\resizebox{\textwidth}{!}{
\begin{tabular}{c|lcc|cccccccc|c}
\toprule
\rowcolor[HTML]{E9F3FE} &  &  &  & \multicolumn{2}{c}{\textbf{VQA-RAD \textuparrow}} & \multicolumn{2}{c}{\textbf{SLAKE \textuparrow}} & \multicolumn{2}{c}{\textbf{PathVQA \textuparrow}} &  &  &  \\ 
\cline{5-10}
\rowcolor[HTML]{E9F3FE}\multirow{-2}{*}{\textbf{Type}} & \multirow{-2}{*}{\textbf{Model}} & \multirow{-2}{*}{\textbf{\# Params}} & \multirow{-2}{*}{\makecell{\textbf{Medical} \\ \textbf{LVLM}}} & \textbf{close} & \textbf{all} & \textbf{close} & \textbf{all} & \textbf{close} & \textbf{all} & \multirow{-2}{*}{\makecell{\textbf{MMMU} \\ \textbf{-Med}}\textuparrow} & \multirow{-2}{*}{\textbf{OMVQA}\textuparrow} & \multirow{-2}{*}{\textbf{Avg. \textuparrow}} \\ 
\midrule \midrule
\multirow{9}{*}{\textbf{Comp. Only}} 
& Med-Flamingo & 8.3B & \Large \ding{51} & 58.6 & 43.0 & 47.0 & 25.5 & 61.9 & 31.3 & 28.7 & 34.9 & 41.4 \\
& LLaVA-Med & 7B & \Large \ding{51} & 60.2 & 48.1 & 58.4 & 44.8 & 62.3 & 35.7 & 30.0 & 41.3 & 47.6 \\
& HuatuoGPT-Vision & 7B & \Large \ding{51} & 66.9 & 53.0 & 59.8 & 49.1 & 52.9 & 32.0 & 42.0 & 50.0 & 50.7 \\
& BLIP-2 & 6.7B & \Large \ding{55} & 43.4 & 36.8 & 41.6 & 35.3 & 48.5 & 28.8 & 27.3 & 26.9 & 36.1 \\
& LLaVA-v1.5 & 7B & \Large \ding{55} & 51.8 & 42.8 & 37.1 & 37.7 & 53.5 & 31.4 & 32.7 & 44.7 & 41.5 \\
& InstructBLIP & 7B & \Large \ding{55} & 61.0 & 44.8 & 66.8 & 43.3 & 56.0 & 32.3 & 25.3 & 29.0 & 44.8 \\
& Yi-VL & 6B & \Large \ding{55} & 52.6 & 42.1 & 52.4 & 38.4 & 54.9 & 30.9 & 38.0 & 50.2 & 44.9 \\
& InternVL2 & 8B & \Large \ding{55} & 64.9 & 49.0 & 66.6 & 50.1 & 60.0 & 31.9 & \underline{43.3} & 54.5 & 52.5\\
& Llama-3.2 & 11B & \Large \ding{55} & 68.9 & 45.5 & 72.4 & 52.1 & 62.8 & 33.6 & 39.3 & 63.2 & 54.7 \\
\midrule
\multirow{5}{*}{\textbf{Comp. \& Gen.}} 
& Show-o & 1.3B & \Large \ding{55} & 50.6 & 33.9 & 31.5 & 17.9 & 52.9 & 28.2 & 22.7 & 45.7 & 42.6 \\
& Unified-IO 2 & 7B & \Large \ding{55} & 46.2 & 32.6 & 35.9 & 21.9 & 52.5 & 27.0 & 25.3 & 33.0 & 33.8 \\
& Janus & 1.3B & \Large \ding{55} & 70.9 & 52.8 & 34.7 & 26.9 & 51.9 & 27.9 & 30.0 & 26.8 & 33.5 \\
& \cellcolor[HTML]{DAE0FB}HealthGPT-M3 & \cellcolor[HTML]{DAE0FB}3.8B & \cellcolor[HTML]{DAE0FB}\Large \ding{51} & \cellcolor[HTML]{DAE0FB}\underline{73.7} & \cellcolor[HTML]{DAE0FB}\underline{55.9} & \cellcolor[HTML]{DAE0FB}\underline{74.6} & \cellcolor[HTML]{DAE0FB}\underline{56.4} & \cellcolor[HTML]{DAE0FB}\underline{78.7} & \cellcolor[HTML]{DAE0FB}\underline{39.7} & \cellcolor[HTML]{DAE0FB}\underline{43.3} & \cellcolor[HTML]{DAE0FB}\underline{68.5} & \cellcolor[HTML]{DAE0FB}\underline{61.3} \\
& \cellcolor[HTML]{DAE0FB}HealthGPT-L14 & \cellcolor[HTML]{DAE0FB}14B & \cellcolor[HTML]{DAE0FB}\Large \ding{51} & \cellcolor[HTML]{DAE0FB}\textbf{77.7} & \cellcolor[HTML]{DAE0FB}\textbf{58.3} & \cellcolor[HTML]{DAE0FB}\textbf{76.4} & \cellcolor[HTML]{DAE0FB}\textbf{64.5} & \cellcolor[HTML]{DAE0FB}\textbf{85.9} & \cellcolor[HTML]{DAE0FB}\textbf{44.4} & \cellcolor[HTML]{DAE0FB}\textbf{49.2} & \cellcolor[HTML]{DAE0FB}\textbf{74.4} & \cellcolor[HTML]{DAE0FB}\textbf{66.4} \\
\bottomrule
\end{tabular}
}
\label{tab:results}
\end{table*}
\begin{table*}[ht]
    \centering
    \caption{The experimental results for the four modality conversion tasks.}
    \resizebox{\textwidth}{!}{
    \begin{tabular}{l|ccc|ccc|ccc|ccc}
        \toprule
        \rowcolor[HTML]{E9F3FE} & \multicolumn{3}{c}{\textbf{CT to MRI (Brain)}} & \multicolumn{3}{c}{\textbf{CT to MRI (Pelvis)}} & \multicolumn{3}{c}{\textbf{MRI to CT (Brain)}} & \multicolumn{3}{c}{\textbf{MRI to CT (Pelvis)}} \\
        \cline{2-13}
        \rowcolor[HTML]{E9F3FE}\multirow{-2}{*}{\textbf{Model}}& \textbf{SSIM $\uparrow$} & \textbf{PSNR $\uparrow$} & \textbf{MSE $\downarrow$} & \textbf{SSIM $\uparrow$} & \textbf{PSNR $\uparrow$} & \textbf{MSE $\downarrow$} & \textbf{SSIM $\uparrow$} & \textbf{PSNR $\uparrow$} & \textbf{MSE $\downarrow$} & \textbf{SSIM $\uparrow$} & \textbf{PSNR $\uparrow$} & \textbf{MSE $\downarrow$} \\
        \midrule \midrule
        pix2pix & 71.09 & 32.65 & 36.85 & 59.17 & 31.02 & 51.91 & 78.79 & 33.85 & 28.33 & 72.31 & 32.98 & 36.19 \\
        CycleGAN & 54.76 & 32.23 & 40.56 & 54.54 & 30.77 & 55.00 & 63.75 & 31.02 & 52.78 & 50.54 & 29.89 & 67.78 \\
        BBDM & {71.69} & {32.91} & {34.44} & 57.37 & 31.37 & 48.06 & \textbf{86.40} & 34.12 & 26.61 & {79.26} & 33.15 & 33.60 \\
        Vmanba & 69.54 & 32.67 & 36.42 & {63.01} & {31.47} & {46.99} & 79.63 & 34.12 & 26.49 & 77.45 & 33.53 & 31.85 \\
        DiffMa & 71.47 & 32.74 & 35.77 & 62.56 & 31.43 & 47.38 & 79.00 & {34.13} & {26.45} & 78.53 & {33.68} & {30.51} \\
        \rowcolor[HTML]{DAE0FB}HealthGPT-M3 & \underline{79.38} & \underline{33.03} & \underline{33.48} & \underline{71.81} & \underline{31.83} & \underline{43.45} & {85.06} & \textbf{34.40} & \textbf{25.49} & \underline{84.23} & \textbf{34.29} & \textbf{27.99} \\
        \rowcolor[HTML]{DAE0FB}HealthGPT-L14 & \textbf{79.73} & \textbf{33.10} & \textbf{32.96} & \textbf{71.92} & \textbf{31.87} & \textbf{43.09} & \underline{85.31} & \underline{34.29} & \underline{26.20} & \textbf{84.96} & \underline{34.14} & \underline{28.13} \\
        \bottomrule
    \end{tabular}
    }
    \label{tab:conversion}
\end{table*}

\noindent \textbf{3rd Stage: Visual Instruction Fine-Tuning.}  
In the third stage, we introduce additional task-specific data to further optimize the model and enhance its adaptability to downstream tasks such as medical visual comprehension (e.g., medical QA, medical dialogues, and report generation) or generation tasks (e.g., super-resolution, denoising, and modality conversion). Notably, by this stage, the word embedding layer and output head have been fine-tuned, only the H-LoRA modules and adapter modules need to be trained. This strategy significantly improves the model's adaptability and flexibility across different tasks.


\section{Experiment}
\label{s:experiment}

\subsection{Data Description}
We evaluate our method on FI~\cite{you2016building}, Twitter\_LDL~\cite{yang2017learning} and Artphoto~\cite{machajdik2010affective}.
FI is a public dataset built from Flickr and Instagram, with 23,308 images and eight emotion categories, namely \textit{amusement}, \textit{anger}, \textit{awe},  \textit{contentment}, \textit{disgust}, \textit{excitement},  \textit{fear}, and \textit{sadness}. 
% Since images in FI are all copyrighted by law, some images are corrupted now, so we remove these samples and retain 21,828 images.
% T4SA contains images from Twitter, which are classified into three categories: \textit{positive}, \textit{neutral}, and \textit{negative}. In this paper, we adopt the base version of B-T4SA, which contains 470,586 images and provides text descriptions of the corresponding tweets.
Twitter\_LDL contains 10,045 images from Twitter, with the same eight categories as the FI dataset.
% 。
For these two datasets, they are randomly split into 80\%
training and 20\% testing set.
Artphoto contains 806 artistic photos from the DeviantArt website, which we use to further evaluate the zero-shot capability of our model.
% on the small-scale dataset.
% We construct and publicly release the first image sentiment analysis dataset containing metadata.
% 。

% Based on these datasets, we are the first to construct and publicly release metadata-enhanced image sentiment analysis datasets. These datasets include scenes, tags, descriptions, and corresponding confidence scores, and are available at this link for future research purposes.


% 
\begin{table}[t]
\centering
% \begin{center}
\caption{Overall performance of different models on FI and Twitter\_LDL datasets.}
\label{tab:cap1}
% \resizebox{\linewidth}{!}
{
\begin{tabular}{l|c|c|c|c}
\hline
\multirow{2}{*}{\textbf{Model}} & \multicolumn{2}{c|}{\textbf{FI}}  & \multicolumn{2}{c}{\textbf{Twitter\_LDL}} \\ \cline{2-5} 
  & \textbf{Accuracy} & \textbf{F1} & \textbf{Accuracy} & \textbf{F1}  \\ \hline
% (\rownumber)~AlexNet~\cite{krizhevsky2017imagenet}  & 58.13\% & 56.35\%  & 56.24\%& 55.02\%  \\ 
% (\rownumber)~VGG16~\cite{simonyan2014very}  & 63.75\%& 63.08\%  & 59.34\%& 59.02\%  \\ 
(\rownumber)~ResNet101~\cite{he2016deep} & 66.16\%& 65.56\%  & 62.02\% & 61.34\%  \\ 
(\rownumber)~CDA~\cite{han2023boosting} & 66.71\%& 65.37\%  & 64.14\% & 62.85\%  \\ 
(\rownumber)~CECCN~\cite{ruan2024color} & 67.96\%& 66.74\%  & 64.59\%& 64.72\% \\ 
(\rownumber)~EmoVIT~\cite{xie2024emovit} & 68.09\%& 67.45\%  & 63.12\% & 61.97\%  \\ 
(\rownumber)~ComLDL~\cite{zhang2022compound} & 68.83\%& 67.28\%  & 65.29\% & 63.12\%  \\ 
(\rownumber)~WSDEN~\cite{li2023weakly} & 69.78\%& 69.61\%  & 67.04\% & 65.49\% \\ 
(\rownumber)~ECWA~\cite{deng2021emotion} & 70.87\%& 69.08\%  & 67.81\% & 66.87\%  \\ 
(\rownumber)~EECon~\cite{yang2023exploiting} & 71.13\%& 68.34\%  & 64.27\%& 63.16\%  \\ 
(\rownumber)~MAM~\cite{zhang2024affective} & 71.44\%  & 70.83\% & 67.18\%  & 65.01\%\\ 
(\rownumber)~TGCA-PVT~\cite{chen2024tgca}   & 73.05\%  & 71.46\% & 69.87\%  & 68.32\% \\ 
(\rownumber)~OEAN~\cite{zhang2024object}   & 73.40\%  & 72.63\% & 70.52\%  & 69.47\% \\ \hline
(\rownumber)~\shortname  & \textbf{79.48\%} & \textbf{79.22\%} & \textbf{74.12\%} & \textbf{73.09\%} \\ \hline
\end{tabular}
}
\vspace{-6mm}
% \end{center}
\end{table}
% 

\subsection{Experiment Setting}
% \subsubsection{Model Setting.}
% 
\textbf{Model Setting:}
For feature representation, we set $k=10$ to select object tags, and adopt clip-vit-base-patch32 as the pre-trained model for unified feature representation.
Moreover, we empirically set $(d_e, d_h, d_k, d_s) = (512, 128, 16, 64)$, and set the classification class $L$ to 8.

% 

\textbf{Training Setting:}
To initialize the model, we set all weights such as $\boldsymbol{W}$ following the truncated normal distribution, and use AdamW optimizer with the learning rate of $1 \times 10^{-4}$.
% warmup scheduler of cosine, warmup steps of 2000.
Furthermore, we set the batch size to 32 and the epoch of the training process to 200.
During the implementation, we utilize \textit{PyTorch} to build our entire model.
% , and our project codes are publicly available at https://github.com/zzmyrep/MESN.
% Our project codes as well as data are all publicly available on GitHub\footnote{https://github.com/zzmyrep/KBCEN}.
% Code is available at \href{https://github.com/zzmyrep/KBCEN}{https://github.com/zzmyrep/KBCEN}.

\textbf{Evaluation Metrics:}
Following~\cite{zhang2024affective, chen2024tgca, zhang2024object}, we adopt \textit{accuracy} and \textit{F1} as our evaluation metrics to measure the performance of different methods for image sentiment analysis. 



\subsection{Experiment Result}
% We compare our model against the following baselines: AlexNet~\cite{krizhevsky2017imagenet}, VGG16~\cite{simonyan2014very}, ResNet101~\cite{he2016deep}, CECCN~\cite{ruan2024color}, EmoVIT~\cite{xie2024emovit}, WSCNet~\cite{yang2018weakly}, ECWA~\cite{deng2021emotion}, EECon~\cite{yang2023exploiting}, MAM~\cite{zhang2024affective} and TGCA-PVT~\cite{chen2024tgca}, and the overall results are summarized in Table~\ref{tab:cap1}.
We compare our model against several baselines, and the overall results are summarized in Table~\ref{tab:cap1}.
We observe that our model achieves the best performance in both accuracy and F1 metrics, significantly outperforming the previous models. 
This superior performance is mainly attributed to our effective utilization of metadata to enhance image sentiment analysis, as well as the exceptional capability of the unified sentiment transformer framework we developed. These results strongly demonstrate that our proposed method can bring encouraging performance for image sentiment analysis.

\setcounter{magicrownumbers}{0} 
\begin{table}[t]
\begin{center}
\caption{Ablation study of~\shortname~on FI dataset.} 
% \vspace{1mm}
\label{tab:cap2}
\resizebox{.9\linewidth}{!}
{
\begin{tabular}{lcc}
  \hline
  \textbf{Model} & \textbf{Accuracy} & \textbf{F1} \\
  \hline
  (\rownumber)~Ours (w/o vision) & 65.72\% & 64.54\% \\
  (\rownumber)~Ours (w/o text description) & 74.05\% & 72.58\% \\
  (\rownumber)~Ours (w/o object tag) & 77.45\% & 76.84\% \\
  (\rownumber)~Ours (w/o scene tag) & 78.47\% & 78.21\% \\
  \hline
  (\rownumber)~Ours (w/o unified embedding) & 76.41\% & 76.23\% \\
  (\rownumber)~Ours (w/o adaptive learning) & 76.83\% & 76.56\% \\
  (\rownumber)~Ours (w/o cross-modal fusion) & 76.85\% & 76.49\% \\
  \hline
  (\rownumber)~Ours  & \textbf{79.48\%} & \textbf{79.22\%} \\
  \hline
\end{tabular}
}
\end{center}
\vspace{-5mm}
\end{table}


\begin{figure}[t]
\centering
% \vspace{-2mm}
\includegraphics[width=0.42\textwidth]{fig/2dvisual-linux4-paper2.pdf}
\caption{Visualization of feature distribution on eight categories before (left) and after (right) model processing.}
% 
\label{fig:visualization}
\vspace{-5mm}
\end{figure}

\subsection{Ablation Performance}
In this subsection, we conduct an ablation study to examine which component is really important for performance improvement. The results are reported in Table~\ref{tab:cap2}.

For information utilization, we observe a significant decline in model performance when visual features are removed. Additionally, the performance of \shortname~decreases when different metadata are removed separately, which means that text description, object tag, and scene tag are all critical for image sentiment analysis.
Recalling the model architecture, we separately remove transformer layers of the unified representation module, the adaptive learning module, and the cross-modal fusion module, replacing them with MLPs of the same parameter scale.
In this way, we can observe varying degrees of decline in model performance, indicating that these modules are indispensable for our model to achieve better performance.

\subsection{Visualization}
% 


% % 开始使用minipage进行左右排列
% \begin{minipage}[t]{0.45\textwidth}  % 子图1宽度为45%
%     \centering
%     \includegraphics[width=\textwidth]{2dvisual.pdf}  % 插入图片
%     \captionof{figure}{Visualization of feature distribution.}  % 使用captionof添加图片标题
%     \label{fig:visualization}
% \end{minipage}


% \begin{figure}[t]
% \centering
% \vspace{-2mm}
% \includegraphics[width=0.45\textwidth]{fig/2dvisual.pdf}
% \caption{Visualization of feature distribution.}
% \label{fig:visualization}
% % \vspace{-4mm}
% \end{figure}

% \begin{figure}[t]
% \centering
% \vspace{-2mm}
% \includegraphics[width=0.45\textwidth]{fig/2dvisual-linux3-paper.pdf}
% \caption{Visualization of feature distribution.}
% \label{fig:visualization}
% % \vspace{-4mm}
% \end{figure}



\begin{figure}[tbp]   
\vspace{-4mm}
  \centering            
  \subfloat[Depth of adaptive learning layers]   
  {
    \label{fig:subfig1}\includegraphics[width=0.22\textwidth]{fig/fig_sensitivity-a5}
  }
  \subfloat[Depth of fusion layers]
  {
    % \label{fig:subfig2}\includegraphics[width=0.22\textwidth]{fig/fig_sensitivity-b2}
    \label{fig:subfig2}\includegraphics[width=0.22\textwidth]{fig/fig_sensitivity-b2-num.pdf}
  }
  \caption{Sensitivity study of \shortname~on different depth. }   
  \label{fig:fig_sensitivity}  
\vspace{-2mm}
\end{figure}

% \begin{figure}[htbp]
% \centerline{\includegraphics{2dvisual.pdf}}
% \caption{Visualization of feature distribution.}
% \label{fig:visualization}
% \end{figure}

% In Fig.~\ref{fig:visualization}, we use t-SNE~\cite{van2008visualizing} to reduce the dimension of data features for visualization, Figure in left represents the metadata features before model processing, the features are obtained by embedding through the CLIP model, and figure in right shows the features of the data after model processing, it can be observed that after the model processing, the data with different label categories fall in different regions in the space, therefore, we can conclude that the Therefore, we can conclude that the model can effectively utilize the information contained in the metadata and use it to guide the model for classification.

In Fig.~\ref{fig:visualization}, we use t-SNE~\cite{van2008visualizing} to reduce the dimension of data features for visualization.
The left figure shows metadata features before being processed by our model (\textit{i.e.}, embedded by CLIP), while the right shows the distribution of features after being processed by our model.
We can observe that after the model processing, data with the same label are closer to each other, while others are farther away.
Therefore, it shows that the model can effectively utilize the information contained in the metadata and use it to guide the classification process.

\subsection{Sensitivity Analysis}
% 
In this subsection, we conduct a sensitivity analysis to figure out the effect of different depth settings of adaptive learning layers and fusion layers. 
% In this subsection, we conduct a sensitivity analysis to figure out the effect of different depth settings on the model. 
% Fig.~\ref{fig:fig_sensitivity} presents the effect of different depth settings of adaptive learning layers and fusion layers. 
Taking Fig.~\ref{fig:fig_sensitivity} (a) as an example, the model performance improves with increasing depth, reaching the best performance at a depth of 4.
% Taking Fig.~\ref{fig:fig_sensitivity} (a) as an example, the performance of \shortname~improves with the increase of depth at first, reaching the best performance at a depth of 4.
When the depth continues to increase, the accuracy decreases to varying degrees.
Similar results can be observed in Fig.~\ref{fig:fig_sensitivity} (b).
Therefore, we set their depths to 4 and 6 respectively to achieve the best results.

% Through our experiments, we can observe that the effect of modifying these hyperparameters on the results of the experiments is very weak, and the surface model is not sensitive to the hyperparameters.


\subsection{Zero-shot Capability}
% 

% (1)~GCH~\cite{2010Analyzing} & 21.78\% & (5)~RA-DLNet~\cite{2020A} & 34.01\% \\ \hline
% (2)~WSCNet~\cite{2019WSCNet}  & 30.25\% & (6)~CECCN~\cite{ruan2024color} & 43.83\% \\ \hline
% (3)~PCNN~\cite{2015Robust} & 31.68\%  & (7)~EmoVIT~\cite{xie2024emovit} & 44.90\% \\ \hline
% (4)~AR~\cite{2018Visual} & 32.67\% & (8)~Ours (Zero-shot) & 47.83\% \\ \hline


\begin{table}[t]
\centering
\caption{Zero-shot capability of \shortname.}
\label{tab:cap3}
\resizebox{1\linewidth}{!}
{
\begin{tabular}{lc|lc}
\hline
\textbf{Model} & \textbf{Accuracy} & \textbf{Model} & \textbf{Accuracy} \\ \hline
(1)~WSCNet~\cite{2019WSCNet}  & 30.25\% & (5)~MAM~\cite{zhang2024affective} & 39.56\%  \\ \hline
(2)~AR~\cite{2018Visual} & 32.67\% & (6)~CECCN~\cite{ruan2024color} & 43.83\% \\ \hline
(3)~RA-DLNet~\cite{2020A} & 34.01\%  & (7)~EmoVIT~\cite{xie2024emovit} & 44.90\% \\ \hline
(4)~CDA~\cite{han2023boosting} & 38.64\% & (8)~Ours (Zero-shot) & 47.83\% \\ \hline
\end{tabular}
}
\vspace{-5mm}
\end{table}

% We use the model trained on the FI dataset to test on the artphoto dataset to verify the model's generalization ability as well as robustness to other distributed datasets.
% We can observe that the MESN model shows strong competitiveness in terms of accuracy when compared to other trained models, which suggests that the model has a good generalization ability in the OOD task.

To validate the model's generalization ability and robustness to other distributed datasets, we directly test the model trained on the FI dataset, without training on Artphoto. 
% As observed in Table 3, compared to other models trained on Artphoto, we achieve highly competitive zero-shot performance, indicating that the model has good generalization ability in out-of-distribution tasks.
From Table~\ref{tab:cap3}, we can observe that compared with other models trained on Artphoto, we achieve competitive zero-shot performance, which shows that the model has good generalization ability in out-of-distribution tasks.


%%%%%%%%%%%%
%  E2E     %
%%%%%%%%%%%%


\section{Conclusion}
In this paper, we introduced Wi-Chat, the first LLM-powered Wi-Fi-based human activity recognition system that integrates the reasoning capabilities of large language models with the sensing potential of wireless signals. Our experimental results on a self-collected Wi-Fi CSI dataset demonstrate the promising potential of LLMs in enabling zero-shot Wi-Fi sensing. These findings suggest a new paradigm for human activity recognition that does not rely on extensive labeled data. We hope future research will build upon this direction, further exploring the applications of LLMs in signal processing domains such as IoT, mobile sensing, and radar-based systems.

\section*{Limitations}
While our work represents the first attempt to leverage LLMs for processing Wi-Fi signals, it is a preliminary study focused on a relatively simple task: Wi-Fi-based human activity recognition. This choice allows us to explore the feasibility of LLMs in wireless sensing but also comes with certain limitations.

Our approach primarily evaluates zero-shot performance, which, while promising, may still lag behind traditional supervised learning methods in highly complex or fine-grained recognition tasks. Besides, our study is limited to a controlled environment with a self-collected dataset, and the generalizability of LLMs to diverse real-world scenarios with varying Wi-Fi conditions, environmental interference, and device heterogeneity remains an open question.

Additionally, we have yet to explore the full potential of LLMs in more advanced Wi-Fi sensing applications, such as fine-grained gesture recognition, occupancy detection, and passive health monitoring. Future work should investigate the scalability of LLM-based approaches, their robustness to domain shifts, and their integration with multimodal sensing techniques in broader IoT applications.


% Bibliography entries for the entire Anthology, followed by custom entries
%\bibliography{anthology,custom}
% Custom bibliography entries only
\bibliography{main}
\newpage
\appendix

\section{Experiment prompts}
\label{sec:prompt}
The prompts used in the LLM experiments are shown in the following Table~\ref{tab:prompts}.

\definecolor{titlecolor}{rgb}{0.9, 0.5, 0.1}
\definecolor{anscolor}{rgb}{0.2, 0.5, 0.8}
\definecolor{labelcolor}{HTML}{48a07e}
\begin{table*}[h]
	\centering
	
 % \vspace{-0.2cm}
	
	\begin{center}
		\begin{tikzpicture}[
				chatbox_inner/.style={rectangle, rounded corners, opacity=0, text opacity=1, font=\sffamily\scriptsize, text width=5in, text height=9pt, inner xsep=6pt, inner ysep=6pt},
				chatbox_prompt_inner/.style={chatbox_inner, align=flush left, xshift=0pt, text height=11pt},
				chatbox_user_inner/.style={chatbox_inner, align=flush left, xshift=0pt},
				chatbox_gpt_inner/.style={chatbox_inner, align=flush left, xshift=0pt},
				chatbox/.style={chatbox_inner, draw=black!25, fill=gray!7, opacity=1, text opacity=0},
				chatbox_prompt/.style={chatbox, align=flush left, fill=gray!1.5, draw=black!30, text height=10pt},
				chatbox_user/.style={chatbox, align=flush left},
				chatbox_gpt/.style={chatbox, align=flush left},
				chatbox2/.style={chatbox_gpt, fill=green!25},
				chatbox3/.style={chatbox_gpt, fill=red!20, draw=black!20},
				chatbox4/.style={chatbox_gpt, fill=yellow!30},
				labelbox/.style={rectangle, rounded corners, draw=black!50, font=\sffamily\scriptsize\bfseries, fill=gray!5, inner sep=3pt},
			]
											
			\node[chatbox_user] (q1) {
				\textbf{System prompt}
				\newline
				\newline
				You are a helpful and precise assistant for segmenting and labeling sentences. We would like to request your help on curating a dataset for entity-level hallucination detection.
				\newline \newline
                We will give you a machine generated biography and a list of checked facts about the biography. Each fact consists of a sentence and a label (True/False). Please do the following process. First, breaking down the biography into words. Second, by referring to the provided list of facts, merging some broken down words in the previous step to form meaningful entities. For example, ``strategic thinking'' should be one entity instead of two. Third, according to the labels in the list of facts, labeling each entity as True or False. Specifically, for facts that share a similar sentence structure (\eg, \textit{``He was born on Mach 9, 1941.''} (\texttt{True}) and \textit{``He was born in Ramos Mejia.''} (\texttt{False})), please first assign labels to entities that differ across atomic facts. For example, first labeling ``Mach 9, 1941'' (\texttt{True}) and ``Ramos Mejia'' (\texttt{False}) in the above case. For those entities that are the same across atomic facts (\eg, ``was born'') or are neutral (\eg, ``he,'' ``in,'' and ``on''), please label them as \texttt{True}. For the cases that there is no atomic fact that shares the same sentence structure, please identify the most informative entities in the sentence and label them with the same label as the atomic fact while treating the rest of the entities as \texttt{True}. In the end, output the entities and labels in the following format:
                \begin{itemize}[nosep]
                    \item Entity 1 (Label 1)
                    \item Entity 2 (Label 2)
                    \item ...
                    \item Entity N (Label N)
                \end{itemize}
                % \newline \newline
                Here are two examples:
                \newline\newline
                \textbf{[Example 1]}
                \newline
                [The start of the biography]
                \newline
                \textcolor{titlecolor}{Marianne McAndrew is an American actress and singer, born on November 21, 1942, in Cleveland, Ohio. She began her acting career in the late 1960s, appearing in various television shows and films.}
                \newline
                [The end of the biography]
                \newline \newline
                [The start of the list of checked facts]
                \newline
                \textcolor{anscolor}{[Marianne McAndrew is an American. (False); Marianne McAndrew is an actress. (True); Marianne McAndrew is a singer. (False); Marianne McAndrew was born on November 21, 1942. (False); Marianne McAndrew was born in Cleveland, Ohio. (False); She began her acting career in the late 1960s. (True); She has appeared in various television shows. (True); She has appeared in various films. (True)]}
                \newline
                [The end of the list of checked facts]
                \newline \newline
                [The start of the ideal output]
                \newline
                \textcolor{labelcolor}{[Marianne McAndrew (True); is (True); an (True); American (False); actress (True); and (True); singer (False); , (True); born (True); on (True); November 21, 1942 (False); , (True); in (True); Cleveland, Ohio (False); . (True); She (True); began (True); her (True); acting career (True); in (True); the late 1960s (True); , (True); appearing (True); in (True); various (True); television shows (True); and (True); films (True); . (True)]}
                \newline
                [The end of the ideal output]
				\newline \newline
                \textbf{[Example 2]}
                \newline
                [The start of the biography]
                \newline
                \textcolor{titlecolor}{Doug Sheehan is an American actor who was born on April 27, 1949, in Santa Monica, California. He is best known for his roles in soap operas, including his portrayal of Joe Kelly on ``General Hospital'' and Ben Gibson on ``Knots Landing.''}
                \newline
                [The end of the biography]
                \newline \newline
                [The start of the list of checked facts]
                \newline
                \textcolor{anscolor}{[Doug Sheehan is an American. (True); Doug Sheehan is an actor. (True); Doug Sheehan was born on April 27, 1949. (True); Doug Sheehan was born in Santa Monica, California. (False); He is best known for his roles in soap operas. (True); He portrayed Joe Kelly. (True); Joe Kelly was in General Hospital. (True); General Hospital is a soap opera. (True); He portrayed Ben Gibson. (True); Ben Gibson was in Knots Landing. (True); Knots Landing is a soap opera. (True)]}
                \newline
                [The end of the list of checked facts]
                \newline \newline
                [The start of the ideal output]
                \newline
                \textcolor{labelcolor}{[Doug Sheehan (True); is (True); an (True); American (True); actor (True); who (True); was born (True); on (True); April 27, 1949 (True); in (True); Santa Monica, California (False); . (True); He (True); is (True); best known (True); for (True); his roles in soap operas (True); , (True); including (True); in (True); his portrayal (True); of (True); Joe Kelly (True); on (True); ``General Hospital'' (True); and (True); Ben Gibson (True); on (True); ``Knots Landing.'' (True)]}
                \newline
                [The end of the ideal output]
				\newline \newline
				\textbf{User prompt}
				\newline
				\newline
				[The start of the biography]
				\newline
				\textcolor{magenta}{\texttt{\{BIOGRAPHY\}}}
				\newline
				[The ebd of the biography]
				\newline \newline
				[The start of the list of checked facts]
				\newline
				\textcolor{magenta}{\texttt{\{LIST OF CHECKED FACTS\}}}
				\newline
				[The end of the list of checked facts]
			};
			\node[chatbox_user_inner] (q1_text) at (q1) {
				\textbf{System prompt}
				\newline
				\newline
				You are a helpful and precise assistant for segmenting and labeling sentences. We would like to request your help on curating a dataset for entity-level hallucination detection.
				\newline \newline
                We will give you a machine generated biography and a list of checked facts about the biography. Each fact consists of a sentence and a label (True/False). Please do the following process. First, breaking down the biography into words. Second, by referring to the provided list of facts, merging some broken down words in the previous step to form meaningful entities. For example, ``strategic thinking'' should be one entity instead of two. Third, according to the labels in the list of facts, labeling each entity as True or False. Specifically, for facts that share a similar sentence structure (\eg, \textit{``He was born on Mach 9, 1941.''} (\texttt{True}) and \textit{``He was born in Ramos Mejia.''} (\texttt{False})), please first assign labels to entities that differ across atomic facts. For example, first labeling ``Mach 9, 1941'' (\texttt{True}) and ``Ramos Mejia'' (\texttt{False}) in the above case. For those entities that are the same across atomic facts (\eg, ``was born'') or are neutral (\eg, ``he,'' ``in,'' and ``on''), please label them as \texttt{True}. For the cases that there is no atomic fact that shares the same sentence structure, please identify the most informative entities in the sentence and label them with the same label as the atomic fact while treating the rest of the entities as \texttt{True}. In the end, output the entities and labels in the following format:
                \begin{itemize}[nosep]
                    \item Entity 1 (Label 1)
                    \item Entity 2 (Label 2)
                    \item ...
                    \item Entity N (Label N)
                \end{itemize}
                % \newline \newline
                Here are two examples:
                \newline\newline
                \textbf{[Example 1]}
                \newline
                [The start of the biography]
                \newline
                \textcolor{titlecolor}{Marianne McAndrew is an American actress and singer, born on November 21, 1942, in Cleveland, Ohio. She began her acting career in the late 1960s, appearing in various television shows and films.}
                \newline
                [The end of the biography]
                \newline \newline
                [The start of the list of checked facts]
                \newline
                \textcolor{anscolor}{[Marianne McAndrew is an American. (False); Marianne McAndrew is an actress. (True); Marianne McAndrew is a singer. (False); Marianne McAndrew was born on November 21, 1942. (False); Marianne McAndrew was born in Cleveland, Ohio. (False); She began her acting career in the late 1960s. (True); She has appeared in various television shows. (True); She has appeared in various films. (True)]}
                \newline
                [The end of the list of checked facts]
                \newline \newline
                [The start of the ideal output]
                \newline
                \textcolor{labelcolor}{[Marianne McAndrew (True); is (True); an (True); American (False); actress (True); and (True); singer (False); , (True); born (True); on (True); November 21, 1942 (False); , (True); in (True); Cleveland, Ohio (False); . (True); She (True); began (True); her (True); acting career (True); in (True); the late 1960s (True); , (True); appearing (True); in (True); various (True); television shows (True); and (True); films (True); . (True)]}
                \newline
                [The end of the ideal output]
				\newline \newline
                \textbf{[Example 2]}
                \newline
                [The start of the biography]
                \newline
                \textcolor{titlecolor}{Doug Sheehan is an American actor who was born on April 27, 1949, in Santa Monica, California. He is best known for his roles in soap operas, including his portrayal of Joe Kelly on ``General Hospital'' and Ben Gibson on ``Knots Landing.''}
                \newline
                [The end of the biography]
                \newline \newline
                [The start of the list of checked facts]
                \newline
                \textcolor{anscolor}{[Doug Sheehan is an American. (True); Doug Sheehan is an actor. (True); Doug Sheehan was born on April 27, 1949. (True); Doug Sheehan was born in Santa Monica, California. (False); He is best known for his roles in soap operas. (True); He portrayed Joe Kelly. (True); Joe Kelly was in General Hospital. (True); General Hospital is a soap opera. (True); He portrayed Ben Gibson. (True); Ben Gibson was in Knots Landing. (True); Knots Landing is a soap opera. (True)]}
                \newline
                [The end of the list of checked facts]
                \newline \newline
                [The start of the ideal output]
                \newline
                \textcolor{labelcolor}{[Doug Sheehan (True); is (True); an (True); American (True); actor (True); who (True); was born (True); on (True); April 27, 1949 (True); in (True); Santa Monica, California (False); . (True); He (True); is (True); best known (True); for (True); his roles in soap operas (True); , (True); including (True); in (True); his portrayal (True); of (True); Joe Kelly (True); on (True); ``General Hospital'' (True); and (True); Ben Gibson (True); on (True); ``Knots Landing.'' (True)]}
                \newline
                [The end of the ideal output]
				\newline \newline
				\textbf{User prompt}
				\newline
				\newline
				[The start of the biography]
				\newline
				\textcolor{magenta}{\texttt{\{BIOGRAPHY\}}}
				\newline
				[The ebd of the biography]
				\newline \newline
				[The start of the list of checked facts]
				\newline
				\textcolor{magenta}{\texttt{\{LIST OF CHECKED FACTS\}}}
				\newline
				[The end of the list of checked facts]
			};
		\end{tikzpicture}
        \caption{GPT-4o prompt for labeling hallucinated entities.}\label{tb:gpt-4-prompt}
	\end{center}
\vspace{-0cm}
\end{table*}
% \section{Full Experiment Results}
% \begin{table*}[th]
    \centering
    \small
    \caption{Classification Results}
    \begin{tabular}{lcccc}
        \toprule
        \textbf{Method} & \textbf{Accuracy} & \textbf{Precision} & \textbf{Recall} & \textbf{F1-score} \\
        \midrule
        \multicolumn{5}{c}{\textbf{Zero Shot}} \\
                Zero-shot E-eyes & 0.26 & 0.26 & 0.27 & 0.26 \\
        Zero-shot CARM & 0.24 & 0.24 & 0.24 & 0.24 \\
                Zero-shot SVM & 0.27 & 0.28 & 0.28 & 0.27 \\
        Zero-shot CNN & 0.23 & 0.24 & 0.23 & 0.23 \\
        Zero-shot RNN & 0.26 & 0.26 & 0.26 & 0.26 \\
DeepSeek-0shot & 0.54 & 0.61 & 0.54 & 0.52 \\
DeepSeek-0shot-COT & 0.33 & 0.24 & 0.33 & 0.23 \\
DeepSeek-0shot-Knowledge & 0.45 & 0.46 & 0.45 & 0.44 \\
Gemma2-0shot & 0.35 & 0.22 & 0.38 & 0.27 \\
Gemma2-0shot-COT & 0.36 & 0.22 & 0.36 & 0.27 \\
Gemma2-0shot-Knowledge & 0.32 & 0.18 & 0.34 & 0.20 \\
GPT-4o-mini-0shot & 0.48 & 0.53 & 0.48 & 0.41 \\
GPT-4o-mini-0shot-COT & 0.33 & 0.50 & 0.33 & 0.38 \\
GPT-4o-mini-0shot-Knowledge & 0.49 & 0.31 & 0.49 & 0.36 \\
GPT-4o-0shot & 0.62 & 0.62 & 0.47 & 0.42 \\
GPT-4o-0shot-COT & 0.29 & 0.45 & 0.29 & 0.21 \\
GPT-4o-0shot-Knowledge & 0.44 & 0.52 & 0.44 & 0.39 \\
LLaMA-0shot & 0.32 & 0.25 & 0.32 & 0.24 \\
LLaMA-0shot-COT & 0.12 & 0.25 & 0.12 & 0.09 \\
LLaMA-0shot-Knowledge & 0.32 & 0.25 & 0.32 & 0.28 \\
Mistral-0shot & 0.19 & 0.23 & 0.19 & 0.10 \\
Mistral-0shot-Knowledge & 0.21 & 0.40 & 0.21 & 0.11 \\
        \midrule
        \multicolumn{5}{c}{\textbf{4 Shot}} \\
GPT-4o-mini-4shot & 0.58 & 0.59 & 0.58 & 0.53 \\
GPT-4o-mini-4shot-COT & 0.57 & 0.53 & 0.57 & 0.50 \\
GPT-4o-mini-4shot-Knowledge & 0.56 & 0.51 & 0.56 & 0.47 \\
GPT-4o-4shot & 0.77 & 0.84 & 0.77 & 0.73 \\
GPT-4o-4shot-COT & 0.63 & 0.76 & 0.63 & 0.53 \\
GPT-4o-4shot-Knowledge & 0.72 & 0.82 & 0.71 & 0.66 \\
LLaMA-4shot & 0.29 & 0.24 & 0.29 & 0.21 \\
LLaMA-4shot-COT & 0.20 & 0.30 & 0.20 & 0.13 \\
LLaMA-4shot-Knowledge & 0.15 & 0.23 & 0.13 & 0.13 \\
Mistral-4shot & 0.02 & 0.02 & 0.02 & 0.02 \\
Mistral-4shot-Knowledge & 0.21 & 0.27 & 0.21 & 0.20 \\
        \midrule
        
        \multicolumn{5}{c}{\textbf{Suprevised}} \\
        SVM & 0.94 & 0.92 & 0.91 & 0.91 \\
        CNN & 0.98 & 0.98 & 0.97 & 0.97 \\
        RNN & 0.99 & 0.99 & 0.99 & 0.99 \\
        % \midrule
        % \multicolumn{5}{c}{\textbf{Conventional Wi-Fi-based Human Activity Recognition Systems}} \\
        E-eyes & 1.00 & 1.00 & 1.00 & 1.00 \\
        CARM & 0.98 & 0.98 & 0.98 & 0.98 \\
\midrule
 \multicolumn{5}{c}{\textbf{Vision Models}} \\
           Zero-shot SVM & 0.26 & 0.25 & 0.25 & 0.25 \\
        Zero-shot CNN & 0.26 & 0.25 & 0.26 & 0.26 \\
        Zero-shot RNN & 0.28 & 0.28 & 0.29 & 0.28 \\
        SVM & 0.99 & 0.99 & 0.99 & 0.99 \\
        CNN & 0.98 & 0.99 & 0.98 & 0.98 \\
        RNN & 0.98 & 0.99 & 0.98 & 0.98 \\
GPT-4o-mini-Vision & 0.84 & 0.85 & 0.84 & 0.84 \\
GPT-4o-mini-Vision-COT & 0.90 & 0.91 & 0.90 & 0.90 \\
GPT-4o-Vision & 0.74 & 0.82 & 0.74 & 0.73 \\
GPT-4o-Vision-COT & 0.70 & 0.83 & 0.70 & 0.68 \\
LLaMA-Vision & 0.20 & 0.23 & 0.20 & 0.09 \\
LLaMA-Vision-Knowledge & 0.22 & 0.05 & 0.22 & 0.08 \\

        \bottomrule
    \end{tabular}
    \label{full}
\end{table*}




\end{document}


\section{Experiments}
\label{sec:experiments}
The experiments are designed to address two key research questions.
First, \textbf{RQ1} evaluates whether the average $L_2$-norm of the counterfactual perturbation vectors ($\overline{||\perturb||}$) decreases as the model overfits the data, thereby providing further empirical validation for our hypothesis.
Second, \textbf{RQ2} evaluates the ability of the proposed counterfactual regularized loss, as defined in (\ref{eq:regularized_loss2}), to mitigate overfitting when compared to existing regularization techniques.

% The experiments are designed to address three key research questions. First, \textbf{RQ1} investigates whether the mean perturbation vector norm decreases as the model overfits the data, aiming to further validate our intuition. Second, \textbf{RQ2} explores whether the mean perturbation vector norm can be effectively leveraged as a regularization term during training, offering insights into its potential role in mitigating overfitting. Finally, \textbf{RQ3} examines whether our counterfactual regularizer enables the model to achieve superior performance compared to existing regularization methods, thus highlighting its practical advantage.

\subsection{Experimental Setup}
\textbf{\textit{Datasets, Models, and Tasks.}}
The experiments are conducted on three datasets: \textit{Water Potability}~\cite{kadiwal2020waterpotability}, \textit{Phomene}~\cite{phomene}, and \textit{CIFAR-10}~\cite{krizhevsky2009learning}. For \textit{Water Potability} and \textit{Phomene}, we randomly select $80\%$ of the samples for the training set, and the remaining $20\%$ for the test set, \textit{CIFAR-10} comes already split. Furthermore, we consider the following models: Logistic Regression, Multi-Layer Perceptron (MLP) with 100 and 30 neurons on each hidden layer, and PreactResNet-18~\cite{he2016cvecvv} as a Convolutional Neural Network (CNN) architecture.
We focus on binary classification tasks and leave the extension to multiclass scenarios for future work. However, for datasets that are inherently multiclass, we transform the problem into a binary classification task by selecting two classes, aligning with our assumption.

\smallskip
\noindent\textbf{\textit{Evaluation Measures.}} To characterize the degree of overfitting, we use the test loss, as it serves as a reliable indicator of the model's generalization capability to unseen data. Additionally, we evaluate the predictive performance of each model using the test accuracy.

\smallskip
\noindent\textbf{\textit{Baselines.}} We compare CF-Reg with the following regularization techniques: L1 (``Lasso''), L2 (``Ridge''), and Dropout.

\smallskip
\noindent\textbf{\textit{Configurations.}}
For each model, we adopt specific configurations as follows.
\begin{itemize}
\item \textit{Logistic Regression:} To induce overfitting in the model, we artificially increase the dimensionality of the data beyond the number of training samples by applying a polynomial feature expansion. This approach ensures that the model has enough capacity to overfit the training data, allowing us to analyze the impact of our counterfactual regularizer. The degree of the polynomial is chosen as the smallest degree that makes the number of features greater than the number of data.
\item \textit{Neural Networks (MLP and CNN):} To take advantage of the closed-form solution for computing the optimal perturbation vector as defined in (\ref{eq:opt-delta}), we use a local linear approximation of the neural network models. Hence, given an instance $\inst_i$, we consider the (optimal) counterfactual not with respect to $\model$ but with respect to:
\begin{equation}
\label{eq:taylor}
    \model^{lin}(\inst) = \model(\inst_i) + \nabla_{\inst}\model(\inst_i)(\inst - \inst_i),
\end{equation}
where $\model^{lin}$ represents the first-order Taylor approximation of $\model$ at $\inst_i$.
Note that this step is unnecessary for Logistic Regression, as it is inherently a linear model.
\end{itemize}

\smallskip
\noindent \textbf{\textit{Implementation Details.}} We run all experiments on a machine equipped with an AMD Ryzen 9 7900 12-Core Processor and an NVIDIA GeForce RTX 4090 GPU. Our implementation is based on the PyTorch Lightning framework. We use stochastic gradient descent as the optimizer with a learning rate of $\eta = 0.001$ and no weight decay. We use a batch size of $128$. The training and test steps are conducted for $6000$ epochs on the \textit{Water Potability} and \textit{Phoneme} datasets, while for the \textit{CIFAR-10} dataset, they are performed for $200$ epochs.
Finally, the contribution $w_i^{\varepsilon}$ of each training point $\inst_i$ is uniformly set as $w_i^{\varepsilon} = 1~\forall i\in \{1,\ldots,m\}$.

The source code implementation for our experiments is available at the following GitHub repository: \url{https://anonymous.4open.science/r/COCE-80B4/README.md} 

\subsection{RQ1: Counterfactual Perturbation vs. Overfitting}
To address \textbf{RQ1}, we analyze the relationship between the test loss and the average $L_2$-norm of the counterfactual perturbation vectors ($\overline{||\perturb||}$) over training epochs.

In particular, Figure~\ref{fig:delta_loss_epochs} depicts the evolution of $\overline{||\perturb||}$ alongside the test loss for an MLP trained \textit{without} regularization on the \textit{Water Potability} dataset. 
\begin{figure}[ht]
    \centering
    \includegraphics[width=0.85\linewidth]{img/delta_loss_epochs.png}
    \caption{The average counterfactual perturbation vector $\overline{||\perturb||}$ (left $y$-axis) and the cross-entropy test loss (right $y$-axis) over training epochs ($x$-axis) for an MLP trained on the \textit{Water Potability} dataset \textit{without} regularization.}
    \label{fig:delta_loss_epochs}
\end{figure}

The plot shows a clear trend as the model starts to overfit the data (evidenced by an increase in test loss). 
Notably, $\overline{||\perturb||}$ begins to decrease, which aligns with the hypothesis that the average distance to the optimal counterfactual example gets smaller as the model's decision boundary becomes increasingly adherent to the training data.

It is worth noting that this trend is heavily influenced by the choice of the counterfactual generator model. In particular, the relationship between $\overline{||\perturb||}$ and the degree of overfitting may become even more pronounced when leveraging more accurate counterfactual generators. However, these models often come at the cost of higher computational complexity, and their exploration is left to future work.

Nonetheless, we expect that $\overline{||\perturb||}$ will eventually stabilize at a plateau, as the average $L_2$-norm of the optimal counterfactual perturbations cannot vanish to zero.

% Additionally, the choice of employing the score-based counterfactual explanation framework to generate counterfactuals was driven to promote computational efficiency.

% Future enhancements to the framework may involve adopting models capable of generating more precise counterfactuals. While such approaches may yield to performance improvements, they are likely to come at the cost of increased computational complexity.


\subsection{RQ2: Counterfactual Regularization Performance}
To answer \textbf{RQ2}, we evaluate the effectiveness of the proposed counterfactual regularization (CF-Reg) by comparing its performance against existing baselines: unregularized training loss (No-Reg), L1 regularization (L1-Reg), L2 regularization (L2-Reg), and Dropout.
Specifically, for each model and dataset combination, Table~\ref{tab:regularization_comparison} presents the mean value and standard deviation of test accuracy achieved by each method across 5 random initialization. 

The table illustrates that our regularization technique consistently delivers better results than existing methods across all evaluated scenarios, except for one case -- i.e., Logistic Regression on the \textit{Phomene} dataset. 
However, this setting exhibits an unusual pattern, as the highest model accuracy is achieved without any regularization. Even in this case, CF-Reg still surpasses other regularization baselines.

From the results above, we derive the following key insights. First, CF-Reg proves to be effective across various model types, ranging from simple linear models (Logistic Regression) to deep architectures like MLPs and CNNs, and across diverse datasets, including both tabular and image data. 
Second, CF-Reg's strong performance on the \textit{Water} dataset with Logistic Regression suggests that its benefits may be more pronounced when applied to simpler models. However, the unexpected outcome on the \textit{Phoneme} dataset calls for further investigation into this phenomenon.


\begin{table*}[h!]
    \centering
    \caption{Mean value and standard deviation of test accuracy across 5 random initializations for different model, dataset, and regularization method. The best results are highlighted in \textbf{bold}.}
    \label{tab:regularization_comparison}
    \begin{tabular}{|c|c|c|c|c|c|c|}
        \hline
        \textbf{Model} & \textbf{Dataset} & \textbf{No-Reg} & \textbf{L1-Reg} & \textbf{L2-Reg} & \textbf{Dropout} & \textbf{CF-Reg (ours)} \\ \hline
        Logistic Regression   & \textit{Water}   & $0.6595 \pm 0.0038$   & $0.6729 \pm 0.0056$   & $0.6756 \pm 0.0046$  & N/A    & $\mathbf{0.6918 \pm 0.0036}$                     \\ \hline
        MLP   & \textit{Water}   & $0.6756 \pm 0.0042$   & $0.6790 \pm 0.0058$   & $0.6790 \pm 0.0023$  & $0.6750 \pm 0.0036$    & $\mathbf{0.6802 \pm 0.0046}$                    \\ \hline
%        MLP   & \textit{Adult}   & $0.8404 \pm 0.0010$   & $\mathbf{0.8495 \pm 0.0007}$   & $0.8489 \pm 0.0014$  & $\mathbf{0.8495 \pm 0.0016}$     & $0.8449 \pm 0.0019$                    \\ \hline
        Logistic Regression   & \textit{Phomene}   & $\mathbf{0.8148 \pm 0.0020}$   & $0.8041 \pm 0.0028$   & $0.7835 \pm 0.0176$  & N/A    & $0.8098 \pm 0.0055$                     \\ \hline
        MLP   & \textit{Phomene}   & $0.8677 \pm 0.0033$   & $0.8374 \pm 0.0080$   & $0.8673 \pm 0.0045$  & $0.8672 \pm 0.0042$     & $\mathbf{0.8718 \pm 0.0040}$                    \\ \hline
        CNN   & \textit{CIFAR-10} & $0.6670 \pm 0.0233$   & $0.6229 \pm 0.0850$   & $0.7348 \pm 0.0365$   & N/A    & $\mathbf{0.7427 \pm 0.0571}$                     \\ \hline
    \end{tabular}
\end{table*}

\begin{table*}[htb!]
    \centering
    \caption{Hyperparameter configurations utilized for the generation of Table \ref{tab:regularization_comparison}. For our regularization the hyperparameters are reported as $\mathbf{\alpha/\beta}$.}
    \label{tab:performance_parameters}
    \begin{tabular}{|c|c|c|c|c|c|c|}
        \hline
        \textbf{Model} & \textbf{Dataset} & \textbf{No-Reg} & \textbf{L1-Reg} & \textbf{L2-Reg} & \textbf{Dropout} & \textbf{CF-Reg (ours)} \\ \hline
        Logistic Regression   & \textit{Water}   & N/A   & $0.0093$   & $0.6927$  & N/A    & $0.3791/1.0355$                     \\ \hline
        MLP   & \textit{Water}   & N/A   & $0.0007$   & $0.0022$  & $0.0002$    & $0.2567/1.9775$                    \\ \hline
        Logistic Regression   &
        \textit{Phomene}   & N/A   & $0.0097$   & $0.7979$  & N/A    & $0.0571/1.8516$                     \\ \hline
        MLP   & \textit{Phomene}   & N/A   & $0.0007$   & $4.24\cdot10^{-5}$  & $0.0015$    & $0.0516/2.2700$                    \\ \hline
       % MLP   & \textit{Adult}   & N/A   & $0.0018$   & $0.0018$  & $0.0601$     & $0.0764/2.2068$                    \\ \hline
        CNN   & \textit{CIFAR-10} & N/A   & $0.0050$   & $0.0864$ & N/A    & $0.3018/
        2.1502$                     \\ \hline
    \end{tabular}
\end{table*}

\begin{table*}[htb!]
    \centering
    \caption{Mean value and standard deviation of training time across 5 different runs. The reported time (in seconds) corresponds to the generation of each entry in Table \ref{tab:regularization_comparison}. Times are }
    \label{tab:times}
    \begin{tabular}{|c|c|c|c|c|c|c|}
        \hline
        \textbf{Model} & \textbf{Dataset} & \textbf{No-Reg} & \textbf{L1-Reg} & \textbf{L2-Reg} & \textbf{Dropout} & \textbf{CF-Reg (ours)} \\ \hline
        Logistic Regression   & \textit{Water}   & $222.98 \pm 1.07$   & $239.94 \pm 2.59$   & $241.60 \pm 1.88$  & N/A    & $251.50 \pm 1.93$                     \\ \hline
        MLP   & \textit{Water}   & $225.71 \pm 3.85$   & $250.13 \pm 4.44$   & $255.78 \pm 2.38$  & $237.83 \pm 3.45$    & $266.48 \pm 3.46$                    \\ \hline
        Logistic Regression   & \textit{Phomene}   & $266.39 \pm 0.82$ & $367.52 \pm 6.85$   & $361.69 \pm 4.04$  & N/A   & $310.48 \pm 0.76$                    \\ \hline
        MLP   &
        \textit{Phomene} & $335.62 \pm 1.77$   & $390.86 \pm 2.11$   & $393.96 \pm 1.95$ & $363.51 \pm 5.07$    & $403.14 \pm 1.92$                     \\ \hline
       % MLP   & \textit{Adult}   & N/A   & $0.0018$   & $0.0018$  & $0.0601$     & $0.0764/2.2068$                    \\ \hline
        CNN   & \textit{CIFAR-10} & $370.09 \pm 0.18$   & $395.71 \pm 0.55$   & $401.38 \pm 0.16$ & N/A    & $1287.8 \pm 0.26$                     \\ \hline
    \end{tabular}
\end{table*}

\subsection{Feasibility of our Method}
A crucial requirement for any regularization technique is that it should impose minimal impact on the overall training process.
In this respect, CF-Reg introduces an overhead that depends on the time required to find the optimal counterfactual example for each training instance. 
As such, the more sophisticated the counterfactual generator model probed during training the higher would be the time required. However, a more advanced counterfactual generator might provide a more effective regularization. We discuss this trade-off in more details in Section~\ref{sec:discussion}.

Table~\ref{tab:times} presents the average training time ($\pm$ standard deviation) for each model and dataset combination listed in Table~\ref{tab:regularization_comparison}.
We can observe that the higher accuracy achieved by CF-Reg using the score-based counterfactual generator comes with only minimal overhead. However, when applied to deep neural networks with many hidden layers, such as \textit{PreactResNet-18}, the forward derivative computation required for the linearization of the network introduces a more noticeable computational cost, explaining the longer training times in the table.

\subsection{Hyperparameter Sensitivity Analysis}
The proposed counterfactual regularization technique relies on two key hyperparameters: $\alpha$ and $\beta$. The former is intrinsic to the loss formulation defined in (\ref{eq:cf-train}), while the latter is closely tied to the choice of the score-based counterfactual explanation method used.

Figure~\ref{fig:test_alpha_beta} illustrates how the test accuracy of an MLP trained on the \textit{Water Potability} dataset changes for different combinations of $\alpha$ and $\beta$.

\begin{figure}[ht]
    \centering
    \includegraphics[width=0.85\linewidth]{img/test_acc_alpha_beta.png}
    \caption{The test accuracy of an MLP trained on the \textit{Water Potability} dataset, evaluated while varying the weight of our counterfactual regularizer ($\alpha$) for different values of $\beta$.}
    \label{fig:test_alpha_beta}
\end{figure}

We observe that, for a fixed $\beta$, increasing the weight of our counterfactual regularizer ($\alpha$) can slightly improve test accuracy until a sudden drop is noticed for $\alpha > 0.1$.
This behavior was expected, as the impact of our penalty, like any regularization term, can be disruptive if not properly controlled.

Moreover, this finding further demonstrates that our regularization method, CF-Reg, is inherently data-driven. Therefore, it requires specific fine-tuning based on the combination of the model and dataset at hand.

\section{Conclusion}
In this work, we propose a simple yet effective approach, called SMILE, for graph few-shot learning with fewer tasks. Specifically, we introduce a novel dual-level mixup strategy, including within-task and across-task mixup, for enriching the diversity of nodes within each task and the diversity of tasks. Also, we incorporate the degree-based prior information to learn expressive node embeddings. Theoretically, we prove that SMILE effectively enhances the model's generalization performance. Empirically, we conduct extensive experiments on multiple benchmarks and the results suggest that SMILE significantly outperforms other baselines, including both in-domain and cross-domain few-shot settings.

% \section*{Impact statement}
% This paper presents work whose goal is to advance the field of Machine Learning. 
% There are many potential societal consequences of our work, none which we feel must be specifically highlighted here.

\section*{Reproducibility Statement}
In order to ensure reproducibility we will release the code at \href{https://github.com/abenechehab/AdaPTS}{https://github.com/abenechehab/AdaPTS}, once the paper has been accepted. The implementation details and hyperparameters are listed in \cref{appendix:implem}.



\bibliography{main}
\bibliographystyle{main}


%%%%%%%%%%%%%%%%%%%%%%%%%%%%%%%%%%%%%%%%%%%%%%%%%%%%%%%%%%%%%%%%%%%%%%%%%%%%%%%
%%%%%%%%%%%%%%%%%%%%%%%%%%%%%%%%%%%%%%%%%%%%%%%%%%%%%%%%%%%%%%%%%%%%%%%%%%%%%%%
% APPENDIX
%%%%%%%%%%%%%%%%%%%%%%%%%%%%%%%%%%%%%%%%%%%%%%%%%%%%%%%%%%%%%%%%%%%%%%%%%%%%%%%
%%%%%%%%%%%%%%%%%%%%%%%%%%%%%%%%%%%%%%%%%%%%%%%%%%%%%%%%%%%%%%%%%%%%%%%%%%%%%%%
\newpage
\appendix
\onecolumn

\textbf{\LARGE Appendix}

\paragraph{Outline.} In~\cref{appendix:theory}, we provide the proofs and a discussion on \cref{prop:solution} and \cref{prop:vae}. We then provide a perspective on Normalizing Flows as adapters in \cref{appendix:flows}. The experimental setup is presented in \cref{appendix:exp_setup}, including all the implementation details in \cref{appendix:implem}. Finally we showcase some additional results in \cref{appendix:results}.  

% Make appear only appendix sections in table of content
\addtocontents{toc}{\protect\setcounter{tocdepth}{2}}

% Change title of table of contents
\renewcommand*\contentsname{\Large Table of Contents}

%\setstretch{1.5} % stretch for table of contents
\tableofcontents
%\noindent\hrulefill
%\setstretch{1} % unstretch for the rest
\clearpage


\documentclass{article}

% Recommended, but optional, packages for figures and better typesetting:
\usepackage{graphicx}
\usepackage{subfigure}
\usepackage{booktabs} % for professional tables
\usepackage[a4paper,top=3cm,bottom=2cm,left=2.5cm,right=2.5cm,marginparwidth=1.75cm]{geometry}
\usepackage{amsmath, amssymb, natbib, graphicx, url,dsfont,datetime,cases,mathtools,amsthm}
\usepackage{thmtools,thm-restate}
\usepackage{hyperref}
\usepackage{authblk}


\usepackage{amsmath}
\usepackage{amssymb}
\usepackage{mathtools}
\usepackage{amsthm}
\usepackage{enumitem}
\usepackage{macros}
\usepackage{tikz}
\usepackage{booktabs}
\usepackage{subfigure,algorithm2e,algorithmic}

% if you use cleveref..
\usepackage[capitalize,noabbrev]{cleveref}

%%%%%%%%%%%%%%%%%%%%%%%%%%%%%%%%
% THEOREMS
%%%%%%%%%%%%%%%%%%%%%%%%%%%%%%%%
\theoremstyle{plain}
\newtheorem{theorem}{Theorem}[section]
\newtheorem{proposition}[theorem]{Proposition}
\newtheorem{lemma}[theorem]{Lemma}
\newtheorem{corollary}[theorem]{Corollary}
\theoremstyle{definition}
\newtheorem{definition}[theorem]{Definition}
\newtheorem{assumption}[theorem]{Assumption}
\theoremstyle{remark}
\newtheorem{remark}[theorem]{Remark}

\title{The Batch Complexity of Bandit Pure Exploration}
\date{}
\author[1]{Adrienne Tuynman}
\author[1]{Rémy Degenne}
\affil[1]{Univ. Lille, Inria, CNRS, Centrale Lille, UMR 9189-CRIStAL, F-59000 Lille, France}
\begin{document}
	\maketitle

\begin{abstract}
In a fixed-confidence pure exploration problem in stochastic multi-armed bandits, an algorithm iteratively samples arms and should stop as early as possible and return the correct answer to a query about the arms distributions.
We are interested in batched methods, which change their sampling behaviour only a few times, between batches of observations.
We give an instance-dependent lower bound on the number of batches used by any sample efficient algorithm for any pure exploration task.
We then give a general batched algorithm and prove upper bounds on its expected sample complexity and batch complexity.
We illustrate both lower and upper bounds on best-arm identification and thresholding bandits.
\end{abstract}
\section{Introduction}
\label{sec:introduction}
The business processes of organizations are experiencing ever-increasing complexity due to the large amount of data, high number of users, and high-tech devices involved \cite{martin2021pmopportunitieschallenges, beerepoot2023biggestbpmproblems}. This complexity may cause business processes to deviate from normal control flow due to unforeseen and disruptive anomalies \cite{adams2023proceddsriftdetection}. These control-flow anomalies manifest as unknown, skipped, and wrongly-ordered activities in the traces of event logs monitored from the execution of business processes \cite{ko2023adsystematicreview}. For the sake of clarity, let us consider an illustrative example of such anomalies. Figure \ref{FP_ANOMALIES} shows a so-called event log footprint, which captures the control flow relations of four activities of a hypothetical event log. In particular, this footprint captures the control-flow relations between activities \texttt{a}, \texttt{b}, \texttt{c} and \texttt{d}. These are the causal ($\rightarrow$) relation, concurrent ($\parallel$) relation, and other ($\#$) relations such as exclusivity or non-local dependency \cite{aalst2022pmhandbook}. In addition, on the right are six traces, of which five exhibit skipped, wrongly-ordered and unknown control-flow anomalies. For example, $\langle$\texttt{a b d}$\rangle$ has a skipped activity, which is \texttt{c}. Because of this skipped activity, the control-flow relation \texttt{b}$\,\#\,$\texttt{d} is violated, since \texttt{d} directly follows \texttt{b} in the anomalous trace.
\begin{figure}[!t]
\centering
\includegraphics[width=0.9\columnwidth]{images/FP_ANOMALIES.png}
\caption{An example event log footprint with six traces, of which five exhibit control-flow anomalies.}
\label{FP_ANOMALIES}
\end{figure}

\subsection{Control-flow anomaly detection}
Control-flow anomaly detection techniques aim to characterize the normal control flow from event logs and verify whether these deviations occur in new event logs \cite{ko2023adsystematicreview}. To develop control-flow anomaly detection techniques, \revision{process mining} has seen widespread adoption owing to process discovery and \revision{conformance checking}. On the one hand, process discovery is a set of algorithms that encode control-flow relations as a set of model elements and constraints according to a given modeling formalism \cite{aalst2022pmhandbook}; hereafter, we refer to the Petri net, a widespread modeling formalism. On the other hand, \revision{conformance checking} is an explainable set of algorithms that allows linking any deviations with the reference Petri net and providing the fitness measure, namely a measure of how much the Petri net fits the new event log \cite{aalst2022pmhandbook}. Many control-flow anomaly detection techniques based on \revision{conformance checking} (hereafter, \revision{conformance checking}-based techniques) use the fitness measure to determine whether an event log is anomalous \cite{bezerra2009pmad, bezerra2013adlogspais, myers2018icsadpm, pecchia2020applicationfailuresanalysispm}. 

The scientific literature also includes many \revision{conformance checking}-independent techniques for control-flow anomaly detection that combine specific types of trace encodings with machine/deep learning \cite{ko2023adsystematicreview, tavares2023pmtraceencoding}. Whereas these techniques are very effective, their explainability is challenging due to both the type of trace encoding employed and the machine/deep learning model used \cite{rawal2022trustworthyaiadvances,li2023explainablead}. Hence, in the following, we focus on the shortcomings of \revision{conformance checking}-based techniques to investigate whether it is possible to support the development of competitive control-flow anomaly detection techniques while maintaining the explainable nature of \revision{conformance checking}.
\begin{figure}[!t]
\centering
\includegraphics[width=\columnwidth]{images/HIGH_LEVEL_VIEW.png}
\caption{A high-level view of the proposed framework for combining \revision{process mining}-based feature extraction with dimensionality reduction for control-flow anomaly detection.}
\label{HIGH_LEVEL_VIEW}
\end{figure}

\subsection{Shortcomings of \revision{conformance checking}-based techniques}
Unfortunately, the detection effectiveness of \revision{conformance checking}-based techniques is affected by noisy data and low-quality Petri nets, which may be due to human errors in the modeling process or representational bias of process discovery algorithms \cite{bezerra2013adlogspais, pecchia2020applicationfailuresanalysispm, aalst2016pm}. Specifically, on the one hand, noisy data may introduce infrequent and deceptive control-flow relations that may result in inconsistent fitness measures, whereas, on the other hand, checking event logs against a low-quality Petri net could lead to an unreliable distribution of fitness measures. Nonetheless, such Petri nets can still be used as references to obtain insightful information for \revision{process mining}-based feature extraction, supporting the development of competitive and explainable \revision{conformance checking}-based techniques for control-flow anomaly detection despite the problems above. For example, a few works outline that token-based \revision{conformance checking} can be used for \revision{process mining}-based feature extraction to build tabular data and develop effective \revision{conformance checking}-based techniques for control-flow anomaly detection \cite{singh2022lapmsh, debenedictis2023dtadiiot}. However, to the best of our knowledge, the scientific literature lacks a structured proposal for \revision{process mining}-based feature extraction using the state-of-the-art \revision{conformance checking} variant, namely alignment-based \revision{conformance checking}.

\subsection{Contributions}
We propose a novel \revision{process mining}-based feature extraction approach with alignment-based \revision{conformance checking}. This variant aligns the deviating control flow with a reference Petri net; the resulting alignment can be inspected to extract additional statistics such as the number of times a given activity caused mismatches \cite{aalst2022pmhandbook}. We integrate this approach into a flexible and explainable framework for developing techniques for control-flow anomaly detection. The framework combines \revision{process mining}-based feature extraction and dimensionality reduction to handle high-dimensional feature sets, achieve detection effectiveness, and support explainability. Notably, in addition to our proposed \revision{process mining}-based feature extraction approach, the framework allows employing other approaches, enabling a fair comparison of multiple \revision{conformance checking}-based and \revision{conformance checking}-independent techniques for control-flow anomaly detection. Figure \ref{HIGH_LEVEL_VIEW} shows a high-level view of the framework. Business processes are monitored, and event logs obtained from the database of information systems. Subsequently, \revision{process mining}-based feature extraction is applied to these event logs and tabular data input to dimensionality reduction to identify control-flow anomalies. We apply several \revision{conformance checking}-based and \revision{conformance checking}-independent framework techniques to publicly available datasets, simulated data of a case study from railways, and real-world data of a case study from healthcare. We show that the framework techniques implementing our approach outperform the baseline \revision{conformance checking}-based techniques while maintaining the explainable nature of \revision{conformance checking}.

In summary, the contributions of this paper are as follows.
\begin{itemize}
    \item{
        A novel \revision{process mining}-based feature extraction approach to support the development of competitive and explainable \revision{conformance checking}-based techniques for control-flow anomaly detection.
    }
    \item{
        A flexible and explainable framework for developing techniques for control-flow anomaly detection using \revision{process mining}-based feature extraction and dimensionality reduction.
    }
    \item{
        Application to synthetic and real-world datasets of several \revision{conformance checking}-based and \revision{conformance checking}-independent framework techniques, evaluating their detection effectiveness and explainability.
    }
\end{itemize}

The rest of the paper is organized as follows.
\begin{itemize}
    \item Section \ref{sec:related_work} reviews the existing techniques for control-flow anomaly detection, categorizing them into \revision{conformance checking}-based and \revision{conformance checking}-independent techniques.
    \item Section \ref{sec:abccfe} provides the preliminaries of \revision{process mining} to establish the notation used throughout the paper, and delves into the details of the proposed \revision{process mining}-based feature extraction approach with alignment-based \revision{conformance checking}.
    \item Section \ref{sec:framework} describes the framework for developing \revision{conformance checking}-based and \revision{conformance checking}-independent techniques for control-flow anomaly detection that combine \revision{process mining}-based feature extraction and dimensionality reduction.
    \item Section \ref{sec:evaluation} presents the experiments conducted with multiple framework and baseline techniques using data from publicly available datasets and case studies.
    \item Section \ref{sec:conclusions} draws the conclusions and presents future work.
\end{itemize}
\section{Lower Bound}
This section presents our main lower bound. As stated above, the lower bound is nearly tight, apart from lower-order terms and the dependency on $\eps$.

\begin{theorem}
\label{thm:lb}
There exists a distribution $\calP$ and a loss function $f$  satisfying Assumption~\ref{assum:lispchitz_smooth} and Assumption~\ref{assump:dia_dominant}, such that for any $(\epsilon,\delta)$-User-level-DP algorithm $\calM$, given i.i.d. dataset $\calD$ drawn from $\calP$, the output of $\calM$ satisfies
\begin{align*}
    \E[F(\calM(\calD))-F(x^*)]\ge GD\cdot \Tilde{\Omega}\Big(\min\Big\{d,\frac{d}{\sqrt{mn}}+\frac{d^{3/2}}{n\epsilon\sqrt{m}}\Big\}\Big).
\end{align*}
%where $F(x):=\E_{z\sim \calP}f(x;z)$ and $x^*=\arg\min_{x\in\calX}f(x)$.
\end{theorem}

The non-private term $GD\frac{d}{\sqrt{mn}}$ represents the information-theoretic lower bound for SCO under these assumptions (see, e.g., Theorem 1 in \cite{agarwal2009information}).  


We construct the hard instance as follows:
let $\calX=[-1,1]^d$ be unit $\ell_\infty$-ball and let $f(x;z)=-\langle x,z\rangle$ for any $x\in \calX$ be the linear function.
Let $z\in[-\sqrt{m},\sqrt{m}]^d$  with $\E_{z\sim\calP}[z]=\mu$.
Then one can easily verify that $f$ satisfies Assumptions~\ref{assum:lispchitz_smooth} and~\ref{assump:dia_dominant} with $G=\sqrt{m},D=1$ and $\beta=0$.
We have
\begin{align}
    F(\calM(\calD))-F(x^*) &= \sum_{i=1}^{d} (\sign(\mu[i])-\calM(\calD)[i])\cdot\mu[i]
   \nonumber \\ &\ge \sum_{i=1}^{d}  |\mu[i]|.\ind\big(\sign(\mu[i])\neq\sign(\calM(\calD)[i])\big). \label{eq:opt_error_to_sign_error}
\end{align}

By~\eqref{eq:opt_error_to_sign_error}, we reduce the optimization error to the weighted sign estimation error.  
Most existing lower bounds rely on the $\ell^2_2$-error of mean estimation.  
We adapt their techniques, especially the fingerprinting lemma, and provide the proof in the Appendix~\ref{sec:lbproof}.
\begin{algorithm}[ht!]
\caption{\textit{NovelSelect}}
\label{alg:novelselect}
\begin{algorithmic}[1]
\State \textbf{Input:} Data pool $\mathcal{X}^{all}$, data budget $n$
\State Initialize an empty dataset, $\mathcal{X} \gets \emptyset$
\While{$|\mathcal{X}| < n$}
    \State $x^{new} \gets \arg\max_{x \in \mathcal{X}^{all}} v(x)$
    \State $\mathcal{X} \gets \mathcal{X} \cup \{x^{new}\}$
    \State $\mathcal{X}^{all} \gets \mathcal{X}^{all} \setminus \{x^{new}\}$
\EndWhile
\State \textbf{return} $\mathcal{X}$
\end{algorithmic}
\end{algorithm}

% !TeX root = ../all.tex
\begin{figure*}[!ht]
	\centering
	\subfigure[Number of samples before stopping in a random BAI instance, logarithmic scale]{\includegraphics[width=0.3\textwidth]{plot_samp.png}\label{fig:exp}
	}\hspace{1em}
	\subfigure[Number of rounds before stopping in a random BAI instance]{\includegraphics[width=0.3\textwidth]{plot_round.png}
		\label{fig:expr}} \hspace{1em}
	\subfigure[Number of samples before stopping in the min. threshold setting, hard instance]{\includegraphics[width=0.3\textwidth]{TaSbad.png}
		
		\label{fig:exptas}}\label{fig:experiments}\caption{Experimental results, $\delta=0.05$, $N=1000$ runs}\end{figure*}
	
\subsection{Experiments on the BAI setting}


	
	
	

Our algorithm PET is near-optimal in round and sample complexities for many pure exploration problems, and has theoretical guarantees for any pure exploration problem. To ascertain its practical performances, we compare it to baselines and state of the art algorithms for best arm identification and thresholding bandits.	

Each experiment is repeated over 1000 runs. All reward distributions are Gaussian with variance 1 and we use the confidence level $\delta = 0.05$, which is chosen for its relevance to statistical practice. We compare
\begin{itemize}[noitemsep]
	\item Round Robin (or uniform sampling), where the stopping rule is checked only at timesteps $(900\times2^r)_{r \ge 1}$;
	\item Track-and-Stop (TaS) \citep{garivierOptimalBestArm2016}, where the empirical value of $w$ is updated only at timesteps $(900\times 2^r)_r$, and the stopping rule is only checked at those times;
	\item Our algorithm PET, with $T_0 = 1$;
	\item Opt-BBAI \citep{jinOptimalBatchedBest2023} with $\alpha = 1.05$ and the quantities described in their Theorem 4.2.
\end{itemize}
The initial batch sizes for TaS and Round Robin were chosen to approximate the initial batch size of our algorithm, to not disadvantage them in terms of round complexity. We modified TaS in order to turn it into a batch algorithm. Note that there is no formal guarantee for the batch or sample complexity of that modification of TaS, but we use it as a sensible baseline. 

For the BAI experiment, we run each algorithm on $10$-arm instances where the best arm has mean $1$, and each other arm $i$ has mean uniformly sampled between $0.6$ and $0.9$.
See Figure~\ref{fig:exp} for the box plots of the sample complexities. The mean is indicated by a black cross.
While both our algorithm and Opt-BBAI use similarly few batches, PET outperforms Opt-BBAI for the sample complexity.
That algorithm is asymptotically optimal as $\delta\rightarrow 0$ but it uses batches that seem to be too large for moderate values of $\delta$ like the $0.05$ we use.

While the batch modification of TaS might seem to be a good alternative for the BAI experiment, there are instances of the thresholding setting where it performs sub-optimally.
That effect that was first observed in \citep{degenneNonAsymptoticPureExploration2019} for the fully online TaS and reflects that, contrary to our results, the sample complexity guarantees of TaS are only asymptotic. 
We run the algorithms on a thresholding bandit with threshold 0.6 and two arms with means 0.5 and 0.6 and observe that batched TaS has high average sample complexity (see Figure~\ref{fig:exptas}; the mean is the black cross), while PET does not.




\section{Conclusion}
In this work, we propose a simple yet effective approach, called SMILE, for graph few-shot learning with fewer tasks. Specifically, we introduce a novel dual-level mixup strategy, including within-task and across-task mixup, for enriching the diversity of nodes within each task and the diversity of tasks. Also, we incorporate the degree-based prior information to learn expressive node embeddings. Theoretically, we prove that SMILE effectively enhances the model's generalization performance. Empirically, we conduct extensive experiments on multiple benchmarks and the results suggest that SMILE significantly outperforms other baselines, including both in-domain and cross-domain few-shot settings.

\section*{Acknowledgements}
	The authors acknowledge the funding of the French National Research Agency under the project FATE (ANR22-CE23-0016-01) and the PEPR IA FOUNDRY project (ANR-23-PEIA-0003). 
	The authors are members of the Inria team Scool.


\bibliographystyle{apalike}
\bibliography{bibli}


%%%%%%%%%%%%%%%%%%%%%%%%%%%%%%%%%%%%%%%%%%%%%%%%%%%%%%%%%%%%%%%%%%%%%%%%%%%%%%%
%%%%%%%%%%%%%%%%%%%%%%%%%%%%%%%%%%%%%%%%%%%%%%%%%%%%%%%%%%%%%%%%%%%%%%%%%%%%%%%
% APPENDIX
%%%%%%%%%%%%%%%%%%%%%%%%%%%%%%%%%%%%%%%%%%%%%%%%%%%%%%%%%%%%%%%%%%%%%%%%%%%%%%%
%%%%%%%%%%%%%%%%%%%%%%%%%%%%%%%%%%%%%%%%%%%%%%%%%%%%%%%%%%%%%%%%%%%%%%%%%%%%%%%
\newpage
\appendix
\onecolumn
% !TeX root = ../all.tex

\section{Proofs of the lower bounds}\label{app:lb}
\subsection{Preliminary lemmas}

For the sake of completeness, we start by restating and proving some results from \citep{taoCollaborativeLearningLimited2019} in slightly more general language.
\begin{definition}
	For some integers $r$ and $n$, define $\tau_\delta^r$ the number of samples before the end of round $r$.
\end{definition}
\begin{lemma}[Generalization of Lemma 27 of \citep{taoCollaborativeLearningLimited2019}]\label{lem:27f}
	For an algorithm, two instances ${\bm\nu}$ and ${\bm\nu}'$ and $r\in\bN$, \[\bP_{{\bm\nu}'}\{R_\delta\geq r+1,\tau_\delta^{r+1} \leq n+m\}\geq \bP_{\bm\nu}\{R_\delta \geq r+1,\tau_\delta^r\leq m\}-\bP_{\bm\nu} \{\tau_\delta>n\} -\Vert\cD_{\bm\nu}^m -\cD_{{\bm\nu}'}^m\Vert_{TV}\] where $\cD^m_{\bm\nu}$ is the distribution of rewards the algorithm got from $\bm\nu$ over $m$ steps.
\end{lemma}

\begin{proof}
	Fix a deterministic algorithm. 
	
	
	First of all, \begin{equation}\label{eq:mpntomn}(R_\delta \geq r+1,\tau_\delta^{r} \leq m,\tau_\delta^{r+1}-\tau_\delta^r < n) \subseteq (R_\delta \geq r+1,\tau_\delta^{r+1}\leq n+m)\end{equation}
	
	And, since $(R_\delta \geq r+1,\tau_\delta^{r} \leq m,\tau_\delta^{r+1}-\tau_\delta^r < n)$ is determined by the first $m$ rewards (at the end of round $r$ using less than $m$ samples, the algorithm must choose the length of round $r+1$), \begin{equation}\label{eq:distprob} \bP_{{\bm\nu}'}\{R_\delta \geq r+1,\tau_\delta^{r} \leq m,\tau_\delta^{r+1}-\tau_\delta^r < n\} \geq \bP_{\bm\nu} \{R_\delta \geq r+1,\tau_\delta^{r} \leq m,\tau_\delta^{r+1}-\tau_\delta^r < n\}  -\Vert\cD_{\bm\nu}^m -\cD_{{\bm\nu}'}^m\Vert_{TV}\end{equation}

On the other hand, \begin{align*}
	(R_\delta \geq r+1,\tau_\delta^r\leq m)\setminus (R_\delta\geq r+1,\tau_\delta^r\leq m,\tau_\delta^{r+1}-\tau_\delta^r < n) &= (R_\delta\geq r+1,\tau_\delta^r\leq m, \tau_\delta^{r+1}-\tau_\delta^r \geq n) \\
	&\subseteq (\tau_\delta >n)
\end{align*} hence \begin{equation}\label{eq:ajoutround}\bP_{\bm\nu}\{R_\delta\geq r+1,\tau_\delta^r\leq m,\tau_\delta^{r+1}-\tau_\delta^r < n\}\geq \bP_{\bm\nu}\{R_\delta \geq r+1,\tau_\delta^r\leq m\}-\bP_{\bm\nu}\{\tau_\delta >n\}\end{equation}

	Hence, using Equations \eqref{eq:mpntomn},~\eqref{eq:distprob} then~\eqref{eq:ajoutround}, \begin{align*}
		\bP_{{\bm\nu}'}\{R_\delta \geq r+1,\tau_\delta^{r+1}\leq n+m\}&\geq \bP_{\bm\nu'} \{R_\delta \geq r+1,\tau_\delta^{r} \leq m,\tau_\delta^{r+1}-\tau_\delta^r < n\}\\
		&\geq \bP_{\bm\nu} \{R_\delta \geq r+1,\tau_\delta^{r} \leq m,\tau_\delta^{r+1}-\tau_\delta^r < n\} -\Vert\cD_{\bm\nu}^m -\cD_{{\bm\nu}'}^m\Vert_{TV} \\
		&\geq \bP_{\bm\nu}\{R_\delta \geq r+1,\tau_\delta^r\leq m\}-\bP_{\bm\nu} \{\tau_\delta>n\} -\Vert\cD_{\bm\nu}^m -\cD_{{\bm\nu}'}^m\Vert_{TV}
	\end{align*}
	
\end{proof}



\begin{lemma}[Generalization of Lemma 26 of \citep{taoCollaborativeLearningLimited2019}]\label{lem:26f_aux}
	For any $\delta$-correct algorithm, for all $m,r\in\bN$ and any two bandit instances ${\bm\nu}, {\bm\nu}'$, 
	we have
	\begin{align*}
	\bP_{\bm\nu}\{R_\delta\geq r+1,\tau_\delta^r\leq m\}
	\ge \bP_{\bm\nu}\{R_\delta\geq r,\tau_\delta^r\leq m\} - 2\delta - \Vert \cD_{\bm\nu}^m - \cD_{{\bm\nu}'}^m \Vert_{TV}
	\: .
	\end{align*}
\end{lemma}

\begin{proof}
Consider the event $\mathcal{F}_1=(R_\delta=r,\tau_\delta \leq m)$.
Denote by $\mathcal{F}_2$ the event that the algorithm returns the best arm of instance ${\bm\nu}$.
Then \(\bP_{\bm\nu}\{\mathcal{F}_1\}=\bP_{\bm\nu}\{\mathcal{F}_1\wedge \mathcal{F}_2\}+\bP_{\bm\nu}\{\mathcal{F}_1\wedge \overline{\mathcal{F}_2}\}\)
	
With $\mathcal{D}_{\bm\nu}^m$ the distribution of rewards over $m$ samples and some ${\bm\nu}'\in Alt_{\bm\nu}$,
\begin{align*}
\bP_{\bm\nu}\{\mathcal{F}_1\wedge \mathcal{F}_2\}
&\leq \bP_{{\bm\nu}'}\{\mathcal{F}_1\wedge \mathcal{F}_2\}+\Vert\cD_{\bm\nu}^m -\cD_{{\bm\nu}'}^m\Vert_{TV}
\\
&\leq \bP_{{\bm\nu}'}\{\mathcal{F}_2\} +\Vert\cD_{\bm\nu}^m -\cD_{{\bm\nu}'}^m\Vert_{TV}
\\
&\leq \delta+\Vert\cD_{\bm\nu}^m -\cD_{{\bm\nu}'}^m\Vert_{TV}
\: .
\end{align*}
On the other hand, $\mathbb{P}_{\bm\nu}\{\mathcal F_1 \wedge \overline{\mathcal F}_2\} \le \mathbb{P}_{\bm\nu}\{\overline{\mathcal F}_2\} \le \delta$.
Therefore $\bP_{\bm\nu}\{\mathcal{F}_1\}\leq 2\delta + \Vert\cD_{\bm\nu}^m -\cD_{{\bm\nu}'}^m\Vert_{TV}$.
Using \( \bP_{\bm\nu}\{R_\delta\geq r+1,\tau_\delta^r\leq m\}\geq \bP_{\bm\nu}\{R_\delta\geq r,\tau_\delta^r\leq m\}-\bP_{\bm\nu}(\mathcal{F}_1)\), we conclude.
\end{proof}


\begin{lemma}\label{lem:26f}
	For any $\delta$-correct algorithm, for all $m,r\in\bN$ and any bandit instance ${\bm\nu}$, 
	we have
	\begin{align*}
	\bP_{\bm\nu}\{R_\delta\geq r+1,\tau_\delta^r\leq m\}
	\ge \bP_{\bm\nu}\{R_\delta\geq r,\tau_\delta^r\leq m\} - 2\delta - \sqrt{\frac{m}{2} (T^\star(\bm\nu))^{-1}}
	\: .
	\end{align*}
\end{lemma}



\begin{proof}
First apply Lemma~\ref{lem:26f_aux} to an arbitrary instance ${\bm\nu}' \in Alt_{\bm\nu}$. Then using Pinsker's inequality yields
\begin{align*}
\Vert \cD_{\bm\nu}^m - \cD_{{\bm\nu}'}^m \Vert_{TV}
\leq \sqrt{\frac{1}{2}\KL(\cD_{\bm\nu}^m\Vert \cD_{{\bm\nu}'}^m)}
= \sqrt{\frac{1}{2} \sum_{i\in[K]} \bE_{\bm\nu}[N_{m,i}] \frac{(\mu_i-\mu_i')^2}{2}}
\end{align*} with $N_{m,i}$ the number of times arm $i$ is pulled before time $m$.

As this is true for all instances ${\bm\nu}'\in Alt_{{\bm\nu}}$, we can obtain an inequality using the infimum over those instances,
\begin{align*}
\inf_{{\bm\nu}' \in Alt_{\bm\nu}} \Vert \cD_{\bm\nu}^m - \cD_{{\bm\nu}'}^m \Vert_{TV}
&\le \sqrt{\frac{m}{2} \inf_{\bm\lambda \in Alt_{{\bm\nu}}} \sum_{i\in[K]} \frac{\bE_{\bm\nu}[N_{m,i}]}{m} \frac{(\mu_i-\lambda_i)^2}{2}}
\\
&\le \sqrt{\frac{m}{2} \sup_{w\in \Sigma_K} \inf_{\bm\lambda \in Alt_{{\bm\nu}}} \sum_{i\in[K]} w_i \frac{(\mu_i-\lambda_i)^2}{2}}
\\
&=   \sqrt{\frac{m}{2} (T^\star(\bm\nu))^{-1}}
\: ,
\end{align*}
by definition of $T^\star$.
\end{proof}





Finally, we also give a technical result to solve inequalities of the form $(k+N^2(a+b\ln N))^N\leq \rho$.

\begin{lemma}\label{lem:suffN}
	Let $\rho \ge e$, $a,b\geq 0$ and $k$ be real numbers, and let $A=\max\{e,k+a\}$.
	Then $N \coloneqq \left\lfloor \frac{\ln \rho}{\ln((\ln \rho)^2(A+b\ln \ln \rho))}\right\rfloor$ satisfies $(k+N^2(a+b\ln N))^N\leq \rho$~.
\end{lemma}

\begin{proof}
	If $N=0$, the equality is $1 \le \rho$, which is true since $\rho \ge e$. Otherwise, $N\geq 1$ and $(\ln \rho)^2(A+b\ln\ln \rho)\geq A\geq e$,
	so $N \le \lfloor \ln\rho / \ln e \rfloor \leq \ln \rho$. Therefore
	\begin{align*}
		N\ln(k+N^2(a+b\ln N))&\leq N\ln (N^2(A+b\ln N))
		\\
		&\leq N\ln((\ln \rho)^2(A+b\ln \ln \rho))
		\\
		&\leq \ln \rho
	\end{align*}
	and finally $(k+N^2(a+b\ln N))^N\leq \rho$~.
\end{proof}

\subsection{The lower bound in the general cases}

We give here a result for any sequence of instances.

\begin{restatable}[]{lemma}{lemrec}\label{lem:rec} 
	Let there be a sequence of instances $({\bm\nu}^n)_{0\leq n\leq N}$ such that the probability of error is bounded by $\delta$ and for any $n\in[0,N-1]$, $c_n \geq \bP_{{\bm\nu}^n} [\tau_\delta >x_n]$.
	Then \begin{align*} \bP_{{\bm\nu}^N}[R_\delta>N] &\geq 1-2N\delta -\sum_{i=0}^{N-1}\left[  c_n+ \sqrt{\frac{X_{n-1}}{2}} \left(\sqrt{ \frac{1}{T^\star(\bm\mu^n)} }  +\sqrt{\sum_{i\in[K]}\frac{\bE[N_{X_{n-1},i}]}{X_{n-1}} \frac{(\mu_i^{n+1}-\mu_i^n)^2}{2\sigma^2}}\right)\right]\end{align*} where $X_n=\sum_{i=-1}^n x_i$, $x_{-1}$ is any positive real number, and $N_{t,i}$ is the number of times arm $i$ is sampled before time $t$.
\end{restatable}

\begin{proof}[Proof of Lemma~\ref{lem:rec}] 
	By lemmas \ref{lem:27f} and \ref{lem:26f}, for any $m$, \begin{align*} 
		\bP_{{\bm\nu}^{n+1}}\{R_\delta\geq n+1,\tau_\delta^{n+1}\leq m+x_n\}  &\geq \bP_{{\bm\nu}^n}\{R_\delta\geq n+1,\tau_\delta^n\leq m\}-c_n-\Vert\cD_{{\bm\nu}^n}^m-\cD_{{\bm\nu}^{n+1}}^m\Vert_{TV}\\
		&\geq \bP_{{\bm\nu}^{n}}\{R_\delta \geq n,\tau_\delta^n\leq m\}-2\delta-\sqrt{\frac{m}{2} (T^\star(\bm\mu^n))^{-1}}\\
		&\hspace{1.5em}-c_n-\sqrt{\frac{1}{2}\sum_{i\in[K]}\bE[N_{m,i}] \frac{(\mu_i^{n+1}-\mu_i^n)^2}{2\sigma^2}}
	\end{align*} and with $X_n=\sum_{i=-1}^{n} x_i$, \begin{align*}\bP_{{\bm\nu}^{n+1}}\{R_\delta\geq n+1,\tau_\delta^{n+1}\leq X_n\} & \geq \bP_{{\bm\nu}^{n}}\{R_\delta\geq n,\tau_\delta^n\leq X_{n-1}\}-2\delta-c_n -\sqrt{\frac{X_{n-1}}{2} (T^\star(\bm\mu))^{-1}}\\
	&\hspace{1.5em} -\sqrt{\frac{X_{n-1}}{2}\sum_{i\in[K]}\frac{\bE[N_{X_{n-1},i}]}{X_{n-1}} \frac{(\mu_i^{n+1}-\mu_i^n)^2}{2\sigma^2}}\end{align*} So that finally \begin{align*}
	\bP_{{\bm\nu}^{N}}\{R_\delta\geq N,\tau_\delta^N\leq X_{N-1}\} &\geq \bP_{{\bm\nu}^{0}}\{R_\delta\geq 0,\tau_\delta^0\leq x_{-1}\}-2N\delta\\
	&\hspace{1.5em} -\sum_{i=0}^{N-1}\left[ c_n+ \sqrt{\frac{X_{n-1}}{2}}\left(\sqrt{ (T^\star(\bm\mu^n))^{-1} } +\sqrt{\sum_{i\in[K]}\frac{\bE[N_{X_{n-1},i}]}{X_{n-1}} \frac{(\mu_i^{n+1}-\mu_i^n)^2}{2\sigma^2}}\right)\right]\end{align*} and we conclude since for any $x_{-1}\geq 0$, $\bP_{{\bm\nu}^0}\{R_\delta\geq 1,\tau_\delta^0\leq x_{-1}\}=1$ (we always use at least 1 round).
\end{proof}

From there, we derive the result for $T^\star(\bm\mu^n)=\zeta^{-n} T^\star(\bm\mu^0)$.

\theorec* 


\begin{proof}[Proof of Lemma~\ref{th:theorec}]
	We apply Lemma~\ref{lem:rec} on the sequence $(\bm\nu^n)_{0\leq n\leq N}$ with $x_{-1}=\gamma T^\star(\bm\mu^0)\log(1/\delta)\frac{1}{\zeta^{-1}-1}$. That way, \begin{align*} X_{n} &= x_{-1} +\sum_{i=0}^n \gamma T^\star(\bm\mu^i)\log(1/\delta)\\
		&=\gamma T^\star(\bm\mu^0)\log(1/\delta)\left( \frac{1}{\zeta^{-1}-1}+\sum_{i=0}^n \zeta^{-i}\right)\\
		&=\gamma T^\star(\bm\mu^0)\log(1/\delta)\frac{\zeta^{-(n+1)}}{\zeta^{-1}-1}
	\end{align*}
	
\end{proof}

Under Assumption~\ref{asm:aff}, we can pick a sequence of instances of means $\bm\mu^{n+1}=x\bm\mu^n+(1-x)\bm y$ and control the sequence of $T^\star(\bm\mu^n)$. That way, we get the following result:
\begin{restatable}[Batch lower bound on affine sequences]{lemma}{lembar}\label{lem:bar}
	For problems on which Assumption~\ref{asm:aff} is satisfied;
	for any algorithm such that, for any Gaussian instance $\bm\nu$ satisfying $T^\star(\bm\mu)\in (T_{\min},T_{\max})$ the probability of error is smaller than $\delta$ and such that $\bP_{\bm\nu}(\tau_\delta>\gamma\log(1/\delta) T^\star(\bm\mu))\leq c$; we have for any $\sigma$-Gaussian instance $\bm\nu$ of complexity $T^\star(\bm\mu)\in (T_{\min},T_{\max})$, for the corresponding $y\in \R$ given by Assumption~\ref{asm:aff} for $\bm\mu$, that $\bP_{\bm\nu} (R_\delta\geq N)\geq 1/2$ for \[N = \min\left\{\frac{\ln \frac{T^\star(\bm\mu)}{T_{\min}}}{\ln\left( \left(\ln \frac{T^\star(\bm\mu)}{T_{\min}}\right)^2 \max\{e,C\} \right)},\frac{1}{2\delta+c}\right\}\] with $C=1+4\gamma\log(\frac{1}{\delta})\left(1+\sqrt{\frac{T^\star(\bm\mu)\Delta^2}{2\sigma^2}}\right)^2$ and $\Delta = \max_i |\mu_i -y|$.
\end{restatable}

\begin{proof}
	Fix some ($\sigma$-Gaussian) instance $\bm\nu^0=\bm\nu$ of complexity $T^\star(\bm\mu)=T_0\in(T_{\min},T_{\max})$. 
	
	For some $\zeta\in (0,1)$ to be fixed later, define the instance of mean $\bm\mu^{n+1}=\zeta^{-1/2}\bm\mu^n+(1-\zeta^{-1/2})\bm y$. We then have $\zeta^nT^\star(\bm\mu^0)= T^\star(\bm\mu^{n})$ by hypothesis. We can thus construct a sequence of instances of length $N$ as long as $\zeta^{N} > \frac{T_{\min}}{T^\star(\bm\mu^0)}$.
	
	\begin{align*} \frac{(\mu_i^{N-n-1}-\mu_i^{N-n})^2}{2\sigma^2} &= \frac{(\zeta^{-(N-n-1)/2}-\zeta^{-(N-n)/2})^2(\mu_i^0-y)^2}{2\sigma^2}\\ &\leq \frac{\zeta^{n-N}\Delta^2}{2\sigma^2} (1-\zeta^{1/2})^2\end{align*}
	We apply Theorem~\ref{th:theorec} on the reversed sequence $\left( \bm\nu_{N-i}\right)_{0\leq i\leq N}$:
	\begin{align*}\bP_{{\bm\nu}}(R_\delta>N) &\geq 1-N(2\delta+c)-\sqrt{\frac{\gamma\log(1/\delta)}{2(\zeta^{-1}-1)}}\times \sum_{i=0}^{N-1} \left[ 1 +\sqrt{T^\star(\bm\mu)\sup_{w\in \Delta_K}\sum_{i\in[K]}w_i \frac{\Delta^2}{2\sigma^2} (1-\zeta^{1/2})^2}\right]\\
		&\geq 1-N\left(2\delta+c+\sqrt{\frac{\gamma\log(1/\delta)}{2(\zeta^{-1}-1)}} \left(1+\sqrt{\frac{T^\star(\bm\mu^0)\Delta^2}{2\sigma^2}}\right)\right)\\
		&\geq 5/8-N(2\delta+c)\end{align*} for \[\zeta =\Bigg( 1+4N^2\gamma\log\left(\frac{1}{\delta}\right)\left(1+\sqrt{\frac{T^\star(\bm\mu^0)\Delta^2}{2\sigma^2}}\right)^2\Bigg)^{-1}\]
	
	
	We can apply Lemma~\ref{lem:suffN} with $\rho = \frac{T^\star(\bm\mu^0)}{T_{\min}}$, $k=1$, $b=0$, and $a=4\gamma\log(1/\delta)\left(1+\sqrt{\frac{T^\star(\bm\mu^0)\Delta^2}{2\sigma^2}}\right)^2$. We get that a sufficient condition is \begin{equation} \label{eq:Nbai} N\leq \frac{\ln \frac{T^\star(\bm\mu^0)}{T_{\min}}}{\ln\left( \left(\ln \frac{T^\star(\bm\mu^0)}{T_{\min}}\right)^2 \max\left\{e,C\right\} \right)}\end{equation} with $C=1+4\gamma\log(1/\delta)\left(1+\sqrt{\frac{T^\star(\bm\mu^0)\Delta^2}{2\sigma^2}}\right)^2$.
	
	Therefore, by picking $N$ that satisfies \eqref{eq:Nbai} and $N\leq \frac{1}{8(2\delta+c)}$, we have that $\bP_{\bm\nu}(R_\delta > N)>\frac{1}{2}$.
\end{proof}




\begin{restatable}[Batch lower bound on affine sequences, expectation constraint]{lemma}{lembarexp}\label{lem:barexp}
	For problems on which Assumption~\ref{asm:aff} is satisfied; 
	for any $\delta$-correct algorithm such that, for any Gaussian instance $\bm\nu$ satisfying $T^\star(\bm\mu)\in (T_{\min},T_{\max})$ $\bE_{\bm\nu}[\tau_\delta]\leq \gamma\log(1/\delta) T^\star(\bm\mu)$, we have for any $\sigma$-Gaussian instance $\bm\nu$ of complexity $T^\star(\bm\mu)\in (T_{\min},T_{\max})$, for the corresponding $y\in\R$ given by Assumption~\ref{asm:aff} for $\bm\mu$ that $\bP_{\bm\nu} (R_\delta\geq N)\geq 1/2$ for \[ N \geq \min\left\{ \frac{\ln \frac{T^\star(\bm\mu)}{T_{\min}}}{\ln\left( \left(\ln \frac{T^\star(\bm\mu)}{T_{\min}}\right)^2 \max\{e,C_\delta'\}  \right)},\frac{1}{3} \ln \frac{T^\star(\bm\mu)}{T_{\min}},\frac{1}{3\delta}\right\}\] with $C_\delta'=\max\left\{e,1+4\gamma\log(1/\delta)\ln \frac{T^\star(\bm\mu)}{T_{\min}}\left(1+\sqrt{\frac{T^\star(\bm\mu)\Delta^2}{2\sigma^2}}\right)^2 \right\}$ and $\Delta = \max_i |\mu_i -y|$.
\end{restatable}

\begin{proof}
For instance $\bm\nu$, for some algorithm satisfying $\bE_{\bm\nu}[\tau_\delta]\leq \gamma \log(1/\delta) T^\star(\bm\mu)$, we have by the Markov inequality that $\bP_{\bm\nu}(\tau_\delta \geq (\gamma/c) T^\star(\bm\mu) \log(1/\delta)) \leq c$.
Applying Lemma~\ref{lem:bar}, $\bP_{\bm\nu} (R_\delta\geq N)\geq 1/2$ for \[N = \min\left\{\frac{\ln \frac{T^\star(\bm\mu)}{T_{\min}}}{\ln\left( \left(\ln \frac{T^\star(\bm\mu)}{T_{\min}}\right)^2 \max\{e,C\} \right)},\frac{1}{2\delta+c}\right\}\] with $C=1+4\gamma/c\log(\frac{1}{\delta})\left(1+\sqrt{\frac{T^\star(\bm\mu)\Delta^2}{2\sigma^2}}\right)^2$ and $\Delta = \max_i |\mu_i -y|$.

Choosing $c = \max\left\{ \delta,\left( \log\frac{T^\star(\bm\mu)}{T_{\min}}\right)^{-1}\right\}$, if $\delta < \left( \log\frac{T^\star(\bm\mu)}{T_{\min}}\right)^{-1}$, then \[ N \geq \min\left\{ \frac{\ln \frac{T^\star(\bm\mu)}{T_{\min}}}{\ln\left( \left(\ln \frac{T^\star(\bm\mu)}{T_{\min}}\right)^2 \max\{e,C_\delta'\}  \right)},\frac{1}{3} \ln \frac{T^\star(\bm\mu)}{T_{\min}}\right\}\] with $C_\delta'=\max\left\{e,1+4\gamma\log(1/\delta)\ln \frac{T^\star(\bm\mu)}{T_{\min}}\left(1+\sqrt{\frac{T^\star(\bm\mu)\Delta^2}{2\sigma^2}}\right)^2 \right\}$.

If $\delta \geq \left( \log\frac{T^\star(\bm\mu)}{T_{\min}}\right)^{-1}$, \[N \geq \min\left\{\frac{\ln \frac{T^\star(\bm\mu)}{T_{\min}}}{\ln\left( \left(\ln \frac{T^\star(\bm\mu)}{T_{\min}}\right)^2 \max\{e,C_\delta'\} \right)},\frac{1}{3\delta}\right\}\]



\end{proof}











\subsection{The top-$k$ and BAI settings}

All that remains is to show that our problems satisfy Assumption~\ref{asm:aff}. We start with top-$k$, and first give a technical result giving a simple formula for $T^\star(\bm\mu)$.

\begin{lemma}\label{lem:baiw}
	For any $\bm w\in \Sigma_K$, \begin{align*}\inf_{\bm\lambda \in Alt_{\bm\mu}} \left( \sum_{i\in [K]} w_i d(\mu_i,\lambda_i)\right) = \min_{\substack{b\geq k+1 \\ a\leq k}} w_a d(\mu_a,\mu_{ab})+w_b d(\mu_b,\mu_{ab})\end{align*} where $\mu_{ab}=\frac{w_a\mu_a +w_b\mu_b}{w_a+w_b}$ (arms are assumed to be ordered, $\mu_1\geq\mu_2\geq \dots$).
\end{lemma}

\begin{lemma}\label{lem:topkgoodbar}
	In the top-$k$ problem, setting $\bm\mu' = x\bm\mu + (1-x)\bm y$ where $\bm y$ is a constant vector and $x>0$, $Alt_{\bm\mu'}=Alt_{\bm\mu}$, $\Delta_i^{\bm\mu'}=x\Delta_i^{\bm\mu}$ and $(T^\star(\bm\mu'))^{-1}=x^2(T^\star(\bm\mu))^{-1}$.
\end{lemma}
\begin{proof}
	First of all, for any two arms $i,j$, $\mu'_i-\mu'_j = x(\mu_i-\mu_j)$ with $x>0$. Therefore, the ordering of arms is conserved, and $Alt_{\bm\mu'}=Alt_{\bm\mu}$. Moreover, since $\Delta_i^{\bm\mu'} = \mu'_i - \mu'_{k+1} =x(\mu_i-\mu_{k+1})=x\Delta_i^{\bm\mu}$ for $i\leq k$ and $\Delta_i^{\bm\mu'} = \mu'_k-\mu'_i=x\Delta_i^{\bm\mu}$ otherwise, we do have $\Delta_i^{\bm\mu'}=x\Delta_i^{\bm\mu}$.
	
	Furthermore,
	\begin{align*}
		(T^\star(\bm\mu'))^{-1}&= \sup_{w\in \Sigma_K}\inf_{\bm\lambda \in Alt_{\bm\mu}} \left( \sum_{i\in [K]} w_i \frac{(\mu_i'-\lambda_i)^2}{2\sigma^2}\right)\\
		&= \sup_{w\in \Sigma_K} \min_{\substack{b\geq k+1 \\ a\leq k}} w_a \frac{(\mu_a'-\mu_{ab}')^2}{2\sigma^2}+w_b \frac{(\mu_b'-\mu_{ab}')^2}{2\sigma^2}
	\end{align*} with \begin{align*}\mu_{ab}'(w)&=\frac{w_a\mu_a' +w_b\mu_b'}{w_a+w_b}=x\mu_{ab}+(1-x)y\end{align*}
	So that \begin{align*}
		(T^\star(\bm\mu'))^{-1}&=x^2\sup_{w\in \Sigma_K}\min_{\substack{b\geq k+1 \\ a\leq k}} w_a \frac{(\mu_a-\mu_{ab})^2}{2\sigma^2}+w_b \frac{(\mu_b-\mu_{ab})^2}{2\sigma^2}\\
		&=x^2(T^\star(\bm\mu))^{-1}
	\end{align*}


\end{proof}

With these results, we can apply Lemmas~\ref{lem:bar} and \ref{lem:barexp}. We see that the value of $y$ does not impact the proof: we thus choose the value that minimizes $\max_i |\mu_i-y|$, which is $y = \frac{\max_i \mu_i+\min_i\mu_i}{2}$.

%\thlbbaib*
	

	
%\begin{proof}\textbf{of Theorem~\ref{th:lbbaib}}
%	Thanks to Lemma~\ref{lem:topkgoodbar}, it suffices to apply Lemma~\ref{lem:bar} with $\nu = \frac{\mu_1+\mu_K}{2}$. We have $\Delta = |\mu_1-\mu_K|/2$.
%\end{proof}	

\subsection{The thresholding setting}

\begin{lemma}\label{lem:tbpgoodbar}
	In the thresholding bandit problem, setting $\bm\mu' = x\bm\mu + (1-x)\bm \tau$ where $\bm \tau$ is the constant vector of value $\tau$ the threshold and $x>0$, $Alt_{\bm\mu'}=Alt_{\bm\mu}$, $\Delta_i^{\bm\mu'}=x\Delta_i^{\bm\mu}$ and $(T^\star(\bm\mu'))^{-1}=x^2(T^\star(\bm\mu))^{-1}$.
\end{lemma}

\begin{proof}
	First of all, for any arm $i$, $\mu'_i -\tau = x(\mu_i-\tau)$ with $x>0$. Therefore, $Alt_{\bm\mu}=Alt_{\bm\mu'}$. Moreover, $\Delta_i^{\bm\mu'}=|\mu'_i-\tau|=x|\mu_i-\tau|=x\Delta_i^{\bm\mu'}$.
	
	Furthermore, \begin{align*}
		(T^\star(\bm\mu'))^{-1}&=\sup_{w\in \Sigma_K} \inf_{\bm\lambda\in Alt_{\bm\mu}} \left( \sum_{i\in[K]} w_i\frac{(\mu'_i-\lambda_i)^2}{2\sigma^2}\right)\\
		&=\sup_{w\in \Sigma_K} \sup_{i\in [K]} w_i \frac{(\mu'_i-\tau)^2}{2\sigma^2}\\
		&=x^2 \sup_{w\in \Sigma_K} \sup_{i\in [K]} w_i \frac{(\mu_i-\tau)^2}{2\sigma^2}\\
		&=x^2(T^\star(\bm\mu))^{-1}
	\end{align*}
\end{proof}



% !TeX root = ../all.tex


\section{Concentration and threshold for the stopping rule}
\label{app:concentration}

We suppose that each arm is sampled once during the first $K$ time steps.
\begin{theorem}
  \label{thm:bound_delta}
  Suppose that the arm distributions are $\sigma^2$-sub-Gaussian. Let $\hat{\mu}_{t,k}$ be the average of arm $k$ at time $t$ and $N_{t,k}$ be the number of times arm $k$ is sampled up to time $t$.
  With probability $1 - \delta$, for all $t > K$,
  \begin{align*}
  \frac{1}{2} \sum_{k=1}^K N_{t,k}\frac{(\hat{\mu}_{t,k}- \mu_k)^2}{2 \sigma^2}
  &\le \frac{K}{2} \overline{W}\left(2\ln \left(\frac{e\pi^2}{6}\right) + \frac{2}{K}\ln \left(\prod_{k=1}^K (1 + \ln N_{t,k})^2\right) + \frac{2}{K}\ln \frac{1}{\delta}\right)
  \: .
  \end{align*}
\end{theorem}



\subsection{Proof of the concentration theorem}

We can assume $w.l.o.g.$ that $\mu_k = 0$ for all $k$ and $\sigma^2 = 1$.

Let $S_{t,k} = \sum_{s=1}^t X_{s,k} \mathbb{I}\{k_s = k\}$.
We want a bound on $\frac{1}{2} \sum_{k=1}^K \frac{S_{t,k}^2}{N_{t,k}}$.


We first remark that $\frac{1}{2}x^2 = \sup_{\lambda} \lambda x - \frac{1}{2}\lambda^2$~. Apply that to $x = S_{t,k}/\sqrt{N_{t,k}}$ to get
\begin{align*}
\sum_{k=1}^K \frac{1}{2}\frac{S_{t,k}^2}{N_{t,k}}
&= \sup_{\lambda_1, \ldots, \lambda_K} \sum_{k=1}^K \left( \lambda_k S_{t,k} - \frac{1}{2}N_{t,k} \lambda_k^2 \right)
\\
&= \sup_{\lambda_1, \ldots, \lambda_K} \sum_{s=1}^t \lambda_{k_s}X_{s,k_s} - \frac{1}{2}\lambda_{k_s}^2
\: .
\end{align*}
The advantage of that formulation is that we can concentrate the sum for any fixed value of $\lambda$ (or any distribution on $\lambda$) thanks to a martingale argument.

\begin{lemma}
For all $\rho \in \mathcal P(\mathbb{R}^K)$, the process $t \mapsto \mathbb{E}_{\lambda \sim \rho}\left[\exp\left(\sum_{s=1}^t \lambda_{k_s}X_{s,k_s} - \frac{1}{2}\lambda_{k_s}^2\right)\right]$ is a non-negative supermartingale with expectation bounded by 1.
\end{lemma}


\begin{corollary}\label{cor:prob_exists_log_ge_le}
For all $\rho \in \mathcal P(\mathbb{R}^K)$ and $x \ge 0$,
\begin{align*}
\mathbb{P}\left(\exists t, \ \ln \mathbb{E}_{\lambda \sim \rho}\left[\exp\left(\sum_{s=1}^t \lambda_{k_s}X_{s,k_s} - \frac{1}{2}\lambda_{k_s}^2\right)\right] \ge x\right) \le e^{-x}
\: .
\end{align*}
Equivalently, for all $\delta \in (0,1]$,
\begin{align*}
\mathbb{P}\left(\exists t, \ \ln \mathbb{E}_{\lambda \sim \rho}\left[\exp\left(\sum_{s=1}^t \lambda_{k_s}X_{s,k_s} - \frac{1}{2}\lambda_{k_s}^2\right)\right] \ge \ln\frac{1}{\delta}\right) \le \delta
\: .
\end{align*}
\end{corollary}

\begin{proof}
Use Ville's inequality and the fact that the process is a non-negative supermartingale.
\end{proof}

We don't want to bound an integral over $\lambda \sim \rho$, but the supremum over $\lambda$, so we need to relate the two quantities.
We do that for Gaussian priors over $\lambda$.

\begin{lemma}\label{lem:integral_exp_eq_log_add}
For $\rho = \mathcal N(0, \mathrm{diag}(\sigma_k^{-2}))$,
\begin{align*}
\ln \mathbb{E}_{\lambda \sim \rho}\left[\exp\left(\sum_{s=1}^t \lambda_{k_s}X_{s,k_s} - \frac{1}{2}\lambda_{k_s}^2\right)\right]
&= -\frac{1}{2}\sum_{k=1}^K \ln(1 + N_{t,k}\sigma_k^{-2}) + \frac{1}{2} \sum_{k=1}^K \frac{S_{t,k}^2}{(N_{t,k} + \sigma_k^2)}
\: .
\end{align*}
\end{lemma}

\begin{proof}
\begin{align*}
&\mathbb{E}_{\lambda \sim \rho}\left[\exp\left(\sum_{s=1}^t \lambda_{k_s}X_{s,k_s} - \frac{1}{2}\lambda_{k_s}^2\right)\right]
\\
&= \prod_k \mathbb{E}_{\lambda_k \sim \mathcal N(0, \sigma_k^{-2})}\left[\exp\left(\lambda_k S_{t,k} - \frac{1}{2}N_{t,k} \lambda_k^2\right)\right]
\\
&= \prod_k \frac{1}{\sqrt{2 \pi \sigma_k^{-2}}}\int_{\lambda_k}\exp\left(\lambda_k S_{t,k} - \frac{1}{2}N_{t,k} \lambda_k^2 - \frac{\sigma_k^2}{2}\lambda_k^2\right)\mathrm{d}\lambda_k
\\ 
&= \prod_k \frac{1}{\sqrt{(1 + N_{t,k}\sigma_k^{-2})}} \frac{1}{\sqrt{2 \pi (N_{t,k} + \sigma_k^2)^{-1}}}
  \int_{\lambda_k} \exp\left(-\frac{1}{2}(N_{t,k} + \sigma_k^2)\left( \lambda_k - \frac{S_{t,k}}{(N_{t,k} + \sigma_k^2)} \right)^2 + \frac{1}{2}\frac{S_{t,k}^2}{N_{t,k} + \sigma_k^2}\right)\mathrm{d}\lambda_k
\\
&= \prod_k \frac{1}{\sqrt{(1 + N_{t,k}\sigma_k^{-2})}} \exp\left(\frac{1}{2}\frac{S_{t,k}^2}{N_{t,k} + \sigma_k^2}\right)
\\ 
\end{align*}

\end{proof}

\begin{corollary}\label{cor:sum_le_eta_mul}
Let $\rho = \mathcal N(0, \mathrm{diag}(\sigma_k^{-2}))$, $\eta_{t,\max} = \max_k \frac{\sigma_k^2}{N_{t,k}}$ and $\eta_{t,\min} = \min_k \frac{\sigma_k^2}{N_{t,k}}$. Then
\begin{align*}
\frac{1}{2} \sum_{k=1}^K \frac{S_{t,k}^2}{N_{t,k}}
&\le (1 + \eta_{t,\max}) \left(\ln \mathbb{E}_{\lambda \sim \rho}\left[\exp\left(\sum_{s=1}^t \lambda_{k_s}X_{s,k_s} - \frac{1}{2}\lambda_{k_s}^2\right)\right]
  + \frac{K}{2}\ln(1 + \eta_{t,\min}^{-1})\right)
\end{align*}

\end{corollary}

\begin{proof}
Using Lemma~\ref{lem:integral_exp_eq_log_add},
\begin{align*}
\frac{1}{2} \sum_{k=1}^K \frac{S_{t,k}^2}{N_{t,k} + \sigma_k^2}
&= \ln \mathbb{E}_{\lambda \sim \rho}\left[\exp\left(\sum_{s=1}^t \lambda_{k_s}X_{s,k_s} - \frac{1}{2}\lambda_{k_s}^2\right)\right]
  + \frac{1}{2} \sum_{k=1}^K \ln(1 + N_{t,k}\sigma_k^{-2})
\: .
\end{align*}
Then
\begin{align*}
\frac{1}{2} \sum_{k=1}^K \frac{S_{t,k}^2}{N_{t,k} + \sigma_k^2}
\ge \frac{1}{2} \sum_{k=1}^K \frac{S_{t,k}^2}{N_{t,k}(1 + \eta_{t,\max})}
= \frac{1}{1 + \eta_{t,\max}}\frac{1}{2} \sum_{k=1}^K \frac{S_{t,k}^2}{N_{t,k}}
\end{align*}
Finally, $N_{t,k} \sigma_k^{-2} \le \eta_{t,\min}^{-1}$.
\end{proof}

If $N_{t,k}$ was a known, unchanging number, we could choose $\sigma_k^2 \propto N_{t,k}$ to get $\eta_{t,\max} = \eta_{t, \min}$, and we would choose it to minimize the right hand side.
The strategy to use that ``known $N_{t,k}$'' case even if they are random is to put geometric grids on the number of pulls of each arm, define distributions that are adapted to each cell of the grid, and combine them into a mixture of Gaussians.

Let $(\eta_{n_1, \ldots, n_K})_{n_1, \ldots, n_K \in \mathbb{N}}$ be non-negative real numbers that will be chosen later.
For $i \in \mathbb{N}$, let $w_i = \frac{6}{\pi^2}\frac{1}{(i+1)^2}$. The weights $(w_i)$ satisfy $\sum_{i \in \mathbb{N}} w_i = 1$, hence also $\sum_{n_1, \ldots, n_K} (\prod_{k=1}^K w_{n_k}) = 1$.

Let $\rho_{n_1, \ldots, n_K} = \bigotimes_{k=1}^K \mathcal N(0, e^{- n_k} \eta_{n_1, \ldots, n_K}^{-1})$. This is a product distribution, with each marginal being a Gaussian with mean 0 and variance that depends on the number of grid cell.

With probability $1 - \delta$, for all $(n_1, \ldots, n_K) \in \mathbb{N}^K$ and all $t$,
\begin{align*}
\ln \mathbb{E}_{\lambda \sim \rho_{n_1, \ldots, n_K}}\left[\exp\left(\sum_{s=1}^t \lambda_{k_s}X_{s,k_s} - \frac{1}{2}\lambda_{k_s}^2\right)\right]
\le \ln \frac{1}{\delta} + \sum_{k=1}^K \ln \frac{1}{w_{n_k}}
\: .
\end{align*}
This is simply an union bound using Corollary~\ref{cor:prob_exists_log_ge_le}, with weight $\prod_{k=1}^K w_{n_k}$ for $\rho_{n_1, \ldots, n_K}$.

In particular, there exists $(n_1, \ldots, n_k)$ such that for all $k \in [K]$, $e^{n_k} \le N_{t,k} \le e^{n_k+1}$.
For that choice, $e^{-1}\eta_{n_1, \ldots, n_K} \le \frac{e^{n_k}\eta_{n_1, \ldots, n_K}}{N_{t,k}} \le \eta_{n_1, \ldots, n_K}$.
For those values, using Corollary~\ref{cor:sum_le_eta_mul} with $\sigma_k^2 = e^{n_k}\eta_{n_1, \ldots, n_K}$, with probability $1 - \delta$,
\begin{align*}
\frac{1}{2} \sum_{k=1}^K \frac{S_{t,k}^2}{N_{t,k}}
&\le (1 + \eta_{n_1, \ldots, n_K}) \left(\ln \frac{1}{\delta} + \sum_{k=1}^K \ln \frac{1}{w_{n_k}}
  + \frac{K}{2}\ln(1 + e \eta_{n_1, \ldots, n_K}^{-1})\right)
\\
&\le (1 + \eta_{n_1, \ldots, n_K}) \left(\ln \frac{e^{K/2}}{\delta} + \sum_{k=1}^K \ln \frac{1}{w_{n_k}}
  + \frac{K}{2}\ln(1 + \eta_{n_1, \ldots, n_K}^{-1})\right)
\\
&= (1 + \eta_{n_1, \ldots, n_K}) \left(\ln \frac{(\sqrt{e}\pi^2/6)^K \prod_{k=1}^K (n_k+1)^2}{\delta}
  + \frac{K}{2}\ln(1 + \eta_{n_1, \ldots, n_K}^{-1})\right)
\end{align*}

This is where we choose $\eta_{n_1, \ldots, n_K}$ to minimize the right hand side.

By Lemma A.3 of \citep{degenneImpactStructureDesign2019}, the minimal value is attained at some $\eta_{n_1, \ldots, n_K}$ such that
\begin{align*}
&(1 + \eta_{n_1, \ldots, n_K}) \left(\ln \frac{(\sqrt{e}\pi^2/6)^K \prod_{k=1}^K (n_k+1)^2}{\delta}
  + \frac{K}{2}\ln(1 + \eta_{n_1, \ldots, n_K}^{-1})\right)
\\
&= \frac{K}{2} \overline{W}\left( 1 + \frac{2}{K}\ln \frac{(\sqrt{e}\pi^2/6)^K \prod_{k=1}^K (n_k+1)^2}{\delta}\right)
\end{align*}

By the choice of $n_k$, it satisfies $n_k \le \ln N_{t,k}$.
We get that with probability $1 - \delta$, for all $t$,
\begin{align*}
\frac{1}{2} \sum_{k=1}^K \frac{S_{t,k}^2}{N_{t,k}}
&\le \frac{K}{2} \overline{W}\left( 1 + \frac{2}{K}\ln \frac{(\sqrt{e}\pi^2/6)^K \prod_{k=1}^K (1 + \ln N_{t,k})^2}{\delta}\right)
\\
&= \frac{K}{2} \overline{W}\left(\frac{2}{K}\ln \left((e\pi^2/6)^K \prod_{k=1}^K (1 + \ln N_{t,k})^2\right) + \frac{2}{K}\ln \frac{1}{\delta}\right)
\: .
\end{align*}

This ends the proof of the theorem.



\subsection{Upper bounds on $\beta(t, \delta)$ and on $\gamma_r$}

We choose the threshold
\begin{align*}
\beta(t, \delta)
&= \frac{K}{2} \overline{W}\left(2\ln \left(\frac{e\pi^2}{6}\right) + \frac{2}{K}\ln \left(\prod_{k=1}^K (1 + \ln N_{t,k})^2\right) + \frac{2}{K}\ln \frac{1}{\delta}\right)
\: .
\end{align*}

We can get an upper bound that is not random by maximizing over $(N_{t,k})_{k \in [K]}$ under the constraint $\sum_{k=1}^K N_{t,k} = t$. We get
\begin{align*}
\beta(t, \delta)
&\le \frac{K}{2} \overline{W}\left(2\ln \left(\frac{e\pi^2}{6}\right) + 4\ln \left(1 + \ln \frac{t}{K}\right) + \frac{2}{K}\ln \frac{1}{\delta}\right)
\: .
\end{align*}
We can get further upper bounds by using $\overline{W}(x) \le x + \ln x + 1/2 \le 2x$. This gives
\begin{align*}
\beta(t, \delta)
&\le 2 K\ln \left(\frac{e\pi^2}{6}\right) + 4 K\ln \left(1 + \ln \frac{t}{K}\right) + 2\ln \frac{1}{\delta}
\\
&\le 2 K\ln \left(\frac{e\pi^2}{6}\right) + 4 K\ln \frac{t}{K} + 2\ln \frac{1}{\delta}
\: .
\end{align*}
The right asymptotic for $\beta(t, \delta)$ as $\delta \to 0$ is $\ln(1/\delta)$. We lost a factor 2 in the upper bound above.




\begin{lemma}\label{lem:gamma_ub_bis}
Let $\gamma_r$ be the solution to $\beta(\bar{t}_r, \delta) = \gamma_r$ , for $\bar{t}_r = 2(K l_{1,r}/T_0 + \gamma_r) T_r$ and $l_{1,r} = 32 T_0 \ln (2\sqrt{2K} T_r)$. Then
\begin{align*}
\gamma_r \le 4 \ln \frac{1}{\delta} + 8K \ln(T_r) + 4K (11 + \ln K)
\: .
\end{align*}
\end{lemma}

\begin{proof}
We use an upper bound for $\beta(t, \delta)$: $\gamma_r$ is bounded from above by the solution $\gamma'_r$ to
\begin{align*}
\gamma = 2 K\ln \left(\frac{e\pi^2}{6}\right) + 4 K \ln \left(2(32 \ln (2\sqrt{2K} T_r) + \frac{\gamma}{K}) T_r\right) + 2\ln \frac{1}{\delta}
\: .
\end{align*}
Then either $\gamma'_r \le 8 K \ln (2\sqrt{2K} T_r)$ or $\gamma'_r$ is less than the solution to
\begin{align*}
\gamma
&= 2 K\ln \left(\frac{e\pi^2}{6}\right) + 4 K \ln \left(10\frac{\gamma T_r}{K}\right) + 2\ln \frac{1}{\delta}
\\
&= 2 K\ln \left(\frac{50 e\pi^2}{3}\right) + 4 K \ln \left(\frac{\gamma T_r}{K}\right) + 2\ln \frac{1}{\delta} 
\: .
\end{align*}


That is,
\begin{align*}
\gamma'_r
&= 4K \overline{W}\left( \frac{1}{2K} \ln \frac{1}{\delta} + \ln(T_r) + \frac{1}{2}\ln\frac{800 e \pi^2}{3} \right)
\\
&\le 4 \ln \frac{1}{\delta} + 8K \ln(T_r) + 4K \ln\frac{800 e \pi^2}{3}
\: .
\end{align*}

At this point, we get
\begin{align*}
\gamma_r
&\le \max\left\{ 4 \ln \frac{1}{\delta} + 8K \ln(T_r) + 4K \ln\frac{800 e \pi^2}{3},  8 K \ln (2\sqrt{2K} T_r)\right\}
\\
&\le 8K \ln(T_r) + \max\left\{ 4 \ln \frac{1}{\delta} + 4K \ln\frac{800 e \pi^2}{3}, 8 K \ln (2\sqrt{2K})\right\}
\\
&\le 8K \ln(T_r) + 4 \ln \frac{1}{\delta} + 4K \ln\frac{800 e \pi^2}{3} + 8 K \ln (2\sqrt{2K})
\\
&\le 8K \ln(T_r) + 4 \ln \frac{1}{\delta} + 4K \ln\left(\frac{6400 e \pi^2}{3}K\right)
\\
&\le 8K \ln(T_r) + 4 \ln \frac{1}{\delta} + 4K (11 + \ln K)
\: .
\end{align*}


\end{proof}

% !TeX root = ../all.tex


\section{Proofs related to the algorithm}\label{app:ub} 

\subsection{Additional Lemmas}

\begin{lemma}\label{lem:subG_concentration}
Let $(X_i)_{i \in \mathbb{N}}$ be i.i.d. $\sigma^2$-sub-Gaussian random variables with mean $\mu$. For $n \in \mathbb{N}$, let $\hat{\mu}_n$ be the average of the first $n$ random variables. Then
\begin{align*}
\mathbb{P}(\exists n \ge N, \ \hat{\mu}_n \ge \mu + \varepsilon) \le e^{- \frac{N \varepsilon^2}{2\sigma^2}}
\: , \\
\mathbb{P}(\exists n \ge N, \ \hat{\mu}_n \le \mu - \varepsilon) \le e^{- \frac{N \varepsilon^2}{2\sigma^2}} \: .
\end{align*}
\end{lemma}

\begin{proof}
Given $(X_1, \ldots, X_N)$, the process $M_n(\lambda) : n \mapsto e^{\lambda\sum_{i=1}^{N+n} (X_i - \mu) - \frac{1}{2}(N+n) \sigma^2 \lambda^2}$ is a nonnegative supermartingale for any $\lambda \in \mathbb{R}$ by the sub-Gaussian hypothesis, with expectation $e^{\lambda\sum_{i=1}^{N} (X_i - \mu) - \frac{1}{2}N \sigma^2 \lambda^2}$ at $n=0$.

By Ville's inequality,
\begin{align*}
\mathbb{P}(\exists n, \  M_n(\lambda) \ge 1/\delta \mid X_1, \ldots, X_N)
\le \delta e^{\lambda\sum_{i=1}^{N} (X_i - \mu) - \frac{1}{2}N \sigma^2 \lambda^2}
\end{align*}

For all $\lambda \in \mathbb{R}$ and all $\delta \in (0,1)$,
\begin{align*}
\mathbb{P}\left(\exists n \ge N, \  \lambda\sum_{i=1}^{n} (X_i - \mu) - \frac{1}{2}n\sigma^2 \lambda^2 \ge \ln(1/\delta)\right)
&= \mathbb{E}\left[\mathbb{P}(\exists n \ge 0, \  M_n(\lambda) \ge 1/\delta \mid X_1, \ldots, X_N)\right]
\\
&\le \delta \mathbb{E}\left[e^{\lambda\sum_{i=1}^{N} (X_i - \mu) - \frac{1}{2}N \sigma^2 \lambda^2}\right]
\\
&\le \delta
\: .
\end{align*}
Reordering, we get, for $\lambda \ge 0$,
\begin{align*}
\mathbb{P}\left(\exists n \ge N, \  \hat{\mu}_n \ge \mu + \frac{1}{2}\sigma^2 \lambda + \frac{1}{N \lambda}\ln\frac{1}{\delta}\right)
&\le \mathbb{P}\left(\exists n \ge N, \  \hat{\mu}_n \ge \mu + \frac{1}{2}\sigma^2 \lambda + \frac{1}{n \lambda}\ln\frac{1}{\delta}\right)
\\
&\le \delta
\: .
\end{align*}

Choose $\delta = e^{-\frac{N \varepsilon^2}{2 \sigma^2}}$ and $\lambda = \frac{\varepsilon}{\sigma^2}$ to obtain
\begin{align*}
\mathbb{P}\left(\exists n \ge N, \  \hat{\mu}_n \ge \mu + \varepsilon \right)
\le e^{-\frac{N \varepsilon^2}{2 \sigma^2}}
\: .
\end{align*}

The second inequality is obtained similarly, with $\lambda \le 0$.
\end{proof}

\begin{lemma}\label{lem:probaE_bis}
	The probability of $\mathcal E_r$ satisfies
	\begin{align*}
	\mathbb{P}(\mathcal E_r) \ge 1-2K\exp(- 2^r l_{1,r} \varepsilon_r^2/2\sigma^2)
	\: .
	\end{align*}
\end{lemma}

\begin{proof}
$\mathcal E_r$ is the event that $\Vert \bm\mu - \tilde{\bm\mu}^r \Vert_\infty \le \varepsilon_r$ and $\Vert \bm\mu - \hat{\bm\mu}^r \Vert_\infty \le \varepsilon_r$.
We use an union bound over the arms to bound the probability of the complement $\mathcal{E}_r^c$.
For each $i \in [K]$, $\tilde{\mu}^r_i$ and $\hat{\mu}^r_i$ are empirical means of at least $2^r l_{1,r}$ samples. We can thus apply Lemma~\ref{lem:subG_concentration} (twice, once for deviations from above and once for deviations from below).
\end{proof}

\begin{lemma}\label{lem:proba_Er}
	Let $p_r \in (0,1]$.
	For the choice $\varepsilon_r = \sqrt{\frac{2\sigma^2}{2^r l_1}\ln\frac{2K}{p_r}}$, the probability of the event $\mathcal E_r$ is $\mathbb{P}(\mathcal E_r) \ge 1 - p_r$.
\end{lemma}


\subsection{Proof of Theorem~\ref{th:alggen}}\label{app:ub_proof}

If $\overline{T}^\star(\hat{B}_r) > T_r$ then the algorithm does not enter the second batch of the phase by definition of the algorithm.

If $\overline{T}^\star(\hat{B}_r) \le T_r$ then by the choice of $\gamma_r$ and Lemma~\ref{lem:sufficientsampling}, under $\mathcal E_r$ the stopping condition is triggered.

We now prove the complexity upper bounds. Let $\mathcal C_r$ be the event that the algorithm attains phase $r$ and does not stop at that phase.
We proved that $\{\overline{T}^\star(\hat{B}_r) \le T_r\} \cap \mathcal E_r \subseteq \mathcal C_r^c$ for all $r$. That is, $\mathcal C_r \subseteq \mathcal E_r^c \cup \{\overline{T}^\star(\hat{B}_r) > T_r\}$.

Recall that $R^* = \min \{r \mid \forall r' \ge r, \ \mathcal E_{r'} \implies \overline{T}^\star(\hat{B}_{r'}) \le T_{r'}\}$.
\begin{align*}
R_\delta
&= \sum_{r=1}^{+\infty} \mathbb{I}(\mathcal C_{r-1}) + \mathbb{I}(\mathcal C_{r-1} \wedge \{\overline{T}^\star(\hat{B}_r) \le T_r\}) \: .
\\
&\le R^* + 2 \sum_{r=R^*+1}^{+\infty} \mathbb{I}(\mathcal C_{r-1}) + \sum_{r=1}^{R^*} \mathbb{I}(\mathcal C_{r-1} \wedge \{\overline{T}^\star(\hat{B}_r) \le T_r\})
\: .
\end{align*}
By definition of $R^*$, for $r \ge R^*$ we have $ \{\overline{T}^\star(\hat{B}_r) > T_r\} \subseteq \mathcal E_r^c$.
Using that property and the inclusion we proved on $\mathcal C_r$ we have, for $r > R^*$,
\begin{align*}
\mathcal C_{r-1}
\subseteq \mathcal E_{r-1}^c \cup \{\overline{T}^\star(\hat{B}_{r-1}) > T_{r-1}\}
\subseteq \mathcal E_{r-1}^c
\: .
\end{align*}
Therefore,
\begin{align*}
R_\delta
&\le R^* + 2 \sum_{r=R^*+1}^{+\infty} \mathbb{I}(\mathcal E_{r-1}^c) + \sum_{r=1}^{R^*} \mathbb{I}(\mathcal C_{r-1} \wedge \{\overline{T}^\star(\hat{B}_r) \le T_r\})
\: .
\end{align*}
When $\mathcal E_r$ happens $T^\star(\bm\mu) \le \overline{T}^\star(\hat{B}_r)$, hence $\{\overline{T}^\star(\hat{B}_r) \le T_r\} \subseteq \mathcal E_r^c \cup \{T^\star(\bm\mu) \le T_r\}$.
\begin{align*}
R_\delta
&\le R^* + 2 \sum_{r=R^*+1}^{+\infty} \mathbb{I}(\mathcal E_{r-1}^c) + \sum_{r=1}^{R^*} \mathbb{I}(\mathcal E_{r}^c)
	+ \sum_{r=1}^{R^*} \mathbb{I}(\{T^\star(\bm\mu) \le T_r\})
\\
&\le R^* + 1 + 2 \sum_{r=1}^{+\infty} \mathbb{I}(\mathcal E_{r}^c) + \sum_{r=1}^{R^*} \mathbb{I}(\{T^\star(\bm\mu) \le T_r\})
\: .
\end{align*}
Finally, $ \sum_{r=1}^{R^*} \mathbb{I}(\{T^\star(\bm\mu) \le T_r\}) = \max\{0, R^* - \lceil \log_2 \frac{T^\star(\bm\mu)}{T_0} \rceil \}$, and the definition of $R^*$ implies $R^* \ge \lceil \log_2 \frac{T^\star(\bm\mu)}{T_0} \rceil$.

We now bound the sample complexity $\tau_\delta$. Since $\bar{t}_r$ is an upper bound on the sample complexity up to phase $r$,
\begin{align*}
\tau_\delta
&\le \sum_{r=1}^{+\infty} \bar{t}_r \mathbb{I}\{C_{r-1}\}
\\
&\le \bar{t}_{R^*} + \sum_{r=R^* + 1}^{+\infty} \bar{t}_r \mathbb{I}\{C_{r-1}\}
\\
&\le \bar{t}_{R^*} + \sum_{r=R^* + 1}^{+\infty} \bar{t}_r \mathbb{I}\{E_{r-1}^c\}
\: .
\end{align*}



\subsection{Proof of the batch and complexity upper bounds}

We finish the proof of Theorem~\ref{thm:compexity_upper_bounds} from where we stopped in the main text. We have
\begin{align*}
\mathbb{E}\left[R_\delta\right]
&\le 2r^* - \lceil \log_2 \frac{T^\star(\bm\mu)}{T_0} \rceil + 1 + 2\sum_{r = 1}^{+\infty} p_r
\: , \\
\mathbb{E}\left[\tau_\delta\right]
&\le \bar{t}_{r^*} + \sum_{r = 1}^{+\infty} p_{r} \bar{t}_{r+1}
\: .
\end{align*}
In those expressions, $r^* = \max\{r_0, r_1\}$ with $r_0 = \min\{r \mid 2 \varepsilon_r \le b(\bm\mu)\}$, $r_1 =  \min \{r \mid T_r \ge e T^\star(\bm\mu)\}$, and
$\bar{t}_r = (K l_{1,r}/T_0 + 2 \gamma_r) T_r$ with $l_{1,r}/T_0 = 32 \ln(2 \sqrt{2K} T_r)$.

The choice of $p_r$ is a trade-off between the sums and $r_0$. We choose $p_r = T_{r+1}^2$.

\paragraph{Bounding the sums}
The sum in the batch complexity is bounded by $T_0^{-2}/3$.
The sum that appears in the sample complexity is
\begin{align*}
\sum_{r = \max\{r_0, r_1\}+1}^{+\infty} p_{r-1} \bar{t}_r
= \sum_{r = \max\{r_0, r_1\}+1}^{+\infty} \frac{\bar{t}_r}{T_r^2}
\: .
\end{align*}
We will need the values of a few sums.
\begin{align*}
\sum_{r=1}^{+\infty} \frac{1}{T_r}
&= \sum_{r=1}^{+\infty} \frac{1}{2^r T_0}
= \frac{1}{T_0}
\: , \\
\sum_{r=1}^{+\infty} \frac{\ln T_r}{T_r}
&= \frac{1}{T_0} \sum_{r=1}^{+\infty} \frac{r \ln2 + \ln T_0}{2^r}
= \frac{\ln (4T_0)}{T_0}
\: .
\end{align*}

Let $c_{K,\delta} = 4 \ln \frac{1}{\delta} + 4K (11 + \ln K)$.
By Lemma~\ref{lem:gamma_ub_bis}, $\gamma_r \le 8K \ln(T_r) + c_{K,\delta}$ .
An upper bound on the sample complexity until the end of phase $r$ is then
\begin{align*}
\bar{t}_r
&= (K l_{1,r}/T_0 + 2\gamma_r) T_r
\\
&= (32 K \ln(2\sqrt{2K}T_r) + 2 \gamma_r) T_r
\\
&\le (48 K \ln(T_r) + 32 K \ln(2\sqrt{2K}) + c_{K,\delta})T_r
\: .
\end{align*}
The sum that appears in the sample complexity is at most
\begin{align*}
\sum_{r = 1}^{+\infty} \frac{\bar{t}_r}{T_r^2}	
&\le \frac{48 K \ln(4T_0) + 32 K \ln(2\sqrt{2K}) + c_{K,\delta}}{T_0}
\: .
\end{align*}

\paragraph{Bound on $r^*$ and $\bar{t}_{r^*}$}

\begin{align*}
\bar{t}_{r^*}
\le (48 K \ln (\max\{T_{r_0}, T_{r_1}\}) + 32 K \ln(2\sqrt{2K}) + c_{K,\delta}) \max\{T_{r_0}, T_{r_1}\}
\: .
\end{align*}


Recall that $r_0 = \min\{r \mid 2 \varepsilon_r \le b(\bm\mu)\}$, $r_1 =  \min \{r \mid T_r \ge e T^\star(\bm\mu)\}$.
If we get an upper bound $n$ on $T_i$, we then have $r_i \le \log_2 \frac{n}{T_0}$.
We get a bound on $T_{r_1}$ from its definition: $T_{r_1-1} \le e T^*(\bm\mu)$ hence $T_{r_1} \le 2 e T^\star(\bm\mu)$.

Since $\varepsilon_{r_0 - 1} \ge b(\bm\mu)/2$, we get an inequality on $T_{r_0 - 1}$.
\begin{align*}
\sqrt{\frac{2\sigma^2}{2^{r_0 - 1}l_{1, r_0-1}} \ln \left( 2K T_{r_0}^2 \right)} \ge \frac{b(\bm\mu)}{2}
\: .
\end{align*}
With the value of $l_{1,r}$ and using $2 T_{r_0 - 1} = T_{r_0}$, this becomes
\begin{align*}
T_{r_0} \le \frac{\sigma^2}{b(\bm\mu)^2}
\: .
\end{align*}

Let $T^\star_b(\bm\mu) = \max\{\frac{\sigma^2}{b(\bm\mu)^2}, 2 e T^\star(\bm\mu)\}$.
We have proved $\max\{T_{r_0}, T_{r_1}\} \le T^\star_b(\bm\mu)$, hence
\begin{align*}
\bar{t}_{r^*}
\le (48 K \ln T^\star_b(\bm\mu) + 32 K \ln(2\sqrt{2K}) + c_{K,\delta}) T^\star_b(\bm\mu)
\end{align*}
and $r^* \le \log_2 \frac{T^\star_b(\bm\mu)}{T_0}$.

\paragraph{Putting things together}

\begin{align*}
\mathbb{E}\left[R_\delta\right]
&\le \log_2 \frac{T^\star_b(\bm\mu)}{T_0} + \log_2 \frac{T^\star_b(\bm\mu)}{T^\star(\bm\mu)} + 1 + T_0^{-2}
\: , \\
\mathbb{E}\left[\tau_\delta\right]
&\le (48 K \ln T^\star_b(\bm\mu) + 32 K \ln(2\sqrt{2K}) + c_{K,\delta}) T^\star_b(\bm\mu)
\\ & \quad + \frac{48 K \ln(4T_0) + 32 K \ln(2\sqrt{2K}) + c_{K,\delta}}{T_0}
\: .
\end{align*}

Let's simplify the sample complexity.
\begin{align*}
32 K \ln(2\sqrt{2K}) + c_{K,\delta}
&= 32 K \ln(2\sqrt{2K}) + 4 \ln \frac{1}{\delta} + 4K (11 + \ln K)
\\
&= 4 \ln \frac{1}{\delta} + 4K (5\ln K + 11 + 4 \ln(8))
\\
&\le 4 \ln \frac{1}{\delta} + 20K (\ln K + 4)
\: .
\end{align*}

\begin{align*}
\mathbb{E}\left[\tau_\delta\right]
&\le (48 K \ln T^\star_b(\bm\mu) + 4 \ln \frac{1}{\delta} + 20K (\ln K + 4)) T^\star_b(\bm\mu)
\\ & \quad + (48 K \ln(4T_0) + 4 \ln \frac{1}{\delta} + 20K (\ln K + 4)) T_0^{-1}
\\
&= 4 \ln \left(\frac{1}{\delta}\right) (T^\star_b(\bm\mu) + T_0^{-1}) + 20K (\ln K + 4) (T^\star_b(\bm\mu) + T_0^{-1})
	+ 48 K (T^\star_b(\bm\mu) \ln T^\star_b(\bm\mu) + T_0^{-1} \ln(4T_0))
\: .
\end{align*}



\subsection{Implication between the two assumptions}

We prove Lemma~\ref{lem:asm2_implies_asm1}.

\begin{lemma}\label{lem:sub_sqrt_inv_TStar} 
Let $\bm\nu$ and $\bm\nu'$ be two instances and let $\omega_{\bm \mu} = \arg\max_\omega \inf_{\lambda \in Alt_{\bm\mu}}\sum_{i=1}^K \omega_i (\mu_i - \lambda_i)^2$. Then
\begin{align*}
\sqrt{T^\star(\bm\mu)^{-1}} - \sqrt{T^\star(\bm\mu')^{-1}}
\le \frac{1}{\sqrt{2\sigma^2}} \Vert \mu - \mu' \Vert_\infty
\: .
\end{align*}
\end{lemma}

\begin{proof}
For $\omega \in \Sigma_K$ and $\bm x \in \mathbb{R}^K$, let $\Vert \bm x \Vert_\omega = \sqrt{\sum_{i=1}^K \omega_i x_i^2}$.
It satisfies the triangle inequality and $\Vert \bm x \Vert_\omega \le \Vert \bm x \Vert_\infty$.
For any $\bm\lambda$ and $\omega$,
\begin{align*}
\Vert \bm\mu - \bm\lambda\Vert_\omega \le \Vert \bm\mu' - \bm\lambda\Vert_\omega + \Vert \bm\mu - \bm\mu'\Vert_\omega
\: .
\end{align*}
We can take an infimum on both sides over lambda in $Alt_{\bm\mu}$ and then apply the result to $\omega_{\bm\mu}$ to get
\begin{align*}
\sqrt{2\sigma^2 T^\star(\bm\mu)^{-1}} \le \inf_{\lambda \in Alt_{\bm\mu}}\Vert \bm\mu' - \bm\lambda\Vert_{\omega_{\bm\mu}} + \Vert \bm\mu - \bm\mu'\Vert_{\omega_{\bm\mu}}
\: .
\end{align*}
Either $Alt_{\bm\mu} = Alt_{\bm\mu'}$ and we can replace by that on the right hand side, or $\bm\mu' \in Alt_{\bm\mu}$. In that second case $\inf_{\lambda \in Alt_{\bm\mu}}\Vert \bm\mu' - \bm\lambda\Vert_{\omega_{\bm\mu}} = 0 \le \inf_{\lambda \in Alt_{\bm\mu'}}\Vert \bm\mu' - \bm\lambda\Vert_{\omega_{\bm\mu}}$. We thus have
\begin{align*}
\sqrt{2\sigma^2 T^\star(\bm\mu)^{-1}} \le \inf_{\lambda \in Alt_{\bm\mu'}}\Vert \bm\mu' - \bm\lambda\Vert_{\omega_{\bm\mu}} + \Vert \bm\mu - \bm\mu'\Vert_{\omega_{\bm\mu}}
\: .
\end{align*}
We maximize over $\omega$ to get $\inf_{\lambda \in Alt_{\bm\mu'}}\Vert \bm\mu' - \bm\lambda\Vert_{\omega_{\bm\mu}} \le \sqrt{2\sigma^2 T^\star(\bm\mu')^{-1}}$, hence
\begin{align*}
\sqrt{2\sigma^2 T^\star(\bm\mu)^{-1}}
&\le \sqrt{2\sigma^2 T^\star(\bm\mu')^{-1}} + \Vert \bm\mu - \bm\mu'\Vert_{\omega_{\bm\mu}}
\\
&\le \sqrt{2\sigma^2 T^\star(\bm\mu')^{-1}} + \Vert \bm\mu - \bm\mu'\Vert_{\infty}
\: .
\end{align*}
After dividing by $\sqrt{2\sigma^2}$, this is the inequality of the lemma.
\end{proof}


\begin{corollary}\label{cor:sub_ln_TStar_le}
For all $\bm\mu$ and $\bm\mu'$,
\begin{align*}
\ln T^\star(\bm\mu') - \ln T^\star(\bm\mu)
\le \sqrt{\frac{2}{\sigma^2}T^\star(\bm\mu')} \ \Vert \bm\mu - \bm\mu' \Vert_\infty
\: .
\end{align*}
\end{corollary}

\begin{proof}
\begin{align*}
\ln T^\star(\bm\mu') - \ln T^\star(\bm\mu)
&= 2 \ln \left( 1 + \frac{\sqrt{T^\star(\bm\mu)^{-1}} - \sqrt{T^\star(\bm\mu')^{-1}}}{\sqrt{T^\star(\bm\mu')^{-1}}} \right)
\\
&\le 2 \frac{\sqrt{T^\star(\bm\mu)^{-1}} - \sqrt{T^\star(\bm\mu')^{-1}}}{\sqrt{T^\star(\bm\mu')^{-1}}}
\\
&\le \sqrt{\frac{2}{\sigma^2}T^\star(\bm\mu')} \ \Vert \bm\mu - \bm\mu' \Vert_\infty
\: .
\end{align*}

\end{proof}


\begin{corollary}\label{cor:abs_sub_ln_TStar_le}
For all $\bm\mu$ and $\bm\mu'$ with $\Vert \bm\mu - \bm\mu' \Vert_\infty \le \sqrt{\sigma^2 / (2 T^\star(\bm\mu))}$,
\begin{align*}
\left\vert \ln T^\star(\bm\mu') - \ln T^\star(\bm\mu) \right\vert
\le \sqrt{\frac{8}{\sigma^2}T^\star(\bm\mu)} \ \Vert \bm\mu - \bm\mu' \Vert_\infty
\: .
\end{align*}
\end{corollary}

\begin{proof}
One of the two inequalities we need to prove is due to Corollary~\ref{cor:sub_ln_TStar_le}. For the other, by the same corollary,
\begin{align*}
\ln T^\star(\bm\mu') - \ln T^\star(\bm\mu)
\le \sqrt{\frac{2}{\sigma^2}T^\star(\bm\mu')} \ \Vert \bm\mu - \bm\mu' \Vert_\infty
\: .
\end{align*}
It remains to show $T^\star(\bm\mu') \le 4 T^\star(\bm\mu)$. By Lemma~\ref{lem:sub_sqrt_inv_TStar} and the the hypothesis on $\Vert \bm\mu - \bm\mu' \Vert_\infty$,
\begin{align*}
\sqrt{T^\star(\bm\mu)^{-1}} - \sqrt{T^\star(\bm\mu')^{-1}}
\le \frac{1}{\sqrt{2\sigma^2}} \Vert \bm\mu - \bm\mu' \Vert_\infty
\le \frac{1}{2}\sqrt{T^\star(\bm\mu)^{-1}} \: .
\end{align*}
Reordering proves the inequality.
\end{proof}



\subsection{Proofs for Top-K and thresholding bandits}
\label{app:topk_threshold}

This section is devoted to the proof of Lemma~\ref{lem:constrBbai}. We start with a preliminary result allowing us to compute $\overline{w}^\star(B)$ once we have found a suitable instance in $B$.


\begin{lemma}\label{lem:wbwst}
	If all instances in $B$ share the same correct answer $i^\star$ and if there exists some mean vector $\bm b\in B$ such that\begin{equation}\label{eq:bworst} \inf_{\bm\nu\in B}\inf_{\bm{\lambda}\in Alt_{\bm b}} \sum_i w_i(\bm{b}) \frac{(\mu_i-\lambda_i)^2}{2\sigma^2}\geq \inf_{\bm{\lambda}\in Alt_{\bm b}} \sum_i w_i(\bm{b}) \frac{(b_i-\lambda_i)^2}{2\sigma^2}\end{equation} where $w(\bm\mu)=\argmax_{w\in \Sigma_K} \inf_{\bm\lambda \in Alt_{\bm\mu}} \sum_i w_i \frac{(\mu_i-\lambda_i)^2}{2\sigma^2}$, then $\overline{T}^\star(B) = \max_{\bm \mu\in B}T^\star(\bm \mu)$.
\end{lemma}

\begin{proof}
	For some $w\in \Sigma_K$, writing $f(w,\bm\mu')=\inf_{\bm\lambda\in Alt_{\bm b}} \sum_i w_i \frac{(\mu'_i-\lambda_i)^2}{2\sigma^2}$ for clarity,
	\begin{align*}
		\inf_{\bm \nu\in \hat{B}_r} f( w,\bm\mu) &\leq f( w,\bm b) &\text{because }\bm b\in \hat{B}_r\\
		&\leq f(w_{\bm b},\bm b)& \text{from the definition of }w_{\bm b}\\
		&\leq \inf_{\bm\mu'\in\hat{B}_r} f(w_{\bm b},\bm\mu')&\text{from the hypothesis}
	\end{align*} so that $w^\star(\bm b) = \argmax_{w\in \Sigma_K} \inf_{\bm\nu\in\hat{B}_r} f(\bm w,\nu)=\overline w^\star(B)$.

\begin{align*}
	\overline{T}^\star(B) &= \left( \inf_{\bm\nu'\in B} \inf_{\bm\lambda \in Alt_{\bm\nu'}} \sum_i \overline{w}^\star_i(B) \frac{(\mu'_i-\lambda_i)^2}{2\sigma^2}\right)^{-1} &\\
	&=\left( \inf_{\bm\nu'\in B} \inf_{\bm\lambda \in Alt_{\bm\nu'}} \sum_i w^\star_i(\bm b) \frac{(\mu'_i-\lambda_i)^2}{2\sigma^2}\right)^{-1}&\\
	&=\left(  \inf_{\bm\lambda \in Alt_{\bm b}} \sum_i w^\star_i(\bm b) \frac{(b_i-\lambda_i)^2}{2\sigma^2}\right)^{-1}&\hspace{-5em}\text{by Equation~\eqref{eq:bworst}}\\
	&= T^\star(\bm b)&
\end{align*}
hence $\overline{T}(B) \leq \max_{\bm\nu\in B} T^\star(\bm\nu)$. By definition, we have the other inequality, and we conclude.
\end{proof}



\lemconstrBbai*

We prove the result separately for top-$k$ and TBP. In both cases, we give a certain mean vector $\bm b$, and then we show it satisfies the premise of Lemma~\ref{lem:wbwst}, then we use that result to show that $\overline{T}^\star(\mathcal{B}_\infty(\bm\mu,\varepsilon))=T^\star(\bm b)$.

\begin{proof}[Proof of Lemma~\ref{lem:constrBbai} for top-$k$]
	Assume without loss of generality that the arms are well ordered, $\mu_1\geq \mu_2\geq \cdots \geq \mu_K$.
	
	If $\mu_{k}-\mu_{k+1}\leq 2\varepsilon$, then there exists $\bm b\in \mathcal B_{\infty}(\bm\mu, \varepsilon)$ such that $b_k=b_{k+1}$. $\overline{T}^\star(\mathcal B_{\infty}(\bm\mu, \varepsilon))\geq \max_{\bm\nu'\in \mathcal B_{\infty}(\bm\mu, \varepsilon)} T^\star(\bm\nu')\geq T^\star(\bm b)=+\infty$, therefore \[\overline{T}^\star(\mathcal B_{\infty}(\bm\mu, \varepsilon))= \max_{\bm\nu'\in \mathcal B_{\infty}(\bm\mu, \varepsilon)} T^\star(\bm\nu')\: .\]
	
	When $\mu_k-\mu_{k+1}>2\varepsilon$, define \[\left\{ \begin{aligned} &b_i= \mu_i-\varepsilon \qquad\text{if }i\leq k \\ &b_i =\mu_i +\varepsilon \qquad\text{if }i\geq k+1\end{aligned}\right. \]
	and, for any $\bm\nu'$, \[w_{\bm\nu'} =\argmax_{w\in \Sigma_K} \inf_{\bm\lambda\in Alt_{\bm\nu'}} \sum_i w_i \frac{(\mu_i'-\lambda_i)^2}{2\sigma^2}.\] 
	
	
	Let there be some $\bm{\nu}'\in \mathcal B_{\infty}(\bm\mu, \varepsilon)$. Then for $i\leq k$, $\mu'_{i} \geq \mu_i-\varepsilon=b_{i}$ and for $i\geq l+1$, $\mu'_i \leq \mu_i +\varepsilon = b_i$, and $Alt_{\bm\nu'}=Alt_{\bm b}$.
	
	We know from Lemma~\ref{lem:baiw} \begin{equation}\label{eq:argmin} \min_{\bm{\lambda}\in Alt_{\bm b}}\sum_i  w_i(\bm{b})\frac{(\mu'_i-\lambda_i)^2}{2\sigma^2} = w_a(\bm b)\frac{(\mu'_a-\mu'_{aj})^2}{2\sigma^2}+w_j(\bm b)\frac{(\mu'_j-\mu'_{aj})^2}{2\sigma^2}\end{equation} for some $a\leq k<k+1\leq j$, and $\mu'_{aj}= \frac{w_a(\bm{b})}{w_a(\bm{b})+w_j(\bm{b})}\mu'_a + \frac{w_j(\bm{b})}{w_a(\bm{b})+w_j(\bm{b})}\mu'_j$. 
	

	\begin{itemize}
		\item If $\mu'_{aj}\in (b_a,\mu'_a)$, then \begin{align*}
			w_a(\bm{b}) (\mu'_a-\mu'_{aj})^2+w_j(\bm{b}) (\mu'_j-\mu'_{aj})^2 &\geq 0+w_j(\bm{b}) (\mu'_j-b_a)^2 \\
			&\geq w_a(\bm{b})(b_a-b_a)^2+w_j(\bm{b})(b_j-b_a)^2\\
		\end{align*}
		\item If $\mu'_{aj}\in (b_j,b_a)$, \begin{align*}
			w_a(\bm{b}) (\mu'_a-\mu'_{aj})^2+w_j(\bm{b}) (\mu'_j-\mu'_{aj})^2 &\geq w_a(\bm{b})(b_a-\mu'_{aj})^2 +w_j(\bm{b})(b_j-\mu'_{aj})^2
		\end{align*}
		\item If $\mu'_{aj} \in (\mu'_j,b_j)$, \begin{align*} w_a(\bm{b}) (\mu'_a-\mu'_{aj})^2+w_j(\bm{b}) (\mu'_j-\mu'_{aj})^2 &\geq w_a(\bm{b})(\mu'_a-b_j)^2+0 \\
			&\geq w_a(\bm{b}) (b_a-b_j)^2 +w_j(\bm{b})(b_j-b_j)^2\end{align*}
	\end{itemize}
	In all three cases, \begin{align*} w_a(\bm{b}) (\mu'_a-\mu'_{aj})^2+w_j(\bm{b}) (\mu'_j-\mu'_{aj})^2 &\geq \inf_{\lambda\in [b_j,b_a]} w_a(\bm{b}) (b_a-\lambda)^2+w_j(\bm{b})(b_j-\lambda)^2\\
		&\geq \inf_{\bm{\lambda}\in Alt_{\bm b}} \sum_i w_i(\bm{b}) (b_i-\lambda_i)^2
	\end{align*}
	and therefore, by Equation~\eqref{eq:argmin}, \[\forall \bm\nu' \in B_{\infty}(\bm\mu, \varepsilon),\; \inf_{\bm{\lambda}\in Alt_{\bm b}} \sum_i w_i(\bm{b}) \frac{(\nu_i-\lambda_i)^2}{2\sigma^2}\geq \inf_{\bm{\lambda}\in Alt_{\bm b}} \sum_i w_i(\bm{b}) \frac{(b_i-\lambda_i)^2}{2\sigma^2} \] and therefore \begin{equation} \inf_{\bm\nu\in\mathcal B_{\infty}(\bm\mu, \varepsilon)}\inf_{\bm{\lambda}\in Alt_{\bm b}} \sum_i w_i(\bm{b}) \frac{(\nu_i-\lambda_i)^2}{2\sigma^2}\geq \inf_{\bm{\lambda}\in Alt_{\bm b}} \sum_i w_i(\bm{b}) \frac{(b_i-\lambda_i)^2}{2\sigma^2}\end{equation}
	
We can thus apply Lemma~\ref{lem:wbwst}, and conclude

\[\overline{T}(\mathcal B_{\infty}(\bm\mu, \varepsilon)) = \max_{\bm\nu\in \mathcal B_{\infty}(\bm\mu, \varepsilon)} T^\star(\bm\nu) \: .\]
\end{proof}









\begin{proof}[Proof of Lemma~\ref{lem:constrBbai} for thresholding bandits]
	If for some $i$, $|\mu_i -\tau|\leq \varepsilon$, then there exists $\bm b\in \mathcal{B}_\infty(\bm\mu,\varepsilon)$ such that $b_i = \tau$. Therefore, $\overline{T}^\star(\mathcal{B}_\infty(\bm\mu,\varepsilon)) \geq \max_{\bm\nu'\in \mathcal{B}_\infty(\bm\mu,\varepsilon)} T^\star(\bm\nu')\geq T^\star(\bm b) =+\infty$, therefore \[\overline{T}^\star(\mathcal{B}_\infty(\bm\mu,\varepsilon)) = \max_{\bm\nu'\in \mathcal{B}_\infty(\bm\mu,\varepsilon)}T^\star(\bm\nu') \: .\]
	
	When $\min_k |\mu_k-\tau|>\varepsilon$, define $U=\{i\in[K]:\mu_i >\tau\}$ and $L=[K]\setminus U$. Define \[\left\{ \begin{aligned} &b_i= \mu_i-\varepsilon &\text{if }i\in U \\ &b_i =\mu_i +\varepsilon &\text{if }i\in L.\end{aligned}\right. \] and for any $\bm\nu'$, $w_{\bm\nu'} =\argmax_{w\in \Sigma_K} \inf_{\bm\lambda \in Alt_{\bm\nu'}} \sum_i w_i\frac{(\mu_i'-\lambda_i)^2}{2\sigma^2}$.
	
	
	
	
	Let there be some $\bm\nu\in \mathcal{B}_\infty(\bm\mu,\varepsilon)$.	We know $\min_{\bm\lambda \in Alt_{\bm b}} \sum_i w_i(\bm b)\frac{(\mu'_i-\lambda_i)^2}{2\sigma^2}=w_j(\bm b) \frac{(\mu'_j-\tau)^2}{2\sigma^2}$ for some $j$. 
	
	For all $i\in U$, $\mu'_i \geq \mu_i-\varepsilon = b_i>\tau$; for all $i\in L$, $\mu'_i \leq \mu_i+\varepsilon = b_i<\tau$. Therefore, \[w_j(\bm b) \frac{(\mu'_j-\tau)^2}{2\sigma^2}\geq w_j(\bm b) \frac{(b_j-\tau)^2}{2\sigma^2}\geq \inf_{\bm\lambda \in Alt_{\bm b}} \sum_i w_i(\bm b) \frac{(b_i-\lambda_i)^2}{2\sigma^2}\]
	We thus have $\forall \bm\nu' \in \mathcal{B}_\infty(\bm\mu,\varepsilon)$, $\inf_{\bm\lambda\in Alt_{\bm b}} \sum_i w_i(\bm b)\frac{(\mu'_i-\lambda_i)^2}{2\sigma^2} \geq \inf_{\bm\lambda \in Alt_{\bm b}} \sum_i w_i(\bm b) \frac{(b_i-\lambda_i)^2}{2\sigma^2}$, and \[\inf_{\nu\in\hat{B}_r}\inf_{\bm\lambda\in Alt_{\bm b}} \sum_i w_i(\bm b)\frac{(\nu_i-\lambda_i)^2}{2\sigma^2} \geq \inf_{\bm\lambda \in Alt_{\bm b}} \sum_i w_i(\bm b) \frac{(b_i-\lambda_i)^2}{2\sigma^2}\] and by Lemma~\ref{lem:wbwst},
\[\overline{T}(\mathcal B_{\infty}(\bm\mu, \varepsilon)) = \max_{\bm\nu\in \mathcal B_{\infty}(\bm\mu, \varepsilon)} T^\star(\bm\nu) \: .\]
\end{proof}

%%%%%%%%%%%%%%%%%%%%%%%%%%%%%%%%%%%%%%%%%%%%%%%%%%%%%%%%%%%%%%%%%%%%%%%%%%%%%%%
%%%%%%%%%%%%%%%%%%%%%%%%%%%%%%%%%%%%%%%%%%%%%%%%%%%%%%%%%%%%%%%%%%%%%%%%%%%%%%%


\end{document}



%%%%%%%%%%%%%%%%%%%%%%%%%%%%%%%%%%%%%%%%%%%%%%%%%%%%%%%%%%%%%%%%%%%%%%%%%%%%%%%
%%%%%%%%%%%%%%%%%%%%%%%%%%%%%%%%%%%%%%%%%%%%%%%%%%%%%%%%%%%%%%%%%%%%%%%%%%%%%%%


\end{document}