Capability evaluations are required to understand and regulate AI systems that may be deployed or further developed. 
Therefore, it is important that evaluations provide an accurate estimation of an AI system’s capabilities. 
However, in numerous cases, previously latent capabilities have been elicited from models, sometimes long after initial release. 
Accordingly, substantial efforts have been made to develop methods for eliciting latent capabilities from models. 
In this paper, we evaluate the effectiveness of capability elicitation techniques by intentionally training \textit{model organisms} -- language models with hidden capabilities that are revealed by a password. We introduce a novel method for training model organisms, based on circuit-breaking, which is more robust to elicitation techniques than standard password-locked models.
We focus on elicitation techniques based on prompting and activation steering, and compare these to fine-tuning methods. Prompting techniques can elicit the actual capability of both password-locked and circuit-broken model organisms in an MCQA setting, while steering fails to do so. For a code-generation task, only fine-tuning can elicit the hidden capabilities of our novel model organism. Additionally, our results suggest that combining techniques improves elicitation. Still, if possible, fine-tuning should be the method of choice to improve the trustworthiness of capability evaluations. We publish our code and results at \url{https://github.com/Felhof/sandbagging-elicitation}