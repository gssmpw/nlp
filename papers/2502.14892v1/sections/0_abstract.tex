\begin{abstract}
Predicting when to initiate speech in real-world environments remains a fundamental challenge for conversational agents. We introduce \frameworkname, a novel framework for real-time speech initiation prediction in egocentric streaming video. By modeling the conversation from the speaker’s first-person viewpoint, \frameworkname is tailored for human-like interactions in which a conversational agent must continuously observe its environment and dynamically decide when to talk.

Our approach bridges the gap between simplified experimental setups and complex natural conversations by integrating four key capabilities: (1) first-person perspective, (2) RGB processing, (3) online processing, and (4) untrimmed video processing. We also present YT-Conversation, a diverse collection of in-the-wild conversational videos from YouTube, as a resource for large-scale pretraining. Experiments on EasyCom and Ego4D demonstrate that \frameworkname outperforms random and silence-based baselines in real time. Our results also highlight the importance of multimodal input and context length in effectively deciding when to speak. Code and data are available at \href{https://jun297.github.io/EgoSpeak/}{website}. 
\end{abstract}







