


% \vspace{-30pt} 
\newglossaryentry{articulation}{
   name=articulation,
   description={defines how smoothly notes are played. One type of articulation marking is \gls{legato}}
}


\newglossaryentry{homophonic}{
   name=homophonic,
   description={music refers to a type of musical \gls{texture} which has one line of melody being played by multiple lines simultaneously}
}

\newglossaryentry{key}{
   name=musical key,
   description={refers to the main notes, scales and chords which a section of music is built from}
}


\newglossaryentry{legato}{
   name=legato,
   description={is a playing technique where a musician smoothly connects notes. The opposite is \textit{staccato} where notes played are detached}
}


\newglossaryentry{monophonic}{
   name=monophonic,
   description={music refers to a type of musical \gls{texture} which consists of a single line or melody}
}


\newglossaryentry{ornament}{
   name=ornaments,
   description={are added notes that are not essential to carry the overall line of the melody (or harmony), but serve instead to decorate a musical line}
}

\newglossaryentry{phrasing}{
   name=phrasing,
   description={is the method in which a musician shapes passages of music to allow for expression}
}


\newglossaryentry{polyphonic}{
   name=polyphony,
   description={refers to a type of musical \gls{texture} which consists of two or more simultaneous lines of independent melody}
}


\newglossaryentry{rubato}{
   name=rubato,
   description={refers to \gls{tempo} deviations musicians may add for expression}
}


% \newglossaryentry{chamber music}{
%    name=chamber music,
%    description={refers to music played by a small group of musicians, such as a string quartet}
% }


\newglossaryentry{semiquaver}{
   name=semiquavers,
   text=semiquaver,
   description={are notes played for $\frac{1}{16}$ of the duration of a semi-breve (the second-longest note value used in modern staff notation). This means they tend to have rather short durations (though this also depends on the underlying \gls{tempo})}
}
\newglossaryentry{demisemiquavers}{
   name=demisemiquavers,
   description={are notes played for $\frac{1}{32}$ of the duration of a semi-breve (the second-longest note value used in modern staff notation). This means they tend to have  very short durations (though this also depends on the underlying \gls{tempo})}
}


\newglossaryentry{sustain pedal}{
   name=sustain pedal,
   description={is a keyboard technique for allowing the strings of the instrument to openly resonate, rather than being damped}
}


\newglossaryentry{texture}{
   name=texture,
   description={refers to the structure of different layers of sound in a piece, and the relationship between them}
}

\newglossaryentry{timbre}{
   name=timbre,
   description={characterises the different acoustic sound qualities that distinct instruments exhibit}
}




\newglossaryentry{glissandi}{
   name=glissando,
   description={is a technique where the instrumentalist slides between notes}
}

\newglossaryentry{tempo}{
   name=tempo,
   description={is the speed of a performance. A tempo marking gives the composer's intended tempo, measured in beats per minute}
}


% \newglossaryentry{pizzicato}{
%    name=pizzicato,
%    description={ is a musical direction for string instrumentalists to pluck the string, as opposed to \gls{arco}}
% }
% \newglossaryentry{arco}{
%    name=arco,
%    description={is a musical direction for instruments in the string family indicating that the bow should be used (as opposed to \gls{pizzicato})}
% }

\newglossaryentry{dynamics}{
   name=dynamics,
   description={define the subjective loudness at which the composer intended marked music passages to be played}
}

\newglossaryentry{harmonic progression}{
   name=harmonic progression,
   description={(or chord progression) is the movement from one chord to another which builds a structural foundation for harmony and \gls{key}}
}

\newglossaryentry{bars}{
   name=musical bars,
   text=bar,
   description={are segments of music bounded by vertical lines known as bar lines. Bars contain a constant number of beats, determined by the piece's \textit{time signature}}
}

\newglossaryentry{rhythm}{
   name=rhythm,
   description={is the arrangement of sounds and silences in time in a perceptibly structured way, often containing a pulse or beat}
}