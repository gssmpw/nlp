% Ensure that chapter* is unaffected and uses default settings
\titleformat{name=\chapter,numberless}
  {\centering\normalfont\Huge\bfseries} % You can customize this if needed
  {}
  {0pt}
  {}
\titlespacing*{name=\chapter,numberless}
  {0pt}{15pt}{40pt} % Adjust these values as needed to match the default or desired spacing

\chapter*{\centering {Musical Score Following using Statistical Inference}}
\addcontentsline{toc}{chapter}{Technical Abstract}
\vspace{-30pt} 
{
  \centering
  % {\bf \uppercase{Technical Abstract}}\par
  \vspace{0.17in}
  \textbf{Josephine Cowley} \par
  St John's College, Cambridge \par
  Supervisor: Professor Simon J. Godsill\par  
}

\vspace{0.15in}

\subsection*{\centering \uppercase{Technical Abstract}}
Musical score following is the real-time mapping of a performance to corresponding locations in a musical score. Score following can be used in a variety of applications including automatic page turning and real-time accompaniment. 
% Motivated by Wilson and Adams's 2013 paper introducing Spectral Mixture kernels for Gaussian Process (GP) regression, we present a 
This report presents a novel approach for score following motivated by Wilson and Adams's 2013 paper, which introduces Spectral Mixture (SM) kernels for Gaussian Process (GP) regression. Since the SM kernel is derived from a Mixture of Gaussians in the frequency domain, it is particularly suitable for modelling the superposed power spectra of musical notes, in which energy is concentrated at multiples of the fundamental frequency of each note.   
% Since the SM kernel is particularly suited for modelling musical signals as the superpositions of individual notes' frequency components.
% This paper presents a novel approach for score following that utilises a Gaussian Process (GP) with the Spectral Mixture kernel introduced by Wilson and Adams in 2013. Our score follower begins by using a GP to statistically infer the musical notes played during 800-sample `audioframes' ($\approx$18 ms) of solo piano music. 
% Motivated by Wilson and Adams's 2013 paper introducing Spectral Mixture kernels for Gaussian Process (GP) regression, we present a novel approach for score following that uses a GP to statistically infer the musical notes played during 800-sample `audioframes' ($\approx$18 ms) of solo piano music. 
Our score follower begins by using a GP to statistically infer the musical notes played during 800-sample `audioframes' ($\approx$18 ms) of solo piano music. 
These predictions are then used in a duration-dependent Hidden Markov Model to predict the most likely score positions in real time. Our two-stage approach achieves successful score following not only on four-part hymns arranged for keyboard, but also on pieces for the violin, oboe, and flute. This showcases the powerful and flexible nature of GPs for statistical inference on musical audio signals. Given the success of this project, we contribute to the literature a first proof of concept of the application of GPs in score
following, and more broadly, in online Music Information Retrieval (MIR) tasks. This project also contributes a working score follower product that renders score position in real time using an adapted open-source user interface. Areas for future work include improving accuracy on repeated notes and during heavy use of sustain pedal, adapting to minor deviations from the score, and modelling multi-instrument works.

%as well as a demonstration of the potential of GPs in the wider context of online Music
%Information Retrieval (MIR) tasks. This project also contributes a working score follower product that can be used in conjunction with any application which receives musical notes as states.

\subsection*{\centering Report Structure}

The report has five parts: \textit{Introduction} (chapters 1–3); \textit{Background} (chapters 4–5);  \textit{Statistical Inference} (chapters 6–8);  \textit{Implementation} (chapters 9–10); and  \textit{Discussion} (chapter 11). \\

\textbf{Part I, \textit{Introduction}}, introduces central concepts for our project. In chapter 1, we open with a brief description of score following and our motivations, as well as a statement of the project goal. In chapter 2, we present important musical preliminaries, exploring the physical phenomena and underlying score features that affect how music is perceived. We also introduce some common standards for music notation. In chapter 3, we formally define the problem of score following. We then take inspiration from manual score following, contrasting it with the challenges faced by automatic score followers. We close with a general framework for score followers, which sets us up for the literature review in Part II and designing our own score follower in Part III.   \\

\textbf{Part II, \textit{Background}}, sets the scene for score following. In chapter 4, we detail several key applications for score following alongside some commercial products. We then carry out a literature review. First, we investigate general score following by chronologically tracing the development of score following over the last four decades. We then examine existing methods for pitch detection and note their drawbacks to motivate our use of GPs. We then perform a literature review of GPs in the context of MIR. Finally, we outline the key recurring challenges from this research. In chapter 5, we lay out the mathematical notation and framework for GPs and detail the Spectral Mixture kernel, which we use in our GP model. \\

\textbf{Part III, \textit{Statistical Inference}}, details the statistical and algorithmic methods we use for score following. In chapter 6, we present a general methodological overview and introduce the high-level framework of our score follower, which contains two critical stages: \textit{GP Model Specification} and \textit{Real-Time Alignment}.  In chapter 7, we commence with Stage 1: \textit{GP Model Specification}. The tasks of this chapter include analysing and modelling musical signals, designing a covariance matrix, tuning our GP's hyperparameters and formulating a log marginal likelihood (LML) function. Finally, we present the results from this chapter, which were largely successful, showing accurate underlying frequency prediction. Chapter 8 focuses on Stage 2: \textit{Real-Time Alignment} and details the statistical inference methods for score following. We use a Hidden Markov Model (HMM) to represent audioframes as observed variables that depend upon latent states (i.e., the underlying notes from the score). We use the LML function from Chapter 7 for the HMM's emission probabilities, and develop a state duration model that uses rhythmic information from the score for transition probabilities. We then implement an online `Windowed' Viterbi algorithm to achieve efficient and near-optimal score following by computing the most likely sequence of latent states. To close, we discuss the results, which were largely positive except for tests where there was excessive use of sustain pedal. \\

\textbf{Part IV, \textit{Implementation}}, presents the final product. In chapter 9, we first outline the framework of the implemented score follower and its modes of operation. We then describe the final product in terms of its components, detailing constituent system architectures and design justifications. In chapter 10, we present the key results of our  score follower, evaluating its performance on several different test cases. Our score follower was able to follow an intermediate monophonic piano piece, a simple 2-part piece and even a 4-part hymn. The follower also performed excellently on various instruments, including the violin, flute and oboe, demonstrating the flexibility of our GP model. The main limitations identified were use of sustain pedal, deviations from the score and repeated notes or passages. \\

\textbf{Part V, \textit{Discussion}}, consists of only chapter 11, which summarises our project. We detail key findings from Parts III and IV, covering the main contributions of this report. We wrap up by assessing areas for future work and conclude with some final reflections.
