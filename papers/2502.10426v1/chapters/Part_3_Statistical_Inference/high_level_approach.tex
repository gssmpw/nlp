\chapter{High-level Approach}{\label{ch:high_level_approach}}
In this chapter, we summarise the overall framework of our score follower. This consists of four stages, of which the third stage is most substantial. Hence, we discuss that stage in the following two chapters. We also provide an overview of the methodology for implementation and evaluation. Finally, we cover the use of programming languages, data sources and software.  


\section{Score Follower Framework}{\label{section:score_follower_framework}}
Our score follower comprises four main steps (as presented in \hyperref[fig:high_level]{Figure \ref*{fig:high_level}}):
\begin{enumerate}
    \item \textbf{Score Feature Extraction:} Important features from the score are extracted, including note onsets and pitches.   
    \item \textbf{Audioframe Extraction:} To perform real-time audio processing, we extract \textit{audioframes}, which are groups of contiguous audio samples, typically between 800 and 2000 samples long.
    \item \textbf{Score Following using Statistical Inference (Stages 1 and 2):} Score following is largely broken down into two statistical inference problems. In Stage 1: \textit{GP Model Specification}, we find the most probable underlying notes of isolated audioframes using a novel Gaussian Process (GP) model. In Stage 2: \textit{Real-Time Alignment}, we consider cumulative results from stage 1 and infer score location over time using Hidden Markov Models and state duration models.
    \item \textbf{Rendering of Score Position:} Using the results from step 3, score positions are displayed by a moving marker upon a digital score.  
\end{enumerate}


\begin{figure}[H]
    \centering
    \includegraphics[width = 0.7\textwidth]{figs/Part_3_Implementation/Stage_2_Alignment/high_level.png}
    \caption{Diagram depicting the high-level score follower framework, where numbers represent the steps being completed.}
    \label{fig:high_level}
\end{figure}

\section{Overview of Methodology}
We began with step 3 and its two stages, since these constitute the bulk of the challenges. The detailed methods for these stages are respectively outlined in \hyperref[ch:model_selection]{chapter \ref*{ch:model_selection}} (Stage 1: \textit{GP Model Specification}) and \hyperref[ch:alignment]{chapter \ref*{ch:alignment}} (Stage 2: \textit{Real-Time Alignment}). We then approached steps 1, 2 and 4 in \hyperref[ch:implementation]{chapter \ref*{ch:implementation}}, which outlines the engineering details of how the final product brings all four steps together. 


\subsection{Evaluation}
Because there is no widely accepted quantitative or objective method for evaluating score followers, we use a qualitative evaluation method based on the subjective evaluation of trained musicians. This is justified because score followers are ultimately to be used by musicians, and should meet their expectations.   %TODO: do i need to mention the score renderer here?



\section{Programming Language}
We chose Python since it is a high-level language with a rich ecosystem for Machine Learning tasks. Packages such as \verb|numpy| contain a significant amount of optimised code in C, which can mitigate the performance reductions resulting from the high-level, interpreted nature of Python. Though there exist Python packages that implement GPs, such as \verb|GPy|, \verb|GPyTorch|, and \verb|GPFlow|, all of our novel GP models were implemented without these packages to allow for flexibility. \verb|Jupyter| notebooks were used as an interactive development environment, especially during the early stages of modelling and hyperparameter optimisation. 




\section{Sources}
\textbf{Kunstder Fugue:} MIDI files were downloaded from the website \textit{Kunstderfugue},\footnote{\href{https://www.kunstderfuge.com/}{https://www.kunstderfuge.com/}} which is a large resource for Classical Music in MIDI format. 

\textbf{Musescore:} MIDI and MusicXML files were downloaded from \textit{Musescore},\footnote{\href{https://musescore.com/sheetmusic?text=bach\%20fugue\%201}{https://musescore.com/sheetmusic?text=bach\%20fugue\%201}} a free sheet music website. 

\textbf{YouTube:} Experimental piano recordings were made by the author, and other recordings were collected on \textit{YouTube}.\footnote{\href{https://www.youtube.com/}{https://www.youtube.com/}}

\section{Software}
\textbf{Audacity:} All performance recordings made by the author were recorded and exported using \textit{Audacity},\footnote{\href{https://www.audacityteam.org/}{https://www.audacityteam.org/}} a widely used free program for recording and editing audio. 

\textbf{Flippy Qualitative TestBench:} The open source tool \textit{Flippy Qualitative Testbench}\footnote{\href{https://github.com/flippy-fyp/flippy-qualitative-testbench/tree/main}{https://github.com/flippy-fyp/flippy-qualitative-testbench/tree/main}} was used to render our score follower and evaluate our results. 