\chapter{A Review of Score Following}{\label{ch:score_following_literature_review}}

In this chapter we set the scene for score following. We commence by presenting a few applications, primarily detailing automatic page turning (APT) and computer-aided accompaniment (CAA). We then carry out a literature review, which begins with historical approaches to general score following, followed by an overview of general pitch detection techniques. The literature review then turns to Gaussian Processes (GPs) in Music Information Retrieval (MIR). Finally, we consider challenges that have been identified in the literature.   

\section{Applications and Commercial Products}
Score following is used in a plethora of potential applications and commercial products:
% Most score followers are designed for a specific application. This helps designers identify the key aims and challenges that the product must address, guiding them when choosing an optimal combination of approaches for each of the steps outlined in \hyperref[subsection:general_approach]{subsection \ref*{subsection:general_approach}}. Thus, understanding the primary goals of previous research provides insight to the problems encountered and explanation for the methods used. 

\subsection{Automatic Page Turning} {\label{subsection:APT}}
Manual page turning during a live performance is a highly specialised skill that demands sight-reading aptitude, constant concentration and an understanding of pre-agreed performance decisions, such as repeats. Therefore, it is not uncommon for concerts to suffer slip-ups due to page turner mistakes.  However, performances make up only a small fraction of the time that page turning is needed. Most page turning occurs in the practice room, where hiring a page turner is infeasible. Thus, page turning is a common source of frustration. \\

The demand for APT solutions is demonstrated by the myriad of commercial products available. However, none use \textit{true} score following. \textbf{Manual APT Devices} rely on gestures from the player to turn the page. Many applications use a physical device, like a foot pedal, to transmit BLE messages to the primary device which turns the page. Examples include \textit{Stomp Bluetooth 4.0}\footnote{\href{https://www.codamusictech.com/products/bluetooth-page-turner-music-pedal-for-tablets}{https://www.codamusictech.com/products/bluetooth-page-turner-music-pedal-for-tablets}} and \textit{Airturn}.\footnote{\href{https://www.airturn.com/}{https://www.airturn.com/}} Though simple and intuitive, manual APT devices distract performers and introduce a new potential point of performance interruption.  \textbf{Scrolling APT Applications} do not require the musician to initiate turns, since they employ a score-scrolling feature. The scroll rate is determined by the \gls{tempo} of a pre-recorded playback of the piece, and therefore scrolling APT applications cannot accommodate spontaneous changes in tempo or unexpected score deviations. Examples include \textit{MobileSheets}\footnote{\href{https://www.zubersoft.com/mobilesheets/}{https://www.zubersoft.com/mobilesheets/}} and \textit{Musicnotes}.\footnote{\href{https://www.musicnotes.com/apps/}{https://www.musicnotes.com/apps/}}

% TODO re add in the unlikely event of space!
% \textbf{APT in the literature:} \cite{eye_gaze_tracking} employs an eye-gaze tracking system, which uses a camera to actively monitor eye position on the score and render the display accordingly. Since musicians frequently divert their eyes away from the score (to other ensemble members, or to their own hands), this system was imperfect.

\subsection{Computer-Aided Accompaniment}
Individual musicians often depend on others—from a single keyboard accompanist to a small ensemble to a whole symphony orchestra—to make music. Computer-aided accompaniment (CAA) aims to provide the supporting lines of a piece such that a single musician can play a multi-instrument work.  Several commercial CAA products exist, including Yamaha’s\textit{CueTIME}\footnote{\href{https://hub.yamaha.com/pianos/p-digital/cuetime-the-software-that-follows-you/}{https://hub.yamaha.com/pianos/p-digital/cuetime-the-software-that-follows-you/}} which provides pre-recorded MIDI instrumentals that ‘wait’ for performer keying information. However, CueTIME is limited to certain Yamaha keyboards. 

% \subsection{Music Analysis} 
% During music analysis, score following could be used as an aid to follow a piece.  Also, if there were a visual marker, then this could be done more efficiently.  

\subsection{Other Applications}
Other score following applications include: performance cues for lighting and camera equipment in live shows; music teaching and pedagogical reasons, such as score study; guided practice, for indication of mistakes; and entertainment, such as karaoke. 

% \subsection{Performance Cues}
% During live streamed concerts, timing of camera shots depends on live music, and complicated methods are currently used to help photographers know their cues. Similarly, in the musical theatre, score following could be used as an aid for stage cues (e.g., lighting). 

% \subsection{Music Teaching and Guided Practice}
% Score following could be used to help improve a musician’s playing. For instance, it could be used as a practice aid, to indicate mistakes or tempo changes. Also, score following could be used for pedagogical reasons during score study.

% \subsection{Entertainment}
% There are many possibilities for entertainment applications. One example could interactive karaoke (using CAA).  Another could be making music concerts more engaging/educational for audience members, by showing a live score follower in a digital programme. 


 

\section{Literature Review}{\label{section:literature_review}}
\subsection{General Approaches to Score Following: Three Eras}

The history of score following can be divided into three eras \cite{lee_2022_final}. The first era largely addressed computer aided accompaniment. In 1984, Dannenberg \cite{dannenberg_1984_algorithm} and Vercoe \cite{vercoe_1984_the} independently presented string-matching approaches using pitch detection. Dannenberg’s group later developed this work to make it robust to \gls{polyphonic}, performance error, \gls{rubato} and \gls{ornament} \cite{dannenberg_1988_new}\cite{bloch_1985_realtime}. Baird went on to develop this into phrase-matching, where whole musical phrases were matched rather than individual notes \cite{baird_1990_the}\cite{baird_1993_artificial}. In 1995, Vantomme introduced the use of temporal features for situations where pitch-based methods were inadequate \cite{vantomme_1995_score}. \\

The second era of score following accommodated performance error using stochastic models. Cano et al.’s work was the first score follower to use a Hidden Markov Model (HMM) and used energy, zero-crossings and fundamental frequency as the observed emissions \cite{cano_1999_scoreperformance}. In 1999, Raphael developed a HMM which instead used spectral features \cite{raphael_1999_automatic}. This work proved seminal, forming the foundation of much research and a commercial product \cite{raphael_2006_aligning}. Dynamic Time Warping (DTW) approaches were also developed during this time in \cite{orio_2001_alignment} and \cite{dixon_2005_match}. \\

Finally, a number of different approaches have emerged since the 2010s. For instance, Shinji et al.’s group explored the use of Conditional Random Fields, and several papers have used Sequential Monte-Carlo methods (Particle Filtering) \cite{sako_2014_ryry}\cite{yamamoto_2013_robust}. Some uniquely interesting approaches include a multimodal neural network that takes images of sheet music \cite{matthiasdorfer_2016_towards} and an eye-tracking technique presented in \cite{noto_2019_adaptive}. 


\subsection{General Approaches to Pitch Detection: Three Categories}
The numerous approaches to pitch detection fall into three categories: time domain, frequency domain, and hybrid approaches \cite{McLeod2008FastAP}. One of the simplest time domain methods is zero-crossing, such as in Cooper and Ng's \cite{zero_crossing}. In 1969, Gold and Rabiner developed a parallel processing technique in \cite{Gold_Rabiner}, which was further developed as a real-time algorithm by Sukkar et al. in \cite{Sukkar}. Autocorrelation methods have been used extensively in both music and speech processing, such as in \cite{Rabiner}. In 1993, Jacovitti and Scarano used an Average Squared Difference Function \cite{Jacovitti}, which de Cheveign and Kawahara further developed in 2002 \cite{Yin}. \\

In the frequency domain, there are many spectral peak estimators, including the peak detection algorithm in McLeod and Wyvill's \cite{Mcleod} and Keiler and Marchand's parabolic fitting technique \cite{Keiler}. In 1999, Wakefield used chromagrams for vocal pitch detection \cite{wakefield99_maveba}. Constant-Q transform coefficients, introduced in \cite{Youngberg}, have been used in several more recent papers \cite{Nakamura}\cite{Chen}. Mel Frequency Ceptral Coefficients were explored in \cite{Logan2000MelFC} for music, rather than in the field of speech processing where it is widely used. \\

Finally, there are several hybrid and other pitch detection techniques, including wavelet transforms \cite{7d015f6ce44c4b199dae7db41389965a}, Cepstrum  analysis \cite{Noll}, and Non-negative Matrix Factorisation \cite{godsill_2006_bayesian}.\\

A common drawback, especially among time domain approaches, is that they do not provide statistical likelihoods for detected pitches. This makes their results difficult to use in higher-level statistical inference models, like those used in score following. This motivates the investigation of Gaussian Processes (GPs) as a time domain method that yields likelihoods of pitches, which can be naturally incorporated in a Bayesian framework for score following.       

\subsection{Gaussian Processes for Music Information Retrieval}
Although there have been several explorations of using GPs to model audio signals \cite{alvarado_2016_gaussian}\cite{wilkinson_2019_gaussian}, this is a novel exploration of using GPs for score following. In 2014, Turner and Sahani used GPs for inferring superposed source components \cite{turner_2014_timefrequency}. Similarly, Liutkus et al. performed undetermined source separation involving GP regression \cite{liutkus_2011_gaussian}. GPs have also been used for jointly estimating the spectral envelope and fundamental frequency of a speech signal, as well as time-domain audio source separation \cite{yoshii_2015_masataka}\cite{yoshii_2013_beyond}. Alvarado and Stowell are particularly relevant contributors to this field. In 2016, they used GPs for pitch estimation and inferring missing segments in a polyphonic recording \cite{alvarado_2016_gaussian}. Their later work presented sparse GPs for source separation using spectrum priors in the time-domain \cite{miscdo_2019_sparse}.



 \section{Challenges}{\label{section:challenges}}
A number of challenges recur throughout the literature:
\begin{itemize}
    \item Player deviation from the score may occur due to mistakes or improvisation. 

    \item Background noise and acoustical effects such as resonating strings can cause issues for pitch detection. 
    \item Score following must work in real-time, and therefore must not overload the CPU. 
    \item Thick \gls{texture}s (e.g. \gls{polyphonic}) increase the complexity of the problem, since there are more simultaneous pitches to detect. 
    \item Evaluation of score followers is not well-standardised. Hence, it is difficult to compare the performance of a score follower to others in the literature.   
\end{itemize}

