\chapter{Music Preliminaries}{\label{ch:music_preliminaries}}
% \epigraph{\textit{What is best in music is not to be found in the notes.}}{Gustav Mahler}

An investigation of score following requires an understanding of the physical properties of music and standard notation practices. This chapter first outlines the perceptual properties of sound and discusses the physical bases for the distinction between musical and non-musical sounds. Then, we asses which of these features are present in the score and are therefore relevant to score following. Finally, we introduce three different standards of musical notation and settle on using symbolic formats like MIDI and MusicXML in this application.


\section{Perception of Music}{\label{subsection:perception_of_music}}
When a trained musician listens to a performance, they make sense of the music by characterising pitch, duration, volume and \gls{timbre} \cite{donnelly_2015_learning}. These features can be explained in terms of underlying physical properties, as well as features from the score and player interpretation. 

\subsection{Perceptual Properties and Physical Correlates}{\label{section:duration}}

\textbf{Pitch} is the perceptual quality of music that makes notes sound higher or lower. Although pitch is determined by \textit{frequency spectra} (a physical phenomenon), different frequency spectra may be perceived as the same pitch. When a pitched instrument plays a note, we perceive the spectrum of frequencies as a single pitch. This frequency spectrum contains energy at the fundamental frequency $f_0$ and at \textit{overtones} (or \textit{partials}). \\ % TODO - put back? -- , the joint set of which makes up partials

\textbf{Duration} is the length of a note’s amplitude-time waveform shape, which is also called the \textit{temporal envelope}. Percussive instruments like conga drums exhibit immediate amplitude decay, whereas sustaining instruments have slowly varying amplitude. The first row of graphs in \hyperref[fig:musical_properties]{Figure \ref*{fig:musical_properties}} shows different instruments' temporal envelopes. \\

Due to psychoacoustics, the perception of \textbf{loudness} is not only dependent on energy or amplitude, but also on frequency and waveform shape \cite{donnelly_2015_learning}. \\

A musical note's \textbf{timbre}, or `tone colour', is associated with how its sound is produced. For instance, a flute, bowed violin, and plucked violin all sound different. One underlying property is the shape of an instrument's \textit{power spectrum}, or its \textit{spectral envelope}. Unlike non-pitched instruments like untuned percussion, pitched instruments exhibit a \textit{harmonic series}, which concentrates energy at multiples of $f_0$. Even then, pitched instruments sound different from one another because the distribution of energy at those harmonics varies greatly. For instance, some instruments exhibit more pronounced even-numbered harmonics. The variation of spectral envelopes over time can be represented by \textit{spectrograms}, like those in the second row of \hyperref[fig:musical_properties]{Figure \ref*{fig:musical_properties}}. Additionally, a note's \textit{temporal envelope} influences its timbre.  \\

All four of these perceptual properties are also affected by the acoustic environment, since resonance, position and noise strongly influence our auditory perception. \\

\begin{figure}[H]
    \centering
    \includegraphics[width=0.75\textwidth]{figs/Part_1_Introduction/musical_properties.png}
    \caption{Examples from \cite{godsill_2006_bayesian} of different instruments' temporal envelopes (top) and spectrograms (bottom). Spectrograms represent the time-varying magnitude spectra—i.e. the modulus of the short-time Fourier transform. Audio data and images are from the RWCP Instrument samples database.}
    \label{fig:musical_properties}
\end{figure}

\subsection{Influences from the Score}{\label{subsection:score_influences}}
We now turn to how the score influences the above perceptual features in music. We also consider the effects of unexpected deviations from the score, highlighting the limitations of using certain features for score following. \\

A score dictates which \textbf{pitches} are intended to be present by specifying musical notes at times. However, the pitches present in a performance may deviate due to performer mistakes. Additionally, performers may embellish a piece with \gls{ornament} or even improvisation. Finally, playing techniques such as \gls{sustain pedal} and \gls{legato} causes pitches from previous notes to continue to sound even after their end in notation. \\

Similar to pitch, note \textbf{durations} are strongly influenced by the score, which specifies \gls{rhythm}, \gls{tempo} and \gls{articulation}. Although some scores have exact tempo markings, many only have qualitative indications (e.g., \textit{Andante} means `walking pace'). Additionally, musicians are often expected to add \gls{rubato} for expression, which distorts rhythm in unspecified ways.\\

Composers specify \gls{dynamics} in the score to describe the intended \textbf{loudness} of certain notes. However, there is no objectively correct loudness given some dynamic. Rather, the loudness of a note is determined by the performer's \textit{interpretation} of those dynamics. Furthermore, musicians alter loudness through \gls{phrasing} or articulation for expression. Thus, loudness differs from pitch and duration in that there is no quantitative representation in the score corresponding to the perceptual feature in question.  \\

% \textbf{Loudness} is influenced by the \gls{dynamics} indicated in the score, but musicians will also alter loudness through \gls{phrasing} or articulation for expression. Importantly, there is no mapping between dynamics and loudness, since dynamics are relative indications and loudness depends greatly on acoustics. This hampers the relevance of loudness to score following. \\

Similar to loudness, composers specify \textbf{\gls{timbre}} with performance directions, which are even more of a matter of subjective interpretation. One exception is that in a multi-instrument work, the choice of instrument predictably influences the resulting timbre. However, since we focus on the solo piano, we cannot make use of this information.\\ 
% Finally, \textbf{\gls{timbre}} is seldom explicitly specified in the score, and like loudness, there is no mapping between timbre indications and spectral/temporal envelopes. Thus, we cannot reliably use timbre for locating score position.\\

Overall, since only pitch and duration correlate strongly with the score, they are the only features we can reliably use to determine score position. This constrains our methods for developing a score follower, and also guides our choice of score notation formats.

\section{Music Notation}
\subsection{Modern Staff Notation}
There are many forms of musical notation, but \textit{modern staff notation} (or \textit{Western music notation}) is commonly used in different genres around the world. We therefore use this notation throughout this report, as seen in \hyperref[fig:score_follower]{Figure \ref*{fig:score_follower}}. 
% TODO rework this section
Even in the digital age, sheet music is stored in visual formats via PDFs and images. However, most score followers do not work directly with these formats,\footnote{Only recently, due to advances in deep learning, have there been attempts to use image processing for alignment directly from sheet music \cite{10.3389/fcomp.2021.718340}.} but rather \textit{symbolic} representations of sheet music like MIDI or MusicXML. 

\subsection{MIDI Notation}{\label{subsection:midi}}
MIDI (\textit{Musical Instrument Digital Interface}\footnote{ \href{https://midi.org/midi-2-0}{https://midi.org/midi-2-0}}) is a technical standard for electronic instruments to communicate. It is also used to efficiently represent both audio recordings and certain score data. MIDI reduces score information down to a few salient features which correspond to the four perceptual features except \gls{timbre}: MIDI note numbers (pitch), note velocity (loudness), and note onsets and offsets (duration), meaning file sizes can be very small. Other score information, like layout and performance indications, is lost.

\subsection{MusicXML}
\textit{MusicXML}\footnote{\href{https://www.musicxml.com/}{https://www.musicxml.com/}} is a feature-wise superset of MIDI. However, MusicXML can also contain information such as the layout of a score in modern staff notation and performance indications. 

\subsection{Choice of Notation}{\label{subsection:choice_notation}}
Despite the expressive benefits of the MusicXML format over MIDI, we choose to use MIDI as the input for our score follower, since it is still the most universally available symbolic format for scores. Moreover, as we have established above, there are only two properties of the score we need for score following—namely pitch and duration—both of which are fully specified in MIDI. However, displaying a fully detailed score requires more information about the score layout and performance directions, and this information is only available in MusicXML. Therefore, the score renderer (see \hyperref[section:renderer]{section \ref*{section:renderer}}) uses MusicXML, which allows conversion to a fully detailed score in modern staff notation. Thus, after connecting the score follower to the score render, we can easily allow trained musicians to evaluate the score follower in modern staff notation.



% However, the score renderer uses MusicXML format (see \hyperref[section:renderer]{section \ref*{section:renderer}}) since the application of displaying the score requires more information about the score layout and performance directions. Conversion from MusicXML to modern staff notation incurs no loss, and so this allows for displaying of modern staff notation in the final product, which allows for easy evaluation by trained musicians.  