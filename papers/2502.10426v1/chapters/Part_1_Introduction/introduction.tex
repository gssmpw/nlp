\chapter{Introduction}{\label{ch:intro}}
% \part{Introduction}
With the proliferation of portable digital devices such as iPads and tablet computers, musicians have begun to utilise these technologies as tools for music-making. Most notably, digital scores offer the practical advantage of storing a musician's entire repertoire, replacing dozens of books and scores. Free online catalogues such as \textit{IMSLP}\footnote{\href{https://imslp.org/}{https://imslp.org/}} provide free instant access to scores, and applications such as \textit{forScore}\footnote{\href{https://forscore.co/}{https://forscore.co/}} provide an integrated library, organisation and practice tool. Hence, these devices are rapidly becoming the norm in practice rooms and are even making their way into concert halls.\\

A logical next step would be for our digital devices to go beyond merely replacing sheet music. These devices have the potential to become an integral part of interactive practice, analysis and feedback. Often, the task of \textit{score following} is a first step in the real-time analysis of a musical performance. This involves mapping real-time positions of a performance to corresponding locations in a score, as depicted in \hyperref[fig:score_follower]{Figure \ref*{fig:score_follower}}. Despite a plethora of research dating back as far as the 1980s, the lack of adoption of score followers reflects the limited performance of these products. This is because score following is no trivial task, and any successful application would need to be highly accurate.
\begin{figure}[H]
    \centering
    \includegraphics[width=0.7\textwidth]{figs/Part_1_Introduction/my_score_follower.png}
    \caption{Illustration of score following: positions (blue) in the musical score are mapped to locations in a live-streamed recording (red) of \textit{O Haupt voll Blut und Wunden} arranged by Bach.}
\label{fig:score_follower}
\end{figure}

\section{Motivation}
It is critical to use a statistical approach for score following due to deviations from the score, which are caused both by inevitable performance errors and deliberate interpretative decisions. Although there exist many statistical approaches for score following (see \hyperref[ch:score_following_literature_review]{chapter \ref*{ch:score_following_literature_review}}), this work is the first attempt in the literature that uses Gaussian Processes (GPs) for this application. A crucial component of GPs is their kernel, which can be fine-tuned to encode the properties of the signals to be modelled. Accordingly, GPs offer a flexible and robust solution for incorporating prior knowledge about the highly structured properties of music. Furthermore, GPs have been successfully used in music transcription and source separation, suggesting they may prove fruitful in the task of score following \cite{miscdo_2019_sparse}. Motivated by Wilson and Adams’s paper \cite{wilson_2013_gaussian}, which introduces Spectral Mixture Kernels for GP Regression, we use GPs to infer the likelihoods that certain notes were played during an audioframe (a group of contiguous audio samples). These predictions can then be used in a second statistical model that predicts score location over time from the note likelihoods. Thus, we develop a two-stage score follower which uses a GP in its first half.

\section{Project Goal}{\label{section:project_goal}}
The overall goal of this project is to use GPs to perform score following
% that would be adequate for use in a page turner 
on simple solo piano pieces. This is not only intended as a proof of concept that GPs can be used for real-time audio processing, but it is also a first step towards a usable score following product, demonstrating how GPs can be incorporated into real user-facing applications. Because of the lack of objective criteria for the success of score followers, we will use the subjective judgement of expert musicians as our primary benchmark. This is especially justified considering that score followers are ultimately to be used by musicians, and should meet their expectations. 

\section{Note to the Reader}
This report assumes a basic understanding of music theory, which includes the musical terms defined in the \hyperref[glossary]{glossary\ref*{glossary}}. These terms are highlighted in blue and linked to their definitions in the glossary upon first occurrence per section. We also assume a basic understanding of Bayesian statistics and GPs. 