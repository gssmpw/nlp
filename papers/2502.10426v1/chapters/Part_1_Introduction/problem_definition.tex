\chapter{Problem Definition}{\label{ch:problem_definition}}


% \epigraph{\textit{In theory there is no difference between theory and practice—in practice there is.}}{Yogi Berra}
This chapter defines the problem of score following. We begin by briefly discussing why score following is a challenging problem, especially when contrasted with score alignment. Next, we examine the techniques used in manual score following (done by human musicians) and assess which are viable in automatic approaches to score following. Finally, we outline a general three-step approach to score following, which is used in the next chapter to discuss the existing literature, as well as to design our implementation.

\section{Defining Score Following}
We will adopt the general definition proposed by Arshia Cont, a prominent figure affiliated with IRCAM,\footnote{\href{https://www.ircam.fr/}{https://www.ircam.fr/}} who offers a concise definition of score following \cite{cont_arshia}:

\begin{definition}
   Score following serves as a real-time mapping interface from Audio abstractions towards Music symbols and from performer(s) live performance to the score in question.
\end{definition}


\subsection{Difficulties for Score Following}
% TODO Change technical abstract
% TODO new title?

Score following is a non-trivial task because every live performance of the same piece, even by the same performer, may vary significantly. As detailed in \hyperref[subsection:perception_of_music]{subsection \ref*{subsection:perception_of_music}}, the score at best partially determines what occurs in a performance of a work, not to mention performer error and acoustic phenomena external to the performance. The space of all possible events that can be considered performances of, say, Beethoven's \textit{Moonlight Sonata} is incredibly diverse, yet a robust score follower must be able abstract from irrelevant differences and focus on the common features derived from the score.

%At a first glance, the problem of score following seems almost contrived. Since we are presumably equipped with a fully specified score, one might believe we could use a linear mapping between performance times and score locations. However, as detailed in \hyperref[subsection:perception_of_music]{subsection \ref*{subsection:perception_of_music}}, there are complex relationships between the underlying physical properties and score, how this influences the performer, their interpretation and, ultimately, music perception. Effectively, a performance of a piece is much more than what is presented on paper, which is why score following is not a trivial task. 
% \hyperref[section:challenges]{section \ref*{section:challenges}} outlines specific challenges.




\subsection{Score Following Vs Score Alignment}{\label{subsection:score_following_v_alignment}}
Score following is not to be confused with score alignment. Score alignment assumes access to the entire performance recording, relaxing the real-time constraint of score following. Therefore, score alignment poses a lesser challenge, since computation time is not critical, and inference can be performed over all the available data. By contrast, for score following, efficiency is highly important, and estimations of score position must be constantly updated upon receiving new data.\\

% TODO: citation for many score alignment solutions
The greater flexibility of score alignment means it has been heavily researched, yielding many robust solutions. Although score following has also long been researched, and several commercial products have been developed (see \hyperref[ch:score_following_literature_review]{chapter \ref*{ch:score_following_literature_review}}), none have proved reliable enough to receive broad uptake among musicians.

\section{Manual and Automatic Score Following}{\label{section:score_following_approaches}
\subsection{Manual Score Following}{\label{subsection:manual_score_following}}
During manual score following, human musicians can effectively utilise all four perceptual features of music: not just pitch and duration, but also loudness and timbre. This is because, in addition to the score, human musicians also have knowledge of the conventions of music performance, subjective memories of loudness and timbre, emotional responses corresponding to performance directions, and in some cases even visual cues. Despite—or because of—this wealth of information, musicians do not consider all these features simultaneously. Rather, they focus their attention on prominent score features that they intuit will help locate themselves within the piece. For instance, rather than trying to track every pitch present, a musician may focus on overall melodic contour (i.e., rising and falling patterns in note pitch), certain noticeable lines (e.g. melody or bass lines), or prominent \gls{harmonic progression}s. Similarly, musicians are likely to use an innate sense of rhythm to count beats of \gls{bars}s to help identify \gls{phrasing}, abstracting from the rhythmic detail of single notes. All this requires knowledge and understanding of music and music theory.

% During score following, a human musician will listen for pitch and \gls{rhythm}, which most clearly show the influence of the score (see \hyperref[subsection:score_influences]{subsection \ref*{subsection:score_influences}}), as well as notable performance directions. However, musicians do not focus on all these features simultaneously. Rather, they focus their attention on prominent features that they intuit will help locate themselves within the piece. For instance, rather than trying to track every pitch present, a musician may focus on overall melodic contour (i.e., rising and falling patterns in note pitch), certain noticeable lines (e.g. melody or bass lines), or prominent \gls{harmonic progression}s. Similarly, musicians are likely to use an innate sense of rhythm to count beats of \gls{bars}s to help identify \gls{phrasing}, abstracting from the rhythmic detail of single notes. Identifying prominent features requires musical understanding and/or music theory knowledge.

\subsection{Automatic Score Following}
Automatic score followers face three distinct challenges compared to manual score followers. First, we have established that automatic score followers are largely limited to using only pitch and duration (see \hyperref[subsection:score_influences]{subsection \ref*{subsection:score_influences}}), since the perceptual features of loudness and \gls{timbre} do not have quantitative representations in the score and are therefore more difficult for a computer to use. Second, while skilled manual score followers use musical understanding to pay attention to only contextually relevant features, this high-level understanding is impossible to imperatively encode.\footnote{Currently, only deep learning techniques offer methods for learning these intuitions. E.g., see \cite{10.3389/fcomp.2021.718340}.} Finally, while human score followers directly perceive pitch and rhythm, computers do not, and must be programmed to infer these features from raw audio. \\

To counter some of the variation caused by irrelevant underlying physical properties (e.g. \gls{timbre}), as well as the possibility of background noise, we design our model for solo piano and assume the audio data contains no significant sounds other than the piano. 

\subsection{General Approach}{\label{subsection:general_approach}}
 Given the lesser resources available to automatic score followers, score followers typically need to follow a three-step approach: \textbf{feature extraction, similarity calculation}, and \textbf{alignment} \cite{chen_2018_an}. Feature extraction involves inferring pitch and rhythmic components from audio. This step mimics basic human auditory perception. Next, similarity calculation looks for correspondences between these features and particular locations in the score. This step requires the ability to read the score's notes and match them to pitches and rhythms. Finally, alignment makes a single best estimate of score position from a sequence of these correspondences. This requires a functional understanding of how single locations in a score `fit together' in context. Different techniques used at each step depends on various implementations, as presented in the next chapter. 

