
\documentclass{article}

\usepackage{silence} 
\WarningFilter{caption}{Unknown document class (or package)}
\WarningFilter{hyperref}{Ignoring empty anchor}

\usepackage{microtype}
\usepackage{graphicx}
\usepackage{subfigure}
\usepackage{booktabs} %

\usepackage{hyperref}


\newcommand{\theHalgorithm}{\arabic{algorithm}}


\usepackage[accepted]{icml2025}

\usepackage{src/macros}

\usepackage{amsmath}
\usepackage{amssymb}
\usepackage{mathtools}
\usepackage{amsthm}

\usepackage[capitalize,noabbrev]{cleveref}

\theoremstyle{plain}
\newtheorem{theorem}{Theorem}[section]
\newtheorem{proposition}[theorem]{Proposition}
\newtheorem{lemma}[theorem]{Lemma}
\newtheorem{corollary}[theorem]{Corollary}
\newtheorem{definition}[theorem]{Definition}
\newtheorem{assumption}[theorem]{Assumption}
\theoremstyle{remark}
\newtheorem{remark}[theorem]{Remark}

\usepackage[textsize=tiny]{todonotes}

\icmltitlerunning{Mechanisms of Projective Composition}

\begin{document}

\twocolumn[
\icmltitle{Mechanisms of Projective Composition of Diffusion Models}



\icmlsetsymbol{equal}{*}

\begin{icmlauthorlist}
\icmlauthor{Arwen Bradley}{equal,comp}
\icmlauthor{Preetum Nakkiran}{equal,comp}
\icmlauthor{David Berthelot}{comp}
\icmlauthor{James Thornton}{comp}
\icmlauthor{Joshua M. Susskind}{comp}
\end{icmlauthorlist}

\icmlaffiliation{comp}{Apple, Cupertino, CA, USA}

\icmlcorrespondingauthor{Arwen Bradley}{arwen\_bradley@apple.com}
\icmlcorrespondingauthor{Preetum Nakkiran}{p\_nakkiran@apple.com}

\icmlkeywords{Machine Learning, ICML}

\vskip 0.3in
]




\printAffiliationsAndNotice{\icmlEqualContribution} %

\begin{abstract}

We study the theoretical foundations of composition in diffusion models, with a particular focus on out-of-distribution extrapolation and length-generalization.
Prior work has shown that composing distributions
via linear score combination can achieve promising results,
including length-generalization in some cases \citep{du2023reduce,liu2022compositional}.
However, our theoretical understanding of how and why such compositions work remains incomplete. In fact, it is not even entirely clear what it means for composition to ``work''.
This paper starts to address these fundamental gaps.
We begin by precisely defining one possible desired result of composition, which we
call \emph{projective composition}.
Then, we investigate: (1) when linear score combinations provably achieve projective composition, (2) whether reverse-diffusion sampling can generate the desired composition, and (3) the conditions under which composition fails. Finally, we connect our theoretical analysis to prior empirical observations where composition has either worked or failed, for reasons that were unclear at the time.
\end{abstract}

\section{Introduction}


\begin{figure}[t]
\centering
\includegraphics[width=0.6\columnwidth]{figures/evaluation_desiderata_V5.pdf}
\vspace{-0.5cm}
\caption{\systemName is a platform for conducting realistic evaluations of code LLMs, collecting human preferences of coding models with real users, real tasks, and in realistic environments, aimed at addressing the limitations of existing evaluations.
}
\label{fig:motivation}
\end{figure}

\begin{figure*}[t]
\centering
\includegraphics[width=\textwidth]{figures/system_design_v2.png}
\caption{We introduce \systemName, a VSCode extension to collect human preferences of code directly in a developer's IDE. \systemName enables developers to use code completions from various models. The system comprises a) the interface in the user's IDE which presents paired completions to users (left), b) a sampling strategy that picks model pairs to reduce latency (right, top), and c) a prompting scheme that allows diverse LLMs to perform code completions with high fidelity.
Users can select between the top completion (green box) using \texttt{tab} or the bottom completion (blue box) using \texttt{shift+tab}.}
\label{fig:overview}
\end{figure*}

As model capabilities improve, large language models (LLMs) are increasingly integrated into user environments and workflows.
For example, software developers code with AI in integrated developer environments (IDEs)~\citep{peng2023impact}, doctors rely on notes generated through ambient listening~\citep{oberst2024science}, and lawyers consider case evidence identified by electronic discovery systems~\citep{yang2024beyond}.
Increasing deployment of models in productivity tools demands evaluation that more closely reflects real-world circumstances~\citep{hutchinson2022evaluation, saxon2024benchmarks, kapoor2024ai}.
While newer benchmarks and live platforms incorporate human feedback to capture real-world usage, they almost exclusively focus on evaluating LLMs in chat conversations~\citep{zheng2023judging,dubois2023alpacafarm,chiang2024chatbot, kirk2024the}.
Model evaluation must move beyond chat-based interactions and into specialized user environments.



 

In this work, we focus on evaluating LLM-based coding assistants. 
Despite the popularity of these tools---millions of developers use Github Copilot~\citep{Copilot}---existing
evaluations of the coding capabilities of new models exhibit multiple limitations (Figure~\ref{fig:motivation}, bottom).
Traditional ML benchmarks evaluate LLM capabilities by measuring how well a model can complete static, interview-style coding tasks~\citep{chen2021evaluating,austin2021program,jain2024livecodebench, white2024livebench} and lack \emph{real users}. 
User studies recruit real users to evaluate the effectiveness of LLMs as coding assistants, but are often limited to simple programming tasks as opposed to \emph{real tasks}~\citep{vaithilingam2022expectation,ross2023programmer, mozannar2024realhumaneval}.
Recent efforts to collect human feedback such as Chatbot Arena~\citep{chiang2024chatbot} are still removed from a \emph{realistic environment}, resulting in users and data that deviate from typical software development processes.
We introduce \systemName to address these limitations (Figure~\ref{fig:motivation}, top), and we describe our three main contributions below.


\textbf{We deploy \systemName in-the-wild to collect human preferences on code.} 
\systemName is a Visual Studio Code extension, collecting preferences directly in a developer's IDE within their actual workflow (Figure~\ref{fig:overview}).
\systemName provides developers with code completions, akin to the type of support provided by Github Copilot~\citep{Copilot}. 
Over the past 3 months, \systemName has served over~\completions suggestions from 10 state-of-the-art LLMs, 
gathering \sampleCount~votes from \userCount~users.
To collect user preferences,
\systemName presents a novel interface that shows users paired code completions from two different LLMs, which are determined based on a sampling strategy that aims to 
mitigate latency while preserving coverage across model comparisons.
Additionally, we devise a prompting scheme that allows a diverse set of models to perform code completions with high fidelity.
See Section~\ref{sec:system} and Section~\ref{sec:deployment} for details about system design and deployment respectively.



\textbf{We construct a leaderboard of user preferences and find notable differences from existing static benchmarks and human preference leaderboards.}
In general, we observe that smaller models seem to overperform in static benchmarks compared to our leaderboard, while performance among larger models is mixed (Section~\ref{sec:leaderboard_calculation}).
We attribute these differences to the fact that \systemName is exposed to users and tasks that differ drastically from code evaluations in the past. 
Our data spans 103 programming languages and 24 natural languages as well as a variety of real-world applications and code structures, while static benchmarks tend to focus on a specific programming and natural language and task (e.g. coding competition problems).
Additionally, while all of \systemName interactions contain code contexts and the majority involve infilling tasks, a much smaller fraction of Chatbot Arena's coding tasks contain code context, with infilling tasks appearing even more rarely. 
We analyze our data in depth in Section~\ref{subsec:comparison}.



\textbf{We derive new insights into user preferences of code by analyzing \systemName's diverse and distinct data distribution.}
We compare user preferences across different stratifications of input data (e.g., common versus rare languages) and observe which affect observed preferences most (Section~\ref{sec:analysis}).
For example, while user preferences stay relatively consistent across various programming languages, they differ drastically between different task categories (e.g. frontend/backend versus algorithm design).
We also observe variations in user preference due to different features related to code structure 
(e.g., context length and completion patterns).
We open-source \systemName and release a curated subset of code contexts.
Altogether, our results highlight the necessity of model evaluation in realistic and domain-specific settings.





\putsec{related}{Related Work}

\noindent \textbf{Efficient Radiance Field Rendering.}
%
The introduction of Neural Radiance Fields (NeRF)~\cite{mil:sri20} has
generated significant interest in efficient 3D scene representation and
rendering for radiance fields.
%
Over the past years, there has been a large amount of research aimed at
accelerating NeRFs through algorithmic or software
optimizations~\cite{mul:eva22,fri:yu22,che:fun23,sun:sun22}, and the
development of hardware
accelerators~\cite{lee:cho23,li:li23,son:wen23,mub:kan23,fen:liu24}.
%
The state-of-the-art method, 3D Gaussian splatting~\cite{ker:kop23}, has
further fueled interest in accelerating radiance field
rendering~\cite{rad:ste24,lee:lee24,nie:stu24,lee:rho24,ham:mel24} as it
employs rasterization primitives that can be rendered much faster than NeRFs.
%
However, previous research focused on software graphics rendering on
programmable cores or building dedicated hardware accelerators. In contrast,
\name{} investigates the potential of efficient radiance field rendering while
utilizing fixed-function units in graphics hardware.
%
To our knowledge, this is the first work that assesses the performance
implications of rendering Gaussian-based radiance fields on the hardware
graphics pipeline with software and hardware optimizations.

%%%%%%%%%%%%%%%%%%%%%%%%%%%%%%%%%%%%%%%%%%%%%%%%%%%%%%%%%%%%%%%%%%%%%%%%%%
\myparagraph{Enhancing Graphics Rendering Hardware.}
%
The performance advantage of executing graphics rendering on either
programmable shader cores or fixed-function units varies depending on the
rendering methods and hardware designs.
%
Previous studies have explored the performance implication of graphics hardware
design by developing simulation infrastructures for graphics
workloads~\cite{bar:gon06,gub:aam19,tin:sax23,arn:par13}.
%
Additionally, several studies have aimed to improve the performance of
special-purpose hardware such as ray tracing units in graphics
hardware~\cite{cho:now23,liu:cha21} and proposed hardware accelerators for
graphics applications~\cite{lu:hua17,ram:gri09}.
%
In contrast to these works, which primarily evaluate traditional graphics
workloads, our work focuses on improving the performance of volume rendering
workloads, such as Gaussian splatting, which require blending a huge number of
fragments per pixel.

%%%%%%%%%%%%%%%%%%%%%%%%%%%%%%%%%%%%%%%%%%%%%%%%%%%%%%%%%%%%%%%%%%%%%%%%%%
%
In the context of multi-sample anti-aliasing, prior work proposed reducing the
amount of redundant shading by merging fragments from adjacent triangles in a
mesh at the quad granularity~\cite{fat:bou10}.
%
While both our work and quad-fragment merging (QFM)~\cite{fat:bou10} aim to
reduce operations by merging quads, our proposed technique differs from QFM in
many aspects.
%
Our method aims to blend \emph{overlapping primitives} along the depth
direction and applies to quads from any primitive. In contrast, QFM merges quad
fragments from small (e.g., pixel-sized) triangles that \emph{share} an edge
(i.e., \emph{connected}, \emph{non-overlapping} triangles).
%
As such, QFM is not applicable to the scenes consisting of a number of
unconnected transparent triangles, such as those in 3D Gaussian splatting.
%
In addition, our method computes the \emph{exact} color for each pixel by
offloading blending operations from ROPs to shader units, whereas QFM
\emph{approximates} pixel colors by using the color from one triangle when
multiple triangles are merged into a single quad.



\subsection{From Bayesian to Frequentist Inference}\label{sec:bayes}

A natural choice for the mixing distribution is the Bayesian posterior, which establishes a fundamental connection between frequentist confidence estimation and Bayesian inference. To explore this relationship, we first formally define the Bayesian inference model.
\begin{assumption}[Bayesian Inference]\label{a:bayes}
    In the Bayesian inference model, the learner defines a prior distribution $\mu_0 \in \sP(\Theta)$ over model parameters (independent of the data), and predicts using the posterior distribution $\mu_t(\theta) \propto \prod_{s=1}^{t-1} p_s(y_s|\theta) \mu_0(\theta)$.
\end{assumption}
The main result of this section establishes that if the mixing distributions are computed according to Bayes' rule, then prior likelihood mixing (\cref{result:prior_mixing}) and sequential likelihood mixing (\cref{result:posterior_mixing}) are equivalent. A further application of Bayes rule shows that any (realizable) Bayesian model can be turned into a $(1-\delta)$-confidence sequence by comparing the log posterior probability $\log \mu_t(\theta)$ to the log prior probability $\log \mu_0(\theta)$. This is known as \emph{prior-posterior ratio confidence set} \citep{waudby2020confidence}: 
\begin{align*}
    C_t =  \left\{ \theta \in \Theta: - \log \mu_t(\theta) \leq  \log \frac{1}{\delta} - \log \mu_0(\theta) \right\} \,.
\end{align*}
The equivalence result is foreshadowed in the works by \citet{waudby2020confidence} and \citet{neiswanger2021uncertainty}, who establish the posterior-ratio confidence set and the connection to the marginal likelihood. The explicit equivalence to the sequential mixing framework, however seems to be absent in prior works, and is formally given in \cref{result:mixing-equivalence} below. 
%As a consequence, the concentration bounds for sequential linear regression by \citet{neiswanger2021uncertainty,flynn2024improved,flynn2024tighter} and earlier work by \cite{abbasi2011improved} are essentially equivalent, as we illustrate below. \todoj{maybe move below Lemma 6, so that 'below' makes sense}

\begin{theorem}[Mixing Equivalence]\label{result:mixing-equivalence} If the mixing distributions are chosen according to Bayes' rule, prior likelihood mixing (\cref{result:prior_mixing}) and sequential mixing (\cref{result:posterior_mixing}) are equivalent.
\end{theorem}
\begin{proof}
   The result follows by applying Bayes' rule recursively to show the following equality, $\sum_{s=1}^t \log \int p_s(y_s|\nu) d\mu_{s-1}(\nu) = \log \int \prod_{s=1}^t p_s(y_s|\nu) d\mu_{0}(\nu)$.
    % \begin{align*}
    %     \sum_{s=1}^t \log \int p_s(y_s|\nu) d\mu_{s-1}(\nu) = \log \int \prod_{s=1}^t p_s(y_s|\nu) d\mu_{0}(\nu) \,. 
    % \end{align*}
\end{proof}

The surprising consequence is, that within the Bayesian inference model, sequential mixing provides no statistical advantage compared to averaging the likelihood over the prior. Less surprisingly though, Bayes' rule can be understood as an incremental update rule to compute the marginal likelihood. In this sense, the equivalence can be re-stated as recovering prior mixing (\cref{result:prior_mixing}) as a special case of sequential mixing (\cref{result:posterior_mixing}). However, note that for mixing distributions outside the Bayesian model, the equivalence does not hold in general, leaving the possibility to find non-Bayesian mixing distributions that achieve faster convergence. We come back to this idea in \cref{sec:oco}.

Next, we state a second implication of Bayes' rule, the prior-posterior ratio confidence set. 
\begin{lemma}[Prior-Posterior Ratio Confidence Set \citep{waudby2020confidence}] \label{lem:posterior_ratio_confidence_set}\\
    For any realizable Bayesian model, the following defines a $(1-\delta)$-confidence sequence:
    % The confidence sequence $C_t = \{\theta \in \Theta: L_t(\theta) \leq \log \frac{1}{\delta} - \log \int \prod_{s=1}^t p_s( y_s|\nu) d\mu_0(\nu)  \}$ can be equivalently written as follows:
\begin{align*}
    C_t &=  \left\{ \theta \in \Theta: - \log \mu_t(\theta) \leq  \log \frac{1}{\delta} - \log \mu_0(\theta) \right\} \,.
\end{align*}
Moreover, the confidence set is equivalent to the construction in \cref{result:prior_mixing,result:posterior_mixing}.
\end{lemma}
\begin{proof}
    Note that $\log \mu_t(\theta) = \log \mu_0(\theta)  + L_t(\theta) - \log \int \prod_{s=1}^t p_s(y_s|\nu) d\mu_{0}(\nu)$ holds for all $\theta \in \Theta$ by Bayes' rule. Substituting the equality into \cref{result:prior_mixing} gives the result.
\end{proof}
The remarkable conclusion is that any realizable Bayesian model can be turned into a frequentist confidence set by thresholding the log posterior probability relative to the log prior probability. As a caveat, it is tempting to think of $C_t$ as a Bayesian credible region, however, the posterior credible probability $\mu_{t-1}(C_t)$ is typically not $1-\delta$. Further, the confidence set, by construction, never rejects parameters in the null set of the prior distribution, unlike in classical Bayesian inference. In any case, a sensible choice is $\Theta = \supp(\mu_0)$, as long as the realizability condition (\cref{a:realizability}) is satisfied, that is, $\theta^* \in \Theta$ defines the true likelihood of the data. For an application of the prior-posterior confidence set to sequential sampling without replacement, we refer to \citet{waudby2020confidence}.

As a consequence of the prior-posterior ratio confidence set and the mixing equivalence, the confidence sets for sequential linear regression by \citet{neiswanger2021uncertainty,flynn2024improved,flynn2024tighter} and earlier work by \cite{abbasi2011improved} are essentially equivalent, as we demonstrate below. Moreover, a lower bound by \citet{lattimore2020bandit} shows that the construction is tight without further assumptions on the data generation distribution. 

\paragraph{Sequential Linear Regression} 
To illustrate the utility of the Bayesian perspective, we consider sequential linear regression with a Gaussian prior and likelihood. To preempt any concerns, we remark that the Gaussian assumption can be relaxed to sub-Gaussian distributions, as we explain in \cref{sec:subgaussian}. Formally, let $\theta^* \in \Theta = \bR^d$, with multivariate Gaussian prior $\cN(\theta_0, V_0^{-1})$ centered at $\theta_0 \in \bR^d$ and prior precision matrix $V_0 \in \bR^{d \times d}$, where commonly $V_0 = \lambda \eye_d \in \bR^{d\times d}$ for a regularizer $\lambda > 0$. The observation likelihood is Gaussian,  $y_t \sim \cN(x_t^\top\theta^*, \sigma^2)$ for a feature vector $x_t \in \bR^d$ and known observation variance $\sigma^2 > 0$. The Gaussian posterior is $\mu_t = \cN(\hat \theta_t^\RLS, V_t^{-1})$, where $\hat \theta_t^\RLS$ is the regularized least squares (RLS) estimate,
\begin{align*}
\hat \theta_t^\RLS = \argmin_{\theta \in \bR^d} \frac{1}{2 \sigma^2} \sum_{s=1}^t \big(\ip{x_s, \theta} - y_s\big)^2 + \frac{1}{2} \|\theta - \theta_0\|_{V_0}^2\,.
\end{align*}
Here, $V_t = \frac{1}{\sigma^2}\sum_{s=1}^t x_s x_s^\top + V_0$ is the posterior precision matrix, and we use the notation $\|\nu\|_A^2 = \nu^\top A \nu$ for $\nu \in \bR^d$ and $A \in \bR^{d\times d}$. The prior and posterior densities are explicitly given as follows:
\begin{align*}
    \mu_0(\theta) &= (2 \pi)^{-2/k} (\det V_0)^{1/2} \exp\big(- \tfrac{1}{2}\|\theta - \theta_0\|_{V_0}^2 \big) \\
    \mu_t(\theta) &= (2 \pi)^{-2/k} (\det V_t)^{1/2} \exp\big(- \tfrac{1}{2}\|\theta - \hat \theta_t^\RLS\|_{V_t}^2 \big)
\end{align*}
Applying \cref{lem:posterior_ratio_confidence_set} with the Gaussian posterior, we get the following $(1-\delta)$-confidence sequence:
\begin{align*}
    C_t^\RLS = \left\{ \theta \in \bR^d : \frac{1}{2}\|\theta - \hat \theta_t^\RLS\|_{V_t}^2 \leq \log \frac{1}{\delta} + \frac{1}{2}\log \det V_t - \frac{1}{2}\log \det V_0 + \frac{1}{2}\|\theta  - \theta_0\|_{V_0}^2 \right\}\,.
\end{align*}
An important feature of the bound is that it scales with the \emph{effective dimension} or \emph{total information gain} $\gamma_t = \frac{1}{2}\log \det V_t - \frac{1}{2}\log \det V_0$ of the data \citep[c.f.~][]{huang2021short}, which can be much smaller than the parameter dimension $d$ \citep{srinivas2009gaussian}. 
Note also that the confidence set does \emph{not} require a known bound on the norm $\|\theta^*\|_2 \leq S$, which is required in all prior work that we are aware of. If such a bound is available, a direct approach is to define the Gaussian prior and posterior directly over the restricted set $\cB_S = \{\theta \in \bR^d : \|\theta\|^2 \leq S\}$. However, in this case, the normalization constant is not easily computed in closed form. Instead, we intersect $C_t^\RLS$ with the norm ball $\cB_S$. Relaxing the confidence set further, and choosing $V_0 = \lambda \eye_d$ and $\theta_0 = 0$, we eventually arrive at
\begin{align*}
    % C_t &\subset \{ \theta \in \Theta : \frac{1}{2 \sigma^2} \|\theta - \hat \theta_t\|_{V_t}^2 \leq \log \frac{1}{\delta} + \log \det V_t - \log \det V_0 + S^2 \}\nonumber\\
    C_t^\RLS \cap \cB_S \subset \left\{ \theta \in \bR^d : \frac{1}{2} \|\theta - \hat \theta_t^\RLS\|_{V_t}^2 \leq \log \frac{1}{\delta} + \frac{1}{2}\log \det V_t - \frac{d}{2}\log \lambda+ \frac{\lambda}{2}S^2 \right\} \,.
\end{align*}
The last line essentially recovers the influential result by \citet{abbasi2011improved}, albeit avoiding a lower-order cross-term, improving the bound by up to a factor of two. 
The proof of \citet{abbasi2011improved} uses the method of mixtures, but mixing the noise martingale $S_t = \sum_{s=1}^t \epsilon_s x_t$ over a centered prior, instead of directly mixing the likelihood ratio. 
More recent work by \cite{flynn2024improved} achieves the tighter result using a similar sequential mixing idea, however, the likelihood framework and connection to Bayesian inference is not mentioned. A direct extension is to heteroscedastic noise, $y_t \sim \cN(x_t^\top\theta^*, \sigma_t^2)$ with known variance $\sigma_t^2$ \citep[c.f.,][]{kirschner2018information}. Another, more involved extension is to unknown mean and variance \cite[c.f.,][]{chowdhury2023bregman}. \looseness=-1

\paragraph{Gaussian Process Regression}
We remark that the confidence set for sequential linear regression can be equivalently stated for non-parametric Gaussian processes regression in infinite-dimensional kernel Hilbert spaces (RKHS) using the `kernel trick'. Our derivation improves (up to a factor of two) the results by \cite{abbasi2012thesis,chowdhury2017kernelized,whitehouse2023sublinear} and recovers more recent results by \cite{neiswanger2021uncertainty,flynn2024tighter}, who do not state the equivalence.

% In particular, we can restate the above confidence set $C_t$ on a separable RKHS space, and project the confidence set onto a specific evaluation $x$, via the reproducing kernel operation $f(x) = f^\top k(\cdot,x)$, to arrive at
% \[  C_t(x) = \{ f(x) | |f(x) - \hat{f}_t(x)| \leq  \} \]
% \todoj{make Gaussian processes explicit}


% Lastly, we remark that discussion extends to the more general class of sub-Gaussian likelihoods, which we discuss in \cref{sec:subgaussian}. 
\section{Our Proposal: Projective-Composition}
\label{sec:composition}

\begin{figure}
    \centering
    \includegraphics[width=0.9\linewidth]{figures/projection-vis.png}
    \caption{
    Distribution $\hat{p}$ is a projective composition
    of $p_1$ and $p_2$ w.r.t. projection functions $(\Pi_1, \Pi_2)$,
    because $\hat{p}$ has the same marginals as $p_1$ when 
    both are post-processed by $\Pi_1$, and analogously for $p_2$.
    }
    \label{fig:projection-vis}
\end{figure}

We now present our formal definition of what it means to ``correctly compose'' distributions.
Our main insight here is, a realistic definition of composition should not
purely be a function of distributions $\{p_1, p_2, \dots \}$, in the way 
the simple product $\hat{p}(x) = p_1(x) p_2(x)$ is purely a function of $p_1, p_2$.
We must also somehow specify 
\emph{which aspects} of each distribution we care about preserving in the composition.
For example, informally, we may want a composition that mimics the style of $p_1$
and the content of $p_2$.
Our definition below of \emph{projective composition} allows us this flexibility.

Roughly speaking, our definition requires specifying a ``feature extractor''
$\Pi_i: \R^n \to \R_k$ associated with every distribution $p_i$.
These functions can be arbitrary, but we usually imagine them as projections\footnote{
We use the term ``projection'' informally here, to convey intuition;
these functions $\Pi_i$ are not necessarily coordinate projections, although this is an important special case (Section~\ref{sec:comp_coord}).
} in
some feature-space, e.g, $\Pi_1(x)$ may be a transform of $x$ which extracts only its style,
and $\Pi_2(x)$ a transform which extracts only its content.
Then, a projective composition is any distribution $\hat{p}$ which
``looks like'' distribution $p_i$ when both are viewed through $\Pi_i$
(see Figure~\ref{fig:projection-vis}).
Formally:

\begin{definition}[Projective Composition] 
\label{def:proj_comp}
Given a collection of distributions $\{p_i\}$ along with
associated ``projection'' functions $\{\Pi_i: \R^n \to \R^k\}$,
we call a distribution $\hat{p}$ a \emph{projective composition} if\footnote{
The notation $\sharp$ refers to push-forward of a probability measure.
}
\begin{equation}
\label{eqn:proj_comp}
\forall i: \quad
\Pi_i \sharp \hat{p} = \Pi_i \sharp p_i.
\end{equation}
That is, when $\hat{p}$ is projected by each $\Pi_i$,
it yields marginals identical to those of $p_i$.
\end{definition}

There are a few aspects of this definition worth emphasizing,
which are conceptually different from 
many prior notions of composition.
First, our definition above does not \emph{construct} a composed distribution;
it merely specifies what properties the composition must have.
For a given set of $\{(p_i, \Pi_i)\}$, there may be many possible distributions $\hat{p}$
which are projective compositions; or in other cases, a projective composition
may not even exist.
Separately, the definition of projective composition does not posit any sort of ``true'' underlying
distribution, nor does it require that the distributions $p_i$ 
are conditionals of an underlying joint distribution.
In particular, projective compositions can be truly ``out of distribution'' with respect to the $p_i$: $\hat{p}$ can be
supported on samples $x$ where none of the $p_i$ are supported.
\begin{figure}[t]
    \centering
    \includegraphics[width=1.0\linewidth]{figures/clevr-color-comp.png}
    \caption{\textbf{Composing yellow objects with objects of other colors.} Yellow objects successfully compose with blue, cyan and magenta objects but not with brown, gray, green, or red objects. Per the histograms (left), in RGB-colorspace yellow has R, G distributed like the background (gray) while B has a distinct distribution peaked closer to zero.
    Taking $M_\text{yellow} \approx \{B\}$, \cref{lem:compose} predicts that standard diffusion can sample from compositions of yellow with any color
    where the B channel is distributed like the background: namely, blue, cyan, magenta per the histograms. (Other colors may theoretically compose per \cref{lem:transform_comp}, but be difficult to sample.) (Additional samples in \cref{fig:clever_color_comp_extra}.)}
    \label{fig:clevr_color_comp}
\end{figure}
\paragraph{Examples.}
We have already discussed the style+content composition of Figure~\ref{fig:style-content}
as an instance of projective composition.
Another even simpler example to keep in mind is 
the following coordinate-projection case.
Suppose we take $\Pi_i: \R^n \to \R$ to be the
projection onto the $i$-th coordinate.
Then, a projective composition of distributions $\{p_i\}$
with these associated functions $\{\Pi_i\}$
means: a distribution where the first coordinate is
marginally distributed identically to the first coordinate of $p_1$,
the second coordinate is marginally distributed as $p_2$, and so on.
(Note, we do not require any independence between coordinates).
This notion of composition would be meaningful if, for example,
we are already working in some disentangled feature space,
where the first coordinate controls the style of the image
the second coordinate controls the texture, and so on.
The CLEVR length-generalization example from Figure~\ref{fig:len_gen}
can also be described as a projective composition in almost an identical way,
by letting $\Pi_i: \R^n \to \R^k$ be a restriction onto the set of
pixels neighboring location $i$. We describe this 
explicitly later in Section~\ref{sec:clevr-details}.


\section{Simple Construction of Projective Compositions}
\label{sec:comp_coord}

It is not clear apriori that projective compositional distributions satisfying Definition \ref{def:proj_comp} ever exist, much less that there is any straightforward way to sample from them.
To explore this, we first restrict attention to perhaps the simplest setting, where the projection functions $\{\Pi_i\}$ are
just coordinate restrictions.
This setting is meant to generalize the intuition we had
in the CLEVR example of Figure~\ref{fig:len_gen},
where different objects were composed in disjoint regions of the image.
We first define the construction of the composed distribution,
and then establish its theoretical properties.








\subsection{Defining the Construction}
Formally, suppose we have a set of distributions
$(p_1, p_2, \ldots, p_k)$ that we wish to compose;
in our running CLEVR example, each $p_i$ is the distribution of images
with a single object at position $i$.
Suppose also we have some reference distribution $p_b$,
which can be arbitrary, but should be thought of as a 
``common background'' to the $p_i$s.
Then, one popular way to construct a composed distribution
is via the \emph{compositional operator} defined below.
(A special case of this construction is used in \citet{du2023reduce}, for example).


\begin{definition}[Composition Operator]
    \label{def:comp_oper}
    Define the \emph{composition operator} $\cC$ acting on an arbitrary set of distributions $(p_b, p_1, p_2, \ldots)$ by
    \begin{align}
    \label{eq:comp_oper}
    \cC[\vec{p}] := \cC[p_b, p_1, p_2, \dots](x) := \frac{1}{Z} p_b(x) \prod_i \frac{p_i(x)}{p_b(x)},
    \end{align}
    where $Z$ is the appropriate normalization constant. We name $\cC[\vec{p}]$ the \emph{composed distribution}, and the score of $\cC[\vec{p}]$ the \emph{compositional score}:
    \begin{align}
    \label{eqn:comp_score}
    &\grad_x \log \cC[\vec{p}](x)  \\
    &= \grad_x \log p_b(x) + \sum_i \left( \grad_x \log p_i(x) - \grad_x \log p_b(x) \right). \notag
    \end{align}
\end{definition}
Notice that if $p_b$ is taken to be the unconditional distribution then this is exactly the Bayes-composition.


\vspace{-0.5em}
\subsection{When does the Composition Operator Work?}
We can always apply the composition operator to any set of distributions,
but when does this actually yield a ``correct'' composition
(according to Definition~\ref{def:proj_comp})?
One special case is when each distribution $p_i$ is
``active'' on a different, non-overlapping set of coordinates.
We formalize this property below
as \emph{Factorized Conditionals} (Definition~\ref{def:factorized}).
The idea is, 
each distribution $p_i$
must have a particular set of ``mask'' coordinates $M_i \subseteq [n]$ which it
samples in a characteristic way,
while independently sampling all other coordinates
from a common background distribution.
If a set of distributions $(p_b, p_1, p_2, \ldots)$ has this
\emph{Factorized Conditional} structure, then 
the composition
operator will produce a projective composition (as we will prove below).



\begin{definition}[Factorized-Conditionals]
\label{def:factorized}

We say a set of distributions $(p_b, p_1, p_2, \dots p_k)$
over $\R^n$
are \emph{Factorized Conditionals} if
there exists a partition of coordinates $[n]$
into disjoint subsets $M_b, M_1, \dots M_k$ such that:
\begin{enumerate}
    \setlength{\itemsep}{1pt}
    \item $(x|_{M_i}, x|_{M_i^c})$ are independent under $p_i$.
    \item $(x|_{M_b}, x|_{M_1}, x|_{M_2}, \dots, x|_{M_k})$
    are mutually independent under $p_b$.
    \item $p_i(x|_{M_i^c}) = p_b(x|_{M_i^c})$.
\end{enumerate}

Equivalently, if we have:
\begin{align}
    p_i(x) &= p_i(x|_{M_i}) p_b(x|_{M_i^c}), \text{ and} \label{eqn:cc-cond}\\
    p_b(x) &= p_b(x|_{M_b}) \prod_{i \in [k]} p_b(x|_{M_i}). \notag
\end{align}
\end{definition}
\vspace{-1em}
Equation~\eqref{eqn:cc-cond} means that each $p_i$
can be sampled by first sampling $x \sim p_b$,
and then overwriting the coordinates of $M_i$
according to some other distribution (which can be specific to distribution $i$).
For instance, the experiment of Figure~\ref{fig:len_gen}
intuitively satisfies this property, since 
each of the conditional distributions could essentially be sampled
by first sampling an empty background image ($p_b$), then ``pasting''
a random object in the appropriate location (corresponding to pixels $M_i$).
If a set of distributions obey this Factorized Conditional structure,
then we can prove that the composition operator $\cC$
yields a correct projective composition,
and reverse-diffusion correctly samples from it.
Below, let $N_t$ denote the noise operator of the
diffusion process\footnote{Our results are agnostic to the specific diffusion noise-schedule and scaling used.} at time $t$.

\begin{theorem}[Correctness of Composition]
\label{lem:compose}
Suppose a set of distributions $(p_b, p_1, p_2, \dots p_k)$
satisfy Definition~\ref{def:factorized},
with corresponding masks $\{M_i\}_i$.
Consider running the reverse-diffusion SDE 
using the following compositional scores at each time $t$:
\begin{align}
s_t(x_t) &:= \grad_x \log \cC[p_b^t, p_1^t, p_2^t, \ldots](x_t),
\end{align}
where $p_i^t := N_t[p_i]$ are the noisy distributions.
Then, the distribution of the generated sample $x_0$ at time $t=0$ is:
\begin{align}
\label{eqn:p_hat}
\hat{p}(x) := p_b(x|_{M_b}) \prod_i p_i(x|_{M_i}).
\end{align}
In particular,
$\hat{p}(x|_{M_i}) = p_i(x|_{M_i})$ for all $i$,
and so
$\hat{p}$ is a projective composition
with respect to projections $\{\Pi_i(x) := x|_{M_i}\}_i$,
per Definition \ref{def:proj_comp}.
\end{theorem}




Unpacking this, Line \ref{eqn:p_hat} says that the final generated distribution
$\hat{p}(x)$ can be sampled by
first sampling
the coordinates $M_b$ according to $p_b$ (marginally),
then independently sampling 
coordinates $M_i$ according to $p_i$ (marginally) for each $i$.
Similarly, by assumption, $p_i(x)$ can be sampled by first sampling the coordinates $M_i$ in some specific way, and then independently sampling the remaining coordinates according to $p_b$. Therefore Theorem \ref{lem:compose} says that $\hat{p}(x)$ samples the coordinates \emph{$M_i$ exactly as they would be sampled by $p_i$}, for each $i$ we wish to compose. 

\begin{proof}(Sketch) \small
Since $\vec{p}$ satisfies Definition \ref{def:factorized}, we have
\begin{align*}
&\cC[\vec{p}](x) := p_b(x) \prod_i \frac{p_i(x)}{p_b(x)} \notag 
= p_b(x) \prod_i \frac{p_b(x_t|_{M_i^c}) p_i(x|_{M_i})}{p_b(x|_{M_i^c})p_b(x|_{M_i})} \notag \\
&= p_b(x) \prod_i \frac{p_i(x|_{M_i})}{p_b(x|_{M_i})} \notag 
= p_b(x|_{M_b}) \prod_i p_i(x_t|_{M_i}) := \hat{p}(x).
\end{align*}
The sampling guarantee follows from the commutativity of composition with the diffusion noising process, i.e. $\cC[\vec{p^t}]= N_t[\cC[\vec{p}]]$. 
The complete proof is in Appendix \ref{app:compose_pf}.
\end{proof}

\begin{remark}
In fact, Theorem~\ref{lem:compose} still holds under any orthogonal transformation of the variables,
because the diffusion noise process commutes with orthogonal transforms.
We formalize this as Lemma~\ref{lem:orthogonal_sampling}.
\end{remark}

\begin{remark}
Compositionality is often thought of in terms of orthogonality between scores.
Definition \ref{def:factorized} implies orthogonality between the score differences that appear in the composed score \eqref{eqn:comp_score}:
$\grad_x \log p_i^t(x_t) - \grad_x \log p_b^t(x_t),$
but the former condition is strictly stronger
(c.f. Appendix \ref{app:score_orthog}).
\end{remark}

\begin{remark}
Notice that the composition operator $\cC$
can be applied to a set of Factorized Conditional
distributions
without knowing the coordinate partition $\{M_i\}$.
That is, we can compose distributions and compute scores
without knowing apriori exactly ``how'' these distributions are supposed to compose
(i.e. which coordinates $p_i$ is active on).
This is already somewhat remarkable, and we will see a much
stronger version of this property in the next section.
\end{remark}

\textbf{Importance of background.}
Our derivations highlight the crucial role of the background
distribution $p_b$ for the composition operator  
(Definition~\ref{def:comp_oper}).
While prior works have taken $p_b$ to be an unconditional distribution and the $p_i$'s its associated conditionals,
our results suggest this is not always the optimal choice -- in particular,
it may not satisfy a Factorized Conditional structure (Definition~\ref{def:factorized}). Figure~\ref{fig:len_gen_monster} demonstrates this empirically: settings (a) and (b) attempt to compose the same distributions using different backgrounds -- empty (a) or unconditional (b) -- with very different results.

\subsection{Approximate Factorized Conditionals in CLEVR.}
\label{sec:clevr-details}

In \cref{fig:len_gen_monster} we explore compositional length-generalization (or lack thereof) in three different setting, two of which (\cref{fig:len_gen_monster}a and \ref{fig:len_gen_monster}c) approximately satisfy \cref{def:factorized}. In this section we explicitly describe how our definition of Factorized Conditionals approximately captures the CLEVR settings of Figures \ref{fig:len_gen_monster}a and \ref{fig:len_gen_monster}c. The setting of \ref{fig:len_gen_monster}b does not satisfy our conditions, as discussed in \cref{sec:problematic-compositions}.

\textbf{Single object distributions with empty background.}
This is the setting of both \cref{fig:len_gen} and \cref{fig:len_gen_monster}a.
The background distribution $p_b$ 
over $n$ pixels is images of an empty scene with no objects.
For each $i \in \{1,\ldots,L\}$ (where $L=4$ in \cref{fig:len_gen} and $L=9$ in \cref{fig:len_gen_monster}a), define the set $M_i \subset [n]$ 
as the set of pixel indices surrounding location $i$.
($M_i$ should be thought of as a ``mask'' that
that masks out objects at location $i$).
Let $M_b := (\cup_i M_i)^c$ be the remaining
pixels in the image.
Then, we claim the distributions $(p_b, p_1, \ldots, p_L)$
form approximately
Factorized Conditionals, with corresponding
coordinate partition $\{M_i\}$.
This is essentially because each distribution $p_i$
matches the background $p_b$ on all pixels except those surrounding
location $i$ (further detail in Appendix~\ref{app:clevr-details}).
Note, however, that the conditions of Definition~\ref{def:factorized}
do not \emph{exactly} hold in the experiment of Figure~\ref{fig:len_gen} -- there is still some dependence between
the masks $M_i$, since objects can cast shadows or even occlude each other.
Empirically, these deviations 
have greater impact
when composing many objects, as seen in \cref{fig:len_gen_monster}a.


\textbf{Bayes composition with cluttered distributions.}
In \cref{fig:len_gen_monster}c we replicate CLEVR experiments in  \citet{du2023reduce, liu2022compositional} where the images contain many objects (1-5) and the conditions label the location of one randomly-chosen object. It turns out the unconditional together with the conditionals can approximately act as Factorized Conditionals in ``cluttered'' settings like this one. The intuition is that if the conditional distributions each contain one specific object plus many independently sampled random objects (``clutter''), then the unconditional distribution \emph{almost} looks like independently sampled random objects, which together with the conditionals \emph{would} satisfy Definition \ref{def:factorized} (further discussion in Appendix \ref{app:clevr-details} and \ref{app:bayes_connect}). This helps to explain the length-generalization observed in \citet{liu2022compositional} and verified in our experiments (\cref{fig:len_gen_monster}c).







\section{Projective Composition in Feature Space}
\label{sec:comp_feature}

\begin{figure}
    \centering
    \includegraphics[width=1.0\linewidth]{figures/feat-space-vis.png}
    \caption{A commutative diagram illustrating Theorem~\ref{lem:transform_comp}.
    Performing composition in pixel space is equivalent 
    to encoding into a feature space ($\cA$),
    composing there,
    and decoding back
    to pixel space ($\cA^{-1}$).
    }
    \label{fig:feat-space-vis}
\end{figure}

So far we have focused on the setting where the projection functions $\Pi_i$ are simply projections onto coordinate subsets $M_i$ in the native space (e.g. pixel space).
This covers simple examples like Figure~\ref{fig:len_gen} but does not include more realistic situations such as Figure~\ref{fig:style-content},
where the properties to be composed are more abstract.
For example a property like ``oil painting'' does not correspond to projection
onto a specific subset of pixels in an image.
However, we may hope that there exists some conceptual feature space
in which ``oil painting'' does correspond to a particular subset of variables.
In this section, we extend our results to the case where the composition occurs in some conceptual feature space, and each distribution to be composed
corresponds to some particular subset of \emph{features}.


Our main result is a featurized analogue of Theorem~\ref{lem:compose}:
if there exists \emph{any} invertible transform $\cA$
mapping into a feature space
where Definition \ref{def:factorized} holds,
then the composition operator (Definition~\ref{def:comp_oper})
yields a projective composition in this feature space, as shown in Figure~\ref{fig:feat-space-vis}.

\begin{theorem}[Feature-space Composition]
\label{lem:transform_comp}
Given distributions $\vec{p} := (p_b, p_1, p_2, \dots p_k)$,
suppose there exists a diffeomorphism $\cA: \R^n \to \R^n$
such that
$(\cA \sharp p_b, \cA \sharp p_1, \dots \cA \sharp p_k)$
satisfy Definition~\ref{def:factorized},
with corresponding partition $M_i \subseteq [n]$.
Then, the composition $\hat{p} := \cC[\vec{p}]$ satisfies:
\begin{align}
\label{eqn:p_hat_A}
\cA \sharp \hat{p}(z)
\equiv
(\cA \sharp p_b (z))|_{M_b} \prod_{i=1}^k (\cA \sharp p_i(z))|_{M_i}.
\end{align}
Therefore, $\hat{p}$
is a projective composition of $\vec{p}$ w.r.t. projection functions
$\{\Pi_i(x) := \cA(x)|_{M_i}\}$.
\end{theorem}
This theorem is remarkable because it means we can
compose distributions $(p_b, p_1, p_2, \dots)$ in the base space,
and this composition will ``work correctly'' in the feature space
automatically (Equation~\ref{eqn:p_hat_A}),
without us ever needing to compute or even know the feature transform $\cA$
explicitly.



Theorem~\ref{lem:transform_comp} may apriori seem too strong
to be true, since it somehow holds for all feature spaces $\cA$
simultaneously.
The key observation underlying Theorem~\ref{lem:transform_comp} 
is that the composition operator $\cC$ behaves
well under reparameterization.
\begin{lemma}[Reparameterization Equivariance]
\label{lem:reparam}
The composition operator of Definition~\ref{def:comp_oper}
is reparameterization-equivariant. That is,
for all diffeomorphisms $\cA: \R^n \to \R^n$
and all tuples of distributions $\vec{p} = (p_b, p_1, p_2, \dots, p_k)$,
\begin{align}
 \cC[ \cA \sharp \vec{p}] =  \cA \sharp \cC[\vec{p}].
\end{align}
\end{lemma}
\arxiv{\footnote{
For example (separate from our goals in this paper):
Classifier-Free-Guidance can be seen as an instance of the composition operator.
Thus, Lemma~\ref{lem:reparam} implies that performing CFG
in latent space is \emph{equivalent} to CFG in pixel-space,
assuming accurate score-models in both cases.}}
\arxiv{This lemma is potentially of independent interest:
reparametrization-equivariance
is a very strong property which is typically not satisfied by
standard operations between probability distributions---
for example, the ``simple product'' $p_1(x)p_2(x)$ does not satisfy it---
so it is mathematically notable that the composition operator 
has this structure.
Lemma~\ref{lem:reparam} and Theorem~\ref{lem:transform_comp}
are proved in Appendix \ref{app:param-indep}.}

This lemma is potentially of independent interest:
equivariance distinguishes the composition operator
from many other common operators
(e.g. the simple product).
Lemma ~\ref{lem:reparam} and Theorem~\ref{lem:transform_comp}
are proved in Appendix \ref{app:param-indep}.

\section{Sampling from Compositions.}
The feature-space Theorem~\ref{lem:transform_comp} is weaker than Theorem~\ref{lem:compose}
in one important way: it does not provide a sampling algorithm.
That is, Theorem~\ref{lem:transform_comp} guarantees that $\hat{p} := \cC[\vec{p}]$
is a projective composition, but does not guarantee that reverse-diffusion
is a valid sampling method.

There is one special case where diffusion sampling \emph{is} guaranteed to work, namely, for orthogonal transforms (which can seen as a straightforward extension of the coordinate-aligned case of \cref{lem:compose}):
\begin{lemma}[Orthogonal transform enables diffusion sampling]
\label{lem:orthogonal_sampling}
If the assumptions of Lemma \ref{lem:transform_comp} hold for $\cA(x) = Ax$, where $A$ is an orthogonal matrix, then running a reverse diffusion sampler with scores $s_t = \grad_x \log \cC[\vec{p}^t]$ generates the composed distribution $\hat{p} = \cC[\vec{p}]$ satisfying \eqref{eqn:p_hat_A}.
\end{lemma}
The proof is given in \cref{app:orthog_sample_pf}.

However, for general invertible transforms, we have no such sampling guarantees.
Part of this is inherent: in the feature-space setting, the 
diffusion noise operator $N_t$ no longer commutes
with the composition operator $\cC$ in general,
 so scores of the noisy composed 
distribution $N_t[\cC[\vec{p}]]$
cannot be computed from scores
of the noisy base distributions $N_t[\vec{p}]$.
Nevertheless, one may hope to sample from the distribution $\hat{p}$
using other samplers besides diffusion, 
such as annealed Langevin Dynamics
or
Predictor-Corrector methods \citep{song2020score}.
We find that the situation is surprisingly subtle:
composition $\cC$ produces distributions which
are in some cases easy to sample (e.g. with diffusion),
yet in other cases apparently hard to sample.
For example, in the
setting of Figure~\ref{fig:clevr_color_comp}, 
our Theorem~\ref{lem:transform_comp} implies
that all pairs of colors should compose equally well
at time $t=0$, since there exist diffeomorphisms
(indeed, linear transforms) between different colors.
However, as we saw,
the diffusion sampler
fails to sample from compositions 
of non-orthogonal colors--- and 
empirically, even more sophisticated
samplers such as Predictor-Correctors
also fail in this setting.
At first glance, it may seem odd that
composed distributions are so hard to sample,
when their constituent distributions are relatively easy to sample.
One possible reason for this below is that the composition operator has extremely poor Lipchitz constant,
so it is possible for a set of distributions $\vec{p}$ to ``vary smoothly''
(e.g. diffusing over time) while their composition $\cC[\vec{p}]$
changes abruptly.
We formalize this in \cref{lem:lipschitz} (further discussion and proof in Appendix \ref{app:lipschitz}).
\begin{lemma}[Composition Non-Smoothness]
\label{lem:lipschitz}
For any set of distributions $\{p_b, p_1, p_2, \dots, p_k\}$,
and any noise scale $t := \sigma$,
define the noisy distributions 
$p_i^t := N_{t}[p_i]$,
and let $q^t$ denote the composed distribution at time $t$: $q^t := \cC[\vec{p}^t]$. Then, for any choice of $\tau > 0$,
there exist distributions $\{p_b, p_1, \dots p_k\}$ over $\R^n$
such that
\begin{enumerate}
    \setlength{\itemsep}{0pt}
    \item For all $i$, the annealing path of $p_i$ is 
    $\cO(1)$-Lipshitz:
    $\forall t, t': W_2(p_i^{t}, p_i^{t'}) \leq \cO(1) |t - t'|$.
    \item The annealing path of $q$ has Lipshitz constant
    at least $\Omega(\tau^{-1})$:
    $\exists t, t': W_2(q^{t}, q^{t'}) \geq \frac{|t - t'|}{2\tau}.$
\end{enumerate}
\end{lemma}



\section{Connections with Prior Observations}
\label{sec:prior_connect}

\begin{figure}
    \centering
    \includegraphics[width=1.0\linewidth]{figures/dog_horse_hat_3.png}
    \caption{{\bf Composing Entangled Concepts.}
    The left image composes the text-conditions ``photo of a dog''
    with ``photo of a horse'', which both control the subject of the image,
    and produces unexpected results.
    In contrast, the right image composes ``photo of a dog''
    with ``photo, with red hat,'' which intuitively correspond
    to disentangled features.
    Both samples from SDXL using score-composition with
    an unconditional background; details in Appendix~\ref{app:sdxl_detail}.
    }
    \label{fig:dog-horse-hat}
\end{figure}



We have presented a mathematical theory of composition.
Although this theoretical model is a simplification of reality (we do not claim its assumptions hold exactly in practice) we believe the spirit of our results carries over to practical settings,
and can help understand empirical observations from prior work.
We now discuss some of these connections.


\textbf{Independence Assumptions and Disentangled Features.}
Our theory relies on a type of independence 
between distributions, related to orthogonality between scores, which we formalize as Factorized Conditionals.
While such conditional structure typically does not exist in pixel-space,
it is plausible that a factorized structure exists in an appropriate \emph{feature space}, as permitted by our theory (Section~\ref{sec:comp_feature}).
In particular, a feature space and distribution with perfectly ``disentangled'' features
would satisfy our assumptions.
Conversely, if distributions are not appropriately disentangled,
our theory predicts that linear score combinations will fail to compose correctly.
This effect is well-known; see \cref{fig:dog-horse-hat}
for an example;
similar failure cases are highlighted
in \citet{liu2022compositional} as well
(such as ``A bird'' failing to compose with ``A flower'').
Regarding successful cases, style and content compositions
consistently work well in practice,
and are often taken to be disentangled features
(e.g. \citet{karras2019style,gatys2016image,zhu2017unpaired}).
Finally, similar in spirit to our theory, methods for successful composition in practice
such as LoRA task arithmetic \cite{zhang2023composing, ilharco2022editing},
typically require some type of approximate ``concept-space orthogonality''.


\textbf{Text conditioning with location information. }
Conditioning on location is a straightforward way to achieve factorized conditionals (provided the objects in different locations are approximately independent) since the required disjointness already holds in pixel-space. 
Many successful text-to-image compositions in \citet{liu2022compositional} use location information in the prompts, either explicitly (e.g. ``A blue bird on a tree'' + ``A red car behind the tree'')
or implicitly
(``A horse'' + ``A yellow flower field''; since horses are typically in the foreground and fields in the background).

\textbf{Unconditional Backgrounds.}
Most prior works on diffusion composition use the Bayes composition, with substantial practical success. 
As discussed in \cref{sec:clevr-details}, Bayes composition may be approximately projective in ``cluttered'' settings,
helping to explain its practical success in text-to-image settings, where images often contain many different possible objects and concepts.




\section{Conclusion}
In this work, we propose a simple yet effective approach, called SMILE, for graph few-shot learning with fewer tasks. Specifically, we introduce a novel dual-level mixup strategy, including within-task and across-task mixup, for enriching the diversity of nodes within each task and the diversity of tasks. Also, we incorporate the degree-based prior information to learn expressive node embeddings. Theoretically, we prove that SMILE effectively enhances the model's generalization performance. Empirically, we conduct extensive experiments on multiple benchmarks and the results suggest that SMILE significantly outperforms other baselines, including both in-domain and cross-domain few-shot settings.



\section*{Acknowledgements}
Acknowledgements: We thank Miguel Angel Bautista Martin,  Etai Littwin, Jason Ramapuram, and Luca Zappella for helpful discussions and feedback throughout this work, and Preetum's dog Papaya for his contributions to Figure 1.



\bibliography{refs}
\bibliographystyle{icml2025}


\newpage
\appendix
\onecolumn
\subsection{Lloyd-Max Algorithm}
\label{subsec:Lloyd-Max}
For a given quantization bitwidth $B$ and an operand $\bm{X}$, the Lloyd-Max algorithm finds $2^B$ quantization levels $\{\hat{x}_i\}_{i=1}^{2^B}$ such that quantizing $\bm{X}$ by rounding each scalar in $\bm{X}$ to the nearest quantization level minimizes the quantization MSE. 

The algorithm starts with an initial guess of quantization levels and then iteratively computes quantization thresholds $\{\tau_i\}_{i=1}^{2^B-1}$ and updates quantization levels $\{\hat{x}_i\}_{i=1}^{2^B}$. Specifically, at iteration $n$, thresholds are set to the midpoints of the previous iteration's levels:
\begin{align*}
    \tau_i^{(n)}=\frac{\hat{x}_i^{(n-1)}+\hat{x}_{i+1}^{(n-1)}}2 \text{ for } i=1\ldots 2^B-1
\end{align*}
Subsequently, the quantization levels are re-computed as conditional means of the data regions defined by the new thresholds:
\begin{align*}
    \hat{x}_i^{(n)}=\mathbb{E}\left[ \bm{X} \big| \bm{X}\in [\tau_{i-1}^{(n)},\tau_i^{(n)}] \right] \text{ for } i=1\ldots 2^B
\end{align*}
where to satisfy boundary conditions we have $\tau_0=-\infty$ and $\tau_{2^B}=\infty$. The algorithm iterates the above steps until convergence.

Figure \ref{fig:lm_quant} compares the quantization levels of a $7$-bit floating point (E3M3) quantizer (left) to a $7$-bit Lloyd-Max quantizer (right) when quantizing a layer of weights from the GPT3-126M model at a per-tensor granularity. As shown, the Lloyd-Max quantizer achieves substantially lower quantization MSE. Further, Table \ref{tab:FP7_vs_LM7} shows the superior perplexity achieved by Lloyd-Max quantizers for bitwidths of $7$, $6$ and $5$. The difference between the quantizers is clear at 5 bits, where per-tensor FP quantization incurs a drastic and unacceptable increase in perplexity, while Lloyd-Max quantization incurs a much smaller increase. Nevertheless, we note that even the optimal Lloyd-Max quantizer incurs a notable ($\sim 1.5$) increase in perplexity due to the coarse granularity of quantization. 

\begin{figure}[h]
  \centering
  \includegraphics[width=0.7\linewidth]{sections/figures/LM7_FP7.pdf}
  \caption{\small Quantization levels and the corresponding quantization MSE of Floating Point (left) vs Lloyd-Max (right) Quantizers for a layer of weights in the GPT3-126M model.}
  \label{fig:lm_quant}
\end{figure}

\begin{table}[h]\scriptsize
\begin{center}
\caption{\label{tab:FP7_vs_LM7} \small Comparing perplexity (lower is better) achieved by floating point quantizers and Lloyd-Max quantizers on a GPT3-126M model for the Wikitext-103 dataset.}
\begin{tabular}{c|cc|c}
\hline
 \multirow{2}{*}{\textbf{Bitwidth}} & \multicolumn{2}{|c|}{\textbf{Floating-Point Quantizer}} & \textbf{Lloyd-Max Quantizer} \\
 & Best Format & Wikitext-103 Perplexity & Wikitext-103 Perplexity \\
\hline
7 & E3M3 & 18.32 & 18.27 \\
6 & E3M2 & 19.07 & 18.51 \\
5 & E4M0 & 43.89 & 19.71 \\
\hline
\end{tabular}
\end{center}
\end{table}

\subsection{Proof of Local Optimality of LO-BCQ}
\label{subsec:lobcq_opt_proof}
For a given block $\bm{b}_j$, the quantization MSE during LO-BCQ can be empirically evaluated as $\frac{1}{L_b}\lVert \bm{b}_j- \bm{\hat{b}}_j\rVert^2_2$ where $\bm{\hat{b}}_j$ is computed from equation (\ref{eq:clustered_quantization_definition}) as $C_{f(\bm{b}_j)}(\bm{b}_j)$. Further, for a given block cluster $\mathcal{B}_i$, we compute the quantization MSE as $\frac{1}{|\mathcal{B}_{i}|}\sum_{\bm{b} \in \mathcal{B}_{i}} \frac{1}{L_b}\lVert \bm{b}- C_i^{(n)}(\bm{b})\rVert^2_2$. Therefore, at the end of iteration $n$, we evaluate the overall quantization MSE $J^{(n)}$ for a given operand $\bm{X}$ composed of $N_c$ block clusters as:
\begin{align*}
    \label{eq:mse_iter_n}
    J^{(n)} = \frac{1}{N_c} \sum_{i=1}^{N_c} \frac{1}{|\mathcal{B}_{i}^{(n)}|}\sum_{\bm{v} \in \mathcal{B}_{i}^{(n)}} \frac{1}{L_b}\lVert \bm{b}- B_i^{(n)}(\bm{b})\rVert^2_2
\end{align*}

At the end of iteration $n$, the codebooks are updated from $\mathcal{C}^{(n-1)}$ to $\mathcal{C}^{(n)}$. However, the mapping of a given vector $\bm{b}_j$ to quantizers $\mathcal{C}^{(n)}$ remains as  $f^{(n)}(\bm{b}_j)$. At the next iteration, during the vector clustering step, $f^{(n+1)}(\bm{b}_j)$ finds new mapping of $\bm{b}_j$ to updated codebooks $\mathcal{C}^{(n)}$ such that the quantization MSE over the candidate codebooks is minimized. Therefore, we obtain the following result for $\bm{b}_j$:
\begin{align*}
\frac{1}{L_b}\lVert \bm{b}_j - C_{f^{(n+1)}(\bm{b}_j)}^{(n)}(\bm{b}_j)\rVert^2_2 \le \frac{1}{L_b}\lVert \bm{b}_j - C_{f^{(n)}(\bm{b}_j)}^{(n)}(\bm{b}_j)\rVert^2_2
\end{align*}

That is, quantizing $\bm{b}_j$ at the end of the block clustering step of iteration $n+1$ results in lower quantization MSE compared to quantizing at the end of iteration $n$. Since this is true for all $\bm{b} \in \bm{X}$, we assert the following:
\begin{equation}
\begin{split}
\label{eq:mse_ineq_1}
    \tilde{J}^{(n+1)} &= \frac{1}{N_c} \sum_{i=1}^{N_c} \frac{1}{|\mathcal{B}_{i}^{(n+1)}|}\sum_{\bm{b} \in \mathcal{B}_{i}^{(n+1)}} \frac{1}{L_b}\lVert \bm{b} - C_i^{(n)}(b)\rVert^2_2 \le J^{(n)}
\end{split}
\end{equation}
where $\tilde{J}^{(n+1)}$ is the the quantization MSE after the vector clustering step at iteration $n+1$.

Next, during the codebook update step (\ref{eq:quantizers_update}) at iteration $n+1$, the per-cluster codebooks $\mathcal{C}^{(n)}$ are updated to $\mathcal{C}^{(n+1)}$ by invoking the Lloyd-Max algorithm \citep{Lloyd}. We know that for any given value distribution, the Lloyd-Max algorithm minimizes the quantization MSE. Therefore, for a given vector cluster $\mathcal{B}_i$ we obtain the following result:

\begin{equation}
    \frac{1}{|\mathcal{B}_{i}^{(n+1)}|}\sum_{\bm{b} \in \mathcal{B}_{i}^{(n+1)}} \frac{1}{L_b}\lVert \bm{b}- C_i^{(n+1)}(\bm{b})\rVert^2_2 \le \frac{1}{|\mathcal{B}_{i}^{(n+1)}|}\sum_{\bm{b} \in \mathcal{B}_{i}^{(n+1)}} \frac{1}{L_b}\lVert \bm{b}- C_i^{(n)}(\bm{b})\rVert^2_2
\end{equation}

The above equation states that quantizing the given block cluster $\mathcal{B}_i$ after updating the associated codebook from $C_i^{(n)}$ to $C_i^{(n+1)}$ results in lower quantization MSE. Since this is true for all the block clusters, we derive the following result: 
\begin{equation}
\begin{split}
\label{eq:mse_ineq_2}
     J^{(n+1)} &= \frac{1}{N_c} \sum_{i=1}^{N_c} \frac{1}{|\mathcal{B}_{i}^{(n+1)}|}\sum_{\bm{b} \in \mathcal{B}_{i}^{(n+1)}} \frac{1}{L_b}\lVert \bm{b}- C_i^{(n+1)}(\bm{b})\rVert^2_2  \le \tilde{J}^{(n+1)}   
\end{split}
\end{equation}

Following (\ref{eq:mse_ineq_1}) and (\ref{eq:mse_ineq_2}), we find that the quantization MSE is non-increasing for each iteration, that is, $J^{(1)} \ge J^{(2)} \ge J^{(3)} \ge \ldots \ge J^{(M)}$ where $M$ is the maximum number of iterations. 
%Therefore, we can say that if the algorithm converges, then it must be that it has converged to a local minimum. 
\hfill $\blacksquare$


\begin{figure}
    \begin{center}
    \includegraphics[width=0.5\textwidth]{sections//figures/mse_vs_iter.pdf}
    \end{center}
    \caption{\small NMSE vs iterations during LO-BCQ compared to other block quantization proposals}
    \label{fig:nmse_vs_iter}
\end{figure}

Figure \ref{fig:nmse_vs_iter} shows the empirical convergence of LO-BCQ across several block lengths and number of codebooks. Also, the MSE achieved by LO-BCQ is compared to baselines such as MXFP and VSQ. As shown, LO-BCQ converges to a lower MSE than the baselines. Further, we achieve better convergence for larger number of codebooks ($N_c$) and for a smaller block length ($L_b$), both of which increase the bitwidth of BCQ (see Eq \ref{eq:bitwidth_bcq}).


\subsection{Additional Accuracy Results}
%Table \ref{tab:lobcq_config} lists the various LOBCQ configurations and their corresponding bitwidths.
\begin{table}
\setlength{\tabcolsep}{4.75pt}
\begin{center}
\caption{\label{tab:lobcq_config} Various LO-BCQ configurations and their bitwidths.}
\begin{tabular}{|c||c|c|c|c||c|c||c|} 
\hline
 & \multicolumn{4}{|c||}{$L_b=8$} & \multicolumn{2}{|c||}{$L_b=4$} & $L_b=2$ \\
 \hline
 \backslashbox{$L_A$\kern-1em}{\kern-1em$N_c$} & 2 & 4 & 8 & 16 & 2 & 4 & 2 \\
 \hline
 64 & 4.25 & 4.375 & 4.5 & 4.625 & 4.375 & 4.625 & 4.625\\
 \hline
 32 & 4.375 & 4.5 & 4.625& 4.75 & 4.5 & 4.75 & 4.75 \\
 \hline
 16 & 4.625 & 4.75& 4.875 & 5 & 4.75 & 5 & 5 \\
 \hline
\end{tabular}
\end{center}
\end{table}

%\subsection{Perplexity achieved by various LO-BCQ configurations on Wikitext-103 dataset}

\begin{table} \centering
\begin{tabular}{|c||c|c|c|c||c|c||c|} 
\hline
 $L_b \rightarrow$& \multicolumn{4}{c||}{8} & \multicolumn{2}{c||}{4} & 2\\
 \hline
 \backslashbox{$L_A$\kern-1em}{\kern-1em$N_c$} & 2 & 4 & 8 & 16 & 2 & 4 & 2  \\
 %$N_c \rightarrow$ & 2 & 4 & 8 & 16 & 2 & 4 & 2 \\
 \hline
 \hline
 \multicolumn{8}{c}{GPT3-1.3B (FP32 PPL = 9.98)} \\ 
 \hline
 \hline
 64 & 10.40 & 10.23 & 10.17 & 10.15 &  10.28 & 10.18 & 10.19 \\
 \hline
 32 & 10.25 & 10.20 & 10.15 & 10.12 &  10.23 & 10.17 & 10.17 \\
 \hline
 16 & 10.22 & 10.16 & 10.10 & 10.09 &  10.21 & 10.14 & 10.16 \\
 \hline
  \hline
 \multicolumn{8}{c}{GPT3-8B (FP32 PPL = 7.38)} \\ 
 \hline
 \hline
 64 & 7.61 & 7.52 & 7.48 &  7.47 &  7.55 &  7.49 & 7.50 \\
 \hline
 32 & 7.52 & 7.50 & 7.46 &  7.45 &  7.52 &  7.48 & 7.48  \\
 \hline
 16 & 7.51 & 7.48 & 7.44 &  7.44 &  7.51 &  7.49 & 7.47  \\
 \hline
\end{tabular}
\caption{\label{tab:ppl_gpt3_abalation} Wikitext-103 perplexity across GPT3-1.3B and 8B models.}
\end{table}

\begin{table} \centering
\begin{tabular}{|c||c|c|c|c||} 
\hline
 $L_b \rightarrow$& \multicolumn{4}{c||}{8}\\
 \hline
 \backslashbox{$L_A$\kern-1em}{\kern-1em$N_c$} & 2 & 4 & 8 & 16 \\
 %$N_c \rightarrow$ & 2 & 4 & 8 & 16 & 2 & 4 & 2 \\
 \hline
 \hline
 \multicolumn{5}{|c|}{Llama2-7B (FP32 PPL = 5.06)} \\ 
 \hline
 \hline
 64 & 5.31 & 5.26 & 5.19 & 5.18  \\
 \hline
 32 & 5.23 & 5.25 & 5.18 & 5.15  \\
 \hline
 16 & 5.23 & 5.19 & 5.16 & 5.14  \\
 \hline
 \multicolumn{5}{|c|}{Nemotron4-15B (FP32 PPL = 5.87)} \\ 
 \hline
 \hline
 64  & 6.3 & 6.20 & 6.13 & 6.08  \\
 \hline
 32  & 6.24 & 6.12 & 6.07 & 6.03  \\
 \hline
 16  & 6.12 & 6.14 & 6.04 & 6.02  \\
 \hline
 \multicolumn{5}{|c|}{Nemotron4-340B (FP32 PPL = 3.48)} \\ 
 \hline
 \hline
 64 & 3.67 & 3.62 & 3.60 & 3.59 \\
 \hline
 32 & 3.63 & 3.61 & 3.59 & 3.56 \\
 \hline
 16 & 3.61 & 3.58 & 3.57 & 3.55 \\
 \hline
\end{tabular}
\caption{\label{tab:ppl_llama7B_nemo15B} Wikitext-103 perplexity compared to FP32 baseline in Llama2-7B and Nemotron4-15B, 340B models}
\end{table}

%\subsection{Perplexity achieved by various LO-BCQ configurations on MMLU dataset}


\begin{table} \centering
\begin{tabular}{|c||c|c|c|c||c|c|c|c|} 
\hline
 $L_b \rightarrow$& \multicolumn{4}{c||}{8} & \multicolumn{4}{c||}{8}\\
 \hline
 \backslashbox{$L_A$\kern-1em}{\kern-1em$N_c$} & 2 & 4 & 8 & 16 & 2 & 4 & 8 & 16  \\
 %$N_c \rightarrow$ & 2 & 4 & 8 & 16 & 2 & 4 & 2 \\
 \hline
 \hline
 \multicolumn{5}{|c|}{Llama2-7B (FP32 Accuracy = 45.8\%)} & \multicolumn{4}{|c|}{Llama2-70B (FP32 Accuracy = 69.12\%)} \\ 
 \hline
 \hline
 64 & 43.9 & 43.4 & 43.9 & 44.9 & 68.07 & 68.27 & 68.17 & 68.75 \\
 \hline
 32 & 44.5 & 43.8 & 44.9 & 44.5 & 68.37 & 68.51 & 68.35 & 68.27  \\
 \hline
 16 & 43.9 & 42.7 & 44.9 & 45 & 68.12 & 68.77 & 68.31 & 68.59  \\
 \hline
 \hline
 \multicolumn{5}{|c|}{GPT3-22B (FP32 Accuracy = 38.75\%)} & \multicolumn{4}{|c|}{Nemotron4-15B (FP32 Accuracy = 64.3\%)} \\ 
 \hline
 \hline
 64 & 36.71 & 38.85 & 38.13 & 38.92 & 63.17 & 62.36 & 63.72 & 64.09 \\
 \hline
 32 & 37.95 & 38.69 & 39.45 & 38.34 & 64.05 & 62.30 & 63.8 & 64.33  \\
 \hline
 16 & 38.88 & 38.80 & 38.31 & 38.92 & 63.22 & 63.51 & 63.93 & 64.43  \\
 \hline
\end{tabular}
\caption{\label{tab:mmlu_abalation} Accuracy on MMLU dataset across GPT3-22B, Llama2-7B, 70B and Nemotron4-15B models.}
\end{table}


%\subsection{Perplexity achieved by various LO-BCQ configurations on LM evaluation harness}

\begin{table} \centering
\begin{tabular}{|c||c|c|c|c||c|c|c|c|} 
\hline
 $L_b \rightarrow$& \multicolumn{4}{c||}{8} & \multicolumn{4}{c||}{8}\\
 \hline
 \backslashbox{$L_A$\kern-1em}{\kern-1em$N_c$} & 2 & 4 & 8 & 16 & 2 & 4 & 8 & 16  \\
 %$N_c \rightarrow$ & 2 & 4 & 8 & 16 & 2 & 4 & 2 \\
 \hline
 \hline
 \multicolumn{5}{|c|}{Race (FP32 Accuracy = 37.51\%)} & \multicolumn{4}{|c|}{Boolq (FP32 Accuracy = 64.62\%)} \\ 
 \hline
 \hline
 64 & 36.94 & 37.13 & 36.27 & 37.13 & 63.73 & 62.26 & 63.49 & 63.36 \\
 \hline
 32 & 37.03 & 36.36 & 36.08 & 37.03 & 62.54 & 63.51 & 63.49 & 63.55  \\
 \hline
 16 & 37.03 & 37.03 & 36.46 & 37.03 & 61.1 & 63.79 & 63.58 & 63.33  \\
 \hline
 \hline
 \multicolumn{5}{|c|}{Winogrande (FP32 Accuracy = 58.01\%)} & \multicolumn{4}{|c|}{Piqa (FP32 Accuracy = 74.21\%)} \\ 
 \hline
 \hline
 64 & 58.17 & 57.22 & 57.85 & 58.33 & 73.01 & 73.07 & 73.07 & 72.80 \\
 \hline
 32 & 59.12 & 58.09 & 57.85 & 58.41 & 73.01 & 73.94 & 72.74 & 73.18  \\
 \hline
 16 & 57.93 & 58.88 & 57.93 & 58.56 & 73.94 & 72.80 & 73.01 & 73.94  \\
 \hline
\end{tabular}
\caption{\label{tab:mmlu_abalation} Accuracy on LM evaluation harness tasks on GPT3-1.3B model.}
\end{table}

\begin{table} \centering
\begin{tabular}{|c||c|c|c|c||c|c|c|c|} 
\hline
 $L_b \rightarrow$& \multicolumn{4}{c||}{8} & \multicolumn{4}{c||}{8}\\
 \hline
 \backslashbox{$L_A$\kern-1em}{\kern-1em$N_c$} & 2 & 4 & 8 & 16 & 2 & 4 & 8 & 16  \\
 %$N_c \rightarrow$ & 2 & 4 & 8 & 16 & 2 & 4 & 2 \\
 \hline
 \hline
 \multicolumn{5}{|c|}{Race (FP32 Accuracy = 41.34\%)} & \multicolumn{4}{|c|}{Boolq (FP32 Accuracy = 68.32\%)} \\ 
 \hline
 \hline
 64 & 40.48 & 40.10 & 39.43 & 39.90 & 69.20 & 68.41 & 69.45 & 68.56 \\
 \hline
 32 & 39.52 & 39.52 & 40.77 & 39.62 & 68.32 & 67.43 & 68.17 & 69.30  \\
 \hline
 16 & 39.81 & 39.71 & 39.90 & 40.38 & 68.10 & 66.33 & 69.51 & 69.42  \\
 \hline
 \hline
 \multicolumn{5}{|c|}{Winogrande (FP32 Accuracy = 67.88\%)} & \multicolumn{4}{|c|}{Piqa (FP32 Accuracy = 78.78\%)} \\ 
 \hline
 \hline
 64 & 66.85 & 66.61 & 67.72 & 67.88 & 77.31 & 77.42 & 77.75 & 77.64 \\
 \hline
 32 & 67.25 & 67.72 & 67.72 & 67.00 & 77.31 & 77.04 & 77.80 & 77.37  \\
 \hline
 16 & 68.11 & 68.90 & 67.88 & 67.48 & 77.37 & 78.13 & 78.13 & 77.69  \\
 \hline
\end{tabular}
\caption{\label{tab:mmlu_abalation} Accuracy on LM evaluation harness tasks on GPT3-8B model.}
\end{table}

\begin{table} \centering
\begin{tabular}{|c||c|c|c|c||c|c|c|c|} 
\hline
 $L_b \rightarrow$& \multicolumn{4}{c||}{8} & \multicolumn{4}{c||}{8}\\
 \hline
 \backslashbox{$L_A$\kern-1em}{\kern-1em$N_c$} & 2 & 4 & 8 & 16 & 2 & 4 & 8 & 16  \\
 %$N_c \rightarrow$ & 2 & 4 & 8 & 16 & 2 & 4 & 2 \\
 \hline
 \hline
 \multicolumn{5}{|c|}{Race (FP32 Accuracy = 40.67\%)} & \multicolumn{4}{|c|}{Boolq (FP32 Accuracy = 76.54\%)} \\ 
 \hline
 \hline
 64 & 40.48 & 40.10 & 39.43 & 39.90 & 75.41 & 75.11 & 77.09 & 75.66 \\
 \hline
 32 & 39.52 & 39.52 & 40.77 & 39.62 & 76.02 & 76.02 & 75.96 & 75.35  \\
 \hline
 16 & 39.81 & 39.71 & 39.90 & 40.38 & 75.05 & 73.82 & 75.72 & 76.09  \\
 \hline
 \hline
 \multicolumn{5}{|c|}{Winogrande (FP32 Accuracy = 70.64\%)} & \multicolumn{4}{|c|}{Piqa (FP32 Accuracy = 79.16\%)} \\ 
 \hline
 \hline
 64 & 69.14 & 70.17 & 70.17 & 70.56 & 78.24 & 79.00 & 78.62 & 78.73 \\
 \hline
 32 & 70.96 & 69.69 & 71.27 & 69.30 & 78.56 & 79.49 & 79.16 & 78.89  \\
 \hline
 16 & 71.03 & 69.53 & 69.69 & 70.40 & 78.13 & 79.16 & 79.00 & 79.00  \\
 \hline
\end{tabular}
\caption{\label{tab:mmlu_abalation} Accuracy on LM evaluation harness tasks on GPT3-22B model.}
\end{table}

\begin{table} \centering
\begin{tabular}{|c||c|c|c|c||c|c|c|c|} 
\hline
 $L_b \rightarrow$& \multicolumn{4}{c||}{8} & \multicolumn{4}{c||}{8}\\
 \hline
 \backslashbox{$L_A$\kern-1em}{\kern-1em$N_c$} & 2 & 4 & 8 & 16 & 2 & 4 & 8 & 16  \\
 %$N_c \rightarrow$ & 2 & 4 & 8 & 16 & 2 & 4 & 2 \\
 \hline
 \hline
 \multicolumn{5}{|c|}{Race (FP32 Accuracy = 44.4\%)} & \multicolumn{4}{|c|}{Boolq (FP32 Accuracy = 79.29\%)} \\ 
 \hline
 \hline
 64 & 42.49 & 42.51 & 42.58 & 43.45 & 77.58 & 77.37 & 77.43 & 78.1 \\
 \hline
 32 & 43.35 & 42.49 & 43.64 & 43.73 & 77.86 & 75.32 & 77.28 & 77.86  \\
 \hline
 16 & 44.21 & 44.21 & 43.64 & 42.97 & 78.65 & 77 & 76.94 & 77.98  \\
 \hline
 \hline
 \multicolumn{5}{|c|}{Winogrande (FP32 Accuracy = 69.38\%)} & \multicolumn{4}{|c|}{Piqa (FP32 Accuracy = 78.07\%)} \\ 
 \hline
 \hline
 64 & 68.9 & 68.43 & 69.77 & 68.19 & 77.09 & 76.82 & 77.09 & 77.86 \\
 \hline
 32 & 69.38 & 68.51 & 68.82 & 68.90 & 78.07 & 76.71 & 78.07 & 77.86  \\
 \hline
 16 & 69.53 & 67.09 & 69.38 & 68.90 & 77.37 & 77.8 & 77.91 & 77.69  \\
 \hline
\end{tabular}
\caption{\label{tab:mmlu_abalation} Accuracy on LM evaluation harness tasks on Llama2-7B model.}
\end{table}

\begin{table} \centering
\begin{tabular}{|c||c|c|c|c||c|c|c|c|} 
\hline
 $L_b \rightarrow$& \multicolumn{4}{c||}{8} & \multicolumn{4}{c||}{8}\\
 \hline
 \backslashbox{$L_A$\kern-1em}{\kern-1em$N_c$} & 2 & 4 & 8 & 16 & 2 & 4 & 8 & 16  \\
 %$N_c \rightarrow$ & 2 & 4 & 8 & 16 & 2 & 4 & 2 \\
 \hline
 \hline
 \multicolumn{5}{|c|}{Race (FP32 Accuracy = 48.8\%)} & \multicolumn{4}{|c|}{Boolq (FP32 Accuracy = 85.23\%)} \\ 
 \hline
 \hline
 64 & 49.00 & 49.00 & 49.28 & 48.71 & 82.82 & 84.28 & 84.03 & 84.25 \\
 \hline
 32 & 49.57 & 48.52 & 48.33 & 49.28 & 83.85 & 84.46 & 84.31 & 84.93  \\
 \hline
 16 & 49.85 & 49.09 & 49.28 & 48.99 & 85.11 & 84.46 & 84.61 & 83.94  \\
 \hline
 \hline
 \multicolumn{5}{|c|}{Winogrande (FP32 Accuracy = 79.95\%)} & \multicolumn{4}{|c|}{Piqa (FP32 Accuracy = 81.56\%)} \\ 
 \hline
 \hline
 64 & 78.77 & 78.45 & 78.37 & 79.16 & 81.45 & 80.69 & 81.45 & 81.5 \\
 \hline
 32 & 78.45 & 79.01 & 78.69 & 80.66 & 81.56 & 80.58 & 81.18 & 81.34  \\
 \hline
 16 & 79.95 & 79.56 & 79.79 & 79.72 & 81.28 & 81.66 & 81.28 & 80.96  \\
 \hline
\end{tabular}
\caption{\label{tab:mmlu_abalation} Accuracy on LM evaluation harness tasks on Llama2-70B model.}
\end{table}

%\section{MSE Studies}
%\textcolor{red}{TODO}


\subsection{Number Formats and Quantization Method}
\label{subsec:numFormats_quantMethod}
\subsubsection{Integer Format}
An $n$-bit signed integer (INT) is typically represented with a 2s-complement format \citep{yao2022zeroquant,xiao2023smoothquant,dai2021vsq}, where the most significant bit denotes the sign.

\subsubsection{Floating Point Format}
An $n$-bit signed floating point (FP) number $x$ comprises of a 1-bit sign ($x_{\mathrm{sign}}$), $B_m$-bit mantissa ($x_{\mathrm{mant}}$) and $B_e$-bit exponent ($x_{\mathrm{exp}}$) such that $B_m+B_e=n-1$. The associated constant exponent bias ($E_{\mathrm{bias}}$) is computed as $(2^{{B_e}-1}-1)$. We denote this format as $E_{B_e}M_{B_m}$.  

\subsubsection{Quantization Scheme}
\label{subsec:quant_method}
A quantization scheme dictates how a given unquantized tensor is converted to its quantized representation. We consider FP formats for the purpose of illustration. Given an unquantized tensor $\bm{X}$ and an FP format $E_{B_e}M_{B_m}$, we first, we compute the quantization scale factor $s_X$ that maps the maximum absolute value of $\bm{X}$ to the maximum quantization level of the $E_{B_e}M_{B_m}$ format as follows:
\begin{align}
\label{eq:sf}
    s_X = \frac{\mathrm{max}(|\bm{X}|)}{\mathrm{max}(E_{B_e}M_{B_m})}
\end{align}
In the above equation, $|\cdot|$ denotes the absolute value function.

Next, we scale $\bm{X}$ by $s_X$ and quantize it to $\hat{\bm{X}}$ by rounding it to the nearest quantization level of $E_{B_e}M_{B_m}$ as:

\begin{align}
\label{eq:tensor_quant}
    \hat{\bm{X}} = \text{round-to-nearest}\left(\frac{\bm{X}}{s_X}, E_{B_e}M_{B_m}\right)
\end{align}

We perform dynamic max-scaled quantization \citep{wu2020integer}, where the scale factor $s$ for activations is dynamically computed during runtime.

\subsection{Vector Scaled Quantization}
\begin{wrapfigure}{r}{0.35\linewidth}
  \centering
  \includegraphics[width=\linewidth]{sections/figures/vsquant.jpg}
  \caption{\small Vectorwise decomposition for per-vector scaled quantization (VSQ \citep{dai2021vsq}).}
  \label{fig:vsquant}
\end{wrapfigure}
During VSQ \citep{dai2021vsq}, the operand tensors are decomposed into 1D vectors in a hardware friendly manner as shown in Figure \ref{fig:vsquant}. Since the decomposed tensors are used as operands in matrix multiplications during inference, it is beneficial to perform this decomposition along the reduction dimension of the multiplication. The vectorwise quantization is performed similar to tensorwise quantization described in Equations \ref{eq:sf} and \ref{eq:tensor_quant}, where a scale factor $s_v$ is required for each vector $\bm{v}$ that maps the maximum absolute value of that vector to the maximum quantization level. While smaller vector lengths can lead to larger accuracy gains, the associated memory and computational overheads due to the per-vector scale factors increases. To alleviate these overheads, VSQ \citep{dai2021vsq} proposed a second level quantization of the per-vector scale factors to unsigned integers, while MX \citep{rouhani2023shared} quantizes them to integer powers of 2 (denoted as $2^{INT}$).

\subsubsection{MX Format}
The MX format proposed in \citep{rouhani2023microscaling} introduces the concept of sub-block shifting. For every two scalar elements of $b$-bits each, there is a shared exponent bit. The value of this exponent bit is determined through an empirical analysis that targets minimizing quantization MSE. We note that the FP format $E_{1}M_{b}$ is strictly better than MX from an accuracy perspective since it allocates a dedicated exponent bit to each scalar as opposed to sharing it across two scalars. Therefore, we conservatively bound the accuracy of a $b+2$-bit signed MX format with that of a $E_{1}M_{b}$ format in our comparisons. For instance, we use E1M2 format as a proxy for MX4.

\begin{figure}
    \centering
    \includegraphics[width=1\linewidth]{sections//figures/BlockFormats.pdf}
    \caption{\small Comparing LO-BCQ to MX format.}
    \label{fig:block_formats}
\end{figure}

Figure \ref{fig:block_formats} compares our $4$-bit LO-BCQ block format to MX \citep{rouhani2023microscaling}. As shown, both LO-BCQ and MX decompose a given operand tensor into block arrays and each block array into blocks. Similar to MX, we find that per-block quantization ($L_b < L_A$) leads to better accuracy due to increased flexibility. While MX achieves this through per-block $1$-bit micro-scales, we associate a dedicated codebook to each block through a per-block codebook selector. Further, MX quantizes the per-block array scale-factor to E8M0 format without per-tensor scaling. In contrast during LO-BCQ, we find that per-tensor scaling combined with quantization of per-block array scale-factor to E4M3 format results in superior inference accuracy across models. 


\end{document}
