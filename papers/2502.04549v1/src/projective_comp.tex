\section{Our Proposal: Projective-Composition}
\label{sec:composition}

\begin{figure}
    \centering
    \includegraphics[width=0.9\linewidth]{figures/projection-vis.png}
    \caption{
    Distribution $\hat{p}$ is a projective composition
    of $p_1$ and $p_2$ w.r.t. projection functions $(\Pi_1, \Pi_2)$,
    because $\hat{p}$ has the same marginals as $p_1$ when 
    both are post-processed by $\Pi_1$, and analogously for $p_2$.
    }
    \label{fig:projection-vis}
\end{figure}

We now present our formal definition of what it means to ``correctly compose'' distributions.
Our main insight here is, a realistic definition of composition should not
purely be a function of distributions $\{p_1, p_2, \dots \}$, in the way 
the simple product $\hat{p}(x) = p_1(x) p_2(x)$ is purely a function of $p_1, p_2$.
We must also somehow specify 
\emph{which aspects} of each distribution we care about preserving in the composition.
For example, informally, we may want a composition that mimics the style of $p_1$
and the content of $p_2$.
Our definition below of \emph{projective composition} allows us this flexibility.

Roughly speaking, our definition requires specifying a ``feature extractor''
$\Pi_i: \R^n \to \R_k$ associated with every distribution $p_i$.
These functions can be arbitrary, but we usually imagine them as projections\footnote{
We use the term ``projection'' informally here, to convey intuition;
these functions $\Pi_i$ are not necessarily coordinate projections, although this is an important special case (Section~\ref{sec:comp_coord}).
} in
some feature-space, e.g, $\Pi_1(x)$ may be a transform of $x$ which extracts only its style,
and $\Pi_2(x)$ a transform which extracts only its content.
Then, a projective composition is any distribution $\hat{p}$ which
``looks like'' distribution $p_i$ when both are viewed through $\Pi_i$
(see Figure~\ref{fig:projection-vis}).
Formally:

\begin{definition}[Projective Composition] 
\label{def:proj_comp}
Given a collection of distributions $\{p_i\}$ along with
associated ``projection'' functions $\{\Pi_i: \R^n \to \R^k\}$,
we call a distribution $\hat{p}$ a \emph{projective composition} if\footnote{
The notation $\sharp$ refers to push-forward of a probability measure.
}
\begin{equation}
\label{eqn:proj_comp}
\forall i: \quad
\Pi_i \sharp \hat{p} = \Pi_i \sharp p_i.
\end{equation}
That is, when $\hat{p}$ is projected by each $\Pi_i$,
it yields marginals identical to those of $p_i$.
\end{definition}

There are a few aspects of this definition worth emphasizing,
which are conceptually different from 
many prior notions of composition.
First, our definition above does not \emph{construct} a composed distribution;
it merely specifies what properties the composition must have.
For a given set of $\{(p_i, \Pi_i)\}$, there may be many possible distributions $\hat{p}$
which are projective compositions; or in other cases, a projective composition
may not even exist.
Separately, the definition of projective composition does not posit any sort of ``true'' underlying
distribution, nor does it require that the distributions $p_i$ 
are conditionals of an underlying joint distribution.
In particular, projective compositions can be truly ``out of distribution'' with respect to the $p_i$: $\hat{p}$ can be
supported on samples $x$ where none of the $p_i$ are supported.
\begin{figure}[t]
    \centering
    \includegraphics[width=1.0\linewidth]{figures/clevr-color-comp.png}
    \caption{\textbf{Composing yellow objects with objects of other colors.} Yellow objects successfully compose with blue, cyan and magenta objects but not with brown, gray, green, or red objects. Per the histograms (left), in RGB-colorspace yellow has R, G distributed like the background (gray) while B has a distinct distribution peaked closer to zero.
    Taking $M_\text{yellow} \approx \{B\}$, \cref{lem:compose} predicts that standard diffusion can sample from compositions of yellow with any color
    where the B channel is distributed like the background: namely, blue, cyan, magenta per the histograms. (Other colors may theoretically compose per \cref{lem:transform_comp}, but be difficult to sample.) (Additional samples in \cref{fig:clever_color_comp_extra}.)}
    \label{fig:clevr_color_comp}
\end{figure}
\paragraph{Examples.}
We have already discussed the style+content composition of Figure~\ref{fig:style-content}
as an instance of projective composition.
Another even simpler example to keep in mind is 
the following coordinate-projection case.
Suppose we take $\Pi_i: \R^n \to \R$ to be the
projection onto the $i$-th coordinate.
Then, a projective composition of distributions $\{p_i\}$
with these associated functions $\{\Pi_i\}$
means: a distribution where the first coordinate is
marginally distributed identically to the first coordinate of $p_1$,
the second coordinate is marginally distributed as $p_2$, and so on.
(Note, we do not require any independence between coordinates).
This notion of composition would be meaningful if, for example,
we are already working in some disentangled feature space,
where the first coordinate controls the style of the image
the second coordinate controls the texture, and so on.
The CLEVR length-generalization example from Figure~\ref{fig:len_gen}
can also be described as a projective composition in almost an identical way,
by letting $\Pi_i: \R^n \to \R^k$ be a restriction onto the set of
pixels neighboring location $i$. We describe this 
explicitly later in Section~\ref{sec:clevr-details}.

