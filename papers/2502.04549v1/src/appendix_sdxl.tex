\section{SDXL experimental details}
\label{app:sdxl_detail}


\subsection{Figure~\ref{fig:style-content}}

The two models composed are
\begin{enumerate}
    \item An SDXL model \citep{podell2023sdxl} fine-tuned 
    on 30 personal photos of the author's dog (Papaya).
    \item SDXL-base-1.0 \citep{podell2023sdxl}
    conditioned on prompt ``an oil painting in the style of van gogh.''
\end{enumerate}

The background score distribution is the unconditional background 
(i.e. SDXL conditioned on the empty prompt).
We use the DDPM sampler \citep{ho2020denoising} with 30 steps,
using the composed score, and CFG guidance weight of $2$ \citet{ho2020denoising}.

Note that using guidance weight $1$ (i.e. no guidance)
also performs reasonably in this case, but is lower quality.


\subsection{Figure~\ref{fig:dog-horse-hat}}
{\bf Left:}
The two score models composed are
\begin{enumerate}
    \item SDXL-base-1.0 \citep{podell2023sdxl}
    conditioned on prompt ``photo of a dog''
    \item SDXL-base-1.0 \citep{podell2023sdxl}
    conditioned on prompt ``photo of a horse''
\end{enumerate}
The background score distribution is the unconditional background 
(i.e. SDXL conditioned on the empty prompt).

For improved sample quality, we use
a Predictor-Corrector method \citep{song2020score}
with the DDPM predictor and the Langevin dynamics corrector,
both operating on the composed score.
We use 100 predictor denoising steps, and 3 Langevin iterations
per step.
We do not use any guidance/CFG.

{\bf Right:}
Identical setting as above, using prompts:
\begin{enumerate}
    \item ``photo of a dog''
    \item ``photo, with red hat''
\end{enumerate}

Note that the DDPM sampler also performed reasonably in this setting,
but Predictor-Corrector methods improved quality.
