\section{Forecasting Visualization}
\label{sm:visualizations}

Here, we show \emph{Powerformer} forecasting visualizations and compare various decay time constants against ground truth and PatcTST~\cite{nie.patchtst.2023a}.
We also highlight plots where both \emph{Powerformer} and PatchTST struggle.
In these cases \emph{Powerformer} generally outperforms better capture faster-moving features, non-periodic features, and is better able to alter periodic predictions.
We believe this is due to \emph{Powerformer's} ability to focus on and amplify local high-frequency features while dampening or removing long-time high-frequency components.
WCMHA achieves this through the power-law masks \fpl{} and \fspl, which amplify local correlations and dampen long-time correlations.

\begin{figure}[!hb]
\begin{minipage}{.5\textwidth}
    \includegraphics[width=1.0\textwidth]{figures/time_series/Electricity/sl512_pl336_var0_id205528.png}
\end{minipage}
\hfill    
\begin{minipage}{.5\textwidth}
    \includegraphics[width=1.0\textwidth]{figures/time_series/Electricity/sl512_pl336_var0_id328304.png}
\end{minipage}
\begin{minipage}{.5\textwidth}
    \includegraphics[width=1.0\textwidth]{figures/time_series/Electricity/sl512_pl336_var0_id785893.png}
\end{minipage}
\hfill    
\begin{minipage}{.5\textwidth}
    \includegraphics[width=1.0\textwidth]{figures/time_series/Electricity/sl512_pl336_var0_id1427458.png}
\end{minipage}
\caption{We show forecasting results on the Electricity dataset for \emph{Powerformer} with \fpl{} and PatchTST. The colored lines represent PatchTST and different \fpl{} decay constants. For these forecasts, the sequence length is 512 and the prediction length is 336.}
\end{figure}




\begin{figure}[!hb]
\begin{minipage}{.5\textwidth}
    \includegraphics[width=1.0\textwidth]{figures/time_series/ETTm1/sl512_pl336_var2_id268.png}
\end{minipage}
\hfill    
\begin{minipage}{.5\textwidth}
    \includegraphics[width=1.0\textwidth]{figures/time_series/ETTm1/sl512_pl336_var3_id3075.png}
\end{minipage}
\begin{minipage}{.5\textwidth}
    \includegraphics[width=1.0\textwidth]{figures/time_series/ETTm1/sl512_pl336_var4_id3095.png}
\end{minipage}
\hfill 
\begin{minipage}{.5\textwidth}
    \includegraphics[width=1.0\textwidth]{figures/time_series/ETTm1/sl512_pl336_var6_id1079.png}
\end{minipage}
\caption{We show forecasting results on the ETTm1 dataset for \emph{Powerformer} with \fpl{} and PatchTST. The colored lines represent PatchTST and different \fpl{} decay constants. For these forecasts, the sequence length is 512 and the prediction length is 336.}
\end{figure}


\begin{figure}[!hb]
\begin{minipage}{.5\textwidth}
    \includegraphics[width=1.0\textwidth]{figures/time_series/Weather/sl512_pl336_var0_id2723.png}
\end{minipage}
\hfill    
\begin{minipage}{.5\textwidth}
    \includegraphics[width=1.0\textwidth]{figures/time_series/Weather/sl512_pl336_var0_id2894.png}
\end{minipage}
\begin{minipage}{.5\textwidth}
    \includegraphics[width=1.0\textwidth]{figures/time_series/Weather/sl512_pl336_var20_id8728.png}
\end{minipage}
\hfill    
\begin{minipage}{.5\textwidth}
    \includegraphics[width=1.0\textwidth]{figures/time_series/Weather/sl512_pl336_var5_id8531.png}
\end{minipage}
\caption{We show forecasting results on the Weather dataset for \emph{Powerformer} with \fpl{} and PatchTST. The colored lines represent PatchTST and different \fpl{} decay constants. For these forecasts, the sequence length is 512 and the prediction length is 336.}
\end{figure}