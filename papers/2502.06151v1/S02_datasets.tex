\section{Datasets\label{sec:datasets}}
We evaluate \emph{Powerformer} on 7 real-world datasets that have become standard public benchmarks for time-series forecast tasks. The datasets can be found and downloaded in Ref.~\cite{wu2021autoformer}, or individually at references below.

\rebuttaledit{
\begin{table}[!h]
    \centering
    \caption{We provide the number of variables and the number of timesteps for each dataset.}
    \label{tab:data_size}
    \vskip 0.1in
    \begin{tabular}{c|cccccc}
        \toprule
        Datasets &  Illness & ETTh* & ETTm* & Weather & Electricity & Traffic \\
        \midrule
        Features &  7 & 7 & 7 & 21 & 321 & 862 \\
        Timesteps & 966 & 17420 & 69680 & 52696 & 26304 & 17544 \\
        \bottomrule
    \end{tabular}
\end{table}
}

\hspace{1cm} \textbf{ETT}\footnote{https://github.com/zhouhaoyi/ETDataset}~\cite{Zhou.informer.2021} provides 7 measurements of the electrical transformer power load (including oil temperature) between July 2016 and July 2018. This is done over 2 regions in China at 2 different sampling rates (hourly and every 15 minutes), resulting in 4 separate datasets (ETTh1, ETTh2, ETTm1, ETTm2). We show the pairwise correlation dependence in Fig.~\ref{fig:ETT_correlations}.

\hspace{1cm} \textbf{Weather}\footnote{https://www.bg-ena.mpg.de/wetter/}~\cite{wu2021autoformer} provides 21 meteorological measurements (air temperature, air pressure, humidity, precipitation, etc.) collected in Germany over the whole of 2020. These measurements are recorded every 10 minutes. We show the pairwise correlation dependence in Fig.~\ref{fig:Weather_correlations}.

\hspace{1cm} \textbf{Electricity}\footnote{https://archive.ics.uci.edu/
    dataset/321/electricityloaddiagrams20112014}~\cite{wu2021autoformer} provides the electricity consumption (kWh) of 321 consumers from 2012 to 2014. These measurements are sampled every hour. We show the pairwise correlation dependence in Fig.~\ref{fig:Electricity_correlations}.
    
\hspace{1cm} \textbf{Traffic}\footnote{http://pems.dot.ca.gov}~\cite{wu2021autoformer} provides occupancy rates on San Francisco Bay Area freeways from 826 sensors. This data comes from the California Department of Transportation and is sampled hourly. We show the pairwise correlation dependence in Fig.~\ref{fig:Traffic_correlations}.


\begin{figure}[!htb]
    \centering
    \includegraphics[width=0.47\linewidth]{figures/correlations_Traffic.pdf}
    \caption{We show the Weather dataset's pairwise correlations as a pairwise separation function. Each colored line represents a different variable in the dataset.}
    \label{fig:Traffic_correlations}
\end{figure}

\begin{figure}[!hb]
\begin{minipage}{.5\textwidth}
    \includegraphics[width=1.0\textwidth]{figures/correlations_Electricity.pdf}
    \caption{We show the Electricity dataset's pairwise correlations as a pairwise separation function. Each colored line represents a different variable in the dataset.}
    \label{fig:Electricity_correlations}
\end{minipage}
\hfill    
\begin{minipage}{.5\textwidth}
    \includegraphics[width=1.0\textwidth]{figures/correlations_Weather.pdf}
    \caption{We show the Weather dataset's pairwise correlations as a pairwise separation function. Each colored line represents a different variable in the dataset.}
    \label{fig:Weather_correlations}
\end{minipage}
\end{figure}

\begin{figure*}[!htb]
    \centering
    \includegraphics[width=0.98\textwidth]{figures/correlations_ETT.pdf}
    \caption{We show the ETT datasets' pairwise correlations as a pairwise separation function. Each colored line represents a different variable in the dataset.}
    \label{fig:ETT_correlations}
\end{figure*}










