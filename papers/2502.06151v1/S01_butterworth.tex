\appendix
\section*{Appendix}

\section{Butterworth Filter}

The Butterworth filter~\cite{butterworth.filter.1930} is often used in signal processing for low-, high-, and band-pass filters.
It is designed to be optimally flat within the passband and sometimes referred to as the ``maximally flat filter."
The motivation behind it's derivation is to have equal sensitivity to frequency modes within the passband.
We leveraged the Butterworth filter to equally weight temporal measurements within the lookback window.
The analog Butterworth filter gain is given by 
\begin{equation}
    f^{\text{BF}}_n(z) = \frac{-1}{\sqrt{1 + \left( \frac{z}{t_\text{c}} \right)^{2n}}} ,
\end{equation}
where $t_c$ is the critical time that sets the width of the filter and the order $n$ determines how fast the gain decays after $t_c$.
In this work we use the digital Butterworth filter gain.
Figure~\ref{fig:maskBF} shows the \fbwo{} and \fbwt{} contributions on the attention score and weights for varying decay constants.

\begin{figure}[!h]
\centering
\includegraphics[width=0.5\linewidth]{figures/butter_filters.pdf}
\caption{We show the effects of the Butterworth filter masks $\left( \fbwnmm \right)$ for orders 1 (dashed lines) and 2 (solid lines), while varying the critical time. 
Panel (a) shows the effects on the attention scores and Panel (b) shows the corresponding effects on the attention weights after applying Softmax to \fbwn.}
\label{fig:maskBF}
\end{figure}

\subsection{Digital Filter Gain\label{sc:butterGain}}
To transform the analog Butterworth filter into a digital filter, the filter design must be discretized.
This discretization is often done using the bilinear transform
\begin{equation}
    z = \frac{2}{T}\frac{z-1}{z+1} ,
\end{equation}
where $T$ is the numerical integration step size used in the trapezoidal rule.

\subsection{Code to Calculate the Gain}
For reproducibility, we provide the code used to calculate the digital Butterworth filter gains.
We multiply the gain by 5 to scale with the attention key-query overlap values.

\begin{verbatim}
    import numpy as np
    import scipy as sp
    
    def butterworth_filter(scale, order, times):
        b, a = sp.signal.butter(order, 0.8, 'lowpass', analog=False)
        t, decay = sp.signal.freqz(b, a)
        t = scale*t/2
        dc = 5*np.log(np.abs(decay))
        decay_interp = sp.interpolate.interp1d(t, dc)
        return decay_interp(times)
\end{verbatim}