\section{Results}\label{sec:results}


We conduct experiments using two dataset categories: standard benchmarks and in-the-wild, as described in Section~\ref{subsec:datasets}. For each ASR model, a leave-one-out strategy is used, training the regression model on 9 benchmark datasets and testing it on the remaining benchmark dataset and all {\nwilds}~in-the-wild datasets to ensure comprehensive evaluation exclusively on out-of-domain data. Additionally, in-domain testing is included in ablation studies, as detailed in Section~\ref{subsec:ablation}. The regression model is trained to predict absolute error counts (word and character levels), which are normalized by the reference length to compute approximate error rates (\(aWER\) and \(aCER\)). We also train regression models to directly predict WER and CER. 




\subsection{Evaluation on In-the-Wild Datasets}\label{subsec:wild_eval}



The wild datasets provide a realistic testbed for evaluating the regression model's ability to approximate error rates under real-world conditions. The results are presented in Table~\ref{tab:main_results_in_the_wild}. High-performing models, like \textit{canary-1b}, demonstrate strong agreement between predicted and actual error rates. For example, on VP\_Accented, \textit{canary-1b} achieves a WER of \(23.2\%\) and an \(aWER\) of \(12.1\%\), with a minimal difference of \(1.1\%\). On Primock57, a clinical consultation dataset, the model shows robustness with a WER of \(16.2\%\) and an \(aWER\) of \(13.4\%\), highlighting its effective generalization across diverse and domain-specific contexts.



For models like \textit{data2vec-audio-large-960h} our approximation is pretty close to actual error rates with difference consistently under \(2\%\) on various datasets. For example, on LibriSpeech-test-noise, the model's actual WER is \(7.2\%\) while the approximated \(aWER\) is \(8.6\%\), showcasing its reliability in noisy conditions. Even on acoustically complex datasets like ATCOsim, where the WER is \(44.0\%\) and the \(aWER\) is \(51.1\%\), the model exhibits a reasonable alignment between approximated and actual error rates.




In contrast, models with high actual error rates, such as \textit{mms-1b-fl102}, show slightly larger deviations, particularly on datasets with challenging conditions. For instance, on ATCOsim, the WER is \(93.4\%\) and the \(aWER\) is \(99.0\%\), resulting in a significant deviation of \(5.6\%\), the highest observed across all in-the-wild datasets. Similarly, on Primock57, where the WER is \(70.2\%\) and the \(aWER\) is \(67.8\%\), the approximation also struggles to align due to the inherently high error rates. This highlights that extreme error cases often correspond to semantically nonsensical outputs, where the distinction between high and extremely high error rates becomes less relevant. 


\begin{table}[!hbt]
\centering
\Large
\resizebox{\columnwidth}{!}{%
\begin{tabular}{p{3cm}p{2cm}p{2cm}p{2cm}p{2cm}}
\toprule
Model              & LS\_Noise & Primock57 & ATCOsim     & VP\_Acc \\ \midrule
w2v2-ls            & 8.8/10.2  & 32.8/35.6 & 43.0/49.5   & 20.4/26.4    \\
can-1b             & 4.1/6.4   & 16.2/13.4 & 30.4/35.5   & 23.2/12.1    \\
d2v-base           & 14.9/16.4 & 39.6/41.7 & 66.0/71.2   & 28.4/33.8    \\
d2v-large          & 7.2/8.6   & 28.3/30.7 & 44.0/51.1   & 21.4/26.5    \\
distil-l-v2        & 7.3/9.2   & 18.3/13.0 & 69.5/66.7   & 14.9/14.5    \\
distil-l-v3        & 6.1/8.3   & 18.4/12.9 & 69.0/63.6   & 14.8/14.0    \\
distil-s.en        & 9.1/10.6  & 19.3/14.7 & 74.9/69.1   & 14.6/14.7    \\
sm4t-l             & 11.2/12.3 & 41.7/37.8 & 75.0/82.5   & 29.3/19.9    \\
sm4t-m             & 14.9/15.6 & 44.1/39.7 & 54.6/60.4   & 30.5/22.5    \\
hub-l-ls-ft        & 7.3/8.8   & 29.5/32.0 & 50.4/56.9   & 21.4/26.6    \\
hub-xl-ls-ft       & 6.8/8.3   & 31.1/32.9 & 46.7/53.0   & 21.8/27.7    \\
mms-1b-a           & 9.5/11.1  & 36.2/34.4 & 63.4/71.8   & 29.9/23.8    \\
mms-1b-f102        & 24.0/24.9 & 70.2/67.8 & 93.4/99.0   & 39.4/38.2    \\
moon-b             & 11.3/12.4 & 19.9/18.5 & 65.5/66.2   & 17.1/20.8    \\
moon-t             & 15.5/17.4 & 29.2/29.5 & 62.9/68.5   & 22.1/26.2    \\
par-ctc-0.6b       & 4.6/7.4   & 16.3/13.8 & 32.9/42.9   & 16.3/13.8    \\
par-ctc-1.1b       & 4.5/6.9   & 16.6/14.1 & 30.9/39.9   & 16.4/12.4    \\
par-rnnt-0.6b      & 3.8/6.9   & 16.3/13.2 & 31.6/41.8   & 17.3/12.6    \\
par-rnnt-1.1b      & 3.5/6.1   & 14.6/13.3 & 27.3/37.6   & 18.1/10.4    \\
par-tdt-1.1b       & 3.4/6.0   & 13.5/13.2 & 28.3/35.7   & 17.9/10.2    \\
pkt-ctc-110m       & 6.1/8.6   & 16.7/13.0 & 39.9/42.4   & 19.2/12.5    \\
sm4t-v2-l          & 7.2/8.4   & 34.6/31.7 & 52.4/57.6   & 33.8/24.5    \\
spchllm-1.5B       & 15.3/16.6 & 42.0/41.8 & 121.1/125.4 & 57.0/59.3    \\
spchllm-2B         & 13.9/15.6 & 39.4/40.3 & 60.6/64.1   & 39.2/44.1    \\
stt-cfc-l          & 5.8/6.8   & 16.1/17.6 & 35.9/38.0   & 18.6/11.5    \\
stt-cfc-s          & 9.7/11.2  & 22.2/24.6 & 43.7/47.7   & 16.4/15.6    \\
stt-fc-cfc-l       & 6.8/10.0  & 17.6/23.9 & 34.9/47.6   & 18.9/13.3    \\
stt-fc-td-l        & 6.0/8.8   & 17.0/20.6 & 34.5/46.5   & 21.1/15.1    \\
w2v2-960h          & 17.4/18.5 & 44.7/47.1 & 68.4/74.0   & 29.9/36.5    \\
w2v2-crelpos       & 5.9/7.4   & 28.5/30.3 & 47.2/54.0   & 22.4/26.7    \\
w2v2-crope         & 6.6/8.1   & 31.7/33.4 & 49.8/56.9   & 21.9/26.3    \\
w2v2-l-960h        & 11.6/12.6 & 37.8/40.2 & 66.4/72.7   & 26.3/33.3    \\
w2v2-l-lv60-s   & 7.8/9.4   & 33.1/35.5 & 40.5/48.8   & 19.3/24.9    \\
w2v2-l-rft-ls      & 10.0/11.5 & 32.2/34.6 & 48.9/55.7   & 22.0/28.6    \\
whisper-l          & 6.2/8.1   & 18.8/13.9 & 65.3/66.9   & 18.7/15.9    \\
whisper-l-v2       & 5.4/6.6   & 19.0/13.1 & 64.8/74.8   & 20.0/18.1    \\
whisper-l-v3       & 4.6/5.9   & 18.7/12.0 & 64.7/73.9   & 19.2/18.1    \\
whisper-l-v3-t & 4.9/6.0   & 18.5/12.3 & 66.0/72.5   & 24.3/23.2    \\
whisper-m.en       & 6.5/7.9   & 19.5/14.0 & 66.2/73.8   & 27.6/26.4    \\
whisper-s.en       & 8.2/9.7   & 20.0/15.1 & 67.1/73.8   & 17.3/17.5    \\
whisper-tiny       & 18.5/20.7 & 30.0/26.6 & 97.6/102.5  & 29.8/33.2   \\ 
\bottomrule
\end{tabular}
}
\caption{Actual and approximated WER ($\downarrow$), separated by a slash, on out-of-distribution wild datasets. The regression model is trained independently for each ASR model on standard benchmarks, making the wild datasets out-of-distribution. See Table~\ref{tab:main_results_in_the_wild_cer} for full names.}
\label{tab:main_results_in_the_wild}
\end{table}



\subsection{Evaluation on Benchmark Datasets}\label{subsec:benchmark_eval}
We summarize results on 10 standard benchmark datasets in Appendix~\ref{appsubsec:results} Tables~\ref{tab:results_benchmark_part1} and~\ref{tab:results_benchmark_part2}. Each table reports actual WER/CER alongside the approximated WER/CER (denoted by aWER/aCER). 

% \begin{figure}[htbp]
%     \centering
%     \begin{subfigure}[b]{1\linewidth}
%         \includegraphics[width=\linewidth]{assets/inthewild_2models.pdf}
%         \caption{In-The-Wild.}
%         \label{fig:WER-in-the-wild}
%     \end{subfigure}
    
%     \vspace{0.3cm} 
%     \begin{subfigure}[b]{1\linewidth}
%         \includegraphics[width=\linewidth]{assets/benchmark_2models.pdf}
%         \caption{Benchmark.}
%         \label{fig:WER-benchmark}
%     \end{subfigure}

%     \caption{Comparison of Actual vs Approximated WER across In-The-Wild and Benchmark Datasets for two Models.}
%     \label{fig:WER-combined}
% \end{figure}


\begin{figure}
    \centering
    \includegraphics[width=1.0\linewidth]{assets/four_models.pdf}
    \caption{Actural and approximated WER for four models across standard benchmark.}
    \label{fig:enter-label}
\end{figure}


Overall, models such as \textit{parakeet-tdt-1.1b} and \textit{whisper-large-v3} show relatively small differences between WER and aWER, indicating reliable approximations. For instance,  the actual WER for \textit{whisper-large-v3} on \textbf{AMI\_IHM} is 19.0\% compared to aWER 17.1\%, 1.9\% gap. Conversely, some challenging datasets (e.g., \textbf{CV11} and \textbf{Earnings22}) reveal larger discrepancies for specific models, particularly those with higher overall error rates. For example, \textit{mms-1b-fl102} exhibits a wide WER/aWER gap on \textbf{Earnings22}, suggesting difficulty handling accented or domain-specific speech. 




In general, high-performing ASR models demonstrate small WER–aWER gaps, indicating that it's easy to approximate when error rates are low. However, models with higher WERs or faced with more acoustically or linguistically challenging test sets tend to show wider divergences. Despite these variations, most results remain within a reasonable margin, highlighting the robustness of our approximation model on diverse out-of-distribution data.

These results underscore the critical role of model quality in achieving reliable approximations. The approximation framework remains effective for high-performing models, while deviations tend to increase in cases of semantically divergent or poorly structured outputs, reflecting the inherent challenges in approximating errors for low-performing systems.







\subsection{Ablation}\label{subsec:ablation}
We conduct ablation experiments to evaluate the robustness of the approximation model and the contributions of its individual components. Using the evaluation setup outlined in Section~\ref{subsec:evaluation}, we select \textit{data2vec-audio-base-960h} as the source model (\(S\)) and \textit{wav2vec2-base-960h} as the target model (\(T\)). The results are summarized in Table~\ref{tab:ablation_baseline_results}, where IID results correspond to Case-I~\ref{para:case_1_data_iid_eval}, and \(D\), \(M\), and \(D+M\) under OOD represent Case II~\ref{para:case_2_data_iid_eval}, Case-III~\ref{para:case_3_data_ood_eval}, and Case-IV~\ref{para:case_4_data_ood_eval}, respectively. The reference model's $r$ value represents the average WER across all datasets. We include reference models with varying $r$ values, such as \textit{whisper-large-v3}~($r=17.8$), \textit{whisper-medium.en}~($r=20.1$), \textit{whisper-tiny}~($r=33.4$), and \textit{mms-1b-fl102}~($r=51.0$).


The results in Table~\ref{tab:ablation_baseline_results} demonstrate the importance of proxy references in improving the regression model's performance. Training without proxy references (\textit{w/o PR}) significantly increases the mean absolute error (MAE) across all conditions. For instance, the IID MAE increases from \(1.03\) (Base) to \(3.13\), and the OOD \(D+M\) MAE rises from \(1.07\) (Base) to \(3.33\), highlighting the essential role of proxy references in approximation.


Increasing the number of high-quality proxy references (\textit{MPR}) further reduces errors. Under IID conditions, the MAE decreases from \(1.00\) with \(n=2\) to \(0.93\) with \(n=5\). Similarly, in OOD \(D+M\), the error drops from \(1.06\) (\textit{MPR}, \(n=2\)) to \(0.95\) (\textit{MPR}, \(n=5\)), demonstrating that multiple high-quality references enhance model robustness.

% 
% \begin{table*}[ht!]
% \resizebox{\textwidth}{!}{%
% \begin{tabular}{lllllllllllllllllllllll}
% \toprule
% \multicolumn{1}{c}{\multirow{2}{*}{Model/Dataset}} & \multicolumn{2}{c}{BERSt-test-validation} & \multicolumn{2}{c}{ami\_ihm} & \multicolumn{2}{c}{common\_voice\_17\_0\_en} & \multicolumn{2}{c}{earnings22\_chunked} & \multicolumn{2}{c}{gigaspeech} & \multicolumn{2}{c}{librispeech\_asr\_test\_clean} & \multicolumn{2}{c}{peoples\_speech} & \multicolumn{2}{c}{slue\_voxceleb} & \multicolumn{2}{c}{spgispeech\_S} & \multicolumn{2}{c}{tedlium-dev-test} & \multicolumn{2}{c}{voxpopuli\_en} \\
% \multicolumn{1}{c}{} & \multicolumn{1}{c}{WER/aWER} & \multicolumn{1}{c}{CER/aCER} & \multicolumn{1}{c}{WER/aWER} & \multicolumn{1}{c}{CER/aCER} & \multicolumn{1}{c}{WER/aWER} & \multicolumn{1}{c}{CER/aCER} & \multicolumn{1}{c}{WER/aWER} & \multicolumn{1}{c}{CER/aCER} & \multicolumn{1}{c}{WER/aWER} & \multicolumn{1}{c}{CER/aCER} & \multicolumn{1}{c}{WER/aWER} & \multicolumn{1}{c}{CER/aCER} & \multicolumn{1}{c}{WER/aWER} & \multicolumn{1}{c}{CER/aCER} & \multicolumn{1}{c}{WER/aWER} & \multicolumn{1}{c}{CER/aCER} & \multicolumn{1}{c}{WER/aWER} & \multicolumn{1}{c}{CER/aCER} & \multicolumn{1}{c}{WER/aWER} & \multicolumn{1}{c}{CER/aCER} & \multicolumn{1}{c}{WER/aWER} & \multicolumn{1}{c}{CER/aCER} \\ \midrule
% asr-wav2vec2-librispeech & 66.5/77.2 & 31.2/38.1 & 28.4/30.5 & 13.8/17.6 & 25.0/29.7 & 11.7/15.0 & 37.3/33.2 & 21.3/16.1 & 16.6/16.5 & 6.9/7.4 & 1.8/3.8 & 0.5/2.2 & 35.6/32.9 & 19.8/17.7 & 19.5/20.4 & 9.8/12.2 & 11.1/12.2 & 4.8/4.7 & 10.3/11.1 & 5.2/5.8 & 14.3/12.6 & 6.6/5.1 \\
% canary-1b & 44.4/58.1 & 20.6/30.2 & 15.4/17.6 & 9.2/12.7 & 8.7/14.2 & 4.1/8.5 & 21.8/16.0 & 15.8/9.1 & 11.1/6.9 & 5.5/4.3 & 1.5/5.7 & 0.5/3.5 & 16.5/22.5 & 11.1/15.2 & 14.9/11.1 & 10.8/8.2 & 3.2/6.7 & 2.0/3.9 & 7.9/7.6 & 5.9/5.0 & 6.4/4.9 & 3.9/3.4 \\
% data2vec-audio-base-960h & 80.3/92.2 & 39.8/46.6 & 39.9/40.4 & 19.9/23.5 & 37.8/42.3 & 18.3/21.7 & 50.8/48.6 & 28.0/25.0 & 23.8/23.5 & 10.1/10.8 & 2.8/4.0 & 0.9/1.6 & 43.4/38.6 & 24.4/20.8 & 26.1/27.6 & 13.0/15.5 & 19.2/19.8 & 8.2/7.9 & 13.6/14.2 & 6.3/6.4 & 18.9/17.5 & 8.5/7.1 \\
% data2vec-audio-large-960h & 66.2/79.0 & 31.1/39.4 & 34.1/36.1 & 16.9/21.2 & 23.3/27.9 & 10.9/14.1 & 37.7/34.5 & 21.2/16.7 & 17.0/16.6 & 7.2/7.4 & 1.8/3.9 & 0.5/1.7 & 35.1/31.3 & 20.0/17.3 & 20.4/22.1 & 10.3/12.9 & 11.3/12.0 & 4.9/4.7 & 9.9/10.6 & 4.5/5.0 & 14.9/13.4 & 6.9/5.5 \\
% distil-large-v2 & 43.9/55.8 & 19.6/27.1 & 17.8/16.8 & 11.2/11.5 & 14.2/19.7 & 7.1/10.6 & 19.3/20.0 & 12.5/13.7 & 12.8/8.2 & 7.1/5.4 & 3.4/6.7 & 1.5/4.2 & 17.4/21.8 & 12.2/14.1 & 16.0/10.8 & 11.4/7.4 & 3.7/7.6 & 1.8/4.5 & 10.4/8.5 & 8.8/5.4 & 9.5/8.2 & 5.8/4.6 \\
% distil-large-v3 & 39.8/53.6 & 17.0/25.7 & 18.5/17.3 & 11.6/11.7 & 13.7/19.4 & 6.6/10.3 & 18.4/19.8 & 12.1/13.0 & 12.2/7.9 & 6.9/5.3 & 2.8/6.6 & 1.2/4.1 & 17.4/21.6 & 12.4/13.8 & 14.4/10.0 & 10.3/6.8 & 3.6/7.4 & 1.8/4.5 & 10.7/9.2 & 8.6/5.7 & 9.3/6.7 & 5.8/4.1 \\
% distil-small.en & 50.9/64.1 & 23.7/30.9 & 18.5/18.4 & 11.1/12.6 & 18.5/23.1 & 9.4/12.5 & 21.2/21.4 & 13.6/14.7 & 13.1/8.6 & 7.3/5.7 & 3.7/7.6 & 1.6/4.5 & 19.0/22.5 & 13.3/14.3 & 15.9/11.4 & 11.3/7.8 & 4.0/7.9 & 1.9/4.7 & 10.8/8.8 & 9.1/5.6 & 10.2/7.4 & 6.4/4.3 \\
% hf-seamless-m4t-large & 65.5/72.7 & 31.6/38.2 & 36.3/33.9 & 25.4/25.1 & 9.5/13.2 & 5.1/7.4 & 30.7/32.8 & 21.1/23.9 & 24.2/21.1 & 16.7/15.7 & 3.2/4.8 & 1.5/2.7 & 38.5/41.5 & 29.2/30.1 & 47.2/42.8 & 39.4/36.1 & 16.2/18.7 & 11.5/13.0 & 19.8/19.1 & 15.7/14.4 & 8.1/6.5 & 5.0/3.6 \\
% hf-seamless-m4t-medium & 63.5/69.5 & 31.3/37.5 & 40.6/37.2 & 29.5/28.9 & 11.3/14.3 & 6.0/7.4 & 33.7/35.9 & 23.9/26.4 & 30.2/28.1 & 22.3/21.7 & 3.8/5.3 & 1.6/2.9 & 43.6/45.7 & 33.6/34.2 & 50.9/47.4 & 43.2/40.3 & 12.9/15.5 & 8.8/10.4 & 27.0/26.2 & 21.3/20.0 & 8.8/7.3 & 5.5/4.4 \\
% hubert-large-ls960-ft & 63.8/76.0 & 28.3/37.0 & 31.1/33.6 & 15.2/19.8 & 24.1/28.8 & 10.6/13.6 & 37.6/34.4 & 20.6/16.3 & 19.3/18.3 & 8.1/7.8 & 2.1/3.7 & 0.6/1.6 & 34.1/31.3 & 18.8/17.7 & 20.8/22.0 & 10.1/12.3 & 11.6/12.4 & 4.9/4.6 & 11.0/12.0 & 5.3/5.4 & 15.0/13.6 & 6.9/5.4 \\
% hubert-xlarge-ls960-ft & 64.3/76.8 & 28.9/37.3 & 31.1/34.3 & 15.0/20.0 & 24.1/28.7 & 10.5/13.9 & 37.3/34.9 & 20.4/15.9 & 18.1/17.4 & 7.3/7.6 & 2.0/3.8 & 0.6/1.7 & 35.5/31.5 & 19.5/16.0 & 20.3/22.1 & 9.9/12.3 & 11.9/12.3 & 4.8/4.5 & 10.1/11.2 & 4.2/5.0 & 14.5/12.8 & 6.7/5.3 \\
% mms-1b-all & 71.2/86.0 & 30.9/39.4 & 37.0/36.2 & 19.1/20.8 & 22.5/27.5 & 8.9/12.5 & 34.1/30.6 & 19.6/15.1 & 19.4/16.9 & 8.3/7.6 & 4.2/6.2 & 1.3/2.7 & 32.2/36.0 & 16.8/18.7 & 27.6/26.3 & 14.6/14.8 & 10.0/12.5 & 3.8/4.9 & 13.5/13.3 & 7.3/6.5 & 8.9/7.6 & 4.4/3.1 \\
% mms-1b-fl102 & 141.3/150.8 & 56.7/62.0 & 75.4/73.3 & 35.1/33.9 & 42.6/45.3 & 17.8/19.9 & 50.6/52.3 & 24.2/26.5 & 37.2/35.7 & 15.7/15.2 & 15.8/17.3 & 5.1/5.9 & 52.4/52.7 & 26.0/25.5 & 51.7/48.7 & 26.1/23.2 & 19.1/22.9 & 5.9/8.6 & 29.7/30.3 & 12.9/12.9 & 22.6/20.5 & 9.3/7.9 \\
% moonshine-base & 55.3/68.5 & 26.3/35.0 & 15.6/24.7 & 9.4/16.7 & 20.8/25.4 & 10.8/13.8 & 24.3/25.6 & 15.9/16.6 & 14.2/10.4 & 8.1/6.8 & 3.4/6.3 & 1.3/3.7 & 26.4/26.2 & 18.1/17.5 & 17.0/13.8 & 11.6/9.1 & 6.4/7.5 & 3.4/3.9 & 5.8/7.0 & 3.4/4.1 & 11.7/9.9 & 6.7/4.6 \\
% moonshine-tiny & 71.8/86.4 & 35.9/42.7 & 21.3/25.3 & 12.8/16.7 & 26.7/31.7 & 14.4/17.3 & 31.2/32.7 & 19.7/20.2 & 16.6/14.1 & 9.1/8.6 & 4.5/7.2 & 1.8/4.2 & 31.8/30.8 & 20.5/19.1 & 20.1/17.2 & 13.4/11.8 & 9.1/9.9 & 4.8/5.3 & 9.8/9.5 & 6.7/5.8 & 14.9/12.9 & 8.2/7.2 \\
% parakeet-ctc-0.6b & 43.4/70.9 & 21.4/37.6 & 17.0/23.1 & 10.0/16.3 & 10.7/21.1 & 5.1/11.2 & 24.7/19.1 & 16.9/11.5 & 12.0/8.6 & 6.1/5.2 & 2.0/5.1 & 0.7/2.5 & 24.2/20.0 & 16.5/12.6 & 13.1/11.0 & 8.7/7.8 & 6.4/7.5 & 3.6/3.8 & 4.3/7.2 & 2.6/4.4 & 7.0/7.2 & 4.1/3.7 \\
% parakeet-ctc-1.1b & 42.8/67.1 & 20.8/35.9 & 15.7/21.4 & 9.0/15.3 & 10.5/20.1 & 5.2/11.0 & 24.0/17.7 & 16.6/10.7 & 12.2/7.9 & 6.2/5.0 & 1.8/5.4 & 0.5/2.6 & 20.7/18.1 & 13.9/11.8 & 13.0/11.6 & 8.7/8.3 & 6.4/7.1 & 3.7/3.6 & 5.0/7.4 & 3.0/4.4 & 6.7/6.5 & 3.9/3.3 \\
% parakeet-rnnt-0.6b & 35.3/67.2 & 16.8/36.2 & 18.8/24.0 & 11.7/17.9 & 8.5/19.9 & 4.2/10.8 & 25.2/18.7 & 17.5/11.5 & 11.7/9.0 & 6.2/5.4 & 1.8/5.5 & 0.6/3.2 & 21.9/17.9 & 15.4/12.2 & 14.5/11.6 & 10.0/8.6 & 4.9/7.3 & 2.9/3.7 & 5.0/6.9 & 3.0/4.0 & 6.4/6.7 & 3.8/3.7 \\
% parakeet-rnnt-1.1b & 33.1/66.5 & 15.1/35.5 & 18.6/23.5 & 11.7/17.2 & 6.7/19.6 & 3.4/10.5 & 25.7/17.9 & 18.4/11.4 & 11.3/8.4 & 6.0/5.0 & 1.5/5.0 & 0.5/3.3 & 23.3/17.2 & 16.6/11.8 & 14.1/10.9 & 9.8/8.9 & 4.5/7.7 & 2.7/4.4 & 5.2/7.7 & 3.5/4.9 & 5.6/6.0 & 3.4/3.2 \\
% parakeet-tdt-1.1b & 33.0/67.0 & 14.9/36.6 & 17.1/23.5 & 10.2/16.9 & 7.2/19.6 & 3.4/10.6 & 24.5/16.6 & 17.1/10.0 & 10.2/7.8 & 4.9/4.7 & 1.3/6.0 & 0.4/2.9 & 24.5/17.9 & 16.6/12.6 & 13.5/10.6 & 9.1/7.9 & 5.4/7.7 & 3.2/4.2 & 4.4/7.1 & 2.8/4.4 & 5.5/6.0 & 3.3/3.1 \\
% parakeet-tdt\_ctc-110m & 40.2/57.1 & 18.6/30.1 & 18.5/18.8 & 10.7/13.6 & 12.7/17.7 & 6.9/10.1 & 22.2/14.8 & 15.7/9.2 & 12.6/8.2 & 6.2/5.0 & 2.6/6.7 & 0.9/3.8 & 16.6/21.4 & 11.5/15.1 & 15.3/11.5 & 10.7/8.3 & 3.8/7.3 & 2.2/4.0 & 5.2/6.6 & 3.3/4.1 & 7.5/6.2 & 4.6/3.1 \\
% seamless-m4t-v2-large & 65.9/79.3 & 31.6/41.0 & 43.0/42.3 & 30.2/30.2 & 8.2/12.3 & 3.9/6.3 & 47.3/47.4 & 33.7/33.9 & 25.7/23.2 & 18.1/17.2 & 2.7/4.4 & 1.0/2.5 & 35.0/36.3 & 25.1/24.9 & 45.1/43.2 & 35.9/34.7 & 11.7/13.6 & 7.6/8.7 & 26.5/25.5 & 21.0/19.1 & 8.0/7.8 & 5.7/5.0 \\
% speechllm-1.5B & 71.4/80.6 & 38.2/45.9 & 67.7/69.3 & 51.5/55.0 & 18.5/22.7 & 10.0/12.7 & 50.8/48.2 & 38.3/35.4 & 27.5/26.0 & 18.1/18.3 & 10.5/12.1 & 7.3/9.2 & 45.1/44.5 & 32.7/32.0 & 60.3/60.8 & 44.1/46.8 & 10.6/10.9 & 6.2/5.8 & 19.4/17.6 & 14.2/11.9 & 30.8/29.9 & 22.0/21.3 \\
% speechllm-2B & 83.2/88.3 & 46.0/51.0 & 38.6/40.8 & 24.3/28.2 & 24.6/28.2 & 16.5/18.3 & 47.3/45.0 & 32.5/30.8 & 24.4/23.6 & 13.5/13.8 & 7.0/9.3 & 4.5/4.8 & 52.9/53.1 & 36.6/35.7 & 36.9/37.8 & 25.8/27.3 & 14.6/15.3 & 8.1/7.5 & 18.8/16.7 & 12.9/9.4 & 28.7/27.7 & 18.5/17.5 \\
% stt\_en\_conformer\_ctc\_large & 51.5/63.2 & 22.5/29.8 & 15.3/19.9 & 7.9/13.4 & 10.4/15.4 & 4.7/8.0 & 24.8/20.0 & 16.4/10.7 & 13.2/10.6 & 5.9/5.6 & 2.2/3.7 & 0.7/2.4 & 24.2/21.5 & 15.2/13.0 & 12.6/14.7 & 7.6/9.4 & 7.9/7.3 & 4.1/3.4 & 5.9/7.7 & 3.3/4.4 & 6.9/5.3 & 3.9/2.7 \\
% stt\_en\_conformer\_ctc\_small & 61.8/75.7 & 28.8/38.0 & 21.2/24.4 & 11.2/15.4 & 19.1/24.1 & 8.9/12.2 & 29.3/25.3 & 19.0/14.1 & 15.5/14.9 & 7.2/7.7 & 3.9/5.4 & 1.4/3.1 & 31.3/26.9 & 19.0/15.7 & 16.6/17.8 & 9.6/11.7 & 10.0/9.4 & 5.1/4.1 & 8.0/9.9 & 3.9/5.2 & 8.9/7.3 & 5.0/3.8 \\
% stt\_en\_fastconformer\_ctc\_large & 48.5/78.1 & 22.9/38.6 & 20.3/24.0 & 11.7/15.3 & 9.5/19.3 & 4.6/10.2 & 27.3/21.5 & 18.3/13.0 & 14.5/14.7 & 7.2/8.2 & 1.9/5.2 & 0.7/2.8 & 26.9/20.5 & 18.3/12.9 & 15.4/14.0 & 9.9/9.7 & 6.9/8.4 & 3.7/4.1 & 5.7/7.8 & 3.1/4.5 & 6.3/6.0 & 3.8/3.3 \\
% stt\_en\_fastconformer\_transducer\_large & 42.4/70.0 & 20.8/36.4 & 19.8/22.0 & 12.9/16.8 & 9.3/18.0 & 4.7/9.8 & 31.5/26.9 & 23.0/18.8 & 13.6/13.2 & 7.4/7.8 & 1.8/3.9 & 0.6/2.6 & 26.5/20.4 & 19.0/13.8 & 16.9/15.1 & 11.3/11.1 & 6.0/7.9 & 3.4/4.0 & 4.9/7.0 & 2.8/4.4 & 6.7/6.9 & 4.1/3.8 \\
% wav2vec2-base-960h & 89.2/99.1 & 40.4/46.9 & 37.9/38.7 & 18.7/21.9 & 40.6/45.7 & 19.5/22.8 & 51.1/48.6 & 28.2/25.4 & 26.2/26.6 & 11.7/12.2 & 3.7/4.5 & 1.1/1.9 & 44.7/40.1 & 24.5/20.6 & 27.3/28.7 & 13.5/15.7 & 21.5/22.4 & 8.9/8.7 & 13.8/14.7 & 6.1/6.6 & 20.5/19.4 & 9.1/7.8 \\
% wav2vec2-conformer-rel-pos-large-960h-ft & 71.1/82.3 & 33.3/40.4 & 35.0/38.7 & 18.5/24.1 & 23.7/28.0 & 10.7/13.7 & 38.4/36.2 & 21.7/17.6 & 18.5/17.2 & 8.5/7.9 & 1.6/3.3 & 0.5/1.5 & 37.3/34.7 & 21.2/19.7 & 20.3/22.1 & 10.6/13.4 & 12.0/12.2 & 5.2/4.6 & 11.7/12.3 & 6.6/6.3 & 14.8/13.1 & 6.9/5.5 \\
% wav2vec2-conformer-rope-large-960h-ft & 68.3/81.3 & 34.2/41.8 & 34.3/36.4 & 18.0/22.7 & 23.6/28.5 & 11.6/15.0 & 36.9/33.9 & 21.4/16.7 & 17.9/17.7 & 7.3/7.6 & 1.8/3.8 & 0.5/1.6 & 35.3/32.1 & 20.4/19.3 & 20.6/22.1 & 10.4/12.9 & 11.7/12.5 & 5.1/4.8 & 10.9/11.8 & 5.5/6.0 & 14.5/13.4 & 6.9/5.4 \\
% wav2vec2-large-960h & 76.7/86.7 & 35.6/41.9 & 34.0/36.4 & 16.4/20.2 & 34.1/38.6 & 16.2/19.4 & 46.4/43.4 & 25.4/21.7 & 20.6/20.5 & 8.6/9.1 & 2.9/4.3 & 0.8/2.1 & 38.9/35.7 & 21.6/19.2 & 23.2/25.1 & 11.5/13.7 & 16.3/17.1 & 6.9/6.6 & 12.2/13.2 & 5.6/6.1 & 18.1/16.7 & 8.2/7.0 \\
% wav2vec2-large-960h-lv60-self & 62.0/79.6 & 30.7/40.4 & 29.1/31.5 & 15.5/19.5 & 23.1/28.8 & 11.0/15.0 & 36.7/32.5 & 20.8/15.7 & 17.6/17.2 & 7.5/8.0 & 1.7/3.5 & 0.5/1.9 & 32.5/29.4 & 18.7/15.7 & 20.2/20.8 & 10.7/12.9 & 10.4/12.2 & 4.3/4.7 & 9.5/10.5 & 4.2/4.9 & 13.5/12.5 & 6.4/5.2 \\
% wav2vec2-large-robust-ft-libri-960h & 64.7/75.7 & 30.0/37.6 & 30.5/33.9 & 13.8/19.2 & 25.0/29.3 & 10.7/13.8 & 37.1/33.5 & 20.5/15.7 & 18.0/17.6 & 7.1/7.7 & 2.8/4.3 & 0.8/2.3 & 36.2/32.6 & 19.2/16.9 & 20.9/23.0 & 9.6/12.5 & 11.8/12.5 & 5.0/4.8 & 10.6/11.9 & 4.8/5.4 & 15.4/14.0 & 7.0/5.8 \\
% whisper-large & 40.6/58.4 & 18.2/31.7 & 18.5/18.3 & 12.3/13.0 & 13.0/18.0 & 6.6/9.3 & 18.8/20.3 & 12.3/14.9 & 12.2/7.7 & 7.1/5.1 & 2.8/5.1 & 1.4/3.5 & 31.2/32.4 & 24.6/23.0 & 17.6/12.6 & 13.7/10.5 & 3.7/7.4 & 2.1/4.4 & 19.3/16.1 & 14.0/10.3 & 8.9/6.0 & 5.5/3.2 \\
% whisper-large-v2 & 38.7/52.0 & 19.0/28.0 & 18.6/17.1 & 12.1/11.8 & 11.3/15.5 & 5.7/8.0 & 19.0/21.5 & 13.0/15.8 & 12.5/7.1 & 7.3/4.9 & 2.8/5.1 & 1.5/3.2 & 18.8/25.2 & 14.2/18.1 & 18.7/15.1 & 14.8/12.1 & 4.1/7.7 & 2.4/4.8 & 28.3/25.3 & 19.4/14.4 & 8.7/6.8 & 5.5/3.8 \\
% whisper-large-v3 & 37.5/50.1 & 17.4/26.1 & 19.0/17.1 & 12.3/12.0 & 9.9/14.5 & 4.9/6.9 & 18.2/20.7 & 12.1/14.9 & 12.5/7.3 & 7.2/4.9 & 2.2/4.0 & 0.9/2.9 & 20.4/27.5 & 15.6/19.5 & 15.6/11.9 & 11.8/8.7 & 3.2/6.3 & 1.7/4.0 & 10.5/8.3 & 8.7/5.5 & 11.0/8.6 & 7.8/6.1 \\
% whisper-large-v3-turbo & 40.4/51.1 & 18.0/25.1 & 19.0/17.5 & 12.3/11.9 & 12.6/16.1 & 6.3/8.1 & 18.8/21.1 & 12.9/15.6 & 12.2/6.8 & 7.1/4.6 & 2.4/4.4 & 1.1/2.5 & 16.0/23.7 & 12.0/16.5 & 15.3/11.9 & 11.5/9.0 & 3.1/6.3 & 1.7/3.9 & 9.9/8.1 & 8.5/5.2 & 13.3/11.1 & 9.8/7.9 \\
% whisper-medium.en & 39.8/52.4 & 18.8/29.3 & 20.3/18.8 & 13.7/14.5 & 14.3/17.8 & 7.2/9.2 & 20.1/22.6 & 13.3/16.5 & 12.8/7.6 & 7.6/5.5 & 3.3/5.5 & 1.8/3.5 & 20.1/25.1 & 15.3/18.0 & 21.2/16.1 & 16.6/13.2 & 4.0/7.7 & 2.2/5.0 & 17.3/14.6 & 18.3/14.2 & 9.0/7.2 & 5.6/3.2 \\
% whisper-small.en & 43.5/55.5 & 20.9/28.1 & 19.8/17.9 & 12.8/12.2 & 17.8/21.9 & 9.3/11.3 & 20.6/22.9 & 13.6/16.3 & 12.8/8.1 & 7.3/5.1 & 3.3/5.6 & 1.4/3.1 & 21.2/25.1 & 16.5/18.0 & 18.2/14.1 & 13.9/10.9 & 3.9/7.5 & 2.1/4.6 & 10.6/8.3 & 15.6/12.0 & 9.5/8.1 & 5.9/3.4 \\
% whisper-tiny & 110.1/130.2 & 56.7/66.6 & 26.7/25.1 & 16.7/17.0 & 33.5/40.3 & 17.7/20.9 & 33.8/35.4 & 22.0/25.4 & 20.6/18.2 & 12.0/11.0 & 7.9/11.2 & 3.4/5.1 & 30.1/31.9 & 21.7/21.5 & 24.0/20.3 & 17.5/14.4 & 8.1/11.9 & 3.9/6.7 & 17.6/15.3 & 13.0/9.5 & 13.2/11.2 & 7.4/6.3 \\
% \bottomrule
% \end{tabular}
% }
% \caption{Comparison of WER/aWER and CER/aCER across Models for Benchmark datasets.}
% \label{tab:results\_benchmark}
% \end{table*}


\begin{table*}[ht!]
\centering
\resizebox{\textwidth}{!}{%
\begin{tabular}{lcccccccccc}
\toprule
\multirow{2}{*}{\textbf{Model}} 
& \multicolumn{2}{c}{\textbf{AMI\_IHM}} 
& \multicolumn{2}{c}{\textbf{CV11}}
& \multicolumn{2}{c}{\textbf{Earnings22}} 
& \multicolumn{2}{c}{\textbf{Gigaspeech}} 
& \multicolumn{2}{c}{\textbf{LibriSpeech\_clean}} 
\\
\cmidrule(lr){2-3} 
\cmidrule(lr){4-5}
\cmidrule(lr){6-7}
\cmidrule(lr){8-9}
\cmidrule(lr){10-11}
& \textbf{WER/aWER} & \textbf{CER/aCER} 
& \textbf{WER/aWER} & \textbf{CER/aCER} 
& \textbf{WER/aWER} & \textbf{CER/aCER}
& \textbf{WER/aWER} & \textbf{CER/aCER}
& \textbf{WER/aWER} & \textbf{CER/aCER}
\\ 
\midrule
asr-wav2vec2-librispeech     & 28.4/30.5 & 13.8/17.6 & 25.0/29.7 & 11.7/15.0 & 37.3/33.2 & 21.3/16.1 & 16.6/16.5 & 6.9/7.4   & 1.8/3.8  & 0.5/2.2  \\
canary-1b                    & 15.4/17.6 & 9.2/12.7  & 8.7/14.2  & 4.1/8.5   & 21.8/16.0 & 15.8/9.1  & 11.1/6.9  & 5.5/4.3   & 1.5/5.7  & 0.5/3.5  \\
data2vec-audio-base-960h     & 39.9/40.4 & 19.9/23.5 & 37.8/42.3 & 18.3/21.7 & 50.8/48.6 & 28.0/25.0 & 23.8/23.5 & 10.1/10.8 & 2.8/4.0  & 0.9/1.6  \\
data2vec-audio-large-960h    & 34.1/36.1 & 16.9/21.2 & 23.3/27.9 & 10.9/14.1 & 37.7/34.5 & 21.2/16.7 & 17.0/16.6 & 7.2/7.4   & 1.8/3.9  & 0.5/1.7  \\
distil-large-v2              & 17.8/16.8 & 11.2/11.5 & 14.2/19.7 & 7.1/10.6  & 19.3/20.0 & 12.5/13.7 & 12.8/8.2  & 7.1/5.4   & 3.4/6.7  & 1.5/4.2  \\
distil-large-v3              & 18.5/17.3 & 11.6/11.7 & 13.7/19.4 & 6.6/10.3  & 18.4/19.8 & 12.1/13.0 & 12.2/7.9  & 6.9/5.3   & 2.8/6.6  & 1.2/4.1  \\
distil-small.en              & 18.5/18.4 & 11.1/12.6 & 18.5/23.1 & 9.4/12.5  & 21.2/21.4 & 13.6/14.7 & 13.1/8.6  & 7.3/5.7   & 3.7/7.6  & 1.6/4.5  \\
hf-seamless-m4t-large        & 36.3/33.9 & 25.4/25.1 & 9.5/13.2  & 5.1/7.4   & 30.7/32.8 & 21.1/23.9 & 24.2/21.1 & 16.7/15.7 & 3.2/4.8  & 1.5/2.7  \\
hf-seamless-m4t-medium       & 40.6/37.2 & 29.5/28.9 & 11.3/14.3 & 6.0/7.4   & 33.7/35.9 & 23.9/26.4 & 30.2/28.1 & 22.3/21.7 & 3.8/5.3  & 1.6/2.9  \\
hubert-large-ls960-ft        & 31.1/33.6 & 15.2/19.8 & 24.1/28.8 & 10.6/13.6 & 37.6/34.4 & 20.6/16.3 & 19.3/18.3 & 8.1/7.8   & 2.1/3.7  & 0.6/1.6  \\
hubert-xlarge-ls960-ft       & 31.1/34.3 & 15.0/20.0 & 24.1/28.7 & 10.5/13.9 & 37.3/34.9 & 20.4/15.9 & 18.1/17.4 & 7.3/7.6   & 2.0/3.8  & 0.6/1.7  \\
mms-1b-all                   & 37.0/36.2 & 19.1/20.8 & 22.5/27.5 & 8.9/12.5  & 34.1/30.6 & 19.6/15.1 & 19.4/16.9 & 8.3/7.6   & 4.2/6.2  & 1.3/2.7  \\
mms-1b-fl102                 & 75.4/73.3 & 35.1/33.9 & 42.6/45.3 & 17.8/19.9 & 50.6/52.3 & 24.2/26.5 & 37.2/35.7 & 15.7/15.2 & 15.8/17.3 & 5.1/5.9  \\
moonshine-base               & 15.6/24.7 & 9.4/16.7  & 20.8/25.4 & 10.8/13.8 & 24.3/25.6 & 15.9/16.6 & 14.2/10.4 & 8.1/6.8   & 3.4/6.3  & 1.3/3.7  \\
moonshine-tiny               & 21.3/25.3 & 12.8/16.7 & 26.7/31.7 & 14.4/17.3 & 31.2/32.7 & 19.7/20.2 & 16.6/14.1 & 9.1/8.6   & 4.5/7.2  & 1.8/4.2  \\
parakeet-ctc-0.6b            & 17.0/23.1 & 10.0/16.3 & 10.7/21.1 & 5.1/11.2  & 24.7/19.1 & 16.9/11.5 & 12.0/8.6  & 6.1/5.2   & 2.0/5.1  & 0.7/2.5  \\
parakeet-ctc-1.1b            & 15.7/21.4 & 9.0/15.3  & 10.5/20.1 & 5.2/11.0  & 24.0/17.7 & 16.6/10.7 & 12.2/7.9  & 6.2/5.0   & 1.8/5.4  & 0.5/2.6  \\
parakeet-rnnt-0.6b           & 18.8/24.0 & 11.7/17.9 & 8.5/19.9  & 4.2/10.8  & 25.2/18.7 & 17.5/11.5 & 11.7/9.0  & 6.2/5.4   & 1.8/5.5  & 0.6/3.2  \\
parakeet-rnnt-1.1b           & 18.6/23.5 & 11.7/17.2 & 6.7/19.6  & 3.4/10.5  & 25.7/17.9 & 18.4/11.4 & 11.3/8.4  & 6.0/5.0   & 1.5/5.0  & 0.5/3.3  \\
parakeet-tdt-1.1b            & 17.1/23.5 & 10.2/16.9 & 7.2/19.6  & 3.4/10.6  & 24.5/16.6 & 17.1/10.0 & 10.2/7.8  & 4.9/4.7   & 1.3/6.0  & 0.4/2.9  \\
parakeet-tdt\_ctc-110m       & 18.5/18.8 & 10.7/13.6 & 12.7/17.7 & 6.9/10.1  & 22.2/14.8 & 15.7/9.2  & 12.6/8.2  & 6.2/5.0   & 2.6/6.7  & 0.9/3.8  \\
seamless-m4t-v2-large        & 43.0/42.3 & 30.2/30.2 & 8.2/12.3  & 3.9/6.3   & 47.3/47.4 & 33.7/33.9 & 25.7/23.2 & 18.1/17.2 & 2.7/4.4  & 1.0/2.5  \\
speechllm-1.5B               & 67.7/69.3 & 51.5/55.0 & 18.5/22.7 & 10.0/12.7 & 50.8/48.2 & 38.3/35.4 & 27.5/26.0 & 18.1/18.3 & 10.5/12.1 & 7.3/9.2  \\
speechllm-2B                 & 38.6/40.8 & 24.3/28.2 & 24.6/28.2 & 16.5/18.3 & 47.3/45.0 & 32.5/30.8 & 24.4/23.6 & 13.5/13.8 & 7.0/9.3  & 4.5/4.8  \\
stt\_en\_conformer\_ctc\_large & 15.3/19.9 & 7.9/13.4  & 10.4/15.4 & 4.7/8.0   & 24.8/20.0 & 16.4/10.7 & 13.2/10.6 & 5.9/5.6   & 2.2/3.7  & 0.7/2.4  \\
stt\_en\_conformer\_ctc\_small & 21.2/24.4 & 11.2/15.4 & 19.1/24.1 & 8.9/12.2  & 29.3/25.3 & 19.0/14.1 & 15.5/14.9 & 7.2/7.7   & 3.9/5.4  & 1.4/3.1  \\
stt\_en\_fastconformer\_ctc\_large & 20.3/24.0 & 11.7/15.3 & 9.5/19.3  & 4.6/10.2  & 27.3/21.5 & 18.3/13.0 & 14.5/14.7 & 7.2/8.2   & 1.9/5.2  & 0.7/2.8  \\
stt\_en\_fastconformer\_transducer\_large 
& 19.8/22.0 & 12.9/16.8 & 9.3/18.0  & 4.7/9.8   & 31.5/26.9 & 23.0/18.8 & 13.6/13.2 & 7.4/7.8   & 1.8/3.9  & 0.6/2.6  \\
wav2vec2-base-960h           & 37.9/38.7 & 18.7/21.9 & 40.6/45.7 & 19.5/22.8 & 51.1/48.6 & 28.2/25.4 & 26.2/26.6 & 11.7/12.2 & 3.7/4.5  & 1.1/1.9  \\
wav2vec2-conformer-rel-pos-large-960h-ft 
& 35.0/38.7 & 18.5/24.1 & 23.7/28.0 & 10.7/13.7 & 38.4/36.2 & 21.7/17.6 & 18.5/17.2 & 8.5/7.9   & 1.6/3.3  & 0.5/1.5  \\
wav2vec2-conformer-rope-large-960h-ft
& 34.3/36.4 & 18.0/22.7 & 23.6/28.5 & 11.6/15.0 & 36.9/33.9 & 21.4/16.7 & 17.9/17.7 & 7.3/7.6   & 1.8/3.8  & 0.5/1.6  \\
wav2vec2-large-960h          & 34.0/36.4 & 16.4/20.2 & 34.1/38.6 & 16.2/19.4 & 46.4/43.4 & 25.4/21.7 & 20.6/20.5 & 8.6/9.1   & 2.9/4.3  & 0.8/2.1  \\
wav2vec2-large-960h-lv60-self
& 29.1/31.5 & 15.5/19.5 & 23.1/28.8 & 11.0/15.0 & 36.7/32.5 & 20.8/15.7 & 17.6/17.2 & 7.5/8.0   & 1.7/3.5  & 0.5/1.9  \\
wav2vec2-large-robust-ft-libri-960h
& 30.5/33.9 & 13.8/19.2 & 25.0/29.3 & 10.7/13.8 & 37.1/33.5 & 20.5/15.7 & 18.0/17.6 & 7.1/7.7   & 2.8/4.3  & 0.8/2.3  \\
whisper-large                 & 18.5/18.3 & 12.3/13.0 & 13.0/18.0 & 6.6/9.3   & 18.8/20.3 & 12.3/14.9 & 12.2/7.7  & 7.1/5.1   & 2.8/5.1  & 1.4/3.5  \\
whisper-large-v2              & 18.6/17.1 & 12.1/11.8 & 11.3/15.5 & 5.7/8.0   & 19.0/21.5 & 13.0/15.8 & 12.5/7.1  & 7.3/4.9   & 2.8/5.1  & 1.5/3.2  \\
whisper-large-v3              & 19.0/17.1 & 12.3/12.0 & 9.9/14.5  & 4.9/6.9   & 18.2/20.7 & 12.1/14.9 & 12.5/7.3  & 7.2/4.9   & 2.2/4.0  & 0.9/2.9  \\
whisper-large-v3-turbo        & 19.0/17.5 & 12.3/11.9 & 12.6/16.1 & 6.3/8.1   & 18.8/21.1 & 12.9/15.6 & 12.2/6.8  & 7.1/4.6   & 2.4/4.4  & 1.1/2.5  \\
whisper-medium.en             & 20.3/18.8 & 13.7/14.5 & 14.3/17.8 & 7.2/9.2   & 20.1/22.6 & 13.3/16.5 & 12.8/7.6  & 7.6/5.5   & 3.3/5.5  & 1.8/3.5  \\
whisper-small.en              & 19.8/17.9 & 12.8/12.2 & 17.8/21.9 & 9.3/11.3  & 20.6/22.9 & 13.6/16.3 & 12.8/8.1  & 7.3/5.1   & 3.3/5.6  & 1.4/3.1  \\
whisper-tiny                  & 26.7/25.1 & 16.7/17.0 & 33.5/40.3 & 17.7/20.9 & 33.8/35.4 & 22.0/25.4 & 20.6/18.2 & 12.0/11.0 & 7.9/11.2 & 3.4/5.1  \\
\bottomrule
\end{tabular}
}
\caption{Actual and approximated WER and CER, separated by a slash, across five standard datasets. The regression model is trained on nine datasets and tested on one, with this process repeated for all datasets, ensuring that the test data is always out-of-distribution.}
\label{tab:results_benchmark_part1}
\end{table*}


\begin{table*}[ht!]
\centering
\resizebox{\textwidth}{!}{%
\begin{tabular}{lcccccccccc}
\toprule
\multirow{2}{*}{\textbf{Model}} 
& \multicolumn{2}{c}{\textbf{peoples\_speech}} 
& \multicolumn{2}{c}{\textbf{slue\_voxceleb}}
& \multicolumn{2}{c}{\textbf{spgispeech\_S}} 
& \multicolumn{2}{c}{\textbf{tedlium-dev-test}}
& \multicolumn{2}{c}{\textbf{voxpopuli\_en}} 
\\
\cmidrule(lr){2-3} 
\cmidrule(lr){4-5}
\cmidrule(lr){6-7}
\cmidrule(lr){8-9}
\cmidrule(lr){10-11}
& \textbf{WER/aWER} & \textbf{CER/aCER} 
& \textbf{WER/aWER} & \textbf{CER/aCER} 
& \textbf{WER/aWER} & \textbf{CER/aCER}
& \textbf{WER/aWER} & \textbf{CER/aCER}
& \textbf{WER/aWER} & \textbf{CER/aCER}
\\
\midrule
asr-wav2vec2-librispeech & 
35.6/32.9 & 19.8/17.7 & 19.5/20.4 & 9.8/12.2 & 11.1/12.2 & 4.8/4.7 & 10.3/11.1 & 5.2/5.8 & 14.3/12.6 & 6.6/5.1 \\

canary-1b & 
16.5/22.5 & 11.1/15.2 & 14.9/11.1 & 10.8/8.2 & 3.2/6.7 & 2.0/3.9 & 7.9/7.6 & 5.9/5.0 & 6.4/4.9 & 3.9/3.4 \\

data2vec-audio-base-960h & 
43.4/38.6 & 24.4/20.8 & 26.1/27.6 & 13.0/15.5 & 19.2/19.8 & 8.2/7.9 & 13.6/14.2 & 6.3/6.4 & 18.9/17.5 & 8.5/7.1 \\

data2vec-audio-large-960h & 
35.1/31.3 & 20.0/17.3 & 20.4/22.1 & 10.3/12.9 & 11.3/12.0 & 4.9/4.7 & 9.9/10.6 & 4.5/5.0 & 14.9/13.4 & 6.9/5.5 \\

distil-large-v2 & 
17.4/21.8 & 12.2/14.1 & 16.0/10.8 & 11.4/7.4 & 3.7/7.6 & 1.8/4.5 & 10.4/8.5 & 8.8/5.4 & 9.5/8.2 & 5.8/4.6 \\

distil-large-v3 & 
17.4/21.6 & 12.4/13.8 & 14.4/10.0 & 10.3/6.8 & 3.6/7.4 & 1.8/4.5 & 10.7/9.2 & 8.6/5.7 & 9.3/6.7 & 5.8/4.1 \\

distil-small.en & 
19.0/22.5 & 13.3/14.3 & 15.9/11.4 & 11.3/7.8 & 4.0/7.9 & 1.9/4.7 & 10.8/8.8 & 9.1/5.6 & 10.2/7.4 & 6.4/4.3 \\

hf-seamless-m4t-large & 
38.5/41.5 & 29.2/30.1 & 47.2/42.8 & 39.4/36.1 & 16.2/18.7 & 11.5/13.0 & 19.8/19.1 & 15.7/14.4 & 8.1/6.5 & 5.0/3.6 \\

hf-seamless-m4t-medium & 
43.6/45.7 & 33.6/34.2 & 50.9/47.4 & 43.2/40.3 & 12.9/15.5 & 8.8/10.4 & 27.0/26.2 & 21.3/20.0 & 8.8/7.3 & 5.5/4.4 \\

hubert-large-ls960-ft & 
34.1/31.3 & 18.8/17.7 & 20.8/22.0 & 10.1/12.3 & 11.6/12.4 & 4.9/4.6 & 11.0/12.0 & 5.3/5.4 & 15.0/13.6 & 6.9/5.4 \\

hubert-xlarge-ls960-ft & 
35.5/31.5 & 19.5/16.0 & 20.3/22.1 & 9.9/12.3  & 11.9/12.3 & 4.8/4.5 & 10.1/11.2 & 4.2/5.0 & 14.5/12.8 & 6.7/5.3 \\

mms-1b-all & 
32.2/36.0 & 16.8/18.7 & 27.6/26.3 & 14.6/14.8 & 10.0/12.5 & 3.8/4.9 & 13.5/13.3 & 7.3/6.5 & 8.9/7.6 & 4.4/3.1 \\

mms-1b-fl102 & 
52.4/52.7 & 26.0/25.5 & 51.7/48.7 & 26.1/23.2 & 19.1/22.9 & 5.9/8.6 & 29.7/30.3 & 12.9/12.9 & 22.6/20.5 & 9.3/7.9 \\

moonshine-base & 
26.4/26.2 & 18.1/17.5 & 17.0/13.8 & 11.6/9.1 & 6.4/7.5 & 3.4/3.9 & 5.8/7.0 & 3.4/4.1 & 11.7/9.9 & 6.7/4.6 \\

moonshine-tiny & 
31.8/30.8 & 20.5/19.1 & 20.1/17.2 & 13.4/11.8 & 9.1/9.9 & 4.8/5.3 & 9.8/9.5 & 6.7/5.8 & 14.9/12.9 & 8.2/7.2 \\

parakeet-ctc-0.6b & 
24.2/20.0 & 16.5/12.6 & 13.1/11.0 & 8.7/7.8  & 6.4/7.5 & 3.6/3.8 & 4.3/7.2 & 2.6/4.4 & 7.0/7.2 & 4.1/3.7 \\

parakeet-ctc-1.1b & 
20.7/18.1 & 13.9/11.8 & 13.0/11.6 & 8.7/8.3  & 6.4/7.1 & 3.7/3.6 & 5.0/7.4 & 3.0/4.4 & 6.7/6.5 & 3.9/3.3 \\

parakeet-rnnt-0.6b & 
21.9/17.9 & 15.4/12.2 & 14.5/11.6 & 10.0/8.6 & 4.9/7.3 & 2.9/3.7 & 5.0/6.9 & 3.0/4.0 & 6.4/6.7 & 3.8/3.7 \\

parakeet-rnnt-1.1b & 
23.3/17.2 & 16.6/11.8 & 14.1/10.9 & 9.8/8.9  & 4.5/7.7 & 2.7/4.4 & 5.2/7.7 & 3.5/4.9 & 5.6/6.0 & 3.4/3.2 \\

parakeet-tdt-1.1b & 
24.5/17.9 & 16.6/12.6 & 13.5/10.6 & 9.1/7.9  & 5.4/7.7 & 3.2/4.2 & 4.4/7.1 & 2.8/4.4 & 5.5/6.0 & 3.3/3.1 \\

parakeet-tdt\_ctc-110m & 
16.6/21.4 & 11.5/15.1 & 15.3/11.5 & 10.7/8.3 & 3.8/7.3 & 2.2/4.0 & 5.2/6.6 & 3.3/4.1 & 7.5/6.2 & 4.6/3.1 \\

seamless-m4t-v2-large & 
35.0/36.3 & 25.1/24.9 & 45.1/43.2 & 35.9/34.7 & 11.7/13.6 & 7.6/8.7 & 26.5/25.5 & 21.0/19.1 & 8.0/7.8 & 5.7/5.0 \\

speechllm-1.5B & 
45.1/44.5 & 32.7/32.0 & 60.3/60.8 & 44.1/46.8 & 10.6/10.9 & 6.2/5.8 & 19.4/17.6 & 14.2/11.9 & 30.8/29.9 & 22.0/21.3 \\

speechllm-2B & 
52.9/53.1 & 36.6/35.7 & 36.9/37.8 & 25.8/27.3 & 14.6/15.3 & 8.1/7.5 & 18.8/16.7 & 12.9/9.4 & 28.7/27.7 & 18.5/17.5 \\

stt\_en\_conformer\_ctc\_large & 
24.2/21.5 & 15.2/13.0 & 12.6/14.7 & 7.6/9.4  & 7.9/7.3 & 4.1/3.4 & 5.9/7.7 & 3.3/4.4 & 6.9/5.3 & 3.9/2.7 \\

stt\_en\_conformer\_ctc\_small & 
31.3/26.9 & 19.0/15.7 & 16.6/17.8 & 9.6/11.7 & 10.0/9.4 & 5.1/4.1 & 8.0/9.9 & 3.9/5.2 & 8.9/7.3 & 5.0/3.8 \\

stt\_en\_fastconformer\_ctc\_large & 
26.9/20.5 & 18.3/12.9 & 15.4/14.0 & 9.9/9.7  & 6.9/8.4 & 3.7/4.1 & 5.7/7.8 & 3.1/4.5 & 6.3/6.0 & 3.8/3.3 \\

stt\_en\_fastconformer\_transducer\_large & 
26.5/20.4 & 19.0/13.8 & 16.9/15.1 & 11.3/11.1 & 6.0/7.9 & 3.4/4.0 & 4.9/7.0 & 2.8/4.4 & 6.7/6.9 & 4.1/3.8 \\

wav2vec2-base-960h & 
44.7/40.1 & 24.5/20.6 & 27.3/28.7 & 13.5/15.7 & 21.5/22.4 & 8.9/8.7 & 13.8/14.7 & 6.1/6.6 & 20.5/19.4 & 9.1/7.8 \\

wav2vec2-conformer-rel-pos-large-960h-ft & 
37.3/34.7 & 21.2/19.7 & 20.3/22.1 & 10.6/13.4 & 12.0/12.2 & 5.2/4.6 & 11.7/12.3 & 6.6/6.3 & 14.8/13.1 & 6.9/5.5 \\

wav2vec2-conformer-rope-large-960h-ft & 
35.3/32.1 & 20.4/19.3 & 20.6/22.1 & 10.4/12.9 & 11.7/12.5 & 5.1/4.8 & 10.9/11.8 & 5.5/6.0 & 14.5/13.4 & 6.9/5.4 \\

wav2vec2-large-960h & 
38.9/35.7 & 21.6/19.2 & 23.2/25.1 & 11.5/13.7 & 16.3/17.1 & 6.9/6.6 & 12.2/13.2 & 5.6/6.1 & 18.1/16.7 & 8.2/7.0 \\

wav2vec2-large-960h-lv60-self & 
32.5/29.4 & 18.7/15.7 & 20.2/20.8 & 10.7/12.9 & 10.4/12.2 & 4.3/4.7 & 9.5/10.5 & 4.2/4.9 & 13.5/12.5 & 6.4/5.2 \\

wav2vec2-large-robust-ft-libri-960h & 
36.2/32.6 & 19.2/16.9 & 20.9/23.0 & 9.6/12.5  & 11.8/12.5 & 5.0/4.8 & 10.6/11.9 & 4.8/5.4 & 15.4/14.0 & 7.0/5.8 \\

whisper-large & 
31.2/32.4 & 24.6/23.0 & 17.6/12.6 & 13.7/10.5 & 3.7/7.4 & 2.1/4.4 & 19.3/16.1 & 14.0/10.3 & 8.9/6.0 & 5.5/3.2 \\

whisper-large-v2 & 
18.8/25.2 & 14.2/18.1 & 18.7/15.1 & 14.8/12.1 & 4.1/7.7 & 2.4/4.8 & 28.3/25.3 & 19.4/14.4 & 8.7/6.8 & 5.5/3.8 \\

whisper-large-v3 & 
20.4/27.5 & 15.6/19.5 & 15.6/11.9 & 11.8/8.7 & 3.2/6.3 & 1.7/4.0 & 10.5/8.3 & 8.7/5.5 & 11.0/8.6 & 7.8/6.1 \\

whisper-large-v3-turbo & 
16.0/23.7 & 12.0/16.5 & 15.3/11.9 & 11.5/9.0 & 3.1/6.3 & 1.7/3.9 & 9.9/8.1 & 8.5/5.2 & 13.3/11.1 & 9.8/7.9 \\

whisper-medium.en & 
20.1/25.1 & 15.3/18.0 & 21.2/16.1 & 16.6/13.2 & 4.0/7.7 & 2.2/5.0 & 17.3/14.6 & 18.3/14.2 & 9.0/7.2 & 5.6/3.2 \\

whisper-small.en & 
21.2/25.1 & 16.5/18.0 & 18.2/14.1 & 13.9/10.9 & 3.9/7.5 & 2.1/4.6 & 10.6/8.3 & 15.6/12.0 & 9.5/8.1 & 5.9/3.4 \\

whisper-tiny & 
30.1/31.9 & 21.7/21.5 & 24.0/20.3 & 17.5/14.4 & 8.1/11.9 & 3.9/6.7 & 17.6/15.3 & 13.0/9.5 & 13.2/11.2 & 7.4/6.3 \\

\bottomrule
\end{tabular}
}
\caption{Actual and approximated WER and CER, separated by a forward slash, across five standard datasets. The regression model is trained on nine datasets and tested on one, with this process repeated for all datasets, ensuring that the test data is always out-of-distribution.}
\label{tab:results_benchmark_part2}
\end{table*}

% Please add the following required packages to your document preamble:
% \usepackage{multirow}
% \usepackage{booktabs} % For \toprule, \midrule, \bottomrule
% \usepackage{array}    % For better alignment

\begin{table}[!hbt]
\centering
\small % Reduce text size (you can also use \footnotesize or \scriptsize)
\renewcommand{\arraystretch}{0.9} % Adjust row height
\setlength{\tabcolsep}{5pt} % Adjust column spacing
\begin{tabular}{lcccc}
\specialrule{2pt}{0.7em}{0.3em} % Custom thick top rule
% \toprule
\multirow{2}{*}{Method} & \multirow{2}{*}{IID} & \multicolumn{3}{c}{OOD}                  \\
                        &                      & D        & M       & D + M \\ \specialrule{2pt}{0.7em}{0.5em} % Custom thick top rule

eWER3(nc=32) & 2.03\textsuperscript{0.07} & 2.09\textsuperscript{0.04} & 2.06 \textsuperscript{0.03} & 2.12\textsuperscript{0.04} \\

eWER3(nc=64) & 1.98\textsuperscript{0.06} & 2.07\textsuperscript{0.05} & 2.00\textsuperscript{0.04} & 2.09\textsuperscript{0.05} \\

\midrule

Base                                    & 1.03\textsuperscript{0.03}          & 1.05\textsuperscript{0.01} & 1.03\textsuperscript{0.02} & 1.07\textsuperscript{0.01}  \\ 

\noalign{\vskip 0.3ex} % extra space above the dashline
\hdashline
\noalign{\vskip 0.6ex} % extra space below the dashline

w/o S                                 & 1.04\textsuperscript{0.03}          & 1.05\textsuperscript{0.01} & 1.04\textsuperscript{0.03} & 1.05\textsuperscript{0.01}  \\ 
w/o PR                                  & 3.13\textsuperscript{0.07}          & 3.22\textsuperscript{0.02} & 3.23\textsuperscript{0.05} & 3.33\textsuperscript{0.02}  \\ 

\noalign{\vskip 0.3ex} % extra space above the dashline
\hdashline
\noalign{\vskip 0.6ex} % extra space below the dashline

w/ MPR (n=2)                            & 1.00\textsuperscript{0.02}          & 1.04\textsuperscript{0.02} & 0.99\textsuperscript{0.02} & 1.06\textsuperscript{0.02}  \\
w/ MPR (n=3)                            & 0.96\textsuperscript{0.02}          & 0.97\textsuperscript{0.01} & 0.95\textsuperscript{0.02} & 0.99\textsuperscript{0.01}  \\
w/ MPR (n=4)                            & 0.95\textsuperscript{0.02}          & 0.96\textsuperscript{0.02} & 0.94\textsuperscript{0.02} & 0.98\textsuperscript{0.02}  \\
w/ MPR (n=5)                            & 0.93\textsuperscript{0.02}          & 0.93\textsuperscript{0.01} & 0.92\textsuperscript{0.02} & 0.95\textsuperscript{0.01}  \\ 

w/MPR (n=10) & 0.90\textsuperscript{0.02} & 0.93\textsuperscript{0.01} & 0.88\textsuperscript{0.02} & 0.95\textsuperscript{0.01} \\
w/MPR (n=20) & 0.89\textsuperscript{0.02} & 0.96\textsuperscript{0.02} & 0.87\textsuperscript{0.02} & 0.96\textsuperscript{0.02} \\

\noalign{\vskip 0.3ex} % extra space above the dashline
\hdashline
\noalign{\vskip 0.6ex} % extra space below the dashline

w/ mMPR (n=3)                           & 0.98\textsuperscript{0.02}          & 0.96\textsuperscript{0.02} & 0.97\textsuperscript{0.02} & 0.98\textsuperscript{0.02}  \\
w/ mMPR (n=5)                           & 0.94\textsuperscript{0.02}          & 0.94\textsuperscript{0.02} & 0.93\textsuperscript{0.01} & 0.96\textsuperscript{0.02}  \\ 

w/mMPR (n=10) & 0.92\textsuperscript{0.02} & 0.94\textsuperscript{0.02} & 0.91\textsuperscript{0.02} & 0.96\textsuperscript{0.02} \\
w/mMPR (n=20) & 1.04\textsuperscript{0.02} & 1.05\textsuperscript{0.01} & 1.02\textsuperscript{0.02} & 1.04\textsuperscript{0.01}
 \\

\noalign{\vskip 0.3ex} % extra space above the dashline
\hdashline
\noalign{\vskip 0.6ex} % extra space below the dashline

\multicolumn{1}{l}{Base (r=17.8)}    & 1.31\textsuperscript{0.04}          & 1.44\textsuperscript{0.02} & 1.31\textsuperscript{0.04} & 1.40\textsuperscript{0.01}  \\
\multicolumn{1}{l}{Base (r=20.1)}    & 1.36\textsuperscript{0.04}          & 1.36\textsuperscript{0.01} & 1.34\textsuperscript{0.03} & 1.34\textsuperscript{0.01}  \\
\multicolumn{1}{l}{Base (r=33.4)}    & 1.55\textsuperscript{0.04}          & 1.69\textsuperscript{0.02} & 1.55\textsuperscript{0.04} & 1.63\textsuperscript{0.02}  \\ 
\multicolumn{1}{l}{Base (r=51.0)}    & 2.03\textsuperscript{0.02}          & 2.10\textsuperscript{0.01} & 2.08\textsuperscript{0.05} & 2.09\textsuperscript{0.01}  \\ 

\noalign{\vskip 0.3ex} % extra space above the dashline
\hdashline
\noalign{\vskip 0.6ex} % extra space below the dashline

\multicolumn{1}{l}{w/o S (r=17.8)} & 1.47\textsuperscript{0.04}          & 1.56\textsuperscript{0.01} & 1.48\textsuperscript{0.04} & 1.54\textsuperscript{0.01}  \\
\multicolumn{1}{l}{w/o S (r=20.1)} & 1.55\textsuperscript{0.02}          & 1.50\textsuperscript{0.01} & 1.55\textsuperscript{0.03} & 1.50\textsuperscript{0.01}  \\
\multicolumn{1}{l}{w/o S (r=33.4)} & 1.79\textsuperscript{0.07}          & 1.89\textsuperscript{0.02} & 1.78\textsuperscript{0.06} & 1.82\textsuperscript{0.02}  \\
\multicolumn{1}{l}{w/o S (r=51.0)} & 2.23\textsuperscript{0.02}          & 2.24\textsuperscript{0.01} & 2.28\textsuperscript{0.04} & 2.21\textsuperscript{0.01} \\ \specialrule{2pt}{0.2em}{0em} % Custom thick top rule


\end{tabular}
\caption{Mean absolute error ($\downarrow$) between predicted word error count and actual error count (in absolute terms) across different configurations. PR - Proxy Reference, S - Similarity, MPR - Multiple PR, D - Data, M - Model. The OOD results are averaged across {\nwilds}~wild datasets. n is the number of proxy references. The r ($\downarrow$) value represents the average WER for proxy reference across 14 datasets.  Superscript represents the standard deviation across five runs.}
\label{tab:ablation_baseline_results}
\end{table}


The quality of references, quantified by the \(r\)-value, also plays a critical role. For example, in IID conditions, the MAE increases from \(1.31\) for \(r=17.8\) to \(2.03\) for \(r=51.0\). A similar trend is observed in OOD \(D+M\), where the MAE rises from \(1.40\) (\(r=17.8\)) to \(2.09\) (\(r=51.0\)). The absence of similarity (\textit{w/o S}) combined with low-quality proxies further degrades performance, underscoring the importance of both high-quality references and similarity measures. These trends are similarly observed for character-level error count approximation, as detailed in Appendix Table~\ref{tab:ablation_baseline_results_cer}.

\begin{table}[ht]
\centering
\caption{Detailed action accuracy with increasing mode size on 10k and 100k games.}
\begin{tabular}{lcc}
\toprule
\textbf{Model Layers} & \textbf{Transformer S-A} & \textbf{\ourmodel S-ASA} \\
\hline
\textit{10k games (660k records)}\\
1 & 21.92 & 11.97 \\
2 & 23.28 & 26.61 \\
 4 & 23.05 & 36.49 \\
 8 & 22.10 & 41.31 \\
 16 & 21.55 & 42.87 \\
 \textit{100k games (6.6M records)}\\
 1 & 29.62 & 11.32 \\
2 & 35.40 & 31.27 \\
4 & 36.93 & 42.57 \\
8 & 36.58 & 48.66 \\
16 & 35.03 & 51.89 \\
 \bottomrule
\end{tabular}
\label{app-tab:scaling}
\end{table}

\paragraph{Scaling Training Data for Regression.}\label{effect_of_scaling_training_data} To evaluate the impact of training data size on the regression model, we scale the data from 1K to 10K examples in increments of 1K. As shown in Figure~\ref{fig:scaling}, the model's performance does not exhibit a clear trend with increasing training data size. Some datasets show slight improvements with more data; others show minimal improvement. This suggests that the regression model is largely agnostic to the size of the training data. In fact, it appears that a relatively small dataset of just 1,000 examples is sufficient to train a robust approximation model. This underscores the model's ability to generalize effectively with limited data, making it an efficient choice for scenarios with constrained datasets.



% \subsection{Qualitative Analysis}
% % Please add the following required packages to your document preamble:
% \usepackage[normalem]{ulem}
% \useunder{\uline}{\ul}{}
\begin{table*}[ht!]
\renewcommand{\arraystretch}{1.3} % Adjust row height for better readability
\setlength{\tabcolsep}{4pt} % Adjust column spacing
\footnotesize % Reduce font size

\begin{tabular}{p{5cm}p{5cm}p{1cm}p{1cm}p{1cm}p{1cm}p{1cm}}
\toprule
Ref                                                                                                                                                                                                                                                                                          & Hypothesis                                                                                                                                                                                                                                                                                           & WER      & aWER  & CER    & aCER  & Sim \\
\midrule
the hrv is not even on                                                                                                                                                                                                                                                                             & H R V IS NOT EVEN ON                                                                                                                                                                                                                                                                                 & 50.00    & 3.26  & 18.18  & 2.61  & 0.71       \\
% because my understanding is that it's because osha is sort of selffunding and they need to generate the revenue to support all these additional inspectors and things that they have decided will be their approach to safety as opposed to the programs like the vpp program and things like that & Because my understanding is that it's because OSHA is sort of self funding and they need to generate the revenue to support all these additional inspectors and things that they have decided will be their approach to safety as opposed to the programs like the VPP program and things like that. & 3.92     & 3.45  & 0.34   & 2.92  & 0.93       \\
CONSEIL WAS MY MANSERVANT                                                                                                                                                                                                                                                                          & conseil was my man's servant                                                                                                                                                                                                                                                                         & 75.00    & 7.39  & 12.00  & 5.54  & 0.88       \\
yeah so book an call support book an appointment to see and i will recommend that meet her face to face                                                                                                                                                                                            & yeah so book an appointment call support book an appointment to see and i would recommend that you do the face to face                                                                                                                                                                               & 23.81    & 19.78 & 22.33  & 11.21 & 0.87       \\
so she has money and then it turns out she's actually uber talented                                                                                                                                                                                                                                & has money she has money yeah and then it turns out she's actually talented                                                                                                                                                                                                                           & 28.57    & 28.63 & 25.37  & 19.12 & 0.56       \\
which is why the november 15th letter was sent asking to open the discussion about the two thousand and eighteen market season                                                                                                                                                                     & which is why the November 15th letter was sent asking to open the discussion about the 2018 market                                                                                                                                                                                                   & 22.73    & 23.45 & 25.40  & 39.91 & 0.91       \\
% Just looking into the fourth quarter firstly, I mean you've reiterated, clearly the guidance, the 10\% grade, you previously specified an H2 grade to get there, I think it would be 11 to 13\% and obviously we can sort of do the calculation around Q4.                                         & Just looking into the fourth quarter firstly, I mean you've reiterated clearly the guidance for the year ten percent growth. I mean, previously specified an H two growth to get there, I think it was eleven percent to thirteen percent and obviously you can sort of do the calculation around Q  & 36.96    & 17.06 & 30.00  & 10.69 & 0.82       \\
We'll continue to look at that and adjust targets as appropriate, but I'm not sure anything changes in terms of the big picture plan for South America. Yes, I mean I would say we are constantly benchmarking each of our countries                                                               & I would say we are constantly benchmarking each of our countries.                                                                                                                                                                                                                                    & 74.42    & 76.63 & 71.56  & 71.40 & 0.37       \\
% also have attached somewhat less less time in answering that question you have said that there is a progress on getting clarity on the guidelines there is an effort on your side and i'm sure that this is the case                                                                               & SO HAVE ATTACHED SOMEWHAT LESS LESS TIME IN ANSWERING THAT QUESTION YOU HAVE SETTLED THAT THERE IS A PROGRESS ON GETTING CLARITY ON THE GUIDE LINES THERE IS AN EFFORT ON YOUR SIDE AND I AM SURE THAT THIS IS THE CASE                                                                              & 12.20    & 26.75 & 4.25   & 14.37 & 0.72       \\
and you give him his carriage with his spats on and his cutaway uh coat you know and his top hat                                                                                                                                                                                                   & H GET IN HIS CARRIAGE WITH HIS SPATS ON AND HIS CUT AWAY COAT YOU KNOW AND HIS TOP HAT AND HIS CHAUFFEUR WOULD TAKE HIM OUT TO WITH THE HORSES NO LESS BEAUTIFULLY                                                                                                                                   & 95.24    & 19.78 & 95.83  & 10.09 & 0.71       \\
okay                                                                                                                                                                                                                                                                                               & I know it's called the "Gateway".                                                                                                                                                                                                                                                                    & 700.00   & 28.94 & 675.00 & 32.85 & 0.33       \\
foreign language.                                                                                                                                                                                                                                                                                  & tangzai sabungi tangzangan and kpanen simbolar to birlivian ain ojong meat aar reanded to it ingaru tampungid debi sampos and tethujun comes west mida                                                                                                                                               & 1,200.00 & 80.39 & 862.50 & 32.26 & 0.22       \\
stronger. our peoples and our economies                                                                                                                                                                                                                                                            & our peoples and our economies are tied                                                                                                                                                                                                                                                               & 50.00    & 14.44 & 47.37  & 8.17  & 0.87       \\
five dollar slaw onion cure                                                                                                                                                                                                                                                                        & \$5.00 Slaw Onion Care                                                                                                                                                                                                                                                                               & 60.00    & 44.22 & 40.74  & 27.24 & 0.74       \\
% foreign language.                                                                                                                                                                                                                                                                                  & thangtai thangtay thangangithanganang cepani simul chum buri lane lane jodong meet arranged to evoke chingaroo tompungi dee thump thump thump thang dee                                                                                                                                              & 1,100.00 & 86.33 & 868.75 & 55.95 & 0.17       \\
ignore\_time\_segment\_in\_scoring                                                                                                                                                                                                                                                                 & Look to Miguel. Farmers like Miguel.                                                                                                                                                                                                                                                                 & 120.00   & 6.84  & 86.67  & 4.02  & 0.92       \\
boy bout seventeen with a red cap no sir indeed i've not  \\   \bottomrule           
\end{tabular}
\caption{
Provide qualitative examples to showcase the effectiveness of our method. 
}
\end{table*}