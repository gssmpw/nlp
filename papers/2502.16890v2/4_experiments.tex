\section{Experiments}
\subsection{Experimental Settings}
This section first introduces the whole experiment settings under a fair comparison.
Secondly, we illustrate the experiment results by comparing \textbf{ReFocus} with the \textbf{TEN} well-acknowledged baselines.
Further, we conducted an ablation study to comprehensively investigate the effectiveness of the `Adaptive Mid-Frequency Energy Optimizer' (\textbf{AMEO}), `Energy-based Key-Frequency Picking Block' (\textbf{EKPB}), and `Key-Frequency Enhanced Training strategy' (\textbf{KET}).

\section{Dataset Generation}
\label{sec:dataset}
\revise{
To train the proposed GNN, we constructed a dataset of building structures and a subset of these structures were subjected to fire simulations using FEA. The dataset generation process is illustrated in \figref{fig:dataset_generation_procedure}. Initially, a total of 33,000 building structures with geometrical details, material properties, and gravity loads were created. Due to randomness in generating these structures, a filter is applied to remove unreasonable data after gravity load simulation, which included 15,377 structures. A trade-off between computational feasibility and model performance is made among the remaining 17,623 structures. As further labeling structures with MIDR requires resource-intensive fire simulations via OpenSeesRT, a large proportion of 16,050 structures is selected as unlabeled dataset. On the other hand, each of the other 1,573 structures was further subjected to 30 different fire simulations, forming the labeled dataset containing $1,573\times 30 = 47,190$ fire cases.} This section details the step-by-step process for generating the dataset, including geometry creation, material property assignment, and simulations due to gravity loads and fire scenarios. 
% To train the proposed neural network, we constructed a dataset comprising building structure data and a subset of fire scenario data. The dataset generation process is illustrated in \figref{fig:dataset_generation_procedure}. 
% A total of 33,000 building structures with geometric details, material properties, and gravity loads were initially created. Out of these, 3,000 structures were selected as labeled data, and the remaining 30,000 were designated as unlabeled data. Further, about half of them filtered out due to instability under gravity loads only. 
\begin{figure*}[h!]
    \centering
    \includegraphics[width=0.8\linewidth]{figures/dataset_filter_procedure.pdf}
    \caption{Workflow for dataset generation (geometry, material property, gravity loads, and fire scenarios).}
    \label{fig:dataset_generation_procedure}
\end{figure*}

\subsection{Geometry Generation}
\label{subsec:geometry_generation}
The geometry of the building structures forms the foundation of the dataset. Regular 
\revise{3D structures} resembling multi-story parking structures or shopping malls were generated, with parameters such as building floor dimensions and story heights selected randomly. Each building structure is composed of multiple rooms, which serve as the basic unit in this study. A room herein is a cuboid space defined by specific length, width, and height. Within a structure, rooms of the same dimensions are uniformly arranged along the length, width, and height, corresponding to the $x$-, $y$-, and $z$-axes, respectively. Structures vary in room size and number of rooms along each axis. Specifically, the room length, width, and height are independently sampled from a uniform distribution within the interval $[2, 5]$ meters along the three directions of the structure. Similarly, the room number along each axis is uniformly sampled independently as an integer within the interval $[2, 7]$, i.e., the maximum number of stories of the buildings simulated in this study is 7.

To introduce variability and simulate real-world scenarios, approximately $8\%$ of structural elements (beams or columns) are randomly removed after initial geometry creation. 
\revise{Such removal is not fire-induced damage, but reflects functional diversity often observed in real buildings, such as open spaces designed for activities in shopping malls, e.g., ice skating rinks. Examples of the generated geometries are illustrated in \figref{fig:example_generated_geometry}, showcasing the diversity and realism of the dataset. This element removal does not affect the definition of room's geometry in the structure and nor does it affect the number of considered fire scenarios.} 

\revise{A range of coefficient of variation values ($3.3\%$ to $17.5\%$) was derived from prior studies that investigated the statistics of geometrical and material properties of structural components of buildings (e.g., \cite{mirza1979variations, lee2004probabilistic}). These studies provide empirical data on the natural variability in parameters such as Young's modulus, yield strength, and dimensions of structural elements due to manufacturing tolerances and material inconsistencies. By selecting $8\%$ for the removal of structural elements in our database, we aimed to maintain a level of variability that is representative of real-world uncertainties while ensuring computational feasibility. This choice ensures that the database captures realistic deviations without introducing extreme cases that may not be commonly encountered in practice.}

\begin{figure*}[h!]
    \centering
    \includegraphics[width=\linewidth]{figures/example_generated_geometry.pdf}
    \caption{Examples of generated structural geometry of different sizes (all dimensions in meters).}
    \label{fig:example_generated_geometry} 
\end{figure*}

{\blockRevise

In this study, we opted for a deterministic square, dimension of $0.1$ m, solid cross-sectional steel elements due to their simplicity in modeling and analysis. Square sections exhibit uniform geometrical properties in all directions, simplifying the computation of structural responses and avoiding complications associated with more complex shapes, such as wide-flange sections, facilitating the computational efficiency and scalability to generate a large dataset. This choice also helps to mitigate issues related to stress concentrations and facilitates a more straightforward representation of structural behavior under thermal loads. 

\textit{Remark:} The selected cross-section provides a comparable flexural rigidity to a $W 130 \times 130 \times 28.1$ wide-flange section (metric units), albeit with significantly higher axial rigidity. This cross-section is acceptable for gravity-load-designed frames under service loading conditions where the models assume fully rigid, moment-resisting beam-column connections for the evaluation of the IDR under thermal loading. This assumption is reasonable in this computational study where the primary interest is to understand the global deformation response of frames under fire conditions. The selection of uniform square cross-sections for both beams and columns, rather than adherence to standard capacity design principles, was made here primarily for computational efficiency and to reduce design parameters in the database generation process. This choice allows for simplified and scalable approach to analyze the fire-induced response of generic steel frames without the need for large section variations, where this study mainly focuses on the fire vulnerability assessment using ML-based predictions. However, if additional loading conditions, e.g., seismic or wind loads, were to be considered, larger sections, strong-column/weak-beam principle, and ductile detailing would be required in the generated buildings for realistic structural behavior under combined loading conditions. Future studies may also consider investigating the influence of variable cross-sectional dimensions and semi-rigid connections on the structural performance under fire conditions. 
} % blockRevise

\subsection{Material Properties}
Steel is chosen as the material for the structures. To reflect real-world variations, we randomly assign one of five slightly different steel material types to each structural element. \revise{
The ranges of material properties are provided in \tabref{tab:material_property_ranges} and the properties are sampled from uniform distributions of the corresponding ranges. These variations simulate differences arising from manufacturing batches or regional material properties. That these properties are at ambient temperature and change when the temperature rises due to a fire. The selection of materials with varying properties is aimed at increasing the diversity of the data. Our goal is to represent as wide a range of data as possible with a limited amount of building structure data, thereby enhancing the generalization ability of the GNN. Our assumed material property ranges are expected to be wider than the real-world conditions based on findings in \cite{mirza1979variations, lee2004probabilistic}. Therefore, we are essentially tackling a more challenging and general task. If we can solve this problem, we are confident that our method will perform equally well or even better in real-world scenarios.
}
\begin{table}[h!]
    \centering
    \caption{Material properties ranges for considered steel structures.}
    \begin{tabular}{lc}
        \toprule
        Property & Range \\
        \midrule
        Young's modulus & [168, 252] GPa \\
        Yield strength & [220, 330] MPa \\
        Strain-hardening ratio & [0.8, 1.2] \% \\
        \bottomrule
    \end{tabular}
    \label{tab:material_property_ranges}
\end{table}

\subsection{Gravity Loads}
Gravity loads are applied to columns and beams based on their \revise{influence (tributary) areas as typically conducted in structural analysis. The considered ``service'' load conditions include the column self-weight and the additional loads directly supported on the beams from their self-weight and weights of the reinforced concrete slabs, people as live load, and building content. An edge beam typically carries approximately half the gravity load supported by a parallel interior beam}. The ranges of gravity loads are listed in \tabref{tab:gravity_load_ranges}. \revise{The loads are sampled from uniform distributions of the corresponding ranges.} Structures that failed to meet an MIDR threshold of $1\%$ under gravity loads were deemed unacceptable designs and filtered out, as such configurations of randomly chosen geometry, material, and gravity load combinations were considered unrealistic from a regulatory and practicality points of view.
\begin{table}[h!]
    \centering
    \caption{Gravity load ranges for considered beams and columns.}
    \begin{tabular}{lc}
        \toprule
        Element & Range (kN/m)  \\
        \midrule
        Column & [0.5, 1.0]  \\
        Edge beam & [1.5, 4.5]  \\
        Interior beam & [3.0, 7.5]  \\
        \bottomrule
    \end{tabular}
    \label{tab:gravity_load_ranges}
\end{table} 

\subsection{Rule-based Thermal Load Generation}
\label{subsec:thermal_load_generation}
To evaluate a building's structural response during a fire event, we employed a simplified rule-based approach for thermal load generation. 
% Previous studies \cite{nan_structuralfire_2023} have demonstrated that steel structures rapidly equilibrate with surrounding gases temperatures due to efficient heat exchange. Consequently, gas temperatures can be directly used as inputs for FEA tools, e.g., OpenSees, simplifying the process of modeling thermal loads. 
% Accurately simulating temperature fields in fire scenarios poses significant challenges. Advanced thermodynamic simulations, such as those performed using Fire Dynamics Simulator (FDS) \cite{mcgrattan_fire_2000}, provide precise temperature predictions. However, these methods are hindered by high computational costs, prolonging execution times, and limited scalability, making them impractical for generating large datasets. Additionally, real-world fire loads often display substantial spatial variability across different rooms \cite{dundar_fire_2023}, resulting in scenario-specific temperature fields with limited generalizability. For example, studies on bridge fires \cite{he_study_2024} have demonstrated that environmental factors, such as wind speeds, can significantly influence temperature distributions. Furthermore, even within identical scenarios, variations in fire modeling methodologies can produce distinctly different temperature fields \cite{zhang_temperature_2020, du_new_2012}. These challenges emphasize the need for efficient and adaptable methods to generate fire temperature data.
% To address these issues, we adopted a rule-based approach to model temperature variations. 
According to \cite{spearpoint_fire_2008}, a typical fire development follows a predictable pattern. During the {\em{growth stage}}, the temperature rises slowly and approximately linearly after ignition. This is followed by the {\em{flashover stage}}, where temperatures increase rapidly to peak values. After reaching the peak, the temperature either stabilizes or continues to rise slowly until the {\em{decay stage}} begins. Inspired by this fire development pattern, we describe the temperature evolution in time, $t$, prior to the decay stage in two distinct stages:
\begin{enumerate}
    \item {\bf{Initial linear increase stage}}: For $t \in [0, t_1)$, temperature increases gradually and linearly as the fire spreads through the building. This stage represents the time before the fire directly affects a structural element.  
    \item {\bf{ISO 834 fire curve stage}}: For $t \in [t_1, t_{\thre}]$, temperature rises rapidly following the ISO 834 curve \cite{ISO834}, modeling the direct impact of the fire on the structural element. 
\end{enumerate}
The slope of the linear temperature increase, $c$, and the transition time, $t_1$, are influenced by the spatial relationship between the fire source and the structural element. For the second stage of temperature evolution, we utilize the ISO 834 curve, a widely accepted standard for fire resistance testing. This standardized fire curve describes the temperature rise over time, enabling rapid and consistent thermal fields across various scenarios. The duration of fire simulation in this study is set to $t_{\thre}=60$ minutes. This value represents the upper limit for the temperature evolution of each structural element, providing a consistent basis for analyzing the structural response to fire.

Let $(x, y, z)$ represents the midpoint of a structural element and $(x_{\subfire}, y_{\subfire}, z_{\subfire})$ the fire source point. \revise{Integer parameters $h$ and $h_{\subfire}$ correspond to the respective floor levels of the element and the fire source}. The temperature evolution for each element is expressed as follows:
\begin{enumerate}
    \item Linear increase stage ($0 < t < t_1$):
    \begin{equation}
    T(t) = c \cdot t,
    \end{equation}
    where $c$, the rate of temperature increase ($^\circ\mathrm{C}/\mathrm{min}$), depends on the height difference between the element, $h$, and the fire source, $h_{\subfire}$:
    \begin{equation}
        c = 
        \begin{cases} 
        5\left/\left(h - h_{\subfire} + 1\right)\right., & h \geq h_{\subfire}, \\
        2\left/\left(h_{\subfire} - h\right)\right., & h < h_{\subfire}.
        \end{cases}
    \end{equation}
     \item ISO 834 stage ($t \geq t_1$):
\begin{equation}
    T(t) = c \cdot t_1 + 345 \log_{10} \left(8 \left(t - t_1\right) + 1\right).
\end{equation}
\end{enumerate}

The transition (arrival) time $t_1$, marking the end of the linear stage, depends on the spatial distance between the fire source and the element. We define the following two Euclidean distances $L_p$ in the $xy$ plane and $L_s$ in the $xyz$ space:
\begin{eqnarray}
L_p & \triangleq & \sqrt{(x - x_{\subfire})^2 + (y - y_{\subfire})^2}, \\
\label{eq:Lp}
L_s & \triangleq & \sqrt{(x - x_{\subfire})^2 + (y - y_{\subfire})^2 + (z - z_{\subfire})^2}.
\label{eq:Ls}
\end{eqnarray}
Accordingly, the transition time, $t_1$, is expressed as follows:
\begin{equation}
    t_1 = 
    \begin{cases}
    \beta_{1} \cdot \left(1 - \exp\left\{- L_s\left/\alpha_{1}\right.\right\}\right), & h > h_{\subfire}, \\
    \beta_{2} \cdot \left(1 - \exp\left\{- L_p\left/\alpha_{2}\right.\right\}\right), & h = h_{\subfire}, \\
    \beta_{3} \cdot \left(1 - \exp\left\{- L_s\left/\alpha_{3}\right.\right\}\right), & h < h_{\subfire} .
    \end{cases}
    \label{eq:t1}
\end{equation}
The parameters $\beta_i$ and $\alpha_i$ for determining $t_1$ are summarized in Table~\ref{tab:fire_spread_parameters}. In this study, we take $r_{\mathrm{up}}=0.95$ and $r_{\mathrm{down}}=0.97$.
\begin{table}[ht]
    \centering
    \caption{Fire spread parameters for $t_1$ calculations.}
    \begin{tabular}{lcc}
        \toprule
        Case  & $\beta_i$ & $\alpha_i$  \\
        \midrule
        $i=1$, Upward spread & $16 \left.\left(1-r_{\mathrm{up}}^{\left|h-h_{\subfire}\right|}\right)\right/\left(1-r_{\mathrm{up}}\right)$ & $10$  \\
        $i=2$, Horizontal spread & $18$ & $18$  \\
        $i=3$, Downward spread & $30 \left.\left(1-r_{\mathrm{down}}^{\left|h-h_{\subfire}\right|}\right)\right/\left(1-r_{\mathrm{down}}\right)$ & $5$  \\
        \bottomrule
    \end{tabular}
    \label{tab:fire_spread_parameters}
\end{table}

\figref{fig:t1_curve} illustrates the $t_1$ curves for various fire scenarios: (1) fire originating on the lower floor, $h-h_{\subfire}=1$ with rapid upward spread, (2) fire on the same floor, $h=h_{\subfire}$ with the fastest spread, and (3) fire on the upper floor, $h_{\subfire}-h=1$ with slow downward spread. The exponential decay in $t_1$ reflects the accelerating fire propagation speed as the distance increases. \figref{fig:t1_curve} also indicates that the employed simplified model is consistent with the Markov chain-based dynamic model given by \cite{cheng_dynamic_2011}, where the rooms at the same floor of the fire point start flashover slightly before the corresponding upper floors. Additionally, $\beta_{1}$ and $\beta_{3}$ are the summation of a geometric sequence, where story level $h$ is the index. The common ratios $r_{\mathrm{up}}<1$ in $\beta_{1}$ and $r_{\mathrm{down}}<1$ in $\beta_{3}$ indicate that the fire speeds up to spread through the next story, which is consistent with the real-world fire spread mechanism given in \cite{hokugo_mechanism_2000}. The temperature profile within the range $t \in [0, t_{\thre}]$ is subsequently used as the thermal load in OpenSeesRT simulations to compute displacements at each structural node at time $t_{\thre}$.
\begin{figure}[h!]
    \centering
    \includegraphics[width=0.8\linewidth]{figures/m204_t1_curve.pdf}
    \caption{Three examples for the $t_1$ curve.}
    \label{fig:t1_curve}
\end{figure}

\revise{
\textit{Remark:} The effects of structural elements, such as concrete floor slabs and partitions, are not explicitly modeled in our approach. Instead, their influence is implicitly captured through the careful selection of the parameters $ \alpha, \beta, r_\mathrm{up} $, and $ r_\mathrm{down} $. This parameterization provides a unified framework for generating temperature fields. Indeed, fire propagation is governed by a multitude of factors and remains an open research question. For instance, if the fire resistance of a floor slab is enhanced by fire protective coating, the corresponding model can account for this by decreasing $\alpha_1$ \& $\alpha_3$, increasing $\beta_1$ \& $\beta_3$, and adopting larger values for $r_\mathrm{up}$ \& $r_\mathrm{down}$, which collectively slow down the vertical spread of fire. Conversely, scenarios involving higher amounts of combustible materials would warrant the opposite adjustments. This flexible and integrated approach avoids the need to design separate models for different fire propagation scenarios while still capturing the essential effects.
}

\revise{
In conclusion, our rule-based approach is a computationally efficient method for approximating fire temperature fields, enabling large-scale dataset generation to train predictive models. By combining ISO 834 fire curves with spatial considerations and embedding structural effects through parameter calibration, the method achieves a balanced trade-off between accuracy and scalability, making it a practical solution for thermal load modeling in fire scenarios. After generating the temperature of each beam or column according to the middle point, the temperature is applied as uniform thermal load to the elements of the structure in question using OpenSeesRT. 
}

% In conclusion, this rule-based approach is a computationally efficient method to approximate fire temperature fields, enabling large-scale dataset generation to train predictive models. By combining ISO 834 fire curves with spatial considerations, the method balances accuracy and scalability, making it a practical solution for thermal load modeling in fire scenarios.

% \subsection{Interstory Drift Ratio}
\subsection{OpenSeesRT Simulation}
\label{subsec:opensees_simulation}

The thermal and mechanical responses of 3D frame structures under combined fire and gravity loads are simulated using OpenSeesRT \cite{perez2024openseesrt}. \revise{In the simulation, the IDR of each node at $t_{\thre}$ is computed using the computed nodal displacements. Each structural model features six degrees of freedom per node (3 translational  and 3 rotational), with linear geometrical transformations (\texttt{geomTransf: Linear}) defining how the element local coordinate systems are mapped to the global coordinate system and assuming small displacements and rotations. Although OpenSeesRT allows a variety of options for modeling finite deformations, in the present simulations and mainly for simplicity, we did not consider large deformations. All bottom nodes (nodes on the ground) are fully constrained in all six degrees of freedom, while degrees of freedom os all other nodes are free.} Material behavior is temperature-dependent and modeled with \texttt{Steel01Thermal}, while fiber-based sections (\texttt{FiberThermal}) capture nonlinear interactions between thermal and mechanical responses at the cross-section level. \revise{Structural elements are represented as displacement-based Euler-Bernoulli beam-columns (\texttt{dispBeamColumnThermal}). This element  formulation accounts for thermal strains (temperature gradients) in the section, which is discretized into fibers. Numerical integration is used along the length of each element using three integration (Gauss) points, one at each end and the third in the middle of the element.}

{\revise{Thermal expansion of steel members plays a crucial role in IDR development. In reality, reinforced concrete floor slabs heat at a different rate than steel members due to their higher thermal mass and lower thermal conductivity. This differential heating can lead to restrained thermal expansion, introducing axial compression in beams and affecting the overall structural response. In this study, explicit {\em{composite action}} between steel members and concrete slabs is not modeled. Instead, our approach focuses on isolating the response of the steel structural frame, which is often the critical load-bearing component in fire scenarios. This assumption aligns with prior studies \cite{Possidente_2024} demonstrating that steel structures reach thermal equilibrium with surrounding gases quickly, allowing the use of uniform thermal loading in fire analysis. Future work could enhance this framework by incorporating slab-beam interaction effects, through a refined FEA for an extended dataset where constraints imposed by floor slabs are explicitly considered.}

The analysis begins with the application of gravity loads, followed by incremental thermal loads simulating the fire exposure. A static nonlinear solver using  \texttt{ExpressNewton} algorithm ensures convergence, while the \texttt{NormDispIncr} test maintains accuracy. An incremental \texttt{LoadControl} scheme with small step sizes is employed to guarantee numerical stability, using 10\% for gravity loads and 1\% for thermal loads. 

\revise{
In the thermal load analysis, uniform thermal load is applied to each beam or column, i.e., the temperature of each element is set to be that at the middle point, according to \secref{subsec:thermal_load_generation}. The \texttt{Steel01Thermal} material allows the properties (e.g., Young's modulus and yield strength) to be adjusted at increasing temperatures according to \cite{EN1993} using its Table 3.1: Reduction factors for the stress-strain relationship of carbon steel at elevated temperatures. For example, if the Young’s modulus at ambient temperature is $E_0$, then as the temperature ($T$) increases, the modulus changes as $E(T) = \eta (T) \times E_0$. \cite{EN1993} directly provides the values of $\eta(T) \in \left[0,1\right] $ at every $100 ^\circ\mathrm{C}$ interval and recommends using linear interpolation to obtain $\eta(T)$ for intermediate values of $T$.
} OpenSeesRT documentation \cite{OpenSeesThermalExamples} provides several examples of thermal analyses.

This modeling framework accommodates variations in material properties, cross-sectional geometries, and temperature profiles, providing robust simulations of structural behavior under fire conditions. The primary settings and configurations for the OpenSeesRT simulations are summarized in \tabref{tab:ops_detail}.
\begin{table}[h!]
    \centering
        \caption{Key settings of OpenSeesRT simulations.}
    \begin{tabular}{l|>{\raggedright\arraybackslash}p{0.6\linewidth}} %
    \toprule
    Modeling Aspect     & Details \\
    \midrule
    Geometry            & 3D models; 6 degrees of freedom per node \\
    Transformation      & geomTransf: Linear \\ 
    Material            & Steel01Thermal \\
    Section             & FiberThermal; Cross-section: $0.1$ m $\times$ $0.1$ m \\ 
    Element type        & {dispBeamColumnThermal} \\ 
    Loading             & Gravity loads: {beamUniform}; Thermal loads: {beamThermal} \\
    Integration scheme  & Incremental {LoadControl}; Step size: $10\%$ (gravity analysis), $1\%$ (thermal analysis) \\
    Nonlinear solver    & {ExpressNewton} algorithm; {UmfPack} solver; Convergence test: {NormDispIncr} tolerance: $10^{-8}$; Maximum \# iterations per step: $1000$. \\ 
    \bottomrule
    \end{tabular}
    \label{tab:ops_detail}
\end{table}

For each structure in the labeled dataset, 30 fire points are selected using a dual-granularity approach, \revise{i.e., two-stage sampling strategy,} to ensure they are well-distributed. Specifically, rooms are sequentially selected, with one fire point randomly chosen within each selected room. If a building is large and contains more than 30 rooms, we randomly select 30 rooms without replacement, i.e., ensuring that no more than one fire point is located in the same room. Conversely, if the building is small and has fewer than 30 rooms, all rooms are initially selected, with one fire point randomly assigned to each room. Additionally, rooms are then selected with replacement until a total of 30 fire points are assigned. \revise{The room-level sampling prioritizes selecting distinct rooms to avoid spatial clustering of fire points, while the point-level sampling ensures intra-room variability. This approach aligns with stratified sampling principles commonly used for efficient spatial representation, where multi-stage sampling strategies optimize coverage and variability, e.g., \cite{arunachalam_generalized_2023}, and enables a more comprehensive characterizing of how the structures respond under fire conditions.}
% This selection method prevents fire points from clustering too closely while maintaining an element of randomness. By distributing fire points in this manner, the 30 fire scenarios are effectively utilized, enabling a more comprehensive characterizing of how the structures respond under fire conditions.

\subsection{Summary of the Dataset Generation}
As discussed in this section and related to  \figref{fig:dataset_generation_procedure}, three key steps were considered in the development of the dataset: 
\begin{enumerate}
    \item {\bf{Filtering process}}: Structures with MIDR exceeding $1\%$ under gravity loads were excluded,  resulting in $1,573$ labeled structures retained for fire simulation and $16,050$ unlabeled structures for training the MFSP predictor.
    \item {\bf{Fire simulations}}: For each retained labeled structure, 30 fire scenarios were simulated using OpenSeesRT, yielding $47,190$ fire cases.
    \item {\bf{Data distribution check}}: MIDR distributions for labeled and unlabeled data under gravity loads were highly similar, because both datasets were generated using the same method. Under fire conditions, the MIDR distribution shifted, reflecting significant structural deformation with values reaching a maximum of about 6\%, an average of 1.70\%, and a standard deviation of 1.12\%. This step ensured a diverse and comprehensive dataset for the proposed predictive framework.
\end{enumerate}
The statistical distribution histograms for MIDR (after applying the $1\%$ filtering threshold \revise{for gravity load responses}) under different loading conditions are plotted in \figref{fig:histogram_mdr}. Figures \ref{fig:histogram_mdr}(a) and \ref{fig:histogram_mdr}(b) show the MIDR distributions of the labeled and unlabeled data, respectively, under gravity loads only. \figref{fig:histogram_mdr}(c) shows the MIDR distribution of the labeled data under the combined effects of gravity and fire loads. Fire load causes the structures to significantly deform, leading to a noticeably \revise{right-skewed} MIDR distribution.

\begin{figure*}[h!]
    \centering
    \includegraphics[width=\linewidth]{figures/histogram_mdr.pdf}
    \caption{Histograms of MIDR for labeled and unlabeled structures with gravity loads and fire cases.}
    \label{fig:histogram_mdr}
\end{figure*}

\revise{
This dataset provides the basis for training and testing the performance of the GNN-based framework. Although we employed a simplified rule-based thermal load generation method compared with conventional CFD-based simulations, the temperature field, the changes of the material properties, and the response of the structures, are all still highly nonlinear and complex. Therefore, it is still a challenging task for the NN to predict the MIDRs based on this dataset.
}
\textbf{Datasets \quad} We conduct extensive experiments on selected \textbf{Eight} widely-used real-world multivariate time series forecasting datasets, including Electricity Transformer Temperature (ETTh1, ETTh2, ETTm1, and ETTm2), Electricity, Traffic, Weather used by Autoformer~\citep{Wu2021autoformer}, and Solar\_Energy datasets proposed in LSTNet~\citep{Lai2018lstnet}.
For a fair comparison, we follow the same standard protocol~\citep{LiuiTransformer} and split all forecasting datasets into training, validation, and test sets by the ratio of 6:2:2 for the ETT dataset and 7:1:2 for the other datasets. The characteristics of these datasets are shown in Table \ref{tab:datasets} (More can be found in the Appendix). 

\textbf{Evaluation protocol \quad} Following TimesNet~\citep{wu2022timesnet}, we use Mean Squared Error (MSE) and Mean Absolute Error (MAE) for the evaluation. We follow the same evaluation protocol, where the input length is set as $T=96$ and the forecasting lengths $F \in \{96, 192, 336, 720\}$. All the experiments are conducted on a single NVIDIA GeForce RTX 4090
with 24G VRAM. The MSE loss function is utilized for model optimization. To foster reproducibility, we make our code, and training scripts available in this \textbf{GitHub Repository}\footnote{\url{https://github.com/Levi-Ackman/ReFocus}}. Full implementation details and other information are in Appendix~\ref{app:exp}.

\textbf{Baseline setting \quad} We carefully choose \textbf{TEN} well-acknowledged forecasting models as our baselines, including 1) Transformer-based methods: iTransformer~\citep{LiuiTransformer}, Crossformer~\citep{zhang2023crossformer}, PatchTST~\citep{Nie2022patchtst}; 2) Linear-based methods: TSMixer~\citep{chen2023tsmixer}, DLinear~\citep{zeng2023dlinear}; 3) TCN-based methods: TimesNet~\citep{wu2022timesnet}, ModernTCN~\citep{donghao2024moderntcn}; 4)Recent cutting-edge frequency-based methods that discussed earlier: FilterNet~\citep{yi2024filternet}, FITS~\citep{xu2024fits}, FreTS~\citep{yi2023fremlp}. These models represent the latest advancements in multivariate time series forecasting and encompass all mainstream prediction model types. The results of ModernTCN, FilterNet, FITS, and FreTS are taken from FilterNet~\citep{yi2024filternet} and other results are taken from iTransformer~\citep{LiuiTransformer}.


\subsection{Experiment Results}
\begin{table*}[th]
  \caption{\small{Multivariate forecasting results with prediction lengths $F\in\{96, 192, 336, 720\}$ and fixed lookback length $T=96$. Results are averaged from all prediction lengths. The best is \boldres{Red} and the second is \secondres{Blue}. The \textbf{Lower} MSE/MAE indicates the better prediction result. Full results are in Appendix~\ref{app:full bench}.}}
  \label{tab:bench_avg}
  \renewcommand{\arraystretch}{0.85} 
  \centering
  \resizebox{1\textwidth}{!}{
  \begin{small}
  \renewcommand{\multirowsetup}{\centering}
  \setlength{\tabcolsep}{1.45pt}
  \begin{tabular}{c|cc|cc|cc|cc|cc|cc|cc|cc|cc|cc|cc}
    \toprule
    {Models} & 
    \multicolumn{2}{c}{\rotatebox{0}{\scalebox{0.75}{\textbf{ReFocus}}}} &
    \multicolumn{2}{c}{\rotatebox{0}{\scalebox{0.8}{FilterNet}}} &
    \multicolumn{2}{c}{\rotatebox{0}{\scalebox{0.8}{iTransformer}}} &
    \multicolumn{2}{c}{\rotatebox{0}{\scalebox{0.8}{ModernTCN}}} &
    \multicolumn{2}{c}{\rotatebox{0}{\scalebox{0.8}{FITS}}} &
    \multicolumn{2}{c}{\rotatebox{0}{\scalebox{0.8}{PatchTST}}} &
    \multicolumn{2}{c}{\rotatebox{0}{\scalebox{0.8}{Crossformer}}} &
    \multicolumn{2}{c}{\rotatebox{0}{\scalebox{0.8}{TimesNet}}} &
    \multicolumn{2}{c}{\rotatebox{0}{\scalebox{0.8}{{TSMixer}}}} &
    \multicolumn{2}{c}{\rotatebox{0}{\scalebox{0.8}{DLinear}}} &
    \multicolumn{2}{c}{\rotatebox{0}{\scalebox{0.8}{FreTS}}}  \\
     &
    \multicolumn{2}{c}{\scalebox{0.8}{\textbf{(Ours)}}} &
    \multicolumn{2}{c}{\scalebox{0.8}{\citeyearpar{yi2024filternet}}} &
    \multicolumn{2}{c}{\scalebox{0.8}{\citeyearpar{LiuiTransformer}}} & 
    \multicolumn{2}{c}{\scalebox{0.8}{\citeyearpar{donghao2024moderntcn}}} & 
    \multicolumn{2}{c}{\scalebox{0.8}{\citeyearpar{xu2024fits}}} & 
    \multicolumn{2}{c}{\scalebox{0.8}{\citeyearpar{Nie2022patchtst}}} & 
    \multicolumn{2}{c}{\scalebox{0.8}{\citeyearpar{zhang2023crossformer}}} &
    \multicolumn{2}{c}{\scalebox{0.8}{\citeyearpar{wu2022timesnet}}} & 
    \multicolumn{2}{c}{\scalebox{0.8}{\citeyearpar{chen2023tsmixer}}} & 
    \multicolumn{2}{c}{\scalebox{0.8}{\citeyearpar{zeng2023dlinear}}} &
    \multicolumn{2}{c}{\scalebox{0.8}{\citeyearpar{yi2023fremlp}}} \\
    \cmidrule(lr){2-3} \cmidrule(lr){4-5}\cmidrule(lr){6-7} \cmidrule(lr){8-9}\cmidrule(lr){10-11}\cmidrule(lr){12-13} \cmidrule(lr){14-15} \cmidrule(lr){16-17} \cmidrule(lr){18-19} \cmidrule(lr){20-21} \cmidrule(lr){22-23}
    {Metric}  & \scalebox{0.8}{MSE} & \scalebox{0.8}{MAE}  & \scalebox{0.8}{MSE} & \scalebox{0.8}{MAE}  & \scalebox{0.8}{MSE} & \scalebox{0.8}{MAE}  & \scalebox{0.8}{MSE} & \scalebox{0.8}{MAE}  & \scalebox{0.8}{MSE} & \scalebox{0.8}{MAE}  & \scalebox{0.8}{MSE} & \scalebox{0.8}{MAE} & \scalebox{0.8}{MSE} & \scalebox{0.8}{MAE} & \scalebox{0.8}{MSE} & \scalebox{0.8}{MAE} & \scalebox{0.8}{MSE} & \scalebox{0.8}{MAE} & \scalebox{0.8}{MSE} & \scalebox{0.8}{MAE} & \scalebox{0.8}{MSE} & \scalebox{0.8}{MAE} \\
    
    \toprule
    \scalebox{0.95}{ETTm1} & \boldres{\scalebox{0.78}{0.387}} & \boldres{\scalebox{0.78}{0.394}} & \scalebox{0.78}{0.392} & \scalebox{0.78}{0.401} & \scalebox{0.78}{0.407} & \scalebox{0.78}{0.410} & \scalebox{0.78}{0.389} & \scalebox{0.78}{0.402} & \scalebox{0.78}{0.415} & \scalebox{0.78}{0.408} & \secondres{\scalebox{0.78}{0.387}} & \secondres{\scalebox{0.78}{0.400}} & \scalebox{0.78}{0.513} & \scalebox{0.78}{0.496} & \scalebox{0.78}{0.400} & \scalebox{0.78}{0.406} & \scalebox{0.78}{0.398} & \scalebox{0.78}{0.407} & \scalebox{0.78}{0.403} & \scalebox{0.78}{0.407}  & \scalebox{0.78}{0.408} & \scalebox{0.78}{0.416} \\

    \midrule
    \scalebox{0.95}{ETTm2} & \boldres{\scalebox{0.78}{0.275}} & \boldres{\scalebox{0.78}{0.320}} & \scalebox{0.78}{0.285} & \scalebox{0.78}{0.328} & \scalebox{0.78}{0.288} & \scalebox{0.78}{0.332} & \secondres{\scalebox{0.78}{0.279}} & \secondres{\scalebox{0.78}{0.322}} & \scalebox{0.78}{0.286} & \scalebox{0.78}{0.328} & \scalebox{0.78}{0.281} & \scalebox{0.78}{0.326} & \scalebox{0.78}{0.757} & \scalebox{0.78}{0.610} & \scalebox{0.78}{0.291} & \scalebox{0.78}{0.333} & \scalebox{0.78}{0.289} & \scalebox{0.78}{0.333} & \scalebox{0.78}{0.350} & \scalebox{0.78}{0.401} & \scalebox{0.78}{0.321} & \scalebox{0.78}{0.368} \\
    
    \midrule
    \scalebox{0.95}{ETTh1} & \boldres{\scalebox{0.78}{0.434}} & \boldres{\scalebox{0.78}{0.433}} & \secondres{\scalebox{0.78}{0.441}} & \scalebox{0.78}{0.439} & \scalebox{0.78}{0.454} & \scalebox{0.78}{0.447} & \scalebox{0.78}{0.446} & \secondres{\scalebox{0.78}{0.433}} & \scalebox{0.78}{0.451} & \scalebox{0.78}{0.440} & \scalebox{0.78}{0.469} & \scalebox{0.78}{0.454} & \scalebox{0.78}{0.529} & \scalebox{0.78}{0.522} & \scalebox{0.78}{0.458} & \scalebox{0.78}{0.450} & \scalebox{0.78}{0.463} & \scalebox{0.78}{0.452} & \scalebox{0.78}{0.456} & \scalebox{0.78}{0.452} & \scalebox{0.78}{0.475} & \scalebox{0.78}{0.463} \\

    \midrule
    \scalebox{0.95}{ETTh2} & \boldres{\scalebox{0.78}{0.371}} & \boldres{\scalebox{0.78}{0.396}} & \scalebox{0.78}{0.383} & \scalebox{0.78}{0.407} & \scalebox{0.78}{0.383} & \scalebox{0.78}{0.407} & \secondres{\scalebox{0.78}{0.382}} & \secondres{\scalebox{0.78}{0.404}} & \scalebox{0.78}{0.383} & \scalebox{0.78}{0.408} & \scalebox{0.78}{0.387} & \scalebox{0.78}{0.407} & \scalebox{0.78}{0.942} & \scalebox{0.78}{0.684} & \scalebox{0.78}{0.414} & \scalebox{0.78}{0.427} & \scalebox{0.78}{0.401} & \scalebox{0.78}{0.417} & \scalebox{0.78}{0.559} & \scalebox{0.78}{0.515} & \scalebox{0.78}{0.472} & \scalebox{0.78}{0.465} \\
    
    \midrule
    \scalebox{0.95}{ECL} & \boldres{\scalebox{0.78}{0.168}} & \boldres{\scalebox{0.78}{0.262}} & \secondres{\scalebox{0.78}{0.173}} & \secondres{\scalebox{0.78}{0.268}} & \scalebox{0.78}{0.178} & \scalebox{0.78}{0.270} & \scalebox{0.78}{0.197} & \scalebox{0.78}{0.282} & \scalebox{0.78}{0.217} & \scalebox{0.78}{0.295} & \scalebox{0.78}{0.205} & \scalebox{0.78}{0.290} & \scalebox{0.78}{0.244} & \scalebox{0.78}{0.334} & \scalebox{0.78}{0.192} & \scalebox{0.78}{0.295} & \scalebox{0.78}{0.186} & \scalebox{0.78}{0.287} & \scalebox{0.78}{0.212} & \scalebox{0.78}{0.300} & \scalebox{0.78}{0.189} & \scalebox{0.78}{0.278} \\

    \midrule
    \scalebox{0.95}{Traffic} & \boldres{\scalebox{0.78}{0.412}} & \boldres{\scalebox{0.78}{0.265}} & \scalebox{0.78}{0.463} & \scalebox{0.78}{0.310} & \secondres{\scalebox{0.78}{0.428}} & \secondres{\scalebox{0.78}{0.282}} & \scalebox{0.78}{0.546} & \scalebox{0.78}{0.348} & \scalebox{0.78}{0.627} & \scalebox{0.78}{0.376} & \scalebox{0.78}{0.481} & \scalebox{0.78}{0.304} & \scalebox{0.78}{0.550} & \scalebox{0.78}{0.304} & \scalebox{0.78}{0.620} & \scalebox{0.78}{0.336} & \scalebox{0.78}{0.522} & \scalebox{0.78}{0.357} & \scalebox{0.78}{0.625} & \scalebox{0.78}{0.383} & \scalebox{0.78}{0.618} & \scalebox{0.78}{0.390} \\


    \midrule
    \scalebox{0.95}{Weather} & \boldres{\scalebox{0.78}{0.245}} & \boldres{\scalebox{0.78}{0.271}} & \secondres{\scalebox{0.78}{0.245}} & \secondres{\scalebox{0.78}{0.272}} & \scalebox{0.78}{0.258} & \scalebox{0.78}{0.279} & \scalebox{0.78}{0.247} & \scalebox{0.78}{0.272} & \scalebox{0.78}{0.249} & \scalebox{0.78}{0.276} & \scalebox{0.78}{0.259} & \scalebox{0.78}{0.281} & \scalebox{0.78}{0.259} & \scalebox{0.78}{0.315} & \scalebox{0.78}{0.259} & \scalebox{0.78}{0.287} & \scalebox{0.78}{0.256} & \scalebox{0.78}{0.279} & \scalebox{0.78}{0.265} & \scalebox{0.78}{0.317} & \scalebox{0.78}{0.250} & \scalebox{0.78}{0.270} \\

    \midrule
    \scalebox{0.95}{Solar\_Energy} & \boldres{\scalebox{0.78}{0.222}} & \boldres{\scalebox{0.78}{0.252}} & \scalebox{0.78}{0.243} & \scalebox{0.78}{0.281} & \secondres{\scalebox{0.78}{0.233}} & \secondres{\scalebox{0.78}{0.262}} & \scalebox{0.78}{0.244} & \scalebox{0.78}{0.286} & \scalebox{0.78}{0.395} & \scalebox{0.78}{0.407} & \scalebox{0.78}{0.270} & \scalebox{0.78}{0.307} & \scalebox{0.78}{0.641} & \scalebox{0.78}{0.639} & \scalebox{0.78}{0.301} & \scalebox{0.78}{0.319} & \scalebox{0.78}{0.260} & \scalebox{0.78}{0.297} & \scalebox{0.78}{0.330} & \scalebox{0.78}{0.401} & \scalebox{0.78}{0.248} & \scalebox{0.78}{0.296} \\

    \bottomrule
  \end{tabular}
    \end{small}
}
\end{table*}
\textbf{Quantitative comparison \quad} Comprehensive forecasting results are listed in Table~\ref{tab:bench_avg}. We leave full forecasting results in APPENDIX to save place. It is quite evident that \textbf{ReFocus} has demonstrated superior predictive performance across all datasets, significantly outperforming the second-best method.  Especially, Compared to the previous SOTA \textbf{iTransformer}, we have reduced the MSE by \textbf{4\%}, \textbf{6\%}, and \textbf{5\%} on the three most challenging benchmarks: Traffic, ECL, and Solar, respectively, indicating a significant breakthrough. These significant improvements indicate that the \textbf{ReFocus} model possesses robust performance and broad applicability in multivariate time series forecasting tasks, especially in tasks with a large number of channels, such as the Solar\_Energy dataset (\textbf{137} channels), ECL dataset (\textbf{321} channels), and Traffic dataset (\textbf{862} channels).

\subsection{Model Analysis}

\begin{table*}[]
\centering
\caption{Average results of long-term forecasting in ablation test. Full results are provided in Table\ref{table: full ablation forecasting} in Appendix \ref{appendix:ablation}}
\label{table:forecasting ablation}
\small % 调整字体大小 small/scriptsize/footnotesize
\setlength{\tabcolsep}{4pt} % 调整列间距
\renewcommand{\arraystretch}{0.8} % 调整行间距
\begin{tabular}{c|c|cc|cc|cc}
\toprule
Models &  & \multicolumn{2}{c|}{\textbf{GTM}} & \multicolumn{2}{c|}{\textbf{GTM(w/o  time \_gra)}} & \multicolumn{2}{c}{\textbf{GTM(w/o  fft)}} \\ \midrule
dataset &  & MSE & MAE & MSE & MAE & MSE & MAE \\ \midrule
ETTh1 & Avg. & \textbf{0.404(3.57\%, 2.42\%)} & \textbf{0.429(2.28\%,0.92\%)} & {\color[HTML]{000000} {\ul 0.414(1.19\%)}} & {\color[HTML]{000000} {\ul 0.433(1.37\%)}} & 0.419 & 0.439 \\
ETTm1 & Avg. & \textbf{0.339(2.87\%,2.59\%)} & \textbf{0.376(2.08\%,1.57\%)} & {\color[HTML]{000000} {\ul 0.348(0.29\%)}} & {\color[HTML]{000000} {\ul 0.382(0.52\%)}} & 0.349 & 0.384 \\
Weather & Avg. & \textbf{0.225(3.43\%,2.60\%)} & \textbf{0.266(3.62\%,3.27\%)} & {\color[HTML]{000000} {\ul 0.231(0.86\%)}} & {\color[HTML]{000000} {\ul 0.275(0.36\%)}} & 0.233 & 0.276 \\
Traffic & Avg. & \textbf{0.385(1.79\%,0.52\%)} & \textbf{0.266(1.85\%,1.12\%)} & {\color[HTML]{000000} {\ul 0.387(1.28\%)}} & {\color[HTML]{000000} {\ul 0.269(0.74\%)}} & 0.392 & 0.271 \\
Electricity & Avg. & \textbf{0.161(2.42\%,1.23\%)} & \textbf{0.254(1.93\%,1.17\%)} & {\color[HTML]{000000} {\ul 0.163(1.21\%)}} & {\color[HTML]{000000} {\ul 0.257(0.77\%)}} & 0.165 & 0.259\\ 
            \bottomrule
\end{tabular}

\end{table*}
\begin{table*}[!h]
\caption{\small{Further ablation of `Key-Frequency Enhanced Training strategy (\textbf{KET})'. `Real' means KET is not performed, i.e. trained on original data. `Pseudo' means trained on Pseudo samples. If both are used (Bottom Line), this means the model is trained on Real and Pseudo samples alternatively, i.e. \textbf{KET}. We list the average results. Full results are in Appendix~\ref{app:full ket}.}
}
\label{tab:ket_avg}
\renewcommand{\arraystretch}{0.85} 
\centering
\setlength{\tabcolsep}{1.45pt}
\resizebox{0.8\textwidth}{!}{
\begin{small}
\renewcommand{\multirowsetup}{\centering}
\setlength{\tabcolsep}{1.45pt}
\begin{tabular}{c|c|cc|cc|cc|cc|cc|cc|cc|cc}

\toprule
\multirow{2}{*}{\scalebox{0.78}{Real}} & \multirow{2}{*}{\scalebox{0.78}{Pseudo}} &
\multicolumn{2}{c}{\scalebox{0.78}{ETTm1}}  & \multicolumn{2}{c}{\scalebox{0.78}{ETTm2}} & \multicolumn{2}{c}{\scalebox{0.78}{ETTh1}} & \multicolumn{2}{c}{\scalebox{0.78}{ETTh2}} & \multicolumn{2}{c}{\scalebox{0.78}{ECL}} & \multicolumn{2}{c}{\scalebox{0.78}{Traffic}} & \multicolumn{2}{c}{\scalebox{0.78}{Weather}}& \multicolumn{2}{c}{\scalebox{0.78}{Solar\_Energy}} \\
\cmidrule(l{10pt}r{10pt}){3-4}\cmidrule(l{10pt}r{10pt}){5-6}\cmidrule(l{10pt}r{10pt}){7-8}\cmidrule(l{10pt}r{10pt}){9-10}\cmidrule(l{10pt}r{10pt}){11-12}\cmidrule(l{10pt}r{10pt}){13-14}\cmidrule(l{10pt}r{10pt}){15-16}\cmidrule(l{10pt}r{10pt}){17-18}
&&\scalebox{0.78}{MSE} & \scalebox{0.78}{MAE}&\scalebox{0.78}{MSE} & \scalebox{0.78}{MAE}&\scalebox{0.78}{MSE} & \scalebox{0.78}{MAE}&\scalebox{0.78}{MSE} & \scalebox{0.78}{MAE}&\scalebox{0.78}{MSE} & \scalebox{0.78}{MAE}&\scalebox{0.78}{MSE} & \scalebox{0.78}{MAE}&\scalebox{0.78}{MSE} & \scalebox{0.78}{MAE}&\scalebox{0.78}{MSE} & \scalebox{0.78}{MAE}\\
\toprule[1pt]


\checkmark & - & \scalebox{0.78}{0.401} & \scalebox{0.78}{0.403} & \scalebox{0.78}{0.283} & \scalebox{0.78}{0.325} & \scalebox{0.78}{0.440} & \scalebox{0.78}{0.437} & \scalebox{0.78}{0.376} & \scalebox{0.78}{0.400} & \scalebox{0.78}{0.178} & \scalebox{0.78}{0.270} & \scalebox{0.78}{0.449} & \scalebox{0.78}{0.289} & \scalebox{0.78}{0.252} & \scalebox{0.78}{0.278} & \scalebox{0.78}{0.232} & \scalebox{0.78}{0.264}\\



- & \checkmark &  \scalebox{0.78}{0.396} & \scalebox{0.78}{0.398} & \scalebox{0.78}{0.280} & \scalebox{0.78}{0.323} & \boldres{\scalebox{0.78}{0.436}} & \boldres{\scalebox{0.78}{0.434}} & \boldres{\scalebox{0.78}{0.372}} & \boldres{\scalebox{0.78}{0.397}} & \scalebox{0.78}{0.175} & \scalebox{0.78}{0.266} & \scalebox{0.78}{0.417} & \scalebox{0.78}{0.271} & \scalebox{0.78}{0.252} & \scalebox{0.78}{0.276} & \scalebox{0.78}{0.277} & \scalebox{0.78}{0.294}\\


\checkmark & \checkmark &  \boldres{\scalebox{0.78}{0.394}} & \boldres{\scalebox{0.78}{0.396}} & \boldres{\scalebox{0.78}{0.279}} & \boldres{\scalebox{0.78}{0.322}} & \scalebox{0.78}{0.437} & \scalebox{0.78}{0.435} & \scalebox{0.78}{0.373} & \scalebox{0.78}{0.398} & \boldres{\scalebox{0.78}{0.171}} & \boldres{\scalebox{0.78}{0.263}} & \boldres{\scalebox{0.78}{0.414}} & \boldres{\scalebox{0.78}{0.268}} & \boldres{\scalebox{0.78}{0.250}} & \boldres{\scalebox{0.78}{0.275}} & \boldres{\scalebox{0.78}{0.228}} & \boldres{\scalebox{0.78}{0.258}}\\


\bottomrule
  \end{tabular}
    \end{small}
}
\end{table*}
\begin{table*}[!h]
\caption{\small{Ablation study of different Key-Frequency Picking strategies. `Softmax' means using softmax function to generate a probability distribution and picking shared Key-Frequency using this distribution. `Max' means always choosing the biggest energy. `Min' means always choosing the smallest energy. We list the average results. Full results are in Appendix~\ref{app:full pick}.}
}
\label{tab:pick_avg}
\renewcommand{\arraystretch}{0.85} 
\centering
\renewcommand{\multirowsetup}{\centering}
\setlength{\tabcolsep}{1.45pt}
\resizebox{0.8\textwidth}{!}{
\begin{small}
\renewcommand{\multirowsetup}{\centering}
\setlength{\tabcolsep}{1.45pt}
\begin{tabular}{c|cc|cc|cc|cc|cc|cc|cc|cc}

\toprule
\multirow{2}{*}{\scalebox{0.78}{Picking Strategy}} &
\multicolumn{2}{c}{\scalebox{0.78}{ETTm1}}  & \multicolumn{2}{c}{\scalebox{0.78}{ETTm2}} & \multicolumn{2}{c}{\scalebox{0.78}{ETTh1}} & \multicolumn{2}{c}{\scalebox{0.78}{ETTh2}} & \multicolumn{2}{c}{\scalebox{0.78}{ECL}} & \multicolumn{2}{c}{\scalebox{0.78}{Traffic}} & \multicolumn{2}{c}{\scalebox{0.78}{Weather}}& \multicolumn{2}{c}{\scalebox{0.78}{Solar\_Energy}} \\

\cmidrule(l{10pt}r{10pt}){2-3}\cmidrule(l{10pt}r{10pt}){4-5}\cmidrule(l{10pt}r{10pt}){6-7}\cmidrule(l{10pt}r{10pt}){8-9}\cmidrule(l{10pt}r{10pt}){10-11}\cmidrule(l{10pt}r{10pt}){12-13}\cmidrule(l{10pt}r{10pt}){14-15}\cmidrule(l{10pt}r{10pt}){16-17}
  &\scalebox{0.78}{MSE} & \scalebox{0.78}{MAE}&\scalebox{0.78}{MSE} & \scalebox{0.78}{MAE}&\scalebox{0.78}{MSE} & \scalebox{0.78}{MAE}&\scalebox{0.78}{MSE} & \scalebox{0.78}{MAE}&\scalebox{0.78}{MSE} & \scalebox{0.78}{MAE}&\scalebox{0.78}{MSE} & \scalebox{0.78}{MAE}&\scalebox{0.78}{MSE} & \scalebox{0.78}{MAE}&\scalebox{0.78}{MSE} & \scalebox{0.78}{MAE}\\
\toprule[1pt]

\scalebox{0.78}{Min} & \scalebox{0.78}{0.388} & \boldres{\scalebox{0.78}{0.392}} & \scalebox{0.78}{0.280} & \scalebox{0.78}{0.323} & \boldres{\scalebox{0.78}{0.432}} & \boldres{\scalebox{0.78}{0.432}} & \scalebox{0.78}{0.371} & \scalebox{0.78}{0.396} & \scalebox{0.78}{0.194} & \scalebox{0.78}{0.281} & \scalebox{0.78}{0.517} & \scalebox{0.78}{0.344} & \scalebox{0.78}{0.378} & \scalebox{0.78}{0.363} & \scalebox{0.78}{0.240} & \scalebox{0.78}{0.270}  \\

\scalebox{0.78}{Max} & \scalebox{0.78}{0.392} & \scalebox{0.78}{0.395} & \scalebox{0.78}{0.279} & \scalebox{0.78}{0.322} & \scalebox{0.78}{0.437} & \scalebox{0.78}{0.435} & \scalebox{0.78}{0.374} & \scalebox{0.78}{0.398} & \scalebox{0.78}{0.172} & \scalebox{0.78}{0.265} & \scalebox{0.78}{0.422} & \scalebox{0.78}{0.273} & \scalebox{0.78}{0.351} & \scalebox{0.78}{0.343} & \scalebox{0.78}{0.230} & \scalebox{0.78}{0.260} \\

\scalebox{0.78}{\textbf{Softmax}} &  \boldres{\scalebox{0.78}{0.387}} & \scalebox{0.78}{0.394} & \boldres{\scalebox{0.78}{0.275}} & \boldres{\scalebox{0.78}{0.320}} & \scalebox{0.78}{0.434} & \scalebox{0.78}{0.433} & \boldres{\scalebox{0.78}{0.371}} & \boldres{\scalebox{0.78}{0.396}} & \boldres{\scalebox{0.78}{0.168}} & \boldres{\scalebox{0.78}{0.262}} & \boldres{\scalebox{0.78}{0.412}} & \boldres{\scalebox{0.78}{0.265}} & \boldres{\scalebox{0.78}{0.346}} & \boldres{\scalebox{0.78}{0.339}} & \boldres{\scalebox{0.78}{0.222}} & \boldres{\scalebox{0.78}{0.252}} \\


\bottomrule
  \end{tabular}
    \end{small}
}
\end{table*}
\begin{table}[!h]
\caption{\small{Multivariate forecasting result of `Energy-based Key-Frequency Picking Block' (EKPB) and other inter-series dependencies modeling backbones. \textbf{*} means `former.' We list the average results. Full results are in Appendix~\ref{app:full ekpb}.}}
\label{tab:ekpb_avg}
\renewcommand{\arraystretch}{0.85} 
\centering
\renewcommand{\multirowsetup}{\centering}
\setlength{\tabcolsep}{1.45pt}
\resizebox{0.5\textwidth}{!}{
\begin{small}
\renewcommand{\multirowsetup}{\centering}
\setlength{\tabcolsep}{1.45pt}
\begin{tabular}{c|cc|cc|cc|cc|cc}

\toprule
\multirow{2}{*}{\scalebox{0.78}{Dataset}} &
\multicolumn{2}{c}{\scalebox{0.78}{\textbf{EKPB}}}  & \multicolumn{2}{c}{\scalebox{0.78}{TSMixer}} & \multicolumn{2}{c}{\scalebox{0.78}{iTrans*}} & \multicolumn{2}{c}{\scalebox{0.78}{Cross*}} & \multicolumn{2}{c}{\scalebox{0.78}{FECAM}} \\

\cmidrule(l{10pt}r{10pt}){2-3}\cmidrule(l{10pt}r{10pt}){4-5}\cmidrule(l{10pt}r{10pt}){6-7}\cmidrule(l{10pt}r{10pt}){8-9}\cmidrule(l{10pt}r{10pt}){10-11}
& \scalebox{0.78}{MSE} & \scalebox{0.78}{MAE}&\scalebox{0.78}{MSE} & \scalebox{0.78}{MAE}&\scalebox{0.78}{MSE} & \scalebox{0.78}{MAE}&\scalebox{0.78}{MSE} & \scalebox{0.78}{MAE}&\scalebox{0.78}{MSE} & \scalebox{0.78}{MAE}\\
\toprule[1pt]

\scalebox{0.78}{ETTm2}& \boldres{\scalebox{0.78}{0.282}} & \boldres{\scalebox{0.78}{0.324}} & \scalebox{0.78}{0.289} & \scalebox{0.78}{0.333} & \scalebox{0.78}{0.288} & \scalebox{0.78}{0.332} & \scalebox{0.78}{0.757} & \scalebox{0.78}{0.610} & \scalebox{0.78}{0.297} & \scalebox{0.78}{0.348}\\

\bottomrule
\scalebox{0.78}{ETTh2}& \boldres{\scalebox{0.78}{0.374}} & \boldres{\scalebox{0.78}{0.399}} & \scalebox{0.78}{0.401} & \scalebox{0.78}{0.417} & \scalebox{0.78}{0.383} & \scalebox{0.78}{0.407} & \scalebox{0.78}{0.942} & \scalebox{0.78}{0.684} & \scalebox{0.78}{0.383} & \scalebox{0.78}{0.407}\\

\bottomrule
\scalebox{0.78}{Weather}& \boldres{\scalebox{0.78}{0.252}} & \boldres{\scalebox{0.78}{0.277}} & \scalebox{0.78}{0.256} & \scalebox{0.78}{0.279} & \scalebox{0.78}{0.258} & \scalebox{0.78}{0.279} & \scalebox{0.78}{0.259} & \scalebox{0.78}{0.315} & \scalebox{0.78}{0.253} & \scalebox{0.78}{0.304}\\

\bottomrule
\scalebox{0.78}{ECL}& \boldres{\scalebox{0.78}{0.176}} & \boldres{\scalebox{0.78}{0.268}} & \scalebox{0.78}{0.186} & \scalebox{0.78}{0.287} & \scalebox{0.78}{0.178} & \scalebox{0.78}{0.270} & \scalebox{0.78}{0.244} & \scalebox{0.78}{0.334} & \scalebox{0.78}{0.199} & \scalebox{0.78}{0.288}\\

\bottomrule
  \end{tabular}
    \end{small}
}
\end{table}
\begin{table}[!h]
\centering
\caption{\small{Model efficiency analysis. We evaluated the \textbf{parameter count}, and the \textbf{inference time} (average of 5 runs on a single NVIDIA 4090 24GB GPU) with $batch\_size= 1$ on \textbf{ECL} dataset. We set the dimension of layer $ dim \in \{256, 512\}$, and the number of network layers $N=2$. The task is \textbf{input-96-forecast-720}. \textbf{*} means `former.' \textbf{Para} means `Parameter count(M).' \textbf{Time} means `inference time(ms).'}}
\phantomsection
\label{tab:efficiency}
\renewcommand{\arraystretch}{0.85} 
\centering
\renewcommand{\multirowsetup}{\centering}
\setlength{\tabcolsep}{1.45pt}
\resizebox{0.5\textwidth}{!}{
\begin{small}
\renewcommand{\multirowsetup}{\centering}
\setlength{\tabcolsep}{1.45pt}
\begin{tabular}{c|cc|cc|cc|cc|cc}
\toprule
\multirow{2}{*}{Dim} & \multicolumn{2}{c}{\textbf{EKPB}} & \multicolumn{2}{c}{Cross*} & \multicolumn{2}{c}{iTrans*} & \multicolumn{2}{c}{TSMixer} & \multicolumn{2}{c}{FECAM}\\
\cmidrule(lr){2-3} \cmidrule(lr){4-5}\cmidrule(lr){6-7} \cmidrule(lr){8-9} \cmidrule(lr){10-11}
& Param & Time & Para & Time & Para & Time & Para & Time & Para & Time  \\
\midrule
256 & \boldres{0.29}	& \boldres{68.91}	& 0.93	& 98.37	& 1.27	& 192.12 & 13.66	& 432.40	& 1.39	& 205.66 \\
512 & \boldres{0.97}	& \boldres{84.54}	& 1.78	& 118.29	& 4.63	& 249.60	& 43.04	& 507.54	& 5.14	& 277.43
\\
\bottomrule
  \end{tabular}
    \end{small}
}
\end{table}
\begin{table}[!h]
\caption{\small{Experiment result of high-pass filter, low-pass filter, RevIN, and AMEO using \textbf{a simple linear projection} as the forecaster on Weather and ETTm1 dataset. We set the input length $T=96$ and forecasting length $F \in \{720, 96\}$.}}
\label{tab:ameo_avg}
\renewcommand{\arraystretch}{0.85} 
\centering
\renewcommand{\multirowsetup}{\centering}
\setlength{\tabcolsep}{1.45pt}
\resizebox{0.5\textwidth}{!}{
\begin{small}
\renewcommand{\multirowsetup}{\centering}
\setlength{\tabcolsep}{1.45pt}
\begin{tabular}{c|c|cc|cc|cc|cc|cc}

\toprule
\multirow{2}{*}{\scalebox{0.78}{Dataset}} & \multirow{2}{*}{\scalebox{0.78}{Length}} &
\multicolumn{2}{c}{\scalebox{0.78}{\textbf{AMEO}}}  & \multicolumn{2}{c}{\scalebox{0.78}{RevIN}} & \multicolumn{2}{c}{\scalebox{0.78}{Low}} & \multicolumn{2}{c}{\scalebox{0.78}{High}} & \multicolumn{2}{c}{\scalebox{0.78}{None}} \\

\cmidrule(l{10pt}r{10pt}){3-4}\cmidrule(l{10pt}r{10pt}){5-6}\cmidrule(l{10pt}r{10pt}){7-8}\cmidrule(l{10pt}r{10pt}){9-10}\cmidrule(l{10pt}r{10pt}){11-12}
  & & \scalebox{0.78}{MSE} & \scalebox{0.78}{MAE}&\scalebox{0.78}{MSE} & \scalebox{0.78}{MAE}&\scalebox{0.78}{MSE} & \scalebox{0.78}{MAE}&\scalebox{0.78}{MSE} & \scalebox{0.78}{MAE}&\scalebox{0.78}{MSE} & \scalebox{0.78}{MAE}\\
\toprule[1pt]
 
\multirow{2}{*}{\scalebox{0.78}{ETTm1}} & \scalebox{0.78}{96} & \boldres{\scalebox{0.78}{0.331}} & \boldres{\scalebox{0.78}{0.365}} & \scalebox{0.78}{0.354} & \scalebox{0.78}{0.375} & \scalebox{0.78}{0.345} & \scalebox{0.78}{0.371} & \scalebox{0.78}{1.097} & \scalebox{0.78}{0.792} & \scalebox{0.78}{0.348} & \scalebox{0.78}{0.375}\\

& \scalebox{0.78}{720} & \boldres{\scalebox{0.78}{0.466}} & \boldres{\scalebox{0.78}{0.440}} & \scalebox{0.78}{0.486} & \scalebox{0.78}{0.448} & \scalebox{0.78}{0.478} & \scalebox{0.78}{0.458} & \scalebox{0.78}{1.106} & \scalebox{0.78}{0.796} & \scalebox{0.78}{0.479} & \scalebox{0.78}{0.456}\\
\bottomrule

\multirow{2}{*}{\scalebox{0.78}{Weather}} & \scalebox{0.78}{96} & \boldres{\scalebox{0.78}{0.164}} & \boldres{\scalebox{0.78}{0.236}} & \scalebox{0.78}{0.194} & \scalebox{0.78}{0.234} & \scalebox{0.78}{0.198} & \scalebox{0.78}{0.258} & \scalebox{0.78}{0.636} & \scalebox{0.78}{0.608} & \scalebox{0.78}{0.198} & \scalebox{0.78}{0.258} \\

& \scalebox{0.78}{720} & \boldres{\scalebox{0.78}{0.331}} & \boldres{\scalebox{0.78}{0.370}} & \scalebox{0.78}{0.365} & \scalebox{0.78}{0.353} & \scalebox{0.78}{0.353} & \scalebox{0.78}{0.387} & \scalebox{0.78}{0.638} & \scalebox{0.78}{0.611} & \scalebox{0.78}{0.352} & \scalebox{0.78}{0.386}\\
\bottomrule
  \end{tabular}
    \end{small}
}
\end{table}

\begin{figure*}[!h]
  \centering
  \includegraphics[width=0.95\linewidth]{Faker/source/assets/jpg/ameo.jpg}
  \caption{\small{The time-frequency domain visualization of the original sequence (ETTm1, \textbf{\textit{the last variate}}), the sequence processed by high-pass and low-pass filters, by RevIN, and by AMEO. We selected the $input-96-forecast-96$ task.}}
  \label{fig:ameo}
\end{figure*}
\begin{figure}[!h]
  \centering
  \includegraphics[width=1\linewidth]{Faker/source/assets/jpg/embed.jpg}
  \caption{\small{T-SNE visualization of the series embeddings with and without `Energy-based Key-Frequency Picking Block' (EKPB) on ECL dataset. We choose the $input-96-forecast-96$ task. Three example variates are highlighted: variates 2\&3 shared a common Key-Frequency, while variate 1 does not.}}
  \label{fig:embed}
  \vspace{-5mm} 
\end{figure}

\textbf{Ablation study of AMEO and KET \quad} To evaluate the contributions of each module in ReFocus, we performed ablation studies on the `Adaptive Mid-Frequency Energy Optimizer (AMEO)' and the `Key-Frequency Enhanced Training (KET)' strategy. The results are summarized in Table~\ref{tab:ablation_avg}. Notably, integrating both modules achieves the best performance, highlighting the effectiveness of their synergy. Additionally, each module delivers substantial improvements over baseline models in most cases.

\textbf{Further study of KET \quad}  We conducted further ablation studies on the KET to demonstrate the importance of alternate training between real and synthetic data. The experimental results in Table~\ref{tab:ket_avg} reveal that while training on pseudo samples can partially enhance the model's generalization performance on the test set, it also tends to cause over-smoothing and training instability on more complex datasets, such as Solar\_Energy. In contrast, training on real and synthetic data alternatively (KET) improves generalization and mitigates over-smoothing and training instability by preserving the spectral dependencies of real samples. More Analyses are in Appendix~\ref{app:further ket}. 

\textbf{Ablation study of different Key-Frequency Picking strategy \quad} We conducted an ablation study on various key-frequency selection strategies. The evaluated methods include Maximum-based, Minimum-based, and \textbf{Softmax-based} random sampling strategies. Our experimental results in Table~\ref{tab:pick_avg} reveal that purely relying on Maximum or Minimum-based strategies may overlook certain critical Key-Frequency. In contrast, the random sampling strategy based on a Softmax probabilistic distribution consistently achieved the best overall performance, particularly on datasets with a larger number of channels and higher complexity—key challenges in multivariate time series forecasting. 

\textbf{Outstanding inter-series modeling ability of the EKPB \quad} We compared `Energy-based Key-Frequency Picking Block' (EKPB) with several well-established backbones, including iTransformer~\citep{LiuiTransformer}, TSMixer~\citep{chen2023tsmixer}, and Crossformer~\citep{zhang2023crossformer}, which have demonstrated exceptional performance in modeling inter-series dependencies. Additionally, we included FECAM
~\citep{jiang2023fecam}, a method also designed for modeling cross-channel frequency-domain dependencies. The results presented in Table~\ref{tab:ekpb_avg} demonstrate that our EKPB outperforms in modeling inter-series dependencies across multiple datasets. Additionally, when comparing the number of parameters and inference time during prediction under identical configurations on the ECL dataset, our EKPB method still outperforms other baselines by a significant margin, as in Table~\ref{tab:efficiency}. To illustrate EKPB's functionality, we visualize the series embeddings with and without its adjustment in Figure~\ref{fig:embed}. The T-SNE visualization of the series embeddings shows that without EKPB, using only the channel-independent strategy~\citep{Nie2022patchtst}, the MSE is 0.171. After applying EKPB, channels sharing Key-Frequency (variates 2\&3) are clustered, while others (variates 1\&3) are separated. This adjustment improves the MSE from 0.171 to 0.145, a \textbf{15\%} reduction. These indicate that EKPB not only achieves better predictive performance but also offers a more resource-efficient solution than other baselines.

\textbf{Superiority of AMEO over RevIN and Filters \quad} We investigated the roles of AMEO, RevIN, and Filters in addressing the Mid-Frequency Spectrum Gap through time-frequency domain visualization analysis. The results presented in Figure~\ref{fig:ameo} align perfectly with our theoretical analysis before. High-pass and low-pass filters fail to address the Mid-Frequency Spectrum Gap and exacerbate this issue. RevIN, on the other hand, merely eliminates the energy of the zero-frequency component while scaling other components using the variance \( \sigma^2 \), which also does not effectively resolve the problem. In contrast, our AMEO successfully amplifies the mid-frequency energy. Furthermore, compared to the original sequence and the sequence processed by RevIN, we observe that the sequence processed by AMEO exhibits significantly higher stationarity with much more stable means and variance.

In Table~\ref{tab:ameo_avg}, the performance of AMEO on two prediction tasks across two datasets consistently surpasses the results achieved by methods based on RevIN and Filters. Furthermore, while Filters and RevIN occasionally lead to degraded performance on certain datasets, AMEO consistently delivers results that outperform the original methods. These findings further highlight the superiority of AMEO over alternative approaches.