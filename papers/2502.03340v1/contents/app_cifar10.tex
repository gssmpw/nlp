\section{Additional Experiments: Visual Domain Detection in Cifar10}\label{app:cifar10}
In this section we present the result for the domain ablation discussed in Section \ref{sect:ablation} conducted on Cifar10 \cite{krizhevsky2009learning}. We explore how the algorithm identifies and groups clients based on the non-IID nature of their data distributions, represented by the Dirichlet concentration parameter $\alpha$. We apply a similar splitting approach, obtaining the following partitions: (1) 90 clients with $\alpha = 0$ and 10 clients with $\alpha = 100$; (2) 90 clients with $\alpha = 0.5$ and 10 clients with $\alpha = 100$; and (3) 40 clients with $\alpha = 100$, 30 clients with $\alpha = 0.5$, and 30 clients with $\alpha = 0$. We evaluate the outcome of this clustering experiment by means of WAS and WADB. Results in Table \ref{tab:ablation1_heter} show that \shortname detects clusters groups clients according to the level of heterogeneity of the group.
\begin{table}[t]
    
    \caption{\small{Clustering with three different splits on Cifar10. \shortname has superior clustering quality across different splits (homogeneous \textit{Hom}, heterogeneous \textit{Het}, extremly heterogeneous \textit{X Het})}}
    \centering
    \small
    \begin{adjustbox}{width=.5\linewidth}
        \label{tab_app:ablation1_heter}
     
        \begin{tabular}{lccccc}
            \toprule
            \textbf{Dataset} & \textbf{(Hom, Het, X Het)} & \makecell{\textbf{Clustering} \\ \textbf{method}} & \textbf{C} & \textbf{WAS} & \textbf{WADB} \\
            \cmidrule(lr){1-6}
          
            \midrule\multirow{9}{*}{Cifar10} 
            & \multirow{3}{*}{(10, 0, 90)} & \texttt{IFCA} & 1 & / &/ \\
            & & \texttt{FeSem} & 3 & -0.0 \scriptsize{$\pm$ 0.1} & 12.0 \scriptsize{$\pm$ 2.0}\\
            & & \shortname & 3 & \textbf{0.1 \scriptsize{$\pm$ 0.0}} & \textbf{0.2 \scriptsize{$\pm$ 0.0}} \\
            \cmidrule(lr){2-6}
            & \multirow{3}{*}{(10, 90, 0)} & \texttt{IFCA} & 1 & / & / \\
            & & \texttt{FeSem} & 3 & -0.0 \scriptsize{$\pm$ 0.0} & 12.0 \scriptsize{$\pm$ 2.0}\\

            & & \shortname & 3 & \textbf{0.2 \scriptsize{$\pm$ 0.0}} & \textbf{0.6 \scriptsize{$\pm$ 0.0}} \\
            \cmidrule(lr){2-6}
            & \multirow{3}{*}{(40, 30, 30)} & \texttt{IFCA} & 2 & -0.2 \scriptsize{$\pm$ 0.0} & \textbf{1.0 \scriptsize{$\pm$ 0.0}} \\
            & & \texttt{FeSem} & 3 & 0.1 \scriptsize{$\pm$ 0.1} & 20.6 \scriptsize{$\pm$ 7.1} \\
            & & \shortname & 3 & \textbf{0.6 \scriptsize{$\pm$ 0.1}} & \textbf{1.0 \scriptsize{$\pm$ 0.4}}\\
            
            \bottomrule
        \end{tabular}
    \end{adjustbox}
    
\end{table}

\begin{table}[t]
    
    \caption{\small Clustering performance of \shortname is assessed on federations with clients from varied domains using clean, noisy, and blurred (Clean, Noise, Blur) images from Cifar10 dataset. It utilizes the Rand Index score \citep{rand1971objective}, where a value close to 1 represents a perfect match between clustering and labels. Consistently \shortname accurately distinguishes all visual domains.}
    \label{tab_app:dom_abl}
    
    \centering
    
    \begin{adjustbox}{width=.5\linewidth}
    \setlength{\tabcolsep}{12pt}
        \begin{tabular}{ccccc}
           \toprule
            \textbf{Dataset} & \textbf{(Clean, Noise, Blur)} & \makecell{\textbf{Clustering} \\ \textbf{method}} & \textbf{C} & \textbf{Rand} \\
            \cmidrule(lr){1-5}
            \multirow{9}{*}{Cifar10} 
            & \multirow{3}{*}{(50, 50, 0)} & \texttt{IFCA} & 1 & 0.5 \scriptsize{$\pm$ 0.0}  \\
            & & \texttt{FeSem} & 2 & 0.49 \scriptsize{$\pm$ 0.2} \\
            & & \shortname & 2 & \textbf{1.0 \scriptsize{$\pm$ 0.0}} \\
            \cmidrule(lr){2-5}
            & \multirow{3}{*}{(50, 0, 50)} & \texttt{IFCA} & 1 & 0.5 \scriptsize{$\pm$ 0.0} \\
            & & \texttt{FeSem} & 2 & 0.5 \scriptsize{$\pm$ 0.1}\\
            & & \shortname & 2 & \textbf{1.0 \scriptsize{$\pm$ 0.0}} \\
            \cmidrule(lr){2-5}
            & \multirow{3}{*}{(40, 30, 30)} & \texttt{IFCA} & 1 & 0.33 \scriptsize{$\pm$ 0.0} \\
            & & \texttt{FeSem} & 3 & 0.34 \scriptsize{$\pm$ 0.1} \\
            & & \shortname & 4 & \textbf{0.9 \scriptsize{$\pm$ 0.0}} \\

            \bottomrule
        \end{tabular}
    \end{adjustbox}
    
\end{table}
