\section{\textcolor{black}{Metrics Used for Evaluation}} \label{app:metrics_choice}
\subsection{\textcolor{black}{Silhouette Score}}
\textcolor{black}{Silhouette Score is a clustering metric that measures the consistency of points within clusters by comparing intra-cluster and nearest-cluster distances \citep{rousseeuw1987silhouettes}. Let us consider a metric space $(M,d)$. For a set of points $\{x_1,\dots, x_N\} \subset M$ and clustering labels $\mathcal{C}_1, \dots, \mathcal{C}_{n_{cl}}$. The Silhouette score of a data point $x_i$ belonging to a cluster $\mathcal{C}_i$ is defined as}
\begin{equation}\label{silhouette_defn1}
    \color{black}
    s_i = \dfrac{b_i - a_i}{\max\{a_i,b_i\}}
\end{equation}
\textcolor{black}{where the values $b_i$ and $a_i$ represent the average intra-cluster distance and the minimal average outer-cluster distance, \ie}\begin{equation}\label{silhouette_defn}
    \color{black}
    \begin{split}
        a_i &= \dfrac{1}{|\mathcal{C}_i| - 1} \sum_{x_j \in \mathcal{C}_i\setminus\{x_i\}} d(x_i, x_j)\\
        b_i & =\min_{j \neq i} \dfrac{1}{|\mathcal{C}_j|} \sum_{x_j \in \mathcal{C}_j} d(x_i,x_j)
    \end{split}
\end{equation}
\textcolor{black}{The value of the Silhouette score ranges between $-1$ and $+1$, \ie $s_i \in [-1,1]$. In particular, a Silhouette score close to 1 indicates well-clustered data points, 0 denotes points near cluster boundaries, and -1 suggests misclassified points. In order to evaluate the overall performance of the clustering, a common choice, that is the one adopted in this paper, is to average the score value for each data point.}
\subsection{\textcolor{black}{Davies-Bouldin Score}}
\textcolor{black}{The Davies-Bouldin Score is a clustering metric that evaluates the quality of clustering by measuring the ratio of intra-cluster dispersion to inter-cluster separation \citep{davies1979cluster}. Let us consider a metric space $(M,d)$, a set of points $\{x_1, \dots, x_N\} \subset M$, and clustering labels $\mathcal{C}_1, \dots, \mathcal{C}_{n_{cl}}$. The Davies-Bouldin score is defined as the average similarity measure $R_{ij}$ between each cluster $\mathcal{C}_i$ and its most similar cluster $\mathcal{C}_j$}:
\begin{equation}\label{db_index_defn1}
    \color{black}
    DB = \dfrac{1}{n_{cl}} \sum_{i=1}^{n_{cl}} \max_{j \neq i} R_{ij}
\end{equation}
\textcolor{black}{where $R_{ij}$ is given by the ratio of intra-cluster distance $S_i$ to inter-cluster distance $D_{ij}$, \ie}
\begin{equation}\label{db_index_defn}
    \color{black}
    R_{ij} = \dfrac{S_i + S_j}{D_{ij}}
\end{equation}
\textcolor{black}{with intra-cluster distance $S_i$ defined as}
\begin{equation}
    \color{black}
    S_i = \dfrac{1}{|\mathcal{C}_i|} \sum_{x_k \in \mathcal{C}_i} d(x_k, c_i)
\end{equation}
\textcolor{black}{where $c_i$ denotes the centroid of cluster $\mathcal{C}_i$, and $D_{ij} = d(c_i, c_j)$ is the distance between centroids of clusters $\mathcal{C}_i$ and $\mathcal{C}_j$. A lower Davies-Bouldin Index indicates better clustering, as it reflects well-separated and compact clusters. Conversely, a higher DBI suggests that clusters are less distinct and more dispersed.}
\subsubsection{\textcolor{black}{Rand Index}} \textcolor{black}{Rand Index is a clustering score that measures the outcome of a clustering algorithm with respect to a ground truth clustering label \citep{rand1971objective}. Let us denote by $a$ the number of pairs that have been grouped in the same clusters, while by $b$ the number of pairs that have been grouped in different clusters, then the Rand-Index is defined as}
\begin{equation}
    \color{black}
    RI = \dfrac{a + b}{\binom{N}{2}}
\end{equation}
\textcolor{black}{where N denotes the number of data points. In our experiments we opted for the Rand Index score to evaluate how the algorithm was able to separate clients in groups of the same level of heterogeneity (which was known a priori and used as ground truth). A Rand Index ranges in $[0,1]$, and a value of 1 signifies a perfect agreement between the identified clusters and the ground truth.}
