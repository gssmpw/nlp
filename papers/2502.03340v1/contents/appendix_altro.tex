\section{Further Experiments} \label{app:other}
In Table \ref{tab_app:fl-algs} we show that \shortname is orthogonal to FL aggregation, which means that any algorithm can be easily embedded in our clustering setting, providing beneficial results, increasing model performance.
\begin{table}[t]
    \caption{\small{\shortname is orthogonal to FL aggregation algorithms, improving their performance in heterogeneous scenarios (Cifar100 with $\alpha = 0.5$ and Femnist with $\alpha = 0.01$). This shows that \shortname and clustering are beneficial in this scenarios.  }}
    \label{tab_app:fl-algs}
    \centering
    \small
    \setlength{\tabcolsep}{4pt} %
    \renewcommand{\arraystretch}{1.1} %
    \begin{adjustbox}{width=.5\linewidth}
    \begin{tabular}{l|cc|cc}
        \toprule
        \textbf{FL method} &  \multicolumn{2}{c|}{\textbf{Cifar100}} & \multicolumn{2}{c}{\textbf{Femnist}} \\
         & No Clusters & \shortname & No Clusters & \shortname \\
        \midrule
        FedAvg &  41.6 {\scriptsize$\pm$ 1.3} & \textbf{53.4} {\scriptsize$\pm$ 0.4} & 76.0 {\scriptsize$\pm$ 0.1} & \textbf{76.1} {\scriptsize$\pm$ 0.1} \\ 
        FedAvgM &  41.5 {\scriptsize$\pm$ 0.5} & \textbf{50.5} {\scriptsize$\pm$ 0.3} & {83.3} {\scriptsize$\pm$ 0.3} & \textbf{83.3} {\scriptsize$\pm$ 0.4} \\ 
        FedProx &  41.8 {\scriptsize$\pm$ 1.0} & \textbf{49.1} {\scriptsize$\pm$ 1.0} & 75.9 {\scriptsize$\pm$ 0.2} & \textbf{76.3} {\scriptsize$\pm$ 0.2} \\ 
        \bottomrule
    \end{tabular}
\end{adjustbox}
\end{table}
\begin{figure}
    \centering
    \includegraphics[width=\linewidth]{figures/tree_levels.pdf}
    \caption{\small{Cluster evolution with respect to the recursive splits in \shortname on Cifar100, projected on the spectral embedded bi-dimensional space. From left to right, top to bottom, we can see that \shortname splits the client into cluster, until a certain level of intra-cluster homogeneity is reached }}
    \label{fig:treelevels}
\end{figure}
\begin{figure}[htbp]
    \centering
    \includegraphics[width=0.6\linewidth]{figures/fedgw_cifar10_matrix_convergence.png}
    \caption{\small{Interaction matrix convergence: on the $y$-axis MSE in logarithmic scale w.r.t. communication rounds in the $x$-axis on Cifar10, with Dirichlet parameter $\alpha = 0.05$.}}
    \label{fig:mse_conv}
\end{figure}
Figure \ref{fig:hom-het} illustrates the clustering results corresponding to varying degrees of heterogeneity, as described in Section \ref{sect:ablation}. As per \shortname, the detection of clusters based on different levels of heterogeneity in the Cifar10 dataset is achieved. Specifically, an examination of the interaction matrix reveals a clear distinction between the two groups.
\begin{figure}[h]
    \centering
    \includegraphics[width=1\linewidth]{figures/spectral_split.pdf}
    \caption{\small{Homogeneous (Cifar10 $\alpha = 100$) vs heterogeneous clustering (Cifar10 $\alpha = 0.05$). The interaction matrix at convergence and the corresponding scaled affinity matrix are on the left. The scatter plot in the 2D plane with spectral embedding is on the right. It is possible to see that the algorithm perfectly separates homogeneous clients (orange) from heterogeneous clients (black) }}
    \label{fig:hom-het}
\end{figure}
In Figure \ref{fig:hom}, we show that in class-balanced scenarios with small heterogeneity, like Cifar10 with $\alpha = 100$, \shortname successfully detects one single cluster. Indeed, in homogeneous scenarios such as this one, the model benefits from accessing more data from all the clients.
\begin{figure}[h]
    \centering
    \includegraphics[width=\linewidth]{figures/spectral_uniform.pdf}
    \caption{\small{Homogeneous case (Cifar10 $\alpha = 100$).  The interaction matrix at convergence and the corresponding scaled affinity matrix are on the left. The scatter plot in the 2D plane with spectral embedding is on the right. In the homogeneous case where no clustering is needed, \textit{FedGW} does not split the clients.}}
    \label{fig:hom}
\end{figure}

Figure \ref{fig:mse_conv} shows how the MSE converges to a small value as the rounds increase for a Cifar10 experiment.

As Figure \ref{fig:class_distr} illustrates, \shortname partitions the Cifar100 dataset into clients based on class distributions. Each cluster's distribution is distinct and non-overlapping, demonstrating the algorithm's efficacy in partitioning data with varying degrees of heterogeneity.
\begin{figure}[t]
    \centering
    \includegraphics[width=\linewidth]{figures/data_distr.pdf}
    \caption{\small{Class distributions among distinct clusters as detected by \shortname on Cifar100. Specifically, we examine the class distributions for each pair of clusters, demonstrating that (1) the clusters were identified by grouping differing levels of heterogeneity and (2) there is, in most cases, an absence of overlapping classes.}}
    \label{fig:class_distr}
\end{figure}
In Figure \ref{fig:dom_ablation}, we report the domain detection on Cifar100, where 40 clients have clean images, 30 have noisy images, and 30 have blurred images. Table \ref{tab:dom_abl} shows that \shortname performs a good clustering, effectively separating the different domains.
\begin{figure}
    \centering
     
    \includegraphics[width=\linewidth]{figures/domain_ablation_c100.pdf}
    \caption{\small{\shortname in the presence of domain imbalance. Three domains on Cifar100: clean clients (unlabeled), noisy clients (+), and blurred clients (x). \textit{Left}: is the interaction matrix $P$ at convergence from which it is possible to see client relations. \textit{Center}: The affinity matrix $W$ computed with respect to the UPVs extracted from $P$, and on which \texttt{FedGW\_Clustering} is performed. We can see that \shortname clusters the clients according to the domain, as proved by results in Table \ref{tab:dom_abl}.}}
    \label{fig:dom_ablation}
   
\end{figure}
