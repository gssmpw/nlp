
\documentclass{article}
\usepackage{multicol}
\usepackage{capt-of}
\usepackage{amsthm}
\usepackage{multicol}

\usepackage{url}
\usepackage{graphicx}
\newtheorem{remark}{Remark}[section]
\usepackage{algorithm}

\usepackage{enumitem}
\usepackage{booktabs}
\usepackage{xspace} 
\usepackage{adjustbox}
\usepackage{booktabs}
\usepackage{makecell}
\usepackage{multirow}
\usepackage{pifont}
\usepackage[table,xcdraw]{xcolor}
\usepackage{parskip}
\usepackage{microtype}
\usepackage{graphicx}
\usepackage{subfigure}
\usepackage{booktabs} %
\newcommand{\ef}[1]{\color{orange}{[EF:#1]}\color{black}}
\newcommand{\mc}[1]{\color{red}{[MC:#1]}\color{black}}
\newcommand{\al}[1]{\color{blue}{[AL:#1]}\color{black}}
\newcommand{\dl}[1]{\color{green}{[DL:#1]}\color{black}}
\newcommand\norm[1]{\left\lVert#1\right\rVert}
\newcommand{\shortname}{\texttt{FedGWC}\xspace}

\makeatletter
\DeclareRobustCommand\onedot{\futurelet\@let@token\@onedot}
\def\@onedot{\ifx\@let@token.\else.\null\fi\xspace}
\def\eg{\emph{e.g}\onedot} \def\Eg{\emph{E.g}\onedot}
\def\ie{\emph{i.e}\onedot} \def\Ie{\emph{I.e}\onedot}
\def\cf{\emph{c.f}\onedot} \def\Cf{\emph{C.f}\onedot}
\def\etc{\emph{etc}\onedot} \def\vs{\emph{vs}\onedot}
\def\wrt{w.r.t\onedot} \def\dof{d.o.f\onedot}
\def\etal{\emph{et al}\onedot}
\makeatother
\usepackage{hyperref}


\newcommand{\theHalgorithm}{\arabic{algorithm}}




\usepackage[accepted]{icml2025}

\usepackage{amsmath}
\usepackage{amssymb}
\usepackage{mathtools}
\usepackage{amsthm}

\usepackage[capitalize,noabbrev]{cleveref}

\theoremstyle{plain}
\newtheorem{theorem}{Theorem}[section]
\newtheorem{proposition}[theorem]{Proposition}
\newtheorem{lemma}[theorem]{Lemma}
\newtheorem{corollary}[theorem]{Corollary}
\theoremstyle{definition}
\newtheorem{definition}[theorem]{Definition}
\newtheorem{assumption}[theorem]{Assumption}
\theoremstyle{remark}
\newcommand\blfootnote[1]{%
  \begingroup
  \renewcommand\thefootnote{}\footnote{#1}%
  \addtocounter{footnote}{-1}%
  \endgroup
}
\usepackage[textsize=tiny]{todonotes}
\usepackage{soul}

\newif\ifcomment
\commenttrue


\definecolor{red200}{HTML}{EF9A9A}
\definecolor{red400}{HTML}{EF5350}
\definecolor{blue50}{HTML}{E3F2FD}
\definecolor{blue100}{HTML}{BBDEFB}
\definecolor{blue200}{HTML}{90CAF9}
\definecolor{green200}{HTML}{A5D6A7}
\definecolor{yellow500}{HTML}{FFEB3B}

\ifcomment
    \newcommand{\marco}[1]{\sethlcolor{red200}\hl{[\textbf{Marco:} #1]}}
    \newcommand{\eros}[1]{\sethlcolor{yellow500}\hl{[\textbf{Eros:} #1]}}
\else
    \newcommand{\marco}[1]{}
    \newcommand{\eros}[1]{}
\fi


\icmltitlerunning{Interaction-Aware Gaussian Weighting for Clustered Federated Learning}

\begin{document}

\twocolumn[
\icmltitle{Interaction-Aware Gaussian Weighting for Clustered Federated Learning}



\icmlsetsymbol{equal}{*}
\icmlsetsymbol{start}{$\dagger$}

\begin{icmlauthorlist}
\icmlauthor{Alessandro Licciardi}{equal,disma,infn}
\icmlauthor{Davide Leo}{equal,dauin}
\icmlauthor{Eros Fan\'i }{dauin}
\icmlauthor{Barbara Caputo}{dauin}
\icmlauthor{Marco Ciccone}{start,vector}
\end{icmlauthorlist}

\icmlaffiliation{dauin}{Department of Computing and Control Engineering, Polytechnic University of Turin, Italy}
\icmlaffiliation{disma}{Department of Mathematical Sciences, Polytechnic University of Turin, Italy}
\icmlaffiliation{vector}{Vector Institute, Toronto, Ontario, Canada}
\icmlaffiliation{infn}{Istituto Nazionale di Fisica Nucleare (INFN), Sezione di Torino, Turin, Italy}

\icmlcorrespondingauthor{Alessandro Licciardi}{alessandro.licciardi@polito.it}


\icmlkeywords{Machine Learning, Federated Learning, Clustered Federated Learning}

\vskip 0.3in
]



\blfootnote{* Equal contribution, $\dagger$ Project started when the author was at Polytechnic University of Turin.  \textsuperscript{1} Department of Mathematical Sciences,
Polytechnic University of Turin, Italy - \textsuperscript{2} Istituto Nazionale di Fisica
Nucleare (INFN), Turin, Italy - \textsuperscript{3} Department of
Computing and Control Engineering, Polytechnic University of
Turin, Italy - \textsuperscript{4} Vector Institute, Toronto, Ontario, Canada. Correspondence to: Alessandro Licciardi \textit{alessandro.licciardi@polito.it}}
\begin{abstract}

Federated Learning (FL) emerged as a decentralized paradigm to train models while preserving privacy. However, conventional FL struggles with data heterogeneity and class imbalance, which degrade model performance.
Clustered FL balances personalization and decentralized training by grouping clients with analogous data distributions, enabling improved accuracy while adhering to privacy constraints. This approach effectively mitigates the adverse impact of heterogeneity in FL.
In this work, we propose a novel clustered FL method, \shortname (Federated Gaussian Weighting Clustering), which groups clients based on their data distribution, allowing training of a more robust and personalized model on the identified clusters. \shortname identifies homogeneous clusters by transforming individual empirical losses to model client interactions with a Gaussian reward mechanism. Additionally, we introduce the \textit{Wasserstein Adjusted Score}, a new clustering metric for FL to evaluate cluster cohesion with respect to the individual class distribution. Our experiments on benchmark datasets show that \shortname outperforms existing FL algorithms in cluster quality and classification accuracy, validating the efficacy of our approach.
\end{abstract}

\documentclass[../main.tex]{subfiles}
\graphicspath{{../images/}}
\makeatletter
\def\input@path{{../images/}}
\makeatother
\begin{document}
\section{Introduction}
\begin{figure}
\centering
\begin{tikzpicture}
\node[inner sep=0pt] (ws) at (0, 0) {
\includegraphics[height=.4\textwidth, trim={10cm 0 10cm 0},clip]{world_space.png}};
\node[inner sep=0pt] (cs) at (6,0) {\includegraphics[height=.4\textwidth, trim={10cm 1cm 10cm 4cm},clip]{conf_space.png}};
\end{tikzpicture}
\vspace{-5pt}
\label{fig:pbrm_intro}
\caption{\textbf{Left}: Shows world space obstacles as grey spheres. Robots start and goal configuration is colored red and green, respectively. Configurations along the computed path are colored transparent blue. \textbf{Right:} Mapped world space scenario to configuration space. Obstacle region is the grey mesh. Red spheres are collision-free regions computed by the neural SCDF. The optimized shortest path in the convex corridor is the blue curve.}
\vspace{-25pt}
\end{figure}
Motion planning is the problem of finding a collision-free trajectory that connects a given start and goal configuration. The planning takes place in the configuration space of the robot. For single body robots, like mobile robots or drones, the configuration space and the world space are usually the same. This simplifies the planning, since explicit obstacle representations are available which enables geometrical tools like separating hyperplanes, smallest distance to obstacles etc., to be used when designing motion planning algorithms. For multi-body robots like manipulators, the situation is completely different. The world space obstacles are usually mapped to non-convex regions, and to make the problem even harder, the mapping is usually not known. Forming explicit representations of the obstacle region in the configuration space is usually too expensive or intractable. Despite all of this, sampling based planners are used with great success, which mainly is due to their use of implicit representations of the obstacle region. The basic idea is to construct a graph in the configuration space that covers and connects the collision-free region. From this graph, a path can be extracted that connects a given start and goal configuration. The approach is computationally expensive, since the graph is constructed with the smallest geometrical building block available, points, which represents a collision-check. Furthermore, the extracted paths from the graph are non-smooth and jagged due to the stochastic nature of the approach. This adds an additional post-processing step to the process, where the paths are shortcutted and smoothened, before the path can be used for tracking. Clearly a lot of time is invested to form this graph and produce smooth paths. Thus, if the obstacles start to move, then all of this work is done in no use, since all points that make up this graph need to be re-verified, which is simply too time consuming to be done in real time.
\\\\
In this work, we want to address the existing drawbacks of the sampling based planners. Our main contribution is an improved motion planner where each vertex in the graph covers a collision-free region in the form of a sphere instead of a point and where the edges are formed with neighboring intersecting spheres. This representation has the advantage of instead of returning piecewise linear paths, returning a sequence of overlapping spheres, i.e. a convex corridor, that connects a given start and goal configuration, illustrated in Figure \ref{fig:pbrm_intro}. This convex corridor allows us to use convex optimization to produce smooth trajectories, instead of computationally expensive post-processing methods. The representation further allows us to estimate the coverage of the collision-free space, which gives us awareness and feedback in the offline roadmap construction phase. Finally, our representation is simple to adapt to moving obstacles, simply requery for the new radii and recheck for intersections. 
\\\\
The spherical collision-free regions are formed using a signed distance function (SDF), which is a function that returns the smallest distance from an arbitrary point to the boundary of an obstacle. As the name implies, the distance is signed, thus if the point is inside the obstacle it is negative otherwise positive. If the distance is positive, a sphere with radius equal to the distance is guaranteed to cover a collision-free region. Using an SDF in motion planning is not new, but what is novel about our approach is that we express the distance in the configuration space instead of the world space and by doing so allows us to form these convex collision-free regions. We refer to the resulting SDF as a signed configuration distance function (SCDF). Computing an SCDF analytically is non-trivial, our approach is therefore to parameterize the SCDF with a deep neural network and learn the mapping by supervised learning. Our resulting neural SCDF can compute distances for different parameter values of obstacle shapes and we also show how multiple distances can be combined, thus making our approach flexible.
\section{Related work}
Motion planning algorithms can roughly be divided into three families, grid-based, sampling based and optimization based methods. Grid-based methods (GBM) discretize the planning space from which a graph is then compiled. A standard search method is A$^\star$ \citep{a_star}, which is classified as an \textit{informed} search method, since it employs a heuristic function to speed up the search. A$^\star$ guarantees to return an optimal path at the level of discretization used. GBMs usually discretize the planning space by a regular lattice and this limits the GBMs to problems with low dimensionality due to the curse of dimensionality. Thus, GBMs are usually limited to single-body robots where the degrees of freedom (DOF) are low. To overcome the inherent scaling problem with the GBMs, stochastic methods are usually used for multi-body robots. These methods are termed as sampling-based methods (SBM) and core members within this family are the rapidly-exploring random trees (RRT) \citep{rrt} and the probabilistic roadmap (PRM) \citep{prm}. RRT grows a tree from the start configuration and explores the collision-free region in a rapid way until it is able to connect to the goal region. RRT is usually improved by bi-directional planning \citep{rrt_connect}, i.e. an additional tree is grown from the goal configuration and the trees are tested for connection after any tree has been expanded. RRT is a single-query method, thus it searches for a path from scratch each time it is queried. Contrary to this, PRM is a multi-query method, which solves for multiple queries without starting from scratch. PRM does this by creating a roadmap (graph) that covers the collision-free space as an offline step. The graph is then used to solve for multiple queries. PRMs are used in cases where the environment does not change since the extra offline step is too computationally costly and needs to be re-done if the environment is changed. In our work, we address this inherent issue by using a different roadmap representation. Our vertices in the graph cover a collision-free region in the form of spheres and we form the edges by checking for intersecting spheres. If something in the environment changes, we recompute the spheres radii and recheck the intersections, without relying on collision detection. We use a trained neural network to compute the sphere radius, therefore querying for the radius can be done fast, hence our representation enables the PRM for dynamic environments.
\\\\
In the recent decades, optimization based methods (OBM) \citep{chomp, schulman, itomp, stomp} have been introduced as an alternative to SBM for multi-body robots. Like the SBM, the OBMs scale well to higher dimensional problems and produce smoother motion. It is common to use a SDF in the optimization since it is a smooth function, thus enabling gradient-based methods. However, the standard way of expressing the SDF is in world space. The distance therefore needs to be mapped to the configuration space by the forward kinematics. This mapping makes the optimization problem a non-linear program (NLP), which is computationally expensive to solve. Recently, a different approach has been proposed. In \cite{mp_gcs} motion planning is formulated as a convex optimization problem by using the graph of convex sets framework \citep{gcs}. The underlying idea is to decompose the collision-free space into intersecting convex sets from which a convex optimization problem is formulated. In cases where an explicit representation of the obstacles in the configuration space exists, like for single-body robots, creating collision-free convex regions can be done fast \citep{iris}. For multi-body robots, this is non-trivial. Existing work does this successfully \citep{iris_nlp, iris_c} by an optimization based approach, but the methods are still too time consuming to be used in the presence of moving obstacles. Our approach is instead to use deep learning to learn an SDF expressed in the configuration space. With this, we can query for shortest distances to the collision boundary, which allows us to expand spherical regions which are collision-free. Our approach is fast and therefore enables our suggested roadmap planner to be used in dynamic environments.
\\\\
Recent research has focused on learning collision detection \citep{fk_kernel_distance, diffco, graphdistnet} by predicting the signed distance between the robot links and the surrounding obstacles in the world space. The learned SDF is used in trajectory optimization but since the distance is expressed in the world space, the problem becomes an NLP and therefore takes a long time to solve. We take a novel approach and suggest to instead express the signed distance in the configuration space. This allows us to improve the PRM at the same time as it enables convex optimization for trajectory optimization, which runs faster and is more reliable than NLP solvers. In \cite{cspf} a learned signed distance function in the configuration space is proposed similar to our approach. However, their approach is restricted to point cloud representations, while we propose to represent the obstacles as parameterized geometric shapes, e.g. spheres. Furthermore, we also show how to use our learned SCDF to improve an existing roadmap planner.
\section{Problem formulation}
A robot is located in the world space, $\W \subset \R^3 $. The unique location of the robot is given by its configuration $\q \in \C$, where $\C$ is the configuration space. The set of points covered by the robots bodies at a certain configuration is expressed as $\B(\q) \subset \W$. The robot is surrounded by $\NrObst$ obstacles $\O = \bigcup_{i=1}^{\NrObst} \O_i$, where  $\O_i \subset \W$. The representation of the obstacle in the configuration space is the set $\C\O_i = \{\q \in \C \: |\: \B(\q) \cap \O_i \neq \emptyset \}$. The obstacle space is formed as $\Co = \bigcup_{i=1}^{\NrObst} \C \O_i$. The complement is referred to as the free space, $\Cf = \C \setminus \Co$. The path planning problem is a tuple, ($\Cf$, $\qStart$, $\qGoal$), where we want to connect a query pair, consisting of a start, $\qStart$, and goal configuration, $\qGoal$, with a geometric path, $\q(s): [0, 1] \mapsto \Cf$, such that $\q(0)=\qStart$ and $\q(1)=\qGoal$, or report correctly when such a path does not exist.
\end{document}

\section{Related Work}
% \subsection{Vision Language Model}
% 시각장애인에서 상황을 설명할 DB가 없으니 만들었다. 그리고 이를 VLM에 튜닝했다.
\subsection{Technical approaches for assisting the visually-impaired}


\subsection{Datasets for visual instruction tuning}

% \begin{figure}
%     \centering
%     \includegraphics[width=0.5\linewidth]{Move_teaser.pdf}
%     \caption{Comparison of different dynamic compute approaches. length of arrow indicates residual transformation per token while width indicates velocity of transformation.}
%     \label{fig:enter-label}
% \end{figure}

\section{Method}
\label{sec:method}
Residual connections play a crucial role in shaping token representations, yet their dynamics remain underexplored in the context of efficient decoding. In this work, we delve deeper into transformer residual dynamics and investigate how modulating residual transformation velocity can improve inference efficiency in token-level processing, optimizing both dense and sparse MoE transformers.


\subsection{Residual Dynamics and Motivation for Multi-rate Residuals} \label{sec:motivation}

To analyze how hidden representations evolve across different layers of a transformer architecture, it's crucial to consider the effect of residual connections. Each transformer decoder layer typically has residual connections across attention and MLP submodules. As the residual stream $h_i$ traverses from interval $E_j$ to $E_{j+1}$, it undergoes a residual transformation given by:  
% \begin{equation}
% \label{eq:slow_residual_transformation}
% H_{E_{j+1}} = H_{E_j} \prod_{i=E_j}^{E_{j+1}} \left( I + \mathcal{A}_i \right) \left( I + \mathcal{M}_i \right) \quad \text{where} \quad \mathcal{A}_i = f(c_i, h_{i}), \mathcal{M}_i = g(h_i)
% \end{equation}

\begin{equation} \label{eq:slow_residual_transformation}
h_{E_{j+1}} = h_{E_j} + \sum_{i=E_j}^{E_{j+1}-1} \left( \mathcal{A}_i(h_i) + \mathcal{M}_i(h_i + \mathcal{A}_i(h_i)) \right) \quad \text{where} \quad \mathcal{A}_i = f(c_i, h_{i}), \mathcal{M}_i = g(h_i). 
\end{equation}

Here, \( \mathcal{A}_i \) denotes the non-linear transformation introduced by the multi-head attention mechanism at layer \( i \), while \( \mathcal{M}_i \) corresponds to the non-linear transformation of the MLP block at the same layer. These transformations depend on the input residual stream \( h_i \) and, in the case of \( \mathcal{A}_i \), the previous contextual representation \( c_i \).\footnote{Normalization layers are typically applied in practice but are omitted here for simplicity of the argument.}


% For easy tokens, the magnitude and direction of this delta transformation become progressively smaller with each successive layer as shown in \cref{fig:delta_transformation}. Consequently, it is feasible to predict these tokens after only a few residual connections, whereas harder tokens necessitate more extensive processing through additional layers.

\begin{figure}[ht]
    \centering
    \begin{subfigure}{0.48\textwidth}
        \centering
        \includegraphics[width=\textwidth]{sections/figures/residual_change.pdf}
        \caption{}
        \label{fig:residual_change}
    \end{subfigure}%
    \hfill
    \begin{subfigure}{0.48\textwidth}
        \centering
        \includegraphics[width=\textwidth]{sections/figures/alignment_wrt_dedicated_model.pdf}
        \caption{}
    \label{fig:alignment_wrt_dedicated_model}
    \end{subfigure}
    \caption{(a) As residual streams propagate through the model, the directional shifts in the residuals become progressively smaller. (b) A dedicated model with $k$ layers achieves a faster rate of change in residual streams and higher alignment than base model leveraging early exit mechanisms at layer $k$.}
    \label{fig}
\end{figure}


To examine whether residual transformations can be accelerated across layers, we conducted experiments using a diverse set of prompts on a pre-trained Phi3 model~\cite{phi3_report}. As illustrated in \cref{fig:residual_change}, we measured the directional shift in residual states as \( 1 - \mathcal{C}(h_{i-1}, h_i) \), where \(\mathcal{C}\) denotes normalized cosine similarity. This shift is notably higher in the initial layers, gradually decreasing in subsequent layers. This behavior allows traditional early exit approaches to effectively accelerate decoding by enabling earlier exits for simpler tokens. However, these approaches typically rely on a distance-based approximation, where the full residual transformation of the model is approximated by the residual transformations of the initial layers. To gain deeper insights into the distance versus velocity aspects of residual transformation, we conducted a comparative study. Specifically, we trained an early exit head at layer $k$ of the Phi3 model, which consists of 32 layers, restricting the distance traveled by each token. To accelerate the residual transformation relative to number of layers, we trained a smaller model consisting of only $k$ layers, while keeping all other hyperparameters consistent. We then compared the next-token prediction accuracy of the early exit head of the base model with that of the smaller model. To ensure an equal number of trainable parameters, we inserted low-rank adapters into the smaller model and trained only these adapters, whereas, in the distance-based approach, we trained solely the early exit head. In addition, to accelerate the residual transformation in smaller model, we distilled the residual streams from the larger model by incorporating a distillation loss ~\cite{sanh2019distilbert} between the residual state at layer \(i\) of the smaller model and the residual state at layer \(4 \times i\) of the larger model. As shown in ~\cref{fig:alignment_wrt_dedicated_model} the smaller model demonstrates a significantly faster rate of change in residual streams, leading to higher next token prediction accuracy after $k$ layers compared to the base model that employs traditional early exit mechanisms after $k$ layers \cite{schuster2022confident, chen2023eellm, varshney-etal-2024-investigating}. This experimental setup, which modifies only the rate of change in residual streams while keeping other factors constant, suggests that dense transformers, trained with a fixed number of layers, may inherently possess a slow residual transformation bias.

This observation raises an intriguing question: if the rate of change in residual streams could be accelerated relative to the number of layers, is it possible to facilitate earlier alignment for a greater proportion of tokens? Earlier alignment would be beneficial to not only facilitate dynamic computation but also for generating speculative tokens efficiently with high acceptance rates in speculative decoding setups ~\cite{leviathan2023fast, chen2023accelerating}. 

%thereby enhancing the efficiency of early exiting? 
 % This bias likely constrains the effectiveness of early exiting, particularly for easier tokens. By addressing this limitation through accelerated residual transformations, we hypothesize that it is possible to substantially improve the efficiency and accuracy of early exit strategies in transformer models.

\subsection{Multi-Rate Residual Transformation} \label{m2r2_method}

To address the slow residual transformation bias described in ~\cref{sec:motivation}, we introduce \textit{accelerated residual streams} that operate at rate $R$ relative to original slow residual stream. We pair slow residual stream, $h$ with an accelerated residual stream, $p$, which has an intrinsic bias towards earlier alignment. Relative to ~\cref{eq:slow_residual_transformation}, accelerated residual transformation from interval $E_j$ to $E_{j+1}$ can be represented as: 

% \begin{equation}
% \label{eq:fast_residual_transformation}
% P_{E_{j+1}} = P_{E_j} \prod_{i=E_j}^{E_{j+1}} \left( I + \hat{\mathcal{A}_i} \right) \left( I + \hat{\mathcal{M}_i} \right) \quad \text{where} \quad \hat{\mathcal{A}_i} = \hat{f}(c_i, P_{i}), \hat{\mathcal{M}_i} = \hat{g}(P_{i})
% \end{equation}


\begin{equation} \label{eq:fast_residual_transformation}
p_{E_{j+1}} = p_{E_j} + \sum_{i=E_j}^{E_{j+1}-1} \left( \hat{\mathcal{A}_i}(p_i) + \hat{\mathcal{M}_i}(p_i + \hat{\mathcal{A}_i}(p_i)) \right) \quad \text{where} \quad \hat{\mathcal{A}_i} = \hat{f}(c_i, p_{i}), \hat{\mathcal{M}_i} = \hat{g}(h_i), 
\end{equation}



where $\hat{\mathcal{A}_i}$ and $\hat{\mathcal{M}_i}$ denote non-linear transformation added by layer $i$ to previous accelerated residual $p_{i}$. Similar to $\mathcal{A}_i$, non-linear transformation $\hat{\mathcal{A}_i}$ attends to same context $c_i$ but uses a different transformation $\hat{f}$ for accelerating $p_{E_j}$ relative to $h_{E_j}$. 

We integrate accelerated residual transformation directly into the base network using parallel accelerator adapters such that rank of accelerator adapters $R_p << d$ where $d$ denotes base model hidden dimension. This setup allows the slow residual stream $h_{E_j}$ to pass through the base model layers while the accelerated residual stream $p_{E_j}$ utilizes these parallel adapters as shown in ~\cref{fig:m2r2_main}. Both slow and accelerated residuals are processed in same forward pass via attention masking and incur negligible additional inference latency in memory bound decoding setups, while in compute bound decoding setups where FLOPs optimization is essential, accelerated residual stream utilizes a fraction of attention heads that of slow residual (see ~\cref{sec:flops_optimization}). Additionally, to maximize the utility of accelerated residual transformations without introducing dedicated KV caches, we propose a shared caching mechanism between the slow and accelerated streams which minimally impact alignment benefits of our approach while offering substantial memory savings (see ~\cref{fig:koala_alignment}). Specifically, the attention operation on the slow residuals \( \text{MHA}(h_t, h_{\leq t}, h_{\leq t}) \) is redefined for accelerated residuals as 
\[
\hat{\mathcal{A}} = MHA(p_t, h_{<t} \oplus p_t, h_{<t} \oplus p_t),
\]
where the accelerated residual at time-step $t$, \( p_t \) attends to the slow residual’s KV cache, facilitating the reuse of contextual information across both residual streams without incurring additional caching costs. Here, \(MHA(q, k, v) \) represents multi-head attention between query \( q \), key \( k \), and value \( v \).

\begin{figure}
    \centering
    \includegraphics[width=0.8\linewidth]{sections//figures/m2r2_main2.pdf}
    \caption{Multi-rate Residuals Framework: Slow residual stream of base model is accompanied by a faster stream that operates at a $2-(J+1)\times$ rate relative to the slow stream, undergoing transformations via accelerator adapters as detailed in \cref{m2r2_method}, where J denotes number of early exit intervals. Colors within the slow and fast residual streams indicate similarity, with matching colors representing the most closely aligned residual states. At the beginning of the forward pass and at each exit point, the accelerated residual state is initialized from the corresponding slow residual state to avoid gradient conflict during training (see ~\cref{sec:grad_conflict}). Early exiting decisions are informed by the Accelerated Residual Latent Attention (ARLA) mechanism, described in \cref{method_arla}, which evaluates residual dynamics across consecutive exit gates.}
    \label{fig:m2r2_main}
\end{figure}

% Furthermore. to maximize the benefits of fast residual transformations without using dedicated KV caches, we propose sharing the fast network’s cache with the slow network. Formally speaking, We modify attention operation on slow residuals $MHA(H_t, H_{<=t}, H_{<=t})$ as $MHA(P_{t}, H_{<t} \oplus P_t, H_{<t}  \oplus P_t)$ such that accelerated residuals attend to previous slow context KV cache, where $MHA(q,k,v)$ denotes multi head attention between query, $q$, key $k$ and value $v$.


\subsection{Enhanced Early Residual Alignment}
Early residual alignment is instrumental in optimizing early exiting, speculative decoding, and Mixture-of-Experts (MoE) inference mechanisms. In this section, we provide a detailed analysis of how accelerated residuals enhance these inference setups.

% By aligning the residual states of intermediate layers with the final output representations, the model can maintain high prediction accuracy even when computations are truncated at earlier layers. This enables more reliable early exiting, reducing the overall computational cost while preserving performance. Additionally, in speculative decoding, early residual alignment allows the model to make confident predictions using faster, partial computations, thereby accelerating inference without sacrificing output quality.


\subsubsection{Early Exiting} \label{method_early_exiting}

A prevalent strategy for enabling early exiting at an intermediate layer $E_{j}$ involves approximating the residual transformation between $E_{j}$ and the final layer $N-1$ using a linear, context independent mapping, $\mathcal{T}$, such that $H_{N-1} \approx \mathcal{T}(H_{E_{j}})$. This approximation has been extensively employed in conventional approaches ~\cite{schuster2022confident, chen2023eellm, varshney-etal-2024-investigating}, providing a computationally efficient means to project the output of deeper layers from intermediate states. Specifically, residual state of layer $N-1$ with this approximation can be expressed as:


% \begin{equation}
% \label{eq: vanila_ea_assumption}
% \Phi(H_{E_{j}}) \sim H_{E_{j}} \prod_{i=E_{j}}^{N}\left( I + \mathcal{A}_i \right) \left( I + \mathcal{M}_i \right) \quad \text{where} \quad \Phi \perp C
% \end{equation}

\begin{equation} \label{eq:early_exiting}
h_{E_j} + \sum_{i=E_j}^{N-1} \left( \mathcal{A}_i(h_i) + \mathcal{M}_i(h_i + \mathcal{A}_i(h_i)) \right) \sim \mathcal{T}(h_{E_{j}})  \quad \text{where} \quad \mathcal{T} \perp c. 
\end{equation}


Here, $\mathcal{A}_i$ and $\mathcal{M}_i$ represent the residual contributions of the multi-head attention and MLP layers, respectively, while $\mathcal{T}$ remains independent of $c$, the preceding context.

This approach is inherently limited by two major factors: first, the assumption of linearity between $h_{E_{j}}$ and $h_{N-1}$ may not hold uniformly for all tokens, particularly when $E_j \ll N$. Second, the linear transformation $\mathcal{T}$ disregards the influence of the context $c$ and fails to account for the latent representations of previous contextual states. In contrast, M2R2 accelerated residual states mitigate both of these challenges by approximating the slow residual transformation of all layers via a faster residual transformation of fewer layers as:
% \begin{equation}
% H_{E_j} \prod_{i=E_j}^{N}\left( I + \mathcal{A}_i \right) \left( I + \mathcal{M}_i \right) \sim P_{E_j} \prod_{i=E_j}^{E_j+1}\left( I + \hat{\mathcal{A}_i} \right) \left( I + \hat{\mathcal{M}_i} \right)
% \end{equation}


\begin{equation} \label{eq:m2r2_approximating_ea}
h_{E_j} + \sum_{i=E_j}^{N-1} \left( \mathcal{A}_i(h_i) + \mathcal{M}_i(h_i + \mathcal{A}_i(h_i)) \right) \sim p_{E_j} + \sum_{i=E_j}^{E_{j+1}-1} \left( \hat{\mathcal{A}_i}(p_i) + \hat{\mathcal{M}_i}(p_i + \hat{\mathcal{A}_i}(p_i)) \right), 
\end{equation}

% \begin{equation} \label{eq:fast_residual_transformation}
% p_{E_{j+1}} = p_{E_j} + \sum_{i=E_j}^{E_{j+1}-1} \left( \hat{\mathcal{A}_i}(p_i) + \hat{\mathcal{M}_i}(p_i + \hat{\mathcal{A}_i}(p_i)) \right) \quad \text{where} \quad \hat{\mathcal{A}_i} = \hat{f}(c_i, p_{i}), \hat{\mathcal{M}_i} = \hat{g}(h_i) 
% \end{equation}






where $p_{E_j}$ is initialized from the slow residual state $h_{E_j}$ at each early exit interval $E_j$ using an identity transformation (see ~\cref{fig:m2r2_main}). As shown in ~\cref{fig:m2r2_residual_sim}, accelerated residuals offer a smoother, more consistent shift in residual direction across layers, in contrast to the abrupt changes typically seen at early exit points in standard early exit methods. Moreover, the normalized cosine similarity between accelerated states at early exit intervals and final residual states is substantially higher compared to traditional early exit techniques, highlighting improved alignment with final layer representations. Traditional adaptive compute methods are constrained by two principal factors: the number of tokens eligible for early exit at intermediate layers and the precision of early exit decision. If residual streams fail to saturate early, the majority of tokens remain ineligible for exit, thereby diminishing potential speedups. Additionally, imprecise delineations between tokens suitable for early exit can lead to underthinking (premature exits that adversely affect accuracy) or overthinking (unnecessary processing that compromises efficiency) ~\cite{zhou2020self, dai2020dynamic}. Enhanced early alignment using ~\cref{eq:m2r2_approximating_ea} helps to address  first issue. To address the second issue we introduce Accelerated Residual Latent Attention, which dynamically assesses the saturation of the residual stream, allowing for a more precise differentiation between tokens that can exit early and those requiring further processing.

% This results in uniform change in residual direction    
% % We keep $\mathcal{A} = \hat{\mathcal{A}}$, while $\hat{\mathcal{M}}$ is accelerated by a factor of $2 - (N_{E}+1)X$ relative to the slower residual transformation $\mathcal{M}$, where $N_E$ represents number of early exiting intervals.
% Figure~\cref{fig:rate_change_comparison} illustrates the comparative rate of change between these transformation streams.



% fig:rate_change_comparison
% - grid plot x axis -> layer id (0, 8) , y axis -> layer id -> dark color cell for max similarity , lighter for lower 
% 
-------------------------------------------------------
Let's consider residual stream $h_i$ traverses through interval $E_j$ to $E_{j+1}$ and undergoes residual transformation given by 
\begin{equation}
h_{E_{j+1}} = h_{E_j} \prod_{i=E_j}^{E_{j+1}} \left( 1 + \delta_i \right)    
\end{equation}

where $\delta_i$ denotes non-linear transformation added by layer $i$. Each non-linear transformation of layer $i$ is a function of previous contextual representation, $c_i$ and input residual stream $h_i-1$ as
$\delta_i = f(c_i, h_{i-1})$ 

One way to exit early at exit $E_j+1$ is to assume that residual transformation from $E_j+1$ to final layer $N-1$ can be approximated by a linear function $\phi$ as $h_{N-1} \sim \Phi(h_{E_j+1})$ and most conventional approaches such as \todo{cite EA papers} use this approach. In other words, 

\begin{equation}
\Phi(h_{E_j+1} \sim h_{E_j+1} \prod_{i=E_j+1}^{N} \left( 1 + \delta_i \right)   
\end{equation}

This approach suffers from two primary issues, linearity assumption from $h_E_j+1$ to $H_N-1$ if often incorrect, particularly when $E_j << N$. More importantly, linear transformation $\Phi$ doesn't consider effect of context $C_i$. M2R2  effectively addresses these issues as accelerated residual stream at interval $E_j+1$ can be represented as 

\begin{equation}
r_{E_{j+1}} = r_{E_j} \prod_{i=E_j}^{E_{j+1}} \left( 1 + \gamma_i \right)    
\end{equation}

where $\gamma_i$ denotes non-linear transformation added by layer $i$ to previous accelerated residual $r_i-1$. Similar to $\delta_i$, non-linear transformation $\gamma_i$ considers context $C_i$ as 
$\gamma_i = g(c_i, r_{i-1})$. So in summary, slow residual transformation is approximated by accelerated residual as: 

\begin{equation}
h_{E_j} \prod_{i=E_j}^{N} \left( 1 + \delta_i \right) \sim h_{E_j} \prod_{i=E_j}^{E_j+1} \left( 1 + \gamma_i \right)
\end{equation}

It's worth noting that accelerated residual $r_i$ and slow residual $h_i$ are processed concurrently at layer $i$ by constructing proper attention mask such as attention of slow residual is represented as 

$MHA(H_it, H_{i<=t}, H_{i<=t}$ while attention of fast residual is computed as 

$MHA(r_it, H_{i<=t}, H_{i<=t}$ where $MHA(q,k,v$ denotes multi head attention between query, $q$, key $k$ and value $v$.


------------------------------------------------------------------

Vertical latent attention on accelerated residual is computed as 
$MHA(S_mt, S(Ej<=i<=m)t, S(Ej<=i<=m)t)$ where $Smt$ denotes query/key/value projection in latent domain at layer $m$ at time $t$. 
------------------------------------------------------------------

Gradient conflict Avoidance: 

Let's consider $w_j$ is a trainable parameter that belongs to a layer between $E_j$ and $E_j+1$. Consider early exit loss at gate $E_j+1$, $L_j+1$, gradient propagation of $w_j$ at another trainable parameter $w_j-n$ can be gives as 

$\sum_{k=E_j-n}^{E_j} \beta_k \frac{\partial L_{E_k}}{\partial w_k}$

where $\beta_j$ denotes backward transformation coefficient for weight $w_j$ to reach gate $E_j$. 
 
On the other hand, gradient propagation in proposed approach can be represented as 

\[
\frac{\partial L_{E_j}}{\partial w_j} = 
\begin{cases} 
\beta_j \frac{\partial L_{E_j}}{\partial w_j} & \text{if } E_j \leq w_j \leq E_{j+1} \\
0 & \text{otherwise}
\end{cases}
\]







% \begin{figure}[ht]
%     \centering
%     \includegraphics[width=0.8\textwidth, height=5cm]{rate_change_comparison.png}
%     \caption{Rate of change comparison between fast and slow residual streams.}
%     \label{fig:rate_change_comparison}
% \end{figure}

%vary k and and plot EA accuracy for larger and smaller models. 

% \begin{figure}[ht]
%     \centering
%     \includegraphics[width=0.5\textwidth,height=5cm]{sections/figures/alignment_comparison_dialogsum.pdf}
%     \caption{Alignment of exited tokens for different early exit layers using traditional early exiting heads, dedicated faster networks, and faster residuals.}
%     \label{fig:small_model_early_exiting}
% \end{figure}


\textbf{Accelerated Residual Latent Attention} \label{method_arla}

In the context of residual streams, we observe that the decision to exit at a given layer can be more effectively informed by analyzing the dynamics of residual stream transformations, instead of solely relying on a classification head applied at the early exit interval $E_j$. To capture the subtle dynamics of residual acceleration, we propose a \textit{Accelerated Residual Latent Attention} (ARLA) mechanism. This approach involves making the exit decision at gate $E_j$ by attending to the residuals spanning from gate $E_{j-1}$ to $E_j$, rather than considering only the residual at gate $E_j$. To minimize the computational overhead associated with exit decision-making, the attention mechanism operates within the latent domain as depicted in ~\cref{fig:arla_arch}. Formally, for each interval $[E_j, E_{j+1}]$, the accelerated residuals are projected into Query ($Q^s_{E_j}, \ldots, Q^s_{E_{j+1}}$), Key ($K^s_{E_j}, \ldots, K^s_{E_{j+1}}$), and Value ($V^s_{E_j}, \ldots, V^s_{E_{j+1}}$) vectors, with latent dimension $d^s$ for $Q^s$, $K^s$, and $V^s$ being significantly smaller than hidden dimension of $p$.\footnote{We use $d^s = 64$ for experiments described in ~\cref{sec:experiments}.} Notably, when the router is allowed to make exit decisions at gate $E_j$ based on residual change dynamics, we observe that the attention is not confined to the residual state at $E_j$ but is distributed across residual states from $E_{j-1}$ to $E_j$, %as illustrated in Figure~\ref{fig:vertical_latent_attention_dynamics}. 
This broader focus on residual dynamics significantly reduces decision ambiguity in early exits, as demonstrated in Figure~\ref{fig:roc_arla}, which contrasts routers based on the last hidden state, and the proposed ARLA router.

%show R -> S transformation. 
%show parameter and flop overhead as compared to adapter on last hidden state.

% \begin{figure}[ht]
%     \centering
%     \includegraphics[width=0.5\textwidth,height=5cm]{sections/figures/roc_arla.pdf}
%     \caption{ROC curves of early exit decision strategies: confidence-based methods (CALM/LITE), routers based on the accelerated hidden state, and latent attention routers.}
%     \label{fig:decision_making_comparison}
% \end{figure}

% \begin{figure}[ht]
%     \centering
%     \includegraphics[width=0.5\textwidth,height=5cm]{vertical_latent_attention.png}
%     \caption{Vertical latent attention mechanism for optimizing early exit decisions by considering residuals from gate \(M\) through \(M-1\).}
%     \label{fig:vertical_latent_attention}
% \end{figure}

\begin{figure}[ht]
    \centering
    \begin{subfigure}{0.52\textwidth}
        \centering
        \includegraphics[width=\textwidth, height = 4cm]{sections/figures/arla_arch.pdf}
        \caption{Accelerated Residual Latent Attention (ARLA): Accelerated residuals between early exit gates are projected into latent domain and attention over residual states within the interval is computed to capture residual dynamics and exit decision is made based on residual saturation.}
        \label{fig:arla_arch}
    \end{subfigure}%
    \hfill
    \begin{subfigure}{0.45\textwidth}
        \centering
        \includegraphics[width=\textwidth, height = 4.5cm]{sections/figures/vla_roc.pdf}
        \caption{ROC classification curves of early exit decision strategies using a linear router used on last residual state ~\cite{schuster2022confident, varshney-etal-2024-investigating, chen2023eellm}  and using ARLA approach that considers residual dynamics. }
        \label{fig:roc_arla}
    \end{subfigure}
    \caption{Effectiveness of ARLA in capturing residual dynamics for early exiting decisions.}


\end{figure}



% \begin{figure}[ht]
%     \centering
%     \includegraphics[width=1\textwidth,height=5cm]{sections/figures/arla.pdf}
%     \caption{fig that plots 32 rows 2 cols heatmap showing attention at each gate}
%     \label{fig:vertical_latent_attention_dynamics}
% \end{figure}

\subsubsection{Self Speculative Decoding} \label{method_self_speculative_decoding}

An alternative means to exploit the early alignment properties of our approach is through the use of accelerated residual states for speculative token sampling to accelerate autoregressive decoding. Speculative decoding aims to speed up memory-bound transformer inference by employing a lightweight draft model to predict candidate tokens, while verifying speculated tokens in parallel and advancing token generation by more than one token per full model invocation \cite{leviathan2023fast, chen2023accelerating, xia2023speculative, miao2023specinfer}. Despite its effectiveness in accelerating large language models (LLMs), speculative decoding introduces substantial complexity in both deployment and training. A separate draft model must be specifically trained and aligned with the target model for each application, which increases the training load and operational complexity ~\cite{chen2023accelerating}. Additionally, this approach is resource-inefficient, as it requires both the draft and target models to be simultaneously maintained in memory during inference \cite{leviathan2023fast, chen2023accelerating}. 

One strategy to address this inefficiency is to leverage the initial layers of the target model itself to generate speculative candidates, as depicted in ~\cite{Tang2024}. While this method reduces the autoregressive overhead associated with speculation, it suffers from suboptimal acceptance rates. This occurs because the linear transformation employed for translating hidden states from layer $k$ to the final layer $N$ is typically a poor approximation, as discussed in ~\cref{sec:motivation} and ~\cref{method_early_exiting}. Our approach resolves this limitation by utilizing accelerated residuals, which demonstrate higher fidelity to their slower counterparts. By utilizing accelerated residuals operating at a rate of $N/k$, where $k$ denotes the number of layers used for candidate speculation, we are able to efficiently generate speculative tokens for decoding.\footnote{We typically set $k = 4$ to balance the trade-off between autoregressive drafting overhead and acceptance rate, as discussed in~\cref{sec:experiments}.}
 This technique not only obviates the need for multiple models during inference but also improves the overall efficiency and effectiveness of speculative decoding.

\begin{figure}
    \centering    \includegraphics[width=1\linewidth]{sections/figures/m2r2_aot_loading.pdf}
    \caption{Ahead-of-Time Expert Loading: M2R2 accelerated residual stream predicts experts required for future layers, reducing reliance on on-demand lazy loading. Speculative pre-loading is efficiently overlapped with computation of multi-head attention (MHA) and MLP transformations. Only incorrectly speculated experts are loaded lazily, resulting in faster inference steps and improved computational efficiency. Here, H indicates LBM Host while D indicates HBM Device.}
    \label{fig:moe_expert_aot_loading}
\end{figure}


\subsubsection{Ahead of Time Expert Loading:} \label{method_aot_expert_loading}

Recent advancements in sparse Mixture-of-Experts (MoE) architectures ~\cite{shazeer2017outrageously, fedus2022switch, artetxe2019massively, lepikhin2020gshard, zoph2022designing} have introduced a paradigm shift in token generation by dynamically activating only a subset of experts per input, achieving superior efficiency in comparison to dense models, particularly under memory-bound constraints of autoregressive decoding \cite{fedus2022switch, zoph2022designing}. This sparse activation approach enables MoE-based language models to generate tokens more swiftly, leveraging the efficiency of selective expert usage and avoiding the overhead of full dense layer invocation. In dense transformer models, pre-loading layers is a common strategy to enhance throughput, as computations of current layer can be overlapped with pre-loading of next layer parameters ~\cite{narayanan2021efficient, shoeybi2020megatron}. However, MoE models face a unique challenge: expert selection occurs dynamically based on previous layer’s output, making it infeasible to preload next layer’s experts in parallel. This limitation results in inherent latency, as expert loading becomes a sequential, on-demand process ~\cite{lepikhin2020gshard, fedus2022switch}.

To address this inefficiency, our method introduces a mechanism with \textit{accelerated residuals}, which not only captures key characteristics of base slower residual states but also exhibit high cosine similarity with their final counterparts (as illustrated in \cref{fig:m2r2_residual_sim}). By employing accelerated residual streams, we can effectively predict the necessary experts for future layers well in advance of their actual invocation. Specifically, using a $2\times$ accelerated residual, the experts needed for layers $2i+2$ and $2i+3$ can be identified while still computing in layer $i$, thus overcoming the bottleneck of sequential, on-demand expert selection and mitigating latency in the decoding pipeline, as shown in \cref{fig:moe_expert_aot_loading}. Note that, we use fixed set of accelerator adapters for transforming accelerated residuals (as discussed in ~\cref{m2r2_method}) while slow residual is transformed via expert routing mechanism. 

Furthermore, our approach integrates a Least Recently Used (LRU) caching strategy, which enhances memory efficiency by replacing the least recently used experts with speculated experts that are anticipated to be needed in upcoming layers. This hybrid approach of preemptive expert loading with LRU caching yields substantial improvements over traditional on-demand loading or standalone caching strategies. By minimizing cache misses and efficiently managing memory, this approach addresses both compute and memory bottlenecks, leading to faster, more resource-efficient token generation in MoE architectures. A comprehensive evaluation of this strategy, in relation to state-of-the-art methods, is provided in \cref{experiments_aot}, and the compute and memory traces on an A100 GPU are detailed in \cref{fig:moe_aot_cuda_trace}.



% Recent advancements in sparse Mixture-of-Experts (MoE) architectures have introduced the concept of utilizing distinct computational paths for different tokens \cite{shazeer2017outrageously}. This approach, wherein only a subset of experts are activated per input, enables MoE-based language models to generate tokens more swiftly compared to their dense counterparts due to memory-bound nature of auto-regressive decoding. In dense models, pre-loading layers in advance is a common strategy to enhance computational efficiency. However, this technique is not applicable to MoE models, where expert selection occurs dynamically based on the outputs of previous layers, preventing parallel pre-fetching of experts.

% Our proposed method addresses this inefficiency. Accelerated residuals, which are highly similar to their slower counterparts (see \cref{fig:similarity}), can reliably predict the necessary experts ahead of time. For instance, by utilizing $2X$ accelerated residual stream, we can predict the experts needed for the layer $2i+1$ and $2i+3$ while carrying out computation in layer $i$. This enables us to commence expert loading significantly earlier, as illustrated in \cref{expert_loading}, effectively mitigating the delays observed with the naive on-demand expert loading. Additionally, our method benefits from incorporating a Least Recently Used (LRU) strategy, where speculated experts replace those that are least recently utilized, resulting in improved performance compared to using either strategy alone. For a comprehensive evaluation, refer to \cref{moe_trace}, which provides a CUDA compute and memory trace of our approach executed on <>.



% A naive solution involves using the residual state of the previous layer along with the gating function of the next layer to predict which experts need to be loaded, and initiating the expert loading process in parallel with the attention computation of the next layer. Yet, as shown in \cref{fig:MOE_attn_vs_loading_time}, the attention computation for medium to long contexts is considerably faster than the expert loading time, making this approach inefficient.




\subsection{Training} \label{method_training}
% This approach is feasible due to the absence of gradient conflicts, as discussed in \cref{sec:grad_conflict}.

To accelerate residual streams, we employ parallel accelerator adapters as described in \cref{m2r2_method}.  For the early exiting use-case outlined in \cref{method_early_exiting}, we define the training objective for these adapters using the following loss function, which combines cross-entropy loss at each exit $E_j$ with distillation loss at each layer $i$. Loss weights coefficients $\alpha_0$ and $\alpha_1$ are employed to balance contribution of corresponding losses.

\begin{align} \label{eq:mr_loss}
L_{\text{m2r2}} = \underbrace{-\alpha_0 \sum_{j=1}^{J} \sum_{t=1}^{T} \log p_{\theta} \left( \hat{y}_t^{E_j} \mid y_{<t}, x \right)}_{\text{cross-entropy loss}} 
+ \underbrace{\alpha_1\sum_{i=1}^{E_{J-1}} \sum_{t=1}^{T} \| \mathbf{p}_{t}^{i} - \mathbf{h}_{t}^{((i - E_{j(i)}) \cdot R_i) + E_{j(i)})} \|^2}_{\text{distillation loss}}.
\end{align}

where $\hat{y}_t^{E_j}$ denotes the predictions from the accelerated residual stream at layer $E_j$ and time step $t$, $y_t$ represents the corresponding ground truth tokens, and $x$ indicates previous context tokens. The distillation loss at each layer $i$ is computed by comparing accelerated residuals at layer $i$ with slow residuals at layer $(i - E_{j(i)}) \cdot R_i + E_{j(i)}$, where $R_i$ denotes the rate of accelerated residuals at layer $i$ while $E_{j(i)}$ represents the most recent gate layer index such that $E_{j(i)} <= i$. \( J \) represents the total number of early exit gates, N denotes number of hidden layers and $E_j$ denotes layer index corresponding to gate index $j$ and \( T \) denotes the sequence length. 

In dynamic compute settings, after training of accelerator adapters, we optimize the query, key, and value parameters governing the ARLA routers (see ~\cref{method_arla}) across all exits in parallel on binary cross entropy loss between predicted decision and ground truth exiting decision. The ground truth labels for the router are determined based on whether the application of the final logit head on $\hat{y}_t^{E_j}$ yields the correct next-token prediction. 


% The objective for this optimization is defined by the following loss function:


%TODO are equations required ? 
% \begin{equation} \label{eq:arla_loss_combined}\small
%     L_{\text{arla}} = -\frac{1}{N} \sum_{t=1}^{T} \left( \sum_{j=1}^{E_n} \left[ O_t^{E_j} \log(\hat{O}_t^{E_j}) + (1 - O_t^{E_j}) \log(1 - \hat{O}_t^{E_j}) \right] \right), \quad \text{where} \quad 
%     O_t^{E_j} = \begin{cases} 
%     1, & \text{if } L(\hat{y}_t^{E_j}) = y_t^{E_j} \\
%     0, & \text{otherwise}
%     \end{cases}
% \end{equation}

% where $\hat{O}_t^{E_j}$ represents the binary predicted logits produced by the vertical latent attention router, as described in \cref{sec:arla}, at gate $E_j$ and time step $t$, and $O_t^{E_j}$ denotes the corresponding ground truth labels. The ground truth labels for the router are determined based on whether the application of the logit head on $\hat{y}_t^{E_j}$ yields the correct next-token prediction. The parameters controlling vertical latent attention are trained concurrently to ensure consistency and efficient use of computational resources.

For self-speculative decoding, as described in \cref{method_self_speculative_decoding}, the training objective remains the same as \cref{eq:mr_loss}, but with the number of intervals set to $J = 1$ and the rate of residual transformation set to $R_n = N/k$, where the first $k$ layers generate speculative candidate tokens. In the context of Ahead-of-Time Expert Loading for Mixture-of-Experts (MoE) models (see \cref{method_aot_expert_loading}), setting the rate of residual transformation to $R_n = 2$ typically offers a good trade-off between the accuracy of expert speculation and AoT pre-loading of experts. 

% Thus, we set $J = 1$ and $E_1 = 16$.


~\subsection{FLOPs Optimization} \label{sec:flops_optimization}

Naively implemented, M2R2 incurs higher FLOP overhead compared to traditional speculative decoding and early exiting approaches such as ~\cite{medusa, schuster2022confident, Tang2024}. However, modern accelerators demonstrate compute bandwidth that exceeds memory access bandwidth by an order of magnitude or more~\cite{databricksLLMInference2023, jouppi2021ten}, meaning increased FLOPs do not necessarily translate to increased decoding latency. Nevertheless, to ensure fair comparison and efficiency in compute bound scenarios, we introduce targeted optimizations.

~\textbf{Attention FLOPs Optimization} For medium-to-long context lengths, attention computation dominates FLOPs in the self-attention layer, surpassing the contribution from MLP layers. Specifically, matrix multiplications involving queries, cached keys, and cached values scale with $l_{kv} * l_{q}$ where $l_{kv}$ denotes previous context length and $l_q$ denotes current query length. Since M2R2 pairs accelerated residuals with slow residuals, a naive implementation results in twice the FLOPs consumption compared to a standard attention layer. To address this, we limit the attention of accelerated residual stream to selectively attend to the top-k most relevant tokens, identified by the slow residual stream based on top attention coefficients\footnote{We set to k = 64 and attend to top 64 tokens as identified by the slow residual stream.}. This is possible since slow and accelerated residual streams are processed in same forward pass and accelerated streams have access to attention coefficients of slow stream. Note that, the faster residual stream still retains the flexibility to assign distinct attention coefficients to these tokens. Furthermore, we design the faster residual stream to employ only 8 attention heads, compared to the 32 heads used in the slow residual stream of the Phi-3 model, reducing query, key, value, and output projection FLOPs by a factor of 1/4. ~\cref{fig:m2r2_num_heads_ablation} indicates effect of using a slicker stream on alignment. As depicted, using $\hat{n}_h = 8$ offers a good trade-off between alignment and FLOPs overhead. 

~\textbf{MLP FLOPs Optimization} The accelerator adapters operating on the accelerated residual stream are intentionally designed with lower rank than their counterparts in the base model. This reduces FLOP overhead by a factor proportional to $hiddenSize / rank$. Additionally, since the faster residual stream uses only 8 attention heads (compared to 32 in the slow residual stream of Phi-3), the subsequent MLP layers process a smaller set of activations, further reducing FLOPs by another factor of 1/4.

These optimizations significantly reduce the FLOP overhead per speculative draft generation, as illustrated in ~\cref{fig:flops_optmization}. Notably, while traditional early-exiting speculative approaches such as DEED require propagating the full slow residual state through the initial layers, incurring substantial computational costs, M2R2 achieves efficient token generation via slimmer, low-rank faster residual streams. In contrast, Medusa introduces considerable FLOP overhead due to per-head computations scaling with $d^2+dv$\footnote{Here $d$ denotes hidden state dimension while $v$ denotes vocab size.}, whereas M2R2 employs low-rank layers for both MLP and language modeling heads, maintaining computational efficiency. All experiments involving the M2R2 approach, as detailed in ~\cref{sec:experiments}, are conducted using these FLOPs optimizations.









% \[
% O_t^{E_j} = 
% \begin{cases} 
% 1, & \text{if } L(\hat{y}_t^{E_j}) = y_t^{E_j} \\
% 0, & \text{otherwise}
% \end{cases}
% \]




%add distillation
% We train accelerator adapters described in \cref{m2r2_method} to accelerate residual streams on next token prediction all in parallel since there are no gradient conflict issues as described in \cref{sec:grad_conflict}.

% \begin{align} \label{eq:mr_loss}
% L_{mr} =  & -\sum_{j = 1}^{E_n} (\sum_{t=1}^{T}\log p_{\theta} (\hat{y}_t^{E_j} | \hat{y}_{<t}, x)) \nonumber
% \end{align}

% where $\hat{y_t^{E_j}}$ denotes predicted logits obtained from accelerated residual stream at gate $E_j$ and time-step $t$ while $y_t^{E_j}$ denotes corresponding truth tokens. 

% Upon training of adapters responsible for accelerating residual streams, we train query, key, value parameters responsible for vertical latent attention of all gates in parallel as

% \begin{equation} \label{eq:arla_loss}
%     L_{arla} = -\frac{1}{N} (\sum_{t=1}^{T}(1\sum_{j=1}^{E_n} \left[ O_t^{E_j} \log(\hat{O}_t^{E_j}) + (1 - o_t^{E_j}) \log(1 - \hat{o_t}_{E_j}) \right]))
% \end{equation}

% where $\hat{O_t^{E_j}}$ denotes binary predicted logits obtained from vertical latent attention router described in \cref{sec:arla} at gate $E_j$ and timestep $t$ while $O_t^{E_j}$ denotes corresponding truth label. Truth labels for router are obtained by computing whether logit head application on $\hat{y}_t^j$ results in true next token prediction. Formally speaking, 

% $O_t^{E_j} = 1 if L(\hat{y_t^{E_j}}) == y_t^{E_j} , 0 otherwise$. 

% Parameters responsible for vertical latent attention are also trained in parallel as well. 

%todo: training slow and fast residuals together and distillation can be two training mdoes. 
%Distillation can be an ablation. 




% Although transformer decoding is memory bound on most mainstream accelerators, there could be scenarios where flop savings are crucial. For instance, on on-device settings power consumption is directly correlated with flops per decoding step and reducing flops does help with overall energy consumption. Vanilla early exiting methods help with flop reduction but suffer from mismatch between training and inference due to early exited tokens. If token at decoding step $t$, $T_t$ exited at layer $E_i$, while token $T_{t+k}$ exits at layer $E_j$ such that $E_i < E_j$, hidden state $H_{t+k}l$ does not have corresponding hidden state $H_tl$ to attend to where $E_i < l <= E_j$. One solution that's often used in literature is to rely on last hidden state available, $H_t{E_j}$, however it tends to be sub-optimal and does affect generation quality \cite{ref}.  To alleviate this mismatch while reducing flops, we train router such that attention mask between token $T_{t+k}$ and token $T_{<t+k}$ is given by: 

% \begin{equation}
%     a_{T_{{t+k}{T_{<t+k}}} = 1 if  E_{T_{<t+k}} >= E{T_{t+k}}
%     else 0
% \end{equation}

% This attention mask enables router to account for exited tokens and get trained accordingly. Since attention mechanism during decoding remains exactly same as that during training, impact on generation quality tends to be minimal as noted in \cref{fig:gen_auality_with_and_without_recompute_attention_show_flops}.  Although MoD does not suffer from training and inference mismatch, we observe that it suffers from discountinuity between pre-training and super-vised fine-tuning resulting in sub-optimal perplexity. On the other hand, our method doesn't not require pre-training , doesn't suffer from discountinuity, and achieves much better perplexity in super-vised fine-tuning and instruction tuning setups as shown in \cref{fig:Mod_vs_m2r2_loss_curves}.






% Our techniques are directly applicable in such scenarios.    




%expert loading with cuda streams in experiments
\section{Theoretical Analysis}

\subsection{Preliminaries}

The Hutchinson trace estimator provides an efficient means to estimate the trace of a matrix, with the following properties:

\begin{lemma}[\cite{Hutchinson89}]
\begin{equation}
    \mathbb{E}[H_m(\mathbf{A})] = \mathrm{Tr}(\mathbf{A}), \; \mathrm{Var}[H_m(\mathbf{A})] \le \frac{2}{m} \mathrm{Tr}^2(\mathbf{A}), \label{eqn:VanHut}
\end{equation}
where $H_m(\mathbf{A})$ is the Hutchinson estimator with $m$ random vectors, $\mathbb{E}[\cdot]$ denotes expectation, and $\mathrm{Var}[\cdot]$ denotes variance.
\end{lemma}

To reduce the variance further, especially for matrices with large dominant eigenvalues, we can leverage a low-rank QR approximation. The Hutch++ estimator improves upon the original Hutchinson method by significantly reducing the variance:

\begin{lemma}[\cite{hutch_pp}] 
Suppose $\mathbf{A}$ is a \emph{symmetric positive semidefinite (PSD)} matrix. Then, the Hutch++ estimator satisfies:
\begin{equation}
    \mathbb{E}[H_m^{++}(\mathbf{A})] = \mathrm{Tr}(\mathbf{A}),\mathrm{Var}[H_m^{++}(\mathbf{A})] \le \frac{18}{m(m-3)} \mathrm{Tr}^2(\mathbf{A}),
\end{equation}
where $H_m^{++}(\mathbf{A})$ is the Hutch++ estimator.
\end{lemma}

It is worth noting that the PSD assumption can be generalized to estimates involving the nuclear norm of non-PSD matrices. However, in this paper, we focus on matrices where large eigenvalues are the primary concern, and thus the PSD assumption always holds.

We define the \emph{relative error} of an estimator $T(\mathbf{A})$ as:
\begin{equation*}
    \varepsilon(T) := \frac{|T(\mathbf{A}) - \mathrm{Tr}(\mathbf{A})|}{|\mathrm{Tr}(\mathbf{A})|}.
\end{equation*}

A key advantage of Hutch++ is its enhanced variance reduction, which leads to significant computational savings compared to the original Hutchinson estimator. Specifically:

\begin{proposition}[\cite{hutch_pp}]
    For a given error threshold $\varepsilon > 0$ and confidence level $\delta > 0$, the following holds with probability at least $1 - \delta$:
    \begin{itemize}
        \item The error of the Hutchinson estimator satisfies $\varepsilon(H_m) < \varepsilon$ when $m = \mathcal{O}\Big(\frac{\log(1/\delta)}{\varepsilon^2}\Big)$.
        \item The error of the Hutch++ estimator satisfies $\varepsilon(H_m^{++}) < \varepsilon$ when $m = \mathcal{O}\Big(\sqrt{\frac{\log(1/\delta)}{\varepsilon^2}} + \log(1/\delta)\Big)$.
    \end{itemize}
\end{proposition}

This result highlights that Hutch++ achieves a quadratic reduction in the required number of samples $m$ for a given accuracy, compared to the original Hutchinson estimator.

\subsection{Error Analysis for Acceleration}
Naïvely applying Hutch++ in generative modeling can incur significant computational costs due to the need for QR decompositions at every iteration. To mitigate this, we propose updating the QR decomposition every $L_s$ iterations. While this approach means the estimated eigenvalues may not always be fully up to date, it strikes a balance between computational efficiency and accuracy.
We now analyze the errors introduced by perturbations in the matrix
$\mathbf{A}$. Let $\widetilde{\mathbf{A}}$ be a perturbed version of
$\mathbf{A}$, by the smoothness of the dynamics, the difference of their traces is proportional to a small
time increment $\eta$ and the frequency of steps in QR updates $L_s$, i.e.,$\mathrm{Tr}(\widetilde{\mathbf{A}}) =\mathrm{Tr}(\mathbf{A}) +
\mathcal{O}((L_s-1)\eta)$. Consider the matrix $Q = \mathrm{QR}(\mathbf{A}S)$,
where $S$ is a random sketching matrix. The \emph{approximate Hutch++ estimator}
for this perturbed matrix is given by:
\begin{equation*}
\scalebox{0.9}{$
    \widetilde{H}_m^{++}(\widetilde{\mathbf{A}}) := \mathrm{Tr}(Q^\top \widetilde{\mathbf{A}} Q) + H_{\frac{m}{3}}\left( (I - QQ^\top) \widetilde{\mathbf{A}} (I - QQ^\top) \right).$}
\end{equation*}

This approximation results from reducing the frequency of QR updates, which is central to our acceleration method. For example, using the frozen QR decomposition (Equation \ref{eqn:acceleration}) over the time subinterval $[t_i, t_{i+L_s}]$, $\frac{\partial z_t}{\partial \mathbf{x}}$ behaves as a perturbation of $\frac{\partial z_t}{\partial \mathbf{x}\left(\lfloor\frac{L}{L_s} t_i\rfloor \cdot \frac{L_s}{L}\right)}$.

We justify the effectiveness of this estimator by proving the following expectation and variance bounds:
\begin{proposition}\label{prop:approximateHut}
\begin{align}
\scalebox{0.9}{$
    \mathbb{E}[\widetilde{H}_m^{++}(\widetilde{\mathbf{A}})] = \mathrm{Tr}(\widetilde{\mathbf{A}})$},\hspace{3.35cm}\tag{14}\label{eqn:approximateExp}\\
\scalebox{0.9}{$
    \mathrm{Var}(\widetilde{H}_m^{++}(\widetilde{\mathbf{A}})) \le \frac{36}{m(m-3)} \mathrm{Tr}^2(\mathbf{A}) +  \frac{1}{m}\mathcal{O}((L_s-1)^2\eta^2).$}\tag{15}\label{eqn:approximateVar}
\end{align}
\end{proposition}
In Equation \ref{eqn:approximateExp}, we observe that the approximate Hutch++ estimator provides the exact trace of $\tilde{\mathbf{A}}$, demonstrating its robustness to the QR decomposition used in the acceleration process. Additionally, in Equation \ref{eqn:approximateVar}, the approximate Hutch++ estimator achieves an approximately quadratic variance reduction similar as the standard Hutch++ estimator.  
Conducting QR decomposition only every \(L_s\) iterations reduces computations while maintaining a variance reduction comparable to the vanilla version. Regarding the specific guidance on the selection of \(L_s\), we refer to interested readers to the complexity analysis section 1.5 in the supplementary file.


\subsection{Error Propagation in Divergence-Based Likelihood Training}\label{sec: error_propagation}

The accumulation of error during the training of divergence-based likelihoods can be analyzed as follows. In the case of Neural ODEs \cite{neural_ode}, by applying the frozen QR decomposition (Equation \ref{eqn:acceleration}) over the time subinterval \( [t_i, t_{i+L_s}] \) for Equation \ref{eq: ode_likelihood}, we estimate the trace term, assuming that the approximate Hutch++ operator is adapted to \( \frac{\partial f}{\partial \mathbf{z}} \) within this interval. Let \( \log \tilde{p}_\theta(t) \) denote the resulting log-likelihood under this approximation. Using Proposition \ref{prop:approximateHut}, we can then establish the following result:

\setcounter{equation}{15}
\begin{proposition}
Let \( M = \max\{|\mathrm{Tr}(\frac{\partial f}{\partial \mathbf{z}})| : t \in [0, T]\} \). Then for \( t \in [0, T] \), we have:
\begin{equation}
\scalebox{0.95}{$
\begin{aligned}
  \mathbb{E}\left[ \log \tilde{p}_\theta(t) \right] &= \log p_\theta(\mathbf{z}(t)),\\
  \mathrm{Var}[\log \tilde{p}_\theta(t)] &\leq T^2 \left[\frac{36M^2}{m(m-3)} + \frac{1}{m}\mathcal{O}((L_s-1)^2\eta^2)\right].
\end{aligned}\label{eqn:loglikelihoodest}$}
\end{equation}
\end{proposition}

Since the divergence-based likelihood of the Schrödinger bridge \citep{forward_backward_SDE, mSB} differs from neural ODEs only by an additional deterministic inner product (see Equation \ref{eq: fb-sde-train-f}) vs. the ISM loss in \cite{VSDM}), the variance reduction achieved by Hutch++ estimators in Equation \ref{eqn:loglikelihoodest} naturally extends to Schrödinger bridge (SB)-based diffusion models \citep{forward_backward_SDE, mSB, VSDM} as well.


\section{Experiments}
In this section, we present the experimental results on widely used FL benchmark datasets \citep{caldas2018leaf} including real-world datasets \citep{hsu2020federated}, comparing the performance of \shortname with other baselines from the literature, including standard FL algorithms and clustering methods.  A detailed description of the implementation settings, datasets and models used for the evaluation are reported in Appendix \ref{app_details}.
\begin{figure}[t]
    \centering
    \includegraphics[width=1.\linewidth]{figures/accuracy_stars.pdf}\vspace{-1.5em}
    \caption{\small{Balanced accuracy on Cifar100 for \shortname (blue curve) with \texttt{FedAvg} aggregation compared to the clustered FL baselines. \shortname detects two splits demonstrating significant improvements in accuracy when clustering is performed, leading also to a faster and more stable convergence than baseline algorithms.}}
    \label{fig:accuracy_jumpcifar100}
    \vspace{-1.3em}
\end{figure}
In Section \ref{exp}, we evaluate our method, \shortname, against various clustering algorithms, including \texttt{CFL} \citep{sattler2020clustered}, \texttt{FeSEM} \citep{long2023multi}, and \texttt{IFCA} \citep{ghosh2020efficient}, and standard FL aggregations \texttt{FedAvg} \citep{mcmahan2017communication}, \texttt{FedAvgM} \citep{asad2020fedopt}, FairAvg \citep{michieli2021all} and \texttt{FedProx} \citep{li2020federated}, showing also that how our approach is orthogonal to conventional FL aggregation methods.

In Section \ref{sec:large_scale}, we underscore that \shortname has the capability to surpass FL methods in real-world and large-scale scenarios \citep{hsu2020federated}.

Finally, in Section \ref{sect:ablation}, we propose analyses on class and domain imbalance, showing that our algorithm successfully detects clients belonging to separate distributions. Further experiments are presented in Appendix \ref{app:other}.

Each client has its own local train and test sets. Algorithm performance is evaluated by averaging the accuracy achieved on the local test sets across the federation, enabling a comparison between FL aggregation and clustered FL approaches (refer to Appendix \ref{app_details} for additional insights). When assessing clustering baselines, we also use the Wasserstein's Adjusted Silhouette Score (WAS) and Wasserstein's Adjusted Davies-Bouldin Score (WADB) to quantify the distributional cohesion among clients, an evaluation performed \textit{a posteriori}. For detection tasks in visual domains (Section \ref{sect:ablation}), we compute the Rand Index \citep{rand1971objective}, a clustering metric that compares the obtained clustering with a ground truth labeling. Further details on the chosen metrics are provided in Appendices \ref{app:metrics_choice} and \ref{app:clustering}.
\subsection{\shortname in heterogeneous settings}\label{exp}
In this section, we analyze the effectiveness of \shortname in mitigating the impact of data heterogeneity compared to standard aggregation methods and other clustered FL algorithms. We conduct experiments on Cifar100 \citep{krizhevsky2009learning} with 100 clients and Femnist \citep{lecun1998mnist} with 400 clients, controlling heterogeneity through a Dirichlet parameter $\alpha$, set to 0.5 for Cifar100 and 0.01 for Femnist, reflecting a realistic class imbalance across clients. Implementation details are provided in Appendix \ref{app_details}.
\begin{table}[t]
    \centering
    \small
        \centering
        \caption{\small{FL baselines in heterogeneous scenarios. Clustering baselines use FedAvg as aggregation mechanism. We emphasize the fact that \shortname and \texttt{CFL} automatically detect the number of clusters, unlike \texttt{IFCA} and \texttt{FeSEM} which require tuning the number of clusters. A higher WAS , denoted by $\uparrow$, and a lower WADB, denoted by $\downarrow$ indicate better clustering outcomes} }
        \label{tab:clustering}
        \begin{adjustbox}{width=\linewidth}
       
        \begin{tabular}{llcccccc}
            \toprule
            & & \makecell{ \textbf{FL} \textbf{method}}& \textbf{C}& \makecell{ \textbf{Automatic} \\ \textbf{Cluster} \\ \textbf{Selection}} & \textbf{Acc} & \textbf{WAS} $\uparrow$ & \textbf{WADB} $\downarrow$ \\
            \midrule
            \multirow{7}{*}{\rotatebox[origin=c]{90}{\textbf{Cifar100} \hspace{.75em}}} & \multirow{4}{*}{\rotatebox[origin=c]{90}{\makecell{Clustered\\ FL}}} &  \texttt{IFCA} &  5 & \ding{55}& 47.5 \scriptsize{$\pm$ 3.5} & -0.8 \scriptsize{$\pm$ 0.2} &  5.2 \scriptsize{$\pm$ 5.1} \\
            & & \texttt{FeSem} & 5 & \ding{55}& 53.4 \scriptsize{$\pm$ 1.8} & -0.3 \scriptsize{$\pm$ 0.1} & 38.4 \scriptsize{$\pm$ 13.0}\\
            & & \texttt{CFL} & 1 &\ding{51}& 41.6 \scriptsize{$\pm$ 1.3} & / & / \\
            & & \shortname & 4 & \ding{51}&\textbf{53.4 \scriptsize{$\pm$ 0.4}} & \textbf{0.1 \scriptsize{$\pm$ 0.0}} & \textbf{2.4 \scriptsize{$\pm$ 0.4}} \\
            \cmidrule{2-8}
            
            & \multirow{3}{*}{\rotatebox[origin=c]{90}{\makecell{Classic\\ FL}}} &  \texttt{FedAvg} &  1 & / &   41.6\scriptsize{$\pm$ 1.3} &  / &  / \\
            & & \texttt{FedAvgM} & 1 & /& 41.5\scriptsize{$\pm$ 0.5}& / & /  \\
            & & \texttt{FedProx} & 1 & /& 41.8\scriptsize{$\pm$ 1.0} & / & / \\
            \midrule
       
            
            \multirow{7}{*}{\rotatebox[origin=c]{90}{\textbf{Femnist} \hspace{.75em}}} & \multirow{4}{*}{\rotatebox[origin=c]{90}{\makecell{Clustered\\ FL}}} &  \texttt{IFCA} &  5 & \ding{55} &  {76.7 \scriptsize{$\pm$ 0.6}} &  \textbf{0.3 \scriptsize{$\pm$ 0.1}} &  \textbf{0.5 \scriptsize{$\pm$ 0.1}} \\
            & & \texttt{FeSem} & 2 & \ding{55}& 75.6 \scriptsize{$\pm$ 0.2} & 0.0 \scriptsize{$\pm$ 0.0} & 25.6 \scriptsize{$\pm$ 7.8}  \\
            & & \texttt{CFL} & 1 & \ding{51}& 76.0 \scriptsize{$\pm$ 0.1} & / & / \\
            & & \shortname & 4 & \ding{51}&76.1 \scriptsize{$\pm$ 0.1} & -0.2 \scriptsize{$\pm$ 0.1} & 18.0 \scriptsize{$\pm$ 6.2}\\
            \cmidrule{2-8}
            & \multirow{3}{*}{\rotatebox[origin=c]{90}{\makecell{Classic\\ FL}}} &  \texttt{FedAvg} &  1 & / &  76.6\scriptsize{$\pm$ 0.1} &  / &  / \\
            & & \texttt{FedAvgM} & 1 & /&  \textbf{83.3}\scriptsize{$\pm$ \textbf{0.3}}& / & /  \\
            & & \texttt{FedProx} & 1 & /& 75.9\scriptsize{$\pm$ 0.2} & / & / \\
            \bottomrule
        \end{tabular}
        \end{adjustbox}
       
      
\end{table}


We compare \shortname against clustered FL baselines (\texttt{IFCA}, \texttt{FeSEM}, \texttt{CFL}) using \texttt{FedAvg} aggregation, as well as standard FL algorithms (\texttt{FedAvg}, \texttt{FedAvgM}, \texttt{FedProx}). For algorithms requiring a predefined number of clusters (\texttt{IFCA}, \texttt{FeSEM}), we report the best result among 2, 3, 4, and 5 clusters, with full tuning details in Appendix \ref{app:tuning}. While \texttt{IFCA} achieves competitive results, its high communication overhead—requiring each client to evaluate models from every cluster in each round—makes it impractical for cross-device FL, serving as an upper bound in our study. \texttt{FeSEM} is more efficient than \texttt{IFCA} but lacks adaptability due to its fixed cluster count. Meanwhile, \texttt{CFL} requires extensive hyperparameter tuning and often produces overly fine-grained clusters or fails to form clusters altogether. In contrast, \shortname requires only one hyperparameter and provides a more practical clustering strategy for cross-device FL.

Table \ref{tab:clustering} presents a comparative analysis of these algorithms in terms of balanced accuracy, WAS, and WADB, using \texttt{FedAvg} as the aggregation method. Higher WAS values indicate better clustering, while lower WADB values suggest better cohesion. On Femnist, clustering-based methods perform worse than standard FL aggregation, but as we move to the more complex and realistic Cifar100 scenario, it becomes evident that clustered FL is necessary to address heterogeneity. In this case, \shortname achieves the best performance in both classification accuracy and clustering quality, with the latter directly influencing the former. The need for clustering grows with increasing heterogeneity, as seen in Table \ref{tab:clustering}: standard FL approaches struggle when trained on a single heterogeneous cluster, whereas clustered FL effectively mitigates the heterogeneity effect. This is particularly relevant for Cifar100, which has a larger number of classes and three-channel images, whereas Femnist consists of grayscale images from only 47 classes.


In Table \ref{tab:clustering}, we present a comparative analysis of these algorithms with respect to balanced accuracy, WAS, and WADB, employing \texttt{FedAvg} as the aggregation strategy. Recall that higher the value of  WAS the better the clustering outcome, as, for WADB, a lower value suggests a better cohesion between clusters. Further details on the metrics used are provided in Appendix \ref{app:metrics_choice}.

Notably, both \shortname and \texttt{CFL} automatically determine the optimal number of clusters based on data heterogeneity, offering a more scalable solution for large-scale cross-device FL. In contrast to \texttt{CFL}, \shortname consistently produced a reasonable number of clusters, even when using the optimal hyperparameters for \texttt{CFL}, which resulted in no splits, thereby achieving performance equivalent to FedAvg. 

 We observe in Figure \ref{fig:accuracy_jumpcifar100} that \shortname exhibits a significant improvement in accuracy on Cifar100 precisely at the rounds where clustering occurs. 

As detailed in Table \ref{tab_app:fl-algs} in Appendix \ref{app:other}, \shortname is orthogonal to any FL aggregation algorithm, \ie any FL method can be easily embedded in \shortname. Our method consistently boosted the performance of FL algorithms for the more heterogeneous settings of Cifar100 and Femnist.
\begin{table*}[t]
\small
\centering
\renewcommand{\arraystretch}{1.5} 
\begin{adjustbox}{width=.7\linewidth}

\begin{tabular}{@{}l|cccccccc@{}}
\toprule
\textbf{Dataset}      & {\shortname} & {\texttt{CFL}}       & \texttt{IFCA} & {FedAvg} & {FedAvgM} & {FedProx} & {FairAvg} \\ \midrule
Google Landmarks      & \textbf{57.4  \scriptsize{$\pm$0.3}} &   40.5  \scriptsize{$\pm$0.2} &  49.4   \scriptsize{$\pm$0.3}    &     40.5  \scriptsize{$\pm$0.2}&  36.4  \scriptsize{$\pm$1.3}  &  40.2  \scriptsize{$\pm$0.6}               &     39.0 \scriptsize{$\pm$0.3}              \\ 
\hline
iNaturalist           & \textbf{47.8  \scriptsize{$\pm$0.2}}   & 45.3  \scriptsize{$\pm$0.1}  &  45.8  \scriptsize{$\pm$0.6}   &     45.3 \scriptsize{$\pm$0.1}            &   37.7  \scriptsize{$\pm$1.4}               &      44.9  \scriptsize{$\pm$0.2}  &      45.1 \scriptsize{$\pm$0.2}            \\ \bottomrule
\end{tabular}
    
\end{adjustbox}
\caption{\small{
Comparison of test accuracy on large scale FL datasets Google Landmarks and iNaturalist, between \shortname and FL baselines -- all clustered FL algorithms use FedAvg aggregation. \shortname outperforms both clustered and standard FL methods detecting 5 and 4 clusters on Landmarks and iNaturalist, respectively.}}\label{tab:largescale}\vspace{-1.5em}
\end{table*}


\subsection{\shortname in Large Scale and Real World Scenarios}\label{sec:large_scale}

We evaluate \shortname on two large-scale, real-world datasets: Google Landmarks \citep{weyand2020google} and iNaturalist \citep{van2018inaturalist}, respectively considering the partitions Landmarks-Users-160K, and iNaturalist-Users-120K, proposed in \citep{hsu2020federated}. Both datasets exhibit high data heterogeneity and involve a large number of clients - approximately 800 for Landmarks and 2,700 for iNaturalist. To simulate a realistic cross-device scenario, we set 10 participating clients per round.
For this experiment, we compare \shortname against clustered FL baselines (\texttt{IFCA}, \texttt{CFL}) and standard FL aggregation methods (\texttt{FedAvg}, \texttt{FedAvgM}, \texttt{FedProx}, \texttt{FairAvg}). The number of clusters for \texttt{IFCA} is tuned between 2 and 5. Due to its high computational cost in large-scale settings, \texttt{FeSEM} was not included in this analysis.
Table \ref{tab:largescale} reports the results: \shortname with \texttt{FedAvg} aggregation achieves 57.4\% accuracy on Landmarks, significantly outperforming all standard FL methods. In this scenario, \shortname detects 5 clusters with the best hyperparameter setting ($\beta = 0.5$), while \texttt{IFCA} identifies 3 clusters.
Similarly, on iNaturalist, \shortname consistently surpasses FL baselines, reaching an average accuracy of 47.8\% with $\beta = 0.5$ (automatically detecting a partition with 4 clusters). Results in Table \ref{tab:largescale} remark that, when dealing with realistic complex decentralized scenarios, standard aggregation methods are not able to mitigate the effects of heterogeneity across the federation, while, on the other hand, clustered FL provides a more efficient solution.
\vspace{-1em}
\subsection{Clustering analysis of \shortname}\label{sect:ablation}
In this section, we investigate the underlying mechanisms behind \shortname’s clustering in heterogeneous scenarios on Cifar100. Further experiments on Cifar10 are presented in Appendix \ref{app:cifar10}.
\vspace{-1.3em}
\paragraph{\shortname detects different client class distributions}
We explore how the algorithm identifies and groups clients based on the non-IID nature of their data distributions, represented by the Dirichlet concentration parameter $\alpha$. For the Cifar100 dataset, we apply a similar splitting approach, obtaining the following partitions: (1) 90 clients with $\alpha = 0$ and 10 clients with $\alpha = 1000$; (2) 90 clients with $\alpha = 0.5$ and 10 clients with $\alpha = 1000$; and (3) 40 clients with $\alpha = 1000$, 30 clients with $\alpha = 0.05$, and 30 clients with $\alpha = 0$. We evaluate the outcome of this clustering experiment by means of WAS and WADB. Results in Table \ref{tab:ablation1_heter} show that \shortname detects clusters groups clients according to the level of heterogeneity of the group.
\begin{table}[h]
    
    \caption{\small{Clustering with three different splits on Cifar100 datasets. \shortname has superior clustering quality across different splits (homogeneous $\alpha = 1000$, heterogeneous $\alpha = 0.05$, extremely heterogeneous $\alpha = 0$}. )}
    \centering
    \small
    \begin{adjustbox}{width=\linewidth}
        \label{tab:ablation1_heter}
     
        \begin{tabular}{lccccc}
            \toprule
            \textbf{Dataset} & \textbf{(Hom, Het, X Het)} & \makecell{\textbf{Clustering} \\ \textbf{method}} & \textbf{C} & \textbf{WAS}$\uparrow$ & \textbf{WADB} $\downarrow$ \\
            \cmidrule(lr){1-6}
        
            \multirow{9}{*}{Cifar100} 
            & \multirow{3}{*}{(10, 0, 90)} & \texttt{IFCA} & 5 & -0.9 \scriptsize{$\pm$ 0.0} & 1.8 \scriptsize{$\pm$ 0.0} \\
            & & \texttt{FeSem} & 5 & -0.8 \scriptsize{$\pm$ 0.2} & 2.6 \scriptsize{$\pm$ 0.6} \\
            & & \shortname & 5 & \textbf{0.1 \scriptsize{$\pm$ 0.1}} & \textbf{0.2 \scriptsize{$\pm$ 0.2}} \\
            \cmidrule(lr){2-6}
            & \multirow{3}{*}{(10, 90, 0)} & \texttt{IFCA} & 5 & -0.0 \scriptsize{$\pm$ 0.0} & \textbf{5.6 \scriptsize{$\pm$ 1.5}} \\
            & & \texttt{FeSem} & 5 & 0.2 \scriptsize{$\pm$ 0.1} & 12.0 \scriptsize{$\pm$ 2.0} \\
            & & \shortname & 5 & \textbf{0.4 \scriptsize{$\pm$ 0.1}} & 6.4 \scriptsize{$\pm$ 2.0} \\
            \cmidrule(lr){2-6}
            
            & \multirow{3}{*}{(40, 30, 30)} & \texttt{IFCA} & 5 & -0.2 \scriptsize{$\pm$ 0.0} & 1.0 \scriptsize{$\pm$ 0.0}\\
            & & \texttt{FeSem} & 5 & -0.2 \scriptsize{$\pm$ 0.0} & 33.2 \scriptsize{$\pm$ 0.0} \\
            & & \shortname & 3 & \textbf{0.4 \scriptsize{$\pm$ 0.2}} & \textbf{0.9 \scriptsize{$\pm$ 0.1}} \\
            \bottomrule
        \end{tabular}
    \end{adjustbox}
    
\end{table}
\vspace{-1.5em}
\paragraph{\shortname detects different visual client domains.}
Here, we focus on scenarios with nearly uniform class imbalance (high $\alpha$ values) but with different visual domains to investigate how \shortname forms clusters in such settings. We incorporated various artificial domains (non-perturbed, noisy, and blurred images) Cifar100 dataset under homogeneous conditions ($\alpha=100.00$). Our results demonstrate that \shortname effectively clustered clients according to these distinct domains. Table \ref{tab:dom_abl} presents the Rand-Index scores, which assess clustering quality based on known domain labels. The high Rand-Index scores, often approaching the upper bound of 1, indicate that \shortname successfully separated clients into distinct clusters corresponding to their respective domains. %
This anaylsis suggests that \shortname may be applicable for detecting malicious clients in FL, pinpointing a potential direction for future research.
\begin{table}[t]
    
    \caption{\small Clustering performance of \shortname is assessed on federations with clients from varied domains using clean, noisy, and blurred (Clean, Noise, Blur) images from Cifar100 datasets. It utilizes the Rand Index score \citep{rand1971objective}, where a value close to 1 represents a perfect match between clustering and labels. Consistently \shortname accurately distinguishes all visual domains. The ground truth number of clusters is respectively 2, 2, and 3.}
    \label{tab:dom_abl}
    
    \centering
    
    \begin{adjustbox}{width=\linewidth}
    \setlength{\tabcolsep}{9pt}
        \begin{tabular}{ccccccc}
           \toprule
            \textbf{Dataset} & \textbf{(Clean, Noise, Blur)} & \makecell{\textbf{Clustering} \\ \textbf{method}} & \textbf{C} &\textbf{\makecell{Automatic\\Cluster \\Selection}}& \makecell{\textbf{Rand} $\uparrow$ \\ \textbf{(max = 1.0)}}  \\
            \cmidrule(lr){1-6}
            
            \multirow{9}{*}{Cifar100} 
            & \multirow{3}{*}{(50, 50, 0)} & \texttt{IFCA} & 1  & \ding{55}& 0.5 \scriptsize{$\pm$ 0.0}  \\
            & & \texttt{FeSem} & 2 &\ding{55}& 0.49 \scriptsize{$\pm$ 0.2}  \\
            & & \shortname & 2 & \ding{51} &\textbf{1.0 \scriptsize{$\pm$ 0.0}} \\
            \cmidrule(lr){2-6}
            & \multirow{3}{*}{(50, 0, 50)} & \texttt{IFCA} & 1 & \ding{55}& 0.5 \scriptsize{$\pm$ 0.0} \\
            & & \texttt{FeSem} & 2 &\ding{55}& 0.51 \scriptsize{$\pm$ 0.1} \\
            & & \shortname & 2 & \ding{51}&\textbf{1.0 \scriptsize{$\pm$ 0.0}} \\
            \cmidrule(lr){2-6}
            
            & \multirow{3}{*}{(40, 30, 30)} & \texttt{IFCA} & 1 &\ding{55} & 0.33 \scriptsize{$\pm$ 0.0}\\
            & & \texttt{FeSem} & 3 &\ding{55} & 0.55 \scriptsize{$\pm$ 0.0}\\
            & & \shortname & 4 & \ding{51} &\textbf{0.6 \scriptsize{$\pm$ 0.0}} \\
            \bottomrule
        \end{tabular}
    \end{adjustbox}
\end{table}



\section{Conclusion Remarks}
This work proposes a RBG graph model for disease spreading via hubs. We study the joint effect of the agent density, hub density, and connection function. The existence of a critical hub density depends only on the boundedness of the support of the connection function, which relates to curbing the traveling distance of individuals. When it comes to dispersion, both the degree distribution and the percolation threshold suggest that increasing dispersion helps spread the disease. The percolation properties of RBG graphs relate to unipartite graphs with modified connection functions. 
An interesting question in this direction is if and when the properties of the RBG graphs can be well represented by unipartite graphs with some modified connection functions. Our conjecture is that for independent connections between different pairs of agents, such representation is unlikely due to the oblivion of the local dependence (present in the RBG models). 
 Another direction is to consider hybrid models where agents may get infected either through common hubs or direct interactions between agents. The former infection mechanism is more centralized than the latter. 

\section*{Impact Statement}
\shortname is an efficient clustering-based approach that improves personalization while reducing communication and computation costs, enhancing the advance in the field of Federated Learning. Providing efficiency and explainability in clustering decisions, \shortname enables more interpretable and scalable federated learning. Its efficiency makes it well-suited for IoT, decentralized AI, and sustainable AI applications, particularly in privacy-sensitive domains.
\section*{Acknowledgments}
{\textcopyright}2025 All rights reserved. The research described in this paper was carried out at the Jet Propulsion Laboratory, California Institute of Technology, under a contract with the National Aeronautics and Space Administration (80NM0018D0004).

\bibliography{reference}
\bibliographystyle{icml2025}


\newpage
\appendix
\onecolumn
\section{Theoretical Results for \shortname}\label{app:fgw}

This section provides algorithms, in pseudo-code, to describe \shortname (see Algorithms \ref{alg:fedgwcluster} and \ref{alg:fedgw_recursion}). Additionally, here we provide the proofs for the convergence results introduced in Section \ref{sec:theory}, specifically addressing the convergence (Theorems \ref{thm_main:1} and \ref{thm_main:weak_conv}) and the formal derivation on the variance bound of the Gaussian weights (Proposition \ref{prop_var_main}). \textcolor{black}{In addition, we also present a sufficient condition, under which is guaranteed that the overall sampling rate of the training algorithm does not increase and remain unchanged during the training process (Theorem \ref{thm:samplerate}).}

\begin{theorem}\label{thm:1}
Let $\{\alpha_t\}_{t = 1}^\infty$ be a sequence of positive real values, and $\{\Gamma_k^t\}_{t=1}^\infty$ the sequence of Gaussian weights. If $\{\alpha_t\}_{t = 1}^\infty \in l^2(\mathbb{N})/l^1(\mathbb{N})$, then $\Gamma_k^t$ converges in $L^2$. Furthermore, for $t\to\infty$, 
\begin{equation}
    \Gamma_k^t \longrightarrow \mu_k\,\,\, a.s.
\end{equation}
\end{theorem}
\begin{proof}
At each communication round, we compute the samples $r_i^{t,s}$ from $R_k^{t,s}$ via a Gaussian transformation of the observed loss in Eq. \ref{eq:reward}. Notice that, due to the linearity of the expectation operator, $\mathbb{E}[\Omega_k^t] = \mu_k$, that is the true, unknown, expected reward. The observed value for the random variable is given by $\omega_k^t = 1/S \sum_{s = 1}^S r_k^{t,s}$, which is sampled from a distribution centered on $\mu_k$.
Each client's weight is updated according to
\begin{equation}\label{weight_formula}
    \gamma_k^{t+1} = (1-\alpha_t)\gamma_t + \alpha_t \omega_k^t\,.
\end{equation}
Since such an estimator follows a Robbins-Monro algorithm, it is proved to converge in $L^2$. In addition, $\Gamma_k^t$ converges to the expectation $\mathbb{E}[\Omega_k^t] = \mu_k$ with probability 1, provided that $\alpha_t$ satisfies $\sum_{t\geq 1}|\alpha_t| = \infty$, and $\sum_{t\geq 1}|\alpha_t|^2 < \infty$ \citep{harold1997stochastic}.
\end{proof}

\begin{theorem}\label{thm:weak_conv}
Let $\alpha \in (0,1)$ be a fixed constant, then in the limit $t \to \infty$, the expectation of the weights converges to the individual theoretical reward $\mu_k$, for each client $k = 1,\dots, K$, \ie,
\begin{equation}
    \mathbb{E}[\Gamma_k^t]\longrightarrow \mu_k\,\,\,t\to\infty\,.
\end{equation}
\end{theorem}
\begin{proof}
Recall that $\gamma_k^{t+1} = (1-\alpha)\gamma_k^t + \alpha \omega_k^t$, where $\omega_k^t$ are samples from $\Omega_k^t$. If we substitute backward the value of $\gamma_k^t$ we can write
\begin{equation}
\gamma_k^{t+1} = (1-\alpha)^2\gamma_k^{t-1} + \alpha \omega_k^t + \alpha(1-\alpha)\omega_k^{t-1}\,.
\end{equation}
By iterating up to the initialization term $\gamma_k^0$ we get the following formulation: 
\begin{equation}\label{explicit}
    \gamma_k^{t+1} = (1-\alpha)^{t+1} \gamma_k^0+ \sum_{\tau = 0}^t \alpha (1-\alpha)^\tau \omega_k^{t-\tau}\,\,.
\end{equation}
Since $\omega_k^t$ are independent and identically distributed samples from $\Omega_k^t$, with expected value $\mu_k$, then the expectation of the weight at the $t$-th communication round would be
\begin{equation}
    \mathbb{E}[\Gamma_k^{t}] = \mathbb{E}\left[(1-\alpha)^{t}\gamma_k^0 + \sum_{\tau = 0}^t \alpha (1-\alpha)^\tau \Omega_k^{t-\tau-1}\right ]\,\,,
\end{equation}
that, due to the linearity of expectation, becomes
\begin{equation}
    \mathbb{E}[\Gamma_k^{t}] = (1-\alpha)^{t}\gamma_k^0 + \sum_{\tau = 0}^t \alpha (1-\alpha)^\tau \mu_k\,\,.
\end{equation}
If we compute the limit
\begin{equation}
    \lim_{t \to \infty}\mathbb{E}[\Gamma_k^{t}] =\lim_{t\to\infty}(1-\alpha)^{t}\gamma_k^0 + \sum_{\tau = 0}^\infty \alpha (1-\alpha)^\tau \mu_k\,\,,
\end{equation}
and since $\alpha\in(0,1)$, the first term tends to zero, and also the geometric series converges. Therefore, the expectation of the weights converges to $\mu_k$, namely
\begin{equation}
    \lim_{t \to \infty} \mathbb{E}[\Gamma_k^t] = \mu_k\,.
\end{equation}
\end{proof}

\begin{proposition}\label{prop_var}
The variance of the weights $\Gamma_k^t$ is smaller than the variance $\sigma_k^2$ of the theoretical rewards $R_k^{t,s}$.
\end{proposition} 
\begin{proof}
From Eq.\ref{explicit}, we can show that $\mathbb{V}ar(\Gamma_k^t)$ converges to a value that depends on $\alpha$ and the number of local training iterations $S$. Indeed
\begin{equation}
\begin{split}
\mathbb{V}ar(\Gamma_k^t) &= \mathbb{V}ar\left( (1-\alpha)^{t} \gamma_k^0 + \sum_{\tau = 0}^t \alpha (1-\alpha)^\tau \Omega_k^{t-\tau-1}\right) \\
&= \sum_{\tau = 0}^t \alpha^2 (1-\alpha)^{2\tau} \mathbb{V}ar(\Omega_k^t) = \dfrac{1}{S}\sum_{\tau = 0}^t \alpha^2 (1-\alpha)^{2\tau}\sigma_k^2
\end{split}
\end{equation}
since $\Omega_k^t = 1/S \sum_{s = 1}^S R_k^{t,s}$\,.

If we compute the limit, that exists finite due to the hypothesis $\alpha \in (0,1)$, we get
\begin{equation}
\lim_{t \to \infty}\mathbb{V}ar(\Gamma_k^t) = \dfrac{\alpha^2\sigma_k^2}{S}\sum_{\tau = 0}^\infty (1-\alpha)^{2\tau} = \dfrac{\alpha}{2-\alpha} \dfrac{\sigma_k^2}{S} <\dfrac{\sigma_k^2}{S}<\sigma_k^2\,\,.
\end{equation}
\end{proof}
We further demonstrate that the interaction matrix $P^t$ identified by \shortname is entry-wise bounded from above, as established in the following proposition.
\begin{proposition}\label{prop:bounded_matrix}
The entries of the interaction matrix $P^t$ are bounded from above, namely for any $t \geq 0$ there exists a positive finite constant $C_t > 0$ such that
\begin{equation}
    P_{kj}^t \leq C_t\,\,.
\end{equation}
And furthermore
\begin{equation}
    \lim_{t \to \infty} C_t = 1\,\,.
\end{equation}
\end{proposition}
\begin{proof}
Without loss of generality we assume that every client of the federation is sampled, and we assume that $\alpha_t = \alpha \in (0,1)$ for any $t \geq 0$. We recall, from Eq.\ref{inter_matrix}, that for any couple of clients $k,j \in \mathcal{P}_t$ the entries of the interaction matrix are updated according to 
\begin{equation}
    P_{kj}^{t+1} = (1-\alpha) P_{kj}^t + \alpha \omega_k^t\,.
\end{equation}
If we iterate backward until $P_{kj}^0$, we obtain the following update
\begin{equation}
     P_{kj}^{t+1} = (1-\alpha)^{t+1} P_{kj}^{0}+ \sum_{\tau = 0}^t \alpha (1-\alpha)^\tau \omega_k^{t-\tau}\,\,.
\end{equation}
We know that, by constructions, the Gaussian rewards $\omega_k^t < 1$  at any time $t$, therefore the following inequality holds
\begin{equation}
    P_{kj}^{t} = (1-\alpha)^{t} P_{kj}^{0}+ \sum_{\tau = 0}^t \alpha (1-\alpha)^\tau \omega_k^{t-\tau-1} \leq (1-\alpha)^{t} P_{kj}^{0}+ \sum_{\tau = 0}^t \alpha (1-\alpha)^\tau\,.
\end{equation}
At any round $t$ we can define the constant $C_t$, as
\begin{equation}
    C_t := (1-\alpha)^t P_{kj}^0 + \alpha \sum_{\tau = 0}^t(1-\alpha)^\tau = (1-\alpha)^t P_{kj}^0 + 1 -(1-\alpha)^{t+1} < \infty\,.
\end{equation}
Moreover, since $\alpha \in (0,1)$, by taking the limit we prove that 
\begin{equation}
    \lim_{t \to \infty} C_t = \lim_{t \to \infty} (1-\alpha)^t P_{kj}^0 + 1 -(1-\alpha)^{t+1} = 1\,.
\end{equation}
\end{proof}

\begin{theorem}{(Sufficient Condition for Sample Rate Conservation)}\label{thm:samplerate} Consider $K_{min}$ as the minimum number of clients permitted per cluster, \ie the cardinality $|\mathcal{C}_n| \geq K_{min}$ for any given cluster $n = 1,\dots, n_{cl}$, and $\rho \in (0,1]$ to represent the initial sample rate. There exists a critical threshold $n^* > 0$ such that, if $K_{min} \geq n^*$ is met, the total sample size does not increase.
\end{theorem}
\begin{proof}
Let us denote by $\rho_n$ the participation rate relative to the $n$-th cluster, \ie
\begin{equation}\label{eq:rho^n}
    \rho_n = \max \left\{\rho, \dfrac{3}{|\mathcal{C}_n|}\right\}
\end{equation}
because, in order to maintain privacy of the clients' information we need to sample at least three clients, therefore $\rho^n$ is at least $3$ over the number of clients belonging to the cluster. The total participation rate at the end of the clustering process is given by
\begin{equation}
    \rho^{\text{global}} = \sum_{n = 1}^{n_{cl}} \dfrac{K_n}{K}
\end{equation}
where $K_n$ denotes the number of clients sampled within the $n$-th cluster. If we focus on the term $K_n$, recalling Equation \ref{eq:rho^n}, we have that
\begin{equation}\label{eq:K_n}
    K_n = \rho_n |\mathcal{C}_n| = \max \left\{\rho, \dfrac{3}{|\mathcal{C}_n|}\right\}\times|\mathcal{C}_n| = \max\{\rho |\mathcal{C}_n|, 3\}\,\,.
\end{equation}
If we write Equation \ref{eq:K_n}, by the means of the positive part function, denoted by $(x)^+ = \max\{0,x\}$, we obtain that
\begin{equation}
    K_n = 3 + \max\{0, \rho |\mathcal{C}_n| - 3\} = 3 + (\rho |\mathcal{C}_n| - 3)^+\,\,.
\end{equation}
Observe that we are looking for a threshold value for which $\rho^{\text{global}} = \rho$, \ie the participation rate remains the same during the whole training process.\\
Let us observe that $K_n = \rho |\mathcal{C}_n| \iff \rho |\mathcal{C}_n| \geq 3 \iff |\mathcal{C}_n| \geq n^* = 3/\rho$. In fact, if we assume that $K_{min} \geq n^*$, then the following chain of equalities holds
\begin{equation*}
    \rho^{\text{global}} = \sum_{n = 1}^{n_{cl}} \dfrac{K_n}{K} = \dfrac{1}{K} \sum_{n = 1}^{n_cl} \rho|\mathcal{C}_{n}| = \dfrac{\rho}{K} \sum_{n = 1}^{n_{cl}}|\mathcal{C}_n| = \dfrac{\rho K}{K} = \rho
\end{equation*}
thus proving that $K_{min} \geq n^*$ is a sufficient condition for not increasing the sampling rate during the training process.
\end{proof}
\begin{algorithm}[t]
\caption{\texttt{FedGW\_Cluster}}\label{alg:fedgwcluster}
   \begin{algorithmic}[1]
     \STATE \textbf{Input:} $P, n_{max}, \mathcal{K}(\cdot,\cdot)$
     \STATE \textbf{Output:} cluster labels $y_{n_{cl}}$, and number of clusters $n_{cl}$
     \STATE Extract UPVs $v_k^j, v_j^k$ from $P$ for any $k,j$
     \STATE $W_{kj}\gets \mathcal{K}(v_k^j,v_j^k)$ for any $k,j$
     \FOR{$n = 2,\dots, n_{max}$}
     \STATE $y_{n} \gets \texttt{Spectral\_Clustering}(W,n)$
     \STATE $DB_n \gets \texttt{Davies\_Bouldin}(W,y_n)$
     \IF{$\min_n DB_n > 1$}
     \STATE $n_{cl} \gets 1$
     \ELSE 
     \STATE$n_{cl} \gets \arg \min_n DB_n$
     \ENDIF
    
    \ENDFOR
   \end{algorithmic}
\end{algorithm}

    
\begin{algorithm}[t]
  \caption{\shortname}\label{alg:fedgw_recursion}
  \begin{algorithmic}[1]
    \STATE \textbf{Input:} $K, T, S, \alpha_t, \epsilon, |\mathcal{P}_t|, \mathcal{K}$
    \STATE \textbf{Output:} $\mathcal{C}^{(1)},\dots, \mathcal{C}^{(N_{cl})}$ and $\theta_{(1)}, \dots, \theta_{(N_{cl})}$ 
    \STATE Initialize $N_{cl}^0\gets 1$
    \vspace{.1cm}
    \STATE Initialize $P^{0}_{(1)} \gets 0_{K\times K}$
    \STATE Initialize $\textrm{MSE}^{0}_{(1)} \gets 1$
    \vspace{.1cm}
    \FOR{$t = 0,\dots,T-1$}
    \STATE $\Delta N^t \gets 0$ for each iterations it counts the number of new clusters that are detected
    \vspace{.1cm}
    \FOR{$n = 1,\dots, N_{cl}^t$}
    
    \STATE Server samples $\mathcal{P}_t^{(n)} \in \mathcal{C}^{(n)}$ and sends the current cluster model $\theta_{(n)}^t$
    \STATE Each client $k \in \mathcal{P}_t^{(n)}$ locally updates $\theta_k^t$ and $l_k^t$, then sends them to the server
    \STATE $\omega_k^t \gets \texttt{Gaussian\_Rewards}(l_k^t, \mathcal{P}_t^{(n)})$, Eq. \ref{eq:reward}
    \STATE $\theta_{(n)}^{t+1}\gets \texttt{FL\_Aggregator}(\theta_k^t, \mathcal{P}_t^{(n)})$
    \STATE $P^{t+1}_{(n)}\gets\texttt{Update\_Matrix}(P^t_{(n)}, \omega_k^t, \alpha_t, \mathcal{P}_t^{(n)})$, according to Eq. \ref{inter_matrix}
    \vspace{.1cm}
    \STATE Update $\textrm{MSE}_{(n)}^{t+1}$
    \vspace{.1cm}
    \IF{$\textrm{MSE}^{t+1}_{(n)} < \epsilon$}
    \vspace{.1cm}
    \STATE Perform $\texttt{FedGW\_Cluster}(P_{(n)}^{t+1}, n_{max}, \mathcal{K})$ on $\mathcal{C}^{(n)}$, providing $n_{cl}$ sub-clusters
    \vspace{.1cm}
    \STATE Update the number of new clusters $\Delta N^t \gets \Delta N^t + n_{cl} -1 $ u
    \vspace{.1cm}
    \STATE Cluster server splits $P_{(n)}^{t+1}$ filtering rows and columns according to the new clusters
    \vspace{.1cm}
    \STATE Re-initialize MSE for new clusters to $1$
    \vspace{.1cm}
    \ENDIF
    \ENDFOR
    \STATE Update the total number of clusters$N_{cl}^{t+1} \gets N_{cl}^t + \Delta N^t$ 
    \ENDFOR
  \end{algorithmic}
\end{algorithm}
\newpage

\section{Theoretical Derivation of the Wasserstein Adjusted Score} \label{app:clustering}
\textcolor{black}{To address the lack of clustering evaluation metrics suited for FL with distributional heterogeneity and class imbalance, we introduced a theoretically grounded adjustment to standard metrics, derived from the Wasserstein distance, Kantorovich–Rubinstein metric \citep{kantorovich1942translocation}. This metric, integrated with popular scores like Silhouette and Davies-Bouldin, enables a modular framework for a posteriori evaluation, effectively comparing clustering outcomes across federated algorithms.}
In this paragraph, we show how the proposed clustering metric that accounts for class imbalance can be derived from a probabilistic interpretation of clustering. 
\begin{definition}
    Let $(M,d)$ be a metric space, and $p \in [1,\infty]$. The Wasserstein distance between two probability measures $\mathbb{P}$ and $\mathbb{Q}$ over $M$ is defined as
    \begin{equation}\label{eq:wass}
        W_p(\mathbb{P}, \mathbb{Q}) =  \inf_{\gamma \in \Gamma(\mathbb{P}, \mathbb{Q})} \mathbb{E}_{(x,y)\sim \gamma}[d(x,y)^p]^{1/p}
    \end{equation}
where $\Gamma(\mathbb{P}, \mathbb{Q})$ is  the set of all the possible couplings of $\mathbb{P}$ and $\mathbb{Q}$ (see Def. \ref{couplings}).
\end{definition}
Furthermore, we need to introduce the notion of coupling of two probability measures.
\begin{definition}\label{couplings}
Let $(M,d)$ be a metric space, and $\mathbb{P}, \mathbb{Q}$ two probability measures over $M$. A coupling $\gamma$ of $\mathbb{P}$ and $\mathbb{Q}$ is a joint probability measure on $M \times M$ such that, for any measurable subset $A \subset M$,
\begin{equation}\label{eq:coupling}
\begin{split}
    \int_A \left(\int_M \gamma(dx, dy) \mathbb{Q}(dy)\right) \mathbb{P}(dx) = \mathbb{P}(A), \\
    \int_A \left(\int_M \gamma(dx, dy) \mathbb{P}(dx)\right) \mathbb{Q}(dy) = \mathbb{Q}(A).
\end{split}
\end{equation}
\end{definition}
Let us recall that the empirical measure over $M$ of a sample of observations $\{x_1, \cdots, x_N\}$ is defined such that for any measurable  set $A \subset M$
\begin{equation}\label{eq:emp_measure}
    \mathbb{P}(A) = \dfrac{1}{N}\sum_{i = 1}^C\delta_{x_i}(A) 
\end{equation}
where $\delta_{x_i}$ is  the Dirac's measure concentrated on the data point $x_i$.\\
In particular, we aim to measure the goodness of a cluster by taking into account the distance between the empirical frequencies between two clients' class distributions and use that to properly adjust the clustering metric. For the sake of simplicity, we assume that the distance $d$ over $M$ is the $L^2$-norm. We obtain the following theoretical result to justify the rationale behind our proposed metric.
\begin{theorem}
    Let $s$ be an arbitrary clustering score. Then, the class-imbalance adjusted score $\tilde{s}$ is exactly the metric $s$ computed with the Wasserstein distance between the empirical measures over each client's class distribution.
\end{theorem}
\begin{proof}
Let us consider two clients; each one has its own sample of observations $\{x_1, \dots, x_C\}$ and $\{y_1, \dots, y_C\}$ where the $i$-th position corresponds to the frequency of training points of class $i$ for each client. We aim to compute the $p$-Wasserstein distance between the empirical measures $\mathbb{P}$ and $\mathbb{Q}$ of the two clients, in particular for any $dx, dy > 0$
\begin{equation}
\begin{split}
    \mathbb{P}(dx) &= \dfrac{1}{N} \sum_{i = 1}^N \delta_{x_i}(dx), \\
    \mathbb{Q}(dy) &= \dfrac{1}{N} \sum_{i = 1}^N \delta_{y_i}(dy)\,\,\,.
\end{split}
\end{equation}
In order to compute $W_p^p(\mathbb{P}, \mathbb{Q})$ we need to carefully investigate the set of all possible coupling measures $\Gamma(\mathbb{P}, \mathbb{Q})$. However, since either $\mathbb{P}$ and $\mathbb{Q}$ are concentrated over countable sets, it is possible to see that the only possible couplings satisfying Eq. \ref{eq:coupling} are the Dirac's measures over all the possible permutations of $x_i$ and $y_i$. In particular, by fixing the ordering of $x_i$, according to the rank statistic $x_{(i)}$, the coupling set can be written as
\begin{equation}
    \Gamma(\mathbb{P}, \mathbb{Q}) = \left\{\dfrac{1}{C} \delta_{(x_{(i)}, y_{\pi(i)})}: \pi \in \mathcal{S}\right\}
\end{equation}
where $\mathcal{S}$  is the set of all possible permutations of $C$ elements. Therefore we could write Eq. \ref{eq:wass} as follows
\begin{equation}
    W_p^p = \min_{\pi \in \mathcal{S}} \int_{M\times M}|x - y|^p \dfrac{1}{N} \sum_{i = 1}^C \delta_{(x_{(i)}, y_{\pi(i)})}(dx,dy)
\end{equation}
since $\mathcal{S}$ is finite, the infimum is a minimum. By exploiting the definition of Dirac's distribution and the linearity of the Lebesgue integral, for any $\pi \in \mathcal{S}$, we get
\begin{equation}
\begin{split}
\int_{M\times M}|x - y|^p \dfrac{1}{C}\sum_{i = 1}^C \delta_{(x_{(i)}, y_{\pi(i)})}(dx,dy) &= \dfrac{1}{C}\sum_{i = 1}^C\int_{M\times M} |x - y|^p\delta_{(x_{(i)}, y_{\pi(i)})}(dx,dy)\\
&=\dfrac{1}{C}\sum_{i = 1}^C |x_{(i)} - y_{\pi(i)}|^p\,\,.
\end{split} 
\end{equation}
Therefore, finding the Wasserstein distance between $\mathbb{P}$ and $\mathbb{Q}$ boils down to a combinatorial optimization problem, that is, finding the permutation $\pi \in \mathcal{S}$ that solves
\begin{equation}\label{eq:min_pi_empirical}
W_p^p(\mathbb{P}, \mathbb{Q}) = \min_{\pi \in \mathcal{S}} \dfrac{1}{C}\sum_{i = 1}^C |x_{(i)}- y_{\pi(i)}|^p\,\,.
\end{equation}
The minimum is achieved when $\pi = \pi^*$ that is the permutation providing the ranking statistic, i.e. $\pi^*(y_i) = y_{(i)}$, since the smallest value of the sum is given for the smallest fluctuations. Thus we conclude that the $p$-Wasserstein distance between $\mathbb{P}$ and $\mathbb{Q}$ is given by
\begin{equation}\label{eq:wass_empirical}
    W_p(\mathbb{P}, \mathbb{Q}) = \left (\dfrac{1}{C} \sum_{i = 1}^C|x_{(i)}- y_{(i)}|^p\right )^{1/p}
\end{equation}
that is the pairwise distance computed between the class frequency vectors, sorted in order of magnitude, for each client, introduced in Section \ref{clustereing_metric}, where we chose $p = 2$.
\end{proof}

\section{Privacy of \shortname}\label{app:privacy}
In the framework of \shortname, clients are required to send only the empirical loss vectors $l_k^{t,s}$ to the server \citep{cho2022towards}. While concerns might arise regarding the potential leakage of sensitive information from sharing this data, it is important to clarify that the server only needs to access aggregated statistics, working on aggregated data. This ensures that client-specific information remains private. Privacy can be effectively preserved by implementing the Secure Aggregation protocol \citep{bonawitz2016practical}, which guarantees that only the aggregated results are shared, preventing the exposure of any raw client data.

\section{Communication and Computational Overhead of \shortname}
\label{app:communication-computation}
\shortname\ minimizes communication and computational overhead, aligning with the requirements of scalable FL systems \citep{mcmahan2016federated}. On the client side, the computational cost remains unchanged compared to the chosen FL aggregation, e.g. FedA, as clients are only required to communicate their local models and a vector of empirical losses after each round. The size of this loss vector, denoted by \( S \), corresponds to the number of local iterations (\ie the product of local epochs and the number of batches) and is negligible w.r.t. the size of the model parameter space, \( |\Theta| \). In our experimental setup, \( S = 8 \), ensuring that the additional communication overhead from transmitting loss values is negligible in comparison to the transmission of model weights. 


All clustering computations, including those based on interaction matrices and Gaussian weighting, are performed exclusively on the server. This design ensures that client devices are not burdened with additional computational complexity or memory demands. The interaction matrix $P$ used in \shortname\ is updated incrementally and involves sparse matrix operations, which significantly reduce both memory usage and computational costs.

These characteristics make \shortname\ particularly well-suited for cross-device scenarios involving large federations and numerous communication rounds.
Moreover, by operating on scalar loss values rather than high-dimensional model parameters, the clustering process in \shortname\ achieves computational efficiency while maintaining effective grouping of clients. The server-side processing ensures that the method remains scalable, even as the number of clients and communication rounds increases. Consequently, \shortname\ meets the fundamental objectives of FL by minimizing costs while preserving privacy and maintaining high performance.


\section{\textcolor{black}{Metrics Used for Evaluation}} \label{app:metrics_choice}
\subsection{\textcolor{black}{Silhouette Score}}
\textcolor{black}{Silhouette Score is a clustering metric that measures the consistency of points within clusters by comparing intra-cluster and nearest-cluster distances \citep{rousseeuw1987silhouettes}. Let us consider a metric space $(M,d)$. For a set of points $\{x_1,\dots, x_N\} \subset M$ and clustering labels $\mathcal{C}_1, \dots, \mathcal{C}_{n_{cl}}$. The Silhouette score of a data point $x_i$ belonging to a cluster $\mathcal{C}_i$ is defined as}
\begin{equation}\label{silhouette_defn1}
    \color{black}
    s_i = \dfrac{b_i - a_i}{\max\{a_i,b_i\}}
\end{equation}
\textcolor{black}{where the values $b_i$ and $a_i$ represent the average intra-cluster distance and the minimal average outer-cluster distance, \ie}\begin{equation}\label{silhouette_defn}
    \color{black}
    \begin{split}
        a_i &= \dfrac{1}{|\mathcal{C}_i| - 1} \sum_{x_j \in \mathcal{C}_i\setminus\{x_i\}} d(x_i, x_j)\\
        b_i & =\min_{j \neq i} \dfrac{1}{|\mathcal{C}_j|} \sum_{x_j \in \mathcal{C}_j} d(x_i,x_j)
    \end{split}
\end{equation}
\textcolor{black}{The value of the Silhouette score ranges between $-1$ and $+1$, \ie $s_i \in [-1,1]$. In particular, a Silhouette score close to 1 indicates well-clustered data points, 0 denotes points near cluster boundaries, and -1 suggests misclassified points. In order to evaluate the overall performance of the clustering, a common choice, that is the one adopted in this paper, is to average the score value for each data point.}
\subsection{\textcolor{black}{Davies-Bouldin Score}}
\textcolor{black}{The Davies-Bouldin Score is a clustering metric that evaluates the quality of clustering by measuring the ratio of intra-cluster dispersion to inter-cluster separation \citep{davies1979cluster}. Let us consider a metric space $(M,d)$, a set of points $\{x_1, \dots, x_N\} \subset M$, and clustering labels $\mathcal{C}_1, \dots, \mathcal{C}_{n_{cl}}$. The Davies-Bouldin score is defined as the average similarity measure $R_{ij}$ between each cluster $\mathcal{C}_i$ and its most similar cluster $\mathcal{C}_j$}:
\begin{equation}\label{db_index_defn1}
    \color{black}
    DB = \dfrac{1}{n_{cl}} \sum_{i=1}^{n_{cl}} \max_{j \neq i} R_{ij}
\end{equation}
\textcolor{black}{where $R_{ij}$ is given by the ratio of intra-cluster distance $S_i$ to inter-cluster distance $D_{ij}$, \ie}
\begin{equation}\label{db_index_defn}
    \color{black}
    R_{ij} = \dfrac{S_i + S_j}{D_{ij}}
\end{equation}
\textcolor{black}{with intra-cluster distance $S_i$ defined as}
\begin{equation}
    \color{black}
    S_i = \dfrac{1}{|\mathcal{C}_i|} \sum_{x_k \in \mathcal{C}_i} d(x_k, c_i)
\end{equation}
\textcolor{black}{where $c_i$ denotes the centroid of cluster $\mathcal{C}_i$, and $D_{ij} = d(c_i, c_j)$ is the distance between centroids of clusters $\mathcal{C}_i$ and $\mathcal{C}_j$. A lower Davies-Bouldin Index indicates better clustering, as it reflects well-separated and compact clusters. Conversely, a higher DBI suggests that clusters are less distinct and more dispersed.}
\subsubsection{\textcolor{black}{Rand Index}} \textcolor{black}{Rand Index is a clustering score that measures the outcome of a clustering algorithm with respect to a ground truth clustering label \citep{rand1971objective}. Let us denote by $a$ the number of pairs that have been grouped in the same clusters, while by $b$ the number of pairs that have been grouped in different clusters, then the Rand-Index is defined as}
\begin{equation}
    \color{black}
    RI = \dfrac{a + b}{\binom{N}{2}}
\end{equation}
\textcolor{black}{where N denotes the number of data points. In our experiments we opted for the Rand Index score to evaluate how the algorithm was able to separate clients in groups of the same level of heterogeneity (which was known a priori and used as ground truth). A Rand Index ranges in $[0,1]$, and a value of 1 signifies a perfect agreement between the identified clusters and the ground truth.}

\newpage
\subsection{Inference Prompts}
For questions from the MATH and AIME benchmarks, we use the following prompt.
\begin{tcolorbox}[title=MATH and AIME Prompt]
Please answer the following question. Think carefully and in a step-by-step fashion. At the end of your solution, put your final result in a boxed environment, e.g. $\boxed{42}$.

\textcolor{blue}{The question would be here.}
\end{tcolorbox}

For questions from the LiveBench Math and LiveBench Reasoning benchmarks, which already come with their own instructions and formatting requests, we do not provide any accompanying prompt and simply submit the model the question verbatim.
\begin{tcolorbox}[title=LiveBench Prompt]
\textcolor{blue}{The question would be here.}
\end{tcolorbox}

\subsection{LM-Based Scoring}
\label{app:scoring}

Given a tuple consisting of a question, ground-truth solution, and candidate response, we grade the correctness of the candidate response by querying a Gemini-v1.5-Pro-002 model to compare the candidate and ground-truth solutions.
This involves repeating the following process five times: (1) send a prompt to the model that provides the question, the correct ground-truth solution, and the candidate response, and asks the model to deliberate on the correctness of the candidate response; and (2) send a followup prompt to the model to obtain a correctness ruling in a structured format.
If a strict majority of (valid) responses to the second prompt evaluate to a JSON object with the key-value pair \tcbox[on line,  boxrule=0.5pt, top=0pt, bottom=0pt, left=1pt, right=1pt]{``student\_final\_answer\_is\_correct'' = True} rather than \tcbox[on line,  boxrule=0.5pt, top=0pt, bottom=0pt, left=1pt, right=1pt]{``student\_final\_answer\_is\_correct'' = False}, the candidate response is labeled correct. Otherwise, the candidate response is labeled incorrect.
These queries are all processed with temperature zero.
The prompts, which can be found at the end of this subsection,
 ask the language model to
(1) identify the final answer of the given response, (2) identify the final answer of the reference (ground truth) response, and (3) determine whether the final answer of the given response satisfactorily matches that of the reference response, ignoring
any non-substantive formatting disagreements.
In line with convention, we instruct our scoring system to ignore the correctness of the logic used to reach the final answer and rather only judge the correctness of the final answer.
The model is asked to label all non-sensical and incomplete responses as being incorrect.

As a form of quality assurance, every scoring output for the Consistency@200 and Verification@200 figures depicted in Table~\ref{tab:main-sota} was manually compared against human scoring.
No discrepancies between automated and human scoring were found on the MATH and AIME datasets for both Consistency@200 and Verification@200.
No discrepancies were found on LiveBench Reasoning for Consistency@200.
For Verification@200, one false positive (answer labeled by automated system as being incorrect but labeled by human as being correct) and one false negative  (answer labeled by automated system as being correct but labeled by human as being incorrect) were identified on LiveBench Reasoning; three false positives and four false negatives were identified on LiveBench Math.
For Consistency@200, two false negatives were identified on LiveBench Math.
This means that LM scoring matched human scoring 99\% of the time, and the choice of human versus automated scoring matters little to our results.

\begin{tcolorbox}[title=Prompt 1, breakable]
You are an accurate and reliable automated grading system. Below are two solutions to a math exam problem: a solution written by a student and the solution from the answer key. Your task is to check if the student's solution reaches a correct final answer. 

Your response should consist of three parts. First, after reading the question carefully, identify the final answer of the answer key's solution. Second, identify the final answer of the student's solution. Third, identify whether the student's final answer is correct by comparing it to the answer key's final answer.

\# The question, answer key, and student solution

The math exam question:

\verb|```|

\textcolor{blue}{The question would be here.}

\verb|```|

The answer key solution:

\verb|```|

\textcolor{blue}{The reference solution would be here.}

\verb|```|

The student's solution:

\verb|```|

\textcolor{blue}{The candidate solution would be here.}

\verb|```|

\# Your response format

Please structure your response as follows. PROVIDE A COMPLETE RESPONSE.

\verb|```|

\# Answer Key Final Answer

Identify the final answer of the answer key solution. That's all you need to do here: just identify the final answer.

A "final answer" can take many forms, depending on what the question is asking for; it can be a number (e.g., "37"), a string (e.g., "ABCDE"), a sequence (e.g., "2,3,4,5"), a letter (e.g., "Y"), a multiple choice option (e.g. "C"), a word (e.g., "Apple"), an algebraic expression (e.g. "$x^2 + 37$"), a quantity with units (e.g. "4 miles"), or any of a number of other options. If a solution concludes that the question is not answerable with the information provided or otherwise claims that there is no solution to the problem, let the final answer be "None". If the solution does not produce any final answer because it appears to be cut off partway or is otherwise non-sensical, let the solution's final answer be "Incomplete solution" (this could only ever possibly happen with the student solution).

YOUR RESPONSE HERE SHOULD BE BRIEF. JUST IDENTIFY WHAT THE QUESTION IS ASKING FOR, AND IDENTIFY THE ANSWER KEY'S FINAL ANSWER. DO NOT ATTEMPT TO ANSWER THE QUESTION OR EVALUATE INTERMEDIATE STEPS.

\# Student Solution Final Answer

Identify the final answer of the student solution.

YOUR RESPONSE HERE SHOULD BE BRIEF. JUST IDENTIFY WHAT THE QUESTION IS ASKING FOR, AND IDENTIFY THE STUDENT'S FINAL ANSWER. DO NOT ATTEMPT TO ANSWER THE QUESTION OR EVALUATE INTERMEDIATE STEPS.

\# Correctness

Simply evaluate whether the student's final answer is correct by comparing it to the answer key's final answer.

Compare the student's final answer against the answer key's final answer to determine if the student's final answer is correct.

* It does not matter how the student reached their final answer, so long as their final answer itself is correct.

* It does not matter how the student formatted their final answer; for example, if the correct final answer is \boxed{7 / 2}, the student may write ***3.5*** or \boxed{\mathrm{three\; and\; a\; half}} or $\boxed{\frac{14}{4}}$. It does not matter if the student's final answer uses the same specific formatting that the question asks for, such as writing multiple choice options in the form "(E)" rather than "***E***".

* It does not matter if the student omitted units such as dollar signs.

* If the student solution appears to be truncated or otherwise incoherent, e.g. due to a technical glitch, then it should be treated as being incorrect.

ONCE AGAIN, DO NOT EVALUATE INTERMEDIATE STEPS OR TRY TO SOLVE THE PROBLEM YOURSELF. THE ANSWER KEY IS ALWAYS RIGHT. JUST COMPARE THE FINAL ANSWERS. IF THEY MATCH, THE STUDENT ANSWER IS CORRECT. IF THEY DO NOT MATCH, THE STUDENT ANSWER IS INCORRECT.

\# Summary

* Answer key final answer: (The final answer of the answer key solution. Please remove any unnecessary formatting, e.g. provide "3" rather than "\boxed{3}", provide "E" rather than "***E***", provide "1, 2, 3" rather than "[1, 2, 3]", provide "4 ounces" rather than "4oz".)

* Student final answer: (The final answer of the student's solution. Please remove any unnecessary formatting, e.g. provide "3" rather than "\boxed{3}", provide "E" rather than "***E***", provide "1, 2, 3" rather than "[1, 2, 3]", provide "4 ounces" rather than "4oz".)

* Student final answer is correct?: (Does the student final answer match the answer key final answer? Please provide "true" or "false".)

\verb|```|

\end{tcolorbox}

\begin{tcolorbox}[breakable, title=Prompt 2]
Please structure your output now as JSON, saying nothing else. Use the following format:
\verb|```|
\{
    "answer\_key\_final\_answer": str (the final answer of the answer key solution; please remove any formatting"),
    "student\_final\_answer": str (the final answer of the student's solution; please remove any formatting"),
    "student\_final\_answer\_is\_correct": true/false,
\}
\end{tcolorbox}

\subsection{Implementation of Consistency@k}
Consistency@k measures the performance of a model by evaluating the correctness of the most common answer reached by the model after being run $k$ times.
An important consideration with implementing consistency@k is that there are many choices for the equivalence relation one can use to define ``the most common answer''.
We define two candidate responses as reaching the same answer if their final answer is the same.
We determine a candidate response's final answer by prompting a language model to identify the final answer from the candidate response; we then strip the extracted final answer of leading and trailing whitespace.
We determine equivalence with a literal string match.
After determining the most common final answer to a question, we use the string \tcbox[on line,  boxrule=0.5pt, top=0pt, bottom=0pt, left=1pt, right=1pt]{``The final answer is \textcolor{blue}{\{final answer\}}''} as the consistency@k response.
Note that we could have instead randomly chosen a candidate response corresponding to the most common final answer, and used that selected response as the consistency@k response---we have found that, because our LM-based scoring system evaluates correctness using only the final answer, this alternative results in the same consistency@k metrics.

\subsection{Benchmark Evaluation Prompts}
\label{app:benchmarkprompts}

The benchmark performances reported in Table~\ref{tab:benchmark} are obtained with the following prompts.
The following prompt is used for the comparison task.
\begin{tcolorbox}[title=Comparison Task Prompt Part 1]
\textcolor{blue}{Question here.}

  Here are two solutions to the above question. You must determine which one is correct. Please think extremely carefully. Do not leap to conclusions. Find out where the solutions disagree, trace them back to the source of their disagreement, and figure out which one is right.

  Solution 1:
  
\textcolor{blue}{First solution here.}

  Solution 2:
  
\textcolor{blue}{Second solution here.}
\end{tcolorbox}
\begin{tcolorbox}[title=Comparison Task Prompt Part 2]
Now summarize your response in a JSON format. Respond in the following format saying nothing else:

  \{
  
     "correct\_solution": 1 or 2
     
  \}
\end{tcolorbox}

The following prompt is used for the scoring task.
\begin{tcolorbox}[title=Scoring Task Prompt Part 1]
\textcolor{blue}{Question here.}

 I include below a student solution to the above question. Determine whether the student solution reaches the correct final answer in a correct fashion; e.g., whether the solution makes two major errors that still coincidentally cancel out. Please be careful and do not leap to conclusions without first reasoning them through.

Solution:

\textcolor{blue}{Solution here.}
\end{tcolorbox}
\begin{tcolorbox}[title=Scoring Task Prompt Part 2]
Now summarize your response in a JSON format. Respond in the following format saying nothing else:

\{
 
 "is\_solution\_correct": 'yes' or 'no'

\}
\end{tcolorbox}

\section{Sensitive Analysis beta value RBF kernel} \label{app:sensitive}

This section provides a sensitivity analysis for the $\beta$ hyper-parameter of the RBF kernel adopted for \shortname. The results of this tuning are shown in Table \ref{tab:sensitive}.

\begin{table}[h]

    \caption{\small{A sensitivity analysis on the RBF kernel hyper-parameter $\beta$ is conducted. We present the balanced accuracy for \shortname on the Cifar10, Cifar100, and Femnist datasets for $\beta \in \{0.1, 0.5, 1.0, 2.0, 4.0\}$. It is noteworthy that \shortname demonstrates robustness to variations in this hyperparameter.}}
    \label{tab:sensitive}
    \centering
    \begin{adjustbox}{width=.6\linewidth}
        \centering
        
        \begin{tabular}{ccccc}
            
            \toprule

            \textbf{$\beta$} & \textbf{Cifar100} & \textbf{Femnist} & \textbf{Google Landmarks} & \textbf{iNaturalist}\\
            
            \midrule

            0.1 &  49.9 & 76.0 & 55.0 &  47.5\\
            0.5  & \textbf{53.4} & 76.0 & \textbf{57.4} & \textbf{47.8}\\
            1.0  & 49.5 & 76.0 & 56.0 & 47.5\\
            2.0  & 50.9 & 75.6 & 57.0 & 47.2 \\
            4.0 & 52.6 & \textbf{76.1} & 55.8 & 47.1 \\
            
            \bottomrule
        
        \end{tabular}
    \end{adjustbox}
\end{table}

\section{Additional Experiments: Visual Domain Detection in Cifar10}\label{app:cifar10}
In this section we present the result for the domain ablation discussed in Section \ref{sect:ablation} conducted on Cifar10 \cite{krizhevsky2009learning}. We explore how the algorithm identifies and groups clients based on the non-IID nature of their data distributions, represented by the Dirichlet concentration parameter $\alpha$. We apply a similar splitting approach, obtaining the following partitions: (1) 90 clients with $\alpha = 0$ and 10 clients with $\alpha = 100$; (2) 90 clients with $\alpha = 0.5$ and 10 clients with $\alpha = 100$; and (3) 40 clients with $\alpha = 100$, 30 clients with $\alpha = 0.5$, and 30 clients with $\alpha = 0$. We evaluate the outcome of this clustering experiment by means of WAS and WADB. Results in Table \ref{tab:ablation1_heter} show that \shortname detects clusters groups clients according to the level of heterogeneity of the group.
\begin{table}[t]
    
    \caption{\small{Clustering with three different splits on Cifar10. \shortname has superior clustering quality across different splits (homogeneous \textit{Hom}, heterogeneous \textit{Het}, extremly heterogeneous \textit{X Het})}}
    \centering
    \small
    \begin{adjustbox}{width=.5\linewidth}
        \label{tab_app:ablation1_heter}
     
        \begin{tabular}{lccccc}
            \toprule
            \textbf{Dataset} & \textbf{(Hom, Het, X Het)} & \makecell{\textbf{Clustering} \\ \textbf{method}} & \textbf{C} & \textbf{WAS} & \textbf{WADB} \\
            \cmidrule(lr){1-6}
          
            \midrule\multirow{9}{*}{Cifar10} 
            & \multirow{3}{*}{(10, 0, 90)} & \texttt{IFCA} & 1 & / &/ \\
            & & \texttt{FeSem} & 3 & -0.0 \scriptsize{$\pm$ 0.1} & 12.0 \scriptsize{$\pm$ 2.0}\\
            & & \shortname & 3 & \textbf{0.1 \scriptsize{$\pm$ 0.0}} & \textbf{0.2 \scriptsize{$\pm$ 0.0}} \\
            \cmidrule(lr){2-6}
            & \multirow{3}{*}{(10, 90, 0)} & \texttt{IFCA} & 1 & / & / \\
            & & \texttt{FeSem} & 3 & -0.0 \scriptsize{$\pm$ 0.0} & 12.0 \scriptsize{$\pm$ 2.0}\\

            & & \shortname & 3 & \textbf{0.2 \scriptsize{$\pm$ 0.0}} & \textbf{0.6 \scriptsize{$\pm$ 0.0}} \\
            \cmidrule(lr){2-6}
            & \multirow{3}{*}{(40, 30, 30)} & \texttt{IFCA} & 2 & -0.2 \scriptsize{$\pm$ 0.0} & \textbf{1.0 \scriptsize{$\pm$ 0.0}} \\
            & & \texttt{FeSem} & 3 & 0.1 \scriptsize{$\pm$ 0.1} & 20.6 \scriptsize{$\pm$ 7.1} \\
            & & \shortname & 3 & \textbf{0.6 \scriptsize{$\pm$ 0.1}} & \textbf{1.0 \scriptsize{$\pm$ 0.4}}\\
            
            \bottomrule
        \end{tabular}
    \end{adjustbox}
    
\end{table}

\begin{table}[t]
    
    \caption{\small Clustering performance of \shortname is assessed on federations with clients from varied domains using clean, noisy, and blurred (Clean, Noise, Blur) images from Cifar10 dataset. It utilizes the Rand Index score \citep{rand1971objective}, where a value close to 1 represents a perfect match between clustering and labels. Consistently \shortname accurately distinguishes all visual domains.}
    \label{tab_app:dom_abl}
    
    \centering
    
    \begin{adjustbox}{width=.5\linewidth}
    \setlength{\tabcolsep}{12pt}
        \begin{tabular}{ccccc}
           \toprule
            \textbf{Dataset} & \textbf{(Clean, Noise, Blur)} & \makecell{\textbf{Clustering} \\ \textbf{method}} & \textbf{C} & \textbf{Rand} \\
            \cmidrule(lr){1-5}
            \multirow{9}{*}{Cifar10} 
            & \multirow{3}{*}{(50, 50, 0)} & \texttt{IFCA} & 1 & 0.5 \scriptsize{$\pm$ 0.0}  \\
            & & \texttt{FeSem} & 2 & 0.49 \scriptsize{$\pm$ 0.2} \\
            & & \shortname & 2 & \textbf{1.0 \scriptsize{$\pm$ 0.0}} \\
            \cmidrule(lr){2-5}
            & \multirow{3}{*}{(50, 0, 50)} & \texttt{IFCA} & 1 & 0.5 \scriptsize{$\pm$ 0.0} \\
            & & \texttt{FeSem} & 2 & 0.5 \scriptsize{$\pm$ 0.1}\\
            & & \shortname & 2 & \textbf{1.0 \scriptsize{$\pm$ 0.0}} \\
            \cmidrule(lr){2-5}
            & \multirow{3}{*}{(40, 30, 30)} & \texttt{IFCA} & 1 & 0.33 \scriptsize{$\pm$ 0.0} \\
            & & \texttt{FeSem} & 3 & 0.34 \scriptsize{$\pm$ 0.1} \\
            & & \shortname & 4 & \textbf{0.9 \scriptsize{$\pm$ 0.0}} \\

            \bottomrule
        \end{tabular}
    \end{adjustbox}
    
\end{table}

\section{Evaluation of IFCA and FeSEM algorithms with different number of clusters} \label{app:tuning}

This section shows the tuning of the number of clusters for the \texttt{IFCA} and \texttt{FeSEM} algorithms, which cannot automatically detect this value. The results of this tuning are shown in Table \ref{tab:tuning_baselines}.

\begin{table}[t]

    \caption{\small{Performance of for baseline algorithms for clustering in FL \texttt{FeSEM}, and \texttt{IFCA}, w.r.t. the number of clusters}}
    \label{tab:tuning_baselines}
    \centering
    \small
    \begin{adjustbox}{width=.35\linewidth}
        \centering
        
        \begin{tabular}{llccc}
            
            \toprule

             & & \makecell{ \textbf{Clustering} \\ \textbf{method}} & \textbf{C} & \textbf{Acc} \\
            
           
            
            \cmidrule{2-5}

            & \multirow{8}{*}{\rotatebox[origin=c]{90}{Cifar100}} & \multirow{4}{*}{\texttt{IFCA}} & 2 & 46.7 \scriptsize{$\pm$ 0.0} \\
            & & & 3 &44.0 \scriptsize{$\pm$ 1.6} \\
            & & & 4 & 45.1 \scriptsize{$\pm$ 2.6} \\
            & & & 5 & 47.5 \scriptsize{$\pm$ 3.5} \\
            
            \cmidrule{3-5}
            
            & & \multirow{4}{*}{\texttt{FeSem}} & 2 & 43.3 \scriptsize{$\pm$ 1.3} \\
            & & & 3 & 48.0 \scriptsize{$\pm$ 1.9} \\
            & & & 4 &50.9 \scriptsize{$\pm$ 1.8} \\
            & & & 5 & 53.4 \scriptsize{$\pm$ 1.8} \\
            
            \cmidrule{2-5}

            & \multirow{8}{*}{\rotatebox[origin=c]{90}{Femnist}} & \multirow{4}{*}{\texttt{IFCA}} & 2 & 76.1 \scriptsize{$\pm$ 0.1} \\
            & & & 3 & 75.9 \scriptsize{$\pm$ 1.9} \\
            & & & 4 & 76.6 \scriptsize{$\pm$ 0.1} \\
            & & & 5 & 76.7 \scriptsize{$\pm$ 0.6} \\
            
            \cmidrule{3-5}
            
            & & \multirow{4}{*}{\texttt{FeSem}} & 2 & 75.6\scriptsize{$\pm$ 0.2} \\
            & & & 3 &75.5\scriptsize{$\pm$ 0.5} \\
            & & & 4 & 75.0\scriptsize{$\pm$ 0.1} \\
            & & & 5 &74.9\scriptsize{$\pm$ 0.1} \\
            
            \bottomrule
        
        \end{tabular}
    \end{adjustbox}
\end{table}

\section{Further Experiments} \label{app:other}
In Table \ref{tab_app:fl-algs} we show that \shortname is orthogonal to FL aggregation, which means that any algorithm can be easily embedded in our clustering setting, providing beneficial results, increasing model performance.
\begin{table}[t]
    \caption{\small{\shortname is orthogonal to FL aggregation algorithms, improving their performance in heterogeneous scenarios (Cifar100 with $\alpha = 0.5$ and Femnist with $\alpha = 0.01$). This shows that \shortname and clustering are beneficial in this scenarios.  }}
    \label{tab_app:fl-algs}
    \centering
    \small
    \setlength{\tabcolsep}{4pt} %
    \renewcommand{\arraystretch}{1.1} %
    \begin{adjustbox}{width=.5\linewidth}
    \begin{tabular}{l|cc|cc}
        \toprule
        \textbf{FL method} &  \multicolumn{2}{c|}{\textbf{Cifar100}} & \multicolumn{2}{c}{\textbf{Femnist}} \\
         & No Clusters & \shortname & No Clusters & \shortname \\
        \midrule
        FedAvg &  41.6 {\scriptsize$\pm$ 1.3} & \textbf{53.4} {\scriptsize$\pm$ 0.4} & 76.0 {\scriptsize$\pm$ 0.1} & \textbf{76.1} {\scriptsize$\pm$ 0.1} \\ 
        FedAvgM &  41.5 {\scriptsize$\pm$ 0.5} & \textbf{50.5} {\scriptsize$\pm$ 0.3} & {83.3} {\scriptsize$\pm$ 0.3} & \textbf{83.3} {\scriptsize$\pm$ 0.4} \\ 
        FedProx &  41.8 {\scriptsize$\pm$ 1.0} & \textbf{49.1} {\scriptsize$\pm$ 1.0} & 75.9 {\scriptsize$\pm$ 0.2} & \textbf{76.3} {\scriptsize$\pm$ 0.2} \\ 
        \bottomrule
    \end{tabular}
\end{adjustbox}
\end{table}
\begin{figure}
    \centering
    \includegraphics[width=\linewidth]{figures/tree_levels.pdf}
    \caption{\small{Cluster evolution with respect to the recursive splits in \shortname on Cifar100, projected on the spectral embedded bi-dimensional space. From left to right, top to bottom, we can see that \shortname splits the client into cluster, until a certain level of intra-cluster homogeneity is reached }}
    \label{fig:treelevels}
\end{figure}
\begin{figure}[htbp]
    \centering
    \includegraphics[width=0.6\linewidth]{figures/fedgw_cifar10_matrix_convergence.png}
    \caption{\small{Interaction matrix convergence: on the $y$-axis MSE in logarithmic scale w.r.t. communication rounds in the $x$-axis on Cifar10, with Dirichlet parameter $\alpha = 0.05$.}}
    \label{fig:mse_conv}
\end{figure}
Figure \ref{fig:hom-het} illustrates the clustering results corresponding to varying degrees of heterogeneity, as described in Section \ref{sect:ablation}. As per \shortname, the detection of clusters based on different levels of heterogeneity in the Cifar10 dataset is achieved. Specifically, an examination of the interaction matrix reveals a clear distinction between the two groups.
\begin{figure}[h]
    \centering
    \includegraphics[width=1\linewidth]{figures/spectral_split.pdf}
    \caption{\small{Homogeneous (Cifar10 $\alpha = 100$) vs heterogeneous clustering (Cifar10 $\alpha = 0.05$). The interaction matrix at convergence and the corresponding scaled affinity matrix are on the left. The scatter plot in the 2D plane with spectral embedding is on the right. It is possible to see that the algorithm perfectly separates homogeneous clients (orange) from heterogeneous clients (black) }}
    \label{fig:hom-het}
\end{figure}
In Figure \ref{fig:hom}, we show that in class-balanced scenarios with small heterogeneity, like Cifar10 with $\alpha = 100$, \shortname successfully detects one single cluster. Indeed, in homogeneous scenarios such as this one, the model benefits from accessing more data from all the clients.
\begin{figure}[h]
    \centering
    \includegraphics[width=\linewidth]{figures/spectral_uniform.pdf}
    \caption{\small{Homogeneous case (Cifar10 $\alpha = 100$).  The interaction matrix at convergence and the corresponding scaled affinity matrix are on the left. The scatter plot in the 2D plane with spectral embedding is on the right. In the homogeneous case where no clustering is needed, \textit{FedGW} does not split the clients.}}
    \label{fig:hom}
\end{figure}

Figure \ref{fig:mse_conv} shows how the MSE converges to a small value as the rounds increase for a Cifar10 experiment.

As Figure \ref{fig:class_distr} illustrates, \shortname partitions the Cifar100 dataset into clients based on class distributions. Each cluster's distribution is distinct and non-overlapping, demonstrating the algorithm's efficacy in partitioning data with varying degrees of heterogeneity.
\begin{figure}[t]
    \centering
    \includegraphics[width=\linewidth]{figures/data_distr.pdf}
    \caption{\small{Class distributions among distinct clusters as detected by \shortname on Cifar100. Specifically, we examine the class distributions for each pair of clusters, demonstrating that (1) the clusters were identified by grouping differing levels of heterogeneity and (2) there is, in most cases, an absence of overlapping classes.}}
    \label{fig:class_distr}
\end{figure}
In Figure \ref{fig:dom_ablation}, we report the domain detection on Cifar100, where 40 clients have clean images, 30 have noisy images, and 30 have blurred images. Table \ref{tab:dom_abl} shows that \shortname performs a good clustering, effectively separating the different domains.
\begin{figure}
    \centering
     
    \includegraphics[width=\linewidth]{figures/domain_ablation_c100.pdf}
    \caption{\small{\shortname in the presence of domain imbalance. Three domains on Cifar100: clean clients (unlabeled), noisy clients (+), and blurred clients (x). \textit{Left}: is the interaction matrix $P$ at convergence from which it is possible to see client relations. \textit{Center}: The affinity matrix $W$ computed with respect to the UPVs extracted from $P$, and on which \texttt{FedGW\_Clustering} is performed. We can see that \shortname clusters the clients according to the domain, as proved by results in Table \ref{tab:dom_abl}.}}
    \label{fig:dom_ablation}
   
\end{figure}



\end{document}
