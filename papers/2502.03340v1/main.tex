
\documentclass{article}
\usepackage{multicol}
\usepackage{capt-of}
\usepackage{amsthm}
\usepackage{multicol}

\usepackage{url}
\usepackage{graphicx}
\newtheorem{remark}{Remark}[section]
\usepackage{algorithm}

\usepackage{enumitem}
\usepackage{booktabs}
\usepackage{xspace} 
\usepackage{adjustbox}
\usepackage{booktabs}
\usepackage{makecell}
\usepackage{multirow}
\usepackage{pifont}
\usepackage[table,xcdraw]{xcolor}
\usepackage{parskip}
\usepackage{microtype}
\usepackage{graphicx}
\usepackage{subfigure}
\usepackage{booktabs} %
\newcommand{\ef}[1]{\color{orange}{[EF:#1]}\color{black}}
\newcommand{\mc}[1]{\color{red}{[MC:#1]}\color{black}}
\newcommand{\al}[1]{\color{blue}{[AL:#1]}\color{black}}
\newcommand{\dl}[1]{\color{green}{[DL:#1]}\color{black}}
\newcommand\norm[1]{\left\lVert#1\right\rVert}
\newcommand{\shortname}{\texttt{FedGWC}\xspace}

\makeatletter
\DeclareRobustCommand\onedot{\futurelet\@let@token\@onedot}
\def\@onedot{\ifx\@let@token.\else.\null\fi\xspace}
\def\eg{\emph{e.g}\onedot} \def\Eg{\emph{E.g}\onedot}
\def\ie{\emph{i.e}\onedot} \def\Ie{\emph{I.e}\onedot}
\def\cf{\emph{c.f}\onedot} \def\Cf{\emph{C.f}\onedot}
\def\etc{\emph{etc}\onedot} \def\vs{\emph{vs}\onedot}
\def\wrt{w.r.t\onedot} \def\dof{d.o.f\onedot}
\def\etal{\emph{et al}\onedot}
\makeatother
\usepackage{hyperref}


\newcommand{\theHalgorithm}{\arabic{algorithm}}




\usepackage[accepted]{icml2025}

\usepackage{amsmath}
\usepackage{amssymb}
\usepackage{mathtools}
\usepackage{amsthm}

\usepackage[capitalize,noabbrev]{cleveref}

\theoremstyle{plain}
\newtheorem{theorem}{Theorem}[section]
\newtheorem{proposition}[theorem]{Proposition}
\newtheorem{lemma}[theorem]{Lemma}
\newtheorem{corollary}[theorem]{Corollary}
\theoremstyle{definition}
\newtheorem{definition}[theorem]{Definition}
\newtheorem{assumption}[theorem]{Assumption}
\theoremstyle{remark}
\newcommand\blfootnote[1]{%
  \begingroup
  \renewcommand\thefootnote{}\footnote{#1}%
  \addtocounter{footnote}{-1}%
  \endgroup
}
\usepackage[textsize=tiny]{todonotes}
\usepackage{soul}

\newif\ifcomment
\commenttrue


\definecolor{red200}{HTML}{EF9A9A}
\definecolor{red400}{HTML}{EF5350}
\definecolor{blue50}{HTML}{E3F2FD}
\definecolor{blue100}{HTML}{BBDEFB}
\definecolor{blue200}{HTML}{90CAF9}
\definecolor{green200}{HTML}{A5D6A7}
\definecolor{yellow500}{HTML}{FFEB3B}

\ifcomment
    \newcommand{\marco}[1]{\sethlcolor{red200}\hl{[\textbf{Marco:} #1]}}
    \newcommand{\eros}[1]{\sethlcolor{yellow500}\hl{[\textbf{Eros:} #1]}}
\else
    \newcommand{\marco}[1]{}
    \newcommand{\eros}[1]{}
\fi


\icmltitlerunning{Interaction-Aware Gaussian Weighting for Clustered Federated Learning}

\begin{document}

\twocolumn[
\icmltitle{Interaction-Aware Gaussian Weighting for Clustered Federated Learning}



\icmlsetsymbol{equal}{*}
\icmlsetsymbol{start}{$\dagger$}

\begin{icmlauthorlist}
\icmlauthor{Alessandro Licciardi}{equal,disma,infn}
\icmlauthor{Davide Leo}{equal,dauin}
\icmlauthor{Eros Fan\'i }{dauin}
\icmlauthor{Barbara Caputo}{dauin}
\icmlauthor{Marco Ciccone}{start,vector}
\end{icmlauthorlist}

\icmlaffiliation{dauin}{Department of Computing and Control Engineering, Polytechnic University of Turin, Italy}
\icmlaffiliation{disma}{Department of Mathematical Sciences, Polytechnic University of Turin, Italy}
\icmlaffiliation{vector}{Vector Institute, Toronto, Ontario, Canada}
\icmlaffiliation{infn}{Istituto Nazionale di Fisica Nucleare (INFN), Sezione di Torino, Turin, Italy}

\icmlcorrespondingauthor{Alessandro Licciardi}{alessandro.licciardi@polito.it}


\icmlkeywords{Machine Learning, Federated Learning, Clustered Federated Learning}

\vskip 0.3in
]



\blfootnote{* Equal contribution, $\dagger$ Project started when the author was at Polytechnic University of Turin.  \textsuperscript{1} Department of Mathematical Sciences,
Polytechnic University of Turin, Italy - \textsuperscript{2} Istituto Nazionale di Fisica
Nucleare (INFN), Turin, Italy - \textsuperscript{3} Department of
Computing and Control Engineering, Polytechnic University of
Turin, Italy - \textsuperscript{4} Vector Institute, Toronto, Ontario, Canada. Correspondence to: Alessandro Licciardi \textit{alessandro.licciardi@polito.it}}
\begin{abstract}

Federated Learning (FL) emerged as a decentralized paradigm to train models while preserving privacy. However, conventional FL struggles with data heterogeneity and class imbalance, which degrade model performance.
Clustered FL balances personalization and decentralized training by grouping clients with analogous data distributions, enabling improved accuracy while adhering to privacy constraints. This approach effectively mitigates the adverse impact of heterogeneity in FL.
In this work, we propose a novel clustered FL method, \shortname (Federated Gaussian Weighting Clustering), which groups clients based on their data distribution, allowing training of a more robust and personalized model on the identified clusters. \shortname identifies homogeneous clusters by transforming individual empirical losses to model client interactions with a Gaussian reward mechanism. Additionally, we introduce the \textit{Wasserstein Adjusted Score}, a new clustering metric for FL to evaluate cluster cohesion with respect to the individual class distribution. Our experiments on benchmark datasets show that \shortname outperforms existing FL algorithms in cluster quality and classification accuracy, validating the efficacy of our approach.
\end{abstract}

\section{Introduction}
\label{sec:introduction}
The business processes of organizations are experiencing ever-increasing complexity due to the large amount of data, high number of users, and high-tech devices involved \cite{martin2021pmopportunitieschallenges, beerepoot2023biggestbpmproblems}. This complexity may cause business processes to deviate from normal control flow due to unforeseen and disruptive anomalies \cite{adams2023proceddsriftdetection}. These control-flow anomalies manifest as unknown, skipped, and wrongly-ordered activities in the traces of event logs monitored from the execution of business processes \cite{ko2023adsystematicreview}. For the sake of clarity, let us consider an illustrative example of such anomalies. Figure \ref{FP_ANOMALIES} shows a so-called event log footprint, which captures the control flow relations of four activities of a hypothetical event log. In particular, this footprint captures the control-flow relations between activities \texttt{a}, \texttt{b}, \texttt{c} and \texttt{d}. These are the causal ($\rightarrow$) relation, concurrent ($\parallel$) relation, and other ($\#$) relations such as exclusivity or non-local dependency \cite{aalst2022pmhandbook}. In addition, on the right are six traces, of which five exhibit skipped, wrongly-ordered and unknown control-flow anomalies. For example, $\langle$\texttt{a b d}$\rangle$ has a skipped activity, which is \texttt{c}. Because of this skipped activity, the control-flow relation \texttt{b}$\,\#\,$\texttt{d} is violated, since \texttt{d} directly follows \texttt{b} in the anomalous trace.
\begin{figure}[!t]
\centering
\includegraphics[width=0.9\columnwidth]{images/FP_ANOMALIES.png}
\caption{An example event log footprint with six traces, of which five exhibit control-flow anomalies.}
\label{FP_ANOMALIES}
\end{figure}

\subsection{Control-flow anomaly detection}
Control-flow anomaly detection techniques aim to characterize the normal control flow from event logs and verify whether these deviations occur in new event logs \cite{ko2023adsystematicreview}. To develop control-flow anomaly detection techniques, \revision{process mining} has seen widespread adoption owing to process discovery and \revision{conformance checking}. On the one hand, process discovery is a set of algorithms that encode control-flow relations as a set of model elements and constraints according to a given modeling formalism \cite{aalst2022pmhandbook}; hereafter, we refer to the Petri net, a widespread modeling formalism. On the other hand, \revision{conformance checking} is an explainable set of algorithms that allows linking any deviations with the reference Petri net and providing the fitness measure, namely a measure of how much the Petri net fits the new event log \cite{aalst2022pmhandbook}. Many control-flow anomaly detection techniques based on \revision{conformance checking} (hereafter, \revision{conformance checking}-based techniques) use the fitness measure to determine whether an event log is anomalous \cite{bezerra2009pmad, bezerra2013adlogspais, myers2018icsadpm, pecchia2020applicationfailuresanalysispm}. 

The scientific literature also includes many \revision{conformance checking}-independent techniques for control-flow anomaly detection that combine specific types of trace encodings with machine/deep learning \cite{ko2023adsystematicreview, tavares2023pmtraceencoding}. Whereas these techniques are very effective, their explainability is challenging due to both the type of trace encoding employed and the machine/deep learning model used \cite{rawal2022trustworthyaiadvances,li2023explainablead}. Hence, in the following, we focus on the shortcomings of \revision{conformance checking}-based techniques to investigate whether it is possible to support the development of competitive control-flow anomaly detection techniques while maintaining the explainable nature of \revision{conformance checking}.
\begin{figure}[!t]
\centering
\includegraphics[width=\columnwidth]{images/HIGH_LEVEL_VIEW.png}
\caption{A high-level view of the proposed framework for combining \revision{process mining}-based feature extraction with dimensionality reduction for control-flow anomaly detection.}
\label{HIGH_LEVEL_VIEW}
\end{figure}

\subsection{Shortcomings of \revision{conformance checking}-based techniques}
Unfortunately, the detection effectiveness of \revision{conformance checking}-based techniques is affected by noisy data and low-quality Petri nets, which may be due to human errors in the modeling process or representational bias of process discovery algorithms \cite{bezerra2013adlogspais, pecchia2020applicationfailuresanalysispm, aalst2016pm}. Specifically, on the one hand, noisy data may introduce infrequent and deceptive control-flow relations that may result in inconsistent fitness measures, whereas, on the other hand, checking event logs against a low-quality Petri net could lead to an unreliable distribution of fitness measures. Nonetheless, such Petri nets can still be used as references to obtain insightful information for \revision{process mining}-based feature extraction, supporting the development of competitive and explainable \revision{conformance checking}-based techniques for control-flow anomaly detection despite the problems above. For example, a few works outline that token-based \revision{conformance checking} can be used for \revision{process mining}-based feature extraction to build tabular data and develop effective \revision{conformance checking}-based techniques for control-flow anomaly detection \cite{singh2022lapmsh, debenedictis2023dtadiiot}. However, to the best of our knowledge, the scientific literature lacks a structured proposal for \revision{process mining}-based feature extraction using the state-of-the-art \revision{conformance checking} variant, namely alignment-based \revision{conformance checking}.

\subsection{Contributions}
We propose a novel \revision{process mining}-based feature extraction approach with alignment-based \revision{conformance checking}. This variant aligns the deviating control flow with a reference Petri net; the resulting alignment can be inspected to extract additional statistics such as the number of times a given activity caused mismatches \cite{aalst2022pmhandbook}. We integrate this approach into a flexible and explainable framework for developing techniques for control-flow anomaly detection. The framework combines \revision{process mining}-based feature extraction and dimensionality reduction to handle high-dimensional feature sets, achieve detection effectiveness, and support explainability. Notably, in addition to our proposed \revision{process mining}-based feature extraction approach, the framework allows employing other approaches, enabling a fair comparison of multiple \revision{conformance checking}-based and \revision{conformance checking}-independent techniques for control-flow anomaly detection. Figure \ref{HIGH_LEVEL_VIEW} shows a high-level view of the framework. Business processes are monitored, and event logs obtained from the database of information systems. Subsequently, \revision{process mining}-based feature extraction is applied to these event logs and tabular data input to dimensionality reduction to identify control-flow anomalies. We apply several \revision{conformance checking}-based and \revision{conformance checking}-independent framework techniques to publicly available datasets, simulated data of a case study from railways, and real-world data of a case study from healthcare. We show that the framework techniques implementing our approach outperform the baseline \revision{conformance checking}-based techniques while maintaining the explainable nature of \revision{conformance checking}.

In summary, the contributions of this paper are as follows.
\begin{itemize}
    \item{
        A novel \revision{process mining}-based feature extraction approach to support the development of competitive and explainable \revision{conformance checking}-based techniques for control-flow anomaly detection.
    }
    \item{
        A flexible and explainable framework for developing techniques for control-flow anomaly detection using \revision{process mining}-based feature extraction and dimensionality reduction.
    }
    \item{
        Application to synthetic and real-world datasets of several \revision{conformance checking}-based and \revision{conformance checking}-independent framework techniques, evaluating their detection effectiveness and explainability.
    }
\end{itemize}

The rest of the paper is organized as follows.
\begin{itemize}
    \item Section \ref{sec:related_work} reviews the existing techniques for control-flow anomaly detection, categorizing them into \revision{conformance checking}-based and \revision{conformance checking}-independent techniques.
    \item Section \ref{sec:abccfe} provides the preliminaries of \revision{process mining} to establish the notation used throughout the paper, and delves into the details of the proposed \revision{process mining}-based feature extraction approach with alignment-based \revision{conformance checking}.
    \item Section \ref{sec:framework} describes the framework for developing \revision{conformance checking}-based and \revision{conformance checking}-independent techniques for control-flow anomaly detection that combine \revision{process mining}-based feature extraction and dimensionality reduction.
    \item Section \ref{sec:evaluation} presents the experiments conducted with multiple framework and baseline techniques using data from publicly available datasets and case studies.
    \item Section \ref{sec:conclusions} draws the conclusions and presents future work.
\end{itemize}
\section{RELATED WORK}
\label{sec:relatedwork}
In this section, we describe the previous works related to our proposal, which are divided into two parts. In Section~\ref{sec:relatedwork_exoplanet}, we present a review of approaches based on machine learning techniques for the detection of planetary transit signals. Section~\ref{sec:relatedwork_attention} provides an account of the approaches based on attention mechanisms applied in Astronomy.\par

\subsection{Exoplanet detection}
\label{sec:relatedwork_exoplanet}
Machine learning methods have achieved great performance for the automatic selection of exoplanet transit signals. One of the earliest applications of machine learning is a model named Autovetter \citep{MCcauliff}, which is a random forest (RF) model based on characteristics derived from Kepler pipeline statistics to classify exoplanet and false positive signals. Then, other studies emerged that also used supervised learning. \cite{mislis2016sidra} also used a RF, but unlike the work by \citet{MCcauliff}, they used simulated light curves and a box least square \citep[BLS;][]{kovacs2002box}-based periodogram to search for transiting exoplanets. \citet{thompson2015machine} proposed a k-nearest neighbors model for Kepler data to determine if a given signal has similarity to known transits. Unsupervised learning techniques were also applied, such as self-organizing maps (SOM), proposed \citet{armstrong2016transit}; which implements an architecture to segment similar light curves. In the same way, \citet{armstrong2018automatic} developed a combination of supervised and unsupervised learning, including RF and SOM models. In general, these approaches require a previous phase of feature engineering for each light curve. \par

%DL is a modern data-driven technology that automatically extracts characteristics, and that has been successful in classification problems from a variety of application domains. The architecture relies on several layers of NNs of simple interconnected units and uses layers to build increasingly complex and useful features by means of linear and non-linear transformation. This family of models is capable of generating increasingly high-level representations \citep{lecun2015deep}.

The application of DL for exoplanetary signal detection has evolved rapidly in recent years and has become very popular in planetary science.  \citet{pearson2018} and \citet{zucker2018shallow} developed CNN-based algorithms that learn from synthetic data to search for exoplanets. Perhaps one of the most successful applications of the DL models in transit detection was that of \citet{Shallue_2018}; who, in collaboration with Google, proposed a CNN named AstroNet that recognizes exoplanet signals in real data from Kepler. AstroNet uses the training set of labelled TCEs from the Autovetter planet candidate catalog of Q1–Q17 data release 24 (DR24) of the Kepler mission \citep{catanzarite2015autovetter}. AstroNet analyses the data in two views: a ``global view'', and ``local view'' \citep{Shallue_2018}. \par


% The global view shows the characteristics of the light curve over an orbital period, and a local view shows the moment at occurring the transit in detail

%different = space-based

Based on AstroNet, researchers have modified the original AstroNet model to rank candidates from different surveys, specifically for Kepler and TESS missions. \citet{ansdell2018scientific} developed a CNN trained on Kepler data, and included for the first time the information on the centroids, showing that the model improves performance considerably. Then, \citet{osborn2020rapid} and \citet{yu2019identifying} also included the centroids information, but in addition, \citet{osborn2020rapid} included information of the stellar and transit parameters. Finally, \citet{rao2021nigraha} proposed a pipeline that includes a new ``half-phase'' view of the transit signal. This half-phase view represents a transit view with a different time and phase. The purpose of this view is to recover any possible secondary eclipse (the object hiding behind the disk of the primary star).


%last pipeline applies a procedure after the prediction of the model to obtain new candidates, this process is carried out through a series of steps that include the evaluation with Discovery and Validation of Exoplanets (DAVE) \citet{kostov2019discovery} that was adapted for the TESS telescope.\par
%



\subsection{Attention mechanisms in astronomy}
\label{sec:relatedwork_attention}
Despite the remarkable success of attention mechanisms in sequential data, few papers have exploited their advantages in astronomy. In particular, there are no models based on attention mechanisms for detecting planets. Below we present a summary of the main applications of this modeling approach to astronomy, based on two points of view; performance and interpretability of the model.\par
%Attention mechanisms have not yet been explored in all sub-areas of astronomy. However, recent works show a successful application of the mechanism.
%performance

The application of attention mechanisms has shown improvements in the performance of some regression and classification tasks compared to previous approaches. One of the first implementations of the attention mechanism was to find gravitational lenses proposed by \citet{thuruthipilly2021finding}. They designed 21 self-attention-based encoder models, where each model was trained separately with 18,000 simulated images, demonstrating that the model based on the Transformer has a better performance and uses fewer trainable parameters compared to CNN. A novel application was proposed by \citet{lin2021galaxy} for the morphological classification of galaxies, who used an architecture derived from the Transformer, named Vision Transformer (VIT) \citep{dosovitskiy2020image}. \citet{lin2021galaxy} demonstrated competitive results compared to CNNs. Another application with successful results was proposed by \citet{zerveas2021transformer}; which first proposed a transformer-based framework for learning unsupervised representations of multivariate time series. Their methodology takes advantage of unlabeled data to train an encoder and extract dense vector representations of time series. Subsequently, they evaluate the model for regression and classification tasks, demonstrating better performance than other state-of-the-art supervised methods, even with data sets with limited samples.

%interpretation
Regarding the interpretability of the model, a recent contribution that analyses the attention maps was presented by \citet{bowles20212}, which explored the use of group-equivariant self-attention for radio astronomy classification. Compared to other approaches, this model analysed the attention maps of the predictions and showed that the mechanism extracts the brightest spots and jets of the radio source more clearly. This indicates that attention maps for prediction interpretation could help experts see patterns that the human eye often misses. \par

In the field of variable stars, \citet{allam2021paying} employed the mechanism for classifying multivariate time series in variable stars. And additionally, \citet{allam2021paying} showed that the activation weights are accommodated according to the variation in brightness of the star, achieving a more interpretable model. And finally, related to the TESS telescope, \citet{morvan2022don} proposed a model that removes the noise from the light curves through the distribution of attention weights. \citet{morvan2022don} showed that the use of the attention mechanism is excellent for removing noise and outliers in time series datasets compared with other approaches. In addition, the use of attention maps allowed them to show the representations learned from the model. \par

Recent attention mechanism approaches in astronomy demonstrate comparable results with earlier approaches, such as CNNs. At the same time, they offer interpretability of their results, which allows a post-prediction analysis. \par


\section{Method}\label{sec:method}
\begin{figure}
    \centering
    \includegraphics[width=0.85\textwidth]{imgs/heatmap_acc.pdf}
    \caption{\textbf{Visualization of the proposed periodic Bayesian flow with mean parameter $\mu$ and accumulated accuracy parameter $c$ which corresponds to the entropy/uncertainty}. For $x = 0.3, \beta(1) = 1000$ and $\alpha_i$ defined in \cref{appd:bfn_cir}, this figure plots three colored stochastic parameter trajectories for receiver mean parameter $m$ and accumulated accuracy parameter $c$, superimposed on a log-scale heatmap of the Bayesian flow distribution $p_F(m|x,\senderacc)$ and $p_F(c|x,\senderacc)$. Note the \emph{non-monotonicity} and \emph{non-additive} property of $c$ which could inform the network the entropy of the mean parameter $m$ as a condition and the \emph{periodicity} of $m$. %\jj{Shrink the figures to save space}\hanlin{Do we need to make this figure one-column?}
    }
    \label{fig:vmbf_vis}
    \vskip -0.1in
\end{figure}
% \begin{wrapfigure}{r}{0.5\textwidth}
%     \centering
%     \includegraphics[width=0.49\textwidth]{imgs/heatmap_acc.pdf}
%     \caption{\textbf{Visualization of hyper-torus Bayesian flow based on von Mises Distribution}. For $x = 0.3, \beta(1) = 1000$ and $\alpha_i$ defined in \cref{appd:bfn_cir}, this figure plots three colored stochastic parameter trajectories for receiver mean parameter $m$ and accumulated accuracy parameter $c$, superimposed on a log-scale heatmap of the Bayesian flow distribution $p_F(m|x,\senderacc)$ and $p_F(c|x,\senderacc)$. Note the \emph{non-monotonicity} and \emph{non-additive} property of $c$. \jj{Shrink the figures to save space}}
%     \label{fig:vmbf_vis}
%     \vspace{-30pt}
% \end{wrapfigure}


In this section, we explain the detailed design of CrysBFN tackling theoretical and practical challenges. First, we describe how to derive our new formulation of Bayesian Flow Networks over hyper-torus $\mathbb{T}^{D}$ from scratch. Next, we illustrate the two key differences between \modelname and the original form of BFN: $1)$ a meticulously designed novel base distribution with different Bayesian update rules; and $2)$ different properties over the accuracy scheduling resulted from the periodicity and the new Bayesian update rules. Then, we present in detail the overall framework of \modelname over each manifold of the crystal space (\textit{i.e.} fractional coordinates, lattice vectors, atom types) respecting \textit{periodic E(3) invariance}. 

% In this section, we first demonstrate how to build Bayesian flow on hyper-torus $\mathbb{T}^{D}$ by overcoming theoretical and practical problems to provide a low-noise parameter-space approach to fractional atom coordinate generation. Next, we present how \modelname models each manifold of crystal space respecting \textit{periodic E(3) invariance}. 

\subsection{Periodic Bayesian Flow on Hyper-torus \texorpdfstring{$\mathbb{T}^{D}$}{}} 
For generative modeling of fractional coordinates in crystal, we first construct a periodic Bayesian flow on \texorpdfstring{$\mathbb{T}^{D}$}{} by designing every component of the totally new Bayesian update process which we demonstrate to be distinct from the original Bayesian flow (please see \cref{fig:non_add}). 
 %:) 
 
 The fractional atom coordinate system \citep{jiao2023crystal} inherently distributes over a hyper-torus support $\mathbb{T}^{3\times N}$. Hence, the normal distribution support on $\R$ used in the original \citep{bfn} is not suitable for this scenario. 
% The key problem of generative modeling for crystal is the periodicity of Cartesian atom coordinates $\vX$ requiring:
% \begin{equation}\label{eq:periodcity}
% p(\vA,\vL,\vX)=p(\vA,\vL,\vX+\vec{LK}),\text{where}~\vec{K}=\vec{k}\vec{1}_{1\times N},\forall\vec{k}\in\mathbb{Z}^{3\times1}
% \end{equation}
% However, there does not exist such a distribution supporting on $\R$ to model such property because the integration of such distribution over $\R$ will not be finite and equal to 1. Therefore, the normal distribution used in \citet{bfn} can not meet this condition.

To tackle this problem, the circular distribution~\citep{mardia2009directional} over the finite interval $[-\pi,\pi)$ is a natural choice as the base distribution for deriving the BFN on $\mathbb{T}^D$. 
% one natural choice is to 
% we would like to consider the circular distribution over the finite interval as the base 
% we find that circular distributions \citep{mardia2009directional} defined on a finite interval with lengths of $2\pi$ can be used as the instantiation of input distribution for the BFN on $\mathbb{T}^D$.
Specifically, circular distributions enjoy desirable periodic properties: $1)$ the integration over any interval length of $2\pi$ equals 1; $2)$ the probability distribution function is periodic with period $2\pi$.  Sharing the same intrinsic with fractional coordinates, such periodic property of circular distribution makes it suitable for the instantiation of BFN's input distribution, in parameterizing the belief towards ground truth $\x$ on $\mathbb{T}^D$. 
% \yuxuan{this is very complicated from my perspective.} \hanlin{But this property is exactly beautiful and perfectly fit into the BFN.}

\textbf{von Mises Distribution and its Bayesian Update} We choose von Mises distribution \citep{mardia2009directional} from various circular distributions as the form of input distribution, based on the appealing conjugacy property required in the derivation of the BFN framework.
% to leverage the Bayesian conjugacy property of von Mises distribution which is required by the BFN framework. 
That is, the posterior of a von Mises distribution parameterized likelihood is still in the family of von Mises distributions. The probability density function of von Mises distribution with mean direction parameter $m$ and concentration parameter $c$ (describing the entropy/uncertainty of $m$) is defined as: 
\begin{equation}
f(x|m,c)=vM(x|m,c)=\frac{\exp(c\cos(x-m))}{2\pi I_0(c)}
\end{equation}
where $I_0(c)$ is zeroth order modified Bessel function of the first kind as the normalizing constant. Given the last univariate belief parameterized by von Mises distribution with parameter $\theta_{i-1}=\{m_{i-1},\ c_{i-1}\}$ and the sample $y$ from sender distribution with unknown data sample $x$ and known accuracy $\alpha$ describing the entropy/uncertainty of $y$,  Bayesian update for the receiver is deducted as:
\begin{equation}
 h(\{m_{i-1},c_{i-1}\},y,\alpha)=\{m_i,c_i \}, \text{where}
\end{equation}
\begin{equation}\label{eq:h_m}
m_i=\text{atan2}(\alpha\sin y+c_{i-1}\sin m_{i-1}, {\alpha\cos y+c_{i-1}\cos m_{i-1}})
\end{equation}
\begin{equation}\label{eq:h_c}
c_i =\sqrt{\alpha^2+c_{i-1}^2+2\alpha c_{i-1}\cos(y-m_{i-1})}
\end{equation}
The proof of the above equations can be found in \cref{apdx:bayesian_update_function}. The atan2 function refers to  2-argument arctangent. Independently conducting  Bayesian update for each dimension, we can obtain the Bayesian update distribution by marginalizing $\y$:
\begin{equation}
p_U(\vtheta'|\vtheta,\bold{x};\alpha)=\mathbb{E}_{p_S(\bold{y}|\bold{x};\alpha)}\delta(\vtheta'-h(\vtheta,\bold{y},\alpha))=\mathbb{E}_{vM(\bold{y}|\bold{x},\alpha)}\delta(\vtheta'-h(\vtheta,\bold{y},\alpha))
\end{equation} 
\begin{figure}
    \centering
    \vskip -0.15in
    \includegraphics[width=0.95\linewidth]{imgs/non_add.pdf}
    \caption{An intuitive illustration of non-additive accuracy Bayesian update on the torus. The lengths of arrows represent the uncertainty/entropy of the belief (\emph{e.g.}~$1/\sigma^2$ for Gaussian and $c$ for von Mises). The directions of the arrows represent the believed location (\emph{e.g.}~ $\mu$ for Gaussian and $m$ for von Mises).}
    \label{fig:non_add}
    \vskip -0.15in
\end{figure}
\textbf{Non-additive Accuracy} 
The additive accuracy is a nice property held with the Gaussian-formed sender distribution of the original BFN expressed as:
\begin{align}
\label{eq:standard_id}
    \update(\parsn{}'' \mid \parsn{}, \x; \alpha_a+\alpha_b) = \E_{\update(\parsn{}' \mid \parsn{}, \x; \alpha_a)} \update(\parsn{}'' \mid \parsn{}', \x; \alpha_b)
\end{align}
Such property is mainly derived based on the standard identity of Gaussian variable:
\begin{equation}
X \sim \mathcal{N}\left(\mu_X, \sigma_X^2\right), Y \sim \mathcal{N}\left(\mu_Y, \sigma_Y^2\right) \Longrightarrow X+Y \sim \mathcal{N}\left(\mu_X+\mu_Y, \sigma_X^2+\sigma_Y^2\right)
\end{equation}
The additive accuracy property makes it feasible to derive the Bayesian flow distribution $
p_F(\boldsymbol{\theta} \mid \mathbf{x} ; i)=p_U\left(\boldsymbol{\theta} \mid \boldsymbol{\theta}_0, \mathbf{x}, \sum_{k=1}^{i} \alpha_i \right)
$ for the simulation-free training of \cref{eq:loss_n}.
It should be noted that the standard identity in \cref{eq:standard_id} does not hold in the von Mises distribution. Hence there exists an important difference between the original Bayesian flow defined on Euclidean space and the Bayesian flow of circular data on $\mathbb{T}^D$ based on von Mises distribution. With prior $\btheta = \{\bold{0},\bold{0}\}$, we could formally represent the non-additive accuracy issue as:
% The additive accuracy property implies the fact that the "confidence" for the data sample after observing a series of the noisy samples with accuracy ${\alpha_1, \cdots, \alpha_i}$ could be  as the accuracy sum  which could be  
% Here we 
% Here we emphasize the specific property of BFN based on von Mises distribution.
% Note that 
% \begin{equation}
% \update(\parsn'' \mid \parsn, \x; \alpha_a+\alpha_b) \ne \E_{\update(\parsn' \mid \parsn, \x; \alpha_a)} \update(\parsn'' \mid \parsn', \x; \alpha_b)
% \end{equation}
% \oyyw{please check whether the below equation is better}
% \yuxuan{I fill somehow confusing on what is the update distribution with $\alpha$. }
% \begin{equation}
% \update(\parsn{}'' \mid \parsn{}, \x; \alpha_a+\alpha_b) \ne \E_{\update(\parsn{}' \mid \parsn{}, \x; \alpha_a)} \update(\parsn{}'' \mid \parsn{}', \x; \alpha_b)
% \end{equation}
% We give an intuitive visualization of such difference in \cref{fig:non_add}. The untenability of this property can materialize by considering the following case: with prior $\btheta = \{\bold{0},\bold{0}\}$, check the two-step Bayesian update distribution with $\alpha_a,\alpha_b$ and one-step Bayesian update with $\alpha=\alpha_a+\alpha_b$:
\begin{align}
\label{eq:nonadd}
     &\update(c'' \mid \parsn, \x; \alpha_a+\alpha_b)  = \delta(c-\alpha_a-\alpha_b)
     \ne  \mathbb{E}_{p_U(\parsn' \mid \parsn, \x; \alpha_a)}\update(c'' \mid \parsn', \x; \alpha_b) \nonumber \\&= \mathbb{E}_{vM(\bold{y}_b|\bold{x},\alpha_a)}\mathbb{E}_{vM(\bold{y}_a|\bold{x},\alpha_b)}\delta(c-||[\alpha_a \cos\y_a+\alpha_b\cos \y_b,\alpha_a \sin\y_a+\alpha_b\sin \y_b]^T||_2)
\end{align}
A more intuitive visualization could be found in \cref{fig:non_add}. This fundamental difference between periodic Bayesian flow and that of \citet{bfn} presents both theoretical and practical challenges, which we will explain and address in the following contents.

% This makes constructing Bayesian flow based on von Mises distribution intrinsically different from previous Bayesian flows (\citet{bfn}).

% Thus, we must reformulate the framework of Bayesian flow networks  accordingly. % and do necessary reformulations of BFN. 

% \yuxuan{overall I feel this part is complicated by using the language of update distribution. I would like to suggest simply use bayesian update, to provide intuitive explantion.}\hanlin{See the illustration in \cref{fig:non_add}}

% That introduces a cascade of problems, and we investigate the following issues: $(1)$ Accuracies between sender and receiver are not synchronized and need to be differentiated. $(2)$ There is no tractable Bayesian flow distribution for a one-step sample conditioned on a given time step $i$, and naively simulating the Bayesian flow results in computational overhead. $(3)$ It is difficult to control the entropy of the Bayesian flow. $(4)$ Accuracy is no longer a function of $t$ and becomes a distribution conditioned on $t$, which can be different across dimensions.
%\jj{Edited till here}

\textbf{Entropy Conditioning} As a common practice in generative models~\citep{ddpm,flowmatching,bfn}, timestep $t$ is widely used to distinguish among generation states by feeding the timestep information into the networks. However, this paper shows that for periodic Bayesian flow, the accumulated accuracy $\vc_i$ is more effective than time-based conditioning by informing the network about the entropy and certainty of the states $\parsnt{i}$. This stems from the intrinsic non-additive accuracy which makes the receiver's accumulated accuracy $c$ not bijective function of $t$, but a distribution conditioned on accumulated accuracies $\vc_i$ instead. Therefore, the entropy parameter $\vc$ is taken logarithm and fed into the network to describe the entropy of the input corrupted structure. We verify this consideration in \cref{sec:exp_ablation}. 
% \yuxuan{implement variant. traditionally, the timestep is widely used to distinguish the different states by putting the timestep embedding into the networks. citation of FM, diffusion, BFN. However, we find that conditioned on time in periodic flow could not provide extra benefits. To further boost the performance, we introduce a simple yet effective modification term entropy conditional. This is based on that the accumulated accuracy which represents the current uncertainty or entropy could be a better indicator to distinguish different states. + Describe how you do this. }



\textbf{Reformulations of BFN}. Recall the original update function with Gaussian sender distribution, after receiving noisy samples $\y_1,\y_2,\dots,\y_i$ with accuracies $\senderacc$, the accumulated accuracies of the receiver side could be analytically obtained by the additive property and it is consistent with the sender side.
% Since observing sample $\y$ with $\alpha_i$ can not result in exact accuracy increment $\alpha_i$ for receiver, the accuracies between sender and receiver are not synchronized which need to be differentiated. 
However, as previously mentioned, this does not apply to periodic Bayesian flow, and some of the notations in original BFN~\citep{bfn} need to be adjusted accordingly. We maintain the notations of sender side's one-step accuracy $\alpha$ and added accuracy $\beta$, and alter the notation of receiver's accuracy parameter as $c$, which is needed to be simulated by cascade of Bayesian updates. We emphasize that the receiver's accumulated accuracy $c$ is no longer a function of $t$ (differently from the Gaussian case), and it becomes a distribution conditioned on received accuracies $\senderacc$ from the sender. Therefore, we represent the Bayesian flow distribution of von Mises distribution as $p_F(\btheta|\x;\alpha_1,\alpha_2,\dots,\alpha_i)$. And the original simulation-free training with Bayesian flow distribution is no longer applicable in this scenario.
% Different from previous BFNs where the accumulated accuracy $\rho$ is not explicitly modeled, the accumulated accuracy parameter $c$ (visualized in \cref{fig:vmbf_vis}) needs to be explicitly modeled by feeding it to the network to avoid information loss.
% the randomaccuracy parameter $c$ (visualized in \cref{fig:vmbf_vis}) implies that there exists information in $c$ from the sender just like $m$, meaning that $c$ also should be fed into the network to avoid information loss. 
% We ablate this consideration in  \cref{sec:exp_ablation}. 

\textbf{Fast Sampling from Equivalent Bayesian Flow Distribution} Based on the above reformulations, the Bayesian flow distribution of von Mises distribution is reframed as: 
\begin{equation}\label{eq:flow_frac}
p_F(\btheta_i|\x;\alpha_1,\alpha_2,\dots,\alpha_i)=\E_{\update(\parsnt{1} \mid \parsnt{0}, \x ; \alphat{1})}\dots\E_{\update(\parsn_{i-1} \mid \parsnt{i-2}, \x; \alphat{i-1})} \update(\parsnt{i} | \parsnt{i-1},\x;\alphat{i} )
\end{equation}
Naively sampling from \cref{eq:flow_frac} requires slow auto-regressive iterated simulation, making training unaffordable. Noticing the mathematical properties of \cref{eq:h_m,eq:h_c}, we  transform \cref{eq:flow_frac} to the equivalent form:
\begin{equation}\label{eq:cirflow_equiv}
p_F(\vec{m}_i|\x;\alpha_1,\alpha_2,\dots,\alpha_i)=\E_{vM(\y_1|\x,\alpha_1)\dots vM(\y_i|\x,\alpha_i)} \delta(\vec{m}_i-\text{atan2}(\sum_{j=1}^i \alpha_j \cos \y_j,\sum_{j=1}^i \alpha_j \sin \y_j))
\end{equation}
\begin{equation}\label{eq:cirflow_equiv2}
p_F(\vec{c}_i|\x;\alpha_1,\alpha_2,\dots,\alpha_i)=\E_{vM(\y_1|\x,\alpha_1)\dots vM(\y_i|\x,\alpha_i)}  \delta(\vec{c}_i-||[\sum_{j=1}^i \alpha_j \cos \y_j,\sum_{j=1}^i \alpha_j \sin \y_j]^T||_2)
\end{equation}
which bypasses the computation of intermediate variables and allows pure tensor operations, with negligible computational overhead.
\begin{restatable}{proposition}{cirflowequiv}
The probability density function of Bayesian flow distribution defined by \cref{eq:cirflow_equiv,eq:cirflow_equiv2} is equivalent to the original definition in \cref{eq:flow_frac}. 
\end{restatable}
\textbf{Numerical Determination of Linear Entropy Sender Accuracy Schedule} ~Original BFN designs the accuracy schedule $\beta(t)$ to make the entropy of input distribution linearly decrease. As for crystal generation task, to ensure information coherence between modalities, we choose a sender accuracy schedule $\senderacc$ that makes the receiver's belief entropy $H(t_i)=H(p_I(\cdot|\vtheta_i))=H(p_I(\cdot|\vc_i))$ linearly decrease \emph{w.r.t.} time $t_i$, given the initial and final accuracy parameter $c(0)$ and $c(1)$. Due to the intractability of \cref{eq:vm_entropy}, we first use numerical binary search in $[0,c(1)]$ to determine the receiver's $c(t_i)$ for $i=1,\dots, n$ by solving the equation $H(c(t_i))=(1-t_i)H(c(0))+tH(c(1))$. Next, with $c(t_i)$, we conduct numerical binary search for each $\alpha_i$ in $[0,c(1)]$ by solving the equations $\E_{y\sim vM(x,\alpha_i)}[\sqrt{\alpha_i^2+c_{i-1}^2+2\alpha_i c_{i-1}\cos(y-m_{i-1})}]=c(t_i)$ from $i=1$ to $i=n$ for arbitrarily selected $x\in[-\pi,\pi)$.

After tackling all those issues, we have now arrived at a new BFN architecture for effectively modeling crystals. Such BFN can also be adapted to other type of data located in hyper-torus $\mathbb{T}^{D}$.

\subsection{Equivariant Bayesian Flow for Crystal}
With the above Bayesian flow designed for generative modeling of fractional coordinate $\vF$, we are able to build equivariant Bayesian flow for each modality of crystal. In this section, we first give an overview of the general training and sampling algorithm of \modelname (visualized in \cref{fig:framework}). Then, we describe the details of the Bayesian flow of every modality. The training and sampling algorithm can be found in \cref{alg:train} and \cref{alg:sampling}.

\textbf{Overview} Operating in the parameter space $\bthetaM=\{\bthetaA,\bthetaL,\bthetaF\}$, \modelname generates high-fidelity crystals through a joint BFN sampling process on the parameter of  atom type $\bthetaA$, lattice parameter $\vec{\theta}^L=\{\bmuL,\brhoL\}$, and the parameter of fractional coordinate matrix $\bthetaF=\{\bmF,\bcF\}$. We index the $n$-steps of the generation process in a discrete manner $i$, and denote the corresponding continuous notation $t_i=i/n$ from prior parameter $\thetaM_0$ to a considerably low variance parameter $\thetaM_n$ (\emph{i.e.} large $\vrho^L,\bmF$, and centered $\bthetaA$).

At training time, \modelname samples time $i\sim U\{1,n\}$ and $\bthetaM_{i-1}$ from the Bayesian flow distribution of each modality, serving as the input to the network. The network $\net$ outputs $\net(\parsnt{i-1}^\mathcal{M},t_{i-1})=\net(\parsnt{i-1}^A,\parsnt{i-1}^F,\parsnt{i-1}^L,t_{i-1})$ and conducts gradient descents on loss function \cref{eq:loss_n} for each modality. After proper training, the sender distribution $p_S$ can be approximated by the receiver distribution $p_R$. 

At inference time, from predefined $\thetaM_0$, we conduct transitions from $\thetaM_{i-1}$ to $\thetaM_{i}$ by: $(1)$ sampling $\y_i\sim p_R(\bold{y}|\thetaM_{i-1};t_i,\alpha_i)$ according to network prediction $\predM{i-1}$; and $(2)$ performing Bayesian update $h(\thetaM_{i-1},\y^\calM_{i-1},\alpha_i)$ for each dimension. 

% Alternatively, we complete this transition using the flow-back technique by sampling 
% $\thetaM_{i}$ from Bayesian flow distribution $\flow(\btheta^M_{i}|\predM{i-1};t_{i-1})$. 

% The training objective of $\net$ is to minimize the KL divergence between sender distribution and receiver distribution for every modality as defined in \cref{eq:loss_n} which is equivalent to optimizing the negative variational lower bound $\calL^{VLB}$ as discussed in \cref{sec:preliminaries}. 

%In the following part, we will present the Bayesian flow of each modality in detail.

\textbf{Bayesian Flow of Fractional Coordinate $\vF$}~The distribution of the prior parameter $\bthetaF_0$ is defined as:
\begin{equation}\label{eq:prior_frac}
    p(\bthetaF_0) \defeq \{vM(\vm_0^F|\vec{0}_{3\times N},\vec{0}_{3\times N}),\delta(\vc_0^F-\vec{0}_{3\times N})\} = \{U(\vec{0},\vec{1}),\delta(\vc_0^F-\vec{0}_{3\times N})\}
\end{equation}
Note that this prior distribution of $\vm_0^F$ is uniform over $[\vec{0},\vec{1})$, ensuring the periodic translation invariance property in \cref{De:pi}. The training objective is minimizing the KL divergence between sender and receiver distribution (deduction can be found in \cref{appd:cir_loss}): 
%\oyyw{replace $\vF$ with $\x$?} \hanlin{notations follow Preliminary?}
\begin{align}\label{loss_frac}
\calL_F = n \E_{i \sim \ui{n}, \flow(\parsn{}^F \mid \vF ; \senderacc)} \alpha_i\frac{I_1(\alpha_i)}{I_0(\alpha_i)}(1-\cos(\vF-\predF{i-1}))
\end{align}
where $I_0(x)$ and $I_1(x)$ are the zeroth and the first order of modified Bessel functions. The transition from $\bthetaF_{i-1}$ to $\bthetaF_{i}$ is the Bayesian update distribution based on network prediction:
\begin{equation}\label{eq:transi_frac}
    p(\btheta^F_{i}|\parsnt{i-1}^\calM)=\mathbb{E}_{vM(\bold{y}|\predF{i-1},\alpha_i)}\delta(\btheta^F_{i}-h(\btheta^F_{i-1},\bold{y},\alpha_i))
\end{equation}
\begin{restatable}{proposition}{fracinv}
With $\net_{F}$ as a periodic translation equivariant function namely $\net_F(\parsnt{}^A,w(\parsnt{}^F+\vt),\parsnt{}^L,t)=w(\net_F(\parsnt{}^A,\parsnt{}^F,\parsnt{}^L,t)+\vt), \forall\vt\in\R^3$, the marginal distribution of $p(\vF_n)$ defined by \cref{eq:prior_frac,eq:transi_frac} is periodic translation invariant. 
\end{restatable}
\textbf{Bayesian Flow of Lattice Parameter \texorpdfstring{$\boldsymbol{L}$}{}}   
Noting the lattice parameter $\bm{L}$ located in Euclidean space, we set prior as the parameter of a isotropic multivariate normal distribution $\btheta^L_0\defeq\{\vmu_0^L,\vrho_0^L\}=\{\bm{0}_{3\times3},\bm{1}_{3\times3}\}$
% \begin{equation}\label{eq:lattice_prior}
% \btheta^L_0\defeq\{\vmu_0^L,\vrho_0^L\}=\{\bm{0}_{3\times3},\bm{1}_{3\times3}\}
% \end{equation}
such that the prior distribution of the Markov process on $\vmu^L$ is the Dirac distribution $\delta(\vec{\mu_0}-\vec{0})$ and $\delta(\vec{\rho_0}-\vec{1})$, 
% \begin{equation}
%     p_I^L(\boldsymbol{L}|\btheta_0^L)=\mathcal{N}(\bm{L}|\bm{0},\bm{I})
% \end{equation}
which ensures O(3)-invariance of prior distribution of $\vL$. By Eq. 77 from \citet{bfn}, the Bayesian flow distribution of the lattice parameter $\bm{L}$ is: 
\begin{align}% =p_U(\bmuL|\btheta_0^L,\bm{L},\beta(t))
p_F^L(\bmuL|\bm{L};t) &=\mathcal{N}(\bmuL|\gamma(t)\bm{L},\gamma(t)(1-\gamma(t))\bm{I}) 
\end{align}
where $\gamma(t) = 1 - \sigma_1^{2t}$ and $\sigma_1$ is the predefined hyper-parameter controlling the variance of input distribution at $t=1$ under linear entropy accuracy schedule. The variance parameter $\vrho$ does not need to be modeled and fed to the network, since it is deterministic given the accuracy schedule. After sampling $\bmuL_i$ from $p_F^L$, the training objective is defined as minimizing KL divergence between sender and receiver distribution (based on Eq. 96 in \citet{bfn}):
\begin{align}
\mathcal{L}_{L} = \frac{n}{2}\left(1-\sigma_1^{2/n}\right)\E_{i \sim \ui{n}}\E_{\flow(\bmuL_{i-1} |\vL ; t_{i-1})}  \frac{\left\|\vL -\predL{i-1}\right\|^2}{\sigma_1^{2i/n}},\label{eq:lattice_loss}
\end{align}
where the prediction term $\predL{i-1}$ is the lattice parameter part of network output. After training, the generation process is defined as the Bayesian update distribution given network prediction:
\begin{equation}\label{eq:lattice_sampling}
    p(\bmuL_{i}|\parsnt{i-1}^\calM)=\update^L(\bmuL_{i}|\predL{i-1},\bmuL_{i-1};t_{i-1})
\end{equation}
    

% The final prediction of the lattice parameter is given by $\bmuL_n = \predL{n-1}$.
% \begin{equation}\label{eq:final_lattice}
%     \bmuL_n = \predL{n-1}
% \end{equation}

\begin{restatable}{proposition}{latticeinv}\label{prop:latticeinv}
With $\net_{L}$ as  O(3)-equivariant function namely $\net_L(\parsnt{}^A,\parsnt{}^F,\vQ\parsnt{}^L,t)=\vQ\net_L(\parsnt{}^A,\parsnt{}^F,\parsnt{}^L,t),\forall\vQ^T\vQ=\vI$, the marginal distribution of $p(\bmuL_n)$ defined by \cref{eq:lattice_sampling} is O(3)-invariant. 
\end{restatable}


\textbf{Bayesian Flow of Atom Types \texorpdfstring{$\boldsymbol{A}$}{}} 
Given that atom types are discrete random variables located in a simplex $\calS^K$, the prior parameter of $\boldsymbol{A}$ is the discrete uniform distribution over the vocabulary $\parsnt{0}^A \defeq \frac{1}{K}\vec{1}_{1\times N}$. 
% \begin{align}\label{eq:disc_input_prior}
% \parsnt{0}^A \defeq \frac{1}{K}\vec{1}_{1\times N}
% \end{align}
% \begin{align}
%     (\oh{j}{K})_k \defeq \delta_{j k}, \text{where }\oh{j}{K}\in \R^{K},\oh{\vA}{KD} \defeq \left(\oh{a_1}{K},\dots,\oh{a_N}{K}\right) \in \R^{K\times N}
% \end{align}
With the notation of the projection from the class index $j$ to the length $K$ one-hot vector $ (\oh{j}{K})_k \defeq \delta_{j k}, \text{where }\oh{j}{K}\in \R^{K},\oh{\vA}{KD} \defeq \left(\oh{a_1}{K},\dots,\oh{a_N}{K}\right) \in \R^{K\times N}$, the Bayesian flow distribution of atom types $\vA$ is derived in \citet{bfn}:
\begin{align}
\flow^{A}(\parsn^A \mid \vA; t) &= \E_{\N{\y \mid \beta^A(t)\left(K \oh{\vA}{K\times N} - \vec{1}_{K\times N}\right)}{\beta^A(t) K \vec{I}_{K\times N \times N}}} \delta\left(\parsn^A - \frac{e^{\y}\parsnt{0}^A}{\sum_{k=1}^K e^{\y_k}(\parsnt{0})_{k}^A}\right).
\end{align}
where $\beta^A(t)$ is the predefined accuracy schedule for atom types. Sampling $\btheta_i^A$ from $p_F^A$ as the training signal, the training objective is the $n$-step discrete-time loss for discrete variable \citep{bfn}: 
% \oyyw{can we simplify the next equation? Such as remove $K \times N, K \times N \times N$}
% \begin{align}
% &\calL_A = n\E_{i \sim U\{1,n\},\flow^A(\parsn^A \mid \vA ; t_{i-1}),\N{\y \mid \alphat{i}\left(K \oh{\vA}{KD} - \vec{1}_{K\times N}\right)}{\alphat{i} K \vec{I}_{K\times N \times N}}} \ln \N{\y \mid \alphat{i}\left(K \oh{\vA}{K\times N} - \vec{1}_{K\times N}\right)}{\alphat{i} K \vec{I}_{K\times N \times N}}\nonumber\\
% &\qquad\qquad\qquad-\sum_{d=1}^N \ln \left(\sum_{k=1}^K \out^{(d)}(k \mid \parsn^A; t_{i-1}) \N{\ydd{d} \mid \alphat{i}\left(K\oh{k}{K}- \vec{1}_{K\times N}\right)}{\alphat{i} K \vec{I}_{K\times N \times N}}\right)\label{discdisc_t_loss_exp}
% \end{align}
\begin{align}
&\calL_A = n\E_{i \sim U\{1,n\},\flow^A(\parsn^A \mid \vA ; t_{i-1}),\N{\y \mid \alphat{i}\left(K \oh{\vA}{KD} - \vec{1}\right)}{\alphat{i} K \vec{I}}} \ln \N{\y \mid \alphat{i}\left(K \oh{\vA}{K\times N} - \vec{1}\right)}{\alphat{i} K \vec{I}}\nonumber\\
&\qquad\qquad\qquad-\sum_{d=1}^N \ln \left(\sum_{k=1}^K \out^{(d)}(k \mid \parsn^A; t_{i-1}) \N{\ydd{d} \mid \alphat{i}\left(K\oh{k}{K}- \vec{1}\right)}{\alphat{i} K \vec{I}}\right)\label{discdisc_t_loss_exp}
\end{align}
where $\vec{I}\in \R^{K\times N \times N}$ and $\vec{1}\in\R^{K\times D}$. When sampling, the transition from $\bthetaA_{i-1}$ to $\bthetaA_{i}$ is derived as:
\begin{equation}
    p(\btheta^A_{i}|\parsnt{i-1}^\calM)=\update^A(\btheta^A_{i}|\btheta^A_{i-1},\predA{i-1};t_{i-1})
\end{equation}

The detailed training and sampling algorithm could be found in \cref{alg:train} and \cref{alg:sampling}.




\section{Theoretical Results and Derivation of \shortname}\label{sec:theory}
In this section, we provide a formal derivation of the algorithm, discussing the mathematical properties of Gaussian Weights and outlining the structured formalism and rationale underlying \shortname.

The stochastic process induced by the optimization algorithm in the local update step, allows the evolution of the empirical loss to be modeled using random variables within a probabilistic framework. We denote random variables with capital letters (\eg , $X$), and their realizations with lowercase letters (\eg, $x$).

The observed loss process $l_k^{t,s}$ is the outcome of a stochastic process $L_k^{t,s}$, and the rewards $r_k^{t,s}$, computed according to Eq. \ref{eq:reward}, are samples from a random reward $R_k^{t,s}$ supported in $(0,1)$, whose expectation $\mathbb{E}[R_k^{t,s}]$ is lower for out-of-distribution clients and higher for in-distribution ones. To estimate the expected reward $\mathbb{E}[R_k^{t,s}]$ we introduce the r.v. $\Omega_k^t = 1/S \sum_{s \in [S]} R_k^{t,s}$, which is an estimator less affected by noisy fluctuations in the empirical loss. Due to the linearity of the expectation operator, the expected reward $\mathbb{E}[R_k^{t,s}]$ for the $k$-th client at round $t$, local iteration $s$ equals the expected Gaussian reward $ \mathbb{E}[\Omega_k^t]$ that, to simplify the notation, we denote by $\mu_k$. $\mu_k$ is the theoretical value that we aim to estimate by designing our Gaussian weights $\Gamma_k^t$ appropriately, as it encodes the ideal reward to quantify the closeness of the distribution of each client $k$ to the main distribution. Note that the process is stationary by construction. Therefore, it does not depend on $t$ but differs between clients, as it reaches a higher value for in-distribution clients and a lower for out-of-distribution clients.

To rigorously motivate the construction of our algorithm and the reliability of the weights, we introduce the following theoretical results. Theorems \ref{thm_main:1} and \ref{thm_main:weak_conv} demonstrate that the weights converge to a finite value and, more importantly, that this limit serves as an unbiased estimator of the theoretical reward $\mu_k$. The first theorem provides a strong convergence result, showing that, with suitable choices of the sequence $\{\alpha_t\}_t$, the expectation of the Gaussian weights $\Gamma_k^t$ converges to $\mu_k$ in $L^2$ and almost surely. In addition, Theorem \ref{thm_main:weak_conv} extends this to the case of constant $\alpha_t$, proving that the weights still converge and remain unbiased estimators of the rewards as $t \to \infty$. 
\begin{theorem}\label{thm_main:1}
Let $\{\alpha_t\}_{t = 1}^\infty$ be a sequence of positive real values, and $\{\Gamma_k^t\}_{t=1}^\infty$ the sequence of Gaussian weights. If $\{\alpha_t\}_{t = 1}^\infty \in l^2(\mathbb{N})/l^1(\mathbb{N})$, then $\Gamma_k^t$ converges in $L^2$. Furthermore, for $t\to\infty$, 
\begin{equation}
    \Gamma_k^t \longrightarrow \mu_k\,\,\, a.s.
\end{equation}
\end{theorem}
\begin{theorem}\label{thm_main:weak_conv}
Let $\alpha \in (0,1)$ be a fixed constant, then in the limit $t \to \infty$, the expectation of the weights converges to the individual theoretical reward $\mu_k$, for each client $k = 1,\dots, K$, \ie,
\begin{equation}
    \mathbb{E}[\Gamma_k^t]\longrightarrow \mu_k\,\,\,t\to\infty\,.
\end{equation}
\end{theorem}
Proposition \ref{prop_var_main} shows that Gaussian weights reduce the variance of the estimate, thus decreasing the error and enabling the construction of a confidence interval for $\mu_k$.
\begin{proposition}\label{prop_var_main}
The variance of the weights $\Gamma_k^t$ is smaller than the variance $\sigma_k^2$ of the theoretical rewards $R_k^{t,s}$.
\end{proposition} 

Complete proofs of Theorems \ref{thm_main:1}, \ref{thm_main:weak_conv}, and Proposition \ref{prop_var_main} are provided in Appendix \ref{app:fgw}. Additionally, Appendix \ref{app:fgw} includes further analysis of \shortname. Specifically, Proposition \ref{prop:bounded_matrix} demonstrates that the entries of the interaction matrix $P$ are bounded, while Theorem \ref{thm:samplerate} establishes a sufficient condition for conserving the sampling rate during the recursive procedure.


\section{Wasserstein Adjusted Score}\label{clustereing_metric}

In the previous Section we observed that when clustering clients according to different heterogeneity levels, the outcome must be evaluated using a metric that assesses the cohesion of individual distributions. In this Section, we introduce a novel metric to evaluate the performance of clustering algorithms in FL. This metric, derived from the Wasserstein distance \citep{kantorovich1942translocation}, quantifies the cohesion of client groups based on their class distribution similarities. 

We propose a general method for adapting clustering metrics to account for class imbalances. This adjustment is particularly relevant when the underlying class distributions across clients are skewed. The formal derivation and mathematical details of the proposed metric are provided in Appendix \ref{app:clustering}. We now provide a high-level overview of our new metric.

Consider a generic clustering metric $s$, e.g. Davies-Bouldin score \citep{davies1979cluster} or the Silhouette score \citep{rousseeuw1987silhouettes}. Let $C$ denote the total number of classes, and $x_i^k$ the empirical frequency of the $i$-th class in the $k$-th client's local training set. Following theoretical reasonings, as shown in Appendix \ref{app:clustering}, the empirical frequency vector for client $k$, denoted by $x_{(i)}^k$, is ordered according to the rank statistic of the class frequencies, \ie  $x_{(i)}^k \geq x_{(i+1)}^k$ for any $i = 1, \dots, C-1$.
The class-adjusted clustering metric $\tilde{s}$ is defined as the standard clustering metric $s$ computed on the ranked frequency vectors $x_{(i)}^k$.  Specifically, the distance between two clients $j$ and $k$ results in
\vspace{-.8em}
\begin{equation}\label{lab_dist_class}
    \dfrac{1}{C}\left(\sum_{i = 1}^C \left | x_{(i)}^k - x_{(i)}^j \right | ^2\right)^{1/2}\,.
\end{equation}
This modification ensures that the clustering evaluation is sensitive to the distributional characteristics of the class imbalance. As we show in Appendix \ref{app:clustering}, this adjustment is mathematically equivalent to assessing the dispersion between the empirical class probability distributions of different clients using the Wasserstein distance, also known as the Kantorovich-Rubenstein metric \citep{kantorovich1942translocation}. This equivalence highlights the theoretical soundness of using ranked class frequencies to better capture variations in class distributions when evaluating clustering outcomes in FL.


\section{Experiments}
In this section, we present the experimental results on widely used FL benchmark datasets \citep{caldas2018leaf} including real-world datasets \citep{hsu2020federated}, comparing the performance of \shortname with other baselines from the literature, including standard FL algorithms and clustering methods.  A detailed description of the implementation settings, datasets and models used for the evaluation are reported in Appendix \ref{app_details}.
\begin{figure}[t]
    \centering
    \includegraphics[width=1.\linewidth]{figures/accuracy_stars.pdf}\vspace{-1.5em}
    \caption{\small{Balanced accuracy on Cifar100 for \shortname (blue curve) with \texttt{FedAvg} aggregation compared to the clustered FL baselines. \shortname detects two splits demonstrating significant improvements in accuracy when clustering is performed, leading also to a faster and more stable convergence than baseline algorithms.}}
    \label{fig:accuracy_jumpcifar100}
    \vspace{-1.3em}
\end{figure}
In Section \ref{exp}, we evaluate our method, \shortname, against various clustering algorithms, including \texttt{CFL} \citep{sattler2020clustered}, \texttt{FeSEM} \citep{long2023multi}, and \texttt{IFCA} \citep{ghosh2020efficient}, and standard FL aggregations \texttt{FedAvg} \citep{mcmahan2017communication}, \texttt{FedAvgM} \citep{asad2020fedopt}, FairAvg \citep{michieli2021all} and \texttt{FedProx} \citep{li2020federated}, showing also that how our approach is orthogonal to conventional FL aggregation methods.

In Section \ref{sec:large_scale}, we underscore that \shortname has the capability to surpass FL methods in real-world and large-scale scenarios \citep{hsu2020federated}.

Finally, in Section \ref{sect:ablation}, we propose analyses on class and domain imbalance, showing that our algorithm successfully detects clients belonging to separate distributions. Further experiments are presented in Appendix \ref{app:other}.

Each client has its own local train and test sets. Algorithm performance is evaluated by averaging the accuracy achieved on the local test sets across the federation, enabling a comparison between FL aggregation and clustered FL approaches (refer to Appendix \ref{app_details} for additional insights). When assessing clustering baselines, we also use the Wasserstein's Adjusted Silhouette Score (WAS) and Wasserstein's Adjusted Davies-Bouldin Score (WADB) to quantify the distributional cohesion among clients, an evaluation performed \textit{a posteriori}. For detection tasks in visual domains (Section \ref{sect:ablation}), we compute the Rand Index \citep{rand1971objective}, a clustering metric that compares the obtained clustering with a ground truth labeling. Further details on the chosen metrics are provided in Appendices \ref{app:metrics_choice} and \ref{app:clustering}.
\subsection{\shortname in heterogeneous settings}\label{exp}
In this section, we analyze the effectiveness of \shortname in mitigating the impact of data heterogeneity compared to standard aggregation methods and other clustered FL algorithms. We conduct experiments on Cifar100 \citep{krizhevsky2009learning} with 100 clients and Femnist \citep{lecun1998mnist} with 400 clients, controlling heterogeneity through a Dirichlet parameter $\alpha$, set to 0.5 for Cifar100 and 0.01 for Femnist, reflecting a realistic class imbalance across clients. Implementation details are provided in Appendix \ref{app_details}.
\begin{table}[t]
    \centering
    \small
        \centering
        \caption{\small{FL baselines in heterogeneous scenarios. Clustering baselines use FedAvg as aggregation mechanism. We emphasize the fact that \shortname and \texttt{CFL} automatically detect the number of clusters, unlike \texttt{IFCA} and \texttt{FeSEM} which require tuning the number of clusters. A higher WAS , denoted by $\uparrow$, and a lower WADB, denoted by $\downarrow$ indicate better clustering outcomes} }
        \label{tab:clustering}
        \begin{adjustbox}{width=\linewidth}
       
        \begin{tabular}{llcccccc}
            \toprule
            & & \makecell{ \textbf{FL} \textbf{method}}& \textbf{C}& \makecell{ \textbf{Automatic} \\ \textbf{Cluster} \\ \textbf{Selection}} & \textbf{Acc} & \textbf{WAS} $\uparrow$ & \textbf{WADB} $\downarrow$ \\
            \midrule
            \multirow{7}{*}{\rotatebox[origin=c]{90}{\textbf{Cifar100} \hspace{.75em}}} & \multirow{4}{*}{\rotatebox[origin=c]{90}{\makecell{Clustered\\ FL}}} &  \texttt{IFCA} &  5 & \ding{55}& 47.5 \scriptsize{$\pm$ 3.5} & -0.8 \scriptsize{$\pm$ 0.2} &  5.2 \scriptsize{$\pm$ 5.1} \\
            & & \texttt{FeSem} & 5 & \ding{55}& 53.4 \scriptsize{$\pm$ 1.8} & -0.3 \scriptsize{$\pm$ 0.1} & 38.4 \scriptsize{$\pm$ 13.0}\\
            & & \texttt{CFL} & 1 &\ding{51}& 41.6 \scriptsize{$\pm$ 1.3} & / & / \\
            & & \shortname & 4 & \ding{51}&\textbf{53.4 \scriptsize{$\pm$ 0.4}} & \textbf{0.1 \scriptsize{$\pm$ 0.0}} & \textbf{2.4 \scriptsize{$\pm$ 0.4}} \\
            \cmidrule{2-8}
            
            & \multirow{3}{*}{\rotatebox[origin=c]{90}{\makecell{Classic\\ FL}}} &  \texttt{FedAvg} &  1 & / &   41.6\scriptsize{$\pm$ 1.3} &  / &  / \\
            & & \texttt{FedAvgM} & 1 & /& 41.5\scriptsize{$\pm$ 0.5}& / & /  \\
            & & \texttt{FedProx} & 1 & /& 41.8\scriptsize{$\pm$ 1.0} & / & / \\
            \midrule
       
            
            \multirow{7}{*}{\rotatebox[origin=c]{90}{\textbf{Femnist} \hspace{.75em}}} & \multirow{4}{*}{\rotatebox[origin=c]{90}{\makecell{Clustered\\ FL}}} &  \texttt{IFCA} &  5 & \ding{55} &  {76.7 \scriptsize{$\pm$ 0.6}} &  \textbf{0.3 \scriptsize{$\pm$ 0.1}} &  \textbf{0.5 \scriptsize{$\pm$ 0.1}} \\
            & & \texttt{FeSem} & 2 & \ding{55}& 75.6 \scriptsize{$\pm$ 0.2} & 0.0 \scriptsize{$\pm$ 0.0} & 25.6 \scriptsize{$\pm$ 7.8}  \\
            & & \texttt{CFL} & 1 & \ding{51}& 76.0 \scriptsize{$\pm$ 0.1} & / & / \\
            & & \shortname & 4 & \ding{51}&76.1 \scriptsize{$\pm$ 0.1} & -0.2 \scriptsize{$\pm$ 0.1} & 18.0 \scriptsize{$\pm$ 6.2}\\
            \cmidrule{2-8}
            & \multirow{3}{*}{\rotatebox[origin=c]{90}{\makecell{Classic\\ FL}}} &  \texttt{FedAvg} &  1 & / &  76.6\scriptsize{$\pm$ 0.1} &  / &  / \\
            & & \texttt{FedAvgM} & 1 & /&  \textbf{83.3}\scriptsize{$\pm$ \textbf{0.3}}& / & /  \\
            & & \texttt{FedProx} & 1 & /& 75.9\scriptsize{$\pm$ 0.2} & / & / \\
            \bottomrule
        \end{tabular}
        \end{adjustbox}
       
      
\end{table}


We compare \shortname against clustered FL baselines (\texttt{IFCA}, \texttt{FeSEM}, \texttt{CFL}) using \texttt{FedAvg} aggregation, as well as standard FL algorithms (\texttt{FedAvg}, \texttt{FedAvgM}, \texttt{FedProx}). For algorithms requiring a predefined number of clusters (\texttt{IFCA}, \texttt{FeSEM}), we report the best result among 2, 3, 4, and 5 clusters, with full tuning details in Appendix \ref{app:tuning}. While \texttt{IFCA} achieves competitive results, its high communication overhead—requiring each client to evaluate models from every cluster in each round—makes it impractical for cross-device FL, serving as an upper bound in our study. \texttt{FeSEM} is more efficient than \texttt{IFCA} but lacks adaptability due to its fixed cluster count. Meanwhile, \texttt{CFL} requires extensive hyperparameter tuning and often produces overly fine-grained clusters or fails to form clusters altogether. In contrast, \shortname requires only one hyperparameter and provides a more practical clustering strategy for cross-device FL.

Table \ref{tab:clustering} presents a comparative analysis of these algorithms in terms of balanced accuracy, WAS, and WADB, using \texttt{FedAvg} as the aggregation method. Higher WAS values indicate better clustering, while lower WADB values suggest better cohesion. On Femnist, clustering-based methods perform worse than standard FL aggregation, but as we move to the more complex and realistic Cifar100 scenario, it becomes evident that clustered FL is necessary to address heterogeneity. In this case, \shortname achieves the best performance in both classification accuracy and clustering quality, with the latter directly influencing the former. The need for clustering grows with increasing heterogeneity, as seen in Table \ref{tab:clustering}: standard FL approaches struggle when trained on a single heterogeneous cluster, whereas clustered FL effectively mitigates the heterogeneity effect. This is particularly relevant for Cifar100, which has a larger number of classes and three-channel images, whereas Femnist consists of grayscale images from only 47 classes.


In Table \ref{tab:clustering}, we present a comparative analysis of these algorithms with respect to balanced accuracy, WAS, and WADB, employing \texttt{FedAvg} as the aggregation strategy. Recall that higher the value of  WAS the better the clustering outcome, as, for WADB, a lower value suggests a better cohesion between clusters. Further details on the metrics used are provided in Appendix \ref{app:metrics_choice}.

Notably, both \shortname and \texttt{CFL} automatically determine the optimal number of clusters based on data heterogeneity, offering a more scalable solution for large-scale cross-device FL. In contrast to \texttt{CFL}, \shortname consistently produced a reasonable number of clusters, even when using the optimal hyperparameters for \texttt{CFL}, which resulted in no splits, thereby achieving performance equivalent to FedAvg. 

 We observe in Figure \ref{fig:accuracy_jumpcifar100} that \shortname exhibits a significant improvement in accuracy on Cifar100 precisely at the rounds where clustering occurs. 

As detailed in Table \ref{tab_app:fl-algs} in Appendix \ref{app:other}, \shortname is orthogonal to any FL aggregation algorithm, \ie any FL method can be easily embedded in \shortname. Our method consistently boosted the performance of FL algorithms for the more heterogeneous settings of Cifar100 and Femnist.
\begin{table*}[t]
\small
\centering
\renewcommand{\arraystretch}{1.5} 
\begin{adjustbox}{width=.7\linewidth}

\begin{tabular}{@{}l|cccccccc@{}}
\toprule
\textbf{Dataset}      & {\shortname} & {\texttt{CFL}}       & \texttt{IFCA} & {FedAvg} & {FedAvgM} & {FedProx} & {FairAvg} \\ \midrule
Google Landmarks      & \textbf{57.4  \scriptsize{$\pm$0.3}} &   40.5  \scriptsize{$\pm$0.2} &  49.4   \scriptsize{$\pm$0.3}    &     40.5  \scriptsize{$\pm$0.2}&  36.4  \scriptsize{$\pm$1.3}  &  40.2  \scriptsize{$\pm$0.6}               &     39.0 \scriptsize{$\pm$0.3}              \\ 
\hline
iNaturalist           & \textbf{47.8  \scriptsize{$\pm$0.2}}   & 45.3  \scriptsize{$\pm$0.1}  &  45.8  \scriptsize{$\pm$0.6}   &     45.3 \scriptsize{$\pm$0.1}            &   37.7  \scriptsize{$\pm$1.4}               &      44.9  \scriptsize{$\pm$0.2}  &      45.1 \scriptsize{$\pm$0.2}            \\ \bottomrule
\end{tabular}
    
\end{adjustbox}
\caption{\small{
Comparison of test accuracy on large scale FL datasets Google Landmarks and iNaturalist, between \shortname and FL baselines -- all clustered FL algorithms use FedAvg aggregation. \shortname outperforms both clustered and standard FL methods detecting 5 and 4 clusters on Landmarks and iNaturalist, respectively.}}\label{tab:largescale}\vspace{-1.5em}
\end{table*}


\subsection{\shortname in Large Scale and Real World Scenarios}\label{sec:large_scale}

We evaluate \shortname on two large-scale, real-world datasets: Google Landmarks \citep{weyand2020google} and iNaturalist \citep{van2018inaturalist}, respectively considering the partitions Landmarks-Users-160K, and iNaturalist-Users-120K, proposed in \citep{hsu2020federated}. Both datasets exhibit high data heterogeneity and involve a large number of clients - approximately 800 for Landmarks and 2,700 for iNaturalist. To simulate a realistic cross-device scenario, we set 10 participating clients per round.
For this experiment, we compare \shortname against clustered FL baselines (\texttt{IFCA}, \texttt{CFL}) and standard FL aggregation methods (\texttt{FedAvg}, \texttt{FedAvgM}, \texttt{FedProx}, \texttt{FairAvg}). The number of clusters for \texttt{IFCA} is tuned between 2 and 5. Due to its high computational cost in large-scale settings, \texttt{FeSEM} was not included in this analysis.
Table \ref{tab:largescale} reports the results: \shortname with \texttt{FedAvg} aggregation achieves 57.4\% accuracy on Landmarks, significantly outperforming all standard FL methods. In this scenario, \shortname detects 5 clusters with the best hyperparameter setting ($\beta = 0.5$), while \texttt{IFCA} identifies 3 clusters.
Similarly, on iNaturalist, \shortname consistently surpasses FL baselines, reaching an average accuracy of 47.8\% with $\beta = 0.5$ (automatically detecting a partition with 4 clusters). Results in Table \ref{tab:largescale} remark that, when dealing with realistic complex decentralized scenarios, standard aggregation methods are not able to mitigate the effects of heterogeneity across the federation, while, on the other hand, clustered FL provides a more efficient solution.
\vspace{-1em}
\subsection{Clustering analysis of \shortname}\label{sect:ablation}
In this section, we investigate the underlying mechanisms behind \shortname’s clustering in heterogeneous scenarios on Cifar100. Further experiments on Cifar10 are presented in Appendix \ref{app:cifar10}.
\vspace{-1.3em}
\paragraph{\shortname detects different client class distributions}
We explore how the algorithm identifies and groups clients based on the non-IID nature of their data distributions, represented by the Dirichlet concentration parameter $\alpha$. For the Cifar100 dataset, we apply a similar splitting approach, obtaining the following partitions: (1) 90 clients with $\alpha = 0$ and 10 clients with $\alpha = 1000$; (2) 90 clients with $\alpha = 0.5$ and 10 clients with $\alpha = 1000$; and (3) 40 clients with $\alpha = 1000$, 30 clients with $\alpha = 0.05$, and 30 clients with $\alpha = 0$. We evaluate the outcome of this clustering experiment by means of WAS and WADB. Results in Table \ref{tab:ablation1_heter} show that \shortname detects clusters groups clients according to the level of heterogeneity of the group.
\begin{table}[h]
    
    \caption{\small{Clustering with three different splits on Cifar100 datasets. \shortname has superior clustering quality across different splits (homogeneous $\alpha = 1000$, heterogeneous $\alpha = 0.05$, extremely heterogeneous $\alpha = 0$}. )}
    \centering
    \small
    \begin{adjustbox}{width=\linewidth}
        \label{tab:ablation1_heter}
     
        \begin{tabular}{lccccc}
            \toprule
            \textbf{Dataset} & \textbf{(Hom, Het, X Het)} & \makecell{\textbf{Clustering} \\ \textbf{method}} & \textbf{C} & \textbf{WAS}$\uparrow$ & \textbf{WADB} $\downarrow$ \\
            \cmidrule(lr){1-6}
        
            \multirow{9}{*}{Cifar100} 
            & \multirow{3}{*}{(10, 0, 90)} & \texttt{IFCA} & 5 & -0.9 \scriptsize{$\pm$ 0.0} & 1.8 \scriptsize{$\pm$ 0.0} \\
            & & \texttt{FeSem} & 5 & -0.8 \scriptsize{$\pm$ 0.2} & 2.6 \scriptsize{$\pm$ 0.6} \\
            & & \shortname & 5 & \textbf{0.1 \scriptsize{$\pm$ 0.1}} & \textbf{0.2 \scriptsize{$\pm$ 0.2}} \\
            \cmidrule(lr){2-6}
            & \multirow{3}{*}{(10, 90, 0)} & \texttt{IFCA} & 5 & -0.0 \scriptsize{$\pm$ 0.0} & \textbf{5.6 \scriptsize{$\pm$ 1.5}} \\
            & & \texttt{FeSem} & 5 & 0.2 \scriptsize{$\pm$ 0.1} & 12.0 \scriptsize{$\pm$ 2.0} \\
            & & \shortname & 5 & \textbf{0.4 \scriptsize{$\pm$ 0.1}} & 6.4 \scriptsize{$\pm$ 2.0} \\
            \cmidrule(lr){2-6}
            
            & \multirow{3}{*}{(40, 30, 30)} & \texttt{IFCA} & 5 & -0.2 \scriptsize{$\pm$ 0.0} & 1.0 \scriptsize{$\pm$ 0.0}\\
            & & \texttt{FeSem} & 5 & -0.2 \scriptsize{$\pm$ 0.0} & 33.2 \scriptsize{$\pm$ 0.0} \\
            & & \shortname & 3 & \textbf{0.4 \scriptsize{$\pm$ 0.2}} & \textbf{0.9 \scriptsize{$\pm$ 0.1}} \\
            \bottomrule
        \end{tabular}
    \end{adjustbox}
    
\end{table}
\vspace{-1.5em}
\paragraph{\shortname detects different visual client domains.}
Here, we focus on scenarios with nearly uniform class imbalance (high $\alpha$ values) but with different visual domains to investigate how \shortname forms clusters in such settings. We incorporated various artificial domains (non-perturbed, noisy, and blurred images) Cifar100 dataset under homogeneous conditions ($\alpha=100.00$). Our results demonstrate that \shortname effectively clustered clients according to these distinct domains. Table \ref{tab:dom_abl} presents the Rand-Index scores, which assess clustering quality based on known domain labels. The high Rand-Index scores, often approaching the upper bound of 1, indicate that \shortname successfully separated clients into distinct clusters corresponding to their respective domains. %
This anaylsis suggests that \shortname may be applicable for detecting malicious clients in FL, pinpointing a potential direction for future research.
\begin{table}[t]
    
    \caption{\small Clustering performance of \shortname is assessed on federations with clients from varied domains using clean, noisy, and blurred (Clean, Noise, Blur) images from Cifar100 datasets. It utilizes the Rand Index score \citep{rand1971objective}, where a value close to 1 represents a perfect match between clustering and labels. Consistently \shortname accurately distinguishes all visual domains. The ground truth number of clusters is respectively 2, 2, and 3.}
    \label{tab:dom_abl}
    
    \centering
    
    \begin{adjustbox}{width=\linewidth}
    \setlength{\tabcolsep}{9pt}
        \begin{tabular}{ccccccc}
           \toprule
            \textbf{Dataset} & \textbf{(Clean, Noise, Blur)} & \makecell{\textbf{Clustering} \\ \textbf{method}} & \textbf{C} &\textbf{\makecell{Automatic\\Cluster \\Selection}}& \makecell{\textbf{Rand} $\uparrow$ \\ \textbf{(max = 1.0)}}  \\
            \cmidrule(lr){1-6}
            
            \multirow{9}{*}{Cifar100} 
            & \multirow{3}{*}{(50, 50, 0)} & \texttt{IFCA} & 1  & \ding{55}& 0.5 \scriptsize{$\pm$ 0.0}  \\
            & & \texttt{FeSem} & 2 &\ding{55}& 0.49 \scriptsize{$\pm$ 0.2}  \\
            & & \shortname & 2 & \ding{51} &\textbf{1.0 \scriptsize{$\pm$ 0.0}} \\
            \cmidrule(lr){2-6}
            & \multirow{3}{*}{(50, 0, 50)} & \texttt{IFCA} & 1 & \ding{55}& 0.5 \scriptsize{$\pm$ 0.0} \\
            & & \texttt{FeSem} & 2 &\ding{55}& 0.51 \scriptsize{$\pm$ 0.1} \\
            & & \shortname & 2 & \ding{51}&\textbf{1.0 \scriptsize{$\pm$ 0.0}} \\
            \cmidrule(lr){2-6}
            
            & \multirow{3}{*}{(40, 30, 30)} & \texttt{IFCA} & 1 &\ding{55} & 0.33 \scriptsize{$\pm$ 0.0}\\
            & & \texttt{FeSem} & 3 &\ding{55} & 0.55 \scriptsize{$\pm$ 0.0}\\
            & & \shortname & 4 & \ding{51} &\textbf{0.6 \scriptsize{$\pm$ 0.0}} \\
            \bottomrule
        \end{tabular}
    \end{adjustbox}
\end{table}



\vspace{-0.2cm}
\section{Impact: Why Free Scientific Knowledge?}
\vspace{-0.1cm}

Historically, making knowledge widely available has driven transformative progress. Gutenberg’s printing press broke medieval monopolies on information, increasing literacy and contributing to the Renaissance and Scientific Revolution. In today's world, open source projects such as GNU/Linux and Wikipedia show that freely accessible and modifiable knowledge fosters innovation while ensuring creators are credited through copyleft licenses. These examples highlight a key idea: \textit{access to essential knowledge supports overall advancement.} 

This aligns with the arguments made by Prabhakaran et al. \cite{humanrightsbasedapproachresponsible}, who specifically highlight the \textbf{ human right to participate in scientific advancement} as enshrined in the Universal Declaration of Human Rights. They emphasize that this right underscores the importance of \textit{ equal access to the benefits of scientific progress for all}, a principle directly supported by our proposal for Knowledge Units. The UN Special Rapporteur on Cultural Rights further reinforces this, advocating for the expansion of copyright exceptions to broaden access to scientific knowledge as a crucial component of the right to science and culture \cite{scienceright}. 

However, current intellectual property regimes often create ``patently unfair" barriers to this knowledge, preventing innovation and access, especially in areas critical to human rights, as Hale compellingly argues \cite{patentlyunfair}. Finding a solution requires carefully balancing the imperative of open access with the legitimate rights of authors. As Austin and Ginsburg remind us, authors' rights are also human rights, necessitating robust protection \cite{authorhumanrights}. Shareable knowledge entities like Knowledge Units offer a potential mechanism to achieve this delicate balance in the scientific domain, enabling wider dissemination of research findings while respecting authors' fundamental rights.

\vspace{-0.2cm}
\subsection{Impact Across Sectors}

\textbf{Researchers:} Collaboration across different fields becomes easier when knowledge is shared openly. For instance, combining machine learning with biology or applying quantum principles to cryptography can lead to important breakthroughs. Removing copyright restrictions allows researchers to freely use data and methods, speeding up discoveries while respecting original contributions.

\textbf{Practitioners:} Professionals, especially in healthcare, benefit from immediate access to the latest research. Quick access to newer insights on the effectiveness of drugs, and alternative treatments speeds up adoption and awareness, potentially saving lives. Additionally, open knowledge helps developing countries gain access to health innovations.

\textbf{Education:} Education becomes more accessible when teachers use the latest research to create up-to-date curricula without prohibitive costs. Students can access high-quality research materials and use LM assistance to better understand complex topics, enhancing their learning experience and making high-quality education more accessible.

\textbf{Public Trust:} When information is transparent and accessible, the public can better understand and trust decision-making processes. Open access to government policies and industry practices allows people to review and verify information, helping to reduce misinformation. This transparency encourages critical thinking and builds trust in scientific and governmental institutions.

Overall, making scientific knowledge accessible supports global fairness. By viewing knowledge as a common resource rather than a product to be sold, we can speed up innovation, encourage critical thinking, and empower communities to address important challenges.

\vspace{-0.2cm}
\section{Open Problems}
\vspace{-0.1cm}

Moving forward, we identify key research directions to further exploit the potential of converting original texts into shareable knowledge entities such as demonstrated by the conversion into Knowledge Units in this work:


\textbf{1. Enhancing Factual Accuracy and Reliability:}  Refining KUs through cross-referencing with source texts and incorporating community-driven correction mechanisms, similar to Wikipedia, can minimize hallucinations and ensure the long-term accuracy of knowledge-based datasets at scale.

\textbf{2. Developing Applications for Education and Research:}  Using KU-based conversion for datasets to be employed in practical tools, such as search interfaces and learning platforms, can ensure rapid dissemination of any new knowledge into shareable downstream resources, significantly improving the accessibility, spread, and impact of KUs.

\textbf{3. Establishing Standards for Knowledge Interoperability and Reuse:}  Future research should focus on defining standardized formats for entities like KU and knowledge graph layouts \citep{lenat1990cyc}. These standards are essential to unlock seamless interoperability, facilitate reuse across diverse platforms, and foster a vibrant ecosystem of open scientific knowledge. 

\textbf{4. Interconnecting Shareable Knowledge for Scientific Workflow Assistance and Automation:} There might be further potential in constructing a semantic web that interconnects publicly shared knowledge, together with mechanisms that continually update and validate all shareable knowledge units. This can be starting point for a platform that uses all collected knowledge to assist scientific workflows, for instance by feeding such a semantic web into recently developed reasoning models equipped with retrieval augmented generation. Such assistance could assemble knowledge across multiple scientific papers, guiding scientists more efficiently through vast research landscapes. Given further progress in model capabilities, validation, self-repair and evolving new knowledge from already existing vast collection in the semantic web can lead to automation of scientific discovery, assuming that knowledge data in the semantic web can be freely shared.

We open-source our code and encourage collaboration to improve extraction pipelines, enhance Knowledge Unit capabilities, and expand coverage to additional fields.

\vspace{-0.2cm}
\section{Conclusion}
\vspace{-0.1cm}

In this paper, we highlight the potential of systematically separating factual scientific knowledge from protected artistic or stylistic expression. By representing scientific insights as structured facts and relationships, prototypes like Knowledge Units (KUs) offer a pathway to broaden access to scientific knowledge without infringing copyright, aligning with legal principles like German \S 24(1) UrhG and U.S. fair use standards. Extensive testing across a range of domains and models shows evidence that Knowledge Units (KUs) can feasibly retain core information. These findings offer a promising way forward for openly disseminating scientific information while respecting copyright constraints.

\section*{Author Contributions}

Christoph conceived the project and led organization. Christoph and Gollam led all the experiments. Nick and Huu led the legal aspects. Tawsif led the data collection. Ameya and Andreas led the manuscript writing. Ludwig, Sören, Robert, Jenia and Matthias provided feedback. advice and scientific supervision throughout the project. 

\section*{Acknowledgements}

The authors would like to thank (in alphabetical order): Sebastian Dziadzio, Kristof Meding, Tea Mustać, Shantanu Prabhat for insightful feedback and suggestions. Special thanks to Andrej Radonjic for help in scaling up data collection. GR and SA acknowledge financial support by the German Research Foundation (DFG) for the NFDI4DataScience Initiative (project number 460234259). AP and MB acknowledge financial support by the Federal Ministry of Education and Research (BMBF), FKZ: 011524085B and Open Philanthropy Foundation funded by the Good Ventures Foundation. AH acknowledges financial support by the Federal Ministry of Education and Research (BMBF), FKZ: 01IS24079A and the Carl Zeiss Foundation through the project "Certification and Foundations of Safe ML Systems" as well as the support from the International Max Planck Research School for Intelligent Systems (IMPRS-IS). JJ acknowledges funding by the Federal Ministry of Education and Research of Germany (BMBF) under grant no. 01IS22094B (WestAI - AI Service Center West), under grant no. 01IS24085C (OPENHAFM) and under the grant DE002571 (MINERVA), as well as co-funding by EU from EuroHPC Joint Undertaking programm under grant no. 101182737 (MINERVA) and from Digital Europe Programme under grant no. 101195233 (openEuroLLM) 

\section*{Impact Statement}
\shortname is an efficient clustering-based approach that improves personalization while reducing communication and computation costs, enhancing the advance in the field of Federated Learning. Providing efficiency and explainability in clustering decisions, \shortname enables more interpretable and scalable federated learning. Its efficiency makes it well-suited for IoT, decentralized AI, and sustainable AI applications, particularly in privacy-sensitive domains.
\section{Acknowledgements}


\bibliography{reference}
\bibliographystyle{icml2025}


\newpage
\appendix
\onecolumn
\section{Theoretical Results for \shortname}\label{app:fgw}

This section provides algorithms, in pseudo-code, to describe \shortname (see Algorithms \ref{alg:fedgwcluster} and \ref{alg:fedgw_recursion}). Additionally, here we provide the proofs for the convergence results introduced in Section \ref{sec:theory}, specifically addressing the convergence (Theorems \ref{thm_main:1} and \ref{thm_main:weak_conv}) and the formal derivation on the variance bound of the Gaussian weights (Proposition \ref{prop_var_main}). \textcolor{black}{In addition, we also present a sufficient condition, under which is guaranteed that the overall sampling rate of the training algorithm does not increase and remain unchanged during the training process (Theorem \ref{thm:samplerate}).}

\begin{theorem}\label{thm:1}
Let $\{\alpha_t\}_{t = 1}^\infty$ be a sequence of positive real values, and $\{\Gamma_k^t\}_{t=1}^\infty$ the sequence of Gaussian weights. If $\{\alpha_t\}_{t = 1}^\infty \in l^2(\mathbb{N})/l^1(\mathbb{N})$, then $\Gamma_k^t$ converges in $L^2$. Furthermore, for $t\to\infty$, 
\begin{equation}
    \Gamma_k^t \longrightarrow \mu_k\,\,\, a.s.
\end{equation}
\end{theorem}
\begin{proof}
At each communication round, we compute the samples $r_i^{t,s}$ from $R_k^{t,s}$ via a Gaussian transformation of the observed loss in Eq. \ref{eq:reward}. Notice that, due to the linearity of the expectation operator, $\mathbb{E}[\Omega_k^t] = \mu_k$, that is the true, unknown, expected reward. The observed value for the random variable is given by $\omega_k^t = 1/S \sum_{s = 1}^S r_k^{t,s}$, which is sampled from a distribution centered on $\mu_k$.
Each client's weight is updated according to
\begin{equation}\label{weight_formula}
    \gamma_k^{t+1} = (1-\alpha_t)\gamma_t + \alpha_t \omega_k^t\,.
\end{equation}
Since such an estimator follows a Robbins-Monro algorithm, it is proved to converge in $L^2$. In addition, $\Gamma_k^t$ converges to the expectation $\mathbb{E}[\Omega_k^t] = \mu_k$ with probability 1, provided that $\alpha_t$ satisfies $\sum_{t\geq 1}|\alpha_t| = \infty$, and $\sum_{t\geq 1}|\alpha_t|^2 < \infty$ \citep{harold1997stochastic}.
\end{proof}

\begin{theorem}\label{thm:weak_conv}
Let $\alpha \in (0,1)$ be a fixed constant, then in the limit $t \to \infty$, the expectation of the weights converges to the individual theoretical reward $\mu_k$, for each client $k = 1,\dots, K$, \ie,
\begin{equation}
    \mathbb{E}[\Gamma_k^t]\longrightarrow \mu_k\,\,\,t\to\infty\,.
\end{equation}
\end{theorem}
\begin{proof}
Recall that $\gamma_k^{t+1} = (1-\alpha)\gamma_k^t + \alpha \omega_k^t$, where $\omega_k^t$ are samples from $\Omega_k^t$. If we substitute backward the value of $\gamma_k^t$ we can write
\begin{equation}
\gamma_k^{t+1} = (1-\alpha)^2\gamma_k^{t-1} + \alpha \omega_k^t + \alpha(1-\alpha)\omega_k^{t-1}\,.
\end{equation}
By iterating up to the initialization term $\gamma_k^0$ we get the following formulation: 
\begin{equation}\label{explicit}
    \gamma_k^{t+1} = (1-\alpha)^{t+1} \gamma_k^0+ \sum_{\tau = 0}^t \alpha (1-\alpha)^\tau \omega_k^{t-\tau}\,\,.
\end{equation}
Since $\omega_k^t$ are independent and identically distributed samples from $\Omega_k^t$, with expected value $\mu_k$, then the expectation of the weight at the $t$-th communication round would be
\begin{equation}
    \mathbb{E}[\Gamma_k^{t}] = \mathbb{E}\left[(1-\alpha)^{t}\gamma_k^0 + \sum_{\tau = 0}^t \alpha (1-\alpha)^\tau \Omega_k^{t-\tau-1}\right ]\,\,,
\end{equation}
that, due to the linearity of expectation, becomes
\begin{equation}
    \mathbb{E}[\Gamma_k^{t}] = (1-\alpha)^{t}\gamma_k^0 + \sum_{\tau = 0}^t \alpha (1-\alpha)^\tau \mu_k\,\,.
\end{equation}
If we compute the limit
\begin{equation}
    \lim_{t \to \infty}\mathbb{E}[\Gamma_k^{t}] =\lim_{t\to\infty}(1-\alpha)^{t}\gamma_k^0 + \sum_{\tau = 0}^\infty \alpha (1-\alpha)^\tau \mu_k\,\,,
\end{equation}
and since $\alpha\in(0,1)$, the first term tends to zero, and also the geometric series converges. Therefore, the expectation of the weights converges to $\mu_k$, namely
\begin{equation}
    \lim_{t \to \infty} \mathbb{E}[\Gamma_k^t] = \mu_k\,.
\end{equation}
\end{proof}

\begin{proposition}\label{prop_var}
The variance of the weights $\Gamma_k^t$ is smaller than the variance $\sigma_k^2$ of the theoretical rewards $R_k^{t,s}$.
\end{proposition} 
\begin{proof}
From Eq.\ref{explicit}, we can show that $\mathbb{V}ar(\Gamma_k^t)$ converges to a value that depends on $\alpha$ and the number of local training iterations $S$. Indeed
\begin{equation}
\begin{split}
\mathbb{V}ar(\Gamma_k^t) &= \mathbb{V}ar\left( (1-\alpha)^{t} \gamma_k^0 + \sum_{\tau = 0}^t \alpha (1-\alpha)^\tau \Omega_k^{t-\tau-1}\right) \\
&= \sum_{\tau = 0}^t \alpha^2 (1-\alpha)^{2\tau} \mathbb{V}ar(\Omega_k^t) = \dfrac{1}{S}\sum_{\tau = 0}^t \alpha^2 (1-\alpha)^{2\tau}\sigma_k^2
\end{split}
\end{equation}
since $\Omega_k^t = 1/S \sum_{s = 1}^S R_k^{t,s}$\,.

If we compute the limit, that exists finite due to the hypothesis $\alpha \in (0,1)$, we get
\begin{equation}
\lim_{t \to \infty}\mathbb{V}ar(\Gamma_k^t) = \dfrac{\alpha^2\sigma_k^2}{S}\sum_{\tau = 0}^\infty (1-\alpha)^{2\tau} = \dfrac{\alpha}{2-\alpha} \dfrac{\sigma_k^2}{S} <\dfrac{\sigma_k^2}{S}<\sigma_k^2\,\,.
\end{equation}
\end{proof}
We further demonstrate that the interaction matrix $P^t$ identified by \shortname is entry-wise bounded from above, as established in the following proposition.
\begin{proposition}\label{prop:bounded_matrix}
The entries of the interaction matrix $P^t$ are bounded from above, namely for any $t \geq 0$ there exists a positive finite constant $C_t > 0$ such that
\begin{equation}
    P_{kj}^t \leq C_t\,\,.
\end{equation}
And furthermore
\begin{equation}
    \lim_{t \to \infty} C_t = 1\,\,.
\end{equation}
\end{proposition}
\begin{proof}
Without loss of generality we assume that every client of the federation is sampled, and we assume that $\alpha_t = \alpha \in (0,1)$ for any $t \geq 0$. We recall, from Eq.\ref{inter_matrix}, that for any couple of clients $k,j \in \mathcal{P}_t$ the entries of the interaction matrix are updated according to 
\begin{equation}
    P_{kj}^{t+1} = (1-\alpha) P_{kj}^t + \alpha \omega_k^t\,.
\end{equation}
If we iterate backward until $P_{kj}^0$, we obtain the following update
\begin{equation}
     P_{kj}^{t+1} = (1-\alpha)^{t+1} P_{kj}^{0}+ \sum_{\tau = 0}^t \alpha (1-\alpha)^\tau \omega_k^{t-\tau}\,\,.
\end{equation}
We know that, by constructions, the Gaussian rewards $\omega_k^t < 1$  at any time $t$, therefore the following inequality holds
\begin{equation}
    P_{kj}^{t} = (1-\alpha)^{t} P_{kj}^{0}+ \sum_{\tau = 0}^t \alpha (1-\alpha)^\tau \omega_k^{t-\tau-1} \leq (1-\alpha)^{t} P_{kj}^{0}+ \sum_{\tau = 0}^t \alpha (1-\alpha)^\tau\,.
\end{equation}
At any round $t$ we can define the constant $C_t$, as
\begin{equation}
    C_t := (1-\alpha)^t P_{kj}^0 + \alpha \sum_{\tau = 0}^t(1-\alpha)^\tau = (1-\alpha)^t P_{kj}^0 + 1 -(1-\alpha)^{t+1} < \infty\,.
\end{equation}
Moreover, since $\alpha \in (0,1)$, by taking the limit we prove that 
\begin{equation}
    \lim_{t \to \infty} C_t = \lim_{t \to \infty} (1-\alpha)^t P_{kj}^0 + 1 -(1-\alpha)^{t+1} = 1\,.
\end{equation}
\end{proof}

\begin{theorem}{(Sufficient Condition for Sample Rate Conservation)}\label{thm:samplerate} Consider $K_{min}$ as the minimum number of clients permitted per cluster, \ie the cardinality $|\mathcal{C}_n| \geq K_{min}$ for any given cluster $n = 1,\dots, n_{cl}$, and $\rho \in (0,1]$ to represent the initial sample rate. There exists a critical threshold $n^* > 0$ such that, if $K_{min} \geq n^*$ is met, the total sample size does not increase.
\end{theorem}
\begin{proof}
Let us denote by $\rho_n$ the participation rate relative to the $n$-th cluster, \ie
\begin{equation}\label{eq:rho^n}
    \rho_n = \max \left\{\rho, \dfrac{3}{|\mathcal{C}_n|}\right\}
\end{equation}
because, in order to maintain privacy of the clients' information we need to sample at least three clients, therefore $\rho^n$ is at least $3$ over the number of clients belonging to the cluster. The total participation rate at the end of the clustering process is given by
\begin{equation}
    \rho^{\text{global}} = \sum_{n = 1}^{n_{cl}} \dfrac{K_n}{K}
\end{equation}
where $K_n$ denotes the number of clients sampled within the $n$-th cluster. If we focus on the term $K_n$, recalling Equation \ref{eq:rho^n}, we have that
\begin{equation}\label{eq:K_n}
    K_n = \rho_n |\mathcal{C}_n| = \max \left\{\rho, \dfrac{3}{|\mathcal{C}_n|}\right\}\times|\mathcal{C}_n| = \max\{\rho |\mathcal{C}_n|, 3\}\,\,.
\end{equation}
If we write Equation \ref{eq:K_n}, by the means of the positive part function, denoted by $(x)^+ = \max\{0,x\}$, we obtain that
\begin{equation}
    K_n = 3 + \max\{0, \rho |\mathcal{C}_n| - 3\} = 3 + (\rho |\mathcal{C}_n| - 3)^+\,\,.
\end{equation}
Observe that we are looking for a threshold value for which $\rho^{\text{global}} = \rho$, \ie the participation rate remains the same during the whole training process.\\
Let us observe that $K_n = \rho |\mathcal{C}_n| \iff \rho |\mathcal{C}_n| \geq 3 \iff |\mathcal{C}_n| \geq n^* = 3/\rho$. In fact, if we assume that $K_{min} \geq n^*$, then the following chain of equalities holds
\begin{equation*}
    \rho^{\text{global}} = \sum_{n = 1}^{n_{cl}} \dfrac{K_n}{K} = \dfrac{1}{K} \sum_{n = 1}^{n_cl} \rho|\mathcal{C}_{n}| = \dfrac{\rho}{K} \sum_{n = 1}^{n_{cl}}|\mathcal{C}_n| = \dfrac{\rho K}{K} = \rho
\end{equation*}
thus proving that $K_{min} \geq n^*$ is a sufficient condition for not increasing the sampling rate during the training process.
\end{proof}
\begin{algorithm}[t]
\caption{\texttt{FedGW\_Cluster}}\label{alg:fedgwcluster}
   \begin{algorithmic}[1]
     \STATE \textbf{Input:} $P, n_{max}, \mathcal{K}(\cdot,\cdot)$
     \STATE \textbf{Output:} cluster labels $y_{n_{cl}}$, and number of clusters $n_{cl}$
     \STATE Extract UPVs $v_k^j, v_j^k$ from $P$ for any $k,j$
     \STATE $W_{kj}\gets \mathcal{K}(v_k^j,v_j^k)$ for any $k,j$
     \FOR{$n = 2,\dots, n_{max}$}
     \STATE $y_{n} \gets \texttt{Spectral\_Clustering}(W,n)$
     \STATE $DB_n \gets \texttt{Davies\_Bouldin}(W,y_n)$
     \IF{$\min_n DB_n > 1$}
     \STATE $n_{cl} \gets 1$
     \ELSE 
     \STATE$n_{cl} \gets \arg \min_n DB_n$
     \ENDIF
    
    \ENDFOR
   \end{algorithmic}
\end{algorithm}

    
\begin{algorithm}[t]
  \caption{\shortname}\label{alg:fedgw_recursion}
  \begin{algorithmic}[1]
    \STATE \textbf{Input:} $K, T, S, \alpha_t, \epsilon, |\mathcal{P}_t|, \mathcal{K}$
    \STATE \textbf{Output:} $\mathcal{C}^{(1)},\dots, \mathcal{C}^{(N_{cl})}$ and $\theta_{(1)}, \dots, \theta_{(N_{cl})}$ 
    \STATE Initialize $N_{cl}^0\gets 1$
    \vspace{.1cm}
    \STATE Initialize $P^{0}_{(1)} \gets 0_{K\times K}$
    \STATE Initialize $\textrm{MSE}^{0}_{(1)} \gets 1$
    \vspace{.1cm}
    \FOR{$t = 0,\dots,T-1$}
    \STATE $\Delta N^t \gets 0$ for each iterations it counts the number of new clusters that are detected
    \vspace{.1cm}
    \FOR{$n = 1,\dots, N_{cl}^t$}
    
    \STATE Server samples $\mathcal{P}_t^{(n)} \in \mathcal{C}^{(n)}$ and sends the current cluster model $\theta_{(n)}^t$
    \STATE Each client $k \in \mathcal{P}_t^{(n)}$ locally updates $\theta_k^t$ and $l_k^t$, then sends them to the server
    \STATE $\omega_k^t \gets \texttt{Gaussian\_Rewards}(l_k^t, \mathcal{P}_t^{(n)})$, Eq. \ref{eq:reward}
    \STATE $\theta_{(n)}^{t+1}\gets \texttt{FL\_Aggregator}(\theta_k^t, \mathcal{P}_t^{(n)})$
    \STATE $P^{t+1}_{(n)}\gets\texttt{Update\_Matrix}(P^t_{(n)}, \omega_k^t, \alpha_t, \mathcal{P}_t^{(n)})$, according to Eq. \ref{inter_matrix}
    \vspace{.1cm}
    \STATE Update $\textrm{MSE}_{(n)}^{t+1}$
    \vspace{.1cm}
    \IF{$\textrm{MSE}^{t+1}_{(n)} < \epsilon$}
    \vspace{.1cm}
    \STATE Perform $\texttt{FedGW\_Cluster}(P_{(n)}^{t+1}, n_{max}, \mathcal{K})$ on $\mathcal{C}^{(n)}$, providing $n_{cl}$ sub-clusters
    \vspace{.1cm}
    \STATE Update the number of new clusters $\Delta N^t \gets \Delta N^t + n_{cl} -1 $ u
    \vspace{.1cm}
    \STATE Cluster server splits $P_{(n)}^{t+1}$ filtering rows and columns according to the new clusters
    \vspace{.1cm}
    \STATE Re-initialize MSE for new clusters to $1$
    \vspace{.1cm}
    \ENDIF
    \ENDFOR
    \STATE Update the total number of clusters$N_{cl}^{t+1} \gets N_{cl}^t + \Delta N^t$ 
    \ENDFOR
  \end{algorithmic}
\end{algorithm}
\newpage

\section{Theoretical Derivation of the Wasserstein Adjusted Score} \label{app:clustering}
\textcolor{black}{To address the lack of clustering evaluation metrics suited for FL with distributional heterogeneity and class imbalance, we introduced a theoretically grounded adjustment to standard metrics, derived from the Wasserstein distance, Kantorovich–Rubinstein metric \citep{kantorovich1942translocation}. This metric, integrated with popular scores like Silhouette and Davies-Bouldin, enables a modular framework for a posteriori evaluation, effectively comparing clustering outcomes across federated algorithms.}
In this paragraph, we show how the proposed clustering metric that accounts for class imbalance can be derived from a probabilistic interpretation of clustering. 
\begin{definition}
    Let $(M,d)$ be a metric space, and $p \in [1,\infty]$. The Wasserstein distance between two probability measures $\mathbb{P}$ and $\mathbb{Q}$ over $M$ is defined as
    \begin{equation}\label{eq:wass}
        W_p(\mathbb{P}, \mathbb{Q}) =  \inf_{\gamma \in \Gamma(\mathbb{P}, \mathbb{Q})} \mathbb{E}_{(x,y)\sim \gamma}[d(x,y)^p]^{1/p}
    \end{equation}
where $\Gamma(\mathbb{P}, \mathbb{Q})$ is  the set of all the possible couplings of $\mathbb{P}$ and $\mathbb{Q}$ (see Def. \ref{couplings}).
\end{definition}
Furthermore, we need to introduce the notion of coupling of two probability measures.
\begin{definition}\label{couplings}
Let $(M,d)$ be a metric space, and $\mathbb{P}, \mathbb{Q}$ two probability measures over $M$. A coupling $\gamma$ of $\mathbb{P}$ and $\mathbb{Q}$ is a joint probability measure on $M \times M$ such that, for any measurable subset $A \subset M$,
\begin{equation}\label{eq:coupling}
\begin{split}
    \int_A \left(\int_M \gamma(dx, dy) \mathbb{Q}(dy)\right) \mathbb{P}(dx) = \mathbb{P}(A), \\
    \int_A \left(\int_M \gamma(dx, dy) \mathbb{P}(dx)\right) \mathbb{Q}(dy) = \mathbb{Q}(A).
\end{split}
\end{equation}
\end{definition}
Let us recall that the empirical measure over $M$ of a sample of observations $\{x_1, \cdots, x_N\}$ is defined such that for any measurable  set $A \subset M$
\begin{equation}\label{eq:emp_measure}
    \mathbb{P}(A) = \dfrac{1}{N}\sum_{i = 1}^C\delta_{x_i}(A) 
\end{equation}
where $\delta_{x_i}$ is  the Dirac's measure concentrated on the data point $x_i$.\\
In particular, we aim to measure the goodness of a cluster by taking into account the distance between the empirical frequencies between two clients' class distributions and use that to properly adjust the clustering metric. For the sake of simplicity, we assume that the distance $d$ over $M$ is the $L^2$-norm. We obtain the following theoretical result to justify the rationale behind our proposed metric.
\begin{theorem}
    Let $s$ be an arbitrary clustering score. Then, the class-imbalance adjusted score $\tilde{s}$ is exactly the metric $s$ computed with the Wasserstein distance between the empirical measures over each client's class distribution.
\end{theorem}
\begin{proof}
Let us consider two clients; each one has its own sample of observations $\{x_1, \dots, x_C\}$ and $\{y_1, \dots, y_C\}$ where the $i$-th position corresponds to the frequency of training points of class $i$ for each client. We aim to compute the $p$-Wasserstein distance between the empirical measures $\mathbb{P}$ and $\mathbb{Q}$ of the two clients, in particular for any $dx, dy > 0$
\begin{equation}
\begin{split}
    \mathbb{P}(dx) &= \dfrac{1}{N} \sum_{i = 1}^N \delta_{x_i}(dx), \\
    \mathbb{Q}(dy) &= \dfrac{1}{N} \sum_{i = 1}^N \delta_{y_i}(dy)\,\,\,.
\end{split}
\end{equation}
In order to compute $W_p^p(\mathbb{P}, \mathbb{Q})$ we need to carefully investigate the set of all possible coupling measures $\Gamma(\mathbb{P}, \mathbb{Q})$. However, since either $\mathbb{P}$ and $\mathbb{Q}$ are concentrated over countable sets, it is possible to see that the only possible couplings satisfying Eq. \ref{eq:coupling} are the Dirac's measures over all the possible permutations of $x_i$ and $y_i$. In particular, by fixing the ordering of $x_i$, according to the rank statistic $x_{(i)}$, the coupling set can be written as
\begin{equation}
    \Gamma(\mathbb{P}, \mathbb{Q}) = \left\{\dfrac{1}{C} \delta_{(x_{(i)}, y_{\pi(i)})}: \pi \in \mathcal{S}\right\}
\end{equation}
where $\mathcal{S}$  is the set of all possible permutations of $C$ elements. Therefore we could write Eq. \ref{eq:wass} as follows
\begin{equation}
    W_p^p = \min_{\pi \in \mathcal{S}} \int_{M\times M}|x - y|^p \dfrac{1}{N} \sum_{i = 1}^C \delta_{(x_{(i)}, y_{\pi(i)})}(dx,dy)
\end{equation}
since $\mathcal{S}$ is finite, the infimum is a minimum. By exploiting the definition of Dirac's distribution and the linearity of the Lebesgue integral, for any $\pi \in \mathcal{S}$, we get
\begin{equation}
\begin{split}
\int_{M\times M}|x - y|^p \dfrac{1}{C}\sum_{i = 1}^C \delta_{(x_{(i)}, y_{\pi(i)})}(dx,dy) &= \dfrac{1}{C}\sum_{i = 1}^C\int_{M\times M} |x - y|^p\delta_{(x_{(i)}, y_{\pi(i)})}(dx,dy)\\
&=\dfrac{1}{C}\sum_{i = 1}^C |x_{(i)} - y_{\pi(i)}|^p\,\,.
\end{split} 
\end{equation}
Therefore, finding the Wasserstein distance between $\mathbb{P}$ and $\mathbb{Q}$ boils down to a combinatorial optimization problem, that is, finding the permutation $\pi \in \mathcal{S}$ that solves
\begin{equation}\label{eq:min_pi_empirical}
W_p^p(\mathbb{P}, \mathbb{Q}) = \min_{\pi \in \mathcal{S}} \dfrac{1}{C}\sum_{i = 1}^C |x_{(i)}- y_{\pi(i)}|^p\,\,.
\end{equation}
The minimum is achieved when $\pi = \pi^*$ that is the permutation providing the ranking statistic, i.e. $\pi^*(y_i) = y_{(i)}$, since the smallest value of the sum is given for the smallest fluctuations. Thus we conclude that the $p$-Wasserstein distance between $\mathbb{P}$ and $\mathbb{Q}$ is given by
\begin{equation}\label{eq:wass_empirical}
    W_p(\mathbb{P}, \mathbb{Q}) = \left (\dfrac{1}{C} \sum_{i = 1}^C|x_{(i)}- y_{(i)}|^p\right )^{1/p}
\end{equation}
that is the pairwise distance computed between the class frequency vectors, sorted in order of magnitude, for each client, introduced in Section \ref{clustereing_metric}, where we chose $p = 2$.
\end{proof}

\section{Privacy of \shortname}\label{app:privacy}
In the framework of \shortname, clients are required to send only the empirical loss vectors $l_k^{t,s}$ to the server \citep{cho2022towards}. While concerns might arise regarding the potential leakage of sensitive information from sharing this data, it is important to clarify that the server only needs to access aggregated statistics, working on aggregated data. This ensures that client-specific information remains private. Privacy can be effectively preserved by implementing the Secure Aggregation protocol \citep{bonawitz2016practical}, which guarantees that only the aggregated results are shared, preventing the exposure of any raw client data.

\section{Communication and Computational Overhead of \shortname}
\label{app:communication-computation}
\shortname\ minimizes communication and computational overhead, aligning with the requirements of scalable FL systems \citep{mcmahan2016federated}. On the client side, the computational cost remains unchanged compared to the chosen FL aggregation, e.g. FedA, as clients are only required to communicate their local models and a vector of empirical losses after each round. The size of this loss vector, denoted by \( S \), corresponds to the number of local iterations (\ie the product of local epochs and the number of batches) and is negligible w.r.t. the size of the model parameter space, \( |\Theta| \). In our experimental setup, \( S = 8 \), ensuring that the additional communication overhead from transmitting loss values is negligible in comparison to the transmission of model weights. 


All clustering computations, including those based on interaction matrices and Gaussian weighting, are performed exclusively on the server. This design ensures that client devices are not burdened with additional computational complexity or memory demands. The interaction matrix $P$ used in \shortname\ is updated incrementally and involves sparse matrix operations, which significantly reduce both memory usage and computational costs.

These characteristics make \shortname\ particularly well-suited for cross-device scenarios involving large federations and numerous communication rounds.
Moreover, by operating on scalar loss values rather than high-dimensional model parameters, the clustering process in \shortname\ achieves computational efficiency while maintaining effective grouping of clients. The server-side processing ensures that the method remains scalable, even as the number of clients and communication rounds increases. Consequently, \shortname\ meets the fundamental objectives of FL by minimizing costs while preserving privacy and maintaining high performance.


\section{\textcolor{black}{Metrics Used for Evaluation}} \label{app:metrics_choice}
\subsection{\textcolor{black}{Silhouette Score}}
\textcolor{black}{Silhouette Score is a clustering metric that measures the consistency of points within clusters by comparing intra-cluster and nearest-cluster distances \citep{rousseeuw1987silhouettes}. Let us consider a metric space $(M,d)$. For a set of points $\{x_1,\dots, x_N\} \subset M$ and clustering labels $\mathcal{C}_1, \dots, \mathcal{C}_{n_{cl}}$. The Silhouette score of a data point $x_i$ belonging to a cluster $\mathcal{C}_i$ is defined as}
\begin{equation}\label{silhouette_defn1}
    \color{black}
    s_i = \dfrac{b_i - a_i}{\max\{a_i,b_i\}}
\end{equation}
\textcolor{black}{where the values $b_i$ and $a_i$ represent the average intra-cluster distance and the minimal average outer-cluster distance, \ie}\begin{equation}\label{silhouette_defn}
    \color{black}
    \begin{split}
        a_i &= \dfrac{1}{|\mathcal{C}_i| - 1} \sum_{x_j \in \mathcal{C}_i\setminus\{x_i\}} d(x_i, x_j)\\
        b_i & =\min_{j \neq i} \dfrac{1}{|\mathcal{C}_j|} \sum_{x_j \in \mathcal{C}_j} d(x_i,x_j)
    \end{split}
\end{equation}
\textcolor{black}{The value of the Silhouette score ranges between $-1$ and $+1$, \ie $s_i \in [-1,1]$. In particular, a Silhouette score close to 1 indicates well-clustered data points, 0 denotes points near cluster boundaries, and -1 suggests misclassified points. In order to evaluate the overall performance of the clustering, a common choice, that is the one adopted in this paper, is to average the score value for each data point.}
\subsection{\textcolor{black}{Davies-Bouldin Score}}
\textcolor{black}{The Davies-Bouldin Score is a clustering metric that evaluates the quality of clustering by measuring the ratio of intra-cluster dispersion to inter-cluster separation \citep{davies1979cluster}. Let us consider a metric space $(M,d)$, a set of points $\{x_1, \dots, x_N\} \subset M$, and clustering labels $\mathcal{C}_1, \dots, \mathcal{C}_{n_{cl}}$. The Davies-Bouldin score is defined as the average similarity measure $R_{ij}$ between each cluster $\mathcal{C}_i$ and its most similar cluster $\mathcal{C}_j$}:
\begin{equation}\label{db_index_defn1}
    \color{black}
    DB = \dfrac{1}{n_{cl}} \sum_{i=1}^{n_{cl}} \max_{j \neq i} R_{ij}
\end{equation}
\textcolor{black}{where $R_{ij}$ is given by the ratio of intra-cluster distance $S_i$ to inter-cluster distance $D_{ij}$, \ie}
\begin{equation}\label{db_index_defn}
    \color{black}
    R_{ij} = \dfrac{S_i + S_j}{D_{ij}}
\end{equation}
\textcolor{black}{with intra-cluster distance $S_i$ defined as}
\begin{equation}
    \color{black}
    S_i = \dfrac{1}{|\mathcal{C}_i|} \sum_{x_k \in \mathcal{C}_i} d(x_k, c_i)
\end{equation}
\textcolor{black}{where $c_i$ denotes the centroid of cluster $\mathcal{C}_i$, and $D_{ij} = d(c_i, c_j)$ is the distance between centroids of clusters $\mathcal{C}_i$ and $\mathcal{C}_j$. A lower Davies-Bouldin Index indicates better clustering, as it reflects well-separated and compact clusters. Conversely, a higher DBI suggests that clusters are less distinct and more dispersed.}
\subsubsection{\textcolor{black}{Rand Index}} \textcolor{black}{Rand Index is a clustering score that measures the outcome of a clustering algorithm with respect to a ground truth clustering label \citep{rand1971objective}. Let us denote by $a$ the number of pairs that have been grouped in the same clusters, while by $b$ the number of pairs that have been grouped in different clusters, then the Rand-Index is defined as}
\begin{equation}
    \color{black}
    RI = \dfrac{a + b}{\binom{N}{2}}
\end{equation}
\textcolor{black}{where N denotes the number of data points. In our experiments we opted for the Rand Index score to evaluate how the algorithm was able to separate clients in groups of the same level of heterogeneity (which was known a priori and used as ground truth). A Rand Index ranges in $[0,1]$, and a value of 1 signifies a perfect agreement between the identified clusters and the ground truth.}

\newpage
\subsection{Inference Prompts}
For questions from the MATH and AIME benchmarks, we use the following prompt.
\begin{tcolorbox}[title=MATH and AIME Prompt]
Please answer the following question. Think carefully and in a step-by-step fashion. At the end of your solution, put your final result in a boxed environment, e.g. $\boxed{42}$.

\textcolor{blue}{The question would be here.}
\end{tcolorbox}

For questions from the LiveBench Math and LiveBench Reasoning benchmarks, which already come with their own instructions and formatting requests, we do not provide any accompanying prompt and simply submit the model the question verbatim.
\begin{tcolorbox}[title=LiveBench Prompt]
\textcolor{blue}{The question would be here.}
\end{tcolorbox}

\subsection{LM-Based Scoring}
\label{app:scoring}

Given a tuple consisting of a question, ground-truth solution, and candidate response, we grade the correctness of the candidate response by querying a Gemini-v1.5-Pro-002 model to compare the candidate and ground-truth solutions.
This involves repeating the following process five times: (1) send a prompt to the model that provides the question, the correct ground-truth solution, and the candidate response, and asks the model to deliberate on the correctness of the candidate response; and (2) send a followup prompt to the model to obtain a correctness ruling in a structured format.
If a strict majority of (valid) responses to the second prompt evaluate to a JSON object with the key-value pair \tcbox[on line,  boxrule=0.5pt, top=0pt, bottom=0pt, left=1pt, right=1pt]{``student\_final\_answer\_is\_correct'' = True} rather than \tcbox[on line,  boxrule=0.5pt, top=0pt, bottom=0pt, left=1pt, right=1pt]{``student\_final\_answer\_is\_correct'' = False}, the candidate response is labeled correct. Otherwise, the candidate response is labeled incorrect.
These queries are all processed with temperature zero.
The prompts, which can be found at the end of this subsection,
 ask the language model to
(1) identify the final answer of the given response, (2) identify the final answer of the reference (ground truth) response, and (3) determine whether the final answer of the given response satisfactorily matches that of the reference response, ignoring
any non-substantive formatting disagreements.
In line with convention, we instruct our scoring system to ignore the correctness of the logic used to reach the final answer and rather only judge the correctness of the final answer.
The model is asked to label all non-sensical and incomplete responses as being incorrect.

As a form of quality assurance, every scoring output for the Consistency@200 and Verification@200 figures depicted in Table~\ref{tab:main-sota} was manually compared against human scoring.
No discrepancies between automated and human scoring were found on the MATH and AIME datasets for both Consistency@200 and Verification@200.
No discrepancies were found on LiveBench Reasoning for Consistency@200.
For Verification@200, one false positive (answer labeled by automated system as being incorrect but labeled by human as being correct) and one false negative  (answer labeled by automated system as being correct but labeled by human as being incorrect) were identified on LiveBench Reasoning; three false positives and four false negatives were identified on LiveBench Math.
For Consistency@200, two false negatives were identified on LiveBench Math.
This means that LM scoring matched human scoring 99\% of the time, and the choice of human versus automated scoring matters little to our results.

\begin{tcolorbox}[title=Prompt 1, breakable]
You are an accurate and reliable automated grading system. Below are two solutions to a math exam problem: a solution written by a student and the solution from the answer key. Your task is to check if the student's solution reaches a correct final answer. 

Your response should consist of three parts. First, after reading the question carefully, identify the final answer of the answer key's solution. Second, identify the final answer of the student's solution. Third, identify whether the student's final answer is correct by comparing it to the answer key's final answer.

\# The question, answer key, and student solution

The math exam question:

\verb|```|

\textcolor{blue}{The question would be here.}

\verb|```|

The answer key solution:

\verb|```|

\textcolor{blue}{The reference solution would be here.}

\verb|```|

The student's solution:

\verb|```|

\textcolor{blue}{The candidate solution would be here.}

\verb|```|

\# Your response format

Please structure your response as follows. PROVIDE A COMPLETE RESPONSE.

\verb|```|

\# Answer Key Final Answer

Identify the final answer of the answer key solution. That's all you need to do here: just identify the final answer.

A "final answer" can take many forms, depending on what the question is asking for; it can be a number (e.g., "37"), a string (e.g., "ABCDE"), a sequence (e.g., "2,3,4,5"), a letter (e.g., "Y"), a multiple choice option (e.g. "C"), a word (e.g., "Apple"), an algebraic expression (e.g. "$x^2 + 37$"), a quantity with units (e.g. "4 miles"), or any of a number of other options. If a solution concludes that the question is not answerable with the information provided or otherwise claims that there is no solution to the problem, let the final answer be "None". If the solution does not produce any final answer because it appears to be cut off partway or is otherwise non-sensical, let the solution's final answer be "Incomplete solution" (this could only ever possibly happen with the student solution).

YOUR RESPONSE HERE SHOULD BE BRIEF. JUST IDENTIFY WHAT THE QUESTION IS ASKING FOR, AND IDENTIFY THE ANSWER KEY'S FINAL ANSWER. DO NOT ATTEMPT TO ANSWER THE QUESTION OR EVALUATE INTERMEDIATE STEPS.

\# Student Solution Final Answer

Identify the final answer of the student solution.

YOUR RESPONSE HERE SHOULD BE BRIEF. JUST IDENTIFY WHAT THE QUESTION IS ASKING FOR, AND IDENTIFY THE STUDENT'S FINAL ANSWER. DO NOT ATTEMPT TO ANSWER THE QUESTION OR EVALUATE INTERMEDIATE STEPS.

\# Correctness

Simply evaluate whether the student's final answer is correct by comparing it to the answer key's final answer.

Compare the student's final answer against the answer key's final answer to determine if the student's final answer is correct.

* It does not matter how the student reached their final answer, so long as their final answer itself is correct.

* It does not matter how the student formatted their final answer; for example, if the correct final answer is \boxed{7 / 2}, the student may write ***3.5*** or \boxed{\mathrm{three\; and\; a\; half}} or $\boxed{\frac{14}{4}}$. It does not matter if the student's final answer uses the same specific formatting that the question asks for, such as writing multiple choice options in the form "(E)" rather than "***E***".

* It does not matter if the student omitted units such as dollar signs.

* If the student solution appears to be truncated or otherwise incoherent, e.g. due to a technical glitch, then it should be treated as being incorrect.

ONCE AGAIN, DO NOT EVALUATE INTERMEDIATE STEPS OR TRY TO SOLVE THE PROBLEM YOURSELF. THE ANSWER KEY IS ALWAYS RIGHT. JUST COMPARE THE FINAL ANSWERS. IF THEY MATCH, THE STUDENT ANSWER IS CORRECT. IF THEY DO NOT MATCH, THE STUDENT ANSWER IS INCORRECT.

\# Summary

* Answer key final answer: (The final answer of the answer key solution. Please remove any unnecessary formatting, e.g. provide "3" rather than "\boxed{3}", provide "E" rather than "***E***", provide "1, 2, 3" rather than "[1, 2, 3]", provide "4 ounces" rather than "4oz".)

* Student final answer: (The final answer of the student's solution. Please remove any unnecessary formatting, e.g. provide "3" rather than "\boxed{3}", provide "E" rather than "***E***", provide "1, 2, 3" rather than "[1, 2, 3]", provide "4 ounces" rather than "4oz".)

* Student final answer is correct?: (Does the student final answer match the answer key final answer? Please provide "true" or "false".)

\verb|```|

\end{tcolorbox}

\begin{tcolorbox}[breakable, title=Prompt 2]
Please structure your output now as JSON, saying nothing else. Use the following format:
\verb|```|
\{
    "answer\_key\_final\_answer": str (the final answer of the answer key solution; please remove any formatting"),
    "student\_final\_answer": str (the final answer of the student's solution; please remove any formatting"),
    "student\_final\_answer\_is\_correct": true/false,
\}
\end{tcolorbox}

\subsection{Implementation of Consistency@k}
Consistency@k measures the performance of a model by evaluating the correctness of the most common answer reached by the model after being run $k$ times.
An important consideration with implementing consistency@k is that there are many choices for the equivalence relation one can use to define ``the most common answer''.
We define two candidate responses as reaching the same answer if their final answer is the same.
We determine a candidate response's final answer by prompting a language model to identify the final answer from the candidate response; we then strip the extracted final answer of leading and trailing whitespace.
We determine equivalence with a literal string match.
After determining the most common final answer to a question, we use the string \tcbox[on line,  boxrule=0.5pt, top=0pt, bottom=0pt, left=1pt, right=1pt]{``The final answer is \textcolor{blue}{\{final answer\}}''} as the consistency@k response.
Note that we could have instead randomly chosen a candidate response corresponding to the most common final answer, and used that selected response as the consistency@k response---we have found that, because our LM-based scoring system evaluates correctness using only the final answer, this alternative results in the same consistency@k metrics.

\subsection{Benchmark Evaluation Prompts}
\label{app:benchmarkprompts}

The benchmark performances reported in Table~\ref{tab:benchmark} are obtained with the following prompts.
The following prompt is used for the comparison task.
\begin{tcolorbox}[title=Comparison Task Prompt Part 1]
\textcolor{blue}{Question here.}

  Here are two solutions to the above question. You must determine which one is correct. Please think extremely carefully. Do not leap to conclusions. Find out where the solutions disagree, trace them back to the source of their disagreement, and figure out which one is right.

  Solution 1:
  
\textcolor{blue}{First solution here.}

  Solution 2:
  
\textcolor{blue}{Second solution here.}
\end{tcolorbox}
\begin{tcolorbox}[title=Comparison Task Prompt Part 2]
Now summarize your response in a JSON format. Respond in the following format saying nothing else:

  \{
  
     "correct\_solution": 1 or 2
     
  \}
\end{tcolorbox}

The following prompt is used for the scoring task.
\begin{tcolorbox}[title=Scoring Task Prompt Part 1]
\textcolor{blue}{Question here.}

 I include below a student solution to the above question. Determine whether the student solution reaches the correct final answer in a correct fashion; e.g., whether the solution makes two major errors that still coincidentally cancel out. Please be careful and do not leap to conclusions without first reasoning them through.

Solution:

\textcolor{blue}{Solution here.}
\end{tcolorbox}
\begin{tcolorbox}[title=Scoring Task Prompt Part 2]
Now summarize your response in a JSON format. Respond in the following format saying nothing else:

\{
 
 "is\_solution\_correct": 'yes' or 'no'

\}
\end{tcolorbox}

\section{Sensitive Analysis beta value RBF kernel} \label{app:sensitive}

This section provides a sensitivity analysis for the $\beta$ hyper-parameter of the RBF kernel adopted for \shortname. The results of this tuning are shown in Table \ref{tab:sensitive}.

\begin{table}[h]

    \caption{\small{A sensitivity analysis on the RBF kernel hyper-parameter $\beta$ is conducted. We present the balanced accuracy for \shortname on the Cifar10, Cifar100, and Femnist datasets for $\beta \in \{0.1, 0.5, 1.0, 2.0, 4.0\}$. It is noteworthy that \shortname demonstrates robustness to variations in this hyperparameter.}}
    \label{tab:sensitive}
    \centering
    \begin{adjustbox}{width=.6\linewidth}
        \centering
        
        \begin{tabular}{ccccc}
            
            \toprule

            \textbf{$\beta$} & \textbf{Cifar100} & \textbf{Femnist} & \textbf{Google Landmarks} & \textbf{iNaturalist}\\
            
            \midrule

            0.1 &  49.9 & 76.0 & 55.0 &  47.5\\
            0.5  & \textbf{53.4} & 76.0 & \textbf{57.4} & \textbf{47.8}\\
            1.0  & 49.5 & 76.0 & 56.0 & 47.5\\
            2.0  & 50.9 & 75.6 & 57.0 & 47.2 \\
            4.0 & 52.6 & \textbf{76.1} & 55.8 & 47.1 \\
            
            \bottomrule
        
        \end{tabular}
    \end{adjustbox}
\end{table}

\section{Additional Experiments: Visual Domain Detection in Cifar10}\label{app:cifar10}
In this section we present the result for the domain ablation discussed in Section \ref{sect:ablation} conducted on Cifar10 \cite{krizhevsky2009learning}. We explore how the algorithm identifies and groups clients based on the non-IID nature of their data distributions, represented by the Dirichlet concentration parameter $\alpha$. We apply a similar splitting approach, obtaining the following partitions: (1) 90 clients with $\alpha = 0$ and 10 clients with $\alpha = 100$; (2) 90 clients with $\alpha = 0.5$ and 10 clients with $\alpha = 100$; and (3) 40 clients with $\alpha = 100$, 30 clients with $\alpha = 0.5$, and 30 clients with $\alpha = 0$. We evaluate the outcome of this clustering experiment by means of WAS and WADB. Results in Table \ref{tab:ablation1_heter} show that \shortname detects clusters groups clients according to the level of heterogeneity of the group.
\begin{table}[t]
    
    \caption{\small{Clustering with three different splits on Cifar10. \shortname has superior clustering quality across different splits (homogeneous \textit{Hom}, heterogeneous \textit{Het}, extremly heterogeneous \textit{X Het})}}
    \centering
    \small
    \begin{adjustbox}{width=.5\linewidth}
        \label{tab_app:ablation1_heter}
     
        \begin{tabular}{lccccc}
            \toprule
            \textbf{Dataset} & \textbf{(Hom, Het, X Het)} & \makecell{\textbf{Clustering} \\ \textbf{method}} & \textbf{C} & \textbf{WAS} & \textbf{WADB} \\
            \cmidrule(lr){1-6}
          
            \midrule\multirow{9}{*}{Cifar10} 
            & \multirow{3}{*}{(10, 0, 90)} & \texttt{IFCA} & 1 & / &/ \\
            & & \texttt{FeSem} & 3 & -0.0 \scriptsize{$\pm$ 0.1} & 12.0 \scriptsize{$\pm$ 2.0}\\
            & & \shortname & 3 & \textbf{0.1 \scriptsize{$\pm$ 0.0}} & \textbf{0.2 \scriptsize{$\pm$ 0.0}} \\
            \cmidrule(lr){2-6}
            & \multirow{3}{*}{(10, 90, 0)} & \texttt{IFCA} & 1 & / & / \\
            & & \texttt{FeSem} & 3 & -0.0 \scriptsize{$\pm$ 0.0} & 12.0 \scriptsize{$\pm$ 2.0}\\

            & & \shortname & 3 & \textbf{0.2 \scriptsize{$\pm$ 0.0}} & \textbf{0.6 \scriptsize{$\pm$ 0.0}} \\
            \cmidrule(lr){2-6}
            & \multirow{3}{*}{(40, 30, 30)} & \texttt{IFCA} & 2 & -0.2 \scriptsize{$\pm$ 0.0} & \textbf{1.0 \scriptsize{$\pm$ 0.0}} \\
            & & \texttt{FeSem} & 3 & 0.1 \scriptsize{$\pm$ 0.1} & 20.6 \scriptsize{$\pm$ 7.1} \\
            & & \shortname & 3 & \textbf{0.6 \scriptsize{$\pm$ 0.1}} & \textbf{1.0 \scriptsize{$\pm$ 0.4}}\\
            
            \bottomrule
        \end{tabular}
    \end{adjustbox}
    
\end{table}

\begin{table}[t]
    
    \caption{\small Clustering performance of \shortname is assessed on federations with clients from varied domains using clean, noisy, and blurred (Clean, Noise, Blur) images from Cifar10 dataset. It utilizes the Rand Index score \citep{rand1971objective}, where a value close to 1 represents a perfect match between clustering and labels. Consistently \shortname accurately distinguishes all visual domains.}
    \label{tab_app:dom_abl}
    
    \centering
    
    \begin{adjustbox}{width=.5\linewidth}
    \setlength{\tabcolsep}{12pt}
        \begin{tabular}{ccccc}
           \toprule
            \textbf{Dataset} & \textbf{(Clean, Noise, Blur)} & \makecell{\textbf{Clustering} \\ \textbf{method}} & \textbf{C} & \textbf{Rand} \\
            \cmidrule(lr){1-5}
            \multirow{9}{*}{Cifar10} 
            & \multirow{3}{*}{(50, 50, 0)} & \texttt{IFCA} & 1 & 0.5 \scriptsize{$\pm$ 0.0}  \\
            & & \texttt{FeSem} & 2 & 0.49 \scriptsize{$\pm$ 0.2} \\
            & & \shortname & 2 & \textbf{1.0 \scriptsize{$\pm$ 0.0}} \\
            \cmidrule(lr){2-5}
            & \multirow{3}{*}{(50, 0, 50)} & \texttt{IFCA} & 1 & 0.5 \scriptsize{$\pm$ 0.0} \\
            & & \texttt{FeSem} & 2 & 0.5 \scriptsize{$\pm$ 0.1}\\
            & & \shortname & 2 & \textbf{1.0 \scriptsize{$\pm$ 0.0}} \\
            \cmidrule(lr){2-5}
            & \multirow{3}{*}{(40, 30, 30)} & \texttt{IFCA} & 1 & 0.33 \scriptsize{$\pm$ 0.0} \\
            & & \texttt{FeSem} & 3 & 0.34 \scriptsize{$\pm$ 0.1} \\
            & & \shortname & 4 & \textbf{0.9 \scriptsize{$\pm$ 0.0}} \\

            \bottomrule
        \end{tabular}
    \end{adjustbox}
    
\end{table}

\section{Evaluation of IFCA and FeSEM algorithms with different number of clusters} \label{app:tuning}

This section shows the tuning of the number of clusters for the \texttt{IFCA} and \texttt{FeSEM} algorithms, which cannot automatically detect this value. The results of this tuning are shown in Table \ref{tab:tuning_baselines}.

\begin{table}[t]

    \caption{\small{Performance of for baseline algorithms for clustering in FL \texttt{FeSEM}, and \texttt{IFCA}, w.r.t. the number of clusters}}
    \label{tab:tuning_baselines}
    \centering
    \small
    \begin{adjustbox}{width=.35\linewidth}
        \centering
        
        \begin{tabular}{llccc}
            
            \toprule

             & & \makecell{ \textbf{Clustering} \\ \textbf{method}} & \textbf{C} & \textbf{Acc} \\
            
           
            
            \cmidrule{2-5}

            & \multirow{8}{*}{\rotatebox[origin=c]{90}{Cifar100}} & \multirow{4}{*}{\texttt{IFCA}} & 2 & 46.7 \scriptsize{$\pm$ 0.0} \\
            & & & 3 &44.0 \scriptsize{$\pm$ 1.6} \\
            & & & 4 & 45.1 \scriptsize{$\pm$ 2.6} \\
            & & & 5 & 47.5 \scriptsize{$\pm$ 3.5} \\
            
            \cmidrule{3-5}
            
            & & \multirow{4}{*}{\texttt{FeSem}} & 2 & 43.3 \scriptsize{$\pm$ 1.3} \\
            & & & 3 & 48.0 \scriptsize{$\pm$ 1.9} \\
            & & & 4 &50.9 \scriptsize{$\pm$ 1.8} \\
            & & & 5 & 53.4 \scriptsize{$\pm$ 1.8} \\
            
            \cmidrule{2-5}

            & \multirow{8}{*}{\rotatebox[origin=c]{90}{Femnist}} & \multirow{4}{*}{\texttt{IFCA}} & 2 & 76.1 \scriptsize{$\pm$ 0.1} \\
            & & & 3 & 75.9 \scriptsize{$\pm$ 1.9} \\
            & & & 4 & 76.6 \scriptsize{$\pm$ 0.1} \\
            & & & 5 & 76.7 \scriptsize{$\pm$ 0.6} \\
            
            \cmidrule{3-5}
            
            & & \multirow{4}{*}{\texttt{FeSem}} & 2 & 75.6\scriptsize{$\pm$ 0.2} \\
            & & & 3 &75.5\scriptsize{$\pm$ 0.5} \\
            & & & 4 & 75.0\scriptsize{$\pm$ 0.1} \\
            & & & 5 &74.9\scriptsize{$\pm$ 0.1} \\
            
            \bottomrule
        
        \end{tabular}
    \end{adjustbox}
\end{table}

\section{Further Experiments} \label{app:other}
In Table \ref{tab_app:fl-algs} we show that \shortname is orthogonal to FL aggregation, which means that any algorithm can be easily embedded in our clustering setting, providing beneficial results, increasing model performance.
\begin{table}[t]
    \caption{\small{\shortname is orthogonal to FL aggregation algorithms, improving their performance in heterogeneous scenarios (Cifar100 with $\alpha = 0.5$ and Femnist with $\alpha = 0.01$). This shows that \shortname and clustering are beneficial in this scenarios.  }}
    \label{tab_app:fl-algs}
    \centering
    \small
    \setlength{\tabcolsep}{4pt} %
    \renewcommand{\arraystretch}{1.1} %
    \begin{adjustbox}{width=.5\linewidth}
    \begin{tabular}{l|cc|cc}
        \toprule
        \textbf{FL method} &  \multicolumn{2}{c|}{\textbf{Cifar100}} & \multicolumn{2}{c}{\textbf{Femnist}} \\
         & No Clusters & \shortname & No Clusters & \shortname \\
        \midrule
        FedAvg &  41.6 {\scriptsize$\pm$ 1.3} & \textbf{53.4} {\scriptsize$\pm$ 0.4} & 76.0 {\scriptsize$\pm$ 0.1} & \textbf{76.1} {\scriptsize$\pm$ 0.1} \\ 
        FedAvgM &  41.5 {\scriptsize$\pm$ 0.5} & \textbf{50.5} {\scriptsize$\pm$ 0.3} & {83.3} {\scriptsize$\pm$ 0.3} & \textbf{83.3} {\scriptsize$\pm$ 0.4} \\ 
        FedProx &  41.8 {\scriptsize$\pm$ 1.0} & \textbf{49.1} {\scriptsize$\pm$ 1.0} & 75.9 {\scriptsize$\pm$ 0.2} & \textbf{76.3} {\scriptsize$\pm$ 0.2} \\ 
        \bottomrule
    \end{tabular}
\end{adjustbox}
\end{table}
\begin{figure}
    \centering
    \includegraphics[width=\linewidth]{figures/tree_levels.pdf}
    \caption{\small{Cluster evolution with respect to the recursive splits in \shortname on Cifar100, projected on the spectral embedded bi-dimensional space. From left to right, top to bottom, we can see that \shortname splits the client into cluster, until a certain level of intra-cluster homogeneity is reached }}
    \label{fig:treelevels}
\end{figure}
\begin{figure}[htbp]
    \centering
    \includegraphics[width=0.6\linewidth]{figures/fedgw_cifar10_matrix_convergence.png}
    \caption{\small{Interaction matrix convergence: on the $y$-axis MSE in logarithmic scale w.r.t. communication rounds in the $x$-axis on Cifar10, with Dirichlet parameter $\alpha = 0.05$.}}
    \label{fig:mse_conv}
\end{figure}
Figure \ref{fig:hom-het} illustrates the clustering results corresponding to varying degrees of heterogeneity, as described in Section \ref{sect:ablation}. As per \shortname, the detection of clusters based on different levels of heterogeneity in the Cifar10 dataset is achieved. Specifically, an examination of the interaction matrix reveals a clear distinction between the two groups.
\begin{figure}[h]
    \centering
    \includegraphics[width=1\linewidth]{figures/spectral_split.pdf}
    \caption{\small{Homogeneous (Cifar10 $\alpha = 100$) vs heterogeneous clustering (Cifar10 $\alpha = 0.05$). The interaction matrix at convergence and the corresponding scaled affinity matrix are on the left. The scatter plot in the 2D plane with spectral embedding is on the right. It is possible to see that the algorithm perfectly separates homogeneous clients (orange) from heterogeneous clients (black) }}
    \label{fig:hom-het}
\end{figure}
In Figure \ref{fig:hom}, we show that in class-balanced scenarios with small heterogeneity, like Cifar10 with $\alpha = 100$, \shortname successfully detects one single cluster. Indeed, in homogeneous scenarios such as this one, the model benefits from accessing more data from all the clients.
\begin{figure}[h]
    \centering
    \includegraphics[width=\linewidth]{figures/spectral_uniform.pdf}
    \caption{\small{Homogeneous case (Cifar10 $\alpha = 100$).  The interaction matrix at convergence and the corresponding scaled affinity matrix are on the left. The scatter plot in the 2D plane with spectral embedding is on the right. In the homogeneous case where no clustering is needed, \textit{FedGW} does not split the clients.}}
    \label{fig:hom}
\end{figure}

Figure \ref{fig:mse_conv} shows how the MSE converges to a small value as the rounds increase for a Cifar10 experiment.

As Figure \ref{fig:class_distr} illustrates, \shortname partitions the Cifar100 dataset into clients based on class distributions. Each cluster's distribution is distinct and non-overlapping, demonstrating the algorithm's efficacy in partitioning data with varying degrees of heterogeneity.
\begin{figure}[t]
    \centering
    \includegraphics[width=\linewidth]{figures/data_distr.pdf}
    \caption{\small{Class distributions among distinct clusters as detected by \shortname on Cifar100. Specifically, we examine the class distributions for each pair of clusters, demonstrating that (1) the clusters were identified by grouping differing levels of heterogeneity and (2) there is, in most cases, an absence of overlapping classes.}}
    \label{fig:class_distr}
\end{figure}
In Figure \ref{fig:dom_ablation}, we report the domain detection on Cifar100, where 40 clients have clean images, 30 have noisy images, and 30 have blurred images. Table \ref{tab:dom_abl} shows that \shortname performs a good clustering, effectively separating the different domains.
\begin{figure}
    \centering
     
    \includegraphics[width=\linewidth]{figures/domain_ablation_c100.pdf}
    \caption{\small{\shortname in the presence of domain imbalance. Three domains on Cifar100: clean clients (unlabeled), noisy clients (+), and blurred clients (x). \textit{Left}: is the interaction matrix $P$ at convergence from which it is possible to see client relations. \textit{Center}: The affinity matrix $W$ computed with respect to the UPVs extracted from $P$, and on which \texttt{FedGW\_Clustering} is performed. We can see that \shortname clusters the clients according to the domain, as proved by results in Table \ref{tab:dom_abl}.}}
    \label{fig:dom_ablation}
   
\end{figure}



\end{document}
