\documentclass{article}


% if you need to pass options to natbib, use, e.g.:
%     \PassOptionsToPackage{numbers, compress}{natbib}
% before loading neurips_2024


% ready for submission
%%%%%%%%%%%%%%%%%%%%%%%%%%%%%%%%%%%%%%%%%%%%%%%%%%%%%%%
%%%%%%%%%%%%%%%    theorems %%%%%%%%%%%%%%%%%%%%%%%%%%%
%%%%%%%%%%%%%%%%%%%%%%%%%%%%%%%%%%%%%%%%%%%%%%%%%%%%%%%
% \usepackage{mdframed}
\usepackage{kantlipsum}

%%%%%%%%%%%%%%%%%%%%%%%%%%%%%%%%%%%%%%%%%%%%%%%%%%%%%%%
%%%%%%%%%%%%%%%    theorems %%%%%%%%%%%%%%%%%%%%%%%%%%%
%%%%%%%%%%%%%%%%%%%%%%%%%%%%%%%%%%%%%%%%%%%%%%%%%%%%%%%
\theoremstyle{plain}
\newtheorem{theorem}{Theorem}[section]
\newtheorem{proposition}[theorem]{Proposition}
\newtheorem{lemma}[theorem]{Lemma}
\newtheorem{example}[theorem]{Example}
\newtheorem{corollary}[theorem]{Corollary}
\theoremstyle{definition}
\newtheorem{definition}[theorem]{Definition}
\newtheorem{assumption}[theorem]{Assumption}
\theoremstyle{remark}
\newtheorem{remark}[theorem]{Remark}


% \titleformat{\subsection}[runin]% runin puts it in the same paragraph
%        {\normalfont\bfseries}% formatting commands to apply to the whole heading
%        {\thesubsection}% the label and number
%        {0.5em}% space between label/number and subsection title
%        {}% formatting commands applied just to subsection title
%        [.]% punctuation or other commands following subsection title


%%%%%%%%%%%%%%%%%%%%%%%%%%%%%%%%%%%%%%%%%%%%%%%%%%%%%%%
%%%%%%%%%%%%%%%  mathematical notations%%%%%%%%%%%%%%%%
% \usepackage[english]{babel}
% \usepackage[utf8]{inputenc}
% \usepackage[T1]{fontenc}

%% Figures, tables and lists
\usepackage[dvipsnames]{xcolor}
\usepackage{paralist}
\usepackage{graphicx}
\usepackage{subcaption}
\usepackage{longtable} 
\usepackage{multirow}
\usepackage{listings}
\usepackage{makecell}
\usepackage{array}
\usepackage{float}
\usepackage{dsfont}
\usepackage{rotating}
\usepackage{booktabs}
\usepackage{enumerate}
\usepackage{tikz}
\usepackage{pgf}
\usepackage{enumitem}
\usepackage{lipsum} % for generating filler text
\usepackage{titlesec}

%% Math
% \usepackage{amssymb, amsthm,bbm}
\usepackage{mathtools}
\usepackage{mathrsfs}
%% References and author info 
\mathtoolsset{showonlyrefs}
\usepackage{natbib}
\usepackage{authblk}
\usepackage{todonotes}
\usepackage{xr-hyper}


%%%%%%%%%%%%%%%%%%%%%%%%%%%%%%%%%%%%%%%%%%%%%%%%%%%%%%%
\newcommand{\R}{\mathbb R}
\newcommand{\EE}{\mathbb{E}}

\DeclareMathOperator{\Tr}{Tr}
\DeclareMathOperator*{\argmin}{argmin}
\DeclareMathOperator*{\argmax}{argmax}

\newcommand{\bs}[1]{\ensuremath{\boldsymbol{#1}}}
\newcommand{\mc}{\mathcal}
\newcommand{\opt}{^\star}


\newcommand{\diff}{\textnormal{d}}


\def \iid {\stackrel{\textnormal{i.i.d.}}{\sim}}
\def \iidtext {\textnormal{i.i.d.}}





%%%%%%%%%%%%%%%%%%%%%%%%%%%%%%%%%%%%%%%%%%%%%%%%%%%%%%%
%%%%%%%%%%%%%%%%%%%%% colors     %%%%%%%%%%%%%%%%%%%%%%
%%%%%%%%%%%%%%%%%%%%%%%%%%%%%%%%%%%%%%%%%%%%%%%%%%%%%%%
\definecolor{myblue}{rgb}{.8, .8, 1}
\definecolor{mathblue}{rgb}{0.2472, 0.24, 0.6} % mathematica's Color[1, 1--3]
\definecolor{mathred}{rgb}{0.6, 0.24, 0.442893}
\definecolor{mathyellow}{rgb}{0.6, 0.547014, 0.24}


% May add more in future.






\usepackage{neurips_2024}
\usepackage{mathtools}
\usepackage{url}
\def\MoG{\texttt{MoG}}
\def\MoLRG{\texttt{MoLRG}}
\def\CSNR{$\mathrm{CSNR}$}
\def\SWD{$\mathrm{SWD}$}
\newcommand{\ZK}[1]{{\color{cyan}[Zekai: #1]}}


% to compile a preprint version, e.g., for submission to arXiv, add add the
% [preprint] option:
%     \usepackage[preprint]{neurips_2024}


% to compile a camera-ready version, add the [final] option, e.g.:
%     \usepackage[final]{neurips_2024}


% to avoid loading the natbib package, add option nonatbib:
%    \usepackage[nonatbib]{neurips_2024}


\usepackage[utf8]{inputenc} % allow utf-8 input
\usepackage[T1]{fontenc}    % use 8-bit T1 fonts
\usepackage{hyperref}       % hyperlinks
\usepackage{url}            % simple URL typesetting
\usepackage{booktabs}       % professional-quality tables
\usepackage{amsfonts}       % blackboard math symbols
\usepackage{nicefrac}       % compact symbols for 1/2, etc.
\usepackage{microtype}      % microtypography
\usepackage{xcolor}         % colors


\title{Understanding Diffusion-based Representation Learning via Low-Dimensional Modeling}


% The \author macro works with any number of authors. There are two commands
% used to separate the names and addresses of multiple authors: \And and \AND.
%
% Using \And between authors leaves it to LaTeX to determine where to break the
% lines. Using \AND forces a line break at that point. So, if LaTeX puts 3 of 4
% authors names on the first line, and the last on the second line, try using
% \AND instead of \And before the third author name.


\author{%
  David S.~Hippocampus\thanks{Use footnote for providing further information
    about author (webpage, alternative address)---\emph{not} for acknowledging
    funding agencies.} \\
  Department of Computer Science\\
  Cranberry-Lemon University\\
  Pittsburgh, PA 15213 \\
  \texttt{hippo@cs.cranberry-lemon.edu} \\
  % examples of more authors
  % \And
  % Coauthor \\
  % Affiliation \\
  % Address \\
  % \texttt{email} \\
  % \AND
  % Coauthor \\
  % Affiliation \\
  % Address \\
  % \texttt{email} \\
  % \And
  % Coauthor \\
  % Affiliation \\
  % Address \\
  % \texttt{email} \\
  % \And
  % Coauthor \\
  % Affiliation \\
  % Address \\
  % \texttt{email} \\
}


\begin{document}


\maketitle


\begin{abstract}
  This work addresses the critical question of why and when diffusion models, despite their generative design, are capable of learning high-quality representations in a self-supervised manner. We hypothesize that diffusion models excel in representation learning due to their ability to learn the low-dimensional distributions of image datasets via optimizing a noise-controlled denoising objective. Our empirical results support this hypothesis, indicating that variations in the representation learning performance of diffusion models across noise levels are closely linked to the quality of the corresponding posterior estimation. Grounded on this observation, we offer theoretical insights into the unimodal representation dynamics of diffusion models as noise scales vary, demonstrating how they effectively learn meaningful representations through the denoising process. We also highlight the impact of the inherent parameter-sharing mechanism in diffusion models, which accounts for their advantages over traditional denoising auto-encoders in representation learning.
\end{abstract}

\section{Introduction}\label{sec:intro}
\section{Introduction}
\label{sec:intro}

\begin{figure*}[tb]
    \centering
    \includegraphics[width=0.848\linewidth]{figs/circuitnn.pdf} 
    \caption{Illustration of differentiable CircuitNN. CircuitNN is designed based on differentiable NAND gates. After DAS is guided by PI and PO pairs of the truth table, CircuitNN can get the precise circuit architecture logic equivalent to the truth table.}
    \label{fig:circuitnn}
\end{figure*}

% 1. Describe the importance of logic synthesis
% 2. Existing Problems
% (a) Neural Architecture Search: Unstable, Predefined Setting, etc.
% (b) Circuit Generation: Probabilistic Model, Logic Equivalence

With the rapid advancement of technology, the scale of integrated circuits (ICs) has expanded exponentially. 
This expansion has introduced significant challenges in chip manufacturing, particularly concerning power and area metrics.
A primary objective in IC design is achieving the same circuit function with fewer transistors, thereby reducing power usage and area occupancy.

Logic synthesis~\cite{hachtel2005logicsynth}, a critical step in electronic design automation (EDA), transforms behavioral-level circuit designs into optimized gate-level circuits, ultimately yielding the final IC layout. 
The primary goal of logic synthesis is to identify the physical implementation with the fewest gates for a given circuit function. 
This task constitutes a challenging NP-hard combinatorial optimization problem. 
Current logic synthesis tools~\cite{brayton2010abc, wolf2013yosys} rely on human-designed heuristics, often leading to sub-optimal outcomes.

Differentiable architecture search (DAS) techniques~\cite{liu2018darts, chu2020darts} offer novel perspectives on addressing challenges in this problem.
Circuit functions can be represented through truth tables, which map binary inputs to their corresponding outputs. 
Truth tables provide a precise representation of input-output relationships, ensuring the design of functionally equivalent circuits.
Inspired by this, researchers~\cite{deepmind2024ai4sys, wang2024tnet} have begun exploring the application of DAS to synthesize circuits directly from truth tables.
Specifically, \citet{deepmind2024ai4sys} proposed CircuitNN, a framework that learns differentiable connection structures with logic gates, enabling the automatic generation of logic circuits from truth tables.
This approach significantly reduces the complexity of traditional circuit generation. 
Building on this, \citet{wang2024tnet} introduced T-Net, a triangle-shaped variant of CircuitNN, incorporating regularization techniques to enhance the efficiency of DAS.

Despite these advancements, several challenges remain. 
The computational complexity of DAS grows quadratically with the number of gates, posing scalability issues.
Although triangle-shaped architecture~\cite{wang2024tnet} partially mitigates this problem, redundancy persists. 
%Additionally, DAS is susceptible to converging to local optima, limiting the ability to search architectures that satisfy the given truth tables~\cite{liu2018darts}. 
%Furthermore, hyperparameters (network depth and layer width) require extensive searches, introducing complexity and prolonging the synthesis process. 
Additionally, DAS is susceptible to converging to local optima~\cite{liu2018darts} and hyperparameters (network depth and layer width) require extensive searches. 
The challenges arise from the vast search space in DAS. 
% Even with predefined settings for CircuitNN, finding a configuration that meets the truth table requires extensive trial and error during the DAS process. 
Intuitively, limiting the search space through predefined parameters (network depth, gates per layer, and connection probabilities) can significantly reduce the complexity.

Recent advances~\cite{openai2023gpt4, abramson2024alphafold3, esser2024sd3, li2024mar} in conditional generative models have demonstrated remarkable performance across language, vision, and graph generation tasks. 
Motivated by these developments, we propose a novel approach to circuit generation that generates preliminary circuit structures to guide DAS in generating refined circuits matching specified truth tables. 
Firstly, we introduce CircuitVQ, a tokenizer with a discrete codebook for circuit tokenization. 
Built upon our Circuit AutoEncoder framework~\cite{hou2022graphmae,li2023maskgae,wu2025mgvga}, CircuitVQ is trained through a circuit reconstruction task. 
Specifically, the CircuitVQ encoder encodes input circuits into discrete tokens using a learnable codebook, while the decoder reconstructs the circuit adjacency matrix based on these tokens.
Subsequently, the CircuitVQ encoder serves as a circuit tokenizer for CircuitAR pretraining, which employs a masked autoregressive modeling paradigm~\cite{chang2022maskgit, li2023mage}. 
In this process, the discrete codes function as supervision signals. 
After training, CircuitAR can generate discrete tokens progressively, which can be decoded into initial circuit structures by the decoder of the CircuitVQ. 
These prior insights can guide DAS in producing refined circuits that match the target truth tables precisely.

Our key contributions can be summarized as follows:
\begin{itemize}
\item We introduce CircuitVQ, a circuit tokenizer that facilitates graph autoregressive modeling for circuit generation, based on our Circuit AutoEncoder framework;
\item Develop CircuitAR, a model trained using masked autoregressive modeling, which generates initial circuit structures conditioned on given truth tables;
\item Propose a refinement framework that integrates differentiable architecture search to produce functionally equivalent circuits guided by target truth tables;
\item Comprehensive experiments demonstrating the scalability and capability emergence of our CircuitAR and the superior performance of the proposed circuit generation approach.
\end{itemize}

% Motivation
% (a) Diffusion (Vision, Graph), Autoregressive (Language, Vision)
% (b) Circuit Generation for Predefined Setting
% (c) Neural Architecture Search for Strict Logic Equivalence

% Contribution
% (a) Circuit Tokenizer (new transformer arch, training strategy)
% (b) CircuitAR (train and gen strategies, post-ar strategy)
% (c) Extensive Evaluation including BitD (Bit Distance) for Scalability


% \section{Preliminaries \& Empirical Studies}\label{sec:problem}
\vspace{-0.12in}
\section{Representation Learning via diffusion models}\label{sec:problem}
\vspace{-0.12in}
\subsection{Problem Formulation}

% We begin by formulating the problem of dynamic benchmarking for LLMs.
A dynamic benchmark is defined as  
$
\small
\mathcal{B}_{\text{dynamic}} = (\mathcal{D}, T(\cdot)), \quad 
\mathcal{D} = (\mathcal{X}, \mathcal{Y}, \mathcal{S}(\cdot))
$
where \( \mathcal{D} \) represents the static benchmark dataset. 
% consisting of input prompts \( \mathcal{X} \), expected outputs \( \mathcal{Y} \), and a scoring function \( \mathcal{S}(\cdot) \) that evaluates the quality of an LLM's outputs by comparing them against \( \mathcal{Y} \). 
The transformation function \( T(\cdot) \) modifies the data set during the benchmarking to avoid possible data contamination.
The dynamic dataset for the evaluation of an LLM can then be expressed as
$
\small
        \mathcal{D}_t = T_t(\mathcal{D}),  \quad
        \forall t \in \{1, \dots, N\}
$
where \( \mathcal{D}_t \) represents the evaluation data set at the timestamp \( t \), and \( N\) is the total timestamp number, which could be finite or infinite. % \ie $N= \infty$.
If the seed dataset $\mathcal{D}$ is empty, the dynamic benchmarking dataset will be created from scratch.



\vspace{-0.15in}
\section{Theoretical Understanding Through Low-Dimensional Models}\label{sec:main}
\vspace{-0.1in}
\documentclass{MITstyle}

%\usepackage[table]{xcolor}
\usepackage{chngcntr}
\usepackage{hyperref}
\usepackage{microtype}

\title{A Lightweight and Extensible Cell Segmentation and Classification Model for Whole Slide Images}

\author{Nikita Shvetsov~$^{1, }$\footnote{Correspondence e-mail: nikita.shvetsov@uit.no}, Thomas K. Kilvaer~$^{2, 3}$, Masoud Tafavvoghi~$^{4}$, Anders Sildnes~$^{1}$, \\ Kajsa Møllersen~$^{4}$, Lill-Tove Rasmussen Busund~$^{5, 6}$, Lars Ailo Bongo~$^{1}$ \\
%
\vspace{1em} % Space between authors and afilliations
%
\normalfont{\small $^{1}$Department of Computer Science, UiT The Arctic University of Norway}\\
\normalfont{\small $^{2}$Department of Oncology, University Hospital of North Norway}\\
\normalfont{\small $^{3}$Department of Clinical Medicine, UiT The Arctic University of Norway}\\
\normalfont{\small $^{4}$Department of Community Medicine, UiT The Arctic University of Norway}\\
\normalfont{\small $^{5}$Department of Medical Biology, UiT The Arctic University of Norway} \\
\normalfont{\small $^{6}$Department of Clinical Pathology, University Hospital of North Norway} %\vspace{2em}
}

\begin{document}
\maketitle

\section*{Abstract}

% \begin{abstract}
% Developing clinically useful cell-level analysis tools in digital pathology remains challenging due to limitations in dataset granularity, inconsistent annotations, computational demands of advanced models, and difficulties in integrating new technologies into clinical workflows. To address these challenges, we propose a multi-faceted solution that enhances data quality, model performance, and usability to create a lightweight and extensible cell segmentation and classification model.

% First, we update data labels by employing a cross-relabeling process that refines the labels of two existing datasets, PanNuke and MoNuSAC, to create a new unified dataset with enhanced granularity, encompassing seven distinct cell types. Second, we leverage the H-Optimus foundation model as a fixed encoder to improve feature representation for simultaneous cell segmentation and classification tasks. Third, to address the computational demands of foundation models, we employ knowledge distillation to reduce model size and complexity while maintaining comparable performance. Finally, to facilitate integration into clinical workflows, we integrate the distilled model into the QuPath software, a widely used open-source platform in digital pathology.

% Our results demonstrate improvements in cell segmentation and classification performance using the H‑Optimus-based model compared to a CNN-based model. Specifically, the average $R^2$ improved from 0.575 to 0.871, and the average $PQ$ score improved from 0.450 to 0.492, indicating better alignment with actual cell counts and enhanced segmentation and classification quality. Furthermore, the distilled student model maintains performance comparable to the larger foundation model while reducing the parameter count by a factor of 48.
% Overall, by reducing computational complexity and integrating it into existing workflows, the proposed approach may significantly impact diagnostic processes, reduce the workload of pathologists, and contribute to improved patient outcomes. Though our approach shows potential enhancements in efficiency and usability of cell segmentation and classification models in digital pathology, extensive validation is needed to deploy these models in clinical practice.
% \end{abstract}

%%% shortened abstract
\begin{abstract}
Developing clinically useful cell-level analysis tools in digital pathology remains challenging due to limitations in dataset granularity, inconsistent annotations, high computational demands, and difficulties integrating new technologies into workflows. To address these issues, we propose a solution that enhances data quality, model performance, and usability by creating a lightweight, extensible cell segmentation and classification model. 

First, we update data labels through cross-relabeling to refine annotations of PanNuke and MoNuSAC, producing a unified dataset with seven distinct cell types. Second, we leverage the H-Optimus foundation model as a fixed encoder to improve feature representation for simultaneous segmentation and classification tasks. Third, to address foundation models' computational demands, we distill knowledge to reduce model size and complexity while maintaining comparable performance. Finally, we integrate the distilled model into QuPath, a widely used open-source digital pathology platform. 

Results demonstrate improved segmentation and classification performance using the H-Optimus-based model compared to a CNN-based model. Specifically, average $R^2$ improved from 0.575 to 0.871, and average $PQ$ score improved from 0.450 to 0.492, indicating better alignment with actual cell counts and enhanced segmentation quality. The distilled model maintains comparable performance while reducing parameter count by a factor of 48. By reducing computational complexity and integrating into workflows, this approach may significantly impact diagnostics, reduce pathologist workload, and improve outcomes. Although the method shows promise, extensive validation is necessary prior to clinical deployment.
\end{abstract}
\clearpage

\section{Introduction}
In digital pathology, accurate segmentation and classification of cells are crucial for many diagnostic, prognostic, and predictive analyses \cite{Jaber_Beziaeva_etal._2019,Lin_Pan_etal._2022,Park_Ock_etal._2022,Shen_Choi_etal._2024}. Nowadays, developments in computational pathology offer multiple solutions \cite{H._Qu_P._Wu_etal._2020,Javed_Mahmood_etal._2020} to utilize cell-level datasets to train machine learning models that solve these problems. The quality and specificity of training datasets are critical for robust and accurate models. Adhering to the principle of "garbage in, garbage out", it is essential to ensure that these datasets are extensively and accurately labeled with distinct classes that reflect the diverse biological characteristics of different cell types. Unfortunately, the number of open-source datasets comprising such high-quality annotations is limited. Existing cell segmentation datasets \cite{Gamper_Koohbanani_etal._2019,Graham_Vu_etal._2019,Verma_Kumar_etal._2021} may offer extensive annotations for certain cell types while providing more general labels for others. For example, in PanNuke, which is one of the largest open-source datasets comprising labeled cells, various types of morphologically and functionally different inflammatory cells like macrophages and lymphocytes are clustered in a broad "inflammatory" class. Consequently, these classes are frequently omitted from analyses or aggregated into broader meta-classes \cite{Gamper_Koohbanani_etal._2020} and likely interfere with other cell classes included in the dataset. This and similar inconsistencies in annotation granularity limit the ability of machine learning models to learn the comprehensive and nuanced features necessary for accurate cell segmentation and classification. To address these challenges, methods for refining and standardizing dataset annotations are essential to enhance the quality of training data.

A complementary approach to mitigate the absence of high-quality training data is the use of foundation models. Foundation models as encoders are defined as large-scale, versatile networks pre-trained on vast, diverse datasets using self-supervised learning, contrasting with convolutional neural network (CNN) pre-trained encoders that rely on supervised learning with labeled data. In practice, foundation models leverage enormous amounts of weakly or unlabeled data from millions of whole slide images (WSIs) and employ self-attention mechanisms to capture long-range dependencies and global context \cite{Chen_Ding_etal._2024,Saillard_Jenatton_etal._2024,Vorontsov_Bozkurt_etal._2024,Xu_Usuyama_etal._2024}. As a consequence, foundation models are able to produce transferable feature representations across different cell types and tissue environments. The feature representations can be leveraged by decoder networks to produce segmentation masks and pixel-level classifications. Because foundation models have comprehensive feature representations, they can be effectively fine-tuned using much smaller amounts of cell-level data compared to the large datasets needed to train models from scratch. Furthermore, foundation models incorporate adversarial training elements or contrastive learning \cite{Chen_Ding_etal._2024,Xu_Usuyama_etal._2024}, enhancing their resilience and adaptability by exposing them to challenging and varied scenarios during training. This may result in more generalizable models, often making them well-suited for diverse and complex tasks in digital pathology.

Despite the inherent advantages of foundation models, their deployment for practical use faces its own obstacles. In particular, they require substantial computational power, financial investments and rigorous testing to ensure reliability and efficacy for a given task \cite{Akkus_Dangott_etal._2022,Dragomir_Cocuz_etal._2022,Go_2022,Jafri_Farooqui_etal._2024}. Moreover, while foundation models enhance feature representation and performance, they depend on the quality of available annotations for decoder fine-tuning and, like any other model, cannot resolve existing inconsistencies or ambiguities in data labels. Therefore, there remains a critical need for solutions that address both data quality and practical deployment considerations.
Further, integrating new technologies into existing clinical workflows often encounters resistance, as it necessitates adjustments to established diagnostic processes. So, there is a need to develop solutions that could be integrated into current practices, minimizing the burden on medical professionals to adopt new tools \cite{King_Williams_etal._2023}.

Existing solutions \cite{Goldsborough_Philps_etal._2024,Hörst_Rempe_etal._2024}, while addressing some aspects of these challenges, fall short in providing a comprehensive approach. To address the data quality and clinical deployment issues, we propose a multi-faceted solution that encompasses data refinement, model optimization, and integration with existing pathology tools (\hyperref[fig:fig1]{Figure 1}). The outcome is a lightweight cell segmentation and classification model that can be integrated into digital pathology workflows for practical clinical use.

\begin{figure}[h!]
    \centering
    \includegraphics[width=\textwidth, height=0.82\textheight, keepaspectratio]{images/Figure_1.pdf}
    \caption{Overview of the proposed solution, including 1) Data refinement using cross-relabeling, 2) Teacher model development and fine tuning, 3) Student model optimization with knowledge distillation and 4) Student model and QuPath integration}
    \label{fig:fig1}
\end{figure}
\clearpage

Our approach begins with preparing the data for the fine-tuning and training of the machine learning models. We create a refined dataset, acquired via cross-relabeling two cell-level datasets, enhancing annotation specificity and consistency of the labeled data. Subsequently, we create a cell segmentation and classification model based on the foundation model. We leverage the foundation model as a fixed encoder and fine-tune a decoder using the refined dataset to improve generalization across diverse tissue- and cell types.
To ensure that the model remains lightweight and deployable in a possibly resource-constrained environment, we employ knowledge distillation to approximate the functionality of the foundation model. Finally, to facilitate the practical application of our model in digital pathology workflows, we integrate it with the QuPath \cite{Bankhead_Loughrey_etal._2017} application. Each methodological component contributes to the overarching goal of enhancing model performance, generalizability, and usability in clinical settings.

The primary contributions of this paper are:
\begin{enumerate}
    \item \textit{Data labels refinement through cross-relabeling:}
    
    We propose a new method for refining labels of cell-level datasets through cross-relabeling. This method employs classification models to re-label broad and ambiguous instances, resulting in a more diverse dataset. Our evaluation demonstrates that these classification models achieve high accuracy on test subsets, indicating the reliability of the method for label refinement.

    \item \textit{Enhanced model performance via foundation models:}
    
    We employ a foundation model as a feature extractor for the cell segmentation and classification task. In comparison with training a CNN model from scratch, the foundation model backbone only needs fine-tuning, which significantly reduces training time, computational resources and data requirements. We show that using a foundation model encoder leads to better performance in cell segmentation and classification networks than using a CNN-based encoder. This improvement may enable the model to generalize more effectively across various tissue types and imaging methods.
    
    \item \textit{Model optimization through knowledge distillation:}
    
    We show that a smaller student model trained using knowledge distillation on the refined dataset obtained via our cross-relabeling approach from a foundation model achieves comparable performance in cell segmentation and quantification tasks. As a result, this model is more suitable for deployment in environments without high-performance computing resources.
    
    \item \textit{Integration with QuPath:}
    
    We integrate the distilled cell segmentation and classification model into QuPath, a widely used open-source digital pathology platform, to accelerate clinical adaptation by enabling pathologists to more easily incorporate advanced computational tools into their existing workflows.
\end{enumerate}

Through these methodological steps, we aim to bridge the gap between advanced machine learning techniques and practical clinical applications, making accurate and efficient digital pathology accessible in a broader range of healthcare settings.

\section{Refining Existing Datasets Using Cross-Relabeling}
To address the limitations of sparse and ambiguous labeling of cell-level datasets, we propose a generalizable cross-relabeling strategy that can be applied to any dataset containing broadly categorized or imprecisely labeled cell types. This approach involves training and subsequently leveraging classification models to refine broad categories into more specific or biologically relevant classes.
When applied to cell-level data, the methodology includes extracting individual cell images from the dataset patches, preprocessing these images to standardize the size and accommodate partial cells, and then training deep learning classifiers capable of distinguishing between the finer cell subtypes within the coarser categories. 
To illustrate our approach, we focus on the PanNuke \cite{Gamper_Koohbanani_etal._2020, Gamper_Koohbanani_etal._2019} and MoNuSAC \cite{Verma_Kumar_etal._2021} datasets that we have used to train models for cell quantification in our previous works \cite{Shvetsov_Grønnesby_etal._2022,Shvetsov_Sildnes_etal._2024}. We find that for better cell differentiation we have to introduce more granular labels. PanNuke includes a broad classification of "inflammatory" cells, encompassing lymphocytes, macrophages, and neutrophils. Each cell type differs significantly in structure, function, and clinical relevance. Conversely, MoNuSAC uses the label "epithelial" for a class that comprises both benign epithelial cells and malignant neoplastic cells. This practice makes it challenging to differentiate between benign and malignant epithelial cells in the dataset, which is a critical distinction when identifying tumor areas within tissue samples. To address these issues, we implement a cross-relabeling strategy as shown in \hyperref[fig:fig2]{Figure 2}. The key components are two classification models: one is trained on singular cell images from PanNuke data to classify the epithelial meta-class into epithelial and neoplastic classes. The other is trained on MoNuSAC to refine the inflammatory class into lymphocytes, neutrophils, and macrophages.

\begin{figure}[h!]
    \centering
    \includegraphics[width=\textwidth]{images/Figure_2.pdf}
    \caption{Refined dataset generation via cross relabeling}
    \label{fig:fig2}
\end{figure}

The refining approach consists of three consecutive steps. The first is the preprocessing step, in which we extract individual cells from both datasets (\hyperref[fig:fig3]{Figure 3}). The specifics of PanNuke and MoNuSAC patch preparation before cell preprocessing are provided in \hyperref[chap:S1]{Appendix S1}.

\begin{figure}[h!]
    \centering
    \includegraphics[width=\textwidth]{images/Figure_3.pdf}
    \caption{Cell instances preprocessing including (1) cell map extraction, (2) bounding box delineation, (3) adjusting cell boxes and (4) cropping and resizing of cell images}
    \label{fig:fig3}
\end{figure}

During preprocessing, we extract cell type maps from the ground truth label mask and calculate bounding boxes around each cell instance. To accommodate partial cells at patch borders, a common issue in cropped patch images, we employ mirror padding and extend the field of view of the cell label by 15 pixels to capture adjacent cells. We then crop and resize the identified regions to $64 \times 64$ pixels using bicubic interpolation.

The preprocessed PanNuke dataset comprises 68,031 neoplastic and 23,207 epithelial cell images, while MoNuSAC comprises  33,104 lymphocytes, 1,252 neutrophils, and 1,695 macrophages, which we subsequently use in training cell classification models and classifying the cell image data \hyperref[fig:S2]{Appendix Figure S2 (1)}. 

The next step is to train two distinct ResNet50-based classifiers tailored to address the specific labeling challenges inherent in each dataset. We use ResNet50 for classification models due to its proven effectiveness for image classification tasks in histopathology \cite{pan2022reviewmachinelearningapproaches}, and its compatibility with small images. For the PanNuke dataset, we design the classifier, trained on MoNuSAC data, to disaggregate the heterogeneous "inflammatory" cell category into distinct subtypes: lymphocytes, macrophages, and neutrophils. Similarly, for the MoNuSAC dataset, the classifier is trained on PanNuke data and distinguishes between benign and malignant epithelial cells within the overarching "epithelial" label. By applying these targeted classifiers to their respective datasets, we assign more specific labels to individual cell instances, thus enabling us to create a unified dataset.
To ensure a balanced representation of classes, we train both models on datasets that had been equalized to match the size of the least represented class. Thus, we obtain datasets comprising 23,207 samples per class for PanNuke and 1,252 samples per class for MoNuSAC data. Next, we partition both of them into training (70\%), validation (20\%), and testing (10\%) subsets. To mitigate the risk of overfitting, we use a single dropout layer with a rate of p=0.5 in both models and data augmentation using randomized color perturbations, rotation, and horizontal and vertical flipping. We employ AdamW optimizer and the cross-entropy loss function for the training criterion.

To evaluate the two trained models, we measure the classification accuracy on the respective test subsets. The accuracies on the test subset for both classifiers are presented in \hyperref[tab:1]{Table 1}. The PanNuke model achieves an average accuracy of 93.57\%, with higher accuracy for neoplastic cells (96.06\%) compared to epithelial cells (86.26\%). The confusion matrix in Figure A3.1 shows that the model predominantly distinguishes accurately between epithelial and neoplastic tissues, with a substantial number of correct classifications and relatively few misclassifications. The MoNuSAC model demonstrates an average accuracy of 98.92\%, excelling in classifying lymphocytes (99.67\%) and macrophages (94.12\%), with lower performance for neutrophils (85.71\%). The confusion matrix in Figure A3.2 shows that the model identifies lymphocytes and performs reasonably well with macrophages and neutrophils.

\begin{table}[h!]
\renewcommand{\arraystretch}{1.5}
  \centering
  \caption{Cell classification results for PanNuke and MoNuSAC trained models (CI 95\%).}
  \label{tab:1}
  \begin{tabular}{|l|c|c|}
   \hline
   %\rowcolor{gray!30}
    Accuracy               & PanNuke model              & MoNuSAC model              \\
    \hline
    Average      & 0.936 (0.931--0.941)         & 0.989 (0.986--0.993)        \\
    \hline
    Neoplastic   & 0.961 (0.956--0.965)         & -                          \\
    \hline
    Epithelial   & 0.863 (0.849--0.877)         & -                          \\
    \hline
    Lymphocytes  & -                          & 0.997 (0.995--0.999)        \\
    \hline
    Neutrophils  & -                          & 0.857 (0.796--0.918)        \\
    \hline
    Macrophages  & -                          & 0.941 (0.906--0.976)        \\
    \hline
  \end{tabular}
\end{table}

Finally, during the last step, we use the model trained on PanNuke data for epithelial cells in MoNuSAC and the model trained on MoNuSAC for the inflammatory cells class in PanNuke. Specifically, we use classifier models to relabel epithelial cells in MoNuSAC and inflammatory cells in PanNuke data. Then we combine cells with refined labels and the rest of the cells in both datasets to create a refined dataset (\hyperref[fig:S2]{Appendix Figure S2 (2)}). The process of relabeling cells and visualizing them on a patch is shown in \hyperref[fig:fig4]{Figure 4}. The cell counts in the refined dataset are provided in \hyperref[tab:S4]{Appendix Table S4}.

\begin{figure}[h!]
    \centering
    \includegraphics[width=\textwidth, height=0.42\textheight, keepaspectratio]{images/Figure_4.pdf}
    \caption{Cell relabeling procedure for epithelial and inflammatory cell classes}
    \label{fig:fig4}
\end{figure}

%\hfill

Relabeling and combining datasets have been explored in a prior study \cite{Parulekar_Kanwat_etal._2023}, where consecutive fine-tuning on multiple datasets was employed to account for hierarchical class label structures. While the method presented in \cite{Parulekar_Kanwat_etal._2023} is intuitive, it often lacks consistency and requires multiple fine-tuning runs, which can be cumbersome and time-consuming. 
In contrast, cross-relabeling simplifies this process by using specialized classification models tailored to each dataset's specific labeling challenges. This approach provides better transparency and produces a unified dataset encompassing seven distinct cell types across multiple tissue samples, enhancing data diversity for further model training or fine-tuning.

Despite these improvements, cross-relabeling does not entirely resolve issues related to poor labeling quality or the amount of labeled data. Specifically, our results show lower accuracies persist for underrepresented classes, such as macrophages, which may stem from a limited sample availability and intrinsic challenges in distinguishing these cells based solely on H\&E staining. Furthermore, while our method enhances label specificity, it relies on the initial quality of the broad labels; thus, any fundamental inaccuracies in the original annotations can propagate through the relabeling process. Addressing the overall problem of limited data labels may require integrating additional data sources or utilizing complementary immunohistochemical staining methods.
Although the reported performance metrics are obtained from evaluations on the native test sets of each dataset, it is important to note that the primary application of these classifiers is to perform cross-relabeling, where a model trained on one dataset (e.g., PanNuke) is applied to another (e.g., MoNuSAC) and vice versa. We acknowledge that a more systematic evaluation of cross-dataset generalization is needed and could be performed in future work.

Overall, the refined dataset produced by our approach can enhance the supervised training or fine-tuning of cell segmentation and classification models, especially those that utilize pre-trained foundation models to improve feature extraction robustness. In addition, these models can detect nuanced classes that enable researchers to conduct more detailed analyses of biological processes in computational pathology.

\section{Foundation models for robust cell segmentation and classification}

Accurate cell segmentation and classification in digital pathology are hindered by limited labeled data and the fact that conventional CNNs are unable to capture global contextual information due to their local receptive field constraints \cite{Gheflati_Rivaz_2022,Yang_Marcus_etal.}. Traditional approaches in cell quantification have predominantly relied on CNN encoders, such as ResNet50, given their proven effectiveness in semantic segmentation tasks \cite{Deshmane_2023,Graham_Vu_etal._2019,Mukasheva_Koishiyeva_etal._2024,Stringer_Wang_etal._2021}. However, approaches that include fine-tuning of pretrained CNNs, data augmentation, and stain normalization to partially increase data variability and address staining differences often fail to achieve the necessary generalization and robustness across diverse tissue types and staining conditions \cite{G._Wang_W._Li_etal._2018,Gao_Bagci_etal._2018,Karim_El_Khoury_Martin_Fockedey_etal._2021}.

To overcome these challenges, we leverage an encoder-decoder network that uses a foundation model as the encoder and a CNN upsampling decoder (\hyperref[fig:fig5]{Figure 5}) for simultaneous cell segmentation and classification in 2D patches extracted from WSIs. Foundation models with transformer-based architectures are viable alternatives to CNN-based encoders \cite{Shamshad_Khan_etal._2023,Sourget_2023}. They enable the creation of more advanced architectures that can decode or transform learned features more effectively \cite{Chen_Duan_etal._2023,Cheng_Misra_etal._2022,Xie_Wang_etal._2021}.

\begin{figure}[h!]
    \centering
    \includegraphics[width=\textwidth]{images/Figure_5.pdf}
    \caption{UNETR-like model with foundational model as backbone}
    \label{fig:fig5}
\end{figure}

By utilizing a transformer-based encoder, we incorporate global contextual information into the feature extraction process, which is a key advantage of such architectures \cite{Chen_Lu_etal._2021}. This foundation model integration facilitates accurate pixel-wise segmentation and classification without the need for extensive encoder training, thereby potentially improving generalization across varied cellular structures and tissue types.
In our implementation, we employ a modified UNETR \cite{Hatamizadeh_Tang_etal._2021} architecture that combines a vision transformer (ViT) \cite{Dosovitskiy_Beyer_etal._2021} encoder with a CNN-based decoder. The encoder utilizes the pretrained H-Optimus foundation model, which contains 1.1 billion parameters and is trained on over 500,000 H\&E stained WSIs \cite{Saillard_Jenatton_etal._2024}. We extract outputs from four evenly spaced transformer blocks $Z_i$, where $i \in [1, 14, 26, 38]$, to serve as residual connections for the CNN decoder. We select these blocks based on our observation that features from non-adjacent levels of the encoder lead to better overall performance on the test subset.

The CNN decoder upsamples the feature representations, acquired from the transformer blocks, to generate an intermediate vector that is handled by two task-specific layers that generate cell segmentation and classification masks. The first task-specific layer is the ‘Cellpose head’,  which is used to delineate cell instances. The layer generates horizontal and vertical gradient maps to form vector fields that are refined through gradient tracking in a post-processing step using the Cellpose algorithm \cite{Stringer_Wang_etal._2021}, known for its efficacy in cell segmentation tasks and generalizability across multiple domains \cite{Pachitariu_Stringer_2022,Stringer_Pachitariu_2024}. The second task-specific layer is the "Cell type head", which assigns labels to individual pixels. In the post-processing step, we determine the output classification label of each segmented cell instance by majority voting over the labeled pixels that comprise the cell in the segmentation map.

To evaluate model performance and measure the impact of adding a foundation model as backbone, we compare it to a ResNet50-based model. ResNet50 is a widely used solution for encoders in segmentation architectures in the medical domain \cite{Deshmane_2023,Graham_Vu_etal._2019,Mukasheva_Koishiyeva_etal._2024,Stringer_Wang_etal._2021}. For the H-Optimus-based model, we utilize frozen weights for the encoder and only fine-tune the decoder to take advantage of the extensive pre-training of the foundation model. For the ResNet50-based model we start with ImageNet \cite{Deng_Dong_etal.} weights and train both encoder and decoder parts. Hyperparameters for the training step are set to be identical, where possible, for comparable evaluation. 
For this evaluation, we deliberately use the PanNuke dataset to provide a standardized and controlled comparison between the H‑Optimus and ResNet50-based models (\hyperref[fig:S2]{Appendix Figure S2 (3)}). Specifically, we use two of the default PanNuke dataset splits (66\%) for training and validation, and reserve the third split (33\%) for testing.

To address the challenge of cell class imbalance in the PanNuke dataset, which is a common characteristic in most cell-level H\&E patch datasets, both models’ training processes employ a weighted loss function comprising cross-entropy and focal loss \cite{Lin_Goyal_etal._2018}. The focal loss component is adjusted with coefficients derived from each cell class' instance frequency, emphasizing learning from underrepresented classes and enhancing the model's sensitivity to rare but significant cellular patterns. The cross-entropy loss is augmented with spectral decoupling regularization \cite{Pezeshki_Kaba_etal._2021,Pohjonen_Stürenberg_etal._2022} and spatially varying label smoothing \cite{Islam_Glocker_2021}, which potentially stabilizes training and improves generalization in case of complex tissue morphologies. For optimization, we employ the \textit{AdamW} \cite{Loshchilov_Hutter_2019} to counter unbalanced class scenarios, with cosine annealing learning rate scheduler.

We utilize the scikit-learn library \cite{Van_der_Walt_Schönberger_etal._2014} and HoVer-Net \cite{Graham_Vu_etal._2019} implementations of $R^2$ (the coefficient of determination) and $PQ$ (panoptic quality) to evaluate our experiments. Complete mathematical formulations and detailed explanations of these metrics are provided in \hyperref[chap:S5]{Appendix S5}. To compute confidence intervals, we use nonparametric bootstrapping, where after calculating the metric on the full sample, we generated 1000 bootstrap replicates by resampling with replacement and then determined the 95\% confidence intervals as the 2.5th and 97.5th percentiles of the resulting empirical distribution.

%\hfill

The model comparisons are summarized in \hyperref[tab:2]{Table 2}. The H‑Optimus-based model achieves higher $R^2$ across all cell classes compared to the ResNet50-based model, which means that its predictions are more closely aligned with the PanNuke cell counts, indicating a stronger correlation with the observed data. Notably, the improvement of $R^2_{dead}$ may be an indicator of better global contextual representations provided by the foundation model backbone. In terms of segmentation and classification quality combined, measured by the PQ score, the H‑Optimus-based model demonstrates notable improvements across most cell classes. Overall, the average $R^2$ improved from 0.575 to 0.871, while the average $PQ$ score improved from 0.450 to 0.492, demonstrating better performance of the H-Optimus-based model.

\begin{table}[h!]
\renewcommand{\arraystretch}{1.5}
  \centering
  \caption{Cell quantification metrics for baseline and proposed models (CI 95\%).}
  \label{tab:2}
  \begin{tabular}{|l|c|c|}
    \hline
    %\rowcolor{gray!30}
    Metric             & Resnet50-based            & H-optimus-based              \\
    \hline
    $R^2_{neoplastic}$    & 0.681 (0.576--0.769)       & \textbf{0.941 (0.917--0.960)} \\
    \hline
    $R^2_{inflammatory}$  & 0.863 (0.778--0.903)       & \textbf{0.949 (0.918--0.966)} \\
    \hline
    $R^2_{connective}$    & 0.600 (0.488--0.698)       & 0.609 (0.436--0.772)          \\
    \hline
    $R^2_{dead}$          & 0.097 (-11.389--0.669)     & 0.925 (0.404--0.982)          \\
    \hline
    $R^2_{epithelial}$    & 0.635 (0.490--0.747)       & \textbf{0.930 (0.886--0.964)} \\
    \hline
    $PQ_{neoplastic}$       & 0.517 (0.499--0.535)       & \textbf{0.589 (0.575--0.604)} \\
    \hline
    $PQ_{inflammatory}$     & 0.455 (0.429--0.482)       & \textbf{0.528 (0.507--0.549)} \\
    \hline
    $PQ_{connective}$       & 0.416 (0.400--0.431)       & \textbf{0.451 (0.436--0.465)} \\
    \hline
    $PQ_{dead}$             & 0.374 (0.342--0.408)       & 0.292 (0.209--0.365)          \\
    \hline
    $PQ_{epithelial}$       & 0.488 (0.460--0.519)       & \textbf{0.599 (0.579--0.618)} \\
    \hline
  \end{tabular}
\end{table}

Our results  show that integrating the H‑Optimus foundation model within the UNETR architecture enhances the model's ability to segment and classify cells across diverse tissues from PanNuke data. The pretrained transformer encoder provides robust feature representations, resulting in higher average $R^2$ and $PQ$ scores compared to the CNN-based model. This leads to more reliable cell quantification and more accurate downstream analysis. Additionally, the streamlined fine-tuning process reduces computational overhead and training time, making the model more adaptable for new data.

Despite these advancements, the foundation model-based approach does not fully resolve all challenges related to cell segmentation and classification. We observe lower metric scores for underrepresented classes in the training data. Furthermore, foundation models typically encompass billions of parameters, resulting in substantial computational and memory requirements. It therefore poses challenges for deployment in resource-constrained environments, limiting their practical applicability in certain clinical settings.

\section{Model optimization via Knowledge Distillation}

To address the limitations posed by the extensive size of foundation models, we implement knowledge distillation — a model compression technique that leverages the teacher-student paradigm \cite{Hinton_Vinyals_etal._2015}. By training a smaller, more efficient student model to replicate the output of a larger, pre-trained teacher model, we retain performance while significantly reducing the model's complexity and resource requirements (\hyperref[fig:fig6]{Figure 6}).

\begin{figure}[h!]
    \centering
    \includegraphics[width=\textwidth, height=0.45\textheight, keepaspectratio]{images/Figure_6.pdf}
    \caption{Knowledge distillation framework for training a student model using a pre-trained teacher}
    \label{fig:fig6}
\end{figure}

We employ knowledge distillation to compress the H‑Optimus-based teacher model into a more efficient student model. The teacher model is the modified UNETR architecture with the H‑Optimus foundation model described in the previous chapter. The student model is based on a UNet architecture augmented with residual connections and incorporates a smaller ViT encoder with 9 million parameters \cite{Steiner_Kolesnikov_etal._2022,Wightman_2019}. 

First, we fine-tune the teacher model using the refined dataset from the cross-relabeling procedure (Section 2). Initially we train the decoder of the teacher model while keeping the encoder weights frozen. We split the refined dataset into train (70\%), validation (20\%) and test (10\%) subsets (\hyperref[fig:S2]{Appendix Figure S2 (4)}). During fine-tuning, we use the train and validation subsets, while leaving the test subset for model evaluation. We set the training procedure and model hyperparameters to be identical to those that were used to demonstrate the utility of foundation models for the simultaneous cell segmentation and classification task.

Next, we perform knowledge distillation from teacher to student using the refined dataset used to fine-tune the teacher model. The student model is trained to replicate the teacher model's outputs. We utilize a specialized loss function that aligns the student's predicted probability distribution with the teacher's, incorporating the teacher's class probability distribution derived from the output. Following the methodology of Hinton et al. \cite{Hinton_Vinyals_etal._2015}, we experiment with various hyperparameter settings for the temperature ($T$) and the balancing coefficients ($\alpha$ and $\beta$) in the loss function. We vary $T$ from 1 to 20 and adjust $\alpha$ and $\beta$ to balance the distillation and student losses. Through iterative tuning and evaluation, we identify that setting $T=14$, $\alpha=0.3$, and $\beta=0.7$ yields a configuration that converges and closely approximates the teacher model's performance during training.

Finally, we assess the performance of both models using the $R^2$ and $PQ$ (defined in \hyperref[chap:S5]{Appendix S5}) on the test set of the refined dataset (\hyperref[tab:3]{Table 3}). We observe that the 95\% confidence intervals overlap for most cell types, so we cannot claim statistically significant performance differences between the teacher and student models. One exception appears in the neoplastic class. The teacher model produces an $R^2$ of 0.919, while the student model shows an $R^2$ of 0.852. In addition, the student model achieves higher $PQ$ values for the neoplastic and connective classes, though the confidence intervals show overlap.

\begin{table}[h!]
\renewcommand{\arraystretch}{1.5}
  \centering
  \caption{Cell quantification metrics for teacher and distilled student models (CI 95\%).}
  \label{tab:3}
  \begin{tabular}{|l|c|c|}
    \hline
    %\rowcolor{gray!30}
    Metric & Teacher & Student \\
    \hline
    $R^2_{neoplastic}$    & \textbf{0.919} (0.898--0.939) & 0.852 (0.800--0.891) \\
    \hline
    $R^2_{lymphocyte}$    & 0.969 (0.956--0.977)         & 0.969 (0.956--0.978) \\
    \hline
    $R^2_{connective}$    & 0.694 (0.548--0.809)         & 0.618 (0.469--0.741) \\
    \hline
    $R^2_{dead}$          & 0.755 (0.400--0.908)         & 0.424 (0.100--0.731) \\
    \hline
    $R^2_{epithelial}$    & 0.922 (0.870--0.958)         & 0.843 (0.738--0.917) \\
    \hline
    $R^2_{macrophage}$    & 0.384 (-0.369--0.724)        & 0.704 (0.352--0.859) \\
    \hline
    $R^2_{neutrofil}$     & 0.854 (0.578--0.929)         & 0.833 (0.502--0.925) \\
    \hline
    $PQ_{neoplastic}$       & 0.581 (0.569--0.593)         & 0.601 (0.588--0.613) \\
    \hline
    $PQ_{lymphocyte}$       & 0.536 (0.520--0.553)         & 0.563 (0.544--0.579) \\
    \hline
    $PQ_{connective}$       & 0.436 (0.421--0.451)         & 0.457 (0.441--0.474) \\
    \hline
    $PQ_{dead}$             & 0.272 (0.235--0.315)         & 0.279 (0.201--0.369) \\
    \hline
    $PQ_{epithelial}$       & 0.522 (0.500--0.545)         & 0.530 (0.506--0.555) \\
    \hline
    $PQ_{macrophage}$       & 0.524 (0.459--0.588)         & 0.474 (0.405--0.543) \\
    \hline
    $PQ_{neutrofil}$        & 0.541 (0.490--0.592)         & 0.565 (0.522--0.607) \\
    \hline
  \end{tabular}
\end{table}


We further decompose the $PQ$ metric into its $SQ$ and $DQ$ components (\hyperref[tab:S6]{Appendix Table S6}). Both models produce nearly identical $SQ$ values, which indicates that they predict instance boundaries with similar precision. Although the student model shows some improvement in $DQ$ scores for certain classes, the confidence intervals overlap and do not confirm a statistically significant difference.

We observe that the student and teacher models yield comparable detection performance despite the student model using a much smaller and simpler architecture. A model with fewer parameters reduces the risk of overfitting when training data are scarce relative to the model’s complexity \cite{Farias_Ludermir_etal._2022}. The knowledge distillation process also encourages the student model to focus on the most generalizable detection features learned from the teacher. These factors enable the student model to achieve similar detection performance across different cell types.

Additionally, considering the model sizes reported in \hyperref[tab:4]{Table 4}, the distilled model achieves a significant reduction compared to the teacher model, with a 48-fold decrease in parameter count and a 5.5-fold reduction in on-disk size. In inference mode, the teacher model requires 16 GB of VRAM for a batch size of 32, while the distilled model only needs 3 GB of VRAM for the same batch size. These reductions make the distilled model significantly more practical for fine-tuning and deployment in resource-constrained environments.

\begin{table}[h!]
\renewcommand{\arraystretch}{1.5}
  \centering
  \caption{Parameter counts and size of teacher and distilled model}
  \label{tab:4}
  \adjustbox{max width=\textwidth}{%
  \begin{tabular}{|l|c|c|c|}
    \hline
    %\rowcolor{gray!30}
    Metric & H-optimus-based (Teacher) & mobileViT-based (Student) & Magnitude of difference \\
    \hline
    Parameters count       & 1,158,917,906   & \textbf{24,093,393}   & \textbf{48x}  \\
    \hline
    Estimated Total Size (MB) & 87,912       & \textbf{15,935}    & \textbf{5.5x} \\
    \hline
  \end{tabular}%
}
\end{table}

%\hfill

With recent advancements in complex network architectures and the use of pretrained encoders to achieve state-of-the-art performance \cite{Baumann_Dislich_etal._2024,Hörst_Rempe_etal._2024} in cell segmentation and classification tasks, model size, computational complexity, and processing times have increased. This limits the scalability and accessibility of these models. As we demonstrate, this may be mitigated using knowledge distillation. Studies in the field of natural language processing have demonstrated the efficacy of knowledge distillation in retaining the capabilities of the teacher model while achieving significant reductions in size and complexity \cite{Huangpu_Gao_2024,Sun_Yu_etal.}. 

We demonstrate the feasibility of knowledge distillation in digital pathology, specifically for cell segmentation and classification tasks. Moreover, we achieve this performance while also significantly reducing the parameter count. In addressing the challenge of knowledge transfer, we found that distillation from a transformer-based model to a smaller transformer is more straightforward than attempting to map transformer features to CNN blocks. In our experiments, using a CNN-based network as a student results in worse cell quantification performance due to the structural constraints of CNN feature space dimensions. 

Although our primary approach relies on a transformer-based student model that performs well, it can be further optimized to incorporate advantages from CNN architectures. For example, employing alternative techniques such as using ViT adapters \cite{Chen_Duan_etal._2023} or $1 \times 1$ convolutions to adjust feature map sizes may be beneficial for harnessing CNN advantages like enhanced local feature extraction. Moreover, if additional performance improvements are desired, the process can be further enhanced by applying supplementary knowledge distillation techniques, such as self-distillation \cite{Zhang_Song_etal._2019} or online distillation \cite{Houyon_Cioppa_etal._2023}.

Despite these promising results, further validation on independent datasets is necessary to fully understand the model's limitations. Underrepresented classes may pose challenges when addressing complex cases. Pathologists need to validate these models to adopt them in clinical settings. While the distilled models are smaller and more deployable, a technological gap persists because pathologists traditionally rely on established methods for inspecting WSIs and diagnosing diseases. Addressing the complexities involved in deploying models for inference and supporting pathologists in adopting new tools is essential for integrating these models into clinical workflows.

\section{Model integration with QuPath}
Digital pathology tools with graphical user interfaces are essential for visualizing and analyzing WSIs. To make our student model useful in clinical pathology workflows, it needs to be integrated into a tool that enables inspecting regions, creating annotations, and providing quantitative analyses of biomarkers. Therefore, we integrate the trained student model from the previous chapter into the QuPath open‑source platform \cite{Bankhead_Loughrey_etal._2017}. QuPath provides the required annotation, visualization, and analysis tools to interpret complex histological data, including workflows for cell segmentation, classification, and quantification (\hyperref[fig:fig7]{Figure 7}). 

\begin{figure}[h!]
    \centering
    \includegraphics[width=\textwidth]{images/Figure_7.pdf}
    \caption{Visualization of model-generated cell quantification annotations (left) and the corresponding unannotated slide (right) in QuPath}
    \label{fig:fig7}
\end{figure}

To identify the regions in a WSI critical for prognosticating tumor development, such as specific tumor areas or border regions without overlapping healthy tissue, the pathologist uses QuPath to outline these regions. Then, the pathologist initiates a cell segmentation and classification script through the QuPath interface for the selected regions. The resulting annotations and quantified cell information are then directly overlaid onto the WSI in the QuPath interface. Additional design and implementation details are in \hyperref[chap:S7]{Appendix S7}. 

Two common approaches for integrating deep learning models into QuPath are Java‑based native QuPath extensions \cite{Goldsborough_Philps_etal._2024} and the execution of RESTful API requests to a model server coupled with handling the response via an extension, as demonstrated in the application of cell segmentation models applied to immunofluorescence images \cite{Sugawara_2023}. While the community is actively working on these integration strategies, there is currently no universal solution that fully addresses all integration and performance requirements.

Extensions may offer better integration with QuPath, allowing slightly improved performance and more widespread usage of the built-in QuPath models, but they lack the flexibility to customize models and modify their behavior. For example, the newest version of QuPath includes models such as StarDist \cite{Weigert_Schmidt} and InstanSeg \cite{Goldsborough_Philps_etal._2024} that can perform cell segmentation. Both models pose limitations when applied to simultaneous cell segmentation and classification. StarDist performs well only on convex, round shapes by design, whereas some neoplastic, inflammatory, and connective cells exhibit complex and non-convex shapes. InstanSeg provides only semantic segmentation without assigning classes to the segmented cells.

%\hfill

In contrast, our approach offers an alternative integration strategy. It utilizes the paquo library to directly interact with QuPath’s internal application programming interface from within Python. This enables data exchange and processing without the need for intermediate conversion steps and provides greater control over model customization, retraining, and the incorporation of custom processing steps.

The integration of our custom model with QuPath underscores its potential to significantly enhance the diagnostic process by reducing the time burden on pathologists and enabling them to focus on more complex interpretative tasks using familiar software. Leveraging a tool that is already well-established among pathologists increases the likelihood of its adoption into daily clinical workflows. The quantitative data generated through the automated workflow is critical for both clinical decision-making and research, facilitating more accurate biomarker analysis, enabling robust statistical evaluations, and supporting hypothesis generation and testing. Additionally, by streamlining cell segmentation and classification, the tool enhances the scalability and reproducibility of pathological assessments, ultimately contributing to improved diagnostic accuracy and patient outcomes.

\section{Conclusion and future work}

In this study, we address critical challenges in digital pathology and tackle the usability and deployment issues of the developed models in standard computing environments without the need for high-performance computing systems. Our multi-faceted approach encompasses data refinement through cross-relabeling, leveraging foundation models for robust cell segmentation and classification, optimizing model performance via knowledge distillation, and integrating the optimized model into the QuPath software for practical application. This approach is used to construct a capable, versatile, and adjustable model for cell segmentation and classification, with enhanced performance and usability.

\begin{sloppypar}
While our approach shows potential in the field of computational pathology, certain limitations persist. 
For example, our implementation currently exhibits lower performance in detecting macrophages. 
This serves as an instance of the broader challenge of accurately identifying complex cell types. In order to address this issue, extending our approach to incorporate additional data sources, exploring alternative modeling approaches, and integrating other imaging modalities such as immunohistochemical staining may help improve detection accuracy. Moreover, although the distilled model reduces computational demands, integrating advanced deep learning models into clinical practice requires addressing technological gaps and potential resistance to adopting new tools within established diagnostic processes.
\end{sloppypar}

Future work could focus on several key areas to refine the proposed approach and facilitate its adoption in clinical environments. Enhancing the cell-relabeling process with additional datasets \cite{Graham_Jahanifar_etal._2021} could improve the representation of underrepresented cell types and enhance overall model performance. Also, incorporating additional data sources, such as multi-modal imaging or complementary staining methods, may address limitations related to cell type differentiation and class imbalance. Exploring other foundation models \cite{Vorontsov_Bozkurt_etal._2024,Zimmermann_Vorontsov_etal._2024} or introducing additional modalities \cite{Ding_Wagner_etal._2024,Vaidya_Zhang_etal._2025} may provide alternative architectures better suited to specific tasks or offer improved efficiency. Implementing more complex knowledge distillation techniques \cite{Houyon_Cioppa_etal._2023,Zhang_Song_etal._2019} could further optimize the model's performance and adaptability. Additionally, deeper integration with QuPath or other digital pathology software could provide pathologists more control over cell quantification analysis directly within the QuPath interface, thereby increasing accessibility and usability. Such enhancements would not only refine model performance but also ensure greater adaptability and scalability within various clinical environments. Finally, extensive validation of the model by pathologists and benchmarking against independent datasets are essential steps toward establishing the model's reliability and fostering confidence in its clinical utility.

\section*{Acknowledgments} 
This work was funded in part by the Research Council of Norway grant no. 309439 SFI Visual Intelligence, and the North Norwegian Health Authority grant no. HNF1521-20.

\bibliographystyle{IEEEtran}
\begin{sloppypar}
\begin{thebibliography}{99}

\bibitem{chaplot2020neural} Chaplot, Devendra Singh, et al. "Neural topological slam for visual navigation." Proceedings of the IEEE/CVF conference on computer vision and pattern recognition. 2020.

\bibitem{maksymets2021thda} Maksymets, Oleksandr, et al. "Thda: Treasure hunt data augmentation for semantic navigation." Proceedings of the IEEE/CVF International Conference on Computer Vision. 2021.

\bibitem{mezghan2022memory} Mezghan, Lina, et al. "Memory-augmented reinforcement learning for image-goal navigation." 2022 IEEE/RSJ International Conference on Intelligent Robots and Systems (IROS). IEEE, 2022.

\bibitem{al2022zero} Al-Halah, Ziad, Santhosh Kumar Ramakrishnan, and Kristen Grauman. "Zero experience required: Plug \& play modular transfer learning for semantic visual navigation." Proceedings of the IEEE/CVF Conference on Computer Vision and Pattern Recognition. 2022.

\bibitem{ye2021auxiliary} Ye, Joel, et al. "Auxiliary tasks and exploration enable objectgoal navigation." Proceedings of the IEEE/CVF international conference on computer vision. 2021.

\bibitem{chaplot2020object} Chaplot, Devendra Singh, et al. "Object goal navigation using goal-oriented semantic exploration." Advances in Neural Information Processing Systems 33 (2020)

\bibitem{ramakrishnan2022poni} Ramakrishnan, Santhosh Kumar, et al. "Poni: Potential functions for objectgoal navigation with interaction-free learning." Proceedings of the IEEE/CVF Conference on Computer Vision and Pattern Recognition. 2022.

\bibitem{ramrakhya2022habitat} Ramrakhya, Ram, et al. "Habitat-web: Learning embodied object-search strategies from human demonstrations at scale." Proceedings of the IEEE/CVF Conference on Computer Vision and Pattern Recognition. 2022.

\bibitem{mousavian2019visual} Mousavian, Arsalan, et al. "Visual representations for semantic target driven navigation." 2019 International Conference on Robotics and Automation (ICRA). IEEE, 2019.

\bibitem{dhariwal2021diffusion} Dhariwal, Prafulla, and Alexander Nichol. "Diffusion models beat gans on image synthesis." Advances in neural information processing systems 34 (2021)

\bibitem{ho2022classifier} Ho, Jonathan, and Tim Salimans. "Classifier-free diffusion guidance." arXiv preprint arXiv:2207.12598 (2022).

\bibitem{nichol2021glide} Nichol, Alex, et al. "Glide: Towards photorealistic image generation and editing with text-guided diffusion models." arXiv preprint arXiv:2112.10741 (2021)

\bibitem{brooks2023instructpix2pix} Brooks, Tim, Aleksander Holynski, and Alexei A. Efros. "Instructpix2pix: Learning to follow image editing instructions." Proceedings of the IEEE/CVF Conference on Computer Vision and Pattern Recognition. 2023.

\bibitem{fu2023guiding} Fu, Tsu-Jui, et al. "Guiding instruction-based image editing via multimodal large language models." arXiv preprint arXiv:2309.17102 (2023).

\bibitem{geng2024instructdiffusion} Geng, Zigang, et al. "Instructdiffusion: A generalist modeling interface for vision tasks." Proceedings of the IEEE/CVF Conference on Computer Vision and Pattern Recognition. 2024.

\bibitem{zhou2024minedreamer} Zhou, Enshen, et al. "Minedreamer: Learning to follow instructions via chain-of-imagination for simulated-world control." arXiv preprint arXiv:2403.12037 (2024).

\bibitem{zhou2023esc} Zhou, Kaiwen, et al. "Esc: Exploration with soft commonsense constraints for zero-shot object navigation." International Conference on Machine Learning. PMLR, 2023.

\bibitem{yu2023l3mvn} Yu, Bangguo, Hamidreza Kasaei, and Ming Cao. "L3mvn: Leveraging large language models for visual target navigation." 2023 IEEE/RSJ International Conference on Intelligent Robots and Systems (IROS). IEEE, 2023.

\bibitem{gadre2023cows} Gadre, Samir Yitzhak, et al. "Cows on pasture: Baselines and benchmarks for language-driven zero-shot object navigation." Proceedings of the IEEE/CVF Conference on Computer Vision and Pattern Recognition. 2023.

\bibitem{shah2023navigation} Shah, Dhruv, et al. "Navigation with large language models: Semantic guesswork as a heuristic for planning." Conference on Robot Learning. PMLR, 2023.

\bibitem{cai2024bridging} Cai, Wenzhe, et al. "Bridging zero-shot object navigation and foundation models through pixel-guided navigation skill." 2024 IEEE International Conference on Robotics and Automation (ICRA). IEEE, 2024.

\bibitem{yu2023co} Yu, Bangguo, Hamidreza Kasaei, and Ming Cao. "Co-NavGPT: Multi-robot cooperative visual semantic navigation using large language models." arXiv preprint arXiv:2310.07937 (2023).

\bibitem{wu2024voronav} Wu, Pengying, et al. "Voronav: Voronoi-based zero-shot object navigation with large language model." arXiv preprint arXiv:2401.02695 (2024).

\bibitem{qin2023mp5} Qin, Yiran, et al. "Mp5: A multi-modal open-ended embodied system in minecraft via active perception." arXiv preprint arXiv:2312.07472 (2023).

\bibitem{du2024learning} Du, Yilun, et al. "Learning universal policies via text-guided video generation." Advances in Neural Information Processing Systems 36 (2024).

\bibitem{ajay2024compositional} Ajay, Anurag, et al. "Compositional foundation models for hierarchical planning." Advances in Neural Information Processing Systems 36 (2024).

\bibitem{liang2024skilldiffuser} Liang, Zhixuan, et al. "Skilldiffuser: Interpretable hierarchical planning via skill abstractions in diffusion-based task execution." Proceedings of the IEEE/CVF Conference on Computer Vision and Pattern Recognition. 2024.

\bibitem{heusel2017gans} Heusel, Martin, et al. "Gans trained by a two time-scale update rule converge to a local nash equilibrium." Advances in neural information processing systems 30 (2017).

\bibitem{zhang2018unreasonable} Zhang, Richard, et al. "The unreasonable effectiveness of deep features as a perceptual metric." Proceedings of the IEEE conference on computer vision and pattern recognition. 2018.

\bibitem{brown2020language} Brown, Tom B. "Language models are few-shot learners." arXiv preprint arXiv:2005.14165 (2020).

\bibitem{podell2023sdxl} Podell, Dustin, et al. "Sdxl: Improving latent diffusion models for high-resolution image synthesis." arXiv preprint arXiv:2307.01952 (2023).

\bibitem{brohan2022rt} Brohan, Anthony, et al. "Rt-1: Robotics transformer for real-world control at scale." arXiv preprint arXiv:2212.06817 (2022).

\bibitem{brohan2023rt} Brohan, Anthony, et al. "Rt-2: Vision-language-action models transfer web knowledge to robotic control." arXiv preprint arXiv:2307.15818 (2023).

\bibitem{li2024manipllm} Li, Xiaoqi, et al. "Manipllm: Embodied multimodal large language model for object-centric robotic manipulation." Proceedings of the IEEE/CVF Conference on Computer Vision and Pattern Recognition. 2024.

\bibitem{shah2023vint} Shah, Dhruv, et al. "ViNT: A foundation model for visual navigation." arXiv preprint arXiv:2306.14846 (2023).

\bibitem{liu2024visual} Liu, Haotian, et al. "Visual instruction tuning." Advances in neural information processing systems 36 (2024).

\bibitem{hu2021lora} Hu, Edward J., et al. "Lora: Low-rank adaptation of large language models." arXiv preprint arXiv:2106.09685 (2021).

\bibitem{qin2023supfusion} Qin, Yiran, et al. "SupFusion: Supervised LiDAR-camera fusion for 3D object detection." Proceedings of the IEEE/CVF International Conference on Computer Vision. 2023.

\bibitem{qin2024worldsimbench} Qin, Yiran, et al. "Worldsimbench: Towards video generation models as world simulators." arXiv preprint arXiv:2410.18072 (2024).

\bibitem{yu2025gamefactory} Yu, Jiwen, et al. "GameFactory: Creating New Games with Generative Interactive Videos." arXiv preprint arXiv:2501.08325 (2025).

\bibitem{zhou2024code} Zhou, Enshen, et al. "Code-as-Monitor: Constraint-aware Visual Programming for Reactive and Proactive Robotic Failure Detection." arXiv preprint arXiv:2412.04455 (2024).

\bibitem{zhang2024ad} Zhang, Zaibin, et al. "AD-H: Autonomous Driving with Hierarchical Agents." arXiv preprint arXiv:2406.03474 (2024).

\bibitem{wang2024toward} Wang, Chaoqun, et al. "Toward Accurate Camera-based 3D Object Detection via Cascade Depth Estimation and Calibration." arXiv preprint arXiv:2402.04883 (2024).

\bibitem{huang2024story3d} Huang, Yuzhou, et al. "Story3d-agent: Exploring 3d storytelling visualization with large language models." arXiv preprint arXiv:2408.11801 (2024).

\bibitem{savinov2018semi} Savinov, Nikolay, Alexey Dosovitskiy, and Vladlen Koltun. "Semi-parametric topological memory for navigation." arXiv preprint arXiv:1803.00653 (2018).

\bibitem{majumdar2022zson} Majumdar, Arjun, et al. "Zson: Zero-shot object-goal navigation using multimodal goal embeddings." Advances in Neural Information Processing Systems 35 (2022): 32340-32352.

\bibitem{yadav2023offline} Yadav, Karmesh, et al. "Offline visual representation learning for embodied navigation." Workshop on Reincarnating Reinforcement Learning at ICLR 2023. 2023.

\bibitem{yadav2023ovrl} Yadav, Karmesh, et al. "Ovrl-v2: A simple state-of-art baseline for imagenav and objectnav." arXiv preprint arXiv:2303.07798 (2023).

\bibitem{sun2024fgprompt} Sun, Xinyu, et al. "FGPrompt: fine-grained goal prompting for image-goal navigation." Advances in Neural Information Processing Systems 36 (2024).

\bibitem{zhu2017target} Zhu, Yuke, et al. "Target-driven visual navigation in indoor scenes using deep reinforcement learning." 2017 IEEE international conference on robotics and automation (ICRA). IEEE, 2017.

\bibitem{koh2024generating} Koh, Jing Yu, Daniel Fried, and Russ R. Salakhutdinov. "Generating images with multimodal language models." Advances in Neural Information Processing Systems 36 (2024).

\bibitem{krantz2022instance} Krantz, Jacob, et al. "Instance-specific image goal navigation: Training embodied agents to find object instances." arXiv preprint arXiv:2211.15876 (2022).

\bibitem{schulman2017proximal} Schulman, John, et al. "Proximal policy optimization algorithms." arXiv preprint arXiv:1707.06347 (2017).

\bibitem{anderson2018evaluation} Anderson, Peter, et al. "On evaluation of embodied navigation agents." arXiv preprint arXiv:1807.06757 (2018).

\bibitem{lin2024navcot} Lin, Bingqian, et al. "NavCoT: Boosting LLM-Based Vision-and-Language Navigation via Learning Disentangled Reasoning." arXiv preprint arXiv:2403.07376 (2024).

\bibitem{NavGPT} Zhou, Gengze, Yicong Hong, and Qi Wu. "Navgpt: Explicit reasoning in vision-and-language navigation with large language models." Proceedings of the AAAI Conference on Artificial Intelligence.

\bibitem{hahn2021no} Hahn, Meera, et al. "No rl, no simulation: Learning to navigate without navigating." Advances in Neural Information Processing Systems 34 (2021): 26661-26673.

\bibitem{li2025t2isafety} Li, Lijun, et al. "T2ISafety: Benchmark for Assessing Fairness, Toxicity, and Privacy in Image Generation." arXiv preprint arXiv:2501.12612 (2025).

\bibitem{an2024agfsync} An, Jingkun, et al. "AGFSync: Leveraging AI-Generated Feedback for Preference Optimization in Text-to-Image Generation." arXiv preprint arXiv:2403.13352 (2024).


\end{thebibliography}
\end{sloppypar}

\clearpage
\beginsupplement
\section*{Appendix}
\renewcommand{\thesubsection}{S\arabic{subsection}}

\subsection{\label{chap:S1}PanNuke and MoNuSAC preprocessing}
The PanNuke dataset comprises a set of 7,901 RGB patches, each with dimensions of $256 \times 256$ pixels, which we set as the standard patch size for our analysis. In contrast, the MoNuSAC dataset encompasses 294 images of heterogeneous dimensions. To standardize the MoNuSAC images with our experiments, we implement a standardization protocol. Specifically, for images exceeding the dimensions of $256 \times 256$ pixels, we segment them into equal-sized patches and apply mirror padding to the remaining portions to avoid information loss at the peripherals. Patches with dimensions less than $128 \times 128$ pixels are excluded from the dataset due to the insufficient resolution to capture relevant cellular details. For patches where either dimension falls between 128 and 256 pixels, we employ upsampling to achieve the standard patch size. As a result, we obtain a total of 2,823 RGB patches derived from the MoNuSAC dataset for subsequent analysis. For additional details on the MoNuSAC data preparation process, refer to the source code \cite{Shvetsov_2025a}.
\clearpage

\subsection{\label{chap:S2}Data usage for the methodology}

\counterwithin{figure}{subsection}
\renewcommand{\thefigure}{S\arabic{subsection}}

\begin{figure}[h!]
    \centering
    \includegraphics[width=\textwidth, height=0.85\textheight, keepaspectratio]{images/A2.pdf}
    \caption{Overview of the methodology for cross-labeling, dataset refinement, and model comparison. (1) Cross-relabeling - training and testing cell classification models, (2) Cross-relabeling - using cell classification models to create refined dataset, (3) Fine-tuning and training models for comparison, (4) Student knowledge distillation with refined dataset}
    \label{fig:S2}
\end{figure}
\clearpage

\subsection{\label{chap:S3}Confusion matrices for classification models}
\counterwithin{figure}{subsection}
\renewcommand{\thefigure}{S\arabic{subsection}.\arabic{figure}}

\begin{figure}[h!]
    \centering
    \includegraphics[width=\textwidth, height=0.4\textheight, keepaspectratio]{images/A3_1.pdf}
    \caption{Confusion matrix for PanNuke trained model}
    \label{fig:S3.1}
\end{figure}

\begin{figure}[h!]
    \centering
    \includegraphics[width=\textwidth, height=0.4\textheight, keepaspectratio]{images/A3_2.pdf}
    \caption{Confusion matrix for MoNuSAC trained model}
    \label{fig:S3.2}
\end{figure}

\clearpage

\subsection{\label{chap:S4}Datasets cell counts}

\counterwithin{table}{subsection}
\renewcommand{\thetable}{S\arabic{subsection}}

\begin{table}[h!]
\renewcommand{\arraystretch}{2.0}
\centering
\caption{\label{tab:S4}Cell counts for PanNuke, MoNuSAC and refined datasets. Numbers in parentheses indicate preprocessed cell counts for cell classifier models training and testing.}
%\adjustbox{max width=\textwidth}{%
\begin{tabular}{|l|c|c|c|}
\hline
%\rowcolor{gray!30}
Cell type & PanNuke & MoNuSAC & Refined \\
\hline
Neoplastic & 77,403 (68,031) & - & 105,451 \\
\hline
Epithelial & 26,572 (23,207) & - & 29,926 \\
\hline
Epithelial (benign and malignant) & - & 31,402 & - \\
\hline
Inflammatory & 32,276 & - & - \\
\hline
Lymphocytes & - & 37,045 (33,104) & 65,275 \\
\hline
Neutrophils & - & 1,355 (1,252) & 3,833 \\
\hline
Macrophage & - & 1,842 (1,695) & 3,410 \\
\hline
Dead & 2,908 & - & 2,908 \\
\hline
Connective & 50,585 & - & 50,585 \\
\hline
\end{tabular}
%
%}
\end{table}



\clearpage

\subsection{\label{chap:S5}Definition of validation metrics}
\counterwithin{equation}{subsection}
\renewcommand{\theequation}{\arabic{equation}}

\subsubsection{\label{chap:S5.1}R\textsuperscript{2}}
The coefficient of determination, denoted as $R^2$, is a statistical measure that represents the proportion of variance in the dependent variable that is predictable from the independent variables. In the context of cell quantification in pathology, $R^2$ is used to assess how well the predicted quantities of different cell types in a patch align with the actual quantities observed in the ground truth data, with higher values representing more accurate quantification. $R^2$ is defined as
\begin{equation*}
R^2 = 1 - \frac{\sum_{i=1}^n (y_i - \hat{y}_i)^2}{\sum_{i=1}^n (y_i - \bar{y})^2},
\end{equation*}
where $y_i$ represents the actual number of cells of a specific type in the $i$-th image, $\hat{y}_i$ represents the predicted number of cells of that type in the $i$-th image, $\bar{y}$ is the mean of the actual numbers across all images, and $n$ is the total number of images in the dataset.

The $R^2$ metric has a range of $(-\infty, 1]$. An $R^2$ of 1 indicates perfect prediction, where all predicted values exactly match the actual values. An $R^2$ of 0 suggests that the model explains none of the variability of the response data around its mean. If $R^2$ is negative, it indicates that the model performs worse than a model that simply predicts the mean of the actual values for all observations.

\subsubsection{\label{chap:S5.2}PQ}
Panoptic Quality ($PQ$) is a comprehensive metric used to evaluate the performance of segmentation models in tasks that require both instance segmentation and classification. $PQ$ provides a single score that encapsulates both the detection accuracy (i.e., how many objects were correctly identified) and the segmentation quality (i.e., how accurately the objects' boundaries were delineated). This metric is particularly useful in multiclass scenarios where each pixel is classified into distinct categories, such as different cell types in pathology images.

$PQ$ is calculated as the product of two terms: Detection Quality ($DQ$) and Segmentation Quality ($SQ$). It can be expressed as
\begin{equation*}
PQ = DQ \cdot SQ,
\end{equation*}
where
\begin{equation*}
DQ = \frac{TP}{TP + 0.5\, FP + 0.5\, FN},
\end{equation*}
\begin{equation*}
SQ = \frac{\sum_{(p, g) \in \mathcal{M}} IoU(p, g)}{TP}.
\end{equation*}
In these formulas, $TP$ denotes the number of correctly matched instances between ground truth and prediction, $FP$ denotes the predicted instances that have no corresponding ground truth, $FN$ denotes the ground truth instances that were not detected, $IoU(p, g)$ is the Intersection over Union for a pair of matched instances $p$ (prediction) and $g$ (ground truth), and $\mathcal{M}$ is the set of matched pairs.

The $PQ$ metric is calculated for each class and is averaged across classes to provide a global performance measure.

The $PQ$ score has a range of $[0, 1.0]$, where a higher score indicates better performance in both detecting and segmenting the instances correctly. A $PQ$ of 1 signifies perfect identification and segmentation of all instances, whereas a $PQ$ of 0 indicates that no instances were correctly identified and segmented.

\clearpage

\subsection{\label{chap:S6}Segmentation and Detection quality metrics for teacher and student models}

\begin{table}[h!]
\renewcommand{\arraystretch}{2.0}
\centering
\caption{Segmentation and detection quality for student and teacher models (CI 95\%)}
\label{tab:S6}
%\adjustbox{max width=\textwidth}{%
\begin{tabular}{|l|c|c|}
\hline
%\rowcolor{gray!30}
Metric & Teacher & Student \\
\hline
$SQ_{neoplastic}$ & 0.819 (0.815--0.823) & 0.824 (0.819--0.828) \\
\hline
$SQ_{lymphocyte}$ & 0.795 (0.788--0.802) & 0.790 (0.783--0.796) \\
\hline
$SQ_{connective}$ & 0.770 (0.762--0.776) & 0.780 (0.772--0.786) \\
\hline
$SQ_{dead}$ & 0.659 (0.623--0.688) & 0.657 (0.624--0.695) \\
\hline
$SQ_{epithelial}$ & 0.780 (0.770--0.790) & 0.788 (0.779--0.797) \\
\hline
$SQ_{macrophage}$ & 0.788 (0.760--0.810) & 0.757 (0.730--0.783) \\
\hline
$SQ_{neutrofil}$ & 0.782 (0.761--0.801) & 0.775 (0.759--0.792) \\
\hline
$DQ_{neoplastic}$ & 0.706 (0.692--0.719) & 0.727 (0.712--0.741) \\
\hline
$DQ_{lymphocyte}$ & 0.675 (0.656--0.698) & 0.713 (0.691--0.734) \\
\hline
$DQ_{connective}$ & 0.566 (0.546--0.584) & 0.583 (0.565--0.602) \\
\hline
$DQ_{dead}$ & 0.410 (0.361--0.465) & 0.435 (0.306--0.561) \\
\hline
$DQ_{epithelial}$ & 0.668 (0.639--0.694) & 0.673 (0.644--0.702) \\
\hline
$DQ_{macrophage}$ & 0.657 (0.583--0.727) & 0.615 (0.531--0.703) \\
\hline
$DQ_{neutrofil}$ & 0.691 (0.625--0.753) & 0.729 (0.679--0.778) \\
\hline
\end{tabular}
%
%}
\end{table}

\clearpage

\subsection{\label{chap:S7}QuPath integration method}
We adopt an integration strategy leveraging the paquo \cite{Bayer_AG} library, a Python package that enables direct interaction with QuPath’s internal API, thereby facilitating seamless data exchange without intermediate conversion steps. The data processing pipeline (\hyperref[fig:S7]{Appendix Figure S7}) begins with the acquisition of WSIs and their associated annotations from QuPath, which are represented as Shapely \cite{Gillies_Wel_etal._2024} polygons. Utilizing paquo, we directly read, create, and modify these annotations and detections within a QuPath project in the Python environment. Images are then cropped using these polygons and processed by cell segmentation and classification models employing standard vision processing toolkits such as OpenCV, pyvips, and PyTorch. Additionally, QuPath employs Groovy scripts to initiate a Python process that starts the entire pipeline from QuPath graphical interface: fetching polygons, extracting images from them, and running deep learning model inference on the cropped images. 
The results are returned to QuPath, leveraging paquo's Python bindings to manipulate QuPath data while minimizing the computational overhead typically associated with cross-environment communication.

\counterwithin{figure}{subsection}
\renewcommand{\thefigure}{S\arabic{subsection}}

\begin{figure}[h!]
    \centering
    \includegraphics[width=\textwidth]{images/A7.pdf}
    \caption{QuPath integration workflow using Python environment}
    \label{fig:S7}
\end{figure}

Compared to traditional workflows that involve exporting annotations as GeoJSON, classifying them in Python, and reimporting them into QuPath, our approach offers several advantages. We eliminate the need to switch between programming languages, providing a cohesive and streamlined development process entirely within QuPath software and removing the necessity to use other tools. Meanwhile, we avoid storing annotations as intermediate JSON files unless required for external use or archiving. By conducting the entire inference and post-processing workflow within the Python environment, we leverage the power and flexibility of Python libraries for image processing and machine learning. This approach also enables adjustments to any set of labels and models, thereby improving its applicability.

%\hfill

The distilled model and QuPath integration code are packaged into a Docker container, enabling streamlined execution with the Docker engine. Detailed integration code and deployment instructions can be found in the GitHub repository \cite{Shvetsov_2025b}.

Despite these benefits, we acknowledge that the paquo library is a proof‑of‑concept project in its early development stage and has not been tested across all versions of QuPath.

\clearpage

\subsection{\label{chap:S8}Data and code availability statement}
All datasets, models, and code used in this study are publicly available and can be obtained from the repositories listed below. 
The PanNuke \cite{Gamper_Koohbanani_etal._2019} and MoNuSAC \cite{Verma_Kumar_etal._2021} datasets are publicly accessible, and download information along with detailed descriptions can be found in their respective articles. Preprocessing scripts for PanNuke and MoNuSAC data, as well as individual cell extraction scripts, are available on GitHub \cite{Shvetsov_2025a}. The H-Optimus foundation model used in our experiments can be downloaded from the HuggingFace repository \cite{hoptimus2024}, and model information is available on GitHub \cite{Saillard_Jenatton_etal._2024}. In addition, the integration code for QuPath and the distilled model packaged in a Docker container are provided in the repository \cite{Shvetsov_2025b}, and paquo Python library is available from the authors GitHub repository \cite{Bayer_AG}.
\clearpage

\end{document}


\section{Additional Experiments}\label{sec:exp}
\vspace{-0.1in}
\begin{figure*}[!h]
    \centering
    \begin{subfigure}[b]{0.8\linewidth}
        \centering
        \includegraphics[width=0.45\linewidth]{images/residual/text/CIReVL_Recall5.png}
        \hfil
        \includegraphics[width=0.45\linewidth]{images/residual/text/pic2word_recall5.png}
        \caption{\textbf{PDV-T}: Impact of $\alpha$ scaling on composed text embeddings}
        \label{fig:residual_text_sub}
    \end{subfigure}
    
    \begin{subfigure}[b]{0.8\linewidth}
        \centering
        \includegraphics[width=0.45\linewidth]{images/residual/image/CIReVL_Recall5.png}
        \hfil
        \includegraphics[width=0.45\linewidth]{images/residual/image/pic2word_recall5.png}
        \caption{\textbf{PDV-I}: Impact of $\alpha$ scaling on composed image embeddings}
        \label{fig:residual_image_sub}
    \end{subfigure}
    
    \begin{subfigure}[b]{0.8\linewidth}
        \centering
        \includegraphics[width=0.45\linewidth]{images/residual/fusion/CIReVL_Recall5.png}
        \hfil
        \includegraphics[width=0.45\linewidth]{images/residual/fusion/pic2word_recall5.png}
        \caption{\textbf{PDV-F}: Impact of varying $\beta$ with on composed fused embeddings}
        \label{fig:residual_fusion_sub}
    \end{subfigure}
    \caption{Impact of changing $\alpha$/$\beta$ on Recall@5 performance across different PDV applications. For each row, results are shown for the CIReVL (left) and Pic2Word (right) baseline methods.}
    \label{fig:residual_all}
\end{figure*}

\section{Experiments} 
\label{sec:exp}
\noindent\textbf{Implementation Details.} We utilize the official implementations of four ZS-CIR baseline methods: CIReVL\footnote{https://github.com/ExplainableML/Vision\_by\_Language} and LDRE \footnote{https://github.com/yzy-bupt/LDRE} as representative caption-based feature extraction approaches and Pic2Word\footnote{https://github.com/google-research/composed\_image\_retrieval} and SEARLE\footnote{https://github.com/miccunifi/SEARLE} as representative pseudo tokenization-based methods. All feature extraction processes follow the original implementations provided by these baseline methods. However, to calculate $\Delta_{PDV}$, we need text embeddings without prompts, which are not provided in the original implementations. For CIReVL and LDRE, we obtain these embeddings by passing the generated image captions directly to CLIP. For Pic2Word and SEARL, we construct the base text embedding by passing the phrase ``a photo of $\langle$token$\rangle$" to CLIP, where $\langle$token$\rangle$ represents the extracted image token obtained via text inversion.

\noindent\textbf{Datasets and Base Vision-Language Models.} Following previous work, we evaluated our method on a suite of datasets including Fashion-IQ \cite{wu2021fashion}, CIRR \cite{liu2021image} and CIRCO \cite{baldrati2023zero}. Our proposed method is a plug-and-play approach requiring no additional training, leveraging only pre-trained models. For feature extraction, we use three CLIP variants: ViT-B/32, ViT-L/14, and ViT-G/14, and used the same pre-trained weights as used by the baseline methods. For image tokenization, we employ the pre-trained Pic2Word model. 

\subsection{Effect of Using the PDV}
We now explore the impact of the three proposed uses of the PDV: Using the PDV to augment text queries (PDV-T, see Sec. \ref{sec:exp1}), using the PDV to augment image queries (PDV-I, see Sec. \ref{sec:exp2}), and using the PDV in queries that fuse image and text data (PDV-F, see Sec. \ref{sec:exp3}).

\begin{table*}
	\footnotesize
	\centering
	\begin{tabular}{l|l|c|c|c|cccccccc}
		\hline
		\textbf{Fashion-IQ} & & & & & \multicolumn{2}{c}{\textbf{Shirt}} & \multicolumn{2}{c}{\textbf{Dress}} & \multicolumn{2}{c}{\textbf{Toptee}} & \multicolumn{2}{c}{\textbf{Average}} \\ \hline
		Backbone & Method& $\beta$ & $\alpha_{I}$& $\alpha_{T}$ & R@10 & R@50 & R@10 & R@50 & R@10 & R@50 & R@10 & R@50 \\
		\hline
		\multirow{6}{*}{ViT-B/32} %
		& SEARLE & - & - & - & 24.14 & 41.81 & 18.39 & 38.08 & 25.91 & 47.02 & 22.81 & 42.30 \\
		& SEARLE + \textbf{PDV-F} & 0.9 & 1.1 & 0.9 & \hli{24.83} & 41.71 & \hli{20.13} & \hli{41.40} & \hli{25.96} & \hli{47.17}  & \hli{23.64} & \hli{43.43} \\
		& CIReVL \textdagger &- & -& -& 28.36 & 47.84 & 25.29 & 46.36 & 31.21 & 53.85 & 28.29 & 49.35 \\
		& CIReVL + \textbf{PDV-F} & 0.75 & 1.4 & 1.4 & \hlb{32.88} & \hlb{52.80} & \hlb{32.67} & \hlb{54.49} & \hlb{38.91} & \hlb{61.81} & \hlb{34.82} & \hlb{56.37} \\
		& LDRE \textdagger & - & - & - & 27.38 & 46.27 & 19.97 & 41.84 & 27.07 & 48.78 & 24.81 & 45.63 \\
		& SEIZE \textdagger & - & - & - & \underline{29.38} & \underline{47.97} & \underline{25.37} & \underline{46.84} & \underline{32.07} & \underline{54.78} & \underline{28.94} & \underline{49.86} \\
		\hline
		\multirow{8}{*}{ViT-L/14} & Pic2Word & & & & 25.96 & 43.52 & 19.63 & 40.90 & 27.28 & 47.83 & 24.29 & 44.08 \\
		& Pic2Word + \textbf{PV-F} & 0.8 & 1.0 & 1.0 & \hli{28.21} & \hli{44.55} & \hli{20.92} & \hli{42.24} & \hli{29.02} & \hli{48.90}& \hli{26.05} & \hli{45.23}\\
		& SEARLE & - & - & - & 26.84 & 45.19 & 20.08 & 42.19 & 28.40 & 49.62 & 25.11 & 45.67 \\
		& SEARLE +\textbf{PDV-F} & 0.8 & 1.2 & 1.0 & \hli{28.66} & \hli{46.76} & \hli{23.60} & \hli{46.41} & \hli{31.00} & \hli{52.32} & \hli{27.75} & \hli{48.50} \\
		& CIReVL \textdagger & & & & 29.49 & 47.40 & 24.79 & 44.76 & 31.36 & 53.65 & 28.55 & 48.57 \\
		
		& CIReVL + \textbf{PDV-F} & 0.55 & 1 & 1.3 & \hlb{37.78} & \hlb{54.22} & \hlb{33.61} & \hlb{56.07} & \hlb{41.61} & \hlb{62.16} & \hlb{37.67} & \hlb{57.48} \\
		& LinCIR & - & - & - & 29.10 & 46.81 & 20.92 & 42.44 & 28.81 & 50.18 & 26.82 & 46.49 \\
        & SEIZE & -& -& -& \underline{33.04} & \underline{53.22} & \underline{30.93} & \underline{50.76} & \underline{35.57} & \underline{58.64} & \underline{33.18} & \underline{54.21} \\
		\hline
        \multirow{6}{*}{ViT-G/14} & Pic2Word  & - & - & - & 33.17 & 50.39 & 25.43 & 47.65 & 35.24 & 57.62 & 31.28 & 51.89\\
         & SEARLE  & - & - & - & 36.46 & 55.35 & 28.16 & 50.32 & 39.83 & 61.45 & 34.81 & 55.71\\
		  & CIReVL \textdagger & -& -& -& 33.71 & 51.42 & 27.07 & 49.53 & 35.80 & 56.14 & 32.19 & 52.36 \\
		& CIReVL + \textbf{PV-F} & 0.6 & 1.4 & 1.4 & \hli{41.90} & \hli{58.19} & \hlb{40.70} & \hlb{62.82} & \underline{\hli{48.09}}& \hli{67.77}& \underline{\hli{43.56}}& \hli{62.93}\\
        & LinCIR & - & - & - & \textbf{46.76} & \underline{65.11} & 38.08& 60.88& \textbf{50.48}& \underline{71.09}& \textbf{45.11} & \underline{65.69}\\
        & SEIZE & - & - & - & \underline{43.60} & \textbf{65.42}& \underline{39.61} & \underline{61.02} & 45.94& \textbf{71.12}& 43.05& \textbf{65.85}\\
		\hline
	\end{tabular}
	\caption{Average recall for different methods on Fashion-IQ validation dataset. \textdagger~denotes that numbers are taken from the original paper.}
	\label{tab:fashion_iq_results}
\end{table*}


\begin{table*}
	\centering
	\footnotesize
	\setlength{\tabcolsep}{4pt}
	\begin{tabular}{ll|c|c|c|cccc|cccc|ccc}
		\hline
		\multicolumn{2}{c|}{\textbf{Dataset}} & & & &  \multicolumn{4}{c|}{\textbf{CIRCO}} & \multicolumn{7}{c}{\textbf{CIRR}} \\
		\hline
		\multicolumn{2}{c|}{Metric} & & & & \multicolumn{4}{c|}{mAP@k} & \multicolumn{4}{c|}{Recall@k} &\multicolumn{3}{c}{$R_s$@k} \\
		\cline{3-16}
		Arch & Method & $\beta$ & $\alpha_I$ & $\alpha_T$ & k=5 & k=10 & k=25 & k=50 & k=1 & k=5 & k=10 & k=50 & k=1 & k=2 & k=3 \\
		\hline
		\multirow{8}{*}{ViT-B/32} 
		& PALAVRA\cite{cohen2022my} \textdagger & -& -& -& 4.61 & 5.32 & 6.33 & 6.80 & 16.62 & 43.49 & 58.51 & 83.95 & 41.61 & 65.30 & 80.94 \\
		& SEARLE \textdagger & -& -&- & 9.35 & 9.94 & 11.13 & 11.84 & 24.00 & 53.42 & 66.82 
		& 89.78 & 54.89 & 76.60 & 88.19 \\
		& SEARLE + \textbf{PDV-F} & 0.9 & 1.4 & 1.2 & \hli{9.99} & \hli{10.50}  & \hli{11.70} & \hli{12.40} & \hli{24.53} & \hli{53.71} & \hli{67.33} & \hli{89.81} & \hli{56.94} & \hli{78.05} & \hli{88.99} \\
		&CIReVL \textdagger & - & - & -& 14.94 & 15.42 & 17.00 & 17.82 & 23.94 & 52.51 & 66.00 & 86.95 & 60.17 & 80.05 & 90.19 \\
		& CIReVL + \textbf{PDV-F} & 0.75 & 1.4 & 1.2 & \hlb{19.90} & \hlb{20.61} & \hlb{22.64} & \hlb{23.52} & \hlb{33.25} & \hlb{64.15} & \hlb{75.23} & \hlb{92.43} & \hlb{65.81} &\underline{\hli{83.76}} &\underline{\hli{92.10}} \\
		& LDRE & -& -& -& 17.81 & 18.04 & 19.73 & 20.67 & 25.69 & 55.52 & 68.77 & 89.86 & 60.10 & 80.58 & 91.04 \\
		& LDRE + \textbf{PDV-F} & 0.75 & 1.4 & 1.4 & \hli{17.80} & \hli{18.78} & \hli{20.61} & \hli{21.56} & \underline{\hli{29.30}} & \underline{\hli{60.39}} & \underline{\hli{72.51}} & \underline{\hli{91.42}} & \hli{63.06} & \hli{82.36} & \hli{91.54} \\
        & SEIZE & -&- &- & \underline{19.04} & \underline{19.64} & \underline{21.55}& \underline{22.49}& 27.47 & 57.42& 70.17 & - & \underline{65.59} & \textbf{84.48}& \textbf{92.77} \\
 		\hline
		\multirow{10}{*}{ViT-L/14}
		& Pic2Word & -& -& -& 6.81 & 7.49 & 8.51 & 9.07 & 23.69 & 51.32 & 63.66 & 86.21 & 53.61 & 74.34 & 87.28 \\
		& Pic2Word + \textbf{PDV-F} & 0.85 & 1.2 & 1.0 & \hli{7.74} &  \hli{8.67} & \hli{9.77} & \hli{10.37} & \hli{23.90} & \hli{51.95} & \hli{64.63} & \hli{87.04} & \hli{53.16}  & \hli{74.07} & \hli{87.08}\\
		& SEARLE \textdagger & - & - & - & 11.68 & 12.73 & 14.33 & 15.12 & 24.24 & 52.48 & 66.29 & 88.84 & 53.76 & 75.01 & 88.19 \\
		& SEARLE + \textbf{PDV-F} & 0.85 & 1.4 & 1.2 & \hli{12.58} & \hli{13.57} & \hli{15.30} & \hli{16.07} & \hli{25.64} & \hli{53.61} & \hli{66.58} & \hli{88.55} & \hli{55.83} & \hli{76.48} & \hli{88.53} \\
		& CIReVL \textdagger & -& -& -& 18.57 & 19.01 & 20.89 & 21.80 & 24.55 & 52.31 & 64.92 & 86.34 & 59.54 & 79.88 & 89.69 \\
		& CIReVL + \textbf{PDV-F} & 0.75 & 1.4 & 1.2 & \hlb{25.67} & \hlb{26.61} & \underline{\hli{28.81}} & \hlb{29.95} & \hlb{36.24} & \hlb{66.17} & \hlb{76.96} & \hlb{92.29} & \hlb{68.07} & \hlb{85.35} & \hlb{93.47} \\
		& LDRE & -& -& -& 22.32 & 23.75 & 25.97 & 27.03 & 26.68 &55.45  & 67.49 & 88.65 & 60.39 & 80.53 & 90.15 \\
		& LDRE + \textbf{PDV-F} & 0.75 & 1.4 & 1.4 & \hli{25.23} & \hli{26.52} & \hlb{28.94} & \hlb{29.95} & \underline{\hli{30.16}} & \underline{\hli{59.98}} & \underline{\hli{71.90}} & \underline{\hli{90.87}} & \hli{63.66} & \hli{82.87} & \hli{91.57} \\

        & LinCIR & - & - & - &12.59 &13.58 &15.00 &15.85 &25.04 &53.25 &66.68 & - &57.11 &77.37 &88.89\\
        & SEIZE & -& -& -& 24.98 & 25.82 &28.24 &\underline{29.35}& 28.65 &57.16& 69.23& - &\underline{66.22} &\underline{84.05} &\underline{92.34} \\
        

        
		\hline
		\multirow{7}{*}{ViT-G/14} & CIReVL \textdagger & -& -& -& 26.77 & 27.59 & 29.96 & 31.03 & 34.65 & 64.29 & 75.06 & 91.66 & 67.95 & 84.87 & 93.21 \\

		& CIReVL + \textbf{PDV-F} & 0.75 & 1.4 & 1.2 & \hli{30.02} & \hli{31.46} & \hli{34.01} & \hli{35.08} & \hli{38.15} &\hli{67.93} & \hli{77.90} & \hli{92.77} & \hli{69.37} & \hli{85.37} & \hli{93.45}  \\
		
		& LDRE & -& -& -& \underline{33.30} & \underline{34.32} & \underline{37.17} & \underline{38.27} & 37.40 & 66.96 & 78.17 & 93.66 & 68.84 & 85.64 & 93.90 \\
		& LDRE + \textbf{PDV-F} & 0.75 & 1.4 & 1.4 & \hlb{34.88} & \hlb{36.41} & \hlb{39.12} & \hlb{40.23} & \hlb{42.51} & \hlb{72.22} & \hlb{81.71} & \hlb{94.94} & \underline{\hli{72.39}} & \underline{\hli{88.34}} & \underline{\hli{94.80}} \\
        & SEARLE & - & - & - & 13.20 &13.85 &15.32 &16.04 & 34.80 & 64.07 & 75.11 &-&68.72 &84.70 &93.23 \\
        & LinCIR & - & - & - & 19.71 &21.01 &23.13 &24.18 &35.25 &64.72 &76.05 & - &63.35 &82.22 &91.98 \\
        & SEIZE & -& -& -& 32.46 & 33.77 &36.46 &37.55 &\underline{38.87} & \underline{69.42} & \underline{79.42} & -&\textbf{74.15} & \textbf{89.23} & \textbf{95.71} \\
		\hline
	\end{tabular}
	\caption{Performance comparison on CIRCO and CIRR test datasets. As in previous works, for CIRCO, mAP@k is reported, while for CIRR both Recall@k and $R_s$@k metrics are used. \textdagger~denotes that numbers are taken from the original paper.}
	\label{tab:circo_cirr_results}
\end{table*}

\noindent{\textbf{Analysing the PDV for Text (PDV-T)}}
\label{sec:exp1}
To investigate how scaling the prompt vector, $\Delta_{PDV}$, affects retrieval performance with composed text embeddings, we conducted experiments using two zero-shot approaches (CIReVL and Pic2Word) with different backbone networks across three datasets. We evaluated the performance by varying the scaling parameter, $\alpha$ (Eq. \ref{eqn:text_embedding}), from -0.5 to 3 by an interval of 0.1.

The results are presented in Figure \ref{fig:residual_text_sub}. To account for scale variations across different experiments, we report relative recall values, where a baseline of zero is established at $\alpha=1$. As shown in Figure \ref{fig:residual_text_sub}, varying $\alpha$ leads to significant changes in relative recall performance\footnote{See supplementary material for Recall@10 and Recall@50 figures}. Our analysis reveals method-specific patterns across datasets. With CIReVL, increasing $\alpha$ improves relative recall on both FashionIQ and CIRCO datasets. In contrast, Pic2Word shows no significant improvement on FashionIQ and CIRR when varying $\alpha$, while CIRCO's performance improves when $\alpha$ is reduced to 0.8-1.0. This divergent behavior is fundamentally linked to each method's ability to generate an accurate $\Delta_{PDV}$. As demonstrated in Tables \ref{tab:fashion_iq_results} and \ref{tab:circo_cirr_results}, CIReVL consistently outperforms Pic2Word across various benchmarks, indicating its superior ability to generate a more accuraute composed query, and thus a more accurate $\Delta_{PDV}$. Consequently, increasing $\alpha$ yields greater benefits for CIReVL compared to Pic2Word.

We visualize the top-5 retrieval results using CIReVL with a ViT-B-32 backbone across three datasets (one reference image from each) under varying $\alpha$ values, as shown in Figure \ref{fig:residual_qual}\red{a}. As $\alpha$ increases, the retrieved results show stronger alignment with the prompt. Conversely, when $\alpha$ exceeds 1, the results include semantically related but unseen variations, while $\alpha$ values below 0.5 yields results opposite to the prompt's intent. For instance, ``brighter blue and sleeveless" retrieves ``dark blue with sleeves," ``plain background" yields ``natural/dark background," and ``young boy" returns ``adult" images.





\noindent{\textbf{Analysing the PDV for Image (PDV-I)}}
\label{sec:exp2}
To evaluate whether $\Delta_{PDV}$ enhances the retrieval performance of image embeddings, we conducted experiments following the protocol described in Section~\ref{sec:exp1}. We modified image embeddings by adding $\Delta_{PDV}$ scaled with $\alpha$ values ranging from -0.5 to 2.0, where $\alpha=0$ represents the original image-only embeddings. As shown in Figure \ref{fig:residual_image_sub}, Recall@K exhibits a positive correlation with $\alpha$ for values below 1. This upward trend continues until $\alpha=2.0$ for CIReVL, while Pic2Word's performance peaks when $\alpha$ reaches 1.4.  The performance of PDV-I was evaluated on the CIRR and CIRCO datasets by comparing it with other visual embedding-based methods, as detailed in Table \ref{tab:circo_cirr_results_pdv-I}. The results reveal that PDV-I achieved marginal improvements over existing approaches.

Following the methodology in Section~\ref{sec:exp1}, we conduct similar visualizations, with results shown in Figure \ref{fig:residual_qual}\red{b}. As with PDV-T, increasing $\alpha$ leads to stronger alignment between retrieved results and the prompt. When $\alpha$ exceeds 0.5, the results exhibit semantic relationships to the query, while $\alpha$ values below 0.5 yield results opposing the prompt's intent.
Notably, PDV-I's top retrievals demonstrate higher visual similarity to reference images compared to PDV-F, as evidenced by the preserved design elements in the clothing item (left) and laptop (middle). This characteristic is particularly valuable for applications include fashion search \cite{wu2021fashion} and logo retrieval \cite{tursun2019component}, where visual similarity plays a crucial role.



\begin{figure*}[!tbh]
	\centering
	\includegraphics[width=0.825\linewidth]{images/qualitative/PV_qual_all_mini.pdf}
	\caption{Visualisation of the impact of $\alpha$/$\beta$ scaling on top-5 retrieval results. CIReVL with ViT-B-32 Clip model is the baseline method used. Representative examples with prompts from three datasets: FashionIQ (left), CIRR (middle), and CIRCO (right) are shown at the top. \textbf{\textcolor{boxgreen}{Green}} and \textbf{\textcolor{boxblue}{blue}} bounding boxes indicate true positives and near-true positives, respectively.}
	\label{fig:residual_qual}
	
\end{figure*}

\noindent{\textbf{Analysing PDV Fusion (PDV-F)}}
\label{sec:exp3}
Finally, we evaluate the effectiveness of fusing image and text-composed embeddings by varying the fusion parameter, $\beta$, from 0 to 1 while maintaining $\alpha=1$
for both PDV-I and PDV-F. At $\beta=0$, the model relies solely on composed image embeddings, while at $\beta=1$, it uses only composed text embeddings. As shown in Figure \ref{fig:residual_fusion_sub}, the fusion of both embeddings consistently outperforms using either embedding type alone. Optimal retrieval performance is typically achieved when $\beta$ is between 0.4 and 0.8.

We similarly visualize the top-5 retrieved results across different $\beta$ values. As shown in Figure \ref{fig:residual_qual}\red{c}, when $\beta$ is small, the retrieved results maintain high visual similarity to the reference image. Conversely, as $\beta$ exceeds 0.5, the results demonstrate stronger semantic alignment with the prompt.



\subsection{ZS-CIR Benchmark Comparison}






\begin{table*}
	\centering
	\footnotesize
	\setlength{\tabcolsep}{4pt}
	\begin{tabular}{l|l|c|cccc|cccc|ccc}
		\hline
		\multicolumn{2}{c|}{\textbf{Dataset}} & & \multicolumn{4}{c|}{\textbf{CIRCO}} & \multicolumn{7}{c}{\textbf{CIRR}} \\
		\hline
		& Metric & & \multicolumn{4}{c|}{mAP@k} & \multicolumn{4}{c|}{Recall@k} & \multicolumn{3}{c}{$R_s$@k} \\
		\cline{2-14}
		Arch & Method & $\alpha_I$ & k=5 & k=10 & k=25 & k=50 & k=1 & k=5 & k=10 & k=50 & k=1 & k=2 & k=3 \\
		\hline
		\multirow{6}{*}{ViT-B/32} 
		& Image-only \textdagger & - & 1.34 & 1.60 & 2.12 & 2.41 & 6.89 & 22.99 & 33.68 & 59.23 & 21.04 & 41.04 & 60.31 \\
		& Text-only \textdagger & - & 2.56 & 2.67 & 2.98 & 3.18 & 21.81 & 45.22 & 57.42 & 81.01 & 62.24 & 81.13 & 90.70 \\
		& Image + Text \textdagger & - & 2.65 & 3.25 & 4.14 & 4.54 & 11.71 & 35.06 & 48.94 & 77.49 & 32.77 & 56.89 & 74.96 \\
		& SEARLE + \textbf{PDV-I} & 1.5 & 4.77 & 5.23  & 6.31 & 6.82 & 16.65 & 42.53 & 55.16 & 81.42 & 44.68 & 67.78 & 82.94\\
		& CIReVL + \textbf{PDV-I} & 2.0 & \textbf{10.29 }& \textbf{10.80} & \textbf{12.23} & \textbf{12.93} & \textbf{27.18} & \textbf{56.53} & \textbf{67.76} & \textbf{87.64} & \textbf{59.81} & \textbf{79.59} & \textbf{90.15}\\
		& LDRE + \textbf{PDV-I} & 2.0 & 8.00 & 8.88 & 10.06 & 10.72 & 23.37 & 51.21 & 63.69 & 85.57 & 55.57 & 76.63 & 88.15\\
		\hline
	\end{tabular}
	\caption{PDV-I performance on CIRCO and CIRR test datasets. Note that the image-only approach utilizes the visual embedding of the reference image, whereas the text-only approach employs the text embedding of the prompt.}
	\label{tab:circo_cirr_results_pdv-I}
\end{table*}

We evaluated PDV-F alongside four baseline approaches (CIReVL, LDRE, Pic2Word, and SEARLE) across three benchmarks. Notably, CIReVL was tested with three different backbones on three datasets, as its models and intermediate results are publicly available. However, for the remaining methods, we conducted partial evaluations due to limited open-source availability or restricted support.

The numerical results are presented in Tables \ref{tab:fashion_iq_results} and \ref{tab:circo_cirr_results}.
On the FashionIQ benchmark, PDV-F yields substantial improvements for all baseline approaches, with CIReVL showing particularly strong gains that scale with backbone size. Similarly, all methods demonstrate significant performance improvements on CIRCO and CIRR datasets. Notably, CIReVL achieves larger improvements compared to other methods, with the most substantial gains observed when using small and medium backbone architectures. Our PDV-F implementation within the CIReVL framework consistently outperformed other state-of-the-art methods, including LinCIR and SEIZE, across most evaluation metrics. Similar to SEIZE, PDV-F offers the advantage of being entirely training-free; however, unlike SEIZE, it does not significantly increase feature extraction computational costs. While LinCIR demonstrates exceptional inference speed, it lacks the training-free nature of our approach, requiring dedicated model training before deployment.  






\section{Conclusion}
In this work, we establish a link between distribution recovery, posterior estimation, and representation learning, providing the first theoretical study of diffusion-based representation learning dynamics across varying noise scales. Using a low-dimensional mixture of low-rank Gaussians, we show that the unimodal representation learning dynamic arises from the interplay between data denoising and class specification. Additionally, our analysis highlights the inherent weight-sharing mechanism in diffusion models, demonstrating its benefits for peak representation performance as well as its limitations in optimizing high-noise regions due to increased complexity. Experiments on both synthetic and real datasets validate our findings.

\bibliographystyle{abbrvnat}
\bibliography{refs}

%%%%%%%%%%%%%%%%%%%%%%%%%%%%%%%%%%%%%%%%%%%%%%%%%%%%%%%%%%%%

\newpage
\appendix

\section{Appendix / supplemental material}
\section{Experimental Setup}\label{app:exp}
\subsection{Datasets}
UnKEBench \cite{UnKE} constructs a dataset containing 1,000 counterfactual unstructured texts, where knowledge is presented in an unstructured and relatively lengthy form, going beyond simple knowledge triplets or linear fact chains. These texts originate from ConflictQA \cite{conflictqa}, a benchmark specifically designed to distinguish LLMs' parameter memory from anti-memory. This approach is crucial for preventing the model from merging knowledge obtained during pretraining with knowledge acquired during the editing process. Moreover, it addresses the key challenge of determining whether the model learns target knowledge during training or editing, ensuring a clear boundary between pretraining knowledge and edited knowledge.

AKEW benchmark \cite{AKEW} considers three aspects: (1) \textit{Structured Facts}: Each structured fact consists of an isolated triplet for editing, sourced from existing datasets or knowledge bases. (2) \textit{Unstructured Facts}: Knowledge is presented in unstructured text form. To enable fair comparisons, each unstructured fact contains the same knowledge update as its corresponding structured fact. Compared to structured facts, unstructured facts exhibit greater complexity in natural language format, as they often encapsulate more implicit knowledge. (3) \textit{Extracted Triplets}: Triplets are extracted from unstructured facts using automated methods to investigate whether they can facilitate knowledge editing methods in handling unstructured knowledge. In this work, we primarily focus on unstructured factual knowledge.

EditEverything dataset integrates question-answering data from multiple domains, forming long and diverse knowledge formats that are more challenging to edit. Specifically, for mathematics, we select longer samples from the Orca-Math dataset \cite{math}, which includes grade school math word problems. For coding, we use the MBPP dataset \cite{code}, which consists of approximately 1,000 crowd-sourced Python programming problems solvable by entry-level programmers, covering programming fundamentals and standard library functionalities. For chemistry, we sample from the Camel-Chemistry dataset \cite{chemistry}, which contains problem-solution pairs generated from 25 chemistry topics, each with 25 subtopics and 32 problems per topic-subtopic pair. Lastly, for the news and poetry categories, since they often contain real-world knowledge that LLMs may already possess, we generate synthetic data using GPT-4o to ensure that the information is not already known by the model.

We present sample instances from the dataset in Figure \ref{fig:sample1}, Figure \ref{fig:sample2}, and Figure \ref{fig:sample3}.

\subsection{Evaluation Metrics} \label{app:exp_metric}
Following previous research on model editing for structured knowledge \cite{ROME, MEND}, existing evaluation metrics primarily focus on triplet-structured knowledge, where the goal is to assess the modification of factual triples (\textit{subject, relation, object}). Specifically, given an LLM $f$, an editing knowledge pair $(x, y)$, an equivalent knowledge query $x_e$, and unrelated knowledge pairs $(x_{loc}, y_{loc})$, three standard evaluation metrics are commonly used:

\textbf{Efficacy.} This metric quantifies the success of modifying the target knowledge in $f_{\mathcal{W}}$. It evaluates whether the edited LLM generates the desired target output $y$ when given the input $x$. Formally, it is defined as:
\begin{equation}
\mathbb{E}\left\{y=\mathop{\arg\max}\limits_{y'}\mathbb{P}_{f}(y'\left|x\right.)\right\}.
\end{equation}

\textbf{Generalization.} This metric assesses whether the model has generalized the newly edited knowledge beyond its specific form. It measures if the LLM correctly produces $y$ when given a semantically equivalent input $x_e$, indicating the degree to which the update propagates correctly across paraphrased or restructured queries:
\begin{equation}
\mathbb{E}\left\{y=\mathop{\arg\max}\limits_{y'}\mathbb{P}_{f}(y'\left|x_e\right.)\right\}.
\end{equation}

\textbf{Specificity.} This metric evaluates whether the editing operation is localized, ensuring that unrelated knowledge remains intact. It measures whether the model's response to an unrelated query $x_{loc}$ remains consistent with its original output $y_{loc}$:
\begin{equation}
\mathbb{E}\left\{y_{loc}=\mathop{\arg\max}\limits_{y'}\mathbb{P}_{f}(y'\left|x_{loc}\right.)\right\}.
\end{equation}

While these metrics are well-suited for structured knowledge editing, they are insufficient for evaluating long-form and diverse-formatted knowledge. Such knowledge is often verbose and complex, making it challenging to assess correctness solely based on Efficacy. In these cases, the model may generate an answer that captures the essential information yet fails an exact-match evaluation. To address this, we primarily follow the existing benchmarks for unstructured knowledge editing, incorporating more flexible evaluation methods suited for long-form responses.

Lexical similarity metrics include BLEU \cite{bleu} and various ROUGE scores (ROUGE-1, ROUGE-2, and ROUGE-L) \cite{rouge}. These are computed based on the \textit{original questions}, \textit{paraphrase question}, and \textit{sub-questions}, providing insights into the lexical and n-gram alignment between the model-generated text and the target answer. These metrics serve as the foundation for assessing the surface-level accuracy of edited content.

Semantic similarity is also considered (Bert Score) \cite{bertscore}, as word-level overlap alone is insufficient to capture the nuanced understanding required by the model. To address this, we utilize embedding-based encoders, specifically the all-MiniLM-L6-v2 model \footnote{https://huggingface.co/sentence-transformers/all-MiniLM-L6-v2}, to measure semantic similarity. This ensures a more balanced evaluation that extends beyond lexical matching, quantifying the depth of the model's comprehension.

\subsection{Baseline Methods}
\begin{itemize}
    \item \textbf{FT-L} \cite{FTw} is a knowledge editing approach that fine-tunes specific layers of the LLM using an autoregressive loss function. We reimplemented this method following the hyperparameter from the original paper.
    
    \item \textbf{MEND} \cite{MEND} is a hypernetwork-based efficient knowledge editing method. It trains a hypernetwork to capture patterns in knowledge updates by mapping low-rank decomposed fine-tuning gradients to LLM parameter modifications, enabling efficient and localized edits. Our implementation follows the original hyperparameter settings and completes training over the full dataset. 
    
    \item \textbf{ROME} \cite{ROME} is a method for modifying factual associations in LLM parameters. It identifies critical neuron activations in MLP layers through perturbation-based knowledge localization and modifies MLP layer weights using Lagrange remainders. Since ROME is not designed for large-scale edits, we follow the original paper’s settings and conduct multiple rounds of single-instance editing for evaluation.
    
    \item \textbf{MEMIT} \cite{MEMIT} extends ROME by enabling batch updates of factual knowledge. It utilizes least squares approximation to modify specific layer parameters across multiple layers, allowing simultaneous updates of large numbers of knowledge facts. We evaluate MEMIT in lifelong editing scenarios using the original paper’s configuration.
    
    \item \textbf{AlphaEdit} \cite{AlphaEdit} is a method designed to mitigate interference in LLM lifelong knowledge editing. It introduces a null-space projection mechanism that ensures parameter updates preserve previously edited knowledge while incorporating new updates. AlphaEdit has demonstrated state-of-the-art (SOTA) performance across multiple evaluation metrics while maintaining strong transferability. We follow the original paper’s hyperparameter configuration in our implementation.
    
    \item \textbf{UnKE} \cite{UnKE} improves knowledge editing by refining both the layer and token dimensions. In the layer dimension, it replaces local key-value storage with a non-local block-based mechanism, enhancing the representation capability of key-value pairs while integrating attention-layer knowledge. In the token dimension, it replaces "term-driven optimization" with "cause-driven optimization," which directly edits the final token while preserving contextual coherence. This eliminates the need for explicit term localization and prevents context loss.
\end{itemize}

\subsection{Implementation Details}
Our AnyEdit and AnyEdit* primarily follow the baseline configurations of MEMIT and UnKE, while other baselines adhere to their original implementation settings. All experiments were conducted on a single A100 GPU (80GB).
\begin{itemize}
    \item \textbf{AnyEdit on Llama3-8B-Instruct:} We select layers 4 to 8 for editing and apply a clamp norm factor of 4. The fact token is defined as the last token. The optimization process involves 25 gradient steps for updating the key-value representations, with a learning rate of 0.5. The loss is applied at layer 31, and we use a weight decay of 0.001. To maintain distributional consistency, we introduce a Kullback-Leibler (KL) regularization term with a factor of 0.0625. Furthermore, we enable hyperparameter $\lambda$ with an update weight of 15,000, using 100,000 samples from the Wikipedia dataset with a data type of float32. The module configurations follow MEMIT, where edits are applied to the MLP down projection layers of the selected transformer blocks. Additionally, for chunked editing, we set a chunk size of 40 tokens with no overlap.
    \item \textbf{AnyEdit on Qwen2.5-7B-Instruct:} Same as the above, except that the loss is applied at layer 27 and the chunk size is set to 50 tokens.
    \item \textbf{AnyEdit* on Llama3-8B-Instruct:} We select layer 7 for editing and apply a clamp norm factor of 4. The fact token is defined as the last token. The optimization process involves updating all parameters in both the attention and MLP layers. The gradient descent process utilizes a learning rate of 0.0002 with 50 optimization steps. For updating key-value representations, we use 25 gradient steps with a learning rate of 0.5. The loss is applied at layer 31, and we use a weight decay of 0.001. To preserve original knowledge, we sample 20 data points to constrain parameter updates. Additionally, for chunked editing, we set a chunk size of 40 tokens with no overlap.
    \item \textbf{AnyEdit* on Qwen2.5-7B-Instruct:} Same as the above, except that the loss is applied at layer 27 and the chunk size is set to 50 tokens.
\end{itemize}

\section{Locate-Then-Edit Paradigm \& Related Proof}
\subsection{Locate-Then-Edit Paradigm}\label{app:model_edit}
Following prior works on model editing, the detailed descriptions of specific methods in this section are based on MEMIT \cite{MEMIT}, AlphaEdit \cite{AlphaEdit} and ECE \cite{ECE}. We adhere to their formulations and methodological explanations to ensure consistency and clarity in presenting these approaches.

The locate-then-edit method primarily focuses on triplet-structured knowledge in the form of $(s, r, o)$, such as modifying $(\text{Olympics}, \text{were held in}, \text{Tokyo})$ to $(\text{Olympics}, \text{were held in}, \text{Paris})$. Given new knowledge $(x_e, y_e)$, a triplet can be represented as $x_e = (s, r)$ and $y_e = o$.

We first refine the auto-regressive language model formulation in Section \ref{sec:method:pre}. Let $f$ be a decoder-only model with $L$ layers, processing input sequence $x = (x_0, x_1, \dots, x_T)$ to predict the next token:
\begin{equation}
    \begin{aligned}
        \vh_t^l(x) &= \vh_t^{l - 1}(x) + \va_t^l(x) + \vm_t^l(x), \\
        \va_t^l &= \text{attn}^l(\vh_0^{l - 1}, \vh_1^{l - 1}, \dots, \vh_t^{l - 1}), \\
        \vm_t^l &= \mW_{\text{out}}^l \sigma(\mW_{\text{in}}^l \gamma(\vh_t^{l - 1}+\va_t^l)),
    \end{aligned}
\end{equation}
where $\vh_t^l$ denotes the hidden state of token $t$ at layer $l$, $\va_t^l$ is the attention output, and $\vm_t^l$ is the feedforward (FFN) output. Here, $\mW_{\text{in}}^l$ and $\mW_{\text{out}}^l$ are weight matrices, $\sigma$ is a nonlinear activation function, and $\gamma$ denotes layer normalization.

\textbf{Key-Value Memory Structure}. Locate-then-edit assumes that factual knowledge is stored in the FFN layers and treats them as linear associative memory \cite{key_value}. Specifically, $\mW_{\text{out}}^l$ is conceptualized as a key-value memory structure:
\begin{equation}
    \begin{aligned}
        \underbrace{\vm_t^l}_{\vv} = \mW_{\text{out}}^l \underbrace{\sigma(\mW_{\text{in}}^l \gamma(\vh_t^{l-1}+\va^l))}_{\vk}. \label{eqapp:define_kv}
    \end{aligned}
\end{equation}
Here, the MLP input-output pair at token $t$ and layer $l$ serves as the key-value pair. Casual Tracing is typically used to locate the target token and layer by injecting Gaussian noise into hidden states and incrementally restoring them to analyze output recovery. For more details, please refer to ROME \cite{ROME}.

\textbf{Computing Key-Value.} For editing knowledge $(x_e, y_e)$, we compute its corresponding key-value pair $(\vk^*, \vv^*)$. The key $\vk^*$ is derived via forward propagation of $x_e$, while the value $\vv^*$ is optimized using gradient descent:
\begin{equation}
    \vv^* = \vv + \arg \min_{\bm{\delta}^l} \left( -\log \mathbb{P}_{f(\vh_t^l + \bm{\delta}^l)} [y_e \mid x_e] \right).
\end{equation}
Here, $f(\vh_t^l + \bm{\delta}^l)$ represents the model output when the FFN output $\vh_t^l$ is replaced with $\vh_t^l + \bm{\delta}^l$. 

Methods such as ROME \cite{ROME}, MEMIT \cite{MEMIT}, and AlphaEdit \cite{AlphaEdit} focus on triplets $(s, r, o)$, selecting the last token of the subject $s$ as the target token. In contrast, UnKE \cite{UnKE} extends to unstructured text, using the last token of $x_e$ as the target.

To insert new knowledge $(\vk^*, \vv^*)$ into the key-value memory, we solve the constrained least squares problem:
\begin{align*}
    \min_{\hat{\mW}} &\quad \left\lVert \hat{\mW}\mK - \mV \right\rVert \\
    \text{s.t.} &\quad \hat{\mW}\vk^* = \vv^*.
\end{align*}
The final parameter update can be computed via ROME/MEMIT/AlphaEdit's closed-form solution or UnKE's gradient-based optimization.

For clarity, let $\tilde{\mW}$ denote the edited weight of $\mW_{\text{out}}^l$ in the MLP, and let $\mW$ represent its original weight. The final parameter update can be computed using the closed-form solutions of ROME/MEMIT/AlphaEdit or the gradient-based optimization method in UnKE.

\textbf{Weights Update in ROME.} The ROME method derives a closed-form solution to the constrained least-squares problem for updating MLP parameters:
\begin{equation}
    \tilde{\mW} = \mW + \frac{(\vv^\ast - \mW\vk^\ast) (\mC^{-1} \vk^\ast) ^ {T}}{(\mC^{-1} \vk^\ast) ^ {T} \vk^\ast},
\end{equation}
where $\mC = \mK \mK^T$. The matrix $\mC$ is estimated using 100,000 samples of hidden states $\vk$ obtained from tokens sampled in-context from the entire Wikipedia dataset.

\textbf{Weights Update in MEMIT.} Since the above solution updates only a single knowledge sample at a time, MEMIT improves upon it by avoiding Lagrange multipliers and instead using a relaxed constraint formulation. The problem is reformulated by maintaining a factual set $\{\mK_1, \mV_1\}$ containing $u$ new associations while preserving the original set $\{\mK_0, \mV_0\}$ with $n$ existing associations:
\begin{equation}
\begin{gathered}
    \mK_0 = \left[\vk_1 \mid \vk_2 \mid \dots \mid \vk_n\right], \quad \mV_0 = \left[\vv_1 \mid \vv_2 \mid \dots \mid \vv_n\right], \\
    \mK_1 = \left[\vk^\ast_{n+1} \mid \vk^\ast_{n+2} \mid \dots \mid \vk^\ast_{n+u}\right], \quad \mV_1 = \left[\vv^\ast_{n+1} \mid \vv^\ast_{n+2} \mid \dots \mid \vv^\ast_{n+u}\right].
\end{gathered}
\end{equation}
Here, $\vk$ and $\vv$ are defined as in Eq.~\ref{eqapp:define_kv}, and their subscripts denote knowledge indices. The objective function is given by:
\begin{equation}
    \tilde{\mW} \triangleq \argmin_{\hat{\mW}} \left( \sum_{i=1}^{n} \left\| \hat{\mW} \vk_i - \vv_i \right\|^2 + \sum_{i=n+1}^{n+u} \left\| \hat{\mW} \vk_i - \vv^\ast_i \right\|^2 \right).
\end{equation}
Applying the normal equation \citep{normal_equation}, the closed-form solution is:
\begin{equation}
    \tilde{\mW} = \left( \mV_1 - \mW \mK_1 \right) \mK_1^T \left( \mK_0 \mK_0^T + \mK_1 \mK_1^T \right)^{-1} + \mW.
\end{equation}

\textbf{Weights Update in AlphaEdit.} AlphaEdit addresses the imbalance between old and new knowledge in lifelong learning. It protects existing knowledge using a null-space projection constraint, ensuring that the update $\bm{\Delta}$ to $\mW_{\text{out}}^l$ is always projected onto the null space of $\mK_0 \mK_0^T$. The final parameter update, refining MEMIT, is:
\begin{equation}
    \tilde{\mW} = \left( \mV_1 - \mW \mK_1 \right) \mK_1^T \mP \left( \mK_p \mK_p^T \mP + \mK_1 \mK_1^T \mP + \mI \right)^{-1}+ \mW.
\end{equation}

\textbf{Weights Update in UnKE.} Unlike previous methods, UnKE considers the entire input to layer $l$, denoted as $f^l$, rather than just the MLP input. The output remains $f^l$'s activation values. The parameter update is applied to the entire layer rather than a single weight matrix. Given the knowledge sets $\{\mK_0, \mV_0\}$ and $\{\mK_1, \mV_1\}$, the optimization objective is formulated as:
\begin{equation}
    \tilde{\Theta}^l \triangleq \argmin_{\hat{\Theta}^l} \left( \sum_{i=1}^{n} \left\|  f_{\hat{\Theta}^l}^l(\vk_i) - \vv_i \right\|^2 + \sum_{i=n+1}^{n+u} \left\|  f_{\hat{\Theta}^l}^l(\vk_i) - \vv^\ast_i \right\|^2 \right),
\end{equation}
where $\Theta^l$ denotes the entire set of parameters in layer $l$. Since a closed-form solution is not feasible, UnKE employs gradient descent to iteratively update $\Theta^l$.

\subsection{Proof of Optimization-Conditional Mutual Information Equivalence} \label{app:proof_cmi}
\begin{theorem}
The optimization objective  
\begin{equation}
    \bm{\delta}^* = \argmin_{\bm{\delta}} \left( -\log \mathbb{P}_{f(\vh_t+\bm{\delta})}(Y \mid X) \right), \label{eq:opt}
\end{equation}  
is equivalent to maximizing the conditional mutual information (CMI) between $X$ and $Y$ given the perturbed hidden state $\vh'$:  
\begin{equation}
    \vh' = \argmax_{\vh'} I(X; Y \mid \vh'). \label{eq:cmi}
\end{equation}
\end{theorem}

\begin{proof}
Starting from the definition of CMI, we expand it via the integral form:  
\begin{equation}
I(X; Y \mid \vh') = \int p(x, y, \vh') \log \frac{p(y \mid x, \vh')}{p(y \mid \vh')} \, dx dy d\vh'.
\end{equation}  
% Applying Bayes’ rule $p(y \mid x, \vh') = \frac{p(x, y \mid \vh')}{p(x \mid \vh')}$, we rewrite the integrand:  
% \begin{equation}
% I(X; Y \mid \vh') = \int p(x, y, \vh') \log \frac{p(x, y \mid \vh')}{p(x \mid \vh') p(y \mid \vh')} \, dx dy d\vh'.
% \end{equation}  
This splits into two entropy terms:  
\begin{align}
I(X; Y \mid \vh') = \underbrace{\int p(x, y, \vh') \log p(y \mid x, \vh') \, dx dy d\vh'}_{\text{Term } \mathcal{A}} - \underbrace{\int p(x, y, \vh') \log p(y \mid \vh') \, dx dy d\vh'}_{\text{Term } \mathcal{B}}. \label{eq:split}
\end{align}  

Term $\mathcal{A}$ simplifies to the expectation:  
\begin{equation}
\mathcal{A} = \mathbb{E}_{p(\vh')} \mathbb{E}_{p(x, y \mid \vh')} \left[ \log p(y \mid x, \vh') \right],
\end{equation}  
while Term $\mathcal{B}$ is independent of $X$ given $\vh'$. Since $\mathcal{B}$ does not affect the optimization over $\vh'$, we focus on maximizing $\mathcal{A}$.  

By definition, $\mathbb{P}_{f(\vh')}(Y \mid X) = p(y \mid x, \vh')$. Thus, minimizing the negative log-likelihood in \eqref{eq:opt} directly maximizes $\mathcal{A}$, which is equivalent to maximizing $I(X; Y \mid \vh')$. Substituting $\vh' = \vh_t + \bm{\delta}^*$, we conclude:  
\begin{equation}
\vh' = \argmax_{\vh'} I(X; Y \mid \vh'),
\end{equation}  
thereby establishing the equivalence.  
\end{proof}

\subsection{Proof of the Decomposition of Mutual Information}\label{app:proof_decom}
To rigorously derive Equation \eqref{eq:final_MI}, we start from the mutual information (MI) decomposition given in Equation \eqref{eq:decom}:
\begin{equation}
    I(X; Y \mid \vh'_1, \dots, \vh'_K) = \sum_{k=1}^{K} I(X; Y_k \mid Y_1, \dots, Y_{k-1}, \vh'_1, \dots, \vh'_K).
\end{equation}

\textbf{Step 1: Application of the First Property.}
The first key property states that later hidden states do not influence earlier token generation:
\begin{equation}
    H(Y_p \mid \vh'_q) = H(Y_p), \quad \text{for } p < q.
\end{equation}
Since mutual information is defined as:
\begin{equation}
    I(X; Y_k \mid Y_1, \dots, Y_{k-1}, \vh'_1, \dots, \vh'_K) = H(Y_k \mid Y_1, \dots, Y_{k-1}, \vh'_1, \dots, \vh'_K) - H(Y_k \mid X, Y_1, \dots, Y_{k-1}, \vh'_1, \dots, \vh'_K).
\end{equation}
Since $\vh'_q$ for $q > k$ does not affect $Y_k$, we can simplify:
\begin{equation}
    H(Y_k \mid Y_1, \dots, Y_{k-1}, \vh'_1, \dots, \vh'_K) = H(Y_k \mid Y_1, \dots, Y_{k-1}, \vh'_1, \dots, \vh'_k).
\end{equation}

\textbf{Step 2: Application of the Second Property.}
The second key property states that once $Y_k$ is determined, conditioning on $Y_k$ subsumes conditioning on $\vh'_k$:
\begin{equation}
    H(\cdot \mid Y_k) = H(\cdot \mid Y_k, \vh'_k).
\end{equation}
Using this, we rewrite the MI term:
\begin{equation}
    I(X; Y_k \mid Y_1, \dots, Y_{k-1}, \vh'_1, \dots, \vh'_K) = I(X; Y_k \mid Y_1, \dots, Y_{k-1}, \vh'_k).
\end{equation}

\textbf{Step 3: Applying the Conditional Mutual Information Decomposition.}
Using the decomposition formula for conditional mutual information, each term can be written as:
\begin{equation}
    I(X; Y_k \mid Y_1, \dots, Y_{k-1}, \vh'_k) = I(X, Y_1, \dots, Y_{k-1}; Y_k \mid \vh'_k) - I(Y_1, \dots, Y_{k-1}; Y_k \mid \vh'_k).
\end{equation}
In the optimization process, since we are given $X$, the second term is ignored:
\begin{equation}
    I(X; Y_k \mid Y_1, \dots, Y_{k-1}, \vh'_k) = I(X, Y_1, \dots, Y_{k-1}; Y_k \mid \vh'_k).
\end{equation}
Substituting this result back into our summation, we arrive at the desired decomposition:
\begin{equation}
    I(X; Y \mid \vh'_1, \dots, \vh'_K) = \sum_{k=1}^{K} I(X, Y_1, \dots, Y_{k-1}; Y_k \mid \vh'_k).
\end{equation}
This completes the proof.

\section{More Experimental Results}\label{app:exp_result}
\subsection{Case Study}

We selected several editing samples from the UnKEBench and EditEverything datasets as case studies to analyze the long-form knowledge edit performance. 

From the experimental results, AnyEdit effectively restores the target text in terms of both textual consistency and generalization to paraphrase questions. In contrast, other baseline methods often fail to achieve effective editing. Although the UnKE method also demonstrates strong editing performance, it exhibits a noticeable performance drop on paraphrase questions.


\subsubsection{Case 1}

\begin{tcolorbox}[boxrule=0.5pt, left=0pt, right=0pt, top=2.5pt, bottom=2.5pt, title={Model Editing Case Study on Llama3-8B-Instruct}]
    \centering
    \renewcommand{\arraystretch}{1.8}
    \begin{tabular}{p{0.18\linewidth} p{0.76\linewidth}}
        \makecell[c]{\textbf{Input}} & \makecell[c]{What is John Mayne's background and experience in journalism?} \\
        \hline
        \makecell[c]{\textbf{Paraphrase}} & \makecell[c]{Can you provide information about John Mayne's journalism background and experience?} \\
        \hline
        \makecell[c]{\textbf{Target Output}} & \makecell*[{{p{13cm}}}]{John Mayne is a journalist who has been working in the field for over a decade. He has written for several reputable publications, including \tar{The New York Times} and \tar{The Guardian}. In an interview with \tar{The Huffington Post}, Mayne discussed his passion for journalism and his commitment to reporting on important issues. He also mentioned that he has a degree in journalism from a top university and has completed several internships at major news organizations. Mayne's extensive experience and education make him a highly qualified journalist and a valuable asset to any newsroom.} \\
        \hline
        \multicolumn{2}{c}{\textbf{\large LLM-edited Generation}} \\ % Bold and centered
        \hline
        \makecell[c]{\textbf{MEMIT}} & \makecell*[{{p{13cm}}}]{John Mayne is a journalist and writer, but I couldn't find much about his personal background. However, he has been in the industry for several years, contributing to various publications. According to LinkedIn, he has worked for major outlets including BBC and \tar{The Guardian}(...) John Mayne's experience spans politics, business, and culture, and he has been recognized for his in-depth reporting.} \\
        \hline
        \makecell[c]{\textbf{AlphaEdit}} & \makecell*[{{p{13cm}}}]{John Mayne is a British journalist and author with a long career in media. He has written for major newspapers such as The Times and The Sunday Times. Mayne has covered diverse topics, including politics and culture, and worked as a foreign correspondent reporting from the Middle East, Africa, and Asia... In addition, he has taught journalism at universities and received multiple awards for his writing.} \\
        \hline
        \makecell[c]{\textbf{UnKE}} & \makecell*[{{p{13cm}}}]{John Mayne is a journalist who has been working in the field for several years. He has written for a number of publications and has a strong track record of producing high-quality, well-researched articles. In addition to his writing experience, Mayne has also worked as an editor and has a strong understanding of the journalism industry. He is well-respected by his peers and is known for his ability to produce engaging and informative articles.} \\
        \hline
        \makecell[c]{\textbf{AnyEdit}} & \makecell*[{{p{13cm}}}]{ John Mayne is a journalist who has been working in the field for over a decade. He has written for several reputable publications, including \tar{The New York Times} and \tar{The Guardian}. In an interview with \tar{The Huffington Post}, Mayne discussed his passion for journalism and his commitment to reporting on important issues. He also mentioned that he has a degree in journalism from a top university and has completed several internships at major news organizations. Mayne's extensive experience and education make him a highly qualified journalist and a valuable asset to any newsroom.} \\
    \end{tabular}
\end{tcolorbox}

\subsubsection{Case 2}

\begin{tcolorbox}[boxrule=0.5pt, left=0pt, right=0pt, top=2.5pt, bottom=2.5pt, title={Model Editing Case Study on Qwen2.5-7B-Instruct}]
    \centering
    \renewcommand{\arraystretch}{1.8}
    \begin{tabular}{p{0.18\linewidth} p{0.76\linewidth}}
        \makecell[c]{\textbf{Input}} & \makecell[c]{find the value of $log_y(x^4)$ if $log_x(y^3) = 2$} \\
        \hline
        \makecell[c]{\textbf{Target Output}} & \makecell*[{{p{13cm}}}]{To find the value of $log_y(x^4)$, we can use the property of logarithms that states: $log_b(a) = 1 / log_a(b)$. Given that $log_x(y^3) = 2$, we can rewrite this as: $y^3 = x^2$. Now, we want to find $log_y(x^4)$. We can use the property mentioned above: $log_y(x^4) = 1 / log_x(y^4)$. But we don't have $log_x(y^4)$; we have $log_x(y^3)$. So, we need to manipulate the expression to use the given information. We know that $y^3 = x^2$, so we can raise both sides to the power of $4/3$ to get $y^4$: $(y^3)^{(4/3)} = (x^2)^{(4/3)}$. $y^4 = x^{(8/3)}$. Now we can write $log_x(y^4)$ in terms of $log_x(y^3)$: $log_x(y^4) = log_x((y^3)^{(4/3)})$. Using the power rule of logarithms ($log_b(a^c) = c * log_b(a)$), we get: $log_x(y^4) = (4/3) * log_x(y^3)$. Since we know $log_x(y^3) = 2$, we can substitute this value in: $log_x(y^4) = (4/3) * 2$. $log_x(y^4) = 8/3$. Now we can find $log_y(x^4)$ using the reciprocal property: $log_y(x^4) = 1 / log_x(y^4)$. $log_y(x^4) = 1 / (8/3)$. $log_y(x^4) = 3/8$. Therefore, the value of $log_y(x^4)$ is \tar{$3/8$}.} \\
        \hline
        \multicolumn{2}{c}{\textbf{\large LLM-edited Generation}} \\ % Bold and centered
        \hline
        \makecell[c]{\textbf{MEMIT}} & \makecell*[{{p{13cm}}}]{To find the value of $log_y(x^4)$, we can use the change of base formula and the given information. Given: $log_x(y^3) = 2$. Using the change of base formula, we can rewrite this as: $log_x(y^3) = (log_y(y^3)) / (log_y(x))$. Since $log_y(y^3) = 3$ (because $y^3$ is the argument raised to the power of 3), we have: $2 = 3 / (log_y(x))$. Now, we can solve for $log_y(x)$: $log_y(x) = 3/2$. Next, we want to find $log_y(x^4)$. Using the power rule of logarithms, we can write: $log_y(x^4) = 4 * log_y(x)$. Substituting the value we found for $log_y(x)$: $log_y(x^4) = 4 * (3/2)$. Simplifying: $log_y(x^4) = 6$. Therefore, the value of $log_y(x^4)$ is $6$.} \\
        \hline
        \makecell[c]{\textbf{AlphaEdit}} & \makecell*[{{p{13cm}}}]{To find the value of $log_y(x^4)$, we can use the change of base formula and the given information: $log_x(y^3) = 2$. Using the change of base formula, we can rewrite this as: $y^3 = x^2$. Now, we want to find $log_y(x^4)$. Using the properties of logarithms, we can rewrite this as: $log_y(x^4) = 4 * log_y(x)$. We can express $log_y(x)$ in terms of $log_x(y)$ using the change of base formula: $log_y(x) = 1 / log_x(y)$. Since we know that $log_x(y^3) = 2$, we can write: $log_x(y^3) = 3 * log_x(y) = 2$. Therefore: $log_x(y) = 2/3$. Substituting this into our expression for $log_y(x)$: $log_y(x) = 1 / (2/3) = 3/2$. Finally, substituting this back into our expression for $log_y(x^4)$: $log_y(x^4) = 4 * (3/2) = 6$. Therefore, the value of $log_y(x^4)$ is $6$.} \\
        \hline
        \makecell[c]{\textbf{UnKE}} & \makecell*[{{p{13cm}}}]{To find the value of $log_y(x^4)$, we can use the property of logarithms that states: $log_b(a) = 1 / log_a(b)$. Given that $log_x(y^3) = 2$, we can rewrite this as $y^3 = x^2$. Now, we can use the property to find $log_y(x^4)$: $log_y(x^4) = 1 / log_x(y^3)$ $= 1 / (1 / x^2)$ $= x^2$. So, the value of $log_y(x^4)$ is $x^2$. Therefore, the correct answer is $x^2$.} \\
        \hline
        \makecell[c]{\textbf{AnyEdit}} & \makecell*[{{p{13cm}}}]{To find the value of $log_y(x^4)$, we can use the property of logarithms that states: $log_b(a) = 1 / log_a(b)$. Given that $log_x(y^3) = 2$, we can rewrite this as: $y^3 = x^2$. Now, we want to find $log_y(x^4)$. We can use the property mentioned above: $log_y(x^4) = 1 / log_x(y^4)$. But we don't have $log_x(y^4)$; we have $log_x(y^3)$. So, we need to manipulate the expression to use the given information. We know that $y^3 = x^2$, so we can raise both sides to the power of $4/3$ to get $y^4$: $(y^3)^{(4/3)} = (x^2)^{(4/3)}$. $y^4 = x^{(8/3)}$. Now we can write $log_x(y^4)$ in terms of $log_x(y^3)$: $log_x(y^4) = log_x((y^3)^{(4/3)})$. Using the power rule of logarithms ($log_b(a^c) = c * log_b(a)$), we get: $log_x(y^4) = (4/3) * log_x(y^3)$. Since we know $log_x(y^3) = 2$, we can substitute this value: $log_x(y^4) = (4/3) * 2$. $log_x(y^4) = 8/3$. Now we can use the property of logarithms: $log_y(x^4) = 1 / log_x(y^4)$. $log_y(x^4) = 1 / (8/3)$. $log_y(x^4) = 3/8$. Therefore, the value of $log_y(x^4)$ is \tar{$3/8$}.} \\
    \end{tabular}
\end{tcolorbox}
\clearpage
\newpage

\subsection{Supplementary Experimental Results on RQ1 \& RQ2} \label{app:exp_result_1}
We present a comprehensive evaluation of all metrics on the UnKEBench and AKEW datasets in Table \ref{tab:app_1} and Table \ref{tab:app_2}. The results demonstrate that UnKE consistently outperforms other baselines across both original and paraphrase question evaluations. Notably, UnKE+, which integrates AnyEdit’s autoregressive editing paradigm, achieves even higher scores in lexical similarity (BLEU, ROUGE-1/2/L) and semantic similarity (BERT Score), indicating its superior ability to preserve and generalize edited knowledge. In contrast, MEMIT and AlphaEdit struggle with paraphrase generalization, showing significantly lower performance on the right side of `/', suggesting that these methods fail to robustly transfer edited knowledge across rephrased contexts. While MEMIT+ and AlphaEdit+ improve over their base versions, their performance still lags behind UnKE and UnKE+.

Overall, UnKE+ achieves the best balance between precise knowledge modification and robust generalization, confirming that combining UnKE with autoregressive fine-tuning leads to stronger and more reliable knowledge editing in LLMs.
\begin{table*}[h]
\caption{Performance comparison in UnKEBench. The `+' symbol indicates results incorporating AnyEdit's autoregressive editing paradigm. The left side of `/' represents the LLM's edited output for original questions, while the right side represents the edited output for paraphrase questions.}
    \label{tab:app_1}
    \centering
    \renewcommand{\arraystretch}{1.2}
    \setlength{\tabcolsep}{4pt}
    \resizebox{\textwidth}{!}{
    \begin{tabular}{l cccc ccc}
        \toprule
        \multirow{2}{*}{\textbf{Method}} & \multicolumn{4}{c}{\textbf{Lexical Similarity}} & \multicolumn{1}{c}{\textbf{Semantic Similarity}} & \textbf{Sub Questions} \\
        \cmidrule(lr){2-5} \cmidrule(lr){6-6} \cmidrule(lr){7-7} 
        & BLEU & ROUGE-1 & ROUGE-2 & ROUGE-L & BERT Score & ROUGE-L \\
        \midrule
        \multicolumn{7}{l}{\textbf{Based on Llama3-8B-Instruct}} \\
        \midrule
        UnKE        & 93.56 / 78.09  & 93.61 / 79.26  & 91.42 / 71.73  & 93.33 / 78.42  & 98.34 / 93.38    & 37.87 \\
        UnKE+       & 99.67 / 84.60  & 99.69 / 86.31  & 99.57 / 81.18  & 99.68 / 85.75  & 99.86 / 94.70    & 41.45 \\
        MEMIT       & 25.57 / 22.88  & 32.67 / 30.75  & 14.51 / 12.37  & 30.49 / 28.65  & 76.21 / 74.25    & 22.56 \\
        MEMIT+      & 88.88 / 81.38  & 93.26 / 86.53  & 90.32 / 80.61  & 92.96 / 85.91  & 97.76 / 95.60    & 41.67 \\
        AlphaEdit   & 21.29 / 20.24  & 28.62 / 27.99  & 11.36 / 10.24  & 26.59 / 25.92  & 73.92 / 72.96    & 20.71 \\
        AlphaEdit+  & 75.02 / 66.35  & 81.70 / 73.47  & 74.35 / 62.74  & 80.92 / 72.22  & 94.19 / 91.51    & 40.56 \\
        \midrule
        \multicolumn{7}{l}{\textbf{Based on Qwen2.5-7B-Instruct}} \\
        \midrule
        UnKE        & 91.92 / 70.61  & 91.41 / 68.47  & 87.75 / 56.34  & 91.01 / 67.00  & 96.97 / 89.17    & 38.12 \\
        UnKE+       & 98.52 / 82.48  & 98.85 / 83.36  & 98.43 / 77.03  & 98.82 / 82.60  & 99.35 / 94.81    & 42.24 \\
        MEMIT       & 45.07 / 40.81  & 40.73 / 36.75  & 19.59 / 15.87  & 38.04 / 34.07  & 78.03 / 76.50    & 24.75 \\
        MEMIT+      & 91.31 / 77.23  & 95.10 / 80.88  & 92.93 / 72.50  & 94.89 / 79.98  & 98.05 / 93.56    & 42.38 \\
        AlphaEdit   & 49.71 / 45.21  & 45.42 / 41.06  & 24.63 / 19.85  & 42.77 / 38.26  & 80.48 / 78.38    & 25.37 \\
        AlphaEdit+  & 97.77 / 83.09  & 98.20 / 84.18  & 97.40 / 77.38  & 98.14 / 83.40  & 99.08 / 94.51    & 41.58 \\
        \bottomrule
    \end{tabular}
    }
    
\end{table*}

\begin{table*}[h]
\caption{Performance comparison in AKEW (Counterfact). The `+' symbol indicates results incorporating AnyEdit's autoregressive editing paradigm. The left side of `/` represents the LLM's edited output for original questions, while the right side represents the edited output for paraphrase questions.}
    \label{tab:app_2}
    \centering
    \renewcommand{\arraystretch}{1.2}
    \setlength{\tabcolsep}{4pt}
    \resizebox{\textwidth}{!}{
    \begin{tabular}{l cccc ccc}
        \toprule
        \multirow{2}{*}{\textbf{Method}} & \multicolumn{4}{c}{\textbf{Lexical Similarity}} & \multicolumn{1}{c}{\textbf{Semantic Similarity}} & \textbf{Sub Questions} \\
        \cmidrule(lr){2-5} \cmidrule(lr){6-6} \cmidrule(lr){7-7} 
        & BLEU & ROUGE-1 & ROUGE-2 & ROUGE-L & BERT Score & ROUGE-L \\
        \midrule
        \multicolumn{7}{l}{\textbf{Based on Llama3-8B-Instruct}} \\
        \midrule
        MEMIT       & 33.44 / 18.13  & 34.46 / 17.44  & 16.29 / 4.74   & 32.20 / 16.10  & 76.44 / 47.80  & 39.98\\
        MEMIT+      & 85.41 / 38.78  & 96.07 / 47.61  & 94.21 / 32.37  & 95.87 / 46.00  & 97.76 / 62.63  & 64.07\\
        UnKE        & 98.43 / 36.99  & 98.43 / 34.58  & 97.78 / 19.37  & 98.37 / 32.89  & 99.62 / 59.62  & 63.22\\
        UnKE+       & 99.98 / 45.23  & 99.98 / 46.57  & 99.96 / 35.41  & 99.98 / 45.31  & 99.95 / 64.24  & 59.03\\
        AlphaEdit   & 23.36 / 16.25  & 26.92 / 15.00  & 10.81 / 3.61   & 24.95 / 13.79  & 72.63 / 44.67  & 35.76 \\
        AlphaEdit+  & 79.60 / 40.67  & 84.49 / 41.11  & 78.00 / 26.60  & 83.76 / 39.51  & 96.51 / 65.14  & 57.05 \\
        \midrule
        \multicolumn{7}{l}{\textbf{Based on Qwen2.5-7B-Instruct}} \\
        \midrule
        MEMIT       & 45.29 / 32.83  & 41.68 / 28.01  & 20.38 / 8.79   & 38.95 / 25.73  & 77.19 / 56.04  & 43.51\\
        MEMIT+      & 90.55 / 44.32  & 95.33 / 45.56  & 93.12 / 27.38  & 95.09 / 43.49  & 98.08 / 65.40  & 55.10\\
        UnKE        & 91.53 / 38.59  & 90.91 / 31.53  & 87.06 / 12.11  & 90.44 / 29.27  & 97.34 / 59.29  & 49.97\\
        UnKE+       & 98.95 / 34.68  & 99.01 / 35.23  & 98.59 / 15.59  & 98.99 / 32.95  & 99.63 / 60.78  & 51.58\\
        AlphaEdit   & 49.97 / 34.65  & 48.15 / 30.02  & 27.76 / 10.38  & 45.55 / 27.69  & 80.66 / 56.99  & 45.12\\
        AlphaEdit+  & 97.61 / 46.97  & 97.80 / 47.63  & 96.89 / 30.31  & 97.73 / 45.84  & 99.10 / 66.10  & 54.99\\
        \bottomrule
    \end{tabular}
    }
    
\end{table*}

\subsection{Supplementary Experimental Results on RQ4}\label{app:exp_result_4}
\begin{figure}[t]
\begin{center}
\includegraphics[width=0.6\linewidth, keepaspectratio]{figures/exp_3.png}
\caption{The relationship between AnyEdit's editing performance and chunk size in long-form diverse-formatted knowledge.}
\label{fig:exp_3}
\end{center}
\end{figure}


 The experimental results of relationship between AnyEdit's editing performance and chunk size in long-form diverse-formatted knowledge are presented in Figure \ref{fig:exp_3}. Based on these results, we draw the following observation:.

\begin{itemize}[leftmargin=*]
    \item \textbf{Obs 7: The editing performance of AnyEdit is influenced by chunk size.}  
    As the chunk size increases beyond a certain threshold, the editing performance of AnyEdit declines. Specifically, when the chunk size is smaller, each iteration of editing becomes more manageable, leading to improved overall performance. However, this improvement comes at the cost of increased editing time due to the larger number of iterations required for longer texts. Conversely, when the chunk size is larger, it becomes challenging to achieve effective edits within a single iteration, resulting in degraded performance. Based on this trade-off, we recommend using a balanced chunk size of 40 for most editing scenarios.
\end{itemize}

\begin{figure}[h]
    \centering
    \includegraphics[width=\textwidth]{figures/data1.png}
    \vspace{-5mm}
    \caption{A Sample of the AKEW (Counterfact) dataset.}
    \label{fig:sample1}
\end{figure}

\begin{figure}[h]
    \centering
    \includegraphics[width=\textwidth]{figures/data2.png}
    \vspace{-5mm}
    \caption{A Sample of the UnKEBench dataset.}
    \label{fig:sample2}
\end{figure}

\begin{figure}[h]
    \centering
    \includegraphics[width=\textwidth]{figures/data3.png}
    \vspace{-5mm}
    \caption{Samples of the EditEverything dataset.}
    \label{fig:sample3}
\end{figure}

\end{document}