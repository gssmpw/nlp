\documentclass[11pt]{article}
\usepackage[top=1in, bottom=1in, left=1in, right=1in]{geometry}
\usepackage{tgpagella}
\usepackage[utf8]{inputenc} % allow utf-8 input
\usepackage[T1]{fontenc}    % use 8-bit T1 fonts


\usepackage{url}
% \usepackage[hidelinks]{hyperref}
\usepackage{hyperref}
\usepackage{amsmath,amsthm,amssymb,amsbsy,amsfonts,amscd,bm}
\usepackage{paralist}
\usepackage{xcolor}
\usepackage{color}
\usepackage{cleveref}
\usepackage{graphicx}
\graphicspath{{./figs/}}
\usepackage{algorithm}
\usepackage{algorithmic}
\usepackage{comment}
\usepackage{multirow}
\usepackage{enumitem}
\usepackage{fancyhdr}
%\usepackage{cite}
%\usepackage{citesort}
% \usepackage{cleveref}
\usepackage{subcaption}
\usepackage{wrapfig}
% Theorems

% Uncomment these if not for icml
\def \remarks {\noindent {\bf Remarks.}\ \ }
\newtheorem{thm}{Theorem}%[section] %(If you want theorem numbered
\newtheorem{lemma}{Lemma}%[section] %%    with section number.
\newtheorem{cor}{Corollary}%[section]
\newtheorem{prop}{Proposition}%[section]
\newtheorem{definition}{Definition}%[section]
\newtheorem{defi}{Definition}%[section]
\newtheorem{assum}{Assumption}
\newtheorem*{goal}{Plan}
\newtheorem{theorem}{Theorem}
\newtheorem{lem}{Lemma}
\theoremstyle{remark}
\newtheorem{remark}{Remark}
% Uncomment above these if not for icml

% \theoremstyle{problem}
% \newtheorem{problem}{Problem}

% fields
\renewcommand{\O}{\mathbb{O}}
\newcommand{\R}{\mathbb{R}}
\def \real    { \mathbb{R} }
\def \reals   { \mathbb{R} }
\def \Real {\mathbb{R}}
\newcommand{\C}{\mathbb{C}}
\def \complex{\mathbb{C}}
\def \Complex {\mathbb{C}}
\newcommand{\Z}{\mathbb{Z}}
\newcommand{\N}{\mathbb{N}}

%%%
\newcommand{\xiao}[1]{\textcolor{orange}{\bf [{\em Xiao:} #1]}}
%%%
\newcommand{\la}{\left\langle}
\newcommand{\ra}{\right\rangle}
\newcommand{\e}{\begin{equation}}
\newcommand{\ee}{\end{equation}}
\newcommand{\en}{\begin{equation*}}
\newcommand{\een}{\end{equation*}}
\newcommand{\eqn}{\begin{eqnarray}}
\newcommand{\eeqn}{\end{eqnarray}}
\newcommand{\bmat}{\begin{bmatrix}}
\newcommand{\emat}{\end{bmatrix}}
% complex numbers
\renewcommand{\Re}[1]{\operatorname{Re}\left\{#1\right\}}
\renewcommand{\Im}[1]{\operatorname{Im}\left\{#1\right\}}
\newcommand{\conj}[1]{\mkern 1.5mu\overline{\mkern-1.5mu#1\mkern-1.5mu}\mkern 1.5mu}

\DeclareMathAlphabet\mathbfcal{OMS}{cmsy}{b}{n}
% probability and stat
\renewcommand{\P}[1]{\operatorname{\mathbb{P}}\left(#1\right)}
\newcommand{\E}{\operatorname{\mathbb{E}}}
\newcommand{\dif}{\operatorname{d}}
\newcommand{\var}{\operatorname{var}}

% calculus
\renewcommand{\d}[1]{\,\mathrm{d}#1}

% constants (written in roman, if wanted)
%\newcommand{\e}{\mathrm{e}}
\renewcommand{\j}{\mathrm{j}}




\newcommand{\mb}{\bm}
\newcommand{\mc}{\mathcal}
\newcommand{\mf}{\mathfrak}
\newcommand{\bb}{\mathbb}
\newcommand{\mrm}{\mathrm}
\newcommand{\mtt}{\mathtt}
\def \vecword  {\operatorname*{vec}}
% linear algebra
% 	vector notation
\newcommand{\vct}[1]{\boldsymbol{#1}}
%   matrices
\newcommand{\mtx}[1]{\boldsymbol{#1}}
%   block vector
\newcommand{\bvct}[1]{\mathbf{#1}}
%   block matrix
\newcommand{\bmtx}[1]{\mathbf{#1}}
%	inner products
\newcommand{\inner}[1]{\left<#1\right>}
\newcommand{\<}{\langle}
\renewcommand{\>}{\rangle}
% 	transpose, Hermitian, pseudo-inverse
\renewcommand{\H}{\mathrm{H}}
\newcommand{\T}{\mathrm{T}}
\newcommand{\pinv}{\dagger}
%	fundamental subspaces
\newcommand{\Null}{\operatorname{Null}}
\newcommand{\Range}{\operatorname{Range}}
\newcommand{\Span}{\operatorname{Span}}
%	operators
\newcommand{\trace}{\operatorname{trace}}
\newcommand{\conv}{\operatorname{conv}}
\newcommand{\rank}{\operatorname{rank}}
\newcommand{\diag}{\operatorname{diag}}
\newcommand{\toep}{\operatorname{Toep}}
\newcommand{\SDP}{\operatorname{SDP}}
\newcommand{\PSD}{\operatorname{PSD}}
\newcommand{\far}{\operatorname{far}}
\newcommand{\near}{\operatorname{near}}
\newcommand{\sign}{\operatorname{sign}}
\newcommand{\sgn}{\operatorname{sgn}}
\newcommand{\Sign}{\operatorname{Sgn}}
\newcommand{\dist}{\operatorname{dist}}
\def \lg        {\langle}
\def \rg        {\rangle}
\def \vec       {\operatorname*{vec}}
%
\newcommand{\dimension}{\operatorname{dim}}

% sets and topology
\newcommand{\set}[1]{\mathbb{#1}}
\newcommand{\closure}{\operatorname{cl}}	% closure
\newcommand{\interior}{\operatorname{int}}
\newcommand{\boundary}{\operatorname{bd}}
\newcommand{\diameter}{\operatorname{diam}}

% functional analysis
\newcommand{\domain}{\operatorname{dom}}
\newcommand{\epigraph}{\operatorname{epi}}
\newcommand{\hypograph}{\operatorname{hypo}}
\newcommand{\linop}[1]{\mathscr{#1}}	% general linear operator

% optimization
\DeclareMathOperator*{\minimize}{\text{minimize}}
\DeclareMathOperator*{\maximize}{\text{maximize}}
\DeclareMathOperator*{\argmin}{\text{arg~min}}
\DeclareMathOperator*{\argmax}{\text{arg~max}}
\def \st {\operatorname*{s.t.\ }}
% other
%\newcommand{\sgn}{\text{sgn}}
\newcommand{\sinc}{\text{sinc}}
\newcommand{\floor}[1]{\left\lfloor #1 \right\rfloor}
\newcommand{\ceil}[1]{\left\lceil #1 \right\rceil}
\newcommand{\eps}{\epsilon}


\newcommand{\wh}{\widehat}
\newcommand{\wc}{\widecheck}
\newcommand{\wt}{\widetilde}
\newcommand{\ol}{\overline}


\newcommand{\norm}[2]{\left\| #1 \right\|_{#2}}
\newcommand{\nrm}[1]{\left\Vert#1\right\Vert}
\newcommand{\abs}[1]{\left| #1 \right|}
\newcommand{\bracket}[1]{\left( #1 \right)}
\newcommand{\parans}[1]{\left(#1\right)}
\newcommand{\innerprod}[2]{\left\langle #1,  #2 \right\rangle}
\newcommand{\prob}[1]{\bb P\left[ #1 \right]}
\newcommand{\expect}[1]{\bb E\left[ #1 \right]}
\newcommand{\function}[2]{#1 \left(#2\right)}
\newcommand{\integral}[4]{\int_{#1}^{#2}\; #3\; #4}
\newcommand{\iter}[1]{^{(#1)}}


\newcommand{\js}[1]{{\color{Mahogany} [\noindent JS: #1]}}

%--------------------------------------------------------------------------
\newcommand{\calA}{\mathcal{A}}
\newcommand{\calB}{\mathcal{B}}
\newcommand{\calC}{\mathcal{C}}
\newcommand{\calD}{\mathcal{D}}
\newcommand{\calE}{\mathcal{E}}
\newcommand{\calF}{\mathcal{F}}
\newcommand{\calG}{\mathcal{G}}
\newcommand{\calH}{\mathcal{H}}
\newcommand{\calI}{\mathcal{I}}
\newcommand{\calJ}{\mathcal{J}}
\newcommand{\calK}{\mathcal{K}}
\newcommand{\calL}{\mathcal{L}}
\newcommand{\calM}{\mathcal{M}}
\newcommand{\calN}{\mathcal{N}}
\newcommand{\calO}{\mathcal{O}}
\newcommand{\calP}{\mathcal{P}}
\newcommand{\calQ}{\mathcal{Q}}
\newcommand{\calR}{\mathcal{R}}
\newcommand{\calS}{\mathcal{S}}
\newcommand{\calT}{\mathcal{T}}
\newcommand{\calU}{\mathcal{U}}
\newcommand{\calV}{\mathcal{V}}
\newcommand{\calW}{\mathcal{W}}
\newcommand{\calX}{\mathcal{X}}
\newcommand{\calY}{\mathcal{Y}}
\newcommand{\calZ}{\mathcal{Z}}

% \hypersetup{
%     colorlinks=true,%
%     citecolor=blue,%
%     filecolor=blue,%
%     linkcolor=blue,%
%     urlcolor=blue
% }

\newcommand{ \Brac }[1]{\left\lbrace #1 \right\rbrace}
\newcommand{ \brac }[1]{\left[ #1 \right]}
\newcommand{ \paren }[1]{ \left( #1 \right) }

\newcommand{\NC}{$\mc {NC}$}

\newcommand{\va}{\vct{a}}
\newcommand{\vb}{\vct{b}}
\newcommand{\vc}{\vct{c}}
\newcommand{\vd}{\vct{d}}
\newcommand{\ve}{\vct{e}}
\newcommand{\vf}{\vct{f}}
\newcommand{\vg}{\vct{g}}
\newcommand{\vh}{\vct{h}}
\newcommand{\vi}{\vct{i}}
\newcommand{\vj}{\vct{j}}
\newcommand{\vk}{\vct{k}}
\newcommand{\vl}{\vct{l}}
\newcommand{\vm}{\vct{m}}
\newcommand{\vn}{\vct{n}}
\newcommand{\vo}{\vct{o}}
\newcommand{\vp}{\vct{p}}
\newcommand{\vq}{\vct{q}}
\newcommand{\vr}{\vct{r}}
\newcommand{\vs}{\vct{s}}
\newcommand{\vt}{\vct{t}}
\newcommand{\vu}{\vct{u}}
\newcommand{\vv}{\vct{v}}
\newcommand{\vw}{\vct{w}}
\newcommand{\vx}{\vct{x}}
\newcommand{\vy}{\vct{y}}
\newcommand{\vz}{\vct{z}}
%
\newcommand{\valpha}{\vct{\alpha}}
\newcommand{\vbeta}{\vct{\beta}}
\newcommand{\vtheta}{\vct{\theta}}
\newcommand{\vepsilon}{\vct{\epsilon}}
\newcommand{\veps}{\vct{\epsilon}}
\newcommand{\vvarepsilon}{\vct{\varepsilon}}
\newcommand{\vgamma}{\vct{\gamma}}
\newcommand{\vlambda}{\vct{\lambda}}
\newcommand{\vnu}{\vct{\nu}}
\newcommand{\vphi}{\vct{\phi}}
\newcommand{\vpsi}{\vct{\psi}}
\newcommand{\vdelta}{\vct{\delta}}
\newcommand{\veta}{\vct{\eta}}
\newcommand{\vxi}{\vct{\xi}}
\newcommand{\vmu}{\vct{\mu}}
\newcommand{\vrho}{\vct{\rho}}
%
\newcommand{\vzero}{\vct{0}}
\newcommand{\vone}{\vct{1}}


\newcommand{\mA}{\mtx{A}}
\newcommand{\mB}{\mtx{B}}
\newcommand{\mC}{\mtx{C}}
\newcommand{\mD}{\mtx{D}}
\newcommand{\mE}{\mtx{E}}
\newcommand{\mF}{\mtx{F}}
\newcommand{\mG}{\mtx{G}}
\newcommand{\mH}{\mtx{H}}
\newcommand{\mJ}{\mtx{J}}
\newcommand{\mK}{\mtx{K}}
\newcommand{\mL}{\mtx{L}}
\newcommand{\mM}{\mtx{M}}
\newcommand{\mN}{\mtx{N}}
\newcommand{\mO}{\mtx{O}}
\newcommand{\mP}{\mtx{P}}
\newcommand{\mQ}{\mtx{Q}}
\newcommand{\mR}{\mtx{R}}
\newcommand{\mS}{\mtx{S}}
\newcommand{\mT}{\mtx{T}}
\newcommand{\mU}{\mtx{U}}
\newcommand{\mV}{\mtx{V}}
\newcommand{\mW}{\mtx{W}}
\newcommand{\mX}{\mtx{X}}
\newcommand{\mY}{\mtx{Y}}
\newcommand{\mZ}{\mtx{Z}}
%
\newcommand{\mGamma}{\mtx{\Gamma}}
\newcommand{\mDelta}{\mtx{\Delta}}
\newcommand{\mLambda}{\mtx{\Lambda}}
\newcommand{\mOmega}{\mtx{\Omega}}
\newcommand{\mPhi}{\mtx{\Phi}}
\newcommand{\mPsi}{\mtx{\Psi}}
\newcommand{\mSigma}{\mtx{\Sigma}}
\newcommand{\mTheta}{\mtx{\Theta}}
\newcommand{\mUpsilon}{\mtx{\Upsilon}}
%
\newcommand{\mId}{\bmtx{I}}
\newcommand{\mEx}{{\bf J}}
\newcommand{\mzero}{{\bf 0}}
\newcommand{\mone}{{\bf 1}}
%
\newcommand{\bva}{\bvct{a}}
\newcommand{\bvh}{\bvct{h}}
\newcommand{\bvw}{\bvct{w}}
\newcommand{\bvx}{\bvct{x}}
\newcommand{\bvy}{\bvct{y}}


\newcommand{\bmM}{\bmtx{M}}
\newcommand{\bmH}{\bmtx{H}}
\newcommand{\bmX}{\bmtx{X}}


\newcommand{\setA}{\set{A}}
\newcommand{\setB}{\set{B}}
\newcommand{\setC}{\set{C}}
\newcommand{\setD}{\set{D}}
\newcommand{\setE}{\set{E}}
\newcommand{\setF}{\set{F}}
\newcommand{\setG}{\set{G}}
\newcommand{\setH}{\set{H}}
\newcommand{\setI}{\set{I}}
\newcommand{\setJ}{\set{J}}
\newcommand{\setK}{\set{K}}
\newcommand{\setL}{\set{L}}
\newcommand{\setM}{\set{M}}
\newcommand{\setN}{\set{N}}
\newcommand{\setO}{\set{O}}
\newcommand{\setP}{\set{P}}
\newcommand{\setQ}{\set{Q}}
\newcommand{\setR}{\set{R}}
\newcommand{\setS}{\set{S}}
\newcommand{\setT}{\set{T}}
\newcommand{\setU}{\set{U}}
\newcommand{\setV}{\set{V}}
\newcommand{\setW}{\set{W}}
\newcommand{\setX}{\set{X}}
\newcommand{\setY}{\set{Y}}
\newcommand{\setZ}{\set{Z}}

\renewcommand{\ss}{{\vspace*{-1mm}}}

\setcounter{MaxMatrixCols}{20}

%\pagestyle{plain}

\graphicspath{{./figs/}}

\newlength{\imgwidth}
\setlength{\imgwidth}{3.125in}

\newcommand{\QW}[1]{\textcolor{red}{\bf [{\em QW:} #1]}}
\newcommand{\note}[1]{\textcolor{blue}{\bf [{\em Note:} #1]}}
\newcommand{\revise}[1]{\textcolor{blue}{#1}}
\newcommand{\edit}[1]{\textcolor{blue}{#1}}

\newboolean{twoColVersion}
\setboolean{twoColVersion}{false}
\newcommand{\twoCol}[2]{\ifthenelse{\boolean{twoColVersion}} {#1} {#2} }


\newcommand{\Poff}{\mathcal{P}_{\textup{off}}}
\newcommand{\Pon}{\mathcal{P}_{\textup{diag}}}

\newcommand{\cy}[1]{\textcolor{red}{\bf [{\em Can:} #1]}}
\newcommand{\qq}[1]{\textcolor{blue}{\bf [{\em Qing:} #1]}}
\newcommand{\pw}[1]{\textcolor{red}{\bf [{\em Peng:} #1]}}
\newcommand{\zz}[1]{\textcolor{blue}{ [{\em Zhihui:} #1]}}
\newcommand{\jz}[1]{\textcolor{red}{ [{\em Jinxin:} #1]}}
\newcommand{\Xiang}[1]{\textcolor{cyan}{\bf [{\em Xiang:} #1]}}

\newcommand*\samethanks[1][\value{footnote}]{\footnotemark[#1]}
\usepackage{authblk}
\usepackage{booktabs}
\usepackage{cleveref}
\usepackage{cite}
\usepackage{natbib}
\usepackage{mathtools}
\usepackage[toc, page]{appendix}
\usepackage{wrapfig}
\definecolor{c1}{HTML}{A7BEAE}
\definecolor{c2}{HTML}{B85042}

\def\MoG{\texttt{MoG}}
\def\MoLRG{\texttt{MoLRG}}
\def\CSNR{$\mathrm{CSNR}$}
\def\SWD{$\mathrm{SWD}$}

% \title{Understanding Diffusion-based Representation Learning via Low-Dimensional Modeling}
\title{Understanding Representation Dynamics of Diffusion Models via Low-Dimensional Modeling}

\author[$\Diamond$]{Xiao Li\thanks{The first two authors contributed equally to the work.}}
\author[$\Diamond$]{Zekai Zhang\samethanks}
\author[$\Diamond$]{Xiang Li}
\author[$\Diamond$]{Siyi Chen}
\author[$\dagger$]{Zhihui Zhu}
\author[$\Diamond$]{Peng Wang}
\author[$\Diamond$]{\\ Qing Qu}


\affil[$\Diamond$]{Department of Electrical Engineering and Computer Science, University of Michigan}
\affil[$\dagger$]{Department of Computer Science \& Engineering, Ohio State University}

\date{}

\date{\today}

\begin{document}

\maketitle


\begin{abstract}
This work addresses the critical question of why and when diffusion models, despite being designed for generative tasks, can excel at learning high-quality representations in a self-supervised manner. To address this, we develop a mathematical framework based on a low-dimensional data model and posterior estimation, revealing a fundamental trade-off between generation and representation quality near the final stage of image generation. Our analysis explains the unimodal representation dynamics across noise scales, mainly driven by the interplay between data denoising and class specification. Building on these insights, we propose an ensemble method that aggregates features across noise levels, significantly improving both clean performance and robustness under label noise. Extensive experiments on both synthetic and real-world datasets validate our findings.

\end{abstract}

\section{Introduction}\label{sec:intro}
\section{Introduction}


\begin{figure}[t]
\centering
\includegraphics[width=0.6\columnwidth]{figures/evaluation_desiderata_V5.pdf}
\vspace{-0.5cm}
\caption{\systemName is a platform for conducting realistic evaluations of code LLMs, collecting human preferences of coding models with real users, real tasks, and in realistic environments, aimed at addressing the limitations of existing evaluations.
}
\label{fig:motivation}
\end{figure}

\begin{figure*}[t]
\centering
\includegraphics[width=\textwidth]{figures/system_design_v2.png}
\caption{We introduce \systemName, a VSCode extension to collect human preferences of code directly in a developer's IDE. \systemName enables developers to use code completions from various models. The system comprises a) the interface in the user's IDE which presents paired completions to users (left), b) a sampling strategy that picks model pairs to reduce latency (right, top), and c) a prompting scheme that allows diverse LLMs to perform code completions with high fidelity.
Users can select between the top completion (green box) using \texttt{tab} or the bottom completion (blue box) using \texttt{shift+tab}.}
\label{fig:overview}
\end{figure*}

As model capabilities improve, large language models (LLMs) are increasingly integrated into user environments and workflows.
For example, software developers code with AI in integrated developer environments (IDEs)~\citep{peng2023impact}, doctors rely on notes generated through ambient listening~\citep{oberst2024science}, and lawyers consider case evidence identified by electronic discovery systems~\citep{yang2024beyond}.
Increasing deployment of models in productivity tools demands evaluation that more closely reflects real-world circumstances~\citep{hutchinson2022evaluation, saxon2024benchmarks, kapoor2024ai}.
While newer benchmarks and live platforms incorporate human feedback to capture real-world usage, they almost exclusively focus on evaluating LLMs in chat conversations~\citep{zheng2023judging,dubois2023alpacafarm,chiang2024chatbot, kirk2024the}.
Model evaluation must move beyond chat-based interactions and into specialized user environments.



 

In this work, we focus on evaluating LLM-based coding assistants. 
Despite the popularity of these tools---millions of developers use Github Copilot~\citep{Copilot}---existing
evaluations of the coding capabilities of new models exhibit multiple limitations (Figure~\ref{fig:motivation}, bottom).
Traditional ML benchmarks evaluate LLM capabilities by measuring how well a model can complete static, interview-style coding tasks~\citep{chen2021evaluating,austin2021program,jain2024livecodebench, white2024livebench} and lack \emph{real users}. 
User studies recruit real users to evaluate the effectiveness of LLMs as coding assistants, but are often limited to simple programming tasks as opposed to \emph{real tasks}~\citep{vaithilingam2022expectation,ross2023programmer, mozannar2024realhumaneval}.
Recent efforts to collect human feedback such as Chatbot Arena~\citep{chiang2024chatbot} are still removed from a \emph{realistic environment}, resulting in users and data that deviate from typical software development processes.
We introduce \systemName to address these limitations (Figure~\ref{fig:motivation}, top), and we describe our three main contributions below.


\textbf{We deploy \systemName in-the-wild to collect human preferences on code.} 
\systemName is a Visual Studio Code extension, collecting preferences directly in a developer's IDE within their actual workflow (Figure~\ref{fig:overview}).
\systemName provides developers with code completions, akin to the type of support provided by Github Copilot~\citep{Copilot}. 
Over the past 3 months, \systemName has served over~\completions suggestions from 10 state-of-the-art LLMs, 
gathering \sampleCount~votes from \userCount~users.
To collect user preferences,
\systemName presents a novel interface that shows users paired code completions from two different LLMs, which are determined based on a sampling strategy that aims to 
mitigate latency while preserving coverage across model comparisons.
Additionally, we devise a prompting scheme that allows a diverse set of models to perform code completions with high fidelity.
See Section~\ref{sec:system} and Section~\ref{sec:deployment} for details about system design and deployment respectively.



\textbf{We construct a leaderboard of user preferences and find notable differences from existing static benchmarks and human preference leaderboards.}
In general, we observe that smaller models seem to overperform in static benchmarks compared to our leaderboard, while performance among larger models is mixed (Section~\ref{sec:leaderboard_calculation}).
We attribute these differences to the fact that \systemName is exposed to users and tasks that differ drastically from code evaluations in the past. 
Our data spans 103 programming languages and 24 natural languages as well as a variety of real-world applications and code structures, while static benchmarks tend to focus on a specific programming and natural language and task (e.g. coding competition problems).
Additionally, while all of \systemName interactions contain code contexts and the majority involve infilling tasks, a much smaller fraction of Chatbot Arena's coding tasks contain code context, with infilling tasks appearing even more rarely. 
We analyze our data in depth in Section~\ref{subsec:comparison}.



\textbf{We derive new insights into user preferences of code by analyzing \systemName's diverse and distinct data distribution.}
We compare user preferences across different stratifications of input data (e.g., common versus rare languages) and observe which affect observed preferences most (Section~\ref{sec:analysis}).
For example, while user preferences stay relatively consistent across various programming languages, they differ drastically between different task categories (e.g. frontend/backend versus algorithm design).
We also observe variations in user preference due to different features related to code structure 
(e.g., context length and completion patterns).
We open-source \systemName and release a curated subset of code contexts.
Altogether, our results highlight the necessity of model evaluation in realistic and domain-specific settings.






\section{Problem Setup}\label{sec:problem}
\section{Problem Formulation} \label{sec:probdef}

This section formally defines the problem of restoring a given pruned network with only using its original pretrained CNN in a way free of data and fine-tuning.



% Unlike many existing works utilize data for identifying unimportant filters as well as fine-tuning to this end, we cannot evaluate the filter importance by data-dependent values like activation maps (\textit{a.k.a.} channels) as our focus in this paper is not to use any training data. Thus, in our problem setting, we can only exploit the values of filters in the original network, and thereby have to make some changes in the remaining filters of the pruned network so that the network can return the output not too much different from the original one.

% No matter how much we carefully select unimportant filters to be pruned, some kinds of retraining process appears inevitable as done by the most existing works to this end. However, since our focus in this paper is not to use any training data, we cannot evaluate the importance of filters by data-dependent values like activation maps (\textit{a.k.a.} channels). 

% To this end, they not only use a careful criterion (\textit{e.g.}, L1-norm), but also fine-tune the network using the original data.
% Most of filter pruning methods try to select filters to be pruned prudently so that pruned network's output be similar to the original network's. To this end, they prune the unimportant filters and then fine-tune the pruned network with using the train data. 

% How can we restore the the pruned networks without any data? In other words, it implies that we cannot use any data-driven values(i.e., activation maps) and we can only exploit the values of original filters. In that case, the only thing we can do maybe changing the weights of remained filters appropriately not to amplify the difference between pruned and unpruned network's outputs through the information of original filters.

\begin{figure*}[t]
	\centering
    \subfigure[\label{fig:matrix:a}Pruning matrix]{\hspace{6mm}\includegraphics[width=0.35\columnwidth]{./figure/LBYL_figure_2_1.pdf}\hspace{6mm}} 
    \subfigure[\label{fig:matrix:b}Delivery matrix for LBYL]{\hspace{6mm}\includegraphics[width=0.35\columnwidth]{./figure/LBYL_figure_2_2.pdf}\hspace{6mm}}
    \subfigure[\label{fig:matrix:c}Delivery matrix for one-to-one]{\hspace{9mm}\includegraphics[width=0.35\columnwidth]{./figure/LBYL_figure_2_3.pdf}\hspace{9mm}} 
    \caption{Comparison between pruning matrix and delivery matrix, where the $4$-th and $6$-th filters are being pruned among $6$ original filters}
	\label{fig:matrix}
	\vspace{-2mm}
\end{figure*}



\subsection{Filter Pruning in a CNN}
Consider a given CNN to be pruned with $L$ layers, where each $\ell$-th layer starts with a convolution operation on its input channels, which are the output of the previous $(\ell-1)$-th layer $\mathbf{A}^{(\ell-1)}$, with the group of convolution filters $\mathbf{W}^{{(\ell)}}$ and thereby obtain the set of \textit{feature maps} $\mathbf{Z}^{(\ell)}$ as follows:
\begin{equation}
\boldsymbol{\mathbf{Z}}^{(\ell)} = {\mathbf{A}^{(\ell-1)} \circledast {\mathbf{W}}^{(\ell)}},
\nonumber
\end{equation}
where $\circledast$ represents the convolution operation. Then, this convolution process is normally followed by a batch normalization (BN) process and an activation function such as ReLU, and the $\ell$-th layer finally outputs an \textit{activation map} $\mathbf{A}^{(\ell)}$ to be sent to the $(\ell+1)$-th layer through this sequence of procedures as:
\begin{equation}
\mathbf{A}^{(\ell)} = \F(\N(\mathbf{Z}^{(\ell)})),
\nonumber
\end{equation}
where $\F(\cdot)$ is an activation function and $\N(\cdot)$ is a BN procedure.

Note that all of $\mathbf{W}^{(\ell)}$, $\mathbf{Z}^{(\ell)}$, and $\mathbf{A}^{(\ell)}$ are tensors such that: $\mathbf{W}^{(\ell)} \in \mathbb{R}^{m \times n \times k \times k}$ and $\mathbf{Z}^{(\ell)},\mathbf{A}^{(\ell)} \in \mathbb{R}^{m \times w \times h}$, where (1) $m$ is the number of filters, which also equals the number of output activation maps, (2) $n$ is the number of input activation maps resulting from the $(\ell-1)$-th layer, (3) $k \times k$ is the size of each filter, and (4) $w \times h$ is the size of each output channel for the $\ell$-th layer.

\smalltitle{Filter pruning as n-mode product}
When filter pruning is performed at the $\ell$-th layer, all three tensors above are consequently modified to their \textit{damaged} versions, namely $\mathbf{\Tilde{W}}^{(\ell)}$, $\mathbf{\Tilde{Z}}^{(\ell)}$, and $\mathbf{\Tilde{A}}^{(\ell)}$, respectively, in a way that: $\mathbf{\Tilde{W}}^{(\ell)} \in \mathbb{R}^{t \times n \times k \times k}$ and $\mathbf{\Tilde{Z}}^{(\ell)},\mathbf{\Tilde{A}}^{(\ell)} \in \mathbb{R}^{t \times w \times h}$, where $t$ is the number of remaining filters after pruning and therefore $t < m$. Mathematically, the tensor of remaining filters, \textit{i.e.}, $\mathbf{\Tilde{W}}^{(\ell)}$, is obtained by the \textit{$1$-mode product} \cite{DBLP:journals/siamrev/KoldaB09} of the tensor of the original filters $\mathbf{W}^{(\ell)}$ with a \textit{pruning matrix} $\boldsymbol{\S} \in \mathbb{R}^{m \times t}$ (see Figure \ref{fig:matrix:a})
as follows:
\begin{eqnarray}\begin{split}\label{eq:pruning}
\mathbf{\Tilde{W}}^{(\ell)} = {\mathbf{W}}^{(\ell)} \times_{1} {\boldsymbol{\S}}^{T},\text{where }\boldsymbol{\S}_{i,k} = 
  \begin{cases} 
   1~ \text{if } i = i'_k \\
   0~ \text{otherwise}
  \end{cases} \\
  \text{s.t. } i, i'_k \in [1, m] 
  \text{ and } k \in [1, t].
  \end{split}
\end{eqnarray}
  
By Eq. (\ref{eq:pruning}), each $i'_k$-th filter is not pruned and the other $(m-t)$ filters are completely removed from $\mathbf{W}^{(\ell)}$ to be $\mathbf{\Tilde{W}}^{(\ell)}$.

This reduction at the $\ell$-th layer causes another reduction for each filter of the $(\ell+1)$-th layer so that $\mathbf{W}^{(\ell+1)}$ is now modified to $\mathbf{\Tilde{W}}^{(\ell+1)} \in \mathbb{R}^{m' \times t \times k' \times k'}$, where $m'$ is the number of filters of size $k' \times k'$ in the $(\ell+1)$-th layer. Due to this series of information losses, the resulting feature map (\textit{i.e.}, $\mathbf{Z}^{(\ell+1)}$) would severely be damaged to be $\mathbf{\Tilde{Z}}^{(\ell+1)}$ as shown below:
\begin{equation}
{\mathbf{\Tilde{Z}}}^{{(\ell+1)}} = \mathbf{\Tilde{A}}^{(\ell)} \circledast {\mathbf{\Tilde{W}}}^{(\ell+1)}~~~\not\approx~~~\mathbf{Z}^{(\ell+1)}
\label{eq:eq}\nonumber
\end{equation}
The shape of $\mathbf{\Tilde{Z}}^{(\ell+1)}$ remains the same unless we also prune filters for the $(\ell+1)$-th layer. If we do so as well, the loss of information will be accumulated and further propagated to the next layers. Note that $\mathbf{\Tilde{W}}^{(\ell+1)}$ can also be represented by the \textit{$2$-mode product} \cite{DBLP:journals/siamrev/KoldaB09} of $\mathbf{W}^{(\ell+1)}$ with the transpose of the same matrix $\boldsymbol{\S}$ as:
\begin{equation} \label{eq:pruning2}
\mathbf{\Tilde{W}}^{(\ell+1)} = {\mathbf{W}}^{(\ell+1)} \times_{2} {\boldsymbol{\S}^T}
\end{equation}




\subsection{Problem of Restoring a Pruned Network without Data and Fine-Tuning}
As mentioned earlier, our goal is to restore a pruned and thus damaged CNN without using any data and re-training process, which implies the following two facts. First, we have to use a pruning criterion exploiting only the values of filters themselves such as L1-norm. In this sense, this paper does not focus on proposing a sophisticated pruning criterion but intends to recover a network somehow pruned by such a simple criterion. Secondly, since we cannot make appropriate changes in the remaining filters by fine-tuning, we should make the best use of the original network and identify how the information carried by a pruned filter can be delivered to the remaining filters.

% For brevity, we formulate our problem here with respect to a specific layer, say $\ell$, and then it can trivially be generalized for the entire network. 
\smalltitle{Delivery matrix}
In order to represent the information to be delivered to the preserved filters, let us first think of what the pruning matrix $\boldsymbol{\S}$ means. As defined in Eq. (\ref{eq:pruning}) and shown in Figure \ref{fig:matrix:a}, each row is either a zero vector (for filters being pruned) or a one-hot vector (for remaining filters), which is intended only to remove filters without delivering any information. Intuitively, we can transform this pruning matrix into a \textit{delivery matrix} that carries information for filters being pruned by replacing some meaningful values with some of the zero values therein. Once we find such an \textit{ideal} $\boldsymbol{\S^*}$, we can plug it into $\boldsymbol{\S}$ of Eq. (\ref{eq:pruning2}) to deliver missing information propagated from the $\ell$-th layer to the filters at the $(\ell+1)$-th layer, which will hopefully generate an approximation $\mathbf{\hat{Z}}^{(\ell+1)}$ close to the original feature map as follows:
\begin{equation} \label{eq:fmap_approx}
{\mathbf{\hat{Z}}}^{{(\ell+1)}} = {\mathbf{\Tilde{A}}^{(\ell)} \circledast ({\mathbf{W}}^{(\ell+1)} \times_{2} {\boldsymbol{\S^*}^T})}
~~~\approx~~~\mathbf{Z}^{(\ell+1)}
\end{equation}
Thus, using the delivery matrix $\boldsymbol{\mathcal{S^*}}$, the information loss caused by pruning at each layer is recovered at the feature map of the next layer.

\smalltitle{Problem statement}
Given a pretrained CNN, our problem aims to find the best delivery matrix $\boldsymbol{\mathcal{S^*}}$ for each layer without any data and training process such that the following \textit{reconstruction error} is minimized:
\begin{equation}
\sum\limits_{i = 1}^{m'}\|{{\mathbf{Z}}_{i}^{{(\ell+1)}}-{\hat{\mathbf{Z}}}_{i}^{{(\ell+1)}}}\|_1,
\label{eq:goal}
\end{equation}
where ${\mathbf{Z}}_i^{{(\ell+1)}}$ and ${\hat{\mathbf{Z}}}_i^{{(\ell+1)}}$ indicate the $i$-th original feature map and its corresponding approximation, respectively, out of $m'$ filters in the $(\ell+1)$-th layer. Note that what is challenging here is that we cannot obtain the activation maps in $\mathbf{A}^{(\ell)}$ and $\mathbf{\Tilde{A}}^{(\ell)}$ without data as they are data-dependent values.

% = \sum\limits_{i = 1}^{m'}\|{{\mathbf{Z}}_{i}^{{(\ell+1)}}-{\mathbf{\Tilde{A}}^{(\ell)} \circledast ({\mathbf{W}}^{(\ell+1)} \times_{2} {\boldsymbol{\mathcal{S^*}^T}})}}\|_{1}


% Our goal is finding the approximation matrix $\boldsymbol{\mathcal{S}}$ to minimize the reconstruction error between the pruned model and the original model without any data, and effectively deliver missing information for pruned filters using this approximation matrix


% $\testit{s}$,which can be represented as below.

% \begin{equation}
% \boldsymbol{\mathcal{S}} =  \underset{{\boldsymbol{\mathcal{S}}}}{\mathrm{argmin}} \sum\limits_{{i} = 1}^{m_{\ell+1}} \|{{\mathbf{Z}}_{i,:,:}^{{(\ell+1)}}-{\hat{\mathbf{Z}}}_{i,:,:}^{{(\ell+1)}}}\|_{1} 
% \label{eq:eq1}
% \end{equation}



% Let us first recall that the ultimate goal of network pruning is to make the output of a pruned network as close as possible to that of its original network. Unlike many existing pruning methods, our focus is not to use any training data at all for the entire pruning and recovery process, and this implies the following two facts. First, we cannot evaluate the filter importance by data-dependent values like activation values or gradients, but have to use a pruning criterion exploiting only the values of filters themselves such as L1-norm. Furthermore, instead of fine-tuning with data, the only thing we can do for the pruned network is to make appropriate changes in the remaining filters by identifying some relationships between pruned filters and the other preserved ones without any support from data. Based on this intuition, this section mathematically and generally defines the problem of restoring a pruned neural network in a manner free of data and fine-tuning.


% Thus, we make approximation matrix $\testit{s}$ $\in$ $\mathbb{R}^{m_{\ell} \times t_{\ell}}$ with relationship between the pruned filter and preserved filters in $\ell$-th layer and then apply it to the original filters in $(\ell+1)$-th layer to compensate for pruned feature maps $\boldsymbol{\hat{\mathbf{Z}}}^{{(\ell+1)}}$ as shown below.
% (\textit{i.e.}, Let $\hat{\mathbf{W}}^{(\ell+1)}$ be ${\mathbf{W}}^{(\ell+1)}$ $\times_2$ ${{\textit{s}}} $, where $\times_2$ is 2-mode matrix product) 

% \begin{equation}
% \mathbf{Z}^{(\ell+1)} = {\mathbf{A}}^{(\ell)} \circledast {\mathbf{W}}^{(\ell+1)}
% \approx {\hat{\mathbf{A}}^{(\ell)} \circledast ({\mathbf{W}}^{(\ell+1)} \times_{2} {{s}}) = {\hat{\mathbf{Z}}}^{{(\ell+1)}}}
% \label{eq:eq}\nonumber
% \end{equation}




% For a Convolutional Neural Network (CNN) with $L$ layers, we denote $\mathcal{A}{^{(\ell-1)}}$ $\in$ $\mathbb{R}^{n_{\ell -1 } \times h_{\ell -1} \times w_{\ell -1}}$ is activation maps at $\ell-1$-th layer, where $n_{\ell -1}$, $h_{\ell -1}$ and $w_{\ell -1}$ are the number of channels, height and width in activation maps, respectively. and we denote $\mathbf{W}^{{(\ell )}}$ $\in$  $\mathbb{R}^{m_{\ell} \times n_{\ell -1}\times k \times k}$ is covolution filters in $\ell$-th layer,where $m_{\ell}$, $n_{\ell-1}$ and $k$ are the number of filters, number of channels and kernel size, respectively. Trough the convolution operation using activation map $\mathcal{A}{^{(\ell-1)}}$ and convolution filter $\mathbf{W}^{{(\ell)}}$ in $\ell$-th layer, the feature maps $\boldsymbol{\mathbf{Z}}^{{(\ell)}}$ $\in$ $\mathbb{R}^{m_{\ell} \times h_{\ell+1} \times w_{\ell+1}}$ is computed as shown as below.


% \begin{equation}
% \boldsymbol{\mathbf{Z}}^{(\ell)} = {\mathcal{A}^{(\ell-1)} \circledast {\mathbf{W}}^{(\ell)}}
% \label{eq:eq1}\nonumber
% \end{equation}
% where $\circledast$ is convolution operation.

% and the feature maps passed through the BN and ReLU layer are activation maps $\mathcal{A}{^{(\ell)}}$ $\in$ $\mathbb{R}^{m_{{\ell}} \times h_{\ell+1} \times w_{\ell+1}} $ in $\ell$-th layer as shown as below.

% \begin{equation}
% \mathcal{A}^{(\ell)} = \mathcal{F}(\mathbf{Z}^{(\ell)} \circledast {\mathbf{W}}^{(\ell)})
% \label{eq:eq2}\nonumber
% \end{equation}
% where $\mathcal{F}$ is the function that implement batch normalization and non-linear activation(\textit{e.g.}, ReLU).

% \smalltitle{Filter Pruning}
% If the filter pruning is performed in $\ell$-th layer, the shape of original filters $\mathbf{W}^{{(\ell)}}$ $\in$ $\mathbb{R}^{m_{\ell} \times n_{\ell-1}\times k \times k}$ is modified to ${\hat {\mathbf{W}}^{(\ell)}}$ $\in$ $\mathbb{R}^{t_{\ell} \times n_{\ell-1}\times k \times k}$, where $t_{\ell}$ $<$ $m_{\ell}$ by pruning criterion. Therefore, the pruned activation maps ${\hat {\mathcal{A}}}{^{({\ell+1})}}$ $\in$ $\mathbb{R}^{t_{{\ell}} \times h_{{\ell+2}} \times w_{{\ell+2}}}$ in (${\ell+1}$)-th layer is computed as below.

% \begin{equation}
% \mathbf{\hat{A}}^{(l+1)} = \mathcal{F}({\mathbf{A}^{(\ell)} \circledast {\mathbf{\hat{W}}}^{(\ell+1)}})
% \label{eq:eq3}\nonumber
% \end{equation}

% Moreover, corresponding channels of each filters in ($\ell +1$)-th layer are sequentially removed. As a result, shape of original filters $\mathbf{W}^{{(\ell+1)}}$ $\in$ $\mathbb{R}^{m_{\ell+1} \times m_{\ell}\times k \times k}$ in ($\ell+1$)-th layer is changed to  ${\hat {\mathbf{W}}^{(\ell+1)}}$ $\in$ $\mathbb{R}^{m_{\ell+1} \times t_{\ell}\times k \times k}$. Although feature maps ${\hat{\mathbf{Z}}}^{{(\ell+1)}}$ $\in$ $\mathbb{R}^{m_{\ell+1} \times h_{\ell+2} \times w_{\ell+2}}$ in ($\ell+1$)-th layer after pruning have same shape with original feature maps ${\mathbf{Z}}^{{(\ell+1)}}$ $\in$ $\mathbb{R}^{m_{\ell+1} \times h_{\ell+2} \times w_{\ell+2}}$, the pruned feature maps $\boldsymbol{\hat{\mathbf{Z}}}^{{(\ell+1)}}$ are damaged.

\section{Study of Representation Dynamics}\label{sec:main}
% In \Cref{sec:main}, we analyzed diffusion representation dynamics with a focus on the denoising process, assuming sufficient training data for learning the underlying distribution. In this section, we explore the impact of the diffusion process (\Cref{subsec:weight_share}) and data complexity (\Cref{subsec:mem_gen}) in shaping diffusion models' representation learning dynamics.




With the setup in \Cref{sec:problem}, this section theoretically investigates the representation dynamics of diffusion models across the noise levels, providing new insights for understanding the representation-generation tradeoff. Moreover, our theoretical studies are corroborated by experimental results on real datasets.

%{\color{blue} In this section, we investigate the representation dynamics of diffusion models, focusing on the intriguing unimodal trend in their representation learning performance across varying noise levels. To uncover the underlying reasons for this behavior, we conduct a theoretical analysis of the posterior estimation quality, $\E[\bm{x}_0 \mid \bm{x}_t]$, in low-dimensional distributions.}





% \vspace{-0.2in}
\subsection{Assumptions of Low-Dimensional Data Distribution}\label{subsec:model}

% \ZK{Refer to latent diffusion? The latent space seems more likely to have a low-dimensional subsuapce structure.}

In this work, we assume that the input data follows a noisy version of the mixture of low-rank Gaussians (\MoLRG) distribution \citep{wang2024diffusion,elhamifar2013sparse, wang2022convergence}, defined as follows.


%\zk{K-subspace or K-class? Need consistency}
\begin{assumption}[$K$-Class Noisy \MoLRG~Distribution]\label{assum:subspace}
\emph{For any sample $\mb x_0$ drawn from the noisy \MoLRG~distribution with $K$ subspaces, the following holds: }
% \qq{edit the equation}
\begin{align}\label{eq:MoG noise}
    \bm x_0 = \bm U_k \bm a + \delta \bm U_k^{\perp} \bm e,\;\text{with prob.}\;\pi_k \geq 0,\; k \in [K].
\end{align}
\emph{Here, $k$ represents the class of $\bm x_0$ and follows a multinomial distribution $k \sim \text{Mult}(K,\pi_k)$, $\bm U_k \in \mathcal{O}^{n \times d_k}$ denotes an orthonormal basis for the $k$-th subspace with its complement $\mb U_k^\perp \in \mathcal{O}^{n \times (n-d_k)}$, $d_k$ is the subspace dimension with $d_k \ll n$, and the coefficient $\bm a \overset{i.i.d.}{\sim} \mathcal{N}(\bm 0, \bm I_{d_k})$ is drawn from the normal distribution. The level of the noise $\bm e \overset{i.i.d.}{\sim} \mathcal{N}(\bm 0, \bm I_{n-d_k})$ is controlled by the scalar $\delta < 1$. }
% $\sum_{k=1}^K \pi_k=1$%Additionally, $\mb U_k^\perp \in \mathcal{O}^{n \times (n-d_k)}$ is the orthogonal complement of $\mb U_k$.
%Here, $\sum_{k=1}^K \pi_k=1$, $\bm U_k \in \mathcal{O}^{n \times d_k}$ denotes an orthonormal basis for the $k$-th subspace, $d_k$ is the subspace dimension with $d_k \ll n$, and the coefficient $\bm a \overset{i.i.d.}{\sim} \mathcal{N}(\bm 0, \bm I_{d_k})$ is drawn from a standard normal distribution \qq{normal means standard Gaussian, remove standard}. For the noise, we assume $\bm e \overset{i.i.d.}{\sim} \mathcal{N}(\bm 0, \bm I_{n-d_k})$ with magnitude controlled by the scalar $\delta < 1$. Additionally, $\mb U_k^\perp \in \mathcal{O}^{n \times (n-d_k)}$ is the orthogonal compliment of $\mb U_k$. 
\end{assumption}
As shown in \Cref{fig:sample}, data from \MoLRG~resides on a union of low-dimensional subspaces, each following a Gaussian distribution with a low-rank covariance matrix representing its basis. The study of Noisy \MoLRG\; distributions is further motivated by the fact that

%The data drawn from \MoLRG~lie on a union of low-dimensional subspaces. Within each subspace, the data follows a Gaussian distribution with a low-rank covariance matrix that represents the subspace basis. Moreover, the study of the Noisy \MoLRG\; distributions is motivated by the facts that
\begin{itemize}[leftmargin=*]
    \vspace{-0.1in}
    \item \emph{\MoLRG\;captures the intrinsic low-dimensionality of image data.} Although real-world image datasets are high-dimensional in terms of pixel count and data volume, extensive empirical studies \citep{gong2019intrinsic,pope2021intrinsic,stanczuk2022your} demonstrated that their intrinsic dimensionality is considerably lower. The \MoLRG~distribution models data in a low-dimensional space with rank $d_k \ll n$ effectively captures this property.
    % \qq{add explanation why MolRG captures the low-dimensionality here}
    \item \emph{The latent space of latent diffusion models is approximately Gaussian.} State-of-the-art large-scale diffusion models \citep{peebles2023scalable, podell2023sdxl} typically employ autoencoders \citep{kingma2013auto} to project images into a low-dimensional latent space, where a KL penalty encourages the learned latent distribution to approximate standard Gaussians \citep{rombach2022high}. Furthermore, recent studies \citep{jing2022subspace, chen2024deconstructing} show that diffusion models can be trained to leverage the intrinsic subspace structure of real-world data.
    % \qq{add: In their training loss, Gaussianity is typically enforced for training the encoder.}\xiao{this is not always true, better be integrated with the previous point}
    
    % \item \emph{Modeling the visual details of real-world image datasets.}\zk{I will shrink} The noise term $\delta \bm U_k^{\perp} \bm e_i$ captures perturbations outside the $k$-th subspace via the orthogonal complement $\bm U_k^{\perp}$, aligning the model with real-world scenarios. These perturbations represent attributes irrelevant to the subspace, such as the background in a bird image or the color and texture of a car. While this additional noise term may be less significant for representation learning, it plays a crucial role in enabling diffusion models to generate high-fidelity samples.
    \item \emph{Modeling the complexity of real-world image datasets.} The noise term $\delta \bm U_k^{\perp} \bm e_i$ captures perturbations outside the $k$-th subspace via the orthogonal complement $\bm U_k^{\perp}$, analogous to insignificant attributes of real-world images, such as the background of an image. While this additional noise term may be less significant for representation learning, it plays a crucial role in enhancing the fidelity of generated samples. %\qq{I feel here there is a bit overstatement. Fine-details are of high frequency, but noise is not. We need to be careful, not mention visual details.}
    % \zk{we can mention the connections with Difan Zou's data assumptions here}
\end{itemize}

%as defined in \Cref{assum:subspace}, motivated by the fact that  Although real-world image datasets are high-dimensional in terms of pixel count and data volume, extensive empirical studies \citep{gong2019intrinsic,pope2021intrinsic,stanczuk2022your} suggest that their intrinsic dimensionality is considerably lower. Moreover, state-of-the-art large-scale diffusion models \citep{peebles2023scalable, podell2023sdxl} commonly employ auto-encoders \citep{kingma2013auto} to map images to a low-dimensional latent space \citep{rombach2022high} for better training efficiency. Consequently, image datasets often reside on a union of low-dimensional manifolds.

%,kamkari2024geometric
%Therefore, there are also many recent studies on how diffusion models learn low-dimensional distributions \citep{wang2024diffusion} \qq{cite other related works from Yuxin Chen, Gen Li, and NYU people, Minshuo Chen}. Moreover, as union of low-dimensional manifolds can be locally approximated by a union of linear subspaces, it motivates us to model the underlying data distribution as a mixture of low-rank Gaussians (\MoLRG) \citep{wang2024diffusion}. The data points generated by \MoLRG~lie on a union of subspaces. Within each subspace, the data follows a Gaussian distribution with a low-rank covariance matrix that represents the subspace basis. Formally, we introduce a noisy version of the \MoLRG~distribution as follows: 

%Numerous studies have shown that real-world datasets, even those as large as ImageNet \citep{russakovsky2015imagenet}, possess low intrinsic dimensionalities \citep{pope2021intrinsic,stanczuk2022your,choi2020stargan}. 

%In light of this, we use the Mixture of Low-Rank Gaussian (\MoLRG) distribution where the data points are generated from a mixture of Gaussian distributions with zero means and low-rank covariance matrices to model the underlying data distribution.

%($\sigma_t$)\ZK{time step with noise scale ($\sigma_t$)}


% \xiao{Add an experiment here to show MoLRG with different delta.}

%Moreover, the Noisy \MoLRG\; is amendable for analysis (see \Cref{lem:E[x_0]_multi}).
%For simplicity of analysis, we let $d_1 = \dots = d_K = d$, and we assume that the basis $\Brac{\mb U_k}$ are orthogonal to each other with $\bm U_k^{T} \bm U_l = \bm 0$ for all $k \neq l$. Additionally, we assume all mixing weights $\Brac{\pi_k}$ are equal with $\pi_1 = \dots = \pi_K = 1/K$, and we define $\mb U_{\perp} = \bigcap_{k=1}^K \bm U_k^{\perp} \in \mathcal{O}^{n \times (n-Kd)}$ to be the noise space that is the orthogonal complement to all basis $\Brac{\mb U_k}_{k=1}^K$. As such, we can derive the ground truth posterior mean $\E\left[\hat{\bm x_0}\vert \bm x_t\right]$ of the noisy \MoLRG~distribution as: 

Moreover, the noisy \MoLRG\; is analytically tractable. For simplicity, we assume equal subspace dimensions ($d_1 = \dots = d_K = d$), orthogonal bases ($\bm U_k^{T} \bm U_l = \bm 0$ for $k \neq l$), and uniform mixing weights ($\pi_1 = \dots = \pi_K = 1/K$). Then, we can derive the ground truth posterior mean $\E\left[\bm x_0 \vert \bm x_t\right]$ for the noisy \MoLRG\; distribution as:

%can be interpreted as class-irrelevant attributes in images (e.g., the background of a bird, the color/texture of a car). These attributes, though unrelated to the primary class, are crucial for the diffusion model to learn in order to generate high-fidelity samples.

%We also assume $\text{span}\left(\bm U_1, \dots, \bm U_K, \bm U_{\perp}\right) = \bm I^{n}$, where $\bm U_{\perp}^T \bm U_l = \bm 0$ for all $l \in [K]$, and $\bm U_k^{\perp} = \begin{bmatrix}
%    \bm U_1 & \dots & \bm U_{k-1} & \bm U_{k+1} & \dots & \bm U_K & \bm U_{\perp}
%\end{bmatrix} \in \mathcal{O}^{n \times D}$ with $D = n - d$ denotes the orthonormal basis of the noise component (i.e., $\bm U_k^{\perp T} \bm U_k = \bm 0$). 

% $\bm U_k^{\perp} = \text{span}\left(\bigcup_{l \neq k} \bm U_l \cup \bm U_{\perp}\right) \in \mathcal{O}^{n \times D}$

% \ZK{We note that our data model can closely resemble the real data distribution, whereas the noise $b \bm U_k^{\perp} \bm e_i$ can be interpreted as the classification-irrelevant attributes of a picture (i.e. the background of a bird picture, the color of a car, etc), which can be important for the model to learn to generate high-fidelity samples.}
% The training objective of diffusion models, as defined in \eqref{eq:dae_loss}, seeks to minimize the denoising loss across all time steps simultaneously using a single model. The success of optimization hence inherently assumes that the difficulty of optimizing the denoising loss remains consistent across varying noise scales. In this section, through the use of simplified models, we demonstrate that this assumption is invalid and the existence of an optimization deficiency in diffusion models, which becomes more pronounced as noise scales increase.

% \begin{figure}[t]
%     \begin{center}
%     \begin{subfigure}{0.31\textwidth}
%     \includegraphics[width = 0.955\textwidth]{figs/single_linear_learned.pdf}
%     \caption{Learned \emph{NDS}} 
%     \end{subfigure} \quad %\hspace*{\fill}
%     \begin{subfigure}{0.31\textwidth}
%     \includegraphics[width = 0.955\textwidth]{figs/single_linear_compare.pdf}
%     \caption{\emph{NDS} decomposition} 
%     \end{subfigure}
%     \begin{subfigure}{0.31\textwidth}
%     \includegraphics[width = 0.975\textwidth]{figs/single_linear_norm.pdf}
%     \caption{Norm of solutions} 
%     \end{subfigure}
%     \end{center}
% \caption{\textbf{Dynamics of representation learning across noise scales in the single Gaussian case.}}
% \label{fig:single_linear}
% \end{figure}

% \begin{figure}[h]
%     \centering
%     \includegraphics[width=0.4\linewidth]{figs/optimal_MSE_MoG.png}
%     \caption{NDS in the MoLRG case(class=3)}
%     \label{fig:ucurve molrg dm}
% \end{figure}

% \xiao{perhaps move the above parts to Section 2.}

% \subsubsection{Representation dynamics within the single Gaussian setting}\label{subsubsec:single_gauss}

% \begin{prop}\label{lem:E[x_0]_single}
%     Given the setup described above, if we set $K=1$ (i.e., the single low-rank Gaussian case), then for each time $t > 0$, we have:
%     \begin{align}\label{eq:E_sinG}
%         \E\left[ \bm x_0\vert \bm x_t\right] = \left(\frac{1}{1+\sigma_t^2}\bm U\bm U^T + \frac{\delta^2}{\delta^2+\sigma_t^2}\bm U_\perp\bm U_\perp^T \right)\bm x_t
%     \end{align}
% \end{prop}

% This result \footnote{Due to the space limitation, we postpone all proofs to the Appendix.} shows that the ground truth posterior mean is a combination of projections of $\bm x_t$ onto the signal space ($\bm U$) and the noise space ($\bm U_{\perp}$), with the signal-to-noise ratio (SNR) determined by $\frac{\delta^2+\sigma_t^2}{\delta^2(1+\sigma_t^2)}$. Since this projection is linear in $\bm x_t$, a linear model suffices to learn the ground truth posterior for each time step $t$. Thus, at each time step, we can train the following linear DAE by minimizing \eqref{eq:dae_loss}:
% \begin{align}%\label{eq:para DAE}
%      \bm x_{\bm \theta_t}(\bm x_t) = \bm W_{2,t} \bm W_{1,t}^T \bm x_t,
%     \end{align}
% Since we are currently dealing with a single Gaussian, we directly measure the representation ability of the learned solution $\widehat{\bm W}_t = \widehat{\bm W}_{2,t} \widehat{\bm W}_{1,t}^T$ by its distance from the signal space $\bm U \bm U^T$. This is because, intuitively, a smaller distance indicates that the linear model more effectively captures the underlying data distribution, and thus better representation. To quantify this, we use the normalized distance to the signal space (which we term as \emph{NDS}), given by $\frac{\|\left( \bm I - \bm U \bm U^T \right) \widehat{\bm W}_t\|_F^2}{\| \widehat{\bm W}_t \|_F^2}$, as a metric to evaluate how well the linear model captures the data structure.

% Now we need to confirm that the unimodal curve occurs within this parameterization. Hence, we conduct experiments using $N=1000$ data samples as defined in (\ref{eq:MoG noise}) and train separate linear DAEs ($\widehat{\bm W}_{2,t} \widehat{\bm W}_{1,t}^T$) for each noise scale using the Adam optimizer \citep{kingma2014adam}. To mimic the training process of diffusion models, we employ the same learning rate, initialization, and training epochs across all experiments. As shown in \Cref{fig:single_linear}(a), the normalized distance to the signal space (\emph{NDS}) for the learned $\widehat{\bm W}_t$ indeed exhibits a reversed unimodal pattern. This enables us to explain the unimodal representation dynamics within this simplified framework.

% \paragraph{Loss decomposition.} In the linear DAE scenario, the learned \emph{NDS} can be decomposed as follows:

% \begin{align}
%     \frac{\|\left( \bm I - \bm U \bm U^T \right) \widehat{\bm W}\|_F^2}{\| \widehat{\bm W} \|_F^2} = \frac{\|\left( \bm I - \bm U \bm U^T \right) \bm W^{\star} \|_F^2}{\| \bm W^{\star} \|_F^2} + \left( \frac{\|\left(\bm U \bm U^T \right) \bm W^{\star}\|_F^2}{\| \bm W^{\star} \|_F^2} - \frac{\|\left(\bm U \bm U^T \right) \widehat{\bm W} \|_F^2}{\| \widehat{\bm W} \|_F^2} \right)
% \end{align}

% where the term on the left-hand side of the equation measures the total distance of $\widehat{\bm W}$ from the signal space (using the \emph{NDS} metric), the first term on the right-hand side measures the total distance of the optimal $\bm W^{\star}$ \footnote{We note that the ground truth (optimal) solution is essentially the expectation of the linear model's closed-form solution.} from the signal space (the optimal \emph{NDS}), and the second term represents the convergence error, which quantifies the difference between the learned and optimal solutions in the signal space. This decomposition reveals that the learned \emph{NDS} consists of two parts: the optimal \emph{NDS} and the error between the learned and true solutions. As shown in \Cref{fig:single_linear}(b), the ground truth \emph{NDS} decreases monotonically as the noise scale increases, while the error between the learned and true solutions grows. The combined effect of these two components forms the reversed unimodal curve observed in \Cref{fig:single_linear}(a). In the following, we explore the dynamics of each term individually, highlighting their behavior as the noise scale changes.
% % As we will demonstrate in the following, the two components drive the upward and downward trends in the representation dynamics, respectively.
% %, taken with respect to $\bm a_i$s and $\bm e_i$s


% \paragraph{First stage: the upward trend.} From \eqref{eq:E_sinG}, we know the ground truth $\bm W_t^{\star}$ is given by $\bm W_t^{\star} = \bm W_{2,t}^{\star} \bm W_{1,t}^{\star T} = \frac{1}{1+\sigma_t^2}\bm U\bm U^T + \frac{\delta^2}{\delta^2+\sigma_t^2}\bm U_\perp\bm U_\perp^T $ which has SNR $\frac{\delta^2+\sigma_t^2}{\delta^2(1+\sigma_t^2)}$. Since $b < 1$, when the noise scale increases, the SNR increases. This suggests that if the learned $\widehat{W}_t$ can converge to the ground truth solution, its distance to the signal space should decrease monotonically as the noise scale grows. Our results in \Cref{fig:single_linear}(b) for the ground truth \emph{NDS} confirm this behavior which also aligns with previous arts that state additive noise serves as implicit regularization for model training . 

% \paragraph{Second stage: the downward trend.} Yet, as illustrated by the red curve in \Cref{fig:single_linear}(b), the convergence error, quantified by $\left( \frac{\|\left(\bm U \bm U^T \right) \bm W^{\star}\|_F^2}{\| \bm W^{\star} \|_F^2} - \frac{\|\left(\bm U \bm U^T \right) \widehat{\bm W} \|_F^2}{\| \widehat{\bm W} \|_F^2} \right)$, increases with larger noise scales. This issue stems directly from the training objectives. As the noise scale rises, both coefficients for the signal and noise space projection diminish, leading to a reduced norm in the ground truth solution $\bm W^{\star}$ and amplified optimization difficulty. Converging to such small-norm solutions typically demands finer control such as smaller initialization, reduced learning rates, or extended training time. Without these adjustments, the model risks either diverging or oscillating around the optimal solution. However, in the diffusion model training regime (which we mimicked in our linear DAE experiments), all noise scales are trained simultaneously, using a single learning rate and fixed training time. This fixed training regime, when applied across varying noise scales, fails to account for the growing complexity of optimization that comes with higher noise levels. As a result, the optimization becomes progressively harder, and the gap between the learned and optimal solutions widens with increasing noise, as our results demonstrate.

% \subsubsection{Generalize to the \MoLRG~distribution}\label{subsubsec:molrg}

% Having established the relationship between subspace learning error and optimization difficulty as noise scale increases in the single Gaussian case with a linear model, we now extend this analysis to show that a similar interplay occurs for more general \MoLRG~distributions and non-linear models.



% \begin{figure}[t]
%     \begin{center}
%     \begin{subfigure}{0.47\textwidth}
%     \includegraphics[width = 0.955\textwidth]{figs/theory_demo.pdf}
%     \caption{xxx} 
%     \end{subfigure} \quad %\hspace*{\fill}
%     \begin{subfigure}{0.47\textwidth}
%     \includegraphics[width = 0.955\textwidth]{figs/theory_demo_2.pdf}
%     \caption{xxx} 
%     \end{subfigure}
%     \end{center}
% \caption{\textbf{xxx}}
% \label{fig:theory_demo}
% \end{figure}

%\qq{both x and $\theta$ should be bolded, needs to make it consistent for all.} 
\begin{figure}[t]
    \begin{center}
    % \includegraphics[width=0.38\textwidth]{figs/molrg_illustration.pdf} % Replace with your image
    \includegraphics[width=0.4\textwidth]{figs/molrg_illustration_v2.pdf}
    \end{center}
    \vspace{-0.25in}
    \caption{\textbf{An illustration of \MoLRG\;with different noise levels.} We visualize samples drawn from noisy~\MoLRG~with noise levels $\delta = 0.1,\;0.3$ and $K=3$.}
    \label{fig:sample}
    \vspace{-0.1in}
\end{figure}

\begin{figure*}[t]
    \begin{center}
    \includegraphics[width = 0.9\textwidth]{figs/Fig1_teaser_zekai.pdf}
    % {figs/teaser_final.pdf}
    \end{center}
\vspace{-0.1in}
\caption{\textbf{Trade-offs between representation quality and generation quality.} The {\color{c1} curve with pentagon markers} demonstrates the transition from fine to coarse granularity in posterior estimation as noise levels increase, corresponding to the monotonic rise in FID. In contrast, the {\color{c2} curve with square markers} reveals an unimodal trend in posterior classification accuracy, achieving peak performance at intermediate noise levels. This occurs when high-level details are filtered out while essential low-level semantic information is preserved, as illustrated by the posterior estimations according to different noise levels shown at the bottom of the figure.}
\label{fig:use_clean}
\end{figure*}

\begin{proposition}\label{lem:E[x_0]_multi}
Suppose the data $\bm x_0$ is drawn from a noisy \MoLRG~data distribution with $K$-class and noise level $\delta$. Let  $\zeta_t = \frac{1}{1 + \sigma_t^2}$ and $\xi_t = \frac{\delta^2}{\delta^2 + \sigma_t^2}$, where $\sigma_t$ is the noise scaling in \eqref{eq:Tweedie}. Then for each time $t > 0$, the optimal posterior estimator $\E\left[\bm x_0 \vert \bm x_t\right]$ has the analytical form: 
\begin{align*}
    \E\left[ \bm x_0 \vert \bm x_t\right] = \sum_{l=1}^K w^{\star}_l(\bm x_t,t) \left( \zeta_t \bm U_l\bm U_l^T + \xi_t \bm U_l^{\perp}\bm U_l^{\perp T} \right) \bm x_t.
\end{align*}
where $w^{\star}_l(\bm x_t,t) = \frac{\exp\left(g_l(\bm x_t, t) \right)}{\sum_{l=1}^K \exp\left(g_l(\bm x_t, t) \right)}$ is a soft-max operator for $g_l(\bm x,t) = \frac{1}{2\sigma_t^2}\zeta_t \|\bm U_l^T \bm x\|^2 + \frac{\delta^2}{2 \sigma_t^2}\xi_t \| \bm U_l^{\perp T} \bm x \|^2 $.
%    \begin{align}
%         \ &w^{\star}_l(\bm x_t,t) := \frac{\exp\left(g_l(\bm x_t, t) \right)}{\sum_{l=1}^K \exp\left(g_l(\bm x_t, t) \right)}, \label{eq:w-k} \\
%           \ &g_l(\bm x,t) = \frac{1}{2\sigma_t^2}\zeta_t \|\bm U_l^T \bm x\|^2 + \frac{\delta^2}{2 \sigma_t^2}\xi_t \| \bm U_l^{\perp T} \bm x \|^2 . \label{eq:softmax}
%    \end{align}
\end{proposition}
%    \begin{align}\label{eq:E_MoG}
%        \begin{split}
%        &\hat{\bm x}_{\bm \theta}^{\star}(\bm x_t, t) := \E\left[ \hat{\bm x_0}\vert \bm x_t\right] \\
%        &\quad =\sum_{l=1}^K w^{\star}_l(\bm x_t,t) \left( \zeta_t \bm U_l\bm U_l^T + \xi_t \bm U_l^{\perp}\bm U_l^{\perp T} \right) \bm x_t 
%        \end{split}
%    \end{align}
%\qq{I feel the discussion on generation quality is unnecessary and can be deferred if possible. We should discuss why we can use the posterior estimation as an indicator of representation quality here instead.}
%\paragraph{Remark.}
The proof can be found in \Cref{app:proof_prop} and it is an extension of the result in \citep{wang2024diffusion}. For $\bm x_0$ following noisy \MoLRG, note that the optimal solution $\hat{\bm x}_{\bm \theta}^{\star}(\bm x_t, t) $ of the training loss \eqref{eq:dae_loss} would exactly be $ \E\left[\bm x_0 \vert \bm x_t\right]$. As such, as illustrated in \Cref{fig:use_clean}, the analytical form of the posterior estimation facilitates the study of generation-representation tradeoff across timesteps:
%In the above proposition, we present the \textbf{ground truth} posterior estimation function that a diffusion model aims to approximate by minimizing the training objective defined in (\ref{eq:dae_loss}). We denote this optimal model as $\hat{\bm x}_{\bm \theta}^{\star}$. Under this optimal setting, the trade-off between generation and representation learning dynamics can be analyzed by evaluating posterior estimations at different time steps $t$, .
\begin{itemize}[leftmargin=*]
    \item \emph{The generation quality.} The generation quality of posterior estimation cam be measured by $||\hat{\bm x}_{\bm \theta}^{\star}(\bm x_t, t) - \bm x_0||^2$. As shown in \Cref{lem:E[x_0]_multi}, this error is minimized at $t=0$ with $\sigma_t=0$, where the true class weight satisfies $w^{\star}_k(\bm x_t) = 1$, yielding $\hat{\bm x}_{\bm \theta}^{\star}(\bm x_t, t) = \bm x_0$. As $t$ increases, higher noise levels $\sigma_t$ decrease $w^{\star}_k(\bm x_t)$, causing a monotonic increase in FID, as seen in \Cref{fig:use_clean}.
    \item \emph{The representation quality.} The representation quality follows a unimodal trend across timesteps \citep{xiang2023denoising,tang2023emergent}, which can be measured through the posterior estimator $\hat{\bm x}_{\bm \theta}^{\star}(\bm x_t, t)$ (see \Cref{subsec:rep-quality}). As shown in \Cref{fig:use_clean}, this unimodal behavior creates a trade-off between generation and representation quality, particularly at smaller $t$ when closer to the original image.
    % \zk{This seems to overlap with the theorem?}
%    To better understand the tradeoff between generation FID and posterior accuracy as illustrated in \Cref{fig:use_clean}, we analyze the unimodal representation dynamics using the ground truth posterior estimation function. This is done by studying the \CSNR~with this function, which we discuss in more detail in the following section.
    % Similarly, to analyze the unimodal behavior of representation quality, an appropriate evaluation metric is required. We discuss this metric in the following section.
\end{itemize}




% \subsection{Measuring Representation Quality}\label{subsec:rep-quality}

\subsection{Measuring Posterior Representation Quality}\label{subsec:rep-quality}
% \qq{this might be moved to earlier sections} 



%In this work, we use classification tasks to study representation learning, focusing on two types of accuracies: (i) feature accuracy: This refers to the classification accuracy achieved by applying linear probing on extracted feature representations, and (ii) posterior accuracy: Since intermediate representations in diffusion models are a byproduct of the posterior estimation process, we directly evaluate the classification accuracy using the posterior estimations ($\hat{\bm x_0} = \bm x_{\bm \theta}(\bm x_0, t)$) to analyze the trade-off between generation quality and representation quality. 

For understanding diffusion-based representation learning, we introduce a metric termed Class-specific Signal-to-Noise Ratio (\CSNR) to quantify the posterior representation quality as follows.

\begin{definition}[Class-specific Signal-to-Noise Ratio]
\emph{Suppose the data $\bm x_0$ follows the noisy \MoLRG\;introduced in \Cref{assum:subspace}. Without loss of generality, let $k$ denote the true class of $\mb x_0$. For its associated posterior estimator $\hat{\bm x}_{\bm \theta}$,} %we define \CSNR\; as:
\begin{align*}%\label{eq:csnr_true}
    \mathrm{CSNR}(\hat{\bm x}_{\bm \theta},t) := \E_k \left[\frac{\E_{\bm x_0}[\|\bm U_k\bm U_k^T\hat{\bm x}_{\bm \theta}(\bm x_0, t)\|^2 \mid k ] }{\E_{\bm x_0}[\sum_{l\neq k}\|\bm U_l\bm U_l^T\hat{\bm x}_{\bm \theta}(\bm x_0, t)\|^2 \mid k ]}\right]
\end{align*}
\emph{Here, $\bm U_k$ represents the basis of the subspace corresponding to the true class to which $\bm x_0$ belongs and the $\bm U_l$s with $l \neq k$ denotes the bases of the subspaces for other classes. }
\end{definition}

%\paragraph{Posterior representation quality based upon noisy \MoLRG.} 


% \begin{align}\label{eq:csnr_true}
%     \mathrm{CSNR}(t, f) := \frac{\E_{\bm x_0}[\|\hat{\bm U}_k\hat{\bm U}_k^Tf(\bm x_0, t)\|^2]}{\E_{\bm x_0}[\sum_{l\neq k}\|\hat{\bm U}_l\hat{\bm U}_l^Tf(\bm x_0, t)\|^2]}
% \end{align}

Intuitively, successful prediction of the class for $\bm x_0$ is achieved when the projection onto the correct class subspace, $\|\bm U_k\bm U_k^T \hat{\bm x}_{\bm \theta}(\bm x_0, t)\|$, preserves larger energy than the projections onto subspaces of any other class, $\|\bm U_l\bm U_l^T \hat{\bm x}_{\bm \theta}(\bm x_0, t)\|$. Thus, \CSNR~measures the ratio of the true class signal to irrelevant noise from other classes at a given noise level $t$, serving as a practical metric for evaluating classification performance and hence the representation quality. In this work, we use posterior representation quality as a proxy for studying the representation dynamics of diffusion models for the following reasons:
%, rather than directly analyzing feature quality, for the following reasons:
\begin{itemize}[leftmargin=*]
    \item \emph{Posterior quality reflects feature quality.} Diffusion models $\hat{\bm{x}}_{\bm{\theta}}$ are trained to perform posterior estimation at a given time step $t$ using corrupted inputs, with the intermediate features emerging as a byproduct of this process. Thus, a more class-representative posterior estimation inherently implies more class-representative intermediate features.
    \item \emph{Model-agnostic analysis.} Our goal is to provide a general analysis independent of specific network architectures and feature extraction protocols. Posterior representation quality offers a unified metric that avoids assumptions tied to particular architectures, making the analysis broadly applicable.
\end{itemize}

%\zk{Also we need to note we use intermediate feature and inner representation interchangeably in our paper.}.
% \zk{The representativeness of the posterior reflects representation quality.}
% Moreover, we note that \CSNR~is applicable to any feature extracting function $f$ \zk{that has subspace structures/linearity within its range}, whether it represents feature extraction or posterior estimation. Accordingly, we adopt \CSNR~as a proxy for classification performance in subsequent analyses and experiments.

% Here, $\hat{\bm U}_k$ represents the basis of the subspace corresponding to the true class to which $f(\bm x_0, t)$ belongs and $\hat{\bm U}_l,l \neq k$ denotes the bases of the subspaces for other classes. Intuitively, successful prediction of the class for $\bm x_0$ is achieved when the projection onto the correct class subspace, $\|\hat{\bm U}_k\hat{\bm U}_k^T f(\bm x_0, t)\|$, is greater than the projections onto subspaces of any other class, $\|\hat{\bm U}_l\hat{\bm U}_l^T f(\bm x_0, t)\|$. Thus, \CSNR~measures the ratio of the true class signal to irrelevant noise from other classes at a given noise level $t$, serving as a practical metric for evaluating classification performance. We note that \CSNR~is applicable to any function $f$, whether it represents feature extraction or posterior estimation. Accordingly, we adopt \CSNR~as a proxy for classification performance in subsequent analyses and experiments.

\subsection{Main Theoretical Results}



% \qq{I feel the main theorem is overly long. You need to introduce the CSNR first, build some intutions on what you want to show, and then describe your results. Otherwise, the reviewer does not understand what you want to show.}

%\ZK{need to be consistent}

% As we discussed in \Cref{subsec:posterior}, based upon the strong correlation between representation quality and the posterior mean estimation, we analyze $\hat{\bm x}_{\mb \theta}^{\star}(\bm x_0, t)$ across different time step $t\in[0,1]$. Here, we use $\mb x_0$ as the input instead of $\mb x_t$ according to our discussion in \Cref{subsec:feature-extraction}. 
Based upon the setup in \Cref{subsec:model} and \Cref{subsec:rep-quality}, we obtain the following results.
%Now, we are ready to state our main theorem as follows:

\begin{theorem}\label{lem:main}(Informal)
% Let data $\bm x_0$ be any arbitrary data point drawn from the \MoLRG~distribution defined in Assumption \ref{assum:subspace} and let $k$ denote the true class $\bm x_0$ belongs to. Then \CSNR~introduced in \eqref{eq:csnr_true} depends on the noise level $\sigma_t$ in the following form: 
%     % Without loss of generality, for any clean $\bm x_0$ from class k (i.e., $\bm x_0 = \bm U_k \bm a_i + \delta\bm U_k^{\perp} \bm e_i$), 
%     % \qq{maybe we should write the decomposition out in the following}
%     \begin{align}\label{eq:csnr}
%         \mathrm{CSNR}(t, \hat{\bm x}_{\textit{approx}}^{\star}) = \frac{1}{(K-1)\delta^2} \cdot\left(\frac{1 + \frac{\sigma_t^2}{\delta^2}h(\hat{w}_k, \delta)}{1 + \frac{\sigma_t^2}{\delta^2}h(\hat{w}_l, \delta)}\right)^2
%     \end{align}
% \qq{I feel maybe we should change $b$ to some greek characters} 
% \qq{needs to remind reviewers what is $\sigma_t$ here}
Suppose the data $\bm x_0$ follows the noisy \MoLRG\;introduced in \Cref{assum:subspace} with $K$ classes and noise level $\delta$, then  the \CSNR~of the optimal denoiser $\hat{\bm x}_\theta^{\star}$ takes the following form:
    % Without loss of generality, for any clean $\bm x_0$ from class k (i.e., $\bm x_0 = \bm U_k \bm a_i + \delta\bm U_k^{\perp} \bm e_i$), 
    % \qq{maybe we should write the decomposition out in the following}
    \begin{align}\label{eq:csnr}
        \mathrm{CSNR}(\hat{\bm x}_\theta^{\star},t) = \frac{1}{(K-1)\delta^2}\cdot \left(\frac{1 + \frac{\sigma_t^2}{\delta^2}h(\hat{w}_t^+, \delta)}{1 + \frac{\sigma_t^2}{\delta^2}h(\hat{w}_t^-, \delta)}\right)^2.
    \end{align}
Here, $h(w, \delta) := (1 - \delta^2)w + \delta^2$ is a monotonically increasing function with respect to $w$. Additionally, $h(\hat{w}_t^+, \delta)$ and $h(\hat{w}_t^-, \delta)$ denote positive and negative class confidence rates with% \qq{does the function below depends on $\sigma_t$? we need to reflect that}\zk{It is a function of $\sigma_t$(or t) and $\delta$, but we omit $\delta$ and use t as a subscript to make it one line} \qq{we can add a bracket below to denote those}
\begin{align*}
\begin{cases}
\hat w_t^+(\sigma_t, \delta) &=\; \mathbb E_k[ \mathbb{E}_{\bm x_0}[w_k(\bm x_0, t)\mid k]], \\
\hat w_t^-(\sigma_t, \delta) &=\; \mathbb E_{k}[\mathbb{E}_{\bm x_0}[w_{l}(\bm x_0, t) \mid k \neq l ]],
\end{cases}
\end{align*}
whose analytical forms can be found in \Cref{app:thm1_proof}.  
\end{theorem}



%$\hat w_t^+ = \mathbb{E}_{\bm x_0}[w_k(\bm x_0, t)]$, $\hat w_t^-= \mathbb{E}_{\bm x_0}[w_{l}(\bm x_0, t)] \;\text{for}\; l\neq k$, and $h(w, \delta) := (1 - \delta^2)w + \delta^2$.
%\begin{align*}&\hat w_t^+ := \mathbb{E}_{\bm x_0}[w_k(\bm x_0, t)]\\
%    &\hat w_t^-:= \mathbb{E}_{\bm x_0}[w_{l}(\bm x_0, t)], l\neq k\\
%    &h(w, \delta) := (1 - \delta^2)w + \delta^2
%\end{align*} 
%where samples $\bm x_0$ are drawn from class $k$ (i.e., $\bm x_0 = \bm U_k \bm a_i + b\bm U_k^{\perp} \bm e_i$)
%we leave the analytical form of $\hat w_t^+$ and $\hat w_t^- $ to the appendix, with which we are able to show that it exhibits a unimodal trend. Note that here $\sigma_t$ denotes the level of additive Gaussian noise introduced during the diffusion training process.
% and $h(w, \delta) := (1 - \delta^2)w + \delta^2$. 
%Since $\delta$ is fixed, $h(w,\delta)$ is a monotonically increasing function with respect to $w$. 
% The term $\hat{\bm x}_{\textit{approx}}^{\star}$ serves as an approximation of $\hat{\bm x}_{\bm \theta}^{\star}$ as defined in \eqref{eq:E_MoG}, obtained by approximating $w_l^{\star}$ in \eqref{eqn:w-k} with $\hat{w}_l$ by taking the expectation inside the softmax with respect to $\bm x_0$.\footnote{We empirically validate the tightness of this approximation in \Cref{fig:assump_validate}.}  
% Note that here $\delta$ represents the magnitude of the fixed intrinsic noise in the data 


    % \xiao{I moved the other part of the thm to next section, not sure if this is better.} \qq{I feel it would be better to move it back, and we can refer later on}
%\qq{just say it is a monotonically increasing function w.r.t. to w}
    % Furthermore, we can decompose $\E_{\bm x_0}[\|f^{\star}(\bm x_0, t)\|^2]$ as:
    % \begin{align}\label{eq:fx_decompose}
    % \begin{split}
    %     \E[\|f^{\star}(\bm x_0, t)\|^2] &= \E\|\bm U_k \bm U_k^Tf^{\star}(\bm x_0, t)\|^2] + \E[\sum_{l \neq k}^K\bm U_l \bm U_l^Tf^{\star}(\bm x_0, t)\|^2] + \E[\|\bm U_{\perp} \bm U_{\perp}^Tf^{\star}(\bm x_0, t)\|^2]
    % \end{split}
    % \end{align}
    % where 
    % \begin{align}\label{eq:fx_decompose_terms}
    % \begin{split}
    %     \E\|\bm U_k \bm U_k^Tf^{\star}(\bm x_0, t)\|^2] &= \left( \frac{\hat{w}_k}{1 + \sigma_t^2} + \frac{(K-1)\delta^2\hat{w}_l}{\delta^2+\sigma_t^2}\right)^2 d \\
    %     \E[\|\bm U_{\perp} \bm U_{\perp}^Tf^{\star}(\bm x_0, t)\|^2] &= \frac{b^6 (n-Kd)}{(\delta^2 + \sigma_t^2)^2} \\
    %     \E[\sum_{l \neq k}^K\bm U_l \bm U_l^Tf^{\star}(\bm x_0, t)\|^2] &= \left( K-1 \right)\left( \frac{\hat{w}_l}{1+\sigma_t^2} + \frac{\delta^2(\hat{w}_k + (K-2)\hat{w}_l)}{\delta^2 + \sigma_t^2} \right)^2 \delta^2 d.
    % \end{split}
    % \end{align}



% In the \MoLRG~setting, the ground truth posterior mean is a combination of $\bm x_t$'s projections onto the signal and noise spaces of each class, with the coefficients determined by a softmax term. Similar to the single Gaussian case, the SNR for the reconstruction is:
% \begin{align}
%     \frac{1}{1 + \sigma_t^2} \;/\; \frac{\delta^2}{\delta^2 + \sigma_t^2} = \frac{\delta^2+\sigma_t^2}{\delta^2(1+\sigma_t^2)}
% \end{align}

% increases as the noise scale grows. Given that the optimal weights involve a softmax term, a linear network is no longer sufficient to capture the optimal solution. Therefore, we employ a more practical MLP-based model to learn the optimal score function.
% \qq{use mathrm for all CSNR}

We defer the formal statement of \Cref{lem:main} and its proof to \Cref{app:thm1_proof}. In the following, we discuss the implications of our result.

\paragraph{The unimodal curve of \CSNR\;across noise levels.}
Intuitively, our theorem shows that unimodal curve is mainly induced by the the interplay between the ``denoising rate" $\sigma_t^2/\delta^2$ and the positive class confidence rate $h(\hat{w}_t^+, \delta)$ as noise level $\sigma_t$ increases. As observed in \Cref{fig:trade_off}, the ``denoising rate" ($\sigma_t^2/\delta^2$) increases monotonically with $\sigma_t$ while the class confidence rate $h(\hat{w}_t^+, \delta)$ monotonically declines. Initially, when $\sigma_t$ is small, the class confidence rate remains relatively stable due to its flat slope, and an increasing ``denoising rate" improves the \CSNR, resulting in improved posterior estimation. However, as indicated by \Cref{lem:E[x_0]_multi}, when $\sigma_t$ becomes too large, $h(\hat{w}_t^+,\delta)$ approaches $h(\hat{w}_t^-,\delta)$, leading to a drop in \CSNR, which limits the ability of the model to project $\bm x_0$ onto the correct signal subspace and ultimately hurts posterior estimation. 



\paragraph{Alignment of \CSNR\;with posterior representation quality.} Although our theory is derived from the noisy \MoLRG\; distribution, it effectively captures real-world phenomena. As shown in \Cref{fig:csnr_molrg_match,fig:cifar}, we conduct experiments on both synthetic (i.e., noisy \MoLRG) and real-world datasets (i.e., CIFAR and ImageNet) to measure $\mathrm{CSNR}(\hat{\bm x}_{\bm \theta},t)$ as well as the posterior probing accuracy. For posterior probing, we use posterior estimations at different timesteps as inputs for classification. The results consistently show that $\mathrm{CSNR}(\hat{\bm x}_{\bm \theta},t)$ follows a unimodal pattern across all cases, mirroring the trend observed in posterior probing accuracy as the noise scale increases. This alignment provides a formal justification for previous empirical findings \citep{xiang2023denoising, baranchuk2021label, tang2023emergent}, which have reported a unimodal trajectory in the representation dynamics of diffusion models with increasing noise levels. Detailed experimental setups are provided in \Cref{app:exp_detail}.
% \qq{need to explain what is posterior probing performance}
% \qq{this sentence is a bit confusing, revise, I suggest to separate into multiple sentences} 

\paragraph{Explanation of generation and representation trade-off.}
Our theoretical findings reveal the underlying rationale behind the generation and representation trade-off: the proportion of data associated with $\delta$ represents class-irrelevant attributes. The unimodal representation learning dynamic thus captures a ``fine-to-coarse" shift \citep{choi2022perception, wang2023diffusion}, where these class-irrelevant attributes are progressively stripped away. During this process, peak representation performance is achieved at a balance point where class-irrelevant attributes are eliminated, while class-essential information is preserved. In contrast, high-fidelity image generation requires capturing the entire data distribution—from coarse structures to fine details—leading to optimal performance at the lowest noise level $\sigma_t$, where class-irrelevant attributes encoded in the $\delta$-term are maximally retained. Thus, our insights explain the trade-off between generation and representation quality. As visualized in \Cref{fig:use_clean} and \Cref{fig:vis_molrg_3class}, representation quality peaks at an intermediate noise level where irrelevant details are stripped away, while generation quality peaks at the lowest noise level, where all details are preserved.


%    The unimodal curve is decided by the interplay    between the ``denoising rate" and the class confidence rate as noise increases. As observed in \Cref{fig:trade_off}, the ``denoising rate" ($\sigma_t^2/\delta^2$) increases monotonically with $\sigma_t$ while the class confidence rate $h(\hat{w}_t^+, \delta)$ monotonically declines. Initially, as $\sigma_t$ increases, the class confidence rate remains relatively stable due to its flat slope (as seen in \Cref{fig:trade_off}), and an increasing ``denoising rate" enhances the \CSNR, resulting in improved posterior estimation. However, as indicated by \Cref{lem:E[x_0]_multi}, when $\sigma_t$ becomes too large, $h(\hat{w}_t^+,\delta)$ approaches $h(\hat{w}_t^-,\delta)$, leading to a drop in \CSNR, which limits the model's ability to project $\bm x_0$ onto the correct signal space and ultimately impairs posterior estimation. 




%\paragraph{Remark.}  Intuitively, the unimodal curve of \CSNR~reflects how the additive noise level $\sigma_t$ in the diffusion process helps counteract the intrinsic data noise $\delta$. The noise ratio ($\sigma_t/\delta$) can be interpreted as the ``denoising" rate, where a larger ratio indicates more data noise being canceled out and vice versa. Meanwhile, $h(\hat{w}_t^+,\delta)$ represents the class confidence rate, with lower values meaning less class-specific information is captured by the model. With $\sigma_t$ increases from $0$ to $\infty$,  the ``denoising rate" rises accordingly, while the class confidence rate decreases monotonically. Thus, from \Cref{lem:main}, we can derive the rationale behind the unimodal behavior of \CSNR.
    

    
%     \item \textbf{Lower bound on the \CSNR.} From (\ref{eq:csnr}), we observe that as $\sigma_t \to 0$ or $\sigma_t \to \infty$, the $\mathrm{CSNR}(t, f^{\star})$ approaches its lower bound at $\frac{1}{(K-1)\delta^2}$, align with the \CSNR~of the clean image. This implies that, even in the worst-case scenario, the optimal posterior estimation retains some class-related information. \qq{I do not find this very informative for our paper, what is the main message we want to convey in this for supporting our paper?}

% \qq{I think here, the most important thing is to explain the following two points: (i) why (10) implies unimodal curve? (ii) what is the implications of the results, or this is related to representation learning? We need to address these two first, then explain other less important stuff }
    %This explains why, even in the memorization regime\ZK{In the memorization regime the optimal score would be different, we should be careful.}, the learned diffusion model retains non-trivial test accuracy, as we will see later in \Cref{fig:phase_transit_2}
  %  \item \textbf{The unimodal curve of representation quality.} Previous studies \citep{xiang2023denoising, baranchuk2021label, tang2023emergent} show that diffusion models exhibit a unimodal representation dynamic as noise increases across tasks like classification, segmentation, and image correspondence. Our theoretical analysis explains this phenomenon: the proportion of data associated with $\delta$ corresponds to class-irrelevant attributes or finer details, leading to a "fine-to-coarse" shift \citep{choi2022perception, wang2023diffusion} where these details are progressively discarded. During this process, peak representation performance is achieved at a balance point where class-irrelevant attributes are eliminated, while class-essential information is preserved.

 %    \item \textbf{Real world analogy and trade-off between generation and representation.}
 %    Previous studies \citep{xiang2023denoising, baranchuk2021label, tang2023emergent} have empirically shown that the representation dynamics of diffusion models follow a unimodal curve as the noise scale increases, across various tasks such as classification, segmentation, and image correspondence. Our theoretical findings reveal the underlying rationale behind this phenomenon: the proportion of data associated with $\delta$ represents class-irrelevant attributes or finer image details. The unimodal representation learning dynamic thus captures a ``fine-to-coarse" shift \citep{choi2022perception, wang2023diffusion}, where these details are progressively stripped away. During this process, peak representation performance is achieved at a balance point where class-irrelevant attributes are eliminated, while class-essential information is preserved.
  %   In contrast, high-fidelity image generation requires capturing the entire data distribution—from coarse structures to fine details—leading to optimal performance at the lowest noise level, where class-irrelevant attributes and finer image details encoded in the $\delta$-term are maximally retained. Thus, our insights clarify the trade-off between generation and representation quality. As visualized in \Cref{fig:use_clean} and \Cref{fig:vis_molrg_3class}, representation quality peaks at an intermediate noise level where irrelevant details are stripped away, while generation quality peaks at the lowest noise level, where all details are preserved.
    % This interpretation is validated by the visualization in \Cref{fig:vis_molrg_3class}.
    
    % In the plot, each class is represented by a colored straight line, while deviations from these lines correspond to the $\delta$-related noise term. Initially, increasing the noise scale effectively cancels out the $\delta$-related data noise, resulting in a cleaner posterior estimation and improved probing accuracy. However, as the noise continues to increase, the class confidence rate drops, leading to an overlap between classes, which ultimately degrades the feature quality and probing performance.
    
    %\ZK{where similar class mix-up phenomenons are observed in \citep{li2024critical}}
    % As the additive noise level $\sigma_t$ increases, the noise ratio ($b_t/\sigma_t$) decreases, and $\hat{w}_k$ diminishes since $g_k(\bm x_0)$ and $g_l(\bm x_0)$ become more similar. In \Cref{fig:ucurve_molrg3class}(c), we plot the noise ratio and $h(w, b)$ as the noise level increases. We can observe that, at the initial stage, the noise ratio declines significantly, while $h(w, b)$ remains nearly unchanged, leading to a larger proportion taken by $h(w, b)$. Since $h(\hat{w}_k, b)$ is much larger than $h(\hat{w}_l, b)$ at this stage, the \CSNR initially increases. However, this trend reverses once $h(\hat{w}_k, b)$ begins to decrease, and the magnitude of $h(w, b)$ outweighs the noise ratio, driving the \CSNR downwards as the ratio $h(\hat{w_2}, b)/h(\hat{w_1}, b)$ decreases.

    % \item \ZK{Correlations between CSNR and representation quality, \cref{fig:ucurve_molrg3class}}
    


\begin{figure}[t] % 'r' for right, 'l' for left
    \centering
    \includegraphics[width=0.4\textwidth]{figs/interplay.pdf}
    \caption{Illustration of the interplay between the denoising rate and the class confidence rate.}
    \label{fig:trade_off}
\end{figure}
\vspace{-0.1in}


\begin{figure}[t]
    \centering
    \includegraphics[width = 0.45\textwidth]{figs/posterior_curve_molrg.pdf}
\caption{\textbf{Posterior probing accuracy and associated \CSNR~dynamics in \MoLRG~data.} We plot the posterior probing accuracy and \CSNR~with the posterior estimations obtained from a learned estimator $\bm{\hat x_\theta}$. both of which exhibit a consistent unimodal pattern. Additionally, we include the optimal \CSNR~, calculated from the ground truth posterior function $\bm{\hat x_\theta}^\star$ defined in \Cref{lem:E[x_0]_multi}, as a reference. The estimator is trained on a 3-class \MoLRG~dataset with data dimension $n=50$, subspace dimension $d=15$, and noise scale $\delta=0.5$.}
\label{fig:csnr_molrg_match}
\end{figure}


% \begin{figure*}[t]
% \begin{center}
%     \begin{subfigure}{0.47\textwidth}
%     \includegraphics[width = 0.955\textwidth]{figs/posterior_curve_molrg.pdf}
%     \caption{Posterior Probing Acc. and $\mathrm{CSNR}(\hat{\bm x}_{\bm \theta},t)$} 
%     \end{subfigure} \quad %\hspace*{\fill}
%     \begin{subfigure}{0.47\textwidth}
%     \includegraphics[width = 0.955\textwidth]{figs/feature_curve_molrg.pdf}
%     \caption{Feature Probing Acc. and $\mathrm{CSNR}(\hat{f}_{\bm \theta},t)$} 
%     \end{subfigure}
%     \end{center}
% \caption{\textbf{Probing accuracy and associated \CSNR~dynamics in \MoLRG~data.} In panel (a), we plot the probing accuracy and the previously discussed \CSNR, both metrics exhibit a consistent unimodal pattern as depicted in \Cref{fig:clean_feature}. Additionally, in (b) we plot the metrics calculated from intermediate features, showing the universality across both posterior and features.}
% \label{fig:csnr_molrg_match}
% \end{figure*}
% mog_csnr.pdf mog_f_csnr.pdf


\begin{figure*}[h]
\begin{center}
    % \begin{subfigure}{0.47\textwidth}
    % \includegraphics[width = 0.955\textwidth]{figs/feature_curve_c10.pdf}
    % \caption{CIFAR10} 
    % \end{subfigure} \quad %\hspace*{\fill}
    % \begin{subfigure}{0.47\textwidth}
    % \includegraphics[width = 0.955\textwidth]{figs/feature_curve_mini.pdf}
    % \caption{MiniImageNet} 
    % \end{subfigure}

    \begin{subfigure}{0.48\textwidth}
    \includegraphics[width = 0.975\textwidth]{figs/posterior_curve_c10.pdf}
    \caption{CIFAR10} 
    \end{subfigure} \quad %\hspace*{\fill}
    \begin{subfigure}{0.48\textwidth}
    \includegraphics[width = 0.975\textwidth]{figs/posterior_curve_mini.pdf}
    \caption{MiniImageNet} 
    \end{subfigure}
    \end{center}
\caption{\textbf{Dynamics of posterior probing accuracy and associated \CSNR~on CIFAR10 and MiniImageNet.} Posterior probing accuracy is plotted alongside $\mathrm{CSNR}(\hat{\bm x}_{\bm \theta},t)$. Probing accuracy is evaluated on the test set, while the empirical \CSNR~is computed from the training set. Both exhibit an aligning unimodal pattern. We use released EDM models \citep{karras2022elucidating} trained on the CIFAR-10 \citep{krizhevsky2009learning} and ImageNet \citep{deng2009imagenet} datasets, evaluating them on CIFAR-10 and MiniImageNet \citep{vinyals2016matching}, respectively. To compute \CSNR~, we apply PCA on the original CIFAR-10/MiniImageNet images to extract the basis $\bm{U}_k$s. Further details can be found in \Cref{app:exp_detail}. }
% \qq{moved here, please revise, this is also the setup of Figure 5? we can refer to Figure 6}}
\label{fig:cifar}
\end{figure*}
%\caption{\textbf{Dynamics of feature probing accuracy and associated \CSNR~on CIFAR10 and MiniImageNet.} Feature probing accuracy is plotted alongside $\mathrm{CSNR}(t, f)$. Probing accuracy is evaluated on the test set, while the empirical \CSNR~is computed from the training set. Both exhibit an aligning unimodal pattern.}
%\caption{\textbf{Dynamics of posterior probing accuracy and associated \CSNR~on CIFAR10 and MiniImageNet.} The top row presents feature probing accuracy alongside $\mathrm{CSNR}(t, f)$, while the bottom row illustrates posterior probing accuracy and $\mathrm{CSNR}(t, \bm x_{\bm \theta})$. Probing accuracy is evaluated on the test set, whereas \CSNR~is computed from the training set. Both feature and posterior probing accuracy exhibit unimodal patterns that align with their respective \CSNR~trends.  }
% \xiao{grid} \qq{revise} 
%\zk{this actually overlaps with fig3, have to reduce, also we can combine it with the remark}




% In contrast, high-fidelity image generation requires capturing the entire data distribution—from coarse structures to fine details—leading to optimal performance at the lowest noise level, where class-irrelevant attributes and finer image details encoded in the $\delta$-term are maximally retained. Thus, our insights clarify the trade-off between generation and representation quality. As visualized in \Cref{fig:use_clean} and \Cref{fig:vis_molrg_3class}, representation quality peaks at an intermediate noise level where irrelevant details are stripped away, while generation quality peaks at the lowest noise level, where all details are preserved.

% Previous studies \citep{xiang2023denoising, baranchuk2021label, tang2023emergent} have empirically shown that the representation dynamics of diffusion models follow a unimodal curve as the noise scale increases, across various tasks such as classification, segmentation, and image correspondence. Our theoretical findings reveal the underlying rationale behind this phenomenon: the proportion of data associated with $\delta$ represents class-irrelevant attributes or finer image details. The unimodal representation learning dynamic thus captures a ``fine-to-coarse" shift \citep{choi2022perception, wang2023diffusion}, where these details are progressively stripped away. During this process, peak representation performance is achieved at a balance point where class-irrelevant attributes are eliminated, while class-essential information is preserved.

% On the other hand, high-fidelity image generation requires capturing the entire data distribution—from coarse structures to fine details—leading to optimal performance at the lowest noise level, where class-irrelevant attributes and finer image details encoded in the $\delta$-term are maximally retained. Therefore, our theoretical insights shed light on explaining the trade-off between generation and representation quality. This explanation is supported by our visualizations in \Cref{fig:use_clean} and \Cref{fig:vis_molrg_3class}, which show that for both synthetic and real datasets, the representation peak consistently occurs at an intermediate noise level where class-irrelevant details are removed, while the generation peak is reached at the smallest noise level, where all details are preserved.
% ==============


% Previous studies \citep{xiang2023denoising, baranchuk2021label, tang2023emergent} have empirically shown that the representation dynamics of diffusion models follow a unimodal curve as the noise scale increases, across various tasks such as classification, segmentation, and image correspondence. Our findings corroborate this observation, as demonstrated in \Cref{fig:clean_feature}, where the representation quality consistently exhibits a unimodal trend, regardless of the specific network architecture, dataset or task. In the following analysis, we argue that this unimodal behavior arises from subtle differences and trade-offs between the requirements of representation learning and the generative nature of diffusion models.

% High-fidelity image generation demands that diffusion models capture every aspect of the data distribution—from coarse structures to fine details. In contrast, representation learning, particularly for high-level tasks such as classification \citep{allen2022feature}, prefers an abstract representation, where finer image details may even act as `noise' that hinders performance. As shown in \Cref{fig:use_clean}, as the noise level increases, the predicted posteriors for clean input $\bm x_0$ transition from `fine' to `coarse' \citep{wang2023diffusion,choi2022perception}, gradually removing fine-grained details. For the classification task in the plot, the best performance is achieved when the posterior estimation retains the essential information while discarding some class-irrelevant details. These findings indicate a trade-off between generative quality and representation performance \citep{chen2024deconstructing}, prompting us to attribute variations in feature quality across noise levels to differences in posterior prediction.


%\subsection{Alignment of CSNR with Representation Quality}\label{subsec:theory_verify}

% \begin{figure*}[t]
%     \centering
%     \includegraphics[width=0.9\linewidth]{figs/vis_molrg.pdf}
%     \vspace{-0.1in}
%     \caption{\textbf{Visualization of posterior estimation for a clean input, higher \CSNR~correspondings to higher classification accuracy.} The \textbf{same} \MoLRG~data is fed into the models; each row represents a different denoising model, and each column corresponds to a different time step with noise scale ($\sigma_t$). The \textcolor{red}{red} box indicates the best posterior estimation and feature probing accuracy.}
%     \label{fig:vis_molrg_3class}
% \end{figure*}



%We conduct experiments on both synthetic and real-world datasets to validate our theoretical insights into the dynamics of representation learning.



%The results for the synthetic and real datasets are shown in \Cref{fig:csnr_molrg_match} and \Cref{fig:cifar}, respectively. As depicted in the plots, $\mathrm{CSNR}(\hat{\bm x}_{\bm \theta},t)$ consistently follows a unimodal pattern, aligning with the trend of posterior probing performance as the noise scale increases.

% We extract intermediate features and posterior predictions at each time step and evaluate the classification performance on the test set using linear and MLP probing, respectively. For \CSNR~calculation, we use the raw inputs (i.e., $\bm x_0$s drawn from the \MoLRG~distribution for the synthetic dataset and the original CIFAR10/MiniImageNet images) to compute the basis $\bm U_k$s, followed by the empirical computation of \CSNR~for posterior predictions, denoted as $\mathrm{CSNR}(\hat{\bm x}_{\bm \theta},t)$. For \CSNR~on feature representations, we compute the basis using the features extracted at each time step and subsequently calculate \CSNR~for the features, denoted as $\mathrm{CSNR}(\hat{f}_{\bm \theta},t)$.

% The results for the synthetic datasets are presented in \Cref{fig:csnr_molrg_match} and real datasets in \Cref{fig:cifar,fig:cifar_2}. As illustrated in the plots, $\mathrm{CSNR}(\hat{\bm x}_{\bm \theta},t)$ consistently exhibits a unimodal pattern similar to the posterior probing performance as the noise scale increases. A similar trend is observed for $\mathrm{CSNR}(\hat{f}_{\bm \theta},t)$, which aligns with the feature probing accuracy. These findings support our theoretical predictions.

% \paragraph{Trade-offs between generation and representation quality.}
% To further illustrate the trade-offs between generation and representation quality, we visualize the posterior estimations and \CSNR~values of the synthetic \MoLRG~dataset at different noise scales in \Cref{fig:vis_molrg_3class}. As observed, the highest generation quality occurs at the smallest noise scale, where the generated samples closely resemble the original inputs. However, optimal generation quality does not necessarily translate to optimal representation quality, as measured by probing performance. Instead, the best representation performance is achieved at intermediate noise scales. This observation aligns with our theoretical insights, suggesting that initially, increasing the noise scale mitigates the $\delta$-related data noise, leading to cleaner posterior estimations and improved probing accuracy. However, as the noise scale continues to rise, the class confidence rate diminishes, causing class overlap and ultimately degrading feature quality and probing performance. Similar trends can be observed in real datasets, where CIFAR10 results are presented at the beginning of this paper in \Cref{fig:use_clean}.

\section{Practical Insights}\label{sec:exp}
% In this section, we first validate our theoretical insights into the representation dynamics of diffusion models under both theoretical and practical settings in \Cref{subsec:emp_verify}. 
We examine the practical implications of our findings in \Cref{subsec:exp_ensemeble}, leveraging feature information at different levels of granularity to enhance robustness. Additionally, we discuss the advantages of diffusion models over traditional single-step DAEs in \Cref{subsec:weight_share}.
% and analyze the influence of data complexity on diffusion-based representation learning in \Cref{subsec:mem_gen}. Detailed experimental setups are provided in \Cref{app:exp_detail}.

% \begin{figure*}[t]
% \begin{center}
%     \begin{subfigure}{0.47\textwidth}
%     \includegraphics[width = 0.955\textwidth]{figs/ucurve_cifar_featureacc.pdf}
%     \caption{Probing Acc. of diffusion models and DAEs} 
%     \end{subfigure} \quad %\hspace*{\fill}
%     \begin{subfigure}{0.47\textwidth}
%     \includegraphics[width = 0.955\textwidth]{figs/ucurve_cifar_csnr.pdf}
%     \caption{CSNR and Posterior Acc.} 
%     \end{subfigure}
%     \end{center}
% \caption{\textbf{Dynamics of feature probing accuracy and \CSNR~on CIFAR10.} Panel (a) shows the feature probing accuracy and Panel (b) shows the \CSNR~and posterior probing accuracy trends computed using the CIFAR10 test dataset, both exhibiting a \textcolor{blue}{matching} unimodal pattern.}
% \label{fig:cifar}
% \end{figure*}

% \subsection{Empirical Validation}\label{subsec:emp_verify}

% \qq{condense, and move this as a subsection of Section 3 instead. This section needs to focus on practical insights: (i) ensemble, (ii) weight sharing, and (iii) memorization to generalization}

% \xiao{Section 4.1.1 has been discussed in Section 2. Section 4.1.2 will be removed to Section 3}
% {\color{blue}
% We divide this subsection into two parts: in \Cref{subsubsec:feature-extraction}, we justify our design choice in the theoretical analysis by considering clean inputs $\bm x_0$ as the input to diffusion models, rather than the conventional approach of using noisy inputs $\bm x_t$. Subsequently, in \Cref{subsubsec:theory_verify}, we validate our theoretical insights by demonstrating that \CSNR~aligns closely with both feature and posterior probing performance.

% \subsubsection{Clean Inputs Enhance Representation Performance Compared to Noisy Inputs}\label{subsubsec:feature-extraction}

% If we treat a diffusion model at each noise level (i.e., $\bm x_{\bm \theta}(\cdot, t)$) as an independent model and aim to compare their representation performance, using the noisy input $\bm x_t$ does not provide a fair basis for comparison. This is because the varying input noise levels at different values of $t$ can significantly influence the results. Therefore, in both the definition of \CSNR~\eqref{eq:csnr_true} and the subsequent analysis, we consistently consider clean inputs $\bm x_0$ to eliminate the confounding effects of input degradation.

% A potential concern, however, is whether using clean inputs during inference contradicts the training paradigm, as diffusion models are trained on corrupted inputs. To address this, we compare the representation performance of models using $\bm x_0$ and $\bm x_t$ as inputs across classification and segmentation tasks, with results presented in \Cref{fig:clean_feature}. Our findings highlight two key insights: (1) using clean inputs $\bm{x}_0$ consistently matches or surpasses the performance achieved with corrupted inputs $\bm{x}_t$, and (2) the performance gap widens as $t$ (or equivalently, $\sigma_t$) increases. These observations suggest that the core representation ability of diffusion models is primarily derived from their denoising objective, while the diffusion process—characterized by the gradual addition and removal of Gaussian noise—plays a minimal role in representation quality.

% This phenomenon draws a parallel with traditional supervised and self-supervised learning paradigms. In these settings, data augmentations such as cropping~\citep{caron2021emerging}, color jittering, or masking~\citep{he2022masked} are applied during training to enhance model robustness and generalization. However, clean, unaugmented images are typically used during inference to ensure optimal performance. Similarly, in diffusion models, additive Gaussian noise serves as a form of data augmentation critical for training~\citep{chen2024deconstructing}, but when the focus shifts to representation learning at inference, using clean images $\bm{x}_0$ is both sufficient and aligned with standard representation learning practices. Thus, we advocate for the use of clean images as the standard input protocol for feature extraction during inference throughout our analysis and future work.

% }

\subsection{Feature Ensembling Across Timesteps Improves Representation Robustness}\label{subsec:exp_ensemeble}

Our theoretical insights imply that features extracted at different timesteps capture varying levels of granularity. Given the high linear separability of intermediate features, we propose a simple ensembling approach across multiple timesteps to construct a more holistic representation of the input. Specifically, in addition to the optimal timestep, we extract feature representations at four additional timesteps—two from the coarse (larger $\sigma_t$) and two from the fine-grained (smaller $\sigma_t$) end of the spectrum. We then train linear probing classifiers for each set and, during inference, apply a soft-voting ensemble by averaging the predicted logits before making a final decision.(experiment details in \Cref{app:exp_detail})

We evaluate this ensemble method against results obtained from the best individual timestep, as well as a self-supervised method MAE \citep{he2022masked}, on both the pre-training dataset and a transfer learning setup. The results, reported in \Cref{tab:ensemble_results} and \Cref{tab:ensemble_results_transfer}, demonstrate that ensembling significantly enhances performance for both EDM \citep{karras2022elucidating} and DiT \citep{peebles2023scalable}, consistently outperforming their vanilla diffusion model counterparts and often surpassing MAE. Moreover, ensembling substantially improves the robustness of diffusion models for classification under label noise.

 \begin{table}[t]
\centering
\resizebox{0.48\linewidth}{!}{
	\begin{tabular}	{l c c c c c } 
		
        \toprule
            % \rule{0pt}{0.1ex} \\ 
		 \textbf{Method} & \multicolumn{5}{c}{\textit{MiniImageNet$^\star$} Test Acc. \%}\\
            % \rule{0pt}{0.1ex} \\ 
		 \midrule
		 \textbf{Label Noise} & Clean & 20\% & 40\% & 60\% & 80\% \\
   %       \midrule
   %       \textit{CIFAR10} \\ 
		 % \midrule
   
   %        EDM & 96.0 & 95.8 & 95.9 & 95.5 & 94.6  \\ 
          
		 % \textbf{EDM (Ensemble)} & 95.7 & 95.8 & 95.9 & 95.4 & 95.0 \\ 
         
         \midrule
   %      \textit{MiniImageNet$^\star$} \\  
		 % \midrule
         MAE & 73.7 & 70.3 & 67.4 & 62.8 & 51.5 \\ 
   
          EDM & 67.2 & 62.9 & 59.2 & 53.2 & 40.1  \\ 
          
		 \textbf{EDM (Ensemble)} & 72.0 & 67.8 & 64.7 & 60.0 & 48.2 \\ 

         
   
          DiT & 77.6 & 72.4 & 68.4 & 62.0 & 47.3  \\ 
          
		 \textbf{DiT (Ensemble)} & \textbf{78.4} & \textbf{75.1} & \textbf{71.9} & \textbf{66.7} & \textbf{56.3} \\ 
         
        \bottomrule
	\end{tabular}}
    \caption{\textbf{Comparison of test performance across different methods under varying label noise levels.} All compared models are publicly available and pre-trained on ImageNet-1K \citep{deng2009imagenet}, evaluated using MiniImageNet classes. Bold font highlights the best result in each scenario.}
    \label{tab:ensemble_results}
    % \vspace{-6mm}
\end{table}

 \begin{table*}[t]
\centering
\resizebox{0.98\linewidth}{!}{
	\begin{tabular}	{l c c c c c | c c c c c | c c c c c} 
		
        \toprule
          % \rule{0pt}{0.05in} \\ 
		 \textbf{Method} & \multicolumn{15}{c}{Transfer Test Acc. \%}\\
         % \rule{0pt}{0.2ex} \\ 
           % \rule{0pt}{0.05in} \\ 
         \midrule
        & \multicolumn{5}{c|}{\textbf{CIFAR100}} & \multicolumn{5}{c|}{\textbf{DTD}} & \multicolumn{5}{c}{\textbf{Flowers102}} \\  
		 \textbf{Label Noise} & Clean & 20\% & 40\% & 60\% & 80\% & Clean & 20\% & 40\% & 60\% & 80\% & Clean & 20\% & 40\% & 60\% & 80\% \\ 
         
         
		 \midrule
         MAE & 63.0 & 58.8 & 54.7 & 50.1 & 38.4 & 61.4 & 54.3 & 49.9 & 40.5 & 24.1 & 68.9 & 55.2 & 40.3 & 27.6 & 9.6 \\ 
   
          EDM & 62.7 & 58.5 & 53.8 & 48.0 & 35.6 & 54.0 & 49.1 & 45.1 & 36.4 & 21.2 & 62.8 & 48.2 & 37.2 & 24.1 & 9.7 \\ 
          
		 \textbf{EDM (Ensemble)} & \textbf{67.5} & \textbf{64.2} & \textbf{60.4} & \textbf{55.4} & \textbf{43.9} & 55.7 & 49.5 & 45.2 & 37.1 & 22.0 & 67.8 & 53.9 & 41.5 & 25.0 & 10.4 \\ 

         
   
          DiT & 64.2 & 58.7 & 53.5 & 46.4 & 32.6 & 65.2 & 59.7 & 53.0 & 43.8 & 27.0 & 78.9 & 65.2 & 52.4 & 34.7 & 13.3 \\ 
          
		 \textbf{DiT (Ensemble)} & 66.4 & 61.8 & 57.6 & 51.3 & 39.2 & \textbf{65.3} & \textbf{60.6} & \textbf{56.1} & \textbf{46.3} & \textbf{30.6} & \textbf{79.7} & \textbf{67.0} & \textbf{54.6} & \textbf{36.6} & \textbf{14.7}  \\ 
         
        \bottomrule
	\end{tabular}}
    \caption{\textbf{Comparison of transfer learning performance across different methods under varying label noise levels.} All compared models are publicly available and pre-trained on ImageNet-1K \citep{deng2009imagenet}, evaluated on different downstream datasets. Bold font highlights the best result in each scenario.  }
    \label{tab:ensemble_results_transfer}
    % \vspace{-6mm}
\end{table*}
%\qq{save space of the first line} 
% \begin{table}[t]
%     \centering
%     \begin{tabular}{c|c c c c}
%     label noise & 0 & 0.4 & 0.6 & 0.8\\
%     \hline
%     cifar10 & 95.8 & 94.72 & 93.7 & 82.83\\
%     % cifar10 (ensemble) & 95.75 & 95.52& 95.12 & 94.04 \\
%     cifar10 (ensemble w/ BN) & 95.65 & 95.63 & 95.36 & 94.98\\
%     MAE(ImageNet transfer)& 83.94 & 80.53 & 76.69 & 63.46\\
%     ResNet(ImageNet transfer) \\
%     \hline
%     % miniImageNet(edm) & 65.64 & 54.48  & 42.26 & 23.58\\
%     % miniImageNet(DiT) & 74.77 & 50.91 & 38.09 & \\
%     miniImageNet(DiT w/BN) & 77.56 & 63.42 & 49.59 & 27.11\\
%     miniImageNet(DiT w/BN ensemble) & 78.44 & 67.91 & 54.30 & 30.18\\
%     % miniImageNet (edm ensemble) & 70.9 & 60.53 & 48.13 & 24.96\\
%     % MAE(ImageNet transfer w/o BN) & 73.37 & 66.27 & 56.32 & 33.37\\
%     MAE(ImageNet transfer w/ BN) & 72.87 & 65.79 & 55.15 & 29.32\\
%     % MAE(ImageNet transfer w/trick) & 69.01 & 61.70 & 52.16 & 29.57\\
%     \hline
%     cifar100 \\
%     MAE(w/ BN)& 63.01 & 53.06 & 42.10 & 22.11\\
%     % MAE(w/o BN)& 58.24 & 51.52 & 40.74 & 21.94\\
%     \end{tabular}
%     \caption{Feature ensemble enhance the overall linear probing accuracy, especially robustness againist label noise}
%     \label{tab:ensemble}
% \end{table}

\subsection{Weight Sharing in Diffusion Models Facilitates Representation Learning}\label{subsec:weight_share}

\begin{figure*}[t]
    \begin{center}
    \begin{subfigure}{0.47\textwidth}
    \includegraphics[width = 0.955\textwidth]{figs/dae_diffusion_c10.pdf}
    \caption{CIFAR10} 
    \end{subfigure} \quad %\hspace*{\fill}
    \begin{subfigure}{0.47\textwidth}
    \includegraphics[width = 0.955\textwidth]{figs/dae_diffusion_c100.pdf}
    \caption{CIFAR100} 
    \end{subfigure}
    \end{center}
    \vspace{-0.1in}
\caption{\textbf{Diffusion models exhibit higher and smoother feature accuracy and similarity compared to individual DAEs.} We train DDPM-based diffusion models and individual DAEs on the CIFAR datasets and evaluate their representation learning performance. Feature accuracy, and feature differences from the optimal features (indicated by {\color{cyan} $\star$}) are plotted against increasing noise levels. The results reveal an inverse correlation between feature accuracy and feature differences, with diffusion models achieving both higher/smoother accuracy and smaller/smoother feature differences compared to DAEs.}
\vspace{-0.05in}
\label{fig:dae_diffusion}
\end{figure*}

% Now that we have established the denoising objective as the primary driver of diffusion models' superior representation learning capabilities, we now turn to a key question: what makes diffusion models outperform traditional single-step denoising autoencoders (DAEs) \citep{vincent2008extracting,vincent2010stacked} and achieve on-par or superior representation performance compared to state-of-the-art self-supervised methods? In this subsection, we reveal that the key advantage of diffusion models over traditional DAEs lies in their inherent weight-sharing mechanism.

Second, we reveal why diffusion models, despite sharing the same denoising objective with classical DAEs, achieve superior representation learning due to their inherent weight-sharing mechanism. By minimizing loss across all noise levels (\ref{eq:dae_loss}), diffusion models enable parameter sharing and interaction among denoising subcomponents, creating an implicit "ensemble" effect. This improves feature consistency and robustness across noise scales compared to DAEs \citep{chen2024deconstructing}, as illustrated in \Cref{fig:dae_diffusion}.

To test this, we trained 10 DAEs, each specialized for a single noise level, alongside a DDPM-based diffusion model on CIFAR10 and CIFAR100. We compared feature quality using linear probing accuracy and feature similarity relative to the optimal features at $\sigma_t = 0.06$ (where accuracy peaks) via sliced Wasserstein distance (\SWD) \citep{doan2024assessing}.

The results in \Cref{fig:dae_diffusion} confirm the advantage of diffusion models over DAEs. Diffusion models consistently outperform DAEs, particularly in low-noise regimes where DAEs collapse into trivial identity mappings. In contrast, diffusion models leverage weight-sharing to preserve high-quality features, ensuring smoother transitions and higher accuracy as noise increases. This advantage is further supported by the \SWD~curve, which reveals an inverse correlation between feature accuracy and feature differences. Notably, diffusion model features remain significantly closer to their optimal state across all noise levels, demonstrating superior representational capacity.

Our finding also aligns with prior results that sequentially training DAEs across multiple noise levels improves representation quality \citep{chandra2014adaptive,geras2014scheduled,zhang2018convolutional}. Our ablation study further confirms that multi-scale training is essential for improving DAE performance on classification tasks in low-noise settings (details in \Cref{app:add_exp}, \Cref{tab:dae_trial}).


\section{Discussion}
In this work, we develop a mathematical framework for analyzing the representation dynamics of diffusion models. By introducing the concept of \CSNR~and leveraging a low-dimensional mixture of low-rank Gaussians, we characterize the trade-off between generative quality and representation quality. Our theoretical analysis explains how the unimodal representation learning dynamics observed across noise scales emerge from the interplay between data denoising and class specification.

Beyond theoretical insights, we propose an ensemble method inspired by our findings that enhances classification performance in diffusion models, both with and without label noise. Additionally, we empirically uncover an inherent weight-sharing mechanism in diffusion models, which accounts for their superior representation quality compared to traditional DAEs. Experiments on both synthetic and real-world datasets validate our findings. Additionally, our findings also open up new avenues for future research that we discuss in the following.

\begin{itemize}[leftmargin=*]
    \item \textbf{Principled diffusion-based representation learning.} While diffusion models have shown strong performance in various representation learning tasks, their application often relies on trial-and-error methods and heuristics. For example, determining the optimal layer and noise scale for feature extraction frequently involves grid searches. Our work provides a theoretical framework to understand representation dynamics across noise scales. A promising future direction is to extend this analysis to include layer-wise dynamics. Combining these insights could pave the way for more principled and efficient approaches to diffusion-based representation learning.

    \item \textbf{Representation alignment for better image generation.} Recent work REPA \citet{yu2024repa} has demonstrated that aligning diffusion model features with features from pre-trained self-supervised foundation models can enhance training efficiency and improve generation quality. By providing a deeper understanding of the representation dynamics in diffusion models, our findings could further advance such representation alignment techniques, facilitating the development of diffusion models with superior training and generation performance.
\end{itemize}

\section*{Acnkowledgement}
We acknowledge funding support from NSF CAREER CCF-2143904, NSF CCF-2212066, NSF CCF-
2212326, NSF IIS 2312842, NSF IIS 2402950, ONR N00014-22-1-2529, and MICDE Catalyst Grant.

\bibliographystyle{abbrvnat}
\bibliography{refs}

\newpage
\appendix
\section{Appendix}


\section{Unexpected Questions}
\label{app:question}
Real-world questions do not always have the correct premises. For example, in the question "\begin{CJK}{UTF8}{gbsn}水俣病的传染途径是什么?\end{CJK}(What is the route of infection for Minamata disease?)", Minamata disease is not an infectious disease. Taking this situation into account, we add a small number of human-written questions with incorrect premises and LLM-generated questions with hard-to-verify premises in the question collection phase. The number of these questions in the total number of questions is about 3\%.

\section{Prompt for LLM Augmentation}
\label{app:aug}

\definecolor{titlecolor}{rgb}{0.9, 0.5, 0.1}
\definecolor{anscolor}{rgb}{0.2, 0.5, 0.8}
\definecolor{labelcolor}{HTML}{48a07e}
\begin{table*}[h]
	\centering
	
 % \vspace{-0.2cm}
	
	\begin{center}
		\begin{tikzpicture}[
				chatbox_inner/.style={rectangle, rounded corners, opacity=0, text opacity=1, font=\sffamily\scriptsize, text width=5in, text height=9pt, inner xsep=6pt, inner ysep=6pt},
				chatbox_prompt_inner/.style={chatbox_inner, align=flush left, xshift=0pt, text height=11pt},
				chatbox_user_inner/.style={chatbox_inner, align=flush left, xshift=0pt},
				chatbox_gpt_inner/.style={chatbox_inner, align=flush left, xshift=0pt},
				chatbox/.style={chatbox_inner, draw=black!25, fill=gray!7, opacity=1, text opacity=0},
				chatbox_prompt/.style={chatbox, align=flush left, fill=gray!1.5, draw=black!30, text height=10pt},
				chatbox_user/.style={chatbox, align=flush left},
				chatbox_gpt/.style={chatbox, align=flush left},
				chatbox2/.style={chatbox_gpt, fill=green!25},
				chatbox3/.style={chatbox_gpt, fill=red!20, draw=black!20},
				chatbox4/.style={chatbox_gpt, fill=yellow!30},
				labelbox/.style={rectangle, rounded corners, draw=black!50, font=\sffamily\scriptsize\bfseries, fill=gray!5, inner sep=3pt},
			]
											
			\node[chatbox_user] (q1) {
				\textbf{System prompt}
				\newline
				\newline
				You are a helpful and precise assistant for segmenting and labeling sentences. We would like to request your help on curating a dataset for entity-level hallucination detection.
				\newline \newline
                We will give you a machine generated biography and a list of checked facts about the biography. Each fact consists of a sentence and a label (True/False). Please do the following process. First, breaking down the biography into words. Second, by referring to the provided list of facts, merging some broken down words in the previous step to form meaningful entities. For example, ``strategic thinking'' should be one entity instead of two. Third, according to the labels in the list of facts, labeling each entity as True or False. Specifically, for facts that share a similar sentence structure (\eg, \textit{``He was born on Mach 9, 1941.''} (\texttt{True}) and \textit{``He was born in Ramos Mejia.''} (\texttt{False})), please first assign labels to entities that differ across atomic facts. For example, first labeling ``Mach 9, 1941'' (\texttt{True}) and ``Ramos Mejia'' (\texttt{False}) in the above case. For those entities that are the same across atomic facts (\eg, ``was born'') or are neutral (\eg, ``he,'' ``in,'' and ``on''), please label them as \texttt{True}. For the cases that there is no atomic fact that shares the same sentence structure, please identify the most informative entities in the sentence and label them with the same label as the atomic fact while treating the rest of the entities as \texttt{True}. In the end, output the entities and labels in the following format:
                \begin{itemize}[nosep]
                    \item Entity 1 (Label 1)
                    \item Entity 2 (Label 2)
                    \item ...
                    \item Entity N (Label N)
                \end{itemize}
                % \newline \newline
                Here are two examples:
                \newline\newline
                \textbf{[Example 1]}
                \newline
                [The start of the biography]
                \newline
                \textcolor{titlecolor}{Marianne McAndrew is an American actress and singer, born on November 21, 1942, in Cleveland, Ohio. She began her acting career in the late 1960s, appearing in various television shows and films.}
                \newline
                [The end of the biography]
                \newline \newline
                [The start of the list of checked facts]
                \newline
                \textcolor{anscolor}{[Marianne McAndrew is an American. (False); Marianne McAndrew is an actress. (True); Marianne McAndrew is a singer. (False); Marianne McAndrew was born on November 21, 1942. (False); Marianne McAndrew was born in Cleveland, Ohio. (False); She began her acting career in the late 1960s. (True); She has appeared in various television shows. (True); She has appeared in various films. (True)]}
                \newline
                [The end of the list of checked facts]
                \newline \newline
                [The start of the ideal output]
                \newline
                \textcolor{labelcolor}{[Marianne McAndrew (True); is (True); an (True); American (False); actress (True); and (True); singer (False); , (True); born (True); on (True); November 21, 1942 (False); , (True); in (True); Cleveland, Ohio (False); . (True); She (True); began (True); her (True); acting career (True); in (True); the late 1960s (True); , (True); appearing (True); in (True); various (True); television shows (True); and (True); films (True); . (True)]}
                \newline
                [The end of the ideal output]
				\newline \newline
                \textbf{[Example 2]}
                \newline
                [The start of the biography]
                \newline
                \textcolor{titlecolor}{Doug Sheehan is an American actor who was born on April 27, 1949, in Santa Monica, California. He is best known for his roles in soap operas, including his portrayal of Joe Kelly on ``General Hospital'' and Ben Gibson on ``Knots Landing.''}
                \newline
                [The end of the biography]
                \newline \newline
                [The start of the list of checked facts]
                \newline
                \textcolor{anscolor}{[Doug Sheehan is an American. (True); Doug Sheehan is an actor. (True); Doug Sheehan was born on April 27, 1949. (True); Doug Sheehan was born in Santa Monica, California. (False); He is best known for his roles in soap operas. (True); He portrayed Joe Kelly. (True); Joe Kelly was in General Hospital. (True); General Hospital is a soap opera. (True); He portrayed Ben Gibson. (True); Ben Gibson was in Knots Landing. (True); Knots Landing is a soap opera. (True)]}
                \newline
                [The end of the list of checked facts]
                \newline \newline
                [The start of the ideal output]
                \newline
                \textcolor{labelcolor}{[Doug Sheehan (True); is (True); an (True); American (True); actor (True); who (True); was born (True); on (True); April 27, 1949 (True); in (True); Santa Monica, California (False); . (True); He (True); is (True); best known (True); for (True); his roles in soap operas (True); , (True); including (True); in (True); his portrayal (True); of (True); Joe Kelly (True); on (True); ``General Hospital'' (True); and (True); Ben Gibson (True); on (True); ``Knots Landing.'' (True)]}
                \newline
                [The end of the ideal output]
				\newline \newline
				\textbf{User prompt}
				\newline
				\newline
				[The start of the biography]
				\newline
				\textcolor{magenta}{\texttt{\{BIOGRAPHY\}}}
				\newline
				[The ebd of the biography]
				\newline \newline
				[The start of the list of checked facts]
				\newline
				\textcolor{magenta}{\texttt{\{LIST OF CHECKED FACTS\}}}
				\newline
				[The end of the list of checked facts]
			};
			\node[chatbox_user_inner] (q1_text) at (q1) {
				\textbf{System prompt}
				\newline
				\newline
				You are a helpful and precise assistant for segmenting and labeling sentences. We would like to request your help on curating a dataset for entity-level hallucination detection.
				\newline \newline
                We will give you a machine generated biography and a list of checked facts about the biography. Each fact consists of a sentence and a label (True/False). Please do the following process. First, breaking down the biography into words. Second, by referring to the provided list of facts, merging some broken down words in the previous step to form meaningful entities. For example, ``strategic thinking'' should be one entity instead of two. Third, according to the labels in the list of facts, labeling each entity as True or False. Specifically, for facts that share a similar sentence structure (\eg, \textit{``He was born on Mach 9, 1941.''} (\texttt{True}) and \textit{``He was born in Ramos Mejia.''} (\texttt{False})), please first assign labels to entities that differ across atomic facts. For example, first labeling ``Mach 9, 1941'' (\texttt{True}) and ``Ramos Mejia'' (\texttt{False}) in the above case. For those entities that are the same across atomic facts (\eg, ``was born'') or are neutral (\eg, ``he,'' ``in,'' and ``on''), please label them as \texttt{True}. For the cases that there is no atomic fact that shares the same sentence structure, please identify the most informative entities in the sentence and label them with the same label as the atomic fact while treating the rest of the entities as \texttt{True}. In the end, output the entities and labels in the following format:
                \begin{itemize}[nosep]
                    \item Entity 1 (Label 1)
                    \item Entity 2 (Label 2)
                    \item ...
                    \item Entity N (Label N)
                \end{itemize}
                % \newline \newline
                Here are two examples:
                \newline\newline
                \textbf{[Example 1]}
                \newline
                [The start of the biography]
                \newline
                \textcolor{titlecolor}{Marianne McAndrew is an American actress and singer, born on November 21, 1942, in Cleveland, Ohio. She began her acting career in the late 1960s, appearing in various television shows and films.}
                \newline
                [The end of the biography]
                \newline \newline
                [The start of the list of checked facts]
                \newline
                \textcolor{anscolor}{[Marianne McAndrew is an American. (False); Marianne McAndrew is an actress. (True); Marianne McAndrew is a singer. (False); Marianne McAndrew was born on November 21, 1942. (False); Marianne McAndrew was born in Cleveland, Ohio. (False); She began her acting career in the late 1960s. (True); She has appeared in various television shows. (True); She has appeared in various films. (True)]}
                \newline
                [The end of the list of checked facts]
                \newline \newline
                [The start of the ideal output]
                \newline
                \textcolor{labelcolor}{[Marianne McAndrew (True); is (True); an (True); American (False); actress (True); and (True); singer (False); , (True); born (True); on (True); November 21, 1942 (False); , (True); in (True); Cleveland, Ohio (False); . (True); She (True); began (True); her (True); acting career (True); in (True); the late 1960s (True); , (True); appearing (True); in (True); various (True); television shows (True); and (True); films (True); . (True)]}
                \newline
                [The end of the ideal output]
				\newline \newline
                \textbf{[Example 2]}
                \newline
                [The start of the biography]
                \newline
                \textcolor{titlecolor}{Doug Sheehan is an American actor who was born on April 27, 1949, in Santa Monica, California. He is best known for his roles in soap operas, including his portrayal of Joe Kelly on ``General Hospital'' and Ben Gibson on ``Knots Landing.''}
                \newline
                [The end of the biography]
                \newline \newline
                [The start of the list of checked facts]
                \newline
                \textcolor{anscolor}{[Doug Sheehan is an American. (True); Doug Sheehan is an actor. (True); Doug Sheehan was born on April 27, 1949. (True); Doug Sheehan was born in Santa Monica, California. (False); He is best known for his roles in soap operas. (True); He portrayed Joe Kelly. (True); Joe Kelly was in General Hospital. (True); General Hospital is a soap opera. (True); He portrayed Ben Gibson. (True); Ben Gibson was in Knots Landing. (True); Knots Landing is a soap opera. (True)]}
                \newline
                [The end of the list of checked facts]
                \newline \newline
                [The start of the ideal output]
                \newline
                \textcolor{labelcolor}{[Doug Sheehan (True); is (True); an (True); American (True); actor (True); who (True); was born (True); on (True); April 27, 1949 (True); in (True); Santa Monica, California (False); . (True); He (True); is (True); best known (True); for (True); his roles in soap operas (True); , (True); including (True); in (True); his portrayal (True); of (True); Joe Kelly (True); on (True); ``General Hospital'' (True); and (True); Ben Gibson (True); on (True); ``Knots Landing.'' (True)]}
                \newline
                [The end of the ideal output]
				\newline \newline
				\textbf{User prompt}
				\newline
				\newline
				[The start of the biography]
				\newline
				\textcolor{magenta}{\texttt{\{BIOGRAPHY\}}}
				\newline
				[The ebd of the biography]
				\newline \newline
				[The start of the list of checked facts]
				\newline
				\textcolor{magenta}{\texttt{\{LIST OF CHECKED FACTS\}}}
				\newline
				[The end of the list of checked facts]
			};
		\end{tikzpicture}
        \caption{GPT-4o prompt for labeling hallucinated entities.}\label{tb:gpt-4-prompt}
	\end{center}
\vspace{-0cm}
\end{table*}
\begin{table*}

\centering

\begin{tabular}{|p{\textwidth}|}
\hline
\\ [2pt]
\par Here is a statement and a corresponding piece of reference text. Please complete the task as follows, strictly following the format I have given for the output:
\par (1) Find all the original key passages in the reference text that directly support the information in the statement (there may be more than one, find each one). Output one original key passage per line and the information in the statement it directly supports in the format “Key passage {number}: {key passage} (information in the supporting statement: {supporting information})”.
\par (2) Please group key passages, each group contains key passages supporting the same or related information in the statement, output one line of the grouping results in the format of “Key passage grouping: Group 1: (first group of key passage numbers), Group 2: (second group of key passage numbers) ...”. For example, if there are 2 pieces of information in the statement, key paragraph 1 supports information 1, key paragraph 2 supports information 2, and key paragraph 3 supports information 1, then the output is “Key Paragraph Grouping: Group 1: (1, 3), Group 2: (2)”.
\par (3) Select a group of key text segments and modify the parts of them that support the information in the statement to meet the following requirements:
\par - The modification should make it impossible for the key passage to fully support the corresponding information in the statement.
\par - The modifications should maintain the logical flow of the key passages and no contradictions between the information in the key passages.
\par - The modification should keep the key paragraph logically coherent in the context of the reference text and not contradict the rest of the reference text.
\par - Modify only the parts that support the information in a statement, leaving the rest unchanged.
\par - If there is more than one key passage in a set, the information in them should remain consistent after revision.
\par You need to try two methods of modification:
\par - Changing the message: modifying the message in one part of the key paragraph to another. Do not make changes that directly conflict with the original information. For example, if the original message is “The Audi A7 Signature Edition has a faster top speed than its predecessor”, an appropriate change would be “The Audi A7 Luxury Edition has a faster top speed than its predecessor”, and an inappropriate change1 would be “The Audi A7 Signature Edition has a slower top speed than its predecessor” (using an antonym, which is in direct conflict with the original message), and inappropriate modification 2 is ‘The top speed of the Audi A7 Signature Edition is not faster than the previous generation’ (adding a negative word, which is in direct conflict with the original message).
\par - Delete Information: Remove information from a place in a key paragraph. If the key paragraph is a complete sentence, it should still be a complete sentence after deleting the information. For example, if the original paragraph reads “Due to weather conditions, the project was delayed until March 15” (complete sentence), an appropriate change would be “Due to weather conditions, the project was delayed until March” (still a complete sentence), an inappropriate change would be “Due to the weather” (no longer a complete sentence).
\par For each method, output the key passage that was modified and check its logical fluency, giving an integer within 1 to 10 as a rating (higher means more fluent). Output one modified key passage per line in the format “{method}-modified key passage {number}: {modified key passage} (logical fluency: {score})”.
\\ [5pt]
\\ [5pt]
\hline

\end{tabular}

\caption{\label{tab:prompt_en} The complete prompt for the LLM augmentation (translated into English).}
\end{table*}

See Table~\ref{tab:prompt} for the prompt for LLM augmentation. Table~\ref{tab:prompt_en} provides an English version.

\section{Instructions for Annotators}
\label{app:ann}
\subsection{First Stage}
In the first stage, we provide the annotators with the question, answer, statement, and cited documents. What LLM considers to be key segments are highlighted in red in the cited documents (see Figure~\ref{fig:stage} for an example). We instruct the annotators to follow the process below:

\par (1) First look at the highlighted text. If the highlighted text fully supports the statement, then the annotation is positive; if the highlighted text contradicts the statement, then the annotation is negative.

\par (2) If the annotation cannot be derived from the highlighted text, then look at the rest of the documents to make the annotation. When the documents fully support the statement, the label is positive, and when there is any information in the statement that contradicts the documents or information that is not mentioned in the documents, the label is negative.

\subsection{Second Stage}

In the second stage, we provide the annotator with the statement and the modified documents. In the documents, the modified parts are highlighted in green, where the dashed and crossed-out text is deleted and the rest is added (see Figure~\ref{fig:stage} for examples). 

For the annotation of whether the quality of the modification is acceptable, the annotators are instructed to note that qualified modifications need to satisfy the following two requirements: (1) There are no contradictions within each modified document. (2) The modified key segments are fluent in their own right and in the context of the document. The annotation for support is the same as the first stage, but based on the modified documents.

 
\section{Input and Training Details}
\label{app:detail}
We input the statement and the cited documents into the model and ask the model to determine whether the statement is fully supported by the documents, outputting yes or no. For input, we label and concatenate the cited documents in order (as shown in Table~\ref{tab:dataset}). For training, we use the following settings: For training, we use the following settings: learning rate is 5e-4, number of epochs is 10, scheduler is cosine scheduler, warmup ratio is 0.03, batch size is 256, and LoRA setting is $r=8$, $a=32$ and 0.1 dropout. We report the model performance for the epoch that achieves the best performance on the dev set.
\label{app:detail}



\section{Related Works}
Language models are known to produce hallucinations - statements that are inaccurate or unfounded~\citep{MaynezNBM20,HuCLGWYG24}. To address this limitation, recent research has focused on augmenting LLMs with external tools such as retrievers~\citep{GuuLTPC20,BorgeaudMHCRM0L22,LiuCtrla2024} and search engines~\citep{WebGPT2021, Komeili0W22, TanGSXLFLWSLS24}. While this approach suggests that generated content is supported by external references, the reliability of such attribution requires careful examination. Recent studies have investigated the validity of these attributions. \citet{DBLP:conf/emnlp/LiuZL23} conducted human evaluations to assess the verifiability of responses from generative search engines. \citet{hu2024evaluate} further investigate the reliability of such attributions when giving adversarial questions to RAG systems. Their findings revealed frequent occurrences of unsupported statements and inaccurate citations, highlighting the need for rigorous attribution verification~\citep{RashkinNLA00PTT23}. However, human evaluation processes are resource-intensive and time-consuming. To overcome these limitations, existing efforts~\citep{GaoDPCCFZLLJG23,DBLP:conf/emnlp/GaoYYC23} proposed an automated approach using Natural Language Inference models to evaluate attribution accuracy. While several English-language benchmarks have been developed for this purpose~\citep{DBLP:conf/emnlp/YueWCZS023}, comparable resources in Chinese are notably lacking. Creating such datasets presents unique challenges, particularly in generating realistic negative samples (unsupported citations).  To address this gap, we introduce the first large-scale Chinese dataset for citation faithfulness detection, developed through a cost-effective two-stage manual annotation process.

\begin{figure*}
    \centering
    \includegraphics[width=0.3\textwidth]{appendix/s1.png}
    \includegraphics[width=0.3\textwidth]{appendix/s2-m.png}
    \includegraphics[width=0.3\textwidth]{appendix/s2-d.png}
    \caption{Examples of interfaces that provide samples to the annotators. The first figure shows an example of the first stage. The last two images show the second stage with the same sample modified (information changed/deleted).}
    \label{fig:stage}
\end{figure*}



\end{document}
