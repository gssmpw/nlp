\documentclass[11pt]{article}
\usepackage[top=1in, bottom=1in, left=1in, right=1in]{geometry}
\usepackage{tgpagella}
\usepackage[utf8]{inputenc} % allow utf-8 input
\usepackage[T1]{fontenc}    % use 8-bit T1 fonts

\input{macro_2}
\usepackage{authblk}
\usepackage{booktabs}
\usepackage{cleveref}
\usepackage{cite}
\usepackage{natbib}
\usepackage{mathtools}
\usepackage[toc, page]{appendix}
\usepackage{wrapfig}
\definecolor{c1}{HTML}{A7BEAE}
\definecolor{c2}{HTML}{B85042}

\def\MoG{\texttt{MoG}}
\def\MoLRG{\texttt{MoLRG}}
\def\CSNR{$\mathrm{CSNR}$}
\def\SWD{$\mathrm{SWD}$}

% \title{Understanding Diffusion-based Representation Learning via Low-Dimensional Modeling}
\title{Understanding Representation Dynamics of Diffusion Models via Low-Dimensional Modeling}

\author[$\Diamond$]{Xiao Li\thanks{The first two authors contributed equally to the work.}}
\author[$\Diamond$]{Zekai Zhang\samethanks}
\author[$\Diamond$]{Xiang Li}
\author[$\Diamond$]{Siyi Chen}
\author[$\dagger$]{Zhihui Zhu}
\author[$\Diamond$]{Peng Wang}
\author[$\Diamond$]{\\ Qing Qu}


\affil[$\Diamond$]{Department of Electrical Engineering and Computer Science, University of Michigan}
\affil[$\dagger$]{Department of Computer Science \& Engineering, Ohio State University}

\date{}

\date{\today}

\begin{document}

\maketitle


\begin{abstract}
This work addresses the critical question of why and when diffusion models, despite being designed for generative tasks, can excel at learning high-quality representations in a self-supervised manner. To address this, we develop a mathematical framework based on a low-dimensional data model and posterior estimation, revealing a fundamental trade-off between generation and representation quality near the final stage of image generation. Our analysis explains the unimodal representation dynamics across noise scales, mainly driven by the interplay between data denoising and class specification. Building on these insights, we propose an ensemble method that aggregates features across noise levels, significantly improving both clean performance and robustness under label noise. Extensive experiments on both synthetic and real-world datasets validate our findings.

\end{abstract}

\section{Introduction}\label{sec:intro}
\section{Introduction}
\label{sec:intro}

\begin{figure*}[tb]
    \centering
    \includegraphics[width=0.848\linewidth]{figs/circuitnn.pdf} 
    \caption{Illustration of differentiable CircuitNN. CircuitNN is designed based on differentiable NAND gates. After DAS is guided by PI and PO pairs of the truth table, CircuitNN can get the precise circuit architecture logic equivalent to the truth table.}
    \label{fig:circuitnn}
\end{figure*}

% 1. Describe the importance of logic synthesis
% 2. Existing Problems
% (a) Neural Architecture Search: Unstable, Predefined Setting, etc.
% (b) Circuit Generation: Probabilistic Model, Logic Equivalence

With the rapid advancement of technology, the scale of integrated circuits (ICs) has expanded exponentially. 
This expansion has introduced significant challenges in chip manufacturing, particularly concerning power and area metrics.
A primary objective in IC design is achieving the same circuit function with fewer transistors, thereby reducing power usage and area occupancy.

Logic synthesis~\cite{hachtel2005logicsynth}, a critical step in electronic design automation (EDA), transforms behavioral-level circuit designs into optimized gate-level circuits, ultimately yielding the final IC layout. 
The primary goal of logic synthesis is to identify the physical implementation with the fewest gates for a given circuit function. 
This task constitutes a challenging NP-hard combinatorial optimization problem. 
Current logic synthesis tools~\cite{brayton2010abc, wolf2013yosys} rely on human-designed heuristics, often leading to sub-optimal outcomes.

Differentiable architecture search (DAS) techniques~\cite{liu2018darts, chu2020darts} offer novel perspectives on addressing challenges in this problem.
Circuit functions can be represented through truth tables, which map binary inputs to their corresponding outputs. 
Truth tables provide a precise representation of input-output relationships, ensuring the design of functionally equivalent circuits.
Inspired by this, researchers~\cite{deepmind2024ai4sys, wang2024tnet} have begun exploring the application of DAS to synthesize circuits directly from truth tables.
Specifically, \citet{deepmind2024ai4sys} proposed CircuitNN, a framework that learns differentiable connection structures with logic gates, enabling the automatic generation of logic circuits from truth tables.
This approach significantly reduces the complexity of traditional circuit generation. 
Building on this, \citet{wang2024tnet} introduced T-Net, a triangle-shaped variant of CircuitNN, incorporating regularization techniques to enhance the efficiency of DAS.

Despite these advancements, several challenges remain. 
The computational complexity of DAS grows quadratically with the number of gates, posing scalability issues.
Although triangle-shaped architecture~\cite{wang2024tnet} partially mitigates this problem, redundancy persists. 
%Additionally, DAS is susceptible to converging to local optima, limiting the ability to search architectures that satisfy the given truth tables~\cite{liu2018darts}. 
%Furthermore, hyperparameters (network depth and layer width) require extensive searches, introducing complexity and prolonging the synthesis process. 
Additionally, DAS is susceptible to converging to local optima~\cite{liu2018darts} and hyperparameters (network depth and layer width) require extensive searches. 
The challenges arise from the vast search space in DAS. 
% Even with predefined settings for CircuitNN, finding a configuration that meets the truth table requires extensive trial and error during the DAS process. 
Intuitively, limiting the search space through predefined parameters (network depth, gates per layer, and connection probabilities) can significantly reduce the complexity.

Recent advances~\cite{openai2023gpt4, abramson2024alphafold3, esser2024sd3, li2024mar} in conditional generative models have demonstrated remarkable performance across language, vision, and graph generation tasks. 
Motivated by these developments, we propose a novel approach to circuit generation that generates preliminary circuit structures to guide DAS in generating refined circuits matching specified truth tables. 
Firstly, we introduce CircuitVQ, a tokenizer with a discrete codebook for circuit tokenization. 
Built upon our Circuit AutoEncoder framework~\cite{hou2022graphmae,li2023maskgae,wu2025mgvga}, CircuitVQ is trained through a circuit reconstruction task. 
Specifically, the CircuitVQ encoder encodes input circuits into discrete tokens using a learnable codebook, while the decoder reconstructs the circuit adjacency matrix based on these tokens.
Subsequently, the CircuitVQ encoder serves as a circuit tokenizer for CircuitAR pretraining, which employs a masked autoregressive modeling paradigm~\cite{chang2022maskgit, li2023mage}. 
In this process, the discrete codes function as supervision signals. 
After training, CircuitAR can generate discrete tokens progressively, which can be decoded into initial circuit structures by the decoder of the CircuitVQ. 
These prior insights can guide DAS in producing refined circuits that match the target truth tables precisely.

Our key contributions can be summarized as follows:
\begin{itemize}
\item We introduce CircuitVQ, a circuit tokenizer that facilitates graph autoregressive modeling for circuit generation, based on our Circuit AutoEncoder framework;
\item Develop CircuitAR, a model trained using masked autoregressive modeling, which generates initial circuit structures conditioned on given truth tables;
\item Propose a refinement framework that integrates differentiable architecture search to produce functionally equivalent circuits guided by target truth tables;
\item Comprehensive experiments demonstrating the scalability and capability emergence of our CircuitAR and the superior performance of the proposed circuit generation approach.
\end{itemize}

% Motivation
% (a) Diffusion (Vision, Graph), Autoregressive (Language, Vision)
% (b) Circuit Generation for Predefined Setting
% (c) Neural Architecture Search for Strict Logic Equivalence

% Contribution
% (a) Circuit Tokenizer (new transformer arch, training strategy)
% (b) CircuitAR (train and gen strategies, post-ar strategy)
% (c) Extensive Evaluation including BitD (Bit Distance) for Scalability


\section{Problem Setup}\label{sec:problem}
\subsection{Problem Formulation}

% We begin by formulating the problem of dynamic benchmarking for LLMs.
A dynamic benchmark is defined as  
$
\small
\mathcal{B}_{\text{dynamic}} = (\mathcal{D}, T(\cdot)), \quad 
\mathcal{D} = (\mathcal{X}, \mathcal{Y}, \mathcal{S}(\cdot))
$
where \( \mathcal{D} \) represents the static benchmark dataset. 
% consisting of input prompts \( \mathcal{X} \), expected outputs \( \mathcal{Y} \), and a scoring function \( \mathcal{S}(\cdot) \) that evaluates the quality of an LLM's outputs by comparing them against \( \mathcal{Y} \). 
The transformation function \( T(\cdot) \) modifies the data set during the benchmarking to avoid possible data contamination.
The dynamic dataset for the evaluation of an LLM can then be expressed as
$
\small
        \mathcal{D}_t = T_t(\mathcal{D}),  \quad
        \forall t \in \{1, \dots, N\}
$
where \( \mathcal{D}_t \) represents the evaluation data set at the timestamp \( t \), and \( N\) is the total timestamp number, which could be finite or infinite. % \ie $N= \infty$.
If the seed dataset $\mathcal{D}$ is empty, the dynamic benchmarking dataset will be created from scratch.



\section{Study of Representation Dynamics}\label{sec:main}
% In \Cref{sec:main}, we analyzed diffusion representation dynamics with a focus on the denoising process, assuming sufficient training data for learning the underlying distribution. In this section, we explore the impact of the diffusion process (\Cref{subsec:weight_share}) and data complexity (\Cref{subsec:mem_gen}) in shaping diffusion models' representation learning dynamics.




With the setup in \Cref{sec:problem}, this section theoretically investigates the representation dynamics of diffusion models across the noise levels, providing new insights for understanding the representation-generation tradeoff. Moreover, our theoretical studies are corroborated by experimental results on real datasets.

%{\color{blue} In this section, we investigate the representation dynamics of diffusion models, focusing on the intriguing unimodal trend in their representation learning performance across varying noise levels. To uncover the underlying reasons for this behavior, we conduct a theoretical analysis of the posterior estimation quality, $\E[\bm{x}_0 \mid \bm{x}_t]$, in low-dimensional distributions.}





% \vspace{-0.2in}
\subsection{Assumptions of Low-Dimensional Data Distribution}\label{subsec:model}

\begin{wrapfigure}[14]{R}{0.44\textwidth}
% \begin{figure}[t]
    \vspace{-0.1in}
    \begin{center}
    % \includegraphics[width=0.38\textwidth]{figs/molrg_illustration.pdf} % Replace with your image
    \includegraphics[width=0.4\textwidth]{figs/molrg_illustration_v2.pdf}
    \end{center}
    \vspace{-0.25in}
    \caption{\textbf{An illustration of \MoLRG\;with different noise levels.} We visualize samples drawn from noisy~\MoLRG~with noise levels $\delta = 0.1,\;0.3$ and $K=3$.}
    \label{fig:sample}
    \vspace{-0.1in}
% \end{figure}
\end{wrapfigure}

% \ZK{Refer to latent diffusion? The latent space seems more likely to have a low-dimensional subsuapce structure.}

In this work, we assume that the input data follows a noisy version of the mixture of low-rank Gaussians (\MoLRG) distribution \citep{wang2024diffusion,elhamifar2013sparse, wang2022convergence}, defined as follows.


%\zk{K-subspace or K-class? Need consistency}
\begin{assum}[$K$-Class Noisy \MoLRG~Distribution]\label{assum:subspace}
\emph{For any sample $\mb x_0$ drawn from the noisy \MoLRG~distribution with $K$ subspaces, the following holds: }
% \qq{edit the equation}
\begin{align}\label{eq:MoG noise}
    \bm x_0 = \bm U_k \bm a + \delta \bm U_k^{\perp} \bm e,\;\text{with prob.}\;\pi_k \geq 0,\; k \in [K].
\end{align}
\emph{Here, $k$ represents the class of $\bm x_0$ and follows a multinomial distribution $k \sim \text{Mult}(K,\pi_k)$, $\bm U_k \in \mathcal{O}^{n \times d_k}$ denotes an orthonormal basis for the $k$-th subspace with its complement $\mb U_k^\perp \in \mathcal{O}^{n \times (n-d_k)}$, $d_k$ is the subspace dimension with $d_k \ll n$, and the coefficient $\bm a \overset{i.i.d.}{\sim} \mathcal{N}(\bm 0, \bm I_{d_k})$ is drawn from the normal distribution. The level of the noise $\bm e \overset{i.i.d.}{\sim} \mathcal{N}(\bm 0, \bm I_{n-d_k})$ is controlled by the scalar $\delta < 1$. }
% $\sum_{k=1}^K \pi_k=1$%Additionally, $\mb U_k^\perp \in \mathcal{O}^{n \times (n-d_k)}$ is the orthogonal complement of $\mb U_k$.
%Here, $\sum_{k=1}^K \pi_k=1$, $\bm U_k \in \mathcal{O}^{n \times d_k}$ denotes an orthonormal basis for the $k$-th subspace, $d_k$ is the subspace dimension with $d_k \ll n$, and the coefficient $\bm a \overset{i.i.d.}{\sim} \mathcal{N}(\bm 0, \bm I_{d_k})$ is drawn from a standard normal distribution \qq{normal means standard Gaussian, remove standard}. For the noise, we assume $\bm e \overset{i.i.d.}{\sim} \mathcal{N}(\bm 0, \bm I_{n-d_k})$ with magnitude controlled by the scalar $\delta < 1$. Additionally, $\mb U_k^\perp \in \mathcal{O}^{n \times (n-d_k)}$ is the orthogonal compliment of $\mb U_k$. 
\end{assum}

As shown in \Cref{fig:sample}, data from \MoLRG~resides on a union of low-dimensional subspaces, each following a Gaussian distribution with a low-rank covariance matrix representing its basis. The study of Noisy \MoLRG\; distributions is further motivated by the fact that

%The data drawn from \MoLRG~lie on a union of low-dimensional subspaces. Within each subspace, the data follows a Gaussian distribution with a low-rank covariance matrix that represents the subspace basis. Moreover, the study of the Noisy \MoLRG\; distributions is motivated by the facts that
\begin{itemize}[leftmargin=*]
    \vspace{-0.1in}
    \item \emph{\MoLRG\;captures the intrinsic low-dimensionality of image data.} Although real-world image datasets are high-dimensional in terms of pixel count and data volume, extensive empirical studies \citep{gong2019intrinsic,pope2021intrinsic,stanczuk2022your} demonstrated that their intrinsic dimensionality is considerably lower. Additionally, recent work \citep{huang2024denoising,liang2024low} has leveraged the intrinsic low-dimensional structure of real-world data to analyze the convergence guarantees of diffusion model sampling. The \MoLRG~distribution, which models data in a low-dimensional space with rank $d_k \ll n$, effectively captures this property.
    % \qq{add explanation why MolRG captures the low-dimensionality here}
    \item \emph{The latent space of latent diffusion models is approximately Gaussian.} State-of-the-art large-scale diffusion models \citep{peebles2023scalable, podell2023sdxl} typically employ autoencoders \citep{kingma2013auto} to project images into a low-dimensional latent space, where a KL penalty encourages the learned latent distribution to approximate standard Gaussians \citep{rombach2022high}. Furthermore, recent studies \citep{jing2022subspace, chen2024deconstructing} show that diffusion models can be trained to leverage the intrinsic subspace structure of real-world data.
    % \qq{add: In their training loss, Gaussianity is typically enforced for training the encoder.}\xiao{this is not always true, better be integrated with the previous point}
    
    % \item \emph{Modeling the visual details of real-world image datasets.}\zk{I will shrink} The noise term $\delta \bm U_k^{\perp} \bm e_i$ captures perturbations outside the $k$-th subspace via the orthogonal complement $\bm U_k^{\perp}$, aligning the model with real-world scenarios. These perturbations represent attributes irrelevant to the subspace, such as the background in a bird image or the color and texture of a car. While this additional noise term may be less significant for representation learning, it plays a crucial role in enabling diffusion models to generate high-fidelity samples.
    \item \emph{Modeling the complexity of real-world image datasets.} The noise term $\delta \bm U_k^{\perp} \bm e_i$ captures perturbations outside the $k$-th subspace via the orthogonal complement $\bm U_k^{\perp}$, analogous to insignificant attributes of real-world images, such as the background of an image. While this additional noise term may be less significant for representation learning, it plays a crucial role in enhancing the fidelity of generated samples. %\qq{I feel here there is a bit overstatement. Fine-details are of high frequency, but noise is not. We need to be careful, not mention visual details.}
    % \zk{we can mention the connections with Difan Zou's data assumptions here}
\end{itemize}

Moreover, the noisy \MoLRG\; is analytically tractable. For simplicity, we assume equal subspace dimensions ($d_1 = \dots = d_K = d$), orthogonal bases ($\bm U_k^{T} \bm U_l = \bm 0$ for $k \neq l$), uniform mixing weights ($\pi_1 = \dots = \pi_K = 1/K$), and define the noise space as $\mb U_{\perp} = \bigcap_{k=1}^K \bm U_k^{\perp} \in \mathcal{O}^{n \times (n-Kd)}$. Then, we can derive the ground truth posterior mean $\E\left[\bm x_0 \vert \bm x_t\right]$ for the noisy \MoLRG\; distribution as:


\begin{figure*}[t]
    \begin{center}
    \includegraphics[width = 0.9\textwidth]{figs/Fig1_teaser_zekai.pdf}
    % {figs/teaser_final.pdf}
    \end{center}
\vspace{-0.1in}
\caption{\textbf{Trade-offs between representation quality and generation quality.} The {\color{c1} curve with pentagon markers} demonstrates the transition from fine to coarse granularity in posterior estimation as noise levels increase, corresponding to the monotonic rise in FID. In contrast, the {\color{c2} curve with square markers} reveals an unimodal trend in posterior classification accuracy, achieving peak performance at intermediate noise levels. This occurs when high-level details are filtered out while essential low-level semantic information is preserved, as illustrated by the posterior estimations according to different noise levels shown at the bottom of the figure.}
\label{fig:use_clean}
\end{figure*}

\begin{prop}\label{lem:E[x_0]_multi}
Suppose the data $\bm x_0$ is drawn from a noisy \MoLRG~data distribution with $K$-class and noise level $\delta$. Let  $\zeta_t = \frac{1}{1 + \sigma_t^2}$ and $\xi_t = \frac{\delta^2}{\delta^2 + \sigma_t^2}$, where $\sigma_t$ is the noise scaling in \eqref{eq:Tweedie}. Then for each time $t > 0$, the optimal posterior estimator $\E\left[\bm x_0 \vert \bm x_t\right]$ has the analytical form: 
\begin{align*}
    \E\left[ \bm x_0 \vert \bm x_t\right] = \sum_{l=1}^K w^{\star}_l(\bm x_t,t) \left( \zeta_t \bm U_l\bm U_l^T + \xi_t \bm U_l^{\perp}\bm U_l^{\perp T} \right) \bm x_t.
\end{align*}
where $w^{\star}_l(\bm x_t,t) = \frac{\exp\left(g_l(\bm x_t, t) \right)}{\sum_{l=1}^K \exp\left(g_l(\bm x_t, t) \right)}$ is a soft-max operator for $g_l(\bm x,t) = \frac{1}{2\sigma_t^2}\zeta_t \|\bm U_l^T \bm x\|^2 + \frac{\delta^2}{2 \sigma_t^2}\xi_t \| \bm U_l^{\perp T} \bm x \|^2 $.
%    \begin{align}
%         \ &w^{\star}_l(\bm x_t,t) := \frac{\exp\left(g_l(\bm x_t, t) \right)}{\sum_{l=1}^K \exp\left(g_l(\bm x_t, t) \right)}, \label{eq:w-k} \\
%           \ &g_l(\bm x,t) = \frac{1}{2\sigma_t^2}\zeta_t \|\bm U_l^T \bm x\|^2 + \frac{\delta^2}{2 \sigma_t^2}\xi_t \| \bm U_l^{\perp T} \bm x \|^2 . \label{eq:softmax}
%    \end{align}
\end{prop}
%    \begin{align}\label{eq:E_MoG}
%        \begin{split}
%        &\hat{\bm x}_{\bm \theta}^{\star}(\bm x_t, t) := \E\left[ \hat{\bm x_0}\vert \bm x_t\right] \\
%        &\quad =\sum_{l=1}^K w^{\star}_l(\bm x_t,t) \left( \zeta_t \bm U_l\bm U_l^T + \xi_t \bm U_l^{\perp}\bm U_l^{\perp T} \right) \bm x_t 
%        \end{split}
%    \end{align}
%\qq{I feel the discussion on generation quality is unnecessary and can be deferred if possible. We should discuss why we can use the posterior estimation as an indicator of representation quality here instead.}
%\paragraph{Remark.}
The proof can be found in \Cref{app:proof_prop} and it is an extension of the result in \citep{wang2024diffusion}. For $\bm x_0$ following noisy \MoLRG, note that the optimal solution $\hat{\bm x}_{\bm \theta}^{\star}(\bm x_t, t) $ of the training loss \eqref{eq:dae_loss} would exactly be $ \E\left[\bm x_0 \vert \bm x_t\right]$. As such, as illustrated in \Cref{fig:use_clean}, the analytical form of the posterior estimation facilitates the study of generation-representation tradeoff across timesteps:
%In the above proposition, we present the \textbf{ground truth} posterior estimation function that a diffusion model aims to approximate by minimizing the training objective defined in (\ref{eq:dae_loss}). We denote this optimal model as $\hat{\bm x}_{\bm \theta}^{\star}$. Under this optimal setting, the trade-off between generation and representation learning dynamics can be analyzed by evaluating posterior estimations at different time steps $t$, .
\begin{itemize}[leftmargin=*]
    \item \emph{The generation quality.} The generation quality of posterior estimation cam be measured by $||\hat{\bm x}_{\bm \theta}^{\star}(\bm x_t, t) - \bm x_0||^2$. As shown in \Cref{lem:E[x_0]_multi}, this error is minimized at $t=0$ with $\sigma_t=0$, where the true class weight satisfies $w^{\star}_k(\bm x_t) = 1$, yielding $\hat{\bm x}_{\bm \theta}^{\star}(\bm x_t, t) = \bm x_0$. As $t$ increases, higher noise levels $\sigma_t$ decrease $w^{\star}_k(\bm x_t)$, causing a monotonic increase in FID, as seen in \Cref{fig:use_clean}.
    \item \emph{The representation quality.} The representation quality follows a unimodal trend across timesteps \citep{xiang2023denoising,tang2023emergent}, which can be measured through the posterior estimator $\hat{\bm x}_{\bm \theta}^{\star}(\bm x_t, t)$ (see \Cref{subsec:rep-quality}). As shown in \Cref{fig:use_clean}, this unimodal behavior creates a trade-off between generation and representation quality, particularly at smaller $t$ when closer to the original image.
    % \zk{This seems to overlap with the theorem?}
%    To better understand the tradeoff between generation FID and posterior accuracy as illustrated in \Cref{fig:use_clean}, we analyze the unimodal representation dynamics using the ground truth posterior estimation function. This is done by studying the \CSNR~with this function, which we discuss in more detail in the following section.
    % Similarly, to analyze the unimodal behavior of representation quality, an appropriate evaluation metric is required. We discuss this metric in the following section.
\end{itemize}




% \subsection{Measuring Representation Quality}\label{subsec:rep-quality}

\subsection{Measuring Posterior Representation Quality}\label{subsec:rep-quality}
% \qq{this might be moved to earlier sections} 



%In this work, we use classification tasks to study representation learning, focusing on two types of accuracies: (i) feature accuracy: This refers to the classification accuracy achieved by applying linear probing on extracted feature representations, and (ii) posterior accuracy: Since intermediate representations in diffusion models are a byproduct of the posterior estimation process, we directly evaluate the classification accuracy using the posterior estimations ($\hat{\bm x_0} = \bm x_{\bm \theta}(\bm x_0, t)$) to analyze the trade-off between generation quality and representation quality. 

For understanding diffusion-based representation learning, we introduce a metric termed Class-specific Signal-to-Noise Ratio (\CSNR) to quantify the posterior representation quality as follows.

\begin{definition}[Class-specific Signal-to-Noise Ratio]
\emph{Suppose the data $\bm x_0$ follows the noisy \MoLRG\;introduced in \Cref{assum:subspace}. Without loss of generality, let $k$ denote the true class of $\mb x_0$. For its associated posterior estimator $\hat{\bm x}_{\bm \theta}$,} %we define \CSNR\; as:
\begin{align*}%\label{eq:csnr_true}
    \mathrm{CSNR}(\hat{\bm x}_{\bm \theta},t) := \E_k \left[\frac{\E_{\bm x_0}[\|\bm U_k\bm U_k^T\hat{\bm x}_{\bm \theta}(\bm x_0, t)\|^2 \mid k ] }{\E_{\bm x_0}[\sum_{l\neq k}\|\bm U_l\bm U_l^T\hat{\bm x}_{\bm \theta}(\bm x_0, t)\|^2 \mid k ]}\right]
\end{align*}
\emph{Here, $\bm U_k$ represents the basis of the subspace corresponding to the true class to which $\bm x_0$ belongs and the $\bm U_l$s with $l \neq k$ denotes the bases of the subspaces for other classes. }
\end{definition}

%\paragraph{Posterior representation quality based upon noisy \MoLRG.} 


% \begin{align}\label{eq:csnr_true}
%     \mathrm{CSNR}(t, f) := \frac{\E_{\bm x_0}[\|\hat{\bm U}_k\hat{\bm U}_k^Tf(\bm x_0, t)\|^2]}{\E_{\bm x_0}[\sum_{l\neq k}\|\hat{\bm U}_l\hat{\bm U}_l^Tf(\bm x_0, t)\|^2]}
% \end{align}

Intuitively, successful prediction of the class for $\bm x_0$ is achieved when the projection onto the correct class subspace, $\|\bm U_k\bm U_k^T \hat{\bm x}_{\bm \theta}(\bm x_0, t)\|$, preserves larger energy than the projections onto subspaces of any other class, $\|\bm U_l\bm U_l^T \hat{\bm x}_{\bm \theta}(\bm x_0, t)\|$. Thus, \CSNR~measures the ratio of the true class signal to irrelevant noise from other classes at a given noise level $t$, serving as a practical metric for evaluating classification performance and hence the representation quality. In this work, we use posterior representation quality as a proxy for studying the representation dynamics of diffusion models for the following reasons:
%, rather than directly analyzing feature quality, for the following reasons:
\begin{itemize}[leftmargin=*]
    \item \emph{Posterior quality reflects feature quality.} Diffusion models $\hat{\bm{x}}_{\bm{\theta}}$ are trained to perform posterior estimation at a given time step $t$ using corrupted inputs, with the intermediate features emerging as a byproduct of this process. Thus, a more class-representative posterior estimation inherently implies more class-representative intermediate features.
    \item \emph{Model-agnostic analysis.} Our goal is to provide a general analysis independent of specific network architectures and feature extraction protocols. Posterior representation quality offers a unified metric that avoids assumptions tied to particular architectures, making the analysis broadly applicable.
\end{itemize}
%\textcolor{orange}{[zzk: posterior representation quality?]} 
%\zk{Also we need to note we use intermediate feature and inner representation interchangeably in our paper.}.
% \zk{The representativeness of the posterior reflects representation quality.}
% Moreover, we note that \CSNR~is applicable to any feature extracting function $f$ \zk{that has subspace structures/linearity within its range}, whether it represents feature extraction or posterior estimation. Accordingly, we adopt \CSNR~as a proxy for classification performance in subsequent analyses and experiments.

% Here, $\hat{\bm U}_k$ represents the basis of the subspace corresponding to the true class to which $f(\bm x_0, t)$ belongs and $\hat{\bm U}_l,l \neq k$ denotes the bases of the subspaces for other classes. Intuitively, successful prediction of the class for $\bm x_0$ is achieved when the projection onto the correct class subspace, $\|\hat{\bm U}_k\hat{\bm U}_k^T f(\bm x_0, t)\|$, is greater than the projections onto subspaces of any other class, $\|\hat{\bm U}_l\hat{\bm U}_l^T f(\bm x_0, t)\|$. Thus, \CSNR~measures the ratio of the true class signal to irrelevant noise from other classes at a given noise level $t$, serving as a practical metric for evaluating classification performance. We note that \CSNR~is applicable to any function $f$, whether it represents feature extraction or posterior estimation. Accordingly, we adopt \CSNR~as a proxy for classification performance in subsequent analyses and experiments.

\subsection{Main Theoretical Results}



% \qq{I feel the main theorem is overly long. You need to introduce the CSNR first, build some intutions on what you want to show, and then describe your results. Otherwise, the reviewer does not understand what you want to show.}

%\ZK{need to be consistent}

% As we discussed in \Cref{subsec:posterior}, based upon the strong correlation between representation quality and the posterior mean estimation, we analyze $\hat{\bm x}_{\mb \theta}^{\star}(\bm x_0, t)$ across different time step $t\in[0,1]$. Here, we use $\mb x_0$ as the input instead of $\mb x_t$ according to our discussion in \Cref{subsec:feature-extraction}. 
Based upon the setup in \Cref{subsec:model} and \Cref{subsec:rep-quality}, we obtain the following results.
%Now, we are ready to state our main theorem as follows:

\begin{theorem}\label{lem:main}(Informal)
% Let data $\bm x_0$ be any arbitrary data point drawn from the \MoLRG~distribution defined in Assumption \ref{assum:subspace} and let $k$ denote the true class $\bm x_0$ belongs to. Then \CSNR~introduced in \eqref{eq:csnr_true} depends on the noise level $\sigma_t$ in the following form: 
%     % Without loss of generality, for any clean $\bm x_0$ from class k (i.e., $\bm x_0 = \bm U_k \bm a_i + \delta\bm U_k^{\perp} \bm e_i$), 
%     % \qq{maybe we should write the decomposition out in the following}
%     \begin{align}\label{eq:csnr}
%         \mathrm{CSNR}(t, \hat{\bm x}_{\textit{approx}}^{\star}) = \frac{1}{(K-1)\delta^2} \cdot\left(\frac{1 + \frac{\sigma_t^2}{\delta^2}h(\hat{w}_k, \delta)}{1 + \frac{\sigma_t^2}{\delta^2}h(\hat{w}_l, \delta)}\right)^2
%     \end{align}
% \qq{I feel maybe we should change $b$ to some greek characters} 
% \qq{needs to remind reviewers what is $\sigma_t$ here}
Suppose the data $\bm x_0$ follows the noisy \MoLRG\;introduced in \Cref{assum:subspace} with $K$ classes and noise level $\delta$, then  the \CSNR~of the optimal denoiser $\hat{\bm x}_\theta^{\star}$ takes the following form:
    % Without loss of generality, for any clean $\bm x_0$ from class k (i.e., $\bm x_0 = \bm U_k \bm a_i + \delta\bm U_k^{\perp} \bm e_i$), 
    % \qq{maybe we should write the decomposition out in the following}
    \begin{align}\label{eq:csnr}
        \mathrm{CSNR}(\hat{\bm x}_\theta^{\star},t) = \frac{1}{(K-1)\delta^2}\cdot \left(\frac{1 + \frac{\sigma_t^2}{\delta^2}h(\hat{w}_t^+, \delta)}{1 + \frac{\sigma_t^2}{\delta^2}h(\hat{w}_t^-, \delta)}\right)^2.
    \end{align}
Here, $h(w, \delta) := (1 - \delta^2)w + \delta^2$ is a monotonically increasing function with respect to $w$. Additionally, $h(\hat{w}_t^+, \delta)$ and $h(\hat{w}_t^-, \delta)$ denote positive and negative class confidence rates with% \qq{does the function below depends on $\sigma_t$? we need to reflect that}\zk{It is a function of $\sigma_t$(or t) and $\delta$, but we omit $\delta$ and use t as a subscript to make it one line} \qq{we can add a bracket below to denote those}
\begin{align*}
\begin{cases}
\hat w_t^+(\sigma_t, \delta) &=\; \mathbb E_k[ \mathbb{E}_{\bm x_0}[w_k(\bm x_0, t)\mid k]], \\
\hat w_t^-(\sigma_t, \delta) &=\; \mathbb E_{k}[\mathbb{E}_{\bm x_0}[w_{l}(\bm x_0, t) \mid k \neq l ]],
\end{cases}
\end{align*}
whose analytical forms can be found in \Cref{app:thm1_proof}.  
\end{theorem}



%$\hat w_t^+ = \mathbb{E}_{\bm x_0}[w_k(\bm x_0, t)]$, $\hat w_t^-= \mathbb{E}_{\bm x_0}[w_{l}(\bm x_0, t)] \;\text{for}\; l\neq k$, and $h(w, \delta) := (1 - \delta^2)w + \delta^2$.
%\begin{align*}&\hat w_t^+ := \mathbb{E}_{\bm x_0}[w_k(\bm x_0, t)]\\
%    &\hat w_t^-:= \mathbb{E}_{\bm x_0}[w_{l}(\bm x_0, t)], l\neq k\\
%    &h(w, \delta) := (1 - \delta^2)w + \delta^2
%\end{align*} 
%where samples $\bm x_0$ are drawn from class $k$ (i.e., $\bm x_0 = \bm U_k \bm a_i + b\bm U_k^{\perp} \bm e_i$)
%we leave the analytical form of $\hat w_t^+$ and $\hat w_t^- $ to the appendix, with which we are able to show that it exhibits a unimodal trend. Note that here $\sigma_t$ denotes the level of additive Gaussian noise introduced during the diffusion training process.
% and $h(w, \delta) := (1 - \delta^2)w + \delta^2$. 
%Since $\delta$ is fixed, $h(w,\delta)$ is a monotonically increasing function with respect to $w$. 
% The term $\hat{\bm x}_{\textit{approx}}^{\star}$ serves as an approximation of $\hat{\bm x}_{\bm \theta}^{\star}$ as defined in \eqref{eq:E_MoG}, obtained by approximating $w_l^{\star}$ in \eqref{eqn:w-k} with $\hat{w}_l$ by taking the expectation inside the softmax with respect to $\bm x_0$.\footnote{We empirically validate the tightness of this approximation in \Cref{fig:assump_validate}.}  
% Note that here $\delta$ represents the magnitude of the fixed intrinsic noise in the data 


    % \xiao{I moved the other part of the thm to next section, not sure if this is better.} \qq{I feel it would be better to move it back, and we can refer later on}
%\qq{just say it is a monotonically increasing function w.r.t. to w}
    % Furthermore, we can decompose $\E_{\bm x_0}[\|f^{\star}(\bm x_0, t)\|^2]$ as:
    % \begin{align}\label{eq:fx_decompose}
    % \begin{split}
    %     \E[\|f^{\star}(\bm x_0, t)\|^2] &= \E\|\bm U_k \bm U_k^Tf^{\star}(\bm x_0, t)\|^2] + \E[\sum_{l \neq k}^K\bm U_l \bm U_l^Tf^{\star}(\bm x_0, t)\|^2] + \E[\|\bm U_{\perp} \bm U_{\perp}^Tf^{\star}(\bm x_0, t)\|^2]
    % \end{split}
    % \end{align}
    % where 
    % \begin{align}\label{eq:fx_decompose_terms}
    % \begin{split}
    %     \E\|\bm U_k \bm U_k^Tf^{\star}(\bm x_0, t)\|^2] &= \left( \frac{\hat{w}_k}{1 + \sigma_t^2} + \frac{(K-1)\delta^2\hat{w}_l}{\delta^2+\sigma_t^2}\right)^2 d \\
    %     \E[\|\bm U_{\perp} \bm U_{\perp}^Tf^{\star}(\bm x_0, t)\|^2] &= \frac{b^6 (n-Kd)}{(\delta^2 + \sigma_t^2)^2} \\
    %     \E[\sum_{l \neq k}^K\bm U_l \bm U_l^Tf^{\star}(\bm x_0, t)\|^2] &= \left( K-1 \right)\left( \frac{\hat{w}_l}{1+\sigma_t^2} + \frac{\delta^2(\hat{w}_k + (K-2)\hat{w}_l)}{\delta^2 + \sigma_t^2} \right)^2 \delta^2 d.
    % \end{split}
    % \end{align}



% In the \MoLRG~setting, the ground truth posterior mean is a combination of $\bm x_t$'s projections onto the signal and noise spaces of each class, with the coefficients determined by a softmax term. Similar to the single Gaussian case, the SNR for the reconstruction is:
% \begin{align}
%     \frac{1}{1 + \sigma_t^2} \;/\; \frac{\delta^2}{\delta^2 + \sigma_t^2} = \frac{\delta^2+\sigma_t^2}{\delta^2(1+\sigma_t^2)}
% \end{align}

% increases as the noise scale grows. Given that the optimal weights involve a softmax term, a linear network is no longer sufficient to capture the optimal solution. Therefore, we employ a more practical MLP-based model to learn the optimal score function.
% \qq{use mathrm for all CSNR}

We defer the formal statement of \Cref{lem:main} and its proof to \Cref{app:thm1_proof}. In the following, we discuss the implications of our result.

\begin{wrapfigure}[12]{R}{0.50\textwidth}
% \begin{figure}[t] % 'r' for right, 'l' for left
    \centering
    \includegraphics[width=0.4\textwidth]{figs/interplay.pdf}
    \caption{Illustration of the interplay between the denoising rate and the class confidence rate.}
    \label{fig:trade_off}
% \end{figure}
\end{wrapfigure}

\paragraph{The unimodal curve of \CSNR\;across noise levels.}
Intuitively, our theorem shows that unimodal curve is mainly induced by the the interplay between the ``denoising rate" $\sigma_t^2/\delta^2$ and the positive class confidence rate $h(\hat{w}_t^+, \delta)$ as noise level $\sigma_t$ increases. As observed in \Cref{fig:trade_off}, the ``denoising rate" ($\sigma_t^2/\delta^2$) increases monotonically with $\sigma_t$ while the class confidence rate $h(\hat{w}_t^+, \delta)$ monotonically declines. Initially, when $\sigma_t$ is small, the class confidence rate remains relatively stable due to its flat slope, and an increasing ``denoising rate" improves the \CSNR, resulting in improved posterior estimation. However, as indicated by \Cref{lem:E[x_0]_multi}, when $\sigma_t$ becomes too large, $h(\hat{w}_t^+,\delta)$ approaches $h(\hat{w}_t^-,\delta)$, leading to a drop in \CSNR, which limits the ability of the model to project $\bm x_0$ onto the correct signal subspace and ultimately hurts posterior estimation. 



\paragraph{Alignment of \CSNR\;with posterior representation quality.} Although our theory is derived from the noisy \MoLRG\; distribution, it effectively captures real-world phenomena. As shown in \Cref{fig:csnr_molrg_match,fig:cifar}, we conduct experiments on both synthetic (i.e., noisy \MoLRG) and real-world datasets (i.e., CIFAR and ImageNet) to measure $\mathrm{CSNR}(\hat{\bm x}_{\bm \theta},t)$ as well as the posterior probing accuracy. For posterior probing, we use posterior estimations at different timesteps as inputs for classification. The results consistently show that $\mathrm{CSNR}(\hat{\bm x}_{\bm \theta},t)$ follows a unimodal pattern across all cases, mirroring the trend observed in posterior probing accuracy as the noise scale increases. This alignment provides a formal justification for previous empirical findings \citep{xiang2023denoising, baranchuk2021label, tang2023emergent}, which have reported a unimodal trajectory in the representation dynamics of diffusion models with increasing noise levels. Detailed experimental setups are provided in \Cref{app:exp_detail}.
% \qq{need to explain what is posterior probing performance}
% \qq{this sentence is a bit confusing, revise, I suggest to separate into multiple sentences} 

\paragraph{Explanation of generation and representation trade-off.}
Our theoretical findings reveal the underlying rationale behind the generation and representation trade-off: the proportion of data associated with $\delta$ represents class-irrelevant attributes. The unimodal representation learning dynamic thus captures a ``fine-to-coarse" shift \citep{choi2022perception, wang2023diffusion}, where these class-irrelevant attributes are progressively stripped away. During this process, peak representation performance is achieved at a balance point where class-irrelevant attributes are eliminated, while class-essential information is preserved. In contrast, high-fidelity image generation requires capturing the entire data distribution—from coarse structures to fine details—leading to optimal performance at the lowest noise level $\sigma_t$, where class-irrelevant attributes encoded in the $\delta$-term are maximally retained. Thus, our insights explain the trade-off between generation and representation quality. As visualized in \Cref{fig:use_clean} and \Cref{fig:vis_molrg_3class}, representation quality peaks at an intermediate noise level where irrelevant details are stripped away, while generation quality peaks at the lowest noise level, where all details are preserved.
    

% \begin{wrapfigure}[19]{R}{0.44\textwidth}
\begin{figure}[t]
    \centering
    \includegraphics[width = 0.52\textwidth]{figs/posterior_curve_molrg.pdf}
\caption{\textbf{Posterior probing accuracy and associated \CSNR~dynamics in \MoLRG~data.} We plot the posterior probing accuracy and \CSNR~with the posterior estimations obtained from a learned estimator $\bm{\hat x_\theta}$. both of which exhibit a consistent unimodal pattern. Additionally, we include the optimal \CSNR~, calculated from the ground truth posterior function $\bm{\hat x_\theta}^\star$ defined in \Cref{lem:E[x_0]_multi}, as a reference. The estimator is trained on a 3-class \MoLRG~dataset with data dimension $n=50$, subspace dimension $d=15$, and noise scale $\delta=0.5$.}
\label{fig:csnr_molrg_match}
\end{figure}
% \end{wrapfigure}


% \begin{figure*}[t]
% \begin{center}
%     \begin{subfigure}{0.47\textwidth}
%     \includegraphics[width = 0.955\textwidth]{figs/posterior_curve_molrg.pdf}
%     \caption{Posterior Probing Acc. and $\mathrm{CSNR}(\hat{\bm x}_{\bm \theta},t)$} 
%     \end{subfigure} \quad %\hspace*{\fill}
%     \begin{subfigure}{0.47\textwidth}
%     \includegraphics[width = 0.955\textwidth]{figs/feature_curve_molrg.pdf}
%     \caption{Feature Probing Acc. and $\mathrm{CSNR}(\hat{f}_{\bm \theta},t)$} 
%     \end{subfigure}
%     \end{center}
% \caption{\textbf{Probing accuracy and associated \CSNR~dynamics in \MoLRG~data.} In panel (a), we plot the probing accuracy and the previously discussed \CSNR, both metrics exhibit a consistent unimodal pattern as depicted in \Cref{fig:clean_feature}. Additionally, in (b) we plot the metrics calculated from intermediate features, showing the universality across both posterior and features.}
% \label{fig:csnr_molrg_match}
% \end{figure*}
% mog_csnr.pdf mog_f_csnr.pdf


\begin{figure*}[h]
\begin{center}
    % \begin{subfigure}{0.47\textwidth}
    % \includegraphics[width = 0.955\textwidth]{figs/feature_curve_c10.pdf}
    % \caption{CIFAR10} 
    % \end{subfigure} \quad %\hspace*{\fill}
    % \begin{subfigure}{0.47\textwidth}
    % \includegraphics[width = 0.955\textwidth]{figs/feature_curve_mini.pdf}
    % \caption{MiniImageNet} 
    % \end{subfigure}

    \begin{subfigure}{0.48\textwidth}
    \includegraphics[width = 0.995\textwidth]{figs/posterior_curve_c10.pdf}
    \caption{CIFAR10} 
    \end{subfigure} \quad %\hspace*{\fill}
    \begin{subfigure}{0.48\textwidth}
    \includegraphics[width = 0.995\textwidth]{figs/posterior_curve_mini.pdf}
    \caption{MiniImageNet} 
    \end{subfigure}
    \end{center}
\caption{\textbf{Dynamics of posterior probing accuracy and associated \CSNR~on CIFAR10 and MiniImageNet.} Posterior probing accuracy is plotted alongside $\mathrm{CSNR}(\hat{\bm x}_{\bm \theta},t)$. Probing accuracy is evaluated on the test set, while the empirical \CSNR~is computed from the training set. Both exhibit an aligning unimodal pattern. We use released EDM models \citep{karras2022elucidating} trained on the CIFAR-10 \citep{krizhevsky2009learning} and ImageNet \citep{deng2009imagenet} datasets, evaluating them on CIFAR-10 and MiniImageNet \citep{vinyals2016matching}, respectively. To compute \CSNR~, we apply PCA on the original CIFAR-10/MiniImageNet images to extract the basis $\bm{U}_k$s. Further details can be found in \Cref{app:exp_detail}. }
% \qq{moved here, please revise, this is also the setup of Figure 5? we can refer to Figure 6}}
\label{fig:cifar}
\end{figure*}
%\caption{\textbf{Dynamics of feature probing accuracy and associated \CSNR~on CIFAR10 and MiniImageNet.} Feature probing accuracy is plotted alongside $\mathrm{CSNR}(t, f)$. Probing accuracy is evaluated on the test set, while the empirical \CSNR~is computed from the training set. Both exhibit an aligning unimodal pattern.}
%\caption{\textbf{Dynamics of posterior probing accuracy and associated \CSNR~on CIFAR10 and MiniImageNet.} The top row presents feature probing accuracy alongside $\mathrm{CSNR}(t, f)$, while the bottom row illustrates posterior probing accuracy and $\mathrm{CSNR}(t, \bm x_{\bm \theta})$. Probing accuracy is evaluated on the test set, whereas \CSNR~is computed from the training set. Both feature and posterior probing accuracy exhibit unimodal patterns that align with their respective \CSNR~trends.  }
% \xiao{grid} \qq{revise} 
%\zk{this actually overlaps with fig3, have to reduce, also we can combine it with the remark}




% In contrast, high-fidelity image generation requires capturing the entire data distribution—from coarse structures to fine details—leading to optimal performance at the lowest noise level, where class-irrelevant attributes and finer image details encoded in the $\delta$-term are maximally retained. Thus, our insights clarify the trade-off between generation and representation quality. As visualized in \Cref{fig:use_clean} and \Cref{fig:vis_molrg_3class}, representation quality peaks at an intermediate noise level where irrelevant details are stripped away, while generation quality peaks at the lowest noise level, where all details are preserved.

% Previous studies \citep{xiang2023denoising, baranchuk2021label, tang2023emergent} have empirically shown that the representation dynamics of diffusion models follow a unimodal curve as the noise scale increases, across various tasks such as classification, segmentation, and image correspondence. Our theoretical findings reveal the underlying rationale behind this phenomenon: the proportion of data associated with $\delta$ represents class-irrelevant attributes or finer image details. The unimodal representation learning dynamic thus captures a ``fine-to-coarse" shift \citep{choi2022perception, wang2023diffusion}, where these details are progressively stripped away. During this process, peak representation performance is achieved at a balance point where class-irrelevant attributes are eliminated, while class-essential information is preserved.

% On the other hand, high-fidelity image generation requires capturing the entire data distribution—from coarse structures to fine details—leading to optimal performance at the lowest noise level, where class-irrelevant attributes and finer image details encoded in the $\delta$-term are maximally retained. Therefore, our theoretical insights shed light on explaining the trade-off between generation and representation quality. This explanation is supported by our visualizations in \Cref{fig:use_clean} and \Cref{fig:vis_molrg_3class}, which show that for both synthetic and real datasets, the representation peak consistently occurs at an intermediate noise level where class-irrelevant details are removed, while the generation peak is reached at the smallest noise level, where all details are preserved.
% ==============


% Previous studies \citep{xiang2023denoising, baranchuk2021label, tang2023emergent} have empirically shown that the representation dynamics of diffusion models follow a unimodal curve as the noise scale increases, across various tasks such as classification, segmentation, and image correspondence. Our findings corroborate this observation, as demonstrated in \Cref{fig:clean_feature}, where the representation quality consistently exhibits a unimodal trend, regardless of the specific network architecture, dataset or task. In the following analysis, we argue that this unimodal behavior arises from subtle differences and trade-offs between the requirements of representation learning and the generative nature of diffusion models.

% High-fidelity image generation demands that diffusion models capture every aspect of the data distribution—from coarse structures to fine details. In contrast, representation learning, particularly for high-level tasks such as classification \citep{allen2022feature}, prefers an abstract representation, where finer image details may even act as `noise' that hinders performance. As shown in \Cref{fig:use_clean}, as the noise level increases, the predicted posteriors for clean input $\bm x_0$ transition from `fine' to `coarse' \citep{wang2023diffusion,choi2022perception}, gradually removing fine-grained details. For the classification task in the plot, the best performance is achieved when the posterior estimation retains the essential information while discarding some class-irrelevant details. These findings indicate a trade-off between generative quality and representation performance \citep{chen2024deconstructing}, prompting us to attribute variations in feature quality across noise levels to differences in posterior prediction.


%\subsection{Alignment of CSNR with Representation Quality}\label{subsec:theory_verify}

% \begin{figure*}[t]
%     \centering
%     \includegraphics[width=0.9\linewidth]{figs/vis_molrg.pdf}
%     \vspace{-0.1in}
%     \caption{\textbf{Visualization of posterior estimation for a clean input, higher \CSNR~correspondings to higher classification accuracy.} The \textbf{same} \MoLRG~data is fed into the models; each row represents a different denoising model, and each column corresponds to a different time step with noise scale ($\sigma_t$). The \textcolor{red}{red} box indicates the best posterior estimation and feature probing accuracy.}
%     \label{fig:vis_molrg_3class}
% \end{figure*}



%We conduct experiments on both synthetic and real-world datasets to validate our theoretical insights into the dynamics of representation learning.



%The results for the synthetic and real datasets are shown in \Cref{fig:csnr_molrg_match} and \Cref{fig:cifar}, respectively. As depicted in the plots, $\mathrm{CSNR}(\hat{\bm x}_{\bm \theta},t)$ consistently follows a unimodal pattern, aligning with the trend of posterior probing performance as the noise scale increases.

% We extract intermediate features and posterior predictions at each time step and evaluate the classification performance on the test set using linear and MLP probing, respectively. For \CSNR~calculation, we use the raw inputs (i.e., $\bm x_0$s drawn from the \MoLRG~distribution for the synthetic dataset and the original CIFAR10/MiniImageNet images) to compute the basis $\bm U_k$s, followed by the empirical computation of \CSNR~for posterior predictions, denoted as $\mathrm{CSNR}(\hat{\bm x}_{\bm \theta},t)$. For \CSNR~on feature representations, we compute the basis using the features extracted at each time step and subsequently calculate \CSNR~for the features, denoted as $\mathrm{CSNR}(\hat{f}_{\bm \theta},t)$.

% The results for the synthetic datasets are presented in \Cref{fig:csnr_molrg_match} and real datasets in \Cref{fig:cifar,fig:cifar_2}. As illustrated in the plots, $\mathrm{CSNR}(\hat{\bm x}_{\bm \theta},t)$ consistently exhibits a unimodal pattern similar to the posterior probing performance as the noise scale increases. A similar trend is observed for $\mathrm{CSNR}(\hat{f}_{\bm \theta},t)$, which aligns with the feature probing accuracy. These findings support our theoretical predictions.

% \paragraph{Trade-offs between generation and representation quality.}
% To further illustrate the trade-offs between generation and representation quality, we visualize the posterior estimations and \CSNR~values of the synthetic \MoLRG~dataset at different noise scales in \Cref{fig:vis_molrg_3class}. As observed, the highest generation quality occurs at the smallest noise scale, where the generated samples closely resemble the original inputs. However, optimal generation quality does not necessarily translate to optimal representation quality, as measured by probing performance. Instead, the best representation performance is achieved at intermediate noise scales. This observation aligns with our theoretical insights, suggesting that initially, increasing the noise scale mitigates the $\delta$-related data noise, leading to cleaner posterior estimations and improved probing accuracy. However, as the noise scale continues to rise, the class confidence rate diminishes, causing class overlap and ultimately degrading feature quality and probing performance. Similar trends can be observed in real datasets, where CIFAR10 results are presented at the beginning of this paper in \Cref{fig:use_clean}.

\section{Practical Insights}\label{sec:exp}
% In this section, we first validate our theoretical insights into the representation dynamics of diffusion models under both theoretical and practical settings in \Cref{subsec:emp_verify}. 
We examine the practical implications of our findings in \Cref{subsec:exp_ensemeble}, leveraging feature information at different levels of granularity to enhance robustness. Additionally, we discuss the advantages of diffusion models over traditional single-step DAEs in \Cref{subsec:weight_share}.
% and analyze the influence of data complexity on diffusion-based representation learning in \Cref{subsec:mem_gen}. Detailed experimental setups are provided in \Cref{app:exp_detail}.

% \begin{figure*}[t]
% \begin{center}
%     \begin{subfigure}{0.47\textwidth}
%     \includegraphics[width = 0.955\textwidth]{figs/ucurve_cifar_featureacc.pdf}
%     \caption{Probing Acc. of diffusion models and DAEs} 
%     \end{subfigure} \quad %\hspace*{\fill}
%     \begin{subfigure}{0.47\textwidth}
%     \includegraphics[width = 0.955\textwidth]{figs/ucurve_cifar_csnr.pdf}
%     \caption{CSNR and Posterior Acc.} 
%     \end{subfigure}
%     \end{center}
% \caption{\textbf{Dynamics of feature probing accuracy and \CSNR~on CIFAR10.} Panel (a) shows the feature probing accuracy and Panel (b) shows the \CSNR~and posterior probing accuracy trends computed using the CIFAR10 test dataset, both exhibiting a \textcolor{blue}{matching} unimodal pattern.}
% \label{fig:cifar}
% \end{figure*}

% \subsection{Empirical Validation}\label{subsec:emp_verify}

% \qq{condense, and move this as a subsection of Section 3 instead. This section needs to focus on practical insights: (i) ensemble, (ii) weight sharing, and (iii) memorization to generalization}

% \xiao{Section 4.1.1 has been discussed in Section 2. Section 4.1.2 will be removed to Section 3}
% {\color{blue}
% We divide this subsection into two parts: in \Cref{subsubsec:feature-extraction}, we justify our design choice in the theoretical analysis by considering clean inputs $\bm x_0$ as the input to diffusion models, rather than the conventional approach of using noisy inputs $\bm x_t$. Subsequently, in \Cref{subsubsec:theory_verify}, we validate our theoretical insights by demonstrating that \CSNR~aligns closely with both feature and posterior probing performance.

% \subsubsection{Clean Inputs Enhance Representation Performance Compared to Noisy Inputs}\label{subsubsec:feature-extraction}

% If we treat a diffusion model at each noise level (i.e., $\bm x_{\bm \theta}(\cdot, t)$) as an independent model and aim to compare their representation performance, using the noisy input $\bm x_t$ does not provide a fair basis for comparison. This is because the varying input noise levels at different values of $t$ can significantly influence the results. Therefore, in both the definition of \CSNR~\eqref{eq:csnr_true} and the subsequent analysis, we consistently consider clean inputs $\bm x_0$ to eliminate the confounding effects of input degradation.

% A potential concern, however, is whether using clean inputs during inference contradicts the training paradigm, as diffusion models are trained on corrupted inputs. To address this, we compare the representation performance of models using $\bm x_0$ and $\bm x_t$ as inputs across classification and segmentation tasks, with results presented in \Cref{fig:clean_feature}. Our findings highlight two key insights: (1) using clean inputs $\bm{x}_0$ consistently matches or surpasses the performance achieved with corrupted inputs $\bm{x}_t$, and (2) the performance gap widens as $t$ (or equivalently, $\sigma_t$) increases. These observations suggest that the core representation ability of diffusion models is primarily derived from their denoising objective, while the diffusion process—characterized by the gradual addition and removal of Gaussian noise—plays a minimal role in representation quality.

% This phenomenon draws a parallel with traditional supervised and self-supervised learning paradigms. In these settings, data augmentations such as cropping~\citep{caron2021emerging}, color jittering, or masking~\citep{he2022masked} are applied during training to enhance model robustness and generalization. However, clean, unaugmented images are typically used during inference to ensure optimal performance. Similarly, in diffusion models, additive Gaussian noise serves as a form of data augmentation critical for training~\citep{chen2024deconstructing}, but when the focus shifts to representation learning at inference, using clean images $\bm{x}_0$ is both sufficient and aligned with standard representation learning practices. Thus, we advocate for the use of clean images as the standard input protocol for feature extraction during inference throughout our analysis and future work.

% }

\subsection{Finding I: Feature Ensembling Across Timesteps Improves Representation Robustness}\label{subsec:exp_ensemeble}

Our theoretical insights imply that features extracted at different timesteps capture varying levels of granularity. Given the high linear separability of intermediate features, we propose a simple ensembling approach across multiple timesteps to construct a more holistic representation of the input. Specifically, in addition to the optimal timestep, we extract feature representations at four additional timesteps—two from the coarse (larger $\sigma_t$) and two from the fine-grained (smaller $\sigma_t$) end of the spectrum. We then train linear probing classifiers for each set and, during inference, apply a soft-voting ensemble by averaging the predicted logits before making a final decision.(experiment details in \Cref{app:exp_detail})

We evaluate this ensemble method against results obtained from the best individual timestep, as well as a self-supervised method MAE \citep{he2022masked}, on both the pre-training dataset and a transfer learning setup. The results, reported in \Cref{tab:ensemble_results} and \Cref{tab:ensemble_results_transfer}, demonstrate that ensembling significantly enhances performance for both EDM \citep{karras2022elucidating} and DiT \citep{peebles2023scalable}, consistently outperforming their vanilla diffusion model counterparts and often surpassing MAE. Moreover, ensembling substantially improves the robustness of diffusion models for classification under label noise.

 \begin{table}[t]
\centering
\resizebox{0.98\linewidth}{!}{
	\begin{tabular}	{l c c c c c } 
		
        \toprule
            % \rule{0pt}{0.1ex} \\ 
		 \textbf{Method} & \multicolumn{5}{c}{\textit{MiniImageNet$^\star$} Test Acc. \%}\\
            % \rule{0pt}{0.1ex} \\ 
		 \midrule
		 \textbf{Label Noise} & Clean & 20\% & 40\% & 60\% & 80\% \\
   %       \midrule
   %       \textit{CIFAR10} \\ 
		 % \midrule
		 % SimCLR ResNet50 & 93.2 & 92.5 & 92.2 & 91.6 & 90.2 \\ 
   
   %        EDM & 96.0 & 95.8 & 95.9 & 95.5 & 94.6  \\ 
          
		 % \textbf{EDM (Ensemble)} & 95.7 & 95.8 & 95.9 & 95.4 & 95.0 \\ 
         
         \midrule
   %      \textit{MiniImageNet$^\star$} \\  
		 % \midrule
		 % SimCLR ResNet50 & 75.1 & 72.1 & 70.1 & 66.6 & \textbf{60.4} \\ 
         MAE & 73.7 & 70.3 & 67.4 & 62.8 & 51.5 \\ 
   
          EDM & 67.2 & 62.9 & 59.2 & 53.2 & 40.1  \\ 
          
		 \textbf{EDM (Ensemble)} & 72.0 & 67.8 & 64.7 & 60.0 & 48.2 \\ 

         
   
          DiT & 77.6 & 72.4 & 68.4 & 62.0 & 47.3  \\ 
          
		 \textbf{DiT (Ensemble)} & \textbf{78.4} & \textbf{75.1} & \textbf{71.9} & \textbf{66.7} & \textbf{56.3} \\ 
         
        \bottomrule
	\end{tabular}}
    \caption{\textbf{Comparison of test performance across different methods under varying label noise levels.} All compared models are publicly available and pre-trained on ImageNet-1K \citep{deng2009imagenet}, evaluated using MiniImageNet classes. Bold font highlights the best result in each scenario.}
    \label{tab:ensemble_results}
    % \vspace{-6mm}
\end{table}

 \begin{table*}[t]
\centering
\resizebox{0.98\linewidth}{!}{
	\begin{tabular}	{l c c c c c | c c c c c | c c c c c} 
		
        \toprule
          % \rule{0pt}{0.05in} \\ 
		 \textbf{Method} & \multicolumn{15}{c}{Transfer Test Acc. \%}\\
         % \rule{0pt}{0.2ex} \\ 
           % \rule{0pt}{0.05in} \\ 
         \midrule
        & \multicolumn{5}{c|}{\textbf{CIFAR100}} & \multicolumn{5}{c|}{\textbf{DTD}} & \multicolumn{5}{c}{\textbf{Flowers102}} \\  
		 \textbf{Label Noise} & Clean & 20\% & 40\% & 60\% & 80\% & Clean & 20\% & 40\% & 60\% & 80\% & Clean & 20\% & 40\% & 60\% & 80\% \\ 
         
         
		 \midrule
		 % SimCLR ResNet50 & 65.8 & 62.9 & \textbf{60.5} & \textbf{56.6} & \textbf{48.4} & 57.1 & 54.6 & 50.0 & 43.5 &	28.3 & 59.5 & 50.8 & 37.5 & 24.4 & 9.5 \\ 
         MAE & 63.0 & 58.8 & 54.7 & 50.1 & 38.4 & 61.4 & 54.3 & 49.9 & 40.5 & 24.1 & 68.9 & 55.2 & 40.3 & 27.6 & 9.6 \\ 
   
          EDM & 62.7 & 58.5 & 53.8 & 48.0 & 35.6 & 54.0 & 49.1 & 45.1 & 36.4 & 21.2 & 62.8 & 48.2 & 37.2 & 24.1 & 9.7 \\ 
          
		 \textbf{EDM (Ensemble)} & \textbf{67.5} & \textbf{64.2} & \textbf{60.4} & \textbf{55.4} & \textbf{43.9} & 55.7 & 49.5 & 45.2 & 37.1 & 22.0 & 67.8 & 53.9 & 41.5 & 25.0 & 10.4 \\ 

         
   
          DiT & 64.2 & 58.7 & 53.5 & 46.4 & 32.6 & 65.2 & 59.7 & 53.0 & 43.8 & 27.0 & 78.9 & 65.2 & 52.4 & 34.7 & 13.3 \\ 
          
		 \textbf{DiT (Ensemble)} & 66.4 & 61.8 & 57.6 & 51.3 & 39.2 & \textbf{65.3} & \textbf{60.6} & \textbf{56.1} & \textbf{46.3} & \textbf{30.6} & \textbf{79.7} & \textbf{67.0} & \textbf{54.6} & \textbf{36.6} & \textbf{14.7}  \\ 
         
        \bottomrule
	\end{tabular}}
    \caption{\textbf{Comparison of transfer learning performance across different methods under varying label noise levels.} All compared models are publicly available and pre-trained on ImageNet-1K \citep{deng2009imagenet}, evaluated on different downstream datasets. Bold font highlights the best result in each scenario.  }
    \label{tab:ensemble_results_transfer}
    % \vspace{-6mm}
\end{table*}
%\qq{save space of the first line} 
% \begin{table}[t]
%     \centering
%     \begin{tabular}{c|c c c c}
%     label noise & 0 & 0.4 & 0.6 & 0.8\\
%     \hline
%     cifar10 & 95.8 & 94.72 & 93.7 & 82.83\\
%     % cifar10 (ensemble) & 95.75 & 95.52& 95.12 & 94.04 \\
%     cifar10 (ensemble w/ BN) & 95.65 & 95.63 & 95.36 & 94.98\\
%     MAE(ImageNet transfer)& 83.94 & 80.53 & 76.69 & 63.46\\
%     ResNet(ImageNet transfer) \\
%     \hline
%     % miniImageNet(edm) & 65.64 & 54.48  & 42.26 & 23.58\\
%     % miniImageNet(DiT) & 74.77 & 50.91 & 38.09 & \\
%     miniImageNet(DiT w/BN) & 77.56 & 63.42 & 49.59 & 27.11\\
%     miniImageNet(DiT w/BN ensemble) & 78.44 & 67.91 & 54.30 & 30.18\\
%     % miniImageNet (edm ensemble) & 70.9 & 60.53 & 48.13 & 24.96\\
%     % MAE(ImageNet transfer w/o BN) & 73.37 & 66.27 & 56.32 & 33.37\\
%     MAE(ImageNet transfer w/ BN) & 72.87 & 65.79 & 55.15 & 29.32\\
%     % MAE(ImageNet transfer w/trick) & 69.01 & 61.70 & 52.16 & 29.57\\
%     \hline
%     cifar100 \\
%     MAE(w/ BN)& 63.01 & 53.06 & 42.10 & 22.11\\
%     % MAE(w/o BN)& 58.24 & 51.52 & 40.74 & 21.94\\
%     \end{tabular}
%     \caption{Feature ensemble enhance the overall linear probing accuracy, especially robustness againist label noise}
%     \label{tab:ensemble}
% \end{table}

\subsection{Finding II: Weight Sharing in Diffusion Models Facilitates Representation Learning}\label{subsec:weight_share}

\begin{figure*}[t]
    \begin{center}
    \begin{subfigure}{0.47\textwidth}
    \includegraphics[width = 0.955\textwidth]{figs/dae_diffusion_c10.pdf}
    \caption{CIFAR10} 
    \end{subfigure} \quad %\hspace*{\fill}
    \begin{subfigure}{0.47\textwidth}
    \includegraphics[width = 0.955\textwidth]{figs/dae_diffusion_c100.pdf}
    \caption{CIFAR100} 
    \end{subfigure}
    \end{center}
    \vspace{-0.1in}
\caption{\textbf{Diffusion models exhibit higher and smoother feature accuracy and similarity compared to individual DAEs.} We train DDPM-based diffusion models and individual DAEs on the CIFAR datasets and evaluate their representation learning performance. Feature accuracy, and feature differences from the optimal features (indicated by {\color{cyan} $\star$}) are plotted against increasing noise levels. The results reveal an inverse correlation between feature accuracy and feature differences, with diffusion models achieving both higher/smoother accuracy and smaller/smoother feature differences compared to DAEs.}
\vspace{-0.05in}
\label{fig:dae_diffusion}
\end{figure*}

% Now that we have established the denoising objective as the primary driver of diffusion models' superior representation learning capabilities, we now turn to a key question: what makes diffusion models outperform traditional single-step denoising autoencoders (DAEs) \citep{vincent2008extracting,vincent2010stacked} and achieve on-par or superior representation performance compared to state-of-the-art self-supervised methods? In this subsection, we reveal that the key advantage of diffusion models over traditional DAEs lies in their inherent weight-sharing mechanism.

Second, we reveal why diffusion models, despite sharing the same denoising objective with classical DAEs, achieve superior representation learning due to their inherent weight-sharing mechanism. By minimizing loss across all noise levels (\ref{eq:dae_loss}), diffusion models enable parameter sharing and interaction among denoising subcomponents, creating an implicit "ensemble" effect. This improves feature consistency and robustness across noise scales compared to DAEs \citep{chen2024deconstructing}, as illustrated in \Cref{fig:dae_diffusion}.

To test this, we trained 10 DAEs, each specialized for a single noise level, alongside a DDPM-based diffusion model on CIFAR10 and CIFAR100. We compared feature quality using linear probing accuracy and feature similarity relative to the optimal features at $\sigma_t = 0.06$ (where accuracy peaks) via sliced Wasserstein distance (\SWD) \citep{doan2024assessing}.

The results in \Cref{fig:dae_diffusion} confirm the advantage of diffusion models over DAEs. Diffusion models consistently outperform DAEs, particularly in low-noise regimes where DAEs collapse into trivial identity mappings. In contrast, diffusion models leverage weight-sharing to preserve high-quality features, ensuring smoother transitions and higher accuracy as noise increases. This advantage is further supported by the \SWD~curve, which reveals an inverse correlation between feature accuracy and feature differences. Notably, diffusion model features remain significantly closer to their optimal state across all noise levels, demonstrating superior representational capacity.

Our finding also aligns with prior results that sequentially training DAEs across multiple noise levels improves representation quality \citep{chandra2014adaptive,geras2014scheduled,zhang2018convolutional}. Our ablation study further confirms that multi-scale training is essential for improving DAE performance on classification tasks in low-noise settings (details in \Cref{app:add_exp}, \Cref{tab:dae_trial}).


\section{Discussion}
In this work, we develop a mathematical framework for analyzing the representation dynamics of diffusion models. By introducing the concept of \CSNR~and leveraging a low-dimensional mixture of low-rank Gaussians, we characterize the trade-off between generative quality and representation quality. Our theoretical analysis explains how the unimodal representation learning dynamics observed across noise scales emerge from the interplay between data denoising and class specification.

Beyond theoretical insights, we propose an ensemble method inspired by our findings that enhances classification performance in diffusion models, both with and without label noise. Additionally, we empirically uncover an inherent weight-sharing mechanism in diffusion models, which accounts for their superior representation quality compared to traditional DAEs. Experiments on both synthetic and real-world datasets validate our findings. Additionally, our findings also open up new avenues for future research that we discuss in the following.

\begin{itemize}[leftmargin=*]
    \item \textbf{Principled diffusion-based representation learning.} While diffusion models have shown strong performance in various representation learning tasks, their application often relies on trial-and-error methods and heuristics. For example, determining the optimal layer and noise scale for feature extraction frequently involves grid searches. Our work provides a theoretical framework to understand representation dynamics across noise scales. A promising future direction is to extend this analysis to include layer-wise dynamics. Combining these insights could pave the way for more principled and efficient approaches to diffusion-based representation learning.

    \item \textbf{Representation alignment for better image generation.} Recent work REPA \citet{yu2024repa} has demonstrated that aligning diffusion model features with features from pre-trained self-supervised foundation models can enhance training efficiency and improve generation quality. By providing a deeper understanding of the representation dynamics in diffusion models, our findings could further advance such representation alignment techniques, facilitating the development of diffusion models with superior training and generation performance.
\end{itemize}

\section*{Acnkowledgement}
We acknowledge funding support from NSF CAREER CCF-2143904, NSF CCF-2212066, NSF CCF-
2212326, NSF IIS 2312842, NSF IIS 2402950, ONR N00014-22-1-2529, and MICDE Catalyst Grant.

\bibliographystyle{abbrvnat}
\bibliography{refs}

\newpage
\appendix
\section{Appendix}
\section{Experimental Setup}\label{app:exp}
\subsection{Datasets}
UnKEBench \cite{UnKE} constructs a dataset containing 1,000 counterfactual unstructured texts, where knowledge is presented in an unstructured and relatively lengthy form, going beyond simple knowledge triplets or linear fact chains. These texts originate from ConflictQA \cite{conflictqa}, a benchmark specifically designed to distinguish LLMs' parameter memory from anti-memory. This approach is crucial for preventing the model from merging knowledge obtained during pretraining with knowledge acquired during the editing process. Moreover, it addresses the key challenge of determining whether the model learns target knowledge during training or editing, ensuring a clear boundary between pretraining knowledge and edited knowledge.

AKEW benchmark \cite{AKEW} considers three aspects: (1) \textit{Structured Facts}: Each structured fact consists of an isolated triplet for editing, sourced from existing datasets or knowledge bases. (2) \textit{Unstructured Facts}: Knowledge is presented in unstructured text form. To enable fair comparisons, each unstructured fact contains the same knowledge update as its corresponding structured fact. Compared to structured facts, unstructured facts exhibit greater complexity in natural language format, as they often encapsulate more implicit knowledge. (3) \textit{Extracted Triplets}: Triplets are extracted from unstructured facts using automated methods to investigate whether they can facilitate knowledge editing methods in handling unstructured knowledge. In this work, we primarily focus on unstructured factual knowledge.

EditEverything dataset integrates question-answering data from multiple domains, forming long and diverse knowledge formats that are more challenging to edit. Specifically, for mathematics, we select longer samples from the Orca-Math dataset \cite{math}, which includes grade school math word problems. For coding, we use the MBPP dataset \cite{code}, which consists of approximately 1,000 crowd-sourced Python programming problems solvable by entry-level programmers, covering programming fundamentals and standard library functionalities. For chemistry, we sample from the Camel-Chemistry dataset \cite{chemistry}, which contains problem-solution pairs generated from 25 chemistry topics, each with 25 subtopics and 32 problems per topic-subtopic pair. Lastly, for the news and poetry categories, since they often contain real-world knowledge that LLMs may already possess, we generate synthetic data using GPT-4o to ensure that the information is not already known by the model.

We present sample instances from the dataset in Figure \ref{fig:sample1}, Figure \ref{fig:sample2}, and Figure \ref{fig:sample3}.

\subsection{Evaluation Metrics} \label{app:exp_metric}
Following previous research on model editing for structured knowledge \cite{ROME, MEND}, existing evaluation metrics primarily focus on triplet-structured knowledge, where the goal is to assess the modification of factual triples (\textit{subject, relation, object}). Specifically, given an LLM $f$, an editing knowledge pair $(x, y)$, an equivalent knowledge query $x_e$, and unrelated knowledge pairs $(x_{loc}, y_{loc})$, three standard evaluation metrics are commonly used:

\textbf{Efficacy.} This metric quantifies the success of modifying the target knowledge in $f_{\mathcal{W}}$. It evaluates whether the edited LLM generates the desired target output $y$ when given the input $x$. Formally, it is defined as:
\begin{equation}
\mathbb{E}\left\{y=\mathop{\arg\max}\limits_{y'}\mathbb{P}_{f}(y'\left|x\right.)\right\}.
\end{equation}

\textbf{Generalization.} This metric assesses whether the model has generalized the newly edited knowledge beyond its specific form. It measures if the LLM correctly produces $y$ when given a semantically equivalent input $x_e$, indicating the degree to which the update propagates correctly across paraphrased or restructured queries:
\begin{equation}
\mathbb{E}\left\{y=\mathop{\arg\max}\limits_{y'}\mathbb{P}_{f}(y'\left|x_e\right.)\right\}.
\end{equation}

\textbf{Specificity.} This metric evaluates whether the editing operation is localized, ensuring that unrelated knowledge remains intact. It measures whether the model's response to an unrelated query $x_{loc}$ remains consistent with its original output $y_{loc}$:
\begin{equation}
\mathbb{E}\left\{y_{loc}=\mathop{\arg\max}\limits_{y'}\mathbb{P}_{f}(y'\left|x_{loc}\right.)\right\}.
\end{equation}

While these metrics are well-suited for structured knowledge editing, they are insufficient for evaluating long-form and diverse-formatted knowledge. Such knowledge is often verbose and complex, making it challenging to assess correctness solely based on Efficacy. In these cases, the model may generate an answer that captures the essential information yet fails an exact-match evaluation. To address this, we primarily follow the existing benchmarks for unstructured knowledge editing, incorporating more flexible evaluation methods suited for long-form responses.

Lexical similarity metrics include BLEU \cite{bleu} and various ROUGE scores (ROUGE-1, ROUGE-2, and ROUGE-L) \cite{rouge}. These are computed based on the \textit{original questions}, \textit{paraphrase question}, and \textit{sub-questions}, providing insights into the lexical and n-gram alignment between the model-generated text and the target answer. These metrics serve as the foundation for assessing the surface-level accuracy of edited content.

Semantic similarity is also considered (Bert Score) \cite{bertscore}, as word-level overlap alone is insufficient to capture the nuanced understanding required by the model. To address this, we utilize embedding-based encoders, specifically the all-MiniLM-L6-v2 model \footnote{https://huggingface.co/sentence-transformers/all-MiniLM-L6-v2}, to measure semantic similarity. This ensures a more balanced evaluation that extends beyond lexical matching, quantifying the depth of the model's comprehension.

\subsection{Baseline Methods}
\begin{itemize}
    \item \textbf{FT-L} \cite{FTw} is a knowledge editing approach that fine-tunes specific layers of the LLM using an autoregressive loss function. We reimplemented this method following the hyperparameter from the original paper.
    
    \item \textbf{MEND} \cite{MEND} is a hypernetwork-based efficient knowledge editing method. It trains a hypernetwork to capture patterns in knowledge updates by mapping low-rank decomposed fine-tuning gradients to LLM parameter modifications, enabling efficient and localized edits. Our implementation follows the original hyperparameter settings and completes training over the full dataset. 
    
    \item \textbf{ROME} \cite{ROME} is a method for modifying factual associations in LLM parameters. It identifies critical neuron activations in MLP layers through perturbation-based knowledge localization and modifies MLP layer weights using Lagrange remainders. Since ROME is not designed for large-scale edits, we follow the original paper’s settings and conduct multiple rounds of single-instance editing for evaluation.
    
    \item \textbf{MEMIT} \cite{MEMIT} extends ROME by enabling batch updates of factual knowledge. It utilizes least squares approximation to modify specific layer parameters across multiple layers, allowing simultaneous updates of large numbers of knowledge facts. We evaluate MEMIT in lifelong editing scenarios using the original paper’s configuration.
    
    \item \textbf{AlphaEdit} \cite{AlphaEdit} is a method designed to mitigate interference in LLM lifelong knowledge editing. It introduces a null-space projection mechanism that ensures parameter updates preserve previously edited knowledge while incorporating new updates. AlphaEdit has demonstrated state-of-the-art (SOTA) performance across multiple evaluation metrics while maintaining strong transferability. We follow the original paper’s hyperparameter configuration in our implementation.
    
    \item \textbf{UnKE} \cite{UnKE} improves knowledge editing by refining both the layer and token dimensions. In the layer dimension, it replaces local key-value storage with a non-local block-based mechanism, enhancing the representation capability of key-value pairs while integrating attention-layer knowledge. In the token dimension, it replaces "term-driven optimization" with "cause-driven optimization," which directly edits the final token while preserving contextual coherence. This eliminates the need for explicit term localization and prevents context loss.
\end{itemize}

\subsection{Implementation Details}
Our AnyEdit and AnyEdit* primarily follow the baseline configurations of MEMIT and UnKE, while other baselines adhere to their original implementation settings. All experiments were conducted on a single A100 GPU (80GB).
\begin{itemize}
    \item \textbf{AnyEdit on Llama3-8B-Instruct:} We select layers 4 to 8 for editing and apply a clamp norm factor of 4. The fact token is defined as the last token. The optimization process involves 25 gradient steps for updating the key-value representations, with a learning rate of 0.5. The loss is applied at layer 31, and we use a weight decay of 0.001. To maintain distributional consistency, we introduce a Kullback-Leibler (KL) regularization term with a factor of 0.0625. Furthermore, we enable hyperparameter $\lambda$ with an update weight of 15,000, using 100,000 samples from the Wikipedia dataset with a data type of float32. The module configurations follow MEMIT, where edits are applied to the MLP down projection layers of the selected transformer blocks. Additionally, for chunked editing, we set a chunk size of 40 tokens with no overlap.
    \item \textbf{AnyEdit on Qwen2.5-7B-Instruct:} Same as the above, except that the loss is applied at layer 27 and the chunk size is set to 50 tokens.
    \item \textbf{AnyEdit* on Llama3-8B-Instruct:} We select layer 7 for editing and apply a clamp norm factor of 4. The fact token is defined as the last token. The optimization process involves updating all parameters in both the attention and MLP layers. The gradient descent process utilizes a learning rate of 0.0002 with 50 optimization steps. For updating key-value representations, we use 25 gradient steps with a learning rate of 0.5. The loss is applied at layer 31, and we use a weight decay of 0.001. To preserve original knowledge, we sample 20 data points to constrain parameter updates. Additionally, for chunked editing, we set a chunk size of 40 tokens with no overlap.
    \item \textbf{AnyEdit* on Qwen2.5-7B-Instruct:} Same as the above, except that the loss is applied at layer 27 and the chunk size is set to 50 tokens.
\end{itemize}

\section{Locate-Then-Edit Paradigm \& Related Proof}
\subsection{Locate-Then-Edit Paradigm}\label{app:model_edit}
Following prior works on model editing, the detailed descriptions of specific methods in this section are based on MEMIT \cite{MEMIT}, AlphaEdit \cite{AlphaEdit} and ECE \cite{ECE}. We adhere to their formulations and methodological explanations to ensure consistency and clarity in presenting these approaches.

The locate-then-edit method primarily focuses on triplet-structured knowledge in the form of $(s, r, o)$, such as modifying $(\text{Olympics}, \text{were held in}, \text{Tokyo})$ to $(\text{Olympics}, \text{were held in}, \text{Paris})$. Given new knowledge $(x_e, y_e)$, a triplet can be represented as $x_e = (s, r)$ and $y_e = o$.

We first refine the auto-regressive language model formulation in Section \ref{sec:method:pre}. Let $f$ be a decoder-only model with $L$ layers, processing input sequence $x = (x_0, x_1, \dots, x_T)$ to predict the next token:
\begin{equation}
    \begin{aligned}
        \vh_t^l(x) &= \vh_t^{l - 1}(x) + \va_t^l(x) + \vm_t^l(x), \\
        \va_t^l &= \text{attn}^l(\vh_0^{l - 1}, \vh_1^{l - 1}, \dots, \vh_t^{l - 1}), \\
        \vm_t^l &= \mW_{\text{out}}^l \sigma(\mW_{\text{in}}^l \gamma(\vh_t^{l - 1}+\va_t^l)),
    \end{aligned}
\end{equation}
where $\vh_t^l$ denotes the hidden state of token $t$ at layer $l$, $\va_t^l$ is the attention output, and $\vm_t^l$ is the feedforward (FFN) output. Here, $\mW_{\text{in}}^l$ and $\mW_{\text{out}}^l$ are weight matrices, $\sigma$ is a nonlinear activation function, and $\gamma$ denotes layer normalization.

\textbf{Key-Value Memory Structure}. Locate-then-edit assumes that factual knowledge is stored in the FFN layers and treats them as linear associative memory \cite{key_value}. Specifically, $\mW_{\text{out}}^l$ is conceptualized as a key-value memory structure:
\begin{equation}
    \begin{aligned}
        \underbrace{\vm_t^l}_{\vv} = \mW_{\text{out}}^l \underbrace{\sigma(\mW_{\text{in}}^l \gamma(\vh_t^{l-1}+\va^l))}_{\vk}. \label{eqapp:define_kv}
    \end{aligned}
\end{equation}
Here, the MLP input-output pair at token $t$ and layer $l$ serves as the key-value pair. Casual Tracing is typically used to locate the target token and layer by injecting Gaussian noise into hidden states and incrementally restoring them to analyze output recovery. For more details, please refer to ROME \cite{ROME}.

\textbf{Computing Key-Value.} For editing knowledge $(x_e, y_e)$, we compute its corresponding key-value pair $(\vk^*, \vv^*)$. The key $\vk^*$ is derived via forward propagation of $x_e$, while the value $\vv^*$ is optimized using gradient descent:
\begin{equation}
    \vv^* = \vv + \arg \min_{\bm{\delta}^l} \left( -\log \mathbb{P}_{f(\vh_t^l + \bm{\delta}^l)} [y_e \mid x_e] \right).
\end{equation}
Here, $f(\vh_t^l + \bm{\delta}^l)$ represents the model output when the FFN output $\vh_t^l$ is replaced with $\vh_t^l + \bm{\delta}^l$. 

Methods such as ROME \cite{ROME}, MEMIT \cite{MEMIT}, and AlphaEdit \cite{AlphaEdit} focus on triplets $(s, r, o)$, selecting the last token of the subject $s$ as the target token. In contrast, UnKE \cite{UnKE} extends to unstructured text, using the last token of $x_e$ as the target.

To insert new knowledge $(\vk^*, \vv^*)$ into the key-value memory, we solve the constrained least squares problem:
\begin{align*}
    \min_{\hat{\mW}} &\quad \left\lVert \hat{\mW}\mK - \mV \right\rVert \\
    \text{s.t.} &\quad \hat{\mW}\vk^* = \vv^*.
\end{align*}
The final parameter update can be computed via ROME/MEMIT/AlphaEdit's closed-form solution or UnKE's gradient-based optimization.

For clarity, let $\tilde{\mW}$ denote the edited weight of $\mW_{\text{out}}^l$ in the MLP, and let $\mW$ represent its original weight. The final parameter update can be computed using the closed-form solutions of ROME/MEMIT/AlphaEdit or the gradient-based optimization method in UnKE.

\textbf{Weights Update in ROME.} The ROME method derives a closed-form solution to the constrained least-squares problem for updating MLP parameters:
\begin{equation}
    \tilde{\mW} = \mW + \frac{(\vv^\ast - \mW\vk^\ast) (\mC^{-1} \vk^\ast) ^ {T}}{(\mC^{-1} \vk^\ast) ^ {T} \vk^\ast},
\end{equation}
where $\mC = \mK \mK^T$. The matrix $\mC$ is estimated using 100,000 samples of hidden states $\vk$ obtained from tokens sampled in-context from the entire Wikipedia dataset.

\textbf{Weights Update in MEMIT.} Since the above solution updates only a single knowledge sample at a time, MEMIT improves upon it by avoiding Lagrange multipliers and instead using a relaxed constraint formulation. The problem is reformulated by maintaining a factual set $\{\mK_1, \mV_1\}$ containing $u$ new associations while preserving the original set $\{\mK_0, \mV_0\}$ with $n$ existing associations:
\begin{equation}
\begin{gathered}
    \mK_0 = \left[\vk_1 \mid \vk_2 \mid \dots \mid \vk_n\right], \quad \mV_0 = \left[\vv_1 \mid \vv_2 \mid \dots \mid \vv_n\right], \\
    \mK_1 = \left[\vk^\ast_{n+1} \mid \vk^\ast_{n+2} \mid \dots \mid \vk^\ast_{n+u}\right], \quad \mV_1 = \left[\vv^\ast_{n+1} \mid \vv^\ast_{n+2} \mid \dots \mid \vv^\ast_{n+u}\right].
\end{gathered}
\end{equation}
Here, $\vk$ and $\vv$ are defined as in Eq.~\ref{eqapp:define_kv}, and their subscripts denote knowledge indices. The objective function is given by:
\begin{equation}
    \tilde{\mW} \triangleq \argmin_{\hat{\mW}} \left( \sum_{i=1}^{n} \left\| \hat{\mW} \vk_i - \vv_i \right\|^2 + \sum_{i=n+1}^{n+u} \left\| \hat{\mW} \vk_i - \vv^\ast_i \right\|^2 \right).
\end{equation}
Applying the normal equation \citep{normal_equation}, the closed-form solution is:
\begin{equation}
    \tilde{\mW} = \left( \mV_1 - \mW \mK_1 \right) \mK_1^T \left( \mK_0 \mK_0^T + \mK_1 \mK_1^T \right)^{-1} + \mW.
\end{equation}

\textbf{Weights Update in AlphaEdit.} AlphaEdit addresses the imbalance between old and new knowledge in lifelong learning. It protects existing knowledge using a null-space projection constraint, ensuring that the update $\bm{\Delta}$ to $\mW_{\text{out}}^l$ is always projected onto the null space of $\mK_0 \mK_0^T$. The final parameter update, refining MEMIT, is:
\begin{equation}
    \tilde{\mW} = \left( \mV_1 - \mW \mK_1 \right) \mK_1^T \mP \left( \mK_p \mK_p^T \mP + \mK_1 \mK_1^T \mP + \mI \right)^{-1}+ \mW.
\end{equation}

\textbf{Weights Update in UnKE.} Unlike previous methods, UnKE considers the entire input to layer $l$, denoted as $f^l$, rather than just the MLP input. The output remains $f^l$'s activation values. The parameter update is applied to the entire layer rather than a single weight matrix. Given the knowledge sets $\{\mK_0, \mV_0\}$ and $\{\mK_1, \mV_1\}$, the optimization objective is formulated as:
\begin{equation}
    \tilde{\Theta}^l \triangleq \argmin_{\hat{\Theta}^l} \left( \sum_{i=1}^{n} \left\|  f_{\hat{\Theta}^l}^l(\vk_i) - \vv_i \right\|^2 + \sum_{i=n+1}^{n+u} \left\|  f_{\hat{\Theta}^l}^l(\vk_i) - \vv^\ast_i \right\|^2 \right),
\end{equation}
where $\Theta^l$ denotes the entire set of parameters in layer $l$. Since a closed-form solution is not feasible, UnKE employs gradient descent to iteratively update $\Theta^l$.

\subsection{Proof of Optimization-Conditional Mutual Information Equivalence} \label{app:proof_cmi}
\begin{theorem}
The optimization objective  
\begin{equation}
    \bm{\delta}^* = \argmin_{\bm{\delta}} \left( -\log \mathbb{P}_{f(\vh_t+\bm{\delta})}(Y \mid X) \right), \label{eq:opt}
\end{equation}  
is equivalent to maximizing the conditional mutual information (CMI) between $X$ and $Y$ given the perturbed hidden state $\vh'$:  
\begin{equation}
    \vh' = \argmax_{\vh'} I(X; Y \mid \vh'). \label{eq:cmi}
\end{equation}
\end{theorem}

\begin{proof}
Starting from the definition of CMI, we expand it via the integral form:  
\begin{equation}
I(X; Y \mid \vh') = \int p(x, y, \vh') \log \frac{p(y \mid x, \vh')}{p(y \mid \vh')} \, dx dy d\vh'.
\end{equation}  
% Applying Bayes’ rule $p(y \mid x, \vh') = \frac{p(x, y \mid \vh')}{p(x \mid \vh')}$, we rewrite the integrand:  
% \begin{equation}
% I(X; Y \mid \vh') = \int p(x, y, \vh') \log \frac{p(x, y \mid \vh')}{p(x \mid \vh') p(y \mid \vh')} \, dx dy d\vh'.
% \end{equation}  
This splits into two entropy terms:  
\begin{align}
I(X; Y \mid \vh') = \underbrace{\int p(x, y, \vh') \log p(y \mid x, \vh') \, dx dy d\vh'}_{\text{Term } \mathcal{A}} - \underbrace{\int p(x, y, \vh') \log p(y \mid \vh') \, dx dy d\vh'}_{\text{Term } \mathcal{B}}. \label{eq:split}
\end{align}  

Term $\mathcal{A}$ simplifies to the expectation:  
\begin{equation}
\mathcal{A} = \mathbb{E}_{p(\vh')} \mathbb{E}_{p(x, y \mid \vh')} \left[ \log p(y \mid x, \vh') \right],
\end{equation}  
while Term $\mathcal{B}$ is independent of $X$ given $\vh'$. Since $\mathcal{B}$ does not affect the optimization over $\vh'$, we focus on maximizing $\mathcal{A}$.  

By definition, $\mathbb{P}_{f(\vh')}(Y \mid X) = p(y \mid x, \vh')$. Thus, minimizing the negative log-likelihood in \eqref{eq:opt} directly maximizes $\mathcal{A}$, which is equivalent to maximizing $I(X; Y \mid \vh')$. Substituting $\vh' = \vh_t + \bm{\delta}^*$, we conclude:  
\begin{equation}
\vh' = \argmax_{\vh'} I(X; Y \mid \vh'),
\end{equation}  
thereby establishing the equivalence.  
\end{proof}

\subsection{Proof of the Decomposition of Mutual Information}\label{app:proof_decom}
To rigorously derive Equation \eqref{eq:final_MI}, we start from the mutual information (MI) decomposition given in Equation \eqref{eq:decom}:
\begin{equation}
    I(X; Y \mid \vh'_1, \dots, \vh'_K) = \sum_{k=1}^{K} I(X; Y_k \mid Y_1, \dots, Y_{k-1}, \vh'_1, \dots, \vh'_K).
\end{equation}

\textbf{Step 1: Application of the First Property.}
The first key property states that later hidden states do not influence earlier token generation:
\begin{equation}
    H(Y_p \mid \vh'_q) = H(Y_p), \quad \text{for } p < q.
\end{equation}
Since mutual information is defined as:
\begin{equation}
    I(X; Y_k \mid Y_1, \dots, Y_{k-1}, \vh'_1, \dots, \vh'_K) = H(Y_k \mid Y_1, \dots, Y_{k-1}, \vh'_1, \dots, \vh'_K) - H(Y_k \mid X, Y_1, \dots, Y_{k-1}, \vh'_1, \dots, \vh'_K).
\end{equation}
Since $\vh'_q$ for $q > k$ does not affect $Y_k$, we can simplify:
\begin{equation}
    H(Y_k \mid Y_1, \dots, Y_{k-1}, \vh'_1, \dots, \vh'_K) = H(Y_k \mid Y_1, \dots, Y_{k-1}, \vh'_1, \dots, \vh'_k).
\end{equation}

\textbf{Step 2: Application of the Second Property.}
The second key property states that once $Y_k$ is determined, conditioning on $Y_k$ subsumes conditioning on $\vh'_k$:
\begin{equation}
    H(\cdot \mid Y_k) = H(\cdot \mid Y_k, \vh'_k).
\end{equation}
Using this, we rewrite the MI term:
\begin{equation}
    I(X; Y_k \mid Y_1, \dots, Y_{k-1}, \vh'_1, \dots, \vh'_K) = I(X; Y_k \mid Y_1, \dots, Y_{k-1}, \vh'_k).
\end{equation}

\textbf{Step 3: Applying the Conditional Mutual Information Decomposition.}
Using the decomposition formula for conditional mutual information, each term can be written as:
\begin{equation}
    I(X; Y_k \mid Y_1, \dots, Y_{k-1}, \vh'_k) = I(X, Y_1, \dots, Y_{k-1}; Y_k \mid \vh'_k) - I(Y_1, \dots, Y_{k-1}; Y_k \mid \vh'_k).
\end{equation}
In the optimization process, since we are given $X$, the second term is ignored:
\begin{equation}
    I(X; Y_k \mid Y_1, \dots, Y_{k-1}, \vh'_k) = I(X, Y_1, \dots, Y_{k-1}; Y_k \mid \vh'_k).
\end{equation}
Substituting this result back into our summation, we arrive at the desired decomposition:
\begin{equation}
    I(X; Y \mid \vh'_1, \dots, \vh'_K) = \sum_{k=1}^{K} I(X, Y_1, \dots, Y_{k-1}; Y_k \mid \vh'_k).
\end{equation}
This completes the proof.

\section{More Experimental Results}\label{app:exp_result}
\subsection{Case Study}

We selected several editing samples from the UnKEBench and EditEverything datasets as case studies to analyze the long-form knowledge edit performance. 

From the experimental results, AnyEdit effectively restores the target text in terms of both textual consistency and generalization to paraphrase questions. In contrast, other baseline methods often fail to achieve effective editing. Although the UnKE method also demonstrates strong editing performance, it exhibits a noticeable performance drop on paraphrase questions.


\subsubsection{Case 1}

\begin{tcolorbox}[boxrule=0.5pt, left=0pt, right=0pt, top=2.5pt, bottom=2.5pt, title={Model Editing Case Study on Llama3-8B-Instruct}]
    \centering
    \renewcommand{\arraystretch}{1.8}
    \begin{tabular}{p{0.18\linewidth} p{0.76\linewidth}}
        \makecell[c]{\textbf{Input}} & \makecell[c]{What is John Mayne's background and experience in journalism?} \\
        \hline
        \makecell[c]{\textbf{Paraphrase}} & \makecell[c]{Can you provide information about John Mayne's journalism background and experience?} \\
        \hline
        \makecell[c]{\textbf{Target Output}} & \makecell*[{{p{13cm}}}]{John Mayne is a journalist who has been working in the field for over a decade. He has written for several reputable publications, including \tar{The New York Times} and \tar{The Guardian}. In an interview with \tar{The Huffington Post}, Mayne discussed his passion for journalism and his commitment to reporting on important issues. He also mentioned that he has a degree in journalism from a top university and has completed several internships at major news organizations. Mayne's extensive experience and education make him a highly qualified journalist and a valuable asset to any newsroom.} \\
        \hline
        \multicolumn{2}{c}{\textbf{\large LLM-edited Generation}} \\ % Bold and centered
        \hline
        \makecell[c]{\textbf{MEMIT}} & \makecell*[{{p{13cm}}}]{John Mayne is a journalist and writer, but I couldn't find much about his personal background. However, he has been in the industry for several years, contributing to various publications. According to LinkedIn, he has worked for major outlets including BBC and \tar{The Guardian}(...) John Mayne's experience spans politics, business, and culture, and he has been recognized for his in-depth reporting.} \\
        \hline
        \makecell[c]{\textbf{AlphaEdit}} & \makecell*[{{p{13cm}}}]{John Mayne is a British journalist and author with a long career in media. He has written for major newspapers such as The Times and The Sunday Times. Mayne has covered diverse topics, including politics and culture, and worked as a foreign correspondent reporting from the Middle East, Africa, and Asia... In addition, he has taught journalism at universities and received multiple awards for his writing.} \\
        \hline
        \makecell[c]{\textbf{UnKE}} & \makecell*[{{p{13cm}}}]{John Mayne is a journalist who has been working in the field for several years. He has written for a number of publications and has a strong track record of producing high-quality, well-researched articles. In addition to his writing experience, Mayne has also worked as an editor and has a strong understanding of the journalism industry. He is well-respected by his peers and is known for his ability to produce engaging and informative articles.} \\
        \hline
        \makecell[c]{\textbf{AnyEdit}} & \makecell*[{{p{13cm}}}]{ John Mayne is a journalist who has been working in the field for over a decade. He has written for several reputable publications, including \tar{The New York Times} and \tar{The Guardian}. In an interview with \tar{The Huffington Post}, Mayne discussed his passion for journalism and his commitment to reporting on important issues. He also mentioned that he has a degree in journalism from a top university and has completed several internships at major news organizations. Mayne's extensive experience and education make him a highly qualified journalist and a valuable asset to any newsroom.} \\
    \end{tabular}
\end{tcolorbox}

\subsubsection{Case 2}

\begin{tcolorbox}[boxrule=0.5pt, left=0pt, right=0pt, top=2.5pt, bottom=2.5pt, title={Model Editing Case Study on Qwen2.5-7B-Instruct}]
    \centering
    \renewcommand{\arraystretch}{1.8}
    \begin{tabular}{p{0.18\linewidth} p{0.76\linewidth}}
        \makecell[c]{\textbf{Input}} & \makecell[c]{find the value of $log_y(x^4)$ if $log_x(y^3) = 2$} \\
        \hline
        \makecell[c]{\textbf{Target Output}} & \makecell*[{{p{13cm}}}]{To find the value of $log_y(x^4)$, we can use the property of logarithms that states: $log_b(a) = 1 / log_a(b)$. Given that $log_x(y^3) = 2$, we can rewrite this as: $y^3 = x^2$. Now, we want to find $log_y(x^4)$. We can use the property mentioned above: $log_y(x^4) = 1 / log_x(y^4)$. But we don't have $log_x(y^4)$; we have $log_x(y^3)$. So, we need to manipulate the expression to use the given information. We know that $y^3 = x^2$, so we can raise both sides to the power of $4/3$ to get $y^4$: $(y^3)^{(4/3)} = (x^2)^{(4/3)}$. $y^4 = x^{(8/3)}$. Now we can write $log_x(y^4)$ in terms of $log_x(y^3)$: $log_x(y^4) = log_x((y^3)^{(4/3)})$. Using the power rule of logarithms ($log_b(a^c) = c * log_b(a)$), we get: $log_x(y^4) = (4/3) * log_x(y^3)$. Since we know $log_x(y^3) = 2$, we can substitute this value in: $log_x(y^4) = (4/3) * 2$. $log_x(y^4) = 8/3$. Now we can find $log_y(x^4)$ using the reciprocal property: $log_y(x^4) = 1 / log_x(y^4)$. $log_y(x^4) = 1 / (8/3)$. $log_y(x^4) = 3/8$. Therefore, the value of $log_y(x^4)$ is \tar{$3/8$}.} \\
        \hline
        \multicolumn{2}{c}{\textbf{\large LLM-edited Generation}} \\ % Bold and centered
        \hline
        \makecell[c]{\textbf{MEMIT}} & \makecell*[{{p{13cm}}}]{To find the value of $log_y(x^4)$, we can use the change of base formula and the given information. Given: $log_x(y^3) = 2$. Using the change of base formula, we can rewrite this as: $log_x(y^3) = (log_y(y^3)) / (log_y(x))$. Since $log_y(y^3) = 3$ (because $y^3$ is the argument raised to the power of 3), we have: $2 = 3 / (log_y(x))$. Now, we can solve for $log_y(x)$: $log_y(x) = 3/2$. Next, we want to find $log_y(x^4)$. Using the power rule of logarithms, we can write: $log_y(x^4) = 4 * log_y(x)$. Substituting the value we found for $log_y(x)$: $log_y(x^4) = 4 * (3/2)$. Simplifying: $log_y(x^4) = 6$. Therefore, the value of $log_y(x^4)$ is $6$.} \\
        \hline
        \makecell[c]{\textbf{AlphaEdit}} & \makecell*[{{p{13cm}}}]{To find the value of $log_y(x^4)$, we can use the change of base formula and the given information: $log_x(y^3) = 2$. Using the change of base formula, we can rewrite this as: $y^3 = x^2$. Now, we want to find $log_y(x^4)$. Using the properties of logarithms, we can rewrite this as: $log_y(x^4) = 4 * log_y(x)$. We can express $log_y(x)$ in terms of $log_x(y)$ using the change of base formula: $log_y(x) = 1 / log_x(y)$. Since we know that $log_x(y^3) = 2$, we can write: $log_x(y^3) = 3 * log_x(y) = 2$. Therefore: $log_x(y) = 2/3$. Substituting this into our expression for $log_y(x)$: $log_y(x) = 1 / (2/3) = 3/2$. Finally, substituting this back into our expression for $log_y(x^4)$: $log_y(x^4) = 4 * (3/2) = 6$. Therefore, the value of $log_y(x^4)$ is $6$.} \\
        \hline
        \makecell[c]{\textbf{UnKE}} & \makecell*[{{p{13cm}}}]{To find the value of $log_y(x^4)$, we can use the property of logarithms that states: $log_b(a) = 1 / log_a(b)$. Given that $log_x(y^3) = 2$, we can rewrite this as $y^3 = x^2$. Now, we can use the property to find $log_y(x^4)$: $log_y(x^4) = 1 / log_x(y^3)$ $= 1 / (1 / x^2)$ $= x^2$. So, the value of $log_y(x^4)$ is $x^2$. Therefore, the correct answer is $x^2$.} \\
        \hline
        \makecell[c]{\textbf{AnyEdit}} & \makecell*[{{p{13cm}}}]{To find the value of $log_y(x^4)$, we can use the property of logarithms that states: $log_b(a) = 1 / log_a(b)$. Given that $log_x(y^3) = 2$, we can rewrite this as: $y^3 = x^2$. Now, we want to find $log_y(x^4)$. We can use the property mentioned above: $log_y(x^4) = 1 / log_x(y^4)$. But we don't have $log_x(y^4)$; we have $log_x(y^3)$. So, we need to manipulate the expression to use the given information. We know that $y^3 = x^2$, so we can raise both sides to the power of $4/3$ to get $y^4$: $(y^3)^{(4/3)} = (x^2)^{(4/3)}$. $y^4 = x^{(8/3)}$. Now we can write $log_x(y^4)$ in terms of $log_x(y^3)$: $log_x(y^4) = log_x((y^3)^{(4/3)})$. Using the power rule of logarithms ($log_b(a^c) = c * log_b(a)$), we get: $log_x(y^4) = (4/3) * log_x(y^3)$. Since we know $log_x(y^3) = 2$, we can substitute this value: $log_x(y^4) = (4/3) * 2$. $log_x(y^4) = 8/3$. Now we can use the property of logarithms: $log_y(x^4) = 1 / log_x(y^4)$. $log_y(x^4) = 1 / (8/3)$. $log_y(x^4) = 3/8$. Therefore, the value of $log_y(x^4)$ is \tar{$3/8$}.} \\
    \end{tabular}
\end{tcolorbox}
\clearpage
\newpage

\subsection{Supplementary Experimental Results on RQ1 \& RQ2} \label{app:exp_result_1}
We present a comprehensive evaluation of all metrics on the UnKEBench and AKEW datasets in Table \ref{tab:app_1} and Table \ref{tab:app_2}. The results demonstrate that UnKE consistently outperforms other baselines across both original and paraphrase question evaluations. Notably, UnKE+, which integrates AnyEdit’s autoregressive editing paradigm, achieves even higher scores in lexical similarity (BLEU, ROUGE-1/2/L) and semantic similarity (BERT Score), indicating its superior ability to preserve and generalize edited knowledge. In contrast, MEMIT and AlphaEdit struggle with paraphrase generalization, showing significantly lower performance on the right side of `/', suggesting that these methods fail to robustly transfer edited knowledge across rephrased contexts. While MEMIT+ and AlphaEdit+ improve over their base versions, their performance still lags behind UnKE and UnKE+.

Overall, UnKE+ achieves the best balance between precise knowledge modification and robust generalization, confirming that combining UnKE with autoregressive fine-tuning leads to stronger and more reliable knowledge editing in LLMs.
\begin{table*}[h]
\caption{Performance comparison in UnKEBench. The `+' symbol indicates results incorporating AnyEdit's autoregressive editing paradigm. The left side of `/' represents the LLM's edited output for original questions, while the right side represents the edited output for paraphrase questions.}
    \label{tab:app_1}
    \centering
    \renewcommand{\arraystretch}{1.2}
    \setlength{\tabcolsep}{4pt}
    \resizebox{\textwidth}{!}{
    \begin{tabular}{l cccc ccc}
        \toprule
        \multirow{2}{*}{\textbf{Method}} & \multicolumn{4}{c}{\textbf{Lexical Similarity}} & \multicolumn{1}{c}{\textbf{Semantic Similarity}} & \textbf{Sub Questions} \\
        \cmidrule(lr){2-5} \cmidrule(lr){6-6} \cmidrule(lr){7-7} 
        & BLEU & ROUGE-1 & ROUGE-2 & ROUGE-L & BERT Score & ROUGE-L \\
        \midrule
        \multicolumn{7}{l}{\textbf{Based on Llama3-8B-Instruct}} \\
        \midrule
        UnKE        & 93.56 / 78.09  & 93.61 / 79.26  & 91.42 / 71.73  & 93.33 / 78.42  & 98.34 / 93.38    & 37.87 \\
        UnKE+       & 99.67 / 84.60  & 99.69 / 86.31  & 99.57 / 81.18  & 99.68 / 85.75  & 99.86 / 94.70    & 41.45 \\
        MEMIT       & 25.57 / 22.88  & 32.67 / 30.75  & 14.51 / 12.37  & 30.49 / 28.65  & 76.21 / 74.25    & 22.56 \\
        MEMIT+      & 88.88 / 81.38  & 93.26 / 86.53  & 90.32 / 80.61  & 92.96 / 85.91  & 97.76 / 95.60    & 41.67 \\
        AlphaEdit   & 21.29 / 20.24  & 28.62 / 27.99  & 11.36 / 10.24  & 26.59 / 25.92  & 73.92 / 72.96    & 20.71 \\
        AlphaEdit+  & 75.02 / 66.35  & 81.70 / 73.47  & 74.35 / 62.74  & 80.92 / 72.22  & 94.19 / 91.51    & 40.56 \\
        \midrule
        \multicolumn{7}{l}{\textbf{Based on Qwen2.5-7B-Instruct}} \\
        \midrule
        UnKE        & 91.92 / 70.61  & 91.41 / 68.47  & 87.75 / 56.34  & 91.01 / 67.00  & 96.97 / 89.17    & 38.12 \\
        UnKE+       & 98.52 / 82.48  & 98.85 / 83.36  & 98.43 / 77.03  & 98.82 / 82.60  & 99.35 / 94.81    & 42.24 \\
        MEMIT       & 45.07 / 40.81  & 40.73 / 36.75  & 19.59 / 15.87  & 38.04 / 34.07  & 78.03 / 76.50    & 24.75 \\
        MEMIT+      & 91.31 / 77.23  & 95.10 / 80.88  & 92.93 / 72.50  & 94.89 / 79.98  & 98.05 / 93.56    & 42.38 \\
        AlphaEdit   & 49.71 / 45.21  & 45.42 / 41.06  & 24.63 / 19.85  & 42.77 / 38.26  & 80.48 / 78.38    & 25.37 \\
        AlphaEdit+  & 97.77 / 83.09  & 98.20 / 84.18  & 97.40 / 77.38  & 98.14 / 83.40  & 99.08 / 94.51    & 41.58 \\
        \bottomrule
    \end{tabular}
    }
    
\end{table*}

\begin{table*}[h]
\caption{Performance comparison in AKEW (Counterfact). The `+' symbol indicates results incorporating AnyEdit's autoregressive editing paradigm. The left side of `/` represents the LLM's edited output for original questions, while the right side represents the edited output for paraphrase questions.}
    \label{tab:app_2}
    \centering
    \renewcommand{\arraystretch}{1.2}
    \setlength{\tabcolsep}{4pt}
    \resizebox{\textwidth}{!}{
    \begin{tabular}{l cccc ccc}
        \toprule
        \multirow{2}{*}{\textbf{Method}} & \multicolumn{4}{c}{\textbf{Lexical Similarity}} & \multicolumn{1}{c}{\textbf{Semantic Similarity}} & \textbf{Sub Questions} \\
        \cmidrule(lr){2-5} \cmidrule(lr){6-6} \cmidrule(lr){7-7} 
        & BLEU & ROUGE-1 & ROUGE-2 & ROUGE-L & BERT Score & ROUGE-L \\
        \midrule
        \multicolumn{7}{l}{\textbf{Based on Llama3-8B-Instruct}} \\
        \midrule
        MEMIT       & 33.44 / 18.13  & 34.46 / 17.44  & 16.29 / 4.74   & 32.20 / 16.10  & 76.44 / 47.80  & 39.98\\
        MEMIT+      & 85.41 / 38.78  & 96.07 / 47.61  & 94.21 / 32.37  & 95.87 / 46.00  & 97.76 / 62.63  & 64.07\\
        UnKE        & 98.43 / 36.99  & 98.43 / 34.58  & 97.78 / 19.37  & 98.37 / 32.89  & 99.62 / 59.62  & 63.22\\
        UnKE+       & 99.98 / 45.23  & 99.98 / 46.57  & 99.96 / 35.41  & 99.98 / 45.31  & 99.95 / 64.24  & 59.03\\
        AlphaEdit   & 23.36 / 16.25  & 26.92 / 15.00  & 10.81 / 3.61   & 24.95 / 13.79  & 72.63 / 44.67  & 35.76 \\
        AlphaEdit+  & 79.60 / 40.67  & 84.49 / 41.11  & 78.00 / 26.60  & 83.76 / 39.51  & 96.51 / 65.14  & 57.05 \\
        \midrule
        \multicolumn{7}{l}{\textbf{Based on Qwen2.5-7B-Instruct}} \\
        \midrule
        MEMIT       & 45.29 / 32.83  & 41.68 / 28.01  & 20.38 / 8.79   & 38.95 / 25.73  & 77.19 / 56.04  & 43.51\\
        MEMIT+      & 90.55 / 44.32  & 95.33 / 45.56  & 93.12 / 27.38  & 95.09 / 43.49  & 98.08 / 65.40  & 55.10\\
        UnKE        & 91.53 / 38.59  & 90.91 / 31.53  & 87.06 / 12.11  & 90.44 / 29.27  & 97.34 / 59.29  & 49.97\\
        UnKE+       & 98.95 / 34.68  & 99.01 / 35.23  & 98.59 / 15.59  & 98.99 / 32.95  & 99.63 / 60.78  & 51.58\\
        AlphaEdit   & 49.97 / 34.65  & 48.15 / 30.02  & 27.76 / 10.38  & 45.55 / 27.69  & 80.66 / 56.99  & 45.12\\
        AlphaEdit+  & 97.61 / 46.97  & 97.80 / 47.63  & 96.89 / 30.31  & 97.73 / 45.84  & 99.10 / 66.10  & 54.99\\
        \bottomrule
    \end{tabular}
    }
    
\end{table*}

\subsection{Supplementary Experimental Results on RQ4}\label{app:exp_result_4}
\begin{figure}[t]
\begin{center}
\includegraphics[width=0.6\linewidth, keepaspectratio]{figures/exp_3.png}
\caption{The relationship between AnyEdit's editing performance and chunk size in long-form diverse-formatted knowledge.}
\label{fig:exp_3}
\end{center}
\end{figure}


 The experimental results of relationship between AnyEdit's editing performance and chunk size in long-form diverse-formatted knowledge are presented in Figure \ref{fig:exp_3}. Based on these results, we draw the following observation:.

\begin{itemize}[leftmargin=*]
    \item \textbf{Obs 7: The editing performance of AnyEdit is influenced by chunk size.}  
    As the chunk size increases beyond a certain threshold, the editing performance of AnyEdit declines. Specifically, when the chunk size is smaller, each iteration of editing becomes more manageable, leading to improved overall performance. However, this improvement comes at the cost of increased editing time due to the larger number of iterations required for longer texts. Conversely, when the chunk size is larger, it becomes challenging to achieve effective edits within a single iteration, resulting in degraded performance. Based on this trade-off, we recommend using a balanced chunk size of 40 for most editing scenarios.
\end{itemize}

\begin{figure}[h]
    \centering
    \includegraphics[width=\textwidth]{figures/data1.png}
    \vspace{-5mm}
    \caption{A Sample of the AKEW (Counterfact) dataset.}
    \label{fig:sample1}
\end{figure}

\begin{figure}[h]
    \centering
    \includegraphics[width=\textwidth]{figures/data2.png}
    \vspace{-5mm}
    \caption{A Sample of the UnKEBench dataset.}
    \label{fig:sample2}
\end{figure}

\begin{figure}[h]
    \centering
    \includegraphics[width=\textwidth]{figures/data3.png}
    \vspace{-5mm}
    \caption{Samples of the EditEverything dataset.}
    \label{fig:sample3}
\end{figure}

\end{document}
