\begin{figure}[!t]
    \centering
    \includegraphics[width=\textwidth]{figure/image/FedTextGrad_Framework.pdf}
    \caption{Illustration of \ours{}, where the upper part (blue boxes) and the lower part (green boxes) represent two different clients. \ul{\textit{Within each client}}, circles represent the prompts, and boxes represent the LLM. \ours{} consists of four steps for local updating, proceeding from left to right. In \texttt{step-1 (Prompt)}, the client is tasked with answering the \textcolor{LimeGreen}{\textit{Query}} by initializing a \textcolor{CornflowerBlue}{\textit{Prompt}} to the LLM to obtain a response. Then, in \texttt{step-2 (Response)}, the LLM performs multi-step reasoning (e.g., CoT) and generates a \textcolor{Orchid}{Response}. In \texttt{step-3 (Evaluation)}, the \textcolor{Orchid}{Response} is evaluated against the ground truth by the LLM, and a \textcolor{Bittersweet}{Evaluation} score is generated. Finally, in \texttt{step-4 (Textual Grad)}, the \textcolor{CornflowerBlue}{\textit{Prompt}} is updated "backward" based on feedback from the LLM. After this, the client sends the \textcolor{red}{\textit{Updated Prompt}} to the server. \ul{\textit{On the server-side}}, the collected prompts from all clients are aggregated by the server, which acts as a trusted third party, and then sent back to the clients, as shown in \texttt{step-5}. Two aggregation strategies are available: simply concatenating the prompts or using the server-side LLM to summarize them. The system iteratively performs local updates (multiple local epochs of \texttt{steps 1-4}) and global aggregation (\texttt{step-5}) for optimization in the FL system.
}
    \label{fig:fedtextgrad_framework}
    \vspace{-5mm}
\end{figure}
