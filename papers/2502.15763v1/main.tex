\immediate\write18{makeindex \jobname.nlo -s nomencl.ist -o \jobname.nls}
%% bare_jrnl.tex
%% V1.4b
%% 2015/08/26
%% by Michael Shell
%% see http://www.michaelshell.org/
%% for current contact information.
%%
%% This is a skeleton file demonstrating the use of IEEEtran.cls
%% (requires IEEEtran.cls version 1.8b or later) with an IEEE
%% journal paper.
%%
%% Support sites:
%% http://www.michaelshell.org/tex/ieeetran/
%% http://www.ctan.org/pkg/ieeetran
%% and
%% http://www.ieee.org/

%%*************************************************************************
%% Legal Notice:
%% This code is offered as-is without any warranty either expressed or
%% implied; without even the implied warranty of MERCHANTABILITY or
%% FITNESS FOR A PARTICULAR PURPOSE! 
%% User assumes all risk.
%% In no event shall the IEEE or any contributor to this code be liable for
%% any damages or losses, including, but not limited to, incidental,
%% consequential, or any other damages, resulting from the use or misuse
%% of any information contained here.
%%
%% All comments are the opinions of their respective authors and are not
%% necessarily endorsed by the IEEE.
%%
%% This work is distributed under the LaTeX Project Public License (LPPL)
%% ( http://www.latex-project.org/ ) version 1.3, and may be freely used,
%% distributed and modified. A copy of the LPPL, version 1.3, is included
%% in the base LaTeX documentation of all distributions of LaTeX released
%% 2003/12/01 or later.
%% Retain all contribution notices and credits.
%% ** Modified files should be clearly indicated as such, including  **
%% ** renaming them and changing author support contact information. **
%%*************************************************************************


% *** Authors should verify (and, if needed, correct) their LaTeX system  ***
% *** with the testflow diagnostic prior to trusting their LaTeX platform ***
% *** with production work. The IEEE's font choices and paper sizes can   ***
% *** trigger bugs that do not appear when using other class files.       ***                          ***
% The testflow support page is at:
% http://www.michaelshell.org/tex/testflow/


\newtheorem{theorem}{theorem}
\newtheorem{lemma}{lemma}
\documentclass[journal]{IEEEtran}
%
% If IEEEtran.cls has not been installed into the LaTeX system files,
% manually specify the path to it like:
% \documentclass[journal]{../sty/IEEEtran}





% Some very useful LaTeX packages include:
% (uncomment the ones you want to load)


% *** MISC UTILITY PACKAGES ***
%
%\usepackage{ifpdf}
% Heiko Oberdiek's ifpdf.sty is very useful if you need conditional
% compilation based on whether the output is pdf or dvi.
% usage:
% \ifpdf
%   % pdf code
% \else
%   % dvi code
% \fi
% The latest version of ifpdf.sty can be obtained from:
% http://www.ctan.org/pkg/ifpdf
% Also, note that IEEEtran.cls V1.7 and later provides a builtin
% \ifCLASSINFOpdf conditional that works the same way.
% When switching from latex to pdflatex and vice-versa, the compiler may
% have to be run twice to clear warning/error messages.






% *** CITATION PACKAGES ***
%
\usepackage{cite}
\usepackage{hyperref}
% cite.sty was written by Donald Arseneau
% V1.6 and later of IEEEtran pre-defines the format of the cite.sty package
% \cite{} output to follow that of the IEEE. Loading the cite package will
% result in citation numbers being automatically sorted and properly
% "compressed/ranged". e.g., [1], [9], [2], [7], [5], [6] without using
% cite.sty will become [1], [2], [5]--[7], [9] using cite.sty. cite.sty's
% \cite will automatically add leading space, if needed. Use cite.sty's
% noadjust option (cite.sty V3.8 and later) if you want to turn this off
% such as if a citation ever needs to be enclosed in parenthesis.
% cite.sty is already installed on most LaTeX systems. Be sure and use
% version 5.0 (2009-03-20) and later if using hyperref.sty.
% The latest version can be obtained at:
% http://www.ctan.org/pkg/cite
% The documentation is contained in the cite.sty file itself.


\usepackage{multirow}



% *** GRAPHICS RELATED PACKAGES ***
%
\ifCLASSINFOpdf
   \usepackage[pdftex]{graphicx}
  % declare the path(s) where your graphic files are
   \graphicspath{{../pdf/}{../jpeg/}}
  % and their extensions so you won't have to specify these with
  % every instance of \includegraphics
   \DeclareGraphicsExtensions{.pdf,.jpeg,.png}
\else
  % or other class option (dvipsone, dvipdf, if not using dvips). graphicx
  % will default to the driver specified in the system graphics.cfg if no
  % driver is specified.
   \usepackage[dvips]{graphicx}
  % declare the path(s) where your graphic files are
   \graphicspath{{../eps/}}
  % and their extensions so you won't have to specify these with
  % every instance of \includegraphics
   \DeclareGraphicsExtensions{.eps}
\fi
% graphicx was written by David Carlisle and Sebastian Rahtz. It is
% required if you want graphics, photos, etc. graphicx.sty is already
% installed on most LaTeX systems. The latest version and documentation
% can be obtained at: 
% http://www.ctan.org/pkg/graphicx
% Another good source of documentation is "Using Imported Graphics in
% LaTeX2e" by Keith Reckdahl which can be found at:
% http://www.ctan.org/pkg/epslatex
%
% latex, and pdflatex in dvi mode, support graphics in encapsulated
% postscript (.eps) format. pdflatex in pdf mode supports graphics
% in .pdf, .jpeg, .png and .mps (metapost) formats. Users should ensure
% that all non-photo figures use a vector format (.eps, .pdf, .mps) and
% not a bitmapped formats (.jpeg, .png). The IEEE frowns on bitmapped formats
% which can result in "jaggedy"/blurry rendering of lines and letters as
% well as large increases in file sizes.
%
% You can find documentation about the pdfTeX application at:
% http://www.tug.org/applications/pdftex





% *** MATH PACKAGES ***
%
\usepackage{amsmath}
% A popular package from the American Mathematical Society that provides
% many useful and powerful commands for dealing with mathematics.
%
% Note that the amsmath package sets \interdisplaylinepenalty to 10000
% thus preventing page breaks from occurring within multiline equations. Use:
\interdisplaylinepenalty=2500
% after loading amsmath to restore such page breaks as IEEEtran.cls normally
% does. amsmath.sty is already installed on most LaTeX systems. The latest
% version and documentation can be obtained at:
% http://www.ctan.org/pkg/amsmath





% *** SPECIALIZED LIST PACKAGES ***
%
%\usepackage{algorithmic}
% algorithmic.sty was written by Peter Williams and Rogerio Brito.
% This package provides an algorithmic environment fo describing algorithms.
% You can use the algorithmic environment in-text or within a figure
% environment to provide for a floating algorithm. Do NOT use the algorithm
% floating environment provided by algorithm.sty (by the same authors) or
% algorithm2e.sty (by Christophe Fiorio) as the IEEE does not use dedicated
% algorithm float types and packages that provide these will not provide
% correct IEEE style captions. The latest version and documentation of
% algorithmic.sty can be obtained at:
% http://www.ctan.org/pkg/algorithms
% Also of interest may be the (relatively newer and more customizable)
% algorithmicx.sty package by Szasz Janos:
% http://www.ctan.org/pkg/algorithmicx




% *** ALIGNMENT PACKAGES ***
%
\usepackage{array}
% Frank Mittelbach's and David Carlisle's array.sty patches and improves
% the standard LaTeX2e array and tabular environments to provide better
% appearance and additional user controls. As the default LaTeX2e table
% generation code is lacking to the point of almost being broken with
% respect to the quality of the end results, all users are strongly
% advised to use an enhanced (at the very least that provided by array.sty)
% set of table tools. array.sty is already installed on most systems. The
% latest version and documentation can be obtained at:
% http://www.ctan.org/pkg/array


% IEEEtran contains the IEEEeqnarray family of commands that can be used to
% generate multiline equations as well as matrices, tables, etc., of high
% quality.




% *** SUBFIGURE PACKAGES ***
%\ifCLASSOPTIONcompsoc
%  \usepackage[caption=false,font=normalsize,labelfont=sf,textfont=sf]{subfig}
%\else
%  \usepackage[caption=false,font=footnotesize]{subfig}
%\fi
% subfig.sty, written by Steven Douglas Cochran, is the modern replacement
% for subfigure.sty, the latter of which is no longer maintained and is
% incompatible with some LaTeX packages including fixltx2e. However,
% subfig.sty requires and automatically loads Axel Sommerfeldt's caption.sty
% which will override IEEEtran.cls' handling of captions and this will result
% in non-IEEE style figure/table captions. To prevent this problem, be sure
% and invoke subfig.sty's "caption=false" package option (available since
% subfig.sty version 1.3, 2005/06/28) as this is will preserve IEEEtran.cls
% handling of captions.
% Note that the Computer Society format requires a larger sans serif font
% than the serif footnote size font used in traditional IEEE formatting
% and thus the need to invoke different subfig.sty package options depending
% on whether compsoc mode has been enabled.
%
% The latest version and documentation of subfig.sty can be obtained at:
% http://www.ctan.org/pkg/subfig




% *** FLOAT PACKAGES ***
%
%\usepackage{fixltx2e}
% fixltx2e, the successor to the earlier fix2col.sty, was written by
% Frank Mittelbach and David Carlisle. This package corrects a few problems
% in the LaTeX2e kernel, the most notable of which is that in current
% LaTeX2e releases, the ordering of single and double column floats is not
% guaranteed to be preserved. Thus, an unpatched LaTeX2e can allow a
% single column figure to be placed prior to an earlier double column
% figure.
% Be aware that LaTeX2e kernels dated 2015 and later have fixltx2e.sty's
% corrections already built into the system in which case a warning will
% be issued if an attempt is made to load fixltx2e.sty as it is no longer
% needed.
% The latest version and documentation can be found at:
% http://www.ctan.org/pkg/fixltx2e


%\usepackage{stfloats}
% stfloats.sty was written by Sigitas Tolusis. This package gives LaTeX2e
% the ability to do double column floats at the bottom of the page as well
% as the top. (e.g., "\begin{figure*}[!b]" is not normally possible in
% LaTeX2e). It also provides a command:
%\fnbelowfloat
% to enable the placement of footnotes below bottom floats (the standard
% LaTeX2e kernel puts them above bottom floats). This is an invasive package
% which rewrites many portions of the LaTeX2e float routines. It may not work
% with other packages that modify the LaTeX2e float routines. The latest
% version and documentation can be obtained at:
% http://www.ctan.org/pkg/stfloats
% Do not use the stfloats baselinefloat ability as the IEEE does not allow
% \baselineskip to stretch. Authors submitting work to the IEEE should note
% that the IEEE rarely uses double column equations and that authors should try
% to avoid such use. Do not be tempted to use the cuted.sty or midfloat.sty
% packages (also by Sigitas Tolusis) as the IEEE does not format its papers in
% such ways.
% Do not attempt to use stfloats with fixltx2e as they are incompatible.
% Instead, use Morten Hogholm'a dblfloatfix which combines the features
% of both fixltx2e and stfloats:
%
% \usepackage{dblfloatfix}
% The latest version can be found at:
% http://www.ctan.org/pkg/dblfloatfix




%\ifCLASSOPTIONcaptionsoff
%  \usepackage[nomarkers]{endfloat}
% \let\MYoriglatexcaption\caption
% \renewcommand{\caption}[2][\relax]{\MYoriglatexcaption[#2]{#2}}
%\fi
% endfloat.sty was written by James Darrell McCauley, Jeff Goldberg and 
% Axel Sommerfeldt. This package may be useful when used in conjunction with 
% IEEEtran.cls'  captionsoff option. Some IEEE journals/societies require that
% submissions have lists of figures/tables at the end of the paper and that
% figures/tables without any captions are placed on a page by themselves at
% the end of the document. If needed, the draftcls IEEEtran class option or
% \CLASSINPUTbaselinestretch interface can be used to increase the line
% spacing as well. Be sure and use the nomarkers option of endfloat to
% prevent endfloat from "marking" where the figures would have been placed
% in the text. The two hack lines of code above are a slight modification of
% that suggested by in the endfloat docs (section 8.4.1) to ensure that
% the full captions always appear in the list of figures/tables - even if
% the user used the short optional argument of \caption[]{}.
% IEEE papers do not typically make use of \caption[]'s optional argument,
% so this should not be an issue. A similar trick can be used to disable
% captions of packages such as subfig.sty that lack options to turn off
% the subcaptions:
% For subfig.sty:
% \let\MYorigsubfloat\subfloat
% \renewcommand{\subfloat}[2][\relax]{\MYorigsubfloat[]{#2}}
% However, the above trick will not work if both optional arguments of
% the \subfloat command are used. Furthermore, there needs to be a
% description of each subfigure *somewhere* and endfloat does not add
% subfigure captions to its list of figures. Thus, the best approach is to
% avoid the use of subfigure captions (many IEEE journals avoid them anyway)
% and instead reference/explain all the subfigures within the main caption.
% The latest version of endfloat.sty and its documentation can obtained at:
% http://www.ctan.org/pkg/endfloat
%
% The IEEEtran \ifCLASSOPTIONcaptionsoff conditional can also be used
% later in the document, say, to conditionally put the References on a 
% page by themselves.




% *** PDF, URL AND HYPERLINK PACKAGES ***
%
%\usepackage{url}
% url.sty was written by Donald Arseneau. It provides better support for
% handling and breaking URLs. url.sty is already installed on most LaTeX
% systems. The latest version and documentation can be obtained at:
% http://www.ctan.org/pkg/url
% Basically, \url{my_url_here}.

\usepackage{enumerate,amssymb,epsfig,euscript,indentfirst,authblk}
\usepackage{subcaption}
\usepackage{booktabs}
\usepackage{nccmath}
\usepackage{lipsum}
\usepackage{mathtools, cuted}
\usepackage{graphics}
\usepackage{algorithm}
% \usepackage{algorithmic}
\usepackage{algpseudocode}
\usepackage{nomencl}
\makenomenclature

\usepackage{etoolbox}

\usepackage{soul,color}
\usepackage{gensymb}
\usepackage{svg}


\newcommand{\nomunit}[1]{%
\renewcommand{\nomentryend}{\hspace*{\fill}#1}}

% *** Do not adjust lengths that control margins, column widths, etc. ***
% *** Do not use packages that alter fonts (such as pslatex).         ***
% There should be no need to do such things with IEEEtran.cls V1.6 and later.
% (Unless specifically asked to do so by the journal or conference you plan
% to submit to, of course. )


% correct bad hyphenation here
\hyphenation{op-tical net-works semi-conduc-tor}


%================================================================================
% Comments
\newcommand{\bowen}[1]{{\color{blue} [Bowen: ``#1'']}}
\newcommand{\kai}[1]{{\color{blue} [Kai: ``#1'']}}
\newcommand{\ruifeng}[1]{{\color{blue} [Ruifeng: ``#1'']}}

% Packages
\usepackage{makecell}

%================================================================================
\begin{document}
%
% paper title
% Titles are generally capitalized except for words such as a, an, and, as,
% at, but, by, for, in, nor, of, on, or, the, to and up, which are usually
% not capitalized unless they are the first or last word of the title.
% Linebreaks \\ can be used within to get better formatting as desired.
% Do not put math or special symbols in the title.
\title{Hybrid Offline-online Scheduling Method for Large Language Model Inference Optimization}
%
%
% author names and IEEE memberships
% note positions of commas and nonbreaking spaces ( ~ ) LaTeX will not break
% a structure at a ~ so this keeps an author's name from being broken across
% two lines.
% use \thanks{} to gain access to the first footnote area
% a separate \thanks must be used for each paragraph as LaTeX2e's \thanks
% was not built to handle multiple paragraphs
%


\author{Bowen~Pang,
        Kai~Li,
        Ruifeng~She,
        and~Feifan~Wang,~\IEEEmembership{Member,~IEEE}
\thanks{B. Pang, K. Li, and R. She are with Noah's Ark Lab, Huawei.}
\thanks{F. Wang is with the Department
of Industrial Engineering, Tsinghua University, Beijing,
100084, China. E-mail: wangfeifan@tsinghua.edu.cn.}
}

% note the % following the last \IEEEmembership and also \thanks - 
% these prevent an unwanted space from occurring between the last author name
% and the end of the author line. i.e., if you had this:
% 
% \author{....lastname \thanks{...} \thanks{...} }
%                     ^------------^------------^----Do not want these spaces!
%
% a space would be appended to the last name and could cause every name on that
% line to be shifted left slightly. This is one of those "LaTeX things". For
% instance, "\textbf{A} \textbf{B}" will typeset as "A B" not "AB". To get
% "AB" then you have to do: "\textbf{A}\textbf{B}"
% \thanks is no different in this regard, so shield the last } of each \thanks
% that ends a line with a % and do not let a space in before the next \thanks.
% Spaces after \IEEEmembership other than the last one are OK (and needed) as
% you are supposed to have spaces between the names. For what it is worth,
% this is a minor point as most people would not even notice if the said evil
% space somehow managed to creep in.



% The paper headers
% \markboth{Journal of \LaTeX\ Class Files,~Vol.~14, No.~8, March~2019}%
% {Shell \MakeLowercase{\textit{WANG et al.}}: Here is the title}
% The only time the second header will appear is for the odd numbered pages
% after the title page when using the twoside option.
% 
% *** Note that you probably will NOT want to include the author's ***
% *** name in the headers of peer review papers.                   ***
% You can use \ifCLASSOPTIONpeerreview for conditional compilation here if
% you desire.




% If you want to put a publisher's ID mark on the page you can do it like
% this:
%\IEEEpubid{0000--0000/00\$00.00~\copyright~2015 IEEE}
% Remember, if you use this you must call \IEEEpubidadjcol in the second
% column for its text to clear the IEEEpubid mark.



% use for special paper notices
%\IEEEspecialpapernotice{(Invited Paper)}



%================================================================================
% make the title area
\maketitle

% As a general rule, do not put math, special symbols or citations
% in the abstract or keywords.
\begin{abstract}
With the development of large language models (LLMs), it has become increasingly important to optimize hardware usage and improve throughput. In this paper, we study the inference optimization of the serving system that deploys LLMs. To optimize system throughput and maximize hardware utilization, we formulate the inference optimization problem as a mixed-integer programming (MIP) model and propose a hybrid offline-online method as solution. The offline method improves large-scale inference systems by introducing a Minimizing Makespan Bin Packing Problem. We further provide a theoretical lower bound computation method. Then, we propose an online sorting and preemptive scheduling method to better utilize hardware. In the online iteration scheduling process, a Lagrangian method is applied to evaluate the cost efficiency of inserting prefill stages versus decode stages at each iteration and dynamically determine when to preempt decoding tasks and insert prefill tasks. Experiments using real-world data from the LLaMA-65B model and the GSM8K dataset demonstrate that system utilization improves from 80.2\% to 89.1\%, and the total inference time decreases from 201.00 to 190.58 seconds. A 100-cases study shows that our method consistently outperforms the baseline method and improves the utilization rate by 8.0\% on average. Finally, we discuss potential future extensions, including stochastic modeling, reinforcement learning-based schedulers, and dynamic decision-making strategies for system throughput and hardware utilization.

\end{abstract}

\def\abstractname{Note to Practitioners}
\begin{abstract}
This work provides optimization tools for enhancing the efficiency of LLM inference systems through advanced scheduling techniques. From the perspective of LLM inference service providers, improved hardware utilization can reduce operational costs by requiring less hardware to maintain the same level of service. From the user’s perspective, reduced inference time translates to faster response times and improved service quality. Furthermore, the proposed scheduling techniques are adaptable to various LLM models, hardware platforms, and datasets, making them highly scalable and broadly applicable to real-world LLM inference scenarios.
\end{abstract}


% Note that keywords are not normally used for peerreview papers.
\begin{IEEEkeywords}
Large Language Model, Inference System, Online Scheduling, Mixed-integer Programming. 
\end{IEEEkeywords}






% For peer review papers, you can put extra information on the cover
% page as needed:
% \ifCLASSOPTIONpeerreview
% \begin{center} \bfseries EDICS Category: 3-BBND \end{center}
% \fi
%
% For peerreview papers, this IEEEtran command inserts a page break and
% creates the second title. It will be ignored for other modes.
\IEEEpeerreviewmaketitle

\section{Introduction}

Despite the remarkable capabilities of large language models (LLMs)~\cite{DBLP:conf/emnlp/QinZ0CYY23,DBLP:journals/corr/abs-2307-09288}, they often inevitably exhibit hallucinations due to incorrect or outdated knowledge embedded in their parameters~\cite{DBLP:journals/corr/abs-2309-01219, DBLP:journals/corr/abs-2302-12813, DBLP:journals/csur/JiLFYSXIBMF23}.
Given the significant time and expense required to retrain LLMs, there has been growing interest in \emph{model editing} (a.k.a., \emph{knowledge editing})~\cite{DBLP:conf/iclr/SinitsinPPPB20, DBLP:journals/corr/abs-2012-00363, DBLP:conf/acl/DaiDHSCW22, DBLP:conf/icml/MitchellLBMF22, DBLP:conf/nips/MengBAB22, DBLP:conf/iclr/MengSABB23, DBLP:conf/emnlp/YaoWT0LDC023, DBLP:conf/emnlp/ZhongWMPC23, DBLP:conf/icml/MaL0G24, DBLP:journals/corr/abs-2401-04700}, 
which aims to update the knowledge of LLMs cost-effectively.
Some existing methods of model editing achieve this by modifying model parameters, which can be generally divided into two categories~\cite{DBLP:journals/corr/abs-2308-07269, DBLP:conf/emnlp/YaoWT0LDC023}.
Specifically, one type is based on \emph{Meta-Learning}~\cite{DBLP:conf/emnlp/CaoAT21, DBLP:conf/acl/DaiDHSCW22}, while the other is based on \emph{Locate-then-Edit}~\cite{DBLP:conf/acl/DaiDHSCW22, DBLP:conf/nips/MengBAB22, DBLP:conf/iclr/MengSABB23}. This paper primarily focuses on the latter.

\begin{figure}[t]
  \centering
  \includegraphics[width=0.48\textwidth]{figures/demonstration.pdf}
  \vspace{-4mm}
  \caption{(a) Comparison of regular model editing and EAC. EAC compresses the editing information into the dimensions where the editing anchors are located. Here, we utilize the gradients generated during training and the magnitude of the updated knowledge vector to identify anchors. (b) Comparison of general downstream task performance before editing, after regular editing, and after constrained editing by EAC.}
  \vspace{-3mm}
  \label{demo}
\end{figure}

\emph{Sequential} model editing~\cite{DBLP:conf/emnlp/YaoWT0LDC023} can expedite the continual learning of LLMs where a series of consecutive edits are conducted.
This is very important in real-world scenarios because new knowledge continually appears, requiring the model to retain previous knowledge while conducting new edits. 
Some studies have experimentally revealed that in sequential editing, existing methods lead to a decrease in the general abilities of the model across downstream tasks~\cite{DBLP:journals/corr/abs-2401-04700, DBLP:conf/acl/GuptaRA24, DBLP:conf/acl/Yang0MLYC24, DBLP:conf/acl/HuC00024}. 
Besides, \citet{ma2024perturbation} have performed a theoretical analysis to elucidate the bottleneck of the general abilities during sequential editing.
However, previous work has not introduced an effective method that maintains editing performance while preserving general abilities in sequential editing.
This impacts model scalability and presents major challenges for continuous learning in LLMs.

In this paper, a statistical analysis is first conducted to help understand how the model is affected during sequential editing using two popular editing methods, including ROME~\cite{DBLP:conf/nips/MengBAB22} and MEMIT~\cite{DBLP:conf/iclr/MengSABB23}.
Matrix norms, particularly the L1 norm, have been shown to be effective indicators of matrix properties such as sparsity, stability, and conditioning, as evidenced by several theoretical works~\cite{kahan2013tutorial}. In our analysis of matrix norms, we observe significant deviations in the parameter matrix after sequential editing.
Besides, the semantic differences between the facts before and after editing are also visualized, and we find that the differences become larger as the deviation of the parameter matrix after editing increases.
Therefore, we assume that each edit during sequential editing not only updates the editing fact as expected but also unintentionally introduces non-trivial noise that can cause the edited model to deviate from its original semantics space.
Furthermore, the accumulation of non-trivial noise can amplify the negative impact on the general abilities of LLMs.

Inspired by these findings, a framework termed \textbf{E}diting \textbf{A}nchor \textbf{C}ompression (EAC) is proposed to constrain the deviation of the parameter matrix during sequential editing by reducing the norm of the update matrix at each step. 
As shown in Figure~\ref{demo}, EAC first selects a subset of dimension with a high product of gradient and magnitude values, namely editing anchors, that are considered crucial for encoding the new relation through a weighted gradient saliency map.
Retraining is then performed on the dimensions where these important editing anchors are located, effectively compressing the editing information.
By compressing information only in certain dimensions and leaving other dimensions unmodified, the deviation of the parameter matrix after editing is constrained. 
To further regulate changes in the L1 norm of the edited matrix to constrain the deviation, we incorporate a scored elastic net ~\cite{zou2005regularization} into the retraining process, optimizing the previously selected editing anchors.

To validate the effectiveness of the proposed EAC, experiments of applying EAC to \textbf{two popular editing methods} including ROME and MEMIT are conducted.
In addition, \textbf{three LLMs of varying sizes} including GPT2-XL~\cite{radford2019language}, LLaMA-3 (8B)~\cite{llama3} and LLaMA-2 (13B)~\cite{DBLP:journals/corr/abs-2307-09288} and \textbf{four representative tasks} including 
natural language inference~\cite{DBLP:conf/mlcw/DaganGM05}, 
summarization~\cite{gliwa-etal-2019-samsum},
open-domain question-answering~\cite{DBLP:journals/tacl/KwiatkowskiPRCP19},  
and sentiment analysis~\cite{DBLP:conf/emnlp/SocherPWCMNP13} are selected to extensively demonstrate the impact of model editing on the general abilities of LLMs. 
Experimental results demonstrate that in sequential editing, EAC can effectively preserve over 70\% of the general abilities of the model across downstream tasks and better retain the edited knowledge.

In summary, our contributions to this paper are three-fold:
(1) This paper statistically elucidates how deviations in the parameter matrix after editing are responsible for the decreased general abilities of the model across downstream tasks after sequential editing.
(2) A framework termed EAC is proposed, which ultimately aims to constrain the deviation of the parameter matrix after editing by compressing the editing information into editing anchors. 
(3) It is discovered that on models like GPT2-XL and LLaMA-3 (8B), EAC significantly preserves over 70\% of the general abilities across downstream tasks and retains the edited knowledge better.
\section{Literature review}
\label{literature}

% ref: Nvidia blog https://developer.nvidia.com/blog/mastering-llm-techniques-inference-optimization/
% ref: Inference paper list https://github.com/DefTruth/Awesome-LLM-Inference
% ref: Review paper https://arxiv.org/abs/2404.14294

\begin{table*}[t]
    \centering
    \caption{Literature Review}
    \begin{tabular}{cccccccccc}
    \toprule
        Work & Throughput & Latency & \makecell{Cache \\ Management} & \makecell{Dynamic \\ Decision} & Uncertainty & \makecell{Request \\ Management} & \makecell{Iteration\\ Management} & Batching & \makecell{Distributed \\ Strategies}\\
    \midrule
        Orca & \checkmark & \checkmark &  &  & \checkmark  &  & \checkmark & \checkmark & \\
        vLLM & \checkmark  & \checkmark  & \checkmark  &  &  &  &  &  & \\
        Sarathi-Serve\cite{agrawal2024sarathi} & \checkmark & \checkmark &  &  &  &  & \checkmark & \checkmark & \\
        Llumnix\cite{sun2024llumnix} &  & \checkmark & \checkmark & \checkmark &  & \checkmark &  &  & \checkmark\\
        InfiniGen\cite{lee2024infinigen} &  & \checkmark & \checkmark &  &  &  &  &  & \\
        dLoRA\cite{wu2024dlora} & \checkmark & \checkmark  & \checkmark & \checkmark &  &  &  & \checkmark & \checkmark\\
        VTC\cite{sheng2024fairness} &  &  &  & \checkmark &  & \checkmark &  &  & \\
        FastServe\cite{wu2023fast} & \checkmark  & \checkmark & \checkmark & \checkmark & \checkmark & \checkmark &  &  & \\
        DistServe\cite{zhong2024distserve} & \checkmark & \checkmark & \checkmark & \checkmark &  &  & \checkmark &  & \checkmark\\
        MoonCake & \checkmark & \checkmark & \checkmark & \checkmark &  &  &  &  & \checkmark \\
    \midrule
        OURS & \checkmark & \checkmark & \checkmark & \checkmark & \checkmark & \checkmark & \checkmark & \checkmark & \\
    \bottomrule
    \end{tabular}

    \label{table:literature_review}
\end{table*}
% Standards in the table:
% cache management: whether caring about KV cache
% dynamic decision: whether making decisons in different states
% uncertainty: whether caring about the ouput length
% request management: whether changing the order of requests
% iteration management: whether caring about the order of PD
% batching: batching strategies
% distributed strategies: whether using GPU clusters


This section provides an overview of existing research and prevalent methodologies in inference optimization for LLMs. Firstly, we introduce the general techniques commonly employed in model serving, which can be seamlessly integrated with our scheduling strategies. Subsequently, we elucidate several classical techniques widely adopted in LLM inference systems, which constitute the cornerstone of our framework and methodologies. In the following, we succinctly introduce recent advancements in inference optimization. Finally, we examine scheduling methods within the existing operations research domain.


% 1. General inference optimization techniques
As LLM inference falls within the broader scope of model serving, a variety of general inference optimization techniques can be effectively utilized. Model compression is one of the quintessential optimization strategies for reducing model size, encompassing techniques such as quantization \cite{jacob2018quantization}, sparsification \cite{child2019generating,bai2024sparsellm}, and distillation \cite{hinton2015distilling}. In addition, the design of more compact structures to replace the original ones is also common. For instance, employing multi-query attention \cite{shazeer2019fast} or grouped-query attention \cite{ainslie2023gqa} in place of the original multi-head attention in Transformer architecture can reduce key-value heads, resulting in a more streamlined model. Nevertheless, both model compression and the design of compact structures can alter model weights, potentially leading to a decline in accuracy. Instead of optimizing the model size, data parallelism (DP) and model parallelism aim to fully leverage the computational power of the devices. In DP \cite{narayanan2021efficient}, the model weights are replicated across multiple devices, allowing different inference requests to be processed in parallel on different devices. Model parallelism distributes the model weights across several devices to minimize the per-device memory footprint of the model weights. Consequently, each device can operate more efficiently by executing a smaller portion of the task. Several model parallelization methods exist, such as pipeline parallelism (PP) \cite{huang2019gpipe}, tensor parallelism (TP) \cite{shoeybi2019megatron}, sequence parallelism (SP) \cite{korthikanti2023reducing}, and context parallelism (CP) \cite{liu2023ring}. Since our scheduling methods are orthogonal to the aforementioned techniques, both model optimization and parallelization strategies can be employed seamlessly in conjunction with our methods to enhance inference efficiency.

% vertical to our methods, we can use our methods along with these techniques

% Techniques such as model compression, pruning, quantization, pipeline parallelism, and distributed computing have been explored extensively. 
% However, these approaches often face limitations related to applicability, overhead, and real-time performance. 

% 2. LLM serving techniques
% Flash attention, speculative decoding, batching (continuous batching), memory management (paged attention), chunked prefill, distributed systems, scheduling
In addition to general model serving techniques, the optimization of LLM inference serving systems primarily involves enhancing the model forward pass. Researchers have improved system efficiency from various perspectives, including kernel fusion, Key-Value (KV) cache management, request management, iteration management, batching, distributed strategies, etc. Here, we present several classic techniques that are prevalent in LLM inference serving systems. FlashAttention \cite{dao2022flashattention} amalgamates the operations of data transfer between hardware components within the attention mechanism to expedite operation execution without compromising model accuracy. Speculative decoding \cite{leviathan2023fast, chen2023accelerating} employs an auxiliary model to generate a preliminary draft, followed by a verification process executed by the main model. This technique enables the serving system to output multiple tokens in a single forward pass instead of one. Orca \cite{yu2022orca} pioneers continuous batching by aggregating different requests at the iteration level. Rather than awaiting the completion of an entire batch before starting the execution of new requests, continuous batching allows new requests to be inserted into the batch while other requests are still in progress. Inspired by memory management strategies in operating systems, vLLM \cite{kwon2023efficient} introduces PagedAttention, wherein the attention key and value vectors are stored as non-contiguous blocks in memory. Continuous batching and PagedAttention significantly increase overall GPU memory utilization during the execution of LLM. SarathiServe \cite{agrawal2024sarathi} introduces chunked-prefills, also known as dynamic SplitFuse or prefill-decode (PD) Fusion, batching together prefill and decode chunks to maximize both computation and bandwidth utilization. Since serving systems are commonly deployed on distributed platforms, numerous strategies have been proposed to exploit distributed characteristics. For example, recent works \cite{patel2024splitwise,zhong2024distserve,hu2024inference} advocate for separating prefill servers from decode servers, also known as PD Separation, due to the distinct computational and bandwidth characteristics of these two stages. For a comprehensive review of these techniques, we recommend \cite{zhou2024survey} for further reference.
These classic techniques form the foundation of inference services and our methods. 

% 3. Recent advancement
Recently, with ongoing advancements in AI system research, numerous innovative inference techniques have been developed, particularly those related to schedulers. We present some representative works and highlight the differences between these approaches and our method in TABLE~\ref{table:literature_review}. Inspired by context switching across CPU cores, Llumnix \cite{sun2024llumnix} proposes a live migration mechanism and a dynamic scheduling policy to reschedule requests across multiple model instances of LLM deployed on GPU clusters, thereby enhancing load balancing and isolation. InfiniGen \cite{lee2024infinigen} addresses the challenge of large KV cache sizes for long-text generation by speculating and prefetching critical KV cache entries. For LoRA models, dLoRA \cite{wu2024dlora} addresses the challenges of serving multiple LoRA models by dynamically merging and unmerging adapters with the base model and migrating requests and adapters between replicas. Sheng et al. \cite{sheng2024fairness} study the fairness problem in LLM serving concerning clients and proposes a novel scheduling algorithm called the virtual token counter (VTC). FastServe \cite{wu2023fast} proposed an innovative skip-join MLFQ scheduler to enable preemption during the autoregression process of LLM inference. In distributed systems, DistServe \cite{zhong2024distserve} tackles the issues of PD interference and resource coupling by disaggregating prefill and decoding computation, and proposes placement algorithms to optimize resource allocation and parallelism strategies for different phases. Mooncake \cite{qin2024mooncake} also proposes a disaggregated architecture that separates the prefill and decode clusters and utilizes a KV cache-centric scheduler to manage the cache flow. While these innovative techniques attempt to address the issues faced by the scheduler to some extent, they scarcely model the scheduling problem formally and are limited to solve the inference optimization problem theoretically.


% 4. Scheduling methods
In the domain of operations research, scheduling is a well-established and extensively utilized approach. For instance, offline scheduling methods are applied in manufacturing systems \cite{wang2020print3D}, healthcare systems \cite{pang2018surgery}, and operations management systems \cite{erdogan2015online}. To address real-time decision-making or multi-stage stochastic scenarios, online scheduling methods have been introduced in these systems \cite{pang2022dynamic, lee2019online}. Nevertheless, none of the systems we examined achieve the rapid decision frequency—down to 10 milliseconds—that is observed in LLM inference systems. Furthermore, traditional stochastic programming models are not suitable for sequential decision-making problems where real-time information is continuously revealed. Therefore, it is imperative to develop and tailor traditional methods for application in this emergent domain of LLM inference. Research on LLM inference optimization can not only enhance hardware resource utilization in the LLM domain but also expand the repertoire available to existing operations research algorithms.


% Scheduling methods.
% Should include two parts: LLM inference applications (fast serve, vllm); and scheduling applications in manufacture.


\section{Problem formulation}
\label{problem_formulation}

\subsection{Background of LLM inference techniques}
High efficiency and low latency are critical in LLM inference, and they depend not only on performance of hardware, such as GPU, but also on how well the hardware is used. As pushing the limit of hardware computing power is costly and faced with slow progress, improved inference scheduling becomes a promising means of achieving the same outcome. Three factors are involved when designing inference scheduling methods: PD management approaches, cache management, and batching strategy.

\begin{figure}
    \centering
    \includegraphics[width=0.98\linewidth]{graph//illustration.pdf}
    \caption{Illustration of LLM inference}
    \label{fig:illustration}
\end{figure}

LLM inference alternately goes through two stages, i.e., prefill and decode stages. In the prefill stage, initial input tokens are processed, and the LLM model's internal states, such as hidden states, are prepared. In the decode stage, the LLM model generates output tokens based on the context established during the prefill stage. The time lengths for prefill stage and decode stage are uncertain. The alternating process continues in parallel, until either the end of the sequence is reached or a predefined stopping criterion is satisfied. The process of LLM inference is illustrated in Fig. \ref{fig:illustration}, where requests, each with a prefill phase and a decode phase, are sent to clients and processed in parallel. The whole process is divided into multiple bins, and each bin consists of a prefill stage and a decode stage. Requests in prefill phase can be served only when the process is in prefill stage, and the same requirement is applied to the decode phase and stage. A client may be idle as either the prefill phase or decode phase is completed but the process still stays in the same stage, causing compromised utilization of computing resources. A single LLM inference in practice typically contains a large number of requests processed by parallel clients. Thus, there is great potential in better utilizing computing resources in LLM inference, but the scheduling problem has a high complexity and, as a real-time decision making in milliseconds, the computing time is limited. The most commonly used approaches to managing prefill are provided below.

% In the context of LLM inference, optimizing task scheduling is critical to maximize efficiency and minimize latency. The primary stages in LLM inference are the prefill and decode phases.

% Prefill Phase: During this stage, initial input tokens are processed, and the model's internal states, such as hidden states, are prepared. This phase establishes the necessary context for subsequent token generation.

% Decode Phase: In this phase, the model generates output tokens based on the context established during the prefill phase. This iterative process continues until either the end of the sequence is reached or a predefined stopping criterion is satisfied. In the Prefill-Decode Competition scenario, the decode phase may be preempted by the prefill phase to enhance hardware utilization.

% Given the distinct characteristics of the prefill and decode phases, previous studies [references] have proposed three strategies for managing these processes:

% 1) Prefill-Decode Competition: In this approach, only one of the prefill or decode operations is processed at any given time on a single hardware node (typically consisting of 8 GPUs). By carefully managing resource contention, this strategy can effectively utilize resources, reduce idle times, and potentially lower overall latency. However, the complexity of scheduling poses a significant challenge, potentially leading to suboptimal performance if not properly managed. Additionally, resource contention might result in bottlenecks during unexpected demand peaks. 

% 2) Prefill-Decode Fusion: This strategy integrates the prefill and decode processes into a single, cohesive operation. Fusion can reduce overhead and enhance throughput by streamlining the inference pipeline. It may also decrease latency due to the alignment of processes. However, this integration may compromise flexibility, restricting the ability to independently optimize each process or tailor responses to varying workload demands.

% 3) Prefill-Decode Separation: This approach separates the prefill and decode phases across multiple GPUs, allowing for independent optimization of each stage. This separation can lead to improvements in efficiency, scalability, and adaptability to different workload types. However, it may introduce additional communication or coordination overhead, which might increase latency if not managed effectively.



% \begin{figure}
%     \centering
%     \includegraphics[width=0.98\linewidth]{graph//PDfusion.pdf}
%     \caption{Illustration of PD fusion}
%     \label{fig:PDfusion}
% \end{figure}

% \begin{figure}
%     \centering
%     \includegraphics[width=0.98\linewidth]{graph//PDseparate.pdf}
%     \caption{Illustration of PD Separate}
%     \label{fig:PDseparate}
% \end{figure}

\begin{figure}
    \centering
    \includegraphics[width=0.98\linewidth]{graph//PDcompetition.pdf}
    \caption{Illustration of PD Competition}
    \label{fig:PDcompetition}
\end{figure}

\begin{itemize}
    \item PD Competition: As illustrated in Fig. \ref{fig:PDcompetition}, in this approach, either prefill or decode stage is processed at any given time for all clients on a single hardware node (typically consisting of 8 GPUs). PD Competition allows decode stage to be preempted by the prefill stage to enhance hardware utilization.
%By carefully managing resource contention, this strategy can effectively utilize resources, reduce idle times, and potentially lower overall latency. However, the complexity of scheduling poses a significant challenge, potentially leading to suboptimal performance if not properly managed. Additionally, resource contention might result in bottlenecks during unexpected demand peaks. 
    \item PD Fusion: This approach integrates the prefill and decode stages into a single cohesive operation, aimed at reducing overhead and enhancing throughput by streamlining the inference pipeline. This approach also attempts to decrease latency through alignment of processes. However, this integration compromises flexibility, restricting the ability to independently optimize each process or tailor responses to varying workload demands.
    \item PD Separation: This approach separates the prefill and decode stages across exclusive sets of GPUs. However, it introduces additional communication or coordination overhead, which increases latency if not properly managed.
%, allowing for independent optimization of each stage. This separation can lead to improvements in efficiency, scalability, and adaptability to different workload types. 

\end{itemize}

As a widely used approach, PD Competition has a high flexibility in effectively utilizing computing resources. Such an approach also allows an inference scheduling method to fit in and further enhance its performance. As is aforementioned, inference scheduling for LLM inference is challenging. This study focuses on the inference scheduling under the PD Competition approach.

The second factor that influences the efficiency of LLM inference is cache management. The KV cache is instrumental during the decoding stage by storing intermediate hidden states from preceding token generation steps. It allows the LLM model to reuse states, significantly accelerating the inference process. Despite its advantages, the KV cache requires to be properly managed. First, the KV Cache size increases with the length of input and output sequences and the number of LLM model layers. This cache size growth results in significant memory consumption, especially in the context of large-sized models and extended sequences. Effective KV cache management avoids memory overflow and sustains high inference speed. Cache management may involve caching only the most relevant hidden states, while discarding or compressing less critical information to optimize resource use. Second, concurrency issues should be addressed. The complexity of managing the KV cache escalates with concurrent inference tasks. Ensuring consistent and conflict-free partitioning and access to the cache for each task is important for performance and accuracy of LLM inference. Besides, although the KV cache alleviates computational load during decoding, it introduces cache access and management overheads. Cache management requires taking into account overall latency. In this study, we proactively compute the KV cache and determine the optimal maximum number of parallel requests, equivalent to the number of clients handled in subsequent stages. Thus, we assume in the inference scheduling problem shown in Fig. \ref{fig:illustration} that the total number of clients is predetermined.

The third factor that influences LLM inference efficiency is batching strategy. Continuous batching, which is introduced by ORCA \cite{yu2022orca}, has emerged as an innovative technique to enhance the inference efficiency by dynamically aggregating incoming requests in real time. Unlike traditional static batching, which waits for a fixed number of requests before processing, continuous batching minimizes latency by reducing idle times between batch executions and adapting to fluctuating workloads. It ensures efficient use of computational resources. By maximizing parallel processing capabilities and aligning with the model architecture's latency and throughput trade-offs, continuous batching significantly enhances the scalability and responsiveness of LLM deployments. Using the continuous batching technique, decoding is allowed to be preempted by the prefill stage. Such a case is shown by the first request of client 2 in Fig. \ref{fig:illustration}, where the decode phase is separated by the prefill stage of bin 2 and is assigned to both the decode stages of bin 1 and 2. In this work, the decision-making process allows decoding to be preempted by the prefill stage, facilitated by the continuous batching technique.

% 注释这部分保留下,batching的文章可以用下
% However, there are specific limitations and challenges associated with batching:
% \begin{itemize}
%     \item Variable Length Sequences: LLMs often process sequences of varying lengths, which complicates the batching process. Batching sequences of different lengths requires padding shorter sequences to match the length of the longest sequence in the batch, which can lead to inefficient memory usage and computational overhead.
%     \item Memory Constraints: Batching large numbers of sequences can strain memory resources, especially when dealing with high-dimensional embeddings and large batch sizes. This can impact the ability to efficiently utilize hardware accelerators and may necessitate careful management of memory allocation.
%     \item Latency Considerations: While batching can improve throughput by processing multiple sequences simultaneously, it can also introduce latency, particularly when waiting for the completion of longer sequences before proceeding to the next batch. This trade-off between latency and throughput requires careful consideration in real-time or interactive applications.
% \end{itemize}

% Dynamic Batching Application: Refer to \href{https://docs.nvidia.com/deeplearning/triton-inference-server/user-guide/docs/examples/jetson/concurrency_and_dynamic_batching/README.html}{this link}
% Dynamic batching is a strategy used in large model inference to optimize the processing of sequences by adjusting batch sizes dynamically based on the characteristics of the input data. Unlike fixed batching, where sequences are grouped into batches of predefined sizes, dynamic batching adapts batch sizes in real-time to improve resource utilization and efficiency.

% While dynamic batching offers flexibility and efficiency benefits, it also faces challenges in maintaining optimal parallelism between prefill and decode rounds, which can lead to inefficiencies and idle time.
% \begin{itemize}
%     \item Sequential Processing Dependencies: Large Language Model (LLM) inference typically involves sequential stages—prefill, where initial states are set up, and decode, where sequences are generated based on these states. These stages often require sequential execution on the same resources due to dependencies, limiting the ability to overlap their execution.

%     \item Variable Batch Completion Times: Even with dynamic batching, sequences within a batch may complete their prefill and decode stages at different times. This variability can result in situations where resources allocated for decode are idle while waiting for prefill to complete on other sequences within the same dynamically sized batch.

%     \item Resource Allocation Efficiency: Efficiently allocating resources between prefill and decode stages becomes crucial. If the system cannot effectively overlap these stages due to sequential dependencies or resource constraints, idle time (or "bubbles") occurs, reducing overall throughput and efficiency.
% \end{itemize}
% By implementing effective overlap strategies, adaptive resource allocation, and optimized batch scheduling policies, it is possible to mitigate these challenges and improve overall efficiency in large model inference tasks.

\begin{table*}
    \centering
    \caption{Notation Table: Sets, Parameters and Decision Variables}
    \begin{tabular}{cl}
    \toprule
        Sets    & Description \\
    \midrule
        $\mathcal{I}=\left\{1,2,\cdots \right\}$  & Set of requests. \\
        % $\mathcal{I}\left(t\right) \subseteq \mathcal{I}$ & Set of remaining requests to be scheduled at time $t$.\\
        $\mathcal{J}=\left\{1,2,\cdots \right\}$  & Set of clients.\\
        $\mathcal{K},\mathcal{K}^\text{p}, \mathcal{K}^\text{d}=\left\{1,2,\cdots \right\}$& Set of bins, prefill stages, and decode stages, respectively. $\mathcal{K} = \mathcal{K}^\text{p} = \mathcal{K}^\text{d}$.\\
        $\mathcal{L}=\left\{1,2,\cdots \right\}$  & Set of all possible levels for the prefill stage.\\
    \midrule
        Parameters          & \\
    \midrule
        $I=\mid \mathcal{I} \mid $                 & Total number of requests.\\
        $J=\mid \mathcal{J} \mid $                 & Total number of clients.\\
        $K=\mid \mathcal{K} \mid $                 & Total number of bins.\\
        $N^\text{p}_i \in \mathbb{Z}^+$             & Input token number of request $i$, for $i\in \mathcal{I}$. \\
        $N^\text{d}_i \in \mathbb{Z}^+$         & Output token number of request $i$, for $i\in \mathcal{I}$. \\
        $N^\text{cap}_l \in \mathbb{Z}^+$        & Maximum token capacity of level $l$, for $l\in \mathcal{L}$. \\  
        $T^\text{d} \in \mathbb{R}^+$            & Decode time per token.\\
        $T^\text{p} \in \mathbb{R}^+$            & Prefill time per token. \\
        $T^\text{p}_l \in \mathbb{R}^+$          & Prefill time of level $l$, for $l\in \mathcal{L}$. \\
    \midrule
        Decision Variables      & \\
    \midrule
        $p_{i,j,k}\in \{0,1\}$ & Assignment of prefill stage $k$ to request $i$ on client $j$ for prefill phase, for $i\in \mathcal{I}$, $j\in \mathcal{J}$, and $k\in \mathcal{K}^\text{p}$.  \\
        $d_{i,j,k}\in \{0,1\}$ & Assignment of decode stage $k$ to request $i$ on client $j$ for decode phase, for $i\in \mathcal{I}$, $j\in \mathcal{J}$, and $k\in \mathcal{K}^\text{d}$.  \\
        % $t^\text{s}_{k}\in \mathbb{R}^+$  & Start time of the $k$th bin, for $k\in \mathcal{K}$.  \\
        % $t^\text{e}_{k}\in \mathbb{R}^+$  & End time of the $k$th bin, for $k\in \mathcal{K}$.  \\
        $t^\text{s,p}_{k}\in \mathbb{R}^+$  & Start time of the $k$th prefill stage, for $k\in \mathcal{K}^\text{p}$.  \\
        $t^\text{s,d}_{k}\in \mathbb{R}^+$  & Start time of the $k$th decode stage, for $k\in \mathcal{K}^\text{d}$.\\
        $n^\text{p}_{k} \in \mathbb{R}^+$ & Time length of the $k$th  prefill stage, for $k\in \mathcal{K}^\text{p}$.  \\
        $n^\text{d}_{k} \in \mathbb{R}^+$ & Time length of the $k$th  decode stage, for $k\in \mathcal{K}^\text{d}$.  \\
        $w_{i,j,k}\in [0,1]$   & The proportion of the decoding phase of request $i$ executed in the $k$th decode stage on client $j$, for $i\in \mathcal{I}$, $j\in \mathcal{J}$, and $k\in \mathcal{K}^\text{d}$. \\
        $x_{i,j}\in \{0,1\}$ & The assignment of request $i$ to client $j$, for $i\in \mathcal{I}$ and $j\in \mathcal{J}$. \\
        $y_{k,l}\in \{0,1\}$    & Indicator variable specifying if the $k$th prefill stage is at level $l$, for $k\in \mathcal{K}^\text{p}$ and $l\in \mathcal{L}$.\\
        $t^\text{max} \in \mathbb{R}^+$ & Total inference time for all requests completed.\\
    \bottomrule
    \end{tabular}
    \label{notation table}
\end{table*}

\subsection{Inference scheduling problem description}
Inference scheduling aims to schedule inference requests using continuous batching and PD Competition, with the goal of minimizing the total inference time while adhering to operational constraints. The problem settings are given as follows and related notations are presented in TABLE \ref{notation table}.

\begin{enumerate}
    \item \textbf
{Requests and processing time:}
    \begin{itemize}
        \item Let \( \mathcal{I}=\left\{ 1,2,\cdots\right\} \) be the set of inference requests. Inference request $i$, for $i \in \mathcal{I}$, has a fixed input token number $N^\text{p}_i \in  \mathbb{Z}^+$ for prefill phase and output token number $N^\text{d}_i \in  \mathbb{Z}^+$ for decode phase. The input token number is assumed to be known, but the output token number is unknown.
        \item Each request has a known prefilling time, linearly related to the total number of input tokens in a bin.
        \item The decoding time is approximately linearly related to the output token number. Let $\mathcal{J}=\left\{1,2,\cdots \right\}$ be the set of clients and $T^\text{d}$ be the decode time per token. The minimal unit of decoding time is equal to the amount of time that each client processes a single token, i.e., $T^\text{d}|\mathcal{J}|$.
    \end{itemize}
    \item \textbf{Bin and batch size:}
    \begin{itemize}
        \item Let $\mathcal{K}=\left\{1,2,\cdots \right\}$ be the set of bins. The inference process is divided into a total of \( K=|\mathcal{K}| \) bins.
        \item The batch size \( |\mathcal{J}| \) represents the maximum number of requests that can be processed simultaneously in a batch.
    \end{itemize}
    \item \textbf{Stages:}
    \begin{itemize}
        \item There are two stages in hardware operation system: prefill and decode. Prefill and decode stages must alternate, ensuring that each stage is dedicated exclusively to one type of operation.
        \item At any given time, the system can perform either prefill or decode stages, but not both.
        \item At any time as the system starts a decode stage, the length of this operation is determined. Meanwhile, all requests processed at this stage will be scheduled. The decision is made based on system state, including information of the duration of the current bin, type of requests being processed, and the set of remaining requests. It is illustrated in Fig. \ref{fig:control}.
    \end{itemize}
    \item \textbf{Assignment and allocation:}
        \begin{itemize}
        \item Each request must be assigned to exactly one prefill stage for processing.
        \item For every request, the prefill phase must be completed by the same client that will subsequently carry out its decode phase. Hence, both phases must be allocated to the same client.
        \item  A client can process only one request at a time. Once a client begins processing a request, it remains occupied and cannot preempt tasks until both the prefill and decode stages of that request are completed.
    \end{itemize}
\end{enumerate}

% \kai{Machine, client, and batch may be confusing. We need to handle this later.}

% Despite the flexibility offered by dynamic batching, maintaining optimal parallelism between prefill and decode stages can be challenging. Variability in the completion times of requests within dynamically sized batches may lead to idle periods in resource utilization. Efficiently scheduling these operations to minimize such idle times while maximizing throughput is crucial for optimizing large model inference.




% \subsection{Model formulation}
% The LLM inference process is formulated as a sequential decision making problem under uncertainty. We use $t\in \mathbb{R}^+$ to denote time point.
% Let $\mathcal{I}\left(t\right) \subseteq \mathcal{I}$ be the set of remaining requests to be scheduled at time $t$. 
%The process of handling a single request is divided into two operations: prefill and decode. The duration of the prefill operation linearly depends on the number of tokens and is assumed to be known. The duration of the decode operation also positively depends on the number of tokens, but the relationship is uncertain and observed to be non-linear. However, since the decoding process outputs one token per client per round and can be preempted by the prefill operation at the end of each round, the time for each round of decoding is only related to the number of parallel clients and can be considered linear.

%Let $N^{\textrm{p}}_i \in \mathbb{Z}^+$ and $N^{\textrm{d}}_i \in \mathbb{Z}^+$, for $i\in \mathcal{I}$, be the token numbers of prefill and decode of request $i$, respectively. Given the number of tokens, denoted by $N\in \mathbb{Z}^+$, the duration of prefill is given by $G^{\textrm{p}}\left(N\right) \in \mathbb{R}^+$ and is linear and positively related with the number of input tokens $N$. Similarly, we use $G^{\textrm{d}}\left(N\right)$ to denote the duration of the decode, and it is also a linear function and positively related with the number of output number $N$. A client can be in either prefill or decode operations, and at any time all clients have to be in the same operation. The inference process is divided into $K\in \mathbb{Z}^+$ bins, and each bin has one prefill operation and then followed by a decode operation. For a request, when the prefill starts, it has to be finished within the same bin. However, the decode operation of a request can be carried out in several consecutive bins if one bin is not long enough. In other words, the decode operation can be preempted by prefill operations. We assume that $K$ is large enough to have all requests to fit in. The online inference process is illustrated by a Grantt chart in Fig. \ref{fig:illustration}. There may be multiple clients, and the bars for the first and second clients are presented. Three bins are shown in the example, and all the clients have to be in the same operation, i.e., either prefill or decode. Among three bins, the first client processes three requests, while the second client only processes two. The first request of the second client takes two bins. 





\begin{figure}
    \centering
    \includegraphics[width=0.9\linewidth]{graph//control.pdf}
    \caption{Online scheduling}
    \label{fig:control}
\end{figure}

\subsection{Deterministic equivalence}
In this hybrid offline-online problem, the offline decision component involves determining the assignment and sequence of given requests on clients.
%which pertains to the decision variables \(t^\text{max}, p_{i,j,k}, d_{i,j,k}, x_{i,j}\). 
As a sequential decision-making problem, the online decision component focuses on determining the length of each bin and the sequence of remaining requests in the future, with the aim of minimizing the total inference time. 
%This part is associated with the decision variables \(t^\text{s,p}{k}, t^\text{s,d}{k}, n^\text{p}{k}, n^\text{d}{k}, p_{i,j,k}, d_{i,j,k}, w_{i,j,k}, y_{k,l}\).

Without considering uncertainty, the offline-online inference scheduling can be formulated as a deterministic equivalence, which is a form of an MIP model as follows. 
%We assume that the prefilling and decoding tokens of each request are known in advance, denoted as $N^\text{p}_i, N^\text{d}_i$, respectively. The optimization of inference throughput concerns the processing time in the prefilling and decoding bins. The decoding time per token is a function of $|\mathcal{J}|$, hence we can treat it as a constant parameter, denoted as $T^\text{d}$. The prefilling time, however, is a linear function of the total number of tokens to be prefilled in a bin. This varying marginal cost implies a nonlinear model, but can be linearized. We define a set of levels $l \in \mathcal{L}$ for the prefilling bins, each with a token capacity $N^{cap}_l$ and the corresponding prefilling time $T^p_l$, and use binary variable $y_{k,l}$ to denote that decoding bin $k$ is of level $l$. Assuming a monotonically non-decreasing function for the prefilling time, in the optimal solution, the token load among the prefilling bins will be balanced, while automatically choosing the minimally containable level for each bin.

\begin{align}
    \label{obj:makespan}
    & \min t^\text{max} \\
    & \nonumber \text{s.t.} \\
    \label{cst:makespan_def}
    & t^\text{max} \geq t^\text{s,d}_{k} + n^\text{d}_{k}, \text{ for } k \in \mathcal{K}^\text{d}, \\
%    \label{cst:bin_def_1}
%    & t^s_0 = 0, \\
    \label{cst:bin_def_2}
    & t^\text{s,p}_{k} - \left(t^\text{s,d}_{k-1} + n^\text{d}_{k-1}\right) \geq 0, \text{ for } k = 2,...,K,\\
    \label{cst:bin_def_3}
    & t^\text{s,d}_{k} - \left(t^\text{s,p}_{k} + n^\text{p}_{k}\right) \geq 0, \text{ for } k = 1,2,...,K,\\
    \label{cst:prefill_length_def_1}
    & n^\text{p}_{k} \geq \sum_{l \in \mathcal{L}} T^\text{p}_l y_{k,l}, \text{ for } k \in \mathcal{K}^\text{p},\\
    \label{cst:prefill_length_def_2}
    & \sum_{i \in \mathcal{I}} \sum_{j \in \mathcal{J}} N^\text{p}_i p_{i,j,k} \leq \sum_{l \in \mathcal{L}} N^\text{cap}_l y_{k,l}, \text{ for } k \in \mathcal{K}^\text{p},\\ 
    \label{cst:level_def}
    & \sum_{l \in \mathcal{L}} y_{k,l} = 1,  \text{ for } k \in \mathcal{K}^\text{p}, \\
    \label{cst:decode_length_def_1}
    & n^\text{d}_{k} \geq T^\text{d} \sum_{i\in \mathcal{I}} N_i^\text{d} w_{i,j,k}, \text{ for }j \in \mathcal{J}, \text{ and } k \in \mathcal{K}^\text{d},\\
    \label{cst:immediate_prefill_decode}
    & d_{i,j,k} - p_{i,j,k} \geq 0, \text{ for }i \in \mathcal{I}, j \in \mathcal{J}, \text{ and } k \in \mathcal{K}^\text{d}, \\
    \label{cst:consecutive_decode_bin}
    \begin{split}
    &M\left(2 - d_{i,j,k_1} - d_{i,j,k_2}\right) + \sum_{k'=k1}^{k_2} d_{i,j,k'} \\ 
    &\quad \geq k_2 - k_1 + 1, \\
    &\quad \text{ for }i \in \mathcal{I}, j \in \mathcal{J}, k_1,k_2 \in \mathcal{K}^\text{d},\text{ and }k_1<k_2, 
    \end{split} \\
    \label{cst:prefill-decode-order}
    \begin{split}
    & M\left(p_{i,j,k_1}-1\right) +d_{i,j,k_2} \leq 0, \\
    &\quad \text{ for }i \in \mathcal{I}, j \in \mathcal{J}, k_1 \in \mathcal{K}^\text{p},k_2\in \mathcal{K}^\text{d}, \text{ and }k_1 > k_2, 
    \end{split} \\
    \label{cst:d_def_1}
    & \sum_{i\in \mathcal{I}} d_{i,j,k} \leq 1, \text{ for }j \in \mathcal{J}, \text{ and } k \in \mathcal{K}^\text{d}, \\
    \label{cst:d_def_2}
    & \sum_{k \in \mathcal{K}^\text{d}} d_{i,j,k} \leq K, \text{ for }i \in \mathcal{I}, \text{ and }j \in \mathcal{J}, \\
    \label{cst:w_def_1}
    & \sum_{k \in \mathcal{K}^\text{d}} w_{i,j,k} = x_{i,j}, \text{ for }i \in \mathcal{I}, \text{ and }j \in \mathcal{J}, \\
    \label{cst:w_def_2}
    & \sum_{j\in \mathcal{J}} \sum_{k \in \mathcal{K}^\text{d}} w_{i,j,k} = 1, \text{ for }i \in \mathcal{I}, \\
    \label{cst:p_def_1}
    & \sum_{i \in \mathcal{I}} p_{i,j,k} \leq 1, \forall j, k, \\
    \label{cst:p_def_2}
    & \sum_k p_{i,j,k} = x_{i,j}, \text{ for }j \in \mathcal{J}, \text{ and } k \in \mathcal{K}^\text{p}, \\
    \label{cst:loc_def}
    & \sum_{j \in \mathcal{J}} x_{i,j} = 1, \text{ for }i \in \mathcal{I}, \\
    \label{cst:var_def_start}
    & x_{i,j} \in \{0,1\}, \text{ for }i \in \mathcal{I}, \text{ and } j \in \mathcal{J}, \\
    & y_{k,l} \in \{0,1\}, \text{ for }k \in \mathcal{K}^\text{p}, \text{ and } j \in \mathcal{J}, \\
    & p_{i,j,k} \in \{0,1\}, \text{ for }i \in \mathcal{I}, j \in \mathcal{J}, k \in \mathcal{K}^\text{p}, \\
    & d_{i,j,k} \in \{0,1\}, \text{ for }i \in \mathcal{I}, j \in \mathcal{J}, k \in \mathcal{K}^\text{d}, \\
    & w_{i,j,k} \in [0,1], \text{ for }i \in \mathcal{I}, j \in \mathcal{J}, k \in \mathcal{K}^\text{d}, \\
    & t^\text{s,p}_{k}, n^\text{p}_{k}, \text{ for }k \in \mathcal{K}^\text{p},\\
    \label{cst:var_def_end}
    & t^\text{s,d}_{k}, n^\text{d}_{k}, \text{ for }k \in \mathcal{K}^\text{d}.
\end{align}
The total inference time is denoted by $t^\text{max}$, which is minimized in the objective function given in Eq. (\ref{obj:makespan}). We let $\mathcal{K}^\text{d}=\left\{1,2,\cdots \right\}$ be the set of decode stages. For any decode stage $k\in \mathcal{K}^\text{d}$, let $t^\text{s,d}_{k}\in \mathbb{R}^+$ and $n^\text{d}_{k}\in \mathbb{R}^+$ denote the start time and time length of the $k$th decode stage, respectively, and thus its end time is $\left(t^\text{s,d}_{k}+n^\text{d}_{k}\right)$. Eq. (\ref{cst:makespan_def}) requires that the total inference time is greater than or equal to the end time of any decode stage. Let $K \in \mathbb{Z}^+$ be the total number of bins, prefill stages, and also decode stages. Any prefill stage should start after the end of its previous decode stage, suggested by Eq. (\ref{cst:bin_def_2}). For any prefill stage $k\in \mathcal{K}^\text{p}$, let $n^\text{p}_{k}\in \mathbb{R}^+$ be the time length of the $k$th prefill stage. Thus, Eq. (\ref{cst:bin_def_3}) suggests that, within the same bin, the decode stage should start after the end of prefill stage. We use $\mathcal{L}=\left\{1,2,\cdots \right\}$  to represent the set of all possible levels for a prefill stage, and the time length of a prefill stage depends on the level. Let $y_{k,l}\in \left\{0,1\right\}$, for $k\in \mathcal{K}^\text{p}$ and $l \in \mathcal{L}$, be an indicator variable. Prefill stage $k$ is at level $l$, if $y_{k,l}=1$. Otherwise, the prefill stage is not at level $l$. We denote by $T^\text{p}_l \in \mathbb{R}^+$ the prefill time of level $l$, for $l\in \mathcal{L}$, and thus the time length of a prefill stage is determined in Eq. (\ref{cst:prefill_length_def_1}). We use $N^\text{cap}_l \in \mathbb{Z}^+$ to denote the maximum token capacity of level $l$, for $l\in \mathcal{L}$, and let $p_{i,j,k}\in \left\{0,1\right\}$ denote assignment of prefill stage $k$ to request $i$ on client $j$ for prefill phase, for $i\in \mathcal{I}$, $j\in \mathcal{J}$, and $k\in \mathcal{K}^\text{p}$. The relationship between $p_{i,j,k}$ and $y_{k,l}$ is given in Eq. (\ref{cst:prefill_length_def_2}). Eq. (\ref{cst:level_def}) suggests that any prefill stage can only be at one level in $\mathcal{L}$. Different than the prefill phase, the decode phase of a request can be served in multiple decode stages. Thus, we introduce $w_{i,j,k}\in [0,1]$ to represent the proportion of the decoding phase of request $i$ executed in the $k$th decode stage on client $j$, for $i\in \mathcal{I}$, $j\in \mathcal{J}$, and $k\in \mathcal{K}^\text{d}$. The time length of a decode stage is provided by Eq. (\ref{cst:decode_length_def_1}). Let $d_{i,j,k}\in \{0,1\}$ be the assignment of decode stage $k$ to request $i$ on client $j$ for decode phase, for $i\in \mathcal{I}$, $j\in \mathcal{J}$, and $k\in \mathcal{K}^\text{d}$. Eqs. (\ref{cst:immediate_prefill_decode})-(\ref{cst:prefill-decode-order}) together sets the rule that the decoding phase of a request must immediately follow the consecutive prefilling phase or its previous decoding phase. Eqs. (\ref{cst:d_def_1})-(\ref{cst:loc_def}) relate the assignment decisions among requests, clients, and bins. Eqs. (\ref{cst:var_def_start})-(\ref{cst:var_def_end}) define the domain for each set of variables, respectively.

Eqs (\ref{obj:makespan})-(\ref{cst:var_def_end}) will hereafter be referred to as ``the original model''. We attempted to solve the original model using commercial MIP solvers, such as Gurobi. However, we found it nearly impossible to solve such a large-scale problem directly. The number of requests, denoted by $|\mathcal{I}|$, is around 1,000 in small cases, while in larger cases, it can be up to 10,000. The number of batch sizes or client numbers, denoted by $|\mathcal{J}|$, can reach up to 200. The number of exclusive bins, denoted by $|\mathcal{K}|$, often is on the same order of magnitude as $|\mathcal{I}|$. %often exceeds the number of requests by a factor of three or more. 
To probe the solving cost, we solved a down-scaled toy case model with merely 100 requests and 20 clients, which took Gurobi more than 3,600 seconds to yield a near-optimal solution without fully closing the dual gap. Solving this problem in its original form is hence evidently impractical and cannot be solved in hours. Therefore, it is necessary to decompose this problem into stages and address it sequentially.

\section{Solution Method}
\label{solution_method}

\begin{figure*}
    \centering
    \includegraphics[width=0.8\linewidth]{graph//solution-method.pdf}
    \caption{Illustration of Solution Method}
    \label{fig:solution-method}
\end{figure*}

\subsection{Method overview}
The original model, provided by Eqs. (\ref{obj:makespan})-(\ref{cst:var_def_end}), demands a solution method capable of making decisions within 10 milliseconds in an asynchronous cycle, as the decoding time per client batch can be around 50 milliseconds. However, the original model is a large-scale MIP model with over 100,000 integer decision variables and constraints, making it difficult to solve even within several hours. Hence, it is vital to develop an efficient solution method that provides the best possible outcomes within the required time frame. As illustrated in Fig. \ref{fig:solution-method}, we propose a \textit{hybrid offline-online method} that structures the scheduling process for large model inference into two main methods: offline requests assignment and online scheduling. Each  method involves a subset of decision variables of the original model and provides timely solutions at each stage. In the figure, we illustrate how the offline-online information, as well as the decision making given by the scheduling models, is obtained and shared in the system.

\textbf{Offline Requests Scheduling}: In this method, a predetermined batch size of clients determines the number of parallel requests. Each client is allocated a balanced number of requests, resulting in an equitable task distribution. This method considers the assignment decisions in the original model as described in constraints (\ref{cst:makespan_def}), (\ref{cst:d_def_1}), and (\ref{cst:w_def_2})--(\ref{cst:p_def_1}). We isolate this part of the model to demonstrate offline training scenarios, such as RLHF (Reinforcement Learning with Human Feedback) training. In this task, requests are typically given and known in advance. Users can manually send their request prompts to the LLMs and wait to receive the outputs. These tasks can implement offline request assignment methods to achieve better throughput. However, for most scenarios such us user using GPT, using the offline method is still limited since solving the MIP model usually takes 10 minutes or more, which cannot meet the rapid iteration requirements of the LLM online decision-making process. Therefore, it is necessary to develop an online method to fill this gap.

\textbf{Online Scheduling}: The online scheduling process comprises two major parts corresponding to two types of online decisions. The first part, \textit{online requests scheduling}, determines which requests are scheduled in the upcoming bin and identifies the next client to serve once a previous request is completed. The second part, \textit{online iteration scheduling}, decides when to conclude the current decoding bin and start a preemptive prefilling bin to enhance overall utilization rates.

In the online requests scheduling part, heuristic methods are employed to determine the optimal order for processing requests in real-time. This approach considers factors such as task priority, current system load, and anticipated resource availability. By effectively prioritizing requests, the system can minimize waiting times and maximize throughput under dynamic operational conditions. This method emphasizes the implementation of the relaxed solutions provided by job assignment and illustrates the constraints (\ref{cst:bin_def_2})--(\ref{cst:prefill-decode-order}) in the original model.

The online iteration scheduling part aims to minimize idle time on machines by strategically allocating computational resources. By dynamically adjusting prefilling and decoding task priorities based on real-time feedback and system constraints, this method enhances overall system efficiency and responsiveness. This proactive scheduling approach minimizes machine idle time and optimizes the utilization of processing resources, thereby improving the overall performance of large language model inference tasks. This method underscores iteration-based optimization and considers the constraints (\ref{cst:w_def_1})--(\ref{cst:w_def_2}) and (\ref{cst:p_def_2})--(\ref{cst:loc_def}) in the original model.

% \textbf{Online Scheduling}: The online scheduling includes two major parts corresponding to two types of online decisions. Online requests scheduling decides the requests scheduled in the next bin and next clients when previous request is finished. Online iteration scheduling decides when to end current decoding bin to start a preemptive prefilling bin in order to improve total utilization rate. 

% The online requests scheduling phase employs heuristic methods to determine the optimal order of processing requests in real-time. This heuristic approach considers factors such as task priority, current system load, and anticipated resource availability. By prioritizing requests effectively, the system can minimize wait times and maximize throughput during dynamic operational conditions. This phase emphasizes the realization of the relaxed solutions given by job assignment and illustrate the constraints \ref{cst:bin_def_1}-\ref{cst:prefill-decode-order} in the original model.

% And the online iteration scheduling phase aims to reduce idle time on machines by strategically allocating computational resources. By dynamically adjusting task priorities based on real-time feedback and system constraints, the online operation scheduling phase enhances overall system efficiency and responsiveness. This proactive scheduling approach minimizes machine idle time and optimizes the utilization of processing resources, thereby improving the overall performance of large language model inference tasks. This phase emphasized the iteration-based optimization and illustrate the constraints \ref{cst:w_def_1}-\ref{cst:w_def_2} and \ref{cst:p_def_2}-\ref{cst:loc_def} in the original model.

The overarching goal of this structured approach is to minimize the total inference time for a specific benchmark, thereby maximizing throughput. By integrating thorough data analysis, efficient task allocation, and adaptive online scheduling strategies, this scheduling solution optimizes the performance of LLM inference processes. This holistic approach not only enhances system efficiency but also supports scalability and reliability in handling complex computational tasks.



\subsection{Offline requests scheduling \& theoretical lower bound}
As previously introduced, we begin by examining the offline request assignment decisions within the original model, with a specific focus on the constraints described in (\ref{cst:makespan_def}), (\ref{cst:d_def_1}), and (\ref{cst:w_def_2}--\ref{cst:p_def_1}). This part of the model is isolated to demonstrate offline training scenarios, such as RLHF training. In this scenario, we tackle the Minimizing Makespan Bin Packing Problem to efficiently address the workload balancing challenge. We assume that the output length is predetermined, and that prefill decode stages do not conflict during problem-solving. Nevertheless, in practical applications and simulations used to evaluate performance, we adhere to these constraints by allocating workload to clients without affecting the uncertainty of output length.

In the offline model outlined in Eqs. (\ref{offline_bin_packing_start})--(\ref{offline_bin_packing_end}), we introduce a new parameter, denoted by $T_i \in \mathbb{R}^+$ for $i\in \mathcal{I}$, representing the estimated decode completion time for request $i$. We also introduce a new decision variable, denoted by $t_j \in \mathbb{R}^+$ for $j\in \mathcal{J}$, to indicate the total decoding time for client $j$.

\begin{align}
    \label{offline_bin_packing_start}
    & \min \max_{j\in \mathcal{J}} t_j \\
    & \nonumber \text{s.t.} \\
    & \sum_{j \in \mathcal{J}} x_{i,j} = 1, \text{ for } i\in \mathcal{I}, \\
    & \sum_{i\in \mathcal{I}} x_{i,j} T_i \leq t_j, \text{ for } j\in \mathcal{J}, \\    
    & x_{i,j} \in \{0,1\}, \text{ for } i\in \mathcal{I},\text{ and }j\in \mathcal{J},\\
    \label{offline_bin_packing_end}
    & t^\text{max}, t_j \in \mathbb{R}^+, \text{ for } j \in  \mathcal{J}.
\end{align}

The offline model also provides a method to calculate the theoretical lower bound for a given set of requests, $\mathcal{I}$. In this method, we assume that prefill and decode phases for all the requests can be separated into two groups, and we calculate the optimal inference time for each group.

Let $t^{\text{p}*} \in \mathbb{R}^+$ and $t^{\text{d}*} \in \mathbb{R}^+$ represent the optimal total prefill and total decode times for all the requests in set $\mathcal{I}$, respectively. The value of $t^{\text{d}*}=\max_{j\in \mathcal{J}} t_j$ is obtained from the objective function value from Eqs. (\ref{offline_bin_packing_start})--(\ref{offline_bin_packing_end}). Let $L=\arg\max_{l \in \mathcal{L}} N_l^{\text{cap}}$, and then the largest prefill time across all levels in $T^\text{p}_l$ is denoted by $T^\text{p}_L$. $N^{\text{cap}}_L$ is the number of maximum number of tokens that can be processed in $T^\text{p}_L$. Then, $t^{\text{p}*}$ can be calculated by the following equation. 

\begin{align}
t^{\text{p}*} \geq T^\text{p}_L  \left\lfloor \frac{\sum_{i\in \mathcal{I}} N_i^{\text{p}}}{N^{\text{cap}}_L} \right\rfloor.
\end{align}
It yields a tight theoretical lower bound $T^{\text{LB}}$ as follows.

\begin{align}
\label{theoreticalLB}
T^{\text{LB}} = t^{\text{p}*} + t^{\text{d}*}.
\end{align}

\subsection{Online requests and iteration scheduling}
In this online part of the LLM inference optimization problem, two critical considerations arise. First, we need to determine which request to send to an available client once the previous request is completed. Second, when a round of decoding stage is finished, we must decide whether to send a preemptive prefill stage or continue this decode stage to the LLM workers for the subsequent time frame.

The first issue presents an online scheduling problem, as illustrated by Eqs. (\ref{cst:w_def_1})--(\ref{cst:w_def_2}) and (\ref{cst:p_def_1})--(\ref{cst:p_def_2}). The primary decision in this context is whether to select a new request to override the original assignment in order to achieve better machine utilization.

A sorting and online preemptive method is illustrated in Algorithm \ref{algorithm1}. This online algorithm first selects the future available requests, denoted as $I_j$, for client $j \in \mathcal{J}$. The set $I_j$ is sorted by $N_i^{\text{p}} + N_i^{\text{d}}$, that is, $N_{i_1}^{\text{p}} + N_{i_1}^{\text{d}}>N_{i_2}^{\text{p}} + N_{i_2}^{\text{d}} \forall i_1<i_2 \in I_j$. Then, for each client, the algorithm calculates future requests and counts the expected remaining tokens $remain\_token(j) \text{ for } j\in \mathcal {J}$ to be processed. Idle clients then greedily select the longest request from busy clients to process. This algorithm utilizes offline information on request assignment to provide timely online assignment decisions.

\begin{algorithm}
\caption{Sorting and Online Preemptive Method}
\label{algorithm1}
\begin{algorithmic}
\Require $\{ I_j = \{ i \mid x_{ij} = 1 \}, \forall j \in \mathcal{J} \}$ 
% \State \textbf{Sort:} $I_j = \{i_1, i_2, \ldots, i_n\}$ by $N_i^{\text{p}} + N_i^{\text{d}}$, $\forall j \in \mathcal{J}$
\For{client $j$ in clients $\mathcal{J}$}
    \If{queue for client $j$ is empty and $I_j \neq \emptyset$}
        \State pop $I_j$ to client $j$
        \State $remain\_token(j) \gets remain\_token(j) - (N_i^{\text{p}} + N_i^{\text{d}})$
    \ElsIf{$\max(remain\_token(j)) > 0$}
        \State pop $\arg\max(remain\_token)$ to client $j$
        \State $remain\_token(j) \gets remain\_token(j) - (N_i^{\text{p}} + N_i^{\text{d}})$
    \EndIf
\EndFor
\end{algorithmic}
\end{algorithm}

Continuing from the previous discussion, the second problem involves a sequential decision-making process, as outlined by Eqs. (\ref{cst:makespan_def})--(\ref{cst:prefill-decode-order}). The main challenge here is to deliver timely and efficient decisions in real time. As previously mentioned, each round of decoding takes approximately 50 milliseconds. Thus, it is essential to ensure that decisions are made within 10 milliseconds to sustain system efficiency. To achieve this, we employ the following method to integrate quick decision-making into the process.

This aspect of decision-making corresponds to the following problem.

\begin{align}
    &\nonumber \min t^\text{max} \\
    & \nonumber \text{s.t.} \\
    \label{tmax}
    & t^\text{max} \geq t_{k}^\text{s,d} + n_{k}^\text{d}, \text{ for } k \in \mathcal{K}^\text{d}, \\
    & t^{\text{s}}_{p,k} - (t^{\text{s}}_{d,k-1} + n^{\text{d}}_{k-1}) \geq 0,  \text{ for } k = 2,...,K,\\
    \label{tsdk}
    & t^{\text{s,d}}_{k} - (t^{\text{s,p}}_{k} + n^{\text{p}}_{k}) \geq 0,  \text{ for } k \in \mathcal{K}^\text{p},\\ 
    \label{npk}
    & n^{\text{p}}_{k} \geq \sum_{l\in \mathcal{L}} T^{\text{p}}_l y_{k,l}, \text{ for } k \in \mathcal{K}^\text{p},\\
    & \sum_{i\in \mathcal{I},j\in \mathcal{J}} N_i^{\text{p}} p_{i,j,k} \leq \sum_{l\in \mathcal{L}} N^{\text{cap}}_l y_{k,l}, \text{ for } k \in \mathcal{K}^\text{p}, \\ 
    & \sum_{l\in \mathcal{L}} y_{k,l} = 1, \text{ for } k \in \mathcal{K}^\text{p}, \\
    \label{ndk}
    & n^{\text{d}}_{k} \geq T^{\text{d}} \sum_{i} N_i^{\text{d}} w_{i,j,k}, \text{ for } j\in \mathcal{J}, \text{ and } k \in \mathcal{K}^\text{d},\\
    & d_{i,j,k} - p_{i,j,k} \geq 0, \text{ for } i\in \mathcal{I}, j\in \mathcal{J}, \text{ and } k \in \mathcal{K}.
\end{align}
By combining Eqs. (\ref{tmax}) and (\ref{tsdk}), we derive that $t^{\text{max}} \geq \max_{k \in \mathcal{K}} \left(t^\text{{s,p}}_{k} + n^\text{p}_{k} + n^\text{d}_{k}\right) $. In the context of online decision-making, the start time $t^\text{{s,p}}_{k}$ is typically influenced by the completion time of preceding tasks. The primary objective is to minimize the total time cost of prefill and decode stages. Consequently, we establish the following equation by integrating the calculations of $n^\text{p}_{k}$ and $n^\text{d}_{k}$ from Eqs. (\ref{npk}) and (\ref{ndk}).

\begin{align}
    \min t^\text{max} \geq \max_{k \in \mathcal{K}} \left( t^\text{s,p}_{k} + \sum_{l\in \mathcal{L}} T^\text{p}_l y_{k,l} + T^\text{d} \sum_{i \in \mathcal{I}, j\in \mathcal{J}} N_i^\text{d} w_{i,j,k} \right).
\end{align}
In this problem, the time cost is divided into two components. The cost for adding a prefill task at any point is given by

\begin{align}
    \frac{\partial t^\text{max}}{\partial y_{k,l}} = \sum_{l \in \mathcal{L}} T^\text{p}_l,
\end{align}
and the cost for adding a decode task at the decision-making moment is expressed by

\begin{align}
    \frac{\partial t^\text{max}}{\partial w_{i,j,k}} = T^\text{d} \sum_{i\in \mathcal{I}} N_i^\text{d}.
\end{align}
Thus, our heuristic method for deciding whether to dispatch a prefill or decode stage to the LLM worker involves comparing the prefill cost $C_p = \sum_l T^p_l$ with the waited decode time $C_d = T^d \sum_{i\in \mathcal {I}, j\in \mathcal{J}} N_i^d w_{i,j,k}$. If $C_p \geq C_d$, the algorithm advises continuing with a round of the decode task and waiting for additional prefill tasks; otherwise, the algorithm recommends executing a round of the prefill task.

% \subsection{An illustrative example}
% \label{a_case_study}

% Here is an example of a table.

% \begin{table}
% \caption{Parameter setting for illustrative example}
% \begin{centering}
% \begin{tabular}{ccccccccc}
% \hline 
% $i$         & 1    &   2  &   3  &   4  &   5  &   6  &   7  &   8 \tabularnewline
% \hline 
% $e_i$       & 0.62  & 0.75 & 0.85 & 0.73 & 0.68 & 0.91 & 0.81 & 0.72  \tabularnewline
% %p_i       & 0.082  & 0.044 & 0.051 & 0.089 & 0.081 & 0.020 & 0.119 & 0.121  \tabularnewline
% $r_i$       & 0.35 & 0.44 & 0.37 & 0.24 & 0.27 & 0.35 & 0.29 & 0.46  \tabularnewline
% $N_i$       &  5   &   6  &   7  &   7  &   7  &   5  &   6  &  6    \tabularnewline
% $T_{i,max}$ &  8   &   8  &   10  &   10 &   8  &   7  &   9  &  9 \tabularnewline
% $T_{i,min}$ &  1   &   1  &   2  &   2  &   1  &   1  &   2  &  1 \tabularnewline
% \hline 
% \end{tabular}
% \par\end{centering}
% \label{setting_illustrative_example}
% \end{table}




% \begin{figure}[t]
% \centering
%   \includegraphics[width=0.99\linewidth]{graph//result_gantt1.png}\\
%   \caption{Baseline: Time Elapsed 201 Throughput 6.58 Utilization: 0.802}
%   \label{layout}
% \end{figure}

% \begin{figure}[t]
% \centering
%   \includegraphics[width=0.99\linewidth]{graph//result_gantt2.png}\\
%   \caption{Baseline: Time Elapsed 201 Throughput 6.58 Utilization: 0.802}
%   \label{layout}
% \end{figure}

% \begin{figure}[t]
% \centering
%   \includegraphics[width=0.99\linewidth]{graph//result_gantt3.png}\\
%   \caption{Simple assignment: Time Elapsed 197.08 Throughput 6.69 Utilization: 0.8549}
%   \label{layout}
% \end{figure}

% \begin{figure}[t]
% \centering
%   \includegraphics[width=0.99\linewidth]{graph//result_gantt4.png}\\
%   \caption{Adjustment: Time Elapsed 191.66 Throughput 6.88 Utilization: 0.8821}
%   \label{layout}
% \end{figure}

% \begin{figure}[t]
% \centering
%   \includegraphics[width=0.99\linewidth]{graph//result_gantt5.png}\\
%   \caption{Online: Time Elapsed 190.58 Throughput 6.92 Utilization: 0.8906}
%   \label{layout}
% \end{figure}

% \begin{figure}[t]
% \centering
%   \includegraphics[width=0.99\linewidth]{graph//result_gantt6.png}\\
%   \caption{RL: Time Elapsed 190.08 Throughput 6.94 Utilization: 0.893}
%   \label{layout}
% \end{figure}
\section{Numerical Experiment}
\label{numerical-experiment}
\subsection{Experiment settings and baseline}
Before the model is solved by our hybrid method, extensive analysis is conducted to evaluate the time taken by the decode and prefill stages on hardware. This analysis provides crucial insights into the computational demands and performance characteristics of each stage. By quantifying these metrics, such as processing times and resource utilization, the data analysis establishes a solid foundation of empirical data. The data serve as reliable support for subsequent decision-making in optimizing scheduling strategies. The basic experiment setting is given in TABLE \ref{Experiment Settings}.


\begin{table}
    \centering
    \caption{Experiment Settings}
    \begin{tabular}{ccc}
    \toprule
    Parameter     & Number & Brief Description\\
    \midrule
    $|\mathcal{I}|$     & 1319 & The GSM8K dataset \\
    $|\mathcal{J}|$     & 200 & Due to hardware memory limit\\
    $\mathbf{E}(N_i^\text{p})$     & 68.43 & The GSM8K dataset input\\
    $\mathbf{E}(N_i^\text{d})$     & 344.83 & The Llama 65B output\\
    $T^\text{p}$     & 0.13 ms/token & The hardware performance on prefilling\\
    $T^\text{d}$     & 0.21 ms/token & The hardware performance on decoding\\
    \bottomrule

    \end{tabular}
    \label{Experiment Settings}
\end{table}

We demonstrate our improvements by utilizing the GSM8K dataset \cite{cobbe2021training}, a comprehensive collection of mathematical word problems specifically designed to assess the problem-solving and reasoning capabilities of language models. This dataset serves as a benchmark for evaluating both arithmetic and logical reasoning skills, essential attributes for advanced language models. Each problem in the dataset is intricately constructed to simulate real-world situations involving numerical relationships, necessitating that models comprehend the problem contextually and perform accurate calculations to derive the correct solution.

The GSM8K dataset comprises 1,319 unique problems as input requests, with an average input length of 68.43 tokens and a standard deviation of 25.04 tokens. For our experiments, we selected the LLaMA-65B language model due to its open-source nature and wide accessibility, making it a suitable candidate for academic and reproducible research. In our tests, the LLaMA-65B model generated responses averaging 344.83 tokens, with a standard deviation of 187.99 tokens. To ensure consistency and focus on quality responses, we constrained the maximum output length to 512 tokens during testing.

Our computational setup is characterized by a robust hardware configuration, consisting of eight Ascend processing units, each equipped with a maximum memory capacity of 64 GB. This formidable hardware infrastructure is essential for facilitating the efficient processing and testing necessary for our experiments. Additionally, we have assessed the KV cache usage for input in this experiment, establishing baseline settings that are also utilized in practical applications. The current hardware, along with the LLM employed, imposes a memory constraint of 1024 blocks of KV cache. Each block can accommodate a maximum of 128 tokens. For the GSM-8k benchmark, the combined maximum input and output for each request requires five blocks. Consequently, this configuration limits us to a maximum of approximately 200 clients running concurrently, calculated by the expression $1024/5 \approx 200 $.

In our experimental setup, we conduct an estimation of the operation time required for prefill and decode stages using over 400 data groups. We find that both prefill and decode times exhibit a linear relationship with the number of tokens involved. Specifically, the prefill time can be calculated as 0.13 milliseconds per token, plus a fixed overhead of 25 milliseconds. For the decode process, the time required for each batch of clients can be estimated as 0.21 milliseconds per token, with an additional fixed overhead of 29 milliseconds. For instance, when processing a parallel batched decode stage involving 200 clients, where each client produces one token per round, the operation would take approximately $200 \times 0.21 + 29 = 71$ milliseconds. In the case of prefill stages, if a batch consists of inputs totaling 5,000 tokens, the estimated time required would be $5000 \times 0.13 + 25 = 675$ milliseconds.

We present a Gantt chart in Fig. \ref{fig:baseline}, generated from an experiment using real-world data and open source LLM, to illustrate the current state of online inference services without the implementation of our proposed method. This chart demonstrates that, in practical scenarios, a significant number of idle periods, or ``bubbles'', occur when no scheduling strategy is employed. Furthermore, in offline scenarios, if the workload among clients is not evenly distributed, substantial machine idle time is observed after the early completion of some client's tasks. Our analysis of this Gantt chart reveals that the overall machine utilization rate is only 80.2\%.

\begin{figure}
    \centering
    \includegraphics[width=0.99\linewidth]{graph//result_gantt_baseline.pdf}
    \begin{flushleft}
    \vspace{-2em}
    \caption*{Utilization rate: 80.2\%. Total inference time: 201.00 seconds.}
    \end{flushleft}
    \vspace{-2em}
    \caption{Result Gantt: Baseline}
    \label{fig:baseline}
\end{figure}

\subsection{Offline request scheduling result}
This offline request scheduling model given by Eqs.  (\ref{offline_bin_packing_start})--(\ref{offline_bin_packing_end}) can be solved using open-source solver SCIP. Due to significantly reduced complexity, optimal solutions can be achieved within 20 minutes comparing to original problem which is not possible to be solved within hours. Although this offline model only addresses workload balancing using estimations of output length, its performance surpasses that of the original version. As illustrated in Fig. \ref{fig:offline_assignment}, the system shows a significant reduction of idle times, and machine utilization is enhanced to 85.5\%. Comparing to the baseline method, this method provides a more balanced request assignment across clients and reduce ``bubbles''. The total inference time can be reduced from 201.00 seconds to 197.08 seconds. Since solving the model still takes relatively long time, we list this method as optional and suggest practitioners use the offline model in typical scenarios such as RLHF training. 

\begin{figure}
    \centering
    \includegraphics[width=0.99\linewidth]{graph//result_gantt_job_assignment.pdf}
    \begin{flushleft}
    \vspace{-2em}
    \caption*{Utilization rate: 85.5\%. Total inference time: 197.08 seconds.}
    \end{flushleft}
    \vspace{-2em}
    \caption{Result Gantt: Offline Request Scheduling}
    \label{fig:offline_assignment}
\end{figure}


\subsection{Online scheduling results}

Incorporating online requests and iteration scheduling methods, as depicted in Fig. \ref{fig:online+offline}, results in a marked improvement in total inference time, showing reductions 190.58s compared to 201.00s in the baseline scenario. Additionally, machine utilization is enhanced to 89.06\%. Comparing to offline scheduling method, these two online methods do not require additional computing and can be used for current online inference.

In Fig. \ref{fig:online_only}, we present the results obtained using only the online scheduling method, without employing the offline scheduling method. As shown, compared to the baseline, the utilization rate improves to 86.19\%, and the total inference time decreases to 193.33 seconds. These results demonstrate that the online method performs well even in the absence of prior knowledge about requests. This scenario is common in the area of LLM inference.

We also calculate the theoretical lower bound using Eq. (\ref{theoreticalLB}). In the specified numerical case utilizing GSM8K, the theoretical bound is 180 seconds, in which $T^{\text{p}*}=13$ seconds and $T^{\text{d}*}=167$ seconds. In this scenario, we reduce the total inference time from 201.00 seconds in the baseline to 190.08 seconds with the hybrid online-offline method. The gap to the optimal value is thus reduced from \(201 - 180 = 21\) seconds to \(190 - 180 = 10\) seconds, representing a reduction of 52.4\% in this ``primal dual'' gap.

% \begin{figure}
%     \centering
%     \includegraphics[width=0.99\linewidth]{graph/result_gantt_online_request.pdf}
%     \begin{flushleft}
%     \vspace{-2em}
%     \caption*{Utilization rate: 88.21\%. Total inference time: 191.66 seconds.}
%     \end{flushleft}
%     \vspace{-2em}
%     \caption{Result Gantt: Offline+Online Request Scheduling}
%     \label{fig:online_request}
% \end{figure}

\begin{figure}
    \centering
    \includegraphics[width=0.99\linewidth]{graph/result_gantt_online_scheduling_only.pdf}
    \begin{flushleft}
    \vspace{-2em}
    \caption*{Utilization rate: 86.19\%. Total inference time: 193.33 seconds.}
    \end{flushleft}
    \vspace{-2em}
    \caption{Result Gantt: Online only Scheduling}
    \label{fig:online_only}
\end{figure}

\begin{figure}
    \centering
    \includegraphics[width=0.99\linewidth]{graph/result_gantt_online_iteration.pdf}
    \begin{flushleft}
    \vspace{-2em}
    \caption*{Utilization rate: 89.06\%. Total inference time: 190.58 seconds.}
    \end{flushleft}
    \vspace{-2em}
    \caption{Result Gantt: Offline+Online Scheduling}
    \label{fig:online+offline}
\end{figure}


To better demonstrate the performance of our online scheduling methods, we present a numerical experiment involving 100 cases in Figs. \ref{fig:result_numerical_utilization} and  \ref{fig:result_numerical_generate_speed}. These cases are randomly generated with the input and output length distributions shown in TABLE \ref{Experiment Settings}. As illustrated in the figures, despite some variations across the 100 cases, our hybrid offline-online method consistently outperforms in both utilization and generation speed. The unit for generation speed is tokens per second, indicating how many tokens the LLM can generate each second. On average, our method achieves an 8.0\% improvement in utilization and an increase of 100.63 tokens per second in generation speed.



\begin{figure}
    \centering
    \includegraphics[width=0.99\linewidth]{graph/result_numerical_utilization.jpg}
    \caption{Utilization rate with 100 cases}
    \label{fig:result_numerical_utilization}
\end{figure}

\begin{figure}
    \centering
    \includegraphics[width=0.99\linewidth]{graph/result_numerical_generate_speed.jpg}
    \caption{Generate speed with 100 cases}
    \label{fig:result_numerical_generate_speed}
\end{figure}
\section{Conclusion }
This paper introduces the Latent Radiance Field (LRF), which to our knowledge, is the first work to construct radiance field representations directly in the 2D latent space for 3D reconstruction. We present a novel framework for incorporating 3D awareness into 2D representation learning, featuring a correspondence-aware autoencoding method and a VAE-Radiance Field (VAE-RF) alignment strategy to bridge the domain gap between the 2D latent space and the natural 3D space, thereby significantly enhancing the visual quality of our LRF.
Future work will focus on incorporating our method with more compact 3D representations, efficient NVS, few-shot NVS in latent space, as well as exploring its application with potential 3D latent diffusion models.



% if have a single appendix:
%\appendix[Proof of the Zonklar Equations]
% or
%\appendix  % for no appendix heading
% do not use \section anymore after \appendix, only \section*
% is possibly needed

% use appendices with more than one appendix
% then use \section to start each appendix
% you must declare a \section before using any
% \subsection or using \label (\appendices by itself
% starts a section numbered zero.)
%


% \appendices
% \section{If we have appendices}
% \label{rest_transitions}
% They can be put here.
% \bowen{Why not using Short Job First method, we should provide an illustration.}

% you can choose not to have a title for an appendix
% if you want by leaving the argument blank
%\section{}
%Appendix two text goes here.


% use section* for acknowledgment
%\section*{Acknowledgment}


%The authors would like to thank...


% Can use something like this to put references on a page
% by themselves when using endfloat and the captionsoff option.
\ifCLASSOPTIONcaptionsoff
  \newpage
\fi



% trigger a \newpage just before the given reference
% number - used to balance the columns on the last page
% adjust value as needed - may need to be readjusted if
% the document is modified later
%\IEEEtriggeratref{8}
% The "triggered" command can be changed if desired:
%\IEEEtriggercmd{\enlargethispage{-5in}}

% references section

% can use a bibliography generated by BibTeX as a .bbl file
% BibTeX documentation can be easily obtained at:
% http://mirror.ctan.org/biblio/bibtex/contrib/doc/
% The IEEEtran BibTeX style support page is at:
% http://www.michaelshell.org/tex/ieeetran/bibtex/
%\bibliographystyle{IEEEtran}
% argument is your BibTeX string definitions and bibliography database(s)
%\bibliography{IEEEabrv,../bib/paper}
%
% <OR> manually copy in the resultant .bbl file
% set second argument of \begin to the number of references
% (used to reserve space for the reference number labels box)
%\begin{thebibliography}{1}

%\bibitem{IEEEhowto:kopka}
%H.~Kopka and P.~W. Daly, \emph{A Guide to \LaTeX}, 3rd~ed.\hskip 1em plus
%  0.5em minus 0.4em\relax Harlow, England: Addison-Wesley, 1999.


%\end{thebibliography}


\bibliographystyle{IEEEtran}
\bibliography{main}

% biography section
% 
% If you have an EPS/PDF photo (graphicx package needed) extra braces are
% needed around the contents of the optional argument to biography to prevent
% the LaTeX parser from getting confused when it sees the complicated
% \includegraphics command within an optional argument. (You could create
% your own custom macro containing the \includegraphics command to make things
% simpler here.)
%\begin{IEEEbiography}[{\includegraphics[width=1in,height=1.25in,clip,keepaspectratio]{mshell}}]{Michael Shell}
% or if you just want to reserve a space for a photo:


\begin{IEEEbiography}[{\includegraphics[width=1in,height=1.25in,clip,keepaspectratio]{graph//BowenPang.jpg}}]{Bowen Pang}
received his Bachelor’s degree and Ph.D. degree from the Department of Industrial Engineering at Tsinghua University, Beijing, China, in 2016 and 2022, respectively. He is currently a researcher at Noah’s Ark Lab, Huawei Technology. His research interests include modeling, analyzing, and solving problems in engineering systems, with applications in healthcare, supply chain, manufacturing, and artificial intelligence. He has been a member of IEEE, IISE, and INFORMS. Please feel free to contact his email pzkaixin@foxmail.com if you are interested in LLM inference optimization area.
\end{IEEEbiography}

\begin{IEEEbiography}[{\includegraphics[width=1in,height=1.25in,clip,keepaspectratio]{graph//KaiLi.jpg}}]{Kai Li}
received the bachelor's degree from YingCai Honors College, University of Electronic Science and Technology of China, Chengdu, China, in 2015, and the Ph.D. degree from the Department of Computer Science and Engineering, Shanghai Jiao Tong University, Shanghai, China, in 2021.
He is currently a researcher at Noah’s Ark Lab, Huawei Technology. 
His research interests include game theory, reinforcement learning, and large language models.
\end{IEEEbiography}

\begin{IEEEbiography}[{\includegraphics[width=1in,height=1.25in,clip,keepaspectratio]{graph//RuifengShe.jpg}}]{Ruifeng She}
 received his B.S. and Ph.D. degrees from the Department of Civil Engineering of the University of Illinois at Urbana-Champaign, USA, in 2018 and 2023, respectively. He is currently a researcher at Noah’s Ark Lab, Huawei Technology. His research interests include optimization of complex systems, reinforcement learning, and the integration of operations research and machine learning. 
\end{IEEEbiography}

\begin{IEEEbiography}[{\includegraphics[width=1in,height=1.25in,clip,keepaspectratio]{graph//FeifanWang.jpg}}]{Feifan Wang}
received the bachelor’s degree from the Department of Industrial Engineering, Zhejiang University of Technology, Hangzhou, China, in 2013, the master’s degree from the Department of Industrial and Systems Engineering, Zhejiang University, Hangzhou, China, in 2016, and the Ph.D. degree from the School of Computing, Informatics, and Decision Systems Engineering, Arizona State University, Tempe, AZ, USA, in 2021. He is currently an Assistant Professor with the Department of Industrial Engineering at Tsinghua University, Beijing, China. His research focuses on modeling, analysis, optimization, and control of complex systems, with applications in healthcare delivery systems and production systems. He is a member of IEEE, IISE, and INFORMS. He was a recipient of multiple awards, including the Design and Manufacturing Best Paper Award from the IISE Transactions, the Best Student Paper Award from IEEE CASE, and the Dean’s Dissertation Award from ASU. He has twice been a finalist for the Best Paper Award on Healthcare Automation from IEEE CASE.
\end{IEEEbiography}




% if you will not have a photo at all:
%\begin{IEEEbiographynophoto}{John Doe}
%Biography text here.
%\end{IEEEbiographynophoto}

% insert where needed to balance the two columns on the last page with
% biographies
%\newpage

%\begin{IEEEbiographynophoto}{Jane Doe}
%Biography text here.
%\end{IEEEbiographynophoto}

% You can push biographies down or up by placing
% a \vfill before or after them. The appropriate
% use of \vfill depends on what kind of text is
% on the last page and whether or not the columns
% are being equalized.

%\vfill

% Can be used to pull up biographies so that the bottom of the last one
% is flush with the other column.
%\enlargethispage{-5in}



% that's all folks
\end{document}


