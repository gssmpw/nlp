\appendix
\section*{Appendix}

\section{Additional Implementaion Details}
\label{appendix:implementationDetails}


% Datasets
\subsection{Datasets}

Total number of samples for each dataset is in Table \ref{table:dataset_stat}.
Each sample includes a question, original answer, conflicting answer, and four types of context: original, conflicting, random-irrelevant, and retrieved-irrelevant.


\paragraph{NaturalQuestionsShort \cite{kwiatkowski-etal-2019-natural}}
Questions consist of real queries issued to the Google search engine.
From a Wikipedia page from the top 5 search results, annotators select a long answer containing enough information to completely infer the answer to the question, and a short answer that comprises the actual answer.
The long answer becomes the context matched with the question, while the short answer being used as the answer.

\paragraph{TriviaQA-web \cite{joshi2017triviaqalargescaledistantly}}
Question-answer pairs are authored by trivia enthusiasts and independently gathered evidence documents that provide high quality supervision for answering the questions.
The web version of TriviaQA is used, where the contexts are retrieved from the results of a Bing search query.

\paragraph{HotpotQA \cite{yang2018hotpotqadatasetdiverseexplainable}}
Questions are diverse and not constrained to any pre-existing knowledge base. Multi-hop reasoning is required to solve the questions.
Paragraphs that provide supporting facts required for reasoning, are given along with the question.
In the original setting, additional distractor paragraphs are augmented in order to increase the difficulty of inference. However, these distractor paragraphs are not used in this setting.

\paragraph{SQuAD \cite{rajpurkar2016squad100000questionsmachine}}
Paragraphs from Wikipedia are presented to crowdworkers, and they are asked to write questions that entail extractive answers.
The answer to each question is a segment of text from the corresponding reading passage.
To remove the uncertainty that excessively long paragraphs bring, QA pairs that do not align with the first 800 tokens are discarded in this setting.

\paragraph{BioASQ \cite{tsatsaronis2015overview}}
BioASQ is a challenge that assesses the ability of systems to semantically index large numbers of biomedical scientific articles and return concise answers to given natural language questions.
Each question is linked to multiple related science articles. The full abstract of each linked article is used as an individual context. Abstracts that do not exactly contain the answer, are discarded.

\paragraph{TextbookQA \cite{Kembhavi_2017_CVPR}}
TextbookQA aims at answering multimodal questions when given a context in formats of text, diagrams and images.
This dataset is collected from lessons from middle school Life Science, Earth Science, and Physical Science textbooks.
Questions that are accompanied with a diagram and "True of False" questions are not used in this setting.

\paragraph{RelationExtraction \cite{levy2017zeroshotrelationextractionreading}}
Given labeled slot-filling examples, relations between entities are transformed into QA pairs using templates. Multiple templates for each type of relation are utilized.
The zero-shot benchmark split of this dataset, which showed that generalization to unseen relations is possible at lower accuracy levels, is used. 


\begin{table}[!t]
\resizebox{\linewidth}{!}{
\begin{tabular}{lc}
\toprule
& \# of Data  \\\midrule
Corpora & \textasciitilde 51,000,000 \\
Extracting Comparative Sentences & 136,300 \\
Extracting and Filtering  & 41,660 \\
\makecell[l]{Property Type Tagging (w/ Toxic Flitering)}  & 12,315 \\\bottomrule
\end{tabular}
}
\caption{The size of the dataset during the data collection pipeline is represented by the number of remaining sentences or noun phrases at each step. The term ``\# of data'' denotes the amount of data left after each processing stage.}
\label{table:construction_statistics}
\end{table}

\subsection{Templates}

% closed book template

\begin{table}[t]
    \centering    

   \begin{tabularx}{\linewidth}{X}
    \toprule
    
\ttfamily
Answer the following questions: \\\\
\ttfamily
\textcolor{brown}{<few-shots>} \\\\
\ttfamily
Question: \textcolor{brown}{<question>} \\
\ttfamily
Answer:
\\
    \bottomrule
\end{tabularx}

    \caption{Template used in closed-book generation.}
    \label{template:cls}
\end{table}

% open book template

\begin{table}[t]
    \centering    

   \begin{tabularx}{\linewidth}{X}
    \toprule
    
\ttfamily
Answer the following questions: \\\\
\ttfamily
\textcolor{brown}{<few-shots>} \\\\
\ttfamily
Context: \textcolor{brown}{<context>} \\
\ttfamily
Question: \textcolor{brown}{<question>} \\
\ttfamily
Answer:
\\
    \bottomrule
\end{tabularx}

    \caption{Template used in the \naive\ open-book generation.}
    \label{template:opn}
\end{table}

% conflicting word generation template

\begin{table}[t]
    \centering    

   \begin{tabularx}{\linewidth}{X}
    \toprule
    
\ttfamily
Answer an entity of the same type as the given keyword. Please note that the keyword is from the given context, and consider the part of speech of the keyword inside the context. You should not give a synonym or alias of the given keyword. The entity and given keyword must have different meanings. Only answer the entity itself without any extra phrases.\\\\
\ttfamily
\textcolor{brown}{<few-shots>} \\\\
% Keyword: Virginia Dare
% Context: <P> Virginia Dare ( born August 18 , 1587 , date of death unknown ) was the first English child born in a New World English overseas possession , and was named after the territory of Virginia , her birthplace . Her parents were Ananias Dare and Eleanor White ( also spelled Ellinor or Elyonor ) . </P>
% Answer: John White

% Keyword: Australia
% Context: Wikipedia [PAR] Copy and paste this wiki - markup : [PAR] { { cite web |url = https://ma.as/65166 |title=86/4138 Tooth Collection : Beer Cans ( 3 ) , 740ml , " Draught Brewed Tooth 's KB Lager " |author = Powerhouse Museum |accessdate=17 January 2017 |publisher = Powerhouse Museum , Australia } } [PAR] Copyright [PAR] Images on this site are reproduced for the purposes of research and study only . Whilst every effort has been made to trace the Copyright holders , we would be grateful for any information concerning Copyright of the images and we will withdraw them immediately on Copyright holder 's request .
% Answer: Japan
\ttfamily
Keyword: \textcolor{brown}{<original-answer>} \\
\ttfamily
Context: \textcolor{brown}{<context>} \\
\ttfamily
Answer:
\\
    \bottomrule
\end{tabularx}

    \caption{Template used when instructing the model to generate a conflicting answer, given the original answer and context.}
    \label{template:conflict}
\end{table}



Template \ref{template:cls} is used to perform closed-book generation for estimating the presence of parametric knowledge.
For \naive\ generation, Template \ref{template:opn} is used.
Template \ref{template:conflict} is employed to generate conflicting answers.

\subsection{Abstention Words}

The predefined abstain words used in evaluations are: [
\texttt{'unanswerable',
    'unknown', 'no known', 'not known', 'do not know'
    'uncertain', 'unclear',
    'no scientific evidence',
    'no definitive answer', 'no right answer', 'no concrete answer',
    'no public information',
    'debate',
    'impossible to know', 'impossible to answer',
    'difficult to predict',
    'not sure',
    'irrelevant', 'not relevant'}]


\subsection{Context Construction}
\label{appendix:datasetConstruction}

To ensure context informativeness and maintain experimental controllability, we have processed the original contexts from the MRQA benchmark by limiting their length and ensuring that the ground-truth answer span is always included.
For each occurrence span of the ground-truth answer in the raw context, we take a 100-word portion surrounding that span and consider it a candidate context.
We then compute the NLI ({\scshape BART-Large}, \citealp{lewis-etal-2020-bart}) score between the question-answer pair and each candidate context, and select the context with the highest NLI score as the original context.

To obtain retrieved-uninformative contexts, {\scshape Contriever-msmarco} \citep{izacard2022unsupervised} is utilized as a retriever model.




\subsection{Hyperparameters for Training}

We train the model for three epochs using the AdamW \cite{loshchilov2017decoupled} optimizer with a learning rate of 0.0001 and a batch size of 16. For efficient fine-tuning, we employ QLoRA \cite{dettmers2023qlora} with rank r=4 and alpha=16.
Training is conducted on a single NVIDIA RTX A6000.



\section{Additional Results}


% Please add the following required packages to your document preamble:
% \usepackage{booktabs}
% \usepackage[table,xcdraw]{xcolor}
% Beamer presentation requires \usepackage{colortbl} instead of \usepackage[table,xcdraw]{xcolor}
\begin{table*}[tp]
% \scriptsize
\centering
\resizebox{0.97\textwidth}{!}{

\begin{tabular}{@{}lrccccccc@{}}
\toprule
\multicolumn{2}{c}{Base} & TriviaQA & NQ & HotpotQA & BioASQ & Squad & TextbookQA & RE \\ \midrule
\multirow{3}{*}{Llama-2} & 7B & 0.6194 / 0.6762 & 0.4177 / 0.5665 & 0.4342 / 0.4675 & 0.5402 / 0.5776 & 0.4859 / 0.5134 & 0.5604 / 0.6318 & 0.5313 / 0.5944 \\
 & 13B & 0.6556 / 0.7067 & 0.4475 / 0.5357 & 0.4503 / 0.4528 & 0.5674 / 0.5487 & 0.5064 / 0.6083 & 0.5691 / 0.6164 & 0.5415 / 0.6471 \\
 & 70B & 0.6718 / 0.7181 & 0.5006 / 0.6008 & 0.4790 / 0.5139 & 0.6205 / 0.6515 & 0.5384 / 0.6392 & 0.5937 / 0.6162 & 0.5622 / 0.6920 \\

\cmidrule{1-9}

\multirow{2}{*}{Llama-3} & 8B & 0.6218 / 0.7100 & 0.4444 / 0.6347 & 0.4529 / 0.4936 & 0.5656 / 0.6454 & 0.4944 / 0.6063 & 0.5627 / 0.6173 & 0.5447 / 0.6613 \\
 & 70B & 0.7120 / 0.7478 & 0.4849 / 0.5824 & 0.5099 / 0.5916 & 0.6625 / 0.6887 & 0.5588 / 0.6780 & 0.6148 / 0.6770 & 0.6006 / 0.7179 \\

\cmidrule{1-9}

Mistral & 7B & 0.6270 / 0.7028 & 0.4444 / 0.5615 & 0.4586 / 0.5471 & 0.5911 / 0.6677 & 0.5111 / 0.6145 & 0.5658 / 0.6244 & 0.5386 / 0.6786 \\

\cmidrule{1-9}

\multirow{7}{*}{Qwen-2.5} & 0.5B & 0.4355 / 0.4680 & 0.3645 / 0.4557 & 0.3614 / 0.3496 & 0.4042 / 0.4778 & 0.4034 / 0.4260 & 0.3923 / 0.4390 & 0.4723 / 0.5390 \\
 & 1.5B & 0.5014 / 0.6333 & 0.4306 / 0.5683 & 0.4057 / 0.5370 & 0.4738 / 0.6465 & 0.4622 / 0.5830 & 0.5194 / 0.6174 & 0.5037 / 0.6714 \\
 & 3B & 0.5145 / 0.6608 & 0.4393 / 0.6106 & 0.4194 / 0.5903 & 0.4996 / 0.6577 & 0.4688 / 0.6460 & 0.5116 / 0.6241 & 0.5086 / 0.6825 \\
 & 7B & 0.5900 / 0.6973 & 0.4529 / 0.6230 & 0.4450 / 0.6191 & 0.5742 / 0.6860 & 0.4929 / 0.6798 & 0.5578 / 0.6659 & 0.5147 / 0.7010 \\
 & 14B & 0.6523 / 0.7240 & 0.4746 / 0.6470 & 0.4616 / 0.6293 & 0.6385 / 0.7220 & 0.5063 / 0.6888 & 0.5663 / 0.6681 & 0.5285 / 0.7035 \\
 & 32B & 0.6364 / 0.7190 & 0.4862 / 0.6484 & 0.4571 / 0.6267 & 0.6435 / 0.7102 & 0.5058 / 0.6891 & 0.5542 / 0.6876 & 0.5334 / 0.7064 \\
 & 72B & 0.6826 / 0.6984 & 0.4865 / 0.5873 & 0.4753 / 0.6188 & 0.6385 / 0.6467 & 0.5364 / 0.6940 & 0.5476 / 0.6567 & 0.5559 / 0.6836 \\ 
 \bottomrule
\end{tabular}}


\caption{Reliability scores of base LLMs across datasets. Each entry presents two inference-time approaches, separated by '/' (\textbf{\naive\ / \absinst}).}
\label{table:eval_v2_table1_base}
\end{table*}
% Please add the following required packages to your document preamble:
% \usepackage{booktabs}
% \usepackage{multirow}
\begin{table*}[tp]
% \scriptsize
\centering
\resizebox{0.97\textwidth}{!}{

\begin{tabular}{@{}lrccccccc@{}}
\toprule
\multicolumn{2}{c}{Instruct} & TriviaQA & NQ & HotpotQA & BioASQ & Squad & TextbookQA & RE \\ \midrule
\multirow{3}{*}{Llama-2} & 7B & 0.6100 / 0.6741 & 0.4133 / 0.5425 & 0.4044 / 0.5568 & 0.5018 / 0.5978 & 0.4643 / 0.6096 & 0.4986 / 0.5701 & 0.5048 / 0.6452 \\
 & 13B & 0.6171 / 0.6555 & 0.4092 / 0.5205 & 0.4007 / 0.5044 & 0.4674 / 0.5251 & 0.4555 / 0.6039 & 0.5040 / 0.5759 & 0.4593 / 0.4464 \\
 & 70B & 0.6637 / 0.7121 & 0.4555 / 0.6010 & 0.4248 / 0.5851 & 0.5203 / 0.6314 & 0.4710 / 0.6351 & 0.5312 / 0.6159 & 0.5099 / 0.6898 \\

\cmidrule{1-9}

 \multirow{2}{*}{Llama-3} & 8B & 0.5489 / 0.6869 & 0.4004 / 0.6086 & 0.3921 / 0.5733 & 0.4921 / 0.6609 & 0.4629 / 0.6744 & 0.4429 / 0.6354 & 0.4718 / 0.6779 \\
 & 70B & 0.6260 / 0.7287 & 0.4185 / 0.6319 & 0.4100 / 0.5961 & 0.4878 / 0.6563 & 0.4628 / 0.6801 & 0.4268 / 0.6283 & 0.4840 / 0.7091 \\

\cmidrule{1-9}


Mistral & 7B & 0.5940 / 0.7172 & 0.4182 / 0.6280 & 0.4216 / 0.5955 & 0.5316 / 0.6903 & 0.4722 / 0.6695 & 0.5012 / 0.6406 & 0.5135 / 0.7049 \\

\cmidrule{1-9}


 \multirow{7}{*}{Qwen-2.5} & 0.5B & 0.3797 / 0.5288 & 0.3571 / 0.4961 & 0.3379 / 0.4582 & 0.3612 / 0.5078 & 0.3973 / 0.5425 & 0.3492 / 0.4875 & 0.4686 / 0.6122 \\
 & 1.5B & 0.5170 / 0.6436 & 0.4146 / 0.5788 & 0.3933 / 0.5698 & 0.4645 / 0.6276 & 0.4595 / 0.6429 & 0.5078 / 0.6180 & 0.4783 / 0.6632 \\
 & 3B & 0.4732 / 0.6081 & 0.3830 / 0.5623 & 0.3257 / 0.4486 & 0.4674 / 0.5868 & 0.4553 / 0.6667 & 0.4590 / 0.6129 & 0.4715 / 0.6471 \\
 & 7B & 0.5427 / 0.6685 & 0.4217 / 0.6038 & 0.3684 / 0.5345 & 0.5291 / 0.6894 & 0.4682 / 0.6800 & 0.5196 / 0.6703 & 0.4856 / 0.7088 \\
 & 14B & 0.5306 / 0.5935 & 0.3987 / 0.4295 & 0.3194 / 0.4327 & 0.4867 / 0.5654 & 0.4571 / 0.5229 & 0.4839 / 0.5480 & 0.4762 / 0.6098 \\
 & 32B & 0.5754 / 0.7097 & 0.4184 / 0.5974 & 0.3453 / 0.5470 & 0.5362 / 0.7029 & 0.4646 / 0.6756 & 0.5019 / 0.6648 & 0.4899 / 0.7040 \\
 & 72B & 0.6189 / 0.7096 & 0.4407 / 0.6172 & 0.3972 / 0.5854 & 0.5423 / 0.6870 & 0.4798 / 0.6934 & 0.5234 / 0.6642 & 0.5037 / 0.7059 \\ 

\bottomrule
\end{tabular}}


\caption{Reliability scores of chat or instruct LLMs across datasets. Each entry presents two inference-time approaches, separated by '/' (\textbf{\naive\ / \absinst}).}
\label{table:eval_v2_table1_chat}


\end{table*}

Table \ref{table:eval_v2_table1_base} presents the exact reliability scores for each dataset, which are averaged in Figure \ref{figure:eval_v2_avg_base}.
Table \ref{table:eval_v2_table1_chat} reports the reliability scores of chat or instruct LLMs. They exhibit similar or lower reliability scores compared to the base models.


\begin{table}[tp]
\centering
\resizebox{\columnwidth}{!}{


\begin{tabular}{r|ccccc}
\toprule
Methods                          & Acc & Abs & C & P & O \\
\midrule
\multicolumn{6}{c}{\textbf{K}nown-\textbf{I}nfo}                        \\
\midrule
\absinst    & 45.38   & 11.65   & 2.93       & 38.95      & 1.09   \\
\baseline & 82.52   & 0.00    & 0.62       & 13.30      & 3.57   \\
\ours   & 84.87   & 0.54    & 0.86       & 11.22      & 2.52   \\
\midrule
\multicolumn{6}{c}{\textbf{U}nknown-\textbf{I}nfo}                      \\
\midrule
\absinst    & 55.50   & 20.97   & 7.90       & 4.76       & 10.87  \\
\baseline & 77.13   & 0.00    & 3.96       & 7.05       & 11.86  \\
\ours   & 77.25   & 2.89    & 5.99       & 5.16       & 8.71   \\
\midrule
\multicolumn{6}{c}{\textbf{K}nown-\textbf{U}ninfo}                      \\
\midrule
\absinst    & 67.28   & 22.17   & 8.00       & 0.23       & 2.34   \\
\baseline & 94.53   & 0.00    & 0.69       & 0.03       & 4.76   \\
\ours   & 76.70   & 18.16   & 0.98       & 0.03       & 4.13   \\

\midrule
\multicolumn{6}{c}{\textbf{U}nknown-\textbf{U}ninfo}                    \\
\midrule
\absinst    & 7.39    & 49.70   & 14.84      & 13.04      & 15.02  \\
\baseline & 15.01   & 0.01    & 2.84       & 34.96      & 47.18  \\
\ours   & 7.01    & 59.73   & 2.71       & 15.99      & 14.55 \\
\bottomrule


\end{tabular}}
\caption{The percentage of instances of accuracy (Acc), abstention (Abs), and error types (C, P, O) across four knowledge-handling scenarios using Llama 3 8B, averaged over in-domain datasets.}
\label{table:analysis_v6_irr_trained_ID}
\end{table}
\begin{table}[tp]
\centering
\resizebox{\columnwidth}{!}{

\begin{tabular}{r|ccccc}
\toprule

Methods                          & Acc   & Abs   & C     & P     & O     \\
\midrule
\multicolumn{6}{c}{\textbf{K}nown-\textbf{I}nfo}         \\
\midrule
\absinst    & 53.31 & 6.03  & 4.44  & 33.14 & 3.07  \\
\baseline & 80.17 & 0.00  & 1.35  & 15.08 & 3.39  \\
\ours   & 85.18 & 1.01  & 1.78  & 10.33 & 1.71  \\
\midrule
\multicolumn{6}{c}{\textbf{U}nknown-\textbf{I}nfo}       \\
\midrule
\absinst    & 60.61 & 10.50 & 13.51 & 2.82  & 12.55 \\
\baseline & 76.27 & 0.00  & 6.50  & 4.32  & 12.92 \\
\ours   & 76.82 & 3.12  & 6.59  & 3.78  & 9.67  \\
\midrule
\multicolumn{6}{c}{\textbf{K}nown-\textbf{U}ninfo}       \\
\midrule
\absinst    & 75.39 & 12.39 & 6.95  & 0.99  & 4.27  \\
\baseline & 86.63 & 0.00  & 0.93  & 0.00  & 12.44 \\
\ours   & 67.22 & 21.25 & 1.45  & 0.11  & 9.97  \\
\midrule
 \multicolumn{6}{c}{\textbf{U}nknown-\textbf{U}ninfo}     \\
\midrule
\absinst    & 4.85  & 36.62 & 19.26 & 15.27 & 24.00 \\
\baseline & 11.08 & 0.01  & 4.89  & 23.24 & 60.78 \\
\ours   & 5.27  & 56.04 & 4.02  & 12.07 & 22.59 \\
\bottomrule

\end{tabular}}
\caption{The percentage of instances of accuracy (Acc), abstention (Abs), and error types (C, P, O) across four knowledge-handling scenarios using Llama 3 8B, averaged over out-of-domain datasets.}
\label{table:analysis_v6_irr_trained_OOD}
\end{table}


The exact values visualized in Figure \ref{figure:eval_v6_ID_OOD} are provided in Table \ref{table:analysis_v6_irr_trained_ID} and Table \ref{table:analysis_v6_irr_trained_OOD}.

% Please add the following required packages to your document preamble:
% \usepackage{multirow}
\begin{table*}[tp]
% \scriptsize
\centering
\resizebox{0.97\textwidth}{!}{

\begin{tabular}{lrccc|ccc|ccc}

\toprule

\multirow{2}{*}{Models} & Datasets $\rightarrow$ & \multicolumn{3}{c}{RE} & \multicolumn{3}{c}{SQuAD} & \multicolumn{3}{c}{TextbookQA} \\
 & Method $\downarrow$ & Acc & Truth & Rely & Acc & Truth & Rely & Acc & Truth & Rely \\

 \midrule

 
\multirow{4}{*}{Llama-2 7B} & \naive & 0.5313 & 0.5313 & 0.5313 & 0.4856 & 0.4859 & 0.4859 & 0.5604 & 0.5604 & 0.5604 \\
 & \absinst & 0.5067 & 0.6039 & 0.5944 & 0.3748 & 0.5410 & 0.5134 & 0.4799 & 0.6667 & 0.6318 \\
 & \baseline & \textbf{0.5995} & 0.5995 & 0.5995 & \textbf{0.5151} & 0.5158 & 0.5158 & \textbf{0.6622} & 0.6622 & 0.6622 \\
 & \ours & 0.4814 & \textbf{0.9100} & \textbf{0.7263} & 0.4033 & \textbf{0.8716} & \textbf{0.6523} & 0.5395 & \textbf{0.8343} & \textbf{0.7474} \\


 \cmidrule{1-11}


 
\multirow{4}{*}{Llama-3 8B} & \naive & 0.5447 & 0.5447 & 0.5447 & 0.4943 & 0.4944 & 0.4944 & 0.5627 & 0.5627 & 0.5627 \\
 & \absinst & 0.4994 & 0.7026 & 0.6613 & 0.4209 & 0.6667 & 0.6063 & 0.4934 & 0.6383 & 0.6173 \\
 & \baseline & \textbf{0.6116} & 0.6116 & 0.6116 & \textbf{0.5412} & 0.5416 & 0.5416 & \textbf{0.6982} & 0.6982 & 0.6982 \\
 & \ours & 0.5228 & \textbf{0.8319} & \textbf{0.7364} & 0.4979 & \textbf{0.7633} & \textbf{0.6929} & 0.6255 & \textbf{0.7775} & \textbf{0.7544} \\


  \cmidrule{1-11}

  
\multirow{4}{*}{Mistral 7B} & \naive & 0.5386 & 0.5386 & 0.5386 & 0.5109 & 0.5111 & 0.5111 & 0.5658 & 0.5658 & 0.5658 \\
 & \absinst & 0.4695 & 0.7672 & 0.6786 & 0.3806 & 0.7537 & 0.6145 & 0.4761 & 0.6572 & 0.6244 \\
 & \baseline & \textbf{0.5944} & 0.5944 & 0.5944 & \textbf{0.5465} & 0.5471 & 0.5471 & \textbf{0.6937} & 0.6937 & 0.6937 \\
 & \ours & 0.4426 & \textbf{0.8528} & \textbf{0.6846} & 0.4058 & \textbf{0.8688} & \textbf{0.6544} & 0.4986 & \textbf{0.8598} & \textbf{0.7293} \\


 

  \cmidrule{1-11}



\multirow{4}{*}{Qwen 2.5 7B} & \naive & 0.5132 & 0.5147 & 0.5147 & 0.4905 & 0.4929 & 0.4929 & 0.5573 & 0.5578 & 0.5578 \\
 & \absinst & 0.4639 & \textbf{0.8504} & \textbf{0.7010} & 0.4493 & \textbf{0.8095} & \textbf{0.6798} & 0.4422 & \textbf{0.7801} & \textbf{0.6659} \\
 & \baseline & \textbf{0.5286} & 0.5286 & 0.5286 & \textbf{0.5007} & 0.5009 & 0.5009 & \textbf{0.5878} & 0.5878 & 0.5878 \\
 & \ours & 0.4653 & 0.8066 & 0.6901 & 0.4525 & 0.7472 & 0.6604 & 0.5395 & 0.6290 & 0.6210 \\

 \bottomrule

 
\end{tabular}}

\caption{Accuracy (Acc), truthfulness (Truth), and reliability (Rely) scores across three out-of-domain datasets. Bold values indicate the highest scores.}
\label{table:reli_train_ood}
\end{table*}
Table \ref{table:reli_train_ood} presents the overall performance on out-of-domain datasets, which are not included in Table \ref{table:reli_train}.