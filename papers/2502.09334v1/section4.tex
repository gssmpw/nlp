\section{\sys Design}
\label{sec:sys_design}

\begin{figure}[!t]
  \centering
  \includegraphics[width=0.8\linewidth]{img/systemoverview_revised.pdf} % Adjust the width as needed
  \vspace{-1em}
  \caption{System overview of \sys.}
  \label{fig:sysoverview}
  \vspace{-1em}
\end{figure}

% \sys's primary goal is to optimize the TTFT, TPOT and E2E goodput of LLM cloud service. 
% \sys's primary goal is to achieve high-performance LLM serving in cloud environments. In this section, we discuss the architecture, general workflow, and main components of \sys.

The architecture overview of \sys is shown in \autoref{fig:sysoverview}. There are three major components, which are the scheduler, workload profiler, and task coordinator. Based on these components, the overall routine of \sys is as follows.
\mytextcircled{1} To launch a serving process, the scheduler generates the deployment plan, which is then utilized to instantiate the model replicas over the cloud GPU resources. 
\mytextcircled{2} During the serving process, the coordinator dispatches the incoming requests across the prefill and decode replicas, and gathers the generated responses. 
\mytextcircled{3} At the same time, the workload profiler consistently monitors the workload and reports to the scheduler. 
\mytextcircled{4} Once a workload shift is detected, the scheduler triggers a lightweight re-scheduling process to adjust the deployment plan for better adaptation to the new workload. Details about each component is demonstrated in~\autoref{appendix:components}.