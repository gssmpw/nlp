%%%%%%%% mlsys 2024 EXAMPLE LATEX SUBMISSION FILE %%%%%%%%%%%%%%%%%

\documentclass{article}

% Recommended, but optional, packages for figures and better typesetting:
\usepackage{microtype}
\usepackage{graphicx}
\usepackage{subfigure}
\usepackage{booktabs} % for professional tables
\usepackage{xspace}
\usepackage{xcolor}
\usepackage{amsmath}
\usepackage{amssymb}
% \usepackage{algorithm}
% \usepackage{algpseudocode}
\usepackage{enumitem}
\usepackage{booktabs} % For better table formatting
\usepackage{multirow} 
\usepackage{soul}
\usepackage[square,numbers]{natbib}
\usepackage{authblk}
\PassOptionsToPackage{hyphens}{url}\usepackage{hyperref}

% \newcommand{\sys}{\textsc{ThunderServe}\xspace}
\newcommand{\sys}{ThunderServe\xspace}
\newcommand\semismall{\fontsize{8.4}{9}\selectfont}
\newcommand{\red}[1]{{\color{red}{#1}}}
\newcommand{\blue}[1]{{\color{blue}{#1}}}
\newcommand{\orange}[1]{{\color{orange}{#1}}}

% \newcommand{\ffc}[1]{{\color{orange}{#1}}}
% \newcommand{\jyh}[1]{{\color{blue}{#1}}}
% \newcommand{\revise}[1]{{\color{blue}{#1}}}

\newcommand{\ffc}[1]{{\color{black}{#1}}}
\newcommand{\jyh}[1]{{\color{black}{#1}}}
\newcommand{\xzyao}[1]{{\color{black}{#1}}}
\newcommand{\revise}[1]{{\color{black}{#1}}}

\DeclareMathOperator*{\argmax}{arg\,max}
\DeclareMathOperator*{\argmin}{arg\,min}

\newcommand{\jyhh}[1]{{\color{black}{#1}}}

\newcommand{\eps}{\varepsilon}

\newcommand{\specialcell}[2][c]{%
	\begin{tabular}[#1]{@{}c@{}}#2\end{tabular}}
\newcommand{\mytextcircled}[1]{\textcircled{\raisebox{-0.8pt}{#1}}}

% hyperref makes hyperlinks in the resulting PDF.
% If your build breaks (sometimes temporarily if a hyperlink spans a page)
% please comment out the following usepackage line and replace
% \usepackage{mlsys2024} with \usepackage[nohyperref]{mlsys2024} above.
\usepackage{hyperref}

% Attempt to make hyperref and algorithmic work together better:
\newcommand{\theHalgorithm}{\arabic{algorithm}}

% Use the following line for the initial blind version submitted for review:
% \usepackage{mlsys2025}

% If accepted, instead use the following line for the camera-ready submission:
\usepackage[accepted]{mlsys2025}

% The \mlsystitle you define below is probably too long as a header.
% Therefore, a short form for the running title is supplied here:
% \mlsystitlerunning{Submission and Formatting Instructions for MLSys 2024}

\begin{document}

\twocolumn[
\mlsystitle{\sys: High-performance and Cost-efficient LLM Serving in Cloud Environments}

% It is OKAY to include author information, even for blind
% submissions: the style file will automatically remove it for you
% unless you've provided the [accepted] option to the mlsys2024
% package.

% List of affiliations: The first argument should be a (short)
% identifier you will use later to specify author affiliations
% Academic affiliations should list Department, University, City, Region, Country
% Industry affiliations should list Company, City, Region, Country

% You can specify symbols, otherwise they are numbered in order.
% Ideally, you should not use this facility. Affiliations will be numbered
% in order of appearance and this is the preferred way.
\mlsyssetsymbol{equal}{*}

\begin{mlsysauthorlist}
\mlsysauthor{Youhe Jiang}{equal,cam}
\mlsysauthor{Fangcheng Fu}{equal,pku}
\mlsysauthor{Xiaozhe Yao}{equal,eth}
\mlsysauthor{Taiyi Wang}{cam}
\mlsysauthor{Bin Cui}{pku}
\mlsysauthor{Ana Klimovic}{eth}
\mlsysauthor{Eiko Yoneki}{cam}
% \mlsysauthor{Buiui Eueu}{ed}
% \mlsysauthor{Aeuia Zzzz}{ed}
% \mlsysauthor{Bieea C.~Yyyy}{to,goo}
% \mlsysauthor{Teoau Xxxx}{ed}
% \mlsysauthor{Eee Pppp}{ed}
\end{mlsysauthorlist}

\mlsysaffiliation{cam}{Department of Computer Science, University of Cambridge, Cambridgeshire, UK}
\mlsysaffiliation{pku}{Department of Computer Science, Peking University, Beijing, China}
\mlsysaffiliation{eth}{Department of Computer Science, ETH Zurich, Zurich, Switzerland}

\mlsyscorrespondingauthor{Eiko Yoneki}{eiko.yoneki@cl.cam.ac.uk}


% You may provide any keywords that you
% find helpful for describing your paper; these are used to populate
% the "keywords" metadata in the PDF but will not be shown in the document
\mlsyskeywords{Machine Learning, MLSys}

\vskip 0.3in

\begin{abstract}
Recent developments in large language models (LLMs) have demonstrated their remarkable proficiency in a range of tasks. Compared to in-house homogeneous GPU clusters, deploying LLMs in cloud environments with diverse types of GPUs is crucial for addressing the GPU shortage problem and being more cost-effective. However, the diversity of network environments and various GPU types on the cloud bring difficulties to achieving high-performance serving. In this work, we propose \sys, a high-performance and cost-efficient LLM serving system for heterogeneous cloud environments. We introduce a \textit{novel scheduling algorithm}, which optimizes the deployment plan of LLM serving to accommodate the heterogeneous resource and network bandwidth conditions in cloud environments. Furthermore, we propose a \textit{lightweight re-scheduling} mechanism, designed to adapt to fluctuating online conditions (e.g., node failures, workload shifts) without the need for costly restarts of ongoing services. Empirical results in both heterogeneous cloud and homogeneous in-house environments reveal that \sys delivers \jyhh{up to a 2.1$\times$ and on average a $1.7\times$} increase in throughput and achieves \jyhh{up to a 2.5$\times$ and on average a $1.5\times$} reduction in latency deadlines compared with state-of-the-art systems given the same price budget, suggesting opting for cloud services provides a more cost-efficient solution.
\end{abstract}
]

% this must go after the closing bracket ] following \twocolumn[ ...

% This command actually creates the footnote in the first column
% listing the affiliations and the copyright notice.
% The command takes one argument, which is text to display at the start of the footnote.
% The \mlsysEqualContribution command is standard text for equal contribution.
% Remove it (just {}) if you do not need this facility.

%\printAffiliationsAndNotice{}  % leave blank if no need to mention equal contribution
\printAffiliationsAndNotice{\mlsysEqualContribution} % otherwise use the standard text.

\section{Introduction}
% Hook(background)
Transformer-based models \cite{Vaswani2017} including BERT \cite{BERT}, RoBERTa \cite{Roberta} and GPT-3 \cite{GPT} have achieved great performance on natural language processing (NLP) tasks. However, all these models suffer from a large number of parameters, which often limits their applications due to high computational cost and memory usage. To overcome this limitation, extensive research has been conducted to reduce the model size of transformer architectures.

Recent works on compressing BERT adopt two primary strategies, pruning \cite{Han} and knowledge distillation (KD) \cite{KD}. Pruning can further be classified into two categories based on how many times pruning and recovery processes are performed: one-shot pruning \cite{SNIP} and iterative pruning \cite{Thinet,LoB}. Even though one-shot pruning is simple and computationally efficient as it conducts only one pruning phase, it tends to be less effective to maintain high accuracy. Therefore, the dominant approach for BERT is taking iterative steps of pruning and recovery while training with original dataset.

Furthermore, recent pruning methods \cite{DynaBERT,block,Xia} attempt to overcome the low pruning performance with the help of KD, which has been successful in maintaining high performance in BERT \cite{DistilB,TinyBERT}. However, the distillation process can be even more time-consuming than iterative pruning, and it is often too complicated to identify what aspects of the teacher model should be matched to the student model, particularly in the BERT architecture.

%figure 1
\begin{figure}[t!]
    \centering
    \includegraphics[width=0.82 \columnwidth]{overflow.pdf}
    \caption{Permutation and Grouping for BERT (PGB), where weight matrices are grouped and all individual weights not belonging to groups are pruned.}
    \label{fig:overview}
\end{figure}

%Motivation
In this paper, we design a novel one-shot pruning method for a task-specific BERT, aiming to achieve both high compression efficiency and high accuracy. Due to the lack of recovery chances, it is quite challenging for one-shot pruning to maintain high performance. Our proposal is to devise a semi-structured pruning method, called \textit{Permutation and Grouping for BERT} (\ours), which combines the benefits of both unstructured and structured pruning. Thus, PGB effectively reduces the model size without sacrificing so much of the accuracy as in unstructured pruning, while ensuring computational efficiency as in structured pruning.

%Methodology Description
Our PGB method is illustrated in Figure \ref{fig:overview}. Basically, we apply a group-based pruning scheme \cite{DGC,Zhao} to each structure of BERT, including multi-head attention (MHA) and feed-forward network (FFN). More specifically, PGB constructs important groups of individual weights for each layer to be preserved and prune all other weights that are not in such a group. Although training with group-based architectures from scratch has been studied with a different purpose \cite{GroupFormer,groupbert}, pruning by weight grouping has never been tackled in the context of transformer-based models. A major challenge is how to preserve the original performance of a given task-specific BERT after pruning unimportant weights. To this end, PGB performs an optimal permutation procedure so that more important weights are clustered as a structure, and adaptively determines the number of important groups for each layer of either MHA or FFN, occasionally dropping the entire layer. Finally, its re-permutation process safely rearranges each weight to its original position. 

%Experiment results
A thorough experimental study is conducted by applying our PGB method to $\text{BERT}_{\text{BASE}}$ \cite{BERT} on the GLUE \cite{GLUE} and SQuAD \cite{SQuAD} benchmarks. Our experimental results show that PGB outperforms the state-of-the-art (SOTA) pruning methods in terms of both efficiency and accuracy.

\section{Background and Related Works} \label{sec:related}
Pretrained language models \cite{BERT,Roberta,GPT} are mostly based on the transformer architecture \cite{Vaswani2017} due to their effectiveness in various NLP tasks. This section first formally describes the typical transformer architecture and then discusses representative techniques for compressing large language models, namely distillation and pruning.


\subsection{Transformer Architecture}
% MHA
The typical transformer architecture consists of encoder and decoder layers, each of which commonly contains two main components: multi-head attention (MHA) and feed-forward network (FFN). In an MHA layer, there are $N_H$ self-attention heads, and each head $h \in [1, N_H]$ involves the following weight matrices, $W_{h}^{Q}, W_{h}^{K}, W_{h}^{V}, W_{h}^{O} \in \mathbb{R}^{{d} \times\frac{d}{N_H}}$. Then, the final output of the MHA layer is computed as follows:
$$
MHA(X)= \sum_{h = 1}^{N_{H}} Attn_{h}(X),
$$
where $Attn_{h}(X)$ is the output of the standard self-attention unit.

% FFN
The output of MHA is then fed into the corresponding FFN layer, which consists of weight matrices $W^{(1)}\in \mathbb{R}^{d \times d_{ffn}}$, $W^{(2)}\in \mathbb{R}^{d_{ffn} \times d}$, $b^{(1)}\in \mathbb{R}^{d_{ffn}}$ and $b^{(2)}\in \mathbb{R}^{d}$, where $d_{ffn}$ (usually $4\times d$) represents the dimension of the
intermediate hidden features. The output of the FFN layer can be computed as follows:
$$
FFN(A)=\sum_{i=1}^{d_{ffn}}GeLU(AW^{(1)}_{:,i}+b^{(1)})W^{(2)}_{i,:}+b^{(2)},
$$
where $A = MHA(X)$.

In transformer-based models, this same structure is repeatedly defined across multiples layers (e.g., 12 layers in $\text{BERT}_{\text{BASE}}$ \cite{BERT}) and each layer has another multiple heads (e.g., 12 heads in BERT$_{\text{BASE}}$ \cite{BERT}). Consequently, they have numerous trainable parameters, which motivates the NLP community to develop various compression methods for these models.


\subsection{Distillation}
Knowledge distillation (KD) \cite{KD} is a compression technique that trains a lightweight student model to mimic the soft output of a large teacher model, leading to competitive performance on downstream tasks. There are some studies where KD methods have been applied to transformer-based models. For example, DistilBERT \cite{DistilB} transfers knowledge to a student model of a fixed-size through a pre-training phase and an optional fine-tuning process. MiniLM \cite{MiniLM} deeply describes the self-attention information of a task-agnostic teacher model by providing a detailed analysis of the self-attention matrices. Both TinyBERT \cite{TinyBERT} and MobileBERT \cite{MoB} transfer knowledge during pretraining using a layer-by-layer strategy. TinyBERT \cite{TinyBERT} additionally performs distillation during fine-tuning. 

Although KD-based methods are shown to be effective at preserving high accuracy, training a student model can be time-consuming; as reported by CoFi \cite{Xia}, TinyBERT \cite{TinyBERT} requires 350 hours and CoFi \cite{Xia} requires 20 hours for compression. Furthermore, 
it is not trivial to effectively distill the knowledge from a multi-layer teacher model with self-attention information to a student model with fewer layers, which involves the problem of layer selection and different loss functions.


\subsection{Pruning}
Pruning \cite{Han} is another popular compression scheme that removes unimportant weights or structures from the target neural network. Following a massive volume of pruning techniques on convolutional neural networks (CNNs), pruning for transformer family has also been studied, falling into either unstructured or structured pruning.

%Unstructured pruning
\paragraph{Unstructured pruning} 
In unstructured pruning, we remove individual weights that are not important, often aiming to reduce the memory storage itself for the target model while maintaining performance, such as the methods based on \textit{lottery ticket hypothesis} \cite{LoB} and \textit{movement pruning} \cite{Mov}. However, in these unstructured pruning methods, it is difficult and complicated to make satisfactory speedup at inference time, often requiring a special hardware.

%Structured pruning
\paragraph{Structured pruning} 
Structured pruning is a simpler and more efficient way to reduce the model size, where we eliminate some structures of parameters that are considered less significant. In the case of transformer architectures, we can remove redundant attention heads \cite{sixteen,voita,SMP}, entire layers of either MHA or FFN \cite{layer1,layer2}, or neurons along with all their connecting weights of FFNs \cite{earlybert}. However, such a coarse-grained pruning scheme inevitably suffers from severe drop in accuracy. Consequently, recent studies \cite{DynaBERT,block,Xia} have explored the combination of pruning with KD to further enhance the performance of compressed networks. Despite higher performance of these combined approaches, they sacrifice efficiency and simplicity in the compression process itself, as KD typically takes lengthy training times and involves complicated matching procedures.

%Semi-structured pruning
\paragraph{Semi-structured pruning} This paper proposes a semi-structured pruning method for BERT in order to achieve a good balance between efficiency and accuracy. To our best knowledge, \textit{block movement pruning} (BMP) \cite{block} is the only one that can be categorized into semi-structured pruning for transformer family, which proposes a block-based pruning approach for each weight matrix of MHA and FFN layers. 

%Group-based pruning 
\paragraph{Group-based pruning} 
Our method is inspired by a grouping strategy, which has been frequently employed to reduce the computational complexity of CNNs. In this approach, a set of filters or channels are grouped together, and the objective is to minimize connectivity and computation between these groups \cite{DGC,Zhao,deeproot,xie,shufflenet}. These grouped architectures are also incorporated into transformer-based models by a few recent works \cite{GroupFormer,groupbert}. However, they focus on designing more efficient architectures to be trained from scratch, not for compressing a trained task-specific model. This work is the first study on group-based pruning on BERT, aiming to compress the task-specific BERT while maintaining its original accuracy.



Having examined several illustrative examples that highlight the limitations of naive applications of current continual learning approaches, we now turn to a systematic analysis of three key conceptual challenges that must be addressed to move the field forward.
We structure our discussion around three fundamental aspects: the nature of continuity in learning problems, 
the choice of appropriate spaces and metrics for measuring similarity, and the role of local objectives in learning.
For each aspect, we first present key considerations that emerge from our analysis, 
followed by specific recommendations for future research directions.


\subsection{On Continuity} 
\begin{tcolorbox}[colback=orange!10,colframe=orange!50,boxsep=-1pt]
\textbf{Considerations: Continuity.}
When designing CL systems, one must examine what continuity means and how it manifests. Two fundamental forms of continuity shape the space of possible approaches: temporal continuity in how tasks evolve, and continuity in the underlying task space itself. These distinct types of continuity create different constraints on learning algorithms and require different treatment.
\end{tcolorbox}

\textbf{Cons \#1: Temporal continuity.}
We will refer to a change over time of the joint distribution of data points and prediction targets as ``drift''.
This is classically handled by assigning a potentially different distribution $\mathcal{D}_i$ to every data point $x_i$ \cite{gama2014survey},
with drift occuring when $\mathcal{D}_i \neq \mathcal{D}_j$.
We advocate here for the approach of \citet{hinder2020towards}, who propose a 
Distribution Process to capture this drift
by associating each datapoint $x_i$ with a time $t_i$, such that two datapoints sharing a time also share the same distribution.
The distributions $D_t$ are defined as Markov kernels in the time domain,
and it is now possible to postulate limiting statements similar to the batch setup or discuss concepts such as the mean distribution over a period of time.

\textbf{Cons \#2: Task continuity.}
While the comparatively simple case of continuously varying mixing coefficients of a discrete task set has been considered under the name ``task-free continual learning'' \cite{Lee2020A, jin2021gradient, shanahan2021encoders}, the possibility of a truly continuous task set has been raised \cite{van2022three}, but we are not aware of a systematic exploration of this setting.
For example, in the task-free setting one might first infer task identity and then use task-specific components \cite{heald2021contextual},
but even if task identity inference is solved,
the lack of a discrete task set in the harder case makes the use of task-specific components no longer trivial.

\begin{tcolorbox}[colback=blue!10,colframe=blue!50,boxsep=-1pt]
\textbf{Recommendations: Continuity.}
Based on our analysis of CL continuity challenges, we propose three key directions for future research:
1) formalizing temporal dynamics through drift processes rather than point-wise distributions, 2) understanding
and managing the impact of data presentation schedules, and 3) developing principled approaches for handling
continuous rather than discrete task spaces.
These recommendations aim to help researchers and practitioners better handle continuous aspects of learning while maintaining theoretical rigor and practical applicability.
\end{tcolorbox}

\textbf{Rec \#1: Use Distribution Processes to capture drift.}
We recommend working with Distribution Processes over datapoint-indexed distributions when modelling drift, as this makes the temporal structure explicit.
In particular, the extent to which temporally close distributions are expected to be similar must then be assumed explicitly rather than implicitly.
It is crucial that these assumptions are understood in continual learning, as they are the core idea underlying the notion of tasks,
and further are the reason we expect forward and backward transfer to be possible.
We believe the improvement in clarity of thinking associated with this new formalization will enhance future work.

\textbf{Rec \#2: Consider schedule dependence.}
Let us formalize a data stream as
an underlying dataset and an order in which this is presented, or \textit{schedule}.
While known in stream learning \cite{gama2014survey},
the effects of such a schedule are considered explicitly only by relatively few CL works \cite{Yoon2020Scalable, wang2022schedule},
and \citet{wang2022schedule} showed that most existing continual learning algorithms suffer drastic fluctuations in performance under different schedules.
After considering the expected temporal correlations of the data stream via drift processes, it is likely that significant permutation symmetries (\eg, discrete task orderings) will remain.
After establishing which permutations of the stream do not constitute meaningful information from which the model should learn,
future work should strive to maximize invariance of CL algorithms to such permutations.

\textbf{Rec \#3: Towards continuous task identity.}
Finally we note that, while discreteness of the underlying task set has been an important and productive underlying assumption in continual learning research, principled methods of handling task identity in the truly continuous case (\eg, section \ref{sec:boxpush} and maybe even \ref{sec:starcraft}) should be developed.
Task-specific components, for example, should still be possible where task identity is not discrete.
When representing a task as, \eg, some embedding in a continuous latent space, however, they are no longer trivial and are indeed interestingly non-trivial.
Such principled approaches should strive to account for the now much richer geometry of the task space.


\subsection{On Spaces} 
\begin{tcolorbox}[colback=orange!10,colframe=orange!50,boxsep=-1pt]
\textbf{Considerations: Spaces.}
When examining CL systems, we encounter three distinct types of continuous spaces: 
parameter space, data space, and function space. Each of these spaces requires careful consideration 
of how to measure ``similarity'' or ``distance'' - a choice that is sometimes forced by the problem 
structure. Even after selecting a space, the choice of metric remains critical, as different metrics 
can capture different aspects of the learning problem. Some scenarios may even require inherently 
asymmetric measures of similarity.
\end{tcolorbox}

\textbf{Cons \#1: The three common spaces.}
Most obviously, we have the continuous space of parameters.
Often we also have a continuous space of possible data items, \eg, arrays of floating point pixel values.
Finally, we have the continuous space of functions representable by our neural network.
If we identify a ``task'' with ``the mapping from inputs to outputs which solves the task'' then it can be seen as a special case of a function space.

When one needs to measure ``similarity'' or ``distance'' in continual learning, one will in general do so in one of these spaces.
Sometimes this is a choice, sometimes it is forced.
For example, when considering a mixture of experts solution to a variety of tasks where the architectures of the neural network models corresponding to the experts differ, it is impossible to measure distance in parameter space.
In this case we must instead consider function space.

\textbf{Cons \#2: Metrics.}
Even once the choice of space is made, ``distances'' are not determined until we choose a metric on that space.
Sometimes there will be a natural choice
(\eg, the Fisher metric in function space for classification tasks, or more generally for tasks where the output is a probability distribution).
In an application such as weight space regularization, there is a simple choice of the Euclidean metric,
but this choice is inherently incapable of identifying more or less important parameters for a given task, and may even violate safety constraints in a case like that of section \ref{sec:constraints}.
The more expressive choice of the Fisher metric as used in Natural Gradient Descent would allow such parameters to be identified.
This may allow a new task to make use of those subspaces of parameter space left unspecified by the preceding tasks.

\textbf{Cons \#3: Divergences.}
Finally, it is often the case that a notion of ``distance'' in a continual learning problem can be identified with a KL divergence, and is thus inherently asymmetrical.
For example, suppose we wish to identify new tasks by measuring the ``distance'' between a memory buffer and a sequence of new datapoints.
If the memory buffer contains datapoints from tasks A and B, but the sequence of new datapoints comes only from task B, is it a new task? Clearly not.
But if this was reversed, and the new datapoints came from A and B, while the buffer came only from B, then the memory buffer would be insufficient to determine correct behaviour on the new points from task A and the answer to the question ``is there a new task'' must be yes.
Consider the case of two 2D Gaussian distributions centered at (0,0) and (1, 0) with isotropic standard deviations 2 and 0.5, respectively.
The KL divergence in one direction is 2.8 bits, but in the other it is 20.5 bits.
Intuitively, this is because samples from the small Gaussian are in-distribution for the large Gaussian, but not vice-versa.
More concretely, in the previously considered application of the Fisher metric to parameter space regularization,
one direction corresponds to measuring distances relative to the Fisher metric measured on the new datapoints, whereas as the other corresponds to using the Fisher metric measured on the buffer.

\begin{tcolorbox}[colback=blue!10,colframe=blue!50,boxsep=-1pt]
\textbf{Recommendations: Spaces.}
Based on our analysis of the different spaces and metrics in continual learning, we propose 
several practical guidelines for developing more effective methods. These recommendations 
focus on making explicit choices about spaces and metrics, recognizing potential asymmetries 
in similarity measures, and considering alternative spaces when standard approaches fail.
\end{tcolorbox}

\textbf{Rec \#1: Choose the correct metric and space.}
Firstly, one must choose the space in which to measure this similarity or distance.
The straightforward option might be to consider raw data such as pixel values, but perhaps semantic differences would be easier to detect in some function space, such as a latent space of a neural network.
Then, having identified the correct space, one must choose a metric on that space.
Even when making ``no choice'' and using the Euclidean metric, one should be mindful of what this means.
For example, when doing weight space regularization, using a quadratic penalty in the Euclidean metric corresponds to the assumption that the appropriate posterior on weights is an isotropic Gaussian.
Making the implications of this ``non-choice'' concrete will allow the implicit assumptions to be sanity-checked.

\textbf{Rec \#2: Remember that the correct notion of similarity may not be symmetric.}
One should also pay attention to any asymmetries in the application of a notion of distance.
Often the ``distance'' measure required in an algorithm will correspond to a KL divergence.
Whether you would like your distance measure to behave like forward KL divergence or reverse KL divergence depends on the purpose of the measure: ``how informative is task A about task B'' will often have a different answer to ``how informative is task B about task A''.
Choosing the wrong direction here will likely result in severe algorithm underperformance, even though both directions agree when the tasks being compared are relatively similar.
Since asymmetries here become most salient when similarity is low, toy examples with large distances should be considered and sanity-checked by comparing both possible directions.

\textbf{Rec \#3: Consider patching broken methods by switching spaces or metrics.}
If a continual learning method fails in some particular application, it may be salvageable by altering the space in which distances are measured.
Suppose, for example that one uses functional regularization in a task where the output of the network is target robot arm pose parameterized by joint angles.
This may fail if task success is dependent on end effector pose, and the sensitivity of end effector pose to joint angle is itself highly dependent on robot pose, due to nonlinear kinematics.
In this case, re-expressing the output in terms of end effector pose via a kinematics model may resolve these difficulties.

\subsection{On Objectives}
\begin{tcolorbox}[colback=orange!10,colframe=orange!50,boxsep=-1pt]
\textbf{Considerations: Objectives.}
Current perspectives on continual learning tend to focus narrowly on accumulating knowledge through 
classification tasks. However, this view may be inherently limiting, as it emphasizes conditional 
knowledge (''which class, given these classes?") over unconditional understanding. The relationship 
between classification, density estimation, and generative modeling suggests broader ways to think 
about knowledge retention in continual learning systems.
\end{tcolorbox}

\textbf{Cons \#1: Accumulating unconditional knowledge.} \\
The knowledge involved in successful classification is inherently of a very conditional nature,
\ie, we answer the question ``given that this datapoint is drawn from the distribution of one of these $N$ classes, which class is it''.
We argue that focusing on classification objectives over density estimation or generative objectives makes continual or lifelong learning unnecessarily overcomplicated.
For example, out of distribution detection is clearly more closely related to density estimation,
and there are whole classes of replay based continual learning algorithms which are closely related to generation.
We believe that building continual learning algorithms on top of narrow classification tasks neglects the potential synergies of introducing generative or density based objectives,
as we shall now discuss.

\begin{tcolorbox}[colback=blue!10,colframe=blue!50,boxsep=-1pt]
\textbf{Recommendations: Objectives.}
Drawing from our analysis of the role of different learning objectives, we propose several 
directions for expanding beyond pure classification in continual learning. These recommendations 
emphasize the potential benefits of incorporating generative and density-based approaches, 
both for avoiding catastrophic forgetting and for more robust task identification.
\end{tcolorbox}

\textbf{Rec \#1: Consider generation for avoiding forgetting.} \\
Where the base task incorporates a generative objective, many challenges related to regularizing on or reviewing data examples from previous tasks are greatly simplified by direct exploitation of this generative function to create synthetic datapoints \cite{Robins95pseudorehearsal}.

\textbf{Rec \#2: Consider densities for task identification.} \\
In the presence of density estimation capabilities available from the base task, it is much easier to assign future datapoints to tasks and to consider questions of task boundaries, be they discrete or continuous.

\textbf{Rec \#3: Consider the energy-based model connection.} \\
Even in the case of primarily classification objectives there seems to be great potential for density estimation via connections to energy-based models \cite{grathwohl2020secretlyenergybased, li2022energybasedforcontinual}.
This could be of great use in the primary evaluation settings common within continual learning.


% \section{\sys Design}
\label{sec:sys_design}

\begin{figure}[!t]
  \centering
  \includegraphics[width=0.8\linewidth]{img/systemoverview_revised.pdf} % Adjust the width as needed
  \vspace{-1em}
  \caption{System overview of \sys.}
  \label{fig:sysoverview}
  \vspace{-1em}
\end{figure}

% \sys's primary goal is to optimize the TTFT, TPOT and E2E goodput of LLM cloud service. 
% \sys's primary goal is to achieve high-performance LLM serving in cloud environments. In this section, we discuss the architecture, general workflow, and main components of \sys.

The architecture overview of \sys is shown in \autoref{fig:sysoverview}. There are three major components, which are the scheduler, workload profiler, and task coordinator. Based on these components, the overall routine of \sys is as follows.
\mytextcircled{1} To launch a serving process, the scheduler generates the deployment plan, which is then utilized to instantiate the model replicas over the cloud GPU resources. 
\mytextcircled{2} During the serving process, the coordinator dispatches the incoming requests across the prefill and decode replicas, and gathers the generated responses. 
\mytextcircled{3} At the same time, the workload profiler consistently monitors the workload and reports to the scheduler. 
\mytextcircled{4} Once a workload shift is detected, the scheduler triggers a lightweight re-scheduling process to adjust the deployment plan for better adaptation to the new workload. Details about each component is demonstrated in~\autoref{appendix:components}.

\section{Experiments}
\subsection{Environment}
\paragraph{Datasets} 



We conduct our experiments using two benchmark datasets, $\text{SQuAD}$ \cite{SQuAD,SQuAD2} and seven tasks (QNLI, QQP, SST-2, CoLA, STS-B, MRPC, and RTE) in GLUE \cite{GLUE} on $\text{BERT}_{\text{BASE}}$ \cite{BERT}.
In our experiments, we use 2K samples from the training data for each benchmark dataset. Also, we perform re-finetuning on the pruned model using the training data of each NLP downstream task and evaluate on the corresponding dev dataset for each task. For detailed information on the benchmarks, please refer to \ref{app:gluedetail}. 

\paragraph{Compared methods}
In order to evaluate the performance of PGB, we conduct comparative experiments with the state-of-the-art structured pruning methods for BERT, including EBERT \cite{ebert}, DynaBERT \cite{DynaBERT}, and CoFi \cite{Xia}. In order to examine their pruning performance, we train each method without using distillation and data augmentation, both of which can commonly be applied to each pruning method.


\paragraph{Implementation details} 
Our PGB method is implemented using Python 3.7.15, PyTorch \cite{Pytorch} and CUDA 11.6, along with the Huggingface library \cite{Wolf}, which incorporates the latest NLP techniques. We set the hyperparmeter $N_{perm}$ that controls the number of sorting operations in the permutation step, and $G_{max}$ that indicates the maximum number of groups, ensuring that the size of the grouped model is within the given budget capacity. All the experiments are conducted on a PC with NVIDIA GeForce RTX A6000. We report that all experimental results are the average of 5 random seeds within a range between 0 and 10.

% Baseline model
As our target model, we use fine-tuned BERT \cite{BERT} for specific tasks. Even though we mainly compare the performance using $\text{BERT}_{\text{BASE}}$, we also conduct experiments using $\text{RoBERTa}_{\text{BASE}}$ and $\text{DistilBERT}_{\text{BASE}}$ \cite{DistilB} (refer to \ref{app:roberta} and \ref{app:distil}). Full experimental details can be found at \ref{app:exdetail}.

\subsection{Main Experimental Results}

\paragraph{Performance comparison}
Table \ref{tab:comACC} and Figure \ref{fig:comacc} show the performance comparison results of PGB with prior structured methods, using the seven tasks of GLUE and SQuAD. To equalize the resulting size of pruned models, we re-implement the released code of the compared methods by ourselves. For CoFi and DynaBERT, since there is no publicly available code for SQuAD, comparative experiments for these two methods were conducted exclusively on the GLUE benchmarks. Table \ref{tab:comACC} demonstrates the corresponding results with pruning rates 50\% and 88\%, where 88\% is the maximum pruning rate that can be achieved by all the compared methods. It is clearly observed that the proposed PGB method outperforms the previous structured pruning methods in all task-specific BERT models, which indicates that PGB is not only the fastest pruning method as observed in Table \ref{time}, but also highly effective to preserve the information of the original model. This empirically implies that we can properly compress transformer architectures with one-shot pruning, without relying on complicated and time-consuming methods.


Figure \ref{fig:comacc} shows how the compression performance changes when varying the target model size in terms of the number of FLOPs. We can observe that PGB generally achieves the best performance among the compared methods. More importantly, PGB tends to work better at higher compression rates, as the performance degradation of PGB gets smaller than those of the other methods as the size of compressed model decreases. These results imply that PGB is capable of maintaining high performance even at extreme compression cases. Considering the compression speed of PGB, we can claim that PGB is a computationally efficient yet accurate compression method for transformer architectures. 

\begin{table}[t!]
\begin{tabular}{clccccc}
\toprule
\multirow{2}{*}{ Pruning ratio} &\multirow{2}{*}{Method}& QNLI & QQP & SST-2 \\
            &       & Acc. & Acc. & Acc.   \\
\hline \hline
0\% & $\text{BERT}_{\text{BASE}}$ & 91.43 & 91.50 & 93.16   \\
\hline
\multirow{4}{*} {50\%} &\multirow{1}{*}{BMP} &  \multirow{1}{*}{89.40} & \multirow{1}{*}{90.30} & \multirow{1}{*}{90.71}    \\
 & \multirow{1}{*}{LayerDrop}   & \multirow{1}{*}{87.60} & \multirow{1}{*}{90.40} & \multirow{1}{*}{90.30}  \\
& \multirow{1}{*}{SNIP}         & \multirow{1}{*}{89.50} & \multirow{1}{*}{88.90} & \multirow{1}{*}{91.80} \\
& \multirow{1}{*}{$\textbf{PGB}$ (Ours)}    & \multirow{1}{*}{\textbf{90.34}} & \multirow{1}{*}{\textbf{91.06}} & \multirow{1}{*}{\textbf{92.30}} \\
\hline
\multirow{2}{*}{80\%}  &\multirow{1}{*}{BMP }   & \multirow{1}{*}{86.40} & \multirow{1}{*}{89.30} &\multirow{1}{*}{89.79}   \\
    & \multirow{1}{*}{\textbf{PGB} (Ours)}  &  \multirow{1}{*}{\textbf{87.12}} & \multirow{1}{*}{\textbf{90.40}} & \multirow{1}{*}{\textbf{89.84}}       \\
\bottomrule
\end{tabular}
\centering
\caption{Performance comparison with other SOTA pruning methods on $\text{BERT}_{\text{BASE}}$.}
\label{tab:more}
\end{table}

%----------------------------------------------------------------------------
\begin{table}
\centering
\begin{tabular}{c|cc|cc}
\toprule
\multirow{2}{*}{Task} & \multicolumn{2}{c|}{40\%}  & \multicolumn{2}{c}{60\%} \\ \cmidrule(r){2-5}
                    & Vanilla  & Re-finetuned & Vanilla & Re-finetuned  \\
\hline

 SST-2 &  92.2  &  92.7  &90.3 & 91.6 \\

QNLI & 90.1  &  91.0   & 88.2&  89.4  \\

 MRPC &  84.8    &  85.1 & 80.9 & 83.5 \\
\bottomrule
\end{tabular}
\centering
\caption{Ablation study in PGB (ours) at 40\% and 60\% of reduced FLOPs before and after re-finetuning using GLUE.}
\label{tab:vanilla}
\end{table}
%-------------------------------------------------------------------------------------------
\paragraph{Comparison with other SOTA methods}
Table \ref{tab:more} presents additional comparisons between PGB and other state-of-the-art (SOTA) pruning methods, using the reported results from their respective experiments. The compared methods include: hybrid-based block movement pruning (BMP) \cite{block}, LayerDrop that drops certain layers of the model \cite{layer2}, and SNIP that prunes redundant mappings in residual modules \cite{TFsnip}. Similar to the findings in Table \ref{tab:comACC}, PGB shows minimum performance degradation. Of particular interest is the comparison with BMP \cite{block}, as it is introduced as another semi-structured pruning method in Section \ref{sec:related}. PGB turns out to outperform BMP even though the reported results of BMP are not from its fully semi-structured pruning scheme, but rather its improved hybrid version yet without distillation.
\begin{figure}[t!]
    \centering
    {\includegraphics[width=0.7\columnwidth]{squad_acc.pdf}}
    \caption{Performance comparison with structured pruning methods varying the Reduced FLOPs ratio on $\text{SQuAD}$ benchmarks.}
    \label{fig:squadacc}
\end{figure}
\begin{table*}[t!]
\centering
\begin{tabular}{l|c|c|c|c}
%\noalign{\smallskip}\noalign{\smallskip}\hline
\toprule
\multirow{2}{*}{Method} & \multicolumn{2}{c|}{Time for Pruning + Re-finetuning}  & \multicolumn{1}{c|}{FLOPs} & \multicolumn{1}{c}{Accuracy} \\ \cmidrule(r){2-5}
 & (epochs)  &  (hours) & (G) & (\%) \\ 
\hline
EBERT \cite{ebert}      &  3 + 3       & $\leq$ 3.5 + 2        &    2.78    & 88.1 \\
DynaBERT \cite{DynaBERT}&  2 + 3        & $\leq$ 25 + 38        &    \textbf{2.60}   &   86.8  \\
CoFi \cite{Xia}     & 20 + 20           & $\leq$ 26 + 23        &    2.97    & 89.8 \\ \hline
 \textbf{PGB (ours)}& \textbf{0 + 3}  & $\leq$ \textbf{0.1 + 2} &    \textbf{2.60}  & \textbf{90.1} \\
%\hline
\bottomrule
\end{tabular}
\caption{Comparison of the pruning efficiency using QQP with 88\% pruning rate on $\text{BERT}_{\text{BASE}}$.}
\label{time}
\end{table*}
\paragraph{Experiments on SQuAD Benchmarks}\label{app:squad} 

We compare additional pruning methods with PGB for $\text{SQuAD}_{v1.1}$ \cite{SQuAD} and $\text{SQuAD}_{v2.0}$ \cite{SQuAD2}. The compared methods exclude knowledge distillation (KD) and data augmentation. Specifically, our approach is compared with PLATON \cite{platon}, representing the state-of-the-art in unstructured pruning, and EBERT \cite{ebert}. In this context, PLATON utilized the extended $\text{PLATON}_{\text{structure}}$ based on structured pruning. Our PGB method shows minimal performance degradation compared to other methods, as shown in Figure \ref{fig:squadacc}.


%Roberta result
\begin{figure*}[t!]
    \centering
    {\includegraphics[width=  \textwidth]{roberta.pdf}}
    \caption{PGB with $\text{BERT}_{\text{BASE}}$ and $\text{RoBERTa}_{\text{BASE}}$ }
    \label{fig:roberta}
\end{figure*}

\paragraph{Efficiency of pruning procedures}
To evaluate the efficiency of each pruning method, we measure how long each method takes to prune the original model and re-finetune the pruned model, as shown in Table \ref{time}. In particular, we report the case of pruning with 88\% on QQP, where the time cost was maximized. Thanks to its one-shot pruning scheme, our PGB method is clearly the fastest pruning method among all the compared methods, while maintaining the best accuracy after pruning and re-finetuning. PGB takes at most 2.1 hours to obtain the final compressed model, whereas most of the other methods takes more than a day to get the same-sized model. Table \ref{time} also reports the computational cost of each compressed model in terms of the number of FLOPs. Although PGB is a semi-structured pruning method, it manages to achieve a comparable model efficiency, without any specific hardware support, to those of the fully structured pruning methods, EBERT \cite{ebert}, DynaBERT \cite{DynaBERT} and CoFi \cite{Xia}.


\subsection{More Experimental Results}
\paragraph{Pruning on DistilBERT}\label{app:distil}
We conduct additional experiments on PGB with pruning rates of 40\% and 60\% for $\text{DistilBERT}_{\text{BASE}}$, and the results are presented in Table \ref{tab:distil}. While $\text{DistilBERT}_{\text{BASE}}$ is half the size of $\text{BERT}_{\text{BASE}}$, as indicated in Table \ref{tab:BERT}, the model configuration remains the same, excluding the number of layers. Consequently, the values of permutation ($N_{perm}$) and the maximum group number ($G_{max}$) are identical. The experimental results in Table \ref{tab:distil} demonstrate that even for a smaller model, our grouping-based pruning scheme does not result in substantial information degradation.  


\begin{table} [t!]
\begin{tabular}{cccccc}
\noalign{\smallskip}\noalign{\smallskip}\hline

Pruning Ratio & Method & QNLI &SST-2 &MRPC & STS-B   \\ 

\hline \hline
{0\%}& $\text{DistilBERT}_{\text{BASE}}$ & \multirow{1}{*}{88.6} & \multirow{1}{*}{91.3} & \multirow{1}{*}{84.8} &\multirow{1}{*}{85.8} \\

{40\%}  &\multirow{2}{*}{PGB (ours)}     &  \multirow{1}{*}{88.1}  & \multirow{1}{*}{91.2} & \multirow{1}{*}{84.6}  & \multirow{1}{*}{85.5}  \\

{60\%}  &         & \multirow{1}{*}{86.5}  & \multirow{1}{*}{89.3} &\multirow{1}{*}{83.8}& \multirow{1}{*}{84.3}  \\

\hline
\end{tabular}
\centering
\caption{Performance of PGB on $\text{DistilBERT}_{\text{BASE}}$ using 40\% and 60\% pruning rates}
\label{tab:distil}
\end{table}


\paragraph{Pruning on RoBERTa}\label{app:roberta}
Figure \ref{fig:roberta} shows the proposed PGB results with $\text{RoBERTa}_{\text{BASE}}$. Similar to BERT, PGB with pruning rates of 60\% on RoBERTa is able to  maintain over 98\% of the performance of the original model. When the Pruning rate surpasses 60\%, RoBERTa demonstrates better performance compared to BERT. However, as the sparsity ratio increases, the performance of BERT surpasses RoBERTa.


\subsection{Ablation Studies}\label{app:ablation}
\paragraph{Performance of PGB without re-finetuning}

Table \ref{tab:vanilla} presents the performance comparison between vanilla PGB, which does not perform re-finetuning, and PGB with re-finetuning. We measure the accuracy of the compressed model with 40\% and 60\%  of reduced FLOPs before and after re-finetuning on task-specific BERT models. We can observe that vanilla PGB itself is already effective to maintain high performance after pruning. For the SST-2 and QNLI tasks, the performance of vanilla PGB is either matches or surpasses that of other pruning methods shown in Figure \ref{fig:comacc}. This demonstrates the capability of our grouped pruning operation to maintain the original performance of large transformer-based models with just a single round of pruning.

\begin{figure}[t!]
    \centering
    {\includegraphics[width=0.7\columnwidth]{group_ablation.pdf}}
  
    \caption{Ablation study with respect to the number of Group ($G_{max}$) with $\tau$=1e-5}
    \label{fig:group_ablation}
\end{figure}

\begin{figure}[t!]
    \centering
    {\includegraphics[width=0.7\columnwidth]{tau_ablation.pdf}}

    \caption{Ablation study with respect to threshold ($\tau$) with $G_{max} = 6$}
    \label{fig:tau_ablation}
\end{figure}

\paragraph{Hyperparameter Sensitivity}
We conduct ablation studies to investigate the sensitivity to the hyperparameter utilized in the proposed PGB method. In Figures \ref{fig:group_ablation} and \ref{fig:tau_ablation}, we present visualization of the accuracy of the compressed model as we vary hyperparameter values, $G_{max}$ and $\tau$. 
Figure \ref{fig:group_ablation} indicates that when the number of groups is limited to a small number (i.e., ${G}_{max}=3$), there is a significant decrease in performance, which is probably because some important weights can be pruned in order to reach each target compressed size, just like in typical structured pruning. At the same time, however, it does not always mean that the larger the $G_{max}$ value, the better the final performance, as the performance also drops in the case of ${G}_{max}=8$. In terms of the threshold $\tau$, Figure \ref{fig:tau_ablation} demonstrates that there is a certain optimal point of $\tau$ to maximize the final performance. In our experiments, we found that $G_{max}=6$ and $\tau=1e-5$ produce the most promising results for compressed models.    



\section{Conclusion}
This paper introduced PGB, one-shot semi-structured pruning with a grouping strategy, as a fast and simple compression approach for transformer-based models. PGB efficiently compresses task-specific BERT models into lightweight and accurate versions within a few hours, contrasting with other SOTA methods that take more than a day to achieve comparable results. By finding an adaptively grouped architecture, PGB combines the advantages of structured pruning and unstructured pruning, offering both computational efficiency and high accuracy. Through extensive experiments, we validated that PGB is a practical solution for quickly compressing complex transformer architectures without 
 significant performance degradation.


% \section{Related Work}
% \label{sec:relatedwork}
% \noindent \textbf{LLM serving systems.} 
% Recent research on LLM serving systems has concentrated on improving hardware efficiency through careful system optimizations \cite{zhao2024llmpq,li2023alpaserve,wu2023fast,kwon2023efficient,tensorrt_llm,liu2023deja,agrawal2023sarathi}. Among them, 
% % AlpaServe \cite{li2023alpaserve} suggests using model parallelism for scaling a single LLM beyond the memory limits of a single device. 
% vLLM \cite{kwon2023efficient} proposes page attention to improve the memory efficiency of the system. 
% % Deja Vu \cite{liu2023deja} reduces LLM inference latency by predicting and leveraging contextual sparsity on-the-fly.
% SARATHI \cite{agrawal2023sarathi} suggests chunk each prefill request and piggyback decoding requests to improve hardware utilization.
% Differently, \sys focuses more on optimizing LLM servicing on cloud heterogeneous resources.


% \noindent \textbf{Phase splitting in LLM serving.} Phase splitting has recently become a hot topic in optimizing LLM serving \cite{qin2024mooncake,jin2024p}. Splitwise and DistServe \cite{patel2023splitwise, zhong2024distserve} improve LLM inference efficiency by splitting prefill and decode phases across separate GPUs, optimizing hardware utilization and resource allocation. 
% TetriInfer \cite{hu2024inference} enhances LLM inference by partitioning prompts into fixed-size chunks, disaggregating prefill and decode replicas, and using a two-level scheduling algorithm for further optimization. 
% % ExeGPT \cite{oh2024exegpt} optimizes LLM inference by proposing novel scheduling strategies based on round-robin allocation and workload-aware allocation policies, decoupling the execution of encoding and decoding for efficient optimization of each phase. 
% % However, it is challenging to transition phase splitting directly to cloud services, mainly due to the diversity of resources and varying network conditions in cloud environments. 
% Our work is inspired by the phase splitting idea, and proposes the good matching between the heterogeneity in workload characteristics of different phases and the heterogeneity in cloud resources for high-performance LLM serving. 


% % 
% % \noindent \textbf{Cost-efficiency in LLM serving.} Recent research has investigated diverse approaches to lowering the cost of LLM serving. SpotServe \cite{miao2023spotserve} reduces LLM serving costs by using preemptible GPU instances, dynamically adapting parallelization for fluctuating workloads. HexGen \cite{jiang2024hexgen} reduces inference costs by deploying generative inference in a decentralized and heterogeneous setting, with asymmetric partitioning and advanced scheduling to enhance performance. 
% % Melange \cite{griggs2024melange} reduces LLM deployment costs by selecting the most cost-efficient GPUs based on model request size, request rate, and latency SLO, treating GPU selection as a cost-aware bin-packing problem to optimize resource allocation. 
% % % \sys adopts a similar concept to utilize heterogeneous GPUs to reduce the serving cost, allowing for cost-efficient cloud LLMs services.
% % Our work has a similar objective, and is the first effort that integrates phase splitting with the heterogeneous GPUs to provide high-performance cloud serving for LLMs.

% \jyhh{

% \noindent \textbf{Heterogeneous GPU Computing.} Recent research has investigated diverse approaches to deploy large models on heterogeneous GPU clusters. 
% % Metis \cite{um2024metis} develops a new search algorithm that automatically finds efficient parallelism plans for distributed training on heterogeneous GPUs. 
% HexGen \cite{jiang2024hexgen} proposes asymmetric partitioning and advanced scheduling to deploy generative inference in a decentralized and heterogeneous setting. 
% Helix \cite{mei2024helix} formulates the heterogeneous GPUs and network connections as a maxflow problem and adopts a mixed integer linear programming algorithm to discover highly optimized strategies to serve LLMs. 
% Our work has a similar objective, but is the first effort that integrates phase splitting with the heterogeneous GPUs to provide high-performance cloud serving for LLMs.
% }
% % There are many other works developed for heterogeneity-aware training \cite{miao2023sdpipe,miao2021heterogeneity,zhang2022mics,um2024metis,acc_par,whale,hap,amp_hetero_model_parallel}, however, our work differs from them in goals.

\vspace{-1em}
\section{Conclusion}
\label{sec:con}
This paper explores the potential of deploying LLM services on clouds. Toward this end, we presented \sys, a system that employs hybrid model parallelism and phase splitting to enhance LLM serving efficiency across heterogeneous cloud GPU clusters.
With \sys, we proposed a novel scheduling algorithm that co-optimizes resource allocation, phase designation, parallelism strategies, and the orchestration of both prefill and decode phases. 
% which specifically addresses the heterogeneity in resources and network bandwidth characterized in cloud environments.
Additionally, we proposed a lightweight re-scheduling mechanism to enhance \sys performance in response to fluctuating online workloads for extremely fast adjustment on clouds. 
We conducted experiments on various workloads in both heterogeneous cloud and homogeneous in-house settings to demonstrate that \sys 
% achieve \jyhh{up to a 2.1$\times$ and on average a 1.7$\times$} increase in throughput and realize \jyhh{up to a 2.5$\times$ and on average a 1.5$\times$} reduction in latency deadlines compared to 
outperforms state-of-the-art systems within the same price budget.


% In the unusual situation where you want a paper to appear in the
% references without citing it in the main text, use \nocite
\nocite{zhang2025sageattention,zhang2024sageattention2,zhang2025spargeattn,jiang2025demystifying}

\bibliography{citation}
\bibliographystyle{mlsys2025}


%%%%%%%%%%%%%%%%%%%%%%%%%%%%%%%%%%%%%%%%%%%%%%%%%%%%%%%%%%%%%%%%%%%%%%%%%%%%%%%
%%%%%%%%%%%%%%%%%%%%%%%%%%%%%%%%%%%%%%%%%%%%%%%%%%%%%%%%%%%%%%%%%%%%%%%%%%%%%%%
% SUPPLEMENTAL CONTENT AS APPENDIX AFTER REFERENCES
%%%%%%%%%%%%%%%%%%%%%%%%%%%%%%%%%%%%%%%%%%%%%%%%%%%%%%%%%%%%%%%%%%%%%%%%%%%%%%%
%%%%%%%%%%%%%%%%%%%%%%%%%%%%%%%%%%%%%%%%%%%%%%%%%%%%%%%%%%%%%%%%%%%%%%%%%%%%%%%

\subsection{Lloyd-Max Algorithm}
\label{subsec:Lloyd-Max}
For a given quantization bitwidth $B$ and an operand $\bm{X}$, the Lloyd-Max algorithm finds $2^B$ quantization levels $\{\hat{x}_i\}_{i=1}^{2^B}$ such that quantizing $\bm{X}$ by rounding each scalar in $\bm{X}$ to the nearest quantization level minimizes the quantization MSE. 

The algorithm starts with an initial guess of quantization levels and then iteratively computes quantization thresholds $\{\tau_i\}_{i=1}^{2^B-1}$ and updates quantization levels $\{\hat{x}_i\}_{i=1}^{2^B}$. Specifically, at iteration $n$, thresholds are set to the midpoints of the previous iteration's levels:
\begin{align*}
    \tau_i^{(n)}=\frac{\hat{x}_i^{(n-1)}+\hat{x}_{i+1}^{(n-1)}}2 \text{ for } i=1\ldots 2^B-1
\end{align*}
Subsequently, the quantization levels are re-computed as conditional means of the data regions defined by the new thresholds:
\begin{align*}
    \hat{x}_i^{(n)}=\mathbb{E}\left[ \bm{X} \big| \bm{X}\in [\tau_{i-1}^{(n)},\tau_i^{(n)}] \right] \text{ for } i=1\ldots 2^B
\end{align*}
where to satisfy boundary conditions we have $\tau_0=-\infty$ and $\tau_{2^B}=\infty$. The algorithm iterates the above steps until convergence.

Figure \ref{fig:lm_quant} compares the quantization levels of a $7$-bit floating point (E3M3) quantizer (left) to a $7$-bit Lloyd-Max quantizer (right) when quantizing a layer of weights from the GPT3-126M model at a per-tensor granularity. As shown, the Lloyd-Max quantizer achieves substantially lower quantization MSE. Further, Table \ref{tab:FP7_vs_LM7} shows the superior perplexity achieved by Lloyd-Max quantizers for bitwidths of $7$, $6$ and $5$. The difference between the quantizers is clear at 5 bits, where per-tensor FP quantization incurs a drastic and unacceptable increase in perplexity, while Lloyd-Max quantization incurs a much smaller increase. Nevertheless, we note that even the optimal Lloyd-Max quantizer incurs a notable ($\sim 1.5$) increase in perplexity due to the coarse granularity of quantization. 

\begin{figure}[h]
  \centering
  \includegraphics[width=0.7\linewidth]{sections/figures/LM7_FP7.pdf}
  \caption{\small Quantization levels and the corresponding quantization MSE of Floating Point (left) vs Lloyd-Max (right) Quantizers for a layer of weights in the GPT3-126M model.}
  \label{fig:lm_quant}
\end{figure}

\begin{table}[h]\scriptsize
\begin{center}
\caption{\label{tab:FP7_vs_LM7} \small Comparing perplexity (lower is better) achieved by floating point quantizers and Lloyd-Max quantizers on a GPT3-126M model for the Wikitext-103 dataset.}
\begin{tabular}{c|cc|c}
\hline
 \multirow{2}{*}{\textbf{Bitwidth}} & \multicolumn{2}{|c|}{\textbf{Floating-Point Quantizer}} & \textbf{Lloyd-Max Quantizer} \\
 & Best Format & Wikitext-103 Perplexity & Wikitext-103 Perplexity \\
\hline
7 & E3M3 & 18.32 & 18.27 \\
6 & E3M2 & 19.07 & 18.51 \\
5 & E4M0 & 43.89 & 19.71 \\
\hline
\end{tabular}
\end{center}
\end{table}

\subsection{Proof of Local Optimality of LO-BCQ}
\label{subsec:lobcq_opt_proof}
For a given block $\bm{b}_j$, the quantization MSE during LO-BCQ can be empirically evaluated as $\frac{1}{L_b}\lVert \bm{b}_j- \bm{\hat{b}}_j\rVert^2_2$ where $\bm{\hat{b}}_j$ is computed from equation (\ref{eq:clustered_quantization_definition}) as $C_{f(\bm{b}_j)}(\bm{b}_j)$. Further, for a given block cluster $\mathcal{B}_i$, we compute the quantization MSE as $\frac{1}{|\mathcal{B}_{i}|}\sum_{\bm{b} \in \mathcal{B}_{i}} \frac{1}{L_b}\lVert \bm{b}- C_i^{(n)}(\bm{b})\rVert^2_2$. Therefore, at the end of iteration $n$, we evaluate the overall quantization MSE $J^{(n)}$ for a given operand $\bm{X}$ composed of $N_c$ block clusters as:
\begin{align*}
    \label{eq:mse_iter_n}
    J^{(n)} = \frac{1}{N_c} \sum_{i=1}^{N_c} \frac{1}{|\mathcal{B}_{i}^{(n)}|}\sum_{\bm{v} \in \mathcal{B}_{i}^{(n)}} \frac{1}{L_b}\lVert \bm{b}- B_i^{(n)}(\bm{b})\rVert^2_2
\end{align*}

At the end of iteration $n$, the codebooks are updated from $\mathcal{C}^{(n-1)}$ to $\mathcal{C}^{(n)}$. However, the mapping of a given vector $\bm{b}_j$ to quantizers $\mathcal{C}^{(n)}$ remains as  $f^{(n)}(\bm{b}_j)$. At the next iteration, during the vector clustering step, $f^{(n+1)}(\bm{b}_j)$ finds new mapping of $\bm{b}_j$ to updated codebooks $\mathcal{C}^{(n)}$ such that the quantization MSE over the candidate codebooks is minimized. Therefore, we obtain the following result for $\bm{b}_j$:
\begin{align*}
\frac{1}{L_b}\lVert \bm{b}_j - C_{f^{(n+1)}(\bm{b}_j)}^{(n)}(\bm{b}_j)\rVert^2_2 \le \frac{1}{L_b}\lVert \bm{b}_j - C_{f^{(n)}(\bm{b}_j)}^{(n)}(\bm{b}_j)\rVert^2_2
\end{align*}

That is, quantizing $\bm{b}_j$ at the end of the block clustering step of iteration $n+1$ results in lower quantization MSE compared to quantizing at the end of iteration $n$. Since this is true for all $\bm{b} \in \bm{X}$, we assert the following:
\begin{equation}
\begin{split}
\label{eq:mse_ineq_1}
    \tilde{J}^{(n+1)} &= \frac{1}{N_c} \sum_{i=1}^{N_c} \frac{1}{|\mathcal{B}_{i}^{(n+1)}|}\sum_{\bm{b} \in \mathcal{B}_{i}^{(n+1)}} \frac{1}{L_b}\lVert \bm{b} - C_i^{(n)}(b)\rVert^2_2 \le J^{(n)}
\end{split}
\end{equation}
where $\tilde{J}^{(n+1)}$ is the the quantization MSE after the vector clustering step at iteration $n+1$.

Next, during the codebook update step (\ref{eq:quantizers_update}) at iteration $n+1$, the per-cluster codebooks $\mathcal{C}^{(n)}$ are updated to $\mathcal{C}^{(n+1)}$ by invoking the Lloyd-Max algorithm \citep{Lloyd}. We know that for any given value distribution, the Lloyd-Max algorithm minimizes the quantization MSE. Therefore, for a given vector cluster $\mathcal{B}_i$ we obtain the following result:

\begin{equation}
    \frac{1}{|\mathcal{B}_{i}^{(n+1)}|}\sum_{\bm{b} \in \mathcal{B}_{i}^{(n+1)}} \frac{1}{L_b}\lVert \bm{b}- C_i^{(n+1)}(\bm{b})\rVert^2_2 \le \frac{1}{|\mathcal{B}_{i}^{(n+1)}|}\sum_{\bm{b} \in \mathcal{B}_{i}^{(n+1)}} \frac{1}{L_b}\lVert \bm{b}- C_i^{(n)}(\bm{b})\rVert^2_2
\end{equation}

The above equation states that quantizing the given block cluster $\mathcal{B}_i$ after updating the associated codebook from $C_i^{(n)}$ to $C_i^{(n+1)}$ results in lower quantization MSE. Since this is true for all the block clusters, we derive the following result: 
\begin{equation}
\begin{split}
\label{eq:mse_ineq_2}
     J^{(n+1)} &= \frac{1}{N_c} \sum_{i=1}^{N_c} \frac{1}{|\mathcal{B}_{i}^{(n+1)}|}\sum_{\bm{b} \in \mathcal{B}_{i}^{(n+1)}} \frac{1}{L_b}\lVert \bm{b}- C_i^{(n+1)}(\bm{b})\rVert^2_2  \le \tilde{J}^{(n+1)}   
\end{split}
\end{equation}

Following (\ref{eq:mse_ineq_1}) and (\ref{eq:mse_ineq_2}), we find that the quantization MSE is non-increasing for each iteration, that is, $J^{(1)} \ge J^{(2)} \ge J^{(3)} \ge \ldots \ge J^{(M)}$ where $M$ is the maximum number of iterations. 
%Therefore, we can say that if the algorithm converges, then it must be that it has converged to a local minimum. 
\hfill $\blacksquare$


\begin{figure}
    \begin{center}
    \includegraphics[width=0.5\textwidth]{sections//figures/mse_vs_iter.pdf}
    \end{center}
    \caption{\small NMSE vs iterations during LO-BCQ compared to other block quantization proposals}
    \label{fig:nmse_vs_iter}
\end{figure}

Figure \ref{fig:nmse_vs_iter} shows the empirical convergence of LO-BCQ across several block lengths and number of codebooks. Also, the MSE achieved by LO-BCQ is compared to baselines such as MXFP and VSQ. As shown, LO-BCQ converges to a lower MSE than the baselines. Further, we achieve better convergence for larger number of codebooks ($N_c$) and for a smaller block length ($L_b$), both of which increase the bitwidth of BCQ (see Eq \ref{eq:bitwidth_bcq}).


\subsection{Additional Accuracy Results}
%Table \ref{tab:lobcq_config} lists the various LOBCQ configurations and their corresponding bitwidths.
\begin{table}
\setlength{\tabcolsep}{4.75pt}
\begin{center}
\caption{\label{tab:lobcq_config} Various LO-BCQ configurations and their bitwidths.}
\begin{tabular}{|c||c|c|c|c||c|c||c|} 
\hline
 & \multicolumn{4}{|c||}{$L_b=8$} & \multicolumn{2}{|c||}{$L_b=4$} & $L_b=2$ \\
 \hline
 \backslashbox{$L_A$\kern-1em}{\kern-1em$N_c$} & 2 & 4 & 8 & 16 & 2 & 4 & 2 \\
 \hline
 64 & 4.25 & 4.375 & 4.5 & 4.625 & 4.375 & 4.625 & 4.625\\
 \hline
 32 & 4.375 & 4.5 & 4.625& 4.75 & 4.5 & 4.75 & 4.75 \\
 \hline
 16 & 4.625 & 4.75& 4.875 & 5 & 4.75 & 5 & 5 \\
 \hline
\end{tabular}
\end{center}
\end{table}

%\subsection{Perplexity achieved by various LO-BCQ configurations on Wikitext-103 dataset}

\begin{table} \centering
\begin{tabular}{|c||c|c|c|c||c|c||c|} 
\hline
 $L_b \rightarrow$& \multicolumn{4}{c||}{8} & \multicolumn{2}{c||}{4} & 2\\
 \hline
 \backslashbox{$L_A$\kern-1em}{\kern-1em$N_c$} & 2 & 4 & 8 & 16 & 2 & 4 & 2  \\
 %$N_c \rightarrow$ & 2 & 4 & 8 & 16 & 2 & 4 & 2 \\
 \hline
 \hline
 \multicolumn{8}{c}{GPT3-1.3B (FP32 PPL = 9.98)} \\ 
 \hline
 \hline
 64 & 10.40 & 10.23 & 10.17 & 10.15 &  10.28 & 10.18 & 10.19 \\
 \hline
 32 & 10.25 & 10.20 & 10.15 & 10.12 &  10.23 & 10.17 & 10.17 \\
 \hline
 16 & 10.22 & 10.16 & 10.10 & 10.09 &  10.21 & 10.14 & 10.16 \\
 \hline
  \hline
 \multicolumn{8}{c}{GPT3-8B (FP32 PPL = 7.38)} \\ 
 \hline
 \hline
 64 & 7.61 & 7.52 & 7.48 &  7.47 &  7.55 &  7.49 & 7.50 \\
 \hline
 32 & 7.52 & 7.50 & 7.46 &  7.45 &  7.52 &  7.48 & 7.48  \\
 \hline
 16 & 7.51 & 7.48 & 7.44 &  7.44 &  7.51 &  7.49 & 7.47  \\
 \hline
\end{tabular}
\caption{\label{tab:ppl_gpt3_abalation} Wikitext-103 perplexity across GPT3-1.3B and 8B models.}
\end{table}

\begin{table} \centering
\begin{tabular}{|c||c|c|c|c||} 
\hline
 $L_b \rightarrow$& \multicolumn{4}{c||}{8}\\
 \hline
 \backslashbox{$L_A$\kern-1em}{\kern-1em$N_c$} & 2 & 4 & 8 & 16 \\
 %$N_c \rightarrow$ & 2 & 4 & 8 & 16 & 2 & 4 & 2 \\
 \hline
 \hline
 \multicolumn{5}{|c|}{Llama2-7B (FP32 PPL = 5.06)} \\ 
 \hline
 \hline
 64 & 5.31 & 5.26 & 5.19 & 5.18  \\
 \hline
 32 & 5.23 & 5.25 & 5.18 & 5.15  \\
 \hline
 16 & 5.23 & 5.19 & 5.16 & 5.14  \\
 \hline
 \multicolumn{5}{|c|}{Nemotron4-15B (FP32 PPL = 5.87)} \\ 
 \hline
 \hline
 64  & 6.3 & 6.20 & 6.13 & 6.08  \\
 \hline
 32  & 6.24 & 6.12 & 6.07 & 6.03  \\
 \hline
 16  & 6.12 & 6.14 & 6.04 & 6.02  \\
 \hline
 \multicolumn{5}{|c|}{Nemotron4-340B (FP32 PPL = 3.48)} \\ 
 \hline
 \hline
 64 & 3.67 & 3.62 & 3.60 & 3.59 \\
 \hline
 32 & 3.63 & 3.61 & 3.59 & 3.56 \\
 \hline
 16 & 3.61 & 3.58 & 3.57 & 3.55 \\
 \hline
\end{tabular}
\caption{\label{tab:ppl_llama7B_nemo15B} Wikitext-103 perplexity compared to FP32 baseline in Llama2-7B and Nemotron4-15B, 340B models}
\end{table}

%\subsection{Perplexity achieved by various LO-BCQ configurations on MMLU dataset}


\begin{table} \centering
\begin{tabular}{|c||c|c|c|c||c|c|c|c|} 
\hline
 $L_b \rightarrow$& \multicolumn{4}{c||}{8} & \multicolumn{4}{c||}{8}\\
 \hline
 \backslashbox{$L_A$\kern-1em}{\kern-1em$N_c$} & 2 & 4 & 8 & 16 & 2 & 4 & 8 & 16  \\
 %$N_c \rightarrow$ & 2 & 4 & 8 & 16 & 2 & 4 & 2 \\
 \hline
 \hline
 \multicolumn{5}{|c|}{Llama2-7B (FP32 Accuracy = 45.8\%)} & \multicolumn{4}{|c|}{Llama2-70B (FP32 Accuracy = 69.12\%)} \\ 
 \hline
 \hline
 64 & 43.9 & 43.4 & 43.9 & 44.9 & 68.07 & 68.27 & 68.17 & 68.75 \\
 \hline
 32 & 44.5 & 43.8 & 44.9 & 44.5 & 68.37 & 68.51 & 68.35 & 68.27  \\
 \hline
 16 & 43.9 & 42.7 & 44.9 & 45 & 68.12 & 68.77 & 68.31 & 68.59  \\
 \hline
 \hline
 \multicolumn{5}{|c|}{GPT3-22B (FP32 Accuracy = 38.75\%)} & \multicolumn{4}{|c|}{Nemotron4-15B (FP32 Accuracy = 64.3\%)} \\ 
 \hline
 \hline
 64 & 36.71 & 38.85 & 38.13 & 38.92 & 63.17 & 62.36 & 63.72 & 64.09 \\
 \hline
 32 & 37.95 & 38.69 & 39.45 & 38.34 & 64.05 & 62.30 & 63.8 & 64.33  \\
 \hline
 16 & 38.88 & 38.80 & 38.31 & 38.92 & 63.22 & 63.51 & 63.93 & 64.43  \\
 \hline
\end{tabular}
\caption{\label{tab:mmlu_abalation} Accuracy on MMLU dataset across GPT3-22B, Llama2-7B, 70B and Nemotron4-15B models.}
\end{table}


%\subsection{Perplexity achieved by various LO-BCQ configurations on LM evaluation harness}

\begin{table} \centering
\begin{tabular}{|c||c|c|c|c||c|c|c|c|} 
\hline
 $L_b \rightarrow$& \multicolumn{4}{c||}{8} & \multicolumn{4}{c||}{8}\\
 \hline
 \backslashbox{$L_A$\kern-1em}{\kern-1em$N_c$} & 2 & 4 & 8 & 16 & 2 & 4 & 8 & 16  \\
 %$N_c \rightarrow$ & 2 & 4 & 8 & 16 & 2 & 4 & 2 \\
 \hline
 \hline
 \multicolumn{5}{|c|}{Race (FP32 Accuracy = 37.51\%)} & \multicolumn{4}{|c|}{Boolq (FP32 Accuracy = 64.62\%)} \\ 
 \hline
 \hline
 64 & 36.94 & 37.13 & 36.27 & 37.13 & 63.73 & 62.26 & 63.49 & 63.36 \\
 \hline
 32 & 37.03 & 36.36 & 36.08 & 37.03 & 62.54 & 63.51 & 63.49 & 63.55  \\
 \hline
 16 & 37.03 & 37.03 & 36.46 & 37.03 & 61.1 & 63.79 & 63.58 & 63.33  \\
 \hline
 \hline
 \multicolumn{5}{|c|}{Winogrande (FP32 Accuracy = 58.01\%)} & \multicolumn{4}{|c|}{Piqa (FP32 Accuracy = 74.21\%)} \\ 
 \hline
 \hline
 64 & 58.17 & 57.22 & 57.85 & 58.33 & 73.01 & 73.07 & 73.07 & 72.80 \\
 \hline
 32 & 59.12 & 58.09 & 57.85 & 58.41 & 73.01 & 73.94 & 72.74 & 73.18  \\
 \hline
 16 & 57.93 & 58.88 & 57.93 & 58.56 & 73.94 & 72.80 & 73.01 & 73.94  \\
 \hline
\end{tabular}
\caption{\label{tab:mmlu_abalation} Accuracy on LM evaluation harness tasks on GPT3-1.3B model.}
\end{table}

\begin{table} \centering
\begin{tabular}{|c||c|c|c|c||c|c|c|c|} 
\hline
 $L_b \rightarrow$& \multicolumn{4}{c||}{8} & \multicolumn{4}{c||}{8}\\
 \hline
 \backslashbox{$L_A$\kern-1em}{\kern-1em$N_c$} & 2 & 4 & 8 & 16 & 2 & 4 & 8 & 16  \\
 %$N_c \rightarrow$ & 2 & 4 & 8 & 16 & 2 & 4 & 2 \\
 \hline
 \hline
 \multicolumn{5}{|c|}{Race (FP32 Accuracy = 41.34\%)} & \multicolumn{4}{|c|}{Boolq (FP32 Accuracy = 68.32\%)} \\ 
 \hline
 \hline
 64 & 40.48 & 40.10 & 39.43 & 39.90 & 69.20 & 68.41 & 69.45 & 68.56 \\
 \hline
 32 & 39.52 & 39.52 & 40.77 & 39.62 & 68.32 & 67.43 & 68.17 & 69.30  \\
 \hline
 16 & 39.81 & 39.71 & 39.90 & 40.38 & 68.10 & 66.33 & 69.51 & 69.42  \\
 \hline
 \hline
 \multicolumn{5}{|c|}{Winogrande (FP32 Accuracy = 67.88\%)} & \multicolumn{4}{|c|}{Piqa (FP32 Accuracy = 78.78\%)} \\ 
 \hline
 \hline
 64 & 66.85 & 66.61 & 67.72 & 67.88 & 77.31 & 77.42 & 77.75 & 77.64 \\
 \hline
 32 & 67.25 & 67.72 & 67.72 & 67.00 & 77.31 & 77.04 & 77.80 & 77.37  \\
 \hline
 16 & 68.11 & 68.90 & 67.88 & 67.48 & 77.37 & 78.13 & 78.13 & 77.69  \\
 \hline
\end{tabular}
\caption{\label{tab:mmlu_abalation} Accuracy on LM evaluation harness tasks on GPT3-8B model.}
\end{table}

\begin{table} \centering
\begin{tabular}{|c||c|c|c|c||c|c|c|c|} 
\hline
 $L_b \rightarrow$& \multicolumn{4}{c||}{8} & \multicolumn{4}{c||}{8}\\
 \hline
 \backslashbox{$L_A$\kern-1em}{\kern-1em$N_c$} & 2 & 4 & 8 & 16 & 2 & 4 & 8 & 16  \\
 %$N_c \rightarrow$ & 2 & 4 & 8 & 16 & 2 & 4 & 2 \\
 \hline
 \hline
 \multicolumn{5}{|c|}{Race (FP32 Accuracy = 40.67\%)} & \multicolumn{4}{|c|}{Boolq (FP32 Accuracy = 76.54\%)} \\ 
 \hline
 \hline
 64 & 40.48 & 40.10 & 39.43 & 39.90 & 75.41 & 75.11 & 77.09 & 75.66 \\
 \hline
 32 & 39.52 & 39.52 & 40.77 & 39.62 & 76.02 & 76.02 & 75.96 & 75.35  \\
 \hline
 16 & 39.81 & 39.71 & 39.90 & 40.38 & 75.05 & 73.82 & 75.72 & 76.09  \\
 \hline
 \hline
 \multicolumn{5}{|c|}{Winogrande (FP32 Accuracy = 70.64\%)} & \multicolumn{4}{|c|}{Piqa (FP32 Accuracy = 79.16\%)} \\ 
 \hline
 \hline
 64 & 69.14 & 70.17 & 70.17 & 70.56 & 78.24 & 79.00 & 78.62 & 78.73 \\
 \hline
 32 & 70.96 & 69.69 & 71.27 & 69.30 & 78.56 & 79.49 & 79.16 & 78.89  \\
 \hline
 16 & 71.03 & 69.53 & 69.69 & 70.40 & 78.13 & 79.16 & 79.00 & 79.00  \\
 \hline
\end{tabular}
\caption{\label{tab:mmlu_abalation} Accuracy on LM evaluation harness tasks on GPT3-22B model.}
\end{table}

\begin{table} \centering
\begin{tabular}{|c||c|c|c|c||c|c|c|c|} 
\hline
 $L_b \rightarrow$& \multicolumn{4}{c||}{8} & \multicolumn{4}{c||}{8}\\
 \hline
 \backslashbox{$L_A$\kern-1em}{\kern-1em$N_c$} & 2 & 4 & 8 & 16 & 2 & 4 & 8 & 16  \\
 %$N_c \rightarrow$ & 2 & 4 & 8 & 16 & 2 & 4 & 2 \\
 \hline
 \hline
 \multicolumn{5}{|c|}{Race (FP32 Accuracy = 44.4\%)} & \multicolumn{4}{|c|}{Boolq (FP32 Accuracy = 79.29\%)} \\ 
 \hline
 \hline
 64 & 42.49 & 42.51 & 42.58 & 43.45 & 77.58 & 77.37 & 77.43 & 78.1 \\
 \hline
 32 & 43.35 & 42.49 & 43.64 & 43.73 & 77.86 & 75.32 & 77.28 & 77.86  \\
 \hline
 16 & 44.21 & 44.21 & 43.64 & 42.97 & 78.65 & 77 & 76.94 & 77.98  \\
 \hline
 \hline
 \multicolumn{5}{|c|}{Winogrande (FP32 Accuracy = 69.38\%)} & \multicolumn{4}{|c|}{Piqa (FP32 Accuracy = 78.07\%)} \\ 
 \hline
 \hline
 64 & 68.9 & 68.43 & 69.77 & 68.19 & 77.09 & 76.82 & 77.09 & 77.86 \\
 \hline
 32 & 69.38 & 68.51 & 68.82 & 68.90 & 78.07 & 76.71 & 78.07 & 77.86  \\
 \hline
 16 & 69.53 & 67.09 & 69.38 & 68.90 & 77.37 & 77.8 & 77.91 & 77.69  \\
 \hline
\end{tabular}
\caption{\label{tab:mmlu_abalation} Accuracy on LM evaluation harness tasks on Llama2-7B model.}
\end{table}

\begin{table} \centering
\begin{tabular}{|c||c|c|c|c||c|c|c|c|} 
\hline
 $L_b \rightarrow$& \multicolumn{4}{c||}{8} & \multicolumn{4}{c||}{8}\\
 \hline
 \backslashbox{$L_A$\kern-1em}{\kern-1em$N_c$} & 2 & 4 & 8 & 16 & 2 & 4 & 8 & 16  \\
 %$N_c \rightarrow$ & 2 & 4 & 8 & 16 & 2 & 4 & 2 \\
 \hline
 \hline
 \multicolumn{5}{|c|}{Race (FP32 Accuracy = 48.8\%)} & \multicolumn{4}{|c|}{Boolq (FP32 Accuracy = 85.23\%)} \\ 
 \hline
 \hline
 64 & 49.00 & 49.00 & 49.28 & 48.71 & 82.82 & 84.28 & 84.03 & 84.25 \\
 \hline
 32 & 49.57 & 48.52 & 48.33 & 49.28 & 83.85 & 84.46 & 84.31 & 84.93  \\
 \hline
 16 & 49.85 & 49.09 & 49.28 & 48.99 & 85.11 & 84.46 & 84.61 & 83.94  \\
 \hline
 \hline
 \multicolumn{5}{|c|}{Winogrande (FP32 Accuracy = 79.95\%)} & \multicolumn{4}{|c|}{Piqa (FP32 Accuracy = 81.56\%)} \\ 
 \hline
 \hline
 64 & 78.77 & 78.45 & 78.37 & 79.16 & 81.45 & 80.69 & 81.45 & 81.5 \\
 \hline
 32 & 78.45 & 79.01 & 78.69 & 80.66 & 81.56 & 80.58 & 81.18 & 81.34  \\
 \hline
 16 & 79.95 & 79.56 & 79.79 & 79.72 & 81.28 & 81.66 & 81.28 & 80.96  \\
 \hline
\end{tabular}
\caption{\label{tab:mmlu_abalation} Accuracy on LM evaluation harness tasks on Llama2-70B model.}
\end{table}

%\section{MSE Studies}
%\textcolor{red}{TODO}


\subsection{Number Formats and Quantization Method}
\label{subsec:numFormats_quantMethod}
\subsubsection{Integer Format}
An $n$-bit signed integer (INT) is typically represented with a 2s-complement format \citep{yao2022zeroquant,xiao2023smoothquant,dai2021vsq}, where the most significant bit denotes the sign.

\subsubsection{Floating Point Format}
An $n$-bit signed floating point (FP) number $x$ comprises of a 1-bit sign ($x_{\mathrm{sign}}$), $B_m$-bit mantissa ($x_{\mathrm{mant}}$) and $B_e$-bit exponent ($x_{\mathrm{exp}}$) such that $B_m+B_e=n-1$. The associated constant exponent bias ($E_{\mathrm{bias}}$) is computed as $(2^{{B_e}-1}-1)$. We denote this format as $E_{B_e}M_{B_m}$.  

\subsubsection{Quantization Scheme}
\label{subsec:quant_method}
A quantization scheme dictates how a given unquantized tensor is converted to its quantized representation. We consider FP formats for the purpose of illustration. Given an unquantized tensor $\bm{X}$ and an FP format $E_{B_e}M_{B_m}$, we first, we compute the quantization scale factor $s_X$ that maps the maximum absolute value of $\bm{X}$ to the maximum quantization level of the $E_{B_e}M_{B_m}$ format as follows:
\begin{align}
\label{eq:sf}
    s_X = \frac{\mathrm{max}(|\bm{X}|)}{\mathrm{max}(E_{B_e}M_{B_m})}
\end{align}
In the above equation, $|\cdot|$ denotes the absolute value function.

Next, we scale $\bm{X}$ by $s_X$ and quantize it to $\hat{\bm{X}}$ by rounding it to the nearest quantization level of $E_{B_e}M_{B_m}$ as:

\begin{align}
\label{eq:tensor_quant}
    \hat{\bm{X}} = \text{round-to-nearest}\left(\frac{\bm{X}}{s_X}, E_{B_e}M_{B_m}\right)
\end{align}

We perform dynamic max-scaled quantization \citep{wu2020integer}, where the scale factor $s$ for activations is dynamically computed during runtime.

\subsection{Vector Scaled Quantization}
\begin{wrapfigure}{r}{0.35\linewidth}
  \centering
  \includegraphics[width=\linewidth]{sections/figures/vsquant.jpg}
  \caption{\small Vectorwise decomposition for per-vector scaled quantization (VSQ \citep{dai2021vsq}).}
  \label{fig:vsquant}
\end{wrapfigure}
During VSQ \citep{dai2021vsq}, the operand tensors are decomposed into 1D vectors in a hardware friendly manner as shown in Figure \ref{fig:vsquant}. Since the decomposed tensors are used as operands in matrix multiplications during inference, it is beneficial to perform this decomposition along the reduction dimension of the multiplication. The vectorwise quantization is performed similar to tensorwise quantization described in Equations \ref{eq:sf} and \ref{eq:tensor_quant}, where a scale factor $s_v$ is required for each vector $\bm{v}$ that maps the maximum absolute value of that vector to the maximum quantization level. While smaller vector lengths can lead to larger accuracy gains, the associated memory and computational overheads due to the per-vector scale factors increases. To alleviate these overheads, VSQ \citep{dai2021vsq} proposed a second level quantization of the per-vector scale factors to unsigned integers, while MX \citep{rouhani2023shared} quantizes them to integer powers of 2 (denoted as $2^{INT}$).

\subsubsection{MX Format}
The MX format proposed in \citep{rouhani2023microscaling} introduces the concept of sub-block shifting. For every two scalar elements of $b$-bits each, there is a shared exponent bit. The value of this exponent bit is determined through an empirical analysis that targets minimizing quantization MSE. We note that the FP format $E_{1}M_{b}$ is strictly better than MX from an accuracy perspective since it allocates a dedicated exponent bit to each scalar as opposed to sharing it across two scalars. Therefore, we conservatively bound the accuracy of a $b+2$-bit signed MX format with that of a $E_{1}M_{b}$ format in our comparisons. For instance, we use E1M2 format as a proxy for MX4.

\begin{figure}
    \centering
    \includegraphics[width=1\linewidth]{sections//figures/BlockFormats.pdf}
    \caption{\small Comparing LO-BCQ to MX format.}
    \label{fig:block_formats}
\end{figure}

Figure \ref{fig:block_formats} compares our $4$-bit LO-BCQ block format to MX \citep{rouhani2023microscaling}. As shown, both LO-BCQ and MX decompose a given operand tensor into block arrays and each block array into blocks. Similar to MX, we find that per-block quantization ($L_b < L_A$) leads to better accuracy due to increased flexibility. While MX achieves this through per-block $1$-bit micro-scales, we associate a dedicated codebook to each block through a per-block codebook selector. Further, MX quantizes the per-block array scale-factor to E8M0 format without per-tensor scaling. In contrast during LO-BCQ, we find that per-tensor scaling combined with quantization of per-block array scale-factor to E4M3 format results in superior inference accuracy across models. 


%%%%%%%%%%%%%%%%%%%%%%%%%%%%%%%%%%%%%%%%%%%%%%%%%%%%%%%%%%%%%%%%%%%%%%%%%%%%%%%
%%%%%%%%%%%%%%%%%%%%%%%%%%%%%%%%%%%%%%%%%%%%%%%%%%%%%%%%%%%%%%%%%%%%%%%%%%%%%%%


\end{document}


% This document was modified from the file originally made available by
% Pat Langley and Andrea Danyluk for ICML-2K. This version was created
% by Iain Murray in 2018. It was modified from a version from Dan Roy in
% 2017, which was based on a version from Lise Getoor and Tobias
% Scheffer, which was slightly modified from the 2010 version by
% Thorsten Joachims & Johannes Fuernkranz, slightly modified from the
% 2009 version by Kiri Wagstaff and Sam Roweis's 2008 version, which is
% slightly modified from Prasad Tadepalli's 2007 version which is a
% lightly changed version of the previous year's version by Andrew
% Moore, which was in turn edited from those of Kristian Kersting and
% Codrina Lauth. Alex Smola contributed to the algorithmic style files.
