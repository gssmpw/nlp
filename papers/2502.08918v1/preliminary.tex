\section{Preliminaries}

\begin{definition}{\textbf{Heterogeneous Graph.}}
    A heterogeneous graph is denoted as $\mathcal{G} = \{\mathcal{V},\mathcal{E}, \mathcal{T}, \mathcal{R}\}$, where $\mathcal{V}$ is the node set with a node type mapping function $\phi: \mathcal{V} \rightarrow \mathcal{T}$, and $\mathcal{E}$ is the edge set with an edge type mapping function $\varphi: \mathcal{E} \rightarrow \mathcal{R}$, respectively. Each node is associated with a node type $\phi(v) \in \mathcal{T}$ and each edge with a relation $\varphi(e) = \varphi(v_i,v_j) \in \mathcal{R}$. A heterogeneous graph can be represented by a set of adjacency matrices $\{\bm A_k\}_{k=i}^K$, where $K=|\mathcal{R}|$ is the number of edge types. $\bm A_k \in \mathbb{R}^{N \times N}$ denotes the adjacency matrix of the $k$-th type edge, where $N$ is the number of nodes in $\mathcal{V}$.
\end{definition}

\begin{definition}{\textbf{Meta-path.}}
    A meta-path $\mathcal{P}$ describes a pattern of connections in the heterogeneous graph, denoted as $T_1 \stackrel{R_1}{\longrightarrow} T_2 \stackrel{R_2}{\longrightarrow} \cdots \stackrel{R_l}{\longrightarrow} T_{l+1}$ (simplified to $T_1 T_2 \cdots T_{l+1}$), where $T_1,T_2,\cdots,T_{l+1} \in \mathcal{T}$ are node types, and $R_1,R_2,\cdots,R_l \in \mathcal{R}$ are relations.
\end{definition}

