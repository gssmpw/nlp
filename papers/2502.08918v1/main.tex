% This is samplepaper.tex, a sample chapter demonstrating the
% LLNCS macro package for Springer Computer Science proceedings;
% Version 2.21 of 2022/01/12
%
\documentclass[runningheads]{llncs}
%
\usepackage[T1]{fontenc}
% T1 fonts will be used to generate the final print and online PDFs,
% so please use T1 fonts in your manuscript whenever possible.
% Other font encondings may result in incorrect characters.
%
\usepackage{graphicx}
% Used for displaying a sample figure. If possible, figure files should
% be included in EPS format.
%
% If you use the hyperref package, please uncomment the following two lines
% to display URLs in blue roman font according to Springer's eBook style:
%\usepackage{color}
%\renewcommand\UrlFont{\color{blue}\rmfamily}
%\urlstyle{rm}
%

\usepackage{bm}
\usepackage{amsmath}
\usepackage{amssymb}
\usepackage{multirow}
\usepackage{subfigure}
\usepackage{makecell}
\usepackage{xspace}
\newcommand{\ourmethod}{CLEAR\xspace}

\begin{document}
%
\title{CLEAR: Cluster-based Prompt Learning \\ on Heterogeneous Graphs}
%
\titlerunning{Prompt Learning on Heterogeneous Graphs}
% If the paper title is too long for the running head, you can set
% an abbreviated paper title here
%
\author{Feiyang Wang$^1$, Zhongbao Zhang$^1$, Junda Ye$^1$, Li Sun$^2$, Jianzhong Qi$^2$}
\institute{$^1$Beijing University of Posts and Telecommunications, Beijing, China \\
$^2$North China Electric Power University, Beijing, China\\
\{fywang, zhongbaozb, jundaye\}@bupt.edu.cn, ccesunli@ncepu.edu.cn\\
jianzhong.qi@unimelb.edu.au
}
%
 \authorrunning{Wang, F., et al.}
% % First names are abbreviated in the running head.
% % If there are more than two authors, 'et al.' is used.
% %
% \institute{Princeton University, Princeton NJ 08544, USA \and
% Springer Heidelberg, Tiergartenstr. 17, 69121 Heidelberg, Germany
% \email{lncs@springer.com}\\
% \url{http://www.springer.com/gp/computer-science/lncs} \and
% ABC Institute, Rupert-Karls-University Heidelberg, Heidelberg, Germany\\
% \email{\{abc,lncs\}@uni-heidelberg.de}}
%
\maketitle              % typeset the header of the contribution
%

\begin{abstract}
Prompt learning has attracted increasing attention in the graph domain as a means to bridge the gap between pretext and downstream tasks. Existing studies on heterogeneous graph prompting typically use feature prompts to modify node features for specific downstream tasks, which do not concern the structure of heterogeneous graphs. Such a design also overlooks information from the meta-paths, which are core to learning the high-order semantics of the heterogeneous graphs. To address these issues, we propose \ourmethod, a \underline{C}luster-based prompt \underline{LEAR}ning model on heterogeneous graphs. We present cluster prompts that reformulate downstream tasks as heterogeneous graph reconstruction. In this way, we align the pretext and downstream tasks to share the same training objective. Additionally, our cluster prompts are also injected into the meta-paths such that the prompt learning process incorporates high-order semantic information entailed by the meta-paths. Extensive experiments on downstream tasks confirm the superiority of \ourmethod. It consistently outperforms state-of-the-art models, achieving up to 5\% improvement on the F1 metric for node classification.

\keywords{Heterogeneous graph \and Prompt learning \and Graph clustering.}
\end{abstract}
%
%
%
\section{Introduction}
\label{sec:introduction}
The business processes of organizations are experiencing ever-increasing complexity due to the large amount of data, high number of users, and high-tech devices involved \cite{martin2021pmopportunitieschallenges, beerepoot2023biggestbpmproblems}. This complexity may cause business processes to deviate from normal control flow due to unforeseen and disruptive anomalies \cite{adams2023proceddsriftdetection}. These control-flow anomalies manifest as unknown, skipped, and wrongly-ordered activities in the traces of event logs monitored from the execution of business processes \cite{ko2023adsystematicreview}. For the sake of clarity, let us consider an illustrative example of such anomalies. Figure \ref{FP_ANOMALIES} shows a so-called event log footprint, which captures the control flow relations of four activities of a hypothetical event log. In particular, this footprint captures the control-flow relations between activities \texttt{a}, \texttt{b}, \texttt{c} and \texttt{d}. These are the causal ($\rightarrow$) relation, concurrent ($\parallel$) relation, and other ($\#$) relations such as exclusivity or non-local dependency \cite{aalst2022pmhandbook}. In addition, on the right are six traces, of which five exhibit skipped, wrongly-ordered and unknown control-flow anomalies. For example, $\langle$\texttt{a b d}$\rangle$ has a skipped activity, which is \texttt{c}. Because of this skipped activity, the control-flow relation \texttt{b}$\,\#\,$\texttt{d} is violated, since \texttt{d} directly follows \texttt{b} in the anomalous trace.
\begin{figure}[!t]
\centering
\includegraphics[width=0.9\columnwidth]{images/FP_ANOMALIES.png}
\caption{An example event log footprint with six traces, of which five exhibit control-flow anomalies.}
\label{FP_ANOMALIES}
\end{figure}

\subsection{Control-flow anomaly detection}
Control-flow anomaly detection techniques aim to characterize the normal control flow from event logs and verify whether these deviations occur in new event logs \cite{ko2023adsystematicreview}. To develop control-flow anomaly detection techniques, \revision{process mining} has seen widespread adoption owing to process discovery and \revision{conformance checking}. On the one hand, process discovery is a set of algorithms that encode control-flow relations as a set of model elements and constraints according to a given modeling formalism \cite{aalst2022pmhandbook}; hereafter, we refer to the Petri net, a widespread modeling formalism. On the other hand, \revision{conformance checking} is an explainable set of algorithms that allows linking any deviations with the reference Petri net and providing the fitness measure, namely a measure of how much the Petri net fits the new event log \cite{aalst2022pmhandbook}. Many control-flow anomaly detection techniques based on \revision{conformance checking} (hereafter, \revision{conformance checking}-based techniques) use the fitness measure to determine whether an event log is anomalous \cite{bezerra2009pmad, bezerra2013adlogspais, myers2018icsadpm, pecchia2020applicationfailuresanalysispm}. 

The scientific literature also includes many \revision{conformance checking}-independent techniques for control-flow anomaly detection that combine specific types of trace encodings with machine/deep learning \cite{ko2023adsystematicreview, tavares2023pmtraceencoding}. Whereas these techniques are very effective, their explainability is challenging due to both the type of trace encoding employed and the machine/deep learning model used \cite{rawal2022trustworthyaiadvances,li2023explainablead}. Hence, in the following, we focus on the shortcomings of \revision{conformance checking}-based techniques to investigate whether it is possible to support the development of competitive control-flow anomaly detection techniques while maintaining the explainable nature of \revision{conformance checking}.
\begin{figure}[!t]
\centering
\includegraphics[width=\columnwidth]{images/HIGH_LEVEL_VIEW.png}
\caption{A high-level view of the proposed framework for combining \revision{process mining}-based feature extraction with dimensionality reduction for control-flow anomaly detection.}
\label{HIGH_LEVEL_VIEW}
\end{figure}

\subsection{Shortcomings of \revision{conformance checking}-based techniques}
Unfortunately, the detection effectiveness of \revision{conformance checking}-based techniques is affected by noisy data and low-quality Petri nets, which may be due to human errors in the modeling process or representational bias of process discovery algorithms \cite{bezerra2013adlogspais, pecchia2020applicationfailuresanalysispm, aalst2016pm}. Specifically, on the one hand, noisy data may introduce infrequent and deceptive control-flow relations that may result in inconsistent fitness measures, whereas, on the other hand, checking event logs against a low-quality Petri net could lead to an unreliable distribution of fitness measures. Nonetheless, such Petri nets can still be used as references to obtain insightful information for \revision{process mining}-based feature extraction, supporting the development of competitive and explainable \revision{conformance checking}-based techniques for control-flow anomaly detection despite the problems above. For example, a few works outline that token-based \revision{conformance checking} can be used for \revision{process mining}-based feature extraction to build tabular data and develop effective \revision{conformance checking}-based techniques for control-flow anomaly detection \cite{singh2022lapmsh, debenedictis2023dtadiiot}. However, to the best of our knowledge, the scientific literature lacks a structured proposal for \revision{process mining}-based feature extraction using the state-of-the-art \revision{conformance checking} variant, namely alignment-based \revision{conformance checking}.

\subsection{Contributions}
We propose a novel \revision{process mining}-based feature extraction approach with alignment-based \revision{conformance checking}. This variant aligns the deviating control flow with a reference Petri net; the resulting alignment can be inspected to extract additional statistics such as the number of times a given activity caused mismatches \cite{aalst2022pmhandbook}. We integrate this approach into a flexible and explainable framework for developing techniques for control-flow anomaly detection. The framework combines \revision{process mining}-based feature extraction and dimensionality reduction to handle high-dimensional feature sets, achieve detection effectiveness, and support explainability. Notably, in addition to our proposed \revision{process mining}-based feature extraction approach, the framework allows employing other approaches, enabling a fair comparison of multiple \revision{conformance checking}-based and \revision{conformance checking}-independent techniques for control-flow anomaly detection. Figure \ref{HIGH_LEVEL_VIEW} shows a high-level view of the framework. Business processes are monitored, and event logs obtained from the database of information systems. Subsequently, \revision{process mining}-based feature extraction is applied to these event logs and tabular data input to dimensionality reduction to identify control-flow anomalies. We apply several \revision{conformance checking}-based and \revision{conformance checking}-independent framework techniques to publicly available datasets, simulated data of a case study from railways, and real-world data of a case study from healthcare. We show that the framework techniques implementing our approach outperform the baseline \revision{conformance checking}-based techniques while maintaining the explainable nature of \revision{conformance checking}.

In summary, the contributions of this paper are as follows.
\begin{itemize}
    \item{
        A novel \revision{process mining}-based feature extraction approach to support the development of competitive and explainable \revision{conformance checking}-based techniques for control-flow anomaly detection.
    }
    \item{
        A flexible and explainable framework for developing techniques for control-flow anomaly detection using \revision{process mining}-based feature extraction and dimensionality reduction.
    }
    \item{
        Application to synthetic and real-world datasets of several \revision{conformance checking}-based and \revision{conformance checking}-independent framework techniques, evaluating their detection effectiveness and explainability.
    }
\end{itemize}

The rest of the paper is organized as follows.
\begin{itemize}
    \item Section \ref{sec:related_work} reviews the existing techniques for control-flow anomaly detection, categorizing them into \revision{conformance checking}-based and \revision{conformance checking}-independent techniques.
    \item Section \ref{sec:abccfe} provides the preliminaries of \revision{process mining} to establish the notation used throughout the paper, and delves into the details of the proposed \revision{process mining}-based feature extraction approach with alignment-based \revision{conformance checking}.
    \item Section \ref{sec:framework} describes the framework for developing \revision{conformance checking}-based and \revision{conformance checking}-independent techniques for control-flow anomaly detection that combine \revision{process mining}-based feature extraction and dimensionality reduction.
    \item Section \ref{sec:evaluation} presents the experiments conducted with multiple framework and baseline techniques using data from publicly available datasets and case studies.
    \item Section \ref{sec:conclusions} draws the conclusions and presents future work.
\end{itemize}
\section{Related Work}
The landscape of large language model vulnerabilities has been extensively studied in recent literature \cite{crothers2023machinegeneratedtextcomprehensive,shayegani2023surveyvulnerabilitieslargelanguage,Yao_2024,Huang2023ASO}, that propose detailed taxonomies of threats. These works categorize LLM attacks into distinct types, such as adversarial attacks, data poisoning, and specific vulnerabilities related to prompt engineering. Among these, prompt injection attacks have emerged as a significant and distinct category, underscoring their relevance to LLM security.

The following high-level overview of the collected taxonomy of LLM vulnerabilities is defined in \cite{Yao_2024}:
\begin{itemize}
    \item Adversarial Attacks: Data Poisoning, Backdoor Attacks
    \item Inference Attacks: Attribute Inference, Membership Inferences
    \item Extraction Attacks
    \item Bias and Unfairness
Exploitation
    \item Instruction Tuning Attacks: Jailbreaking, Prompt Injection.
\end{itemize}
Prompt injection attacks are further classified in \cite{shayegani2023surveyvulnerabilitieslargelanguage} into the following: Goal hijacking and \textbf{Prompt leakage}.

The reviewed taxonomies underscore the need for comprehensive frameworks to evaluate LLM security. The agentic approach introduced in this paper builds on these insights, automating adversarial testing to address a wide range of scenarios, including those involving prompt leakage and role-specific vulnerabilities.

\subsection{Prompt Injection and Prompt Leakage}

Prompt injection attacks exploit the blending of instructional and data inputs, manipulating LLMs into deviating from their intended behavior. Prompt injection attacks encompass techniques that override initial instructions, expose private prompts, or generate malicious outputs \cite{Huang2023ASO}. A subset of these attacks, known as prompt leakage, aims specifically at extracting sensitive system prompts embedded within LLM configurations. In \cite{shayegani2023surveyvulnerabilitieslargelanguage}, authors differentiate between prompt leakage and related methods such as goal hijacking, further refining the taxonomy of LLM-specific vulnerabilities.

\subsection{Defense Mechanisms}

Various defense mechanisms have been proposed to address LLM vulnerabilities, particularly prompt injection and leakage \cite{shayegani2023surveyvulnerabilitieslargelanguage,Yao_2024}. We focused on cost-effective methods like instruction postprocessing and prompt engineering, which are viable for proprietary models that cannot be retrained. Instruction preprocessing sanitizes inputs, while postprocessing removes harmful outputs, forming a dual-layer defense. Preprocessing methods include perplexity-based filtering \cite{Jain2023BaselineDF,Xu2022ExploringTU} and token-level analysis \cite{Kumar2023CertifyingLS}. Postprocessing employs another set of techniques, such as censorship by LLMs \cite{Helbling2023LLMSD,Inan2023LlamaGL}, and use of canary tokens and pattern matching \cite{vigil-llm,rebuff}, although their fundamental limitations are noted \cite{Glukhov2023LLMCA}. Prompt engineering employs carefully designed instructions \cite{Schulhoff2024ThePR} and advanced techniques like spotlighting \cite{Hines2024DefendingAI} to mitigate vulnerabilities, though no method is foolproof \cite{schulhoff-etal-2023-ignore}. Adversarial training, by incorporating adversarial examples into the training process, strengthens models against attacks \cite{Bespalov2024TowardsBA,Shaham2015UnderstandingAT}.

\subsection{Security Testing for Prompt Injection Attacks}

Manual testing, such as red teaming \cite{ganguli2022redteaminglanguagemodels} and handcrafted "Ignore Previous Prompt" attacks \cite{Perez2022IgnorePP}, highlights vulnerabilities but is limited in scale. Automated approaches like PAIR \cite{chao2024jailbreakingblackboxlarge} and GPTFUZZER \cite{Yu2023GPTFUZZERRT} achieve higher success rates by refining prompts iteratively or via automated fuzzing. Red teaming with LLMs \cite{Perez2022RedTL} and reinforcement learning \cite{anonymous2024diverse} uncovers diverse vulnerabilities, including data leakage and offensive outputs. Indirect Prompt Injection (IPI) manipulates external data to compromise applications \cite{Greshake2023NotWY}, adapting techniques like SQL injection to LLMs \cite{Liu2023PromptIA}. Prompt secrecy remains fragile, with studies showing reliable prompt extraction \cite{Zhang2023EffectivePE}. Advanced frameworks like Token Space Projection \cite{Maus2023AdversarialPF} and Weak-to-Strong Jailbreaking Attacks \cite{zhao2024weaktostrongjailbreakinglargelanguage} exploit token-space relationships, achieving high success rates for prompt extraction and jailbreaking.

\subsection{Agentic Frameworks for Evaluating LLM Security}

The development of multi-agent systems leveraging large language models (LLMs) has shown promising results in enhancing task-solving capabilities \cite{Hong2023MetaGPTMP, Wang2023UnleashingTE, Talebirad2023MultiAgentCH, Wu2023AutoGenEN, Du2023ImprovingFA}. A key aspect across various frameworks is the specialization of roles among agents \cite{Hong2023MetaGPTMP, Wu2023AutoGenEN}, which mimics human collaboration and improves task decomposition.

Agentic frameworks and the multi-agent debate approach benefit from agent interaction, where agents engage in conversations or debates to refine outputs and correct errors \cite{Wu2023AutoGenEN}. For example, debate systems improve factual accuracy and reasoning by iteratively refining responses through collaborative reasoning \cite{Du2023ImprovingFA}, while AG2 allows agents to autonomously interact and execute tasks with minimal human input.

These frameworks highlight the viability of agentic systems, showing how specialized roles and collaborative mechanisms lead to improved performance, whether in factuality, reasoning, or task execution. By leveraging the strengths of diverse agents, these systems demonstrate a scalable approach to problem-solving.

Recent research on testing LLMs using other LLMs has shown that this approach can be highly effective \cite{chao2024jailbreakingblackboxlarge, Yu2023GPTFUZZERRT, Perez2022RedTL}. Although the papers do not explicitly employ agentic frameworks they inherently reflect a pattern similar to that of an "attacker" and a "judge". \cite{chao2024jailbreakingblackboxlarge}  This pattern became a focal point for our work, where we put the judge into a more direct dialogue, enabling it to generate attacks based on the tested agent response in an active conversation.

A particularly influential paper in shaping our approach is Jailbreaking Black Box Large Language Models in Twenty Queries \cite{chao2024jailbreakingblackboxlarge}. This paper not only introduced the attacker/judge architecture but also provided the initial system prompts used for a judge.
% Consider a lasso optimization procedure with potentially distinct regularization penalties:
% \begin{align}
%     \hat{\beta} = \arg\min_{\beta}\{\|y-X\beta\|^2_2+\sum_{i=1}^{N}\lambda_i|\beta_i|\}.
% \end{align}
\subsection{Supervised Data-Driven Learning}\label{subsec:supervised}
We consider a generic data-driven supervised learning procedure. Given a dataset \( \mathcal{D} \) consisting of \( n \) data points \( (x_i, y_i) \in \mathcal{X} \times \mathcal{Y} \) drawn from an underlying distribution \( p(\cdot|\theta) \), our goal is to estimate parameters \( \theta \in \Theta \) through a learning procedure, defined as \( f: (\mathcal{X} \times \mathcal{Y})^n \rightarrow \Theta \) 
that minimizes the predictive error on observed data. 
Specifically, the learning objective is defined as follows:
\begin{align}
\hat{\theta}_f := f(\mathcal{D}) = \arg\min_{\theta} \mathcal{L}(\theta, \mathcal{D}),
\end{align}
where \( \mathcal{L}(\cdot,\mathcal{D}) := \sum_{i=1}^{n} \mathcal{L}(\cdot, (x_i, y_i))\), and $\mathcal{L}$ is a loss function quantifying the error between predictions and true outcomes. 
Here, $\hat{\theta}_f$ is the parameter that best explains the observed data pairs \( (x_i, y_i) \) according to the chosen loss function \( \mathcal{L} (\cdot) \).

\paragraph{Feature Selection.}
Feature selection aims to improve model \( f \)'s predictive performance while minimizing redundancy. 
%Formally, given data \( X \), response \( y \), feature set \( \mathcal{F} \), loss function \( \mathcal{L}(\cdot) \), and a feature limit \( k \), the objective is:
% \begin{align}
% \mathcal{S}^* = \arg \min_{\mathcal{S} \subseteq \mathcal{F}, |\mathcal{S}| \leq k} \mathcal{L}(y, f(X_\mathcal{S})) + \lambda R(\mathcal{S}),
% \end{align}
% where \( X_\mathcal{S} \) is the submatrix of \( X \) for selected features \( \mathcal{S} \), \( \lambda \) is a regularization parameter, and \( R(\mathcal{S}) \) penalizes feature redundancy.
 State-of-the-art techniques fall into four categories: (i) filter methods, which rank features based on statistical properties like Fisher score \citep{duda2001pattern,song2012feature}; (ii) wrapper methods, which evaluate model performance on different feature subsets \citep{kohavi1997wrappers}; (iii) embedded methods, which integrate feature selection into the learning process using techniques like regularization \citep{tibshirani1996LASSO,lemhadri2021lassonet}; and (iv) hybrid methods, which combine elements of (i)-(iii) \citep{SINGH2021104396,li2022micq}. This paper focuses on embedded methods via Lasso, benchmarking against approaches from (i)-(iii).

\subsection{Language Modeling}
% The objective of language modeling is to learn a probability distribution \( p_{LM}(x) \) over sequences of text \( x = (X_1, \ldots, X_{|x|}) \), such that \( p_{LM}(x) \approx p_{text}(x) \), where \( p_{text}(x) \) represents the true distribution of natural language. This process involves estimating the likelihood of token sequences across variable lengths and diverse linguistic structures.
% Modern large language models (LLMs) are trained on vast datasets spanning encyclopedias, news, social media, books, and scientific papers \cite{gao2020pile}. This broad training enables them to generalize across domains, learn contextual knowledge, and perform zero-shot learning—tackling new tasks using only task descriptions without fine-tuning \cite{brown2020gpt3}.
Language modeling aims to approximate the true distribution of natural language \( p_{\text{text}}(x) \) by learning \( p_{\text{LM}}(x) \), a probability distribution over text sequences \( x = (X_1, \ldots, X_{|x|}) \). Modern large language models, trained on diverse datasets \citep{gao2020pile}, exhibit strong generalization across domains, acquire contextual knowledge, and perform zero-shot learning—solving new tasks using only task descriptions—or few-shot learning by leveraging a small number of demonstrations \citep{brown2020gpt3}.
\paragraph{Retrieval-Augmented Generation (RAG).} Retrieval-Augmented Generation (RAG) enhances the performance of generative language models by  integrating a domain-specific information retrieval process  \citep{lewis2020retrieval}. The RAG framework comprises two main components: \textit{retrieval}, which extracts relevant information from external knowledge sources, and \textit{generation}, where an LLM generates context-aware responses using the prompt combined with the retrieved context. Documents are indexed through various databases, such as relational, graph, or vector databases \citep{khattab2020colbert, douze2024faiss, peng2024graphretrievalaugmentedgenerationsurvey}, enabling efficient organization and retrieval via algorithms like semantic similarity search to match the prompt with relevant documents in the knowledge base. RAG has gained much traction recently due to its demonstrated ability to reduce incidence of hallucinations and boost LLMs' reliability as well as performance \citep{huang2023hallucination, zhang2023merging}. 
 
% image source: https://medium.com/@bindurani_22/retrieval-augmented-generation-815c1ae438d8
\begin{figure}
    \centering
\includegraphics[width=1.03\linewidth]{fig/fig1.pdf}
\vspace{-0.6cm}
\scriptsize 
    \caption{Retrieval Augmented Generation (RAG) based $\ell_1$-norm weights (penalty factors) for Lasso. Only feature names---no training data--- are included in LLM prompt.} 
    \label{fig:rag}
\end{figure}
% However, for the RAG model to be effective given the input token constraints of the LLM model used, we need to effectively process the retrieval documents through a procedure known as \textit{chunking}.

\subsection{Task-Specific Data-Driven Learning}
LLM-Lasso aims to bridge the gap between data-driven supervised learning and the predictive capabilities of LLMs trained on rich metadata. This fusion not only enhances traditional data-driven methods by incorporating key task-relevant contextual information often overlooked by such models, but can also be especially valuable in low-data regimes, where the learning algorithm $f:\mathcal{D}\rightarrow\Theta$ (seen as a map from datasets $\mathcal{D}$ to the space of decisions $\Theta$) is susceptible to overfitting.

The task-specific data-driven learning model $\tilde{f}:\mathcal{D}\times\mathcal{D}_\text{meta}\rightarrow\Theta$ can be described as a metadata-augmented version of $f$, where a link function $h(\cdot)$ integrates metadata (i.e. $\mathcal{D}_\text{meta}$) to refine the original learning process. This can be expressed as:
\[
\tilde{f}(\mathcal{D}, \mathcal{D}_\text{meta}) := \mathcal{T}(f(\mathcal{D}),  h(\mathcal{D}_{\text{meta}})),
\]
where the functional $\mathcal{T}$ takes the original learning algorithm $f(\mathcal{D})$ and transforms it into a task-specific learning algorithm $\tilde{f}(\mathcal{D}, \mathcal{D}_\text{meta})$ by incorporating the metadata $\mathcal{D}_\text{meta}$. 
% In particular, the link function $h(\mathcal{D}_{\text{meta}})$ provides a structured mechanism summarizing the contextual knowledge.

There are multiple approaches to formulate $\mathcal{T}$ and $h$.
%to ``inform" the data-driven model $f$ of %meta knowledge. 
For instance, LMPriors \citep{choi2022lmpriorspretrainedlanguagemodels} designed $h$ and $\mathcal{T}$ such that $h(\mathcal{D}_{\text{meta}})$ first specifies which features to retain (based on a probabilistic prior framework), and then $\mathcal{T}$ keeps the selected features and removes all the others from the original learning objective of $f$. 
Note that this approach inherently is restricted as it selects important features solely based on $\mathcal{D}_\text{meta}$ without seeing $\mathcal{D}$.

In contrast, we directly embed task-specific knowledge into the optimization landscape through regularization by introducing a structured inductive bias. This bias guides the learning process toward solutions that are consistent with metadata-informed insights, without relying on explicit probabilistic modeling. Abstractly, this can be expressed as:
\begin{align}
    \!\!\!\!\!\hat{\theta}_{\tilde{f}} := \tilde{f}(\mathcal{D},\mathcal{D}
    _\text{meta})= \arg\min_{\theta} \mathcal{L}(\theta, \mathcal{D}) + \lambda R(\theta, \mathcal{D}_{\text{meta}}),
\end{align}
where \( \lambda \) is a regularization parameter, \( R(\cdot) \) is a regularizer, and $\theta$ is the prediction parameter.
%We explain our framework with more details in the following section.


% Our research diverges from both aforementioned approaches by positioning the LLM not as a standalone feature selector but as an enhancement to data-driven models through an embedded feature selection method, L-LASSO. L-LASSO incorporates domain expertise—auxiliary natural language metadata about the task—via the LLM-informed LASSO penalty, which is then used in statistical models to enhance predictive performance. This method integrates the rich, context-sensitive insights of LLMs with the rigor and transparency of statistical modeling, bridging the gap between data-driven and knowledge-driven feature selection approaches. To approach this task, we need to tackle two key components: (i). train an LLM that is expert in the task-specific knowledge; (ii). inform data-driven feature selector LASSO with LLM knowledge.

% In practice, this involves combining techniques like prompt engineering and data engineering to develop an effective framework for integrating metadata into existing data-driven models. We will go through this in detail in Section \ref{mthd} and \ref{experiment}.


\section{Model}
\label{sec:model}
Let $[N] = \{1, 2, \dots, N \}$ be a set of $N$ agents.
We examine an environment in which a system interacts with the agents over $T$ rounds.
Every round $t\leq T$ comprises $N$ \emph{sessions}, each session represents an encounter of the system with exactly one agent, and each agent interacts exactly once with the system every round.
I.e., in each round $t$ the agents arrive sequentially. 


\paragraph{Arrival order} The \emph{arrival order} of round $t$, denoted as $\ordv_t=(\ord_t(1),\dots, \ord_t(N))$, is an element from set of all permutations of $[N]$. Each entry $q$ in $\ordv_t$ is the index of the agent that arrives in the $q^{\text{th}}$ session of round $t$.
For example, if $\ord_t(1) = 2$, then agent $2$ arrives in the first session of round $t$.
Correspondingly, $\ord_t^{-1}(i)=q$ implies that agent $i$ arrives in the $q^{\text{th}}$ session of round $t$. 

As we demonstrate later, the arrival order has an immediate impact on agent rewards. We call the mechanism by which the arrival order is set \emph{arrival function} and denote it by $\ordname$. Throughout the paper, we consider several arrival functions such as the \emph{uniform arrival} function, denoted by $\uniord$, and the \emph{nudged arrival} $\sugord$; we introduce those formally in Sections~\ref{sec:uniform} and~\ref{sec:nudge}, respectively.

%We elaborate more on this concept in Section~\ref{sec: arrival}.


\paragraph{Arms} We consider a set of $K \geq 2$ arms, $A = \{a_1, \ldots, a_K\}$. The reward of arm $a_i$ in round $t$ is a random variable $X_i^t \sim \mathcal{D}^t_i$, where the rewards $(X_i^t)_{i,t}$ are mutually independent and bounded within the interval $[0,1]$. The reward distribution $\mathcal{D}^t_i$ of arm $a_i$, $i\in [K]$ at round $t\in T$ is assumed to be non-stationary but independent across arms and rounds. We denote the realized reward of arm $a_i$ in round $t$ by $x_i^t$. We assume \emph{reward consistency}, meaning that rewards may vary between rounds but remain constant within the sessions of a single round. Specifically, if an arm $a_i$ is selected multiple times during round~$t$, each selection yields the same reward $x_i^t$, where the superscript $t$ indicates its dependence on the round rather than the session. This consistency enables the system to leverage information obtained from earlier sessions to make more informed decisions in later sessions within the same round. We provide further details on this principle in Subsection~\ref{subsec:information}.


\paragraph{Algorithms} An algorithm is a mapping from histories to actions. We typically expect algorithms to maximize some aggregated agent metric like social welfare. Let $\mathcal H^{t,q}$ denote the information observed during all sessions of rounds $1$ to $t-1$ and sessions $1$ to $q-1$ in round $t$.  The history $\mathcal H^{t,q}$ is an element from $(A \times [0,1])^{(t-1) \cdot N +q-1}$, consisting of pairs of the form (pulled arm, realized reward). Notice that we restrict our attention to \emph{anonymous} algorithms, i.e., algorithms that do not distinguish between agents based on their identities. Instead, they only respond to the history of arms pulled and rewards observed, without conditioning on which specific agent performed each action.
%In the most general case, algorithms make decisions at session $q$ of round $t$  based on the entire history $\mathcal H^{t,q}$ and the index of the arriving agent $\ord_t(q)$. %Furthermore, we sometimes assume that algorithms have Bayesian information, i.e., algorithms are aware of the distributions $(\mathcal D_i)^K_{i=1}$. 
Furthermore, we sometimes assume that algorithms have Bayesian information, meaning they are aware of the reward distributions $(\mathcal{D}^t_i)_{i,t}$. If such an assumption is required to derive a result, we make it explicit. %Otherwise, we do not assume any additional knowledge about the algorithm’s information. %This distinction allows us to analyze both general algorithms without prior distributional knowledge and specialized algorithms that leverage Bayesian information.


\paragraph{Rewards} Let $\rt{i}$ denote the reward received by agent $i \in [N]$ at round $t$, and let $\Rt{i}$ denote her cumulative reward at the end of round $t$, i.e., $\Rt{i} = \sum_{\tau=1}^{t}{r^{\tau}_{i}}$. We further denote the \emph{social welfare} as the sum of the rewards all agents receive after $T$ rounds. Formally, $\sw=\sum^{N}_{i=1}{R^T_i}$. We emphasize that social welfare is independent of the arrival order. 


\paragraph{Envy}
We denote by $\adift{i}{j}$ the reward discrepancy of agents $i$ and $j$ in round $t$; namely, $\adift{i}{j}= \rt{i} - \rt{j}$. %We call this term \omer{name??} reward discrepancy in round $t$. 
The (cumulative) \emph{envy} between two agents at round $t$ is the difference in their cumulative rewards. Formally, $\env_{i,j}^t= \Rt{i} - \Rt{j}$ is the envy after $t$ rounds between agent $i$ and $j$. We can also formulate envy as the sum of reward discrepancies, $\env_{i,j}^t= \sum^{t}_{\tau=1}{\adif{i}{j}^\tau}$. Notice that envy is a signed quantity and can be either positive or negative. Specifically, if $\env_{i,j}^t < 0$, we say that agent $i$ envies agent $j$, and if $\env_{i,j}^t > 0$, agent $j$ envies agent $i$. The main goal of this paper is to investigate the behavior of the \emph{maximal envy}, defined as
\[
\env^t = \max_{i,j \in [N]} \env^t_{i,j}.
\]
For clarity, the term \emph{envy} will refer to the maximal envy.\footnote{ We address alternative definitions of envy in Section~\ref{sec:discussion}.} % Envy can also be defined in alternative ways, such as by averaging pairwise envy across all agents. We address average envy in Section~\ref{sec:avg_envy}.}
Note that $\env_{i,j}^t$ are random variables that depend on the decision-making algorithm, realized rewards, and the arrival order, and therefore, so is $\env^t$. If a result we obtain regarding envy depends on the arrival order $\ordname$, we write $\env^t(\ordname)$. Similarly, to ease notation, if $\ordname$ can be understood from the context, it is omitted.



\paragraph{Further Notation} We use the subscript $(q)$ to address elements of the $q^{\text{th}}$ session, for $q\in [N]$.
That is, we use the notation $\rt{(q)}$ to denote the reward granted to the agent that arrives in the $q^{\text{th}}$ session of round $t$ and $\Rt{(q)}$ to denote her cumulative reward. %Additionally, we introduce the notation $\at{(q)}$ to denote the arm pulled in that session.
Correspondingly, $\sdift{q}{w} = \rt{(q)} - \rt{(w)}$ is the reward discrepancy of the agents arriving in the $q^{\text{th}}$ and $w^{\text{th}}$ sessions of round $t$, respectively. 
To distinguish agents, arms, sessions and rounds, we use the letters $i,j$ to mark agents and arms, $q,w$ for sessions, and $t,\tau$ for rounds.


\subsection{Example}
\label{sec: example}
To illustrate the proposed setting and notation, we present the following example, which serves as a running example throughout the paper.

\begin{table}[t]
\centering
\begin{tabular}{|c|c|c|c|}
\hline
$t$ (round) & $\ordv_t$ (arrival order) & $x_1^t$ & $x_2^t$ \\ \hline
1           & 2, 1                     & 0.6     & 0.92    \\ \hline
2           & 1, 2                     & 0.48    & 0.1     \\ \hline
3           & 2, 1                     & 0.15    & 0.8     \\ \hline
\end{tabular}
\caption{
    Data for Example~\ref{example 1}.
}
\label{tbl: example}
\end{table}

\begin{algorithm}[t]
\caption{Algorithm for Example~\ref{example 1}}
\label{alguni}
\DontPrintSemicolon 
\For{round $t = 1$ to $T$}{
    pull $a_{1}$ in the first session\label{alguniexample: first}\\
    \lIf{$x^t_1 \geq \frac{1}{2}$}{pull $a_{1}$ again in second session \label{alguniexample: pulling a again}}
    \lElse{pull $a_{2}$ in second session \label{alguniexample: sopt else}}
}
\end{algorithm}


\begin{example}\label{example 1}
We consider $K=2$ uniform arms, $X_1,X_2 \sim \uni{0,1}$, and $N=2$ for some $T\geq 3$. We shall assume arm decision are made by Algorithm~\ref{alguni}: In the first session, the algorithm pulls $a_1$; if it yields a reward greater than $\nicefrac{1}{2}$, the algorithm pulls it again in the second session (the ``if'' clause). Otherwise, it pulls $a_2$.



We further assume that the arrival orders and rewards are as specified in Table~\ref{tbl: example}. Specifically, agent 2 arrives in the first session of round $t=1$, and pulling arm $a_2$ in this round would yield a reward of $x^1_2 = 0.92$. Importantly, \emph{this information is not available to the decision-making algorithm in advance} and is only revealed when or if the corresponding arms are pulled.




In the first round, $\boldsymbol{\eta}^1 = \left(2,1\right)$; thus, agent 2 arrives in the first session.
The algorithm pulls arm $a_1$, which means, $a^1_{(1)} = a_1$, and the agent receives $r_{2}^1=r_{(1)}^1=x_1^1=0.6$.
Later that round, in the second session, agent 1 arrives, and the algorithm pulls the same arm again since $x^1_1 = 0.6 \geq \nicefrac{1}{2}$ due to the ``if'' clause.
I.e., $a^1_{(2)} = a_1$ and $r_{1}^1 = r_{(2)}^1 = x_1^1 = 0.6$.
Even though the realized reward of arm $a_2$ in that round is higher ($0.92$), the algorithm is not aware of that value.
At the end of the first round, $R^1_1 = R^1_{(2)} = R^1_2 = R^1_{(1)} = 0.6$. The reward discrepancy is thus $\adif{1}{2}^1 = \adif{2}{1}^1= \sdif{2}{1}^1 = 0.6 - 0.6 =0$. 



In the second round, agent 1 arrives first, followed by agent 2.
Firstly, the algorithm pulls arm $a_1$ and agent 1 receives a reward of $r_{1}^2 = r_{(1)}^2 = x_1^2 = 0.48$.
Because the reward is lower than $\nicefrac{1}{2}$, in the second session the algorithm pulls the other arm ($a^2_{(2)} = a_2$), granting agent 2 a reward of $r_{2}^2 = r_{(2)}^2 = x_2^2 = 0.1$.
At the end of the second round, $R^2_1 = R^2_{(1)} = 0.6 + 0.48 = 1.08$ and $R^2_2 = R^2_{(2)} = 0.6 + 0.1 = 0.7$. Furthermore, $\sdif{2}{1}^2 = \adif{2}{1}^2 = r^2_{2} - r^2_{1} = 0.1 - 0.48 = -0.38$.

In the third and final round, agent 2 arrives first again, and receives a reward  of $0.15$ from $a_1$. When agent 1 arrives in the second session, the algorithm pulls arm $a_2$, and she receives a reward of $0.8$. As for the reward discrepancy, $\sdif{2}{1}^3 = \adif{2}{1}^3 = r^3_{2} - r^3_{1} = 0.15 - 0.8 = -0.75$. 

Finally, agent 1 has a cumulative reward of $R^3_1 = R^3_{(2)} = 0.6 + 0.48 + 0.8 = 1.88$, whereas agent~2 has a cumulative reward of $R^3_2 = R^3_{(1)} = 0.6 + 0.1 + 0.15 = 0.85$. In terms of envy, $\env^1_{1,2}= \adif{1}{2}^1 =0$, $\env^2_{1,2}=\adif{1}{2}^1+\adif{1}{2}^2= 0.38$, and $\env^3_{1,2} = -\env^3_{2,1} = R^3_1-R^3_2 = 1.88-0.85 = 1.03$, and consequently the envy in round 3 is $\env^3 = 1.03$.
\end{example}


\subsection{Information Exploitation}
\label{subsec:information}

In this subsection, we explain how algorithms can exploit intra-round information.
Since rewards are consistent in the sessions of each round, acquiring information in each session can be used to increase the reward of the following sessions.
In other words, the earlier sessions can be used for exploration, and we generally expect agents arriving in later sessions to receive higher rewards.
Taken to the extreme, an agent that arrives after all arms have been pulled could potentially obtain the highest reward of that round, depending on how the algorithm operates.

To further demonstrate the advantage of late arrival, we reconsider Example~\ref{example 1} and Algorithm~\ref{alguni}. 
The expected reward for the agent in the first session of round $t$ is $\E{\rt{(1)}}=\mu_1=\frac{1}{2}$, yet the expected reward of the agent in the second session is
\begin{align*}
\E{\rt{(2)}}=\E{\rt{(2)}\mid X^t_1 \geq \frac{1}{2} }\prb{X^t_1 \geq \frac{1}{2}} + \E{\rt{(2)}\mid X^t_1 < \frac{1}{2} }\prb{X^t_1 < \frac{1}{2}};
\end{align*}
thus, $\E{\rt{(2)}} =\E{X^t_1\mid X^t_1 \geq \frac{1}{2} }\cdot \frac{1}{2} + \mu_2\cdot\frac{1}{2} = \frac{5}{8}$.
Consequently, the expected welfare per round is $\E{\rt{(1)}+\rt{(2)}}=1+\frac{1}{8}$, and the benefit of arriving in the second session of any round $t$ is $\E{\rt{(2)} - \rt{(1)}} = \frac{1}{8}$. This gap creates envy over time, which we aim to measure and understand.
%This discrepancy generates envy over time, and our paper aims to better understand it.
\subsection{Socially Optimal Algorithms}
\label{sec: sw}
Since our model is novel, particularly in its focus on the reward consistency element, studying social welfare maximizing algorithms represents an important extension of our work. While the primary focus of this paper is to analyze envy under minimal assumptions about algorithmic operations, we also make progress in the direction of social welfare optimization. See more details in Section~\ref{sec:discussion}.%Due to space limitations, we defer the discussion on socially optimal algorithms to  \ifnum\Includeappendix=0{the appendix}\else{Section~\ref{appendix:sociallyopt}}\fi.




% Since our model is novel and specifically the reward consistency element, it might be interesting to study social welfare optimization. While the main focus of our paper is to study envy under minimal assumptions on how the algorithm operates, we take steps toward this direction as well. Due to space limitations, we defer the discussion on socially optimal algorithms to  \ifnum\Includeappendix=0{the appendix}\else{Section~\ref{appendix:sociallyopt}}\fi.  We devise a socially optimal algorithm for the two-agent case, offer efficient and optimal algorithms for special cases of $N>2$ agents, and an inefficient and approximately optimal algorithm for any instance with $N>2$. Moreover, we address the welfare-envy tradeoff in Section~\ref{sec:extensions}.


% Social welfare, unlike envy, is entirely independent of the arrival order. While the main focus of our paper is to study envy under minimal assumptions on how the algorithm operates, socially optimal algorithms might also be of interest. Due to space limitations, we defer the discussion on socially optimal algorithms to  \ifnum\Includeappendix=0{the appendix}\else{Section~\ref{appendix:sociallyopt}}\fi. We devise a socially optimal algorithm for the two-agent case, offer efficient and optimal algorithms for special cases of $N>2$ agents, and an inefficient and approximately optimal algorithm for any instance with $N>2$. %Furthermore, we treat the welfare-envy tradeoff of the special case of Example~\ref{example 1}.



\section{Experiments}
\label{sec:exp}
Following the settings in Section \ref{sec:existing}, we evaluate \textit{NovelSum}'s correlation with the fine-tuned model performance across 53 IT datasets and compare it with previous diversity metrics. Additionally, we conduct a correlation analysis using Qwen-2.5-7B \cite{yang2024qwen2} as the backbone model, alongside previous LLaMA-3-8B experiments, to further demonstrate the metric's effectiveness across different scenarios. Qwen is used for both instruction tuning and deriving semantic embeddings. Due to resource constraints, we run each strategy on Qwen for two rounds, resulting in 25 datasets. 

\subsection{Main Results}

\begin{table*}[!t]
    \centering
    \resizebox{\linewidth}{!}{
    \begin{tabular}{lcccccccccc}
    \toprule
    \multirow{3}*{\textbf{Diversity Metrics}} & \multicolumn{10}{c}{\textbf{Data Selection Strategies}} \\
    \cmidrule(lr){2-11}
    & \multirow{2}*{\textbf{K-means}} & \multirow{2}*{\vtop{\hbox{\textbf{K-Center}}\vspace{1mm}\hbox{\textbf{-Greedy}}}}  & \multirow{2}*{\textbf{QDIT}} & \multirow{2}*{\vtop{\hbox{\textbf{Repr}}\vspace{1mm}\hbox{\textbf{Filter}}}} & \multicolumn{5}{c}{\textbf{Random}} & \multirow{2}{*}{\textbf{Duplicate}} \\ 
    \cmidrule(lr){6-10}
    & & & & & \textbf{$\mathcal{X}^{all}$} & ShareGPT & WizardLM & Alpaca & Dolly &  \\
    \midrule
    \rowcolor{gray!15} \multicolumn{11}{c}{\textit{LLaMA-3-8B}} \\
    Facility Loc. $_{\times10^5}$ & \cellcolor{BLUE!40} 2.99 & \cellcolor{ORANGE!10} 2.73 & \cellcolor{BLUE!40} 2.99 & \cellcolor{BLUE!20} 2.86 & \cellcolor{BLUE!40} 2.99 & \cellcolor{BLUE!0} 2.83 & \cellcolor{BLUE!30} 2.88 & \cellcolor{BLUE!0} 2.83 & \cellcolor{ORANGE!20} 2.59 & \cellcolor{ORANGE!30} 2.52 \\    
    DistSum$_{cosine}$  & \cellcolor{BLUE!30} 0.648 & \cellcolor{BLUE!60} 0.746 & \cellcolor{BLUE!0} 0.629 & \cellcolor{BLUE!50} 0.703 & \cellcolor{BLUE!10} 0.634 & \cellcolor{BLUE!40} 0.656 & \cellcolor{ORANGE!30} 0.578 & \cellcolor{ORANGE!10} 0.605 & \cellcolor{ORANGE!20} 0.603 & \cellcolor{BLUE!10} 0.634 \\
    Vendi Score $_{\times10^7}$ & \cellcolor{BLUE!30} 1.70 & \cellcolor{BLUE!60} 2.53 & \cellcolor{BLUE!10} 1.59 & \cellcolor{BLUE!50} 2.23 & \cellcolor{BLUE!20} 1.61 & \cellcolor{BLUE!30} 1.70 & \cellcolor{ORANGE!10} 1.44 & \cellcolor{ORANGE!20} 1.32 & \cellcolor{ORANGE!10} 1.44 & \cellcolor{ORANGE!30} 0.05 \\
    \textbf{NovelSum (Ours)} & \cellcolor{BLUE!60} 0.693 & \cellcolor{BLUE!50} 0.687 & \cellcolor{BLUE!30} 0.673 & \cellcolor{BLUE!20} 0.671 & \cellcolor{BLUE!40} 0.675 & \cellcolor{BLUE!10} 0.628 & \cellcolor{BLUE!0} 0.591 & \cellcolor{ORANGE!10} 0.572 & \cellcolor{ORANGE!20} 0.50 & \cellcolor{ORANGE!30} 0.461 \\
    \midrule    
    \textbf{Model Performance} & \cellcolor{BLUE!60}1.32 & \cellcolor{BLUE!50}1.31 & \cellcolor{BLUE!40}1.25 & \cellcolor{BLUE!30}1.05 & \cellcolor{BLUE!20}1.20 & \cellcolor{BLUE!10}0.83 & \cellcolor{BLUE!0}0.72 & \cellcolor{ORANGE!10}0.07 & \cellcolor{ORANGE!20}-0.14 & \cellcolor{ORANGE!30}-1.35 \\
    \midrule
    \midrule
    \rowcolor{gray!15} \multicolumn{11}{c}{\textit{Qwen-2.5-7B}} \\
    Facility Loc. $_{\times10^5}$ & \cellcolor{BLUE!40} 3.54 & \cellcolor{ORANGE!30} 3.42 & \cellcolor{BLUE!40} 3.54 & \cellcolor{ORANGE!20} 3.46 & \cellcolor{BLUE!40} 3.54 & \cellcolor{BLUE!30} 3.51 & \cellcolor{BLUE!10} 3.50 & \cellcolor{BLUE!10} 3.50 & \cellcolor{ORANGE!20} 3.46 & \cellcolor{BLUE!0} 3.48 \\ 
    DistSum$_{cosine}$ & \cellcolor{BLUE!30} 0.260 & \cellcolor{BLUE!60} 0.440 & \cellcolor{BLUE!0} 0.223 & \cellcolor{BLUE!50} 0.421 & \cellcolor{BLUE!10} 0.230 & \cellcolor{BLUE!40} 0.285 & \cellcolor{ORANGE!20} 0.211 & \cellcolor{ORANGE!30} 0.189 & \cellcolor{ORANGE!10} 0.221 & \cellcolor{BLUE!20} 0.243 \\
    Vendi Score $_{\times10^6}$ & \cellcolor{ORANGE!10} 1.60 & \cellcolor{BLUE!40} 3.09 & \cellcolor{BLUE!10} 2.60 & \cellcolor{BLUE!60} 7.15 & \cellcolor{ORANGE!20} 1.41 & \cellcolor{BLUE!50} 3.36 & \cellcolor{BLUE!20} 2.65 & \cellcolor{BLUE!0} 1.89 & \cellcolor{BLUE!30} 3.04 & \cellcolor{ORANGE!30} 0.20 \\
    \textbf{NovelSum (Ours)}  & \cellcolor{BLUE!40} 0.440 & \cellcolor{BLUE!60} 0.505 & \cellcolor{BLUE!20} 0.403 & \cellcolor{BLUE!50} 0.495 & \cellcolor{BLUE!30} 0.408 & \cellcolor{BLUE!10} 0.392 & \cellcolor{BLUE!0} 0.349 & \cellcolor{ORANGE!10} 0.336 & \cellcolor{ORANGE!20} 0.320 & \cellcolor{ORANGE!30} 0.309 \\
    \midrule
    \textbf{Model Performance} & \cellcolor{BLUE!30} 1.06 & \cellcolor{BLUE!60} 1.45 & \cellcolor{BLUE!40} 1.23 & \cellcolor{BLUE!50} 1.35 & \cellcolor{BLUE!20} 0.87 & \cellcolor{BLUE!10} 0.07 & \cellcolor{BLUE!0} -0.08 & \cellcolor{ORANGE!10} -0.38 & \cellcolor{ORANGE!30} -0.49 & \cellcolor{ORANGE!20} -0.43 \\
    \bottomrule
    \end{tabular}
    }
    \caption{Measuring the diversity of datasets selected by different strategies using \textit{NovelSum} and baseline metrics. Fine-tuned model performances (Eq. \ref{eq:perf}), based on MT-bench and AlpacaEval, are also included for cross reference. Darker \colorbox{BLUE!60}{blue} shades indicate higher values for each metric, while darker \colorbox{ORANGE!30}{orange} shades indicate lower values. While data selection strategies vary in performance on LLaMA-3-8B and Qwen-2.5-7B, \textit{NovelSum} consistently shows a stronger correlation with model performance than other metrics. More results are provided in Appendix \ref{app:results}.}
    \label{tbl:main}
    \vspace{-4mm}
\end{table*}


\begin{table}[t!]
\centering
\resizebox{\linewidth}{!}{
\begin{tabular}{lcccc}
\toprule
\multirow{2}*{\textbf{Diversity Metrics}} & \multicolumn{3}{c}{\textbf{LLaMA}} & \textbf{Qwen}\\
\cmidrule(lr){2-4} \cmidrule(lr){5-5} 
& \textbf{Pearson} & \textbf{Spearman} & \textbf{Avg.} & \textbf{Avg.} \\
\midrule
TTR & -0.38 & -0.16 & -0.27 & -0.30 \\
vocd-D & -0.43 & -0.17 & -0.30 & -0.31 \\
\midrule
Facility Loc. & 0.86 & 0.69 & 0.77 & 0.08 \\
Entropy & 0.93 & 0.80 & 0.86 & 0.63 \\
\midrule
LDD & 0.61 & 0.75 & 0.68 & 0.60 \\
KNN Distance & 0.59 & 0.80 & 0.70 & 0.67 \\
DistSum$_{cosine}$ & 0.85 & 0.67 & 0.76 & 0.51 \\
Vendi Score & 0.70 & 0.85 & 0.78 & 0.60 \\
DistSum$_{L2}$ & 0.86 & 0.76 & 0.81 & 0.51 \\
Cluster Inertia & 0.81 & 0.85 & 0.83 & 0.76 \\
Radius & 0.87 & 0.81 & 0.84 & 0.48 \\
\midrule
NovelSum & \textbf{0.98} & \textbf{0.95} & \textbf{0.97} & \textbf{0.90} \\
\bottomrule
\end{tabular}
}
\caption{Correlations between different metrics and model performance on LLaMA-3-8B and Qwen-2.5-7B.  “Avg.” denotes the average correlation (Eq. \ref{eq:cor}).}
\label{tbl:correlations}
\vspace{-2mm}
\end{table}

\paragraph{\textit{NovelSum} consistently achieves state-of-the-art correlation with model performance across various data selection strategies, backbone LLMs, and correlation measures.}
Table \ref{tbl:main} presents diversity measurement results on datasets constructed by mainstream data selection methods (based on $\mathcal{X}^{all}$), random selection from various sources, and duplicated samples (with only $m=100$ unique samples). 
Results from multiple runs are averaged for each strategy.
Although these strategies yield varying performance rankings across base models, \textit{NovelSum} consistently tracks changes in IT performance by accurately measuring dataset diversity. For instance, K-means achieves the best performance on LLaMA with the highest NovelSum score, while K-Center-Greedy excels on Qwen, also correlating with the highest NovelSum. Table \ref{tbl:correlations} shows the correlation coefficients between various metrics and model performance for both LLaMA and Qwen experiments, where \textit{NovelSum} achieves state-of-the-art correlation across different models and measures.

\paragraph{\textit{NovelSum} can provide valuable guidance for data engineering practices.}
As a reliable indicator of data diversity, \textit{NovelSum} can assess diversity at both the dataset and sample levels, directly guiding data selection and construction decisions. For example, Table \ref{tbl:main} shows that the combined data source $\mathcal{X}^{all}$ is a better choice for sampling diverse IT data than other sources. Moreover, \textit{NovelSum} can offer insights through comparative analyses, such as: (1) ShareGPT, which collects data from real internet users, exhibits greater diversity than Dolly, which relies on company employees, suggesting that IT samples from diverse sources enhance dataset diversity \cite{wang2024diversity-logD}; (2) In LLaMA experiments, random selection can outperform some mainstream strategies, aligning with prior work \cite{xia2024rethinking,diddee2024chasing}, highlighting gaps in current data selection methods for optimizing diversity.



\subsection{Ablation Study}


\textit{NovelSum} involves several flexible hyperparameters and variations. In our main experiments, \textit{NovelSum} uses cosine distance to compute $d(x_i, x_j)$ in Eq. \ref{eq:dad}. We set $\alpha = 1$, $\beta = 0.5$, and $K = 10$ nearest neighbors in Eq. \ref{eq:pws} and \ref{eq:dad}. Here, we conduct an ablation study to investigate the impact of these settings based on LLaMA-3-8B.

\begin{table}[ht!]
\centering
\resizebox{\linewidth}{!}{
\begin{tabular}{lccc}
\toprule
\textbf{Variants} & \textbf{Pearson} & \textbf{Spearman} & \textbf{Avg.} \\
\midrule
NovelSum & 0.98 & 0.96 & 0.97 \\
\midrule
\hspace{0.10cm} - Use $L2$ distance & 0.97 & 0.83 & 0.90\textsubscript{↓ 0.08} \\
\hspace{0.10cm} - $K=20$ & 0.98 & 0.96 & 0.97\textsubscript{↓ 0.00} \\
\hspace{0.10cm} - $\alpha=0$ (w/o proximity) & 0.79 & 0.31 & 0.55\textsubscript{↓ 0.42} \\
\hspace{0.10cm} - $\alpha=2$ & 0.73 & 0.88 & 0.81\textsubscript{↓ 0.16} \\
\hspace{0.10cm} - $\beta=0$ (w/o density) & 0.92 & 0.89 & 0.91\textsubscript{↓ 0.07} \\
\hspace{0.10cm} - $\beta=1$ & 0.90 & 0.62 & 0.76\textsubscript{↓ 0.21} \\
\bottomrule
\end{tabular}
}
\caption{Ablation Study for \textit{NovelSum}.}
\label{tbl:ablation}
\vspace{-2mm}
\end{table}

In Table \ref{tbl:ablation}, $\alpha=0$ removes the proximity weights, and $\beta=0$ eliminates the density multiplier. We observe that both $\alpha=0$ and $\beta=0$ significantly weaken the correlation, validating the benefits of the proximity-weighted sum and density-aware distance. Additionally, improper values for $\alpha$ and $\beta$ greatly reduce the metric's reliability, highlighting that \textit{NovelSum} strikes a delicate balance between distances and distribution. Replacing cosine distance with Euclidean distance and using more neighbors for density approximation have minimal impact, particularly on Pearson's correlation, demonstrating \textit{NovelSum}'s robustness to different distance measures.






\section{Conclusion}
In this work, we propose a simple yet effective approach, called SMILE, for graph few-shot learning with fewer tasks. Specifically, we introduce a novel dual-level mixup strategy, including within-task and across-task mixup, for enriching the diversity of nodes within each task and the diversity of tasks. Also, we incorporate the degree-based prior information to learn expressive node embeddings. Theoretically, we prove that SMILE effectively enhances the model's generalization performance. Empirically, we conduct extensive experiments on multiple benchmarks and the results suggest that SMILE significantly outperforms other baselines, including both in-domain and cross-domain few-shot settings.

\bibliographystyle{splncs04}
\bibliography{ref}

\end{document}
