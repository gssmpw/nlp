\section{Related Work}

\subsection{Graph Neural Networks}

Graph Neural Networks (GNNs) have emerged as powerful tools for learning representations of graph-structured data \cite{GCN,GAT,nips24sun}. Neighbor-based heterogeneous GNNs \cite{RGCN,HetGNN} generally capture the relational information from nodes of different types based on the connectivity of the graph. Path-based heterogeneous GNNs \cite{GTN,HGT,HAN,MAGNN}, on the other hand, focus on capturing high-order semantic information to learn node representations under the guidance of meta-paths.

\vspace{-2mm}

\subsection{Graph Pre-training}

Graph pre-training aims to mine graph information for downstream tasks in an unsupervised manner \cite{www25sun}. Recently, graph contrastive learning (GCL) \cite{GraphCL,GCC,GCA,aaai22selfMG,aaai23sun} has shown competitive performance in graph pre-training. The principle of GCL is to maximize the mutual information between positive sample pairs. For example, DMGI \cite{DMGI} maximizes the mutual information between local patches of a graph and the global representation of the entire graph. HeCo \cite{HeCo} introduces both network-schema and meta-path views for cross-view contrastive learning. SHGP \cite{SHGP} generates pseudo labels serving as self-supervised signals to guide node learning.

\vspace{-2mm}

\subsection{Prompt Learning}

As a powerful paradigm, prompt learning has attracted increasing attention with its flexibility and effectiveness. GPPT \cite{GPPT} introduces a token pair consisting of candidate label class and node entity, which reformulates the node classification task as edge prediction. Methods \cite{GPF,VNT-GPPE} inject feature prompts into node features, which can be optimized with few-shot labels for specific downstream tasks. To further relieve the difficulties of transferring prior knowledge to downstream domains, methods \cite{All,GraphPrompt} unify different downstream tasks as graph-level tasks with learnable prompts. HetGPT \cite{HetGPT} integrates a virtual class prompt and a heterogeneous feature prompt for heterogeneous graph learning. Further, HGPrompt \cite{HGPrompt} designs a unified graph prompt learning method for both heterogeneous and homogeneous graphs with dual templates and dual prompts.