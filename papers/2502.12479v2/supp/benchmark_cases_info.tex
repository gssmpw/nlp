\paragraph{Visualization:} We divide the benchmark set into three groups: Group 1 for single-segment motifs, Group 2 for double-segment motifs and Group 3 for multiple-segment motifs, with 10 problems for each group. We visualize the motif problems in \Cref{fig:group_1_problems,fig:group_2_problems,fig:group_3_problems}.

\paragraph{Example motif specification PDB:}  We provide an example motif specification in \Cref{fig:example_motif}.


\clearpage % Ensure the tcolorbox content is on its own page
\pagebreak

{
\begin{figure}[ht]
    \centering
    \includegraphics[width=1.0\textwidth]{resources/motif_group_1_fig.png} % Load the PDF
    \caption{Visualization of single-segment motifs in PyMol.  Colored by segment.  Side-chain atoms are shown only for positions for which amino acid type is fixed in the problem specification. ``\dag''  denotes the problem has a variation case with a different specified scaffold length.}
    \label{fig:group_1_problems} % Label for referencing
\end{figure}
}

\clearpage % Ensure the tcolorbox content is on its own page
\pagebreak

{
\begin{figure}[ht]
    \centering
    \includegraphics[width=1.0\textwidth]{resources/motif_group_2_fig.png} % Load the PDF
    \caption{Visualization of double-segment motifs in PyMol. Colored by segment.  Side-chain atoms are shown only for positions for which amino acid type is fixed in the problem specification. ``\dag''  denotes the problem has a variation case with a different specified scaffold length. ``\ddag'' denotes the problem has a variation case with different specified motif segments from the same reference protein.}
    \label{fig:group_2_problems} % Label for referencing
\end{figure}
}

\clearpage % Ensure the tcolorbox content is on its own page
\pagebreak

{
\begin{figure}[ht]
    \centering
    \includegraphics[width=1.0\textwidth]{resources/motif_group_3_fig.png} % Load the PDF
    \caption{Visualization of multiple-segment motifs in PyMol. Colored by segment.  Side-chain atoms are shown only for positions for which amino acid type is fixed in the problem specification. ``\ddag'' denotes the problem has a variation case with different specified motif segments from the same reference protein.}
    \label{fig:group_3_problems} % Label for referencing
\end{figure}
}

\clearpage % Ensure the tcolorbox content is on its own page
\pagebreak

{
\input{resources/example_motif_4XOJ.tex}
}
\clearpage % Ensure the tcolorbox content is on its own page
\pagebreak
