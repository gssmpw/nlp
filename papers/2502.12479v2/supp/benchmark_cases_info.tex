\paragraph{Visualization:} We divide the benchmark set into three groups: Group 1 for single-segment motifs, Group 2 for double-segment motifs and Group 3 for multiple-segment motifs, with 10 problems for each group. We visualize the motif problems in \Cref{fig:group_1_problems,fig:group_2_problems,fig:group_3_problems}.

\paragraph{Example motif specification PDB:}  We provide an example motif specification in \Cref{fig:example_motif}.


\clearpage % Ensure the tcolorbox content is on its own page
\pagebreak

{
\begin{figure}[ht]
    \centering
    \includegraphics[width=1.0\textwidth]{resources/motif_group_1_fig.png} % Load the PDF
    \caption{Visualization of single-segment motifs in PyMol.  Colored by segment.  Side-chain atoms are shown only for positions for which amino acid type is fixed in the problem specification. ``\dag''  denotes the problem has a variation case with a different specified scaffold length.}
    \label{fig:group_1_problems} % Label for referencing
\end{figure}
}

\clearpage % Ensure the tcolorbox content is on its own page
\pagebreak

{
\begin{figure}[ht]
    \centering
    \includegraphics[width=1.0\textwidth]{resources/motif_group_2_fig.png} % Load the PDF
    \caption{Visualization of double-segment motifs in PyMol. Colored by segment.  Side-chain atoms are shown only for positions for which amino acid type is fixed in the problem specification. ``\dag''  denotes the problem has a variation case with a different specified scaffold length. ``\ddag'' denotes the problem has a variation case with different specified motif segments from the same reference protein.}
    \label{fig:group_2_problems} % Label for referencing
\end{figure}
}

\clearpage % Ensure the tcolorbox content is on its own page
\pagebreak

{
\begin{figure}[ht]
    \centering
    \includegraphics[width=1.0\textwidth]{resources/motif_group_3_fig.png} % Load the PDF
    \caption{Visualization of multiple-segment motifs in PyMol. Colored by segment.  Side-chain atoms are shown only for positions for which amino acid type is fixed in the problem specification. ``\ddag'' denotes the problem has a variation case with different specified motif segments from the same reference protein.}
    \label{fig:group_3_problems} % Label for referencing
\end{figure}
}

\clearpage % Ensure the tcolorbox content is on its own page
\pagebreak

{

\begin{figure}
    \centering
\begin{tcolorbox}[
  title={},
  fonttitle=\bfseries,       % Bold caption text
  colback=white,             % Background color
  colframe=black,            % Border color
  coltitle=white,            % Title text color
  sharp corners,             % Square corners
  boxrule=0.5mm,             % Thickness of the border
  toptitle=0.2mm,              % Space above the title
  bottomtitle=1mm,           % Space below the title
  titlerule=1mm              % Remove the title bar rule
]

\small % Reduce font size
\begin{verbatim}
REMARK 1 Reference PDB ID: 4XOJ
REMARK 2 Motif Segment Placement in Reference PDB: 39;A;43;B;90;C;46
REMARK 3 Length for Designed Scaffolds: 150
ATOM      1  N   HIS A   1      -2.924  -3.724   2.088  1.00  0.00           N  
ATOM      2  CA  HIS A   1      -1.871  -3.781   3.096  1.00  0.00           C  
ATOM      3  C   HIS A   1      -1.802  -2.517   3.945  1.00  0.00           C  
ATOM      4  O   HIS A   1      -1.071  -2.501   4.936  1.00  0.00           O  
ATOM      5  CB  HIS A   1      -0.502  -4.109   2.485  1.00  0.00           C  
ATOM      6  CG  HIS A   1       0.119  -2.995   1.722  1.00  0.00           C  
ATOM      7  ND1 HIS A   1       0.033  -2.839   0.365  1.00  0.00           N  
ATOM      8  CD2 HIS A   1       0.846  -1.950   2.150  1.00  0.00           C  
ATOM      9  CE1 HIS A   1       0.703  -1.723   0.052  1.00  0.00           C  
ATOM     10  NE2 HIS A   1       1.215  -1.148   1.102  1.00  0.00           N  
TER
ATOM     11  N   ASP B   1      -4.487  -6.677  -3.743  1.00  0.00           N  
ATOM     12  CA  ASP B   1      -4.063  -5.623  -2.793  1.00  0.00           C  
ATOM     13  C   ASP B   1      -4.922  -4.362  -3.009  1.00  0.00           C  
ATOM     14  O   ASP B   1      -4.495  -3.371  -3.582  1.00  0.00           O  
ATOM     15  CB  ASP B   1      -2.583  -5.335  -2.926  1.00  0.00           C  
ATOM     16  CG  ASP B   1      -2.042  -4.468  -1.808  1.00  0.00           C  
ATOM     17  OD1 ASP B   1      -2.752  -4.286  -0.791  1.00  0.00           O  
ATOM     18  OD2 ASP B   1      -0.880  -4.003  -1.955  1.00  0.00           O  
TER
ATOM     19  N   GLY C   1       5.361   5.214   1.901  1.00  0.00           N  
ATOM     20  CA  GLY C   1       4.400   6.252   2.188  1.00  0.00           C  
ATOM     21  C   GLY C   1       3.501   6.577   1.006  1.00  0.00           C  
ATOM     22  O   GLY C   1       2.479   7.251   1.188  1.00  0.00           O  
ATOM     23  N   UNK C   2       3.821   6.075  -0.188  1.00  0.00           N  
ATOM     24  CA  UNK C   2       2.965   6.247  -1.351  1.00  0.00           C  
ATOM     25  C   UNK C   2       1.905   5.154  -1.494  1.00  0.00           C  
ATOM     26  O   UNK C   2       0.929   5.369  -2.228  1.00  0.00           O  
ATOM     27  N   SER C   3       2.100   4.015  -0.827  1.00  0.00           N  
ATOM     28  CA  SER C   3       1.156   2.896  -0.909  1.00  0.00           C  
ATOM     29  C   SER C   3      -0.271   3.352  -0.782  1.00  0.00           C  
ATOM     30  O   SER C   3      -0.604   4.170   0.069  1.00  0.00           O  
ATOM     31  CB  SER C   3       1.374   1.891   0.223  1.00  0.00           C  
ATOM     32  OG  SER C   3       2.357   0.946  -0.128  1.00  0.00           O  
TER
END   
\end{verbatim}
\end{tcolorbox}
    \caption{Example motif specification in PDB format (problem 27, 4XOJ).  }
    \label{fig:example_motif}
\end{figure}
}
\clearpage % Ensure the tcolorbox content is on its own page
\pagebreak
