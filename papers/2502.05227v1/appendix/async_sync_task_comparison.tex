We perform a closer analysis of the difficulty between the synchronous and asynchronous datasets by adapting the first 3 tasks of the asynchronous dataset to synchronous variants. We do this by making cooking instant by setting the time delay to 0. The results in Table~\ref{tab:dataset-comparison} demonstrate that \react{} \gptfo{} performs better in the synchronous variants compared to the asynchronous setting.

\begin{table}[h]
    \small
    \centering
    \begin{tabular}{lcc}
        \toprule
        & \textbf{Synchronous (\%)} & \textbf{Asynchronous (\%)} \\  
        \midrule
        $\hyperref[fig:0_async]{[1 ]}$ \includegraphics[width=1.5cm]{assets/task_specific_assets_expanded_svg/0_async.pdf} & 50.0 & 20.0 \\
        $\hyperref[fig:1_async]{[2 ]}$ \includegraphics[width=1.5cm]{assets/task_specific_assets_expanded_svg/1_async.pdf} & 60.0 & 30.0 \\
        $\hyperref[fig:2_async]{[3 ]}$ \includegraphics[width=2cm]{assets/task_specific_assets_expanded_svg/2_async.pdf} & 50.0 & 40.0 \\
        \bottomrule
    \end{tabular}
    \caption{ReAct \gptfo{} performance on 3 synchronous variant tasks for comparison.}
    \label{tab:dataset-comparison}
\end{table}