%\begin{figure}[ht]
    \centering
    % Left column (longest JSON)
    \noindent
    \begin{minipage}[t]{0.48\textwidth}  % Left column (long JSON)
        \vspace{0pt}  % Align at the top of the column
        \begin{tcolorbox}[colframe=black, boxrule=0.5mm, width=\textwidth, boxsep=0pt, left=0pt, right=0pt, top=0pt, bottom=0pt]
        \begin{minted}[fontsize=\scriptsize, breaklines, frame=none, bgcolor=white,
            xleftmargin=0pt, xrightmargin=0pt, texcomments=true, escapeinside=||,
            highlightlines=true, style=colorful, 
            highlightcolor=\color{red}, highlightcolor=\color{green}] {json}
{
    "version": "1.0.0",

    "player": {
        "robot": {
            "front": "robot_front.png",
            "back": "robot_back.png",
            "left": "robot_left.png",
            "right": "robot_right.png"
        }
    },

    "floor": "floorkitchen.png",

    "item": {
        "constants": {
            "STATION_ITEM_OFFSET" : 0.25,
            "X_SCALE_FACTOR": 0.125,
            "Y_SCALE_FACTOR": 0.75
        },
        "entities": {
            "chicken": {
                "assets": {
                    "default": "chicken.png",
                    "cooked": {
                        "asset": "cookedchicken.png",
                        "predicates": ["iscooked"]
                    },
                    "fried": {
                        "asset": "friedchicken.png",
                        "predicates": ["isfried"]
                    }
                },
                "constants": {}
            }, ...
        }
    },

    "station": {
        "constants": {},
        "entities": {
            "fryer": {
                "assets": {
                    "default": "fryer.png"
                },
                "constants": {}
            },...
        }
    },
}
        \end{minted}
        \end{tcolorbox}
    \end{minipage}
    \hfill
    % Right column (stacked two JSONs)
    \begin{minipage}[t]{0.48\textwidth}  % Right column
        \vspace{0pt}
        % First stacked JSON
        \begin{tcolorbox}[colframe=black, boxrule=0.5mm, width=\textwidth, boxsep=0pt, left=0pt, right=0pt, top=0pt, bottom=0pt]
        \begin{minted}[fontsize=\scriptsize, breaklines, frame=none, bgcolor=white,
            xleftmargin=0pt, xrightmargin=0pt, texcomments=true, escapeinside=||,
            highlightlines=true, style=colorful, 
            highlightcolor=\color{red}, highlightcolor=\color{green}] {json}
{
    "version": "1.0.0",
    "width": 3,
    "height": 3,
    "config": {
        "num_cuts": {
            "lettuce": 3,
            "default": 3
        },
        "cook_time": {
            "patty": 3,
            "default": 3
        }
    },
    "stations": [
        {
            "name": "board",
            "x": 0,
            "y": 1,
            "id": "A"
        }
    ],
    "items": [
        {
            "name": "lettuce",
            "x": 0,
            "y": 1,
            "stack-level": 0,
            "predicates": ["iscuttable"],
            "id": "a"
        }
    ],
    "players": [
        {
            "name": "robot",
            "x": 0,
            "y": 0,
            "direction": [0, 1]
        }
    ],
    "goal_description": "Cut the lettuce on the board until it is cut",
    "goal": [
        {
            "predicate": "iscut",
            "args": ["lettuce"],
            "ids": ["a"]
        }
    ]
}
        \end{minted}
        \end{tcolorbox}

        \vspace{1em}  % Space between stacked JSONs

        % Second stacked JSON
        \begin{tcolorbox}[colframe=black, boxrule=0.5mm, width=\textwidth, boxsep=0pt, left=0pt, right=0pt, top=0pt, bottom=0pt]
        \begin{minted}[fontsize=\scriptsize, breaklines, frame=none, bgcolor=white,
            xleftmargin=0pt, xrightmargin=0pt, texcomments=true, escapeinside=||,
            highlightlines=true, style=colorful, 
            highlightcolor=\color{red}, highlightcolor=\color{green}] {json}
{
    "mouse_click_actions": [
        {
            "name": "move",
            "input_instructions": {
                "button": "left",
                "click_on": "s2"
            }
        },...
    ],
    "keyboard_actions": [
        {
            "name": "cook",
            "input_instructions": {
                "key": "e",
                "at": "s1"
            }
        },...
    ]
}
        \end{minted}
        \end{tcolorbox}
    \end{minipage}
    
    \caption{Combined JSONs: (Left) Rendering JSON, (Top-right) Environment JSON, (Bottom-right) Input JSON.}
    \label{fig:combined_jsons}
\end{figure}

To provide flexibility in task and environment creation, an environment JSON (Figure~\ref{fig:environment_json_example}) is used to define the problem. The size of the grid used can be specified, and positions of objects in the item can be specified using coordinates. Predicates that are specific to an item can also be specified. In conjunction with the flexible goal creation described in Section 2, objects in the environment can be given specific ids, if the goal must be satisfied for specific objects. Additionally, if the environment requires a different number of cuts to complete cutting, or a different cook time, these values can be configured in the JSON. 
\begin{figure}[t!]
    \centering
    \noindent
    \includegraphics[width=0.8\textwidth]{assets/JSONs/env_json.png}
    \caption{Environment JSON for a lettuce cutting task.}
    \label{fig:environment_json_example}
\end{figure}

Adding objects to the environment is also simple. To add a new object, the necessary predicates for that object can be added to the domain JSON, and its corresponding image can be added to the rendering JSON (Figure~\ref{fig:rendering_json_example}). If there are different images for the object depending on the predicates that are true in the environment, these can also be specified. The images can also be scaled or offset using the rendering JSON. 
\begin{figure}[t!]
    \centering
    \noindent
    \includegraphics[width=0.96\textwidth]{assets/JSONs/rendering_json.png}
    \caption{Rendering JSON}
    \label{fig:rendering_json_example}
\end{figure}

To specify what button to press for each action, we use an input JSON (Figure~\ref{fig:input_json_example}). If the action requires a mouse click, we can specify where the player needs to click to perform the action. If the action requires a key press, we specify which button to press for which action, and where the player needs to be to perform the action. 
\begin{figure}[t!]
    \centering
    \noindent
    \includegraphics[width=0.96\textwidth]{assets/JSONs/input_json.png}
    \caption{Input JSON}
    \label{fig:input_json_example}
\end{figure}

