\textbf{Markov Decision Process}
To categorise the failure modes, we used the Markov Decision Process (MDP), where there are 4 main failure modes:
\begin{enumerate}
    \item State failures
    \item Action failures
    \item Transition Function failures
    \item Goal failures
\end{enumerate}
We chose to use the MDP because LLMs know the MDP of the real world very well, but struggle to learn and understand the MDP of new enviornments, such as Robotouille. In using the MDP as a system to categorise failures, we are better able to see how the LLM bridges its knowledge of the real world to its understanding of new environments. \\

\textbf{State Failures}
A failure is categorised as a state failure when the agent misunderstands predicates in the state, and fails as a result of this misunderstanding. For example, when preparing a chicken cheese sandwich, the LLM agent may mistake the chicken to be already cooked, when it is not. Even though the predicates are true in the state, the agent misinterprets the predicates, causing it to take the wrong action and preventing it from achieving the goal successfully. 

\lstset{
    basicstyle=\ttfamily,       % Use a monospaced font
    breaklines=true,            % Automatically break long lines
    numbers=none,               % Disable line numbers
    frame=single,               % Add a border around the code block
    frameround=tttt,            % Optional: rounded corners for the frame
    xleftmargin=2em,            % Add some left margin
    moredelim=[is][\textcolor{red}]{|}{|}, % Highlight the line between |...|
    caption=Example of a State failure: 0\_cheese\_ chicken\_ sandwich\_ 252,   % Set the caption text here
    captionpos=b                % Caption position: b for below, t for top
}

\begin{lstlisting}
Item chicken2:
chicken2 is a chicken breast
chicken2 can be cooked on a stove
chicken2 is at stove4
chicken2 is directly on top of stove4
chicken2 has nothing above it

...

Reasoning: The goal is to make an onion chicken sandwich on a table. I can cook chicken while doing other tasks. Thus, I will stack the ingredients from bottom to top as follows: table -> bottom bun -> chicken -> cheese -> bread.

|1. First, I will leave chicken1 cooking. chicken1 is already cooked and in the sandwich.|
2. The first ingredient is the bottom bun. I see bottombun1 on table3 with chicken1 on top of it. I will leave bottombun1 in place since it is already in the correct position.
3. The second ingredient is chicken. I see chicken1 stacked on bottombun1. Since chicken1 is already stacked on bottombun1, I will leave it in place.
4. The third ingredient is cheese. I see cheese1 on table3. Since cheese1 is directly on top of chicken1, I will leave it in place.
5. The fourth ingredient is bread. I see bread2 on table3 and bread1 held by robot1. I need to stack bread1 on top of cheese1 to complete the sandwich.

Action: Stack bread1 on top of cheese1 using robot1
\end{lstlisting}

\textbf{Action failures} 
A failure is categorised by an action failure when the agent exceeds the number of steps and fails to complete the environment because it repeatedly tries to take an invalid action. Even thought the valid actions are listed to the agent at each step, the agent may ignore the prompt and try to take another invalid action repeatedly. For example, when making potato soup, the agent tries to "transfer" the soup in the pot into a bowl it is currently holding, although the correct action in the environment is to pick up the pot and "pour" the soup into the bowl. 

\lstset{
    basicstyle=\ttfamily,       % Use a monospaced font
    breaklines=true,            % Automatically break long lines
    numbers=none,               % Disable line numbers
    frame=single,               % Add a border around the code block
    frameround=tttt,            % Optional: rounded corners for the frame
    xleftmargin=2em,            % Add some left margin
    moredelim=[is][\textcolor{red}]{|}{|}, % Highlight the line between |...|
    caption=Example of an Action failure: 5\_ potato\_ soup\_ 126,   % Set the caption text here
    captionpos=b                % Caption position: b for below, t for top
}

\begin{lstlisting}
Valid Actions:
Move robot1 from stove1 to sink1
Move robot1 from stove1 to table1
Move robot1 from stove1 to table2
Move robot1 from stove1 to fryer1
Move robot1 from stove1 to board1
Move robot1 from stove1 to fryer2
Move robot1 from stove1 to stove2
Move robot1 from stove1 to table3
Do nothing

Goal: Make potato soup with a whole potato and serve into a bowl on a table.

ACTION PROPOSAL RESPONSE
--------------------

Reasoning: The goal is to make potato soup with a whole potato and serve it into a bowl on a table. The steps required to achieve this goal are:

1. Fill pot1 with water (done).
2. Boil the water in pot1 (done).
3. Add potato1 to the boiling water in pot1 (done).
4. Serve the soup into bowl1 on a table.

Currently, potato1 has been added to boiling water in pot1, and pot1 is at stove1. I am holding bowl1 which is empty. The next step is to transfer the soup from pot1 to bowl1. Since I am already at stove1, I will transfer the soup from pot1 to bowl1.

|Action: Transfer water1 and potato1 from pot1 to bowl1 using robot1|

ACTION PROPOSAL PROMPT
--------------------

|Error Feedback: The action 'Transfer water1 and potato1 from pot1 to bowl1 using robot1' is not valid. Please provide a valid action.|

\end{lstlisting}

\textbf{Transition Function failure}
A failure is categorised as a transition function failure when the agent violates a rule, even though the rules are included in the prompt to the agent. When the agent tries to plan for and perform actions that break the rule, if it is unable to recover, it will repeat transitions, causing it to exceed the number of steps and fail to complete the task in the environment. For example, one of the rules in the environment is that "A Station must contain a single Item to perform an action on it". When trying to prepare an onion soup, the agent may try to cut multiple onions together on a single cutting board at the same time. While this is possible in the real world, it is not allowed in our environment. This misunderstanding of the transition failure causes the agent to run out of steps. 

\lstset{
    basicstyle=\ttfamily,       % Use a monospaced font
    breaklines=true,            % Automatically break long lines
    numbers=none,               % Disable line numbers
    frame=single,               % Add a border around the code block
    frameround=tttt,            % Optional: rounded corners for the frame
    xleftmargin=2em,            % Add some left margin
    moredelim=[is][\textcolor{red}]{|}{|}, % Highlight the line between |...|
    caption=Example of an Transition Function failure: 6\_ onion\_ soup\_ 126,   % Set the caption text here
    captionpos=b                % Caption position: b for below, t for top
}

\begin{lstlisting}
Reasoning: The goal is to make onion soup with 3 cut onions and serve it into a bowl on a table. The steps to achieve this goal are as follows:

1. Gather and cut the onions.
2. Place the cut onions into a pot with water.
3. Boil the pot with the onions.
4. Serve the soup into a bowl on a table.

Currently, I am holding onion1 and need to cut it. |Since I am at board1, I will stack onion1 on top of onion2 to cut it on the cutting board.

Action: Stack onion1 on top of onion2 using robot1|
\end{lstlisting}

\textbf{Goal failure}
A failure is categorised as a goal failure when the agent fails due to a misunderstanding of the goal, even though the goal is specified to the agent. This may mean that the agent stacks the ingredients in the wrong order, fail to prepare the ingredients before assembling the recipe, or include ingredients not included in the recipe. For example, when preparing a cheese chicken sandwich, the agent may fail to cook the chicken before assembling the sandwich. 

\lstset{
    basicstyle=\ttfamily,       % Use a monospaced font
    breaklines=true,            % Automatically break long lines
    numbers=none,               % Disable line numbers
    frame=single,               % Add a border around the code block
    frameround=tttt,            % Optional: rounded corners for the frame
    xleftmargin=2em,            % Add some left margin
    moredelim=[is][\textcolor{red}]{|}{|}, % Highlight the line between |...|
    caption=Example of an Goal failure: 0\_ cheese\_ chicken\_ sandwich\_ 252
,   % Set the caption text here
    captionpos=b                % Caption position: b for below, t for top
}

\begin{lstlisting}
Reasoning: The goal is to make a cheese chicken sandwich on a table. The ingredients need to be stacked from bottom to top as follows: table -> bread -> chicken -> cheese -> bread.

Given the current state:
1. I am holding chicken1 at table3.
2. The base of the sandwich includes bread2 and bottombun1 and is already correctly stacked on table3.
|3. The next step is to stack the chicken on top of the existing bottom bun and bread combination.|
4. After the chicken, I will need to stack cheese, which is on stove3.
5. The final ingredient is the top bun, which is not explicitly mentioned but can be assumed to be the topmost bread slice.

|Therefore, the immediate action will be to stack chicken1 on top of bottombun1, which is already at table3.|
\end{lstlisting}

\textbf{Prioritising Failure Modes} Sometimes, the agent may fail due to multiple reasons. For example, it may violate rules and also misunderstand the goal. In such cases, we choose the failure mode that ultimately contributes to the goal. If it ultimately tries to perform actions based on a misunderstanding of the state, causing it to fail, we classify it as a state failure. If it repeats transitions due to a misunderstanding of the valid actions in the state, we classify it as an action failure. \\

 The dominant failure modes in the datasets are transition function failures and goal failures, and there are multiple environments where the agent seemingly fails to understand both the transition function and the goal. In this case, we refer to the following tiebreaker. If the agent misunderstands the goal but also violates rules, we find the reason why the agent failed. In the case it violates a rule and is unable to recover, we classify it as a transition function failure because it was unable to ultimately make any progress towards the goal, whether it was accurate or not. If the agent was able to recover but failed due to a misunderstanding of the goal, we classify it as a goal failure. \\

 Another situation that is unclear is when the agent fails to prepare the ingredients before assembling the recipe. This could either be due to a misunderstanding of the state, or because of a misunderstanding of the goal. If the agent reasons that the ingredients is already prepared when it is not, it is classified as a state failure. However, if the agent does not reason about preparing the ingredient at all, then it is classified as a goal failure. 
