In this section, we annotated for the transition failures on the synchronous and asynchronous datasets whether (1) the LLM agent recovers from a failure and (2) whether it repeats its mistake after recovering.

% \GG{Su Yean - crisply define what a mistake is and isn't. there are 3 cases. (1) where it violates a rule (obviously a mistake). (2) where it recovers but repeats the mistake much later (is this the same mistake?). (3) where it repeats the same mistake at a different station (same rule, different context). 1 paragraph
% }

A mistake occurs when the agent violates a rule at a certain station for a specific action. When the agent makes a mistake, there are 4 cases:
\begin{enumerate}
    \item The agent violates a rule and is unable to recover
    \item The agent violates a rule at a station for a specific action, but is able to recover. After recovery, they do not make any more mistakes; they do not repeat the mistake after recovering. 
    \item The agent violates a rule at a station, recovers, but is later repeats the mistake by trying to perform the same action at the same type of station. In this case, they repeat the mistake after recovering. 
    \item The agent violates a rule at a station, recovers, and does not repeat the mistake by trying to violate the same rule for the same action at the same type of action. However, they violate the same rule for a different action at a different type of station. In this case, we say that they do not repeat their mistake. 
\end{enumerate}

On the synchronous dataset, the transition failures account for 32.1\% (17) of the total failures. Of these failures, 58.8\% (10) recovered from the mistake. Of the failures that recovered from their mistake, 90\% (9) did not repeat the same mistake.

On the asynchronous dataset, the transition failures account for 58.5\% (52) of the total failures. Of these failures, 40.4\% (21) recovered from the mistake. Of the failures that recovered from their mistake, 57.1\% (12) did not repeat the same mistake.

% \GG{Su Yean - put concise point 2; just clarify that despite recovering a failure can still be a transition failure if the recovery process took too long and exhausted step limit.}

In the case where the agent is able to recover from a mistake, the agent may still fail to complete the task because they recovery process took too long and exhausted the step limit. Then, this failure would be categorised as a Transition Function failure. 

% \GG{Su Yean can you provide some specifics on how you annotated this?}

% \GG{(1) does repeating its mistake after recovering meaning the exact same mistake or the same kind of mistake? so if i and failing to put lettuce on a cutting board cause something is fixing it, then I recover from this, but then I accidentally leave something else on it and start being dumb again, is this considered repeating the mistake because its the same kind of mistake?}

% \GG{(2) so the LLM agent manages to recover from its mistake and it doesn't make the same mistake again. but these are transition failures - if its not repeating its mistake anymore but you still classified it as transition failure then what led to the failure? did it take too long to get to the goal because of the repeated failures? did it recover from the mistake you annotated but come across another one later? in this case, then which mistake did you annotate?}