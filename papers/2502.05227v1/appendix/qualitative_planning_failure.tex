To gain more insight into why LLM agents are unable to efficiently complete asynchronous tasks successfully, we created a new baseline by repeating the rules to the agent before every action. However, this caused the agent to sometimes be even more inefficient, as it naively tries to follow rules without proper planning. \\

In this example, the agent needs to cook a chicken on a stove, but all the stoves are currently occupied. It first follows the rule "A Station must contain a single Item to perform an action on it" and picks up the item from the stove. Then, it follows the rule "A Player can only hold a single Item at a time". Since it is currently holding the item it does not need, and it wants to cook the chicken, it places the item back onto the stove to free its hands. This causes the agent to repeatedly pick up the item from the stove and place it back onto the stove. (Fig.~\ref{fig:qualitative_planning_fail})

\begin{figure}[t!]
    \centering
    \includegraphics[width=0.5\textwidth]{assets/env_figures_svg/qualitative_planning_failure.pdf}
    \caption{0\_ cheese\_ chicken\_ sandwich\_ 84: All stoves are occupied, and agent naively follows rules, causing it to repeat transitions}
    \label{fig:qualitative_planning_fail}
\end{figure}

\lstset{
    basicstyle=\ttfamily,       % Use a monospaced font
    breaklines=true,            % Automatically break long lines
    numbers=none,               % Disable line numbers
    frame=single,               % Add a border around the code block
    frameround=tttt,            % Optional: rounded corners for the frame
    xleftmargin=2em,            % Add some left margin
    moredelim=[is][\textcolor{red}]{|}{|}, % Highlight the line between |...|
    caption= Log for 0\_ cheese\_ chicken\_ sandwich\_ 84,   % Set the caption text here
    captionpos=b                % Caption position: b for below, t for top
}

\begin{lstlisting}
Reasoning: The goal is to make a cheese chicken sandwich on a table. The sandwich will be stacked in the following order: table -> bread -> chicken -> cheese -> bread.

1. The first ingredient, bread1, is already on table2 and will serve as the base for the sandwich. I will leave it in place.

2. The second ingredient is chicken. I see chicken1 on table3. I must cook chicken1 before stacking it on bread1. |stove4 currently has bread3 on it, which is not needed for the sandwich. I will pick up bread3 from stove4 to make space for cooking chicken1.|

Action: Pick up bread3 from stove4 using robot1

...

Reasoning: The goal is to prepare a cheese chicken sandwich on a table. To achieve this, I need to stack the ingredients from bottom to top as follows: table -> bread -> chicken -> cheese -> bread.

1. The first ingredient, bread1, is already on table2 and will serve as the base for the sandwich. I will leave it in place.

2. The second ingredient is chicken. I see chicken1 on table3. |I must cook chicken1 before stacking it on bread1. stove4 is currently empty, so I will place bread3 there to free up my hands.|

Action: Place bread3 on stove4 using robot1
\end{lstlisting}