\vspace{-1em}
\section{Experiments}
\vspace{-1em}

\subsection{Baselines}

We evaluate LLMs on \robotouille{} using the following baselines: \io{}, \iocot{}, and \react{}. \io{} takes as input the initial state, including valid actions and goal, and outputs an plan directly. \iocot{} \citep{wei2023chainofthoughtpromptingelicitsreasoning} also takes as input the initial state but outputs a plan with chain of thought before each action that estimates the resulting state. Instead of outputting the entire plan, \react{} \citep{yao2022react} outputs reasoning and the next action given the current state, and receives the next state before repeating. 
We use an ablated version of \react{} that only keeps the reasoning and action of the previous timestep in context (along with the base prompt and in-context examples); the improved performance and cost-effectiveness is detailed in Appendix~\ref{app:react-ablations}. 
Each baseline receives a single in-context example on a training example excluded from the testing set. We use temperature 0.7 for all models. All prompts and few-shot examples are located in our codebase \href{https://github.com/portal-cornell/robotouille}{here}.
%are included in Appendix~\ref{app:prompts}.

% \textbf{Metrics} We use 4 different metrics to evaluate LLM baselines. Success rate is determined by reaching the goal within 1.5 times the optimal number of steps for the given instance. Steps to go is the optimal number of steps to reach the goal from the final state of a failure; this metric is normalized with the optimal number of steps to reach the goal from the initial state. Optimality rate is the ratio between the number of steps taken and the optimal steps to reach the goal. Repeated transitions are the number of duplicate transitions included in the plan.
% % \GG{Some formulas here would make things more concrete. Can introduce after problem formulation has consistent math}

\subsection{Results and Analysis}

% \subsubsection{Overall Results}
% \begin{itemize}[leftmargin=*]
%     \item The best baseline, \gptfo{} \react{}, only achieves 47\% on the synchronous dataset and 11\% on the asynchronous dataset. See Sec~\ref{sec:success}.
%     \item Dominant failure modes on the asynchronous dataset are similar to those in the synchronous dataset indicating that simple LLM failures are inhibiting asynchronous planning. See Sec~\ref{sec:failures}.
%     \item Further investigations on the low asynchronous performance reveal that better feedback incorporation and reliable self-verification are crucial future work directions to boost performance. See Sec~\ref{sec:follow-ups}.
% \end{itemize}

\subsubsection{Overall Takeaways}
\begin{itemize}[leftmargin=*]
    \item \textbf{Closed-loop agents are superior}: The best baseline, \gptfo{} \react{}, achieves 47\% on the synchronous dataset and 11\% on the asynchronous dataset, surpassing open-loop approaches \io{} and \iocot{} (Finding 1, Sec~\ref{sec:success}).
    \item \textbf{Poor feedback incorporation leads to decreased asynchronous performance}: Despite being closed-loop, \gptfo{} \react{} failures often make little progress towards the goal (Finding 3, Sec~\ref{sec:success}) due to poor failure recovery (Finding 5, Sec~\ref{sec:failures}). We find that boosting priors improves performance (Finding 7, Sec~\ref{sec:follow-ups}) but discuss better feedback methods in Section~\ref{dis:feedback}.
    \item \textbf{Synchronous and asynchronous failures are closely related}: Both synchronous and asynchronous failures are dominated by rule violations and goal misinterpretation (Finding 4, Sec~\ref{sec:failures}). We hypothesize that this is due to poor failure recovery (Finding 5, Sec~\ref{sec:failures}) and agents that recover efficiently could boost performance in both settings.
    \item \textbf{Task prioritization is critical in asynchronous planning}: Proper prioritization of subtasks in asynchronous settings significantly boosts performance (Finding 6, Sec~\ref{sec:follow-ups}).
\end{itemize}

\subsubsection{Success and Optimality}
\label{sec:success}

% \textbf{Question 1.} \textit{How successful are baselines?}
\textbf{Finding 1.} Closed-loop baselines outperform open-loop baselines.

Table~\ref{tab:successes} shows the success rates of various LLMs baselines on the synchronous and asynchronous datasets. Table~\ref{tab:tasks-results} shows the task-specific success rates of baselines using \gptfo{}. Success rate is determined by reaching the goal within 1.5 times the optimal number of steps for the given instance. Baselines exceeding this step limit are terminated.
% TODO: Include open-source
% \NS{and Table~\ref{tab:async-baselines}}

Among all the LLM baselines, \react{} with the \gptfo{} model performs the best on the synchronous and asynchronous datasets. \io{} performs worst for most LLMs while \iocot{} improves performance. We qualitatively observed \texttt{gemini-1.5-flash} failing with \react{} since it attempts to solve the few-shot example goal rather than the current environment goal. This is likely due to the long context examples, aligning with findings in \cite{liu2023lostmiddlelanguagemodels} where LLMs struggle with simple tasks in long contexts. We investigate PLaG (BaG) \cite{lin2024graphenhancedlargelanguagemodels} and Reflexion \cite{shinn2023reflexionlanguageagentsverbal} performance on the asynchronous dataset in Appendix~\ref{app:async-baselines} and achieved small performance improvements.
%\NS{However, it is important to note that the Gemini model notably performs poorly with \react{}. We discuss more about this in Section~\ref{app:gemini-discussion}.}

When considering task-specific success over \gptfo{} baselines, \react{} generally achieves higher performance per task. While we list the horizon length as a crude difficulty metric, it is evident that success rate is not solely dependent on it. We investigate this further in Appendix~\ref{app:horizon_complexity}. We also investigate different agent failure modes in more depth in Section~\ref{sec:failures}.

\begin{table}[!h]
    \small
    \centering
    \begin{tabular}{llllllll}
        \toprule
        & \multicolumn{3}{c}{\textbf{Synchronous (\%)}} & \multicolumn{3}{c}{\textbf{Asynchronous (\%)}} \\ 
        \cmidrule(r){2-4} \cmidrule(r){5-7}
        & \textbf{I/O} & \textbf{I/O CoT} & \textbf{ReAct} & \textbf{I/O} & \textbf{I/O CoT} & \textbf{ReAct} \\ 
        \midrule
        \gptfo{} & 4.00 & 14.0 & \textbf{47.0} & 1.00 & 1.00 & \textbf{11.0}       \\
        \texttt{gpt-4o-mini} & 4.00 & 10.0 & 11.0 & 0.00    & 1.00 & 0.00     \\
        % \texttt{gpt-4-0125-preview}   &     &               &       \\
        % \texttt{gpt-4-0613}   &     &               &       \\
        % \texttt{gpt-3.5-turbo-0125} &     &               &       \\
        \texttt{gemini-1.5-flash} & 0.00 & 13.0 & 0.00 & 0.00 & 0.00 & 0.00      \\
        % \texttt{gemini-1.5-pro} &     &               &       &     &               &       \\
        % \texttt{gemini-1.0-pro} &     &               &       \\
        % \texttt{claude-3.5-sonnet} &     &               &       &     &               &       \\
        % \texttt{claude-3-opus} &     &               &       \\
        % \texttt{claude-3-sonnet} &     &               &       \\
        \texttt{claude-3-haiku} & 1.00 & 2.00 & 2.00 & 0.00 & 0.00 & 0.00      \\
        % \midrule
        % \texttt{Meta-Llama-3.1-70b-Instruct}  &     &               &       &     &               &       \\
        % \texttt{Meta-Llama-3.1-8b-Instruct}  &     &               &       &     &               &       \\
        % \texttt{Meta-Llama-8b-Instruct}  &     &               &       &     &               &       \\
        % \texttt{Llama-2-70b-chat-hf}  &     &               &       &     &               &       \\
        % \texttt{Llama-2-13b-chat-hf}  &     &               &       &     &               &       \\
        % \texttt{Llama-2-7b-chat-hf}  &     &               &       &     &               &       \\
        % % \texttt{Qwen2-72B-Instruct}  &     &               &       &     &               &       \\
        % % \texttt{Qwen2-7B-Instruct}  &     &               &       &     &               &       \\
        % \texttt{Mixtral-8x22B-Instruct-v0.1}  &     &               &       &     &               &       \\
        % \texttt{Codestral-22B-v0.1}  &     &               &       &     &               &       \\
        % \texttt{Mamba-Codestral-7B-v0.1} &     &               &       &     &               &       \\
        % \texttt{Phi-3.5-MoE-instruct}  &     &               &       &     &               &       \\
        % \texttt{Phi-3-medium-4k-instruct}  &     &               &       &     &               &       \\
        % \texttt{Phi-3-small-128k-instruct}  &     &               &       &     &               &       \\
        % \texttt{gemma-2-27b-it} &     &               &       &     &               &       \\
        % \texttt{gemma-2-9b-it} &     &               &       &     &               &       \\
        \bottomrule
    \end{tabular}
    \caption{Success rates of state-of-the-art LLMs on the synchronous and asynchronous datasets.}
    \label{tab:successes}
\end{table}
% % Table looks weird in terms of multiples of 10

\begin{table}[!h]
    \small
    \centering
    \begin{tabular}{llll|l}
        \toprule
        & \textbf{I/O} & \textbf{I/O} & \textbf{ReAct} & \textbf{Horizon} \\
        & & \textbf{CoT} & & \textbf{Length} \\
        \midrule
        \multicolumn{5}{c}{\textbf{Synchronous (\%)}} \\
        \midrule
        $\hyperref[fig:0_sync]{[1 ]}$ \includegraphics[width=1cm]{assets/task_specific_assets_expanded_svg/0_sync.pdf}  & 20.0 & 40.0 & \textbf{70.0} & 10       \\
        $\hyperref[fig:1_sync]{[2 ]}$ \includegraphics[width=1cm]{assets/task_specific_assets_expanded_svg/1_sync.pdf}  & 0.00  & 20.0 & \textbf{80.0} & 14       \\
        $\hyperref[fig:2_sync]{[3 ]}$ \includegraphics[width=1.5cm]{assets/task_specific_assets_expanded_svg/2_sync.pdf}  & 10.0 & 30.0 & \textbf{80.0} & 24  \\
        $\hyperref[fig:3_sync]{[4 ]}$ \includegraphics[width=1cm]{assets/task_specific_assets_expanded_svg/3_sync.pdf}  & 0.00 & 10.0 & \textbf{40.0} & 10 \\
        $\hyperref[fig:4_sync]{[5 ]}$ \includegraphics[width=1.5cm]{assets/task_specific_assets_expanded_svg/4_sync.pdf}  & 0.00 & 0.00 & \textbf{60.0} & 15 \\
        $\hyperref[fig:5_sync]{[6 ]}$ \includegraphics[width=2.5cm]{assets/task_specific_assets_expanded_svg/5_sync.pdf}  & 10.0 & \textbf{20.0} & \textbf{20.0} & 23 \\
        $\hyperref[fig:6_sync]{[7 ]}$ \includegraphics[width=2.5cm]{assets/task_specific_assets_expanded_svg/6_sync.pdf}  & 0.00 & 0.00 & \textbf{50.0} & 36 \\
        $\hyperref[fig:7_sync]{[8 ]}$ \includegraphics[width=3cm]{assets/task_specific_assets_expanded_svg/7_sync.pdf}  & 0.00 & 10.0 & \textbf{30.0} & 44 \\
        $\hyperref[fig:8_sync]{[9 ]}$ \includegraphics[width=4cm]{assets/task_specific_assets_expanded_svg/8_sync.pdf}  & 0.00 & 10.0 & \textbf{20.0} & 63 \\
        $\hyperref[fig:9_sync]{[10]}$ \includegraphics[width=4cm]{assets/task_specific_assets_expanded_svg/9_sync.pdf} & 0.00 & 0.00 & \textbf{20.0} & 57 \\
        \midrule
        \textbf{Total}   & 4.00 & 14.0 & \textbf{47.0} & \\
        \addlinespace[0.5em]
        \midrule
        \multicolumn{5}{c}{\textbf{Asynchronous (\%)}} \\
        \midrule
        $\hyperref[fig:0_async]{[1 ]}$ \includegraphics[width=1.5cm]{assets/task_specific_assets_expanded_svg/0_async.pdf} & 10.0 & 0.00 & \textbf{20.0} & 21       \\
        $\hyperref[fig:1_async]{[2 ]}$ \includegraphics[width=1.5cm]{assets/task_specific_assets_expanded_svg/1_async.pdf} & 0.00 & 0.00 & \textbf{30.0} & 27      \\
        $\hyperref[fig:2_async]{[3 ]}$ \includegraphics[width=2cm]{assets/task_specific_assets_expanded_svg/2_async.pdf} & 0.00 & 0.00 & \textbf{40.0} & 37      \\
        $\hyperref[fig:3_async]{[4 ]}$ \includegraphics[width=2cm]{assets/task_specific_assets_expanded_svg/3_async.pdf} & 0.00 & 0.00 & \textbf{10.0} & 42      \\
        $\hyperref[fig:4_async]{[5 ]}$ \includegraphics[width=2.5cm]{assets/task_specific_assets_expanded_svg/4_async.pdf} & 0.00 & \textbf{10.0} & 0.00 & 46 \\
        $\hyperref[fig:5_async]{[6 ]}$ \includegraphics[width=1.0cm]{assets/task_specific_assets_expanded_svg/5_async.pdf} & 0.00 & 0.00 & \textbf{10.0} & 19      \\
        $\hyperref[fig:6_async]{[7 ]}$ \includegraphics[width=2.0cm]{assets/task_specific_assets_expanded_svg/6_async.pdf} & 0.00 & 0.00 & 0.00 & 42      \\
        $\hyperref[fig:7_async]{[8 ]}$ \includegraphics[width=2.5cm]{assets/task_specific_assets_expanded_svg/7_async.pdf} & 0.00 & 0.00 & 0.00 & 46      \\
        $\hyperref[fig:8_async]{[9 ]}$ \includegraphics[width=3.5cm]{assets/task_specific_assets_expanded_svg/8_async.pdf} & 0.00 & 0.00 & 0.00 & 68      \\
        $\hyperref[fig:9_async]{[10]}$ \includegraphics[width=5cm]{assets/task_specific_assets_expanded_svg/9_async.pdf} & 0.00 & 0.00 & 0.00 & 82      \\
        \midrule
        \textbf{Total}   & 1.00 & 1.00 & \textbf{11.0} & \\
        \addlinespace[0.5em]
        \bottomrule
    \end{tabular}
    \caption{\gptfo{} performance on the synchronous and asynchronous datasets. 
    }
    \label{tab:tasks-results}
\end{table}

% \textbf{Question 2.} \textit{How close to optimal are successes?}
\textbf{Finding 2}. Asynchronous successes are less optimal than synchronous ones.

Fig.~\ref{fig:histogram_optimality} shows a histogram of the binned optimality rates on the successful runs of \gptfo{} \react{} on the synchronous and asynchronous datasets. Optimality rate is $\frac{\|\hat\tau\|}{\|\tau^*\|}$ where $\|\hat\tau\|$ is the number of steps taken by an agent and $\|\tau^*\|$ is the optimal number of steps to reach the goal from the initial state.

For the synchronous dataset, 55.3\% of successful attempts are optimal compared to the asynchronous dataset where only 9.1\% of successful attempts are optimal. We expect this since the order that tasks are done in the synchronous setting does not affect optimality compared to the asynchronous setting. We also see for the asynchronous dataset that 63.6\% of successful attempts are suboptimal in the $(1, 1.25]$ bucket. We qualitatively observe that while the LLM agent usually prioritizes asynchronous subtasks, suboptimal runs were due to inefficient actions, such as waiting while cooking. We further investigate the agent's subtask prioritization in Section~\ref{sec:follow-ups}.

\begin{figure}[!h]
    \centering
    \includegraphics[width=0.75\textwidth]{assets/final_svg/optimality.pdf}
    \caption{Histogram of the optimality rate for \gptfo{} \react{} successes on the synchronous and asynchronous datasets. The 1 bin includes attempts that were optimal. Attempts between $(1, 1.5]$ are suboptimal but classified as successful. Attempts greater than an optimality rate of 1.5 are classified as failures.
    % \GG{Yuki suggests a proportional histogram would make more sense due to unequal data sizes}
    % \YW{nitpick: all your plots with numbers could really benefit from increasing the font of the numbers}
    }
    \label{fig:histogram_optimality}
\end{figure}

% \textbf{Question 3.} \textit{How far off are failures from the goal?}
\textbf{Finding 3.} Asynchronous failures make little progress toward the goal.

Fig.~\ref{fig:histogram_steps_to_go} shows a histogram of the binned normalized steps to go on the failed runs of \gptfo{} \react{} on the synchronous and asynchronous datasets. Steps to go is $\frac{\|\tau^*_{\text{left}}\|}{\|\tau^*\|}$ where $\|\tau^*_{\text{left}}\|$ are the optimal number of steps left to reach the goal from the final state in a failed run and normalization factor $\|\tau^*\|$ is the optimal number of steps to reach the goal from the initial state.

% The synchronous and asynchronous datasets included 7 and 16 attempts in the $(0, 0.5]$ bucket respectively that made progress to the goal but exceeded the step limit. 
% % \GG{TODO: Su yean can we say anything special here about asynchronous having async fails? Answer: not necessarily which is what we expect. typically async bad with soup.} 
% There were 24 and 52 attempts in the $(0.5, 1.0]$ bucket respectively. The largest portion of asynchronous failures were in this bucket showing that most attempts made little to no progress towards the goal. The attempts in buckets beyond $1.0$ made progress away from the goal. Both datasets account for 22 and 19 attempts, where 3 of the synchronous attempts were excessively away from the goal, 2 of which were twice as far from where they started.

\begin{figure}[!h]
    \centering
    \includegraphics[width=0.75\textwidth]{assets/final_svg/steps_to_go.pdf}
    \caption{Histogram of the normalized steps to go for \gptfo{} \react{} failures on the synchronous and asynchronous datasets. The 0 to 0.5 bucket includes attempts that were making progress towards the goal while the 0.5 to 1 bucket includes attempts that made little to no progress towards the goal. Buckets greater than 1 includes attempts that traversed further away from the goal.
    % \YW{You should put numbers above the histogram instead of telling the number in your analysis paragraph.}
    % \GG{Yuki suggests a proportional histogram would make more sense due to unequal data sizes}
    }
    \label{fig:histogram_steps_to_go}
\end{figure}

For the asynchronous dataset, about 58.6\% of the failures are in the $(0.5, 1.0]$ bucket, which shows that most attempts made little to no progress towards the goal. We also see this in the synchronous dataset, with 41.5\% of failures in the $(0.5, 1.0]$ bucket. We show quantitative results of \gptfo{} \react{}'s ineffective failure recovery in Section~\ref{sec:failures} suggesting that failures on the asynchronous dataset are mainly due to little progress being made. In contrast, we see 45.3\% failures in the synchronous dataset from $(1.0, \infty)$ which show that most attempts make progress away from the goal. The asynchronous dataset only has 25.3\% failures from $(1.0, \infty)$. We present qualitatively annotated failures in Section~\ref{sec:failures} that suggest failures on the synchronous dataset are due to misunderstanding the goal. 

\subsubsection{Failure Mode Analysis}
\label{sec:failures}

% \textbf{Question 4.} \textit{What are the dominant failure modes?}
\textbf{Finding 4.} Dominant failures in both settings stem from rule violations and goal misinterpretations.

Fig.~\ref{fig:piechart_taxonomy} shows a nested piechart that captures failure modes of \gptfo{} \react{} on the synchronous and asynchronous datasets. We define our failure modes in terms of uncertainty over the MDP of the environment. The 4 main failure categories include uncertainty in the state (S), actions (A), transition function (T) and the goal (G). For a detailed description of the subcategories and dataset annotation, see Appendix~\ref{app:taxonomy-details}.

For synchronous failures, the uncertainty in the goal accounts for the majority at 64.1\% followed by the uncertainty in the transition function at 32.1\%. Goal failures could be due to (1) an incorrect understanding at the start of the plan or (2) a mistake during plan execution, such as using an ingredient without cutting it, which is incorrectly believed to satisfy the goal. We observe that case (1) occurs 28.3\% of the time under Bad Start; the LLM agent restates goals incorrectly for complex tasks with strict ordering dependencies like Task 6 or tasks with many diverse ingredients like Task 10 which we show in Appendix~\ref{app:qualitative-bad-starts}. We observe that case (2) occurs 35.8\% of the time under the remaining subcategories; although the LLM agent starts with a correct goal, it misunderstands the goal during execution by choosing the wrong action.
% Case (2) occurs; although the LLM agent stated the goal correctly, it misremember the goal during execution thereby choosing the wrong action (e.g. using an ingredient without cutting it, don't use this example though since you stated it before). 
% make easy to understand
For transition failures, violating the `one item at a station' rule accounts for the majority of failures at 24.5\%. We qualitatively observe that the agent attempts to use cutting stations for ingredient preparation while other items occupy the station; however, we also observe that once the agent has recovered from this failure it is unlikely to repeat it which we show in Appendix~\ref{app:unlikely-repeat}.

\begin{figure}[t!]
    \centering
    \includegraphics[width=\textwidth]{assets/first_draft/FINAL_piecharts.png}
    \caption{Nested pie chart of \gptfo{} \react{} failure modes capturing uncertainties in the MDP. The main categories are on the outer circle representing the uncertainty in the state space (S), action space (A), transition function (T), or reward/goal (G). The subcategories on the inner circle represent the dominant cause of failure and are described further in Appendix~\ref{app:taxonomy-details}.}
    \label{fig:piechart_taxonomy}
\end{figure}

For asynchronous failures, the inverse is true with uncertainty in the transition function accounting for 56.8\% of failures and uncertainty in the goal accounting for 34.1\% of failures. Similar to the synchronous failures, violating the `one item at a station' rule dominates failures at 53.4\%. This is due to the increased number of unique stations in the asynchronous setting compared to the synchronous setting which increases the potential number of recoveries necessary. In the synchronous setting, which only uses the cutting board station, an agent may need to recover once from violating the 'one item at a station' rule. In the asynchronous setting, which uses stoves, fryers, and sinks, an agent, in the worst case, may need to recover from violating rules on each station in a task.

We point out that while we designed the synchronous and asynchronous datasets to test different capabilities of LLM agents, we mainly observe similar transition failures in both settings. This demonstrates the need to improve LLM agents at following environment constraints to improve their decision-making ability. We investigate this further in Section~\ref{sec:follow-ups}.

% \textbf{Question 5.} \textit{How effective are failure recoveries?}
\textbf{Finding 5.} Asynchronous recovery is worse than synchronous recovery.

Fig.~\ref{fig:histogram_repeated_transitions} shows a histogram of the repeated transitions of \gptfo{} \react{} runs on the synchronous and asynchronous datasets. We use repeated transitions as a proxy for measuring \react{}'s effectiveness at recovering from failure.

% For the failures, approximately 66\% of the attempts have 6 or less repeated transitions which correspond to attempts that recovered effectively but misunderstood the goal; this aligns with goal failures being dominant for the synchronous setting in Fig.~\ref{fig:piechart_taxonomy}. In the asynchronous setting, we also see successes with little to no repeated transitions. For the failures, approximately 55\% of the attempts have 7 or more repeated transitions which correspond to attempts that were ineffective at recovery, with the worst attempt having 29 repeated transitions; this aligns with transition failures being dominant for the asynchronous setting in Fig.~\ref{fig:piechart_taxonomy}.

In both the synchronous and asynchronous datasets, we see that the majority of successes have 0 repeated transitions; few successes have repeated transitions but successfully recover. For failures, the asynchronous dataset's lower and upper quartiles are 103.1\% and 55.8\% larger than the synchronous dataset's quartiles. This means that failures on the asynchronous dataset are expected to have higher repeated transitions; this ineffectiveness at recovery aligns with the transition failures being dominant for the asynchronous setting in Fig.~\ref{fig:piechart_taxonomy}. Similarly, since the synchronous dataset has lower quartiles than the asynchronous dataset, we expect to see less repeated transitions which suggests less transition failures. We do a further investigation in Appendix~\ref{app:stochasticity} by introducing stochasticity and find that even in synchronous settings LLMs struggle at recovery.

% Since the agent is still failing in these runs despite making progress towards its goal, its goal must be misaligned with the task; this aligns with goal failures being dominant for the synchronous setting in Fig.~\ref{fig:piechart_taxonomy}.

\begin{figure}[!h]
    \centering
    \includegraphics[width=0.75\textwidth]{assets/final_svg/repeated_transitions.pdf}
    \caption{Histogram of the repeated transitions of \gptfo{} \react{} runs on the synchronous and asynchronous datasets. The median and quartiles of the asynchronous dataset are generally higher than those of the synchronous dataset, indicating higher repeated transitions.}
    \label{fig:histogram_repeated_transitions}
\end{figure}

\subsubsection{Follow-Up Investigation}
\label{sec:follow-ups}

From the previous experiments, we conclude that LLM agents struggle in the asynchronous dataset due to simple failures that arise in the synchronous dataset. In order to have a better understanding of how to improve LLM agent capabilities on asynchronous planning, we look into asynchronous subtask prioritization and boosting performance.

% \textbf{Question 6.} \textit{Does asynchronous subtask prioritization affect performance?}
\textbf{Finding 6.} Proper asynchronous prioritization boosts performance.

% In addition to observing performance changes, we annotate the asynchronous setting runs with stronger rule priors on Tasks 1 to 3 with whether the asynchronous tasks (like cooking) are done first or later. For the 12 successes, 9 runs prioritize asynchronous tasks and 3 runs do not. For the 18 failures, there was an even split between prioritization. We expect most successes to be prioritizing asynchronous tasks since this aligns with the in-context example provided; however despite the example, there are still instances where asynchronous tasks are not prioritized. Further increasing priors on which tasks to prioritize would easily fix this issue; however, an agent should be capable of reasoning about and auditing whether its trajectory is optimal without excess domain-specific knowledge. We discuss methods for self-correcting agents in Section~\ref{dis:correct}.

% If we assume that the synchronous planning capabilities of LLM agents is perfect, then performance on the asynchronous dataset entirely depends on subtask prioritization. We expect that runs where the agent prioritizes asynchronous subtasks will succeed while those where it doesn't will be suboptimal or be a failure in the worst case. We annotated the subtask prioritization of \gptfo{} \react{} in the initial experiments and found that in 72.7\% of successes, the agent prioritizes asynchronous subtasks while in 52.8\% of failures, the agent doesn't prioritize asynchronous subtasks. The former result aligns closely to what we expect while the latter does not. Since the agent is imperfect, it can correctly prioritize asynchronous subtasks and still encounter failure cases related to synchronous planning. We also expect the agent to prioritize asynchronous subtasks in successful runs even though it is imperfect because the in-context example focuses on this prioritization; however, it is undesirable that some successes do not prioritize asynchronous subtasks despite an example that does so. An agent should be capable of auditing its own reasoning and plan to ensure that it is optimal. We discuss methods for reliable self-verification in Section~\ref{dis:correct}.

Efficient asynchronous planning requires prioritizing subtasks that can be performed asynchronously. We investigate how success rate changes with asynchronous task prioritization to understand the impact of asynchronous planning on the results. Our hypothesis is that prioritizing asynchronous subtasks leads to higher success rates because the planned trajectory is shorter and reaches the goal within the maximum step limit. We find that the success rate conditioned on prioritization is 16\% compared to 6\% without, supporting that prioritization achieves higher success rate. An agent should be capable of auditing its own reasoning and plan to ensure that its prioritization correctly targets asynchronous subtasks. We discuss methods for reliable self-verification in Section~\ref{dis:correct}.

% \textbf{Question 7.} \textit{Would asynchronous performance improve by increasing priors over the transition function?}
\textbf{Finding 7.} Stronger priors improve asynchronous performance.

The dominant failures of \gptfo{} \react{} on the asynchronous dataset were transition failures. We investigate how we can improve performance by increasing the priors over the transition function. We create an augmented method, \reactp{}, that prompts \react{} with more details about the rules of \robotouille{}. 
%See Appendix~\ref{app:prompts} for differences in prompting.


Fig.~\ref{fig:piechart_followup} shows nested pie charts of the failure modes on Tasks 1 to 3 of the asynchronous dataset from the \gptfo{} \react{} experiments in Table~\ref{tab:tasks-results} and from \gptfo{} \reactp{}.

We observe a statistically insignificant change in performance, where the success rate for \gptfo{} \react{} is $0.30 \pm 0.085$ and \gptfo{} \reactp{} is $0.40 \pm 0.050$. We also observe failures relating to violating the 'one item at station' rule decrease from $38.1\%$ for \gptfo{} \react{} (8 failures) to $22.2\%$ for \gptfo{} \reactp{} (4 failures) accounting for a $50\%$ decrease in these transition failures. While this shows that increasing priors over rules decreases transition failures as expected, overall performance did not improve due to other failures that arose. We note that state failures increase from $23.8\%$ for \gptfo{} \react{} (5 failures) to $38.9\%$ for \gptfo{} \reactp{} (7 failures). These failures are due to misunderstandings with the state description provided; specifically, the agent assumes that meat on a stove always implies it is cooked. Augmenting \reactp{} over state priors would presumably improve performance but is impractical because it requires excessive effort from a domain-expert and wouldn't generalize to new domains. We discuss methods for incorporating state feedback in Section~\ref{dis:feedback}.

\begin{figure}[!h]
    \centering
    \includegraphics[width=\textwidth]{assets/final_svg/react_prior_piechart.pdf}
    \caption{Nested pie chart of failure modes capturing uncertainties in the MDP of \gptfo{} \reactp{} on Tasks 1 to 3 (30 problems) of the asynchronous dataset using \gptfo{} \react{} and \gptfo{} \reactp{}.}
    \label{fig:piechart_followup}
\end{figure}