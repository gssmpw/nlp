
\section{Dataset Details}
\label{sec:baselines}
\vspace{-1em}

In this section we discuss the contents of the synchronous and asynchronous datasets and their differences. We provide discussion of the in-context example tasks and multi-agent dataset in Appendix~\ref{app:dataset-extra}. Each dataset contains 10 unique tasks and has 10 procedurally generated instances. Table~\ref{tab:tasks-results} and Appendix~\ref{app:dependency-graphs} include visual representations of the tasks and dependency graphs respectively.

% \textbf{Datasets} We curate 3 datasets to test the synchronous, asynchronous, and multi-agent planning capabilities of LLMs. Each dataset contains 10 unique tasks and have 10 procedurally generated instances. We evaluate baselines on the synchronous and asynchronous datasets. Due to low baseline performance, we leave evaluation on the harder multi-agent dataset to future work.

\textbf{Synchronous Dataset} This dataset consists of tasks involving assembling sandwiches and burgers with ingredients that may need to be cut. Any ingredients that can be cooked are initialized as cooked. Tasks 1 to 3 involve assembling sandwiches of increasing difficulty where Task 1 only involves assembling and Task 2 and 3 involve cutting ingredients. Tasks 4 to 7 involve assembling burgers which differ from sandwiches in that the burger buns have ordering constraints with distinct buns that go on the top and the bottom. Unlike other tasks, Task 6 enforces a strict ordering constraint on the placement of all ingredients. Finally, Tasks 8 to 10 involve the preparation of 2 recipes which increase in difficulty from identical sandwiches, identical burgers, and finally a sandwich and burger with different ingredients.

\textbf{Asynchronous Dataset} This dataset consists of tasks including sandwiches and burgers from before but also fried recipes and soup. Unlike the synchronous dataset, ingredients that can be cooked are initialized as uncooked; this allows for asynchronous planning. Tasks 1 to 3 use the same ingredients as those in the synchronous setting except for an added ingredient which must be cooked or fried. We studied these tasks in Appendix~\ref{app:async-sync-task-comparison} for a closer one-to-one comparison with synchronous tasks and found that asynchronous tasks are more difficult. Tasks 4 and 5 involve making a burger and a fried recipe; Task 4 includes french fries which requires cutting a potato then frying while Task 5 includes fried onions which is the same process with an onion. Tasks 6 to 7 introduce a new recipe, soup, which involves filling a pot with water from a sink, boiling the water, putting ingredients inside, and finally serving in a bowl. Of these subtasks, filling a pot with water and boiling the water are steps that can be done asynchronously with other tasks. Finally, Tasks 8 to 10 involve making soup along with increasing numbers of sandwiches and burgers.