
The \toolname game demonstrated~\cite{DBLP:conf/icst/StraubingerBF24} that even young learners can engage
with testing concepts in an engaging and fun way, but the game design
precludes the important programming concept of loops. To address this
problem, we introduced a new game concept with loop-based recipes to
teach children about loops. The narrative of our game extends the one
introduced by \toolname and provides a satisfying story of healing
the infected critters collected as part of the original gameplay. The
gameplay uses the established approach of building on mutation testing
to introduce a testing challenge. Importantly, though, the concept of
recipes allows loops to become an integral aspect of the game. Our
study with 29 secondary school students confirms that the
loop-integrated levels promote active engagement, although they also
present challenges in understanding loops and conditionals.


To address these challenges, there are many possible avenues for
future exploration. The game could more gradually introduce different concepts,
making a clear distinction between single iterations and full loop
behaviors. This step-by-step approach, coupled with tutorials and
guided assistance, would help children develop a stronger
understanding before moving on to more complex levels. Adding an
in-game hint system could provide real-time guidance, especially for
loops and conditionals, offering helpful tips on structuring tests and
handling variables to keep players engaged and reduce frustration. By
incorporating adjustable difficulty levels, particularly in the loop
stages, the game could better accommodate for children with different
skill levels, ensuring a smoother progression through increasingly
challenging concepts. Visual aids that clearly illustrate how abstract
variables interact, such as rounds in a loop, would help connect the
game's mechanics to programming concepts, making it easier for
children to grasp how loops work and their effects on gameplay. These
enhancements would make \toolname more intuitive and engaging,
ultimately supporting young learners as they master testing
fundamental programming concepts.
\vspace{1em}

%%% Local Variables:
%%% mode: latex
%%% TeX-master: "../paper"
%%% End:
