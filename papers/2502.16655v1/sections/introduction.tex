Coding is not only the foundation of software engineering, but it is also becoming a prominent part of school curricula worldwide, even for young learners. However, learning to code presents challenges, often hindered by students' misconceptions about fundamental coding concepts like loops and variables~\cite{qian2017students,sorva2012visual}. 
%
To support students in overcoming these difficulties, testing has been suggested as an effective approach to enhance their understanding of programming principles~\cite{denny2019closer,prather2018metacognitive,wrenn2019executable,prasad2023conceptual}. Research indicates that introducing testing concepts early can significantly improve students' grasp of coding practices and their ability to create more reliable programs~\cite{DBLP:conf/acse/Carrington97,jones2001experiential,DBLP:conf/iticse/MarreroS05}. 
%
Unfortunately, engaging younger learners in abstract concepts like software testing remains a major challenge~\cite{DBLP:conf/iticse/Luxton-ReillySA18a,DBLP:conf/acse/Carrington97,jones2001experiential,DBLP:conf/iticse/MarreroS05}.

Gamification has become a promising strategy to tackle engagement issues in education~\cite{DBLP:conf/mindtrek/DeterdingDKN11,DBLP:journals/computers/NardoFFMMS24}. Specifically, serious games designed for learning have proven effective in teaching various programming and debugging skills~\cite{DBLP:conf/icer/MiljanovicB17,DBLP:conf/its/MuratetDTV12}. 
%
For software testing, games like Code Defenders~\cite{DBLP:conf/sigcse/FraserGKR19} and Code Immunity Boost~\cite{hsueh2023design} have shown the benefits of gamifying testing concepts to engage learners. However, these games are primarily aimed at students with more advanced mechanical, reading, and programming skills, which limits their accessibility for younger or less experienced learners~\cite{DBLP:journals/jss/GarousiRLA20}.

The serious game \toolname~\cite{DBLP:conf/icst/StraubingerCF23,DBLP:conf/icst/StraubingerBF24} introduces a novel approach designed to bridge the gap between the challenges of engaging with testing concepts through gamification and the abilities of younger learners. Like advanced testing games such as Code Defenders and Code Immunity Boost, \toolname is based on the concept of mutation testing, where artificially seeded bugs are used to assess test effectiveness and inspire new tests. In \toolname, this concept is subtly woven into a creative storyline where critters run short code snippets as they move through a landscape of tiles, and players must create portals to separate critters running correct code from those running mutated code.
%
Unlike other testing games, \toolname is specifically designed to accommodate the needs of younger learners by utilizing block-based programming, which removes the syntax-related challenges of traditional text-based coding~\cite{maloney2010scratch}. This has been shown to be effective in engaging students in testing activities~\cite{DBLP:conf/icst/StraubingerBF24}.

Although \toolname has proven effective, the approach of having critters execute code snippets during each step of their journey inherently limits the range of programming concepts that can be addressed. While \toolname successfully covers basic sequences, conditional branches, and variables, it notably lacks the integration of loops into its gameplay. Loops are a crucial programming concept that children often find challenging to grasp~\cite{DBLP:journals/ijcci/GomesFT18,DBLP:conf/sigcse/GroverB17,bentz2023novice} and for which they are especially prone for errors~\cite{DBLP:journals/tse/Ntafos88}. %, making them highly relevant for testing.

To address this gap, in this paper we present a new game concept that enhances the \toolname serious game by making loops and the testing of loop-based programs a central element of its mutation testing-based gameplay. 
%
This updated concept builds on the existing \toolname storyline by introducing second-stage levels: after rescuing the healthy critters through the original gameplay, players use loop-based recipes to create healing potions. In this stage, players must differentiate between critters following the correct recipe and those following mutated versions, requiring them to understand loops and assert the expected behavior effectively.

Overall, the contributions of this paper are as follows:
%
\begin{itemize}
  %
\item We introduce a novel game concept that combines mutation testing
  with loop testing.
  %
\item We integrate the loop testing game concept into the \toolname
  narrative and gameplay.
  %
\item We implement the new game concept as an extension of \toolname,
  and provide multiple playable loop-levels.
  %
\item We empirically evaluate \toolname and the new loop levels
  involving 29 school students.
  %
\item We discuss the challenges the children encountered and suggest
  potential solutions to address them.
  %
\end{itemize}

Our study involving secondary school students revealed that these new
levels promote active engagement, though they also present unique
challenges in grasping loop concepts. These findings suggest potential
strategies for enhancing the way programming and testing are taught to
younger learners.


%%% Local Variables:
%%% mode: latex
%%% TeX-master: "../paper"
%%% End:
