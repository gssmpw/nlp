\begin{figure}
	\centering
	\includegraphics[width=0.8\linewidth]{img/unlockedSecondStage.png}
	\caption{Second part of the scoreboard after a successful game}
	\label{fig:baseScoreDialogUnlockedSecond}
\end{figure}


In order to integrate loop testing into \toolname, we extend existing
base levels of \toolname, as described in \cref{sec:codecritters},
with new second-stage levels. Like the base levels, the second stages
also use the concept of mutation testing but place a stronger emphasis
on a previously overlooked topic: loops. These loop levels are not
available from the start; they must be unlocked by scoring at least
800 points in the corresponding base levels. This design encourages
players to first develop their skills and deepen their understanding
of the basic game concepts. \Cref{fig:baseScoreDialogUnlockedSecond}
displays the conclusion dialog shown to players who completed a base
level with a perfect score earning three stars and thus enabling the
second stage.


\subsection{Story}

The healthy critters are determined to find a way to cure their sick companions, so they seek help from a friendly wizard. The wizard agrees to assist them but needs the critters to gather magical berries to brew a healing potion. Armed with a magical recipe detailing how to collect the berries, along with a basket and supplies provided by the wizard, the critters set off on their journey. However, the paths to the magical berries have been long neglected, making it difficult for the critters to keep their recipes clean and unspoiled.

\subsection{Gameplay}

\Cref{fig:loopLevel1screen} shows the game screen during gameplay of the first loop level. Following the basic structure of the base levels, the game board appears on the left, with the corresponding Recipe Under Test (RUT) displayed on the right. The game board consists of an 8-by-8 grid of tiles, featuring a cyclic path with a collection basket at its start, at least one berry bush, and a signpost along the way. 

Since critters cannot carry all the required berries at once, they must gather them in multiple rounds, gradually filling the wizard's basket. The RUT is structured as a loop, representing the critters' repeated trips to gather berries, with instructions specifying how to collect them. For example, when following the instructions in \cref{fig:loopLevel1screen}, a critter completes three rounds and picks exactly one red berry in each round. If the recipe is smudged or corrupted (i.e., a mutant), however, a critter might for example collect two berries instead of one, deviating from the valid instructions. Mutants must be sent back to the wizard to receive a clean recipe. They can be visually identified by the dirt-smudged piece of paper they carry.

The objective of this level, similar to the base levels, is to identify these mutants among the berry collectors using assertion tests. Instead of portals, signposts serve as checkpoints to distinguish and send back the critters carrying corrupted recipes. The signposts can be filled with the same familiar block-based code as in the base levels to assert relevant conditions. This toolbox offers a wider range of options, including if-else blocks, to handle the more complex tests required for loop-based scenarios.

\begin{figure}
	\centering
	\includegraphics[width=\linewidth]{img/validTest2.png}
	\caption{Correct but long test for loop level 1}
	\label{fig:validTestLong}
\end{figure}

\begin{figure}
	\centering
	\includegraphics[width=\linewidth]{img/validTest1.png}
	\caption{Correct but short and efficient test for loop level 1}
	\label{fig:validTestShort}
\end{figure}

These blocks can be used to create customized tests that differentiate between valid and corrupted berry collectors, allowing for multiple approaches. For example, two versions of a correct test for level 1 (\cref{fig:loopLevel1screen}) are illustrated in \cref{fig:validTestLong} and \cref{fig:validTestShort}. Both tests are equally effective in verifying the correct berry collection during each round—whether by using if-conditions for an explicit check in each round or by expecting the number of berries to match the number of rounds completed. Although technically signposts are similar to the test cases created in the base levels, conceptually loop tests are similar to loop invariants~\cite{DBLP:journals/csur/FuriaMV14}. 

Once the signpost is set up, the game can be started, and the critters begin their rounds of gathering berries. Each time a critter passes the signpost, the test is executed. If the test fails, indicating that the critter has a corrupted recipe, the critter exits at the crossing to return to the wizard for a new recipe. On the other hand, critters who pass the test have the correct recipe and continue with their gathering mission. The game ends when the last critter either completes its task or is sent back, followed by a dialog displaying the player's final score.

\begin{figure}[t]
	\centering
	\includegraphics[width=\linewidth]{img/scoreDialogLoops.png}
	\caption{Scoreboard after finishing a loop level}
	\label{fig:scoreLoop}
\end{figure}

\Cref{fig:scoreLoop} shows the conclusion dialog after successfully completing a loop level. In each loop level, there are ten critters on their journey, with a varying number of mutants to detect. The scoring system is similar to that of the base levels, with the player's score primarily based on the percentage of correctly identified mutants and valid collectors. The maximum score is 1000 points or three stars.

A unique feature of the loop levels is the penalty for late detection, which emphasizes the importance of early error identification in repetitive tasks. Points are deducted if mutants remain in the game for longer than they should. In other words, if a mutant is not caught in the first round where its mutation affects the outcome, each additional round it remains undetected results in a penalty of 25 points.


\subsection{Progression}

\begin{figure}
	\centering
	\includegraphics[width=\linewidth]{img/indexPageAllLevelsGrey.png}
	\caption{Level overview on the homepage with unlocked base and locked (grey) loop levels}
	\label{fig:allLevels}
\end{figure}

Together with the new loop levels, \toolname features a total of twenty levels: ten increasingly difficult base levels and ten corresponding second-stage levels. \Cref{fig:allLevels} displays the starting page of \toolname, organized into three difficulty categories, showcasing all twenty available levels.

As players advance through the loop levels, both the Recipes Under Test and the game boards become more complex to increase the challenge. For example, \cref{fig:otherLoopLevel} illustrates a more advanced level that introduces a second type of berry and incorporates an if-condition related to the critters' shirt color into the recipe. The game board also includes an alternative route, visually representing this if-condition.

\begin{figure*}
	\centering
	\includegraphics[width=0.8\linewidth]{img/otherLoopLevel.png}
	\caption{Gameboard and instructions for loop level 2}
	\label{fig:otherLoopLevel}
\end{figure*}

\Cref{fig:otherLoopLevelTest} shows an example of a correct test, where the test ensures that the appropriate amount of either red or pink berries is collected, depending on the critter's shirt color.

\begin{figure}
	\centering
	\includegraphics[width=\linewidth]{img/otherLevelExampleTest.png}
	\caption{Correct test for loop level in \cref{fig:otherLoopLevel}}
	\label{fig:otherLoopLevelTest}
\end{figure}

As the game progresses, the complexity of the RUTs and game boards gradually increases, leading up to the final loop level. This gradual increase in difficulty helps to reinforce the fundamentals of loops and loop testing. The final loop level introduces the concept of nested loops, one of the most challenging loop structures to understand and test.


%%% Local Variables:
%%% mode: latex
%%% TeX-master: "../paper"
%%% End:
