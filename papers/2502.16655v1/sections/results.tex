\subsection{RQ 1: How do children play \toolname?}


\begin{figure*}[t]
	\centering
	\begin{subfigure}[t]{0.45\textwidth}
		\centering
		\includegraphics[width=\textwidth]{img/all-games}
		\caption{Number of games played over time}
		\label{fig:timegames}
	\end{subfigure}
	\hfill
	\begin{subfigure}[t]{0.45\textwidth}
		\centering
		\includegraphics[width=\textwidth]{img/current-levels-time}
		\caption{Current level played over time}
		\label{fig:timelevel}
	\end{subfigure}
	
	\caption{Differences between the players over time divided into base and loop levels}
	\label{fig:difftime}
\end{figure*}

During the experiment, the children played a total of 533 base-level games and 191 loop-level games, averaging 18.38 base-level games and 7.96 loop-level games per child. On average, they explored 4.03 base levels and 2.71 loop levels, successfully completing 1.72 base levels and 1.00 loop levels with a score of at least two out of three stars.
To unlock the corresponding loop level, they needed to achieve at least two stars in the base level. On average, the children needed about three attempts to unlock a loop level, resulting in them playing the base levels three times each. Once the loop level was unlocked, players often attempted it immediately but would return to the base level to maximize their score if they could not make quick progress. Therefore, the children performed better on base levels due to familiarity, while loop levels posed greater challenges, prompting repeated attempts.

During gameplay, the children placed an average of 31.45 portals—most of the time consisting of three blocks each—in the base levels and utilized 32.15 blocks in the loop level signposts (\cref{fig:boxmines}).
They also detected an average of 137.31 mutants in the base levels and identified 16.71 faulty recipes in the loop levels (\cref{fig:boxmutants}). Additionally, the players allowed 29.24 healthy critters to reach the tower in the base levels, and on average, 6.58 collectors successfully gathered the correct amount of berries in the loop levels (\cref{fig:boxhumans}). These numbers vary significantly because (1) the children played more base than loop levels, and (2) the base levels feature more critters on the gameboard compared to the loop levels. Additionally, in the base levels, players could place and remove portals freely along the path, allowing for more flexible experimentation. In contrast, blocks in the loop levels could only be set at specific signposts, which required more precise decision-making and offered a wider variety of blocks for testing. This makes it challenging, if not impossible, to directly compare the base levels with the loop levels, which is why we refrain from doing so and present the data as it is.

When examining the number of games played over time (see \cref{fig:timegames}), it is clear that there are significant differences among the children. Some played as few as seven games in total, while others played up to 46 games during the 60-minute experiment. This wide range suggests that the children either have varying levels of programming skills or that some were not fully engaged in the activity. Observations during the experiment indicated the latter in some cases, as a few children attempted to cheat by modifying blocks and scores using the developer consoles in their browsers. This occurred only a few times, and altering the data in the browser does not affect any information in the database.
Nevertheless, they interacted with \toolname in one way or another. The teacher also noted that the children seemed more engaged with \toolname than they usually are with their typical tasks, where they tend to get distracted more often.

\cref{fig:timelevel} illustrates the progression of levels played, showing the current level (base or loop) that each player opened at any given minute during the experiment on average for all players. Initially, only the base levels were played since the loop levels were locked. After a few minutes, when the first loop levels were unlocked, many children attempted to play either loop level one or two. However, many quickly returned to loop level one to first understand how these loop levels worked. The progression through base levels increased steadily during the first third of the experiment, then remained mostly stable for the rest of the session, with a slight decline toward the end. Meanwhile, the number of loop levels played increased significantly in the second third of the experiment as more children managed to unlock them, reaching a peak around the 40-minute mark and stabilizing afterward, with most children playing loop levels one through three.

\summary{RQ 1}{Our study reveals wide variations in gameplay patterns, level progression, and involvement, with some even attempting to manipulate the game. Despite these differences, overall engagement was higher than in their usual activities, with most children gradually advancing through the levels after an initial learning phase.}

\subsection{RQ 2: How do children interact with the new loop-based levels in \toolname?}

\begin{figure*}
	\centering
	\begin{subfigure}[t]{0.45\textwidth}
		\centering
		\includegraphics[width=\textwidth]{img/total-levels-time}
		\caption{Correctness per loop level}
		\label{fig:correctnesslevel}
	\end{subfigure}
	\hfill
	\begin{subfigure}[t]{0.45\textwidth}
		\centering
		\includegraphics[width=\textwidth]{img/total-ratios-time}
		\caption{Correctness over time for loop levels}
		\label{fig:correctnessratio}
	\end{subfigure}
	
	\caption{Differences in correctness on level and time basis}
	\label{fig:diffcorrecttime}
	\vspace{-1em}
\end{figure*}

\begin{figure}
	\centering
	\includegraphics[width=\linewidth]{img/survey}
	\caption{Survey responses to ``How much did you enjoy\ldots''}
	\label{fig:survey}
\end{figure}

\Cref{fig:timelevel} shows that most players concentrated on loop levels 1 and 2, with only a few attempting levels 3 and 4 before returning to the earlier levels, which accounts for the significantly lower ratios in levels 3 and 4. Even fewer students tackled level 5, but those who did were more successful at identifying incorrect recipes than in levels 3 and 4 (\cref{fig:correctnesslevel}), despite level 5 being the most challenging due to its introduction of nested loops. This suggests that only the strongest players attempted level 5, leading to higher average success rates. \cref{fig:correctnesslevel} shows that the ratios of incorrect recipes and completed collectors remain fairly consistent in their differences across each level. Across all levels, the ratio of detected incorrect recipes was consistently higher than that of completed collectors, indicating a focus on identifying code errors over recognizing correct code. Although both are connected since they travel the same path, many children wrote tests that correctly identified all the incorrect recipes but also mistakenly caught the collectors due to flawed testing. During this testing, the number of blocks used in level 1 was more than twice the optimal amount, resulting in a ratio of more than one, but this number decreased in levels 2 and 3, likely reflecting the players' learning curve. The increase in block usage for level 4 in \cref{fig:correctnesslevel} is due to multiple if-conditions within the loop, requiring more blocks to handle its complexity.

Looking at the average ratios over time in \cref{fig:correctnessratio}, a different pattern emerges. Initially, the ratio of correctly identified recipes (collectors in \cref{fig:correctnessratio}) is near its maximum, while the ratio of detected incorrect recipes (recipes in \cref{fig:correctnessratio}) starts at zero. This suggests that players initially allowed the levels to run without adding tests to familiarize themselves with the loop levels. As the experiment progressed, the ratios for both recipes and collectors fluctuated similarly between 0.3 and 0.6, with a noticeable drop around two-thirds of the way through the experiment. This decline might indicate that players transitioned from the first to the second level, requiring time to adjust to the increased difficulty. After this dip, both ratios improved, showing that players adapted and learned how to handle the new level. Towards the end, the ratio of detected incorrect recipes decreased again, possibly indicating that some children attempted new levels but could not make further progress as time ran out. The peak in the ratio of blocks added during the second third of the experiment indicates that many players initially attempted to use as many blocks as possible to cover all scenarios, often exceeding a ratio of one, which would represent the optimal number of blocks. This approach was gradually abandoned, likely because they did not see a corresponding increase in accuracy, as reflected in the steady ratios for both recipes and collectors. This indicates that the students learned that including too many assertions in a single test can create confusion rather than improve results, making it difficult to identify what went wrong.

\summary{RQ 2}{Children playing the new loop-based levels were better at identifying errors in code than recognizing correct code, with most focusing on the simpler levels and adapting their strategies as difficulty increased.}

\subsection{RQ 3: Do children enjoy playing the loop levels of \toolname?}

\Cref{fig:survey} summarizes the results of the exit survey, showing that 58\% of the children enjoyed playing \toolname. While they liked the elements of the base levels—such as portals, critters, and mutants—they were more uncertain about the loop levels, particularly the wizards' instructions and the signposts. The signposts of the loop levels were less liked by the players than other features, with 38\% of players expressing dissatisfaction, a sentiment frequently noted in the free-text responses.  It was not that they disliked the loop levels more than the base levels; instead, they wished the portals they enjoyed in the base levels could also be used in the loop levels, as they were already familiar with them. Post experiment, we addressed this feedback by replacing the signposts with movable portals like those in the base levels in a newer version of \toolname.  Conversely, many children expressed their enjoyment of playing \toolname in the free-text comments, stating that they liked it and encouraged further development of the game with additional levels and stories. Notably, two participants even logged into \toolname from home. % after the session ended.

\summary{RQ 3}{The children enjoyed playing \toolname, favoring the base levels over the loop levels due to their familiarity with the portals.}
%%% Local Variables:
%%% mode: latex
%%% TeX-master: "../paper"
%%% End:
