The experiment revealed several challenges that children faced while interacting with \toolname, which varied by game level and were especially noticeable with advanced programming concepts like loops and conditionals.

A major hurdle was understanding loops and how they were represented in the game. Many participants struggled to connect loop iterations (rounds) with the expected outcomes (like berry collection per round). In level 1, most players initially misunderstood how loops worked, leading to unsuccessful attempts at designing tests to distinguish mutants from valid collectors. Though they often figured it out after two or three tries, the process highlighted a significant learning curve. Despite nearly a year of programming lessons, many students lacked a clear grasp of fundamental loop concepts, indicating that their learning focused more on using loops in a limited setup rather than understanding the underlying principles.

\begin{figure}
	\centering
	\includegraphics[width=\linewidth]{img/invalidTest1.png}
	\caption{Incorrect test for loop level 1}
	\label{fig:invalidTest1}
\end{figure}

\begin{figure}
	\centering
	\includegraphics[width=\linewidth]{img/invalidTest2.png}
	\caption{Incorrect test for loop level 1}
	\label{fig:invalidTest2}
\end{figure}

One example of an incorrect test for level 1 is shown in \cref{fig:invalidTest1}. Since the wizard's instructions specify that the critters should walk three rounds, the player was likely trying to prevent them from walking a fourth round. However, the way the test was set up resulted in the opposite effect: all critters were sent back in the first round because their \texttt{roundsCount} was not equal to four. Another example is the test shown in \cref{fig:invalidTest2}. This test contains three assertions that compare the current number of collected red berries to one, two, and three. Although it seems that the player grasped the principles of loops, they misunderstood that all assertions are evaluated during each pass of a critter. As a result, this leads to all recipes being marked as incorrect, since the number of berries cannot simultaneously be one, two, and three.

Level 2 introduced if-conditions within loops, which significantly increases the complexity. Most students found it difficult to apply conditional logic to differentiate between critters based on their attributes. This difficulty was amplified by the need to evaluate these conditions within the loop repeatedly, resulting in frequent mistakes. Only a small fraction of the participants (four out of 20) completed level 2 successfully, underscoring the lack of foundational programming knowledge—a gap that \toolname was not originally designed to address.
%
The introduction of nested loops in later levels added another layer of complexity, requiring participants to manage multiple variables over several iterations. This abstract reasoning was a steep challenge, and no participants fully mastered these levels.
%
While most students engaged well with the base levels, their interest dropped with the advanced loop levels. Some even attempted to alter the game's code using developer tools, possibly out of frustration with the game mechanics or their lack of progress. This behavior indicates a need to better support motivation and understanding, especially when the difficulty increases.

To help with these challenges, future versions of \toolname could introduce loops and conditionals more gradually, emphasizing the difference between individual iterations and overall loop behavior. Breaking down complex concepts and incorporating tutorials could strengthen foundational skills before progressing to harder levels. Adding an in-game hint system could also provide real-time support, guiding players through loops and conditional logic. Hints suggesting ways to structure tests or manage variables could help refine their approach, reducing frustration and maintaining engagement.
%%% Local Variables:
%%% mode: latex
%%% TeX-master: "../paper"
%%% End:
