To assess the effectiveness of the new loop levels in \toolname, we conducted a controlled experiment focused on addressing the following research questions:

\begin{itemize}
	\item \textbf{RQ 1}: How do children play \toolname?
	\item \textbf{RQ 2}: How do children interact with the new loop-based levels in \toolname?
	\item \textbf{RQ 3}: Do children enjoy playing the loop levels of \toolname?
\end{itemize}

\subsection{Experiment Setup}

The controlled experiment was carried out at \school in two sessions in July 2024.

\subsubsection{Experiment Environment}

Through a collaboration with \school, a secondary school that prepares students for higher education, we conducted our experiment in their academic environment. We were given access to two sessions of their elective robotics course, each lasting 90 minutes and attended by around 15 students each. Since this course is elective, all students had shown interest in programming. Before the experiment, these students had ten months of experience with a block-based programming language developed by Lego,\footnote{\url{https://education.lego.com/en-gb/lessons/ev3-robot-trainer/}} which provided them with a solid understanding of block-based programming—the foundation of \toolname.

We had already used the original version of \toolname~\cite{DBLP:conf/icst/StraubingerBF24} during a prior session at the school, meaning that all the students were already familiar with the base levels. To avoid overwhelming the students in the limited time available, five base levels (1, 2, 6, 7, and 10) and five second-stage levels (1, 4, 6, 8, and 10) were selected. A total of 29 students participated in our experiment, conducted across two sessions. All participants were male, with no female students involved. The majority were in 5th to 7th grade (ages 11–13), with only three students from the 8th grade (ages 14–15). The experiment was organized in the school's computer lab, with each student working on their own computer.

\subsubsection{Experiment Procedure}

During the first ten minutes, we provided a recap of \toolname, covering its storyline, game mechanics, and the specifics of the new loop levels.
After the introduction, participants were tasked to play \toolname independently, without guidance on which level to choose, whether they should collect all points before moving on, or focusing on loop levels only. This gameplay session lasted for 60 minutes and ended with the conclusion of the experiment, providing enough time for participants to complete our exit survey. The survey starts with general questions about participants' gender and grade level, followed by seven questions on a five-point Likert scale that assess their enjoyment of various aspects of \toolname. It also includes open-ended prompts for participants to share additional thoughts or feedback.

\subsubsection{Experiment Analysis}

Our analysis centers on presenting results for all participants, highlighting the differences between the base and loop levels, as well as the challenges faced.

\subsubsection{RQ 1: How do children play \toolname?}

To answer this research question, we focus on understanding how children interact with the game. This analysis will help us identify the strategies and behaviors they use when playing the different levels, especially the newly introduced loop levels. We examine the data gathered from the experiment, covering both the base and loop levels, to gain these insights. First, we look at the average number of (1) completed and (2) attempted levels. Next, we examine the participants' activity throughout the experiment, focusing on any behavioral differences between base and loop levels.
This includes tracking which levels they played at various times and how many games they played during specific time intervals. We also perform a comparative analysis of three key metrics: the total number of (1) generated test cases (i.e., portals, signposts, and code blocks), (2) identified bugs (i.e., mutants and recipes), and (3) recognized correct code (i.e., healthy critters and collectors). This data is analyzed separately for the base and loop levels and is presented using box plots.

\begin{figure*}
	\vspace{-1.5em}
	\centering
	\begin{subfigure}[t]{0.325\textwidth}
		\centering
		\includegraphics[width=\textwidth]{img/portals}
		\vspace{-3.5em}
		\caption{Number of created portals/blocks}
		\label{fig:boxmines}
	\end{subfigure}
	\hfill
	\begin{subfigure}[t]{0.325\textwidth}
		\centering
		\includegraphics[width=\textwidth]{img/mutants-killed}
		\vspace{-3.5em}
		\caption{Number of killed mutants/recipes}
		\label{fig:boxmutants}
	\end{subfigure}
	\hfill
	\begin{subfigure}[t]{0.325\textwidth}
		\centering
		\includegraphics[width=\textwidth]{img/humans-finished}
		\vspace{-3.5em}
		\caption{Number of finished critters/collectors}
		\label{fig:boxhumans}
	\end{subfigure}
	
	\caption{Statistics on the use of \toolname divided into base and loop levels}
	\label{fig:testboxes}
\end{figure*}

\subsubsection{RQ 2: How do children interact with the new loop-based levels in \toolname?}

To answer this research question, we aim to assess how the children interact with the new loop levels of \toolname. Analyzing their behavior will help us understand their grasp of key concepts like loops and conditional logic while testing code. This insight will also highlight areas where they face challenges, guiding potential improvements in the game's design to enhance their learning experience. Specifically, we analyze three key ratios: (1) the number of utilized versus required blocks in signposts to assess their ability to express loop invariants, (2) the identified errors in recipes as a measure of their success, and (3) the correctly identified recipes. We examine these ratios across levels and over time to identify variations in difficulty and to track the skill development throughout the experiment.

%Move to discussion section
%\subsubsection{RQ 3: What difficulties did the children face while playing \toolname?}

%To answer this research question, we will analyze the data and results from the previous questions to identify the difficulties children encountered while playing both the base and loop levels. Additionally, we will examine the free-text responses from the exit survey to capture the challenges the children personally perceived. We will then discuss these challenges and their implications for the future of secondary education.

\subsubsection{RQ 3: Do children enjoy playing the loop levels of \toolname?}

By examining their feedback and in-game behavior, we can determine which elements of the loop levels appeal to the children and which areas may need improvement to enhance the gaming experience. This insight will guide us in refining the game's design to make the learning process more engaging and effective. We analyze the responses from the exit survey, presenting the data through stacked bar charts that display the questions and their respective percentages.
%

\subsection{Threats to Validity} 

\paragraph{Threats to Internal Validity} Participants' previous experience with block-based programming and \toolname may influence the outcomes of the experiment. Introducing loop levels in \toolname without prior exposure to block-based programming could overwhelm some children. Additionally, participants might feel pressured to provide socially desirable responses in the exit survey, which could distort the data. To mitigate this risk, we encouraged them to answer honestly and without hesitation.

\paragraph{Threats to External Validity} The small sample size and lack of diversity in gender and grade levels restrict the ability to generalize our findings to a broader population of children. Since the participants were already enrolled in a robotics course and had an interest in programming, the results may not accurately reflect the experiences of children in more typical settings. Furthermore, the brief duration of the experiment may not capture long-term effects or usage patterns of \toolname, which could influence how interactions with the tool evolve over time.

%%% Local Variables:
%%% mode: latex
%%% TeX-master: "../paper"
%%% End:
