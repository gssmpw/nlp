\section{Numerical Results}


\begin{figure}[t!]
    \centering
    \includegraphics[width=\linewidth]{dataset.pdf}
    \caption{Illustration of the utilized dataset. The region of interest within the mask is highlighted by red dashed lines.}
    \label{fig:dataset}
\end{figure}

\subsection{Experiment Details }

\minisection{Dataset} LithoBench~\cite{zheng2024lithobench} is a well-known layout pattern dataset consisting of 133,496 tiles organized into several subsets. For our work, we focus exclusively on the MetalSet subset, which is generated from the ICCAD-13 benchmark~\cite{banerjee2013iccad} and contains 16,472 tiles, each measuring 2048$\times$2048 nm$^2$. For each tile, we crop the central region with dimensions of 812$\times$756 nm$^2$ and utilize commercial tools to generate both the aerial image and the corresponding resist image. The aerial image is a grayscale representation indicating the intensity distribution in the region of interest, while the resist image is a binary representation that illustrates the resist result. We show several examples in \Cref{fig:dataset}. Both the commercial resist tools and our TorchResist framework ensure that the resolutions of aerial and resist images are consistent at 7$nm$ per pixel during the resist simulation process. We treat the generated aerial images and resist images as the inputs and target outputs of our TorchResist and baseline resist methods. We randomly select 20\% of whole dataset as calibration set and treat the remaining ones as the test set. The calibration/test split is uniform across all resist methods.

\minisection{Parameter Optimization} Some of the parameters are pre-determined based on domain knowledge. For example, in our settings, we set the absorption coefficient of the resist, $\alpha = A + B$, to 6.186$/\text{nm}$, with $A = 0$. The resist thickness is fixed at 75$\,\text{nm}$. The number of $n$ in the development model is set to five for our experiments. Additionally, we always set the boundary sharpness factor $s$ to six.

To calibrate the remaining parameters in TorchResist, we use the popular gradient descent method. We adjust the parameters on the training set for a total of 9 epochs. The learning rate is initialized at 1e-2 and is scaled by a factor of 0.3 after every 3 epochs. The optimizer used is Adam~\cite{kingma2014adam} with $\beta_1, \beta_2 = 0.9, 0.999$. The batch size is set to 16. The entire training process takes approximately 1 hour on a single NVIDIA A100 GPU.

\subsection{Evaluation}

\minisection{Evaluation Metrics} 
After training is completed, we fix the parameters of TorchResist. The predicted developed depth image is first up-sampled to 1$\,\text{nm}$ per pixel and then thresholded by the threshold $\tau$, as explained in \Cref{eq:object}. The up-sampling algorithm used is bilinear interpolation. We compare the predictions of TorchResist on the test set with the corresponding ground truth to evaluate its performance. Quantitatively, we employ three evaluation metrics: Pixel Difference, Edge Placement Error (EPE)-mean, and EPE-max. 

Pixel Difference is the normalized $L_0$-norm between the prediction and the ground truth:
\begin{equation}
     \text{Pixel Difference} = \frac{\| \rmW - \Gamma(f_\theta(\rmR), \tau)\|_0}{\# \text{pixels in total}} \times 100\%.
\end{equation}

EPE estimates the difference between the edges of the ground truth and the predicted edges. EPE-mean is the average EPE value across all edges in a tile, while EPE-max is the maximum EPE value in a tile. The reported values represent the averages across all tiles in the test set.


\begin{table}[t!]
    \centering
    \caption{The performance comparison of the resist model on LithoBench. The lithography model is a commercial tool.}
    \label{tab:acc}
    \resizebox{\textwidth}{!}{
    \begin{tabular}{@{}lccccc@{}}
        \toprule
         Resist Model & Pixel Difference (\%) & EPEMean (nm) & EPEmax (nm) & Differentiable & Depth Simulation \\ 
        \midrule
         Fixed Thres.\cite{banerjee2013iccad} & 0.59 & 1.52 & 4.45 & \ding{55} & \ding{55} \\
         Variable Thres.\cite{randall1999variable} & 0.49 & 1.21 & 3.95 & \ding{55} & \ding{55} \\
         TorchResist & 0.22 & 0.73 & 2.87 & \checkmark & \checkmark \\
        \bottomrule
    \end{tabular}
    }
\end{table}

\begin{figure}[t!]
    \centering
    \includegraphics[width=\linewidth]{result.pdf}
    \caption{Illustration of the predictions of TorchResist. We also compare the predictions with groundtruth for the reference. The resolution of all the figures are 1 nm/pixel.}
    \label{fig:result}
\end{figure}

\minisection{Baseline Methods} We compare our TorchResist with two baseline methods: Fixed Threshold~\cite{banerjee2013iccad,zheng2024lithobench} and Variable Threshold~\cite{randall1999variable}. The Fixed Threshold method is the simplest resist method, which applies a fixed threshold to the aerial images to obtain the final binary resist results. The Variable Threshold method uses an adaptive threshold to determine the resist result, given by:
\begin{equation}
    \tau_{var} = M_1 + M_2 R_{\text{max}},
\end{equation}
where $M_1$ and $M_2$ are constants to be determined, and $R_{\text{max}}$ is the maximum value of the aerial image in a local region. We fine-tune the constants for both baselines on the training set and evaluate their performance on the test set.

\minisection{Results} We predict the resist values for the aerial images provided by commercial tools using all three resist methods and compare the results in \Cref{tab:acc}. We also show visualizations of the predicted results in \Cref{fig:result}. The results indicate that TorchResist outperforms both threshold-based methods, demonstrating its superior model capacity. Furthermore, TorchResist can simultaneously output both the binary resist and the development depth, which is beneficial for downstream tasks that require 3D simulation results.

\minisection{Efficiency and Scale Robustness} We also evaluate the efficiency of TorchResist on a single NVIDIA 3090 GPU. We conduct experiments at both 1nm/pixel and 7nm/pixel resolution and report the average processing time for a single mask. The results are summarized in \Cref{tab:time}. We also test the scale robustness of TorchResist by comparing the results at different resolutions. We should note the all the model parameters keep the same under different resolutions. The average pixel difference between them is 0.17\%. Some examples at different resolutions are shown in \Cref{fig:robust}. The result shows the scale robustness of TorchResist, and the inference cost can be largely reduced without an obvious trade-off in precision.



\minisection{Extension on Open-Source Lithography Models} There are two popular open-source lithography simulators that are widely used: FUILT\cite{fuilt} and ICCAD13\cite{banerjee2013iccad}. However, both simulators lack reliable resist modeling, which limits their overall capabilities. To address this limitation and support further research tasks that depend on accurate resist modeling, we introduce two additional variants of TorchResist: TorchResist-F and TorchResist-I. These variants are derived from the outputs of the respective lithography simulators and commercial resist results. 

The evaluation results for these variants are provided in \Cref{tab:other} for reference. It is important to note that the only difference between these variants and the original TorchResist lies in the parameter values, which are adjusted to account for the differences in lithography simulations. Consequently, the evaluation metrics for these variants should not be directly compared with the results presented in the main table.





\begin{figure}[t]
    \centering
    \includegraphics[width=.8\linewidth]{robust.pdf}
    \caption{Comparison between results obtained at different resolutions.}
    \label{fig:robust}
\end{figure}








\begin{table}[t!]
    \centering
    \caption{Comparison of model efficiency, measured by the average time required to process a 2µm $\times$ 2µm patch at different resolutions. Inference is performed using TorchResist on a single NVIDIA 3090 GPU.}
    \label{tab:time}
    \begin{tabular}{c|cc}
    \toprule
        Resolution & Cost Time(s) & Ratio\\
        \midrule
        7 nm/pixel&  0.04  &1.00 \\
         1 nm/pixel&  1.98  &48.96 \\
         \bottomrule
    \end{tabular}
\end{table}

\begin{table}[t!]
    \centering
    \caption{The performance of TorchResist with different open-source lithography models.}
    \label{tab:other}
  
    \begin{tabular}{@{}lcccc@{}}
        \toprule
         Litho Model&Resist Model & Pixel Difference (\%) & EPEMean (nm) & EPEmax (nm) \\ 
        \midrule
         FUILT\cite{fuilt} & TorchResist-F & 1.77&7.03 & 31.91 \\
         ICCAD13\cite{banerjee2013iccad} & TorchResist-I & 3.39 & 10.62 & 59.98\\
        \bottomrule
    \end{tabular}
    
\end{table}