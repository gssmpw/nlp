\section{Algorithms}

\subsection{Resist Models}

\minisection{Exposition Model} The exposure of the resist is the initial step in the process. As described in previous work \cite{dill1975characterization}, when light passes through the resist without reflection, the Lambert-Beer law can be applied to characterize the optical absorption:
\begin{equation}
    \frac{d\rmI}{dh} = -\rmI \sum_i a_i m_i,
    \label{eq:LBL}
\end{equation}
where $\rmI$ represents the light intensity, $h$ denotes the distance from the resist-air interface, and $a_i$ and $m_i$ are the molar absorption coefficient and molar concentration of the $i$-th component, respectively. We consider three absorbing species: the inhibitor, the base resin, and the reaction products. For a positive photoresist, \Cref{eq:LBL} can be further specified as:
\begin{equation}
    \frac{\partial \rmI(h,t)}{\partial h} = -\rmI(h,t) \left[a_1 \vm_1(h,t) + a_2 \vm_2(h,t) + a_3 \vm_3(h,t)\right],
    \label{eq:depth}
\end{equation}
where $a_1$, $a_2$, and $a_3$ are the molar absorption coefficients of the inhibitor, base resin, and reaction products, respectively. Similarly, $\vm_1$, $\vm_2$, and $\vm_3$ represent the molar concentrations of the inhibitor, base resin, and reaction products. The variables $h$ and $t$ denote the depth in the film and the exposure time, respectively.
 The destruction of inhibitor can be obtained via,
\begin{equation}
    \frac{\partial \vm_1(h,t)}{\partial t} = - \vm_1(h,t)\rmI(h,t)C,\label{eq:time}
\end{equation}
where $C$ is the fractional decay rate of inhibitor per unit intensity.

Assuming the boundary condition of light intensity, the aerial image $\rmR(x,y)$, is given by optical lithography simulation and the lamp intensity is consistent during the exposure, we have,
\begin{equation}
    \rmI(0,t) = \rmR. \label{eq:aerial}
\end{equation}
Considering the initial inhibitor uniformity, resin uniformity and resin does not bleach,
\begin{equation}
\begin{aligned}
    \vm_1(h,0) &= m_{10}; \\
    \vm_2(h,t) &= m_{20}. \label{eq:init}
\end{aligned}
\end{equation}
And reaction product is generated from inhibitor, and the amount of substance is conserved.
\begin{equation}
    \vm_3(h,t) = m_{10} - \vm_1(h,t). \label{eq:amount}
\end{equation}
By substituting \Cref{eq:aerial,eq:init,eq:amount} into \Cref{eq:depth,eq:time}, normalizing, and replacing constants, we obtain:
\begin{equation}
\begin{aligned}
    \frac{\partial \rmI(h,t)}{\partial h} &= -\rmI(h,t)[A\rmM(h,t) + B]; \\
    \frac{\partial \rmI(h,t)}{\partial t} &= -\rmI(h,t) \rmM(h,t)C, \label{eq:expo}
\end{aligned}
\end{equation}
where $\rmM(h,t) = \frac{\vm_1(h,t)}{m_{10}}$ is fractional inhibitor concentration. $A = (a_1-a_3)m_{10}$, $B = (a_2m_{20} + a_3m_{10})$ and $C$ are the constant that should be further calibrated. \Cref{eq:expo} can be further solved with the following initial conditions and boundary conditions,
\begin{equation}
    \begin{aligned}
        \rmM(h,0) &= 1; \\
        \rmM(0,t) &= \exp(-\rmR C t);\\
        \rmI(h,0) &= \rmR\exp[-(A + B)h];\\
        \rmI(0,t) &= \rmR.\label{eq:condition}
    \end{aligned}
\end{equation}


\minisection{Development model} The bulk development model proposed in \cite{mack1987development} can describe the reaction of developer with the resist. Assuming $k_D$ is the rate of diffusion of developer to resist surface, $k_R$ is the rate constant, the rate of development can be described with,
\begin{equation}
    \vr = \frac{k_Dk_R\rmD\vm_3^n}{k_D+k_R\vm_3^n}, \label{eq:speed}
\end{equation}
where $\rmD$ is the bulk developer concentration and we assume $n$ molecules of product $\vm_3$ react with the developer to dissolve a resin molecule. By using \Cref{eq:amount} and the fractional inhibitor concentration $\rmM(h,t)$, \Cref{eq:speed} can be rewritten as,
\begin{equation}
     \vr = \frac{k_D\rmD(1-\rmM)^n}{k_D/k_Rm_{10}^n+(1-\rmM)^n}. 
\end{equation}
As described in \Cref{eq:condition}, when resist unexposed ($t=0$), we have $\rmM=1$ and the rate is zero. When resist completely exposed, we have $\rmM = 0$ and the rate is equal to $r_\text{max}$,
\begin{equation}
    r_\text{max} = \frac{k_D\rmD}{k_D/k_Rm_{10}^n+1}.
\end{equation}
Let $a$ be a constant,
\begin{equation}
    a = k_D/k_Rm_{10}^n.
\end{equation}
The physical meaning of $a$ is an inflection point in the rate curve. By letting,
\begin{equation}
    \frac{d^2\vr}{d\rmM^2} = 0,
\end{equation}
we have,
\begin{equation}
    a = \frac{n+1}{n-1}(1-m_\text{TH})^n,
\end{equation}
where $m_\text{TH}$ is the value of $\rmM$ at the inflection point. By replacing the constant and taking the finite dissolution rate of unexposed resist ($r_\text{min}$) into consideration, the final rate model is,
\begin{equation}
    \vr = r_\text{max}\frac{(a +1)(1-\rmM)^n}{a+(1-\rmM)^n} + r_\text{min},
\end{equation}
where $m_\text{TH}$, $r_\text{max}$, and $r_\text{min}$ should be determined experimentally.

Once the development rate is obtained, the front of developer can be computed by finding the time required to reach each point $\rmT(z,x,y)$, and we have,
\begin{equation}
    \left\vert \nabla\rmT(z,x,y)\right\vert = \frac{1}{\vr(z,x,y)}.
\end{equation}
If we use a simplified model where we only consider the vertical development path, the time can be computed as,
\begin{equation}
    \rmT(z,x,y) = \int_0^h\frac{dh}{\vr(z,x,y)}.
\end{equation}
However, the development path is general not strictly vertical, the fast-marching level-set methods and their variants\cite{osher2001level,sethian1996fast,dai2014fast} can be employed to solve this problem. Given the speed field $\vr$, the time required to reach each point in the field can be computed. And the final development result is the envelop $\rmT(z,x,y) = t_\text{dev}$, where $t_\text{dev}$ is the development time.


In summary, there is a group of parameters should be further decided in TorchResist. Then we decide the values with popular numerical methods. 

% 



\subsection{Parameter Optimization}



We have formulated the resist process in previous subsection, and denote it as $f_\theta(\cdot)$, several parameters $\theta$ still wait for further calibration. In our case, the calibration is conducted on a dataset of gray-scale aerial image and binary wafer image pairs $\{(\rmR_i,\rmW_i)\}_{i=1}^N$. The target of the optimization is to minimize the difference of the prediction and target by optimizing the parameter $\theta$ and $\tau$,
\begin{equation}
      \theta, \tau = \argmin_{\theta,\tau} \| \rmW - \Gamma(f_\theta(\rmR) , \tau)\|_0, \label{eq:object}
\end{equation}
where $\Gamma(\cdot, \tau)$ is a threshold function to binaryize the output of resist simulator and $\tau$ is an adjustable threshold. Obviously, directly optimize \Cref{eq:object} is difficult due to the exist of $L0$ norm and threshold function $\Gamma$. 

\minisection{Differentiable Object Functions} To address is issue, a widely applied trick is employing the sigmoid function $\sigma(\cdot)$ and replacing the $L0$-norm with the binary cross-entropy (BCE) between the predicted value and groundtruth.
\begin{gather*}
        \sigma(h) = \frac{e^h}{1+e^h}, \\
        \text{BCE}(h_1, h_2)= -[h_2\log(h_1) + (1-h_2)\log(1-h_1)],
\end{gather*}
where $h_1\in[0,1]$ is the prediction and $h_2\in\{0,1\}$ is the target. And the differentiable object function gives,
\begin{equation}
    \text{loss} = \text{BCE}(\sigma(sf_\theta(\rmR) - s\tau),\rmW), \label{eq:softobject}
\end{equation}
where $s$ is a scale factor to sharpen the boundary of prediction. We should note that every step in the formulation of resist model $f_\theta$ is differentiable, and we implement the process with modern automatic differentiation framework PyTorch\cite{paszke2019pytorch}. Therefore, the optimization of \Cref{eq:softobject} can be easily achieved with popular gradient-decent methods, which we will further detailed in next section. 


