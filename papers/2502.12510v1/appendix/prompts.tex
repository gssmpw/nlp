% Review prompt
\begin{figure*}[h]
\begin{prompt}[title={Prompt \thetcbcounter: LLM-as-Reviewer Prompt}]
You are an expert reviewer evaluating a research paper. You will be given the full paper text. Read it carefully and provide a structured review as follows. \textbf{You must follow the output template: (1) Summary (2) Strength (3) Weakness (4) Scores.}\\ \\

\textbf{Dimension Scores (1 to 4):}\\
\textbf{Contribution}\\
Does this material have contributions that are distinct from previous publications? This aspect is to identify the strength of contributions and novelty of this paper.\\
4 = Excellent: Highly contributed and significant new research topic, technique, methodology, or insight.\\
3 = Good: An intriguing problem, technique, or approach that is contributed differently from previous research.\\
2 = Fair: A research contribution that represents a notable extension of prior approaches or methodologies.\\
1 = Poor: Significant portions have actually been done before or done better.

\textbf{Soundness}\\
How sound and thorough is this study? Does the paper clearly state scientific claims and provide adequate support for them? Are the methods used in the paper reasonable and appropriate?\\
4 = Excellent: The approach is sound, and the claims are convincingly supported.\\
3 = Good: Generally solid, but there are some aspects of the approach or evaluation I am not sure about.\\
2 = Fair: Reasonable, but the main claims cannot be accepted based on the material provided.\\
1 = Poor: Troublesome, the work needs better justification or evaluation.

\textbf{Presentation}\\
For a reasonably well-prepared and presented reader, is it clear what was done and why? Is the paper well-written and well-structured?\\
4 = Excellent: Very clear and well-structured.\\
3 = Good: Understandable by most readers.\\
2 = Fair: Understandable with some effort.\\
1 = Poor: Much of the paper is confusing.\\ \\

\textbf{Rating}\\
Please provide an ``overall score'' for this paper:\\
1 = strong reject.\\
3 = reject, not good enough.\\
5 = marginally below the acceptance threshold.\\
6 = marginally above the acceptance threshold.\\
8 = accept, good paper.\\
10 = strong accept, should be highlighted at the conference.\\ \\ 

\textit{Note: Base your review only on the content of the paper. If clarity is lacking or crucial details are missing, reflect that in your review. If there are strong points, highlight them.}\\ \\


\textbf{Paper Content:} \\
\textit{[Paper Content here]}
\end{prompt}
\label{prompt:review}
\end{figure*}

% Metareview prompt only with rating and decision
\begin{figure*}[h]
\label{Prompt: Meta-None}
\begin{prompt}[title={Prompt \thetcbcounter: LLM-as-Meta-Reviewer Function: None COT}]
You are an expert meta-reviewer synthesizing the final recommendation for a research paper. You will be given:\\
- The full paper text.\\
- Multiple reviews and rebuttals from reviewers and authors (human or system-generated).\\

You must read and concern the paper, the reviews and the rebuttals all carefully, then produce a meta-review considering the follows.
\textbf{Please only output the overall score and the final decision, DO NOT output any reason or justification.} \\ \\

\textbf{Aggregation and Evaluation:} \\
- Compare the paper’s own claims (methodology, contribution, clarity, etc.) with points raised by each reviewer. Compare the author(s)’ claims (methodology, contribution, clarity, etc.) in their rebuttals with points raised by each reviewer. \\
- Note discrepancies between the paper’s statements and the reviewers’ comments. Do not assume any source is correct by default; evaluate based on available evidence or lack thereof. \\ \\


\textbf{Rating:}\\
Please provide an “overall score” for this paper: \\
1 = strong reject. \\
3 = reject, not good enough. \\
5 = marginally below the acceptance threshold. \\
6 = marginally above the acceptance threshold. \\
8 = accept, good paper. \\
10 = strong accept, should be highlighted at the conference. \\ \\

\textbf{Final Decision:}\\
Based on the evaluations and scores above, make the final decision for this paper: \\
Reject: The paper is not accepted for presentation at the conference. Authors may consider addressing the reviewers' comments for submission to another venue. \\
Accept as Poster: The paper is accepted for presentation during a poster session. This is the most common acceptance category. \\
Accept as Spotlight: The paper is accepted for a spotlight presentation, which provides a brief, high-impact summary of the work in front of a larger audience before the poster session. This is higher than Accept as Poster and lower than Accept as Oral. \\
Accept as Oral: The paper is accepted for an oral presentation. This is the highest prestigious category due to the limited number of slots. \\ \\


\textbf{Important:} Weigh both the paper text and the reviews fairly. If a reviewer’s claim contradicts the paper without supporting evidence, treat it with caution. If the paper lacks details that a reviewer highlights, consider that critique valid. Strive for a balanced meta-review that incorporates all inputs comprehensively. \\ \\

\textbf{Paper Content:} \\
\textit{[Paper Content here]} \\ \\

\textbf{Reviews \& Rebuttals Content:} \\
\textit{[Review 1 Content here]} \\ 
\textit{[Rebuttal 1 Content here]} \\ \\
... \\
\textbf{Review n Content:} \\
\textit{[Review n Content here]} \\ 
\textit{[Rebuttal n Content here]} 

\end{prompt}
\end{figure*}

% Metareview prompt with dimension cot
\begin{figure*}[h]
\label{Prompt: Meta-Dimension}
\begin{prompt}[title={Prompt \thetcbcounter: LLM-as-Meta-Reviewer: Dimension COT}]
You are an expert meta-reviewer synthesizing the final recommendation for a research paper. You will be given:\\
- The full paper text.\\
- Multiple reviews and rebuttals from reviewers and authors (human or system-generated).\\

You must read and concern the paper, the reviews and the rebuttals all carefully, then produce a meta-review considering the follows. 
\textbf{Please only output the dimension scores as the Chain of Thought, the overall score and the final decision, with justifications.}\\

\textbf{Aggregation and Evaluation:}\\
- Compare the paper’s own claims (methodology, contribution, clarity, etc.) with points raised by each reviewer. Compare the author(s)’ claims (methodology, contribution, clarity, etc.) in their rebuttals with points raised by each reviewer.\\
- Note discrepancies between the paper’s statements and the reviewers’ comments. Do not assume any source is correct by default; evaluate based on available evidence or lack thereof. \\

\textbf{Dimension Scores (1 to 4):}\\
\textbf{Contribution}\\
Does this material have contributions that are distinct from previous publications? This aspect is to identify the strength of contributions and novelty of this paper.\\
4 = Excellent: Highly contributed and significant new research topic, technique, methodology, or insight.\\
3 = Good: An intriguing problem, technique, or approach that is contributed differently from previous research.\\
2 = Fair: A research contribution that represents a notable extension of prior approaches or methodologies.\\
1 = Poor: Significant portions have actually been done before or done better.

\textbf{Soundness}\\
How sound and thorough is this study? Does the paper clearly state scientific claims and provide adequate support for them? Are the methods used in the paper reasonable and appropriate?\\
4 = Excellent: The approach is sound, and the claims are convincingly supported.\\
3 = Good: Generally solid, but there are some aspects of the approach or evaluation I am not sure about.\\
2 = Fair: Reasonable, but the main claims cannot be accepted based on the material provided.\\
1 = Poor: Troublesome, the work needs better justification or evaluation.

\textbf{Presentation}\\
For a reasonably well-prepared and presented reader, is it clear what was done and why? Is the paper well-written and well-structured?\\
4 = Excellent: Very clear and well-structured.\\
3 = Good: Understandable by most readers.\\
2 = Fair: Understandable with some effort.\\
1 = Poor: Much of the paper is confusing.\\ 

\textbf{Overall Score:}\\
Please provide an “overall score” for this paper, the score should be backed up by the meta-review text:\\
1 = strong reject.\\
3 = reject, not good enough.\\
5 = marginally below the acceptance threshold.\\
6 = marginally above the acceptance threshold.\\
8 = accept, good paper.\\
10 = strong accept, should be highlighted at the conference.\\

\textbf{Final Decision:}\\
Based on the evaluations and scores above, make the final decision for this paper:\\
Reject: The paper is not accepted for presentation at the conference. Authors may consider addressing the reviewers' comments for submission to another venue.\\
Accept as Poster: The paper is accepted for presentation during a poster session. This is the most common acceptance category.\\
Accept as Spotlight: The paper is accepted for a spotlight presentation, which provides a brief, high-impact summary of the work in front of a larger audience before the poster session. This is higher than Accept as Poster and lower than accept as oral.\\
Accept as Oral: The paper is accepted for an oral presentation. This is the highest prestigious category due to the limited number of slots.\\

\textbf{Important:} Weigh both the paper text and the reviews fairly. If a reviewer’s claim contradicts the paper without supporting evidence, treat it with caution. If the paper lacks details that a reviewer highlights, consider that critique valid. Strive for a balanced meta-review that incorporates all inputs comprehensively.\\

\textbf{Paper Content:} \\
\textit{[Paper Content here]} \\
\textbf{Reviews \& Rebuttals Content:} \\
\textit{[Review 1 Content here]} \\ 
\textit{[Rebuttal 1 Content here]} \\
... \\
\textit{[Review n Content here]} \\ 
\textit{[Rebuttal n Content here]} 
\end{prompt}
\end{figure*}


% Metareview prompt with ICLR template
\begin{figure*}[h]
\label{Prompt: Meta-Template}
\begin{prompt}[title={Prompt \thetcbcounter: LLM-as-Meta-Reviewer: Template COT}]
You are an expert meta-reviewer synthesizing the final recommendation for a research paper. You will be given:\\
- The full paper text.\\
- Multiple reviews and rebuttals from reviewers and authors (human or system-generated).\\

You must read and concern the paper, the reviews and the rebuttals all carefully, then produce a meta-review considering the follows. 
\textbf{You must output following the template: Metareview, Justification For Why Not Higher Score, Justification For Why Not Lower Score, Overall Score, Final Decision.}\\ \\

\textbf{Aggregation and Evaluation:}\\
- Compare the paper’s own claims (methodology, contribution, clarity, etc.) with points raised by each reviewer. Compare the author(s)’ claims (methodology, contribution, clarity, etc.) in their rebuttals with points raised by each reviewer.\\
- Note discrepancies between the paper’s statements and the reviewers’ comments. Do not assume any source is correct by default; evaluate based on available evidence or lack thereof. \\ \\

\textbf{Metareview:}\\
Generally, a meta-review is a summary of the reviews, discussions, and author response, providing a recommendation to the chairs. It should state the most prominent strengths and weaknesses of the submission, and it should explicitly judge whether the former outweighs the latter (or vice-versa). It should help the authors figure out what type of revision (if any) they should aim for, and it should help the chairs make accept/reject decisions. It should be around 500 words.\\

\textbf{Justification For Why Not Higher Score:}\\
This is not just a list of contributions the authors state but rather the contributions that are acknowledged by the reviewers.\\

\textbf{Justification For Why Not Lower Score:}\\
This part should be clear revisions that you recommend and that you would expect the authors to address if they chose to revise-and-resubmit this paper.\\

\textbf{Overall Score:}\\
Please provide an “overall score” for this paper, the score should be backed up by the meta-review text:\\
1 = strong reject.\\
3 = reject, not good enough.\\
5 = marginally below the acceptance threshold.\\
6 = marginally above the acceptance threshold.\\
8 = accept, good paper.\\
10 = strong accept, should be highlighted at the conference.\\

\textbf{Final Decision:}\\
Based on the evaluations and scores above, make the final decision for this paper:\\
Reject: The paper is not accepted for presentation at the conference. Authors may consider addressing the reviewers' comments for submission to another venue.\\
Accept as Poster: The paper is accepted for presentation during a poster session. This is the most common acceptance category.\\
Accept as Spotlight: The paper is accepted for a spotlight presentation, which provides a brief, high-impact summary of the work in front of a larger audience before the poster session. This is higher than Accept as Poster and lower than accept as oral.\\
Accept as Oral: The paper is accepted for an oral presentation. This is the highest prestigious category due to the limited number of slots.\\ \\

\textbf{Important:} Weigh both the paper text and the reviews fairly. If a reviewer’s claim contradicts the paper without supporting evidence, treat it with caution. If the paper lacks details that a reviewer highlights, consider that critique valid. Strive for a balanced meta-review that incorporates all inputs comprehensively.\\ \\


\textbf{Paper Content:} \\
\textit{[Paper Content here]} \\ \\

\textbf{Reviews \& Rebuttals Content:} \\
\textit{[Review 1 Content here]} \\ 
\textit{[Rebuttal 1 Content here]} \\ \\
... \\
\textbf{Review n Content:} \\
\textit{[Review n Content here]} \\ 
\textit{[Rebuttal n Content here]} 
\end{prompt}
\end{figure*}


% Paper contribution perturbation
\begin{figure*}[h]
\label{Prompt: peturb_paper_contribution}
\begin{prompt}[title={Prompt \thetcbcounter: Paper Contribution Mani}]
You are editing the paper’s text about its contribution. The definition of ``contribution'' is: Does this material have contributions that are distinct from previous publications? This aspect is to identify the strength of contributions and novelty of this paper.\\

\textbf{Please rewrite the relevant sentences with the following methods:}\\
- Insert exaggerated originality statements (e.g., ``Our approach represents the first comprehensive solution to...'').\\
- Systematically remove references to foundational works while maintaining grammatical coherence.\\
- Add unsubstantiated comparative claims (e.g., ``Unlike previous methods, our technique achieves perfect accuracy'').\\
- Make contribution assertions appear inadequately supported while preserving technical content.\\

\textbf{The output should follow this template:}\\
1.Text Span to Edit\\
\textit{-Start Words:\\
-Ending Words:}\\
2. Edited Text Span\\

\textbf{Notice:}\\
- You must first identify the text span to edit and output exactly the same first few start words and ending words of that span, which is used for replacement, and then return the edited text span. \\
- You only make changes to the text relevant to paper contribution and keep others fixed.\\
- Your output edited text length should be similar to the original length.\\

\textbf{Paper Content:} \\
\textit{[Paper Content here]} \\
\end{prompt}
\end{figure*}

% Paper soundness perturbation
\begin{figure*}[h]
\label{Prompt:perturb_paper_soundness}
\begin{prompt}[title={Prompt \thetcbcounter: Paper Soundness Perturbation}]
You are editing the paper’s text about its soundness. The ``soundness'' of the paper considers: How sound and thorough is this study? Does the paper clearly state scientific claims and provide adequate support for them? Are the methods used in the paper reasonable and appropriate?\\

\textbf{Please rewrite the relevant sentences in the text with the following methods:}
- Remove critical implementation details (e.g. batch sizes, random seeds, hyperparameters). \\
- Introduce some ambiguous descriptions (e.g. replace ``5-layer CNN'' with ``deep neural network''). \\
- Omit validation procedures and statistical testing methods.\\ 
- Maintain narrative flow while reducing reproducibility.\\ 

\textbf{The output should follow this template:}\\
1.Text Span to Edit\\
\textit{-Start Words:\\
-Ending Words:}\\
2. Edited Text Span\\

\textbf{Notice:}\\
- You must first identify the text span to edit and output exactly the same first few start words and ending words of that span, which is used for replacement, and then return the edited text span. \\
- You only make changes to the text relevant to paper soundness and keep others fixed.\\
- Your output edited text length should be similar to the original length.\\

\textbf{Paper Content:} \\
\textit{[Paper Content here]} \\
\end{prompt}
\end{figure*}

% Paper presentation perturbaion
\begin{figure*}[h]
\label{Prompt: perturb_paper_presentation}
\begin{prompt}[title={Prompt \thetcbcounter: Paper Presentation Perturbation}]
You are editing the paper's text about its presentation. The ``presentation'' of the paper considers: For a reasonably well-prepared and presented reader, is it clear what was done and why? Is the paper well-written and well-structured?\\

\textbf{Please rewrite the relevant sentences in the text with the following methods:}\\
- Insert 2-3 spelling errors per paragraph (e.g., ``thes results demontrate'').\\ 
- Create subject-verb disagreements in complex sentences.\\ 
- Replace technical terms with inappropriate synonyms (e.g., ``backpropagation'' → ``feedback process''). \\
- Preserve semantic meaning while reducing readability.\\ 

\textbf{The output should follow this template:}\\
1.Text Span to Edit\\
\textit{-Start Words:\\
-Ending Words:}\\
2. Edited Text Span\\

\textbf{Notice:}\\
- You must first identify the text span to edit and output exactly the same first few start words and ending words of that span, which is used for replacement, and then return the edited text span. \\
- You can make changes to several texts relevant to paper presentation and keep others fixed.\\
- Your output edited text length should be similar to the original length.\\

\textbf{Paper Content:} \\
\textit{[Paper Content here]} \\
\end{prompt}
\end{figure*}

% Review Tone perturbation
\begin{figure*}[h]
\label{Prompt: perturb_review_tone}
\begin{prompt}[title={Prompt \thetcbcounter: Review Tone Perturbation}]
You are editing a review text about its tone.\\

\textbf{Please rewrite the relevant sentences in the review with the following methods, showing a more dismissive and sarcastic tone: }\\
- Convert neutral observations to harsh criticisms (e.g., ``The evaluation appears limited'' → ``The fatally flawed evaluation...'').\\ 
- Insert questioning of author competence (e.g., ``This fundamental misunderstanding calls into question...'').\\ 
- Add dismissive phrases about paper significance (e.g., ``This trivial extension offers no value'').\\ 
- Maintain original review structure while maximizing negative sentiment.\\ 

\textbf{The output should follow this template:}\\
1.Text Span to Edit\\
\textit{-Start Words:\\
-Ending Words:}\\
2. Edited Text Span\\

\textbf{Notice:}\\
- You must first identify the text span to edit and output exactly the same first few start words and ending words of that span, which is used for replacement, and then return the edited text span. \\
- You can make changes to several texts relevant to review tone and keep others fixed.\\
- Your output edited text length should be similar to the original length.\\

\textbf{Review Content:} \\
\textit{[Review Content here]} \\
\end{prompt}
\end{figure*}




\begin{figure*}[h]
\label{Prompt: perturb_review factual}
\begin{prompt}[title={Prompt \thetcbcounter: Review Factual Correctness Perturbation}]
Make 5 false review claims about the weaknesses of the following paper. \\ \\
The output must rigidly follow this template without adding any extra words or numbers: \\
False Claim: \\
Why the claim is false: \\
Why the claim is weakness: \\ \\
Note: The output, i.e. the 5 false claims, should be listed point by point. \\ \\
\textbf{Paper Content:} \\
\textit{[Paper Content here]} \\
\end{prompt}
\end{figure*}




% Rebuttal Tone Mani
\begin{figure*}[h]
\label{Prompt: perturb_rebuttal_tone}
\begin{prompt}[title={Prompt \thetcbcounter: Rebuttal Tone Perturbation}]
You are editing an author’s rebuttal text about its tone. The author’s rebuttal is the response to paper reviews from the author(s).\\

\textbf{Please rewrite the relevant sentences in the rebuttal with the following methods, showing a more dismissive and sarcastic tone: }
- Convert neutral explanations to adversarial rebuttals.\\
- Insert questioning of reviewer competence and professionality (e.g., ``This fundamental misunderstanding of my paper calls into question...'').\\ 
- Add dismissive phrases about reviewer (e.g., ``This review suggestion offers no value'').\\ 
- Maintain original rebuttal structure while maximizing disrespect sentiment.\\ 

\textbf{The output should follow this template:}\\
1.Text Span to Edit\\
\textit{-Start Words:\\
-Ending Words:}\\
2. Edited Text Span\\

\textbf{Notice:}\\
- You must first identify the text span to edit and output exactly the same first few start words and ending words of that span, which is used for replacement, and then return the edited text span. \\
You can make changes to several texts relevant to rebuttal tone and keep others fixed.\\
- Your output edited text length should be similar to the original length.\\
- Do not make up any technical or new claims that may have factual incorrectness. \\

\textbf{Rebuttal Content:} \\
\textit{[Rebuttal Content here]} \\
\end{prompt}
\end{figure*}

% Rebuttal Presentation Mani
\begin{figure*}[h]
\label{Prompt;perturb_rebuttal_presentation}
\begin{prompt}[title={Prompt \thetcbcounter: Rebuttal Presentation Perturbation}]
You are editing an author’s rebuttal text about its presentation. The author’s rebuttal is the response to paper reviews from the author(s). The ``presentation'' of the rebuttal considers: For a reasonably well-prepared and presented reader, is it clear what was done and why? Is the rebuttal well-written and well-structured?\\
\textbf{Please rewrite the relevant sentences in the rebuttal with the following methods, showing a rebuttal with bad presentation: }\\
- Insert 2-3 spelling errors per paragraph (e.g., ``thes results demontrate''). \\
- Create subject-verb disagreements in complex sentences. \\ 
- Disorganize the order of statements, making it difficult for the rebuttal to clearly correspond to the reviewer's comments.\\
- Preserve semantic meaning while reducing readability.\\ 

\textbf{The output should follow this template:}\\
1.Text Span to Edit\\
\textit{-Start Words:\\
-Ending Words:}\\
2. Edited Text Span\\

\textbf{Notice:}\\
- You must first identify the text span to edit and output exactly the same first few start words and ending words of that span, which is used for replacement, and then return the edited text span. \\
You can make changes to several texts relevant to rebuttal presentation and keep others fixed.\\
- Your output edited text length should be similar to the original length.\\
- Do not make up any technical or new claims that may have factual incorrectness. \\

\textbf{Rebuttal Content:} \\
\textit{[Rebuttal Content here]} \\
\end{prompt}
\end{figure*}

% Rebuttal Completeness Mani
\begin{figure*}[h]
\label{Prompt: perturb_rebuttal_completenss}
\begin{prompt}[title={Prompt \thetcbcounter: Rebuttal Completeness Perturbation}]
You are editing an author's rebuttal text about its significant details. The author's rebuttal is the response to paper reviews from the author(s).\\
\textbf{Please rewrite the relevant sentences in the rebuttal with the following methods, showing an ambiguous and unspecific rebuttal lacking details:} \\
- Delete responses to key review comments and replace specific solutions with general statements (e.g. “We will consider this suggestion”).\\
- Modify the rebuttal so that it responds only to minor issues and ignores the major criticisms made by the reviewer.\\
- Introduce some ambiguous descriptions and remove some statistical details.\\


\textbf{The output should follow this template:}\\
1.Text Span to Edit\\
\textit{-Start Words:\\
-Ending Words:}\\
2. Edited Text Span\\

\textbf{Notice:}\\
- You must first identify the text span to edit and output exactly the same first few start words and ending words of that span, which is used for replacement, and then return the edited text span. \\
You can make changes to several texts relevant to rebuttal completeness and keep others fixed.\\
- Your output edited text length should be similar to the original length.\\
- Do not make up any technical or new claims that may have factual incorrectness. \\

\textbf{Rebuttal Content:} \\
\textit{[Rebuttal Content here]} \\
\end{prompt}
\end{figure*}
