\section{Related Work}
Analog circuit design optimization has increasingly benefited from machine learning approaches, particularly reinforcement learning (RL) and Bayesian optimization (BO), given the complexity and high dimensionality of the design space. In RL, methods such as \cite{prl} employ prioritized sampling of replay buffers for more effective exploration, while \cite{autockt} utilizes algorithms like proximal policy optimization \cite{ppo} to optimize stochastic policies efficiently. Additionally, \cite{gcn_rl} integrates graph neural networks to enable transfer learning for circuit optimization across different technology nodes.

Bayesian optimization \cite{BO}, on the other hand, is widely adopted for its ability to balance exploration and exploitation in expensive simulations. Techniques like \cite{batch_BO} employ multiple acquisition functions to suggest design points, taking into account trade-offs between different acquisition strategies. Meanwhile, \cite{parallel_BO} improves the efficiency of BO through batch querying and parallel execution.

Recently, large language models (LLMs) have shown promise in analog circuit design automation. LADAC \cite{LADAC} automates design parameter selection using LLM-driven suggestions. In-context learning \cite{in_context_learning} and chain of thought prompting \cite{chain_of_thought} have enhanced LLMs’ ability to reason through complex design trade-offs and generate high-quality solutions. More recently, ADO-LLM \cite{ado-llm} integrates BO with in-context learning, leveraging LLMs to suggest design points based on prior knowledge.

However, for more efficient analog design, emulating human designers through information reuse is crucial. While \cite{gcn_rl} applies transfer learning through graph-based optimization, it lacks a semantic understanding of the design space. To address this, we propose LLM-USO, which combines the strengths of LLMs and BO to achieve more efficient analog design optimization with enhanced information reuse.