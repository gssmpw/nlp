\section{Related Work}
Analog circuit design optimization has increasingly benefited from machine learning approaches, particularly reinforcement learning (RL) and Bayesian optimization (BO), given the complexity and high dimensionality of the design space. In RL, methods such as ____ employ prioritized sampling of replay buffers for more effective exploration, while ____ utilizes algorithms like proximal policy optimization ____ to optimize stochastic policies efficiently. Additionally, ____ integrates graph neural networks to enable transfer learning for circuit optimization across different technology nodes.

Bayesian optimization ____, on the other hand, is widely adopted for its ability to balance exploration and exploitation in expensive simulations. Techniques like ____ employ multiple acquisition functions to suggest design points, taking into account trade-offs between different acquisition strategies. Meanwhile, ____ improves the efficiency of BO through batch querying and parallel execution.

Recently, large language models (LLMs) have shown promise in analog circuit design automation. LADAC ____ automates design parameter selection using LLM-driven suggestions. In-context learning ____ and chain of thought prompting ____ have enhanced LLMs’ ability to reason through complex design trade-offs and generate high-quality solutions. More recently, ADO-LLM ____ integrates BO with in-context learning, leveraging LLMs to suggest design points based on prior knowledge.

However, for more efficient analog design, emulating human designers through information reuse is crucial. While ____ applies transfer learning through graph-based optimization, it lacks a semantic understanding of the design space. To address this, we propose LLM-USO, which combines the strengths of LLMs and BO to achieve more efficient analog design optimization with enhanced information reuse.