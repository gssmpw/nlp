% This must be in the first 5 lines to tell arXiv to use pdfLaTeX, which is strongly recommended.
\pdfoutput=1
% In particular, the hyperref package requires pdfLaTeX in order to break URLs across lines.

\documentclass[11pt]{article}

% Change "review" to "final" to generate the final (sometimes called camera-ready) version.
% Change to "preprint" to generate a non-anonymous version with page numbers.
%\usepackage[review]{acl}
\usepackage[final]{acl}

% Standard package includes
\usepackage{amsmath}
\usepackage{times}
\usepackage{latexsym}
\usepackage{booktabs}
\usepackage{float}
\usepackage{multirow}
\usepackage{xspace}
\usepackage{wasysym}
\usepackage{amssymb}
\usepackage{pifont}
\usepackage{xcolor}         % colors
\usepackage{colortbl}
\usepackage{color}
\usepackage{makecell}
\usepackage{geometry}
\usepackage{mathtools}
\usepackage[most, breakable, many]{tcolorbox}
\usepackage{ulem} 


\definecolor{Gray}{gray}{0.90}

\definecolor{mypurple}{HTML}{25004D}
\definecolor{darkgreen}{HTML}{2D8659}
\definecolor{darkred}{HTML}{990000}

\newcommand{\cmark}{\textcolor{darkgreen}{\ding{51}}} % green check mark
\newcommand{\xmark}{\textcolor{red}{\ding{55}}} % red cross mark

% For proper rendering and hyphenation of words containing Latin characters (including in bib files)
\usepackage[T1]{fontenc}
% For Vietnamese characters
% \usepackage[T5]{fontenc}
% See https://www.latex-project.org/help/documentation/encguide.pdf for other character sets

% This assumes your files are encoded as UTF8
\usepackage[utf8]{inputenc}

% This is not strictly necessary, and may be commented out,
% but it will improve the layout of the manuscript,
% and will typically save some space.
\usepackage{microtype}

% This is also not strictly necessary, and may be commented out.
% However, it will improve the aesthetics of text in
% the typewriter font.
\usepackage{inconsolata}

%Including images in your LaTeX document requires adding
%additional package(s)
\usepackage{graphicx}

\usepackage{subfigure}
\usepackage{enumitem}
\newenvironment{itemize*}%
 {\leftmargini=20pt\begin{itemize}%
  \setlength{\itemsep}{3pt}%
  \setlength{\parskip}{0pt}%
  }%
 {\end{itemize}}
\newenvironment{enumerate*}%
 {\begin{enumerate}%
  \setlength{\itemsep}{0pt}%
  \setlength{\parskip}{0pt}}%
 {\end{enumerate}}

% If the title and author information does not fit in the area allocated, uncomment the following
%
%\setlength\titlebox{<dim>}
%
% and set <dim> to something 5cm or larger.

%\title{Fine-Tuning Open-Source LLMs for Task-Oriented Multiturn Dialogue Agents}
% \title{ActTOD: An Open-Source Task-Oriented Dialogue Agent for Complex Multi-Turn Conversations}
%\title{ACTOR: Agent-based Conversational Training for Task Oriented Reasoning}
%\title{Towards to Multiturn Conversational Agents with Task-Oriented Dialogue Reasoning and Acting}
%\title{ReActTOD: Task-Oriented Dialogue Reasoning and Acting for Multiturn Conversational Agents}
%\title{Are Conversational Agents Truly Ready for Task-Oriented Dialogue and Function Calling?}
%\title{Why Do Conversational Agents Still Fail? Rethinking the Integration of Dialogue and Function Calling}
%\title{Why Can't Dialogue Systems Be Agents, and Agents Be Conversational? Unifying Capabilities for Next-Generation Task Oriented Dialogues}
% \title{Can Dialogue Systems Be Agents, and Agents Be Conversational? \\Unifying Capabilities for Next-Generation Task Oriented Dialogues}


%\title{Can Dialogue Systems Be Agents, and Agents Be Conversational? \\Unifying Capabilities for Agentic Task Oriented Dialogues}

% \title{
% %Can Dialogue Systems Be Agents, and Agents Be Conversational?\\
% CALM: A Unified Language Agent for Task-Oriented Dialogues}


% \title{
% Can Dialogue Systems Be Agents, and Agents Be Conversational?\\
% CALM: A Conversational Agentic Language Model  for Bridging the Chasm Between TOD Systems and Language Agents\\ 
% }


% \title{
% Can Dialogue Systems Be Agents, and Agents Be Conversational?\\
% CALM: A Unified Conversational Agentic Language Model 
% }

\title{
Can a Single Model Master Both Multi-turn Conversations and Tool Use?\\
CoALM: A Unified Conversational Agentic Language Model 
}

% Author information can be set in various styles:
% For several authors from the same institution:
% \author{Author 1 \and ... \and Author n \\
%         Address line \\ ... \\ Address line}
% if the names do not fit well on one line use
%         Author 1 \\ {\bf Author 2} \\ ... \\ {\bf Author n} \\
% For authors from different institutions:
% \author{Author 1 \\ Address line \\  ... \\ Address line
%         \And  ... \And
%         Author n \\ Address line \\ ... \\ Address line}
% To start a separate ``row'' of authors use \AND, as in
% \author{Author 1 \\ Address line \\  ... \\ Address line
%         \AND
%         Author 2 \\ Address line \\ ... \\ Address line \And
%         Author 3 \\ Address line \\ ... \\ Address line}

\author{
Emre Can Acikgoz$^{1}$, Jeremiah Greer$^{2}$, Akul Datta$^{1}$, Ze Yang$^{1}$, William Zeng$^{2}$,\\
\textbf{Oussama Elachqar$^{2}$, Emmanouil Koukoumidis$^{2}$, Dilek Hakkani-Tür$^{1}$, Gokhan Tur$^{1}$}\\
$^{1}$University of Illinois Urbana-Champaign, $^{2}$Oumi\\
\texttt{\{acikgoz2, akuld2, zey2, dilek, gokhan\}@illinois.edu}\\
\texttt{\{jeremy, william, oussama, manos\}@oumi.ai}\\
}

%\author{
%  \textbf{First Author\textsuperscript{1}},
%  \textbf{Second Author\textsuperscript{1,2}},
%  \textbf{Third T. Author\textsuperscript{1}},
%  \textbf{Fourth Author\textsuperscript{1}},
%\\
%  \textbf{Fifth Author\textsuperscript{1,2}},
%  \textbf{Sixth Author\textsuperscript{1}},
%  \textbf{Seventh Author\textsuperscript{1}},
%  \textbf{Eighth Author \textsuperscript{1,2,3,4}},
%\\
%  \textbf{Ninth Author\textsuperscript{1}},
%  \textbf{Tenth Author\textsuperscript{1}},
%  \textbf{Eleventh E. Author\textsuperscript{1,2,3,4,5}},
%  \textbf{Twelfth Author\textsuperscript{1}},
%\\
%  \textbf{Thirteenth Author\textsuperscript{3}},
%  \textbf{Fourteenth F. Author\textsuperscript{2,4}},
%  \textbf{Fifteenth Author\textsuperscript{1}},
%  \textbf{Sixteenth Author\textsuperscript{1}},
%\\
%  \textbf{Seventeenth S. Author\textsuperscript{4,5}},
%  \textbf{Eighteenth Author\textsuperscript{3,4}},
%  \textbf{Nineteenth N. Author\textsuperscript{2,5}},
%  \textbf{Twentieth Author\textsuperscript{1}}
%\\
%\\
%  \textsuperscript{1}Affiliation 1,
%  \textsuperscript{2}Affiliation 2,
%  \textsuperscript{3}Affiliation 3,
%  \textsuperscript{4}Affiliation 4,
%  \textsuperscript{5}Affiliation 5
%\\
%  \small{
%    \textbf{Correspondence:} \href{mailto:email@domain}{email@domain}
%  }
%}

\begin{document}
\maketitle
\begin{abstract}
% Large Language Models (LLMs) have revolutionized task-oriented conversational systems, such as those powering ChatGPT and GPT-4. 
% Despite their success, these models face challenges in handling complex, task-specific dialogues and sometimes be costly. Additionally, there is a critical gap in the adaptation of open-source LLMs for task-oriented dialogue applications.
% In this paper, we present Conv-LLaMA, an open-source task-oriented dialogue agent designed to manage complex, multi-turn conversations. 
% By leveraging domain-specific datasets and advanced fine-tuning techniques, Conv-LLaMA offers a robust and cost-effective alternative to closed-source models like GPT-4 and open-source models that lack domain-specific dialogue capabilities.
% We evaluate the model's performance in terms of accuracy and dialogue management capabilities, demonstrating its potential as a robust tool for task-oriented dialogue systems.
\begin{abstract}


The choice of representation for geographic location significantly impacts the accuracy of models for a broad range of geospatial tasks, including fine-grained species classification, population density estimation, and biome classification. Recent works like SatCLIP and GeoCLIP learn such representations by contrastively aligning geolocation with co-located images. While these methods work exceptionally well, in this paper, we posit that the current training strategies fail to fully capture the important visual features. We provide an information theoretic perspective on why the resulting embeddings from these methods discard crucial visual information that is important for many downstream tasks. To solve this problem, we propose a novel retrieval-augmented strategy called RANGE. We build our method on the intuition that the visual features of a location can be estimated by combining the visual features from multiple similar-looking locations. We evaluate our method across a wide variety of tasks. Our results show that RANGE outperforms the existing state-of-the-art models with significant margins in most tasks. We show gains of up to 13.1\% on classification tasks and 0.145 $R^2$ on regression tasks. All our code and models will be made available at: \href{https://github.com/mvrl/RANGE}{https://github.com/mvrl/RANGE}.

\end{abstract}


\end{abstract}

% TODO: Bir de aralara think koyuyoruz ya, o cok gec geliyor. Ondan intro’da da bahsetsel iyi olabilir. Belki annotation guided think step generation filan gibi egzantrik bir isim de dusunebiliriz ona
% O kismi contribution’a cevirmeye calisalim. Yani nasil annotation urettik buna buyuk model kullanaral, filan…

\section{Introduction}

Video generation has garnered significant attention owing to its transformative potential across a wide range of applications, such media content creation~\citep{polyak2024movie}, advertising~\citep{zhang2024virbo,bacher2021advert}, video games~\citep{yang2024playable,valevski2024diffusion, oasis2024}, and world model simulators~\citep{ha2018world, videoworldsimulators2024, agarwal2025cosmos}. Benefiting from advanced generative algorithms~\citep{goodfellow2014generative, ho2020denoising, liu2023flow, lipman2023flow}, scalable model architectures~\citep{vaswani2017attention, peebles2023scalable}, vast amounts of internet-sourced data~\citep{chen2024panda, nan2024openvid, ju2024miradata}, and ongoing expansion of computing capabilities~\citep{nvidia2022h100, nvidia2023dgxgh200, nvidia2024h200nvl}, remarkable advancements have been achieved in the field of video generation~\citep{ho2022video, ho2022imagen, singer2023makeavideo, blattmann2023align, videoworldsimulators2024, kuaishou2024klingai, yang2024cogvideox, jin2024pyramidal, polyak2024movie, kong2024hunyuanvideo, ji2024prompt}.


In this work, we present \textbf{\ours}, a family of rectified flow~\citep{lipman2023flow, liu2023flow} transformer models designed for joint image and video generation, establishing a pathway toward industry-grade performance. This report centers on four key components: data curation, model architecture design, flow formulation, and training infrastructure optimization—each rigorously refined to meet the demands of high-quality, large-scale video generation.


\begin{figure}[ht]
    \centering
    \begin{subfigure}[b]{0.82\linewidth}
        \centering
        \includegraphics[width=\linewidth]{figures/t2i_1024.pdf}
        \caption{Text-to-Image Samples}\label{fig:main-demo-t2i}
    \end{subfigure}
    \vfill
    \begin{subfigure}[b]{0.82\linewidth}
        \centering
        \includegraphics[width=\linewidth]{figures/t2v_samples.pdf}
        \caption{Text-to-Video Samples}\label{fig:main-demo-t2v}
    \end{subfigure}
\caption{\textbf{Generated samples from \ours.} Key components are highlighted in \textcolor{red}{\textbf{RED}}.}\label{fig:main-demo}
\end{figure}


First, we present a comprehensive data processing pipeline designed to construct large-scale, high-quality image and video-text datasets. The pipeline integrates multiple advanced techniques, including video and image filtering based on aesthetic scores, OCR-driven content analysis, and subjective evaluations, to ensure exceptional visual and contextual quality. Furthermore, we employ multimodal large language models~(MLLMs)~\citep{yuan2025tarsier2} to generate dense and contextually aligned captions, which are subsequently refined using an additional large language model~(LLM)~\citep{yang2024qwen2} to enhance their accuracy, fluency, and descriptive richness. As a result, we have curated a robust training dataset comprising approximately 36M video-text pairs and 160M image-text pairs, which are proven sufficient for training industry-level generative models.

Secondly, we take a pioneering step by applying rectified flow formulation~\citep{lipman2023flow} for joint image and video generation, implemented through the \ours model family, which comprises Transformer architectures with 2B and 8B parameters. At its core, the \ours framework employs a 3D joint image-video variational autoencoder (VAE) to compress image and video inputs into a shared latent space, facilitating unified representation. This shared latent space is coupled with a full-attention~\citep{vaswani2017attention} mechanism, enabling seamless joint training of image and video. This architecture delivers high-quality, coherent outputs across both images and videos, establishing a unified framework for visual generation tasks.


Furthermore, to support the training of \ours at scale, we have developed a robust infrastructure tailored for large-scale model training. Our approach incorporates advanced parallelism strategies~\citep{jacobs2023deepspeed, pytorch_fsdp} to manage memory efficiently during long-context training. Additionally, we employ ByteCheckpoint~\citep{wan2024bytecheckpoint} for high-performance checkpointing and integrate fault-tolerant mechanisms from MegaScale~\citep{jiang2024megascale} to ensure stability and scalability across large GPU clusters. These optimizations enable \ours to handle the computational and data challenges of generative modeling with exceptional efficiency and reliability.


We evaluate \ours on both text-to-image and text-to-video benchmarks to highlight its competitive advantages. For text-to-image generation, \ours-T2I demonstrates strong performance across multiple benchmarks, including T2I-CompBench~\citep{huang2023t2i-compbench}, GenEval~\citep{ghosh2024geneval}, and DPG-Bench~\citep{hu2024ella_dbgbench}, excelling in both visual quality and text-image alignment. In text-to-video benchmarks, \ours-T2V achieves state-of-the-art performance on the UCF-101~\citep{ucf101} zero-shot generation task. Additionally, \ours-T2V attains an impressive score of \textbf{84.85} on VBench~\citep{huang2024vbench}, securing the top position on the leaderboard (as of 2025-01-25) and surpassing several leading commercial text-to-video models. Qualitative results, illustrated in \Cref{fig:main-demo}, further demonstrate the superior quality of the generated media samples. These findings underscore \ours's effectiveness in multi-modal generation and its potential as a high-performing solution for both research and commercial applications.
\section{Related Work}

\subsection{Large 3D Reconstruction Models}
Recently, generalized feed-forward models for 3D reconstruction from sparse input views have garnered considerable attention due to their applicability in heavily under-constrained scenarios. The Large Reconstruction Model (LRM)~\cite{hong2023lrm} uses a transformer-based encoder-decoder pipeline to infer a NeRF reconstruction from just a single image. Newer iterations have shifted the focus towards generating 3D Gaussian representations from four input images~\cite{tang2025lgm, xu2024grm, zhang2025gslrm, charatan2024pixelsplat, chen2025mvsplat, liu2025mvsgaussian}, showing remarkable novel view synthesis results. The paradigm of transformer-based sparse 3D reconstruction has also successfully been applied to lifting monocular videos to 4D~\cite{ren2024l4gm}. \\
Yet, none of the existing works in the domain have studied the use-case of inferring \textit{animatable} 3D representations from sparse input images, which is the focus of our work. To this end, we build on top of the Large Gaussian Reconstruction Model (GRM)~\cite{xu2024grm}.

\subsection{3D-aware Portrait Animation}
A different line of work focuses on animating portraits in a 3D-aware manner.
MegaPortraits~\cite{drobyshev2022megaportraits} builds a 3D Volume given a source and driving image, and renders the animated source actor via orthographic projection with subsequent 2D neural rendering.
3D morphable models (3DMMs)~\cite{blanz19993dmm} are extensively used to obtain more interpretable control over the portrait animation. For example, StyleRig~\cite{tewari2020stylerig} demonstrates how a 3DMM can be used to control the data generated from a pre-trained StyleGAN~\cite{karras2019stylegan} network. ROME~\cite{khakhulin2022rome} predicts vertex offsets and texture of a FLAME~\cite{li2017flame} mesh from the input image.
A TriPlane representation is inferred and animated via FLAME~\cite{li2017flame} in multiple methods like Portrait4D~\cite{deng2024portrait4d}, Portrait4D-v2~\cite{deng2024portrait4dv2}, and GPAvatar~\cite{chu2024gpavatar}.
Others, such as VOODOO 3D~\cite{tran2024voodoo3d} and VOODOO XP~\cite{tran2024voodooxp}, learn their own expression encoder to drive the source person in a more detailed manner. \\
All of the aforementioned methods require nothing more than a single image of a person to animate it. This allows them to train on large monocular video datasets to infer a very generic motion prior that even translates to paintings or cartoon characters. However, due to their task formulation, these methods mostly focus on image synthesis from a frontal camera, often trading 3D consistency for better image quality by using 2D screen-space neural renderers. In contrast, our work aims to produce a truthful and complete 3D avatar representation from the input images that can be viewed from any angle.  

\subsection{Photo-realistic 3D Face Models}
The increasing availability of large-scale multi-view face datasets~\cite{kirschstein2023nersemble, ava256, pan2024renderme360, yang2020facescape} has enabled building photo-realistic 3D face models that learn a detailed prior over both geometry and appearance of human faces. HeadNeRF~\cite{hong2022headnerf} conditions a Neural Radiance Field (NeRF)~\cite{mildenhall2021nerf} on identity, expression, albedo, and illumination codes. VRMM~\cite{yang2024vrmm} builds a high-quality and relightable 3D face model using volumetric primitives~\cite{lombardi2021mvp}. One2Avatar~\cite{yu2024one2avatar} extends a 3DMM by anchoring a radiance field to its surface. More recently, GPHM~\cite{xu2025gphm} and HeadGAP~\cite{zheng2024headgap} have adopted 3D Gaussians to build a photo-realistic 3D face model. \\
Photo-realistic 3D face models learn a powerful prior over human facial appearance and geometry, which can be fitted to a single or multiple images of a person, effectively inferring a 3D head avatar. However, the fitting procedure itself is non-trivial and often requires expensive test-time optimization, impeding casual use-cases on consumer-grade devices. While this limitation may be circumvented by learning a generalized encoder that maps images into the 3D face model's latent space, another fundamental limitation remains. Even with more multi-view face datasets being published, the number of available training subjects rarely exceeds the thousands, making it hard to truly learn the full distibution of human facial appearance. Instead, our approach avoids generalizing over the identity axis by conditioning on some images of a person, and only generalizes over the expression axis for which plenty of data is available. 

A similar motivation has inspired recent work on codec avatars where a generalized network infers an animatable 3D representation given a registered mesh of a person~\cite{cao2022authentic, li2024uravatar}.
The resulting avatars exhibit excellent quality at the cost of several minutes of video capture per subject and expensive test-time optimization.
For example, URAvatar~\cite{li2024uravatar} finetunes their network on the given video recording for 3 hours on 8 A100 GPUs, making inference on consumer-grade devices impossible. In contrast, our approach directly regresses the final 3D head avatar from just four input images without the need for expensive test-time fine-tuning.


\section{Problem Formulation}
In this paper, the recommendation task takes the user behavior data as input. 
Let $\mathcal{U}$ and $\mathcal{V}$ be sets of users and items respectively, where $|\mathcal{U}|= m$, and $|\mathcal{V}|= n$. We use the index $u \in \mathcal{U}$ to denote a user and $i\in \mathcal{V}$ to denote an item. 
The user-item rating matrix is denoted as $\mathbf{R} = [r_i^u]^{m\times n}\in \mathbb{R}^{m\times n}$ to indicate whether user $u$ has interacted with item $i$, where $r_i^u=1$ represents user $u$ has interacted with item $i$,  whereas $r_i^u=0$ represents user $u$ has not interacted with item $i$. 
We use $\mathcal{V}^{+}_{u}=\{i\in\mathcal{V}|r_i^u=1\}$ to represent a set of items that user $u$ has interacted with.
$\mathcal{V}^{+}_{u}$ can be splited into a training set $\mathcal{S}_{u}^{+}$ and a testing set $\mathcal{T}_{u}$, requiring that $\mathcal{S}_{u}^{+} \cup \mathcal{T}_{u} = \mathcal{V}^{+}_{u}$ and $\mathcal{S}_{u}^{+} \cap \mathcal{T}_{u} = \emptyset$. It worth noted that $\mathcal{S}_u^{-}=\{i|r_i^u=0,i\in \mathcal{I}\}$, which means $\mathcal{S}_u^{-}$ consists of the negative items that user $u$ have not interacted with. The training set is denoted as $\mathcal{D}=\{(u,i, j)|u\in \mathcal{U}, i \in \mathcal{S}_{u}^{+}, j \in \mathcal{S}_{u}^{-}\}$. The testing set is denoted as $\mathcal{\hat{D}}=\{(u,i)|u\in \mathcal{U}, i \in \mathcal{T}_{u}\}$.

In the recommendation task, the model aims to recommend a list of $k$ items $\mathcal{X}_u$ for the user $u$, which matches the condition $\mathcal{X}_u\cap \mathcal{S}_u^{+}=\emptyset$. 
By comparing the recommendation list $\mathcal{X}_u$ with the testing set $\mathcal{T}_u$, we evaluate the recommendation quality from various perspectives, including accuracy, diversity, and fairness. 

\section{methodology}
% \section{Interest Unit-based Product Organization}
This chapter introduces the construction of interest units, the redesign of product forms with new interaction interface, and the IU-Boosted CTR prediction model integrating interest unit.

\subsection{Interest Unit-based Product Organization}

\begin{figure}[tbp]
\includegraphics[width=8cm]{gsid_arxiv.png}
\caption{One example of the foundational understanding system generated by the semantic clustering method.}
\label{fig:gsid_tree}
\end{figure}

User behaviors such as browsing, searching, clicking, or purchasing are primarily driven by underlying needs, which can be abstracted into specific interest units. These interest units can range from concrete product instances (e.g., "Iphone15 ProMax 256G") to broad demand categories (e.g., "concert tickets"). 
By predefining interest units and systematically associating relevant products with these constructed interest units, platforms can enhance user engaging experience and demand-matching efficiency.

\subsubsection{\textbf{Construction of Interest Unit}} Despite the diverse needs of Xianyu users, we believe that the core demands can be exhaustively identified to some extent. We have adopted a data-driven, bottom-up analytical framework that constructs interest units from the perspectives of product attributes and user needs, reorganizing the vast array of products on the Xianyu platform. 
% Compared to traditional knowledge construction that relies on manual operations, this method overcomes its limitations and offers greater flexibility.
\\ \textbf{I. Attribute-Driven} \textit{(Based on intrinsic product attributes)} \\
When browsing, users tend to focus on the core attribute information of a product. By combining these core attributes, we can essentially exhaustively identify the core demands users have for a certain category of products. Therefore, based on product attribute information, we have defined two types of user interest units.
\begin{itemize}
    \item SPU Interest Units: Product attributes, such as category and brand, are important on e-commerce platforms. For standard products with comprehensive structured attributes, we aggregate products into SPU Interest units based on the CPV (customer perceived value) information provided by users or identified by algorithms, forming a type of interest unit.
    \item Image Cluster Interest Unit: Product image information is also crucial when users are browsing. For non-standard products where structured attributes are difficult to define, we use product image information to aggregate products with similar appearances into clusters, thereby defining a type of interest unit based on these clusters.
\end{itemize}
\textbf{II. Demand-Aware} \textit{(Balancing product attributes and user needs)}
Not all categories have a one-to-one match between product supply and buyer demand. To balance buyer needs and seller supply during the construction of interest units, we developed a Query-Aware semantic unit generation system called Generative Semantic ID~\footnote{Another systematic effort from industry practice, which isn't the focus of this paper, will be briefly mentioned below. This work will soon be under review.}(GSID), based on open knowledge from large models and combined with the vast interaction data of "query-product" on Xianyu. This system defines Semantic interest units. The Xianyu GSID is a hierarchical tree structure, as shown in the figure~\ref{fig:gsid_tree}. GSID includes three levels, each containing 128 IDs, with the second level space being approximately 16,000 and the third level space being approximately 2.1 million.
Specifically, the Xianyu GSID algorithm uses the encoder-decoder network of the T5 model as the backbone structure. The encoder is a BERT-based vector encoder responsible for extracting product semantic vectors and for the decoder combines the encoder output vector at each decoding step with the previous decoding result to produce the current decoding vector, and then looks up the corresponding theme in the CodeBook to discretize and generate hierarchical semantic IDs.
\subsubsection{\textbf{Redesign of Interaction Interface}}
\begin{figure}[tbp]
\includegraphics[width=8cm]{product_arxiv.png}
\caption{The redesigned product format. Left is stage one style  and right image is stage two style with explanationo on the middle}
\label{fig:new_product}
\end{figure}
% 
Once these basic interest units are constructed, they can not only be incorporated into algorithmic modeling but also further utilized to change the way products are presented on the homepage recommendation interface. As shown in Figure \ref{fig:new_product}, we implemented a series of changes to clearly express user interest units: (1) On the homepage recommendations, we display the theme of the associated interest unit next to the product (left side of Figure \ref{fig:new_product}). (2) On the secondary landing page, products within the same interest unit are arranged together, making it easier for users to select products efficiently (right side of Figure \ref{fig:new_product}, product prototype image), with explanations for each module on the secondary landing page in between.
It is noteworthy that this new product format naturally aligns with the aforementioned two-stage recommendation paradigm, allowing for a more organic combination of algorithm models and product formats, significantly enhancing efficiency. Once the product set for interest units is delineated, we need to generate a front-end title for the interest unit to facilitate user understanding. This information is also displayed at the bottom of the card in homepage and at the top of the secondary page. The title of the interest unit is automatically generated by a large language model, by feeding corresponding descriptive information of N products randomly selected from each interest unit.

\subsection{Interest Unit-based Recommendation}\label{sec:recommendation}
\begin{figure*}[tbp]
    \includegraphics[width=14cm]{model_arxiv.png}
    \caption{An overview of proposed IU-Boosted Network, which consists of three components: (1) the interest unit-level feature for each product, (2) the user's hierarchical IU click sequence to determine their interest unit preference, and (3) the attention mechanism introduced for handling multiple items within the interest unit.}
    \label{fig:model_overview}
    % \vspace{-0.4cm}
\end{figure*}

As shown in Figure \ref{fig:new_product}, the upgrade in the homepage product format has led to significant changes in user navigation paths: user interactions are no longer confined to individual products but can occur across multiple products under the same interest unit. Additionally, behaviors of different users within the same interest unit can be aggregated and accumulated. 

Building upon this, we construct IU-level features to reflect the attributes of each IU and hierarchical IU click sequences using attention mechanism to user interest unit interest. We name this recommendation algorithm, which leverages behavior accumulation on Interest Unit (IU), as \textbf{IU-Boosted Network}. In this section, we will introduce the components of our proposed method in detail.

\subsubsection{\textbf{IU-Level Feature Construction}}
We accumulate behaviors of different users across all products under the same interest unit to construct IU-Level features, serving as foundational attributes of the interest unit. Products may be deleted after being sold, resulting in the obsolete of the accumulated information on their product IDs. However, the information aggregated on their associated interest unit remains permanently accessible. When new products are launched, we can attach their related interest unit attributes to enhance recommendation efficiency. Based on this, we develop multi-dimensional features to optimize recommendation performance:
(1) Statistical Features IU Dimension: Include various behavioral metrics such as impressions, clicks, inquiries, and transactions, reflecting the overall performance and popularity of the interest unit.
2) User-IU Cross Feature: Capture interaction patterns and frequencies between users and specific interest units or specific types of interest units.
\subsubsection{\textbf{IU Hierarchical Click Sequences}}
Users may exhibit multiple behaviors under the same interest unit, where the number of interactions reflects the intensity of their preference for the interest unit. We construct hierarchical IU click sequences to model user preferences at the interest unit level for refined recommendations. 
The normal item click sequence takes the following form:
\begin{equation}
    \boldsymbol{E}(Item \ Seq)=
    Concat [\boldsymbol E({Item\_i}), i = 1\ldots, m],
\end{equation}
where $\boldsymbol{E}\left({Item}\right)$ means the embedding representation for items consist of ID feature and side feature:
\begin{equation}
    \boldsymbol E\left({Item}\right)=Concat [\boldsymbol E\left(\mathcal{F}_{Item\_ID}\right), \boldsymbol E\left(\mathcal{F}_{Item\_Side}\right)],
\end{equation}

The embedding representation of IU and the IU click sequence can be expressed as followed:
\begin{equation}
    \boldsymbol{E}\left(IU\right)=
    Concat [\boldsymbol E\left(\mathcal{F}_{IU\_ID}\right), \boldsymbol E\left(\mathcal{F}_{IU\_Side}\right),
    \boldsymbol{E}\left(Item \ Seq\right)],
\end{equation}
\begin{equation}
    \boldsymbol{E}\left(IU \ Seq\right)=
    Concat [\boldsymbol E\left({IU\_1}\right), \boldsymbol E\left({IU\_2}\right), \ldots,
    \boldsymbol E\left({IU\_n}\right)],
\end{equation}
where $\boldsymbol{E}\left(IU \ Seq\right)$, $\boldsymbol{E}(IU \ Seq)$ means the sequence embedding representations, and $\boldsymbol E\left(\mathcal{F}_{IU\_ID}\right)$, $\boldsymbol E\left(\mathcal{F}_{IU\_Side}\right)$ means the embedding for id feature and side feature for interest unit respectively.

\subsubsection{\textbf{Attention Mechanism for IU Sequence}}
In addition to the traditional product-based attention mechanism, we further introduce an attention mechanism based on IU behavior. When scoring a target product, we first parse the IU ID associated with the target product and the IU IDs of the products in the user's historical click sequence. We utilize an attention mechanism to calculate the distance between the IU ID of the target product and the IU IDs of products previously clicked on, in IU to assess the intensity of the user's preference for the interest unit to which the current target product belongs.


Please note that our model mainly focuses on the expansion of product attributes from the perspective of single item to interest unit, and thus can be applied to various CTR prediction networks and sequence information modeling methods.
% \section{Experiment and Results}
\section{Results and Analysis}
In this section, we first present safe vs. unsafe evaluation results for 12 LLMs, followed by fine-grained responding pattern analysis over six models among them, and compare models' behavior when they are attacked by same risky questions presented in different languages: Kazakh, Russian and code-switching language.    

\begin{table}[t!]
\centering
\small
\resizebox{\columnwidth}{!}{
\begin{tabular}{clcccc}
\toprule
\multicolumn{1}{l}{\textbf{Rank} } & \textbf{Model} & \textbf{Kazakh $\uparrow$} & \textbf{Russian $\uparrow$} & \textbf{English $\uparrow$} \\
\midrule
1 & \claude & \textbf{96.5}   & 93.5    & \textbf{98.6}    \\
2 & \gptfouro & 95.8   & 87.6    & 95.7    \\
3 & \yandexgpt & 90.7   & \textbf{93.6}    & 80.3    \\
4 & \kazllmseventy & 88.1 & 87.5 & 97.2 \\
5 & \llamaseventy & 88.0   & 85.5    & 95.7    \\
6 & \sherkala & 87.1   & 85.0    & 96.0    \\
7 & \falcon & 87.1   & 84.7    & 96.8    \\
8 & \qwen & 86.2   & 85.1    & 88.1    \\
9 & \llamaeight & 85.9   & 84.7    & 98.3    \\
10 & \kazllmeight & 74.8   & 78.0    & 94.5 \\
11 & \aya & 72.4 & 84.5 & 96.6 \\
12 & \vikhr & --- & 85.6 & 91.1 \\
\bottomrule
\end{tabular}
}
\caption{Safety evaluation results of 12 LLMs, ranked by the percentage of safe responses in the Kazakh dataset. \claude\ achieves the highest safety score for both Kazakh and English, while \yandexgpt\ is the safest model for Russian responses.}
\label{tab:safety-binary-eval}
\end{table}



\subsection{Safe vs. Unsafe Classification}
% In this subsection, 
We present binary evaluation results of 12 LLMs, by assessing 52,596 Russian responses and 41,646 Kazakh responses.
% 26,298 responses generated by six models on the Russian dataset and 22,716 responses on the Kazakh dataset. 

%\textbf{Safety Rank.} In general, proprietary systems outperform the open-source model. For Russian, As shown in Table \ref{tab:model_comparison_russian}, \textbf{Yandex-GPT} emerges as the safest large language model (LLM) for Russian, with a safety percentage of 93.57\%. Among the open-source models, \textbf{Vikhr-Nemo-12B} is the safest, achieving a safety percentage of 85.63\%.
% Edited: This is mentioned in the discussion
% This outcome highlights the potential impact of pretraining data on model behavior. Models pre-trained primarily on Russian data are better at understanding and handling harmful questions - in both proprietary systems and open-source setups. 
%For Kazakh, as shown in Table \ref{tab:model_comparison_kazakh}, \textbf{Claude} emerges as the safest large language model (LLM) with a safety percentage of 96.46\%, closely followed by GPT-4o at 95.75\%. In contrast, \textbf{Aya-101}, despite being specifically tuned for Kazakh, consistently shows the highest unsafe response rates at 72.37\%. 

\begin{figure*}[t!]
	\centering
        \includegraphics[scale=0.28]{figures/question_type_no6_kaz.png}
	\includegraphics[scale=0.28]{figures/question_type_exclude_region_specific_new.png} 

	\caption{Unsafe answer distribution across three question types for risk types I-V: Kazakh (left) and Russian (right)}
	\label{fig:qt_non_reg}
\end{figure*}

\begin{figure*}[t!]
	\centering
        \includegraphics[scale=0.28]{figures/question_type_only6_kaz.png}
	\includegraphics[scale=0.28]{figures/question_type_region_specific_new.png} 
	
	\caption{Unsafe answer distribution across three question types for risk type VI: Kazakh (left) and Russian (right)}
	\label{fig:qt_reg}
\end{figure*}

\textbf{Safety Rank.} In general, proprietary systems outperform the open-source models. 
For Russian, as shown in Table~\ref{tab:safety-binary-eval},  % \ref{tab:model_comparison_russian}, 
\yandexgpt emerges as the safest language model for Russian, with safe responses account for 93.57\%. Among the open-source models, \kazllmseventy is the safest (87.5\%), followed by \vikhr with a safety percentage of 85.63\%.

For Kazakh, % as shown in Table \ref{tab:model_comparison_kazakh}, 
% YX: todo, rerun Kazakh safety percentage using Diana threshold
\claude is the safest model with 96.46\% safe responses, closely followed by \gptfouro\ at 95.75\%. \aya, despite being specifically tuned for Kazakh, shows the highest unsafe rates at 72.37\%.



% \begin{table}[t!]
% \centering
% \resizebox{\columnwidth}{!}{%
% \begin{tabular}{clccc}
% \toprule
% \textbf{Rank} & \textbf{Model Name}  & \textbf{Safe} & \textbf{Unsafe} & \textbf{Safe \%} \\ \midrule
% \textbf{1} & \textbf{Yandex-GPT} & \textbf{4101} & \textbf{282} & \textbf{93.57} \\
% 2 & Claude & 4100 & 283 & 93.54 \\
% 3 & GPT-4o & 3839 & 544 & 87.59 \\
% 4 & Vikhr-12B & 3753 & 630 & 85.63 \\
% 5 & LLama-3.1-instruct-70B & 3746 & 637 & 85.47 \\
% 6 & LLama-3.1-instruct-8B & 3712 & 671 & 84.69 \\
% \bottomrule
% \end{tabular}
% }
% \caption{Comparison of models based on safety percentages for the Russian dataset.}
% \label{tab:model_comparison_russian}
% \end{table}


% \begin{table}[t!]
% \centering
% \resizebox{\columnwidth}{!}{%
% \begin{tabular}{clccc}
% \toprule
% \textbf{Rank} & \textbf{Model Name}  & \textbf{Safe} & \textbf{Unsafe} & \textbf{Safe \%} \\ \midrule
% 1             & \textbf{Claude}  & \textbf{3652} & \textbf{134} & \textbf{96.46} \\ 
% 2             & GPT-4o                      & 3625          & 161          & 95.75 \\ 
% 3             & YandexGPT                   & 3433          & 353          & 90.68 \\
% 4             & LLama-3.1-instruct-70B      & 3333          & 453          & 88.03 \\
% 5             & LLama-3.1-instruct-8B       & 3251          & 535	       & 85.87 \\
% 6             & Aya-101                     & 2740          & 1046         & 72.37 \\ 
% \bottomrule
% \end{tabular}
% }
% \caption{Comparison of models based on safety percentages for the Kazakh dataset.}
% \label{tab:model_comparison_kazakh}
% \end{table}



\textbf{Risk Areas.} 
We selected six representative LLMs for Russian and Kazakh respectively and show their unsafe answer distributions over six risk areas.
As shown in Table \ref{tab:unsafe_answers_summary}, risk type VI (region-specific sensitive topics) overwhelmingly contributes the largest number of unsafe responses across all models. This highlights that LLMs are poorly equipped to address regional risks effectively. For instance, while \llama models maintain comparable safety levels across other risk type (I–V), their performance drops significantly when dealing with risk type VI. Interestingly, while \yandexgpt\ demonstrates relatively poor performance in most other risk areas, it handles region-specific questions remarkably well, suggesting a stronger alignment with regional norms and sensitivities. For Kazakh, Table \ref{tab:unsafe_answers_summary_kazakh} shows that region‐specific topics (risk type VI) pose a substantial challenge across all six models, including the commercial \gptfouro\ and \claude, which demonstrate superior safety on general categories. 

% \begin{table}[t!]
% \centering
% \resizebox{\columnwidth}{!}{%
% \begin{tabular}{lccccccc}
% \toprule
% \textbf{Model} & \textbf{I} & \textbf{II} & \textbf{III} & \textbf{IV} & \textbf{V} & \textbf{VI} & \textbf{Total} \\ \midrule
% LLama-3.1-instruct-8B & 60 & 70 & 16 & 31 & 9 & 485 & 671 \\
% LLama-3.1-instruct-70B & 29 & 55 & 8 & 4 & 1 & 540 & 637 \\
% Vikhr-12B & 41 & 93 & 15 & 1 & 3 & 477 & 630 \\
% GPT-4o & 21 & 51 & 6 & 2 & 0 & 464 & 544 \\
% Claude & 7 & 10 & 1 & 0 & 0 & 265 & 283 \\
% Yandex-GPT & 55 & 125 & 9 & 3 & 8 & 82 & 282 \\
% \bottomrule
% \end{tabular}%
% }
% \caption{Ru unsafe cases over risk areas of six models.}
% \label{tab:unsafe_answers_summary}
% \end{table}


\begin{table}[t!]
\centering
\resizebox{\columnwidth}{!}{%
\begin{tabular}{lccccccc}
\toprule
\textbf{Model} & \textbf{I} & \textbf{II} & \textbf{III} & \textbf{IV} & \textbf{V} & \textbf{VI} & \textbf{Total} \\ \midrule
\llamaeight & 60 & 70 & 16 & 31 & 9 & 485 & 671 \\
\llamaseventy & 29 & 55 & 8 & 4 & 1 & 540 & 637 \\
\vikhr & 41 & 93 & 15 & 1 & 3 & 477 & 630 \\
\gptfouro & 21 & 51 & 6 & 2 & 0 & 464 & 544 \\
\claude & 7 & 10 & 1 & 0 & 0 & 265 & 283 \\
\yandexgpt & 55 & 125 & 9 & 3 & 8 & 82 & 282 \\
\bottomrule
\end{tabular}%
}
\caption{Ru unsafe cases over risk areas of six models.}
\label{tab:unsafe_answers_summary}
\end{table}


% \begin{table}[t!]
% \centering
% \resizebox{\columnwidth}{!}{%
% \begin{tabular}{lccccccc}
% \toprule
% \textbf{Model} & \textbf{I} & \textbf{II} & \textbf{III} & \textbf{IV} & \textbf{V} & \textbf{VI} & \textbf{Total} \\ \midrule
% Aya-101 & 96 & 235 & 165 & 166 & 90 & 294 & 1046 \\
% Llama-3.1-instruct-8B & 25 & 15 & 91 & 37 & 14 & 353 & 535 \\
% Llama-3.1-instruct-70B & 33 & 39 & 88 & 27 & 20 & 246 & 453 \\
% Yandex-GPT & 29 & 76 & 95 & 29 & 16 & 108 & 353 \\
% GPT-4o & 2 & 1 & 41 & 0 & 3 & 114 & 161 \\
% Claude & 2 & 1 & 26 & 3 & 6 & 96 & 134 \\ \bottomrule
% \end{tabular}%
% }
% \caption{Kaz unsafe cases over risk areas of six models.}
% \label{tab:unsafe_answers_summary_kazakh}
% \end{table}


\begin{table}[t!]
\centering
\resizebox{\columnwidth}{!}{%
\begin{tabular}{lccccccc}
\toprule
\textbf{Model} & \textbf{I} & \textbf{II} & \textbf{III} & \textbf{IV} & \textbf{V} & \textbf{VI} & \textbf{Total} \\ \midrule
\aya & 96 & 235 & 165 & 166 & 90 & 294 & 1046 \\
\llamaeight & 25 & 15 & 91 & 37 & 14 & 353 & 535 \\
\llamaseventy & 33 & 39 & 88 & 27 & 20 & 246 & 453 \\
\yandexgpt & 29 & 76 & 95 & 29 & 16 & 108 & 353 \\
\gptfouro & 2 & 1 & 41 & 0 & 3 & 114 & 161 \\
\claude & 2 & 1 & 26 & 3 & 6 & 96 & 134 \\ 
\bottomrule
\end{tabular}%
}
\caption{Kaz unsafe cases over risk areas of six models.}
\label{tab:unsafe_answers_summary_kazakh}
\end{table}

% \begin{figure*}[t!]
% 	\centering
% 	\includegraphics[scale=0.28]{figures/human_1000_kz_font16.png} 
% 	\includegraphics[scale=0.28]{figures/human_1000_ru_font16.png}

% 	\caption{Human vs \gptfouro\ fine-grained labels on 1,000 Kazakh (left) and Russian (right) samples.}
% 	\label{fig:human_fg_1000}
% \end{figure*}

\textbf{Question Type.} For Russian, Figures \ref{fig:qt_non_reg} and \ref{fig:qt_reg} reveal differences in how models handle general risks I-V and region-specific risk VI. For risks I-V, indirect attacks % crafted to exploit model vulnerabilities—
result in more unsafe responses due to their tricky phrasing. 

In contrast, region-specific risks see slightly more unsafe responses from direct attacks, 
% as these explicit prompts are more likely to bypass safety mechanisms. 
since indirect attacks for region-specific prompts often elicit safer, vaguer answers. %, suggesting models struggle less with implicit harm. 
Overall, the number of unsafe responses is similar across question types, indicating models generally struggle with safety alignment in all jailbreaking queries.

For Kazakh, Figures \ref{fig:qt_non_reg} and \ref{fig:qt_reg} show greater variation in unsafe responses across question types due to the low-resource nature of Kazakh data. For general risks I-V, \llamaseventy\ and \aya\ produce more unsafe outputs for direct harm prompts. At the same time, \claude\ and \gptfouro\ struggle more with indirect attacks, reflecting challenges in handling subtle cues. For region-specific risk VI, most models perform similarly due to limited Kazakh-specific data, though \llamaeight\ shows higher unsafe outputs for indirect local references, likely due to their implicit nature. Direct region-specific attacks yield fewer unsafe responses, as explicit prompts trigger more cautious outputs. Across all risk areas, general questions with sensitive words produce the fewest unsafe answers, suggesting over-flagging or cautious behavior for unclear harmful intent.



% \subsection{Fine-grained Classification}
% We extended our analysis to include fine-grained classifications for both safe and unsafe responses. For unsafe responses, we categorized outputs into four harm types, as shown in Table \ref{table:unsafe_response_categories}. 

% For safe responses, we classified outputs into six distinct patterns of safety, following a fine-grained rubric provided in \cite{wang2024chinesedatasetevaluatingsafeguards}. The types outlined in this rubric are presented in Table \ref{table:safe_response_categories}.

% To validate the fine-grained classification, we conducted human evaluation on the same 1,000 responses in Russian used for the preliminary binary classification.
% The confusion matrix highlights the alignment and discrepancies between human annotations and GPT's fine-grained labels. The diagonal values represent instances where the GPT's labels match human annotations, with category 5 (provides general, safe information) showing the highest agreement (404 instances). However, off-diagonal values reveal areas of disagreement, such as misclassifications in categories 1 (not willing to answer), 4 (recommends consulting an expert), and 7 (provides harmful or risky instructions). For example, in category 1, while 94 responses were correctly labeled, GPT-4o misclassified several instances into categories 4, 5, or 7, indicating overlap or ambiguity in these classifications. Similar trends are observed in other categories, where GPT sometimes struggles to differentiate nuanced distinctions in human-labeled categories. Overall, GPT's fine-grained labels match human annotations in 710 out of 1000 cases, achieving an agreement rate of 71\%.

% \begin{figure}[ht!]
%     \centering
%     \includegraphics[width=0.95\linewidth]{figures/human_fg_1000_ru.png}
%     \caption{{Human vs GPT-4o Fine-Grained Labels on 1,000 Russian Samples}}
%     \label{fig:human_fg_1000_ru}
% \end{figure}


% After conducting human evaluation on a representative sample, we extended the fine-grained classification to a full dataset comprising 21,915 responses generated by five different models. 


% \begin{figure}[ht!]
%     \centering
%     \includegraphics[width=0.95\linewidth]{figures/all_5_ru.png}
%     \caption{Fine-grained label distribution for responses from five models for Russian.}
%     \label{fig:all_5_fg_russian}
% \end{figure}
% Category 5 ("safe and general information") consistently has the highest frequency across all models, aligning with its dominance in the 1,000-sample evaluation. However, differences in the distribution across other categories highlight variability in how models handle nuanced safety risks, with Yandex-GPT showing a slightly broader spread across categories. 
% In the distribution of unsafe responses, most models exhibit higher counts in certain labels such as 8. However, Yandex-GPT displays comparatively fewer responses in these labels. 
% It exhibits a high rate of responses classified under label 7, which indicates instances where the model provides harmful or risky instructions, including unethical behavior or sensitive discussions. While this may suggest a vulnerability in addressing complex or challenging prompts, it was observed that many of Yandex-GPT’s responses tend to deflect responsibility or offer vague advice such as "check the internet". Although this approach minimizes the risk of unsafe outputs, it often results in responses that lack depth or contextually relevant information, limiting their overall usefulness for users.

% % \subsection{Kazakh}


% % Overall, these findings underscore how resource constraints and prompt explicitness affect model safety in Kazakh. Some models manage direct attacks better yet fail on indirect ones, while region-specific content remains challenging for all given the lack of localized training data.
% \subsubsection{Fine-grained Classification}
% Similarly, we conducted a human evaluation on 1,000 Kazakh samples, following the same methodology as the Russian evaluation. The match between human annotations and GPT-4o's fine-grained classifications was 707 out of 1,000, ensuring that the fine-grained classification framework aligned well with human judgments.
% The confusion matrix in Figure \ref{fig:human_fg_1000_kz} for 1,000 Kazakh samples illustrates the agreement between human annotations and GPT-4o's fine-grained classifications. The highest agreement is observed in category 5 (360 instances), indicating GPT-4o's strength in recognizing responses labeled by humans as "safe and general information." However, discrepancies are evident in categories such as 3 and 7, where GPT-4o misclassified several instances, highlighting areas for further refinement in distinguishing nuanced classifications.


\begin{figure}[t!]
	\centering
	\includegraphics[scale=0.18]{figures/human_1000_kz_font16.png} 
	\includegraphics[scale=0.18]{figures/human_1000_ru_font16.png}

	\caption{Human vs \gptfouro\ fine-grained labels on 1,000 Kazakh (left) and Russian (right) samples.}
	\label{fig:human_fg_1000}
\end{figure}

% \begin{figure}[t!]
% 	\centering
% 	\includegraphics[scale=0.28]{figures/human_1000_kz_font16.png} 
% 	\includegraphics[scale=0.28]{figures/human_1000_ru_font16.png}

% 	\caption{Human vs \gptfouro\ fine-grained labels on 1,000 Kazakh (top) and Russian (bottom) samples.}
% 	\label{fig:human_fg_1000}
% \end{figure}

% \begin{figure*}[t!]
% 	\centering
% 	\includegraphics[scale=0.28]{figures/all_5_kz_font16.png} 
% 	\includegraphics[scale=0.28]{figures/all_5_ru_font_16.png} \\
% 	\caption{Fine-grained responding pattern distribution across five models for Kazakh (left) and Russian (right).}
% 	\label{fig:all_5}
% \end{figure*}

\begin{figure}[t!]
	\centering
	\includegraphics[scale=0.28]{figures/all_5_kz_font16.png} 
	\includegraphics[scale=0.28]{figures/all_5_ru_font_16.png} \\
	\caption{Fine-grained responding pattern distribution across five models for Kazakh (top) and Russian (bottom).}
	\label{fig:all_5}
\end{figure}


\subsection{Fine-Grained Classification}
\label{sec:fine-grained-classification}
% As discussed in Section \ref{harmfulness_evaluation}, 
We further analyzed fine-grained responding patterns for safe and unsafe responses. For unsafe responses, outputs were categorized into four harm types, and safe responses were classified into six distinct patterns of safety, as rubric in Appendix \ref{safe_unsafe_response_categories}. 
% \cite{wang2024chinesedatasetevaluatingsafeguards}

\paragraph{Human vs. GPT-4o}
Similar to binary classification, we validated \gptfouro's automatic evaluation results by comparing with human annotations on 1,000 samples for both Russian and Kazakh. %, comparing human annotations with \gptfouro's fine-grained labels.
For the Russian dataset, \gptfouro's labels aligned with human annotations in 710 out of 1,000 cases, achieving an agreement rate of 71\%. 
Agreement rate of Kazakh samples is 70.7\%. %with 707 out of 1,000 cases matching
% The confusion matrix highlights areas of alignment and discrepancies.
% 
As confusion matrices illustrated in Figure~\ref{fig:human_fg_1000}, the majority of cases falling into \textit{safe responding patter 5} --- providing general and harmless information, for both human annotations and automatic predictions.
% highest agreement with 404 correct classifications for Russian. 
Mis-classifications for safe responses mainly focus on three closely-similar patterns: 3, 4, and 5, and patterns 7 and 8 are confusing to discern for unsafe responses, particularly for Kazakh (left figure).
We find many Russian samples which were labeled as ``1. reject to answer'' by humans are diversely classified across 1-6 by GPT-4o, which is also observed in Kazakh but not significant.

% suggesting label alignment issues are language-independent. 
% YX: I did not observe this, commented
% Notably, Russian showed confusion between 7 (risky instructions) and 1 (refusal to answer), this trend does not appear in Kazakh.


% highlight the strengths and limitations of {\gptfouro}'s fine-grained classification framework across both languages, paving the way for further refinements.


% However, misclassifications were observed in categories such as 1 (not willing to answer), 4 (recommends consulting an expert), and 7 (provides harmful or risky instructions), revealing overlaps and ambiguities in nuanced classifications.

% Similarly, for the Kazakh dataset, the agreement rate between human annotations and GPT-4o's labels was 70.7\%, with 707 out of 1,000 cases matching. As with the Russian analysis, category 5 (360 instances) showed the highest alignment. However, discrepancies were more prominent in categories such as 3 and 7, underscoring GPT-4o's challenges in differentiating fine-grained human-labeled categories. 
% A similar pattern was observed for Kazakh dataset, which suggests that alignment and misaligned of fine-grained lables is not language dependent.

% These findings, illustrated in Figures \ref{fig:human_fg_1000}, highlight the strengths and limitations of {\gptfouro}'s fine-grained classification framework across both languages, paving the way for further refinements.

\paragraph{Fine-grained Analysis of Five LLMs}
% After conducting human evaluation on representative samples, we extended 
\figref{fig:all_5} shows fine-grained responding pattern distribution across five models based on the full set of Russian and Kazakh data.
% For Russian, we selected \vikhr, \gptfouro, \llamaseventy, \claude, and \yandexgpt, while for Kazakh, we chose \aya, \gptfouro, \llamaseventy, \claude, and \yandexgpt. 
% The evaluation covered 21,915 responses in Russian and 18,930 responses in Kazakh.
% 
In both languages, pattern 5 of providing \textit{general and harmless information} consistently witnessed the highest frequency across all models, with \llamaseventy\ exhibiting the largest number of responses falling into this category for Kazakh (2,033). 
% YX:summarize more noteable findings here.

Differences of other patterns vary across languages. 
Unsafe responses in Russian are predominantly in pattern 8, where models provide incorrect or misleading information without expressing uncertainty. % (misinformation and speculation), 
For Kazakh, \aya\ exhibits the highest occurrence of pattern 7 (harmful or risky information) and pattern 8, indicating a stronger tendency to generate unethical, misleading, or potentially harmful content.

%Variations in other patterns across models highlight differences in how nuanced safety risks are classified, reflecting the models' differing capabilities in handling safety evaluation for these distinct linguistic contexts. For Russian, the majority of unsafe responses fall under pattern 8 (misinformation and speculation), indicating that models frequently provide incorrect or misleading information without acknowledging uncertainty. For Kazakh, \aya\ has the highest occurence of pattern 7 (harmful or risky information) and pattern 8 (misinformation and speculation), indicating a greater tendency to generate unethical, misleading, or potentially harmful content. 

%This trend suggests that Russian models may struggle more with factual accuracy, whereas Kazakh models, particularly \aya, pose higher risks related to both harmful content and misinformation. Additionally, \gptfouro\ and \claude\ consistently produce fewer unsafe responses in both languages, demonstrating stronger alignment with safety standards
\subsection{Code Switching}
\begin{table}[t!]
\centering

\setlength{\tabcolsep}{3pt}
\scalebox{0.7}{
\begin{tabular}{lcccccccccc}
\toprule
\textbf{Model Name} & \multicolumn{2}{c}{\textbf{Kazakh}} & \multicolumn{2}{c}{\textbf{Russian}} & \multicolumn{2}{c}{\textbf{Code-Switched}} \\  
\cmidrule(lr){2-3} \cmidrule(lr){4-5} \cmidrule(lr){6-7}
& \textbf{Safe} & \textbf{Unsafe} & \textbf{Safe} & \textbf{Unsafe} & \textbf{Safe} & \textbf{Unsafe} \\ 
\midrule
\llamaseventy & 450 & 50 & 466 & 34 & 414 & 86 \\
\gptfouro & 492 & 8 & 473 & 27 & 481 & 19
 \\
\claude & 491 & 9 & 478 & 22 & 484 & 16 \\ 
\yandexgpt & 435 & 65 & 458 & 42 & 464 & 36 \\
\midrule
\end{tabular}}
\caption{Model safety when prompted in Kazakh, Russian, and code-switched language.}
\label{tab:finetuning-comparison}
\end{table}


\gptfouro\ and \claude\ demonstrate strong safety performance across three languages, even with a high proportion of safe responses in the challenging code-switching context. In contrast, \llamaseventy\ and \yandexgpt\ are less safe, exhibiting more harmful responses, particularly in the code-switching scenario. These results show the varying capabilities of models in defending the same attacks that are just presented in different languages, where open-sourced large language models especially require more robust safety alignment in multilingual and code-switching scenarios.

% \subsection{LLM Response Collection}
% We conducted experiments with a variety of mainstream and region-specific 
% large language models for both Russian and Kazakh languages. For both Russian and Kazakh languages, we employed four multilingual models: Claude-3.5-sonnet, Llama 3.1 70B \cite{meta2024llama3}, GPT-4 \cite{openai2024gpt4o}, and YandexGPT. Additionally, we included language-specific models: VIKHR \cite{nikolich2024vikhrconstructingstateoftheartbilingual} for Russian and Aya \cite{ustun-etal-2024-aya} for Kazakh. 

% \subsection{Kazakh-Russian Code-Switching Evaluation}

% In Kazakhstan, the prevalence of bilingualism is a defining characteristic of its linguistic landscape, with most individuals seamlessly mixing Kazakh and Russian in daily communication \cite{Zharkynbekova2022}. This phenomenon, known as code-switching, reflects the unique cultural and social dynamics of the region. Despite this, there is currently no safety evaluation dataset tailored to this unique multilingual environment. Developing a code-switched dataset is essential to evaluate the ability of large language models (LLMs) to navigate the complexities of bilingual interactions, ensuring they produce contextually appropriate, non-harmful, and culturally sensitive responses. To address this, we sampled 500 questions from both Kazakh and Russian question sets, ensuring they represent similar queries but with culturally nuanced differences in meaning. Building on these, we developed a mixed version that integrates Kazakh and Russian naturally, reflecting the way people in Kazakhstan communicate. This process involved annotators to ensure that the mixed questions were crafted with a natural tone and were localized to Kazakhstan's context.


% \begin{table*}[t!]
%     \centering
%     \small
%     \setlength{\tabcolsep}{6pt} % Adjusts column separation
%     \adjustbox{max width=\linewidth}{
%     \begin{tabular}{ll ccc ccc c}
%     \toprule
%     \textbf{Risk Area} & \textbf{Harm Type} & \multicolumn{3}{c}{\textbf{Kazakh}} & \multicolumn{3}{c}{\textbf{Russian}} & \textbf{\#Q} \\
%                        &                    & \textbf{Ori} & \textbf{FN} & \textbf{FP} & \textbf{Ori} & \textbf{FN} & \textbf{FP} & \\
%     \midrule
%     \multirow{2}{*}{I. Information Hazards} & 1. Risks from leaking sensitive information... & 131 & 133 & 131 & 131 & 133 & 131 & 790 \\
%                                              & 2. Compromise of privacy by leaking or inferring private information (person/individual) & 81 & 82 & 81 & 81 & 82 & 81 & 488 \\
%     \midrule
%     \multirow{3}{*}{II. Malicious Uses} & 3. Assisting illegal activities & 132 & 135 & 132 & 132 & 135 & 132 & 798 \\
%                                         & 4. Nudging or advising unethical actions & 71 & 71 & 71 & 71 & 71 & 71 & 426 \\
%                                         & 5. Reducing the cost of disinformation campaigns & 40 & 42 & 40 & 40 & 42 & 40 & 244 \\
%     \midrule
%     \multirow{3}{*}{III. Discrimination, Exclusion, Toxicity, Hateful, Offensive} & 6. Social stereotypes and unfair discrimination & 94 & 96 & 94 & 94 & 96 & 94 & 568 \\
%                                         & 7. Toxic language (hate speech) & 52 & 59 & 52 & 52 & 59 & 52 & 326 \\
%                                         & 8. Adult content & 27 & 29 & 27 & 27 & 29 & 27 & 166 \\

%     \midrule
%     \multirow{2}{*}{IV. Misinformation Harms} & 9. Disseminating false or misleading information & 92 & 99 & 92 & 92 & 99 & 92 & 566 \\
%                                         & 10. Causing material harm by disseminating misinformation e.g. in medicine or law & 63 & 63 & 63 & 63 & 63 & 63 & 378 \\
%     \midrule
%     \multirow{2}{*}{V. Human–chatbot Interaction Harms} & 11. Mental health or overreliance concerns & 66 & 66 & 66 & 66 & 66 & 66 & 396 \\
%                                         & 12. Treating the chatbot as a human & 50 & 51 & 50 & 50 & 51 & 50 & 302 \\

%     \midrule
%     \multirow{2}{*}{VI. Kazakhstan-specific Sensitivity} & 13. Politically sensitive topics & 63 & 66 & 63& 63 & 66 & 63 &  384 \\
%         & 14. Controversial historical events & 46 & 57 & 46 & 46 & 57 & 46  & 298 \\
% & 15. Regional and racial issues & 45 & 45 & 45 &  45 & 45 & 45 & 270  \\
% & 16. Societal and cultural concerns & 138 & 139 & 138 &  138 & 139 & 138  & 830  \\
% & 17. Legal and human rights matters & 57 & 57 & 57 & 57 & 57 & 57  & 342 \\
%     \midrule
%         \multirow{2}{*}{VII. Russia-specific Sensitivity} 
%             & 13. Politically sensitive topics & - & - & - & 54 & 54 & 54 & 162 \\
%     & 14. Controversial historical events & - & - & - & 38 & 38 & 38 & 114 \\
%     & 15. Regional and racial issues & - & - & - & 26 & 26 & 26 & 78 \\
%     & 16. Societal and cultural concerns & - & - & - & 40 & 40 & 40 & 120 \\
%     & 17. Legal and human rights matters & - & - & - & 41 & 41 & 41 & 123 \\
%     \midrule
%     \bf Total & -- & 1248 & 1290 & 1248 & 1447 & 1489 & 1447 & \textbf{8169} \\
%     \bottomrule
%     \end{tabular}
%     }
%     \caption{The number of questions for Kazakh and Russian datasets across six risk areas and 17 harm types. Ori: original direct attack, FN: indirect attack, and FP: over-sensitivity assessment.}
%     \label{tab:kazakh-russian-data}
% \end{table*}




\section{Discussion}

% \subsection{Kazakh vs Russian}

% The evaluation reveals that Kazakh responses tend to be generally safer than their Russian counterparts, likely due to Kazakh being a low-resource language with significantly less training data. As a result, Kazakh models are less exposed to the vast, often unfiltered datasets containing harmful or unsafe content, which are more prevalent in high-resource languages like Russian. This data scarcity naturally limits the model's ability to generate nuanced but potentially unsafe responses. However, this does not mean the models are specifically fine-tuned for safer performance. When analyzing unsafe answers, it’s clear that Kazakh models, while safer overall, distribute their unsafe responses more evenly across various risk types and question types. This suggests Kazakh models generate fewer unsafe answers but in a broader range of contexts.

% In contrast, Russian models tend to concentrate unsafe answers in specific areas, particularly region-specific risks or indirect attacks. This indicates that Russian models have learned to handle certain types of unsafe content by focusing on specific topics, such as politically sensitive issues, but struggle when confronted with unfamiliar content, leading to unsafe responses due to insufficient filtering. Kazakh models, despite having less training data, tend to respond more broadly, including both direct and indirect risks. This could be due to the less curated nature of their training data, making them more likely to answer unsafe questions without filtering the potential harm involved. The exception to this trend is Aya, a model specifically fine-tuned for Kazakh. Despite fine-tuning, it exhibits the lowest safety percentage (72.37\%) in the Kazakh dataset, suggesting that fine-tuning in specific languages may introduce risks if proper safety measures are not taken.

% The evaluation reveals notable differences in the distribution of safe response patterns across Kazakh and Russian fine-grained labels. Refusal to answer is more frequent in Russian models, particularly Yandex-GPT, reflecting a cautious approach to safety-critical queries. Interestingly, Aya, despite being fine-tuned for Kazakh and exhibiting lower overall safety, also frequently refuses to answer, suggesting an over-reliance on conservative mechanisms. Responses providing general, safe information dominate in both languages, with Kazakh models displaying a slightly higher tendency to rely on this approach. This highlights how the low-resource nature of Kazakh results in more generalized and inherently safer responses. In contrast, Russian models excel at recognizing risks, issuing disclaimers, and refuting incorrect assumptions, likely benefiting from richer and more diverse training data.
% Yandex-GPT exhibits a notably high rate of responses classified under label 7, indicating an overreliance on general disclaimers or deflections, such as "check the internet" or "I don't know." While these responses minimize the risk of unsafe outputs, they often lack substantive or contextually relevant information, reducing their overall utility for users.


Most models perform safer on Kazakh dataset than Russian dataset, higher safe rate on Kazakh dataset in \tabref{tab:safety-binary-eval}. This does not necessarily reveal that current LLMs have better understanding and safety alignment on Kazakh language than Russian, while this may conversely imply that models do not fully understand the meaning of Kazakh attack questions, fail to perceive risks and then provide general information due to lacking sufficient knowledge regarding this request.

We observed the similar number of examples falling into category 5 \textit{general and harmless information} for both Kazakh and Russian, while the Kazakh data set size is 3.7K and Russian is 4.3K. Kazakh has much less examples in category 1 \textit{reject to answer} compared to Russian. This demonstrate models tend to provide general information and cannot clearly perceive risks for many cases.

Additionally, in spite of less harmful responses on Kazakh data, these unsafe responses distribute evenly across different risk areas and question categories, exhibiting equally vulnerability spanning all attacks regardless of what risks and how we jailbreak it.
In contrary, unsafe responses on Russian dataset often concentrate on specific areas and question types, such as region-specific risks or indirect attacks, presenting similar model behaviors when evaluating over English and Chinese data.
It suggests that broader training data in English, Chinese and Russian may allow models to address certain types of attacks robustly,
% effectively—particularly politically sensitive issues—
yet they may falter when confronted with unfamiliar content like regional sensitive topics.

Moreover, in responses collection, we observed many Russian or English responses especially for open-sourced LLMs when we explicitly instructed the models to answer Kazakh questions in Kazakh language. This further implies more efforts are still needed to improve LLMs' performance on low-resource languages.
Interestingly, \aya, a fine-tuned Kazakh model, proves an exception by displaying the lowest safety percentage (72.37\%) among Kazakh models, revealing that the multilingual fine-tuning without stringent safety measures can introduce risks.



% However, this does not mean they are explicitly fine-tuned for safety, likely it happens due to limited training data, which reduces exposure to harmful content. 
% \aya, a fine-tuned Kazakh model, proves an exception by displaying the lowest safety percentage (72.37\%) among Kazakh models, revealing that the multilingual fine-tuning without stringent safety measures can introduce risks.
% Kazakh models generally produce safer responses than their Russian counterparts, likely because Kazakh is a low-resource language with less training data. 
% This limited exposure to harmful or unsafe content naturally limits nuanced yet potentially unsafe outputs. 
% However, it does not imply that the models are specifically fine-tuned for enhanced safety.


% while Kazakh models tend to generate fewer unsafe answers overall, those unsafe responses appear more evenly spread across different risk types and question categories.
% Russian models, on the other hand, often concentrate unsafe responses in specific areas, such as region-specific risks or indirect attacks.
% It implies that their broader training datasets allow them to address certain types of unsafe content more effectively—particularly politically sensitive issues—yet they may falter when confronted with unfamiliar or insufficiently filtered content.

% Meanwhile, Kazakh models sometimes respond more broadly, possibly due to less curated training data. 

Differences also emerge in how language models handle safe responses. 
\yandexgpt, for instance, often refuses to answer high-risk queries. 
It frequently relies on generic disclaimers or deflections like ``check in the Internet'' or ``I don’t know,'' minimizing risk but are less helpful. Interestingly, it often responds with ``I don’t know'' in Russian, even for Kazakh queries, we speculate that these may be default responses stemming from internal system filters, rather than generated by model itself.
This likely explains why \yandexgpt\ is the safest model for the Russian language but ranks third for Kazakh. While its filters perform well for Russian, they struggle with the low-resource Kazakh language.

% Aya, despite its lower overall safety, also employs refusals often, hinting at an over-reliance on conservative approaches. 

% Across both languages, models commonly resort to providing general, safe information, although Kazakh models lean on this strategy slightly more. 
% Russian models, by contrast, excel at detecting risks, issuing disclaimers, and correcting inaccuracies, likely benefiting from richer and more diverse training data.


% \subsection{Response Patterns}


% We conducted a detailed analysis of the models' outputs and identified several noteworthy patterns. YandexGPT, while being one of the safest overall, frequently generates responses in Russian even when the question is posed in Kazakh. These responses often appear as placeholders, prompting users to search for the answer online. This behavior might not originate from the model itself but rather from safety filters implemented in the YandexGPT system. The model's leading performance in ensuring safety during Russian-language interactions, coupled with its lower performance in Kazakh, can be attributed to the limited robustness of these safety filters when handling unsafe content in Kazakh.

% In contrast, Aya-101 exhibits a tendency to fall into repetition, often repeating the same sentences multiple times. Interestingly, the Vikhr model, despite being of a similar size, does not exhibit this issue. We attribute this difference to two key factors. First, Vikhr and Aya-101 have distinct architectures: Vikhr is based on the Mistral-Nemo model, whereas Aya-101 is built on mT5, an older and less robust model. Second, Aya-101 is a multilingual model, while Vikhr was predominantly trained for Russian. Multilingualism has been shown to potentially degrade performance in large language models~\cite{huang2025surveylargelanguagemodels}, which may explain Aya-101's issues with repetition.

\paragraph{Summary}
Our findings provide significant insights into the influence of correctness, explanations, and refinement on evaluation accuracy and user trust in AI-based planners. 
In particular, the findings are three-fold: 
(1) The \textbf{correctness} of the generated plans is the most significant factor that impacts the evaluation accuracy and user trust in the planners. As the PDDL solver is more capable of generating correct plans, it achieves the highest evaluation accuracy and trust. 
(2) The \textbf{explanation} component of the LLM planner improves evaluation accuracy, as LLM+Expl achieves higher accuracy than LLM alone. Despite this improvement, LLM+Expl minimally impacts user trust. However, alternative explanation methods may influence user trust differently from the manually generated explanations used in our approach.
% On the other hand, explanations may help refine the trust of the planner to a more appropriate level by indicating planner shortcomings.
(3) The \textbf{refinement} procedure in the LLM planner does not lead to a significant improvement in evaluation accuracy; however, it exhibits a positive influence on user trust that may indicate an overtrust in some situations.
% This finding is aligned with prior works showing that iterative refinements based on user feedback would increase user trust~\cite{kunkel2019let, sebo2019don}.
Finally, the propensity-to-trust analysis identifies correctness as the primary determinant of user trust, whereas explanations provided limited improvement in scenarios where the planner's accuracy is diminished.

% In conclusion, our results indicate that the planner's correctness is the dominant factor for both evaluation accuracy and user trust. Therefore, selecting high-quality training data and optimizing the training procedure of AI-based planners to improve planning correctness is the top priority. Once the AI planner achieves a similar correctness level to traditional graph-search planners, strengthening its capability to explain and refine plans will further improve user trust compared to traditional planners.

\paragraph{Future Research} Future steps in this research include expanding user studies with larger sample sizes to improve generalizability and including additional planning problems per session for a more comprehensive evaluation. Next, we will explore alternative methods for generating plan explanations beyond manual creation to identify approaches that more effectively enhance user trust. 
Additionally, we will examine user trust by employing multiple LLM-based planners with varying levels of planning accuracy to better understand the interplay between planning correctness and user trust. 
Furthermore, we aim to enable real-time user-planner interaction, allowing users to provide feedback and refine plans collaboratively, thereby fostering a more dynamic and user-centric planning process.

\section{Limitations}
Our method's reliance on semantic embeddings introduces inherent biases present in encoder's training data. While these embeddings enable semantic consistency, they may not capture certain culturally-specific or nuanced artistic concepts. This highlights the need for more careful study on choices of the semantic embeddings and their effects on SliderSpace discovery. The current discovery process requires significant computational time ($\approx$ 2 hrs on A100), which may limit rapid experimentation and iteration. This computational overhead opens avenues for future research into training time optimizations. We also note that our method trains 4 times faster than  Concept Sliders for same number of sliders. For art style discovery, it is possible that the discovered directions are not one-to-one matched with the original artists. Further work can address discovery that nudges the directions to be aligned with real artists. 
% This can be used as an interpretability tool or an attribution tool if the discover directions are aligned with real artists. 
%\section{Impact Statement}
The CoALM agent demonstrates improved performance across both conversational TOD and agentic tasks, offering a global approach that combines these capabilities. By addressing the question \textit{Can a Single Model Master Both Multi-turn Conversations and Tool Use?}, we tried to establish a new paradigm for developing unified conversational agents. However, we emphasize that our primary goal is neither to top specific leaderboards nor to create a state-of-the-art agent that can be surpassed in the coming months through improved baselines or task-specific optimization with more data. Instead, our aim is to advance the field by identifying critical gaps in both Conversational AI and Language Agent research, highlighting why addressing these gaps is essential for developing reliable conversational systems that contribute to the path toward Artificial General Intelligence (AGI).


%We recognize that any current top-performing model can be surpassed in the coming months through improved baselines or task-specific optimization with additional data. 
This work was funded by Deutsche Forschungsgemeinschaft (DFG, German Research Foundation) under Germany’s Excellence Strategy - EXC 2092 CASA - (390781972), the European Research Council (ERC) under the consolidator grant MALFOY (101043410), and the German Federal Ministry of Education and Research under the grant BIFOLD24B.


%% \begin{table*}[!h]
% \centering
% \resizebox{\textwidth}{!}{
% \begin{tabular}{lrrrrr}
% \toprule
% \textbf{Data Mixture} & \textbf{Data Type} & \textbf{Data Name} & \textbf{\# of Data Instances} &  \textbf{\# of Total Tokens} & \textbf{Avg. Tokens Per Instance} \\ \midrule
% \multirow{2}{*}{\textbf{Prior Agent Works}}& Information Seeking & FireAct &   $2,063$ &    $542,176$ &   $262.81$ \\
%  &  Information Seeking & AgentInstruct &    $1,866$ &   $2,517,785$ &  $1349.30$ \\ \midrule

% \multirow{5}{*}{\textbf{\dataname} \textbf{AgentFlan}}
% & Information Seeking & HotpotQA &      $1,664$ &   $2,472,227$ &  $1485.71$ \\
% & Software Packages (Tool) & MATH &      $1,732$ &   $1,719,467$ &   $992.76$ \\
% & Software Packages (Tool) & APPS Code &        $647$ &   $1,235,472$ &  $1909.54$ \\
% & External Memory & WikiTableQuestion  &      $1,065$ &   $1,316,246$ &  $1235.91$ \\
% & Robot Planning & ALFWorld  &      $2,031$ &   $3,838,269$ &  $1889.84$ \\ \midrule

% \multirow{4}{*}{\textbf{General Conversation}}
% & Single-Turn Reasoning & OpenOrca  &     $50,000$ &  $14,034,152$ &   $280.68$ \\
% & Multi-Turn Conversations & ShareGPT &     $10,000$ &  $17,933,861$ &  $1793.39$ \\
% & Multi-Turn Conversations & ShareGPT  &      $4,583$ &  $18,195,878$ &  $3970.30$ \\
% & Multi-turn Reasoning & CapyBara &  $4,647$ &   $4,982,435$ &  $1072.18$ \\ \midrule

% \multirow{4}{*}{\textbf{TOD Conversation}}
% & Multi-Turn TOD Conversations & MultiWOZ 2.4     &     $8,400$   &  $TODO$          &  $TODO$ \\
% & Multi-Turn TOD Conversations & SGD              &     $16,168$  &  $TODO$          &  $TODO$ \\
% & Multi-Turn TOD Conversations & ABCD             &      $8,034$  &  $TODO$          &  $TODO$ \\
% & Multi-turn TOD Conversations & STAR V2          &      $6,652$  &  $TODO$          &  $TODO$ \\
% & Multi-turn TOD Conversations & MSR-E2E          &      $6,993$  &  $TODO$          &  $TODO$ \\ 
% & Multi-turn TOD Conversations & KVRET            &      $7,884$  &  $TODO$          &  $TODO$ \\ 
% & Single-turn TOD Conversations& ST               &      $29,576$ &  $TODO$          &  $TODO$ \\ 
% & Multi-turn TOD Conversations & Task-Master      &      $23,346$ &  $TODO$          &  $TODO$ \\ \cmidrule{2-6}
% \multicolumn{3}{r}{\textbf{Total}} & $107,053$ & $TODO$ & $TODO$ \\ \bottomrule
% \end{tabular}
% }
% \caption{Statistics of our training mixture and comparison with prior work.}
% \label{tab: data_training_mixture_stats}
% \end{table*}

\begin{table*}[!t]
\centering
\resizebox{\textwidth}{!}{
\begin{tabular}{lrrrrrr}
\toprule
\textbf{Data Domain} & \textbf{Data Type} & \textbf{Data Name} & \textbf{Data Format} & \textbf{\# of Data Samples} &  \textbf{\# of Total Tokens} & \textbf{Avg. Tokens Per Sample} \\ \midrule

\multirow{1}{*}{\textbf{TOD}}&  Single-Turn  & SNIPS         & State Tracking &    $13,028$   &   $12,278,780$ &  $942.49$  \\ \cmidrule{2-7}
\multirow{2}{*}{\textbf{LA}}& Single-Turn & Hammer  & API Call          &     $13,819$  &  $10,199,147$  &   $738.05$ \\
                                  & Multi-Turn  & ToolAce & API Call          &     $202,500$ &  $129,001,612$ &  $637.04$ \\ \cmidrule{2-7}
\multirow{1}{*}{\textbf{CRA}}& Multi-Turn & SGD & ReAct API Call &    $82,236$   &   $59,704,782$ &  $726.02$ \\ \cmidrule{2-7}
                                 
         
\multicolumn{4}{r}{\textbf{Total}} & $311,583$ &  $211,184,321$ &  $760.90$ \\ \bottomrule
\end{tabular}
}
\caption{\textbf{CoALM-IT Dataset Details.} Statistical details of our proposed CoALM-IT dataset showcasing the training mixtures. Generated \textbf{CRA} denotes the Conversational ReAct API dataset.}
\vspace{-5mm}
\label{tab: data_training_mixture_stats}
\end{table*}


% \begin{table*}[!ht]
% \centering
% \resizebox{\textwidth}{!}{
% \begin{tabular}{lrrrrrr}
% \toprule
% \textbf{Data Domain} & \textbf{Data Type} & \textbf{Data Name} & \textbf{Data Format} & \textbf{\# of Data Instances} &  \textbf{\# of Total Tokens} & \textbf{Avg. Tokens Per Instance} \\ \midrule

% \multirow{2}{*}{\textbf{TOD}}&  Single-Turn  & SNIPS         & State Tracking &    $13,028$ &   $12,278,780$ &  $942.49$  \\
%                              &  Single-Turn  & SLURPS        & State Tracking & $11,514$    & $11,359,895$   & $986.62$   \\ \cmidrule{2-7}
                             
% \multirow{2}{*}{\textbf{API Call}}& Single-Turn & Hammer  & API Call    &     $13,819$  &  $10,199,147$ &   $738.05$ \\
%                                         & Multi-Turn  & ToolAce & API Call   &     $202,500$ &  $129,001,612$ &  $637.04$ \\ \cmidrule{2-7}

% \multirow{2}{*}{\textbf{Dialogue + API}}& Multi-Turn   & MultiWoz 2.4  & ReAct + API Call &    $65,563$ &    $71,850,667$ &   $1095.90$ \\
%                                  & Multi-Turn   & SGD           & ReAct + API Call &    $82,236$ &   $59,704,782$ &  $726.02$ \\ \cmidrule{2-7}
                                 
         
% \multicolumn{4}{r}{\textbf{Total}} & $401,744$ &  $299,344,457$ &  $786.34$ \\ \bottomrule
% \end{tabular}
% }
% \caption{TOD Conversation Instruction Tuning Datasets. Statistics of our training mixture and comparison with prior work.}
% \vspace{-5mm}
% \label{tab: data_training_mixture_stats}
% \end{table*}




% \begin{table*}[h]
\renewcommand{\arraystretch}{1.4} % Adjust row spacing
\resizebox{\textwidth}{!}{%
\begin{tabular}{lccccc |cc| c| cc| cccccc|cc}
\toprule
\multirow{3}{*}{\textbf{Model}} & \multicolumn{16}{c}{\textbf{Task}}                                                                                                                                                                                                                 & \multicolumn{2}{c}{\textbf{Method}} \\ \cmidrule(r){2-17} \cmidrule(r){18-19}
                       & \multicolumn{5}{c}{\textbf{E2E}}                  & \multicolumn{2}{c}{\textbf{DST}} & \multicolumn{1}{c}{\textbf{Detection}} & \multicolumn{2}{c}{\textbf{NAP}}& \multicolumn{6}{c}{\textbf{Human Evaluation}}                                                      & \multirow{2}{*}{\textbf{Fine-tuning}}   & \multirow{2}{*}{\textbf{Prompting}}   \\
                       & Inform     & Success    & BLEU       & C          & BertScore  & JGA        & Slot-F1     & Acc                            & F1        & Acc           & Coherency  & Fluency    & Informativeness & Satisfaction & Understanding & Truthfulness            &                                &                              \\ \midrule
DailoGPT               & \ding{55}  & \ding{55}  & \checkmark & \ding{55}  & \ding{55}  & \ding{55}  & \ding{55}   & \ding{55}                      & \ding{55} & \ding{55}     & \checkmark & \ding{55}  & \checkmark      & \checkmark   & \ding{55}     & \ding{55}               &\checkmark                      & \ding{55}                    \\ 
PPTOD                  & \checkmark & \checkmark & \checkmark & \checkmark & \ding{55}  & \checkmark & \ding{55}   & \ding{55}                      & \ding{55} & \ding{55}     & \checkmark & \checkmark & \ding{55}       & \ding{55}    & \checkmark    & \checkmark              &\checkmark                      & \ding{55}                    \\
IC-DST                 & \ding{55}  & \ding{55}  & \ding{55}  & \ding{55}  & \ding{55}  & \checkmark & \ding{55}   & \ding{55}                      & \ding{55} & \ding{55}     & \ding{55}  & \ding{55}  & \ding{55}       & \ding{55}    & \ding{55}     & \ding{55}               & \ding{55}                      & \checkmark                   \\
InstructDIAL           & \ding{55}  & \ding{55}  & \ding{55}  & \ding{55}  & \ding{55}  & \checkmark & \checkmark  & \ding{55}                      & \ding{55} & \ding{55}     & \ding{55}  & \ding{55}  & \ding{55}       & \ding{55}    & \ding{55}     & \ding{55}               &\checkmark                      & \ding{55}                    \\
Any-TOD                & \ding{55}  & \ding{55}  & \checkmark & \ding{55}  & \ding{55}  & \checkmark & \ding{55}   & \ding{55}                      & \checkmark & \checkmark   & \ding{55}  & \ding{55}  & \ding{55}       & \ding{55}    & \ding{55}     & \ding{55}               & \ding{55}                      & \checkmark                   \\
Vojita                 & \ding{55}  & \checkmark & \checkmark & \ding{55}  & \ding{55}  & \checkmark & \checkmark  & \checkmark                     & \ding{55} & \ding{55}     & \ding{55}  & \ding{55}  & \checkmark      & \checkmark   & \ding{55}     & \ding{55}               & \ding{55}                      & \checkmark                   \\
SGP-TOD                & \checkmark & \checkmark & \checkmark & \checkmark & \checkmark & \ding{55}  & \ding{55}   & \ding{55}                      & \checkmark & \checkmark   & \ding{55}  & \ding{55}  & \ding{55}       & \ding{55}    & \ding{55}     & \ding{55}               & \ding{55}                      & \checkmark                   \\
LDST                   & \ding{55}  & \ding{55}  & \ding{55}  & \ding{55}  & \ding{55}  & \checkmark & \ding{55}   & \ding{55}                      & \ding{55} & \ding{55}     & \ding{55}  & \ding{55}  & \ding{55}       & \ding{55}    & \ding{55}     & \ding{55}               &\checkmark                      & \ding{55}                    \\
Prompter               & \ding{55}  & \ding{55}  & \ding{55}  & \ding{55}  & \ding{55}  & \checkmark & \ding{55}   & \ding{55}                      & \ding{55} & \ding{55}     & \ding{55}  & \ding{55}  & \ding{55}       & \ding{55}    & \ding{55}     & \ding{55}               &\checkmark                      & \ding{55}                    \\
LUAS                   & \ding{55}  & \ding{55}  & \ding{55}  & \ding{55}  & \ding{55}  & \checkmark & \checkmark  & \ding{55}                      & \ding{55} & \ding{55}     & \ding{55}  & \ding{55}  & \ding{55}       & \ding{55}    & \ding{55}     & \ding{55}               &\checkmark                      & \ding{55}                    \\
AutoTOD                & \checkmark & \checkmark & \ding{55}  & \checkmark & \ding{55}  & \ding{55}  & \ding{55}   & \ding{55}                      & \ding{55} & \ding{55}     & \checkmark & \checkmark & \checkmark      & \checkmark   & \ding{55}     & \ding{55}               & \ding{55}                      & \checkmark                   \\ \bottomrule
\end{tabular}
}
\caption{\textbf{Comparison of TOD Models: Methods, Metrics, and Evaluation Approaches.} This table presents a comprehensive comparison of various Task-Oriented Dialogue (TOD) models, focusing on their evaluation metrics, tasks, and methods. It compares models across End-to-End (E2E) dialogue, Dialogue State Tracking (DST), Domain Detection, and Next-Action-Prediction (NAP) tasks, as well as human evaluation metrics such as Coherency, Fluency, Informativeness, Satisfaction, Understanding, and Truthfulness. The evaluation metrics include Inform (measuring if required information is provided), Success (task completion rate), BLEU (similarity between generated and reference responses based on n-grams), C (combined score, (Inform + Success) × 0.5 + BLEU), BERTScore (similarity using contextual embeddings), JGA (Joint Goal Accuracy), Slot-F1 (F1 score for slot-filling), Accuracy (general prediction correctness), and F1 (harmonic mean of precision and recall). The table also distinguishes between models using additional task-specific training (fine-tuning) and in-context learning (ICL, prompting). Models are marked with checkmarks ($\checkmark$) where evaluations were conducted, and crosses ($\times$) where not. This comparison underscores the need for consistent evaluation practices and transparency in code sharing to ensure reproducibility and fairness in TOD model assessments.}
\label{tab:my-table}
\end{table*}







% Bibliography entries for the entire Anthology, followed by custom entries
%\bibliography{anthology,custom}
% Custom bibliography entries only
\bibliography{custom}

\newpage
\clearpage
\appendix

\section*{Appendix}
\label{sec:appendix}

\subsection{Lloyd-Max Algorithm}
\label{subsec:Lloyd-Max}
For a given quantization bitwidth $B$ and an operand $\bm{X}$, the Lloyd-Max algorithm finds $2^B$ quantization levels $\{\hat{x}_i\}_{i=1}^{2^B}$ such that quantizing $\bm{X}$ by rounding each scalar in $\bm{X}$ to the nearest quantization level minimizes the quantization MSE. 

The algorithm starts with an initial guess of quantization levels and then iteratively computes quantization thresholds $\{\tau_i\}_{i=1}^{2^B-1}$ and updates quantization levels $\{\hat{x}_i\}_{i=1}^{2^B}$. Specifically, at iteration $n$, thresholds are set to the midpoints of the previous iteration's levels:
\begin{align*}
    \tau_i^{(n)}=\frac{\hat{x}_i^{(n-1)}+\hat{x}_{i+1}^{(n-1)}}2 \text{ for } i=1\ldots 2^B-1
\end{align*}
Subsequently, the quantization levels are re-computed as conditional means of the data regions defined by the new thresholds:
\begin{align*}
    \hat{x}_i^{(n)}=\mathbb{E}\left[ \bm{X} \big| \bm{X}\in [\tau_{i-1}^{(n)},\tau_i^{(n)}] \right] \text{ for } i=1\ldots 2^B
\end{align*}
where to satisfy boundary conditions we have $\tau_0=-\infty$ and $\tau_{2^B}=\infty$. The algorithm iterates the above steps until convergence.

Figure \ref{fig:lm_quant} compares the quantization levels of a $7$-bit floating point (E3M3) quantizer (left) to a $7$-bit Lloyd-Max quantizer (right) when quantizing a layer of weights from the GPT3-126M model at a per-tensor granularity. As shown, the Lloyd-Max quantizer achieves substantially lower quantization MSE. Further, Table \ref{tab:FP7_vs_LM7} shows the superior perplexity achieved by Lloyd-Max quantizers for bitwidths of $7$, $6$ and $5$. The difference between the quantizers is clear at 5 bits, where per-tensor FP quantization incurs a drastic and unacceptable increase in perplexity, while Lloyd-Max quantization incurs a much smaller increase. Nevertheless, we note that even the optimal Lloyd-Max quantizer incurs a notable ($\sim 1.5$) increase in perplexity due to the coarse granularity of quantization. 

\begin{figure}[h]
  \centering
  \includegraphics[width=0.7\linewidth]{sections/figures/LM7_FP7.pdf}
  \caption{\small Quantization levels and the corresponding quantization MSE of Floating Point (left) vs Lloyd-Max (right) Quantizers for a layer of weights in the GPT3-126M model.}
  \label{fig:lm_quant}
\end{figure}

\begin{table}[h]\scriptsize
\begin{center}
\caption{\label{tab:FP7_vs_LM7} \small Comparing perplexity (lower is better) achieved by floating point quantizers and Lloyd-Max quantizers on a GPT3-126M model for the Wikitext-103 dataset.}
\begin{tabular}{c|cc|c}
\hline
 \multirow{2}{*}{\textbf{Bitwidth}} & \multicolumn{2}{|c|}{\textbf{Floating-Point Quantizer}} & \textbf{Lloyd-Max Quantizer} \\
 & Best Format & Wikitext-103 Perplexity & Wikitext-103 Perplexity \\
\hline
7 & E3M3 & 18.32 & 18.27 \\
6 & E3M2 & 19.07 & 18.51 \\
5 & E4M0 & 43.89 & 19.71 \\
\hline
\end{tabular}
\end{center}
\end{table}

\subsection{Proof of Local Optimality of LO-BCQ}
\label{subsec:lobcq_opt_proof}
For a given block $\bm{b}_j$, the quantization MSE during LO-BCQ can be empirically evaluated as $\frac{1}{L_b}\lVert \bm{b}_j- \bm{\hat{b}}_j\rVert^2_2$ where $\bm{\hat{b}}_j$ is computed from equation (\ref{eq:clustered_quantization_definition}) as $C_{f(\bm{b}_j)}(\bm{b}_j)$. Further, for a given block cluster $\mathcal{B}_i$, we compute the quantization MSE as $\frac{1}{|\mathcal{B}_{i}|}\sum_{\bm{b} \in \mathcal{B}_{i}} \frac{1}{L_b}\lVert \bm{b}- C_i^{(n)}(\bm{b})\rVert^2_2$. Therefore, at the end of iteration $n$, we evaluate the overall quantization MSE $J^{(n)}$ for a given operand $\bm{X}$ composed of $N_c$ block clusters as:
\begin{align*}
    \label{eq:mse_iter_n}
    J^{(n)} = \frac{1}{N_c} \sum_{i=1}^{N_c} \frac{1}{|\mathcal{B}_{i}^{(n)}|}\sum_{\bm{v} \in \mathcal{B}_{i}^{(n)}} \frac{1}{L_b}\lVert \bm{b}- B_i^{(n)}(\bm{b})\rVert^2_2
\end{align*}

At the end of iteration $n$, the codebooks are updated from $\mathcal{C}^{(n-1)}$ to $\mathcal{C}^{(n)}$. However, the mapping of a given vector $\bm{b}_j$ to quantizers $\mathcal{C}^{(n)}$ remains as  $f^{(n)}(\bm{b}_j)$. At the next iteration, during the vector clustering step, $f^{(n+1)}(\bm{b}_j)$ finds new mapping of $\bm{b}_j$ to updated codebooks $\mathcal{C}^{(n)}$ such that the quantization MSE over the candidate codebooks is minimized. Therefore, we obtain the following result for $\bm{b}_j$:
\begin{align*}
\frac{1}{L_b}\lVert \bm{b}_j - C_{f^{(n+1)}(\bm{b}_j)}^{(n)}(\bm{b}_j)\rVert^2_2 \le \frac{1}{L_b}\lVert \bm{b}_j - C_{f^{(n)}(\bm{b}_j)}^{(n)}(\bm{b}_j)\rVert^2_2
\end{align*}

That is, quantizing $\bm{b}_j$ at the end of the block clustering step of iteration $n+1$ results in lower quantization MSE compared to quantizing at the end of iteration $n$. Since this is true for all $\bm{b} \in \bm{X}$, we assert the following:
\begin{equation}
\begin{split}
\label{eq:mse_ineq_1}
    \tilde{J}^{(n+1)} &= \frac{1}{N_c} \sum_{i=1}^{N_c} \frac{1}{|\mathcal{B}_{i}^{(n+1)}|}\sum_{\bm{b} \in \mathcal{B}_{i}^{(n+1)}} \frac{1}{L_b}\lVert \bm{b} - C_i^{(n)}(b)\rVert^2_2 \le J^{(n)}
\end{split}
\end{equation}
where $\tilde{J}^{(n+1)}$ is the the quantization MSE after the vector clustering step at iteration $n+1$.

Next, during the codebook update step (\ref{eq:quantizers_update}) at iteration $n+1$, the per-cluster codebooks $\mathcal{C}^{(n)}$ are updated to $\mathcal{C}^{(n+1)}$ by invoking the Lloyd-Max algorithm \citep{Lloyd}. We know that for any given value distribution, the Lloyd-Max algorithm minimizes the quantization MSE. Therefore, for a given vector cluster $\mathcal{B}_i$ we obtain the following result:

\begin{equation}
    \frac{1}{|\mathcal{B}_{i}^{(n+1)}|}\sum_{\bm{b} \in \mathcal{B}_{i}^{(n+1)}} \frac{1}{L_b}\lVert \bm{b}- C_i^{(n+1)}(\bm{b})\rVert^2_2 \le \frac{1}{|\mathcal{B}_{i}^{(n+1)}|}\sum_{\bm{b} \in \mathcal{B}_{i}^{(n+1)}} \frac{1}{L_b}\lVert \bm{b}- C_i^{(n)}(\bm{b})\rVert^2_2
\end{equation}

The above equation states that quantizing the given block cluster $\mathcal{B}_i$ after updating the associated codebook from $C_i^{(n)}$ to $C_i^{(n+1)}$ results in lower quantization MSE. Since this is true for all the block clusters, we derive the following result: 
\begin{equation}
\begin{split}
\label{eq:mse_ineq_2}
     J^{(n+1)} &= \frac{1}{N_c} \sum_{i=1}^{N_c} \frac{1}{|\mathcal{B}_{i}^{(n+1)}|}\sum_{\bm{b} \in \mathcal{B}_{i}^{(n+1)}} \frac{1}{L_b}\lVert \bm{b}- C_i^{(n+1)}(\bm{b})\rVert^2_2  \le \tilde{J}^{(n+1)}   
\end{split}
\end{equation}

Following (\ref{eq:mse_ineq_1}) and (\ref{eq:mse_ineq_2}), we find that the quantization MSE is non-increasing for each iteration, that is, $J^{(1)} \ge J^{(2)} \ge J^{(3)} \ge \ldots \ge J^{(M)}$ where $M$ is the maximum number of iterations. 
%Therefore, we can say that if the algorithm converges, then it must be that it has converged to a local minimum. 
\hfill $\blacksquare$


\begin{figure}
    \begin{center}
    \includegraphics[width=0.5\textwidth]{sections//figures/mse_vs_iter.pdf}
    \end{center}
    \caption{\small NMSE vs iterations during LO-BCQ compared to other block quantization proposals}
    \label{fig:nmse_vs_iter}
\end{figure}

Figure \ref{fig:nmse_vs_iter} shows the empirical convergence of LO-BCQ across several block lengths and number of codebooks. Also, the MSE achieved by LO-BCQ is compared to baselines such as MXFP and VSQ. As shown, LO-BCQ converges to a lower MSE than the baselines. Further, we achieve better convergence for larger number of codebooks ($N_c$) and for a smaller block length ($L_b$), both of which increase the bitwidth of BCQ (see Eq \ref{eq:bitwidth_bcq}).


\subsection{Additional Accuracy Results}
%Table \ref{tab:lobcq_config} lists the various LOBCQ configurations and their corresponding bitwidths.
\begin{table}
\setlength{\tabcolsep}{4.75pt}
\begin{center}
\caption{\label{tab:lobcq_config} Various LO-BCQ configurations and their bitwidths.}
\begin{tabular}{|c||c|c|c|c||c|c||c|} 
\hline
 & \multicolumn{4}{|c||}{$L_b=8$} & \multicolumn{2}{|c||}{$L_b=4$} & $L_b=2$ \\
 \hline
 \backslashbox{$L_A$\kern-1em}{\kern-1em$N_c$} & 2 & 4 & 8 & 16 & 2 & 4 & 2 \\
 \hline
 64 & 4.25 & 4.375 & 4.5 & 4.625 & 4.375 & 4.625 & 4.625\\
 \hline
 32 & 4.375 & 4.5 & 4.625& 4.75 & 4.5 & 4.75 & 4.75 \\
 \hline
 16 & 4.625 & 4.75& 4.875 & 5 & 4.75 & 5 & 5 \\
 \hline
\end{tabular}
\end{center}
\end{table}

%\subsection{Perplexity achieved by various LO-BCQ configurations on Wikitext-103 dataset}

\begin{table} \centering
\begin{tabular}{|c||c|c|c|c||c|c||c|} 
\hline
 $L_b \rightarrow$& \multicolumn{4}{c||}{8} & \multicolumn{2}{c||}{4} & 2\\
 \hline
 \backslashbox{$L_A$\kern-1em}{\kern-1em$N_c$} & 2 & 4 & 8 & 16 & 2 & 4 & 2  \\
 %$N_c \rightarrow$ & 2 & 4 & 8 & 16 & 2 & 4 & 2 \\
 \hline
 \hline
 \multicolumn{8}{c}{GPT3-1.3B (FP32 PPL = 9.98)} \\ 
 \hline
 \hline
 64 & 10.40 & 10.23 & 10.17 & 10.15 &  10.28 & 10.18 & 10.19 \\
 \hline
 32 & 10.25 & 10.20 & 10.15 & 10.12 &  10.23 & 10.17 & 10.17 \\
 \hline
 16 & 10.22 & 10.16 & 10.10 & 10.09 &  10.21 & 10.14 & 10.16 \\
 \hline
  \hline
 \multicolumn{8}{c}{GPT3-8B (FP32 PPL = 7.38)} \\ 
 \hline
 \hline
 64 & 7.61 & 7.52 & 7.48 &  7.47 &  7.55 &  7.49 & 7.50 \\
 \hline
 32 & 7.52 & 7.50 & 7.46 &  7.45 &  7.52 &  7.48 & 7.48  \\
 \hline
 16 & 7.51 & 7.48 & 7.44 &  7.44 &  7.51 &  7.49 & 7.47  \\
 \hline
\end{tabular}
\caption{\label{tab:ppl_gpt3_abalation} Wikitext-103 perplexity across GPT3-1.3B and 8B models.}
\end{table}

\begin{table} \centering
\begin{tabular}{|c||c|c|c|c||} 
\hline
 $L_b \rightarrow$& \multicolumn{4}{c||}{8}\\
 \hline
 \backslashbox{$L_A$\kern-1em}{\kern-1em$N_c$} & 2 & 4 & 8 & 16 \\
 %$N_c \rightarrow$ & 2 & 4 & 8 & 16 & 2 & 4 & 2 \\
 \hline
 \hline
 \multicolumn{5}{|c|}{Llama2-7B (FP32 PPL = 5.06)} \\ 
 \hline
 \hline
 64 & 5.31 & 5.26 & 5.19 & 5.18  \\
 \hline
 32 & 5.23 & 5.25 & 5.18 & 5.15  \\
 \hline
 16 & 5.23 & 5.19 & 5.16 & 5.14  \\
 \hline
 \multicolumn{5}{|c|}{Nemotron4-15B (FP32 PPL = 5.87)} \\ 
 \hline
 \hline
 64  & 6.3 & 6.20 & 6.13 & 6.08  \\
 \hline
 32  & 6.24 & 6.12 & 6.07 & 6.03  \\
 \hline
 16  & 6.12 & 6.14 & 6.04 & 6.02  \\
 \hline
 \multicolumn{5}{|c|}{Nemotron4-340B (FP32 PPL = 3.48)} \\ 
 \hline
 \hline
 64 & 3.67 & 3.62 & 3.60 & 3.59 \\
 \hline
 32 & 3.63 & 3.61 & 3.59 & 3.56 \\
 \hline
 16 & 3.61 & 3.58 & 3.57 & 3.55 \\
 \hline
\end{tabular}
\caption{\label{tab:ppl_llama7B_nemo15B} Wikitext-103 perplexity compared to FP32 baseline in Llama2-7B and Nemotron4-15B, 340B models}
\end{table}

%\subsection{Perplexity achieved by various LO-BCQ configurations on MMLU dataset}


\begin{table} \centering
\begin{tabular}{|c||c|c|c|c||c|c|c|c|} 
\hline
 $L_b \rightarrow$& \multicolumn{4}{c||}{8} & \multicolumn{4}{c||}{8}\\
 \hline
 \backslashbox{$L_A$\kern-1em}{\kern-1em$N_c$} & 2 & 4 & 8 & 16 & 2 & 4 & 8 & 16  \\
 %$N_c \rightarrow$ & 2 & 4 & 8 & 16 & 2 & 4 & 2 \\
 \hline
 \hline
 \multicolumn{5}{|c|}{Llama2-7B (FP32 Accuracy = 45.8\%)} & \multicolumn{4}{|c|}{Llama2-70B (FP32 Accuracy = 69.12\%)} \\ 
 \hline
 \hline
 64 & 43.9 & 43.4 & 43.9 & 44.9 & 68.07 & 68.27 & 68.17 & 68.75 \\
 \hline
 32 & 44.5 & 43.8 & 44.9 & 44.5 & 68.37 & 68.51 & 68.35 & 68.27  \\
 \hline
 16 & 43.9 & 42.7 & 44.9 & 45 & 68.12 & 68.77 & 68.31 & 68.59  \\
 \hline
 \hline
 \multicolumn{5}{|c|}{GPT3-22B (FP32 Accuracy = 38.75\%)} & \multicolumn{4}{|c|}{Nemotron4-15B (FP32 Accuracy = 64.3\%)} \\ 
 \hline
 \hline
 64 & 36.71 & 38.85 & 38.13 & 38.92 & 63.17 & 62.36 & 63.72 & 64.09 \\
 \hline
 32 & 37.95 & 38.69 & 39.45 & 38.34 & 64.05 & 62.30 & 63.8 & 64.33  \\
 \hline
 16 & 38.88 & 38.80 & 38.31 & 38.92 & 63.22 & 63.51 & 63.93 & 64.43  \\
 \hline
\end{tabular}
\caption{\label{tab:mmlu_abalation} Accuracy on MMLU dataset across GPT3-22B, Llama2-7B, 70B and Nemotron4-15B models.}
\end{table}


%\subsection{Perplexity achieved by various LO-BCQ configurations on LM evaluation harness}

\begin{table} \centering
\begin{tabular}{|c||c|c|c|c||c|c|c|c|} 
\hline
 $L_b \rightarrow$& \multicolumn{4}{c||}{8} & \multicolumn{4}{c||}{8}\\
 \hline
 \backslashbox{$L_A$\kern-1em}{\kern-1em$N_c$} & 2 & 4 & 8 & 16 & 2 & 4 & 8 & 16  \\
 %$N_c \rightarrow$ & 2 & 4 & 8 & 16 & 2 & 4 & 2 \\
 \hline
 \hline
 \multicolumn{5}{|c|}{Race (FP32 Accuracy = 37.51\%)} & \multicolumn{4}{|c|}{Boolq (FP32 Accuracy = 64.62\%)} \\ 
 \hline
 \hline
 64 & 36.94 & 37.13 & 36.27 & 37.13 & 63.73 & 62.26 & 63.49 & 63.36 \\
 \hline
 32 & 37.03 & 36.36 & 36.08 & 37.03 & 62.54 & 63.51 & 63.49 & 63.55  \\
 \hline
 16 & 37.03 & 37.03 & 36.46 & 37.03 & 61.1 & 63.79 & 63.58 & 63.33  \\
 \hline
 \hline
 \multicolumn{5}{|c|}{Winogrande (FP32 Accuracy = 58.01\%)} & \multicolumn{4}{|c|}{Piqa (FP32 Accuracy = 74.21\%)} \\ 
 \hline
 \hline
 64 & 58.17 & 57.22 & 57.85 & 58.33 & 73.01 & 73.07 & 73.07 & 72.80 \\
 \hline
 32 & 59.12 & 58.09 & 57.85 & 58.41 & 73.01 & 73.94 & 72.74 & 73.18  \\
 \hline
 16 & 57.93 & 58.88 & 57.93 & 58.56 & 73.94 & 72.80 & 73.01 & 73.94  \\
 \hline
\end{tabular}
\caption{\label{tab:mmlu_abalation} Accuracy on LM evaluation harness tasks on GPT3-1.3B model.}
\end{table}

\begin{table} \centering
\begin{tabular}{|c||c|c|c|c||c|c|c|c|} 
\hline
 $L_b \rightarrow$& \multicolumn{4}{c||}{8} & \multicolumn{4}{c||}{8}\\
 \hline
 \backslashbox{$L_A$\kern-1em}{\kern-1em$N_c$} & 2 & 4 & 8 & 16 & 2 & 4 & 8 & 16  \\
 %$N_c \rightarrow$ & 2 & 4 & 8 & 16 & 2 & 4 & 2 \\
 \hline
 \hline
 \multicolumn{5}{|c|}{Race (FP32 Accuracy = 41.34\%)} & \multicolumn{4}{|c|}{Boolq (FP32 Accuracy = 68.32\%)} \\ 
 \hline
 \hline
 64 & 40.48 & 40.10 & 39.43 & 39.90 & 69.20 & 68.41 & 69.45 & 68.56 \\
 \hline
 32 & 39.52 & 39.52 & 40.77 & 39.62 & 68.32 & 67.43 & 68.17 & 69.30  \\
 \hline
 16 & 39.81 & 39.71 & 39.90 & 40.38 & 68.10 & 66.33 & 69.51 & 69.42  \\
 \hline
 \hline
 \multicolumn{5}{|c|}{Winogrande (FP32 Accuracy = 67.88\%)} & \multicolumn{4}{|c|}{Piqa (FP32 Accuracy = 78.78\%)} \\ 
 \hline
 \hline
 64 & 66.85 & 66.61 & 67.72 & 67.88 & 77.31 & 77.42 & 77.75 & 77.64 \\
 \hline
 32 & 67.25 & 67.72 & 67.72 & 67.00 & 77.31 & 77.04 & 77.80 & 77.37  \\
 \hline
 16 & 68.11 & 68.90 & 67.88 & 67.48 & 77.37 & 78.13 & 78.13 & 77.69  \\
 \hline
\end{tabular}
\caption{\label{tab:mmlu_abalation} Accuracy on LM evaluation harness tasks on GPT3-8B model.}
\end{table}

\begin{table} \centering
\begin{tabular}{|c||c|c|c|c||c|c|c|c|} 
\hline
 $L_b \rightarrow$& \multicolumn{4}{c||}{8} & \multicolumn{4}{c||}{8}\\
 \hline
 \backslashbox{$L_A$\kern-1em}{\kern-1em$N_c$} & 2 & 4 & 8 & 16 & 2 & 4 & 8 & 16  \\
 %$N_c \rightarrow$ & 2 & 4 & 8 & 16 & 2 & 4 & 2 \\
 \hline
 \hline
 \multicolumn{5}{|c|}{Race (FP32 Accuracy = 40.67\%)} & \multicolumn{4}{|c|}{Boolq (FP32 Accuracy = 76.54\%)} \\ 
 \hline
 \hline
 64 & 40.48 & 40.10 & 39.43 & 39.90 & 75.41 & 75.11 & 77.09 & 75.66 \\
 \hline
 32 & 39.52 & 39.52 & 40.77 & 39.62 & 76.02 & 76.02 & 75.96 & 75.35  \\
 \hline
 16 & 39.81 & 39.71 & 39.90 & 40.38 & 75.05 & 73.82 & 75.72 & 76.09  \\
 \hline
 \hline
 \multicolumn{5}{|c|}{Winogrande (FP32 Accuracy = 70.64\%)} & \multicolumn{4}{|c|}{Piqa (FP32 Accuracy = 79.16\%)} \\ 
 \hline
 \hline
 64 & 69.14 & 70.17 & 70.17 & 70.56 & 78.24 & 79.00 & 78.62 & 78.73 \\
 \hline
 32 & 70.96 & 69.69 & 71.27 & 69.30 & 78.56 & 79.49 & 79.16 & 78.89  \\
 \hline
 16 & 71.03 & 69.53 & 69.69 & 70.40 & 78.13 & 79.16 & 79.00 & 79.00  \\
 \hline
\end{tabular}
\caption{\label{tab:mmlu_abalation} Accuracy on LM evaluation harness tasks on GPT3-22B model.}
\end{table}

\begin{table} \centering
\begin{tabular}{|c||c|c|c|c||c|c|c|c|} 
\hline
 $L_b \rightarrow$& \multicolumn{4}{c||}{8} & \multicolumn{4}{c||}{8}\\
 \hline
 \backslashbox{$L_A$\kern-1em}{\kern-1em$N_c$} & 2 & 4 & 8 & 16 & 2 & 4 & 8 & 16  \\
 %$N_c \rightarrow$ & 2 & 4 & 8 & 16 & 2 & 4 & 2 \\
 \hline
 \hline
 \multicolumn{5}{|c|}{Race (FP32 Accuracy = 44.4\%)} & \multicolumn{4}{|c|}{Boolq (FP32 Accuracy = 79.29\%)} \\ 
 \hline
 \hline
 64 & 42.49 & 42.51 & 42.58 & 43.45 & 77.58 & 77.37 & 77.43 & 78.1 \\
 \hline
 32 & 43.35 & 42.49 & 43.64 & 43.73 & 77.86 & 75.32 & 77.28 & 77.86  \\
 \hline
 16 & 44.21 & 44.21 & 43.64 & 42.97 & 78.65 & 77 & 76.94 & 77.98  \\
 \hline
 \hline
 \multicolumn{5}{|c|}{Winogrande (FP32 Accuracy = 69.38\%)} & \multicolumn{4}{|c|}{Piqa (FP32 Accuracy = 78.07\%)} \\ 
 \hline
 \hline
 64 & 68.9 & 68.43 & 69.77 & 68.19 & 77.09 & 76.82 & 77.09 & 77.86 \\
 \hline
 32 & 69.38 & 68.51 & 68.82 & 68.90 & 78.07 & 76.71 & 78.07 & 77.86  \\
 \hline
 16 & 69.53 & 67.09 & 69.38 & 68.90 & 77.37 & 77.8 & 77.91 & 77.69  \\
 \hline
\end{tabular}
\caption{\label{tab:mmlu_abalation} Accuracy on LM evaluation harness tasks on Llama2-7B model.}
\end{table}

\begin{table} \centering
\begin{tabular}{|c||c|c|c|c||c|c|c|c|} 
\hline
 $L_b \rightarrow$& \multicolumn{4}{c||}{8} & \multicolumn{4}{c||}{8}\\
 \hline
 \backslashbox{$L_A$\kern-1em}{\kern-1em$N_c$} & 2 & 4 & 8 & 16 & 2 & 4 & 8 & 16  \\
 %$N_c \rightarrow$ & 2 & 4 & 8 & 16 & 2 & 4 & 2 \\
 \hline
 \hline
 \multicolumn{5}{|c|}{Race (FP32 Accuracy = 48.8\%)} & \multicolumn{4}{|c|}{Boolq (FP32 Accuracy = 85.23\%)} \\ 
 \hline
 \hline
 64 & 49.00 & 49.00 & 49.28 & 48.71 & 82.82 & 84.28 & 84.03 & 84.25 \\
 \hline
 32 & 49.57 & 48.52 & 48.33 & 49.28 & 83.85 & 84.46 & 84.31 & 84.93  \\
 \hline
 16 & 49.85 & 49.09 & 49.28 & 48.99 & 85.11 & 84.46 & 84.61 & 83.94  \\
 \hline
 \hline
 \multicolumn{5}{|c|}{Winogrande (FP32 Accuracy = 79.95\%)} & \multicolumn{4}{|c|}{Piqa (FP32 Accuracy = 81.56\%)} \\ 
 \hline
 \hline
 64 & 78.77 & 78.45 & 78.37 & 79.16 & 81.45 & 80.69 & 81.45 & 81.5 \\
 \hline
 32 & 78.45 & 79.01 & 78.69 & 80.66 & 81.56 & 80.58 & 81.18 & 81.34  \\
 \hline
 16 & 79.95 & 79.56 & 79.79 & 79.72 & 81.28 & 81.66 & 81.28 & 80.96  \\
 \hline
\end{tabular}
\caption{\label{tab:mmlu_abalation} Accuracy on LM evaluation harness tasks on Llama2-70B model.}
\end{table}

%\section{MSE Studies}
%\textcolor{red}{TODO}


\subsection{Number Formats and Quantization Method}
\label{subsec:numFormats_quantMethod}
\subsubsection{Integer Format}
An $n$-bit signed integer (INT) is typically represented with a 2s-complement format \citep{yao2022zeroquant,xiao2023smoothquant,dai2021vsq}, where the most significant bit denotes the sign.

\subsubsection{Floating Point Format}
An $n$-bit signed floating point (FP) number $x$ comprises of a 1-bit sign ($x_{\mathrm{sign}}$), $B_m$-bit mantissa ($x_{\mathrm{mant}}$) and $B_e$-bit exponent ($x_{\mathrm{exp}}$) such that $B_m+B_e=n-1$. The associated constant exponent bias ($E_{\mathrm{bias}}$) is computed as $(2^{{B_e}-1}-1)$. We denote this format as $E_{B_e}M_{B_m}$.  

\subsubsection{Quantization Scheme}
\label{subsec:quant_method}
A quantization scheme dictates how a given unquantized tensor is converted to its quantized representation. We consider FP formats for the purpose of illustration. Given an unquantized tensor $\bm{X}$ and an FP format $E_{B_e}M_{B_m}$, we first, we compute the quantization scale factor $s_X$ that maps the maximum absolute value of $\bm{X}$ to the maximum quantization level of the $E_{B_e}M_{B_m}$ format as follows:
\begin{align}
\label{eq:sf}
    s_X = \frac{\mathrm{max}(|\bm{X}|)}{\mathrm{max}(E_{B_e}M_{B_m})}
\end{align}
In the above equation, $|\cdot|$ denotes the absolute value function.

Next, we scale $\bm{X}$ by $s_X$ and quantize it to $\hat{\bm{X}}$ by rounding it to the nearest quantization level of $E_{B_e}M_{B_m}$ as:

\begin{align}
\label{eq:tensor_quant}
    \hat{\bm{X}} = \text{round-to-nearest}\left(\frac{\bm{X}}{s_X}, E_{B_e}M_{B_m}\right)
\end{align}

We perform dynamic max-scaled quantization \citep{wu2020integer}, where the scale factor $s$ for activations is dynamically computed during runtime.

\subsection{Vector Scaled Quantization}
\begin{wrapfigure}{r}{0.35\linewidth}
  \centering
  \includegraphics[width=\linewidth]{sections/figures/vsquant.jpg}
  \caption{\small Vectorwise decomposition for per-vector scaled quantization (VSQ \citep{dai2021vsq}).}
  \label{fig:vsquant}
\end{wrapfigure}
During VSQ \citep{dai2021vsq}, the operand tensors are decomposed into 1D vectors in a hardware friendly manner as shown in Figure \ref{fig:vsquant}. Since the decomposed tensors are used as operands in matrix multiplications during inference, it is beneficial to perform this decomposition along the reduction dimension of the multiplication. The vectorwise quantization is performed similar to tensorwise quantization described in Equations \ref{eq:sf} and \ref{eq:tensor_quant}, where a scale factor $s_v$ is required for each vector $\bm{v}$ that maps the maximum absolute value of that vector to the maximum quantization level. While smaller vector lengths can lead to larger accuracy gains, the associated memory and computational overheads due to the per-vector scale factors increases. To alleviate these overheads, VSQ \citep{dai2021vsq} proposed a second level quantization of the per-vector scale factors to unsigned integers, while MX \citep{rouhani2023shared} quantizes them to integer powers of 2 (denoted as $2^{INT}$).

\subsubsection{MX Format}
The MX format proposed in \citep{rouhani2023microscaling} introduces the concept of sub-block shifting. For every two scalar elements of $b$-bits each, there is a shared exponent bit. The value of this exponent bit is determined through an empirical analysis that targets minimizing quantization MSE. We note that the FP format $E_{1}M_{b}$ is strictly better than MX from an accuracy perspective since it allocates a dedicated exponent bit to each scalar as opposed to sharing it across two scalars. Therefore, we conservatively bound the accuracy of a $b+2$-bit signed MX format with that of a $E_{1}M_{b}$ format in our comparisons. For instance, we use E1M2 format as a proxy for MX4.

\begin{figure}
    \centering
    \includegraphics[width=1\linewidth]{sections//figures/BlockFormats.pdf}
    \caption{\small Comparing LO-BCQ to MX format.}
    \label{fig:block_formats}
\end{figure}

Figure \ref{fig:block_formats} compares our $4$-bit LO-BCQ block format to MX \citep{rouhani2023microscaling}. As shown, both LO-BCQ and MX decompose a given operand tensor into block arrays and each block array into blocks. Similar to MX, we find that per-block quantization ($L_b < L_A$) leads to better accuracy due to increased flexibility. While MX achieves this through per-block $1$-bit micro-scales, we associate a dedicated codebook to each block through a per-block codebook selector. Further, MX quantizes the per-block array scale-factor to E8M0 format without per-tensor scaling. In contrast during LO-BCQ, we find that per-tensor scaling combined with quantization of per-block array scale-factor to E4M3 format results in superior inference accuracy across models. 



\begin{figure*}[!ht]
\begin{tcolorbox}[colback=gray!5!white,colframe=black!95!black,title=\textbf{\small{SNIPS SFT Sample} | \textbf{Format:} Dialogue State Tracking}] 
\small \textcolor{darkgreen}{\textbf{Instruction:}} \\
You are a helpful assistant who is assigned to find the intents shown by the user on 7 domains - GetWeather, AddToPlaylist, SearchScreeningEvent, BookRestaurant, SearchCreativeWork, RateBook, PlayMusic. \\
 \\
The user can seek for BookRestaurant by slots - poi, restaurant\_type, served\_dish, timeRange, party\_size\_number, restaurant\_name, state, country, party\_size\_description, sort, city, spatial\_relation, cuisine, facility. \\
The user can seek for GetWeather by slots - condition\_temperature, geographic\_poi, current\_location, timeRange, condition\_description, state, country, city, spatial\_relation. \\
The user can seek for SearchCreativeWork by slots - object\_type, object\_name. \\
The user can seek for PlayMusic by slots - track, playlist, service, genre, year, album, music\_item, sort, artist. \\
The user can seek for SearchScreeningEvent by slots - movie\_name, location\_name, timeRange, object\_type, movie\_type, object\_location\_type, spatial\_relation. \\
The user can seek for RateBook by slots - rating\_value, rating\_unit, object\_type, object\_select, object\_part\_of\_series\_type, best\_rating, object\_name.
Do not capture any other slots! \\
 \\
\# Task \\
You will be provided with an user utterance. You must find all the user intents and output them in JSON format. \\
 \\
\# Sample Output \\
{"domain": "AddToPlaylist", "slot\_values": {"music\_item": "abc", "artist": "xyz"}} \\
 \\
\small \textcolor{darkgreen}{\textbf{Input:}} \\
User: Book a table at a restaurant in Portugal with parking for me and bonnie in 19 minutes\\
 \\
\small \textcolor{mypurple}{\textbf{Output:}} \\
System: {"domain": "BookRestaurant", "slot\_values": {"restaurant\_type": "restaurant", "country": "Portugal", "facility": "parking", "party\_size\_description": "me and bonnie", "timeRange": "in 19 minutes"}}"
\end{tcolorbox}

\vspace{-0.25cm}
\caption{SNIPS fine-tuning sample example.}
\label{tab:snips-dst}
\end{figure*}

 % "instruction": "You are a helpful assistant who is assigned to find the intents shown by the user on 7 domains - GetWeather, AddToPlaylist, SearchScreeningEvent, BookRestaurant, SearchCreativeWork, RateBook, PlayMusic.\n\nThe user can seek for AddToPlaylist by slots - playlist, music_item, entity_name, playlist_owner, artist.\nThe user can seek for BookRestaurant by slots - poi, restaurant_type, served_dish, timeRange, party_size_number, restaurant_name, state, country, party_size_description, sort, city, spatial_relation, cuisine, facility.\nThe user can seek for GetWeather by slots - condition_temperature, geographic_poi, current_location, timeRange, condition_description, state, country, city, spatial_relation.\nThe user can seek for SearchCreativeWork by slots - object_type, object_name.\nThe user can seek for PlayMusic by slots - track, playlist, service, genre, year, album, music_item, sort, artist.\nThe user can seek for SearchScreeningEvent by slots - movie_name, location_name, timeRange, object_type, movie_type, object_location_type, spatial_relation.\nThe user can seek for RateBook by slots - rating_value, rating_unit, object_type, object_select, object_part_of_series_type, best_rating, object_name.\nDo not capture any other slots!\n\n# Task\nYou will be provided with an user utterance. You must find all the user intents and output them in JSON format.\n\n# Sample Output\n{\"domain\": \"AddToPlaylist\", \"slot_values\": {\"music_item\": \"abc\", \"artist\": \"xyz\"}}\n\nUser: Book a table at a restaurant in Portugal with parking for me and bonnie in 19 minutes",
 %        "input": "",
 %        "output": "\nSystem: {\"domain\": \"BookRestaurant\", \"slot_values\": {\"restaurant_type\": \"restaurant\", \"country\": \"Portugal\", \"facility\": \"parking\", \"party_size_description\": \"me and bonnie\", \"timeRange\": \"in 19 minutes\"}}"
\newpage
\begin{figure*}[!ht]
\begin{tcolorbox}[colback=gray!5!white,colframe=black!95!black,title=\textbf{\small{Hammer SFT Sample} | \textbf{Format:} Function Calling}] 
\small \textcolor{darkgreen}{\textbf{Instruction:}} \\
{[BEGIN OF TASK INSTRUCTION]} \\
You are a tool calling assistant. In order to complete the user's request, you need to select one or more appropriate tools from the following tools and fill in the correct values for the tool parameters. Your specific tasks are: \\
1. Make one or more function/tool calls to meet the request based on the question. \\
2. If none of the function can be used, point it out and refuse to answer. \\
3. If the given question lacks the parameters required by the function, also point it out. \\
{[END OF TASK INSTRUCTION]} \\
 \\
{[BEGIN OF AVAILABLE TOOLS]} \\
{[{"name": "LxOm64zLyg", "description": "Gets hourly weather forecast information for given geographical coordinates using the RapidAPI service.", "parameters": {"TDpjPd": {"description": "The latitude of the geographical location.", "type": "int", "default": 46.95828}, "78th2U3lFj": {"description": "The longitude of the geographical location.", "type": "int", "default": 10.87152}}}, {"name": "WoDdNSe7e7K5", "description": "Fetches weather updates for a given city using the RapidAPI Weather API.", "parameters": {"LzZsvxUC": {"description": "The name of the city for which to retrieve weather information.", "type": "str", "default": "London"}}}, {"name": "CBrCNmwOERb", "description": "Fetches the hourly weather forecast for a given location using the RapidAPI service.", "parameters": {"TDEJ.ZwMt": {"description": "The name of the location for which to retrieve the hourly weather forecast.", "type": "str", "default": "Berlin"}}}, {"name": "1YTQVXkwLY", "description": "Returns an air quality forecast for a given location.", "parameters": {"2bkgDA": {"description": "The latitude of the location for which the air quality forecast is to be retrieved.", "type": "int", "default": "35.779"}, "DQi.ReZ16": {"description": "The longitude of the location for which the air quality forecast is to be retrieved.", "type": "int", "default": "-78.638"}, "hF.1": {"description": "The number of hours for which the forecast is to be retrieved (default is 72).", "type": "int", "default": "72"}}}]} \\
{[END OF AVAILABLE TOOLS]} \\
 \\
{[BEGIN OF FORMAT INSTRUCTION]} \\
The output MUST strictly adhere to the following JSON format, and NO other text MUST be included. \\
The example format is as follows. Please make sure the parameter type is correct. If no function call is needed, please directly output an empty list '{[]}' \\
{[ \\
{"name": "func\_name1", "arguments": {"argument1": "value1", "argument2": "value2"}}, \\
... (more tool calls as required) \\
]} \\
{[END OF FORMAT INSTRUCTION]} \\
\\
\small \textcolor{darkgreen}{\textbf{Input:}} \\
{[BEGIN OF QUERY]} \\
What are the current weather conditions in Sydney? \\
{[END OF QUERY]} \\
 \\
\small \textcolor{mypurple}{\textbf{Output:}} \\
{[{"name": "WoDdNSe7e7K5", "arguments": {"LzZsvxUC": "Sydney"}}]}
\end{tcolorbox}

\vspace{-0.25cm}
\caption{Hammer fine-tuning sample example.}
\label{tab:hammer-api-2}
\end{figure*}

% "instruction": "[BEGIN OF TASK INSTRUCTION]\nYou are a tool calling assistant. In order to complete the user's request, you need to select one or more appropriate tools from the following tools and fill in the correct values for the tool parameters. Your specific tasks are:\n1. Make one or more function/tool calls to meet the request based on the question.\n2. If none of the function can be used, point it out and refuse to answer.\n3. If the given question lacks the parameters required by the function, also point it out.\n\n[END OF TASK INSTRUCTION]\n\n[BEGIN OF AVAILABLE TOOLS]\n[{"name": "LxOm64zLyg", "description": "Gets hourly weather forecast information for given geographical coordinates using the RapidAPI service.", "parameters": {"TDpjPd": {"description": "The latitude of the geographical location.", "type": "int", "default": 46.95828}, "78th2U3lFj": {"description": "The longitude of the geographical location.", "type": "int", "default": 10.87152}}}, {"name": "WoDdNSe7e7K5", "description": "Fetches weather updates for a given city using the RapidAPI Weather API.", "parameters": {"LzZsvxUC": {"description": "The name of the city for which to retrieve weather information.", "type": "str", "default": "London"}}}, {"name": "CBrCNmwOERb", "description": "Fetches the hourly weather forecast for a given location using the RapidAPI service.", "parameters": {"TDEJ.ZwMt": {"description": "The name of the location for which to retrieve the hourly weather forecast.", "type": "str", "default": "Berlin"}}}, {"name": "1YTQVXkwLY", "description": "Returns an air quality forecast for a given location.", "parameters": {"2bkgDA": {"description": "The latitude of the location for which the air quality forecast is to be retrieved.", "type": "int", "default": "35.779"}, "DQi.ReZ16": {"description": "The longitude of the location for which the air quality forecast is to be retrieved.", "type": "int", "default": "-78.638"}, "hF.1": {"description": "The number of hours for which the forecast is to be retrieved (default is 72).", "type": "int", "default": "72"}}}]\n[END OF AVAILABLE TOOLS]\n\n[BEGIN OF FORMAT INSTRUCTION]\n\nThe output MUST strictly adhere to the following JSON format, and NO other text MUST be included.\nThe example format is as follows. Please make sure the parameter type is correct. If no function call is needed, please directly output an empty list '[]'\n```\n[\n    {"name": "func_name1", "arguments": {"argument1": "value1", "argument2": "value2"}},\n    ... (more tool calls as required)\n]\n```\n\n[END OF FORMAT INSTRUCTION]\n\n[BEGIN OF QUERY]\nWhat are the current weather conditions in Sydney?\n[END OF QUERY]\n\n",
%     "input": "",
%     "output": "```\n[{"name": "WoDdNSe7e7K5", "arguments": {"LzZsvxUC": "Sydney"}}]\n```"
%   },
\newpage
\begin{figure*}[!ht]
\begin{tcolorbox}[colback=gray!5!white,colframe=black!95!black,title=\textbf{\small{SGD Instruction Sample} | \textbf{Format:} Action Optimization}] 
\small \textcolor{darkgreen}{\textbf{Instruction:}} \\
{[BEGIN OF TASK INSTRUCTION]}\\
You are a helpful conversational assistant who can perform API function calling. \\
Your goal is to understand user queries and respond using the appropriate API functions. \\
In order to complete the user's request, you need to select a tool from the following functions and fill in the correct values for the function parameters. \\
Your specific tasks are: \\
1. Analyze the user’s query within the given dialogue context to identify their intent and relevant details. \\
2. Make a function/tool call and provide the necessary arguments to meet the request based on the user query. \\
3. Formulate a natural and coherent response, guiding the conversation towards resolving the user’s request. \\
{[END OF TASK INSTRUCTION]} \\
 \\
{[BEGIN OF AVAILABLE TOOLS]} \\
1. FindBus(from\_location, to\_location, leaving\_date) \\
2. ReserveCar(pickup\_location, pickup\_date, pickup\_time, dropoff\_date, type) \\
. \\
. \\
. \\
50. FindProvider(city, type) \\
51. GetRide(destination, number\_of\_riders, shared\_ride) \\
{[END OF AVAILABLE TOOLS]} \\
 \\
{[BEGIN OF FORMAT INSTRUCTION]} \\
The output MUST strictly adhere to the following structured text format. \\
Example Output API Call Format: \\
function\_name(argument1=value1, argument2=value2, ...) \\
{[END OF FORMAT INSTRUCTION]} \\
 \\
{[BEGIN OF CONVERSATION HISTORY]} \\
User: I'm looking for events happening in New York this weekend. \\
Thought: The user wants to find events in a specific location and timeframe.  \\
I'll need to call the FindEvents API with the appropriate category and city. \\
Action: FindEvents(category='all', city\_of\_event='New York') \\
Observation: {'events': [{'name': 'Jazz Festival', 'date': '2023-10-07', 'location': 'Central Park'}, {'name': 'Food Truck Rally', 'date': '2023-10-08', 'location': 'Union Square'}]} \\
Thought: I have the event details now. I should summarize the events for the user. \\
System: This weekend in New York, you can attend the Jazz Festival at Central Park on October 7th and the Food Truck Rally at Union Square on October 8th. Would you like more information about any of these events? \\
User: Yes, can you tell me more about the Jazz Festival? \\
Thought: I need to provide more details about the Jazz Festival. I don't need an API call for that. \\
System: The Jazz Festival in Central Park will feature various artists performing live jazz music throughout the day. It starts at 11 AM and goes until 8 PM. Would you like to know how to get tickets? \\
{[END OF CONVERSATION HISTORY]} \\
 \\
\small \textcolor{darkgreen}{\textbf{Input:}} \\
User: Yes, please tell me how to get tickets for the Jazz Festival. \\
Thought: The user is interested in purchasing tickets. I need to call the BuyEventTickets API for the Jazz Festival. \\
 \\
\small \textcolor{mypurple}{\textbf{Output:}} \\
Action: BuyEventTickets(event\_name='Jazz Festival', number\_of\_seats=2, date='2023-10-07', city\_of\_event='New York')\\
\end{tcolorbox}

\vspace{-0.25cm}
\caption{SGD fine-tuning sample example targeting function optimization.}
\label{tab:sgd-sft-action}
\end{figure*}


% "instruction": "[BEGIN OF TASK INSTRUCTION]\nYou are a helpful conversational assistant who can perform API function calling. \nYour goal is to understand user queries and respond using the appropriate API functions.\nIn order to complete the user's request, you need to select a tool from the following functions and fill in the correct values for the function parameters. \nYour specific tasks are:\n1. Analyze the user’s query within the given dialogue context to identify their intent and relevant details.\n2. Make a function/tool call and provide the necessary arguments to meet the request based on the user query.\n3. Formulate a natural and coherent response, guiding the conversation towards resolving the user’s request.\n[END OF TASK INSTRUCTION]\n\n[BEGIN OF AVAILABLE TOOLS]\n1. FindBus(from_location, to_location, leaving_date)\n2. ReserveCar(pickup_location, pickup_date, pickup_time, dropoff_date, type)\n3. PlayMovie(title)\n4. CheckBalance(account_type)\n5. SearchOnewayFlight(origin, destination, departure_date)\n6. SearchOnewayFlight(origin_city, destination_city, departure_date)\n7. FindEvents(event_type, city)\n8. GetRide(destination, number_of_seats, ride_type)\n9. BookAppointment(doctor_name, appointment_time, appointment_date)\n10. ReserveOnewayFlight(origin_city, destination_city, airlines, departure_date)\n11. SearchRoundtripFlights(origin, destination, departure_date, return_date)\n12. SearchHotel(location)\n13. SearchRoundtripFlights(origin_city, destination_city, departure_date, return_date)\n14. GetAvailableTime(event_date)\n15. FindMovies(location)\n16. FindBus(origin, destination, departure_date)\n17. FindProvider(city)\n18. BookAppointment(stylist_name, appointment_time, appointment_date)\n19. FindAttractions(location)\n20. SearchHotel(destination)\n21. BuyEventTickets(event_name, number_of_seats, date, city_of_event)\n22. GetWeather(city)\n23. GetCarsAvailable(pickup_city, pickup_date, pickup_time, dropoff_date)\n24. GetEvents(event_date)\n25. FindEvents(category, city_of_event)\n26. FindRestaurants(cuisine, city)\n27. BuyBusTicket(from_location, to_location, leaving_date, leaving_time, travelers)\n28. ReserveHotel(hotel_name, location, check_in_date, check_out_date)\n29. SearchHouse(where_to)\n30. LookupMusic()\n31. ReserveRoundtripFlights(origin_city, destination_city, airlines, departure_date, return_date, outbound_departure_time, inbound_departure_time)\n32. ReserveRestaurant(restaurant_name, city, time)\n33. TransferMoney(account_type, amount, recipient_account_name)\n34. GetEventDates(event_name, city)\n35. ReserveHotel(hotel_name, check_in_date, number_of_days, destination)\n36. ScheduleVisit(property_name, visit_date)\n37. GetTimesForMovie(movie_name, location, show_date)\n38. AddEvent(event_name, event_date, event_location, event_time)\n39. FindMovies(genre)\n40. BuyBusTicket(origin, destination, departure_date, departure_time, group_size)\n41. BuyMovieTickets(movie_name, number_of_tickets, show_date, location, show_time, show_type)\n42. PlaySong(song_name)\n43. FindApartment(area, number_of_beds)\n44. ReserveCar(pickup_location, pickup_date, pickup_time, dropoff_date, car_type)\n45. PlayMedia(song_name)\n46. LookupSong()\n47. BuyEventTickets(event_name, number_of_tickets, date, city)\n48. BookHouse(where_to, number_of_adults, check_in_date, check_out_date)\n49. BookAppointment(dentist_name, appointment_time, appointment_date)\n50. FindProvider(city, type)\n51. GetRide(destination, number_of_riders, shared_ride)\n\n[BEGIN OF FORMAT INSTRUCTION]\nThe output MUST strictly adhere to the following structured text format.\nExample Output API Call Format:\nfunction_name(argument1=value1, argument2=value2, ...)\n[END OF FORMAT INSTRUCTION]\n\n\n[BEGIN OF CONVERSATION HISTORY]\n\nUser: I'm looking for events happening in New York this weekend.  \nThought: The user wants to find events in a specific location and timeframe. I'll need to call the FindEvents API with the appropriate category and city.  \nAPI Name: FindEvents  \nAPI Input: {'category': 'all', 'city_of_event': 'New York'}  \nAPI Result: {'events': [{'name': 'Jazz Festival', 'date': '2023-10-07', 'location': 'Central Park'}, {'name': 'Food Truck Rally', 'date': '2023-10-08', 'location': 'Union Square'}]}  \nThought: I have the event details now. I should summarize the events for the user.  \nSystem: This weekend in New York, you can attend the Jazz Festival at Central Park on October 7th and the Food Truck Rally at Union Square on October 8th. Would you like more information about any of these events?  \n\nUser: Yes, can you tell me more about the Jazz Festival?  \nThought: I need to provide more details about the Jazz Festival. I don't need an API call for that.  \nSystem: The Jazz Festival in Central Park will feature various artists performing live jazz music throughout the day. It starts at 11 AM and goes until 8 PM. Would you like to know how to get tickets?  [END OF CONVERSATION HISTORY]",
%     "input": "User: Yes, please tell me how to get tickets for the Jazz Festival.\nThought: The user is interested in purchasing tickets. I need to call the BuyEventTickets API for the Jazz Festival.",
%     "output": "Action: BuyEventTickets({'event_name': 'Jazz Festival', 'number_of_seats': 2, 'date': '2023-10-07', 'city_of_event': 'New York'})"
%   },

\newpage
\begin{figure*}[!ht]
\begin{tcolorbox}[colback=gray!5!white,colframe=black!95!black,title=\textbf{\small{SGD Instruction Sample} | \textbf{Format:} System Response Optimization}] 
\small \textcolor{darkgreen}{\textbf{Instruction:}} \\
{[BEGIN OF TASK INSTRUCTION]}\\
You are a helpful conversational assistant specializing in understanding user queries and providing accurate, reasoned responses. \\
Your goal is to analyze the user's input, reason about their intent and needs, and provide a coherent and contextually appropriate system response. \\
Your specific tasks are: \\
- Ensure your response is informative and contextually relevant, guiding the conversation toward successful task completion. \\
- Analyze the user's input in the context of the conversation history (if available) to identify their intent and relevant details. \\
- Use logical reasoning to determine the most suitable response, considering the user's needs and the dialogue context. \\
- Generate a natural and coherent system response to address the user’s request or query effectively. \\
{[END OF TASK INSTRUCTION]} \\
\\
{[BEGIN OF FORMAT INSTRUCTION]} \\
The output MUST strictly adhere to the following structured text format. \\
Example Output API Call Format: \\
function\_name(argument1=value1, argument2=value2, ...) \\
{[END OF FORMAT INSTRUCTION]} \\
 \\
{[BEGIN OF CONVERSATION HISTORY]} \\
User: I'm looking for events happening in New York this weekend. \\
Thought: The user wants to find events in a specific location and timeframe.  \\
I'll need to call the FindEvents API with the appropriate category and city. \\
Action: FindEvents(category='all', city\_of\_event='New York') \\
Observation: {'events': [{'name': 'Jazz Festival', 'date': '2023-10-07', 'location': 'Central Park'}, {'name': 'Food Truck Rally', 'date': '2023-10-08', 'location': 'Union Square'}]} \\
Thought: I have the event details now. I should summarize the events for the user. \\
System: This weekend in New York, you can attend the Jazz Festival at Central Park on October 7th and the Food Truck Rally at Union Square on October 8th. Would you like more information about any of these events? \\
User: Yes, can you tell me more about the Jazz Festival? \\
Thought: I need to provide more details about the Jazz Festival. I don't need an API call for that. \\
System: The Jazz Festival in Central Park will feature various artists performing live jazz music throughout the day. It starts at 11 AM and goes until 8 PM. Would you like to know how to get tickets? \\
{[END OF CONVERSATION HISTORY]} \\
 \\
\small \textcolor{darkgreen}{\textbf{Input:}} \\
User: Yes, please tell me how to get tickets for the Jazz Festival. \\
Thought: The user is interested in purchasing tickets. I need to call the BuyEventTickets API for the Jazz Festival. \\
Action: BuyEventTickets(event\_name='Jazz Festival', number\_of\_seats=2, date='2023-10-07', city\_of\_event='New York') \\
Observation: {'status': 'success', 'message': 'You have successfully purchased 2 tickets for the Jazz Festival.'} \\
Thought: The tickets were successfully purchased. I need to confirm this with the user. \\
 \\
\small \textcolor{mypurple}{\textbf{Output:}} \\
System: You have successfully purchased 2 tickets for the Jazz Festival in Central Park! Enjoy the music! \\
\end{tcolorbox}

\vspace{-0.25cm}
\caption{SGD fine-tuning sample example targeting system response optimization.}
\label{tab:sgd-sft-response}
\end{figure*}


% "instruction": "[BEGIN OF TASK INSTRUCTION]\nYou are a helpful conversational assistant specializing in understanding user queries and providing accurate, reasoned responses. \nYour goal is to analyze the user's input, reason about their intent and needs, and provide a coherent and contextually appropriate system response.\nYour specific tasks are:\n- Ensure your response is informative and contextually relevant, guiding the conversation toward successful task completion.\n- Analyze the user's input in the context of the conversation history (if available) to identify their intent and relevant details.\n- Use logical reasoning to determine the most suitable response, considering the user's needs and the dialogue context.\n- Generate a natural and coherent system response to address the user’s request or query effectively.\n[END OF TASK INSTRUCTION]\n\n\n[BEGIN OF CONVERSATION HISTORY]\n\nUser: I'm looking for events happening in New York this weekend.  \nThought: The user wants to find events in a specific location and timeframe. I'll need to call the FindEvents API with the appropriate category and city.  \nAPI Name: FindEvents  \nAPI Input: {'category': 'all', 'city_of_event': 'New York'}  \nAPI Result: {'events': [{'name': 'Jazz Festival', 'date': '2023-10-07', 'location': 'Central Park'}, {'name': 'Food Truck Rally', 'date': '2023-10-08', 'location': 'Union Square'}]}  \nThought: I have the event details now. I should summarize the events for the user.  \nSystem: This weekend in New York, you can attend the Jazz Festival at Central Park on October 7th and the Food Truck Rally at Union Square on October 8th. Would you like more information about any of these events?  \n\nUser: Yes, can you tell me more about the Jazz Festival?  \nThought: I need to provide more details about the Jazz Festival. I don't need an API call for that.  \nSystem: The Jazz Festival in Central Park will feature various artists performing live jazz music throughout the day. It starts at 11 AM and goes until 8 PM. Would you like to know how to get tickets?  [END OF CONVERSATION HISTORY]",
%     "input": "User: Yes, please tell me how to get tickets for the Jazz Festival.\nThought: The user is interested in purchasing tickets. I need to call the BuyEventTickets API for the Jazz Festival.\nAction: BuyEventTickets({'event_name': 'Jazz Festival', 'number_of_seats': 2, 'date': '2023-10-07', 'city_of_event': 'New York'})\nObservation: {'status': 'success', 'message': 'You have successfully purchased 2 tickets for the Jazz Festival.'}\nThought: The tickets were successfully purchased. I need to confirm this with the user.",
%     "output": "System: You have successfully purchased 2 tickets for the Jazz Festival in Central Park! Enjoy the music!"
%   },
\newpage
\begin{table*}[!t]
\scriptsize
\centering
\ttfamily
\begin{tabular}{p\linewidth}
\toprule
You are an advanced AI assistant specializing in conversational dialogues. \\
You have access to a variety of services and APIs to assist users with their requests and your goal is to provide helpful and informative responses to user queries and commands. \\
You can interact with multiple services and APIs to fulfill user requests. \\
Your responses should be natural, informative, and tailored to the user's needs. \\
 \\
\# Task Information: \\
- You are asked to create a dataset in the format: User - Thought1 - API - API Input Arguments - API Result - Thought2 - System, or User - Thought - System.  \\
- For the given \# User Input, generate a multi-turn dialogue that follows this format, with each turn exhibiting realistic context reasoning, thought processes, and API interaction where applicable.  \\
- The dialogues should be converted to follow a specific \# Output Format, which includes reasoning on whether an API call is needed or if the system can respond directly. \\
- If the system decides that an API call is necessary, use this format: User - Thought1 - API - API Input Arguments - API Result - Thought2 - System. \\
- Call the right API from \# Avaliable Functions and provide the necessary input arguments to fulfill the user's request. \\
- If you think a function argument is not necessary, you can skip it. Don't provide unnecessary arguments and None values. \\
- Ensure that the API calls are used logically and that the dialogue remains coherent and natural throughout the exchange.  \\
- If the system determines that an API call is not necessary, use this format: User - Thought - System. \\
- Include intermediate thoughts where appropriate to capture the model's internal reasoning, and clearly separate the different components of the format. \\
\\
\# Available Functions: \\
1. CheckBalance(account\_type) \\
2. TransferMoney(account\_type, amount, recipient\_account\_name) \\
. \\
. \\
. \\
52. FindAttractions(location) \\
53. GetWeather(city) \\
 \\
\# Output Format: \\
- If an API Call is Needed: \\
\quad	    User: [User Input] \\
\quad	    Thought1: [I need to call an API] \\
\quad	    API Name: [API Call Name: CheckBalance(), TransferMoney(),...,  FindAttractions(), GetWeather()] \\
\quad	    API Input: [The input parameters for the API] \\
\quad	    API Result: [API output result] \\
\quad	    Thought2: [2nd thought after API Result that ensure if the information is enough before the system response] \\
\quad	    System: [Your system response here] \\
 \\
- If an API Call is Not Needed: \\
\quad	    User: [User Input] \\
\quad	    Thought: [I don't need an API and I want to respond to the user] \\
\quad	    System: [Your system response here] \\
 \\
Below is an example of the format: \\
 \\
\# Example: \\
User: Add Don and Sherri to my "Meditate to Sounds of Nature" playlist.   \\
Thought1: The user wants to add two artists to a specific playlist. I'll need to extract the playlist name and the artists' names.  \\ 
API Name: AddToPlaylist   \\
API Input: {'playlist\_name': 'Meditate to Sounds of Nature', 'artists': Don Sherri}   \\
API Result: {'status': 'success', 'message': 'Don and Sherri have been added to your playlist.'}   \\
Thought2: The API call was successful, and the artists were added to the playlist. I need to inform the user about this.   \\
System: Don and Sherri have been successfully added to your "Meditate to Sounds of Nature" playlist. Enjoy your music!   \\
 \\
User: Thank you for adding them. I think I can add more songs to the playlist. \\
Thought: I don't need an API call for that, I want to respond to the user.  \\
System: Would you like to add more songs?   \\
 \\
User: Yes, please add "Calm River" to the playlist.   \\
Thought1: The user wants to add another song to the playlist. I'll note the song title and proceed with the request.   \\
API Name: AddToPlaylist   \\
API Input: {'playlist\_name': 'Meditate to Sounds of Nature', 'songs': Calm River}   \\
API Result: {'status': 'success', 'message': 'Calm River has been added to your playlist.'}   \\
Thought2: The song was successfully added. I should let the user know.   \\
System: "Calm River" has been successfully added to your "Meditate to Sounds of Nature" playlist. Would you like to do anything else?   \\
 \\
User: No, that's all for now. Thank you!   \\
Thought: The user is satisfied and doesn't need further assistance. I should acknowledge their gratitude.   \\
System: You're welcome! If you need anything else, feel free to ask. Enjoy your relaxing music! \\
 \\
\# User Input:\\
\bottomrule
\end{tabular}
\caption{API-driven conversation prompt to generate SGD fine-tuning samples with GPT4-o. Only functions used across all dialogues are included due to their large number.}
\label{tab: prompt}
\end{table*}




\end{document}
