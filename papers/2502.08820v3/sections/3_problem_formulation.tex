% \begin{table*}[!h]
% \centering
% \resizebox{\textwidth}{!}{
% \begin{tabular}{lrrrrr}
% \toprule
% \textbf{Data Mixture} & \textbf{Data Type} & \textbf{Data Name} & \textbf{\# of Data Instances} &  \textbf{\# of Total Tokens} & \textbf{Avg. Tokens Per Instance} \\ \midrule
% \multirow{2}{*}{\textbf{Prior Agent Works}}& Information Seeking & FireAct &   $2,063$ &    $542,176$ &   $262.81$ \\
%  &  Information Seeking & AgentInstruct &    $1,866$ &   $2,517,785$ &  $1349.30$ \\ \midrule

% \multirow{5}{*}{\textbf{\dataname} \textbf{AgentFlan}}
% & Information Seeking & HotpotQA &      $1,664$ &   $2,472,227$ &  $1485.71$ \\
% & Software Packages (Tool) & MATH &      $1,732$ &   $1,719,467$ &   $992.76$ \\
% & Software Packages (Tool) & APPS Code &        $647$ &   $1,235,472$ &  $1909.54$ \\
% & External Memory & WikiTableQuestion  &      $1,065$ &   $1,316,246$ &  $1235.91$ \\
% & Robot Planning & ALFWorld  &      $2,031$ &   $3,838,269$ &  $1889.84$ \\ \midrule

% \multirow{4}{*}{\textbf{General Conversation}}
% & Single-Turn Reasoning & OpenOrca  &     $50,000$ &  $14,034,152$ &   $280.68$ \\
% & Multi-Turn Conversations & ShareGPT &     $10,000$ &  $17,933,861$ &  $1793.39$ \\
% & Multi-Turn Conversations & ShareGPT  &      $4,583$ &  $18,195,878$ &  $3970.30$ \\
% & Multi-turn Reasoning & CapyBara &  $4,647$ &   $4,982,435$ &  $1072.18$ \\ \midrule

% \multirow{4}{*}{\textbf{TOD Conversation}}
% & Multi-Turn TOD Conversations & MultiWOZ 2.4     &     $8,400$   &  $TODO$          &  $TODO$ \\
% & Multi-Turn TOD Conversations & SGD              &     $16,168$  &  $TODO$          &  $TODO$ \\
% & Multi-Turn TOD Conversations & ABCD             &      $8,034$  &  $TODO$          &  $TODO$ \\
% & Multi-turn TOD Conversations & STAR V2          &      $6,652$  &  $TODO$          &  $TODO$ \\
% & Multi-turn TOD Conversations & MSR-E2E          &      $6,993$  &  $TODO$          &  $TODO$ \\ 
% & Multi-turn TOD Conversations & KVRET            &      $7,884$  &  $TODO$          &  $TODO$ \\ 
% & Single-turn TOD Conversations& ST               &      $29,576$ &  $TODO$          &  $TODO$ \\ 
% & Multi-turn TOD Conversations & Task-Master      &      $23,346$ &  $TODO$          &  $TODO$ \\ \cmidrule{2-6}
% \multicolumn{3}{r}{\textbf{Total}} & $107,053$ & $TODO$ & $TODO$ \\ \bottomrule
% \end{tabular}
% }
% \caption{Statistics of our training mixture and comparison with prior work.}
% \label{tab: data_training_mixture_stats}
% \end{table*}

\begin{table*}[!t]
\centering
\resizebox{\textwidth}{!}{
\begin{tabular}{lrrrrrr}
\toprule
\textbf{Data Domain} & \textbf{Data Type} & \textbf{Data Name} & \textbf{Data Format} & \textbf{\# of Data Samples} &  \textbf{\# of Total Tokens} & \textbf{Avg. Tokens Per Sample} \\ \midrule

\multirow{1}{*}{\textbf{TOD}}&  Single-Turn  & SNIPS         & State Tracking &    $13,028$   &   $12,278,780$ &  $942.49$  \\ \cmidrule{2-7}
\multirow{2}{*}{\textbf{LA}}& Single-Turn & Hammer  & API Call          &     $13,819$  &  $10,199,147$  &   $738.05$ \\
                                  & Multi-Turn  & ToolAce & API Call          &     $202,500$ &  $129,001,612$ &  $637.04$ \\ \cmidrule{2-7}
\multirow{1}{*}{\textbf{CRA}}& Multi-Turn & SGD & ReAct API Call &    $82,236$   &   $59,704,782$ &  $726.02$ \\ \cmidrule{2-7}
                                 
         
\multicolumn{4}{r}{\textbf{Total}} & $311,583$ &  $211,184,321$ &  $760.90$ \\ \bottomrule
\end{tabular}
}
\caption{\textbf{CoALM-IT Dataset Details.} Statistical details of our proposed CoALM-IT dataset showcasing the training mixtures. Generated \textbf{CRA} denotes the Conversational ReAct API dataset.}
\vspace{-5mm}
\label{tab: data_training_mixture_stats}
\end{table*}


% \begin{table*}[!ht]
% \centering
% \resizebox{\textwidth}{!}{
% \begin{tabular}{lrrrrrr}
% \toprule
% \textbf{Data Domain} & \textbf{Data Type} & \textbf{Data Name} & \textbf{Data Format} & \textbf{\# of Data Instances} &  \textbf{\# of Total Tokens} & \textbf{Avg. Tokens Per Instance} \\ \midrule

% \multirow{2}{*}{\textbf{TOD}}&  Single-Turn  & SNIPS         & State Tracking &    $13,028$ &   $12,278,780$ &  $942.49$  \\
%                              &  Single-Turn  & SLURPS        & State Tracking & $11,514$    & $11,359,895$   & $986.62$   \\ \cmidrule{2-7}
                             
% \multirow{2}{*}{\textbf{API Call}}& Single-Turn & Hammer  & API Call    &     $13,819$  &  $10,199,147$ &   $738.05$ \\
%                                         & Multi-Turn  & ToolAce & API Call   &     $202,500$ &  $129,001,612$ &  $637.04$ \\ \cmidrule{2-7}

% \multirow{2}{*}{\textbf{Dialogue + API}}& Multi-Turn   & MultiWoz 2.4  & ReAct + API Call &    $65,563$ &    $71,850,667$ &   $1095.90$ \\
%                                  & Multi-Turn   & SGD           & ReAct + API Call &    $82,236$ &   $59,704,782$ &  $726.02$ \\ \cmidrule{2-7}
                                 
         
% \multicolumn{4}{r}{\textbf{Total}} & $401,744$ &  $299,344,457$ &  $786.34$ \\ \bottomrule
% \end{tabular}
% }
% \caption{TOD Conversation Instruction Tuning Datasets. Statistics of our training mixture and comparison with prior work.}
% \vspace{-5mm}
% \label{tab: data_training_mixture_stats}
% \end{table*}




\section{Preliminaries}
A Conversational Agent, at its core, must understand user intents, maintain context across multi-turn interactions, and respond contextually. Beyond traditional TOD tasks, modern conversational agents are also expected to exhibit agentic abilities, like tool calling, planning, and decision making, to fulfill complex user requests. An effective conversational agent integrates these capabilities as skills, ensuring natural and relevant interactions while efficiently completing the user’s objectives. The detailed task formulations for TOD systems and LA are provided in Appendix \ref{app: problem-formulation}.

\vspace{-1ex}

\subsection{Why we need both TOD and LA Capabilities?}

Multi-turn interactions are critical for refining ambiguous user requests. For example, when a user says "Find me a hotel", the system can ask clarifying questions to clarify the user's intention (e.g., location, price range) instead of returning generic results. This ensures meaningful and task-specific conversations. That said, traditional TOD systems excel at handling these multi-turn interactions but over a small set of APIs (e.g., query\_restaurant, book\_hotel) \cite{ye-etal-2022-multiwoz}. By training on structured dialogue flows, they achieve high task success rates in controlled scenarios (e.g., standard booking or reservation tasks) without requiring complex function-calling capabilities. However, these systems struggle to adapt to new services (e.g., airline, retail) without expensive re-training. 

In real-world settings, users may need to access a wide variety of APIs (e.g., search\_direct\_flight, get\_product\_details). This is where LA shines: they leverage LLMs and can rapidly learn how to use unseen new tools since they are already proficient with determining when to invoke an API and decide which API to use from a diverse set of available functions. Without these skills, agents fail to fulfill complex user goals, limiting their utility. 

Together, these skills form the backbone of a unified conversational agents, enabling them to transition from being passive responders to proactive collaborators capable of managing intricate tasks and sustaining user engagement. 


%\vspace{-1ex}
\subsection{Can TOD Systems Solve Function Calling Tasks?}
The benchmark results demonstrate the limitations of TOD systems in function calling scenarios. Despite achieving top performance on MultiWOZ metrics as in Table \ref{tab:tod-results}, these systems show significantly lower accuracy on both API-Bank (Table \ref{tab:apibank}) and BFCL (Table \ref{tab:bfcl}) benchmarks. This performance gap reveals that TOD systems' traditional strengths in dialogue management do not translate well to handling diverse, unseen, and complex API calls.


\subsection{Can LAs Handle Task-oriented Multi-turn Conversations?}
Conversely, agentic models like ToolAce \cite{Liu2024ToolACE}, Hammer \cite{Lin2024Hammer}, and Granite \cite{abdelaziz-etal-2024-granite} while achieving accurate results on API-Bank and BFCL V3, perform poorly on MultiWOZ's task completion metrics. These results highlight a critical weakness: while they deliver strong performance on function execution tasks, they fall short in maintaining coherent multi-turn conversations and properly fulfilling user intents. Their specialized optimization for tool calling impairs their dialogue management abilities, indicating that current LAs need more balanced capabilities to handle task-oriented conversations more effectively.

