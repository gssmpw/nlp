\begin{figure*}[!h]
\begin{tcolorbox}[colback=gray!5!white,colframe=black!95!black,title=\textbf{\small{MultiWOZ Instruction Sample} | \textbf{Format:} Action Optimization}] 
\small \textcolor{darkgreen}{\textbf{Instruction:}} \\
You are an advanced Conversational AI Agent responsible for understanding user queries and determining the appropriate action by selecting the correct API function. \\
For each query, you need to: \\
    1. Thought: Analyze the user's request to understand the intention and extract relevant information such as names, entities, and specific actions. \\
    2. Act: Choose the most appropriate API function from the provided set of available functions and determine the correct input parameters based on the user's query. \\ 
\\
\# Task Information: \\
- Given a Conversation History between a user and a system (if available), your objective is to produce the next system response using interleaving Thought, Action, and Observation steps continue the dialogue smoothly and effectively. \\
- Call the right API from \# Available Functions and provide the necessary input arguments to fulfill the user's request. \\
- If you think a function argument is not necessary, you can skip it. Don't provide unnecessary arguments and None values. \\
- Ensure that the API calls are used logically and that the dialogue remains coherent and natural throughout the exchange.  \\
 \\
\# Available Functions: \\
1. query\_attractions(area, name, type) \\
2. query\_hotels(area, internet, name, parking, pricerange, stars, type) \\
3. book\_hotel(day, people, stay, name) \\
4. query\_restaurants(area, food, name, pricerange) \\
5. book\_restaurant(day, people, time) \\
6. book\_taxi(arrive\_by, day, departure, destination, leave\_at) \\
 \\
\# Output Format \\
Output your response in the 'output' field, considering the following guidelines: \\
1. Address the user's current query or request accurately. \\
2. Use relevant information from the conversation history. \\
3. If the user's input requires an API call or external service, provide a clear rationale for this decision in your response. \\
4. If the Observation of an action is provided then just generate response. \\
5. Ensure your response is informative and contextually relevant, guiding the conversation toward successful task completion. \\
 \\
\# Conversation History: \\
User: Is there any expensive seafood restaurant in town? \\
Thought: I need to call an API to find a restaurant that fits the criteria of being expensive and serving seafood. \\
API Name: query\_restaurants   \\
API Input: {'area': 'Cambridge', 'food': 'seafood', 'pricerange': 'expensive'}  \\
API Result: [{'name': 'loch fyne', 'address': 'The Little Rose 37 Trumpington Street', 'phone': '01223362433', 'pricerange': 'expensive'}] \\
Thought: The API returned one suitable restaurant, Loch Fyne. I can inform the user about it and ask if they would like to make a reservation.  \\
System: Yes, there is an upscale seafood restaurant called Loch Fyne which is centrally located. Would you like me to make a reservation for you?" \\
 \\
\small \textcolor{darkgreen}{\textbf{Input:}} \\
User: Can I get a booking for 8 people at 19:45 on Friday? Send me the reference number. \\
 \\
\small \textcolor{mypurple}{\textbf{Output:}} \\
Thought: I need to call an API to book a table for 8 people at the specified time and day. \\
Action: book\_restaurant({'day': 'friday', 'people': 8, 'time': '19:45'})\\
\end{tcolorbox}

\vspace{-0.25cm}
\caption{MultiWOZ fine-tuning sample example targeting function optimization.}
\label{tab:multiwoz-sft-action}
\end{figure*}
