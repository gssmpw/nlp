\section{Results}
We first report the time per optimization step in Section \ref{results:optimization_step_time}.
Then, we present the results comparing FlexDeMo's performance in terms of validation loss in Section \ref{results:1gbpsvs100mbps}, before integrating both measures into efficiency (i.e., validation loss by time) in Section \ref{results:validation_per_minute_of_training_time}.
Last, we compare FlexDeMo against the DeMo baseline in Section~\ref{results:hybrid_sharding}.

\begin{table*}[htb]
    \centering
    \caption{Average time per step in seconds, for inter-node connections with different bandwidths. Top1--Top32 denote different variants of FlexDeMo, while Deto-full and AdamW are the baselines.
    \textbf{Bold} denotes the best for given bandwidth. \underline{Underline} denotes overall best.}
    \label{tab:average_time_per_step_in_seconds}
    \begin{tabular}{lcccccc}
        \toprule
        \textbf{Bandwidth} & \textbf{Top1} & \textbf{Top8} & \textbf{Top16} &\textbf{Top32} & \textbf{Deto-full} & \textbf{AdamW} \\
        \midrule
        100 Mbps & \textbf{0.956} & 4.676 & 9.175 & 21.29 & 12.98 & 11.35 \\
        500 Mbps & \textbf{0.241} & 1.016 & 2.147 & 3.222 & 2.829 & 2.304 \\
        1000 Mbps & \underline{\textbf{0.172}} & 0.585 & 1.112 & 2.155 & 1.505 & 1.276 \\
        \midrule
    \end{tabular}
\end{table*}
\subsection{Speed: Average Time per Optimization Step}\label{results:optimization_step_time}
We analyze the time per step for 100, 500 and 1000Mbps network connections. We report the results of all experiment configurations in Table \ref{tab:average_time_per_step_in_seconds} and visualized in Figure~\ref{fig:time-vs-bandwidth}, where it is clear how each FlexDeMo configuration versus AdamW and Full gradient syncronization scales as the bandwidth decreases. This is especially significant at bandwidths lower than 500Mbps, which is a level of bandwidths comparable to those available in HPCs, in practice.
As expected we see that \texttt{Top1}, communicating the least amount of data delivers the fastest step-time of all configurations and, further, is faster as the bandwidth increases. Furthermore, we observe that the step-time increases with TopK. AdamW is consequently faster than full synchronization.



\begin{figure}[h]
    \centering
    \includegraphics[width=0.6\textwidth]{Figures/time-vs-bw_svg-raw.pdf}
    \caption{Average time per step when training a T5-Small model for FSDP (adamw), full gradient synchronisation (deto-full), and FlexDeMo with different TopK values, and inter-node connections with different bandwidths.}\label{fig:time-vs-bandwidth}
\end{figure}

\subsection{Efficacy: Validation Loss per Epoch}\label{results:1gbpsvs100mbps}
We report the impact on training and validation losses employing FlexDeMo. 

Note that the bandwidth limit does not affect the training dynamics with respect to time-insensitive variables (i.e., number of optimization steps, number of epochs). However, for completeness, we report the validation loss achieved in both $1$Gbps and $100$Mbps conditions.
Note that the training loss shown in the figures is generally lower than validation loss in the Figure, as we report average training loss over an entire epoch and the model was still improving throughout the epoch.

In both conditions AdamW displays the best validation loss, closely followed by FlexDeMo in its Top8, Top1, and, Top16 configurations, respectively (see Figure~\ref{fig:100mbpsvs1gbps}).
The only difference between the $1$Gbps and $100$Mbps conditions is that Top8 exhibits lower training loss for the $100$Mbps connection, which we attribute to random effects. Full gradient synchronization performs worst with both slow learning and low performance.

\begin{figure*}[htbp]
    \centering
    \begin{subfigure}[b]{0.49\textwidth}
        \centering
        \includegraphics[width=\textwidth]{Figures/val-train-loss-bw1000_svg-raw.pdf}
        \caption{1Gbps}
        \label{fig:100mbpsvs1gbps:1gbps}
    \end{subfigure}
    \hfill
    \begin{subfigure}[b]{0.49\textwidth}
        \centering
        \includegraphics[width=\textwidth]{Figures/val-train-loss-bw100_svg-raw.pdf}
        \caption{100Mbps}
        \label{fig:100mbpsvs1gbps:100mbs}
    \end{subfigure}
    \caption{ Training (dashed) and validation (solid) loss curves for FSDP (adamw), full gradient synchronisation (deto-full), and FlexDeMo comparing the effect of TopK 1, 8, 16, and 32 using a fixed chunk-size of 128 on a T5-small model over a 1Gbps and 100Mbps network connection. Each point is the average loss per epoch. Note that validation loss is measured at the end of each epoch, while training loss is the average over steps.}\label{fig:100mbpsvs1gbps}
\end{figure*}


\subsection{Efficiency: Validation Loss per Minute of Training Time}\label{results:validation_per_minute_of_training_time}
Here, we report validation loss by time to see how the gains in speed translate to efficiency.
Figure \ref{fig:validation_per_minute} shows the effect of employing FlexDeMo on the validation loss over minutes of training time for 1Gbps and 100Mbps connections in Figures \ref{fig:100mbpsvs1gbps:training:1gbs} and \ref{fig:100mbpsvs1gbps:training:100mbps}, respectively. AdamW yields the best performance in both scenarios, while FlexDeMo demonstrates to be more efficient, consequently a smaller loss in performance.  At 1Gbps AdamW is marganilly delayed, achieving a validation loss of $0.074506$ after $19.998$ minutes, comparable with DeMo's $0.074837$ at $12.167$ minutes ($39.16\%$ faster). At 100Mbps, DeMo reaches a validation loss of $0.074656$ in $69.302$ minutes, while AdamW takes significantly longer with $438.665$ minutes ($633.0\%$ slower) to achieve a comparable loss of $0.074508$. Full gradient synchronization yields significant sub-optimal performance exhibiting higher loss. Top1 demonstrates to be the best choice, with decreasing performance as K increases for both communication-speeds. However, we are training on a small dataset where the larger TopKs would promote overfitting, hence worse resulting performance.


\begin{figure*}[ht]
    \centering
    \begin{subfigure}[b]{0.49\textwidth}
        \centering
        \includegraphics[width=\textwidth]{Figures/val-train-loss-wall-time-bw1000_svg-raw.pdf}
        \caption{1Gbps}
        \label{fig:100mbpsvs1gbps:training:1gbs}
    \end{subfigure}
    \hfill
    \begin{subfigure}[b]{0.49\textwidth}
        \centering
        \includegraphics[width=\textwidth]{Figures/val-train-loss-wall-time-bw100_svg-raw.pdf}
        \caption{100Mbps}
        \label{fig:100mbpsvs1gbps:training:100mbps}
    \end{subfigure}
    \caption{Comparison of validation (solid) and training (dashed) loss per minute of training time, for FSDP (adamw), full gradient synchronisation (deto-full), and FlexDeMo, in a bandwidth limited setup (1 Gbps and 100 Mbps) with 2 nodes.  Each point is the average loss per epoch.}
    \label{fig:validation_per_minute}
\end{figure*}



\subsection{DeMo vs. FlexDeMo}\label{results:hybrid_sharding}
We investigate the impact of a hybrid sharding strategy in a local cluster between 2 nodes with 2 accelerators each. Experiments are conducted with T5-Small on a 500Mpbs inter-node connection bandwidth.
We run two different configurations.
In the first configuration, we shard the model to 1 accelerator at a time, affectively giving us 4 replication groups on the 2 nodes. This configuration corresponds to DeMo \cite{peng2024demodecoupledmomentumoptimization}. In the second configuration, we shard the model to 2 accelerators at a time, effectively giving us 2 replication groups, which are the 2 nodes. This configuration corresponds to FlexDeMo and its hybrid sharding strategy. The 2 accelerators within each node are NV-Linked, making the network connection the bottleneck of investigation. We report the results in Table \ref{tab:hybrid_sharding}. FlexDeMo is significantly more efficient in this setting, being $1.75$, $2.0$, $2.52$, and $2.54$ times faster for TopKs of $1$, $8$, $16$, and $32$, respectively.

\begin{table*}
    \centering
   \caption{Average time per step in seconds, width FSDP with 1 shard and FSDP employing the hybrid-sharding strategy accross 2 nodes on 500Mpbs network connection. \texttt{Deto-Full} denotes full gradient synchronization. \texttt{TopK} denotes the TopK hyperparameter. \textbf{Bold} denotes the best for a given optimizer. \underline{Underline} denotes overall best.}\label{tab:hybrid_sharding}
    \begin{tabular}{cccccccc}
    \toprule
         & \textbf{Shards} & \textbf{Deto-Full} & \textbf{Top1} & \textbf{Top8} & \textbf{Top16} & \textbf{Top32} \\
        \midrule
        DeMo & 1 & 3.06 & 0.42 & 2.06 & 5.16 & 9.82 \\
        \midrule
        FlexDeMo & 2 & \textbf{2.64} & \textbf{\underline{0.24}} & \textbf{1.03} & \textbf{2.04} & \textbf{3.86} \\
        \midrule
    \end{tabular}

\end{table*}
