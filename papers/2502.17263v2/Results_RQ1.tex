\subsection{RQ1: How Do UX Professionals Raise Usability Issues Differently Than Other Contributors?}

\subsubsection{Usability Dimensions}

\begin{figure}[t]
  \centering
    \includegraphics[width=\columnwidth]{images/CompareotherandUX2.pdf}
    \caption{Percentage of issues posted by UX professionals and other contributors touching on different usability dimensions categorized by Nielsen heuristics.}
    \Description{A bar chart where the x-axis is the percentage and the y-axis includes nine usability heuristics. For UX professionals, 23.66 percent of the posted issues were related to heuristic \#8 Design, 19.35 percent related to heuristic \#7 Flexibility, 18.28 percent related to heuristic \#4 Consistency, 11.83 percent to \#5 Error Prevention, 9.68 percent to \#3 User control, 8.60 percent to \#1 System status, 3.23 percent to \#6, 2.15 percent to \#9, and 2.15 percent to \#10. For other contributors, 24.51 percent of the posted issues were related to heuristic \#8 Design, 40.10 percent related to heuristic \#7 Flexibility, 8.82 percent related to heuristic \#4 Consistency, 6.86 percent to \#5 Error Prevention, 4.90 percent to \#1 System status, 0.98 percent to \#6, and 7.84 percent to \#9.}
    \label{fig:frequency-Nielsen-heuristics}
\end{figure}

The usability issues posted by UX professionals reported a diverse range of usability concerns; see Figure~\ref{fig:frequency-Nielsen-heuristics}. We found that UX professionals considered most of the usability dimensions, covering a wider and more balanced range than other contributors. Similar to other contributors, UX professionals paid attention to \textit{\#7: Flexibility and efficiency of use} and \textit{\#8: Aesthetic and minimalist design}. However, their primary focus was also on \textit{\#4: Consistency and standards} and \textit{\#5: Error prevention}, while issues related to \textit{\#9: Help users recognize, diagnose, and recover from errors} were comparatively less frequent.

\begin{figure}[t]
  \centering
  \begin{minipage}
    [b]{\columnwidth}
    \includegraphics[width=\columnwidth]{images/SentimentDistbution2.pdf}
    \subcaption{Sentiments}
    \Description{A bar chart where the x-axis is the percentage and the y-axis includes three sentiments: neutral, negative, and positive. For UX professionals, 85.87 percent of the posted issues were neutral, 7.61 percent were negative, and 6.52 percent were positive. For other contributors, 57.84 percent of the posted issues were neutral, 19.61 percent were negative, and 22.55 percent were positive.}
    \label{fig:frequency-sentiment-compare}
  \end{minipage}
  \hfill
  \begin{minipage}[b]{\columnwidth}
    \includegraphics[width=\columnwidth]{images/ToneDistribution2.pdf}
    \subcaption{Tones} 
    \Description{A bar chart where the x-axis is the percentage and the y-axis includes five tones: sad, polite, frustrated, excited, and no tone. For UX professionals, 55.91 percent of the posted issues had no tone, 17.20 percent were sad, 13.98 percent were polite, 7.53 percent were frustrated, and 5.38 percent were excited. For other contributors, 28.43 percent of the posted issues had no tone, 41.18 percent were sad, 13.73 percent were polite, 8.82 percent were frustrated, and 7.84 percent were excited.}
    \label{fig:frequency-tone-compare}
  \end{minipage}
   \caption{Percentage of issues posted by UX professionals and other contributors that included different sentiments and tones.}
    \label{fig:frequency-emotion}
\end{figure}

\subsubsection{Sentiment and Tones}
The UX professionals more frequently applied \textit{neutral} sentiment and \textit{no tone} in comparison to others (see Figures~\ref{fig:frequency-sentiment-compare} and~\ref{fig:frequency-tone-compare}). Notably, other contributors frequently used the \textit{sad} tone when posting usability issues. This means that issues posted by UX professionals were more factual than emotional. This aligns with our impression when reading those issue posts.

\begin{figure}[t]
  \centering
    \includegraphics[width=\columnwidth]{images/CompareArgumentdistribution.pdf}
    \caption{Percentage of issues posted by UX professionals and other contributors that included a claim and a premise.}
    \Description{A bar chart where the x-axis is the percentage and the y-axis includes three types: claim and premise, claim only, and no argument. For UX professionals, 93.55 percent of the posted issues had both claim and premise and 6.45 percent had only claim. For other contributors, 68.42 percent of the posted issues had both claim and premise, 28.95 percent had only claim, and 2.63 percent were not argumentative.}
    \label{fig:frequency-argument-structure}
\end{figure}

\subsubsection{Argument Structure}
Usability issues posted by UX professionals often had a solid premise to support their argument and they frequently adopted an argumentative structure of \textit{claim and premise}, as depicted in Figure~\ref{fig:frequency-argument-structure}. This is different than the other contributors, who more frequently reported issues without any premise.
