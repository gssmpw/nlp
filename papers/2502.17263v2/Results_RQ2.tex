\subsection{RQ2: How Do UX Professionals Follow Up on the Usability Issues They Posted?}

\subsubsection{Frequency of Follow-Ups}

Among the 93 usability issues posted by the four UX professionals, we found that they followed up on 31 (33.3\%) of them in the issue comments. In the remaining 62 issues that they did not follow up with, we found that most were resolved right after the issue post or at most after a few comments.

Focusing on UX professionals' general following-up behavior as commenters, some interesting observations appeared from our preliminary analysis. First, none of the UX professionals participated with other UX professionals; they only collaborated on the usability discussion threads that they instantiated. Additionally, all the \textit{claim-only} issues were not followed up by the UX professionals. Given the limited sample size of our dataset, these preliminary observations need to be further investigated in future work.

\subsubsection{UX Professionals' Purposes for Following up on Their Usability Issues}
Through the inductive coding process, we identified the following purposes of the four UX professionals when commenting in discussion threads about their usability issues; Figure~\ref{fig:frequency-purpose} shows the frequency of these purposes.

\begin{figure*}[t]
    \centering
    \includegraphics[width=\textwidth]{images/PurposeDistribution3.pdf}
    \caption{Frequency of UX professionals' purposes for following up on their usability issues}
    \Description{A bar chart where the x-axis is the frequency of posted comments and the y-axis includes the six purposes of following up. For "sharing an idea/suggestion/design", the frequency was 19; for "giving extra information/example", the frequency was 12; for "asking questions/help", the frequency was 5; for "planning", the frequency was 5; for "highlighting an ongoing problem", the frequency was 3; and for "expressing enthusiasm", the frequency was 3.}
    \label{fig:frequency-purpose}
\end{figure*}

\addvspace{4pt}
\textbf{Sharing an idea/suggestion/design:} These commenters offer insights, suggestions, personal experiences, opinions, or design recommendations related to the topic under discussion. They contribute creative or evaluative input. One instance of this purpose is from Atom's UX professional in issue \#1006:

\begin{quote}
    \textit{Awesome, my real secret was I was waiting for @benogle's thoughts to go forward :) Next matter to discuss regarding these tabs is... Separate tabs from UI \& Syntax Styles: I've been noticing in my discovery that many themes have dark tabs/syntax and a light theme around the web. Others may prefer lighter UI\/Tabs with a dark UI. One thought would be to make them separate from the UI and the Syntax. For myself I'd match them up `usually' with the syntax and not the UI. Anyone with strong thoughts on this?}
\end{quote}

As another example, one of VSCode's UX professionals commented in issue \#18132: 

\begin{quote}
    \textit{@sandy081 nice work. As far as the title for the default settings, I vote that we leave it out unless we have indications from users that it's needed (perhaps we could try it first without it). However, here is an idea for a dismissible header. IMAGE. Here was another read-only concept I had (but I think the darker background is more clear). Just showing this as another example. I think we could do without it, since you're right it might raise questions on other editors that are read only.}
\end{quote}

\addvspace{4pt}
\textbf{Giving extra information/example:} The participants of comments contributed additional details, examples, or context to enrich the discussion. Their aim was to provide supplementary information or examples for a better understanding of a discussed topic. Examples of this type of purpose are VSCode UX professionals in issue \#4331: ``\textit{This will also be relevant to \#4100}'' and the male UX professional in Jupyter Lab of \#6615:

\begin{quote}
    \textit{@ellisonbg the folder icon is rendering in the correct alignment on my JupyterLab. But I haven't updated anything since yesterday; maybe it's a regression? I'm on Chrome {Version 74.0.3729.169 (Official Build) (64-bit)} and the latest OS.}
\end{quote}

\addvspace{4pt}
\textbf{Asking questions/help:} The comment posters sought clarification or assistance by posing queries or requesting guidance related to the posted issues. They aimed to gather information, guidance, or solutions. For example, Atom's UX professionals in issue \#964: ``\textit{What will it take to make this happen? Are we talking styles to move the drawer or more than that?}'' Also, Jupyter Lab male UX professionals in issue \#7967:

\begin{quote}
    \textit{I had a couple of questions about the existing UI. Is anybody attached to the @ symbols? They are a bit repetitive, I think we can go forward without them, but if somebody has a good reason to leave them in I am unaware of please let me know. Some of the extensions in the existing UI don't have orgs or usernames (see screenshot). Is this something the extension developer is doing on purpose?}
\end{quote}

\addvspace{4pt}
\textbf{Planning:} The UX professionals also discussed in the comments future strategies, proposed plans, or outlined potential courses of action to address the issue or improve the situation. They engage in forward-thinking discussions. One example of this type is from VSCode's UX professional in issue \#9861: ``\textit{Feel free to move to August if it's not worth the risk fixing it quickly.}'' and ``\textit{@isidorn FYI - this is the task I'll work on once your change is in Master. I'll put this on the June Milestone.}''

\addvspace{4pt}
\textbf{Highlighting an ongoing problem:} In these comments, the UX professionals drew attention to persistent issues, emphasizing the need for resolution or further investigation. They aimed to underscore existing problems for collective awareness and action. As an example for this type of purpose, a VSCode UX professional commented in issue \#3682: ``\textit{[IMAGE] This is me web inspecting it so you can see the problem (which is otherwise harder to see).}''

\addvspace{4pt}
\textbf{Expressing enthusiasm:} The commenters sometimes exhibited excitement, positive feedback, or encouragement regarding the topic or solution being discussed. Their goal in the comment was to express support or appreciation. One example of this is from Atom's UX professionals in issue \#840: ``\textit{Look forward to updating to check this out.}''
