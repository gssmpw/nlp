We focused on analyzing the characteristics of the usability issues posted by UX professionals and understanding how they followed up on their usability issues on the ITSs. In this section, we first synthesize our findings related to the participation of UX professionals in the usability issue posting and discussions, and then we discuss the implications of these findings.

\subsection{Synthesizing the Primary Findings}

\subsubsection{Scarcity of UX Professionals' Engagement in OSS Usability Issue Discussions}
Our results yielded crucial evidence that UX professionals' involvement in the OSS issue discussions is very scarce; i.e., only four individuals were identified as UX professionals from the 224 usability issue posters. This limited presence raises concerns about the potential impact on UX quality and overall user satisfaction in OSS projects. We recognized that one possible reason behind their limited involvement in the issue tracking system is that they may conduct and discuss their work related to usability, such as user studies and UX design, outside of these systems. However, we argue that because the central development decisions, project management, and collaboration for OSS take place within ITSs, the limited participation of UX professionals on these platforms creates a potential disconnection between the design considerations that they offer and the actual development workflow. This disconnection is also problematic as it may lead to a lack of alignment between design goals, overall project objectives, and lack of user satisfaction. Thus, while we recognize that UX and usability-related tasks go beyond issue discussions, understanding the implications of limited contribution by UX professionals in these systems is essential for addressing challenges and fostering more effective collaboration in OSS development.

\subsubsection{Specific Issue Reporting and Follow-Up Style of UX Professionals}
A fascinating aspect of UX professionals' contributions was their distinct issue-reporting style and issue follow-up behavior. Usability issues posted by UX professionals covered a wider range of usability aspects than those posted by other contributors. UX professionals also almost always backed up their usability-related arguments with premises and tended to approach usability issues using their expertise, without leveraging emotional tone or sentiment. Moreover, when following up on the issues they posted, the UX professionals often made efforts to ensure that their points were clearly understood and their concerns were properly addressed. These findings all indicated that the expertise of the UX professionals indeed affected the way they posted usability issues. The issues and comments posted by the UX professionals read confident, well-explained, and sufficiently supported. These attributes may help raise the awareness of usability in the OSS communities, allowing them to adopt a user-centric mindset and better prioritize these issues. Together, our findings highlighted the multifaceted nature of UX professionals' contributions to usability issue discussions within OSS projects, although their current involvement is rare. These insights may provide directions for further investigation and potential strategies for enhancing the collaboration and engagement of UX professionals in OSS communities in the future.

\subsection{Implication to Practice and Research}
Our results about the actual contributions provided by UX professionals carry important implications for both OSS practices and future research. According to our study's insights into UX professionals' distinctive reporting styles, developing collaboration tools tailored to their communication patterns could support more effective collaboration within OSS communities. For instance, platforms that encourage factual and experience-based reporting, frequently adopted by the UX professionals in our dataset, may enhance their involvement and further benefit other stakeholders (e.g., end users) in creating usability issues. These features can also enhance issue summarization techniques (e.g.,~\cite{Gilmer2023}) to support collaborative issue comprehension and synthesis. Moreover, designing tools that match UX designers with OSS projects based on the designers' backgrounds and skills, as well as the OSS projects' characteristics, could encourage both project maintainers to actively seek support from UX professionals and UX professionals to contribute to OSS projects. Besides, tools could be designed to identify specific usability issue types (e.g., based on usability heuristics) and, depending on the issue type, help UX professionals apply their expertise more effectively in discussing and resolving the issues.

Further, understanding how UX professionals posted and discussed usability allows us to envision collaboration strategies that can maximize their involvement within OSS projects. Integrating their expertise in the entire development life-cycle, including expert review, user testing, and design would offer opportunities to shape the project direction from the beginning. Dedicated spaces for UX-focused discussions on OSS products would facilitate knowledge and experience sharing as well as collective problem-solving among UX professionals. Besides, incorporating UX-related metrics in OSS platforms such as GitHub would highlight the influence of UX contributors, reinforcing the value of their role within the OSS communities.

\subsection{Limitations and Future Work}
This research suffers from multiple limitations that can be addressed in future work. First, we recognize that this research analyzed only a few usability issues reported by UX professionals and other contributors based on an existing dataset from previous research~\cite{sanei2023characterizing}. To enhance generalizability, future studies should aim to expand the sample of OSS usability issues. Similarly, this study only investigated a few popular OSS projects. Our rationale was that popular projects may attract more UX professional contributions. However, smaller projects may have unique characteristics that we did not capture. Further, we did not investigate collaboration platforms other than the ITS (GitHub Issues in this case). We chose this platform since it is the place where most of the public decision-making of OSS is happening. Yet, we acknowledge that some UX discussions may occur on other platforms, including private channels of the projects that we cannot observe. Finally, this research is based on an analysis of artifacts created by OSS community members. Although this method provided rich information about the main focuses of the OSS communities regarding UX professionals in reporting usability issues, it cannot reveal personal cognitive aspects such as goals, motivations, and challenges of usability issue posters. Therefore, by conducting direct user studies with UX professionals who contribute to the OSS communities, future work could have a richer understanding of their roles in OSS.
