Open-source software (OSS) relies on the distributed, often asynchronous, collaborative efforts of developers, designers, and users to improve usability~\cite{wang2020open, Cheng2018, Hellman2022}. This collaboration frequently happens in issue tracking systems (ITSs), such as GitHub Issues~\cite{githubGitHubIssues}, in which various bugs, enhancement requests, and other project-related topics were raised, discussed, and traced. These topics frequently touched on usability aspects of the product~\cite{sanei2023characterizing}. Previous studies on usability discussions on the ITSs~\cite{Cheng2018, sanei2023characterizing} indicated that contributors with UX backgrounds may play a crucial role in enhancing the usability of OSS, providing reports, feedback, and insights that contribute to user-centered improvements. In this study, we investigate the impacts of UX professionals and their expertise in collaborating with the OSS community within the ITS. 

While many contributors may raise and discuss usability issues, individuals with UX design backgrounds bring a unique perspective. They do not only identify problems but also offer solutions rooted in user-centric design principles~\cite{ccetin2007analysis}. Moreover, utilizing their knowledge, UX experts' inputs can go beyond identifying existing bugs and problems in OSS and push into anticipating and preventing potential user difficulties and improving user satisfaction. Recognizing and encouraging the active collaboration and engagement of UX experts within the larger OSS communities can significantly elevate the overall usability and quality of OSS~\cite{hedberg2009integrating, khalajzadeh2022diverse, Rajanen2023Usability}. 

However, previous exploration~\cite{sanei2023characterizing} indicated that the participation of contributors who have UX backgrounds in issue discussions may not be very frequent. This is partially because ITSs are often created with developers in mind, ignoring the needs of other stakeholders, such as end users and designers. Although scarce, UX professionals' participation in issue discussions can still reveal important insights about how they, and their expertise, can influence the style of their contribution in a developer-centric environment, which subsequently impacts the OSS community and the software design. Their participation in ITSs is also crucial for their voices to be heard since design and implementation decisions are often made on those platforms. By analyzing their contribution, we hope to highlight the necessity for their design expertise in the OSS community, emphasizing the potential opportunities for creating more inclusive OSS tools and development environments.

In particular, we conducted a mixed-methods analysis on usability issue discussions in five widely used OSS projects (Jupyter Lab, Google Colab, CoCalc, VSCode, and Atom), based on a dataset created in previous study~\cite{sanei2023characterizing}. We first focused on discovering the background and experience of usability issue posters to identify UX professionals from the dataset. After we found the UX professionals, we explored the following two research questions to characterize their contributions to the projects.

\textbf{RQ1: How do UX professionals raise usability issues differently than other contributors?}
Although there are studies that focus on usability issue discussions in OSS development~\cite{arora2013state, garcia2017challenges, sharma2018usability, sanei2023characterizing}, there is no research in investigating issues posted by UX professionals in OSS. To address this gap, we conducted an analysis to identify the aspects of usability (based on Nielsen's heuristics~\cite{nielsen2005ten}) that have been reported by UX professionals, the types of sentiment and tone expressed in those reports, and the argumentative discourse embedded in the posted usability issues. The findings indicated that there is a limited number of UX professionals in projects who participated in issue discussions (around one in each project). Compared with other participants, UX professionals considered a wider spectrum of usability issues, sufficiently supported their stances when reporting issues, and reported issues mostly based on facts, not their emotions. 

\textbf{RQ2: How do UX professionals follow up on the usability issues they posted?}
After characterizing the issues reported by the UX professionals, we investigated how they continued to contribute to resolving those issues they posted. Concretely, we analyzed their behavior after posting usability issues, when following up on those issues in comments. Through this analysis, we found that UX professionals engaged in about 1/3 of their issues as commenters. An inductive coding on the purpose of their follow-up comments indicated that their main goals were to share ideas/opinions/experiences and provide supplementary details so that their feedback or suggestions could be better understood, taken seriously, and addressed by the OSS developers.

In summary, this study concentrated on examining how UX professionals participated in OSS ITSs, exploring their contribution to reporting and following up on usability issues. Overall, this investigation revealed the distinctive contribution but minimal participation of UX professionals in addressing usability within the ITS. Our results provided some insights that can inform strategies to encourage their active involvement, leading to enriching OSS communities, fostering OSS usability quality, and improving end-user satisfaction.