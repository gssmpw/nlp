\documentclass[sigconf]{acmart}
\usepackage{graphicx} % Required for inserting images
\usepackage{subcaption}

\copyrightyear{2025}
\acmYear{2025}
\setcopyright{rightsretained}
\acmConference[CHI EA '25]{Extended Abstracts of the CHI Conference on Human Factors in Computing Systems}{April 26-May 1, 2025}{Yokohama, Japan}
\acmBooktitle{Extended Abstracts of the CHI Conference on Human Factors in Computing Systems (CHI EA '25), April 26-May 1, 2025, Yokohama, Japan}\acmDOI{10.1145/3706599.3720063}
\acmISBN{979-8-4007-1395-8/2025/04}

\begin{document}

%
\title[Unveiling the Scarce Contributions of UX Professionals to Usability Issue Discussions of OSS Projects]{Untold Stories: Unveiling the Scarce Contributions of UX Professionals to Usability Issue Discussions of Open Source Software Projects}

\author{Arghavan Sanei}
\email{arghavan.sanei@polymtl.ca}
\affiliation{%
  \institution{Polytechnique Montreal}
  \city{Montreal}
  \state{Quebec}
  \country{Canada}
}
\author{Jinghui Cheng}
\email{Jinghui.cheng@polymtl.ca}
\affiliation{%
  \institution{Polytechnique Montreal}
  \city{Montreal}
  \state{Quebec}
  \country{Canada}
}

\begin{abstract}
Previous work established that open source software (OSS) projects can benefit from the involvement of UX professionals, who offer user-centric perspectives and contributions to improve software usability. However, their participation in OSS issue discussions (places where design and implementation decisions are often made) is relatively scarce since those platforms are created with a developer-centric mindset. Analyzing a dataset sampled from five OSS projects, this study identifies UX professionals' distinct approaches to raising and following up on usability issues. Compared to other contributors, UX professionals addressed a broader range of usability issues, well-supported their stances, and were more factual than emotional. They also actively engage in discussions to provide additional insights and clarifications in comments following up on the issues they posted. Results from this study provide useful insights for increasing UX professionals' involvement in OSS communities to improve usability and end-user satisfaction.
\end{abstract}

\begin{CCSXML}
<ccs2012>
<concept>
<concept_id>10003120.10003130.10003233.10003597</concept_id>
<concept_desc>Human-centered computing~Open source software</concept_desc>
<concept_significance>300</concept_significance>
</concept>
<concept>
<concept_id>10011007.10010940.10011003.10011687</concept_id>
<concept_desc>Software and its engineering~Software usability</concept_desc>
<concept_significance>500</concept_significance>
</concept>
<concept>
<concept_id>10003120.10003130.10003131.10003570</concept_id>
<concept_desc>Human-centered computing~Computer supported cooperative work</concept_desc>
<concept_significance>300</concept_significance>
</concept>
</ccs2012>
\end{CCSXML}

\ccsdesc[500]{Human-centered computing~Open source software}
\ccsdesc[500]{Software and its engineering~Software usability}
\ccsdesc[500]{Human-centered computing~Computer supported cooperative work}

\keywords{Usability Discussions, Open Source Software, UX Professionals}

\maketitle

\section{Introduction}
% 
% 
The widespread integration of communication networks and smart devices in modern control systems has increased the vulnerability of industrial systems to online cyber-attacks, e.g., Industroyer, Blackenergy, etc \citep{osti_1505628}.
% Modern control systems have seen a large push to include communication networks and smart devices to increase performance, made possible by improvements in communication device cost and energy consumption. This trend has been coupled with the usage of open-standard communication protocols among industrial control systems, making them vulnerable to online cyber-attacks such as Industroyer, Blackenergy, etc \citep{osti_1505628}. 
To counter this, methods have been developed to improve security by achieving attack detection, mitigation, and monitoring, among others \citep{sandberg2022secure}. This paper focuses on active attack diagnosis to mitigate stealthy attacks. 
%
%\subsection{Literature review}

Active diagnosis techniques rely on the inclusion of additional moduli to control systems
% inclusion within the control system of additional moduli 
to alter the behavior of the system compared to information known by the attacker. 
For instance, the concept of additive watermarking was introduced in \cite{mo2015physical}, where noise signals of known mean and variance are added at the plant and compensated for it at the controller. 
This compensation, however, is not exact, causing some performance degradation. Thus, trade-offs between performance and detectability  are necessary \citep{zhu2023detection}.
% A later work \citep{zhu2023detection} designs the watermark signal by trading performance for detection. Thus, although additive watermarking serves as a good detection scheme, they endure performance losses even in the nominal case. 

In encrypted control \citep{darup2021encrypted}, the sensor data is encrypted, sent to the controller, and then operated on directly. Encrypted input signals are sent back to the plant for decryption. Although encryption is widespread in IT security, in control systems it presents some concerns, such as the introduction of time delays \citep{stabile2024verifiable}, while it may present inherent weaknesses \citep{alisic2023model}.
% they are not preferred as they introduce time delays \citep{stabile2024verifiable} which can cause instability, and some encryption schemes can be very weak  \citep{alisic2023model}. 

In moving target defense \citep{griffioen2020moving}, the plant is augmented with fictitious dynamics, known to the controller. The plant output is transmitted to the controller along with the fictitious states over a network under attack. 
The additional measurements then aide in the detection of attacks. 
This comes at the cost of higher communication bandwidth needs, which increases rapidly with the dimension of the augmented systems.
% Since the dynamics of the fictitious dynamics are exactly known to the controller, the attack is detected easily. However, when the scale of the system increases, the communication bandwidth used by moving the target defense approach increases rapidly. 

Other recently proposed works include two-way coding \citep{fang2019two}, a weak encryuption technique, and dynamic masking \citep{abdalmoaty2023privacy}, which enhances privacy as well as security, have been shown to be effective against zero-dynamics attacks.
% Two-way coding \citep{fang2019two} and dynamic masking \citep{abdalmoaty2023privacy} are other recently proposed approaches. Two-way coding is another form of weak encryption technique whilst dynamic masking proposes an architecture that enhances both privacy and security. These schemes are shown to be effective against zero dynamics attacks but remain to be studied for other classes of attacks. 
% Recent extensions include \citep{mukherjee2021secure,ramos2024privacy}.
% Some other works which are related are \citep{mukherjee2021secure}, an extension of \cite{fang2019two}. The work \citep{ramos2024privacy} is an extension of moving target defense for multi-agent systems. 
Furthermore, filtering techniques for attack detection are proposed by \cite{murguia2020security,hashemi2022codesign,escudero2023safety}, while not focusing on stealthy attacks.
% The works \citep{murguia2020security,hashemi2022codesign,escudero2023safety} develop filtering techniques to guarantee safety, without being focused on stealthy covert attacks.

Multiplicative watermarking (mWM) has been proposed by the authors as a diagnosis technique \citep{ferrari2020switching}. mWM consists of a pair of filters on each communication channel between the plant and its controller; the scheme is affine to weak encryption, whereby ``encoding'' and ``decoding'' are done by changing signals' dynamic characteristics through inverse pairs of filters. This enables original signals to be recovered exactly, and thus does not lead to performance degradation.
% A multiplicative watermark is an affine to a weak encryption technique, through which the signal is ``encoded'' by a filter, changing its dynamic behavior. The use of inverse pairs means that the original signal can be recovered, through ``decoding'' via an inverse filter. As such, differently to techniques based on additive watermarking, no performance is lost due to the injection of noise, and there are no bandwidth limitations.

%\subsection{Contributions}
One of the critical features of multiplicative watermarking is that to detect stealthy attacks, the mWM filter parameters must be switched over time. In this paper, an algorithm to optimally design the mWM parameters after a switching event is presented, enhancing detection performance, without changing the switching time.
% This is done without changing the switching time, which is taken as given.

\textcolor{black}{
To formalize the filter design problem, we suppose the defender is interested in optimal performance against adversaries injecting covert attacks with matched system parameters \citep{smith2015covert}, including the mWM parameters prior to the switch. This scenario represents a worst case where malicious agents can take full control of the system while remaining undetected.
Thus, the attack strategy is explicitly included within the formulation of the closed-loop system, and the mWM filters are chosen by solving an optimization problem minimizing the attack-energy-constrained output-to-output gain (AEC-OOG) \citep{anand2023risk}, a variation of the output-to-output gain proposed in  \cite{teixeira2015strategic}.
}
The main contributions of this paper are:
% We consider an adversary injecting a covert attack with matched system parameters \citep{smith2015covert}, i.e., an attacker with full knowledge of the control system parameters, including those of the mWM filters before the switch. This scenario is taken as a worst case, as it has been shown that this class of attacks can be made stealthy. To quantitatively define a cost, the output-to-output gain (OOG) \citep{teixeira2015strategic} is leveraged,
% a metric introduced to evaluate the impact of an additive attack in a control system. %Specifically, OOG evaluates the worst-case performance loss that an attacker injecting an undetectable attack can obtain. 
% Here, the maximum performance loss caused by a stealthy adversary with limited energy is taken, the attack-energy-constrained OOG (AEC-OOG) \citep{anand2023risk}. The main contributions of this paper are:
\begin{enumerate}
%[label=\alph*.]
\item The problem of optimally designing the switching mWM filters is formulated as an optimization problem, with the AEC-OOG is taken as the objective;%where the AEC-OOG is taken as the impact metric; 
\item The worst-case scenario of a covert attack with exact knowledge of plant and mWM filter parameters is embedded within the design problem;
% The optimization problem is defined to incorporate the worst-case scenario of a covert attack with exact knowledge of plant and mWM filter parameters;
\item The feasibility of the optimization problem is shown to be dependent only on stability conditions; 
\item A solution scheme is proposed to promote randomization of the mWM filter parameters such that an eavesdropping adversary cannot remain stealthy.
\end{enumerate} 

This builds on the results of \cite{ferrari2020switching}, where the focus was on the design of the switching protocols, rather than the parameters themselves.
Compared to previous work \citep{gallo2021design}, this paper introduces an optimization problem which is always feasible (thanks to the use of AEC-OOG in the objective), while also considering a more sophisticated class of covert attacks, where the presence of watermark is known to the adversary. 
Moreover, this paper poses a different objective than \citep{zhang2023hybrid}; indeed, while \citep{zhang2023hybrid} provided a design strategy to ensure certain privacy properties, in this paper we address the problem of optimal parameter design following a switching event.


%\subsection{Organization}
The rest of the paper is organized as follows. 
After formulating the problem in Section~\ref{sec:PF}, we propose our design algorithm in Section~\ref{sec:main}, and analyze its properties. It is then evaluated through a numerical example in Section~\ref{sec:NE}, and concluding remarks are given Section~\ref{sec:Con}.
% We provide the problem background in Section~\ref{sec:PF}. We formulate the design problem in Section~\ref{sec:main}, together with an analysis of its properties. The proposed algorithm is evaluated through a numerical example in Section \ref{sec:NE}. Concluding remarks are offered in Section \ref{sec:Con}.
\section{Methods}
We based our analyses on the labeled data created in previous work~\cite{sanei2023characterizing}. The dataset distinguished 305 usability issues from five popular OSS projects (Jupyter Lab,
Google Colab, CoCalc, VSCode, and Atom) and identified their posters. In this paper, we focus on individuals who have ever posted a usability issue in that dataset. 

\subsection{Discovering the Role of Issue Posters}\label{sec: Discovering_role}

To detect the background of the usability issue posters in the dataset, we checked each user's \textit{Profile page} on GitHub, examining their bios, shared personal websites, LinkedIn pages, and/or shared resumes. If they have not shared these information, we searched for their LinkedIn profiles using their full names to extract details on their backgrounds and expertise. We considered their job titles posted in the information acquired this way and categorized them into (1) UX professionals, (2) managers, (3) data scientists, and (4) developers. UX professionals were defined as those indicating positions such as \textit{UX designer} and \textit{user interface and user experience designer}.

Among the 224 usability issue posters in the dataset, we were able to identify the role of 180 users. Within those 180 users, 121 (67.2\%) were developers, 34 (18.9\%) identified as data scientists, 21 (11.7\%) held managerial positions, and only four (2.2\%) were UX professionals. The UX professionals included one male contributed to \textit{VSCode}, another male contributed to \textit{Atom}, and two involved in \textit{Jupyter Lab} project, one male and one female. Notably, there were no UX professionals involved in \textit{CoCalc} and \textit{Google Colab} projects in our data sample. For easier referencing, in the following we call the UX professionals of VSCode as \textit{VSCode\_pro}, Atom \textit{Atom\_pro}, male of Jupyter Lab as \textit{Jupyter\_pro\_M} and female as \textit{Jupyter\_pro\_F}.

\subsection{Characteristics of Issues Posted by UX Professionals (RQ1)}

Once we identified the roles of the usability issue posters, we extracted all the issues posted by the four UX professionals across the five OSS projects. Next, we analyzed the extracted issues by adopting the following steps. First, following the approach outlined in \cite{sanei2023characterizing}, we labeled each issue with either usability or non-usability; and for each usability issue, we identified the main \textit{usability dimension} touched by the issue using the ten Nielsen heuristics~\cite{nielsen2005ten}. Then, similar to \cite{sanei2021impacts}, we identified the specific \textit{sentiment} and \textit{tone} expressed by the UX professionals when posting the usability issues. In our study, the sentiment captures the valence of the emotion that includes three categories (positive, negative, and neutral), while the tone describes emotion with seven affective factors (excited, frustrated, impolite, polite, sad, satisfied, and sympathetic). Subsequently, we analyzed the \textit{argument structure} of the usability issues to better understand the discursive device that the issue posters adopted to convince other discussion participants. We particularly identified whether a \textit{claim} and a \textit{premise} appeared in a usability issue post, using criteria proposed in prior work~\cite{skitalinskaya_learning_2021, wachsmuth_argumentation_2017, dowden1993logical}. Statements were considered as claims if they explicitly indicate the position or stance of the issue posters to the discussed usability issues; and premise means that a statement contains reasoning, evidence, or examples that support a stance. We compared how the above characteristics (i.e., usability dimensions, sentiments, tones, and argument structures) differed in issues posted by UX professionals and those without UX expertise.

\subsection{UX Professionals' Purpose Following Up on Issues (RQ2)}

% After investigating how UX professionals posted the usability issues, we recognized the importance of understanding their participation afterwards, particularly in following up on the discussion threads of the issues they posted. 
Thus, we first isolated comments made by the UX professionals posted to the usability issues they created within the datasets. Then, we employed an inductive content analysis~\cite{wamboldt1992content, Hsieh2005} and categorized the various purposes behind their contributions in posting each comment. For our analysis, the \textit{purpose} specifies the distinct goal that a particular comment serves within the context of the discussion thread. The purpose of a comment may vary based on its content and the immediate objective of the issue posters to write in the discussion to address one specific comment posted by another contributor. We grouped the identified purposes into themes through an iterative approach conducted by the two authors.

\section{Results}
In our analysis, we found 105 issues reported by the four UX professionals in all the issues of VSCode, Atom, and Jupyter Lab (a total of 139,948 issues). A majority of the issues (93 out of 105, or 88.6\%) focused on usability, only one (0.9\%) discussed a bug and 11 (10.5\%) were related to reporting \textit{UX meetings} for the VSCode project.

\vspace{-2mm}
\section{RQ1: Users’ Current Impression of PHC}
\vspace{-2mm}
%\subsection{Current Online Identity Verification Practices}

\textbf{Current Verification Practices.} 
When discussing identity[personhood] verification, 
participants most commonly mentioned financial services, including online banking and investment platforms, as well as health services, government-related processes, and cross-border regional verifications in both
%, and two-factor authentication (2FA) used by various institutions. 
%The discussion highlighted both 
digital and physical forms of credentials. Several participants mentioned they are required to upload government IDs (e.g., social security number, driver's license) when creating an account for financial services- as P7 said--\textit{``For Robinhood, it asked for uploading my government-issued IDs like driver's license and passport.''} 
%In addition to financial services, the applications where participants experienced verification processes include a wide range of applications. 
Similarly, P2 explained their verification experience in the government services requiring multiple information \textit{``Last year when I was requesting my tax filing documents in IRS. To access them, I had to verify my identity with my face, as well as information from government-issued IDs to confirm my identity. [id.me~\cite{irsIdentityVerification}].''} Beyond financial and government services, identity verification has also become essential in marketplace apps,
%that rely on trust to facilitate peer-to-peer transactions. 
P1 illustrated this trend with Airbnb, explaining  \textit{``If you do book an Airbnb, at least as a renter, you would need to verify your government id before being eligible to book your first stay.''} 
%This indicates that the scenarios we prepared for succeeding sessions are representative of typical user occasions. 

%\textbf{Confusion Between Verification and Authentication} Many participants did not distinguish between verification and authentication. In particular, when asked about experiences with biometric verification, they often shared daily experiences related to authentication, e.g., (P3): \textit{"It's I think most of the apps required me such fingerprints or related identification. For example, when I open my bank apps, I can just use my fingers to login to my account."} Participants indicated frequent use of fingerprints or face authentications on mobile phones. For instance, P1 mentioned stored biometric information on devices, \textit{"Because my android phone already have my fingerprint to log into my phone. And then this app just to reuse my fingerprint, which is already registered on my phone."}

%\textbf{Perception Before \& After Educational Video}
%As discussed in the method section \ref{sec:study_protocol}, we revisited questions about participants' understanding of PHC to compare their responses before and after watching the educational video to assess whether the video impacted their original knowledge of PHC. 


\iffalse
\textbf{Misconception \& Lack of Understanding of PHC.}
We observed that users sometimes \fixme{how may misunderstand, add the participants add in parenthesis} misunderstood PHCs. 
%These misconceptions represent critical points that should be carefully addressed in communication when aiming for the widespread adoption of PHC in the future. 
P10 explained the benefit \fixme{why benefit is misconception? you already have another theme for benefit} of PHC, stating \textit{"I am thinking that imagine you are a criminal or an online criminal. For example, with some records that you are not allowed to use some services online. For federal government, they need to verify your credentials to be able to track you or just not allow you to do activities."} This comment was interpreted as PHC enhancing traceability to assist in criminal investigations \fixme{how come this is a misconception? this perhaps a narrow understanding of PHC, same goes for tracking data aspects, explain to reflect on it}. 
\ayae{I incorrectly develop this theme. I flagged "misunderstanding" to narrow understanding. Let me skip this theme. }
\fi
%Such potential traceability was also mentioned in the context of tech companies tracking data, saying \textit{"But what is more of a concern for me is that some of these big tech companies like Google, like Facebook, like even Apple, Amazon. They use that information to track my searches and activities."} However, PHCs are fundamentally intended to serve as a means of anonymously verifying users' personhood without compromising their privacy.  \fixme{add 2 interesting quotes that have clear misconception about PHC. none of the quote above dictate clear misconception.} 

\textbf{Trade-offs between known and unknown privacy guarantee.} 
Participants often made trade-offs between familiar security guarantees associated with traditional verification methods over the less clear assurances of emerging PHC. As P18 mentioned she heard about World ID as a personhood credential and remarked \textit{``
%If I’m signing in for the first time or doing a person verification with World ID, it will be able to tell if I am signing in for the first time? I guess that means they still keep some sort of data from the iris scan. 
They scan your iris, create a unique code, and re-verify me again later time. I guess that means they still keep some sort of data from the iris scan. I would still stay with email or traditional verification as I know what they are keeping. Even though world app guarantees privacy with hi-tech, I don’t exactly know how much privacy I have in human tests. I’m not saying it’s a bad idea, it's just i am unsure.''} 
%#### use this in discussion We also identified another potential factor: age, which may influence perceptions of PHC and the types of data used in credential issuance. 
P23 with age ranges from 45-52 noted--\textit{``it's just easier to do with email or my physical photo id in an old system 
%like getting ssn, 
Newer technology is supposed to be faster and more user-friendly, but to me, iris scan or selfie scan, I can't even know if I am doing it correctly. I’m getting to the point where I can say I’m old-fashioned.''} This underscores the generational divide in preferences with emerging technologies compared to established methods.




%\textbf{Post-understanding: Found similarities with other technologies} P5 interpreted the term as a verification method based on personal characteristics, which differ from person to person, such as behavioral patterns, commenting \textit{"It's like, personality based or like activities humans particularly perform. It sounds similar to the pattern of people walking. Those are also things people use for verification."} After watching the video, he updated his perception of PHC \textit{"I thought it was similar to SSL technology. Like in SSL, they will give the certificate to 3rd party who encrypts and has those certificates. To maintain their integrity, they will pass on the kind of certificate to web browsers. So when a user goes through a particular website, they will be able to cross-verify using the public keys available."}, linking it to technologies such as SSL, which uses encryption and hashing for certificates.


%\subsection{Perceived Benefits of Personhood Credentials}
%\fixme{there needs to be titles of sub-themes under this benefit, it's hard to read. Sub-themes could be -- convenience of not carrying physical id, etc -- whatever unique /interesting subthemes came up in the interview}
\iffalse
\textbf{Perceived Benefits of PHC: Reduced Personal Information Exposure} %\fixme{this section is not interesting enough, is there quote where participants had to provide smany info like physical id, digital verification to get unique identifier? if so, then those examples / quotes will go well with this theme. please look for peoples' experience related things that was stressful or challenging, current quotes is just statement, not experience at all. we need to present themes through experience related quotes.} \ayae{reflected}
\tanusree{note: may remove this theme if space needed}Participants identified various benefits of PHC, frequently involving reduced risks related to personal data exposure, such as minimizing personal data exposure, reducing repeated checks for different online services, preventing data misuse, and enhancing security against fake accounts through trusted, privacy-focused third-party systems. As P3 mentioned about having different ids to even perform regular tasks \textit{``when I am in a foreign country, or even in dealing with housing, I don't trust the leasing process and people involved, sometimes the sites used are sketchy, but I had to verify my ID, paystub, ssn, residence permit or passport that would reveal all of my personal details. It would reveal my permanent address. Almost everything, most of the time I am not comfortable at all. But with PHC, it's like they use a minimal of verification.''} P13 on contrast mentioned different verification method for different sites, \textit{``it feel like I am putting my information into a lot of different sites, [..]. finance with KYC, student id with capturing photo, traveling with passport. With PHC, much of these can be improved''} P1 explained how PHC can increase confidence with unfamiliar entities \textit{``All of that information doesn't need to be filled in each time, rather verifying with a unique identifier. Even there is a breach, its less likely to link my personal data''}
%you fill a form for submission and the potential conference that if I am signing up for an account with a 3rd party, which I have not heard before. I'm not comfortable with even in the case of a breach. 
%They would not have access to my personal information, but just rather the identifier id. So that would be something I'd be comfortable with.''} 

%with the most frequently mentioned the reduced personal information exposure in the issuing process. P3 expressed perceived benefit with her anxious experience of sharing sensitive information for verification, \textit{``I think PHCs, essentially, they collect certain information like Iris scanning and use that to verify your identity online without exposing your personal details. So personal details, such as maybe my name or my postcode, or if I were to verify my ID through scanning like a residence, permit or passport that would reveal all of my personal details. It would reveal my address. My full name even my like my number, such as the passport number and things like that. But with PHC, it's like they use a minimal of verification.''} P13 reflected her thoughtless data sharing when asked about PHC benefits, \textit{``I think at the moment it's kind of like it does feel like I am putting my information into a lot of different sites, probably without thinking too much about it. So I think that I'd feel a lot better doing that.''}
%Several also emphasized advantages in addressing online identity crises caused by AI.
%P4 expressed \textit{``I think what I glean from it was there was to provide without giving away too much personal information, but it can still sort of verify your uniqueness.''} In addition to this, 
%P11 summarized the benefits of PHC, \textit{``Personhood credentials is a way in which a person can confirm that they are not a bot or an AI, and that they're a real person without giving away their personally identifiable information.''} 
%P11 discussed online malicious attacks, \textit{``I think that it will tackle like fake accounts, multiple accounts and bots massively. And I can see it being used broadly because you're not giving away any really important information... And having to get this identification through your government, or something works really well because you don't mind giving them the information. And then, the kind of redacted information that you give to social media sites, increases safety and stuff, but doesn't put your data at risk.''} 
%P1 explained increased confidence in privacy through PHCs for data sharing with unfamiliar entities \textit{``All of that information doesn't need to be filled in each time, you fill a form for submission and the potential conference that if I am signing up for an account with a 3rd party, which I have not heard before. I'm not comfortable with even in the case of a breach. They would not have access to my personal information, but just rather the identifier id. So that would be something I'd be comfortable with.''} 
%\textbf{Efficiency of Verification Process} 
%\fixme{this section can be merged to reduced personal info. do you have experience related quotes to present this effectiveness. current quote is not interesting} \ayae{reflected}Another key benefit highlighted by participants was the efficiency of the verification process through less exposure to personal information. P11 shared her perspective on the lengthy verification process, \textit{``I think it streamline the process because I mean some websites, some social media do request you to upload documents to verify your identity these days, and it is quite a long process sometimes, like I said,  sometimes your ID gets rejected. So maybe using this new method would be quicker. And you could set up an account a lot easier, and that'd be good.''}
%P3 emphasized the convenience of less repeated verification, stating, \textit{"Having an easier login from a trusted source. So it's ease of access for the user." } 
\fi


\textbf{Perceived Benefits of PHC: Fairness in Representation.} Participants discussed fairness and the potential benefits of PHCs. P19 shared her experience with online study platforms which is one of the primary means of earning, \textit{``often time I got "time out or returned" in this platform and earlier I thought I am slow responding to survey request I received on my account. Lately, it happens too often, feels like there are bots or a group of people are more proactive in participating in surveys. honestly, it reduces the chances for me being selected for research studies ''} She emphasized that personhood verification could help filter out bots and increase her chances as a genuine participant.  P21 who is a dedicated gamer who find it disheartening to encounter players, likely bots, who level up so quickly. To emphasize the benefit of PHC, he mentioned \textit{``“ I play online games with other people around the world, like shooting games. We regularly encounter bots in these games, which basically dilutes our experience. As an experienced player, I can usually tell when someone isn’t a real person
based on how they play. personhood verification could be a good feature.''}




%\subsection{Perceived Concerns of Personhood Credentials}
%\fixme{there needs to be titles of sub themes under this benefit, it's hard to read }


\textbf{Perceived Concerns: Power, Control, Security} 
%\fixme{the title is too plain, I remember there was interesting themes in the coding and i don't see those here. please go back to the excel to rewrite perceived concerns. first summarize these concerns from the list below I added in the comment line and add 2 quotes which are most interesting. } \fixme{not addressed well. as said before you need to first summarize the concerns in a line and then add a quote on interesting ones.  this seem interesting -"failed to detect criminal information"  and "uncertain regulation", "centralized power"}\ayae{reflected on 1/9}
% Concerns: PHC/credentials getting stolen (due to hacking)
% Concerns: credential's validity period
% Concerns: untrustworthy PHC issuer
% Concerns: centralized data storage
% Concerns: Issuers can be hacked
% Concerns: PHC can de-anonymized
% Concerns: information will be stored centrally, this may lead to power can be abused later
% Concerns: uncertain regulations
% Concerns: failed to detect criminal information
% Concerns: malicious attacker create fake PHC
% Concerns: data linkage of credential and information stored in service providers
Participants expressed concerns about potential risks, including the centralized power of the issuer, uncertain regulations of emerging technology, authenticity of PHC, and misuse of anonymity.
For instance, P6 pointed out that centralization of credential information with PHC issuers could lead to power being misused, saying, \textit{"you're gonna give all your information to a small group or institutions to issue PHC. So they have the power that can be abused later."} 
%P10 expressed concerns regarding the authenticity of PHCs, particularly in scenarios where credentials could be falsified. They noted, \textit{"I don't know if it's possible to fake government-issued ID. I feel like I'm probably concerned about an inauthentic PHC."}\fixme{this seems off compared to the excel} 
P10 also highlighted the risks in high-stakes situations, such as employment background verification, where an inauthentic PHC could have serious consequences. He noted \textit{"I don't know if it's possible to fake government-issued ID. I feel like I'm probably concerned about an inauthentic PHC."} P4 in a similar topic expressed concerns about how much data will be revealed to an employer if they only share PHC \textit{``For instance, I had a job offer that required details about my criminal record from five years ago. It would probably reveal more even if I only share PHC. how much anonymity I have.''}

%"That's good. I feel like for some type of government position. I'd be more open to doing that. depending on the company. And the, I guess, like intensity of their background check. And like how far they're really trying to go back. It would depend on what Phcs you would disclose. If you get to pick, but again for a background, check it. I would be fine using a Phd.

%Sure. Yeah. So, for example, like, I had a job offer that was asking about 5 years prior, like a crime record so like for that type of severity where they're kind of looking pretty far back, in my opinion, from 5 years I would probably be more that would probably need more Phc credentials than yeah one. But one that's just like a general. You are a real person type background check, which I actually don't really know how common that is, but like less severe."


%Another significant concern is the anonymity of PHCs. One participant emphasized that for background checks, providing personal information is crucial to ensure a thorough review of candidates' histories, expressing, \textit{"For the background check, I would provide my personal information. Because it might be needed information to fully check their background and if the hired employees doesn't have any negative histories or criminal records, or something like that. That's important for the company. So that's why I would provide the personal information instead of PHC."}
%P4 -- "Possibly that again the somebody trying to steal the you know the the image, or whatever you're trying to upload. So I think this is a general cyber security concern."
%a single point of failure in centralized control, data governance and identity integrity issues.
%P6 was concerned about the possibility of their biometric credentials being hacked, saying \fixme{not interesting quote}
%\textit{"I worry that if someone were to steal your biometric data like your iris scan that then they could just have access to all of your accounts through that one thing through the one iris scan. So I don't want there to be like one single weak point that could just topple the entire thing like that."}
%P17 discussed uncertain regulations and expected clear guidelines on data practices of PHC systems while it involves more sensitive process than a password and email \textit{"if there is a good regulation that what extent this PHC system is able to share personal information to governments or other institutes and what security might be in place."}
%P17 -- "My main concern is that, for example, someone could record your iris in that case, or something like that, and manage to trick the verification system just by recording your eyes, or something like that. I don't know if that's really like a stupid thought. But that's what I was imagining, like as a concern, that maybe people could record your face or your iris, and maybe trick the system. I don't know if that's possible. Yeah, that that would be a really concerning situation. Because if everything like, if all my passwords were set up in a by a PHC, I'm guessing that if they if all my information gets leaked, people could access all. My, for example. bank account information, or yeah, health information. And yeah, that that would be really concerning."
%\fixme{one quote used twice, not clear} This topic can be interpreted as an explainability issue. P10 emphasized the importance of transparency in data practices, \textit{"I'm curious where this (credential) is saved, though, like what server holds all this information. That would be cause I feel like recently, we've seen more stuff about like banking apps, having data breaches or something, which is a concern with your data saved."} 


%Another significant concern is the anonymity of PHCs. One participant emphasized that for background checks, providing personal information is crucial to ensure a thorough review of candidates' histories, expressing, \textit{"For the background check, I would provide my personal information. Because it might be needed information to fully check their background and if the hired employees doesn't have any negative histories or criminal records, or something like that. That's important for the company. So that's why I would provide the personal information instead of PHC."}
%P4 -- "Possibly that again the somebody trying to steal the you know the the image, or whatever you're trying to upload. So I think this is a general cyber security concern."

\textbf{PHC Preference Dilemma: Physical and Digital Verification.}
Many participants often relate their offline experiences when describing their preference for credentials and issuance systems.
For instance, some of them described the inconvenience of carrying physical IDs %with offline experience of verification in daily life
, saying \textit{``When you need to buy some alcohol drinks, take a flight, you need to show your driver's license or passport. %And when you need to buy some in the market or the grocery stores, and when you take a flight, you also need to use these IDs. 
But it's not a very convenient, because I have to carry the physical ID all the time.''} Another participant pointed out the potential risk of IDs being stolen, citing, \textit{``I think this information may be lost, or maybe stolen. So it's a risk.''} 
While participants highlighted the inconvenience and physical vulnerabilities of traditional IDs, they simultaneously shared the limitations of digital identifiers, such as verification through phone number [OTP] during international travels.

%which is often dependent on  .P\fixme{} noting, \textit{"And I think the challenge I met is since I need to use my Chinese number to receive the code. The good thing is that I can receive the code in the US. But I'm not sure if I can receive the code in other countries. So I think that's very inconvenient if I'm traveling abroad."}\fixme{what does this quote convey, who said this. no participant id here} \ayae{This quote explains that if the credential is a digital identifier such as a phone number, there will be problems when traveling internationally. This quote from pilot study P1, I tried to find alternative quotes from main study, but it was unique. Should I skip this part? }\fixme{i can understand from what you say, but not from the quote. this is researcher's job to make sure the quote is understandable}

%\textbf{Data Handling by PHC Issuers} Another significant concern was how PHC issuers manage and handle the credential data they collected. Participants showed uncertainties regarding what happens to their personal information after verification and whether it is securely stored or potentially mishandled. For example, P4 shared that \textit{"I'm not sure if they tell us explicitly how they're gonna use, do they going to share our information with 3rd party or with government. If they ask for it, I'm not sure about that. These are my concerns."}

%\textbf{Centralized PHC} Several participants expressed concerns over the centralized power held by PHC issuers, which they feared could lead to misuse or overextension of control. P4 commented \textit{"But I think there's a concern on the other side that they're gonna be one kind of you're gonna give all your information to a small group or an institution. So they have the power that can be abused later."} illustrating the apprehension about the concentration of power in the hands of a single entity as a PHC issuer.
\iffalse
\textbf{Trustworthiness of PHC Issuer} The issue of trustworthiness in PHC issuers was also discussed, with participants expressing doubts about the reliability of the entities managing their data. Some worried whether these issuers could be trusted to protect their personal information and maintain their privacy. P7 described \textit{"My only concern is, how do we trust the Phc Am I using it? Yeah, because if if they are the one entity who handles all these kind of information, and if they are collecting everything about us from the government. And whether this entity is considered like a NGO, or that's a governmental body, or that if that's a private sector, so these are like some nuances about how to define or how to build this kind of issuer, because we have these giant issuers like Equifax or like these giant companies who are handling credits or those kind of financial aspect of the US structure. And they're like giants. They have these kind of monopoly over how they handle things. Is it going to be something similar? If so, how people can trust them? I don't want to share my personal information anymore, with like some random website...But how the way that the issuer can be understood? I think that would be the decision factor for users to start engaging with the system."}

\textbf{How Users Evaluate the Trustworthiness of Stakeholders} When discussing the current verification process, participants often indicated that there are organizations they can trust and others they cannot. Thus, we asked how or why they have trust with specific stakeholders. P5 explained his trust in banks comes from regulatory aspects: \textit{"They are being monitored by federal agencies. Their activities are monitored. They are under a lot of regulations. So there is a monitoring system that is tracking banks, at least Major Banks or Banks that I know of and other financial services they are under a lot of regulations."} P4 highlighted another aspect resulting in their trust in banks,  which is that the accessible customer support, stating \textit{"You know what sort of steps have taken so you could probably have a read of that, and any concerns that you might have. So you probably always have the opportunity that you could always speak to them first, because obviously they probably have agents available that you could speak to about any concerns. "} On the other hand, we observed that many participants expressed distrust towards social media platforms. P3 shared a formative experience of her social media account almost being hacked, noting \textit{"I've had people experience...Who then tried to get like hack into my account. So the one hacker would hack one person's account, and because, obviously, in your social media, you have access to other people's accounts as well, like I've had people send me really dodgy links, which I knew like instantly, that if I clicked on that link, it would mean that my account would get hacked as well."} Some participants elaborated on how they evaluate the trustworthiness of companies or stakeholders when they share their sensitive information in general. P6 reflected \textit{"I guess if the company was suggested to me, or I was referred to the company through the services of like Capital One or a big company that's been around forever. That would give more validity to me."} P7 provided insight from his own experience with a suspicious verification process, saying \textit{"I do a lot of research… What I did was I went and searched on the Internet. So I went on Reddit, and I looked for different keywords, and my conclusion was that it's a very standard process. It's not like the dealership wants to collect the social security for themselves. They wanted to do the background check, and they wanted to see if the information that they provided was matching with my information from the US. "}
\fi
%\textbf{Offline Verification in Daily Life} Some of the participants shared their offline experience that required identity verification, especially in the context of international travel and immigration procedures,  as P3 described \textit{"So in the past I've had to use my fingerprints, for example, when I came to the UK. When I had to renew my passport in the embassy of my country, they would take my fingerprint there. So they would take a photo of my current passport, and then they would take a photo of myself, but then they would also require me to give my fingerprints as well."}  


\iffalse
\textbf{Motivational Experiences for Using PHC}
\fixme{give a clear theme title. what motivation? this title doesn't convey to RQs. think critically}
When discussing the benefits of PHCs, participants shared their experiences with deceptive online activities stemming from digital identity crises ranging from bots to scams. P10 explained the issue of bot accounts on social media, noting \textit{"I get random bots following me on my twitter a lot which I'm assuming. If they had this PHC Implemented, we would probably see a reduction in those types of bots...I'm assuming there's bots on Instagram as well, and I'm sure most social media accounts is probably bots."} P11 also reflected on fake news spread by bots, \textit{"I know that bots and AI and stuff are utilized quite a lot on social media sites, and to spread like fake news, and to support certain people's agendas and the kind of a tool that that are used in order to."} Accordingly, identity verification is recognized as a key area of concern in the context of social media platforms for end-users due to the proliferation of bot accounts and the dissemination of fake news. 

In addition, some participants emphasized the significance of identity verification in relation to cryptocurrency. P13 stated scams in cryptocurrency, \textit{"I think with cryptocurrency, everyone's scamming everyone all the time...And the companies who do the id verification basically decided it might be legally difficult. So they just stop. So that's probably the main problem I've had."} P12 indicated trade-off between privacy and traceability in crypto wallets, \textit{"I don't really like identity you put in your facial verification, or your Id or your driver's license on trust. But for Binance, Bybit and KuCoin, they actually request that because it's actually a big company or anything. But I'll see it's just a very diverse app that you can do a lot of things on. It needs the privacy of other people. It needs the privacy of other people and yours to cooperate because of scams and fraud."}

\textbf{Context Matters for Users' Preference}
 \fixme{this needs to expand a lot, this is one of the main section Section 2 for all the scenarios, and one of the main results. we also have post survey for this result. move some visuals here and rewrite this }
As outlined in the method section \ref{sec:study_protocol}, we explored participants’ perceptions and preferences regarding PHC across five distinct scenarios. Participants’ responses indicated the necessity of verification methods depending on the specific scenario. In some contexts, they considered PHC unnecessary and suggested simpler methods like email or phone verification. As P5 stated, \textit{"I love to talk, to, chatgpt, and bard about weird stuff, and I think a lot of people like to talk to them about weird stuff. And so I would like the option to not log in via biometrics, and have a lesser experience and not have my identity tied to it. So I personally, I would not like to have my identity tied to my chat Gpt, or my bard, or Gemini."}, they explained that online platforms, which do not require extensive personal information, are sufficient with simple verifications.
\fi

\vspace{-2mm}
\section{RQ2: Factors Influence on How People Would like to Verify Themselves }
\vspace{-2mm}
%\textbf{Context Matters for Users' Preference}
%As outlined in the method section \ref{sec:study_protocol}, we explored 
In this section, we discuss various factors that influence people's preferences including application type, credential types, stakeholders as issuers, and architecture type. %across five distinct scenarios. %In this section, we present factors that 
\vspace{-2mm}
%\subsection{Resistance towards Digital Identifier}
%\subsection{Reevaluating Offline Experience}
\subsection{PHC Onboarding: Online Vs Offline}
\vspace{-2mm}
Participants identified the onboarding process as a key factor in managing PHCs. While some preferred a hybrid approach over a fully online system, others favored a fully online process for its convenience. Many drew parallels to how they opened their first bank accounts, emphasizing the importance of in-person verification. They described visiting the bank, presenting their IDs and passports to a bank official, and then receiving their account. Reflecting on that experience, participants indicated that PHCs, especially for financial services, should be issued in a similar way. One of the perceptions was the high risk of security of a fully online system where user would use their own browser at home to upload certain ids and passports to receive PHC compared to bank officials doing the same thing in their protected system. P22 said \textit{``There is a high risk with a fully online system where users would use their own browsers at home to upload certain IDs and passports to receive a PHC. Compared to that, having a bank official verify the information in their protected system feels much safer.''} 
%This sentiment highlights the perceived security benefits of involving intermediaries in the PHC issuing process. 
However, some highlighted the impracticalities of in-person verification as P18 mentioned \textit{``I live in west coast, now i need to travel to a location to have my irises scanned by the Orb device, which is not practical for me.''}


%Mental model: security perception Online vs. Direct Data Submission for verification
%#######"I think the benefit is that a lot of these details would be something that I would know about myself. So, for example, when I try to call the bank say, for example, I have an issue, and I call the bank they would have to verify that I'm the person who is the account holder. So in that case they would have to ask me verification questions, such as maybe my postcode or my date of birth or my full name or even in like, maybe my phone number. But any kind of personal information that would then tie back to me. And if say, for example, the benefit is that if someone else were to call in my case but they didn't know these details, and the bank wouldn't give out My any of my personal information, or the bank would know that my account is compromised, and then they could take quick action, maybe freeze the funds in my account. And yeah, so I think, like, Phc is important" (P3)
%######"I have done online healthcare with like telemed where they watch you on a camera. And you tell them, hey, I think I have a cold. I have a runny nose and a sore throat, and they say, Yeah, you have a cold I've done that before. I don't have any problem with that. They in this case they did not. I mean they they asked some verbal questions like, What's my address? What are the last 4 of my social security number. And that's that's information I don't mind giving out, because I know who's asking for it, and I know why they're asking for it. And I don't have any problem with that giving out the last 4 of your social. That's not a problem. Nobody can steal your entire social security number just from having a piece of it. However, my doctor's office when I go in person they have a microphone in the in the treatment room and the microphone records everything that I say. And it's recorded by AI computer. Who then trans poses that and puts it into my medical medical history, and I do not like that, because there are some things I tell my doctor. I don't want them to write down. And so, before I go into the treatment room, I ask the doctor to turn the AI off so that he has to manually write down the notes and have them put into my record. And what what I tell that doctor at that point is now personal. I don't feel like everything I've told him is being recorded. So that's my personality. I just don't like everything, because the more information that goes into something the more can be inappropriately used." (P8)


%particularly for high-stakes services like financial transactions.
%\fixme{Credential: resistance to phisycal ID carrying.  I remember interesting quotes like buying alcohol and some scenario people mentioned where carrying physical id all the time is cumbersome. then "Across border: Limitation in access/interoperability in verification method"} 
%\ayae{Unfortunetely, I observed this theme only in pilot study. Could we include them?: https://docs.google.com/spreadsheets/d/1PxN9fYU1CT5AuELanIjmtY6900U35gHn/edit?usp=sharing&ouid=102374555302641952762&rtpof=true&sd=true} \fixme{yes, please add the quote here, I also some across in my interviews as well} 
%\ayae{reflected on 1/9}

%Pilot P1 -- "But it's not a very convenient sometimes, because I have to take the physical id like in the Us. Like my driver's license or my passport. So it's not very convenient, and in China. We don't usually take our id with us, because, only when we need to like, go to a hotel or go to the Government department for some service. We we only take our id that time, but in the Us. I have to take my id yeah, every day."

%Pilot P2 -- "I think this information may be lost, or maybe stolen by other people. So it's a risk. Think the passport and the Id card. This this information may may be lost, or maybe this is the account."

%Main P8 -- "Everything is able to be stolen so. It might work in the short term until people that do bad things figure out how to steal it and use it. But you know. we came up with paper versions of authentication, like a driver's license, because we needed something to identify people. But now we realize that that's not as secure as it can be. So when you go to, maybe start a new job, you have to bring more than one form of identification to prove. So I I think that maybe in the beginning, using biometrics would feel secure, and it would be until it's been until a way is found to misuse it. just like a paper identification, like a driver's license."

\iffalse
\vspace{-2mm}
\subsection{Types of Application}
\vspace{-2mm}
% \fixme{this needs to expand a lot,  and one of the main results. we also have post survey for this result. move some visuals here and rewrite this } \fixme{if these figures are explained in later section, why did you add this paragraph here? this section is about application types. start with a outline like "participants had various preference across application types..... For instance, for X, Y , Z applications, .... ...."} 
Participants had various preferences across application types, reflecting context-specific needs in identity verification. 
%For instance, finance and healthcare services were commonly regarded as having a higher level of security and trust due to their sensitive nature. When we asked their perceptions regarding PHCs in healthcare scenario, P2 said \textit{"Still pretty similar to the banking example. Because, in the health context, it has specific regulations to protect data privacy, because it's still a high risk context. So I think it's my perception would be pretty similar to the banking service."} P17 highlighted similarities while also addressing nuanced differences in the sensitivity of the data, stating, \textit{"I feel similarly to the banking account topic. I would say that I feel even less concerned with these health applications, because it's also private information. But I don't think that it's as dangerous that someone has that information compared to my banking information."} Given the nature of finance and healthcare applications, 
%Participants expressed a positive attitude toward using PHCs, recognizing their value in enhancing security and privacy. \textit{"I think if PHCs hold, identification is secured. Because I think my personal data is not going anywhere. I think me using PHC in healthcare system is also the same with the financial. Because keeping my data secured is the main important thing."}
%Participants’ responses indicated their preferences are depending on each scenario where different type of applications are considered as service providers as seen in Figures \ref{fig:credential},  \ref{fig:issuer}, \ref{fig:architecture}. Results of preferences regarding credentials, issuers, and architecture will be addressed in subsequent sections. This section discusses the variations in participants' acceptance of PHCs observed across different scenario sessions. 
 %\fixme{not adequate, I don't see any preference regarding financial service and related quote, healthcare and related quote, govt and related quote and employment background and related quote, do add those. You added quotes for the less important ones, like social media and llm. Do keep those. I will decide later. add the quotes from other application type. you are supposed to explain similarities and difference on people's preference of PHC in different application types here.} \ayae{reflected on 1/9}
 Regarding government, finance and health services, most participants were positive about using PHCs. As P4 explained, \textit{"I think it'd be happy to go ahead and use PHC if it's something that you're applying for, like you said benefits or something like tax breaks, then you've got those factors to consider...probably because they already obviously hold a lot of information about you for working purposes, tax purposes, all that kind of thing. So if you're only providing verification, I think maybe not going to be a major concern apart from the usual. "}, they felt less concerned since the government already holds a significant amount of information about them. In the employment background check scenario, participants showed receptiveness to using PHCs, for instance, P9 recognized the non-invasive nature of data collection in PHC, saying, \textit{"I really don't love with job applications having to provide so much information before you even have the job. It feels very invasive, but if all I have to do is show my personal credentials, which come from something as simple as like a fingerprint or an iris scan. I really like that. That doesn't feel invasive."}
 In contexts like social media and LLM applications, participants viewed PHCs as unnecessary, favoring simpler verification methods like email or phone numbers. For instance, P5 noted \textit{"For social media apps, I really do not feel comfortable doing that...I would seriously consider not using it and use another app that does not ask for such information. First of all, they are regulated, but not like banks.
 %They have more freedom to do a lot of things that I may not agree with...They might use your information and your tweets or messages for a lot of other reasons, like training their large language models. Because of that, such institutions can use your information for many reasons are in the news business a lot of times, and I don't see any reason of providing PHC information for them.
 "} Moreover, P7 suggested bot's behavior prediction instead of PHC as a way to protect the core model in LLM scenario, saying \textit{"
 %I think there are ways that they can predict attacks and they can prevent it. They can design some triggers, so that before their core models being under attack by defaults, they can prevent that and shut down the accounts very soon...So I think it make more sense from the corporates to take those steps as opposed to asking us to provide them with our sensitive information, so that they core model would not be attacked.
LLM companies should proactively predict and prevent attacks by designing triggers to shut down compromised LLM account users, rather than requesting sensitive user information to issue PHC to protect core models."} In the same line P23 noted that- \textit{``I and my sister use chatGPT in a shared premium account because its costly to pay for 2 accounts. PHC means one person one account, so not working for us.''}
%This contrasting differences in trends between finance and healthcare applications and social media and LLM applications are also evident in the quantitative data.
 %Figures \ref{fig:credential}, \ref{fig:issuer}, \ref{fig:architecture} show similar distributions for finance and healthcare, as well as for social media and LLM applications.
 %P6 stated, \textit{"I love to talk, to, chatgpt, and bard about weird stuff, and I think a lot of people like to talk to them about weird stuff. And so I would like the option to not log in via biometrics, and have a lesser experience and not have my identity tied to it. So I personally, I would not like to have my identity tied to my chat Gpt, or my bard, or Gemini."}


%Trust with healthcare services
%######"Yeah, I'd be absolutely fine with that. I mean, here in the Uk, we have the Nhs, so we have kind of a system of like identity verification. We all have, like an Nhs number which is associated with us as a person. And so I think that data kind of already exists for healthcare for us. so it wouldn't make me feel nervous giving them any more information about me personally, because, yeah, they kind of already have it all." (P11)

%Distrust with LLM companies
%###### "Well, I already have privacy concerns with AI models like Chat Gpt. When I created a jet chat Gpt account, I created a new email address so that it didn't have any of my information from my current email address. So I use a different email address for Chat Gpt, and the reason I did that is because I don't know how the data is being used. The other problem I have with Chat Gpt is, if I give it personal information, and someone misuses that information. I'm now at risk. And so I've never given any personal information that could be used against me like I haven't given it my place of birth. I haven't given it the name of my 1st pet. I haven't given it, you know, the 1st car I ever drove, because those are security questions that are used when you call your bank and they have to identify you. So I'm I don't give it my social security number, you know. So it's really up to you to protect yourself. just like when you're walking down the street. If you see a situation that you think is dangerous. Why would you keep walking down the street? You would turn around and go the other direction so that you're not putting yourself in danger. And I think we have as humans. We have to be wary of chat Gpt type models because we don't know how that information is going to be used in the future, or how it's even being used now. In fact, the people that created jet chat gpt, they don't know what they're gonna do with all the information? They're just collecting a lot of information right now and trying to find out. " (P8)
\fi

\vspace{-2mm}
\subsection{Data Requirements to Issue Credential}
\vspace{-2mm}
% \fixme{Are the themes added in this section from the result of section 2. please incorporate themes from section 2 scenario based discussion. we also have post survey for this result. move some visuals here and rewrite this } \ayae{reflected}
Data requirement in issuing PHC credentials was one of the main factors. Participants generally categorized data requirements into three main types: \ding{202} Government-issued IDs, such as social security numbers, passports, and driver's licenses; \ding{203} Biometrics, including face, fingerprints, and iris; and \ding{204} Digital identifiers, such as phone numbers and email addresses. To grasp their preferences comprehensively, Figure \ref{fig:credential} presents quantitative results of credential preference from post-survey.
%(The questions are listed in Appendix \ref{post_survey}). 
For instance, a government-issued ID is the most preferred data to issue PHC across applications except social media and llm applications. Phone number is most preferred in llm application followed by Iris scan.

%\fixme{you need to explain the figure 5 Post-Survey Results: Credential Preference briefly here, what it represent, only showing figure is not enough, very bad practice considered by reviewers. just add a line to sum up the figure} \ayae{reflected}


\textbf{Familiarity with traditional (e.g. govt id)} 
Several participants highlighted their preference for their PHC associating with government-issued IDs as they were most familiar with this approach and perceived it as reliable.
%noting their trust in widely recognized and standardized forms of verification\fixme{it seems like you somehow only focused on trust aspect of physical id. think broader. this is very repetitive. people not only prefer govt id, they actually talked about physical form of id. then say more eprevalenly govt id}. 
To add some complexity, P11 mentioned the practicability of different types of government id- \textit{``Driver's license is pretty common, and  we usually bring that all the time. It's easy to bring and easy to take a photo and upload. A passport or a social security number is not the one that people usually bring. So if we need to verify that we 1st need to come back home and then search it, search them and then provide it. So it takes, additional steps. But a driver's license pretty easy.''} 
%I would probably prefer to use just a standard identity document like my passport or my driver's license or something official because then I would know that everyone else was using official documents. So it was all above board, and it was reliable. I feel like it's more reliable if we're using like official documents.

%\ayae{Comments focusing on the physical aspects were seen in the pilot, can I include them? >> quote from pilot study "Driver's license is pretty common, and you know we usually bring that all the time. It's easy to bring and easy to take a photo and easy to upload. That's the reason. A passport or a social security number is not the one that people usually bring. So if we need to verify that we 1st need to come back home and then search it, search them and then provide it. So it takes, additional steps. But driver's license pretty easy." }
%At the same time, P11 compared their preference to biometrics, stating \textit{"I trust it more because, for example, like my passport, it proves that I am me and that I am a resident in this country, whereas biometric data, although it is it? Is you? There's no kind of attached to it. It's giving you like more validity as a person." } 

\iffalse
\textbf{Ease of Biometric Verification}
\fixme{in first line mentioned a list of benefit participants mentioned about biometrics, like 
Biometrics is easier no need to remember credentials and if there is anything. For any themes first things to provide the overall result as a list and then add quotes that's interesting } \ayae{reflected}
Among the participants, some indicated a preference for biometrics, highlighting its significant convenience and ease of use. P4 expressed \textit{"I think maybe the biometric. Maybe iris scanning probably my preference. Seems like the easiest one to do. I think it'd be the ease of it, taking the selfie sort of thing and uploading that. And obviously it's seems kind of more fun, I would say, compared to maybe doing some of the other ones. It seems like pretty quick way to do it. And I go with that."} This participant underscored data minimization with biometric, noting \textit{"Because obviously, the government id is gonna have a lot of your personal information. So I think that's the thing we're trying to minimize here. So with the iris scanning I think it's just the minimal and still being secure."} 
P8 emphasized that biometrics eliminate the need for users to remember passwords, stating, \textit{"
Well, the benefits, of course, are simplification. I don't have to remember a password for everything. I can just use biometrics like my face, or a fingerprint, or something like that, or an eye retinal scan."}
%PX ------ As a biometric authentication method compared to like face id, which, although is very convenient, is not as easy or as friction free as fingerprint id. So that's 1 thing which I I really would like Apple to bring back. Patch id for the iphone. That's 1 personal comment. But in general I think they do perform well enough for me to not have any problems with them on a day to day basis.
\fi

\textbf{Sensitivity, Security, \& Efficiency Across Different Biometrics}
We found that participants favored biometrics due to their ease of use, no requirement to remember passwords minimized exposure of personal data, and their functionality as standalone credentials. Highlighting efficacy and privacy P4 stated - \textit{``government id is gonna have a lot of your personal information. biometrics here minimize data-just biomarker. like with iris or fingerprint, it's just the minimal and still being secure.''} 
We also found that individuals have varying preferences across biometrics types, such as fingerprints, facial recognition, selfie, and iris scanning. Firstly, participants considered fingerprints as less invasive and less sensitive.
%\fixme{add a quote that present why fingerprints are less  invasive. the quote you added before is not representative}
%as reflected in the following comments: \textit{"I feel like a fingerprint is definitely preferable. It just seems easier. It also seems less invasive. (P9)"}, \textit{"For me, and I understand that fingerprint a meaningless, it's just my personal preference. (P13)"} 
In contrast, facial recognition was discussed as a more sensitive biometric method, as illustrated by P5, \textit{"I would say fingerprint is fine, but face is too much. you can be identified in public settings in streets. If they have camera [surveillance], anyone can just trace your whereabouts."}  Regarding iris recognition, most participants expressed their views without having direct experience with this method. Notably, P8 said-
%discussed that the iris might represent a more secure biometric method, 
\textit{"iris verification is probably the more secure of all of the biometrics. Because I know that fingerprints can be regenerated, and facial recognition can be regenerated. But irises are not easily generated. The eye is a very complex organ in the body. So if I had a choice between the 3 forms of biometrics, I would choose iris, because I think it's the most secure."} In content of efficacy, P7 emphasized the efficiency of face compared to fingerprints, stating \textit{"I think facial recognition is better and more technologically improved, it is quick.''} On the same note, 
P23 highlighted challenges with their father fingerprint, noting that it had become difficult to identify due to the type of construction work they engaged in, causing the prints to blur over time.
%Biometric depends on devices
%###### "First, I use Pixel, Google, Pixel and Google Pixel uses fingerprints. At the back of the phone. So in order, one of the ways that you could identify yourself was to use the fingerprint reading... I switched to iphone. And then they also change the verification with the platform change as well. So because iphone they don't have like a fingerprint." (P7)
%###### "So as an iphone user, I use like on a on a Mac user, I use touch id and face id a lot in my like regular authentication practices. So I guess the benefit of touch, id or face id is the verification happens on device or on apple side, where they store your fingerprint or face data, and they only provide this as like a client service for all other apps or all the other services that want to verify your Id. So that way, it gives me a little bit of confidence that my biometric data doesn't leave like Apple's ecosystem, or outside of my device, and gives me that sense of security that I can give biometric access as a way to log in, because it's a very convenient way to log in across different services." (P1)
%###### "I think most of the Id checker platforms I've used or like anywhere that I've had to upload or any anywhere that needed to check my id or verify my id like they've mostly will ask for my face, because the phone that I use that's like it's an iphone 14 pro, so it doesn't have like that fingerprint option anymore. So you just do like a face id scam. So most places I've seen so far recently, I haven't come across any fingerprint verification. It's mostly been like they would make me take a photo of myself, and I have to fit my photo in a frame that they provide, and then they use the photo I provide of myself against the Id that I've provided to verify that I'm the person on the Id." (P3)

%Perception about biometric: face is faster than fingerprint 
%###### "I think I found with fingerprint recognition. Sometimes it can seem a little glitchy like there are times when I've been. You're inputting your fingerprint, basically. And it seems to get stuck. Or it won't fully be able to scan for some reason, whereas I found, with the facial recognition that seems to be more like user friendly. And faster as well. The facial recognition seems faster." (P13)

%Perception about biometric: it will change after certain period time
%###### "So if, for example. The only time I think the photo wouldn't match is say, for example, if the photo was really really old, say, when you were a child, and now you're 20 years older and you look very different. So I think that's the only time when maybe the software that's used to create like a match between the 2 is going to have difficulty, whereas, like with a fingerprint scan. it's fingerprint scan. " (P3)
%####### "Yeah, that's a very good question, because write the exact that that this time time frame I I use 2 different mobile phones. First, st I use Pixel, Google, Pixel and Google Pixel uses fingerprints. At the back of the phone. So in order, one of the ways that you could identify yourself was to use the fingerprint reading. So it wasn't I. I wasn't successful, as I mentioned, because of the passport and the issues with Robin Hood. and I think it took couple of weeks or even month, and I switched to iphone. And then they also change the verification with the platform change as well. So because iphone they don't have like a fingerprint." (P7)


%. So I use a stand on my desk, and I place my phone on it...So whenever I put my phone on the stand on my desk and I need to log into something, I don't need to reach my phone and use my fingers instead, the iPhone can read my face, and I would have a much quicker login, I would say that's the one of the reason that I would prefer."}
%\fixme{another code I saw "Perception about biometric: face is faster than fingerprint" this seems to capture efficiency aspect. add an interesting quote on this. Also about this "Perception about biometric: it will change after certain period time" - add a quotes} \ayae{reflected. pause the second theme which was found only in pilot study.}
\iffalse
\textbf{Simpler Verification with Digital Identifiers}
%Credential preferences varied across scenarios, with notable differences observed particularly in the social media and LLM application scenarios as shown in Figure \ref{fig:credential}. 
While government-issued IDs were the most dominant choice in other scenarios, phone numbers were the most preferred option in social media and llm application scenarios (Figure \ref{fig:credential}).
%these two scenarios. 
P6 described \textit{"If I have to require something for issuing PHC, I would rather it be the phone number. I would even just give them an email address, more anonymous than a phone number. For my personal use of AI, I don't feel a need at all for them to know who I am."} 
%Participants preferred to remain anonymous by using simpler verification methods through digital identifier credentials.
\fi

\textbf{Combination of Multiple Credentials}
Participants emphasized the importance of using multiple type of credentials to enhance security. P13 indicated this preference, \textit{combination of facial scan and fingerprint combination. If you're wanting to prove this is the person. Then it would make sense to have more than one biometric. I don't know if it's possible to fake somebody's. If there are completely two different biometrics, it would be more difficult to fake.''} In cases like financial service, many participants preferred PHC issuance based on both physical id (e.g. gov id) and biometric (e.g. iris).
%\textbf{Trade-off on Digital Identifiers} Participants also shared their preference with a trade-off on digital identifiers, noting both risks and advantages. P3 acknowledged the potential risks with \textit{"Email address is more likely leaked or stolen by other people. So I think it has many risks related to using an email address."} However, P3 preferred using an email address as a credential, supplementing by \textit{"But on the other hand, an email address will not leak my personal information too much. so I think, for the pros and cons, I like this."} 

\begin{figure}[!t]
    \centering
    \begin{subfigure}{0.48\linewidth}
        \centering
        \includegraphics[width=\linewidth]{Fig/credential_chart.png}
        \caption{Results of credential preference: Which types of credential would you prefer to use as personhood verification for each scenario? (Multiple selections were allowed.) }
        \label{fig:credential}
    \end{subfigure}
    \hfill
    \begin{subfigure}{0.48\linewidth}
        \centering
        \includegraphics[width=\linewidth]{Fig/issuer_chart.png}
        \caption{Results of issuer preference: Which type of issuer would you prefer to issue and manage your PHC for each scenario (Multiple selections were allowed.)}
        \label{fig:issuer}
    \end{subfigure}
    \vspace{-10pt}
\end{figure}


\vspace{-2mm}
\subsection{Stakeholder Types}
\vspace{-2mm}
% \fixme{Are the themes added in this section from the result of section 2. please incorporate themes from section 2 scenario-based discussion if there is anything interestingthere. we also have a post survey for this result. move some visuals here and rewrite this. moved some paragraphs from RQ1 which is suited here. Try to make coherence, remove repetition} \ayae{reflected}

\textbf{Control and Practicality: Preferences of PHC Issuers.}
Participants expressed varied levels of acceptance regarding the preferred issuers of PHC issuers. Across all application scenarios (Figure~\ref{fig:issuer}), government entities emerged as the most trusted issuers for the majority of participants, followed by nonprofit organizations (NPOs). P5 noted-- \textit{"Government is my preferred. IIf there are certain organizations are leveraging 3rd party organizations, they should be regulated and under government supervision. The least favored one is private companies without supervision, like commercial companies doing their own."} 
However, practical considerations influenced participants' views in certain domains, particularly social media and LLM applications, where private companies were rated as acceptable issuers by some. As P19 said \textit{``I don't see it happening where social media will involve govt vetted PHC. its just not practical. and truthfully, i don't want gov issued phc for social media, i am not fool to allow govt to another layer of surveillance.''} In contrast, a group of participants favored nonprofit organizations for domains like healthcare and social media, valuing their balance of trust and regulation --P10 stating \textit{``So NPOs I feel like they do have that government backing, they could be another trustworthy source, but not as intense as the govt. to balance it out.''}This contrast underscores the complex trade-offs individuals consider when evaluating trust, regulatory oversight, and practicality in selecting entities to issue PHCs. 
%== discussion sectionWhile government oversight remains the most trusted option, concerns about privacy, surveillance, and domain-specific practicality shape preferences for nonprofit and private entities in certain applications.
%It showed preference in managing PHC under government supervision and further explained that they did not extend the same recognition to private companies or organizations without governmental restrictions, which was 
%Interestingly, participants chose NPOs 
%particularly evident against social media platforms. P1 expressed their concerns, stating \textit{"I'm I don't think I'd be comfortable yet, at least, looking at like a social media organization or an organization that does not have governmental restrictions or things like that for sharing their data."} 
%This contrast highlights the complex considerations individuals have when evaluating trust and level of regulatory strictness of different entities for managing private data in PHC issuing process.

\iffalse
\textbf{Context-based Issuer Preference}
As shown in Figure \ref{fig:issuer}, the distribution of preferred issuers varied across scenarios, aside from the government being the most favored option. 
%For example, financial institutions were the second most preferred after the government, while private entities were more commonly preferred in the social media scenario.
In the social media and LLM scenarios, NPOs were selected more frequently than in other scenarios. As a reason for this preference, P16 expressed concerns about the use of data for profit-driven purposes, stating, \textit{"I think, mostly because of the information, like tracking and sharing situation that already happens. Like a social media or like tech company, that provides this for you, if one tech company were to be responsible for that, they would just aggregate so much personal data and also sell it because if they're for profit.  How else are they going to make money? So I think that's mostly."}

%P10 --- "So my first thought is, it seems, bizarre to have the government control social media. And then again, we talked about the selling of data with private companies. So I was more hesitant with that. Finances didn't make much sense, and neither did education. So NPOs I feel like they do have that government backing with their type, their credentials as an NPO. So they could be another trustworthy source, but not as intense as the Government, and it would be more people would be less hesitant to be and my data is going right into the government. That's why I picked NPO."

%\textbf{Trustworthiness of PHC Issuer} The issue of trustworthiness in PHC issuers was also discussed, with participants expressing doubts about the reliability of the entities managing their data. Some worried whether these issuers could be trusted to protect their personal information and maintain their privacy. P7 described \textit{"My only concern is, how do we trust the Phc Am I using it? Yeah, because if if they are the one entity who handles all these kind of information, and if they are collecting everything about us from the government. And whether this entity is considered like a NGO, or that's a governmental body, or that if that's a private sector, so these are like some nuances about how to define or how to build this kind of issuer, because we have these giant issuers like Equifax or like these giant companies who are handling credits or those kind of financial aspect of the US structure. And they're like giants. They have these kind of monopoly over how they handle things. Is it going to be something similar? If so, how people can trust them? I don't want to share my personal information anymore, with like some random website...But how the way that the issuer can be understood? I think that would be the decision factor for users to start engaging with the system."}
\fi
\textbf{Trustworthiness of Stakeholders}
%\fixme{make this para concise, you can trim the quote as well and shorten, doesn't need to have entire quote, can edit. this theme should not take more than a para} \ayae{reflected}
When discussing the current verification process, participants often indicated that there are organizations they can trust and others they cannot. 
For instance,  Some worried whether these issuers could be trusted to protect their personal information and maintain their privacy. P7 described these concerns and discussed how trust in issuers can be a deciding factor. \textit{"My only concern is, how do we trust the PHC am I using it? Because they are the entity who handles all these information, and they are collecting everything about us from the government...I don't want to share my personal information anymore, with some random website...But how the way the issuer can be understood? I think that would be the decision factor for users to start engaging with the system."}
%Considering the importance of stakeholder trust, we asked how or why they have trust with specific stakeholders. Based on their responses, 
Trustworthiness are associated with various aspects of stakeholders like regulations, customer support, past experiences with issues.
P5 explained his trust in banks comes from regulatory aspects: \textit{"They are being monitored by federal agencies. Their activities are monitored. They are under a lot of regulations. So there is a monitoring system that is tracking banks."} P4 emphasized her trust in banks with accessible customer support, which allows individuals to review the necessary steps and address any concerns they might have.
%P4 highlighted another aspect resulting in their trust in banks,  which is that the accessible customer support, stating \textit{"You know what sort of steps have taken so you could probably have a read of that, and any concerns that you might have. So you probably always have the opportunity that you could always speak to them first, because obviously they probably have agents available that you could speak to about any concerns. "} 
On the other hand, we observed that many participants expressed distrust towards social media platforms. P3 shared an experience of her social media account almost being hacked, 
noting \textit{"Someone tried to hack into my account...I've had people send me really dodgy links, which I knew instantly, that if I clicked on that link, it would mean that my account would get hacked as well."} 
%\fixme{make this paragraph concise} \ayae{updated}
%\fixme{following quotes and evaluation of trustworthiness not clear}\ayae{cut following quotes, no significant implications to RQ}
%Some participants elaborated on how they evaluate the trustworthiness of companies or stakeholders when they share their sensitive information in general. 
%P6 reflected \textit{"I guess if the company was suggested to me, or I was referred to the company through the services of like Capital One or a big company that's been around forever. That would give more validity to me."} 
%P7 provided insight from his own experience with a suspicious verification process, saying \textit{"I do a lot of research… What I did was I went and searched on the Internet. So I went on Reddit, and I looked for different keywords, and my conclusion was that it's a very standard process. It's not like the dealership wants to collect the social security for themselves. They wanted to do the background check, and they wanted to see if the information that they provided was matching with my information from the US. "}

%\textbf{Variability in Trust Among Individuals} While there was a general consensus on preferred entities for data trust, individual perspectives varied regarding which organizations were considered trustworthy. For example, P4 expressed distrust toward the government, commenting \textit{"Government is not kind of trying to protect most of the citizens’ privacy and security, but at the same time it’s going to give more power to the government."} They also showed relative trust in social media companies, explaining \textit{"They probably invest in the privacy and security part. So I’m kind of less concerned."} These contrasting viewpoints highlight how personal beliefs and experiences shape individual trust preferences, underscoring the diversity of opinion in stakeholder trust.
\vspace{-2mm}
\subsection{Architecture Types}
\vspace{-2mm}
%\fixme{before you start any subsection or section add 1-2 lines to summarize. for example people had a range of preference on architectures. Architecture preference: decentralized improve security of centrailized data storage, Architecture preference: decentralized mitigates privacy concerns, Architecture preference: decentralized-users can choose preferred issuers, Architecture preference: decentralized and sector-based, Architecture preference: clear regulation/policy for decentralized,Architecture preference: centralized with biometric and govt. ID Architecture preference: centralized for simplicity Architecture preference: middle - decentralized but oversight by gov. First summarize these in 2 lines and then the interesting themes you discussed below} \ayae{reflected} 
We observed diverse preferences for PHC architectures, with participants highlighting decentralized models for improving centralized data security and enabling user choice of issuers. Others preferred the simplicity of centralized systems, and hybrid models blending decentralization with government oversight.
 \begin{figure}[!t]
 \vspace{5pt}
	\centering
	\includegraphics[width=0.6\linewidth]{Fig/architecture_chart.png}
 \vspace{-15pt}
	\caption{Results of architecture preference: Which type of system would you prefer to issue and manage your PHC for each service provider?%\fixme{caption needs to be self explanatory, keep consistent graphics and color for all graphs}
    }
\label{fig:architecture}
\vspace{-15pt}
\end{figure}

\textbf{Decentralized vs. Centralized}
Participants highlighted the risks of centralized storage, preferring decentralization for its enhanced security. P8 stated \textit{"
%Definitely decentralized because 
you've created a wall between any problems with data theft, with a centralized data storage. They only have to breach the walls of one castle with decentralized. They have to breach the walls of two different castles, which makes it a lot more complex and most people won't. Most people won't go through the trouble of doing that. 
%Like, I said, if you're really dedicated and you will.
"} 
%\textbf{Simplicity of Centralized Architecture}
In contrast, participants who expressed a preference for a centralized approach highlighted its simplicity and ease of use. P21 stated \textit{``I feel like central authority can respond to any threat aftermath more effectively for centralize issuance and I guess central, like if gov is issuer then it will have global legitimacy, i can use it for customs.''}
%P4 stated \textit{"They both got benefits. But because the ease of simplicity maybe the single PHC issuer probably. So you've got the different sort of providers, the health and the the financial and you've got the trustworthy sort of should in this case, the government which can be used for the multiple applications. So you got the user, then it can use that same Id and on multiple platforms, just making ease of use."} This perspective underscores the appeal of centralized systems for their ability of identity management across multiple platforms while maintaining trust through a single reliable issuer.

\textbf{In Between Centralization and Decentralization}
Some participants expressed preferences for issuers with a balanced approach of centralized and decentralized systems. 
%P9 explained \textit{"For instance, one participant shared, "My initial reaction is that I would prefer centralized because it just seems easier. It seems a little safer that there's just one PHC I'm getting and it also just seems easier for me as the user. But I now only have one PHC to remember or hold on to and pass on to various people. But then it's a little tricky, because I see the pros of decentralized that it might be a little bit more secure because I now have, if someone hacked or stole, or whatever my PHC, they don't have access to everything if it's decentralized. So they might only get my social media or like banking or healthcare services would be secure. So I think centralized sounds easier, but decentralized sounds better."}
P1 highlighted a mixed preference, favoring decentralization while emphasizing the government's oversight, \textit{"I think it's sort of in the middle. I think decentralized would be the right way to go, because. as expecting the Government to have enough resources to keep verifying Phcs would be hard...But the verification, like the issuing authority, is still a government, and the Government still has oversight on how these verification systems work."} P3 on the other hand mentioned having decentralized system to issue PHC with having govt as one of the trusted issuers to sign the phc. 

\iffalse
\textbf{Design preferences of architecture.} \tanusree{we can potentially cut this theme if needed}
Design preferences reflect diverse priorities over accessibility, trust, and security. P1 emphasize the need for accessible platforms that ensure users can verify their credentials anytime and anywhere, saying \textit{"Accessible for people to log in at any time or any place across the world for their online verification and need education programs to prevent social engineering in such new verification system
%Because if these platforms go down and verifications are stopped, it would be a very, very big impact for people. It should be like the uptime and how well they are accessible for consumers.
"}
%Trust is also highlighted through the expectation that PHC issuers should be government operated or supervised for credibility, \textit{"I would expect the PHC issuer to be either part of the government or directly supervised by the Government. (P5)"} 
Many also emphasized for secure algorithms with valid time constriant for PHC. 
%education programs to prevent social engineering are essential for enhancing security, P2 saying, \textit{"For example, protect the algorithm, the algorithm make the system more robust. And also there's a lot of like education training things to prevent social engineering."}. 
In the same line, P23 mentioned the integration of number of threshold for factors in verification to allow people choice what data and how many data they would like to provide to have their phc issued. This demonstrate people' varying level of expectation for usability and security. 

%recommended to provide layered protection during interactions with service providers. P6 described \textit{"You will need a 2-factor authorization. Whenever you use your face or your fingerprint to log into the app, you should simultaneously get a code on your phone that only lasts for 2 to 5 min. So that you're not just relying upon the biometric data, you have to have the phone that you signed up with."}

%\fixme{summarize these with adding a quote of valide time constraints Design preference: adding valid time constraints Design preference: third-party commitments Design preference: encrypted credentials Design preference: checking organizations Design preference: transparency-what information is going to be shared Design preference: segregating database-preventing access to irrelevant or sensitive information} \ayae{summarized design preferences. (Above lists are from pilot study)} \fixme{where did you summarized? is it "accessibility, trust, and security" ? this is so high level, need to be specific like adding valid time constraints, encrypted credentials, segregating database-preventing access, etc. its fine to add pilot study results}
\fi
\vspace{-2mm}
\section{RQ3: Design Suggestions for PHC}
\vspace{-2mm}
%\subsection{Participants' Needs for PHC Designs}
Participants expressed specific needs for PHC systems, primarily focusing on enhancing security and trust while accommodating diverse needs. From the design sessions, we identified the following themes.
%P4 preferred a decentralized approach, stating, \textit{"I prefer the decentralized version just because you’re not giving all the power or like responsibility to government, although government might have all the information."} 
%Additionally, transparency was identified as a critical requirement. 
%P4 also noted, \textit{"I think any of this institute, if they take this responsibility, they need to be regulated about how they’re gonna handle user privacy security. They should be transparent about their process."} This emphasis on transparency and accountability highlights the need for PHC issuers to demonstrate their commitment to protecting user privacy and security as trusted issuers.

%\section{RQ3: Participants' Suggestions for PHC Designs}
%Participants suggested several design improvements for PHC systems. 
%\tanusree{do we only have these two design themes? we discussed more before} \ayae {}


\textbf{Design Theme 1: Time-bounded Credential for Privacy}
%\tanusree{ will add the user sketch here.}
Some users expect credentials to verify personhood with the least amount of personal data, avoiding detailed personal or biometric data collection for a certain period. Another expectation in the same line was the portability of preferred credentials if they can be used across platforms without re-verifying frequently. To address the amount of data and period of such data collection that has been used for issuing credentials, some features or design concepts were -- limited validity period, proof without storage or pseudonymization of the data when storing it. P21 mentioned - \textit{``I want a credential that works like a trusted pass—valid only for a set time (e.g., 30 days, 6 months, or 1 year), with reminders before it expires. It should prove I am a real person without storing my sensitive details or tying them to my real-world identity, like some level of anonymity. Just give me the freedom to be verified without being exposed.'' }, which is depicted in Figure \ref{fig:P21}.
% \begin{figure}[!t]
%	\centering
%	\includegraphics[width=\linewidth]{Fig/sketch_P21_area.png}
%	\caption{Illustration to depict time-bounded credential with retention and expiration date based on different data types}
%\label{fig:P21}
%\end{figure}


\textbf{Design Theme 2: Sensitivity-based Usable Credential Choice for PHC}
Multiple participants expressed that their preferences are sensitivity-dependent as presented in Figure \ref{fig:P3}. They suggested a design to incorporate choices for end users based on their perceived level of security needs across services in healthcare, finance, social media, etc.
%P14 considered system designs based on the trustworthiness of PHC issuers. Scenarios can be categorized into two types: when issuer is trustful and when it is not trustful.
%\fixme{i don't understand what it mean?}. 
%P4 suggested that in cases where the PHC issuer is untrustworthy, additional verification or oversight by the government should be implemented. \fixme{i don't understand what it mean?} 
%\ayae{Revisiting P4's transcripts, his point is preference to additional verification when interacting social media. We can skip this quote in this section.}
%For instance, P14 suggested using government-issued IDs or facial recognition as credentials for trustworthy PHC issuers, whereas fingerprints or iris scans would be preferable for untrustworthy issuers. 
%\fixme{i am struggling to understand why you present all the narrative about trustworthiness when even its so clear that its about sensitivity of the application area } \ayae{I agree with that, I was too much highlighted the trustworthiness of PHC issuers. Let me comment out this part.}
To illustrate the concept, P3, P4 provides a design where she made a choice of government-issued id or face to obtain a PHC for services in healthcare and finance
%to tailor the choice of credentials depending on the trustworthiness of the service provider, stating that 
In her words- \textit{"I think it very much depends on the scenario. For example, places where which are like the most sensitive. Creating a bank account or an account on a government website, or with a healthcare provider. I think, in terms of financial security, or on the government side, my preferred would be a face scan or a video call to verify my face and uploading a doc like my government issued residence permit, or my passport. Whereas in cases which are doesn't have that much of a security concern. So, having an account on ChatGPT, or a social media account, I would be okay with like fingerprint scan or iris scan."} She considers fingerprint and iris are less sensitive since face scan can potentially reveal one's identity, stating \textit{"If I scan my whole face, you can figure out my identity. I think the iris, fingerprint scanning is poses less of a risk, because it's only scanning your iris''}
%I'm sure there is some kind of system that could figure out who I am and other personal information about myself... But I think the iris scanning is poses less of a risk, because it's only scanning your iris... I think fingerprint scan, or iris scanning is probably like the better way to identify a person than having to scan their whole face or having or asking them to upload an entire id. "}
%\fixme{add this participants rationale why she considered iris scan / fingerprint less sensitive than face/video / id}\ayae{updated}
%\fixme{you are contradicting your earlier statement, read the theme again to understand what you are presenting here}\ayae{updated}

\begin{figure*}[!h]
 \begin{subfigure}{0.48\textwidth}
     \centering
     \includegraphics[width=0.8\textwidth]{Fig/sketch_P21_area.png}
     \captionsetup{width=\textwidth, font=footnotesize} 
     \caption{Illustration to depict time-bounded credential with retention and expiration date based on different data types}
     \label{fig:P21}
 \end{subfigure}
 \hfill
 \begin{subfigure}{0.48\textwidth}
 \centering
     %\raisebox{0.2cm{
     \includegraphics[width=0.8\textwidth]{Fig/sketch_P3_area.png}
     %}
     \captionsetup{width=\textwidth, font=footnotesize} 
     \caption{Illustration of choice for users to choose various data requirements for PHC issuance for different applications}
     \label{fig:P3}
 \end{subfigure}
 \hfill
 \begin{subfigure}{0.48\textwidth}
 \centering
     %\raisebox{0.5cm}{
     \includegraphics[width=0.8\textwidth]{Fig/sketch_P2.png}
     %}
     \captionsetup{width=\textwidth, font=footnotesize} 
     \caption{Illustration to depict comprehensive visually interactive human check with video chat for humanness cues, environment check}
     \label{fig:P2}
 \end{subfigure}
 \hfill
 \begin{subfigure}{0.48\textwidth}
     \centering
     %\raisebox{-2cm}{
     \includegraphics[width=0.8\textwidth]{Fig/sketch_P13.png}
     %}
     \captionsetup{width=\textwidth, font=footnotesize} 
     \caption{Illustration to depict periodic biometrics, dynamic authentication, geo-restricted access}
     \label{fig:P13}
 \end{subfigure}
\hfill
 %\bigskip
 \begin{subfigure}{0.48\textwidth}
 \centering
     \raisebox{1cm}{
    \includegraphics[width=0.8\textwidth]{Fig/sketch_P9_area.png}
     }
     \captionsetup{width=\textwidth, font=footnotesize} 
     \caption{Illustration to depict single issuer system with the government}
     \label{fig:P9}
 \end{subfigure}
 \hfill
 \begin{subfigure}{0.48\textwidth}
     \centering\includegraphics[width=0.8\textwidth]{Fig/sketch_P7_area.png}
     \captionsetup{width=\textwidth, font=footnotesize} 
     \caption{Illustration to depict decentralization for PHC governance}
     \label{fig:P7}
 \end{subfigure}
    \caption{Design suggestions for PHC: each represents the following design themes (a) Time-bounded credential for privacy; (b) Sensitivity-based usable credential choice for PHC; (c) Comprehensive visually interactive
human check; (d) Mitigate PHC misuse; (e)(f) Distribute power across issuer: decentralized vs centralized. }
    \label{fig:four_sketches}
\end{figure*}

% \begin{figure}[!t]
%	\centering
%	\includegraphics[width=\linewidth]{Fig/sketch_P3_area.png}
%	\caption{Illustration of choice for users to choose various data requirements for PHC issuance for different applications}
%\label{fig:P3}
%\end{figure}

\textbf{Design Theme 3: Comprehensive Visually Interactive Human Check.}
Another theme comes up where participants explain how \textit{``video chat''} offers several advantages in preventing social attacks during personhood credential issuance by triangulating real-time interaction, visual verification, and environment check. As P2 said \textit{``video chat could be a way to avoid social attacks in PHC issuing. Yes. this way service provider would ask you a lot of questions to make sure you are not controlled by someone else, and they would ask to move my camera, take videos of my whole room to make sure there's no one else beside me. And of course, it's it would. It would not be a 100\% thing, all the other things [only face, ids] are still the old stuff. This could be a new step to developing PHC algorithm.''} P2 further emphasized how human interaction is still a key when stepping into new technological innovation where in the issuance process, trained staff can assess subtle cues and inconsistencies that automated systems might miss, potentially detecting social engineering attempts or ask personalized questions based on the user's responses and behavior, making it difficult for attackers to prepare scripted answers (Figure \ref{fig:P2}). 

% \begin{figure}[!t]
%	\centering
%	\includegraphics[width=\linewidth]{Fig/sketch_P2.png}
%	\caption{Illustration to depict comprehensive visually interactive human check with video chat for humanness cues, environment check}
%\label{fig:P2}
%\end{figure}


\iffalse
\begin{figure}[!t]
 \vspace{5pt}
	\centering
	\includegraphics[width=\linewidth]{Fig/sketch_p1.png}
 \vspace{-15pt}
	\caption{Sketch by P1: Government involvement in PHC Issuers’ system}
\label{fig:sketch_p1}
\vspace{-15pt}
\end{figure}


\begin{figure}[!t]
 \vspace{5pt}
	\centering
	\includegraphics[width=\linewidth]{Fig/sketch_p7.png}
 \vspace{-15pt}
	\caption{Sketch by P7: Blockchain-based Issuer}
\label{fig:sketch_p7}
\vspace{-15pt}
\end{figure}

\begin{figure}[!t]
 \vspace{5pt}
	\centering
	\includegraphics[width=\linewidth]{Fig/sketch_p13.png}
 \vspace{-15pt}
	\caption{Sketch by P13: Different Layer of Security}
\label{fig:sketch_p13}
\vspace{-15pt}
\end{figure}

\begin{figure}[!t]
 \vspace{5pt}
	\centering
	\includegraphics[width=\linewidth]{Fig/sketch_p3.png}
 \vspace{-15pt}
	\caption{Sketch by P3: Context-dependent Preference}
\label{fig:sketch_p3}
\vspace{-15pt}
\end{figure}


\begin{figure*}[!t]
	\centering
	\includegraphics[width=\linewidth]{Fig/sketch_all.png}
	\caption{Sketches from Design Session \fixme{each subcaption needs to be self explanatory}}
\label{fig:sketch_all}
\end{figure*}
\fi


% \begin{figure}[!t]
%	\centering
%	\includegraphics[width=\linewidth]{Fig/sketch_P13.png}
%	\caption{Illustration to depict system design to ensure primary PHC users is using the credential with periodic biometrics, dynamic Authentication, and geo-restricted access}
%\label{fig:P2}
%\end{figure}

\textbf{Design Theme 4: Mitigate PHC Misuse: Periodic Biometrics, Dynamic Authentication, and Geo-Restricted Access.}
Participants suggested enhanced key management for the PHC system to address potential misuse, such as identity selling, credential sharing, and unauthorized account setup (Figure \ref{fig:P13}). Suggestions included periodic biometric verification (e.g., facial recognition or fingerprint scans during login) or random re-authentication to ensure that only the credential holder can access the system (P6, P8, P13, P18, P23). P13 highlighted the risk of credential sharing during onboarding, where users may rely on friends or family for setup assistance, potentially leading to unauthorized access. To mitigate this, P13 proposed combining biometrics with dynamic authentication methods, such as time-sensitive push notifications. Additionally, P13 suggested implementing geo-fencing to restrict access from unfamiliar locations or devices, reducing the risk of misuse if credentials are leaked. These suggestions aim to ensure that PHCs are used securely and by the intended user across different services.


% \begin{figure}[!t]
%	\centering
%	\includegraphics[width=\linewidth]{Fig/sketch_P7_area.png}
%	\caption{Illustration to depict decentralization for PHC governance}
%\label{fig:P2}
%\end{figure}

\textbf{Design 5: Distribute Power Across Issuer: Decentralized vs Centralized} 
Participants (P5, P9, P10) frequently highlighted the importance of a single issuer for PHC, prioritizing a unified point of trust. For example, P10 suggested the government as the primary issuer for various services, arguing that involving third parties, such as insurers in health contexts, complicates credential management (Figure \ref{fig:P9}). 
Conversely, others (P1, P7, P11) supported a multi-issuer approach, leveraging blockchain-based systems to enable decentralized storage and sector-specific distribution of power, addressing trust and governance concerns in PHC. 
%As illustrated in Figure \ref{fig:sketch_all} (b), this approach proposes a decentralized infrastructure where multiple entities sync. 
The main design concept focus on ensuring that any misuse of data by one entity would be immediately flagged by others and distributed control among different issuers when included as decentralized issuance systems. P7 illustrated the concept by 
stating \textit{"The important part is, they will have the same information all at once and they are synced together. So someone is uploading their document or info, all these companies are going to have it in sync together, so that if there is any misuse of information is going on here, the other companies would get to know it."}  (Figure \ref{fig:P7})

%Sure. there is always going to be a risk of companies getting together and misusing it. But we in this case, we're just reducing the risk of those kind of data misusage. So that's basically, what I'm trying to do is mimicking a blockchain structure so that everyone has the information in real time, and everyone has a copy of it. So that's what I'm trying to achieve here.
%\fixme{add a lien what does it mean by sector based decentralization}\ayae{updated}
%This approach underscores the value of decentralization in creating a more resilient and transparent system, reducing the reliance on any single entity and building greater trust in the PHC issuer.
%======= will think later-----Sector-based decentralization is also proposed as a system where responsibilities are distributed across multiple entities, with each entity specializing in a specific sector (e.g., social media, finance). Instead of a centralized authority handling all data, each sector operates only relevant data to its domain, reducing the risk of data monopolization and excessive data exposure.

%\textbf{Design Theme 6: Having Single PHC Issuer (Emphasized on Government).}
%Participants often (P5, P9, P10) emphasize single issuer preference for PHC to maintain a single point of trust. To illustrate this, P10 suggested government to be the significant issuer for different services where involving third parties like insurance provider in health context can be cumbersome in the credential handling process. Simialrly, P5 conceptualize PHC issuer to be directly from government or supervised by government, exception where tech company might handle PHC 

%two-factor authentication (2FA) for the use of PHC. P8 has security concerns of PHC being hacked and described 2FA as a simple and least expensive way to secure his account, noting \textit{"The simplest solution is to do 2FA...This can be your text message. And then, we could do or email. So that will unlock your account. That's probably the least expensive way to handle it to use a 2FA service."} P13 indicated the combination of 2FA and biometrics would strengthen security -  \textit{"I'd just be concerned that once a person's PHC has been set up that they could then like set up accounts for other people. It would at the moment I mentioned, I use the Google Authenticator. When you log into a web and to a site it asks for a number. So I go my Google authenticator, and if that was combined with the biometrics, maybe that would work."} as explained in Figure \ref{fig:sketch_all} (c). By introducing another authentication process, end users can rest assured that the PHC user is who they claim to be.

\iffalse
\textbf{Design Theme 1: Government involvement in PHC issuer's system} 
Several participants mentioned government involvement as part of the ideal structure for PHCs. P1 expressed an expectation for some form of government involvement to ensure the reliability of the PHC issuer \textit{"I'd expect my PHC issuer and the government to work together to provide me sort of an id, and their data would be stored, I guess, more on a Phc issuer side that under government. But I would expect both of these organizations to work together."}, along with a sketch as shown in Figure \ref{fig:sketch_all} (a). 

\textbf{Design Theme 2: Decentralization} One suggestion focused on \textbf{blockchain-based issuer} to address concerns of trust and governance in PHC. As illustrated in Figure \ref{fig:sketch_all} (b), this approach proposes a decentralized infrastructure where multiple entities sync. This design mimics blockchain principles, ensuring that any misuse of data by one entity would be immediately flagged by others, stating that \textit{"The important part is, they will have the same information all at once and they are synced together. So someone is uploading their document or their information, all these companies are going to have it in sync together, so that if there is any misuse of information is going on here, the other companies would get to know it. Sure. there is always going to be a risk of companies getting together and going to misuse it. But we in this case, we're just reducing the risk of those kind of data misusage. So that's basically, what I'm trying to do is mimicking a blockchain structure so that everyone has the information in real time, and everyone has a copy of it. So that's what I'm trying to achieve here."} This approach underscores the value of decentralization in creating a more resilient and transparent system, reducing the reliance on any single entity and building greater trust in the PHC issuer.

\textbf{Design Theme 4: Different Layer of Security}
Some participants preferred to have two-factor authentication (2FA) for the use of PHC. P8 has security concerns of PHC being hacked and described 2FA as a simple and least expensive way to secure his account, noting \textit{"The simplest solution is to do 2FA...This can be your text message. And then, we could do or email. So that will unlock your account. That's probably the least expensive way to handle it to use a 2FA service."} P13 indicated the combination of 2FA and biometrics would strengthen security - 
\textit{"I'd just be concerned that once a person's PHC has been set up that they could then like set up accounts for other people. It would at the moment I mentioned, I use the Google Authenticator. When you log into a web and to a site it asks for a number. So I go my Google authenticator, and if that was combined with the biometrics, maybe that would work."} as explained in Figure \ref{fig:sketch_all} (c). By introducing another authentication process, end users can rest assured that the PHC user is who they claim to be.
\fi

\section{Discussion}
\section{Discussion and Future Work}\label{sec:discussion}
This paper pioneers the novel approach of selective response, showing that withholding responses can be a powerful tool for GenAI systems. By opting not to answer every query as accurately as it can---particularly when new or complex topics emerge---GenAI can encourage user participation on community-driven platforms and thereby generate more high-quality data for future training. This mechanism ultimately enhances GenAI's long-term performance and revenue. From a welfare perspective, our results indicate that such selective engagement can also benefit users, leading to better solutions and increased overall satisfaction. Since this work is the first to address selective response strategies for GenAI, numerous promising directions remain for future research; we highlight some of them below. 

First, from a technical standpoint, all of the results in this paper rely on Assumption~\ref{assumption: data lip}, involving the lipshitz condition of the accuracy function and the sensitivity parameter $\beta$. Future work could seek to relax this assumption. Furthermore, our constrained optimization approach in Subsection~\ref{sec: welfare constrained revenue maximization} could be extended to approximate the optimal (continuous) strategy instead of the optimal discrete strategy.

Second, our stylized model adopts the simplifying---though unrealistic---assumption that only a single GenAI platform exists. Admittedly, this makes it easier to focus on the idea of selective responses, and indeed, this assumption is pivotal in keeping our analysis tractable. Future research could explore scenarios with multiple GenAI platforms and human-centered forums. In such settings, one platform's selective response might redirect users not only to forums but also to competing GenAI platforms, leading to the tragedy of the commons \cite{hardin1968tragedy}: Although all GenAI platforms benefit from fresh data generation, none may choose to respond selectively if it means losing users to competitors. 

Third, we assumed Forum behaves non-strategically. In reality, human-centered platforms often monetize their data by selling it to GenAI platforms, adding a further layer of strategic interaction for GenAI. Moreover, data transfer between the platforms can form the basis for collaboration: GenAI could employ selective response to bolster Forum content creation, and Forum could, in turn, attribute that content to GenAI for subsequent use in retraining.


%Third, we make the (again) simplifying assumption that Forum is non-strategic. However, in practice, human-centered platforms can sell their data to GenAI platforms. This adds additional considerations for GenAI. Furthermore, data transmission between the platforms can also become the basis for collaboration: GenAI can use selective response to ensure enough content is generated in Forum, and Forum could provide the data attributed to this mechanism back to GenAI. 


%Second, this paper makes the simplifying yet unrealistic assumption of the existence of one GenAI platform. Indeed, this simplifies many aspects and allows us to analyze selective responses. Future work could address the data generation process with more than one GenAI platform and possibly several human-centered forums. In such a case, selective response of one GenAI platform can either drive users to forums or to other GenAI platforms; thus, we might face a tragedy of the commons situation~\ref{hardin1968tragedy}, where all GenAI platforms are interested in fresh data generation but none volunteer to selectively respond and lose users. 

%This paper examines the competition between a generative AI platform and human-based platforms, challenging the assumption that always providing answers is optimal. We analyzed the impact of withholding answers on GenAI's revenue and developed an efficient approximately optimal algorithm for this purpose. We further explored how withholding affects users, showing that it can lead to better outcomes compared to always answering. Specifically, we demonstrated that withholding can Pareto-dominate this strategy and derived the necessary and sufficient conditions for that. Finally, we proposed a second approximately optimal algorithm that maximizes GenAI's revenue while ensuring users are better off than when GenAI answers all queries.

%On a more conceptual level, our model assumes that GenAI’s data comes solely from the competing platform (Forum). Future research could explore a scenario where GenAI can purchase additional data from a third party. This extension could provide valuable insights into the interplay between withholding answers and data purchasing, and whether these two strategies can complement each other or must be traded off.
\section{Conclusion}
Software development is increasingly conceived as a collaboration activity between developers and AIs. Indeed, IDEs already implement features to enable interactive development, with AI suggesting implementations that are reused by developers.

Although multiple studies show this interaction can be successful, there is still limited understanding of how the models must be configured and used in the context of code generation tasks. This study addresses this gap, systematically investigating the impact of several key parameters, including the repeated submission of a prompt to accommodate for the non-deterministic nature of the models.

Our study reveals several key findings about the usage of ChatGPT. In particular, we discovered how creativity, although up to a limited extent, is useful to increase the range of methods whose code can be generated correctly. A major role is played by parameter top-p, which is commonly underrated, and instead has a major impact on the correctness of the results, with lower values producing better results. Finally, prompts should be submitted multiple times, with $5$ repetitions combined with a temperature of $1.2$ resulting in an effective configuration in our experiments.  

Future work concerns two main research directions. One is about replicating this experiment with other AI assistants, to validate our findings in multiple contexts. The second research direction concerns finding strategies to deal with the need to submit the same prompt multiple times to obtain a useful result, and thus developing approaches able to select or merge multiple responses automatically. 

\begin{acks}
This work is partially supported by the Alfred P. Sloan Foundation (G-2021-16745) and the Natural Sciences and Engineering Research Council of Canada (RGPIN-2018-04470).
\end{acks}

\balance
\bibliographystyle{ACM-Reference-Format}
\bibliography{references}

\end{document}
