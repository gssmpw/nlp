% This must be in the first 5 lines to tell arXiv to use pdfLaTeX, which is strongly recommended.
\pdfoutput=1
% In particular, the hyperref package requires pdfLaTeX in order to break URLs across lines.

\documentclass[11pt]{article}

% Change "review" to "final" to generate the final (sometimes called camera-ready) version.
% Change to "preprint" to generate a non-anonymous version with page numbers.
% \usepackage[review]{acl}

\usepackage[final]{acl}
% Standard package includes
\usepackage{times}
\usepackage{latexsym}
%\pagestyle{plain}

\usepackage{comment}
\usepackage{hyperref}
\usepackage{url}
\usepackage{graphicx} % Required for inserting images
\usepackage{multirow}
\usepackage{ragged2e}
\usepackage{booktabs}
\usepackage{adjustbox}
\usepackage{todonotes}
\usepackage{subcaption}
\usepackage{amsmath}
\usepackage{xcolor}
\usepackage{adjustbox}
%\usepackage{biblatex}

% For proper rendering and hyphenation of words containing Latin characters (including in bib files)
\usepackage[T1]{fontenc}
% For Vietnamese characters
% \usepackage[T5]{fontenc}
% See https://www.latex-project.org/help/documentation/encguide.pdf for other character sets

% This assumes your files are encoded as UTF8
\usepackage[utf8]{inputenc}

% This is not strictly necessary, and may be commented out,
% but it will improve the layout of the manuscript,
% and will typically save some space.
\usepackage{microtype}

% This is also not strictly necessary, and may be commented out.
% However, it will improve the aesthetics of text in
% the typewriter font.
\usepackage{inconsolata}

%Including images in your LaTeX document requires adding
%additional package(s)
\usepackage{graphicx}

% If the title and author information does not fit in the area allocated, uncomment the following
%
%\setlength\titlebox{<dim>}
%
% and set <dim> to something 5cm or larger.

\title{Multi-Novelty:Improve the Diversity and Novelty of Contents Generated by Large Language Models via inference-time Multi-Views Brainstorming}

% Author information can be set in various styles:
% For several authors from the same institution:
% \author{Author 1 \and ... \and Author n \\
%         Address line \\ ... \\ Address line}
% if the names do not fit well on one line use
%         Author 1 \\ {\bf Author 2} \\ ... \\ {\bf Author n} \\
% For authors from different institutions:
% \author{Author 1 \\ Address line \\  ... \\ Address line
%         \And  ... \And
%         Author n \\ Address line \\ ... \\ Address line}
% To start a separate ``row'' of authors use \AND, as in
% \author{Author 1 \\ Address line \\  ... \\ Address line
%         \AND
%         Author 2 \\ Address line \\ ... \\ Address line \And
%         Author 3 \\ Address line \\ ... \\ Address line}
%\author{
%  \textbf{First Author\textsuperscript{1}},
%  \textbf{Second Author\textsuperscript{1,2}},
%  \textbf{Third T. Author\textsuperscript{1}},
%  \textbf{Fourth Author\textsuperscript{1}},
%\\
%  \textbf{Fifth Author\textsuperscript{1,2}},
%  \textbf{Sixth Author\textsuperscript{1}},
%  \textbf{Seventh Author\textsuperscript{1}},
%  \textbf{Eighth Author \textsuperscript{1,2,3,4}},
%\\
%  \textbf{Ninth Author\textsuperscript{1}},
%  \textbf{Tenth Author\textsuperscript{1}},
%  \textbf{Eleventh E. Author\textsuperscript{1,2,3,4,5}},
%  \textbf{Twelfth Author\textsuperscript{1}},
%\\
%  \textbf{Thirteenth Author\textsuperscript{3}},
%  \textbf{Fourteenth F. Author\textsuperscript{2,4}},
%  \textbf{Fifteenth Author\textsuperscript{1}},
%  \textbf{Sixteenth Author\textsuperscript{1}},
%\\
%  \textbf{Seventeenth S. Author\textsuperscript{4,5}},
%  \textbf{Eighteenth Author\textsuperscript{3,4}},
%  \textbf{Nineteenth N. Author\textsuperscript{2,5}},
%  \textbf{Twentieth Author\textsuperscript{1}}
%\\
%\\
%  \textsuperscript{1}Affiliation 1,
%  \textsuperscript{2}Affiliation 2,
%  \textsuperscript{3}Affiliation 3,
%  \textsuperscript{4}Affiliation 4,
%  \textsuperscript{5}Affiliation 5
%\\
%  \small{
%    \textbf{Correspondence:} \href{mailto:email@domain}{email@domain}
%  }
%}
\author{
    \textbf{Arash Lagzian\textsuperscript{1}}, 
    \textbf{Srinivas Anumasa\textsuperscript{1}}, 
    \textbf{Dianbo Liu\textsuperscript{1}} \\
    \textsuperscript{1}National University of Singapore \\
    \texttt{alagzian@visitor.nus.edu.sg} \\
    \texttt{\{srinu\_pd, dianbo\}@nus.edu.sg} \\
}
\begin{document}
\maketitle
\begin{abstract}
Large Language Models (LLMs) demonstrate remarkable proficiency in generating accurate and fluent text. However, they often struggle with diversity and novelty, leading to repetitive or overly deterministic responses. These limitations stem from constraints in training data, including gaps in specific knowledge domains, outdated information, and an over-reliance on textual sources. Such shortcomings reduce their effectiveness in tasks requiring creativity, multi-perspective reasoning, and exploratory thinking, such as LLM based AI scientist agents and creative artist agents . To address this challenge, we introduce inference-time  multi-view brainstorming method, a novel approach that enriches input prompts with diverse perspectives derived from both textual and visual sources, which we refere to as 
\textit{"Multi-Novelty"}. By incorporating additional contextual information as diverse starting point for chain of thoughts, this method enhances the variety and creativity of generated outputs. Importantly, our approach is model-agnostic, requiring no architectural modifications and being compatible with both open-source and proprietary LLMs.
% We evaluate our method and framework on over 909,500 generated outputs from various well-known LLMs, demonstrating significant improvements in output diversity and novelty while maintaining quality and relevance. \noindent\textbf{Code and Dataset:} Available on the \href{https://github.com/arashlagzian/MultyNovelty-ACL2025}{GitHub}.  

\end{abstract}

\begin{figure*}[t]
    \centering
    \includegraphics[width=\textwidth]{images/Figure_overview.png} 
    \caption{Overview of our proposed methodology for enriching the diversity and novelty of LLM-generated content using multi-view embeddings. Our approach starts with diverse input prompts (e.g., philosophical questions, problem-solving scenarios, imaginative tasks) and generates multiple textual and visual views (n=50 per each prompt) through multi-view generators. These multi-view embeddings are then fed into various open-source and closed-source LLMs, such as Qwen, DeepSeek-R1, and GPT-4o, to generate responses with enriched diversity and novelty. We created the 909kPR dataset consisting of 909,500 generated answers across different models. Finally, the DNC framework evaluates the generated responses using three measures: diversity, novelty, and correctness.}
    \label{fig:overview}
\end{figure*}

\section{Introduction}

Rapid advances in large language models (LLMs) have spurred an ongoing debate on the usefulness of these models on tasks that require human-level creativity such as LLM based AI scientist agents and art or design assistant. On the one hand, LLMs have already been used in creative writing \cite{Bellemare2024, Orwig2024},
poetry generation \cite{porter2024}, idea generation \cite{lee2024, si2024can} and even originality \cite{Guzik2023}. On the other hand, some studies suggest that LLM creativity is significantly weaker than human creativity \cite{Chakrabarty2024} and that LLM-generated stories exhibit observable shortcomings \cite{sato2023, levenson2023}. A recent user study found that, the use of an AI assistant in writing appears to enhance the creativity of individual writers, however, it reduces the collective diversity of novel content produced by multiple writers \cite{Doshi2024}. Recent studies on AI scientists suggest novelty as a major issue in using LLM for scientific study \cite{lu2024ai}. This suggests that we should examine the distribution of
LLM creations for a given prompt instead of each
creation individually. Language models like GPT-4o \cite{openai2023gpt4} can produce repetitive answers in story telling and the answer not the same as original text \cite{xu2024echoes}. 

Existing methods primarily rely on input prompts to generate responses, often resulting in limited diversity. To address this, we propose leveraging multiple views for prompting to enhance response diversity. However, ensuring the correctness of these diverse responses is equally crucial. Recognizing the importance of both novelty and accuracy, we introduce a novel framework designed to generate diverse and novel responses while also providing a mechanism to evaluate these aspects. \\
Our key contributions in this work are as follows:
\begin{itemize}
\item We propose an architecture-independent approach to enrich generated text in terms of both novelty and diversity. By incorporating multiple views of the text embedding or image embedding , our method encourages the model to produce more diverse and novel outputs.
\item We introduce a framework to quantitatively assess the responses generated based on diversity, novelty, and correctness.
\item We conducted extensive experiments to demonstrate the effectiveness of our approach, evaluating over $909k$ generated responses and showcasing its improvements over existing LLM models including GPT-4o and DeepSeek-R1
\end{itemize}



\section{Related Work}
Creative writing \cite{Kobak2024, lee2024} is on the rise; however, some studies suggest that content generated by human users tends to be more creative \cite{Kefford2023}. This study show that ChatGPT's ideas are more purchased from Wharton MBA students. There is an ongoing debate about whether LLMs can enhance creativity. To explore this, \cite{lee2024} demonstrates that when participants were tasked with generating creative ideas for everyday purposes, their creativity improved. However, \cite{Begus2023} finds that AI-generated narratives often lack imagination and typically include plot twists in a more casual manner. Additionally, \cite{Chakrabarty2024} invited expert writers to evaluate stories generated by LLMs versus those created by professional writers using a standard creativity test. Their findings suggest that LLM-generated stories are less creative compared to those written by professionals.
Empirical studies have underscored this issue. For example, \cite{si2024can} conducted qualitative analyses involving human judgment and found that after generating 500 samples, 50\% were non-repetitive ideas. However, in the following 1,500 generations, only an additional 50\% of non-repetitive ideas were produced. Alarmingly, in the final 2,000 rounds, just 12.5\% of the generated ideas were non-repetitive. This suggests that while an individual LLM output may appear novel, when generating multiple outputs, the LLM tends to become repetitive, lacking the diversity necessary to effectively enhance collective creativity.
This decline underscores the resource inefficiency and diminishing returns in prolonged LLM-generated content. \\
\cite{McCoy2023} suggests that novelty in LLM outputs can be detected by ensuring "the text must not have been copied from the training data." However, a more recent study by \cite{xu2024echoes} argues that this definition is superficial. In their experiment on story continuation, they demonstrate that while GPT-4ogenerated samples may meet this standard, the generated continuations are still quite conventional and lack diversity.
\cite{Shaib2024} analyze different existing scores that can help measure diversity in LLM outputs, but these metrics all focus on surface-level features such as n-gram overlaps. \cite{Ghosal2022} indicate that "identifying novel text is not straightforward because the text many have less lexical overlap yet convey the same information." and to the best of our knowledge there is no study to evaluate the diversity, novelty, and correctness of the generated outputs at the same time.

\section{Proposed Method}


\subsection{Multi-view Embedding}
The inability of existing LLM models to generate diverse and novel text persists even after fine-tuning the temperature parameter. We propose that instead of solely adjusting the temperature, prompting models from multiple perspectives can effectively encourage the generation of more diverse and novel text. Figure \ref{fig:overview} offers an overview of our approach. Instead of directly interacting with the LLM model to generate a response, we first interact with a multi-view generator to create several perspectives of the given prompt. These generated views are then fed into the LLM model to produce the final response. The following section will give a more detailed overview of our approach.
%According to the different limitation of large language models in generating diverse and novel outputs even by using temperature parameter, we aim to introduce multi-view embedding method as a model agnostic method to leverage textual or visual sources related to the input prompt to help model generate more diverse and novel answers and keep relevancy and correctness. In this work, views consist of textual or visual image sources and overview of our model depicted in Figure \ref{fig:overal_view}. 
%In sections \ref{sec:text_view}, and \ref{sec:image_view} we explain about text view embedding and image view embedding in our method and according to our experiments this method improve the model's understanding, output diversity, and enhanced creativity.
In sections \ref{sec:text_view} and \ref{sec:image_view}, we explain the concepts of text view embedding and image view embedding within our method. Our experiments show that this approach enhances the model's understanding, increases output diversity, and boosts creativity.

% \begin{figure}[h!]
%     \centering
%     \begin{minipage}{0.48\textwidth}
%         \centering
%         \includegraphics[height=0.7\textwidth]{images/044.jpg}
%         \subcaption{Comparison between existing text generation methods and our method.}
%         \label{fig:overal_view}
%     \end{minipage}
%     \begin{minipage}{0.44\textwidth}
%         \centering
%         \includegraphics[height=0.7\textwidth]{images/022.jpg}
%         \subcaption{Text Multi-View Embedding.}
%         \label{fig:text_views}
%     \end{minipage}
%     \caption{(a) Comparison of existing text generation methods and our method. (b) Illustration of the Text Multi-View Embedding.}
%     \label{fig:combined_figure}
% \end{figure}.
% \begin{figure*}[t]
%     \centering
%     \begin{minipage}{0.48\linewidth}
%         \centering
%         \includegraphics[width=\linewidth]{images/044.jpg}
%         \subcaption{Comparison between existing text generation methods and our method.}
%         \label{fig:overal_view}
%     \end{minipage} \hfill
%     \begin{minipage}{0.48\linewidth}
%         \centering
%         \includegraphics[width=\linewidth]{images/022.jpg}
%         \subcaption{Text Multi-View Embedding.}
%         \label{fig:text_views}
%     \end{minipage}
%     \caption{(a) Comparison of existing text generation methods and our method. (b) Illustration of the Text Multi-View Embedding.}
%     \label{fig:combined_figure}
% \end{figure*}

% \begin{figure}[h!]
%     \centering
%     \makebox[\textwidth][l]{ % Shift both images to the left
%     \hspace{-0.2cm}
%         \begin{minipage}{0.48\textwidth}
%             \centering
%             \includegraphics[height=0.7\textwidth]{images/044.jpg}
%             \subcaption{Comparison between existing text generation methods and our method.}
%             \label{fig:overal_view}
%         \end{minipage}
%         %\hspace{0.02\textwidth}
%         \begin{minipage}{0.48\textwidth}
%             \centering
%             \includegraphics[height=0.65\textwidth]{images/022.jpg}
%             \subcaption{Text Multi-View Embedding.}
%             \label{fig:text_views}
%         \end{minipage}
%     }
%     \caption{(a) Comparison of existing text generation methods and our method. (b) Illustration of the Text Multi-View Embedding.}
%     \label{fig:combined_figure}
% \end{figure}


% \begin{figure}[h!]
%     \centering
%     \includegraphics[width=0.65\textwidth]{images/04.jpg} 
%     \caption{Compression between existing text generation methods and our method.}
%     \label{fig:overal_view}
% \end{figure}

\subsubsection{Text View Embedding}
\label{sec:text_view}
%Text Multi-View Embedding is used to enrich the input prompt by generating multiple diverse perspectives or representations of the same concept and then combining them before feeding them into a model. This technique aims to provide a more comprehensive and context-rich input via textual sources. These sources can crawl via internet or added manually, or even generate by a language model. In this work, we used GPT-4o as our text multi-view generator and all texts are in English language. Figure \ref{fig:text_views} depict the overall text view embedding section in our method In \cite{dipper2024} also used GPT-4o to get diverse aspects from the math questions and this work lead them to have more performance on reasoning task.

Text Multi-view Embedding enhances the input prompt by generating multiple diverse perspectives or representations of the same concept, which are then combined before being fed into the model. This approach aims to provide a more comprehensive and context-rich input using various textual sources. These sources can be gathered from the internet, added manually, or even generated by a language model. In this work, we utilize GPT-4o as our text multi-view generator, with all texts in English. Figure \ref{fig:text_views} illustrates the text view embedding section of our method. In \cite{dipper2024}, GPT-4o was also used to extract diverse perspectives from math questions, resulting in improved performance on reasoning tasks.

\begin{figure}[t]  % [t], [h], or [ht], depending on your preference
    \centering
    \includegraphics[width=\linewidth]{images/022.png}
    \caption{Text Multi-View Embedding.}
    \label{fig:text_views}
\end{figure}

\subsubsection{Image View Embedding}
\label{sec:image_view}
In addition to Text Multi-view Embedding, we also introduce Image Multi-view Embedding. Images contain rich contextual information and, unlike text, can offer diverse and multiple perspectives. 
Image Multi-view Embedding is used to enhance the input prompt by incorporating multiple image-based perspectives. This method starts by crawling for images related to the input prompt, which serve as visual representations of the concept. Once relevant images are retrieved, the Qwen-2VL \cite{qwen2vl} vision-language model is used to describe each image. These descriptions capture the visual content in textual form; however, they may lack consistency in writing style or contain structural issues. To improve the quality and coherence of these descriptions, we use a language model, in this case GPT-4o-mini, to rewrite and refine the original descriptions. The refined descriptions ensure that the textual representation of visual content is well-structured and stylistically consistent.  Instead of directly concatenating the descriptions with the input prompt, we use the refined descriptions as additional context when generating the final response. The model utilizes the detailed information from the rewritten image descriptions to provide richer, more accurate answers to the input prompt. The process of image view embedding is shown in the Figure \ref{fig:image_views} and Figure \ref{fig:happiness} in appendix \ref{appendix:a} shows an example of an image obtained by crawling over the internet using the input prompt, with its corresponding Image view and Answer.

\begin{figure}[ht!]
    \centering
    \includegraphics[width=0.48\textwidth]{images/05.jpg} 
    \caption{This figure illustrates the process of preparing image view embeddings and provides an example for 10 input prompts. Row 1 displays 10 prompts from various subjects. Row 2 shows images crawled based on each input prompt. Row 3 presents the descriptions generated for the images, and Row 4 contains the rewritten descriptions, which serve as our image view embeddings.}
    \label{fig:image_views}
\end{figure}

% This approach increases the semantic richness and contextual awareness of the input. Visual content often highlights aspects that textual prompts alone might overlook, making it possible to generate more comprehensive and creative outputs. The rewriting process ensures high-quality input, which enhances the overall coherence and informativeness of the model’s responses. Figure \ref{fig:happiness} in appendix \ref{appendix:a} shows an example of an image obtained by crawling over the internet using the input prompt, with its corresponding Image view and Answer. %image, image view related to the image, and final answer generated by model.








\subsection{Metrics}
\label{sec:metrics}
To evaluate the responses generated by different LLMs for a given input prompt, it is crucial to consider multiple aspects of the generated text in order to quantify the model's performance. Existing works typically focus on one aspect—such as novelty, diversity, or correctness—individually. In contrast, our work takes a comprehensive approach by considering all these aspects to assess the model's performance.
%For evaluating the answers of different LLMs to the input prompt we should consider different aspects some works only attention to the diversity like: some another try to evaluate the answers in novelty detection but non of them can't capture the enrich information about the output of the LLMs. In this work we introduce a framework to measure the diversity, novelty, and correctness of the answers at the same time. Now we discuss each one in details below:
\subsubsection{Diversity Measure}
\label{subsec:diversity_measure}
The diversity in the text generated by LLMs can be measured from various aspects. Two of the most important aspects are:

\begin{itemize} 
\item Diversity across different text responses. 
\item Diversity within different tokens generated in a single text response. 
\end{itemize}

In this work, first we try to describe each metric and then plot the importance of each one in diversity measurement and special cases for each metric and finally talk about the results.

\paragraph{MTLD:} Lexical diversity is a key measure of linguistic richness in generated text. To evaluate this, we use the Measure of Textual Lexical Diversity (MTLD) \cite{McCarthy2010}. MTLD calculates the mean length of text segments that maintain a predefined type-token ratio (TTR). This approach overcomes limitations of traditional TTR metrics, which are sensitive to text length. Following \cite{McCarthy2010}, we set the TTR threshold to 0.72, as this value has been empirically validated to balance sensitivity and robustness across a variety of text datasets.


% in other works like \cite{Shaib2024} try to use n\_gram metric to measure diversity in lexical level we try to explain it why normilized MLTD is better see appendix~\ref{appendix:A}.

\paragraph{Semantic Diversity of Text (SDT)}
We measure semantic diversity using Term Frequency–Inverse Document Frequency (TF-IDF) representations of all responses. First, each response is transformed into a TF-IDF vector based on term usage and distinctiveness. Then, we compute the pairwise cosine similarity among these TF-IDF vectors to quantify how similar the responses are. Because higher similarity implies lower diversity, we define our overall semantic diversity metric as:

\[
\text{SDT} = 1 - \overline{\text{Sim}},
\]

where \( \overline{\text{Sim}} \) is the average of all pairwise cosine similarities between the TF-IDF vectors of the responses. Consequently, larger values of SDT indicate greater variation in term usage—i.e., more semantic diversity. This formulation ensures that near-identical responses (high cosine similarity) yield low diversity scores, while dissimilar responses produce high diversity scores.
% Semantic diversity based on Term Frequency-Inverse Document Frequency (TF-IDF) captures the variation in term usage relative to the importance of terms across multiple outputs. By assigning weights to terms based on their frequency and distinctiveness, TF-IDF provides insights into the uniqueness of generated content. High semantic diversity is reflected in less repetition of terms and greater differentiation in word usage across outputs. This metric is instrumental in evaluating language models’ ability to generate contextually diverse and meaningful text, which is particularly valuable in creative and informative content generation tasks.

% Given a set of responses $r_j$ for a given  $j^{th}$ prompt. 
% The \textit{TF-IDF} score for a term \( t \) in response \( r_j \) is calculated as:
% \[
% \text{TF-IDF}(t, r_j) = \text{TF}(t, r_j) \times \text{IDF}(t)
% \]
% where 
% The Term Frequency for a term \( t \) in the set of responses \( r_j \) is calculated as: $
% \text{TF}(t, r_j) = \frac{\text{Number of times term } t \text{ appears in } r_j}{\text{Total number of terms in } r_j}
% $
% and the Inverse Document Frequency for a term \( t \) is calculated as:
% $
% \text{IDF}(t) = \log \left( \frac{m}{\text{DF}(t)} \right)
% $.
% Where, \( m \) is the total number of prompts  and \( \text{DF}(t) \) is the number of set of responses that contain the term \( t \). \textcolor{red}{To measure the \textit{semantic diversity}, we aggregate the \textit{TF-IDF} scores of all terms in all the set of responses.} % This can be computed as the average or sum of the TF-IDF scores for all unique terms \( t_1, t_2, ..., t_n \) in document \( d \), where \( n \) is the number of unique terms in document \( d \). 
% \[
% \textbf{SDT} =  \frac{1}{m} \sum_{j=1}^{m}\frac{1}{n_j} \sum_{i=1}^{n_j} \text{TF-IDF}(t^j_i, r_j)
% \]

\paragraph{Semantic Diversity of Embeddings (SDE)}
BERT-based semantic diversity measures the variation in the \emph{semantic content} of generated text. 
Unlike lexical diversity, which focuses on surface-level differences in word usage, BERT embeddings capture 
deeper contextual differences by mapping sentences into a dense semantic space. We compute pairwise cosine 
similarity among the embeddings of all responses and then take the average cosine distance 
(i.e., \(1 - \text{cosine similarity}\)) to quantify diversity. Higher semantic diversity indicates a broader 
range of meanings and greater contextual richness. This approach leverages the contextualized representations 
of BERT~\citep{bert2019}, offering a robust and nuanced assessment of the diversity and coherence of 
model-generated outputs.
The \textbf{Semantic Diversity of Embeddings} (\(\textbf{SDE}\)) can be quantified by averaging the 
pairwise \emph{cosine distance} between all embeddings. Given a set of embeddings 
\(\{\mathbf{e}_1, \mathbf{e}_2, \dots, \mathbf{e}_n\}\), the cosine similarity between two embeddings 
\(\mathbf{e}_i\) and \(\mathbf{e}_j\) is defined as:
$
\text{CosineSimilarity}(\mathbf{e}_i, \mathbf{e}_j) 
= \frac{\mathbf{e}_i \cdot \mathbf{e}_j}{\|\mathbf{e}_i\| \|\mathbf{e}_j\|}.
$
The corresponding cosine distance is:
$
1 - \text{CosineSimilarity}(\mathbf{e}_i, \mathbf{e}_j).
$
Thus, the overall \textbf{SDE} is:
\[
\textbf{SDE} = \frac{2}{n(n-1)} \sum_{1 \leq i < j \leq n} 
\biggl(1 - \frac{\mathbf{e}_i \cdot \mathbf{e}_j}{\|\mathbf{e}_i\| \|\mathbf{e}_j\|}\biggr),
\]
where \(n\) is the total number of embeddings (responses).
% Semantic diversity using BERT measures the variation in semantic content across generated text outputs. Unlike lexical diversity, which focuses on surface-level differences in word usage, BERT-based semantic diversity captures deeper contextual differences by embedding sentences into a dense semantic space. We compute pairwise cosine similarity between answers embeddings and subtract mean value from one, so higher semantic diversity, reflecting a greater range of meaning and contextual richness in our setting. This approach leverages the contextualized representations of BERT \cite{bert2019}, offering a robust and nuanced evaluation of the diversity and coherence of model-generated content.

% The \textbf{SDE} of embeddings can be quantified by calculating the average pairwise \textit{cosine distance} between the embeddings of the responses. Given a set of embeddings \( \{ \mathbf{e}_1, \mathbf{e}_2, \dots, \mathbf{e}_n \} \), the cosine similarity between two embeddings \( \mathbf{e}_i \) and \( \mathbf{e}_j \) is defined as:
% $
% \text{Cosine Similarity}(\mathbf{e}_i, \mathbf{e}_j) = \frac{\mathbf{e}_i \cdot \mathbf{e}_j}{\|\mathbf{e}_i\| \|\mathbf{e}_j\|}
% $
% The cosine distance is then can be computed as: $
%  1 - \text{Cosine Similarity}(\mathbf{e}_i, \mathbf{e}_j)
% $
% \textcolor{red}{The \textbf{SDE} of a set of embeddings is then defined as the average cosine distance between all pairs of embeddings}:
% \[
% \textbf{SDE} = \frac{2}{n(n-1)} \sum_{1 \leq i < j \leq n} \left( 1 - \frac{\mathbf{e}_i \cdot \mathbf{e}_j}{\|\mathbf{e}_i\| \|\mathbf{e}_j\|} \right)
% \]

% %Where:
% %\begin{itemize}
% %    \item \( E = \{ \mathbf{e}_1, \mathbf{e}_2, \dots, \mathbf{e}_n \} \) is the set of embeddings.
% %    \item \( n \) is the number of embeddings in the set.
% %    \item \( \mathbf{e}_i \cdot \mathbf{e}_j \) is the dot product of the embeddings \( \mathbf{e}_i \) and \( \mathbf{e}_j \).
% %    \item \( \|\mathbf{e}_i\| \) and \( \|\mathbf{e}_j\| \) are the norms (magnitudes) of the embeddings \( \mathbf{e}_i \) and \( \mathbf{e}_j \), respectively.
% %\end{itemize}

\paragraph{Self-BLUE}
Self-BLEU evaluates the diversity of generated text by measuring how similar each generated sample is to the others. Unlike traditional BLEU, which compares a generated sample to reference texts, Self-BLEU treats each sample as a reference for the others. A high Self-BLEU score signals redundancy, reflecting repeated patterns or limited variability \cite{selfblue2018}. In our setting we subtract the mean value across all answers from 1 to measure the diversity across all answers.

\paragraph{Lexical Entropy}

%Lexical entropy measures the uncertainty or unpredictability in the choice of words within generated text. It provides a quantitative assessment of lexical variety, with higher entropy indicating greater diversity in word usage. We compute Shannon entropy (in bits) over all tokens in the responses, based on their relative frequencies. Given a set of responses, each token’s probability is calculated, and the entropy is computed as the sum of \( -p \cdot \log_2(p) \) for all unique tokens. This approach captures how evenly distributed the token usage is across the text. A higher lexical entropy reflects less repetition and richer vocabulary, which is essential for evaluating the linguistic creativity of language models.

It quantifies the uncertainty or variability in word choice within generated text, serving as a measure of lexical diversity. Higher entropy indicates a broader and more varied vocabulary, while lower entropy suggests repetitive or predicTable word usage. We calculate Shannon entropy (in bits) over all tokens in the generated responses, based on their relative frequencies. For a given set of responses, the probability of each unique token is determined, and the entropy is computed as: $
H = -\sum_{i} p_i \log_2 p_i
$ where \( p_i \) represents the probability of the \( i \)-th token. A higher lexical entropy signifies reduced repetition and a richer vocabulary, making it a valuable indicator of linguistic creativity in language models.




% \begin{table}[h]
% \centering
% \caption{Diversity Metrics for different models. Higher values indicate better performance. $num\_samples = 100$ , $max\_length = 125$}
% \label{tab:mean_results}
% \begin{adjustbox}{width=\textwidth,center}
% \begin{tabular}{lcccccc}
% \toprule
% Model & \multicolumn{2}{c}{num\_samples = 100} & \multicolumn{2}{c}{num\_samples = 250} & \multicolumn{2}{c}{num\_samples = 500} \\
% \cmidrule(lr){2-3} \cmidrule(lr){4-5} \cmidrule(lr){6-7}
%  & GPT-40 & SBERT & GPT-40 & SBERT & GPT-40 & SBERT \\
% \midrule
% gpt-2 & 0 & 0 & 0 & 0 & 0 & 0 \\
% gpt-2 + text view & 0 & 0 & 0 & 0 & 0 & 0 \\
% qwen & 0 & 0 & 0 & 0 & 0 & 0 \\
% qwen + text view & 0 & 0 & 0 & 0 & 0 & 0 \\
% qwen + image view & 0 & 0 & 0 & 0 & 0 & 0 \\
% gpt40 mini & 0 & 0 & 0 & 0 & 0 & 0 \\
% gpt40 mini + text view & 0 & 0 & 0 & 0 & 0 & 0 \\
% gpt40 mini + image view & 0 & 0 & 0 & 0 & 0 & 0 \\
% gpt4o & 0 & 0 & 0 & 0 & 0 & 0 \\
% gpt4o + text view & 0 & 0 & 0 & 0 & 0 & 0 \\
% gpt40+ image view & 0 & 0 & 0 & 0 & 0 & 0 \\
% deepseekr-R1 & 0 & 0 & 0 & 0 & 0 & 0 \\
% deepseekr-R1 + text view & 0 & 0 & 0 & 0 & 0 & 0 \\
% deepseekr-R1 + image view & 0 & 0 & 0 & 0 & 0 & 0 \\
% \bottomrule
% \end{tabular}
% \end{adjustbox}
% \end{table}



\subsubsection{Novelty Measure}
Novelty detection is critical for assessing large language model (LLM) outputs, as it differentiates existing knowledge from newly introduced information. This task can be framed as Natural Language Inference (NLI), wherein one determines whether a hypothesis contains content not found in a premise set. Prior studies have employed both NLI-based methods and embedding techniques such as SBERT \cite{Ghosal2022} to measure semantic similarity, illustrating the importance of novelty detection for refining LLM-generated text.

In our work, we wanted to assess the inherent capability of LLMs to perform as NLI models in detecting novel versus non-novel documents. To achieve this, we used \textbf{TAP-DNLD 1.0} \cite{Ghosal18} a dataset consisting of around $2.8k$ novel and $2.7k$ non-novel documents, and each document belongs to one of the ten categories and each target document labeld versus three source document by human. For comparison , first we sampled from this dataset in three different seeds and in each seed from each category sampled 5 novel and 5 non-novel document to have a balance data and totally 300 document then We used GPT-4o and SBERT as novelty detector to have better comparison and results reported in Table \ref{tab:classification_metrics_sbert_gpt}.

\begin{table}[ht]
    \centering
    % \vspace{1em}
    \resizebox{\linewidth}{!}{ % Adjust Table size to fit page width
    \begin{tabular}{ l c c c c c }
        \hline
        \textbf{Seed} & \textbf{Model} & \textbf{Accuracy} & \textbf{Precision} & \textbf{Recall} & \textbf{F1-score} \\
        \hline
        \multirow{2}{*}{Seed0} 
        & SBERT  & 0.6700 & 0.6269 & 0.8400 & 0.7179 \\ 
        & GPT-4O & \textbf{0.6900} & \textbf{0.6415} & \textbf{0.8600}  & \textbf{0.7347 } \\ 
        \hline
        \multirow{2}{*}{Seed1} 
        & SBERT  & \textbf{0.7200} & \textbf{0.6528} & \textbf{0.9400} & \textbf{0.7705} \\ 
        & GPT-4O & 0.6600  & 0.6086  & 0.9000  & 0.7260  \\ 
        \hline
        \multirow{2}{*}{Seed2} 
        & SBERT  & 0.6900 & 0.6338 & 0.9000 & 0.7438 \\ 
        & GPT-4O & \textbf{0.6950 } & \textbf{0.6368 } & \textbf{0.9100}  & \textbf{0.7492 } \\ 
        \hline
    \end{tabular}
    }
    \caption{Classification metrics to evaluate novelty detectors GPT-4o and SBERT models on dataset \textbf{TAP-DNLD 1.0} \cite{Ghosal18}.}
    \label{tab:classification_metrics_sbert_gpt}
\end{table}

% \begin{table}[h]
%     \centering
%     \caption{Classification metrics (Mean ± Std) for GPT-4O and SBERT across different seeds. Higher values per comparison are bolded.}
%     \vspace{1em}
%     \resizebox{\textwidth}{!}{ % Adjust Table size to fit page width
%     \begin{tabular}{ l c c c c c }
%         \hline
%         \textbf{Seed} & \textbf{Model} & \textbf{Accuracy} & \textbf{Precision} & \textbf{Recall} & \textbf{F1-score} \\
%         \hline
%         \multirow{2}{*}{Seed0} 
%         & SBERT  & 0.6700 & 0.6269 & \textbf{0.8400} & 0.7179 \\ 
%         & GPT-4O & \textbf{0.6900 ± 0.0200} & \textbf{0.6415 ± 0.0107} & 0.8600 ± 0.0400 & \textbf{0.7347 ± 0.0216} \\ 
%         \hline
%         \multirow{2}{*}{Seed1} 
%         & SBERT  & \textbf{0.7200} & \textbf{0.6528} & \textbf{0.9400} & \textbf{0.7705} \\ 
%         & GPT-4O & 0.6600 ± 0.0200 & 0.6086 ± 0.0164 & 0.9000 ± 0.0000 & 0.7260 ± 0.0117 \\ 
%         \hline
%         \multirow{2}{*}{Seed2} 
%         & SBERT  & 0.6900 & 0.6338 & \textbf{0.9000} & 0.7438 \\ 
%         & GPT-4O & \textbf{0.6950 ± 0.0250} & \textbf{0.6368 ± 0.0204} & 0.9100 ± 0.0100 & \textbf{0.7492 ± 0.0175} \\ 
%         \hline
%     \end{tabular}
%     }
%     \label{tab:classification_metrics_sbert_gpt}
% \end{table}

After establishing the baseline performance of these models, we utilized GPT-4o and SBERT as novelty detectors to evaluate the novelty of generated outputs across different language models, as well as to assess the impact of our method on the novelty of generated responses. The novelty detection process is illustrated in Figure \ref{fig:novelty detector}. In this approach, for each set of prompt outputs, the first generated response is considered novel by default. Subsequent responses are sequentially analyzed by the novelty detector, where each new answer is compared against previously identified novel responses. This comparison is conducted in a hypothesis-premise framework, where the new response serves as the hypothesis, and previously identified novel answers form the premise set. If the new answer introduces additional information not present in the premise set, it is classified as a novel answer and added to the premise set. Otherwise, it is labeled as redundant.

\begin{figure}[ht!]
    \centering
    \includegraphics[width=\linewidth]{images/06.jpg} 
    \caption{Novelty detection method by using GPT-4o and SBERT.}
    \label{fig:novelty detector}
\end{figure}

 



% \begin{table}[h]
%     \centering
%     \resizebox{\textwidth}{!}{ % Adjust Tablesize to fit page width
%     \begin{tabular}{|c|c|c|c|c|c|}
%         \hline
%         \textbf{Seed} & \textbf{Model} & \textbf{Accuracy} & \textbf{Precision} & \textbf{Recall} & \textbf{F1-score} \\
%         \hline
%         \multirow{2}{*}{Seed0} & SBERT  & 0.6700 & 0.6269  & \textbf{0.8400}  & 0.7179 \\
%                                & GPT-4O & \textbf{0.6900 ± 0.0200} & \textbf{0.6415 ± 0.0107} & 0.8600 ± 0.0400 & \textbf{0.7347 ± 0.0216} \\
%         \hline
%         \multirow{2}{*}{Seed1} & SBERT  & \textbf{0.7200} & \textbf{0.6528}  & \textbf{0.9400}  & \textbf{0.7705} \\
%                                & GPT-4O & 0.6600 ± 0.0200 & 0.6086 ± 0.0164 & 0.9000 ± 0.0000 & 0.7260 ± 0.0117 \\
%         \hline
%         \multirow{2}{*}{Seed2} & SBERT  & 0.6900 & 0.6338  & \textbf{0.9000}  & 0.7438 \\
%                                & GPT-4O & \textbf{0.6950 ± 0.0250} & \textbf{0.6368 ± 0.0204} & 0.9100 ± 0.0100 & \textbf{0.7492 ± 0.0175} \\
%         \hline
%     \end{tabular}
%     } % End resizebox
%     \caption{Classification metrics (Mean ± Std) for GPT-4O and SBERT across different seeds. Higher values per comparison are bolded.}
%     \label{tab:classification_metrics_sbert_gpt}
% \end{table}
\subsubsection{Correctness Measure}
\label{subsubsec: correctness_measure}
Answers could be novel or even diverse but completely irrelevant to the input prompt (see example in appendix \ref{sec:correctness_measure_example}), or they can have different structures. In this section we want to answer these two questions:
\begin{itemize}
    \item Is the generated answer correct and relevant to the given input prompt?  
    \item How well does the generated answer adhere to proper language structure and grammatical accuracy?  
\end{itemize}
%To answer the first question we design an experiment to evaluate the language models able to detect relevant answers to the question or not. We used GPT-WritingPrompts dataset \cite{gptstpries} data. This dataset consist of around $97k$ unique prompts and answers from both humans and GPT3.5 but we only used the answers of humans.
To address the first question, we designed an experiment to assess whether language models can accurately detect relevant answers to a given prompt. For this evaluation, we used the GPT-WritingPrompts dataset \cite{gptstpries}, which contains approximately $97k$ unique prompts along with responses from both humans and GPT-3.5. However, in our study, we exclusively utilized the human-generated answers. %We samples in ten different seeds from this data and per each seed $1k$ data sampled randomly, then we cluster them into 10 different clusters and try to pick 100 prompt randomly and for each prompt we select one relevant answer here we called it "Correct" and one irrelevant answer from another cluster and called it "incorrect" to ensure that these answers are different. 
We sampled data using ten different random seeds, selecting 1K samples per seed. These samples were then clustered into ten distinct groups. From each cluster, we randomly picked $100$ prompts. For each selected prompt, we assigned one relevant answer from the same cluster, labeling it as "Correct," and one irrelevant answer from a different cluster, labeling it as "Incorrect," ensuring that the chosen answers were distinct. Since the answers are approximately $500$ words long, we assume that if an answer is related to the prompt, its summary will also be relevant, and similarly, if an answer is irrelevant, its summary will remain unrelated. Based on this assumption, we use the GPT-4o model to generate summaries for both correct and incorrect answers, setting the maximum length to 250 words. Next, we evaluate the ability of two well-known LLMs, GPT-4o and DeepSeekV3 \cite{deepseekv3}, to determine the correctness of the answers. The results, presented in Table \ref{tab:mean_std_metrics_sum}, indicate that GPT-4o outperforms DeepSeekV3 in detecting correctness. An example prompt and its related responses are provided in Appendix \ref{appendix:b}. Based on these findings, we use GPT-4o for correctness detection in our experiments.
\begin{table}[ht]
    \centering
    %\vspace{1em}
    \resizebox{\linewidth}{!}{ % Adjust Tablesize to fit page width
    \begin{tabular}{ c c c c c }
        \hline
        \textbf{Model} & \textbf{Accuracy} & \textbf{Precision} & \textbf{Recall} & \textbf{F1-score} \\
        \hline
        DeepSeekV3  & 0.8410 ± 0.0223 & \textbf{0.9564 ± 0.0285} & 0.7270 ± 0.0447 & 0.8191 ± 0.0301 \\
        \hline
        GPT-4O       & \textbf{0.8980 ± 0.0214} & 0.9468 ± 0.0255 & \textbf{0.8310 ± 0.0316} & \textbf{0.8946 ± 0.0228} \\
        \hline
    \end{tabular}
    } % End resizebox
    \caption{Results on classification task to detect correct answer across all $10$ seeds for DeepSeekV3 and GPT-4o models. Complete results can be found in Table \ref{tab:classification_metrics_correctness}, Appendix \ref{appendix:a}.}
    \label{tab:mean_std_metrics_sum}
\end{table}



%To answer the second question, we first designed an experiment to evaluate the language models based on their ability to give a score to the generated English texts. Based on this experiment we used IELTS Writing text task 2 with labeld scores \cite{Mazlumi2023} from kaggle dataset. This dataset consist of 642 writing text for task 1 and 793 writing text for task 2. only task 2 is related to our work and we use questions and answers to the task 2 to evaluate language models. We sampled randomly in 3 different seeds with 100 samples in each one in two iterations to evaluate the quality of the generated text and gave a score between 1 and 9 by two well-known model and results shown in Table\ref{tab:structure_score_comperession}. Based on this experiment we used DeepSeekV3  model to evaluate the answers from the grammatical and English structure aspects and give each answer a score between 1 and 10. the template of prompts for both experiments available in Appendix~\ref{appendix:b}.

To address the second question, we designed an experiment to evaluate language models based on their ability to assign scores to generated English texts. For this evaluation, we utilized the IELTS Writing Task 2 dataset with labeled scores \cite{Mazlumi2023} from Kaggle. This dataset contains 642 responses for Task 1 and 793 responses for Task 2. Since only Task 2 is relevant to our study, we used its questions and answers to assess language models. We randomly sampled 100 responses in three different seeds, conducting two iterations to evaluate text quality. Two well-known models assigned scores between 1 and 9 (based on the IELTS writing score range), and the results are presented in Table \ref{tab:structure_score_comperession}. Based on this experiment, we selected the DeepSeekV3 model to evaluate responses in terms of grammar and overall English structure, assigning each answer a score between 1 and 10. The prompt templates for both experiments are provided in Appendix~\ref{appendix:b}. Based on this experiment we used DeepSeekV3 model to evaluate the outputs.
% \begin{comment}
\begin{table}[ht]
    \centering
    \resizebox{\linewidth}{!}{ % Adjust Table size to fit page width
    \begin{tabular}{ c c c }
        \hline
        \textbf{Seed} & \textbf{GPT-4o} & \textbf{DeepSeekV3} \\
        \hline
        Seed0  & 2.3775 ± 0.0350 & \textbf{2.0775 ± 0.0025} \\
        \hline
        Seed1  & 2.2288 ± 0.0512 & \textbf{2.0300 ± 0.0050} \\
        \hline
        Seed2  & 2.1562 ± 0.0587 & \textbf{1.9050 ± 0.0500} \\
        \hline
    \end{tabular}
    } % End resizebox
    \caption{MSE (Mean ± Std) for GPT-4o and DeepSeekV3 per seed for two iterations. Lower values are bolded.}
    \label{tab:structure_score_comperession}
\end{table}



% \end{comment}

% \begin{table}[h]
%     \centering
%     \begin{minipage}{0.48\textwidth}
%         \centering
%         \caption{MSE (Mean ± Std) for GPT-4O and DeepSeekV3 per seed for two iterations. Lower values are bolded.}
%         \resizebox{0.95\textwidth}{!}{ % Adjust Table size to fit page width
%         \begin{tabular}{ c c c }
%             \hline
%             \textbf{Seed} & \textbf{GPT-4O} & \textbf{DeepSeekV3} \\
%             \hline
%             Seed0  & 2.3775 ± 0.0350 & \textbf{2.0775 ± 0.0025} \\
%             \hline
%             Seed1  & 2.2288 ± 0.0512 & \textbf{2.0300 ± 0.0050} \\
%             \hline
%             Seed2  & 2.1562 ± 0.0587 & \textbf{1.9050 ± 0.0500} \\
%             \hline
%         \end{tabular}
%         }
%         \label{tab:structure_score_comperession}
%     \end{minipage} \hfill
%     \begin{minipage}{0.48\textwidth}
%         \centering
%         \caption{Mean score ranging from 1 to 10 for correctness score results from English structure aspects for $num\_sample = 100$ per each prompt (10 prompts).}
%         \vspace{1em}
%         \begin{tabular}{lc}
%             \toprule
%             Model & DeepSeekV3  \\ %(\%)
%             \midrule
%             GPT2 & 3.29 \\
%             GPT2 + Text View & 2.40 \\ \hline
%             GPT4o & 8.07 \\
%             GPT4o + Text View & 8.05 \\
%             GPT4o + Image View & 8.06 \\ \hline
%             DeepSeek-R1 & 7.96 \\
%             DeepSeek-R1 + Text View & 7.15 \\
%             DeepSeek-R1 + Image View & 6.10 \\
%             \bottomrule
%         \end{tabular}
%         \label{tab:correctness_english_structure}
%     \end{minipage}
% \end{table}


\section{Experimental Results}
In Section \ref{subsec:diversity_measure}, we introduced various diversity metrics. In this section, we provide experimental results to assess the diversity of different language models. Additionally, we applied the multi-view embedding method to four open-source models—GPT-2 Medium \cite{gpt2}, Qwen2.5-1.5B \cite{qwen2.5-1.5b}, Llama 3.2-3B \cite{llama32} ,and DeepSeek-R1-7B \cite{deepseekr1-7b}—as well as two API-based models, GPT-4o \cite{openai2023gpt4} and GPT-4o Mini \cite{openai2024gpt4omini}\footnote{We conducted all expermiments on one A100 GPU and spent 750\$ for API models.}. Our experiments utilized 10 diverse prompts spanning multiple domains (provided in Appendix \ref{appendix:b}), instructing each model to generate responses. This process was repeated across different sample sizes ranging from $100$ to $2000$, with a fixed maximum sequence length of $125$, resulting in a total of $909500$ generated responses and for having fair compression we used the same parameters for all models like $tempereture = 0.9$ and $top\_k = 0.95$. The generated outputs were then evaluated in terms of diversity, novelty, and correctness. Table \ref{tab:mean_results_diversity} presents the diversity measurement results. The experimental findings demonstrate that our method enhances diversity across all models. In some cases, we observed up to a threefold increase in the diversity of generated outputs.

\begin{table}[ht]
    \centering
    %\vspace{1em}
    \resizebox{\linewidth}{!}{
    \begin{tabular}{ l c c c c c }
        \hline
        \textbf{Models} & \textbf{MTLD} & \textbf{TF-IDF} & \textbf{Lexical Entropy} & \textbf{Semantic Embedding} & \textbf{Self-BLEU} \\
        \hline
        GPT-2                      & \textbf{41.36 ± 0.04}  & 0.87 ± 0.01  & 8.16 ± 0.09  & 0.55 ± 0.00  & 0.59 ± 0.11  \\
        GPT-2 + Text View          & 34.95 ± 1.14  & \textbf{0.90 ± 0.01}  & \textbf{8.35 ± 0.13}  & \textbf{0.65 ± 0.01}  & \textbf{0.65 ± 0.09}  \\
        \hline
        Qwen-2.5                       & 67.19 ± 0.14  & 0.80 ± 0.01  & 7.91 ± 0.05  & 0.22 ± 0.00  & 0.25 ± 0.09  \\
        Qwen-2.5 + Text View           & 67.64 ± 0.10  & \textbf{0.86 ± 0.01}  & \textbf{8.46 ± 0.07}  & \textbf{0.38 ± 0.10}  & \textbf{0.39 ± 0.10}  \\
        Qwen-2.5 + Image View          & \textbf{70.28 ± 0.04}  & 0.84 ± 0.01  & 8.30 ± 0.07  & 0.30 ± 0.00  & 0.37 ± 0.11  \\
        \hline 
        LLama-3.2                  & 52.45 ± 0.18 & 0.72 ± 0.01 & 7.41 ± 0.05 & 0.23 ± 0.00 & 0.18 ± 0.07\\
        LLama-3.2 + Text View      & 54.23 ± 0.05 & \textbf{0.84 ± 0.01} & \textbf{8.21 ± 0.09} & \textbf{0.43 ± 0.00} & \textbf{0.36 ± 0.11}\\
        LLama-3.2 + Image View     & \textbf{54.46} ± 0.07 & 0.76 ± 0.01 & 7.70 ± 0.06 & 0.31 ± 0.00 & 0.27 ± 0.07\\
        \hline
        DeepSeek-R1                & 52.51 ± 0.10  & 0.78 ± 0.01  & 7.64 ± 0.05  & 0.24 ± 0.00  & 0.24 ± 0.05  \\
        DeepSeek-R1 + Text View    & \textbf{54.61 ± 0.21}  & \textbf{0.85 ± 0.01}  & \textbf{8.21 ± 0.08}  & \textbf{0.39 ± 0.00}  & \textbf{0.36 ± 0.09}  \\
        DeepSeek-R1 + Image View   & 54.36 ± 0.05  & 0.83 ± 0.01  & 8.07 ± 0.06  & \textbf{0.39 ± 0.00}  & 0.34 ± 0.10  \\
        \hline
        GPT-4O Mini                & 57.95 ± 0.02  & 0.63 ± 0.01  & 7.06 ± 0.02  & 0.11 ± 0.00  & 0.09 ± 0.05  \\
        GPT-4O Mini + Text View    & \textbf{59.59 ± 0.11}  & \textbf{0.81 ± 0.03}  & \textbf{7.93 ± 0.03}  & \textbf{0.33 ± 0.00}  & \textbf{0.24 ± 0.11}  \\
        GPT-4O Mini + Image View   & 57.74 ± 2.59  & 0.79 ± 0.02  & 7.79 ± 0.03  & 0.30 ± 0.01  & 0.23 ± 0.08  \\
        \hline
        GPT-4O                     & 57.18 ± 0.71  & 0.66 ± 0.01  & 7.01 ± 0.14  & 0.11 ± 0.02  & 0.08 ± 0.04  \\
        GPT-4O + Text View         & \textbf{58.25 ± 0.03}  & 0.79 ± 0.01  & \textbf{7.86 ± 0.05}  & \textbf{0.31 ± 0.00}  & 0.23 ± 0.08  \\
        GPT-4O + Image View        & 55.75 ± 1.49  & \textbf{0.79 ± 0.01}  & 7.80 ± 0.08  & \textbf{0.32 ± 0.01}  & \textbf{0.26 ± 0.10}  \\
        \hline
    \end{tabular}
    }
    \caption{Mean ± standard deviation of diversity metrics across six different sample sizes ranging from $100$ to $2000$ per each prompt (10 prompts), with $max\_length=125$. Detailed results are provided in Appendix \ref{appendix:a}.}
    \label{tab:mean_results_diversity}
\end{table}

Table \ref{tab:novelty_gpt_ds} illustrates the novelty score of two well-known large language models GPT-4o and DeepSeek-R1 and impact of our method. By incorporating multi-view embeddings, we enriched the input representation, leading to the generation of more novel and diverse responses from these models. This analysis provides valuable insights into how multi-view embedding strategies influence novelty detection and enhance the creativity of LLM outputs. The results show that for GPT-4o, our approach in some cases led to around ninefold improvement in novelty when text or image view embeddings were applied. Similarly, for DeepSeek-R1—one of the most inherently creative models—our method resulted in approximately a twofold increase in the novelty score. Additional results for other models, evaluated solely using SBERT as the novelty detector, are presented in Table \ref{tab:novelty_sbert_2} in Appendix \ref{appendix:a}. % plot should be added here


% two columns for GPT4o results
% \begin{table}
% \centering
% \caption{Results on percentage of novelty score across different models according two novelty detectors GPT-4o and SBERT}
% \label{tab:novelty_gpt_ds}
% \vspace{1em}
% \begin{tabular}{lcccccc}
% \toprule
%  & \multicolumn{2}{c}{num\_samples = 100} & \multicolumn{2}{c}{num\_samples = 250} & \multicolumn{2}{c}{num\_samples = 500} \\
% \cmidrule(lr){2-3} \cmidrule(lr){4-5} \cmidrule(lr){6-7}
% Model & GPT-4o & SBERT & GPT-4o & SBERT & GPT-4o & SBERT \\
% \midrule
% GPT4o & 10.60 & 5.4 & 0 & 3.52 & 0 & 2.9 \\
% GPT4o + Text View & 29.30 & \underline{40.2} & 0 & \textbf{32.08} & 0 & \textbf{24.48} \\
% GPT4o + Image View & 42.60 & \textbf{47.3} & 0 & \underline{30.44} & 0 & \underline{22.54} \\ \hline
% DeepSeek-R1 & 27.40 & 43.6 & 0 & 36.24 & 0 & 31.82 \\
% DeepSeek-R1 + Text View & 46.20 & \textbf{75.3} & 0 & \underline{63.88} & 0 & \underline{55.86} \\
% DeepSeek-R1 + Image View & 40.60 & \underline{73.7} & 0 & \textbf{65.04} & 0 & \textbf{56.18} \\
% \bottomrule
% \end{tabular}
% \end{table}

\begin{table}
\centering
\resizebox{\linewidth}{!}{
\begin{tabular}{lcccccc}
\toprule
 & \multicolumn{2}{c}{num\_samples = 100} & \multicolumn{2}{c}{num\_samples = 250} & \multicolumn{1}{c}{num\_samples = 500} \\
\cmidrule(lr){2-3} \cmidrule(lr){4-5} \cmidrule(lr){6-6}
Model & GPT-4o & SBERT & GPT-4o & SBERT & SBERT \\
\midrule
GPT-4o & 10.60 & 5.4 & 7.28 & 3.52 & 2.9 \\
GPT-4o + Text View & \underline{29.30} & \underline{40.2} & \underline{20.4} & \textbf{32.08} & \textbf{24.48} \\
GPT-4o + Image View & \textbf{42.60} & \textbf{47.3} & \textbf{25.52} & \underline{30.44} & \underline{22.54} \\ \hline
DeepSeek-R1 & 27.40 & 43.6 & 18.6 & 36.24 & 31.82 \\
DeepSeek-R1 + Text View & \textbf{46.20} & \textbf{75.3} & \textbf{38.22} & \underline{63.88} & \underline{55.86} \\
DeepSeek-R1 + Image View & \underline{40.60} & \underline{73.7} & \underline{27.56} & \textbf{65.04} & \textbf{56.18} \\
\bottomrule
\end{tabular}}
\caption{Results on percentage of novelty score across different models according to two novelty detectors, GPT-4o and SBERT. More results for other models exist in Table \ref{tab:novelty_sbert_2}.}
\label{tab:novelty_gpt_ds}
\end{table}

The evaluation of answer correctness across different models is presented in Table \ref{tab:correctness_combined_single}, considering two aspects introduced in Section \ref{subsubsec: correctness_measure}. We assess the correctness of generated answers based on their relevance to the input prompt, using GPT-4o as the correctness evaluator and evaluates the correctness of language models from an English language structure perspective with DeepSeekV3. The results indicate that all models achieve high correctness scores in this aspect. Figure \ref{fig:final_comparison} summarizes the performance of our proposed method (multi-view embedding) alongside standard model variants. We report three key metrics: diversity, novelty, and correctness. Diversity quantifies the variety of generated responses on scale of [0,1], novelty measures the proportion of novel content (expressed in percentages), and correctness reflects the relevance of the generated text to the given prompt.
Overall, the data show that applying our method increases both diversity and novelty across various models. For instance, with Qwen, the baseline achieves a diversity of 0.41, novelty of 32.7\%, and correctness of 93.77\%. When we apply the text-view, diversity rises to 0.56 and novelty to 68.7\%, albeit with a correctness drop to 76.6\%. Similarly, the image-view offers 0.50 diversity and 57.2\% novelty at a correctness of 82.5\%. These patterns exemplify a trade-off: while the generated content becomes more varied and inventive, alignment with the prompt can decrease.
To further evaluate whether these gains hold when only correct (i.e., relevant) answers are considered, we also compute diversity and novelty exclusively for the subset of correct responses. Even under this stricter measurement, the Qwen variants (text view and image view) maintain higher diversity and novelty than the baseline. This finding underscores that our method fosters creativity and variety without entirely sacrificing correctness..

\begin{table}[t] % [t] or [h] depending on your preference
\centering
\resizebox{\linewidth}{!}{%
\begin{tabular}{lcc}
\toprule
 & \multicolumn{2}{c}{\textbf{Correctness Measure}} \\
\cmidrule(lr){2-3}
\textbf{Model} & \textbf{Correctness (\%)} & \textbf{Language Structure ([1,10])} \\
\midrule
GPT-2 & 7.20 & 3.29 \\
GPT-2 + Text View & 2.40 & 2.40 \\
\hline
Qwen & 93.77 & 7.57 \\
Qwen + Text View & 76.60 & 7.52 \\
Qwen + Image View & 82.50 & 6.64 \\
\hline
LlaMa-3.2 & 85.60 & 8.42 \\
LlaMa-3.2 + Text View & 83.10 & 7.38 \\
LlaMa-3.2 + Image View & 90.30 & 7.79 \\
\hline
DeepSeek-R1 & 91.80 & 7.96 \\
DeepSeek-R1 + Text View & 81.00 & 7.15 \\
DeepSeek-R1 + Image View & 53.90 & 6.10 \\
\hline
GPT-4o mini & 99.81 & 8.70 \\
GPT-4o mini + Text View & 91.00 & 8.12 \\
GPT-4o mini + Image View & 87.00 & 7.71 \\
\hline
GPT-4o & 99.60 & 8.07 \\
GPT-4o + Text View & 92.60 & 8.05 \\
GPT-4o + Image View & 94.60 & 8.06 \\
\bottomrule
\end{tabular}%
}
\caption{Correctness (\%) based on GPT-4o and language structure based on DeepSeekV3 ([1,10]). $num\_sample =100$ per each prompt (10 prompts).}
\label{tab:correctness_combined_single}
\end{table}


% \begin{table*}[h!]
%     \centering
%     \begin{minipage}[t]{0.48\linewidth} % Align from top
%         \centering
%         %\vspace{1em}
%         \vspace{0pt} % Align from top
%         \begin{tabular}{lc}
%             \toprule
%             Model & GPT-4o  \\ 
%             \midrule
%             GPT-2 & 7.20 \\
%             GPT-2 + Text View & 2.40 \\ \hline
%             Qwen & 93.77 \\
%             Qwen + Text View & 76.60 \\
%             Qwen + Image View & 82.50 \\ \hline
%             LlaMa-3.2 & 85.6 \\
%             LlaMa-3.2 + Text View & 83.1 \\
%             LlaMa-3.2 + Image View & 90.3 \\ \hline
%             DeepSeek-R1 & 91.80 \\
%             DeepSeek-R1 + Text View & 81.00 \\
%             DeepSeek-R1 + Image View & 53.90 \\ \hline
%             GPT-4o mini & 99.81 \\
%             GPT-4o mini + Text View & 91.00 \\
%             GPT-4o mini + Image View & 87.00 \\ \hline
%             GPT-4o & 99.60 \\
%             GPT-4o + Text View & 92.60 \\
%             GPT-4o + Image View & 94.60 \\            
%             \bottomrule
%         \end{tabular}
%         \label{tab:correctness_gpt4}
%         \subcaption{Correctness results (\%) }
%     \end{minipage}
%     \hfill
%     \begin{minipage}[t]{0.48\linewidth} % Align from top
%         \centering
%         %\vspace{1em}
%         \vspace{0pt} % Align from top
%         \begin{tabular}{lc}
%             \toprule
%             Model & DeepSeekV3  \\ 
%             \midrule
%             GPT-2 & 3.29 \\
%             GPT-2 + Text View & 2.40 \\ \hline
%             Qwen & 0 \\
%             Qwen + Text View & 0 \\
%             Qwen + Image View & 0 \\ \hline
%             LlaMa-3.2 & 0 \\
%             LlaMa-3.2 + Text View & 0 \\
%             LlaMa-3.2 + Image View & 0 \\ \hline
%             DeepSeek-R1 & 7.96 \\
%             DeepSeek-R1 + Text View & 7.15 \\
%             DeepSeek-R1 + Image View & 6.10 \\ \hline
%             GPT-4o mini & 0 \\
%             GPT-4o mini + Text View & 00 \\
%             GPT-4o mini + Image View & 0 \\ \hline
%             GPT-4o & 8.07 \\
%             GPT-4o + Text View & 8.05 \\
%             GPT-4o + Image View & 8.06 \\
%             \bottomrule
%         \end{tabular}
%         \subcaption{Mean score [1,10] correctness from English structure aspect.}
%         \label{tab:correctness_english_structure}
%     \end{minipage}
%     \caption{(a) and (b) show correctness results for different evaluation aspects for $num\_sample = 100$ per prompt (10 prompts).}
%     \label{tab:correctness_combined}
% \end{table*}



% \begin{table}
% \centering
% \caption{correctness results in (\%) for $num\_sample = 100$ per each prompt (10 prompts).}
% \label{tab:correctness_gpt4}
% \vspace{1em}
% \begin{tabular}{lc}
% \toprule
% Model & GPT-4o  \\ %(\%)
% \midrule
% GPT4o mini & 99.81 \\
% GPT4o mini + Text View & 91.00 \\
% GPT4o mini + Image View & 87.00 \\ \hline
% Qwen & 93.77 \\
% Qwen + Text View & 76.60 \\
% Qwen + Image View & 82.50 \\ \hline
% GPT4o & 99.60 \\
% GPT4o + Text View & 92.60 \\
% GPT4o + Image View & 94.60 \\ \hline
% DeepSeek-R1 & 91.80 \\
% DeepSeek-R1 + Text View & 81.00 \\
% DeepSeek-R1 + Image View & 53.90 \\
% \bottomrule
% \end{tabular}
% \end{table}

% \begin{table}
% \centering
% \caption{Mean score ranging from 1 to 10 for correctness score results form English structure aspects for $num\_sample = 100$ per each prompt (10 prompts).}
% \label{tab:correctness_english structure}
% \vspace{1em}
% \begin{tabular}{lc}
% \toprule
% Model & DeepSeekV3  \\ %(\%)
% \midrule
% GPT2 & 3.29 \\
% GPT2 + Text View & 2.40 \\ \hline
% GPT4o & 8.07 \\
% GPT4o + Text View & 8.05 \\
% GPT4o + Image View & 8.06 \\ \hline
% DeepSeek-R1 & 7.96 \\
% DeepSeek-R1 + Text View & 7.15 \\
% DeepSeek-R1 + Image View & 6.10 \\
% \bottomrule
% \end{tabular}
% \end{table}
\begin{figure*}[t]
  \centering
  \includegraphics[width=0.45\linewidth]{images/qwen.png} \hspace{3em}
  \includegraphics[width=0.45\linewidth]{images/gpt2.png} \\[+1mm]
  \includegraphics[width=0.45\linewidth]{images/llama.png}\hspace{3em}
  \includegraphics[width=0.45\linewidth]{images/dsr1.png} \\[+1mm]
  \includegraphics[width=0.45\linewidth]{images/4omini.png} \hspace{3em}
  \includegraphics[width=0.45\linewidth]{images/4o.png}
  \caption{Diversity, Novelty, and Correctness (All answers) vs (Correct answers) For $num\_samples = 100$. Diversity measure is $Self\_BLUE$, Novelty measure is based on $SBERT$, and Correctness measure is based on $GPT-4o$.}
  \label{fig:final_comparison}
\end{figure*}




% \section{Ablation Study}
% While our method significantly enhances the diversity and novelty of generated outputs across different language models, we conducted an ablation study to examine its impact on correctness and English language structure. The results indicate that increasing output diversity and novelty can sometimes lead to a decline in correctness and grammatical quality. In particular, as shown in Table \ref{tab:correctness_combined}, applying multi-view embeddings led to a slight decrease in correctness scores, especially for models that originally produced highly relevant responses. This suggests that while the model explores a broader range of responses, it may deviate from the most precise or expected answer. Similarly, Table \ref{tab:correctness_english_structure} demonstrates that the grammatical correctness of responses is also affected, with models generating more varied but less structurally refined answers. These findings highlight the trade-off between novelty and correctness, a well-known challenge in text generation. While increasing diversity enriches the model's expressiveness and reduces repetitiveness, it can also introduce syntactic errors or responses that, while novel, may not align perfectly with human expectations. %This analysis underscores the importance of balancing novelty and correctness, particularly for applications where factual accuracy and linguistic quality are critical.

\section{Conclusion}
%In this work we introduce multi-view embedding, a model-agnostic method to enrich the input prompt from different textual and visual sources to increase the diversity and novelty of the generated answers for both opensource and closed source LLMs. Also we introduce a framework to measure the diversity, novelty, and correctness of generated results of LLMs and validate both our method and framework on real world datasets and also $469k$ generated answers from different LLMs 
In this work, we present multi-view embedding, a model-agnostic approach that enriches the input prompt with diverse textual and visual sources to enhance the diversity and novelty of generated responses for both open-source and closed-source LLMs. Additionally, we introduce a framework to assess the diversity, novelty, and correctness of the generated outputs from LLMs. We validate both our method and framework using real-world datasets and $909k$ generated answers from various LLMs.

\section{Future Work}
%As shown in this paper, using textual or visual views increase the output diversity and creativity of existing language models. In our future works we want to address this issue another type of views like videos, songs, maps, or even mathematical equation as multi-view embeddings what effect on the model performance. We also want to extend our introduced framework from measuring diversity, novelty, and correctness to another areas like effectiveness and usefulness to provide more accurate evaluation of behavior of large language models.
As demonstrated in this paper, incorporating textual or visual views enhances the output diversity and creativity of existing language models. In our future work, we aim to explore the impact of additional view types, such as videos, songs, maps, or even mathematical equations, as multi-view embeddings on model performance. Furthermore, we plan to extend our proposed framework to assess not only diversity, novelty, and correctness, but also effectiveness and usefulness, in order to provide a more comprehensive evaluation of the behavior of large language models.

\clearpage
\clearpage
\section{Limitations} 
Despite the promising results, our method heavily depends on the quality and relevance of the views used during generation. If text or image views are not closely related to the input prompt, the model may produce outputs with lower correctness and reduced novelty. Additionally, generating these views—particularly in large numbers—can itself pose challenges. In our work, we use GPT-4o as a multi-view text generator; however, if many distinct views are requested, their diversity and novelty may diminish. This means that even if the underlying mechanism can foster creativity, oversaturating the process with a high volume of views can lead to repetitive or less innovative content. Finally, while our approach can enhance creativity and diversity for a moderate number of outputs, producing a large number of responses inevitably reduces both novelty and variation. These factors suggest that our method is most effective for situations where a bounded number of creative yet coherent outputs is desired, rather than an unbounded volume of responses.






















% \section{Engines}

% To produce a PDF file, pdf\LaTeX{} is strongly recommended (over original \LaTeX{} plus dvips+ps2pdf or dvipdf).
% The style file \texttt{acl.sty} can also be used with
% lua\LaTeX{} and
% Xe\LaTeX{}, which are especially suitable for text in non-Latin scripts.
% The file \texttt{acl\_lualatex.tex} in this repository provides
% an example of how to use \texttt{acl.sty} with either
% lua\LaTeX{} or
% Xe\LaTeX{}.

% \section{Preamble}

% The first line of the file must be
% \begin{quote}
% \begin{verbatim}
% \documentclass[11pt]{article}
% \end{verbatim}
% \end{quote}

% To load the style file in the review version:
% \begin{quote}
% \begin{verbatim}
% \usepackage[review]{acl}
% \end{verbatim}
% \end{quote}
% For the final version, omit the \verb|review| option:
% \begin{quote}
% \begin{verbatim}
% \usepackage{acl}
% \end{verbatim}
% \end{quote}

% To use Times Roman, put the following in the preamble:
% \begin{quote}
% \begin{verbatim}
% \usepackage{times}
% \end{verbatim}
% \end{quote}
% (Alternatives like txfonts or newtx are also acceptable.)

% Please see the \LaTeX{} source of this document for comments on other packages that may be useful.

% Set the title and author using \verb|\title| and \verb|\author|. Within the author list, format multiple authors using \verb|\and| and \verb|\And| and \verb|\AND|; please see the \LaTeX{} source for examples.

% By default, the box containing the title and author names is set to the minimum of 5 cm. If you need more space, include the following in the preamble:
% \begin{quote}
% \begin{verbatim}
% \setlength\titlebox{<dim>}
% \end{verbatim}
% \end{quote}
% where \verb|<dim>| is replaced with a length. Do not set this length smaller than 5 cm.

% \section{Document Body}

% \subsection{Footnotes}

% Footnotes are inserted with the \verb|\footnote| command.\footnote{This is a footnote.}

% \subsection{Tables and figures}

% See Table~\ref{tab:accents} for an example of a table and its caption.
% \textbf{Do not override the default caption sizes.}

% \begin{table}
%   \centering
%   \begin{tabular}{lc}
%     \hline
%     \textbf{Command} & \textbf{Output} \\
%     \hline
%     \verb|{\"a}|     & {\"a}           \\
%     \verb|{\^e}|     & {\^e}           \\
%     \verb|{\`i}|     & {\`i}           \\
%     \verb|{\.I}|     & {\.I}           \\
%     \verb|{\o}|      & {\o}            \\
%     \verb|{\'u}|     & {\'u}           \\
%     \verb|{\aa}|     & {\aa}           \\\hline
%   \end{tabular}
%   \begin{tabular}{lc}
%     \hline
%     \textbf{Command} & \textbf{Output} \\
%     \hline
%     \verb|{\c c}|    & {\c c}          \\
%     \verb|{\u g}|    & {\u g}          \\
%     \verb|{\l}|      & {\l}            \\
%     \verb|{\~n}|     & {\~n}           \\
%     \verb|{\H o}|    & {\H o}          \\
%     \verb|{\v r}|    & {\v r}          \\
%     \verb|{\ss}|     & {\ss}           \\
%     \hline
%   \end{tabular}
%   \caption{Example commands for accented characters, to be used in, \emph{e.g.}, Bib\TeX{} entries.}
%   \label{tab:accents}
% \end{table}

% As much as possible, fonts in figures should conform
% to the document fonts. See Figure~\ref{fig:experiments} for an example of a figure and its caption.

% Using the \verb|graphicx| package graphics files can be included within figure
% environment at an appropriate point within the text.
% The \verb|graphicx| package supports various optional arguments to control the
% appearance of the figure.
% You must include it explicitly in the \LaTeX{} preamble (after the
% \verb|\documentclass| declaration and before \verb|\begin{document}|) using
% \verb|\usepackage{graphicx}|.

% \begin{figure}[t]
%   \includegraphics[width=\columnwidth]{example-image-golden}
%   \caption{A figure with a caption that runs for more than one line.
%     Example image is usually available through the \texttt{mwe} package
%     without even mentioning it in the preamble.}
%   \label{fig:experiments}
% \end{figure}

% \begin{figure*}[t]
%   \includegraphics[width=0.48\linewidth]{example-image-a} \hfill
%   \includegraphics[width=0.48\linewidth]{example-image-b}
%   \caption {A minimal working example to demonstrate how to place
%     two images side-by-side.}
% \end{figure*}

% \subsection{Hyperlinks}

% Users of older versions of \LaTeX{} may encounter the following error during compilation:
% \begin{quote}
% \verb|\pdfendlink| ended up in different nesting level than \verb|\pdfstartlink|.
% \end{quote}
% This happens when pdf\LaTeX{} is used and a citation splits across a page boundary. The best way to fix this is to upgrade \LaTeX{} to 2018-12-01 or later.

% \subsection{Citations}

% \begin{table*}
%   \centering
%   \begin{tabular}{lll}
%     \hline
%     \textbf{Output}           & \textbf{natbib command} & \textbf{ACL only command} \\
%     \hline
%     \citep{Gusfield:97}       & \verb|\citep|           &                           \\
%     \citealp{Gusfield:97}     & \verb|\citealp|         &                           \\
%     \cite{Gusfield:97}       & \verb|\cite|           &                           \\
%     \citeyearpar{Gusfield:97} & \verb|\citeyearpar|     &                           \\
%     \citeposs{Gusfield:97}    &                         & \verb|\citeposs|          \\
%     \hline
%   \end{tabular}
%   \caption{\label{citation-guide}
%     Citation commands supported by the style file.
%     The style is based on the natbib package and supports all natbib citation commands.
%     It also supports commands defined in previous ACL style files for compatibility.
%   }
% \end{table*}

% Table~\ref{citation-guide} shows the syntax supported by the style files.
% We encourage you to use the natbib styles.
% You can use the command \verb|\cite| (cite in text) to get ``author (year)'' citations, like this citation to a paper by \cite{Gusfield:97}.
% You can use the command \verb|\citep| (cite in parentheses) to get ``(author, year)'' citations \citep{Gusfield:97}.
% You can use the command \verb|\citealp| (alternative cite without parentheses) to get ``author, year'' citations, which is useful for using citations within parentheses (e.g. \citealp{Gusfield:97}).

% A possessive citation can be made with the command \verb|\citeposs|.
% This is not a standard natbib command, so it is generally not compatible
% with other style files.

% \subsection{References}

% \nocite{Ando2005,andrew2007scalable,rasooli-tetrault-2015}

% The \LaTeX{} and Bib\TeX{} style files provided roughly follow the American Psychological Association format.
% If your own bib file is named \texttt{custom.bib}, then placing the following before any appendices in your \LaTeX{} file will generate the references section for you:
% \begin{quote}
% \begin{verbatim}
% \bibliography{custom}
% \end{verbatim}
% \end{quote}

% You can obtain the complete ACL Anthology as a Bib\TeX{} file from \url{https://aclweb.org/anthology/anthology.bib.gz}.
% To include both the Anthology and your own .bib file, use the following instead of the above.
% \begin{quote}
% \begin{verbatim}
% \bibliography{anthology,custom}
% \end{verbatim}
% \end{quote}

% Please see Section~\ref{sec:bibtex} for information on preparing Bib\TeX{} files.

% \subsection{Equations}

% An example equation is shown below:
% \begin{equation}
%   \label{eq:example}
%   A = \pi r^2
% \end{equation}

% Labels for equation numbers, sections, subsections, figures and tables
% are all defined with the \verb|\label{label}| command and cross references
% to them are made with the \verb|\ref{label}| command.

% This an example cross-reference to Equation~\ref{eq:example}.

% \subsection{Appendices}

% Use \verb|\appendix| before any appendix section to switch the section numbering over to letters. See Appendix~\ref{sec:appendix} for an example.

% \section{Bib\TeX{} Files}
% \label{sec:bibtex}

% Unicode cannot be used in Bib\TeX{} entries, and some ways of typing special characters can disrupt Bib\TeX's alphabetization. The recommended way of typing special characters is shown in Table~\ref{tab:accents}.

% Please ensure that Bib\TeX{} records contain DOIs or URLs when possible, and for all the ACL materials that you reference.
% Use the \verb|doi| field for DOIs and the \verb|url| field for URLs.
% If a Bib\TeX{} entry has a URL or DOI field, the paper title in the references section will appear as a hyperlink to the paper, using the hyperref \LaTeX{} package.

% \section*{Limitations}

% Since December 2023, a "Limitations" section has been required for all papers submitted to ACL Rolling Review (ARR). This section should be placed at the end of the paper, before the references. The "Limitations" section (along with, optionally, a section for ethical considerations) may be up to one page and will not count toward the final page limit. Note that these files may be used by venues that do not rely on ARR so it is recommended to verify the requirement of a "Limitations" section and other criteria with the venue in question.

% \section*{Acknowledgments}

% This document has been adapted
% by Steven Bethard, Ryan Cotterell and Rui Yan
% from the instructions for earlier ACL and NAACL proceedings, including those for
% ACL 2019 by Douwe Kiela and Ivan Vuli\'{c},
% NAACL 2019 by Stephanie Lukin and Alla Roskovskaya,
% ACL 2018 by Shay Cohen, Kevin Gimpel, and Wei Lu,
% NAACL 2018 by Margaret Mitchell and Stephanie Lukin,
% Bib\TeX{} suggestions for (NA)ACL 2017/2018 from Jason Eisner,
% ACL 2017 by Dan Gildea and Min-Yen Kan,
% NAACL 2017 by Margaret Mitchell,
% ACL 2012 by Maggie Li and Michael White,
% ACL 2010 by Jing-Shin Chang and Philipp Koehn,
% ACL 2008 by Johanna D. Moore, Simone Teufel, James Allan, and Sadaoki Furui,
% ACL 2005 by Hwee Tou Ng and Kemal Oflazer,
% ACL 2002 by Eugene Charniak and Dekang Lin,
% and earlier ACL and EACL formats written by several people, including
% John Chen, Henry S. Thompson and Donald Walker.
% Additional elements were taken from the formatting instructions of the \emph{International Joint Conference on Artificial Intelligence} and the \emph{Conference on Computer Vision and Pattern Recognition}.

% Bibliography entries for the entire Anthology, followed by custom entries
%\bibliography{anthology,custom}
% Custom bibliography entries only
%\bibliographystyle{acl_natbib}
 \bibliography{acl_latex}
% % This must be in the first 5 lines to tell arXiv to use pdfLaTeX, which is strongly recommended.
\pdfoutput=1
% In particular, the hyperref package requires pdfLaTeX in order to break URLs across lines.

\documentclass[11pt]{article}

% Change "review" to "final" to generate the final (sometimes called camera-ready) version.
% Change to "preprint" to generate a non-anonymous version with page numbers.
\usepackage[final]{acl}

% Standard package includes
\usepackage{times}
\usepackage{latexsym}

% For proper rendering and hyphenation of words containing Latin characters (including in bib files)
\usepackage[T1]{fontenc}
% For Vietnamese characters
% \usepackage[T5]{fontenc}
% See https://www.latex-project.org/help/documentation/encguide.pdf for other character sets

% This assumes your files are encoded as UTF8
\usepackage[utf8]{inputenc}

% This is not strictly necessary, and may be commented out,
% but it will improve the layout of the manuscript,
% and will typically save some space.
\usepackage{microtype}

% This is also not strictly necessary, and may be commented out.
% However, it will improve the aesthetics of text in
% the typewriter font.
\usepackage{inconsolata}

%Including images in your LaTeX document requires adding
%additional package(s)
\usepackage{graphicx}
\usepackage{xcolor}

% If the title and author information does not fit in the area allocated, uncomment the following
%
%\setlength\titlebox{<dim>}
%
% and set <dim> to something 5cm or larger.

\usepackage{booktabs}
\usepackage{hyperref}
\usepackage{multirow}
\usepackage{multicol}
\usepackage[most]{tcolorbox}
\usepackage{adjustbox}
\usepackage{graphicx}
\usepackage{fullpage}
\usepackage{times}
\usepackage{fancyhdr,graphicx,amsmath,amssymb}
%\usepackage[ruled,vlined]{algorithm2e}
\usepackage{algorithm}
\usepackage{algpseudocode}
\usepackage{booktabs}
\usepackage{adjustbox}
\usepackage{url}
\usepackage{hyperref}
\usepackage{amssymb}
\usepackage{marvosym}
\usepackage{multirow}
\usepackage{subcaption}
\DeclareMathOperator*{\argmax}{arg\,max}
\DeclareMathOperator*{\argmin}{arg\,min}


\newtcolorbox{promptbox}[2][]{
  colback=gray!10,
  colframe=gray!50,
  arc=3mm,
  boxrule=1pt,
  left=10pt,
  right=10pt,
  top=8pt,
  bottom=8pt,
  before skip=12pt,
  after skip=12pt,
  fonttitle=\bfseries,
  title=#2,
  #1
}

\title{Quality-Aware Decoding: Unifying Quality Estimation and Decoding}

% Author information can be set in various styles:
% For several authors from the same institution:
% \author{Author 1 \and ... \and Author n \\
%         Address line \\ ... \\ Address line}
% if the names do not fit well on one line use
%         Author 1 \\ {\bf Author 2} \\ ... \\ {\bf Author n} \\
% For authors from different institutions:
% \author{Author 1 \\ Address line \\  ... \\ Address line
%         \And  ... \And
%         Author n \\ Address line \\ ... \\ Address line}
% To start a separate ``row'' of authors use \AND, as in
% \author{Author 1 \\ Address line \\  ... \\ Address line
%         \AND
%         Author 2 \\ Address line \\ ... \\ Address line \And
%         Author 3 \\ Address line \\ ... \\ Address line}

\author{Sai Koneru$^{1}$,
  Matthias Huck$^{2}$,
  Miriam Exel$^{2}$, \textnormal{and}
  Jan Niehues$^{1}$ \\
  $^{1}$ Karlsruhe Institute of Technology \\
  $^{2}$ SAP SE, Dietmar-Hopp-Allee 16, 69190 Walldorf, Germany \\
  \texttt{\{sai.koneru, jan.niehues\}@kit.edu} \\
  \texttt{\{matthias.huck, miriam.exel\}@sap.com}}

%\author{
%  \textbf{First Author\textsuperscript{1}},
%  \textbf{Second Author\textsuperscript{1,2}},
%  \textbf{Third T. Author\textsuperscript{1}},
%  \textbf{Fourth Author\textsuperscript{1}},
%\\
%  \textbf{Fifth Author\textsuperscript{1,2}},
%  \textbf{Sixth Author\textsuperscript{1}},
%  \textbf{Seventh Author\textsuperscript{1}},
%  \textbf{Eighth Author \textsuperscript{1,2,3,4}},
%\\
%  \textbf{Ninth Author\textsuperscript{1}},
%  \textbf{Tenth Author\textsuperscript{1}},
%  \textbf{Eleventh E. Author\textsuperscript{1,2,3,4,5}},
%  \textbf{Twelfth Author\textsuperscript{1}},
%\\
%  \textbf{Thirteenth Author\textsuperscript{3}},
%  \textbf{Fourteenth F. Author\textsuperscript{2,4}},
%  \textbf{Fifteenth Author\textsuperscript{1}},
%  \textbf{Sixteenth Author\textsuperscript{1}},
%\\
%  \textbf{Seventeenth S. Author\textsuperscript{4,5}},
%  \textbf{Eighteenth Author\textsuperscript{3,4}},
%  \textbf{Nineteenth N. Author\textsuperscript{2,5}},
%  \textbf{Twentieth Author\textsuperscript{1}}
%\\
%\\
%  \textsuperscript{1}Affiliation 1,
%  \textsuperscript{2}Affiliation 2,
%  \textsuperscript{3}Affiliation 3,
%  \textsuperscript{4}Affiliation 4,
%  \textsuperscript{5}Affiliation 5
%\\
%  \small{
%    \textbf{Correspondence:} \href{mailto:email@domain}{email@domain}
%  }
%}

\begin{document}
\maketitle
\begin{abstract}
% Neural Machine Translation (NMT) has achieved high-quality translations in many scenarios, pushing the boundaries of tasks such as instruction-following and multimodal translation. 
Quality Estimation (QE) models for Neural Machine Translation (NMT) predict the quality of the hypothesis without having access to the reference.
An emerging research direction in NMT involves the use of QE models, which have demonstrated high correlations with human judgment and can enhance translations through Quality-Aware Decoding. Although several approaches have been proposed based on sampling multiple candidate translations and picking the best candidate, none have integrated these models directly into the decoding process. In this paper, we address this by proposing a novel token-level QE model capable of reliably scoring partial translations. We build a uni-directional QE model for this, as decoder models are inherently trained and efficient on partial sequences. We then present a decoding strategy that integrates the QE model for Quality-Aware decoding and demonstrate that the translation quality improves when compared to the N-best list re-ranking with state-of-the-art QE models (up to $1.39$ XCOMET-XXL $\uparrow$). Finally, we show that our approach provides significant benefits in document translation tasks, where the quality of N-best lists is typically suboptimal\footnote{Code can be found at \url{https://github.com/SAP-samples/quality-aware-decoding-translation}}
\end{abstract}
\section{Introduction}

Large language models (LLMs) have significantly impacted various Natural Language Processing (NLP) tasks \citep{brown2020language, jiang2023mistral, dubey2024llama}, including Neural Machine Translation (NMT). The field of NMT is transitioning from using dedicated encoder-decoder transformers \citep{vaswani2017attention, nllb2024scaling} to leveraging decoder-only LLM-based translation models \citep{kocmi2024findings}. This shift is driven by LLMs' ability to retain knowledge, handle large contexts, and follow instructions, learned during extensive pre-training \citep{xu2024contrastive, alves2024tower}. As a result, LLM-based MT models have achieved state-of-the-art translation quality \citep{kocmi2024findings}.

In parallel, Quality Estimation (QE) has become a well-researched subfield within NMT. QE models are trained to predict the quality of a translation without requiring access to the reference \citep{rei2021references,rei2022cometkiwi}. Interestingly, QE models can achieve performance in assessing translation quality that is comparable to MT evaluation models, which do have access to the reference \citep{zerva2024findings}.

This led to the question: "\textit{Can we integrate QE into the current translation process to improve quality?}" Incorporating QE into NMT offers several benefits. First, having a expert QE model guiding the decoding can further improve the quality. Second, by adapting the QE model with feedback from human annotators, we can generate future translations guided with the newly obtained feedback.

\begin{figure*}[!ht]
\includegraphics[width=\textwidth]{Figures/nbestlist_problem.png}
 \caption{Example from WMT'23 English → German \#ID: 10: The paragraph begins with 'Department of Homeland Security,' which should be translated as 'Ministerium für \textbf{I}nnere Sicherheit.' However, the top 25 beams do not contain the correct translation and begin with an error, making N-best list re-ranking insufficient. Although the top-5 tokens at the decoding contain the correct forms 'Inn' or 'Inner,' the probabilities split among them giving highest mass to the incorrect token 'inn.' Quality-Aware decoding can prevent errors with earlier integration.}
\label{fig:nbestlist}
\end{figure*}


Several approaches have been explored to integrate QE into the translation process. These include re-ranking the N-best list \citep{fernandes2022quality}, applying minimum Bayes risk (MBR) decoding on a quality-filtered N-best list \citep{tomani2024quality}, and training additional models for post-editing based on QE-predicted errors \citep{treviso2024xtower}. However, all these methods operate on fully generated sequences before the QE model can exert influence. Integrating QE earlier in the decoding process, referred in this paper as \textit{Quality-Aware Decoding}, could enhance translation quality and reduce reliance on the N-best list. This is especially relevant when dealing with long inputs as good translations during decoding are likely to be pruned and may need sampling larger number of finished hypothesis. We illustrate this in Figure \ref{fig:nbestlist}.

To achieve this, a QE model capable of predicting the quality of partial translations is required. However, current leading QE models face challenges in this area, as they are typically not trained to predict scores for incomplete hypotheses. \textit{Therefore, developing QE models that can handle partial translations is essential for implementing Quality-Aware Decoding during the translation process}.

In this work, we propose adapting LLM-based MT models to perform QE on partial translations and incorporating this model into the decoding. We create a token-level synthetic QE dataset using WMT Multidimensional Quality Metrics (MQM) data \citep{burchardt2013multidimensional, freitag2024llms}. We then adapt a uni-directional LLM-based MT model to predict whether a token is \textit{Good} or \textit{Bad}. Training QE models on these token-level tasks alleviates the data challenge and allows us to exploit the MQM data while simultaneously making the task easier for the model compared to predicting a score directly.

\begin{figure*}[!ht]
\includegraphics[width=\textwidth]{Figures/annotation_scheme.png}
 \caption{Token-level label annotation scheme using the MQM error tags. \textit{MASK} indicates that this token label will not be used in training to prevent incorrect learning signal.}
\label{fig:annotation}
\end{figure*}

Furthermore, integrating the QE model into NMT during decoding is not trivial, as we need to combine the QE estimates during decoding. Therefore, we use the decoding strategy from \citet{koneru2024plug}, and modify it to incorporate token-level predictions efficiently with the adapted QE model to provide real-time feedback during the decoding process. We summarize our main findings and contributions below.

\begin{itemize}
    \item We present a novel uni-directional QE model which estimates quality on incomplete hypotheses by averaging the probabilities of each token being classified as \textit{Good}. 
    
    %We demonstrate that it achieves improved correlation with human annotations on WMT 23 English $\rightarrow$ German, compared to the log probabilities of the same LLM-based NMT model.

    \item We propose a decoding strategy that combines the token-level QE model on partial hypothesis and the NMT model to perform Quality-Aware Decoding. 
    
    \item We show through experiments that early integration is essential and the translation quality is improved even when compared to re-ranking the N-best list with state-of-the-art QE models.

    \item We highlight the significance of our approach in document translation scenarios, where post-generation QE techniques fall short due to their reliance on the quality of the N-best list, a challenge that becomes more difficult as the input length increases.
\end{itemize}



\section{Quality-Aware Decoding}

The primary objective of this paper is to achieve Quality-Aware Decoding in MT. To accomplish this, it is essential to predict the quality of partial translations and integrate this information during the decoding process. Our approach proposes using one NMT model for generating translations and another adapted NMT model to predict the quality of the candidate translations produced by the first model.

First, we explain why relying solely on the NMT model to predict the quality of a hypothesis is insufficient and why an additional model is necessary. Next, we outline the adaptation of the NMT model for QE on partial translations, detailing the creation of a token-level QE dataset, the modifications made to the NMT model for this task, and the process of estimating the sentence-level quality score. Finally, we describe the algorithm used to incorporate the QE score into the decoding process.

\subsection{Decomposing Decoding: Translation + QE}
NMT models generate a token-by-token sequence and provide the probability of each token at the decoding step. The average of the log-probabilities is often used as a proxy to score the candidate during search. 

While NMT models are capable of generating high-quality translations, using the average log-probabilities of hypotheses as a scoring metric tends to yield poor correlation with actual translation quality \citep{eikema2020map, freitag2020bleu}. In many cases, a translation can continue in several different ways, all of which may be acceptable. If the starting tokens for these continuations differ, the probability mass may be spread across multiple options which is used during the search. However, from a quality perspective, all these continuations could still achieve a high score, as the QE scores are independent and need not sum to $1$.

Therefore, we propose a expert model that focuses on quality to estimate the scores better during decoding and  improve the search space leading to a better hypothesis.


% Therefore, relying solely on the average log-probabilities during decoding is not ideal, as it computes the score independently for each token and does not fully correlate with the overall quality of the current hypothesis.

\subsection{Quality Estimation on Partial Sequences}

% NMT models decode sequences token-by-token. 
To provide a quality score during decoding, the QE model must be capable of handling incomplete sequences. It should not penalize a sequence if there is a potential extension that could lead to a perfect translation.

Estimating the score in this way is not feasible with current QE models, such as COMET \citep{rei2021references}, as they were not trained for this specific task and cannot provide reliable scores in the context of partial translations. Hence, we need to develop a partial QE system.

When building a partial QE system, several factors need to be considered. First, should the model use a uni-directional or bi-directional architecture? A \textbf{uni-directional} model is more efficient, as it allows for caching the hidden states, which can then be used for subsequent steps without re-encoding, unlike a bi-directional model.

Next, we need to decide whether to predict the QE score at the sequence level or at the token level. For \textbf{token-level QE}, we can directly use data from MQM annotations, as we already know which tokens are \textit{Good} or \textit{Bad}. However, for segment-level scoring, we need to consider how to synthetically create the training data. 

% Additionally, COMET models are encoder-only architectures pre-trained on full sentences, rather than partial sentences as required in this case. Moreover, predicting the score of partial translations naturally favors decoder-only models due to their efficiency. New tokens only need to process the preceding sequence, avoiding the need to re-encode the entire sequence. Additionally, this approach simplifies training, as we do not require synthetically shorter samples. 

%  Furthermore, there is no readily available dataset containing partial translations along with their quality scores. Hence, we need to design the adaptation process with a QE model that is uni-directional and exploit already available human annotated data.

% \subsubsection{Token-level Quality Estimation}

Therefore, we decide adapt the uni-directional model into a token-level QE system that predicts whether each token is \textit{Good} or \textit{Bad} (a binary decision) by adding an additional classifier head. This adaptation enables us to estimate the score for a sequence by calculating the average probability that each token is classified as \textit{Good}. We hypothesize that adapting the model in this way, rather than directly predicting the score, provides greater stability, as the last hidden states inherently contain token-level information and do not require mapping the entire sequence to a single score.

For training this model, we leverage the WMT MQM data containing error annotations in MT outputs. We can treat tokens before an error as \textit{Good} and those containing inside an error as \textit{Bad}. Then, we can train in uni-directional manner where each token's label is predicted using only the preceding context in the hypothesis. This is crucial as we only have the preceding context to estimate the quality for partial hypothesis.

\subsubsection{Learning the Right Signal}

\begin{algorithm*}[!t]
\caption{Computing merged score of partial hypothesis with translation and token-level QE models.}
\begin{algorithmic}[1]
\setlength{\baselineskip}{1.2em}
\Procedure{MergeScore}{}
    \State \textbf{Input:}   Hypothesis tokens $h_1, h_2, h_3, \dots, h_{n}$, Translation Model $\mathcal{M}_{NMT}$, QE model $\mathcal{M}_{QE}$, Source sentence $\mathcal{S}$, Re-ranking weight $\alpha$,
    \State \textbf{Output:} $merged\_score$
    \State $Score_{NMT} \gets \frac{1}{n}\sum \log \mathcal{P}(h_1, h_2,\dots, h_{n}|\mathcal{S};\mathcal{M}_{NMT})$ 
    \State $Score_{QE} \gets \frac{1}{n}\sum \log \mathcal{P}(0_{1}, 0_{2},\dots,0_{n} | h_1, h_2,\dots, h_{n},\mathcal{S};\mathcal{M}_{QE})$ 
    \State $merged\_score \gets (\alpha) \times Score_{NMT} + (1 - \alpha) \times Score_{QE}$
\EndProcedure
\end{algorithmic}
\label{alg:joint}
\end{algorithm*}

The straightforward approach to creating labels is to assign $1$ to all tokens within the error span and $0$ otherwise. However, MQM annotations can mark errors from words to phrases, and the starting tokens of an error span may not always be wrong. This is illustrated in Figure \ref{fig:annotation}.

For example, consider the German sentence \textit{"Ich spiele Tennis"} translated by three different NMT systems, each annotated with MQM error labels. In this work, we focus on learning a binary decision: whether an error is present, ignoring error severity.

\textbf{System 1: No error}: The translation \textit{"I play Tennis"} is perfect, and all tokens are labeled as "\textit{Good}."

\textbf{System 2: Partial error}: The translation \textit{"I played Tennis"} has an error in the verb form ("played" instead of "play"). The error is in the token span \textit{"played"}, but not all tokens in this span are incorrect (e.g., "pla" is correct). Assigning a "\textit{Bad}" label to the entire span would lead to incorrect learning. A more refined approach is needed to mark errors accurately at the token level.

\textbf{System 3: Full error}: The translation \textit{"I enjoy Tennis"} contains an error in \textit{"enjoy"}, so all tokens in this span should be labeled as "\textit{Bad}."

It is not trivial to decide when the prefix of an error span is correct/incorrect. To achieve accurate labeling, we propose the following scheme:

\begin{itemize} \item Apply a \texttt{<MASK>} operation to all tokens within the error span. \item Only the last token in the span is assigned the label "\textit{Bad}", as the error is considered complete at the end of the span. \end{itemize}

If the error token is in the middle, we still train the model to predict "\textit{Bad}" in the end and let the model determine which tokens should be part of the error span during inference. This approach ensures that errors are identified without explicitly defining the error span. 

\subsubsection{Sequence-Level Quality Estimation}


After fine-tuning a token-level classification model to predict the quality of the tokens, we still need to map these predictions into a sequence-level score that can be integrated during the decoding process. There are several potential ways to achieve this.

One approach is to simply count how many tokens are classified as \textit{Bad} in the current hypothesis. However, this method has limitations. The number of errors should be normalized based on the length of the hypothesis to account for varying sizes. Additionally, converting the probabilities into a fixed number of error tokens would need to account for different error types according to the MQM format, as each error counts differently.

To avoid such strict scoring schemes, we take a simpler approach. We average the log probabilities of all tokens that are classified as \textit{Good}. This method inherently accounts for the length of the hypothesis, and it provides a score on the scale of log probabilities, which aligns with the decoding process. Therefore, we use this averaged log probability as a proxy metric for the QE score, where a higher score indicates better quality
(\textbf{Line 5} in Algorithm \ref{alg:joint}).

\subsubsection{Fusing Translation and Quality}

We can use a token-level QE system to evaluate the quality of a source and partial hypothesis during decoding. However, integrating these probabilities into all candidates is computationally expensive, as each beam considers extensions equal to the vocabulary size.

To address this, we adopt a simplified decoding strategy from \citet{koneru2024plug}, which ensembles models with different vocabularies. By adapting the same MT model for token-level QE, we simplify the merging process, as the vocabularies match. This restriction is reasonable, as it is also beneficial to leverage the knowledge learned by the specialized MT for token-level QE.

The core idea is to re-rank the top candidates at each decoding step using the QE model. After re-ranking, the translation and QE scores are merged, and the process repeats until the end-of-sentence token is generated, for each beam. This strategy allows us to efficiently incorporate the QE model’s estimate, improving translation quality.

During decoding, at each step, we have scores for $n$ beams and $V$ possible extensions from the vocabulary. In typical beam search, we select the top $n$ extensions and expand the hypothesis. To make the decoding process Quality-aware, we estimate the quality of these extensions. Since estimating all extensions is computationally expensive, we limit the candidates by selecting a specified number of top candidates.

To achieve this, we use a hyper-parameter $topk$, which selects the best $topk$ extensions for each beam. For each of these top $topk$ extensions, we compute a combined score, detailed in Algorithm \ref{alg:joint}. This combined score incorporates both the translation model score and the quality estimation score, ensuring the quality is considered during decoding.

For a top extension at decoding step $n$, let the current tokens be $h_1, h_2, h_3, \dots, h_n$. The NMT model score is computed as the average log probabilities of each token (Line 4). For the token-level QE model, we compute the average probability of each token being classified as '\textit{Good}' (Line 5). The merged score is equal to weighted linear combination of these probabilities, with weight $\alpha$ (Line 6).

Thus, to make the decoding process Quality-Aware, we first train a token-level QE system by adapting the same NMT model to ensure vocabulary matching. We then combine the scores from both models to improve the sequence estimates explored during search.


\begin{table*}[!ht]
\resizebox{2\columnwidth}{!}{
\begin{tabular}{@{}ccccc@{}}
\toprule
\multicolumn{1}{c|}{Model}            & \multicolumn{1}{c|}{Beams}                & \multicolumn{1}{c|}{Re-ranking}              & MetricX ($\downarrow$)     & XCOMET-XXL ($\uparrow$)    \\ \midrule
\multicolumn{5}{c}{\textit{English $\rightarrow$ German}}                                                                                                          \\ \midrule
\multicolumn{1}{c|}{Tower}            & \multicolumn{1}{c|}{5}                    & \multicolumn{1}{c|}{\_}                      & 2.52          & 86.93          \\
\multicolumn{1}{c|}{Tower}            & \multicolumn{1}{c|}{25}                   & \multicolumn{1}{c|}{XCOMET-XL QE}            & 2.37          & 87.79          \\
\multicolumn{1}{c|}{Tower}            & \multicolumn{1}{c|}{25}                   & \multicolumn{1}{c|}{Tower QE} & 2.38          & 87.40          \\
\multicolumn{1}{c|}{Tower + Tower QE} & \multicolumn{1}{c|}{5 (25* for Tower QE)} & \multicolumn{1}{c|}{\_}                      & 2.12          & 88.95          \\
\multicolumn{1}{c|}{Tower + Tower QE} & \multicolumn{1}{c|}{5 (25* for Tower QE)} & \multicolumn{1}{c|}{XCOMET-XL QE}            & \textbf{2.09} & \textbf{89.08} \\ \midrule
\multicolumn{5}{c}{\textit{Chinese $\rightarrow$ English}}                                                                                                         \\ \midrule
\multicolumn{1}{c|}{Tower}            & \multicolumn{1}{c|}{5}                    & \multicolumn{1}{c|}{\_}                      & 2.42          & 88.91          \\
\multicolumn{1}{c|}{Tower}            & \multicolumn{1}{c|}{25}                   & \multicolumn{1}{c|}{XCOMET-XL QE}            & 2.30          & 89.49          \\
\multicolumn{1}{c|}{Tower}            & \multicolumn{1}{c|}{25}                   & \multicolumn{1}{c|}{Tower QE} & 2.32          & 89.51          \\
\multicolumn{1}{c|}{Tower + Tower QE} & \multicolumn{1}{c|}{5 (25* for Tower QE)} & \multicolumn{1}{c|}{\_}                      & 2.26          & 89.82          \\
\multicolumn{1}{c|}{Tower + Tower QE} & \multicolumn{1}{c|}{5 (25* for Tower QE)} & \multicolumn{1}{c|}{XCOMET-XL QE}            & \textbf{2.24} & \textbf{90.00} \\ \bottomrule
\end{tabular}
}
\caption{Translation Quality on WMT23 English $\rightarrow$ German Test set. Both XCOMET and MetricX columns use reference for reporting translation quality where as XCOMET-XL QE does not use for re-ranking.}
\label{tab:qadecoding}
\end{table*}

\begin{table}[!ht]
\resizebox{\columnwidth}{!}{
\centering
\begin{tabular}{@{}c|ccc@{}}
\toprule
                                                                                      & Pearson        & Spearmann      & Kendall        \\ \midrule
COMETQE                                                                               & \textbf{44.41} & 41.29          & 31.19          \\ \midrule
COMETQE-XL                                                                            & 41.23          & \textbf{42.17} & \textbf{31.84} \\ \midrule
Tower Avg. Log Prob                                                                        & 32.32          & 16.74          & 12.77          \\ \midrule
\begin{tabular}[c]{@{}c@{}}Tower QE\end{tabular} & 40.56          & 33.96          & 25.87          \\ \bottomrule
\end{tabular}
}
\caption{Correlation on WMT 23 for English $\rightarrow$ German Test set. The scores are calculated after removing the few sentences labeled for hallucination detection. Best scores according to each coefficient are highlighted in \textbf{bold}.}
\label{tab:correlation}
\end{table}

\section{Experimental Setup}
\paragraph{Datasets:} We focus on two language directions given their availability of MQM data: English $\rightarrow$ German and Chinese $\rightarrow$ English. To train our token-level QE systems, we use the MQM datasets\footnote{https://github.com/google/wmt-mqm-human-evaluation} from WMT \citep{freitag2021experts}. Specifically, we use the datasets until 2022 for training, 2024 for validation, and 2023 for testing \citep{kocmi2024findings}. This setup is consistent with all the other QE metrics, and we do not use any additional data beyond these datasets.
\vspace{-0.1cm}
\paragraph{Models:} 
Our proposed approach achieves Quality-Aware decoding by combining an NMT model with a token-level QE model, where we adapt the same NMT for QE by adding a classification head. We use the state-of-the-art NMT model, Tower 7B\footnote{Unbabel/TowerInstruct-7B-v0.2} \citep{alves2024tower}, which provides high-quality translations and has already been exposed to MQM data during instruction-tuning. This ensures that the gains observed in our approach stem from integrating Quality-Aware decoding into the NMT process, rather than introducing new data. Additional details on training and hyper-parameters are provided in Appendix \ref{sec:training_detail}.
\vspace{-0.1cm}
\paragraph{Metrics:}
For reporting the translation quality, we consistently use XCOMET-XXL\footnote{Unbabel/XCOMET-XXL} \citep{guerreiro2024xcomet} and MetricX\footnote{google/metricx-24-hybrid-xl-v2p6} \citep{juraska2024metricx} \textbf{with the reference}. To compare with N-best list re-ranking, we use the XCOMET-XL QE\footnote{Unbabel/XCOMET-XL} \textbf{without the reference}. This approach allows us to avoid biasing toward a single metric during the re-ranking process and enables us to measure the gains achieved by differently trained metrics. 

\section{Results}



\begin{table*}[!ht]
\centering
\resizebox{2\columnwidth}{!}{
\begin{tabular}{@{}ccccc@{}}
\toprule
\multicolumn{1}{c|}{Model}            & \multicolumn{1}{c|}{Beams}                        & \multicolumn{1}{c|}{Re-ranking}               & MetricX ($\downarrow$) & XCOMET-XXL ($\uparrow$) \\ \midrule
\multicolumn{5}{c}{\textit{English $\rightarrow$ German}}                                                                                                          \\ \midrule
\multicolumn{1}{c|}{Tower}            & \multicolumn{1}{c|}{25}                           & \multicolumn{1}{c|}{XCOMET-XL QE}             & 2.37     & 87.79      \\
\multicolumn{1}{c|}{Tower}            & \multicolumn{1}{c|}{25}                           & \multicolumn{1}{c|}{Tower QE}         & 2.38     & 87.40      \\
\multicolumn{1}{c|}{Tower}            & \multicolumn{1}{c|}{25}                           & \multicolumn{1}{c|}{Tower Distill QE} & 2.38     & 87.39      \\
\multicolumn{1}{c|}{Tower + Tower QE} & \multicolumn{1}{c|}{5 (25* for Tower QE)}         & \multicolumn{1}{c|}{\_}                       & 2.12     & \textbf{88.95}      \\
\multicolumn{1}{c|}{Tower + Tower QE} & \multicolumn{1}{c|}{5 (25* for Tower Distill QE)} & \multicolumn{1}{c|}{\_}                       & \textbf{2.11}     & 88.76      \\ \bottomrule
\end{tabular}
}
\caption{Performance of Unidirectional QE trained with/without distillation on WMT23 English $\rightarrow$ German Test set. Best scores according to each metric are highlighted in \textbf{bold}.}
\label{tab:towerdistill}
\end{table*}


\begin{table*}[!ht]
\centering
\resizebox{2\columnwidth}{!}{
\begin{tabular}{@{}cccccc@{}}
\toprule
\multicolumn{1}{c|}{Model}            & \multicolumn{1}{c|}{Beams}                & \multicolumn{1}{c|}{Re-ranking}       & XCOMET-XL ($\uparrow$)     & \multicolumn{1}{c|}{XCOMET-XXL ($\uparrow$)}     & Impact                                                                                       \\ \midrule
\multicolumn{6}{c}{\textit{Paragraph-Level}}                                                                                                                                                                                                                                    \\ \midrule
\multicolumn{1}{c|}{Tower}            & \multicolumn{1}{c|}{25}                   & \multicolumn{1}{c|}{XCOMET-XL QE}     & \textbf{86.56} & \multicolumn{1}{c|}{87.79}          & \multirow{3}{*}{\begin{tabular}[c]{@{}c@{}}$\delta$ = + 1.16\\ (88.95 - 87.79)\end{tabular}} \\
\multicolumn{1}{c|}{Tower}            & \multicolumn{1}{c|}{25}                   & \multicolumn{1}{c|}{Tower QE} & 85.40          & \multicolumn{1}{c|}{87.40}          &                                                                                              \\
\multicolumn{1}{c|}{Tower + Tower QE} & \multicolumn{1}{c|}{5 (25* for Tower QE)} & \multicolumn{1}{c|}{\_}               & 86.36          & \multicolumn{1}{c|}{\textbf{88.95}} &                                                                                              \\ \midrule
\multicolumn{6}{c}{\textit{Sentence-Level}}                                                                                                                                                                                                                                     \\ \midrule
\multicolumn{1}{c|}{Tower}            & \multicolumn{1}{c|}{25}                   & \multicolumn{1}{c|}{XCOMET-XL QE}     & \textbf{86.42}          & \multicolumn{1}{c|}{87.68}          & \multirow{3}{*}{\begin{tabular}[c]{@{}c@{}}$\delta$ = + 0.38\\ (88.06 - 87.68)\end{tabular}} \\
\multicolumn{1}{c|}{Tower}            & \multicolumn{1}{c|}{25}                   & \multicolumn{1}{c|}{Tower QE} & 85.23          & \multicolumn{1}{c|}{87.41}          &                                                                                              \\
\multicolumn{1}{c|}{Tower + Tower QE} & \multicolumn{1}{c|}{5 (25* for Tower QE)} & \multicolumn{1}{c|}{\_}               & 85.96          & \multicolumn{1}{c|}{\textbf{88.06}}          &                                                                                              \\ \bottomrule
\end{tabular}
}
\caption{Impact of integrating Unidirectional QE during decoding with paragraphs vs sentences on WMT23 English $\rightarrow$ German Test set. $\delta$ denotes the improvement in translation quality from re-ranking N-best list with XCOMET-XL QE to integrating unidirectional Tower QE during the decoding. Best scores according to each metric are highlighted in \textbf{bold}.}
\label{tab:sentvspara}
\end{table*}



We conduct a series of experiments to validate the effectiveness of Quality-Aware decoding and identify the scenarios where it provides the most benefit. First, we evaluate whether our token-level QE model can better estimate sequence quality compared to the log probabilities of the NMT model. Next, we assess the impact of Quality-Aware decoding by comparing it with other approaches to determine if it improves translation quality. We also perform an ablation study to examine whether training the QE model on errors from the same NMT model enhances its performance. Finally, we explore the impact of source sentence length to highlight the limitations of N-best list re-ranking.

\subsection{Quality Estimation Performance}

First, we evaluate the agreement between the Tower-based token-level QE model (\textbf{Tower QE}) and human scores for a given hypothesis. It is only beneficial if we achieve higher correlation than the average of the NMT model log probabilities to show the need to integrate it during decoding. Therefore, we report the correlation with human scores of different models on WMT 23 English $\rightarrow$ German in Table \ref{tab:correlation}. 

We observe that the best-performing systems are the Comet QE models, which predict a single score using the full hypothesis. This is expected, as these models assess quality after the hypothesis is fully generated. In contrast, both log probabilities and Tower QE scores are based on the predicted token of each decoding step, using only the preceding context. Log probabilities perform poorly in this setup, while our proposed model, Tower QE, achieves twice the correlation with human judgments compared to log probabilities, despite scoring token by token with preceding context. This result highlights the potential of integrating our approach into the decoding process.

\subsection{Unified Decoding for NMT}


To validate the effectiveness of our unified decoding approach, we compare it with several baselines in Table \ref{tab:qadecoding}. First, we evaluate whether our approach outperforms generating translations with the NMT model alone. Next, we check if the quality of translations improves compared to N-best list re-ranking. To make the setups comparable, we set $topk$ and $num\_beams$ to $5$ and compare with re-ranking the top $25$ beams using XCOMET-XL. Finally, to demonstrate that re-ranking the N-best list remains a viable and complementary approach, we re-rank the top $5$ beams obtained from Quality-Aware decoding using the same QE model. 

We find that re-ranking with XCOMET-XL and Tower QE yields similar results, indicating that our partial QE model does not over-fit to any specific metric. Furthermore, we observe that the unified decoding approach outperforms N-best list re-ranking across both metrics in both language pairs. For example, the MetricX score improves from $2.37$ to $2.12$ for English $\rightarrow$ German. Note that Tower has already seen this data during instruction-tuning and the improvement is not from new data but from Quality-Aware decoding. Moreover, re-ranking the top $5$ beams obtained from unified decoding with XCOMET-XL leads to a slight further improvement in quality. This highlights the robustness and generalizability of our approach across different evaluation metrics.
%\footnotetext{\href{https://github.com/WMT-QE-Task/wmt-qe-2023-data}{WMT 23 English $\rightarrow$ German QE Data}}

\subsection{Adapting for Tower Errors}

We use the MQM annotations from WMT to train our Tower QE model, which contains error annotations from other systems. However, a viable alternative would be to adapt Tower QE specifically to the errors it typically makes. To maintain a similar data setup, we first generate translations using Tower on these source sentences. Then, we annotate the generated hypotheses with XCOMET-XL using the reference and fine-tune Tower QE on this synthetic dataset, which we refer to as \textbf{Tower Distill QE}. We evaluate the performance of the new distill QE model and report the results in Table \ref{tab:towerdistill}.

We observe that the distilled QE model performs very similarly to the QE model trained on errors from other systems. This indicates that there was no significant benefit in adapting the QE model to the specific errors typically made by Tower. However, further analysis on larger datasets and different domains is needed to fully validate the effectiveness of the distillation approach as the current synthetic data generated is small.

\subsection{Sentence vs Document-level Translation}

From Table \ref{tab:qadecoding}, we observe that the gains for English $\rightarrow$ German (paragraph-level) are much higher than for Chinese $\rightarrow$ English (sentence-level). We hypothesize that this discrepancy arises from the length of the sentences, as the N-best list re-ranking is likely sufficient for shorter sentences. To confirm this, we take the English paragraphs and split them into sentences using a tokenizer while tracking the paragraph IDs. We then perform the entire decoding process similarly, and later join the sentences back using the paragraph IDs before evaluation. We report the results in Table \ref{tab:sentvspara}.

We define the impact as the improvement in translation quality from re-ranking the N-best list with XCOMET-XL QE to integrating Tower QE. Comparing the results at the paragraph level to those at the sentence level, we observe that the impact decreases, which confirms our hypothesis. Additionally, we obtain better scores at the document level, further highlighting the potential benefits of Quality-Aware Decoding.

\section{Related Work}

\textbf{Integrating QE in NMT:} Several advancements have been made in improving QE for NMT over the years \citep{rei2021references, rei2022cometkiwi, blain2023findings, zerva2024findings, guerreiro2024xcomet}. These developments have led to the integration of QE in various ways.
One common approach involves applying QE after generating multiple sequences through techniques such as QE re-ranking \citep{fernandes2022quality, faria2024quest} or Minimum Bayes Risk (MBR) decoding \citep{tomani2024quality}. Another direction focuses on removing noisy data using QE models, followed by fine-tuning on high-quality data \citep{xu2024contrastive, finkelstein2024introducing}. \citet{vernikos2024don} proposes to generate diverse translations as a first step and then combine them. We perform this explicitly by integrating the QE directly into decoding.
Recently, \citet{zhang2024learning} exploited the MQM data by training models to penalize tokens within an error span, improving translation quality. In contrast, our approach adopts a modular framework, where we propose an expert QE model that is trained independently for targeted training. This modular approach aims to improve performance by decomposing the task into separate translation and QE components.

\textbf{Reward Modeling in NLG:}  Quality-Aware decoding shares several similarities with controllable text generation methods, particularly in the use of an additional "Quality/Reward" model that guides the decoding. A well-explored approach for controlling text is altering the decoding with a reward model (Weighted Decoding) \citep{yang2021fudge}. This method modifies the decoding by adjusting token probabilities based on the reward model, allowing for more controlled generation.
Similarly, \citet{deng-raffel-2023-reward} also used a uni-directional reward model, with the aim of maintaining efficiency during generation. This approach minimizes computational complexity while still benefiting from the guiding influence of the reward model. Moreover, recent work by \citet{li-etal-2024-reinforcement} introduced a token-level reinforcement learning-based reward model, providing more fine-grained feedback that enhances control over text generation at a granular level. While similar, the key contribution in our work lies in the development of the first uni-directional QE model for translation. 


\section{Conclusion}
We have shown the importance of Quality-Aware decoding to improve translation quality, rather than relying on post-generation techniques. In this work, we demonstrated how MQM data can be used to build a uni-directional token-level QE model, which is then integrated into the decoding process. Through a series of experiments, we showed that our Quality-Aware decoding approach results in measurable improvements in translation quality. Notably, we did not introduce new training data to the NMT model, and show that the gains stem from Quality-Aware decoding.


\section{Limitations}
While our Quality-Aware decoding improves translation quality, it adds considerable computational complexity to the inference process. Theoretically, this approach would double the time needed to generate a translation and require additional memory to utilize the token-level QE model. One potential solution to mitigate this issue could be to use token-level QE as a reward model for training via Reinforcement Learning.

Additionally, we trained our model on a limited set of human-annotated MQM data. However, current QE models, such as XCOMET, are capable of predicting error tags using the reference with reasonable quality. This suggests that further improvements could be achieved if these models were trained on larger-scale datasets, providing more nuanced feedback and refining translation quality even further.

Lastly, our proposed token-level QE model does not account for error severity. Ideally, it should be able to predict the category of errors, allowing for more nuanced feedback and enabling the model to generate translations with only minor errors when necessary.


% Bibliography entries for the entire Anthology, followed by custom entries
%\bibliography{anthology,custom}
% Custom bibliography entries only
\bibliography{custom}

\appendix

\section{Appendix}
\label{sec:appendix}

% \begin{table*}[!ht]
% \centering
% \begin{tabular}{@{}ccccc@{}}
% \toprule
% \multicolumn{1}{c|}{Model}            & \multicolumn{1}{c|}{Beams}                & \multicolumn{1}{c|}{Re-ranking}              & XCOMET-XL      & XCOMET-XXL     \\ \midrule
% \multicolumn{5}{c}{\textit{English $\rightarrow$ German}}                                                                                                          \\ \midrule
% \multicolumn{1}{c|}{Tower}            & \multicolumn{1}{c|}{5}           & \multicolumn{1}{c|}{\_}                      & 84.93          & 86.93          \\
% \multicolumn{1}{c|}{Tower}            & \multicolumn{1}{c|}{25}                   & \multicolumn{1}{c|}{\textbf{\_}}             & 84.87 & 86.45          \\
% \multicolumn{1}{c|}{Tower MBR}        & \multicolumn{1}{c|}{25}                   & \multicolumn{1}{c|}{\_}                      & 85.23          & 87.09          \\
% \multicolumn{1}{c|}{Tower}            & \multicolumn{1}{c|}{25}                   & \multicolumn{1}{c|}{XCOMET-XL QE}            & 86.56          & 87.79          \\
% \multicolumn{1}{c|}{Tower}            & \multicolumn{1}{c|}{5}                    & \multicolumn{1}{c|}{Tower QE} & 85.34          & 87.33          \\
% \multicolumn{1}{c|}{Tower}            & \multicolumn{1}{c|}{25}                   & \multicolumn{1}{c|}{Tower QE} & 85.40          & 87.40          \\
% \multicolumn{1}{c|}{Tower + Tower QE} & \multicolumn{1}{c|}{5 (25* for Tower QE)} & \multicolumn{1}{c|}{\_}                      & 86.36          & 88.95          \\
% \multicolumn{1}{c|}{Tower + Tower QE} & \multicolumn{1}{c|}{5 (25* for Tower QE)} & \multicolumn{1}{c|}{XCOMET-XL QE}            & \textbf{86.88} & \textbf{89.08} \\ \midrule
% \multicolumn{5}{c}{\textit{Chinese $\rightarrow$ English}}                                                                                                         \\ \midrule
% \multicolumn{1}{c|}{Tower}            & \multicolumn{1}{c|}{5}                    & \multicolumn{1}{c|}{\_}                      & 85.38          & 88.91          \\
% \multicolumn{1}{c|}{Tower}            & \multicolumn{1}{c|}{25}                   & \multicolumn{1}{c|}{\_}                      & 85.29          & 88.71          \\
% \multicolumn{1}{c|}{Tower MBR}        & \multicolumn{1}{c|}{25}                   & \multicolumn{1}{c|}{\_}                      & 86.00          & 89.23          \\
% \multicolumn{1}{c|}{Tower}            & \multicolumn{1}{c|}{25}                   & \multicolumn{1}{c|}{XCOMET-XL QE}            & 87.04          & 89.49          \\
% \multicolumn{1}{c|}{Tower}            & \multicolumn{1}{c|}{5}                    & \multicolumn{1}{c|}{Tower QE} & 85.64          & 89.10          \\
% \multicolumn{1}{c|}{Tower}            & \multicolumn{1}{c|}{25}                   & \multicolumn{1}{c|}{Tower QE} & 85.93          & 89.51          \\
% \multicolumn{1}{c|}{Tower + Tower QE} & \multicolumn{1}{c|}{5 (25* for Tower QE)} & \multicolumn{1}{c|}{\_}                      & 86.01          & 89.82          \\
% \multicolumn{1}{c|}{Tower + Tower QE} & \multicolumn{1}{c|}{5 (25* for Tower QE)} & \multicolumn{1}{c|}{XCOMET-XL QE}            & \textbf{86.67} & \textbf{90.00} \\ \bottomrule
% \end{tabular}
% \caption{COMET scores on WMT23 English $\rightarrow$ German Test set. Both XCOMET metric columns use reference for reporting translation quality and do not when used for re-ranking }
% \end{table*}


\subsection{Training details}
\label{sec:training_detail}

We use the transformers library \citep{wolf-etal-2020-transformers} for training and inference with Tower-Instruct V2.  For adapting Tower to token-level QE, we use LoRA \citep{hulora} based fine-tuning with an additional classifier head. Therefore, we only train the adapters and the weights for classification head.

We add the adapters to the modules \textit{q\_proj,k\_proj,v\_proj,gate\_proj,up\_proj} and \textit{down\_proj}. We set a batch size for each device to 12 initially and enable \textit{auto\_find\_batch\_size} to \textit{True} on 4 NVIDIA RTX A6000 GPU's. For having a  larger batch size during training, we set \textit{gradient\_accumulation\_steps} to 6. We use a \textit{learning\_rate} of $1e^{-5}$. We set the \textit{eval\_steps} to $50$ and \textit{num\_train\_epochs} to $10$. The other parameters are set to default.

Using the cross-entropy loss for token-level QE directly is insufficient due to the fact that the majority of tokens are classified as '\textit{Good}'. Hence, we find that the weighted cross-entropy loss is essential when fine-tuning the model. For the training on human MQM data, we set the weights to $0.05,0.95$ to '\textit{Good}' and '\textit{Bad}' labels respectively. In the case of distilling from XCOMET, we observed more errors. Therefore, we find that setting them $0.2,0.8$ to '\textit{Good}' and '\textit{Bad}' labels respectively provided stable training.

We train on data until WMT'22 for training and use WMT'24 for validation. We calculate the macro '\textit{F1}' on token-level predictions as the validation metric and stop training if it does not improve for 10 consecutive \textit{eval\_steps}.

\subsection{Partial vs Full Sequence Quality Estimation}

We also compare the difference in performance between our proposed token-level QE for partial sequences with Tower trained for full sequence QE. We achieve this by adding a regression head to predict the score at the end-of-sentence token. Hence, the model uses the source and hypothesis to predict the score using regression head at the end.

We fine-tune the model using only direct assesment data \citep{zerva2024findings} (\textbf{Tower Full DA}). Furthermore, we use this as initialisation and continue fine-tuning on the MQM data (\textbf{Tower Full DA + MQM}). We also use LoRA similarly to the previous model with a regression head to adapt the model. We report the scores in Table \ref{tab:correlation_ablation}.

We see that the both Tower QE models based on full sentences outperforms the partial model. However, this is expected as it has seen the entire context and was also trained on larger amounts of data. Nonetheless, the partial model still achieves much higher correlaiton that the log probabilities showcasing its potential for Quality-Aware decoding.

\subsection{Robustness to re-ranking weight}

In our method, we introduce a hyperparameter, $\alpha$, to merge the probabilities from the token-level QE model and the translation model. This section analyzes the impact of $\alpha$ on the final translation quality.

To efficiently evaluate its effect, we re-rank the N-best list using different values of $\alpha$. This approach allows us to estimate the ideal value of $\alpha$ without the need for joint decoding multiple times. If the re-ranking model (in this case, Tower QE) is beneficial, we expect that any $\alpha$ less than 1 will improve translation quality, as it demonstrates that incorporating the probabilities from the QE model is helpful.

We visualize this impact in Figure \ref{fig:mainfigure}. The results show that using an $\alpha$ less than 1 leads to improved translation quality in both scenarios. This indicates that relying entirely on the NMT model does not yield the best results and highlights the importance of the Tower QE model.

Thus, we emphasize that re-ranking the N-best list provides an effective way to tune the value of $\alpha$, and it remains robust to different values.

\begin{figure*}[!htpb]
\begin{promptbox}[title={Tower Translation Prompt}]
    \small
    <|im\_start|>user\\
    Translate the sentence from English into German.\\
    English: \{src\_sent\}\\
    German:\\
    <|im\_end|>\\
    <|im\_start|>assistant
\end{promptbox}

\begin{promptbox}[title={Tower Token-Level QE Prompt}]
    \small
    English:\{src\_sent\}\\
    German: \{tgt\_sent\}
\end{promptbox}
\caption{Prompts used in our experiments for translation and QE model. \{src\_sent\} and \{tgt\_sent\} represent the source and target sentence. We replace the language with Chinese and English when experimenting with that language pair.}
\end{figure*}

\begin{figure*}[!htpb]
    \centering
    % First subfigure
    \begin{subfigure}[b]{0.5\textwidth}
        \centering
        \includegraphics[width=\textwidth]{Figures/alphas_ende_25.png} % Replace with your image path
        \caption{English $\rightarrow$ German}
        \label{fig:subfigure1}
    \end{subfigure}
    
    \vspace{0.5cm} % Adjust space between the two subfigures

    % Second subfigure
    \begin{subfigure}[b]{0.5\textwidth}
        \centering
        \includegraphics[width=\textwidth]{Figures/alphas_zhen_25.png} % Replace with your image path
        \caption{Chinese $\rightarrow$ English}
        \label{fig:subfigure2}
    \end{subfigure}
    
    \caption{Impact of $\alpha$ when re-ranking with token-level Tower QE on WMT'23 Test sets.}
    \label{fig:mainfigure}
\end{figure*}


\begin{table*}[!ht]
\centering
\begin{tabular}{@{}c|ccc@{}}
\toprule
                                                                                      & Pearson        & Spearmann      & Kendall        \\ \midrule
COMETQE                                                                               & \textbf{44.41} & 41.29          & 31.19          \\ \midrule
COMETQE-XL                                                                            & 41.23          & \textbf{42.17} & \textbf{31.84} \\ \midrule
\begin{tabular}[c]{@{}c@{}}COMETQE Scratch\\      Fine-tuned (ours)\end{tabular}      & 36.32          & 33.66          & 25.24          \\ \midrule
Tower Log Prob                                                                        & 32.32          & 16.74          & 12.77          \\ \midrule
\begin{tabular}[c]{@{}c@{}}Tower Partial QE\end{tabular} & 40.56          & 33.96          & 25.87          \\ \midrule
Tower Full DA                                                                        & 33.67          & 36.46          & 27.38          \\ \midrule
Tower Full DA + MQM                                                                 & 32.03          & 40.85          & 30.38          \\ \bottomrule
\end{tabular}
\caption{Full Correlation results on WMT 23 for English $\rightarrow$ German Test set. Partial indicates that the QE model predict scores via token-level where as full indicates predicting the score at the end-of-sentence token. The scores are calculated after removing the few sentences labelled for hallucination detection. Best scores according to each coefficient are highlighted in \textbf{bold}.}
\label{tab:correlation_ablation}
\end{table*}




\end{document}


\appendix
\section{Appendix: Detailed Experimental Results}

\label{appendix:a}
\begin{figure}[ht!]
    \centering
    \noindent
    \begin{minipage}{0.35\textwidth} % Reduced image size
        \includegraphics[width=\textwidth]{images/download_42.jpeg} % Replace with your image path
    \end{minipage}%
    \hspace{1em} % Add horizontal space between image and text
    \begin{minipage}{0.4\textwidth}
        \vspace{1em} % Add vertical space at the top
        \justifying % Ensures text is justified
        \noindent\textbf{Prompt:} \textit{What is the meaning of true happiness in life?} \\
        \noindent\textbf{Image View:} \textit{The scene shows a vibrant outdoor concert where a large crowd is gathered, with many people expressing their enjoyment by raising their hands in the air. Bright lights illuminate the area, particularly focusing on the stage in the background. The audience is lively and energetic, with some individuals recording the experience on their phones. The atmosphere is filled with excitement and celebration, characteristic of a live music event.}
    \end{minipage}

    \vspace{1em} % Add space between the image and the answer
    \justifying
    \noindent\textbf{Answer:} \textit{True happiness in life often encompasses experiences that bring joy, connection, and personal growth through various activities such as participating in music concerts like the one described. Engaging in shared experiences with others creates bonds and enriches our social lives. Music, being an art form rooted deeply in human culture, has long been associated with positive emotions; it can evoke strong feelings of joy, community, and self-expression. In such a vibrant outdoor concert setting, the collective emotion displayed by the attendees—raising their hands in the air, indicating enthusiasm, and enthusiastically recording the moment with smartphones—are vivid expressions of true happiness and engagement.}

    \caption{An example of Image view related to the input prompt and answered by Qwen2.5-1.5B model.}
    \label{fig:happiness}
\end{figure}

\begin{table}[ht]
    \centering
    %\vspace{1em}
    \resizebox{\linewidth}{!}{ % Adjust Table size to fit page width
    \begin{tabular}{ l c c c c c }
        \hline
        \textbf{Models} & \textbf{MTLD} & \textbf{TF-IDF} & \textbf{Lexical Entropy} & \textbf{Semantic Embedding} & \textbf{Self-BLEU} \\
        \hline
        GPT-2                      & \textbf{41.44}  & 0.85  & 7.98  & 0.54  & 0.76  \\
    
        GPT-2 + Text View          & 35.91  & \textbf{0.88}  & \textbf{8.11}  & \textbf{0.64}  & \textbf{0.80}  \\
        \hline
        Qwen-2.5                       & 67.00  & 0.78  & 7.81  & 0.22  & 0.41  \\
        
        Qwen-2.5 + Text View           & 67.77  & \textbf{0.85}  & \textbf{8.32}  & \textbf{0.39}  & \textbf{0.56}  \\
        
        Qwen-2.5 + Image View          & \textbf{70.24}  & 0.82  & 8.15  & 0.29  & 0.50  \\
        \hline
        LLama-3.2                  & 52.20 & 0.70 & 7.32 & 0.23 & 0.31\\
        LLama-3.2 + Text View      & 54.22 & \textbf{0.82} & \textbf{8.04} & \textbf{0.44} & \textbf{0.54}\\
        LLama-3.2 + Image View     & \textbf{54.39} & 0.74 & 7.59 & 0.31 & 0.41\\
        \hline
        DeepSeek-R1             & 52.34  & 0.76  & 7.54  & 0.24  & 0.39  \\
        
        DeepSeek-R1 + Text View & \textbf{55.07}  & \textbf{0.84}  & \textbf{8.07}  & \textbf{0.40}  & \textbf{0.52}  \\
        
        DeepSeek-R1 + Image View & 54.37  & 0.82  & 7.95  & 0.39  & \textbf{0.52}  \\
        \hline
        GPT-4o Mini                & 57.98  & 0.61  & 7.03  & 0.11  & 0.16  \\
        
        GPT-4o Mini + Text View    & 59.68  & \textbf{0.77}  & \textbf{7.87}  & \textbf{0.34}  & \textbf{0.39}  \\
        
        GPT-4o Mini + Image View   & \textbf{60.19}  & \textbf{0.77}  & 7.76  & 0.32  & 0.38  \\
        \hline
        GPT-4o                     & 57.17  & 0.64  & 6.85  & 0.10  & 0.15  \\
        
        GPT-4o + Text View         & \textbf{58.25}   & 0.76  & \textbf{7.75}  & \textbf{0.31}  & 0.38  \\
        
        GPT-4o + Image View        & 54.66   & \textbf{0.77}  & 7.68  & \textbf{0.31}  & \textbf{0.44}  \\
        \hline
        
    \end{tabular}
    } % End resizebox
    \caption{Diversity Metrics for different models. Higher values indicate better performance. $num\_samples = 100$ , $max\_length = 125$}
    \label{tab:num_samples100}
\end{table}

\begin{table}[ht]
    \centering
    %\vspace{1em}
    \resizebox{\linewidth}{!}{ % Adjust Table size to fit page width
    \begin{tabular}{ l c c c c c }
        \hline
        \textbf{Models} & \textbf{MTLD} & \textbf{TF-IDF} & \textbf{Lexical Entropy} & \textbf{Semantic Embedding} & \textbf{Self-BLEU} \\
        \hline
        GPT-2                      & \textbf{41.35}  & 0.86  & 8.11  & 0.55  & 0.68  \\
    
        GPT-2 + Text View          & 35.69  & \textbf{0.89}  & \textbf{8.28}  & \textbf{0.64}  & \textbf{0.72}  \\
        \hline
        Qwen-2.5                       & 67.45  & 0.79  & 7.88  & 0.22  & 0.31  \\
        
        Qwen-2.5 + Text View           & 67.68  & \textbf{0.85}  & \textbf{8.43}  & \textbf{0.39}  & 0.47  \\
        
        Qwen-2.5 + Image View          & \textbf{70.27}  & 0.83  & 8.26  & 0.30  & \textbf{0.48}  \\
        \hline
        LLama-3.2                  & 52.52 & 0.71 & 7.38 & 0.22 & 0.23\\
        LLama-3.2 + Text View      & 54.31 & \textbf{0.83} & \textbf{8.16} & \textbf{0.43} & \textbf{0.44}\\
        LLama-3.2 + Image View     & \textbf{54.59} & 0.75 & 7.66 & 0.30 & 0.31\\
        \hline
        DeepSeek-R1             & 52.41  & 0.78  & 7.60  & 0.24  & 0.30  \\
        
        DeepSeek-R1 + Text View & \textbf{54.44}  & \textbf{0.85}  & \textbf{8.12}  & \textbf{0.40}  & \textbf{0.45}  \\
        
        DeepSeek-R1 + Image View & 54.42  & 0.83  & 8.05  & 0.39  & 0.41  \\
        \hline
        GPT-4o Mini                & 57.94  & 0.62  & 7.05  & 0.11  & 0.16  \\
        
        GPT-4o Mini + Text View    & 59.79  & \textbf{0.79}  & \textbf{7.91}  & \textbf{0.33}  & \textbf{0.38}  \\
        
        GPT-4o Mini + Image View   & \textbf{60.38}  & 0.78  & 7.82  & 0.31  & 0.28  \\
        \hline
        GPT-4o                     & 57.40  & 0.66  & 6.88  & 0.09  & 0.10  \\
        
        GPT-4o + Text View         & \textbf{58.21}   & 0.78  & \textbf{7.85}  & \textbf{0.31}  & 0.29  \\
        
        GPT-4o + Image View        & 55.23   & \textbf{0.79}  & 7.77  & \textbf{0.31}  & \textbf{0.34}  \\
        \hline
        
    \end{tabular}
    } % End resizebox
    \caption{Diversity Metrics for different models. Higher values indicate better performance. $num\_samples = 250$, $max\_length = 125$.}
    \label{tab:num_sample250}
\end{table}

\begin{table}[ht]
    \centering
    %\vspace{1em}
    \resizebox{\linewidth}{!}{ % Adjust Table size to fit page width
    \begin{tabular}{ l c c c c c }
        \hline
        \textbf{Models} & \textbf{MTLD} & \textbf{TF-IDF} & \textbf{Lexical Entropy} & \textbf{Semantic Embedding} & \textbf{Self-BLEU} \\
        \hline
        GPT-2                      & \textbf{41.30}  & 0.87  & 8.17  & 0.55  & 0.60  \\
    
        GPT-2 + Text View          & 35.77  & \textbf{0.90}  & \textbf{8.36}  & \textbf{0.64}  & \textbf{0.65}  \\
        \hline
        Qwen-2.5                       & 67.13  & 0.80  & 7.91  & 0.22  & 0.25  \\
        
        Qwen-2.5 + Text View           & 67.62  & \textbf{0.86}  & \textbf{8.48}  & \textbf{0.39}  & \textbf{0.40}  \\
        
        Qwen-2.5 + Image View          & \textbf{70.36}  & 0.84  & 8.32  & 0.30  & 0.36  \\
        \hline
        LLama-3.2                  & 52.60 & 0.72 & 7.43 & 0.23 & 0.18\\
        LLama-3.2 + Text View      & 54.22 & \textbf{0.84} & \textbf{8.23} & \textbf{0.43} & \textbf{0.37}\\
        LLama-3.2 + Image View     & \textbf{54.45} & 0.76 & 7.71 & 0.31 & 0.26\\
        \hline 
        DeepSeek-R1             & 52.56  & 0.79  & 7.66  & 0.24  & 0.24  \\
        
        DeepSeek-R1 + Text View & \textbf{54.52}  & \textbf{0.86}  & \textbf{8.22}  & \textbf{0.39}  & \textbf{0.37}  \\
        
        DeepSeek-R1 + Image View & 54.27  & 0.84  & 8.08  & \textbf{0.39}  & 0.34  \\
        \hline
        GPT-4o Mini                & 57.95  & 0.63  & 7.07  & 0.11  & 0.07  \\
        
        GPT-4o Mini + Text View    & 59.53  & \textbf{0.79}  & \textbf{7.93}  & \textbf{0.33}  & \textbf{0.22}  \\
        
        GPT-4o Mini + Image View   & \textbf{60.36}  & \textbf{0.79}  & 7.85  & 0.31  & 0.21  \\
        \hline
        GPT-4o                     & 56.99  & 0.67  & 6.89  & 0.09  & 0.07  \\
        
        GPT-4o + Text View         &  \textbf{58.22}  &  \textbf{0.79} & \textbf{7.86}  & 0.30  & 0.23  \\
        
        GPT-4o + Image View        & 55.29   & 0.79  & \textbf{7.79}  & \textbf{0.31}  & \textbf{0.27}  \\
        \hline
        
    \end{tabular}
    } % End resizebox
    \caption{Diversity Metrics for different models. Higher values indicate better performance. $num\_samples = 500$, $max\_length = 125$.}
    \label{tab:num_samples500}
\end{table}

\begin{table}[ht]
    \centering
    %\vspace{1em}
    \resizebox{\linewidth}{!}{ % Adjust Table size to fit page width
    \begin{tabular}{ l c c c c c }
        \hline
        \textbf{Models} & \textbf{MTLD} & \textbf{TF-IDF} & \textbf{Lexical Entropy} & \textbf{Semantic Embedding} & \textbf{Self-BLEU} \\
        \hline
        GPT-2                      & \textbf{41.33}  & 0.87  & 8.21  & 0.54  & 0.53  \\
    
        GPT-2 + Text View          & 35.66  & \textbf{0.90}  & \textbf{8.43}  & \textbf{0.64}  & \textbf{0.59}  \\
        \hline
        Qwen-2.5                       & 67.10  & 0.81  & 7.94  & 0.22  & 0.20  \\
        
        Qwen-2.5 + Text View           & 67.44  & \textbf{0.87}  & \textbf{8.50}  & \textbf{0.38}  & \textbf{0.34}  \\
        
        Qwen-2.5 + Image View          & \textbf{70.29}  & 0.85  & 8.34  & 0.30  & 0.30  \\
        \hline
        LLama-3.2                  & 52.63 & 0.73 & 7.44 & 0.23 & 0.14\\
        LLama-3.2 + Text View      & 54.22 & \textbf{0.84} & \textbf{8.26} & \textbf{0.43} & 0.31\\
        LLama-3.2 + Image View     & \textbf{54.43} & 0.77 & 7.74 & 0.31 & \textbf{0.21}\\
        \hline
        DeepSeek-R1             & 52.54  & 0.79  & 7.67  & 0.24  & 0.19  \\
        
        DeepSeek-R1 + Text View & \textbf{54.55}  & \textbf{0.86}  & \textbf{8.27}  & \textbf{0.39}  & \textbf{0.31}  \\
        
        DeepSeek-R1 + Image View & 54.38  & 0.84  & 8.11  & \textbf{0.39}  & 0.28  \\
        \hline
        GPT-4o Mini                & 57.95  & 0.64  & 7.08  & 0.11  & 0.05  \\
        
        GPT-4o Mini + Text View    & \textbf{59.57}  & \textbf{0.88}  & \textbf{7.95}  & \textbf{0.33}  & 0.17  \\
        
        GPT-4o Mini + Image View   & 55.76  & 0.81  & 7.79  & 0.29  & \textbf{0.19}  \\
        \hline
        GPT-4o                     & 57.83  & 0.66  & 7.14  & 0.13  & 0.06  \\
        
        GPT-4o + Text View         & \textbf{58.29}   & 0.79  & \textbf{7.90}  & \textbf{0.31}  & 0.18  \\
        
        GPT-4o + Image View        & 55.12   & \textbf{0.80}  & 7.80  & \textbf{0.31}  & \textbf{0.21}  \\
        \hline
        
    \end{tabular}
    } % End resizebox
    \caption{Diversity Metrics for different models. Higher values indicate better performance. $num\_samples = 1000$, $max\_length = 125$.}
    \label{tab:num_samples1000}
\end{table}

\begin{table}[ht]
    \centering
    %\vspace{1em}
    \resizebox{\linewidth}{!}{ % Adjust Table size to fit page width
    \begin{tabular}{ l c c c c c }
        \hline
        \textbf{Models} & \textbf{MTLD} & \textbf{TF-IDF} & \textbf{Lexical Entropy} & \textbf{Semantic Embedding} & \textbf{Self-BLEU} \\
        \hline
        GPT-2                      & \textbf{41.39}  & 0.87  & 8.24  & 0.55  & 0.49  \\
    
        GPT-2 + Text View          & 33.30  & \textbf{0.90}  & \textbf{8.47}  & \textbf{0.66}  & \textbf{0.58}  \\
        \hline
        Qwen-2.5                       & 67.19  & 0.81  & 7.95  & 0.22  & 0.18  \\
        
        Qwen-2.5 + Text View           & 67.64   & \textbf{0.87}  & \textbf{8.52}  & \textbf{0.38}  & \textbf{0.30}  \\

        Qwen-2.5 + Image View          & \textbf{70.30}  & 0.85  & 8.35  & 0.30  & 0.27  \\
        \hline
        LLama-3.2                  & 52.53 & 0.73 & 7.44 & 0.23 & 0.12\\
        LLama-3.2 + Text View      & 54.28 & \textbf{0.85} & \textbf{8.28} & \textbf{0.43} & \textbf{0.27}\\
        LLama-3.2 + Image View     & \textbf{54.40} & 0.77 & 7.75 & 0.31 & 0.18\\
        \hline
        DeepSeek-R1             & \textbf{52.59}  & 0.79  & 7.68  & 0.24  & 0.17  \\
        
        DeepSeek-R1 + Text View & \textbf{54.53}  & \textbf{0.86}  & \textbf{8.28}  & \textbf{0.39}  & \textbf{0.28}  \\
        
        DeepSeek-R1 + Image View & 54.32  & 0.84  & 8.12  & \textbf{0.39}  & 0.24  \\
        \hline
        GPT-4o Mini                & 57.91  & 0.64  & 7.08  & 0.11  & 0.04  \\
        
        GPT-4o Mini + Text View    & \textbf{59.50}  & \textbf{0.80}  & \textbf{7.96}  & \textbf{0.33}  & 0.15  \\
        
        GPT-4o Mini + Image View   & 54.86  & 0.81  & 7.77  & 0.29  & \textbf{0.16}  \\
        \hline
        GPT-4o                     & 57.90  & 0.66  & 7.14  & 0.13  & 0.05  \\
        
        GPT-4o + Text View         & \textbf{58.26}   & 0.79  & \textbf{7.90}  & \textbf{0.31}  & 0.15  \\
        
        GPT-4o + Image View        & 55.14   & \textbf{0.80}  & 7.82  & \textbf{0.31}  & \textbf{0.18}  \\
        \hline
        
    \end{tabular}
    } % End resizebox
    \caption{Diversity Metrics for different models. Higher values indicate better performance. $num\_samples = 1500$, $max\_length = 125$.}
    \label{tab:num_samples1500}
\end{table}
%%%%%%%%%%%%
\begin{table}[ht]
    \centering
    %\vspace{1em}
    \resizebox{\linewidth}{!}{ % Adjust Table size to fit page width
    \begin{tabular}{ l c c c c c }
        \hline
        \textbf{Models} & \textbf{MTLD} & \textbf{TF-IDF} & \textbf{Lexical Entropy} & \textbf{Semantic Embedding} & \textbf{Self-BLEU} \\
        \hline
        GPT-2                      & \textbf{41.36}  & 0.87  & 8.25  & 0.55  & 0.46  \\
    
        GPT-2 + Text View          & 33.39  & \textbf{0.90}  & \textbf{8.48}  & \textbf{0.67}  & \textbf{0.55}  \\
        \hline
        Qwen-2.5                       & 67.25 & 0.81  & 7.95  & 0.22  & 0.16  \\
        
        Qwen-2.5 + Text View           & 67.68   & \textbf{0.87}  & \textbf{8.53}  & \textbf{0.38}  & \textbf{0.28}  \\

        Qwen-2.5 + Image View          & \textbf{70.24}  & 0.85  & 8.36  & 0.30  & 0.25  \\
        \hline
        LLama-3.2                  & 52.21 & 0.73 & 7.45 & 0.23 & 0.11\\
        LLama-3.2 + Text View      & 54.14 & \textbf{0.85} & \textbf{8.29} & \textbf{0.43} & \textbf{0.25}\\
        LLama-3.2 + Image View     & \textbf{54.49} & 0.77 & 7.75 & 0.30 & 0.16\\
        \hline
        DeepSeek-R1             & 52.62  & 0.79  & 7.69  & 0.24  & 0.15  \\
        
        DeepSeek-R1 + Text View & \textbf{54.53}  & \textbf{0.86}  & \textbf{8.29}  & \textbf{0.39} & \textbf{0.26}  \\
        
        DeepSeek-R1 + Image View & 54.38  & 0.84  & 8.12  & \textbf{0.39}  & 0.22  \\
        \hline
        GPT-4o Mini                & 57.98  & 0.64  & 7.08  & 0.11  & 0.03  \\
        
        GPT-4o Mini + Text View    & \textbf{59.50}  & 0.80  & \textbf{7.97}  & \textbf{0.33}  & 0.13  \\
        
        GPT-4o Mini + Image View   & 54.90  & \textbf{0.81}  & 7.78  & 0.29  & \textbf{0.14}  \\
        \hline
        GPT-4o                     & 55.78  & 0.67  & 7.16  & 0.12  & 0.05  \\
        
        GPT-4o + Text View         & 58.26   & \textbf{0.80}  & 7.91  & 0.31  & 0.14  \\
        
        GPT-4o + Image View        &\textbf{59.05}   & \textbf{0.80}  & \textbf{7.94}  & \textbf{0.35}  & \textbf{0.15}  \\
        \hline
        
    \end{tabular}
    } % End resizebox
    \caption{Diversity Metrics for different models. Higher values indicate better performance. $num\_samples = 2000$, $max\_length = 125$.}
    \label{tab:num_samples2000}
\end{table}
%%%%%%%%%%%%

\begin{table}[ht]
    \centering
    % \vspace{1em}
    \resizebox{\linewidth}{!}{ % Adjust Table size to fit page width
    \begin{tabular}{ c c c c c c}
        \hline
        \textbf{Seed} & \textbf{Model} & \textbf{Accuracy} & \textbf{Precision} & \textbf{Recall} & \textbf{F1-score} \\
        \hline
        \multirow{2}{*}{Seed0}  & DeepSeekV3  & 0.8450 & \textbf{0.9367} & 0.7400 & 0.8268 \\
                                & GPT-4o       & \textbf{0.9100} & 0.9271 & \textbf{0.8900} & \textbf{0.9082} \\
        \hline
        \multirow{2}{*}{Seed1}  & DeepSeekV3  & 0.8100 & \textbf{0.9198} & 0.6800 & 0.7816 \\
                                & GPT-4o       & \textbf{0.8700} & 0.9111 & \textbf{0.8200} & \textbf{0.8632} \\
        \hline
        \multirow{2}{*}{Seed2}  & DeepSeekV3  & 0.8450 & 0.9726 & 0.7100 & 0.8187 \\
                                & GPT-4o       & \textbf{0.9150} & \textbf{0.9663} & \textbf{0.8600} & \textbf{0.9091} \\
        \hline
        \multirow{2}{*}{Seed3}  & DeepSeekV3  & 0.7900 & 0.9265 & 0.6300 & 0.7500 \\
                                & GPT-4o       & \textbf{0.8650} & \textbf{0.9195} & \textbf{0.8000} & \textbf{0.8556} \\
        \hline
        \multirow{2}{*}{Seed4}  & DeepSeekV3  & 0.8750 & 0.9000 & 0.7600 & 0.8588 \\
                                & GPT-4o       & \textbf{0.9300} & \textbf{0.9778} & \textbf{0.8800} & \textbf{0.9268} \\
        \hline
        \multirow{2}{*}{Seed5}  & DeepSeekV3  & 0.8300 & 0.9342 & 0.7200 & 0.8068 \\
                                & GPT-4o       & \textbf{0.9200} & \textbf{0.9773} & \textbf{0.8600} & \textbf{0.9149} \\
        \hline
        \multirow{2}{*}{Seed6}  & DeepSeekV3  & 0.8450 & \textbf{0.9859} & 0.7000 & 0.8187 \\
                                & GPT-4o       & \textbf{0.8650} & 0.9011 & \textbf{0.8000} & \textbf{0.8556} \\
        \hline
        \multirow{2}{*}{Seed7}  & DeepSeekV3  & 0.8550 & \textbf{0.9863} & 0.7200 & 0.8324 \\
                                & GPT-4o       & \textbf{0.9150} & 0.9462 & \textbf{0.8800} & \textbf{0.9123} \\
        \hline
        \multirow{2}{*}{Seed8}  & DeepSeekV3  & 0.8400 & \textbf{0.9857} & 0.6900 & 0.8118 \\
                                & GPT-4o       & \textbf{0.8850} & 0.9753 & \textbf{0.7900} & \textbf{0.8729} \\
        \hline
        \multirow{2}{*}{Seed9}  & DeepSeekV3  & 0.8450 & \textbf{0.9600} & 0.7200 & 0.8229 \\
                                & GPT-4o       & \textbf{0.8750} & 0.9032 & \textbf{0.8400} & \textbf{0.8750} \\
        \hline
    \end{tabular}
    } % End resizebox
    \caption{Classification metricss for GPT-4o and DeepSeekV3 across different seeds to evaluate the ability of these models to detecet the correct answer.}
    \label{tab:classification_metrics_correctness}
\end{table}

\begin{table}
\centering

%\vspace{1em}
\resizebox{\linewidth}{!}{ % Adjust Table size to fit page width
\begin{tabular}{lccc}
\toprule
 & \multicolumn{1}{c}{num\_samples = 100} & \multicolumn{1}{c}{num\_samples = 250} & \multicolumn{1}{c}{num\_samples = 500} \\
\cmidrule(lr){2-2} \cmidrule(lr){3-3} \cmidrule(lr){4-4}
Model & SBERT & SBERT & SBERT \\
\midrule
GPT2 & 98.10 & 96.76 & 95.46 \\
GPT2 + Text View & 99.70 & 98.88 & 98.72 \\
\hline
GPT-4o-mini & 5.71 & 3.44 & 2.50 \\
GPT-4o-mini + Text View & 47.50 & 32.24 & 24.30 \\
GPT-4o-mini + Image View & 47.30 & 30.44 & 18.70 \\
\hline
Qwen2.5-1.5B & 32.70 & 25.44 & 22.58 \\
Qwen2.5-1.5B + Text View & 68.70 & 59.60 & 52.64 \\
Qwen2.5-1.5B + Image View & 57.20 & 48.80 & 45.14 \\
\bottomrule
\end{tabular}}
\caption{Results on percentage of novelty score across different models according to the novelty detector SBERT}
\label{tab:novelty_sbert_2}
\end{table}

\clearpage
\noindent\textbf{Example for correct and incorrect answer:}\label{sec:correctness_measure_example}\\
The generated answers to the input prompt could be novel, or even diverse but incorrect. Pay attention to the following example:.\\
{\ttfamily
\justifying
Prompt: "what is the meaning of true happiness in life?"\\
Answer 1: (First answer$\rightarrow$ novel, correct)\\
"True happiness in life is achieving a balance between personal fulfillment and meaningful connections with others, along with good health and peace of mind."\\
Answer 2: (novel, and correct)\\
"Happiness means being content with what you have, staying optimistic in the face of challenges, and cherishing the moments shared with family and friends."\\
Answer 3: (non-novel, and correct)\\
“Happiness is living a healthy life, having supportive relationships, achieving your goals, and feeling grateful for what you have.”\\
Answer 4: (novel, but incorrect)\\
"To make the best pizza dough, you need high-quality flour, water, yeast, and a bit of olive oil. Let it rise for at least 24 hours for optimal texture."\\

\par}

\section{Appendix: Prompts and more information}
\label{appendix:b}

\noindent\textbf{Prompts used in this work}\\
We Used 10 prompts from different subjects to improve generalization on our results.\\
{\ttfamily
\justifying
1.Philosophical Question:\\
"What is the meaning of true happiness in life?",\\

2.Hypothetical Scenario:\\
"If humans could live on Mars, what challenges would they face and how could they overcome them?",\\

3.Creative Thinking Prompt:\\
"Can you describe an imaginary city where technology and nature exist in perfect harmony?",\\

4.Practical Advice Question:\\
"What are the most effective ways to learn a new language quickly?",\\

5.Exploration of Abstract Concepts:\\
"How would you explain the concept of time to someone who has never experienced it?",\\
    
6.Scientific Exploration:\\
"What are the possible effects of artificial intelligence on scientific research in the next decade?",\\

7.Ethical Dilemma:\\
"Is it ever justifiable to prioritize technological advancement over environmental protection?",\\

8.Problem-Solving Question:\\
"How can cities effectively reduce traffic congestion without compromising accessibility?",\\

9.Imaginative Scenario:\\
"If animals could communicate with humans, how would that change our world?",\\

10.Personal Reflection Prompt:\\
"What qualities make someone a great leader, and how can those qualities be developed?"\\
    
\par}
\noindent\textbf{Evaluate the correctness of prompt answer experiment details} \\
This is an example of a random prompt from seed 0 with it's correct answer and selected incorrect answer from another cluster. in Figure \ref{fig:prompt_answer_correctness} the visualization of distance between two answers demonstrated.\\

{\ttfamily
\justifying
Prompt:\\
"In a isolated research facility , a researcher walks past a janitor , only to stop later and realize that we do n't have a janitor ."\\
"correct\_answer":\\
"Something was off . But something was always off in these hallways . The corridors slightly too long , slightly too narrow for two people to walk comfortably abreast . As if someone had googled exactly what it was that made people unconsciously uncomforTable and crammed every last wretched detail into the place . Shifting in the starched collar , he sniffed and pulled the pen from the clipboard . <newline> Trying to trace the events that lead to him being here , it was nigh on impossible . A odd comment by a professor there , reading a specific page of a newspaper , there was no defining reason that spurred him here . <newline> That said , no one could deny that it was a bold career move , and a bold move for him in general . He was in Antarctica for Christ 's sake . <newline> It brought him no small joy that on slow days he would pace the corridors of the facility , for lack of something better to do . He had worked out the optimal route based on path efficiency , and had divided his routine between Surveying tasks so that the laps loosely resembled a set - three reps of the uniformly grey corridors , spaced by intervals of 45 minutes . <newline> So on the second rep of the routine , passing through a adjunct corridor connecting both polar fieldwork hub A and a room masquerading as a laboratory , that he saw a Figure that the very end of the hallway . <newline> It 's not my job to speculate on a person 's intelligence obviously . But what I can say is it was a pretty damn long corridor and our protagonist had enough time to realise that something was off . Instead , the approach happened , he adopted his socially trained response to the presence of an unknown but nonthreatening human , an easy smile , and nodded as he went past the scruffy Figure . <newline> He took in some details . The janitors uniform , one that looked from the lack of utility more parodic that anything , as if dressed Halloween or something . A slight gait , no hat , two buttons done up interrupted by the third undone and then the fourth one buttoned up by the neck . <newline> He rounded the corner . <newline> There was a pause . <newline> He reprocessed the moment . <newline> Turning , he popped his head back around the corner , and looked back down the corridor , visibly straining . <newline> 'Mom ? ' <newline>"\\
"incorrect\_answer":\\
"*God is everywhere . * <newline> <newline> When I was a little boy and my mom brought her friends home I used to go to the church . They used to keep it open all the time . Back then churches were n't afraid of thieves . <newline> <newline> I used to sit in the pews at night and then catch the morning sermon . *God is in all of us . * used to say the priest . His words would wash over me , comforting and familiar . I was part of something great , all humanity connected by God 's love . That 's what started me on killing people too . I always chose people for whom salvation was a myth . And I think I helped them . Even if they did n't know it . <newline> <newline> Now I 've been chosen . I played god with people 's lives , now I am God . I did n't bother asking why . It 's obvious . <newline> <newline> Here I am now . Silently watching . I will not punish the deserving . I will turn a blind eye when the innocents suffer . The greedy , the cheaters , the murderers and the rapists shall live good lives . Their victims will with death upon themselves every day of their lives . The world will be an uglier place ."\\
"correct\_summary":\\
"In the unsettling corridors of an Antarctic facility, designed to evoke discomfort with their narrow and elongated design, our protagonist reflects on how he ended up there. The journey to this remote place wasn't marked by a clear decision, but rather a series of subtle influences like passing comments and newspaper articles. Despite the oddity of his situation, relocating to such an isolated location was viewed as a bold career move. To combat the monotony, he developed a routine of pacing the dreary hallways, structuring his day around these efficient laps interspersed with his work.\texttt{\textbackslash n\textbackslash n One} day, as he completed his usual circuit and passed through a connecting corridor, he noticed a Figure at the far end. Instinctively, he reacted with a default nod and smile, a social conditioning when encountering strangers. Something about the person seemed odd, their janitorial uniform appeared almost like a parody, missing elements of practicality, and the person walked with a slight limp. After rounding a corner, the protagonist paused, puzzled, and reviewed the encounter in his mind. Curiosity piqued, he turned back, peering down the corridor with noticeable effort to see the Figure once more. The moment replays, and to his shock, he exclaims, \"Mom?\" suggesting the surprising recognition of his mother in this unexpected place.",\\
"incorrect\_summary":\\
"The narrator reflects on their childhood experiences at church, where they found comfort in a priest's message that \"God is in all of us.\" This message led them to view humanity as interconnected through God's love. However, this belief also fueled their justification for killing, targeting those they deemed without hope of salvation. The narrator believed they were aiding these individuals, whether or not they realized it. Now, they claim to have ascended to a god-like status, implying they have the power to determine others' fates without questioning why they have been chosen for this role. The narrator adopts a detached and indifferent stance towards justice and morality, deciding not to punish those who might deserve it and ignoring the suffering of innocent people. They suggest that they will allow wrongdoers to thrive while their victims live in misery, predicting that the world will become a more \"ugly\" place as a result."\\
"prompt\_cluster":\\
6\\
"incorrect\_cluster":\\
7\\
  
\par}

\begin{figure}[ht!]
    \centering
    \includegraphics[width=0.55\textwidth]{images/tsne_example_seed_0.png} 
    \caption{An example of selecting correct and incorrect prompt for a prompt and summarized answers.}
    \label{fig:prompt_answer_correctness}
\end{figure}

\noindent\textbf{Template to evaluate the IELTS score prediction ability} \\
{\ttfamily
\justifying
You are an IELTS examiner. Please evaluate the following essay and give a score between 1.0 and 9.0 based on the IELTS Writing Band Descriptors. The essay should be scored based on Task Response,  Coherence and Cohesion, Lexical Resource, and Grammatical Range and Accuracy.\\
Question: \{question\} \\
Essay: \{essay\} \\
Please provide only a score between 1.0 and 9.0.
\par}

\noindent\textbf{Template to get the summary of prompt answers} \\
{\ttfamily
\justifying
Please summarize the following text in no more than \{max\_words\} words:\\
\{text\}\\
\par}

\noindent\textbf{Template to evaluate the correctness of answers} \\
{\ttfamily
\justifying
PROMPT:\\
\{prompt\_text\}\\
ANSWER:\\
\{summarized\_answer\}\\
Question: Is this answer relevant to the prompt, or is it irrelevant??\\
Please respond with exactly one word: "relevant" or "irrelevant".\\
\par}

\begin{figure*}[t]
    \centering
    \includegraphics[width=\linewidth]{images/allsamples_diversity.png}
    \caption{Diversity plot for three variants standard model, with text view, and with image view across all models and six different number of samples per each prompt (10 prompts).}
    \label{fig:allsamples_diversity}
\end{figure*}

\end{document}
