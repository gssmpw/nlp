\documentclass[journal]{vgtc}                     % final (journal style)
% \documentclass[journal]{IEEEtran}                     % final (journal style)
%\documentclass[journal,hideappendix]{vgtc}        % final (journal style) without appendices
%\documentclass[review,journal]{vgtc}              % review (journal style)
%\documentclass[review,journal,hideappendix]{vgtc} % review (journal style)
%\documentclass[widereview]{vgtc}                  % wide-spaced review
%\documentclass[preprint,journal]{vgtc}            % preprint (journal style)


%% Uncomment one of the lines above depending on where your paper is
%% in the conference process. ``review'' and ``widereview'' are for review
%% submission, ``preprint'' is for pre-publication in an open access repository,
%% and the final version doesn't use a specific qualifier.

%% If you are submitting a paper to a conference for review with a double
%% blind reviewing process, please use one of the ``review'' options and replace the value ``0'' below with your
%% OnlineID. Otherwise, you may safely leave it at ``0''.
\onlineid{0}

%% In preprint mode you may define your own headline. If not, the default IEEE copyright message will appear in preprint mode.
%\preprinttext{To appear in IEEE Transactions on Visualization and Computer Graphics.}

%% In preprint mode, this adds a link to the version of the paper on IEEEXplore
%% Uncomment this line when you produce a preprint version of the article 
%% after the article receives a DOI for the paper from IEEE
%\ieeedoi{xx.xxxx/TVCG.201x.xxxxxxx}

%% declare the category of your paper, only shown in review mode
\vgtccategory{Research}

%% please declare the paper type of your paper to help reviewers, only shown in review mode
%% choices:
%% * algorithm/technique
%% * application/design study
%% * evaluation
%% * system
%% * theory/model
\vgtcpapertype{please specify}

\newcommand{\hfedit}[1]{{\color{red} #1}}
\newcommand{\hfcomment}[1]{\hfedit{[HF: #1]}}


%% Paper title.
\title{Make the Fastest Faster: Importance Mask for Interactive Volume Visualization using Reconstruction Neural Networks}

% Author ORCID IDs should be specified using \authororcid like below inside
% of the \author command. ORCID IDs can be registered at https://orcid.org/.
% Include only the 16-digit dashed ID.
\author{%
  \authororcid{Jianxin Sun}{0000-0002-9627-9397},
  \authororcid{David Lenz}{0000-0002-2587-2783}, 
  \authororcid{Hongfeng Yu}{0000-0002-0596-8227}, and
  \authororcid{Tom Peterka}{0000-0002-0525-3205}
}
% \author{}

\authorfooter{
  %% insert punctuation at end of each item
  \item
  	Jianxin Sun and Hongfeng Yu are with the University of Nebraska-Lincoln.
  	E-mail: jianxin.sun@huskers.unl.edu, yu@cse.unl.edu.
  \item
  	David Lenz and Tom Peterka are with Argonne National Laboratory.
  	E-mail: dlenz@anl.gov, tpeterka@mcs.anl.gov.

  % \item Martha Stewart is with Martha Stewart Enterprises at Microsoft
  % Research.
  % 	E-mail: martha.stewart@example.com.
}

%% Abstract section.
\abstract{
Visualizing a large-scale volumetric dataset with high resolution is challenging due to the high computational time and space complexity. Recent deep-learning-based image inpainting methods significantly improve rendering latency by reconstructing a high-resolution image for visualization in constant time on GPU from a partially rendered image where only a small portion of pixels go through the expensive rendering pipeline. However, existing methods need to render every pixel of a predefined regular sampling pattern. In this work, we provide Importance Mask Learning (IML) and Synthesis (IMS) networks which are the first attempts to learn importance regions from the sampling pattern to further minimize the number of pixels to render by jointly considering the dataset, user's view parameters, and the downstream reconstruction neural networks. Our solution is a unified framework to handle various image inpainting-based visualization methods through the proposed differentiable compaction/decompaction layers. Experiments show our method can further improve the overall rendering latency of state-of-the-art volume visualization methods using reconstruction neural network for free when rendering scientific volumetric datasets. Our method can also directly optimize the off-the-shelf pre-trained reconstruction neural networks without elongated retraining.
}

%% Keywords that describe your work. Will show as 'Index Terms' in journal
%% please capitalize first letter and insert punctuation after last keyword
\keywords{Large-scale data, interactive visualization, deep learning, reconstruction neural network}

% A teaser figure can be included as follows
\teaser{
  \centering
  \includegraphics[trim=0 0 0 0,clip,width=\linewidth]{figures/teaser_small.png}
  \caption{The proposed IML and IMS Networks can further improve the rendering latency of already fast volume visualization methods using reconstruction neural networks, like the super-resolution EnhanceNet (a and b) and foveated rendering FoVolNet(c and d), with similar rendering quality. For each subfigure, the rightmost portion is the baseline using the traditional ray casting method, the leftmost portion is the interactive visualization method using reconstruction neural network, and the middle portion is the proposed method using our IML/S Net that empowers the respective reconstruction neural network.
  }
  \label{fig:teaser}
}

%% Uncomment below to disable the manuscript note
%\renewcommand{\manuscriptnotetxt}{}

%% Copyright space is enabled by default as required by guidelines.
%% It is disabled by the 'review' option or via the following command:
%\nocopyrightspace


%%%%%%%%%%%%%%%%%%%%%%%%%%%%%%%%%%%%%%%%%%%%%%%%%%%%%%%%%%%%%%%%
%%%%%%%%%%%%%%%%%%%%%% LOAD PACKAGES %%%%%%%%%%%%%%%%%%%%%%%%%%%
%%%%%%%%%%%%%%%%%%%%%%%%%%%%%%%%%%%%%%%%%%%%%%%%%%%%%%%%%%%%%%%%

%% Tell graphicx where to find files for figures when calling \includegraphics.
%% Note that due to the \DeclareGraphicsExtensions{} call it is no longer necessary
%% to provide the the path and extension of a graphics file:
%% \includegraphics{diamondrule} is completely sufficient.
\graphicspath{{figs/}{figures/}{pictures/}{images/}{./}} % where to search for the images

%% Only used in the template examples. You can remove these lines.
\usepackage{tabu}                      % only used for the table example
\usepackage{booktabs}                  % only used for the table example
\usepackage{lipsum}                    % used to generate placeholder text
\usepackage{mwe}                       % used to generate placeholder figures

%% We encourage the use of mathptmx for consistent usage of times font
%% throughout the proceedings. However, if you encounter conflicts
%% with other math-related packages, you may want to disable it.
\usepackage{mathptmx}                  % use matching math font

\usepackage{microtype}                 % use micro-typography (slightly more compact, better to read)
\PassOptionsToPackage{warn}{textcomp}  % to address font issues with \textrightarrow
\usepackage{textcomp}                  % use better special symbols
\usepackage{mathptmx}                  % use matching math font
\usepackage{times}                     % we use Times as the main font
\renewcommand*\ttdefault{txtt}         % a nicer typewriter font
\usepackage{cite}                      % needed to automatically sort the references
\usepackage{tabu}                      % only used for the table example
\usepackage{booktabs}                  % only used for the table example
\usepackage{adjustbox}
\usepackage{tabularx} % in the preamble
\usepackage{array}
%\usepackage{subfigure}
\usepackage{caption}
\usepackage{subcaption}
\usepackage{gensymb}
\usepackage{amsmath}
\usepackage{amssymb}    % For additional math symbols
\usepackage{enumitem}
\usepackage{algorithm}
\usepackage{algpseudocode}
\usepackage{multirow}
\usepackage{array}
\usepackage{wrapfig}
\usepackage{graphicx}
\usepackage{soul}



\newcolumntype{P}[1]{>{\centering\arraybackslash}p{#1}}
\newcolumntype{M}[1]{>{\centering\arraybackslash}m{#1}}
\newcolumntype{C}{>{\centering\arraybackslash}X}
%\usepackage[caption=false, font=normalsize, labelfont=sf, textfont=sf]{subfig}
%\usepackage[caption=false,font=footnotesize]{subfig}
%\usepackage{subfig}

\captionsetup{belowskip=-5pt}

% \usepackage{subcaption}
\captionsetup{compatibility=false}

\usepackage{interval}

\usepackage{mathrsfs}

\newcommand{\Biography}[3]{
\vspace{0.9cm}
\noindent\includegraphics[#3]{#1}
\noindent\small #2}


\begin{document}

%%%%%%%%%%%%%%%%%%%%%%%%%%%%%%%%%%%%%%%%%%%%%%%%%%%%%%%%%%%%%%%%
%%%%%%%%%%%%%%%%%%%%%% START OF THE PAPER %%%%%%%%%%%%%%%%%%%%%%
%%%%%%%%%%%%%%%%%%%%%%%%%%%%%%%%%%%%%%%%%%%%%%%%%%%%%%%%%%%%%%%%

%% The ``\maketitle'' command must be the first command after the
%% ``\begin{document}'' command. It prepares and prints the title block.
%% the only exception to this rule is the \firstsection command
% \firstsection{Introduction}

% \maketitle

\section{Introduction}
\label{sec:introduction}
The business processes of organizations are experiencing ever-increasing complexity due to the large amount of data, high number of users, and high-tech devices involved \cite{martin2021pmopportunitieschallenges, beerepoot2023biggestbpmproblems}. This complexity may cause business processes to deviate from normal control flow due to unforeseen and disruptive anomalies \cite{adams2023proceddsriftdetection}. These control-flow anomalies manifest as unknown, skipped, and wrongly-ordered activities in the traces of event logs monitored from the execution of business processes \cite{ko2023adsystematicreview}. For the sake of clarity, let us consider an illustrative example of such anomalies. Figure \ref{FP_ANOMALIES} shows a so-called event log footprint, which captures the control flow relations of four activities of a hypothetical event log. In particular, this footprint captures the control-flow relations between activities \texttt{a}, \texttt{b}, \texttt{c} and \texttt{d}. These are the causal ($\rightarrow$) relation, concurrent ($\parallel$) relation, and other ($\#$) relations such as exclusivity or non-local dependency \cite{aalst2022pmhandbook}. In addition, on the right are six traces, of which five exhibit skipped, wrongly-ordered and unknown control-flow anomalies. For example, $\langle$\texttt{a b d}$\rangle$ has a skipped activity, which is \texttt{c}. Because of this skipped activity, the control-flow relation \texttt{b}$\,\#\,$\texttt{d} is violated, since \texttt{d} directly follows \texttt{b} in the anomalous trace.
\begin{figure}[!t]
\centering
\includegraphics[width=0.9\columnwidth]{images/FP_ANOMALIES.png}
\caption{An example event log footprint with six traces, of which five exhibit control-flow anomalies.}
\label{FP_ANOMALIES}
\end{figure}

\subsection{Control-flow anomaly detection}
Control-flow anomaly detection techniques aim to characterize the normal control flow from event logs and verify whether these deviations occur in new event logs \cite{ko2023adsystematicreview}. To develop control-flow anomaly detection techniques, \revision{process mining} has seen widespread adoption owing to process discovery and \revision{conformance checking}. On the one hand, process discovery is a set of algorithms that encode control-flow relations as a set of model elements and constraints according to a given modeling formalism \cite{aalst2022pmhandbook}; hereafter, we refer to the Petri net, a widespread modeling formalism. On the other hand, \revision{conformance checking} is an explainable set of algorithms that allows linking any deviations with the reference Petri net and providing the fitness measure, namely a measure of how much the Petri net fits the new event log \cite{aalst2022pmhandbook}. Many control-flow anomaly detection techniques based on \revision{conformance checking} (hereafter, \revision{conformance checking}-based techniques) use the fitness measure to determine whether an event log is anomalous \cite{bezerra2009pmad, bezerra2013adlogspais, myers2018icsadpm, pecchia2020applicationfailuresanalysispm}. 

The scientific literature also includes many \revision{conformance checking}-independent techniques for control-flow anomaly detection that combine specific types of trace encodings with machine/deep learning \cite{ko2023adsystematicreview, tavares2023pmtraceencoding}. Whereas these techniques are very effective, their explainability is challenging due to both the type of trace encoding employed and the machine/deep learning model used \cite{rawal2022trustworthyaiadvances,li2023explainablead}. Hence, in the following, we focus on the shortcomings of \revision{conformance checking}-based techniques to investigate whether it is possible to support the development of competitive control-flow anomaly detection techniques while maintaining the explainable nature of \revision{conformance checking}.
\begin{figure}[!t]
\centering
\includegraphics[width=\columnwidth]{images/HIGH_LEVEL_VIEW.png}
\caption{A high-level view of the proposed framework for combining \revision{process mining}-based feature extraction with dimensionality reduction for control-flow anomaly detection.}
\label{HIGH_LEVEL_VIEW}
\end{figure}

\subsection{Shortcomings of \revision{conformance checking}-based techniques}
Unfortunately, the detection effectiveness of \revision{conformance checking}-based techniques is affected by noisy data and low-quality Petri nets, which may be due to human errors in the modeling process or representational bias of process discovery algorithms \cite{bezerra2013adlogspais, pecchia2020applicationfailuresanalysispm, aalst2016pm}. Specifically, on the one hand, noisy data may introduce infrequent and deceptive control-flow relations that may result in inconsistent fitness measures, whereas, on the other hand, checking event logs against a low-quality Petri net could lead to an unreliable distribution of fitness measures. Nonetheless, such Petri nets can still be used as references to obtain insightful information for \revision{process mining}-based feature extraction, supporting the development of competitive and explainable \revision{conformance checking}-based techniques for control-flow anomaly detection despite the problems above. For example, a few works outline that token-based \revision{conformance checking} can be used for \revision{process mining}-based feature extraction to build tabular data and develop effective \revision{conformance checking}-based techniques for control-flow anomaly detection \cite{singh2022lapmsh, debenedictis2023dtadiiot}. However, to the best of our knowledge, the scientific literature lacks a structured proposal for \revision{process mining}-based feature extraction using the state-of-the-art \revision{conformance checking} variant, namely alignment-based \revision{conformance checking}.

\subsection{Contributions}
We propose a novel \revision{process mining}-based feature extraction approach with alignment-based \revision{conformance checking}. This variant aligns the deviating control flow with a reference Petri net; the resulting alignment can be inspected to extract additional statistics such as the number of times a given activity caused mismatches \cite{aalst2022pmhandbook}. We integrate this approach into a flexible and explainable framework for developing techniques for control-flow anomaly detection. The framework combines \revision{process mining}-based feature extraction and dimensionality reduction to handle high-dimensional feature sets, achieve detection effectiveness, and support explainability. Notably, in addition to our proposed \revision{process mining}-based feature extraction approach, the framework allows employing other approaches, enabling a fair comparison of multiple \revision{conformance checking}-based and \revision{conformance checking}-independent techniques for control-flow anomaly detection. Figure \ref{HIGH_LEVEL_VIEW} shows a high-level view of the framework. Business processes are monitored, and event logs obtained from the database of information systems. Subsequently, \revision{process mining}-based feature extraction is applied to these event logs and tabular data input to dimensionality reduction to identify control-flow anomalies. We apply several \revision{conformance checking}-based and \revision{conformance checking}-independent framework techniques to publicly available datasets, simulated data of a case study from railways, and real-world data of a case study from healthcare. We show that the framework techniques implementing our approach outperform the baseline \revision{conformance checking}-based techniques while maintaining the explainable nature of \revision{conformance checking}.

In summary, the contributions of this paper are as follows.
\begin{itemize}
    \item{
        A novel \revision{process mining}-based feature extraction approach to support the development of competitive and explainable \revision{conformance checking}-based techniques for control-flow anomaly detection.
    }
    \item{
        A flexible and explainable framework for developing techniques for control-flow anomaly detection using \revision{process mining}-based feature extraction and dimensionality reduction.
    }
    \item{
        Application to synthetic and real-world datasets of several \revision{conformance checking}-based and \revision{conformance checking}-independent framework techniques, evaluating their detection effectiveness and explainability.
    }
\end{itemize}

The rest of the paper is organized as follows.
\begin{itemize}
    \item Section \ref{sec:related_work} reviews the existing techniques for control-flow anomaly detection, categorizing them into \revision{conformance checking}-based and \revision{conformance checking}-independent techniques.
    \item Section \ref{sec:abccfe} provides the preliminaries of \revision{process mining} to establish the notation used throughout the paper, and delves into the details of the proposed \revision{process mining}-based feature extraction approach with alignment-based \revision{conformance checking}.
    \item Section \ref{sec:framework} describes the framework for developing \revision{conformance checking}-based and \revision{conformance checking}-independent techniques for control-flow anomaly detection that combine \revision{process mining}-based feature extraction and dimensionality reduction.
    \item Section \ref{sec:evaluation} presents the experiments conducted with multiple framework and baseline techniques using data from publicly available datasets and case studies.
    \item Section \ref{sec:conclusions} draws the conclusions and presents future work.
\end{itemize}
\section{RELATED WORK}
\label{sec:relatedwork}
In this section, we describe the previous works related to our proposal, which are divided into two parts. In Section~\ref{sec:relatedwork_exoplanet}, we present a review of approaches based on machine learning techniques for the detection of planetary transit signals. Section~\ref{sec:relatedwork_attention} provides an account of the approaches based on attention mechanisms applied in Astronomy.\par

\subsection{Exoplanet detection}
\label{sec:relatedwork_exoplanet}
Machine learning methods have achieved great performance for the automatic selection of exoplanet transit signals. One of the earliest applications of machine learning is a model named Autovetter \citep{MCcauliff}, which is a random forest (RF) model based on characteristics derived from Kepler pipeline statistics to classify exoplanet and false positive signals. Then, other studies emerged that also used supervised learning. \cite{mislis2016sidra} also used a RF, but unlike the work by \citet{MCcauliff}, they used simulated light curves and a box least square \citep[BLS;][]{kovacs2002box}-based periodogram to search for transiting exoplanets. \citet{thompson2015machine} proposed a k-nearest neighbors model for Kepler data to determine if a given signal has similarity to known transits. Unsupervised learning techniques were also applied, such as self-organizing maps (SOM), proposed \citet{armstrong2016transit}; which implements an architecture to segment similar light curves. In the same way, \citet{armstrong2018automatic} developed a combination of supervised and unsupervised learning, including RF and SOM models. In general, these approaches require a previous phase of feature engineering for each light curve. \par

%DL is a modern data-driven technology that automatically extracts characteristics, and that has been successful in classification problems from a variety of application domains. The architecture relies on several layers of NNs of simple interconnected units and uses layers to build increasingly complex and useful features by means of linear and non-linear transformation. This family of models is capable of generating increasingly high-level representations \citep{lecun2015deep}.

The application of DL for exoplanetary signal detection has evolved rapidly in recent years and has become very popular in planetary science.  \citet{pearson2018} and \citet{zucker2018shallow} developed CNN-based algorithms that learn from synthetic data to search for exoplanets. Perhaps one of the most successful applications of the DL models in transit detection was that of \citet{Shallue_2018}; who, in collaboration with Google, proposed a CNN named AstroNet that recognizes exoplanet signals in real data from Kepler. AstroNet uses the training set of labelled TCEs from the Autovetter planet candidate catalog of Q1–Q17 data release 24 (DR24) of the Kepler mission \citep{catanzarite2015autovetter}. AstroNet analyses the data in two views: a ``global view'', and ``local view'' \citep{Shallue_2018}. \par


% The global view shows the characteristics of the light curve over an orbital period, and a local view shows the moment at occurring the transit in detail

%different = space-based

Based on AstroNet, researchers have modified the original AstroNet model to rank candidates from different surveys, specifically for Kepler and TESS missions. \citet{ansdell2018scientific} developed a CNN trained on Kepler data, and included for the first time the information on the centroids, showing that the model improves performance considerably. Then, \citet{osborn2020rapid} and \citet{yu2019identifying} also included the centroids information, but in addition, \citet{osborn2020rapid} included information of the stellar and transit parameters. Finally, \citet{rao2021nigraha} proposed a pipeline that includes a new ``half-phase'' view of the transit signal. This half-phase view represents a transit view with a different time and phase. The purpose of this view is to recover any possible secondary eclipse (the object hiding behind the disk of the primary star).


%last pipeline applies a procedure after the prediction of the model to obtain new candidates, this process is carried out through a series of steps that include the evaluation with Discovery and Validation of Exoplanets (DAVE) \citet{kostov2019discovery} that was adapted for the TESS telescope.\par
%



\subsection{Attention mechanisms in astronomy}
\label{sec:relatedwork_attention}
Despite the remarkable success of attention mechanisms in sequential data, few papers have exploited their advantages in astronomy. In particular, there are no models based on attention mechanisms for detecting planets. Below we present a summary of the main applications of this modeling approach to astronomy, based on two points of view; performance and interpretability of the model.\par
%Attention mechanisms have not yet been explored in all sub-areas of astronomy. However, recent works show a successful application of the mechanism.
%performance

The application of attention mechanisms has shown improvements in the performance of some regression and classification tasks compared to previous approaches. One of the first implementations of the attention mechanism was to find gravitational lenses proposed by \citet{thuruthipilly2021finding}. They designed 21 self-attention-based encoder models, where each model was trained separately with 18,000 simulated images, demonstrating that the model based on the Transformer has a better performance and uses fewer trainable parameters compared to CNN. A novel application was proposed by \citet{lin2021galaxy} for the morphological classification of galaxies, who used an architecture derived from the Transformer, named Vision Transformer (VIT) \citep{dosovitskiy2020image}. \citet{lin2021galaxy} demonstrated competitive results compared to CNNs. Another application with successful results was proposed by \citet{zerveas2021transformer}; which first proposed a transformer-based framework for learning unsupervised representations of multivariate time series. Their methodology takes advantage of unlabeled data to train an encoder and extract dense vector representations of time series. Subsequently, they evaluate the model for regression and classification tasks, demonstrating better performance than other state-of-the-art supervised methods, even with data sets with limited samples.

%interpretation
Regarding the interpretability of the model, a recent contribution that analyses the attention maps was presented by \citet{bowles20212}, which explored the use of group-equivariant self-attention for radio astronomy classification. Compared to other approaches, this model analysed the attention maps of the predictions and showed that the mechanism extracts the brightest spots and jets of the radio source more clearly. This indicates that attention maps for prediction interpretation could help experts see patterns that the human eye often misses. \par

In the field of variable stars, \citet{allam2021paying} employed the mechanism for classifying multivariate time series in variable stars. And additionally, \citet{allam2021paying} showed that the activation weights are accommodated according to the variation in brightness of the star, achieving a more interpretable model. And finally, related to the TESS telescope, \citet{morvan2022don} proposed a model that removes the noise from the light curves through the distribution of attention weights. \citet{morvan2022don} showed that the use of the attention mechanism is excellent for removing noise and outliers in time series datasets compared with other approaches. In addition, the use of attention maps allowed them to show the representations learned from the model. \par

Recent attention mechanism approaches in astronomy demonstrate comparable results with earlier approaches, such as CNNs. At the same time, they offer interpretability of their results, which allows a post-prediction analysis. \par


\subsection{Greedies}
We have two greedy methods that we're using and testing, but they both have one thing in common: They try every node and possible resistances, and choose the one that results in the largest change in the objective function.

The differences between the two methods, are the function. The first one uses the median (since we want the median to be >0.5, we just set this as our objective function.)

Second one uses a function to try to capture more nuances about the fact that we want the median to be over 0.5. The function is:

\[
\text{score}(\text{opinion}) =
\begin{cases} 
\text{maxScore}, & \text{if } \text{opinion} \geq 0.5 \\
\min\left(\frac{50}{0.5 - \text{opinion}}, \frac{\text{maxScore}}{2}\right), & \text{if } \text{opinion} < 0.5 
\end{cases}
\] 

Where we set maxScore to be $10000$.

\subsection{find-c}
Then for the projected methods where we use Huber-Loss, we also have a $find-c$ version (temporary name). This method initially finds the c for the rest of the run. 

The way it does it it randomly perturbs the resistances and opinions of every node, then finds the c value that most closely approximates the median for all of the perturbed scenarios (after finding the stable opinions). 

\section{Experiments}
\label{sec:exp}
Following the settings in Section \ref{sec:existing}, we evaluate \textit{NovelSum}'s correlation with the fine-tuned model performance across 53 IT datasets and compare it with previous diversity metrics. Additionally, we conduct a correlation analysis using Qwen-2.5-7B \cite{yang2024qwen2} as the backbone model, alongside previous LLaMA-3-8B experiments, to further demonstrate the metric's effectiveness across different scenarios. Qwen is used for both instruction tuning and deriving semantic embeddings. Due to resource constraints, we run each strategy on Qwen for two rounds, resulting in 25 datasets. 

\subsection{Main Results}

\begin{table*}[!t]
    \centering
    \resizebox{\linewidth}{!}{
    \begin{tabular}{lcccccccccc}
    \toprule
    \multirow{3}*{\textbf{Diversity Metrics}} & \multicolumn{10}{c}{\textbf{Data Selection Strategies}} \\
    \cmidrule(lr){2-11}
    & \multirow{2}*{\textbf{K-means}} & \multirow{2}*{\vtop{\hbox{\textbf{K-Center}}\vspace{1mm}\hbox{\textbf{-Greedy}}}}  & \multirow{2}*{\textbf{QDIT}} & \multirow{2}*{\vtop{\hbox{\textbf{Repr}}\vspace{1mm}\hbox{\textbf{Filter}}}} & \multicolumn{5}{c}{\textbf{Random}} & \multirow{2}{*}{\textbf{Duplicate}} \\ 
    \cmidrule(lr){6-10}
    & & & & & \textbf{$\mathcal{X}^{all}$} & ShareGPT & WizardLM & Alpaca & Dolly &  \\
    \midrule
    \rowcolor{gray!15} \multicolumn{11}{c}{\textit{LLaMA-3-8B}} \\
    Facility Loc. $_{\times10^5}$ & \cellcolor{BLUE!40} 2.99 & \cellcolor{ORANGE!10} 2.73 & \cellcolor{BLUE!40} 2.99 & \cellcolor{BLUE!20} 2.86 & \cellcolor{BLUE!40} 2.99 & \cellcolor{BLUE!0} 2.83 & \cellcolor{BLUE!30} 2.88 & \cellcolor{BLUE!0} 2.83 & \cellcolor{ORANGE!20} 2.59 & \cellcolor{ORANGE!30} 2.52 \\    
    DistSum$_{cosine}$  & \cellcolor{BLUE!30} 0.648 & \cellcolor{BLUE!60} 0.746 & \cellcolor{BLUE!0} 0.629 & \cellcolor{BLUE!50} 0.703 & \cellcolor{BLUE!10} 0.634 & \cellcolor{BLUE!40} 0.656 & \cellcolor{ORANGE!30} 0.578 & \cellcolor{ORANGE!10} 0.605 & \cellcolor{ORANGE!20} 0.603 & \cellcolor{BLUE!10} 0.634 \\
    Vendi Score $_{\times10^7}$ & \cellcolor{BLUE!30} 1.70 & \cellcolor{BLUE!60} 2.53 & \cellcolor{BLUE!10} 1.59 & \cellcolor{BLUE!50} 2.23 & \cellcolor{BLUE!20} 1.61 & \cellcolor{BLUE!30} 1.70 & \cellcolor{ORANGE!10} 1.44 & \cellcolor{ORANGE!20} 1.32 & \cellcolor{ORANGE!10} 1.44 & \cellcolor{ORANGE!30} 0.05 \\
    \textbf{NovelSum (Ours)} & \cellcolor{BLUE!60} 0.693 & \cellcolor{BLUE!50} 0.687 & \cellcolor{BLUE!30} 0.673 & \cellcolor{BLUE!20} 0.671 & \cellcolor{BLUE!40} 0.675 & \cellcolor{BLUE!10} 0.628 & \cellcolor{BLUE!0} 0.591 & \cellcolor{ORANGE!10} 0.572 & \cellcolor{ORANGE!20} 0.50 & \cellcolor{ORANGE!30} 0.461 \\
    \midrule    
    \textbf{Model Performance} & \cellcolor{BLUE!60}1.32 & \cellcolor{BLUE!50}1.31 & \cellcolor{BLUE!40}1.25 & \cellcolor{BLUE!30}1.05 & \cellcolor{BLUE!20}1.20 & \cellcolor{BLUE!10}0.83 & \cellcolor{BLUE!0}0.72 & \cellcolor{ORANGE!10}0.07 & \cellcolor{ORANGE!20}-0.14 & \cellcolor{ORANGE!30}-1.35 \\
    \midrule
    \midrule
    \rowcolor{gray!15} \multicolumn{11}{c}{\textit{Qwen-2.5-7B}} \\
    Facility Loc. $_{\times10^5}$ & \cellcolor{BLUE!40} 3.54 & \cellcolor{ORANGE!30} 3.42 & \cellcolor{BLUE!40} 3.54 & \cellcolor{ORANGE!20} 3.46 & \cellcolor{BLUE!40} 3.54 & \cellcolor{BLUE!30} 3.51 & \cellcolor{BLUE!10} 3.50 & \cellcolor{BLUE!10} 3.50 & \cellcolor{ORANGE!20} 3.46 & \cellcolor{BLUE!0} 3.48 \\ 
    DistSum$_{cosine}$ & \cellcolor{BLUE!30} 0.260 & \cellcolor{BLUE!60} 0.440 & \cellcolor{BLUE!0} 0.223 & \cellcolor{BLUE!50} 0.421 & \cellcolor{BLUE!10} 0.230 & \cellcolor{BLUE!40} 0.285 & \cellcolor{ORANGE!20} 0.211 & \cellcolor{ORANGE!30} 0.189 & \cellcolor{ORANGE!10} 0.221 & \cellcolor{BLUE!20} 0.243 \\
    Vendi Score $_{\times10^6}$ & \cellcolor{ORANGE!10} 1.60 & \cellcolor{BLUE!40} 3.09 & \cellcolor{BLUE!10} 2.60 & \cellcolor{BLUE!60} 7.15 & \cellcolor{ORANGE!20} 1.41 & \cellcolor{BLUE!50} 3.36 & \cellcolor{BLUE!20} 2.65 & \cellcolor{BLUE!0} 1.89 & \cellcolor{BLUE!30} 3.04 & \cellcolor{ORANGE!30} 0.20 \\
    \textbf{NovelSum (Ours)}  & \cellcolor{BLUE!40} 0.440 & \cellcolor{BLUE!60} 0.505 & \cellcolor{BLUE!20} 0.403 & \cellcolor{BLUE!50} 0.495 & \cellcolor{BLUE!30} 0.408 & \cellcolor{BLUE!10} 0.392 & \cellcolor{BLUE!0} 0.349 & \cellcolor{ORANGE!10} 0.336 & \cellcolor{ORANGE!20} 0.320 & \cellcolor{ORANGE!30} 0.309 \\
    \midrule
    \textbf{Model Performance} & \cellcolor{BLUE!30} 1.06 & \cellcolor{BLUE!60} 1.45 & \cellcolor{BLUE!40} 1.23 & \cellcolor{BLUE!50} 1.35 & \cellcolor{BLUE!20} 0.87 & \cellcolor{BLUE!10} 0.07 & \cellcolor{BLUE!0} -0.08 & \cellcolor{ORANGE!10} -0.38 & \cellcolor{ORANGE!30} -0.49 & \cellcolor{ORANGE!20} -0.43 \\
    \bottomrule
    \end{tabular}
    }
    \caption{Measuring the diversity of datasets selected by different strategies using \textit{NovelSum} and baseline metrics. Fine-tuned model performances (Eq. \ref{eq:perf}), based on MT-bench and AlpacaEval, are also included for cross reference. Darker \colorbox{BLUE!60}{blue} shades indicate higher values for each metric, while darker \colorbox{ORANGE!30}{orange} shades indicate lower values. While data selection strategies vary in performance on LLaMA-3-8B and Qwen-2.5-7B, \textit{NovelSum} consistently shows a stronger correlation with model performance than other metrics. More results are provided in Appendix \ref{app:results}.}
    \label{tbl:main}
    \vspace{-4mm}
\end{table*}


\begin{table}[t!]
\centering
\resizebox{\linewidth}{!}{
\begin{tabular}{lcccc}
\toprule
\multirow{2}*{\textbf{Diversity Metrics}} & \multicolumn{3}{c}{\textbf{LLaMA}} & \textbf{Qwen}\\
\cmidrule(lr){2-4} \cmidrule(lr){5-5} 
& \textbf{Pearson} & \textbf{Spearman} & \textbf{Avg.} & \textbf{Avg.} \\
\midrule
TTR & -0.38 & -0.16 & -0.27 & -0.30 \\
vocd-D & -0.43 & -0.17 & -0.30 & -0.31 \\
\midrule
Facility Loc. & 0.86 & 0.69 & 0.77 & 0.08 \\
Entropy & 0.93 & 0.80 & 0.86 & 0.63 \\
\midrule
LDD & 0.61 & 0.75 & 0.68 & 0.60 \\
KNN Distance & 0.59 & 0.80 & 0.70 & 0.67 \\
DistSum$_{cosine}$ & 0.85 & 0.67 & 0.76 & 0.51 \\
Vendi Score & 0.70 & 0.85 & 0.78 & 0.60 \\
DistSum$_{L2}$ & 0.86 & 0.76 & 0.81 & 0.51 \\
Cluster Inertia & 0.81 & 0.85 & 0.83 & 0.76 \\
Radius & 0.87 & 0.81 & 0.84 & 0.48 \\
\midrule
NovelSum & \textbf{0.98} & \textbf{0.95} & \textbf{0.97} & \textbf{0.90} \\
\bottomrule
\end{tabular}
}
\caption{Correlations between different metrics and model performance on LLaMA-3-8B and Qwen-2.5-7B.  “Avg.” denotes the average correlation (Eq. \ref{eq:cor}).}
\label{tbl:correlations}
\vspace{-2mm}
\end{table}

\paragraph{\textit{NovelSum} consistently achieves state-of-the-art correlation with model performance across various data selection strategies, backbone LLMs, and correlation measures.}
Table \ref{tbl:main} presents diversity measurement results on datasets constructed by mainstream data selection methods (based on $\mathcal{X}^{all}$), random selection from various sources, and duplicated samples (with only $m=100$ unique samples). 
Results from multiple runs are averaged for each strategy.
Although these strategies yield varying performance rankings across base models, \textit{NovelSum} consistently tracks changes in IT performance by accurately measuring dataset diversity. For instance, K-means achieves the best performance on LLaMA with the highest NovelSum score, while K-Center-Greedy excels on Qwen, also correlating with the highest NovelSum. Table \ref{tbl:correlations} shows the correlation coefficients between various metrics and model performance for both LLaMA and Qwen experiments, where \textit{NovelSum} achieves state-of-the-art correlation across different models and measures.

\paragraph{\textit{NovelSum} can provide valuable guidance for data engineering practices.}
As a reliable indicator of data diversity, \textit{NovelSum} can assess diversity at both the dataset and sample levels, directly guiding data selection and construction decisions. For example, Table \ref{tbl:main} shows that the combined data source $\mathcal{X}^{all}$ is a better choice for sampling diverse IT data than other sources. Moreover, \textit{NovelSum} can offer insights through comparative analyses, such as: (1) ShareGPT, which collects data from real internet users, exhibits greater diversity than Dolly, which relies on company employees, suggesting that IT samples from diverse sources enhance dataset diversity \cite{wang2024diversity-logD}; (2) In LLaMA experiments, random selection can outperform some mainstream strategies, aligning with prior work \cite{xia2024rethinking,diddee2024chasing}, highlighting gaps in current data selection methods for optimizing diversity.



\subsection{Ablation Study}


\textit{NovelSum} involves several flexible hyperparameters and variations. In our main experiments, \textit{NovelSum} uses cosine distance to compute $d(x_i, x_j)$ in Eq. \ref{eq:dad}. We set $\alpha = 1$, $\beta = 0.5$, and $K = 10$ nearest neighbors in Eq. \ref{eq:pws} and \ref{eq:dad}. Here, we conduct an ablation study to investigate the impact of these settings based on LLaMA-3-8B.

\begin{table}[ht!]
\centering
\resizebox{\linewidth}{!}{
\begin{tabular}{lccc}
\toprule
\textbf{Variants} & \textbf{Pearson} & \textbf{Spearman} & \textbf{Avg.} \\
\midrule
NovelSum & 0.98 & 0.96 & 0.97 \\
\midrule
\hspace{0.10cm} - Use $L2$ distance & 0.97 & 0.83 & 0.90\textsubscript{↓ 0.08} \\
\hspace{0.10cm} - $K=20$ & 0.98 & 0.96 & 0.97\textsubscript{↓ 0.00} \\
\hspace{0.10cm} - $\alpha=0$ (w/o proximity) & 0.79 & 0.31 & 0.55\textsubscript{↓ 0.42} \\
\hspace{0.10cm} - $\alpha=2$ & 0.73 & 0.88 & 0.81\textsubscript{↓ 0.16} \\
\hspace{0.10cm} - $\beta=0$ (w/o density) & 0.92 & 0.89 & 0.91\textsubscript{↓ 0.07} \\
\hspace{0.10cm} - $\beta=1$ & 0.90 & 0.62 & 0.76\textsubscript{↓ 0.21} \\
\bottomrule
\end{tabular}
}
\caption{Ablation Study for \textit{NovelSum}.}
\label{tbl:ablation}
\vspace{-2mm}
\end{table}

In Table \ref{tbl:ablation}, $\alpha=0$ removes the proximity weights, and $\beta=0$ eliminates the density multiplier. We observe that both $\alpha=0$ and $\beta=0$ significantly weaken the correlation, validating the benefits of the proximity-weighted sum and density-aware distance. Additionally, improper values for $\alpha$ and $\beta$ greatly reduce the metric's reliability, highlighting that \textit{NovelSum} strikes a delicate balance between distances and distribution. Replacing cosine distance with Euclidean distance and using more neighbors for density approximation have minimal impact, particularly on Pearson's correlation, demonstrating \textit{NovelSum}'s robustness to different distance measures.






\section{Discussion of Assumptions}\label{sec:discussion}
In this paper, we have made several assumptions for the sake of clarity and simplicity. In this section, we discuss the rationale behind these assumptions, the extent to which these assumptions hold in practice, and the consequences for our protocol when these assumptions hold.

\subsection{Assumptions on the Demand}

There are two simplifying assumptions we make about the demand. First, we assume the demand at any time is relatively small compared to the channel capacities. Second, we take the demand to be constant over time. We elaborate upon both these points below.

\paragraph{Small demands} The assumption that demands are small relative to channel capacities is made precise in \eqref{eq:large_capacity_assumption}. This assumption simplifies two major aspects of our protocol. First, it largely removes congestion from consideration. In \eqref{eq:primal_problem}, there is no constraint ensuring that total flow in both directions stays below capacity--this is always met. Consequently, there is no Lagrange multiplier for congestion and no congestion pricing; only imbalance penalties apply. In contrast, protocols in \cite{sivaraman2020high, varma2021throughput, wang2024fence} include congestion fees due to explicit congestion constraints. Second, the bound \eqref{eq:large_capacity_assumption} ensures that as long as channels remain balanced, the network can always meet demand, no matter how the demand is routed. Since channels can rebalance when necessary, they never drop transactions. This allows prices and flows to adjust as per the equations in \eqref{eq:algorithm}, which makes it easier to prove the protocol's convergence guarantees. This also preserves the key property that a channel's price remains proportional to net money flow through it.

In practice, payment channel networks are used most often for micro-payments, for which on-chain transactions are prohibitively expensive; large transactions typically take place directly on the blockchain. For example, according to \cite{river2023lightning}, the average channel capacity is roughly $0.1$ BTC ($5,000$ BTC distributed over $50,000$ channels), while the average transaction amount is less than $0.0004$ BTC ($44.7k$ satoshis). Thus, the small demand assumption is not too unrealistic. Additionally, the occasional large transaction can be treated as a sequence of smaller transactions by breaking it into packets and executing each packet serially (as done by \cite{sivaraman2020high}).
Lastly, a good path discovery process that favors large capacity channels over small capacity ones can help ensure that the bound in \eqref{eq:large_capacity_assumption} holds.

\paragraph{Constant demands} 
In this work, we assume that any transacting pair of nodes have a steady transaction demand between them (see Section \ref{sec:transaction_requests}). Making this assumption is necessary to obtain the kind of guarantees that we have presented in this paper. Unless the demand is steady, it is unreasonable to expect that the flows converge to a steady value. Weaker assumptions on the demand lead to weaker guarantees. For example, with the more general setting of stochastic, but i.i.d. demand between any two nodes, \cite{varma2021throughput} shows that the channel queue lengths are bounded in expectation. If the demand can be arbitrary, then it is very hard to get any meaningful performance guarantees; \cite{wang2024fence} shows that even for a single bidirectional channel, the competitive ratio is infinite. Indeed, because a PCN is a decentralized system and decisions must be made based on local information alone, it is difficult for the network to find the optimal detailed balance flow at every time step with a time-varying demand.  With a steady demand, the network can discover the optimal flows in a reasonably short time, as our work shows.

We view the constant demand assumption as an approximation for a more general demand process that could be piece-wise constant, stochastic, or both (see simulations in Figure \ref{fig:five_nodes_variable_demand}).
We believe it should be possible to merge ideas from our work and \cite{varma2021throughput} to provide guarantees in a setting with random demands with arbitrary means. We leave this for future work. In addition, our work suggests that a reasonable method of handling stochastic demands is to queue the transaction requests \textit{at the source node} itself. This queuing action should be viewed in conjunction with flow-control. Indeed, a temporarily high unidirectional demand would raise prices for the sender, incentivizing the sender to stop sending the transactions. If the sender queues the transactions, they can send them later when prices drop. This form of queuing does not require any overhaul of the basic PCN infrastructure and is therefore simpler to implement than per-channel queues as suggested by \cite{sivaraman2020high} and \cite{varma2021throughput}.

\subsection{The Incentive of Channels}
The actions of the channels as prescribed by the DEBT control protocol can be summarized as follows. Channels adjust their prices in proportion to the net flow through them. They rebalance themselves whenever necessary and execute any transaction request that has been made of them. We discuss both these aspects below.

\paragraph{On Prices}
In this work, the exclusive role of channel prices is to ensure that the flows through each channel remains balanced. In practice, it would be important to include other components in a channel's price/fee as well: a congestion price  and an incentive price. The congestion price, as suggested by \cite{varma2021throughput}, would depend on the total flow of transactions through the channel, and would incentivize nodes to balance the load over different paths. The incentive price, which is commonly used in practice \cite{river2023lightning}, is necessary to provide channels with an incentive to serve as an intermediary for different channels. In practice, we expect both these components to be smaller than the imbalance price. Consequently, we expect the behavior of our protocol to be similar to our theoretical results even with these additional prices.

A key aspect of our protocol is that channel fees are allowed to be negative. Although the original Lightning network whitepaper \cite{poon2016bitcoin} suggests that negative channel prices may be a good solution to promote rebalancing, the idea of negative prices in not very popular in the literature. To our knowledge, the only prior work with this feature is \cite{varma2021throughput}. Indeed, in papers such as \cite{van2021merchant} and \cite{wang2024fence}, the price function is explicitly modified such that the channel price is never negative. The results of our paper show the benefits of negative prices. For one, in steady state, equal flows in both directions ensure that a channel doesn't loose any money (the other price components mentioned above ensure that the channel will only gain money). More importantly, negative prices are important to ensure that the protocol selectively stifles acyclic flows while allowing circulations to flow. Indeed, in the example of Section \ref{sec:flow_control_example}, the flows between nodes $A$ and $C$ are left on only because the large positive price over one channel is canceled by the corresponding negative price over the other channel, leading to a net zero price.

Lastly, observe that in the DEBT control protocol, the price charged by a channel does not depend on its capacity. This is a natural consequence of the price being the Lagrange multiplier for the net-zero flow constraint, which also does not depend on the channel capacity. In contrast, in many other works, the imbalance price is normalized by the channel capacity \cite{ren2018optimal, lin2020funds, wang2024fence}; this is shown to work well in practice. The rationale for such a price structure is explained well in \cite{wang2024fence}, where this fee is derived with the aim of always maintaining some balance (liquidity) at each end of every channel. This is a reasonable aim if a channel is to never rebalance itself; the experiments of the aforementioned papers are conducted in such a regime. In this work, however, we allow the channels to rebalance themselves a few times in order to settle on a detailed balance flow. This is because our focus is on the long-term steady state performance of the protocol. This difference in perspective also shows up in how the price depends on the channel imbalance. \cite{lin2020funds} and \cite{wang2024fence} advocate for strictly convex prices whereas this work and \cite{varma2021throughput} propose linear prices.

\paragraph{On Rebalancing} 
Recall that the DEBT control protocol ensures that the flows in the network converge to a detailed balance flow, which can be sustained perpetually without any rebalancing. However, during the transient phase (before convergence), channels may have to perform on-chain rebalancing a few times. Since rebalancing is an expensive operation, it is worthwhile discussing methods by which channels can reduce the extent of rebalancing. One option for the channels to reduce the extent of rebalancing is to increase their capacity; however, this comes at the cost of locking in more capital. Each channel can decide for itself the optimum amount of capital to lock in. Another option, which we discuss in Section \ref{sec:five_node}, is for channels to increase the rate $\gamma$ at which they adjust prices. 

Ultimately, whether or not it is beneficial for a channel to rebalance depends on the time-horizon under consideration. Our protocol is based on the assumption that the demand remains steady for a long period of time. If this is indeed the case, it would be worthwhile for a channel to rebalance itself as it can make up this cost through the incentive fees gained from the flow of transactions through it in steady state. If a channel chooses not to rebalance itself, however, there is a risk of being trapped in a deadlock, which is suboptimal for not only the nodes but also the channel.

\section{Conclusion}
This work presents DEBT control: a protocol for payment channel networks that uses source routing and flow control based on channel prices. The protocol is derived by posing a network utility maximization problem and analyzing its dual minimization. It is shown that under steady demands, the protocol guides the network to an optimal, sustainable point. Simulations show its robustness to demand variations. The work demonstrates that simple protocols with strong theoretical guarantees are possible for PCNs and we hope it inspires further theoretical research in this direction.
\section{Conclusion}
In this work, we propose a simple yet effective approach, called SMILE, for graph few-shot learning with fewer tasks. Specifically, we introduce a novel dual-level mixup strategy, including within-task and across-task mixup, for enriching the diversity of nodes within each task and the diversity of tasks. Also, we incorporate the degree-based prior information to learn expressive node embeddings. Theoretically, we prove that SMILE effectively enhances the model's generalization performance. Empirically, we conduct extensive experiments on multiple benchmarks and the results suggest that SMILE significantly outperforms other baselines, including both in-domain and cross-domain few-shot settings.
% \section*{Acknowledgment}
This work was supported by the National Natural Science Foundation of China (62441239,~U23A20319,~62172056,~62472394,~62192784, \\ U22B2038) as well as the 8th Young Elite Scientists Sponsorship Program by CAST (2022QNRC001).


%% \section{Introduction} %for journal use above \firstsection{..} instead
% This template is for papers of VGTC-sponsored conferences such as IEEE VIS, IEEE VR, and ISMAR which are published as special issues of TVCG.
% The template does not contain the respective dates of the conference/journal issue, these will be entered by IEEE as part of the publication production process.
% Therefore, \textbf{please leave the copyright statement at the bottom-left of this first page untouched}.


% \section{Author Details}

% You should specify ORCID IDs for each author (see \url{https://orcid.org/}  to register) for disambiguation and long-term contact preservation.
% Use \verb|\authororcid{Author Name}{0000-0000-0000-0000}| for each author, replacing the ``Author Name'' and using the 16-digit (hyphenated) ORCID ID for the second parameter.
% The template shows an example without ORCID IDs for two of the authors.
% ORCID IDs should be provided in all cases.

% Each author's affiliations have to be provided in the author footer on the bottom-left corner of the first page.
% It is permitted to merge two or more people from the same institution as long as they are shown in the same order as in the overall author sequence on the top of the first page.
% For example, if authors A, B, C, and D are from institutions 1, 2, 1, and 2, respectively, then it is ok to use 2 bullets as follows:
% \begin{itemize}
%   \item A and C are with Institution 1. E-mail: \{a\,$|$\,c\}@i1.com\,.

%   \item B and D are with Institution 2. E-mail: \{b\,$|$\,d\}@i2.org\,.
% \end{itemize}


% \section{Hyperlinks and Cross References}

% The style uses the \verb|hyperref| package which can typeset clickable hyperlinks using \verb|\href{...}{...}|, hyperlinked URLs using \verb|\url{...}|, and turns references into internal links.

% The style also uses \verb|cleveref| to automatically and consistently format cross references.
% We recommend that you use the \verb|\cref{label}| and \verb|\Cref{label}| calls instead of \verb|Figure~\ref{label}| or similar.
% \verb|\Cref| should be used when starting a sentence to spell out the reference (e.g.\ ``Section'') while \verb|\cref| should be used when referencing within a sentence to abbreviate (e.g.\ ``Sec.'').
% Here are examples for use within a sentence: \cref{fig:vis_papers}, \cref{tab:vis_papers}, \cref{sec:supplement_inst,sec:references_inst}, \cref{eq:sum}.
% The following sentences all start with a reference, so use \verb|\Cref|.
% \Cref{fig:vis_papers} is a \verb|figure| environment.
% \Cref{tab:vis_papers} is a \verb|table| environment.
% \Cref{sec:supplement_inst,sec:references_inst} are \verb|section| environments.
% \Cref{eq:sum} is an \verb|equation| environment.


% \section{Figures}

% \subsection{Loading figures}

% The style automatically looks for image files with the correct extension (eps for regular \LaTeX; pdf, png, and jpg for pdf\LaTeX), in a set of given subfolders defined above using \verb|\graphicspath|: figures/, pictures/, images/.
% It is thus sufficient to use \verb|\includegraphics{CypressView}| (instead of \verb|\includegraphics{pictures/CypressView.jpg}|).
% Figures should be in CMYK or Grey scale format, otherwise, colour shifting may occur during the printing process.

% \subsection{Vector figures}

% Vector graphics like svg, eps, pdf are best for charts and other figures with text or lines.
% They will look much nicer and crisper and any text in them will be more selectable, searchable, and accessible.

% \subsection{Raster figures}

% Of the raster graphics formats, screenshots of user interfaces and text, as well as line art, are better shown with png.
% jpg is better for photographs.
% Make sure all raster graphics are captured in high enough resolution so they look crisp and scale well.

% \subsection{Figures on the first page}

% The teaser figure should only have the width of the abstract as the template enforces it.
% The use of figures other than the optional teaser is not permitted on the first page.
% Other figures should begin on the second page.
% Papers submitted with figures other than the optional teaser on the first page will be refused.

% \subsection{Subfigures}

% You can add subfigures using the \texttt{subcaption} package that is automatically loaded.
% Inside a \verb|figure| environment, create a \verb|subfigure| environment.
% See \cref{fig:ex_subfigs} for an example.
% You can reference individual figures, either fully using \verb|\cref| (\cref{fig:ex_subfigs_a,fig:ex_subfigs_b}) or by letter using \verb|\subref|.
% E.g., \subref{fig:ex_subfigs_b}, \subref{fig:ex_subfigs_c}.
% Note that \verb|\subref| only works for one label at a time.

% \begin{figure}[tbp]
%   \centering
%   \begin{subfigure}[b]{0.45\columnwidth}
%   	\centering
%   	\includegraphics[width=\textwidth]{example-image-a}
%   	\caption{The letter A.}
%   	\label{fig:ex_subfigs_a}
%   \end{subfigure}%
%   \hfill%
%   \begin{subfigure}[b]{0.45\columnwidth}
%   	\centering
%   	\includegraphics[width=\textwidth]{example-image-b}
%   	\caption{The letter B.}
%   	\label{fig:ex_subfigs_b}
%   \end{subfigure}%
%   \\%
%   \begin{subfigure}[b]{0.45\columnwidth}
%   	\centering
%   	\includegraphics[width=\textwidth]{example-image-c}
%   	\caption{The letter C.}
%   	\label{fig:ex_subfigs_c}
%   \end{subfigure}%
%   \subfigsCaption{Example of adding subfigures with the \texttt{subcaption} package.}
%   \label{fig:ex_subfigs}
% \end{figure}

% \subsection{Figure Credits}
% \label{sec:figure_credits_inst}

% In the \hyperref[sec:figure_credits]{Figure Credits} section at the end of the paper, you should credit the original sources of any figures that were reproduced or modified.
% Include any license details necessary, as well as links to the original materials whenever possible.
% For credits to figures from academic papers, include a citation that is listed in the \textbf{References} section.
% An example is provided \hyperref[sec:figure_credits]{below}.


% \section{Equations and Tables}

% Equations can be added like so:

% \begin{equation}
%   \label{eq:sum}
%   \sum_{j=1}^{z} j = \frac{z(z+1)}{2}
% \end{equation}

% Tables, such as \cref{tab:vis_papers} can also be included.


% \begin{table}[tb]
%   \caption{%
%   	VIS/VisWeek accepted/presented papers: 1990--2016.%
%   }
%   \label{tab:vis_papers}
%   \scriptsize%
%   \centering%
%   \begin{tabu}{%
%   	  r%
%   	  	*{7}{c}%
%   	  	*{2}{r}%
%   	}
%   	\toprule
%   	year & \rotatebox{90}{Vis/SciVis} &   \rotatebox{90}{SciVis conf} &   \rotatebox{90}{InfoVis} &   \rotatebox{90}{VAST} &   \rotatebox{90}{VAST conf} &   \rotatebox{90}{TVCG @ VIS} &   \rotatebox{90}{CG\&A @ VIS} &   \rotatebox{90}{VIS/VisWeek} \rotatebox{90}{incl.\ TVCG/CG\&A}   &   \rotatebox{90}{VIS/VisWeek} \rotatebox{90}{w/o TVCG/CG\&A}   \\
%   	\midrule
%   	2016 & 30 &   & 37 & 33 & 15 & 23 & 10 & 148 & 115 \\
%   	2015 & 33 & 9 & 38 & 33 & 14 & 17 & 15 & 159 & 127 \\
%   	2014 & 34 &   & 45 & 33 & 21 & 20 &    & 153 & 133 \\
%   	2013 & 31 &   & 38 & 32 &    & 20 &    & 121 & 101 \\
%   	2012 & 42 &   & 44 & 30 &    & 23 &    & 139 & 116 \\
%   	2011 & 49 &   & 44 & 26 &    & 20 &    & 139 & 119 \\
%   	2010 & 48 &   & 35 & 26 &    &    &    & 109 & 109 \\
%   	2009 & 54 &   & 37 & 26 &    &    &    & 117 & 117 \\
%   	2008 & 50 &   & 28 & 21 &    &    &    &  99 &  99 \\
%   	2007 & 56 &   & 27 & 24 &    &    &    & 107 & 107 \\
%   	2006 & 63 &   & 24 & 26 &    &    &    & 113 & 113 \\
%   	2005 & 88 &   & 31 &    &    &    &    & 119 & 119 \\
%   	2004 & 70 &   & 27 &    &    &    &    &  97 &  97 \\
%   	2003 & 74 &   & 29 &    &    &    &    & 103 & 103 \\
%   	2002 & 78 &   & 23 &    &    &    &    & 101 & 101 \\
%   	2001 & 74 &   & 22 &    &    &    &    &  96 &  96 \\
%   	2000 & 73 &   & 20 &    &    &    &    &  93 &  93 \\
%   	1999 & 69 &   & 19 &    &    &    &    &  88 &  88 \\
%   	1998 & 72 &   & 18 &    &    &    &    &  90 &  90 \\
%   	1997 & 72 &   & 16 &    &    &    &    &  88 &  88 \\
%   	1996 & 65 &   & 12 &    &    &    &    &  77 &  77 \\
%   	1995 & 56 &   & 18 &    &    &    &    &  74 &  74 \\
%   	1994 & 53 &   &    &    &    &    &    &  53 &  53 \\
%   	1993 & 55 &   &    &    &    &    &    &  55 &  55 \\
%   	1992 & 53 &   &    &    &    &    &    &  53 &  53 \\
%   	1991 & 50 &   &    &    &    &    &    &  50 &  50 \\
%   	1990 & 53 &   &    &    &    &    &    &  53 &  53 \\
%   	\midrule               
%   	\textbf{sum} & \textbf{1545} & \textbf{9} & \textbf{632} & \textbf{310} & \textbf{50} & \textbf{123} & \textbf{25} & \textbf{2694} & \textbf{2546} \\
%   	\bottomrule
%   \end{tabu}%
% \end{table}

% \begin{figure}[tb]% specify a combination of t, b, p, or h for top, bottom, on its own page, or here
%   \centering % avoid the use of \begin{center}...\end{center} and use \centering instead (more compact)
%   \includegraphics[width=\columnwidth]{paper-count-2016}
%   \caption{%
%   	A visualization of the 1990--2016 data from \cref{tab:vis_papers}, recreated based on Fig.\ 1 from \cite{Isenberg:2017:VMC}.%
%   }
%   \label{fig:vis_papers}
% \end{figure}


% \section{Supplemental Material Instructions}
% \label{sec:supplement_inst}

% In support of transparent research practices and long-term open science goals, you are encouraged to make your supplemental materials available on a publicly-accessible repository.
% Please describe the available supplemental materials in the \hyperref[sec:supplemental_materials]{Supplemental Materials} section.
% These details could include (1) what materials are available, (2) where they are hosted, and (3) any necessary omissions.


% \section{References}
% \label{sec:references_inst}

% An example of the reference formatting is provided in the \textbf{References} section at the end.


% \subsection{Include DOIs}

% All references which have a DOI should have it included in the bib\TeX\ for the style to display.
% The DOI can be entered with or without the \url{https://doi.org/} prefix.

% \subsection{Narrow DOI option}

% The \verb|-narrow| versions of the bibliography style use the font \verb|PTSansNarrow-TLF| for typesetting the DOIs in a compact way.
% This font needs to be available on your \LaTeX\ system.
% It is part of the \href{https://www.ctan.org/pkg/paratype}{\texttt{paratype} package}, and many distributions (such as MikTeX) have it automatically installed.
% If you do not have this package yet and want to use a \verb|-narrow| bibliography style then use your \LaTeX\ system's package installer to add it.
% If this is not possible you can also revert to the respective bibliography styles without the \verb|-narrow| in the file name.
% DVI-based processes to compile the template apparently cannot handle the different font so, by default, the template file uses the \texttt{abbrv-doi} bibliography style.

% \subsection{Disabling hyperlinks}

% To avoid adding hyperlinks to the references (the default) you can use \verb|\bibliographystyle{abbrv-doi}| instead of \verb|\bibliographystyle{abbrv-doi-hyperref}|.
% By default, the DOI field in a bib\TeX\ entry is turned into a hyperlink.

% See the examples in the bib\TeX\ file and the bibliography at the end of this template.

% \subsection{Guidelines for bibTeX}

% \begin{itemize}
%   \item All bibliographic entries should be sorted alphabetically by the last name of the first author.
%         This \LaTeX/bib\TeX\ template takes care of this sorting automatically.
%   \item Merge multiple references into one; e.\,g., use \cite{Max:1995:OMF,Kitware:2003} (not \cite{Kitware:2003}\cite{Max:1995:OMF}).
%         Within each set of multiple references, the references should be sorted in ascending order.
%         This \LaTeX/bib\TeX\ template takes care of both the merging and the sorting automatically.
%   \item Verify all data obtained from digital libraries, even ACM's DL and IEEE Xplore  etc.\ are sometimes wrong or incomplete.
%   \item Do not trust bibliographic data from other services such as Mendeley.com, Google Scholar, or similar; these are even more likely to be incorrect or incomplete.
%   \item Articles in journal---items to include:
%         \begin{itemize}
%   	      \item author names
%   	      \item title
%   	      \item journal name
%   	      \item year
%   	      \item volume
%   	      \item number
%   	      \item month of publication as variable name (i.e., \{jan\} for January, etc.; month ranges using \{jan \#\{/\}\# feb\} or \{jan \#\{-{}-\}\# feb\})
%         \end{itemize}
%   \item Use journal names in proper style: correct: ``IEEE Transactions on Visualization and Computer Graphics'', incorrect: ``Visualization and Computer Graphics, IEEE Transactions on''
%   \item Papers in proceedings---items to include:
%         \begin{itemize}
%   	      \item author names
%   	      \item title
%   	      \item abbreviated proceedings name: e.g., ``Proc.\textbackslash{} CONF\_ACRONYNM'' without the year; example: ``Proc.\textbackslash{} CHI'', ``Proc.\textbackslash{} 3DUI'', ``Proc.\textbackslash{} Eurographics'', ``Proc.\textbackslash{} EuroVis''
%   	      \item year
%   	      \item publisher
%   	      \item town with country of publisher (the town can be abbreviated for well-known towns such as New York or Berlin)
%         \end{itemize}

%   \item Article/paper title convention: refrain from using curly brackets, except for acronyms/proper names/words following dashes/question marks etc.; example:\\\\
%         %
%         The paper ``Marching Cubes: A High Resolution 3D Surface Construction Algorithm'' should be entered as ``\{M\}arching \{C\}ubes: A High Resolution \{3D\} Surface Construction Algorithm'' or  ``\{M\}arching \{C\}ubes: A high resolution \{3D\} surface construction algorithm''.
%         It will then be typeset as ``Marching Cubes: A high resolution 3D surface construction algorithm''
%   \item For all entries:
%         \begin{itemize}
%   	      \item DOI can be entered in the DOI field as plain DOI number or as DOI url.
%   	      \item Provide full page ranges AA-{}-BB
%         \end{itemize}
%   \item When citing references, do not use the reference as a sentence object; e.g., wrong: ``In \cite{Lorensen:1987:MCA} the authors describe \dots'', correct: ``Lorensen and Cline \cite{Lorensen:1987:MCA} describe \dots''
% \end{itemize}

% \section{Appendices}
% \label{sec:appendices_inst}

% Appendices can be specified using \verb|\appendix|.
% For example, our Troubleshooting instructions in
% \iflabelexists{appendix:troubleshooting}
%   {\cref{appendix:troubleshooting}}
%   {the appendix of the full paper at \url{https://osf.io/nrmyc}}.

% Note that the paper submission has to end after the \textbf{References} section and within the page limit of the conference you are submitting to.
% Any version of Appendices or the paper with Appendices included has to be submitted separately as supplementary material.
% You can use the \verb|hideappendix| class option to remove everything after \verb|\appendix|.
% We encourage you to submit a full version of your paper to a preprint server with any appendices included.

% You can use the \verb|\iflabelexists| macro to cross reference an appendix from the main text, but only if that label (i.e.\ the appendix) actually exists.
% For example, above we use 

% \begin{verbatim}
% \iflabelexists{appendix:troubleshooting}
%   {\cref{appendix:troubleshooting}}
%   {the appendix of the full paper at
%    \url{https://osf.io/XXXXX}}.
% \end{verbatim}

% in order to cross-reference to the appendix with \verb|\cref| if it exists, but if the appendix is commented out then we will simply create a hyperlinked URL to it.


% \section{Filler Text to Flush Out the Paper}

% \lipsum[1-2]% Just add some more arbitrary text so we see a fuller paper example


% \section*{Supplemental Materials}
% \label{sec:supplemental_materials}

% Refer to the instructions for this section (\cref{sec:supplement_inst}).
% Below is an example you can follow that includes the actual supplemental material for this template:

% All supplemental materials are available on OSF at \url{https://doi.org/10.17605/OSF.IO/2NBSG}, released under a CC BY 4.0 license.
% In particular, they include (1) Excel files containing the data for and analyses for creating \cref{tab:vis_papers} and \cref{fig:vis_papers}, (2) figure images in multiple formats, and (3) a full version of this paper with all appendices.
% Our other code is intellectual property of a corporation---Starbucks Research---and there is no feasible way to share it publicly.


% \section*{Figure Credits}
% \label{sec:figure_credits}

% Refer to the instructions for this section (\cref{sec:figure_credits_inst}).
% Here are the actual figure credits for this template:

% \Cref{fig:teaser} image credit: Scott Miller / Special to the Vancouver Sun, January 22, 2009, page A6.

% \Cref{fig:vis_papers} is a partial recreation of Fig.\ 1 from \cite{Isenberg:2017:VMC}, which is in the public domain.


% %% if specified like this the section will be ommitted in review mode
% \acknowledgments{%
% 	The authors wish to thank A, B, and C.
%   This work was supported in part by a grant from XYZ (\# 12345-67890).%
% }





% \bibliographystyle{abbrv-doi-hyperref-narrow}
\bibliographystyle{abbrv-doi}
% \bibliographystyle{abbrv-doi-narrow}

\bibliography{template}

% \Biography{figures/JianxinSun_photo.jpeg}{\textbf{Jason Sun} is a Ph.D. candidate in the School of Computing at the University of Nebraska-Lincoln. He received a MS degree in Electrical and Computer Engineering from Purdue University and a BS degree in Electrical Engineering from Harbin Institute of Technology. His research interests are computer vision, machine learning, data modeling, and visualization. His current research focuses on visualizing large-scale scientific multi-dimensional datasets by leveraging deep learning techniques}{width=3cm,height=3.8cm}

% \vspace{15mm}

% \begin{wrapfigure}[10]{l}{1in}
%         \raisebox{11pt}[\dimexpr\height-0.0\baselineskip\relax]{\includegraphics[width=1in,height=1.25in,clip,keepaspectratio]{figures/JianxinSun_photo.jpeg}}%
% \end{wrapfigure}
% \noindent\textbf{Jianxin Sun} is a Ph.D. candidate in the School of Computing at the University of Nebraska-Lincoln. He received an MS degree in Electrical and Computer Engineering from Purdue University and a BS degree in Electrical Engineering from Harbin Institute of Technology. His research interests are computer vision, machine learning, data modeling, and visualization. His current research focuses on visualizing large-scale scientific multi-dimensional datasets by leveraging deep learning techniques.

% \vspace{12mm}

% \begin{wrapfigure}[10]{l}{1in}
%         \raisebox{14pt}[\dimexpr\height-0.0\baselineskip\relax]{\includegraphics[width=1in,height=1.25in,clip,keepaspectratio]{figures/XinyanXie_photo.jpeg}}%
% \end{wrapfigure}
% \noindent\textbf{Xinyan Xie} is a Ph.D. candidate in the School of Computing at the University of Nebraska-Lincoln. She received an MS degree in Statistics from the same university. Her research includes machine learning, computational geometry, 3D object reconstruction, and detection. Her current research focuses on feature learning from hyperspectral data, detection, localization, and classification of 3D objects in the form of point clouds, and machine learning applications in plant science.

% \vspace{12mm}

% \begin{wrapfigure}[10]{l}{1in}
%         \raisebox{14pt}[\dimexpr\height-0.0\baselineskip\relax]{\includegraphics[width=1in,height=1.25in,clip,keepaspectratio]{figures/HongfengYu_photo.jpeg}}%
% \end{wrapfigure}
% \noindent\textbf{Hongfeng Yu} is an associate professor in the School of Computing at the University of Nebraska-Lincoln. He received a Ph.D. degree in Computer Science from the University of California-Davis. Dr. Yu’s research concentrates on big data analysis and visualization, high-performance computing, and user interfaces and interaction.

% \vfill

% \begin{wrapfigure}{l}{25mm}
%         \raisebox{0pt}[\dimexpr\height-1\baselineskip\relax]{\includegraphics[width=1in,height=1.25in,clip,keepaspectratio]{figures/XinyanXie_photo.jpeg}}%
%         \end{wrapfigure}
% \noindent\textbf{Xinyan Xie} is a Ph.D. candidate in the School of Computing at the University of Nebraska-Lincoln. She received a MS degree in Statistics from the same university. Her research includes machine learning, computational geometry, 3D object reconstruction and detection. Her current research focuses on feature learning from hyperspectral data, detection, localization, and classification of 3D object in form of point clouds, and machine learning applications in plant science.

% \vfill

% \begin{wrapfigure}{l}{25mm}
%         \raisebox{0pt}[\dimexpr\height-1\baselineskip\relax]{\includegraphics[width=1in,height=1.25in,clip,keepaspectratio]{figures/HongfengYu_photo.jpeg}}%
%         \end{wrapfigure}
% \noindent\textbf{Hongfeng Yu} is an associate professor in the School of Computing at the University of Nebraska-Lincoln. He received a Ph.D. degree in Computer Science from the University of California-Davis. Dr. Yu’s research concentrates on big data analysis and visualization, high-performance computing, and user interfaces and interaction.




% \appendix % You can use the `hideappendix` class option to skip everything after \appendix

% \section{About Appendices}
% Refer to \cref{sec:appendices_inst} for instructions regarding appendices.

% \section{Troubleshooting}
% \label{appendix:troubleshooting}

% \subsection{ifpdf error}

% If you receive compilation errors along the lines of \texttt{Package ifpdf Error: Name clash, \textbackslash ifpdf is already defined} then please add a new line \verb|\let\ifpdf\relax| right after the \verb|\documentclass[journal]{vgtc}| call.
% Note that your error is due to packages you use that define \verb|\ifpdf| which is obsolete (the result is that \verb|\ifpdf| is defined twice); these packages should be changed to use \verb|ifpdf| package instead.


% \subsection{\texttt{pdfendlink} error}

% Occasionally (for some \LaTeX\ distributions) this hyper-linked bib\TeX\ style may lead to \textbf{compilation errors} (\texttt{pdfendlink ended up in different nesting level ...}) if a reference entry is broken across two pages (due to a bug in \verb|hyperref|).
% In this case, make sure you have the latest version of the \verb|hyperref| package (i.e.\ update your \LaTeX\ installation/packages) or, alternatively, revert back to \verb|\bibliographystyle{abbrv-doi}| (at the expense of removing hyperlinks from the bibliography) and try \verb|\bibliographystyle{abbrv-doi-hyperref}| again after some more editing.

\end{document}

