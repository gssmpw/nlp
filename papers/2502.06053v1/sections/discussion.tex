\section{Limitations and Future Work}
In this paper, we provide a way to further optimize the already fast volume visualization method leveraging RecNN. We concentrate on the rationale and validation of the proposed method, as well as the evaluation of how much rendering latency can be reduced while maintaining similar rendering quality. We demonstrate that the network can successfully learn the important pixels from the C-Image by jointly considering various inputs and configurations of a volume visualization system. 

One limitation of the current design is that the mean value used to determine the percentage of C-Image as important pixels is a predefined hyperparameter. While users can adjust the mean value during inference, our experiments indicate that the network trained with a specific mean value $m$ achieves the best reconstruction quality when the inference mean value is also close to $m$. Using a mean value significantly higher than $m$, resulting in a large number of pixels being rendered as important pixels, does not improve reconstruction quality and may even lead to worse results. It would be highly beneficial to develop a method that enables the network to also learn an adaptive mean value, resulting in a dynamic percentage of important pixels in the C-Image that adjusts based on the underlying rendering results. This approach can also assist in quickly identifying the optimal ORP for a given reconstruction quality error tolerance. 

Various aspects of the pipeline need detailed investigation in the future. In this work, we only investigate the static fixed foveated rendering, it is worth investigating whether dynamic foveal rendering could impact reconstruction quality, rendering latency, and the pattern of the IM learned. IML Net is trained only using pixel-wise loss (MSE) in our work, other loss configurations, like structural similarity-based loss (SSIM) and perceptual loss, can be introduced to evaluate how the loss function affects the pattern of the learned IM. It is worth investigating which factors influence the distribution of important pixels and how many pixels are needed to render a specific frame. Factors such as image complexity, region size, and color dynamics may all contribute to determining which pixels are selected as important pixels. We also see the potential of utilizing the proposed IML Net to generate IMs with various levels of detail (LoD), enabling the construction of a multi-resolution representation to further optimize rendering performance. 



