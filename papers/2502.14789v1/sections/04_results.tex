\begin{figure}
\centering
\begin{tabular}{c@{}c@{}c@{}c@{}c}
\hspace{-0.3cm}
\rotatebox{90}{\hspace{0.2cm} \tiny{Input}} %
\includegraphics[width=0.19\linewidth]{figures/gt_pca/sedan/_DSC1567.jpg} &
\includegraphics[width=0.19\linewidth]{figures/gt_pca/sedan/_DSC1569.jpg} &
\includegraphics[width=0.19\linewidth]{figures/gt_pca/sedan/_DSC1570.jpg} &
\includegraphics[width=0.19\linewidth]{figures/gt_pca/sedan/_DSC1571.jpg} & 
\includegraphics[width=0.19\linewidth]{figures/gt_pca/sedan/_DSC1579.jpg} \\
\hspace{-0.3cm}
\rotatebox{90}{\hspace{0.2cm} \tiny{PCA}} %
\includegraphics[width=0.19\linewidth]{figures/gt_pca/sedan/_DSC1567_feat_pca.png} &
\includegraphics[width=0.19\linewidth]{figures/gt_pca/sedan/_DSC1569_feat_pca.png} &
\includegraphics[width=0.19\linewidth]{figures/gt_pca/sedan/_DSC1570_feat_pca.png} &
\includegraphics[width=0.19\linewidth]{figures/gt_pca/sedan/_DSC1571_feat_pca.png} & 
\includegraphics[width=0.19\linewidth]{figures/gt_pca/sedan/_DSC1579_feat_pca.png} \\
\hspace{-0.3cm}
\rotatebox{90}{\hspace{0.2cm} \tiny{Zoom}} %
\includegraphics[trim={2cm 2.2cm 2cm 0.4cm},clip, width=0.19\linewidth]{figures/gt_pca/sedan/_DSC1567_feat_pca.png} &
\includegraphics[trim={1.5cm 2.2cm 2.5cm 0.4cm}, clip,width=0.19\linewidth]{figures/gt_pca/sedan/_DSC1569_feat_pca.png} &
\includegraphics[trim={2.1cm 2cm 1.9cm 0.6cm}, clip, width=0.19\linewidth]{figures/gt_pca/sedan/_DSC1570_feat_pca.png} &
\includegraphics[trim={1.9cm 2.2cm 2.1cm 0.4cm}, clip, width=0.19\linewidth]{figures/gt_pca/sedan/_DSC1571_feat_pca.png} & 
\includegraphics[trim={1.5cm 1.7cm 2.5cm 0.9cm}, clip, width=0.19\linewidth]{figures/gt_pca/sedan/_DSC1579_feat_pca.png} \\
\end{tabular}
\vspace{-0.4cm}
\caption{PCA of DINOv2 features for ground-truth input views of the Sedan scene from real-world dataset of \cite{verbin2022refnerf}. We zoom in on the windshield, illustrating differences in corresponding locations between views.}
\label{fig:gt_pca}
\vspace{-0.3cm}
\end{figure}





\section{Experiments} \label{sec:results}

We evaluate our representation on multiple vectors. 
First, we demonstrate that input features from DINOv2 contain both view-dependent and view-independent information. 
Second, we consider the ability to segment objects in 3D space. Unlike previous methods, our disentangled feature fields capture both diffuse and reflective properties and allow for a better reconstruction of the semantic components in the scene. We also enable the segmentation of the reflective component of objects in isolation from different novel views.
Third, we consider the ability to remove view-dependent (reflective) components in the scene for specific semantic objects segmented using our approach. Fourth, we consider the ability to edit the scene, where one can manipulate or change only specific objects and their dependent (reflective) properties in isolation.  

\paragraph{Datasets.} We evaluate our method on synthetic scenes from the Shiny Blender \cite{verbin2022refnerf} dataset as well as real-world scenes from the RefNeRF real-world \cite{verbin2022refnerf} dataset and Mip-NeRF-360 dataset \cite{barron2022mipnerf360}. We consider a diverse set of scenes and objects from both real world and synthetic scenes, featuring multiple objects and variable lighting conditions, highlighting the generality of our approach. 
In particular, we consider 11 scenes and 25 objects from 3 real-world and synthetic datasets. This is comparable to recent work such as DFF or RefNeRF.  We consider the standard train-view/novel-view splits provided by the respective datasets and evaluate our model on such novel views. 
Additionally, our method supports incorporating arbitrary 2D semantic features extracted from models like CLIP-Lseg and DINOv2. 
The webpage provides associated videos depicting novel views and corresponding segmentation, removal, and editing results. We also provide implementation details in the appendix. 

\subsection{Superpixel Segmentation} \label{sec:segmentation}

Superpixels are perceptually meaningful, connected regions that group pixels based on similarities in color or other features, first introduced by \citet{renmalik2003}. Since then, various algorithmic approaches have been developed \citep{achanta2012, felzenszwalb2004}. Defining appropriate local regions is crucial for extracting spatial features in spectral-spatial models. While fixed-size windows (e.g., \citet{ertem2020}) have shown promising results, they constrain the ability to fully capture spatial context. In contrast, superpixels provide adaptive regions that enhance discriminative information, as demonstrated by \citet{fang2015kernals}. \citet{cui2018} further highlighted this by employing a superpixel-based random walker to refine an SVM probability map with significant success. Additionally, Cui et al. showed that superpixel spectra are more stable and less sensitive to noise than individual pixel spectra, making superpixel-based approaches more robust to image noise.


% The most common algorithm used in clustering based superpixel methods is Lloyd's algorithm \citep{lloyd2006}, a modified version of the popular k-means clustering algorithm. In the context of Lloyd's algorithm, let us first formalise the definition of a superpixel segmentation.

\begin{definition}[Superpixel Segmentation]  
Given an image \( I : \Lambda \to \mathbb{R}^d \), where \( \Lambda \subset \mathbb{Z}^2 \) represents the image domain, superpixel segmentation partitions \( \Lambda \) into a set of regions \(\{S_i\}_{i=1}^n\). Each superpixel \( S_i \) is defined as \( S_i = \{x \in \Lambda : f(x) = i\} \), where \( f: \Lambda \to \{1, \ldots, n\} \) is a labeling function that assigns each pixel \( x \) to one of the \( n \) superpixels based on a feature function.  
\end{definition}


We propose using the Felzenszwalb segmentation algorithm \citep{felzenszwalb2004} as an enhancement to superpixel-based methods, which predominantly rely on SLIC \citep{achanta2010slic} and its variants.

\begin{figure}[h]
    \centering
    \includegraphics[width=\linewidth]{method/images/felz_salinas.pdf}
    \caption{The Salinas HSI segmented using Felzenszwalb segmentation algorithm. \citep{felzenszwalb2004} The first figure shows a false-colored RGB image and other 3 shows the image segmented using a minimum size of 50, 100, 200 pixels respectively.}
    \label{fig:felz_salinas}
\end{figure}

% \begin{table}[h!]
%     \centering
%     \begin{tabular}{|c|c|c|c|c|}
%         \hline
%         Method & OA & AA & KA & No. \\
%         \hline
%         SLIC & 0.9969 & 0.9946 & 0.9966 & 2210 \\
%         \hline
%         Felzenszwalb & \textbf{0.9987} & \textbf{0.9955} & \textbf{0.9976} & 517 \\
%         \hline
%         Quickshift & 0.9978 & 0.9955 & 0.9976 & 274 \\
%         \hline
%     \end{tabular}
%     \caption{Segmentation performance if each superpixel was assigned by its most common class among its pixels, and then compared against ground truth labels. This test was performed on the SALINAS dataset. All parameters are set by default according to the skimage library \citep{skimage2014}, except for \texttt{n\_segments = 2000} for SLIC, and \texttt{min\_size = 100} for Felzenszwalb.}
%     \label{tab:my_label}
% \end{table}












\begin{table}[ht]
\centering

\scalebox{0.8}{ %
\begin{tabular}{lcccc|cc}
\toprule
\multirow{2}{*}{Scene (Objects)} & \multirow{2}{*}{Ours} & Ours & Ours & Ours & \multirow{2}{*}{DFF} & DFF+  \\
&  & implicit & total & optt & & Ref \\ 
\midrule
        Bicycle (Bench, Wheels) & \textbf{0.583} & 0.381 & 0.407 & 0.444 & 0.518 & 0.504  \\ 
        Counter (Mitten, Plant,
        & \multirow{2}{*}{\textbf{0.824}} & \multirow{2}{*}{0.705} & \multirow{2}{*}{0.641} & \multirow{2}{*}{0.664} & \multirow{2}{*}{0.713} & \multirow{2}{*}{0.629} \\ 
        Tray) \\
        Garden (Ball, Plant, 
        & \multirow{2}{*}{\textbf{0.840}} & \multirow{2}{*}{0.813} & \multirow{2}{*}{0.786} & \multirow{2}{*}{0.389} & \multirow{2}{*}{0.824} & \multirow{2}{*}{0.700} \\
        Tabletop) \\
        Kitchen (Lego) & \textbf{0.871} & 0.761 & 0.631 & 0.778 & 0.856 & 0.850 \\ 
        Gardenspheres (Cone, 
        & \multirow{2}{*}{\textbf{0.825}} &  \multirow{2}{*}{0.663} & \multirow{2}{*}{0.647} & \multirow{2}{*}{0.619} & \multirow{2}{*}{0.776} & \multirow{2}{*}{0.770} \\ 
        Head, Spheres) \\
        Sedan (Bonnet-top, Wind-
        & \multirow{2}{*}{\textbf{0.643}} &  \multirow{2}{*}{0.337} & \multirow{2}{*}{0.377} & \multirow{2}{*}{0.485} & \multirow{2}{*}{0.625} & \multirow{2}{*}{0.512} \\ 
        shield, Hubcaps, Wheel) \\
        Toycar (Body, Wheel) & \textbf{0.709} & 0.499 & 0.176 & 0.283 & 0.689 & 0.654 \\ 
        Teapot (Cover) & \textbf{0.762} & 0.125 & 0.654 & 0.768 & 0.529 & 0.399 \\ 
        \midrule
        Mean (Real World Scenes) & \textbf{0.757} & 0.628 & 0.594 & 0.523 & 0.691 & 0.582\\ 
        \midrule
        Car (Windshield, Wheels) & \textbf{0.799} & 0.437 & 0.402 & 0.088 & 0.523 & 0.445 \\
        Toaster (Body, Toasts) & 0.906 & 0.844 & 0.833 & 0.839 & \textbf{0.940} & 0.787 \\ 
        Helmet (Body, Windshield) & \textbf{0.863} & 0.795 & 0.705 & 0.894 & 0.731 & 0.472 \\ 
        \midrule
        Mean (Shiny Blender) & \textbf{0.856} & 0.568 & 0.550 & 0.648 & 0.731 & 0.527 \\ 
        \bottomrule
    \end{tabular}
}




    
\caption{Mean IoU for segmentation of objects from the  Shiny Blender synthetic dataset~\cite{verbin2022refnerf} (bottom) and real-world scenes (top). First four scenes are taken from the real world RefNeRF~\cite{verbin2022refnerf} dataset while the other four real world scenes are from the Mip-NeRF-360 dataset. 
On the RHS, we compare our approach DFF~\cite{kobayashi2022decomposing} and a baseline where DFF is optimized for features while RefNeRF is optimized for appearance (see \cref{sec:semantic_segmentation}). On the LHS, we also consider variants of our approach: (1). Ours-implicit: Our approach, but with implicit, instead of explicit, representation of physical properties [s.a. roughness], (2). Ours-total: Our approach, but using the total features [dependent and independent] for segmentation, (3). Our-optt: Our approach, where we optimized the color and features together. For each scene, we show, in brackets, the segmented objects. 
}
\label{tab:segmentation_iou}
\vspace{-0.3cm}
\end{table}




\begin{figure}[t]
\centering
\begin{tabular}{c@{}c@{}c@{}c@{}c}
\hspace{-0.3cm}
\rotatebox{90}{\hspace{0.1cm}\tiny{Novel View}}
\includegraphics[width=0.19\linewidth]{figures/segmentation/gardenspheres/ours/no_seg/0.png} &
\includegraphics[width=0.19\linewidth]{figures/segmentation/gardenspheres/ours/no_seg/2.png} &
\includegraphics[width=0.19\linewidth]{figures/segmentation/gardenspheres/ours/no_seg/4.png} &
\includegraphics[width=0.19\linewidth]{figures/segmentation/gardenspheres/ours/no_seg/6.png} & 
\includegraphics[width=0.19\linewidth]{figures/segmentation/gardenspheres/ours/no_seg/8.png} \\

\hspace{-0.3cm}
\rotatebox{90}{\hspace{0.5cm}\tiny{Sphere} }
\includegraphics[width=0.19\linewidth]{figures/segmentation/gardenspheres/ours/spheres/0.png} &
\includegraphics[width=0.19\linewidth]{figures/segmentation/gardenspheres/ours/spheres/2.png} &
\includegraphics[width=0.19\linewidth]{figures/segmentation/gardenspheres/ours/spheres/4.png} &
\includegraphics[width=0.19\linewidth]{figures/segmentation/gardenspheres/ours/spheres/6.png} & 
\includegraphics[width=0.19\linewidth]{figures/segmentation/gardenspheres/ours/spheres/8.png} \\

\hspace{-0.3cm}
\rotatebox{90}{\hspace{0.3cm}\tiny{Reflection}}
\includegraphics[width=0.19\linewidth]{figures/segmentation/gardenspheres/ours/highlight/0.png} &
\includegraphics[width=0.19\linewidth]{figures/segmentation/gardenspheres/ours/highlight/2.png} &
\includegraphics[width=0.19\linewidth]{figures/segmentation/gardenspheres/ours/highlight/4.png} &
\includegraphics[width=0.19\linewidth]{figures/segmentation/gardenspheres/ours/highlight/6.png} & 
\includegraphics[width=0.19\linewidth]{figures/segmentation/gardenspheres/ours/highlight/8.png} \\



\end{tabular}
\vspace{-0.3cm}
\caption{Segmentation of the spheres for novel views of the Garden-spheres real-world scene using either the full segmentation of the sphere (second row) or only the reflective component of the spheres (third row). }
\vspace{-0.3cm}
\label{fig:segmentation_combination_1}
\end{figure}















\subsection{Feature Analysis}
\label{sec:feature_analysis}




We first consider whether the distilled DINOv2 features capture both view-dependent and view-independent components. To this end, we visualize the PCA of the features for five ground-truth views of the Sedan scene from the real-world RefNeRF dataset. As seen in Fig.~\ref{fig:gt_pca}, while the features appear similar, there are notable differences, particularly in reflective regions such as the windshield (zoomed in). As further evidence, we note the recent work of \cite{el2024probing}, which demonstrates that features obtained from large foundation models, DINOv2 in particular, are not 3D view consistent. As such, applying our model has the advantage of disentangling view-dependent and view-independent components of a given feature view and can enhance downstream applications that require view-independent feature representations. This is illustrated in Sec.~\ref{sec:semantic_segmentation}, where we show that our view-independent feature field yields better 3D segmentation results than using the full (both dependent and independent) feature field or baselines.
Beyond 3D segmentation, one can render ground truth views only using the view-independent component, discarding their view-dependent component.  





\subsection{Semantic Segmentation}
\label{sec:semantic_segmentation}

\begin{figure}[bh!]
\centering
\begin{tabular}{c@{}c@{}c}

\hspace{-0.3cm}
\rotatebox{90}{\hspace{0.4cm}\tiny{Novel View}}
\includegraphics[width=0.32\linewidth]{figures/no_edit/sedan/rgb/13.png} &
\includegraphics[width=0.32\linewidth]{figures/no_edit/sedan/rgb/14.png} &
\includegraphics[width=0.32\linewidth]{figures/no_edit/sedan/rgb/15.png} \\


\hspace{-0.3cm}
\rotatebox{90}{\hspace{0cm}\tiny{Bonet-top+W-shield}}
\includegraphics[width=0.32\linewidth]{figures/seg_combination/sedan/full_hl/13.png} &
\includegraphics[width=0.32\linewidth]{figures/seg_combination/sedan/full_hl/14.png} &
\includegraphics[width=0.32\linewidth]{figures/seg_combination/sedan/full_hl/15.png} \\

\hspace{-0.3cm}
\rotatebox{90}{\hspace{0.5cm}\tiny{Bonet-top}}
\includegraphics[width=0.32\linewidth]{figures/seg_combination/sedan/bonet_top_hl/13.png} &
\includegraphics[width=0.32\linewidth]{figures/seg_combination/sedan/bonet_top_hl/14.png} &
\includegraphics[width=0.32\linewidth]{figures/seg_combination/sedan/bonet_top_hl/15.png} \\

\hspace{-0.3cm}
\rotatebox{90}{\hspace{0.5cm}\tiny{Windsheild}}
\includegraphics[width=0.32\linewidth]{figures/seg_combination/sedan/windshield_hl/13.png} &
\includegraphics[width=0.32\linewidth]{figures/seg_combination/sedan/windshield_hl/14.png} &
\includegraphics[width=0.32\linewidth]{figures/seg_combination/sedan/windshield_hl/15.png} \\

\end{tabular}
\caption{
Segmentation of the reflective component of different semantic components of the real-world car scene. The first row displays three novel views. We then demonstrate the segmentation of the reflective component of (1). Both the bonnet-top and the windshield (second row), (2). The bonnet-top (third row), and (3). The windshield (fourth row).}
\label{fig:segmentation_combination_2}
\vspace{-0.4cm}

\end{figure}








We consider our method's capability in segmenting both independent and view-dependent components. 



\subsubsection{Full Object Segmentation.}

In Fig.\ref{fig:segmentation}, we visualize our segmentation of three novel views for a real-world scene and a synthetic scene. The segmentation is obtained by clicking on each object and using our disentangled view-independent feature component, described in Sec.~\ref{sec:objectsegmentation}. We compare our method to Distilled Feature Field (DFF)~\cite{kobayashi2022decomposing}, the closest method to ours, which uses a single view-independent feature field. As RefNeRF is an improved appearance model, we also consider another baseline whereby appearance is obtained as in RefNeRF, using both view-dependent and view-independent feature fields, while semantics is obtained as in DFF, using a view-independent feature field. 
As can be seen, our method results in a superior segmentation and can successfully segment both reflective and non-reflective regions well, while the baseline is worse, particularly in reflective regions. Additional results are provided in the webpage. 


For numerical evaluation, we compare the segmentation of objects for both synthetic scenes and real-world scenes. We manually obtain ground-truth segmentations by first applying the Segment Anything Model~\cite{kirillov2023segment} and subsequently manually refining masks (see examples in the webpage). 
As can be seen in Tab.~\ref{tab:segmentation_iou}, our method results in better mean IoU scores. 


\subsubsection{Explicit modeling}
Our approach leverages explicit components to model the view-dependent and view-independent feature fields. Specifically, we model roughness and reflections explicitly, as detailed in Eq.~\ref{eq:omega_r} and Eq.~\ref{eq:reflectence_color} in Sec.~\ref{sec:method}. This enables novel applications that incorporate physical properties, such as object roughness editing. However, it is unclear whether explicit molding is preferable to implicit modeling of view-dependent and view-independent feature fields when one is interested in segmentation only. 
To evaluate the impact of explicit modeling, we considered a baseline that implicitly models view-dependent and independent feature fields by excluding the Reflection, IDE and roughness components (Eq.~\ref{eq:omega_r} and Eq.~\ref{eq:reflectence_color}). 
As seen in Tab.~\ref{tab:segmentation_iou}, this results in inferior performance. 




\subsubsection{View Dependent Segmentation.}                                                                              


We consider the ability to segment the view-dependent reflective surfaces of given objects. 
To this end, we begin by segmenting semantic objects using the view-independent features according to Sec.~\ref{sec:objectsegmentation} and subsequently select only a subset of points corresponding to features from the view-dependent (reflectance) feature field. 
Fig.~\ref{fig:segmentation_combination_1} illustrates for different novel views our success in segmentation for the Garden-spheres scene, for both the entire sphere as well as only the view-dependent reflection. 
Fig.~\ref{fig:segmentation_combination_2} illustrates the ability to segment the reflective region of the (1). bonnet-top and the windshield, (2). bonnet-top, and (3). windshield. 
As can be seen, only the reflective region of the desired semantic entity is depicted. 
Additional examples 
are provided in the webpage. 











\subsection{Removal}










\begin{figure}
\centering
\begin{tabular}{c@{~}c@{~}c@{~}c}


\rotatebox{90}{\hspace{0.1cm} \tiny{Novel View} }
\includegraphics[width=0.21\linewidth]{figures/segmentation/gardenspheres/ours/no_seg/0.png} &
\includegraphics[width=0.21\linewidth]{figures/segmentation/gardenspheres/ours/no_seg/2.png} &
\includegraphics[width=0.21\linewidth]{figures/segmentation/gardenspheres/ours/no_seg/4.png} &
\includegraphics[width=0.21\linewidth]{figures/segmentation/gardenspheres/ours/no_seg/6.png} \\


\rotatebox{90}{\hspace{0.4cm} \tiny{Full} }
\includegraphics[width=0.21\linewidth]{figures/edit_new/gardenspheres/spheres/color_full/0.png} &
\includegraphics[width=0.21\linewidth]{figures/edit_new/gardenspheres/spheres/color_full/2.png} &
\includegraphics[width=0.21\linewidth]{figures/edit_new/gardenspheres/spheres/color_full/4.png} &
\includegraphics[width=0.21\linewidth]{figures/edit_new/gardenspheres/spheres/color_full/6.png} \\

\rotatebox{90}{\hspace{0.0cm} \tiny{Independent} }
\includegraphics[width=0.21\linewidth]{figures/edit_new/gardenspheres/spheres/color/ours/0.png} &
\includegraphics[width=0.21\linewidth]{figures/edit_new/gardenspheres/spheres/color/ours/2.png} &
\includegraphics[width=0.21\linewidth]{figures/edit_new/gardenspheres/spheres/color/ours/4.png} &
\includegraphics[width=0.21\linewidth]{figures/edit_new/gardenspheres/spheres/color/ours/6.png}  \\








\end{tabular}



\begin{tabular}{c@{~}c@{~}c}


\rotatebox{90}{\hspace{0.7cm} \tiny{Novel View} }
\includegraphics[width=0.26\linewidth]{figures/segmentation/toaster/ours/non_seg/0.png} &
\includegraphics[width=0.26\linewidth]{figures/segmentation/toaster/ours/non_seg/20.png} &
\includegraphics[width=0.26\linewidth]{figures/segmentation/toaster/ours/non_seg/34.png} \\

\rotatebox{90}{\hspace{0.9cm} \tiny{Full} }
\includegraphics[width=0.26\linewidth]{figures/edit_new/toaster/color_full/0.png} &
\includegraphics[width=0.26\linewidth]{figures/edit_new/toaster/color_full/20.png} &
\includegraphics[width=0.26\linewidth]{figures/edit_new/toaster/color_full/34.png} \\

\rotatebox{90}{\hspace{0.9cm} \tiny{Independent} }
\includegraphics[width=0.26\linewidth]{figures/edit_new/toaster/color/0.png} &
\includegraphics[width=0.26\linewidth]{figures/edit_new/toaster/color/20.png} &
\includegraphics[width=0.26\linewidth]{figures/edit_new/toaster/color/34.png}  \\


\end{tabular}
\caption{
Editing the color of i). The spheres in the real-world gardenspheresand (ii). The toaster in the synthetic toaster, either (1). using all radiance fields, (\emph{full}) or (2). using the independent component only (\emph{indepdent}). }
\label{fig:edit_color_1}
\vspace{-0.7cm}
\hspace{0.3cm}
\end{figure}
















\begin{table*}[t]
    \caption{Median p-values for watermark detection using \textbf{UP} (Untargeted Training Data Paraphrasing), \textbf{TP} (Targeted Training Data Paraphrasing), and \textbf{WN} (Watermark Neutralization), compared against direct training (No Attack) and unwatermarked conditions (Unw.). \raisebox{0.5ex}{\colorbox{blue!30}{\quad}} indicates high watermark confidence, \raisebox{0.5ex}{\colorbox{blue!10}{\quad}} indicates low watermark confidence, and unshaded cells indicate insufficient evidence for watermark presence. Student model used in this table is Llama-7b, results for Llama-3.2-1b can be found in Appendix \ref{sec:remove_llama3_2}.}
    \centering
    \resizebox{0.80\textwidth}{!}{
    \begin{tabular}{cccccccc}
        \toprule
        \multicolumn{2}{c}{\textbf{Watermarking Scheme}} & \textbf{Token Num}. & \textbf{Unw.} & \textbf{No Attack} & \textbf{UP} & \textbf{TP} & \textbf{WN} \\
        \midrule
        \multirow{9}{*}{KGW}& \multirow{3}{*}{$n=1$} & 1k & 5.75e-01 & \cellcolor{blue!30}{8.97e-29} & \cellcolor{blue!10}6.82e-03 & 8.27e-01 & 8.20e-02 \\
        & & 2k & 5.71e-01 & \cellcolor{blue!30}6.49e-55 & \cellcolor{blue!10}2.43e-04 & 6.99e-01 & 2.72e-02 \\
        & & 3k & 6.01e-01 & \cellcolor{blue!30}2.68e-81 & \cellcolor{blue!30}9.68e-06 & 7.98e-01 & 1.21e-02 \\
        \cmidrule{2-8}
        & \multirow{3}{*}{$n=2$} & 10k & 4.80e-01 & \cellcolor{blue!30}4.12e-28 & \cellcolor{blue!10}2.18e-03 & 6.88e-01 & 9.85e-02 \\
        & & 20k & 4.47e-01 & \cellcolor{blue!30}4.12e-53 & \cellcolor{blue!10}1.29e-05 & 7.37e-01 & 3.35e-02 \\
        & & 30k & 3.62e-01 & \cellcolor{blue!30}8.26e-79 & \cellcolor{blue!30}1.05e-07 & 7.43e-01 & 1.24e-02 \\
        \cmidrule{2-8}
        & \multirow{3}{*}{$n=3$} & 100k & 3.40e-01 & \cellcolor{blue!10}1.85e-03 & 8.52e-01 & 4.30e-01 & 5.95e-01 \\
        & & 300k & 3.41e-01 & \cellcolor{blue!30}8.98e-09 & 9.51e-01 & 3.23e-01 & 6.80e-01 \\
        & & 1 million & 4.84e-01 & \cellcolor{blue!30}1.67e-23 & 8.63e-01 & 3.69e-01 & 8.63e-01 \\
        \midrule
        \multirow{9}{*}{SynthID-Text}& \multirow{3}{*}{$n=1$} & 1k & 9.67e-01 & \cellcolor{blue!10}1.46e-05 & 7.10e-01 & 9.98e-01 & 9.44e-01\\
        & & 2k & 9.95e-01 & \cellcolor{blue!30}1.08e-09 & 7.69e-01 & 9.96e-01 & 9.88e-01 \\
        & & 3k & 9.99e-01 & \cellcolor{blue!30}1.02e-13 & 8.05e-01 & 9.98e-01 & 9.97e-01 \\
        \cmidrule{2-8}
        & \multirow{3}{*}{$n=2$} & 10k & 4.23e-01 & \cellcolor{blue!30}6.67e-11 & 1.10e-01 & 4.97e-01 & 1.52e-01 \\
        & & 20k & 3.82e-01 & \cellcolor{blue!30}8.83e-20 & 4.29e-02 & 5.42e-01 & 7.30e-02 \\
        & & 30k & 3.09e-01 & \cellcolor{blue!30}1.65e-28 & 1.53e-02 & 4.45e-01 & 4.40e-02 \\
        \cmidrule{2-8}
        & \multirow{3}{*}{$n=3$} & 100k & 9.98e-01 & 5.28e-01 & 9.92e-01 & 9.94e-01 & 9.87e-01\\
        & & 300k & 9.76e-01 & 5.78e-01 & 9.99e-01 & 9.91e-01 & 9.49e-01\\
        & & 1 million & 9.87e-01 & 5.83e-01 & 9.99e-01 & 9.92e-01 & 9.85e-01\\
        \bottomrule
    \end{tabular}
    }
    \label{tab:removal}
\end{table*}







Several works in the literature~\cite{mirzaei2023spin, weder2023removing, wang2023inpaintnerf360} consider the problem of 3D ``inpainting" or 3D object ``removal", where one wishes to remove a 3D foreground object, resulting in a realistic-looking background scene. Given our structurally disentangled representation, we can now explore a novel capability of ``inpainting" or ``removing" the reflective part of an object. This can be achieved by selecting the 3D points corresponding to a given object (using a click) and then rendering for those points only the color component from the view-independent radiance field.

Fig.~\ref{fig:removal_1} shows the removal of the reflective component of the bonnet-top and windshield for the car scene and the Garden-spheres scene. Our method enables the removal of the reflected radiance from the object and the retention of its diffuse color.  %

\subsection{Editing}







\subsubsection{Color Editing.}

We consider the ability to edit the color of an object while adhering to its reflections. In Fig.~\ref{fig:edit_color_1}, we change the color of the segmented 3D points of (i). the spheres, for the real-world Garden-spheres scene, and (ii). the toaster body for the synthetic toaster scene. This color change occurs either (1). using both radiance fields (independent and reflection) or (2). using the independent component only, leaving the view-dependent reflective component unchanged. Using the latter results in a more natural manipulation that correctly adheres to reflections. Additional examples are shown in the webpage. 
To assess our color editing abilities numerically, we conducted a user study on colored objects. We consider 5 colors and 5 scenes (1 object each) and asked users to assess: 1. Color faithfulness (``how well does the desired color match the object?"), 2. Realism (``how realistic does object look?"), 3. Reflections match the unedited scene (``how well do the reflections match the unedited scene?"). We considered 25 users and obtained a MOS score (1-worse, 5-best) of \textbf{4.6}/\textbf{3.1}/\textbf{4.6} vs. 2.0/2.8/\textbf{4.6} in comparison to DFF for questions 1/2/3 respectively. 

\subsubsection{Roughness Editing.}


Next, we consider the ability to manipulate physical components introduced by our architectural design. In Eq.~\ref{eq:reflectence_color}, we utilize the roughness parameter $\kappa$ that controls the roughness. To this end, we consider the ability to manipulate the roughness of individual objects in the scene. We do so by segmenting the 3D points of an object and varying the roughness parameter $\kappa$ for those points. Fig.~\ref{fig:roughness_2} illustrates two examples: (i). the spheres in the real-world scene of the Garden-spheres scene, and (ii). the helmet case for the synthetic helmet scene.  
Additional examples are provided in the webpage. 







\subsubsection{Ablation Study.}


In Fig.~\ref{fig:indep_ablation}, we consider an ablation in which our segmentation is performed using not only the independent feature component $\mathbf{f}_{indep}$, but also the independent and dependent components together, $\mathbf{f}_{indep} + \mathbf{f}_{ref}$. As can be seen, this results in a worse segmentation. 

We also consider two additional ablations: (1) We optimized the appearance and feature model together, as opposed to first training the appearance model only and then the feature model (see Sec.~\ref{sec:training_objective}). For 3D segmentation (as in Tab~\ref{tab:segmentation_iou}), on average, it is worse, e.g., we get mIOU 0.648 (vs our 0.856) for Shiny Blender. (2). We also conducted an ablation where we removed the tone mapping function (Eq.~\ref{eq:learnable_weights_1}). We observe that appearance is slightly worse. Specifically, for the Gardenphere scene we obtain PSNR/LPIPS/SSIM of 28.8/0.180/0.809 vs our 28.9/0.180/0812. This results in a similar minor performance drop for 3D segmentation.



\subsubsection{Limitations.}

While our method is designed for segmenting and editing reflective regions of semantic objects in a scene, it cannot do so at the instance-based level.
For example, for the spheres in Fig.~\ref{fig:teaser}, selecting one of the spheres will result in capturing and editing both spheres. 
As in standard multiview reconstruction, errors can occur when the number of multiview ground truth features is not sufficient, or when they are noisy erroneous. 
We note that our task is highly unsupervised, as we aim to disentangle semantics and appearance in 3D space, given only 2D entangled appearance and semantics supervision. Improved physical modeling of, e.g., reflections and enhanced and generalizable feed-forward models may result in improved performance. 


