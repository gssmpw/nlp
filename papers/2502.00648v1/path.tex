\section{The Path Ahead}
\label{sec:path}

We have laid out an argument for using agency as the lens through which to evaluate the potential harms of generative AI tools to society. We
have also discussed the advancements needed both theoretically and methodologically to be able to use the concept of agency in this way.
Agent-based simulations will put these new models into motion, so that we can evaluate particular scenarios. The path ahead is not sequential;
several of these problems can and should be worked on in parallel.
At the same time, it is worth keeping in mind that
generative AI tools can also be enormously beneficial in multiple ways:

\begin{itemize}
    \item By enabling computing substrates with intuitive, seamless interactions, thus truly democratizing computing-based infrastructure.
    \item By providing information in an easy-to-understand and unbiased way, thus creating a more informed and educated population.
    \item By improving decision-making in complex systems, thus increasing efficiency and reducing waste.
\end{itemize}

Thus, perhaps the best application of the new theory, model, and simulations would be:

\smallskip
\noindent
\textbf{Evaluating better futures:} How can AI systems and humans act collaboratively to augment agency, instead of just adversarially? What new
questions does that pose for a theory of agency?

It is useful to take a step back at this stage and note that such a theory, if we are able to develop it, could also be used to think about
other problems as well. There are other, equally urgent, problems facing the world today, such as climate change, which require acting on a
global scale. Will these ideas extend to the collective agency needed for effective action at that scale?