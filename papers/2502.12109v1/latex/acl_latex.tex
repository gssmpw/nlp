% This must be in the first 5 lines to tell arXiv to use pdfLaTeX, which is strongly recommended.
\pdfoutput=1
% In particular, the hyperref package requires pdfLaTeX in order to break URLs across lines.

\documentclass[11pt]{article}

% Change "review" to "final" to generate the final (sometimes called camera-ready) version.
% Change to "preprint" to generate a non-anonymous version with page numbers.
\usepackage[preprint]{acl}

% Standard package includes
\usepackage{times}
\usepackage{latexsym}

% For proper rendering and hyphenation of words containing Latin characters (including in bib files)
\usepackage[T1]{fontenc}
% For Vietnamese characters
% \usepackage[T5]{fontenc}
% See https://www.latex-project.org/help/documentation/encguide.pdf for other character sets

% This assumes your files are encoded as UTF8
\usepackage[utf8]{inputenc}

% This is not strictly necessary, and may be commented out,
% but it will improve the layout of the manuscript,
% and will typically save some space.
\usepackage{microtype}

% This is also not strictly necessary, and may be commented out.
% However, it will improve the aesthetics of text in
% the typewriter font.
\usepackage{inconsolata}

%Including images in your LaTeX document requires adding
%additional package(s)
\usepackage{graphicx}


\usepackage{arydshln}
\usepackage{array}
\usepackage{amsmath}
\usepackage{amssymb}
\usepackage{booktabs}
\usepackage{colortbl}
\usepackage{etoolbox}
\usepackage{enumitem}
\usepackage{makecell}
\usepackage{multirow}
\usepackage{xspace}
\usepackage{xcolor}
\usepackage{graphicx}
\usepackage{spverbatim}
\usepackage{float}
\usepackage{tabularx}

% If the title and author information does not fit in the area allocated, uncomment the following
%
%\setlength\titlebox{<dim>}
%
% and set <dim> to something 5cm or larger.

%\title{Personality Structured Interview for Large Language Model Personality Simulation}
\title{Personality Structured Interview for Large Language Model Simulation in Personality Research}
% Author information can be set in various styles:
% For several authors from the same institution:
% \author{Author 1 \and ... \and Author n \\
%         Address line \\ ... \\ Address line}
% if the names do not fit well on one line use
%         Author 1 \\ {\bf Author 2} \\ ... \\ {\bf Author n} \\
% For authors from different institutions:
% \author{Author 1 \\ Address line \\  ... \\ Address line
%         \And  ... \And
%         Author n \\ Address line \\ ... \\ Address line}
% To start a separate ``row'' of authors use \AND, as in
% \author{Author 1 \\ Address line \\  ... \\ Address line
%         \AND
%         Author 2 \\ Address line \\ ... \\ Address line \And
%         Author 3 \\ Address line \\ ... \\ Address line}

% \author{First Author \\
%   Affiliation / Address line 1 \\
%   Affiliation / Address line 2 \\
%   Affiliation / Address line 3 \\
%   \texttt{email@domain} \\\And
%   Second Author \\
%   Affiliation / Address line 1 \\
%   Affiliation / Address line 2 \\
%   Affiliation / Address line 3 \\
%   \texttt{email@domain} \\}

\author{
    Pengda Wang$^{1}$, Huiqi Zou$^{3}$, Hanjie Chen$^{2}$, Tianjun Sun$^{1}$, Ziang Xiao$^{3}$$^{*}$, and Frederick L. Oswald$^{1}$$^{*}$ \\
    $^1$Department of Psychological Sciences, Rice University\\
    $^2$Department of Computer Science, Rice University\\ 
    $^3$Department of Computer Science, Johns Hopkins University \\
    \texttt{\{pw32,hc86,ts110,foswald\}@rice.edu;}
    \texttt{\{hzou11,ziang.xiao\}@jhu.edu} \\ 
}

%\author{
%  \textbf{First Author\textsuperscript{1}},
%  \textbf{Second Author\textsuperscript{1,2}},
%  \textbf{Third T. Author\textsuperscript{1}},
%  \textbf{Fourth Author\textsuperscript{1}},
%\\
%  \textbf{Fifth Author\textsuperscript{1,2}},
%  \textbf{Sixth Author\textsuperscript{1}},
%  \textbf{Seventh Author\textsuperscript{1}},
%  \textbf{Eighth Author \textsuperscript{1,2,3,4}},
%\\
%  \textbf{Ninth Author\textsuperscript{1}},
%  \textbf{Tenth Author\textsuperscript{1}},
%  \textbf{Eleventh E. Author\textsuperscript{1,2,3,4,5}},
%  \textbf{Twelfth Author\textsuperscript{1}},
%\\
%  \textbf{Thirteenth Author\textsuperscript{3}},
%  \textbf{Fourteenth F. Author\textsuperscript{2,4}},
%  \textbf{Fifteenth Author\textsuperscript{1}},
%  \textbf{Sixteenth Author\textsuperscript{1}},
%\\
%  \textbf{Seventeenth S. Author\textsuperscript{4,5}},
%  \textbf{Eighteenth Author\textsuperscript{3,4}},
%  \textbf{Nineteenth N. Author\textsuperscript{2,5}},
%  \textbf{Twentieth Author\textsuperscript{1}}
%\\
%\\
%  \textsuperscript{1}Affiliation 1,
%  \textsuperscript{2}Affiliation 2,
%  \textsuperscript{3}Affiliation 3,
%  \textsuperscript{4}Affiliation 4,
%  \textsuperscript{5}Affiliation 5
%\\
%  \small{
%    \textbf{Correspondence:} \href{mailto:email@domain}{email@domain}
%  }
%}

\begin{document}
% \maketitle
{\makeatletter\acl@finalcopytrue
  \maketitle
}
\begin{abstract}

Although psychometrics researchers have recently explored the use of large language models (LLMs) as proxies for human participants, LLMs often fail to generate heterogeneous data with human-like diversity, which diminishes their value in advancing social science research. 
To address these challenges, we explored the potential of the theory-informed Personality Structured Interview (PSI) as a tool for simulating human responses in personality research. 
In this approach, the simulation is grounded in nuanced real-human interview transcripts that target the personality construct of interest.
We have provided a growing set of 357 structured interview transcripts from a representative sample\footnote{Dataset and code will be shared later.}, each containing an individual's response to 32 open-ended questions carefully designed to gather theory-based personality evidence.
Additionally, grounded in psychometric research, we have summarized an evaluation framework to systematically validate LLM-generated psychometric data.
Results from three experiments demonstrate that well-designed structured interviews could improve human-like heterogeneity in LLM-simulated personality data and predict personality-related behavioral outcomes (i.e., organizational citizenship behaviors and counterproductive work behavior).
We further discuss the role of theory-informed structured interviews in LLM-based simulation and outline a general framework for designing structured interviews to simulate human-like data for psychometric research.  



\end{abstract}

\def\thefootnote{*}\makeatletter\def\Hy@Warning#1{}\makeatother\footnotetext{Co-corresponding authors.}

\def\thefootnote{\arabic{footnote}}



\section{Introduction}

% Personality is a fascinating topic that has been widely researched in psychology and consistently validated as a reliable indicator of various behavioral and psychological outcomes. 
% For instance, conscientiousness has been associated with job performance and academic success~\citep{barrick1991big}, while neuroticism strongly predicts mental health issues like anxiety and depression~\citep{john2008paradigm}. 
% Moreover, personality traits significantly impact interpersonal relationships, leadership effectiveness, and even physical health~\citep{judge2002personality, robins2002s}. 
% Given the predictive power of personality regarding behavioral and psychological outcomes, many studies on large language models (LLMs) aim to explore the personality traits exhibited by these models and investigate their potential to simulate personality (e.g., \citealt{lee2024llms}, \citealt{huang2023revisiting}, \citealt{ran2024capturing}).

% In addition to predicting and shaping behavior, AI's use as a simulation tool can also help explore problems and scenarios without the need to experiment on real systems. 
% \citet{messeri2024artificial} discuss the vision of ``AI as Surrogate,'' emphasizing the potential of AI tools as ideal research participants. 
% They argue that AI can rapidly answer hundreds of questions without experiencing fatigue and requires significantly less motivation to provide reliable answers compared to humans. 
% With proper training, generative AI tools (e.g., LLMs) may be able to represent a wide range of human experiences and perspectives. 
% Furthermore, simulations based on LLMs could be more cost-effective, as they can minimize or even eliminate the need for expensive human samples and circumvent common logistical barriers.




% What is personality research, and why is it important
% Why is it hard (esp. why do we need simulation data)? (try to avoid time/money as motivation; try to say hard-to-reach sample, complicated context, potentially unethical experiments,etc.)
% What are the current challenges with LLM-based simulation? (human-like diversity, eval)
% Describe our method
% - What is a structured interview? -> need to highlight what the structure is (derived from theory [you can consider those theories as priors, the subsequent behaviors are grounded on those theories])
% - Why structured interviews can address existing challenges?
% Describe our eval method
% - why and how 
% Describe our experiment design and results
% Summary with highlighted contribution



% What is personality research, and why is it important

% Personality is a fascinating topic that has been widely researched in psychology. 
% Well-developed measures of personality demonstrate empirical patterns of relationships with various behavioral and psychological outcomes that mirror the patterns that one would expect if personality was, in fact, being measured reliably~\citep{roberts2007power}. 
% For instance, measures of conscientiousness have been consistently and positively correlated with job performance and academic success~\citep{barrick1991big}, whereas neuroticism as a measured personality trait is strongly and positively correlated with mental health issues, such as anxiety and depression~\citep{john2008paradigm}. 
% Moreover, personality traits show statistically and practically significant correlations with interpersonal relationships, leadership effectiveness, and even physical health~\citep{judge2002personality, robins2002s}. 

%In Psychology, personality research aims to understand individual differences in important domains such as behavior, thought processes, and interpersonal relationships (e.g.,~\citealp{maher1994personality,roberts2007power,roberts2022personality}).

Personality research is an important field in psychology for understanding how individual traits shape significant life outcomes and trajectories, including career success, mental health, and overall well-being (e.g.,~\citealp{judge2002personality,roberts2007power,robins2002s}).
For instance, measures of conscientiousness have been consistently and positively correlated with job performance and academic success~\citep{barrick1991big}, whereas measures of neuroticism are found to be strongly and positively correlated with mental health issues, such as anxiety and depression~\citep{john2008paradigm}. 

%Overall, personality research is a critical field in Psychology for understanding how individual traits shape significant life outcomes and trajectories, including career success, mental health, and overall well-being (e.g.,~\citealp{judge2002personality, robins2002s}).




% Why is it hard (esp. why do we need simulation data)? (try to avoid time/money as motivation; try to say hard-to-reach sample, complicated context, potentially unethical experiments,etc.)

% However, studying these topics in social science often entails significant time and financial investments, as research frequently requires multiple rounds of data collection or additional validation across various contexts. 


However, studying personality requires understanding habits of thoughts, attitudes, and behaviors across diverse contexts and cultures, which is often a research challenge. 
For example, some research questions involve rare or extreme cases, making it difficult to gather sufficient samples (e.g.,~\citealp{lynam2001using}). 
Others require longitudinal designs to track personality development or dynamic social interactions (e.g.,~\citealp{damian2019sixteen,roberts2006patterns}). 
Additionally, certain experimental manipulations, like inducing stress or altering life circumstances, may be feasible or unethical for causal insights~\cite {fisher2021decoding}.
If a method could accurately simulate the human personality distributions, they could supplement and accelerate personality research by offering a scalable, cost-effective approach to supplement traditional data collection and experimentation~\citep{messeri2024artificial}.

% What are the current challenges with LLM-based simulation? (human-like diversity, eval)
Several prior studies have supported this possibility, showing that LLMs can generate responses for personality scales that reflect personality traits resembling those of humans (e.g., \citealt{lee2024llms}, \citealt{huang2023revisiting}). 
%However, other studies have emphasized that although current methods effectively replicate human personality patterns in terms of broad personality traits, they continue to face significant limitations in capturing individual differences at the item level (e.g., \citealt{wang2024not}).
However, studies found such methods face significant limitations in capturing individual differences at the item level (e.g., \citealt{wang2024not}).
Information at the item level provides a more granular understanding of how traits manifest in specific individuals and situations rather than relying solely on broad trait averages.
However, without additional empirical information, items confound these sources of variance.
Moreover, key psychometric challenges remain in modeling LLM responses, as they may not accurately reflect the intended personality construct.




% Describe our method
% - What is a structured interview? -> need to highlight what the structure is (derived from theory [you can consider those theories as priors, the subsequent behaviors are grounded on those theories])
% - Why structured interviews can address existing challenges?
% First, describe what we did? I felt there is something missing

To address this gap, we explored the potential of the Personality Structured Interview (PSI), which employs theory-informed questions to elicit information embedded in human narrative responses that are relevant to the target personality constructs of interest in LLM simulations.
We believe this sophisticated linguistic approach can inform and enhance the measurement of human-like traits in heterogeneous LLM-simulated personality data, potentially leading to a greater understanding of personality.

% These interview questions are carefully crafted based on established theories to capture the depth and complexity of personality construct information as they are naturally embedded in human narrative responses that LLMs will train themselves on.


% Therefore, this paper primarily focuses on enhancing the process of LLM simulation of population level distribution of personality traits. 
% To achieve this, we developed a personality structured interview (PSI) consisting of 32 questions.
% These interview questions are specifically theory-informed to capture the richness of personality traits, as embedded in human narrative responses that LLMs will train themselves on.
% This approach has been usefully compared and contrasted with other methods found in the recent research literature, such as relying on dialogue information~\citep{zhang2018personalizing}, using adjective-based dimensional categorization~\citep{serapio2023personality}, or transforming personality scale items into open-ended responses~\citep{wang2024incharacter, ran2024capturing}.


% Describe our eval method
% - why and how 
% Describe our experiment design and results

Grounded in psychometric research (e.g.,~\citealp{cronbach1951coefficient,cronbach1955construct}), we have summarized a framework for evaluating psychometric data that fully considers its hierarchical structure, in which observed responses (item level) are mapped to latent traits (domain level). 
This evaluation framework encompasses a range of analyses, from overall descriptive statistics to a more in-depth examination of psychometric performance.
%Through three experiments—comparing simulated data with actual human samples, evaluating different methods for simulating human personality distributions, and testing whether the PSI method simulation can demonstrate personality-related behavior—--we confirmed our hypothesis: the PSI method outperforms other approaches and can simulate personality-related behavior.

Based on our evaluation framework, we conducted three experiments. 
Those three experiments evaluated our method's effectiveness in replicating individual personality traits, simulating human-like personality distribution, and demonstrating personality-related behaviors with respect to established personality theory. 
We found our PSI method improves human-like diversity in personality traits simulation and recovers nuances in personality-related behaviors. 



\begin{figure*}[h]
    \centering
    \includegraphics[height=5cm,width=\linewidth]{latex/Figures/PSI_LLM.pdf}
    \caption{Overview of the Development of the Personality Structured Interview and Experimental Implementation in This Study \\
\textit{Note:} For an overview of the personality structured interview development process, refer to §\ref{personality_interview_question} and Appendix~\ref{sec:appendix PSI}. The structured interview development framework is detailed in Appendix~\ref{sec:appendix_psychometric}. Information on data collection and the human sample is provided in §\ref{experiment_setup} and Appendix~\ref{sec:appendix dataset}. The LLM simulation process is described in §\ref{experiment_setup} and Appendix~\ref{sec:appendix_prompt}. Details on the experimental setup and results of the three experiments can be found in §\ref{experiment_setup}, §\ref{results}, and Appendix~\ref{sec:appendix results}. For the psychometric data evaluation framework, see Appendix~\ref{sec:appendix evaluation_framework}.}
\label{tab:overview}
\end{figure*}


%The rest of the paper presents the PSI method, describes its development process, and demonstrates its effectiveness through three experiments (see Figure~\ref{tab:overview} for an overview of the PSI's development and experimental implementation in this study). 
%The PSI development framework can also be applied to other psychometric data simulations (see Appendix~\ref{sec:appendix_psychometric}).
%We also provide the PSI dataset with 357 structured interview transcripts, researchers can use it for various purposes, including (but not limited to) simulating population samples in social sciences research or exploring interactions and cooperation (behavior) among LLM-based agents with diverse personalities.
%The paper further emphasizes the importance of establishing scientific evaluation standards for LLM-simulated psychometric data from a psychological perspective and proposes an evaluation framework (see Appendix~\ref{sec:appendix evaluation_framework}).
%Additionally, it can serve as an introductory guide for researchers interested in obtaining a foundational understanding of personality psychology and incorporating it into their research.


In summary, our paper offers three contributions:
\begin{itemize}
    \setlength{\itemsep}{-4pt}
    \item A theory-informed LLM-based simulation method (PSI) for personality research and a development framework (see Appendix~\ref{sec:appendix_psychometric}). 
    \item A growing dataset, now with 357 structured interview transcripts. 
    \item An evaluation framework for LLM-simulated psychometric data that is grounded in psychological theories (see Appendix~\ref{sec:appendix evaluation_framework}).
\end{itemize}

We further discuss the potential of PSI methods in advancing research (see Appendix~\ref{sec:appendix_psychometric}). 
We believe the framework behind PSI could be generalized to other psychological constructs and other psychometric data to develop theory-informed simulation methods for human understanding. 




\section{Personality Research and Related Works}
Here we provide a concise overview of key personality concepts and discuss related works.
% The content is generally fine. I think the question your readers want to answer through this section is 
%1. What is personality research, why it is important (I think you covered)
%2. What kinds of personality research out there where your PSI method could help? (you don't need to say how PSI could help but you want to setup the stage where your readers can easily grasp why you are building PSI and what's the impact of PSI
% In NLP, what work has been done in the past and why they fall short, e.g., How PSI can address the gap. You also want to highlight that there is a lack of evaluation framework, so social scientists have concerns about using this method (interpretability is also another concern). 

% \citet{mackinnon1944structure} proposed that personality can be defined in two ways. 
% On the one hand, personality refers to the internal factors that explain an individual's consistent behavior across time, situations, and cultures, such as temperament and interpersonal strategies. 
% %These factors also drive social behavior. 
% On the other hand, personality encompasses a person’s general interpersonal characteristics perceived and described by others. Therefore, personality is also closely linked to one’s reputation. 
% The first definition emphasizes internal drives, whereas the second focuses on external behaviors and others’ perceptions. 
% Together, they highlight the significance of personality for individuals' habits of thoughts, feelings, and behaviors as they operate in social contexts when interacting with others~\citep{hogan1996personality}.

\citet{mackinnon1944structure} proposed two definitions of personality. 
One emphasizes internal factors like temperament and interpersonal strategies that drive consistent behavior across time, situations, and cultures. 
The other focuses on interpersonal characteristics as perceived by others, linking personality to reputation. 
The former highlights internal drives, while the latter centers on external behaviors and social perception. 
Together, they underscore personality's role in shaping thought patterns and behaviors in social interactions~\citep{hogan1996personality}.

Personality encodes rich and complex information in language and text~\citep{goldberg1990alternative, saucier2001lexical}. In fact, the Five-Factor Model (FFM) of personality is extensively researched (e.g.,~\citealp{costa2008revised,john1999big,mccrae1997personality}); it is directly based on the \textit{lexical hypothesis}, which posits that individual differences that are important in human interactions (e.g., have survival value across cultures) are often encoded in some or all languages of the world. 
The five factors are openness, conscientiousness, extraversion, agreeableness, and neuroticism (OCEAN).
From~\citeposs{galton1884measurement} early research to~\citeposs{allport1936trait} systematic organization and factor analysis of lexical terms, and through further developments by~\citet{norman1963toward} and~\citet{goldberg1990alternative}, the FFM theory has progressively evolved (see Appendix~\ref{sec:appendix personality_structure} for further details on the structure of personality). 
In other words, because language and text contain rich and complex information about personality, LLMs may capture and model such encoding by learning from vast amounts of training data. 

There are many studies to-date on LLMs that focus on personality, where researchers aim to benchmark their psychological profiles~\citep{lee2024llms, li2024quantifying, pellert2023ai}, investigate their ability to simulate personality~\citep{huang2023revisiting, jiang2023personallm, serapio2023personality}, and examine whether LLM agents align with character personality settings and how these settings influence their generated dialogues~\citep{shao2023character, wang2024incharacter, xu2024character}.

Currently, common methods for LLM simulation of human personality distributions have several limitations. 
The Persona-Chat dataset (Persona method) by~\citet{zhang2018personalizing} was originally designed to enhance the engagement of chit-chat models by increasing personalization. 
Therefore, its focus is more on enhancing personalization, rather than capturing personality traits. 
The adjective-based categorization method (Shape method) proposed by~\citet{serapio2023personality} effectively simulates profile-specific patterns of high/low standing on personality dimensions. 
However, this method presents a challenge in simulating human personality distributions, as restricting each case to a single dimension fails to capture the inherently multidimensional and hierarchical nature of personality data (e.g.,~\citealp{kachur2020assessing}).


% In contrast, real-world human personality responses typically follow a normal distribution~\citep{kachur2020assessing}.
% However, since human personality distributions are often assumed to be normal~\citep{kachur2020assessing}, this method of simulating samples will typically rely on the assumption of multivariate normal distributions. 
% However, real-world samples may exhibit more complex distribution patterns. 
% Moreover, as this approach relies on the assumed idealized relationship between adjectives and personality traits, it may lead to responses that appear overly idealized, compromising authenticity.


%a paragraph on what gap did we address?
%Considering LLMs' ability to perform in-context learning with textual statistical information, we believe that information extracted from the theory-informed structured interview can enable LLMs to simulate a more nuanced personality distribution. 

To address the aforementioned limitations of personality-simulated data, we explore the use of theory-informed PSI transcription to simulate personality data. 
% This method enables LLMs to generate a more nuanced personality distribution by simulating responses based on information that is theory-informed and construct-related.



% Currently, common methods for LLM simulation of human personality distributions include simulation based on descriptions containing demographic information and persona descriptions (e.g., \textit{``I sleep in late during the day. I listen to metal music.''}; the Persona-Chat dataset constructed by~\citealp{zhang2018personalizing}; Silicon Persona by~\citealp{petrov2024limited}); simulation based on the 70 bipolar adjectives proposed by~\citet{goldberg1992development} (e.g., \textit{``You are extremely friendly, extremely energetic, extremely assertive, extremely bold, and extremely active.''};~\citealp{serapio2023personality}); and simulation based on two hours in-depth interview transcriptions~\citep{park2024generative}. 

% However, these methods either fail to provide sufficient information for LLMs to simulate human personality distributions or are overly cumbersome. 
% Considering LLMs' ability to perform in-context learning based on textual statistical information, we believe that LLM responses produced from structured interview questions designed specifically to capture personality information will better simulate the human personality distributions (see more detail in \S\ref{personality_interview_question}).



% \subsection{The Structure of Personality}
% A key question in personality-related LLM research pertains to personality structure: What is the nature and breadth of the personality traits we want to simulate the human personality distributions?
% In the research literature, personality structures often emerge from applying factor analysis to individuals' responses to a large number of personality-relevant items. 
% This approach is what has been used to identify the five personaltiy factors in the FFM. 
% Moreover, personality is better understood in terms of one's continuous standing on each of multiple dimensions rather than as static types or profiles. 
% Research data clearly supports this view~\citep{wilmot2015contemporary, wilmot2019direct}. 
% Dividing individuals into limited categories (e.g., 16 types in MBTI) artificially segments continuous dimensions into discrete units, which may overlook important individual differences~\citep{ones2018Wiernik}. 

% Another question is whether we should incorporate personality traits and structures beyond the FFM to achieve a more comprehensive understanding of personality, given that additional personality variables and alternative structures have been proposed over the years. 
% It turns out that there can be conceptual overlap among these models (e.g.,~\citealp{hough2015beyond}). 

% For example, a meta-analysis by~\citet{joseph2010emotional} revealed that emotional intelligence (EI) shows statistically and practically significant relationships with neuroticism and extroversion within the FFM. 
% In fact, when controlling for personality variables, the unique contribution of EI almost disappears. 
% Similarly, \citet{crede2017much} conducted a meta-analysis on grit and found that its core components can largely be explained by conscientiousness, with little added predictive validity beyond that. 
% Moreover, although the HEXACO structure introduces an honesty-humility dimension, the remaining five dimensions align closely with the FFM structure~\citep{lee2004psychometric, lee2006further}. 
% Research by \citet{cutler2023deep} further indicates that nearly all personality semantic information can be classified within the FFM structure. 

% These studies collectively suggest that although further subdivision of personality structures might provide new perspectives, it often leads to conceptual redundancy and measurement complexity without necessarily enhancing predictive validity or theoretical value. 
% Rather than pursuing a wide range of personality frameworks, we will delve deeper into the FFM structure, which is very widely accepted by personality psychologists.



\section{Personality Structured Interview}
\label{personality_interview_question}

% As previously mentioned, the current methods for simulating human personality distributions have several limitations. 
% The Persona-Chat dataset (Persona method) by~\citet{zhang2018personalizing} was originally designed to enhance the engagement of chit-chat models by increasing personalization. 
% Therefore, its focus is more on enhancing personalization rather than capturing personality traits. 
% The adjective-based categorization method (Shape method) proposed by~\citet{serapio2023personality} effectively simulates specific high/low personality dimensions using adjectives. 
% However, using this method to simulate human personality distributions may lead to the assumption that personality traits are uniformly distributed.
% In contrast, real-world human personality responses typically follow a normal distribution~\citep{kachur2020assessing}.

% Considering LLMs' ability to perform in-context learning with textual statistical information, we believe that structured interview questions can enable LLMs to simulate a more nuanced personality distribution. 



% However, in-depth interviews may contain excessive information~\citep{park2024generative}, making it challenging to extract relevant data for simulation. 
% It would be ideal to collect data specifically tailored to the aspects we aim to model. 
% A theory-informed structured interview is one such method for ensuring that the extracted information reflects the target construct.

The theory-informed structured interview mentioned in this study differs from previous interview approaches used in LLM-based simulations, such as the Life Interview ~\citep{park2024generative}, which broadly inquires about various aspects of an individual’s life.
In contrast, the theory-informed structured interview is specifically designed to target the personality constructs that we aim to model.

% A theory-informed approach is essential not only for ensuring greater accuracy in data simulation but also for enhancing interpretability. 
% Given that the primary goal of the simulation is to generate data that closely align with actual data, we seek not only numerical similarity at the personality scale level but also theoretical coherence in personality-related behaviors expressed by the LLM. 
% By firmly grounding data generation by the LLM in theory, this approach ensures that the simulated data do not merely reproduce statistical properties but also reflect meaningful theoretical patterns, making them more interpretable in psychological and applied research contexts. 
% This interpretability aids in understanding data generation mechanisms, enabling transparent comparisons with human responses and preserving the constructs' psychological significance.

A theory-informed approach is essential not only for improving the accuracy of data simulation but also for enhancing interpretability. 
Our goal is to generate data that not only aligns with actual data at the personality scale level but also exhibits theoretically coherent personality-related behaviors. 
By grounding LLM-generated data in theory, we ensure that the simulation goes beyond reproducing statistical properties to capturing meaningful psychological patterns. 
This enhances interpretability, facilitates transparent comparisons with human responses, and preserves the psychological validity of the constructs.



%say why theory-informed is important, not just for accuracy but also for interpretability

% We believe that PSI can capture more personality related information from textual data, helping the model better learn how to display the corresponding personality traits. 
% By using information obtained from human samples, we can resample this data to simulate human samples without violating the distribution characteristics inherent in human samples.

%I think you should consider moving those paragraphs to later. Right now, the reader has no idea what PSI is, what kind of questions are included, or how it was being developed. Introduce what PSI first will help your readers understand why the framework is important and why you want to empathize that psi is different than others, 
% I think the general structure of Sec. 3 could be 
%what is the goal of building good simulation
%What PSI is and how PSI is designed to address each goal
%how PSI can be used for other simulation task (I think this could go the discussion)
%how psi compared to others, why our definition is different than other works in nlp.



Learning from psychometric theories, we summarized a framework for developing structured interviews, which can serve as a template for creating interviews designed to collect targeted information for simulating specific types of psychometric data (see Appendix~\ref{sec:appendix_psychometric}).
We present a PSI example designed to capture personality-related information, with relevant personality theories and question design detailed in §\ref{question_design}.


We want to further emphasize that the ``structured interview questions'' mentioned here differ from those described by~\citet{wang2024incharacter,ran2024capturing}, who refers to converting personality scale items into open-ended questions. 
The main issue with their approach is that, while it attempts to rewrite items as open-ended questions to elicit richer responses, it may fail to achieve the intended effect. 
For instance, in their example, \textit{``Values artistic, aesthetic experiences''} is rewritten as \textit{``Do you value artistic, aesthetic experiences?''}
Although this question appears open-ended, it tends to prompt a simple \textit{``yes''} or \textit{``no''} response and function as if it is close-ended, limiting the exploration of the respondent’s deeper thoughts or reasoning~\citep{trull1998structured}.

Moreover, such questions differ from the behavioral or situational questions typically found in structured interviews. 
Structured interviews usually involve more specific scenarios or tasks that encourage respondents to describe past experiences or hypothetical reactions~\citep{campion1997review}. 
For example, to explore a respondent's views on artistic and aesthetic experiences, a more effective question might be, \textit{``Please describe a moment when you felt deeply inspired by an artistic or aesthetic experience?''} 
This encourages detailed, contextual responses beyond a simple \textit{``yes''} or \textit{``no.''}




\subsection{PSI Questions Design}
\label{question_design}
How can we design questions more effectively and obtain textual information with greater depth in measuring personality?
McAdams' three levels of personality theory and the corresponding, theory-informed interviews offer a valuable direction (e.g.,~\citealp{mcadams1995we,mcadams1996personality,mcadams2001psychology}). 

According to McAdams’s theory, personality can be divided into three levels: traits (broad personality characteristics), personal concerns (goals, values, strategies), and narrative identity (the story individuals tell themselves about their lives). 
The latter two levels, in particular, reveal how individuals act in specific situations and developmental stages and how they organize and make sense of their life experiences.

The key to obtaining more meaningful textual information lies in guiding individuals toward higher levels of self-expression. Rather than simply recording mundane daily events, we should encourage reflection on pivotal life moments, significant relationships, and future aspirations. 
For example, effective questions could be: \textit{``Can you describe an event that changed the trajectory of your life?''} or \textit{``Tell me about a moment you are most proud of.''}
By designing questions like these, we can elicit deeper, more meaningful narratives, enriching the personality information.
% and increasing its value for research purposes.

We adapted and modified~\citet{mcadams1995we,mcadams1996personality,mcadams2001psychology} theory-based interview and narrative identity approach, incorporating elements from the structured interview of the FFM (SIFFM;~\citealp{trull1998structured}). 
The initial draft of the question pool was prepared and revised by subject matter experts (SMEs) through discussion. 
The SMEs included two doctoral students, one postdoctoral researcher, and one professor, all specializing in personality psychology.
Subsequently, we conducted pilot testing with six undergraduate research assistants from a personality research lab. 

% say more about who are the person contributed to the interview data, are they from a representative sample? where did you recruit them etc.
% you want a descriptive stats on how long each interview is. how long did each person spend
%also, you need to say how those interviews are being used. 
%and what other questions people answered.

As a result, the PSI, comprising 32 questions, was developed, as shown in Table~\ref{tab:structured_interview_questions} in Appendix~\ref{sec:appendix PSI}. 


\subsection{PSI Dataset}
\label{dataset_psi}

Using this set of questions, we designed a chatbot to conduct structured interviews, facilitating data collection (Institutional Review Board approval was obtained).
Because data collection is ongoing, we used responses collected through the end of December 2024. After excluding incomplete responses and those failing attention checks, the final sample consisted of 357 participants. On average, participants were in their early 30s (\textit{M} = 33.30 years, \textit{SD} = 13.06), with 52.40\% identifying as men, and 44.80\% as women.

To ensure a demographically diverse and broadly representative sample, we recruited participants from both undergraduate and working adult populations, ensuring a diverse range of backgrounds in terms of educational, employment, and life experience. 
Undergraduate participants were recruited from a large public university in the Midwest and received research credit for their participation. 
Working adult participants were sourced from two widely recognized, high-quality crowdsourcing platforms—Prolific\footnote{https://www.prolific.com/} and CloudResearch Connect\footnote{https://connect.cloudresearch.com/}—which are known for their diverse and demographically representative participant pools. 
All working adult participants were compensated for their time.
The average interview duration was 34 minutes.

In addition to completing the structured interview, participants provided demographic information and responded to both the personality scale and the personality-related behaviors scale (see Appendix~\ref{sec:appendix dataset} for detailed descriptions of data collection and examples of the dataset).
Responses to PSI were subsequently incorporated into the prompt to generate simulated data (see Appendix~\ref{sec:appendix_prompt} for the specific prompts used in the simulation process).

%if the evaluation framework is a contribution, you want to have a section to introduce it here. Basically moving some of the content in experimental setup to this section. 



\section{Experimental Setup}
\label{experiment_setup}
We designed three primary experiments to verify the effectiveness of the PSI method and its advantages over other methods: 
(1) Response Similarity (§\ref{response_similarity}): evaluating the degree of similarity between responses generated by LLMs based on the PSI method and actual corresponding human responses on personality scales; 
(2) Human Personality Distribution Simulation (§\ref{method_comparison}): comparing the PSI method with other methods in simulating human personality distributions; (3) Personality-Related Behavioral Performance (§\ref{related_behavior}): testing the models' performance using the PSI method to simulate personality-related behaviors.
General settings shared by all experiments are described in §\ref{general_settings}, while specific differences are detailed in their respective sections.

\subsection{General Settings}
\label{general_settings}

\paragraph{Psychological Scale}
We used Big Five Inventory-2 (BFI-2;~\citealp{soto2017next}): The BFI-2 is designed to capture three core facets of each of the FFM personality factors. 
Each facet is measured by two positively worded items and two negatively worded items, resulting in 60 items in total. 
Human respondents and LLMs were instructed to indicate the degree to which they agree with each item on a 5-point scale (1 = ``Strongly disagree'', 2 = ``Somewhat disagree'', 3 = ``Neither agree nor disagree'', 4 = ``Somewhat agree'', 5 = ``Strongly agree''). 
Specific scale items and scoring criteria can be found in Table~\ref{fig:bfi-2}, Table~\ref{fig:bfi-2_domain} and Table~\ref{fig:bfi-2_facet} in Appendix~\ref{sec:appendix BFI-2} .

\paragraph{LLMs}
We tested two LLMs, encompassing both closed-source and open-source models: GPT-4o (gpt-4o-2024-08-06)~\citep{gpt4o}, and Llama 3 (llama-3-70b-instruct)~\citep{llama3modelcard}. 
We include these models to allow for cross-validation and comparative analysis, ensuring that the findings are robust and not specific to a single model. 

GPT-4o and Llama 3 are among the current leading models, making them ideal for assessing the latest advancements in LLM performance. 
To ensure reproducibility, the temperature of the LLMs was set to zero to generate deterministic responses. 
% For the specific prompts used to generate simulated data, please refer to Appendix~\ref{sec:appendix_prompt}.



\subsection{Response Similarity}
\label{response_similarity}
The main purpose of this experiment is to assess the similarity between responses generated by LLMs based on the PSI method and human self-reports. 
% Following the structured interview, participants also provided self-reported personality ratings, which were used to conduct this initial assessment of LLM personality traits under the PSI method.

% \paragraph{Corresponding Human Respondents}
% \label{human_data_info}
% The human respondents in this experiment also served as the interviewees for the PSI method used in this paper. 

% Institutional Review Board approval was obtained.
% Because we are still actively collecting data, we used data collected up until the end of December 2024. 
% After excluding responses that failed attention checks or were incomplete, the final sample size was 357. 
% Among these, participants were on average in their early 30s (\textit{M} = 33.30 years, \textit{SD} = 13.06), 52.40\% identified as men, 44.80\% as women, and 2.80\% as other genders.

% The human respondents consisted of undergraduate students and working adults. 
% Undergraduate participants, recruited from a large public university in the Midwest, received research credit for their participation. 
% Working adult participants were recruited from two high-quality crowdsourcing platforms—Prolific\footnote{https://www.prolific.com/} and CloudResearch Connect\footnote{https://connect.cloudresearch.com/}—and were compensated for their participation.

\paragraph{Metrics}
Because the data generated by LLMs correspond on a one-to-one basis with data from human respondents, we used the Mean Absolute Error (MAE) and Pearson correlation coefficient (\( r \)) to examine the strength of the similarity between them. 
Smaller MAE and higher \( r \) indicate greater similarity.
For the \( r \) calculation formula, see Appendix~\ref{sec:appendix additional_setting}.
~\citet{park2024generative} employed a Life Interview simulation, which is currently the most information-rich method for personality simulation. 
It is also one of the few papers that provides a direct one-to-one performance comparison with human samples.
The reported MAE and \( r \) with corresponding human samples are likely the best values available so far. 
We used this as a standard for comparison.

If data from human respondents are regarded as the gold-standard criterion, this test can also be considered an examination of criterion-related validity, which assesses the correlation between a measure and an external standard~\citep{cronbach1955construct}.


\subsection{Human Personality Distribution Simulation}
\label{method_comparison}
The main purpose of this experiment is to investigate further the differences among various methods of simulating human personality distributions. 
We compared the PSI method, the Persona method—which relies on dialogue information~\citep{zhang2018personalizing}—and the Shape method, which uses adjective-based dimensional categorization~\citep{serapio2023personality}, to observe differences when simulating a normal human sample.
The specific descriptions of the Persona method and Shape method are shown in Appendix~\ref{sec:appendix additional_setting}.


\paragraph{Human Sample Criterion}
The human samples used for this experiment were collected as part of a broader project related to personality assessment through Prolific and received Institutional Review Board approval. 
Respondents were instructed to complete a set of demographic questions, the BFI-2, and a set of criterion measures. The respondents were compensated with \$3.75 for their participation. 
In total, 1,559 respondents provided valid responses. 
On average, participants were in their early 40s (\textit{M} = 42.29 years, \textit{SD} = 11.79), 50.80\% identifying as men, 49.20\% identifying as women.

\paragraph{Metrics}
To quantify the degree of similarity between human and LLM responses, we leveraged multiple metrics. 
At the domain and facet levels, we compared the mean (\textit{M}; sample mean of each domain and facet), standard deviation (\textit{SD}; sample standard deviation of each domain and facet), Cronbach's alpha, and correlations among scores. 
At the item level, we also compared the mean and standard deviation of item scores (sample mean and standard deviation of each item). 

Specifically, we also used MAE and \( r \) to quantify similarities in these metrics.
Here, however, MAE and \( r \) are based on sample-level analysis rather than the one-to-one correspondence seen in the previous experiment. 
Our focus is on whether, when simulating an overall human sample, similarities can be observed across various measurement levels—including items, facets, and domains. 

Additionally, separately for human and LLM responses, we fitted a three-factor confirmatory factor analysis (CFA;~\citealp{joreskog1969general}) model (TFM; where facets are ``factors'') to the responses to each domain of the BFI-2. 
Aside from the facet structure of each domain, we also used the facet scores as indicators and fitted the FFM. 
Model fit, standardized factor loadings, and latent correlations among the facets were also compared between human and LLM responses. 
Tucker’s congruence coefficient (TCC;~\citealp{tucker1951}) and MAE of factor loadings were used to quantify the similarity between factor solutions from human and LLM responses. 
For the CFA model, model fit information, and the TCC calculation formula, see Appendix~\ref{sec:appendix additional_setting}.
This evaluation framework is adaptable and can be used to analyze the fidelity of other simulated psychometric data (see Appendix~\ref{sec:appendix evaluation_framework} for further details).



\subsection{Personality-Related Behavioral Performance}
\label{related_behavior}
The primary aim of this experiment is to explore whether LLMs, when assigned specific personality settings, exhibit behaviors theoretically aligned with those personalities.
This is an exploratory study, as LLMs generate responses based on statistical probabilities derived from the training corpus~\citep{yang2024babbling}, making it uncertain whether assigning personality settings (through the PSI data) will influence the model to act consistently with that personality. 
Fortunately, the data we have collected through the PSI method includes LLM ratings of human behaviors, providing a valuable basis to test this hypothesis.

\paragraph{Personality-Related Behavior}
We collected self-report data on two classic types of workplace behaviors: organizational citizenship behavior (OCB) and counterproductive work behavior (CWB). 
Both types of behaviors are widely supported by research as being related to personality (e.g.,~\citealp{organ1995meta,berry2007interpersonal}).

These data were collected using the scale developed by~\citet{spector2010counterproductive} and prompted the LLM to respond to the same questions (for a detailed description of~\citeposs{spector2010counterproductive} measures, see Appendix~\ref{sec:appendix additional_setting}; for specific prompt details, refer to Appendix~\ref{sec:appendix_prompt}).

\paragraph{Metrics}
We primarily focus on the \( r \) between personality domains from different data sources and OCB, as well as CWB.
We anticipate that the \( r \) between self-reported personality domains and OCB/CWB in LLM simulation will closely align with those observed in the human participants.




\section{Experimental Results}
\label{results}

Here, we present the results of the three experiments mentioned above: Response Similarity (§\ref{response_similarity_result}), Human Personality Distribution Simulation (§\ref{method_comparison_result}), and Personality-Related Behavioral Performance (§~\ref{related_behavior_result}).



\subsection{Response Similarity Results}
\label{response_similarity_result}
Table~\ref{tab:mae_corr_1:1} provides a clear illustration of the similarity between the personality generated using the PSI method and the corresponding human self-reported personality data. 
The results show that regardless of the model used, the average correlation is around .5, suggesting a moderately strong positive relationship and a significant association between the two variables.
In particular, compared to the Life Interview method~\citep{park2024generative}, the PSI method consistently outperforms or matches the Life Interview approach across nearly all metrics, as indicated by its similar or lower MAE and similar or higher \( r \).

Notably, the Life Interview method relies on up to two hours of interview data, while the PSI method achieves comparable results using interview data consisting of just 32 questions, with an average duration of about 34 minutes.
This further underscores the advantage of the PSI method in effectively leveraging LLMs to simulate personality profiles.


\setlength\tabcolsep{8pt}
\begin{table*}[h]
\scriptsize
\centering
\begin{tabular}{lccc|ccc}
    \toprule
    \multirow{2}{*}{\textbf{Domain}} & \multicolumn{3}{c|}{\textbf{MAE}} & \multicolumn{3}{c}{\textbf{\( r \)}} \\
    
     \cmidrule(lr){2-4} \cmidrule(lr){5-7}
     & \textbf{PSI GPT-4o} & \textbf{PSI Llama3} & \textbf{Life Interview} & \textbf{PSI GPT-4o} & \textbf{PSI Llama3} & \textbf{Life Interview} \\
    \midrule
    Extraversion & \textbf{0.58} & 0.75 & 0.72 & \textbf{.64} & .43 & .45 \\
    Agreeableness & \textbf{0.53} & \textbf{0.56} & 0.60 & \textbf{.41} & \textbf{.36} & .35 \\
    Conscientiousness & \textbf{0.59} & 0.66 & 0.63 & .46 & .40 & .52 \\
    Neuroticism  & \textbf{0.63} & \textbf{0.59} & 0.75 & .63 & .57 & .68 \\
    Openness & 0.80 & 1.27 & 0.62 & \textbf{.43} & \textbf{.39} & .39 \\
    \bottomrule
\end{tabular}
\caption{MAE and \( r \) of Human Response and LLM Response Across PSI GPT-4o, PSI Llama3, and Life Interview (see~\citealp{park2024generative}, page 35 Table 3) for BFI-2 Each Personality Domain 
\\
\textit{Note:} \textit{n} = 357 for PSI method; \textit{n} = 1,052 for Life Interview method. The bold text highlights the aspects where the PSI method performs better than the Life Interview method.}
\label{tab:mae_corr_1:1}
\end{table*}




\subsection{Human Personality Distribution Simulation Results}
\label{method_comparison_result}

\paragraph{Descriptive Statistics}
We compared the means and standard deviations of the different methods at three levels: personality item, facet, and domain.
MAE and \( r \) for both the means and standard deviations are shown in Table~\ref{tab:method_comparison_ds}. 
The MAE for the means reflected the average difference between the LLM responses and the human responses at each level, whereas the MAE for the standard deviations revealed the difference in variability between the two datasets.
The \( r \) further illustrated the linear relationship between the two datasets, with values closer to one indicating a stronger correlation. 
Detailed means and standard deviations for human responses and LLM responses at the item, facet, and domain levels are provided in Tables~\ref{tab:item_mean},~\ref{tab:item_sd},~\ref{tab:facet_mean},~\ref{tab:facet_sd},~\ref{tab:domain_mean}, and~\ref{tab:domain_sd}  in Appendix~\ref{sec:appendix results.1}.

Table~\ref{tab:method_comparison_ds} shows that the PSI method demonstrates better performance in simulating human samples compared to the Persona and Shape methods at the item level. 
Although at the facet and domain level, some results from other methods slightly outperform PSI (e.g., Shape GPT-4o in \textit{M}), the process of aggregating scores from item to facet to domain levels reduces the impact of extreme values, smoothing them out at higher levels. 
Therefore, the PSI method’s advantage at the more granular item level becomes more important.

Additionally, PSI consistently demonstrates statistically significant positive correlations with the \textit{SD} of human samples across items, facets, and domains, whereas the other methods exhibit either negative or non-significant correlations. 
Simulating the \textit{SD} is a challenge for the other methods. 
Although the Shape method increases response variability and introduces more variance, it still struggles to replicate the true variance observed in human samples. 
The PSI method proposed in this paper addresses this issue to a certain extent.





\setlength\tabcolsep{8pt}
\begin{table*}[h]
\scriptsize
  \centering
    \begin{tabular}{llcccccccccccc}
    \toprule
    \multirow{3}{*}{\textbf{Model}} 
    & \multicolumn{4}{c}{\textbf{Item Level}} 
    & \multicolumn{4}{c}{\textbf{Facet Level}} 
    & \multicolumn{4}{c}{\textbf{Domain Level}} \\
    \cmidrule(lr){2-5} \cmidrule(lr){6-9} \cmidrule(lr){10-13} 
    & \multicolumn{2}{c}{\textbf{MAE}} & \multicolumn{2}{c}{\textbf{\( r \)}} 
    & \multicolumn{2}{c}{\textbf{MAE}} & \multicolumn{2}{c}{\textbf{\( r \)}} 
    & \multicolumn{2}{c}{\textbf{MAE}} & \multicolumn{2}{c}{\textbf{\( r \)}} \\
    \cmidrule(lr){2-3} \cmidrule(lr){4-5} \cmidrule(lr){6-7} \cmidrule(lr){8-9} \cmidrule(lr){10-11} \cmidrule(lr){12-13}
    & \textit{\textbf{M}} & \textit{\textbf{SD}} & \textit{\textbf{M}} & \textit{\textbf{SD}} 
    & \textit{\textbf{M}} & \textit{\textbf{SD}} & \textit{\textbf{M}} & \textit{\textbf{SD}}
    & \textit{\textbf{M}} & \textit{\textbf{SD}} & \textit{\textbf{M}} & \textit{\textbf{SD}} \\ 
    
    \rowcolor[rgb]{ .949,  .953,  .961} \multicolumn{13}{c}{Persona} \\
    
    GPT-4o & 0.48 & 0.44 & .50 & -.13 & 0.42 & 0.36 & .64 & -.24 & 0.42 & 0.30 & .69 & -.64 \\
    Llama3 & 0.51 & 0.26 & .57 & .02 & \textbf{0.33} & \textbf{0.15} & .80 & .31 & \textbf{0.31} & \textbf{0.06} & .80 & \textbf{.76} \\

    \rowcolor[rgb]{ .949,  .953,  .961} \multicolumn{13}{c}{Shape} \\

    GPT-4o & 0.58 & \textbf{0.16} & .38 & -.06 & 0.50 & 0.18 & \textbf{.82} & -.41 & 0.46 & 0.21 & \textbf{.84} & -.77 \\
    Llama3 & 0.62 & 0.25 & .45 & -.26 & 0.52 & 0.38 & .78 & -.64 & 0.45 & 0.31 & .82 & -.65 \\

    \rowcolor[rgb]{ .949,  .953,  .961} \multicolumn{13}{c}{PSI} \\

    GPT-4o & \textbf{0.48} & 0.34 & \textbf{.68} & \textbf{.27} & 0.40 & 0.22 & .71 & \textbf{.49} & 0.38 & 0.15 & .67 & .69 \\
    Llama3 & 0.66 & 0.36 & .50 & .14 & 0.54 & 0.27 & .57 & .19 & 0.49 & 0.21 & .54 & .51 \\
    
    \bottomrule
    \end{tabular}
  \caption{MAE and \( r \) for BFI-2 Human Responses and Different Methods LLM Responses at Item, Facet, and Domain Levels \\
  \textit{Note:} \textit{n} = 1,559 for human responses, \textit{n} = 297 for Shape Llama3, and \textit{n} = 300 for other LLM responses. Some sample sizes are below 300, because certain generated data exceeded reasonable thresholds (1-5) for specific items and were excluded from the analysis. The bold text indicates the best-performing data in that column.}
  \label{tab:method_comparison_ds}
\end{table*}




\paragraph{Psychometric Performance}
Psychometric performance includes multiple components; here, we primarily present model fit and structural validity (i.e., factor loadings). 
For other aspects, such as scale reliability and discriminant validity, please refer to Appendix~\ref{sec:appendix results.1}.


\textbf{Model Fit:}
For both the human sample criterion and different methods of LLM responses, the TFM was fitted to each BFI-2 domain, and the FFM was fitted to all the data. 
TFM fit information is shown in Table~\ref{tab:model_fit_tfm}, and FFM fit information can be seen in Table~\ref{tab:model_fit_ffm} in Appendix~\ref{sec:appendix results.1}.



From Table~\ref{tab:model_fit_tfm}, it can be observed that the model fit indices of the Persona method and the PSI method are relatively similar to those of the human sample, while the Shape method performs slightly worse. 
Table~\ref{tab:model_fit_ffm} reflects a similar trend, where the Shape method shows notable discrepancies compared to the human sample in the RMSEA and SRMR model fit indices (see Appendix~\ref{sec:appendix additional_setting} for the explanation of RMSEA, SRMR, and other model fit indices).


\textbf{Structural Validity:}
TCC was used to evaluate the similarity of factor loadings between human responses and LLM responses.
TCC results for the TFM of each BFI-2 domain are shown in Table~\ref{tab:tcc_tfm}, while results for the FFM are shown in Table~\ref{tab:tcc_ffm} in Appendix~\ref{sec:appendix results.1}.


The values of the TCC are generally high and thus supportive of factor-loading correspondence, with all methods showing strong alignment with the human sample (a TCC above .95 indicates good similarity, while a TCC of .85 to .94 suggests fair similarity;~\citealp{lorenzo2006tucker}). 
However, it is important to note that the TCC focuses on overall profile similarity, largely ignoring differences in absolute values (which might be too lenient). 
Therefore, we also need to examine the results of the specific standardized factor loadings.

Tables~\ref{tab:fc_tfm} and~\ref{tab:fc_ffm} in Appendix~\ref{sec:appendix results.1} present the standardized factor loadings.
From these tables, it is clear that the PSI method outperforms the Shape method in terms of similarity to human samples, with its factor loadings more closely matching human data. 
However, its results are relatively close to those of the Persona method, yet there are clearly differences in some factor loadings compared to the human sample.

It is important to note that the highest factor loadings do not necessarily indicate the best performance, as we are evaluating the similarity between the factor loadings of the human data and the LLM responses. 
Higher factor loadings suggest stronger correlations between the factor and the items, but this is not always ideal. 
It indicates that LLMs may struggle to make refined distinctions among items within a facet, meaning they may have difficulty determining which items are more or less related to the factor, unlike human respondents, who exhibit greater differentiation.

Tables~\ref{tab:ifc_tfm} and~\ref{tab:ifc_ffm} in Appendix~\ref{sec:appendix results.1} present the specific inter-factor correlations.
Aligning with our previous findings, the PSI method generally performs better than the other two methods when simulating human personality distributions, especially at a more fine-grained level. 
For instance, in the TFM results, other methods encountered anomalies, such as inter-factor correlations exceeding one (accompanied by a warning message when fitting the model), but the PSI method generally performed well.

The other results (e.g., scale reliability and discriminant validity) can be found in Appendix~\ref{sec:appendix results.1}. 
Overall, the results produced by the PSI method generally outperform those of the Persona method and the Shape method.
However, there is still a certain gap compared to the human sample.




\subsection{Personality-Related Behavioral Performance Results}
\label{related_behavior_result}

Table~\ref{tab:ocb_cwb_gpt4o} presents the correlations between the personality dimensions and OCB/CWB reported by PSI GPT-4o, with a comparison to human self-reported data. 
The relevant results for PSI Llama3 are provided in Table~\ref{tab:ocb_cwb_llama3} in Appendix~\ref{sec:appendix results.2}, while Figures~\ref{tab:ocb_cwb_all_gpt4o} and Figure~\ref{tab:ocb_cwb_all_llama3} display the complete correlation matrices for GPT-4o and Llama3, respectively.




% \setlength\tabcolsep{8pt} % Adjust column separation
% \begin{table}[h]
% \small
% \centering
% \begin{tabular}{lcc|cc}
%     \toprule
%     \multirow{2}{*}{\textbf{Domain}} 
%     & \multicolumn{2}{c|}{\textbf{OCB}} 
%     & \multicolumn{2}{c}{\textbf{CWB}} \\

%     \cmidrule(lr){2-3} 
%     \cmidrule(lr){4-5}
%     & \textbf{Human} & \textbf{PSI} 
%     & \textbf{Human} & \textbf{PSI} \\
%     \midrule
%     Ext  & .42  & .54  & -.01 & -.08 \\
%     Agr  & .18  & .36  & -.30 & -.54 \\
%     Con  & .12  & .45  & -.35 & -.47 \\
%     Neu  & -.21 & -.35 & .23  & .36 \\
%     Ope  & .17  & .02  & -.17 & .01 \\
%     \bottomrule
% \end{tabular}
% \caption{Comparison of OCB and CWB Correlations with Personality Domains: Human vs. PSI GPT-4o \\
% \textit{Note:} \textit{n} = 357 for human and PSI method.}
% \label{tab:ocb_cwb_gpt4o}
% \end{table}



The results demonstrate that the correlations simulated by LLMs between OCB/CWB and various personality dimensions closely resemble the patterns seen in human self-reported data, except for the openness domain.
When human reports show positive, negative, or no correlation, the LLM simulations generally exhibit consistent trends in the corresponding directions.

However, the correlations generated by LLM simulations are often higher, likely because the model primarily relies on the ``typical'' or ``idealized'' knowledge structures absorbed from its training corpus during simulation, consistently repeating and amplifying such associations. 
In contrast, human self-reported data contains more random noise, leading to relatively weaker correlations.


\section{Conclusion}
In summary, this study proposes a novel method for LLM to simulate human personality distributions---PSI and provides a comprehensive explanation of its development process and corresponding dataset. 

% Through three experiments, we verified the effectiveness of this method. 
% The results show that PSI performs comparably or even better in simulating personality compared to the outcomes derived from two-hour interviews in~\citep{park2024generative}, and when simulating human samples, PSI is generally better than the existing methods, both at a finer granularity (the item level) and in terms of overall psychometric indicators. 
% Finally, we explored whether the personality information set through PSI can be used to simulate personality-related behavior. 
% Although LLMs can achieve near-human standards to some extent, they still exhibit a certain degree of `idealized' simulation tendencies. 
Through three experiments, we validated PSI's effectiveness. 
Results show that PSI matches or even outperforms two-hour interviews~\citep{park2024generative} in personality simulation and surpasses existing methods in terms of both item-level accuracy and overall psychometric indicators. 
One experiment also examined PSI’s ability to simulate personality-related behavior, revealing that although LLMs approach human-like performance, they still exhibit idealized tendencies.


In summary, a theory-informed structured interview like PSI can better simulate human-like psychometric data. 
We explore the role of theory-informed structured interviews in simulation in greater detail and examine their potential to advance research, both of which are elaborated upon in Appendix~\ref{sec:appendix_psychometric}.



\section*{Limitations}
This study has several limitations.
First, we explored the potential of PSI in advancing research. 
Our single experiment demonstrates that theory-informed questions can effectively elicit information embedded in human narrative responses—--particularly those relevant to the target personality construct—--allowing LLMs to simulate the corresponding behavior. 
However, additional experiments are needed to investigate its broader applications.

Second, the personality assessment in this paper relies on self-reported personality scales, which can be viewed as a limitation of the study. 
However, these self-report scales are specifically designed for humans to target personality traits across individuals, in ways that cannot be directly located within LLMs themselves (see Appendix~\ref{sec:appendix_psychometric} for a detailed discussion on psychometrics). 
Furthermore, self-report scales are used often because respondents have privileged access to their own personality, and even with their well-known limitations (e.g., faking or social desirability in self-reports; see Appendix~\ref{sec:appendix results.3} for the social desirability analysis), they remain valuable to evaluate the personality traits exhibited by LLMs. 
However, because the primary goal of this study is to have the LLM simulate human responses and behaviors that correspond to extracted personality information, it is reasonable to assess LLMs using methods designed to measure human personality traits.

Moreover, this paper tests only two types of LLMs, which may somewhat limit its scope, given the array of LLMs that are now available. 
However, the conclusions drawn from our two models were highly consistent, and we have reason to believe that the advantages of the PSI method are generally applicable across different LLMs. 
Furthermore, we designed three different experiments and conducted comprehensive psychometric analyses of various methods within LLMs, ensuring greater rigor and reliability of the results.

\section*{Ethical Statement}
We hereby confirm that all authors of this study are aware of the provided ACL Code of Ethics and comply with the Code of Conduct.

\paragraph{Human Sample Data}
This paper discusses the comparison between multiple sets of human data and results generated by LLMs. 
The collection of human data strictly adhered to relevant ethical guidelines and received approval from the Institutional Review Board. 
To ensure fair treatment of participants and proper recognition of their contributions, reasonable compensation or course credit (for student participants) was provided.

Throughout the research process, we placed a strong emphasis on transparency and openness, also ensuring that all participants signed informed consent forms. 
Additionally, to safeguard data privacy and uphold ethical standards, the publicly available dataset underwent rigorous screening to exclude any personally identifiable information and was shared only with explicit public consent.


% \section*{Acknowledgments}






\bibliography{custom}

\clearpage
\newpage

\appendix

\section{Appendix: The Structure of Personality}
\label{sec:appendix personality_structure}

A key question in personality-related LLM research pertains to personality structure: What is the nature and breadth of the personality traits we want to simulate the human personality distributions?
In the research literature, personality structures often emerge from applying factor analysis to individuals' responses to a large number of personality-relevant items. 
This approach is what has been used to identify the five personaltiy factors in the FFM. 
Moreover, personality is better understood in terms of one's continuous standing on each of multiple dimensions rather than as static types or profiles. 
Research data clearly supports this view~\citep{wilmot2015contemporary, wilmot2019direct}. 
Dividing individuals into limited categories (e.g., 16 types in MBTI) artificially segments continuous dimensions into discrete units, which may overlook important individual differences~\citep{ones2018Wiernik}. 

Another question is whether we should incorporate personality traits and structures beyond the FFM to achieve a more comprehensive understanding of personality, given that additional personality variables and alternative structures have been proposed over the years. 
It turns out that there can be conceptual overlap among these models (e.g.,~\citealp{hough2015beyond}). 

For example, a meta-analysis by~\citet{joseph2010emotional} revealed that emotional intelligence (EI) shows statistically and practically significant relationships with neuroticism and extroversion within the FFM. 
In fact, when controlling for personality variables, the unique contribution of EI almost disappears. 
Similarly, \citet{crede2017much} conducted a meta-analysis on grit and found that its core components can largely be explained by conscientiousness, with little added predictive validity beyond that. 
Moreover, although the HEXACO structure introduces an honesty-humility dimension, the remaining five dimensions align closely with the FFM structure~\citep{lee2004psychometric, lee2006further}. 
Research by \citet{cutler2023deep} further indicates that nearly all personality semantic information can be classified within the FFM structure. 

These studies collectively suggest that although further subdivision of personality structures might provide new perspectives, it often leads to conceptual redundancy and measurement complexity without necessarily enhancing predictive validity or theoretical value. 
Rather than pursuing a wide range of personality frameworks, we will delve deeper into the FFM structure, which is very widely accepted by personality psychologists.


\clearpage
\newpage

\section{Appendix: Personality Structured Interview Questions}
\label{sec:appendix PSI}

Table~\ref{tab:structured_interview_questions} presents the final set of 32 questions that form the basis of our personality structured interview. 
This framework was developed by adapting and modifying McAdams's life history interview and narrative identity approach~\citep{mcadams1995we,mcadams1996personality,mcadams2001psychology}, while also incorporating components from the Structured Interview of the Five-Factor Model (SIFFM;~\citealp{trull1998structured}).

The initial draft of the question pool was created and refined through collaborative discussions among subject matter experts (SMEs). 
The SME team consisted of two doctoral students, one postdoctoral researcher, and one professor, all specializing in personality psychology.
To ensure the quality and clarity of the questions, pilot testing was conducted with six undergraduate research assistants from a personality research lab.
This iterative process of development and feedback led to the construction of the personality structured interview. 

The development process is similar to the development of psychological tests or scales (psychometric).
We have provided more details on psychometrics and the development framework in Appendix~\ref{sec:appendix_psychometric}.


\begin{table}[h]
\centering
\scriptsize 
\begin{tabular}{@{}p{0.5cm}p{15cm}@{}} 
\# & \textbf{Questions} \\
1 & To get us started, where are you from? Where did you grow up and what was the place like? \\
2 & Thinking back, what kind of student were you in school? \\
3 & Did you have a teacher or teachers that were influential? If so, why? What were they like? \\
4 & What was your favorite subject in school, and why? \\
5 & What was your least favorite subject in school, and why? \\
6 & Still thinking back, who were your heroes when you were young and why? \\
7 & When you were little, what did you want to be when you grew up? And why? \\
8 & What were your dreams and plans when you graduated from high school? What made you have those dreams or plans? \\
9 & If you had complete freedom, what would your dream job be, and why? \\
10 & How have your dreams and goals changed throughout your life? \\
11 & Shifting gears to your childhood, how would you describe the personalities of people in the family you grew up in? For example, what were your parents and/or siblings like? \\
12 & How are you similar or different from your parents and/or siblings? \\
13 & How do you think your similarities and/or differences influenced your relationship with them? \\
14 & What was the best part of your childhood? \\
15 & What do you think were the worst parts of your childhood? \\
16 & Switching gears a little bit, what was your first paid job? How old were you then? (If this is not applicable to you, then please put `NA') \\
17 & What other jobs have you had? (If this is not applicable to you, then please put `NA') \\
18 & What do you do now for a living? And why did you choose it? \\
19 & Please describe your typical work day. \\
20 & What is the best and worst part of your current work? \\
21 & Did you serve in the military? Please tell us about that experience, what was the best and worst part of it? \\
22 & Moving on, what are your adult friendships like? \\
23 & How are your adult friendships different from your childhood friendships? \\
24 & What are your strongest qualities as a friend? In other words, what makes you a great friend to have? \\
25 & What about your weakest qualities in friendships? In other words, what do you struggle with when you are trying to be a friend to someone? \\
26 & Moving onto more general questions, when thinking about your life in general, what are you most proud of? \\
27 & What hobbies or other interests do you have? \\
28 & What things frighten you now? \\
29 & What were some things that frightened you most as a child? \\
30 & What are the three biggest news events that have occurred in your lifetime? \\
31 & If you had the power to solve one and only one problem in the world, what would it be, and why? \\
32 & Tell me about a time when you did not know if you would make it. How did you overcome that challenge? \\
\end{tabular}
\caption{Structured Interview Questions}
\label{tab:structured_interview_questions}
\end{table}


\clearpage
\newpage

\section{Appendix: Psychometrics and Structured Interview Development Framework}
\label{sec:appendix_psychometric}

Psychometrics is a field of psychology dedicated to the theory and practice of psychological measurement. 
It primarily focuses on quantifying psychological traits, behaviors, and abilities through systematic testing and analysis. 
Psychological traits, such as personality dimensions, cognitive abilities, and emotional states, are inherently abstract constructs that cannot be measured directly. 
Therefore, psychometricians rely on tools like surveys, questionnaires, scales, or structured interviews to infer these traits through observable indicators or responses.

\paragraph{Measuring Psychological Traits}
To measure a psychological trait, psychometricians typically operationalize the trait by identifying observable behaviors or self-identities that correlate with the underlying construct. 
For example, extraversion can be measured by assessing behaviors such as sociability, assertiveness, and enjoyment of social interactions, or by examining identities such as seeing oneself as an outgoing person and believing that one thrives in social situations.

These behaviors/identities are translated into measurable items (for scale; e.g., \textit{``I enjoy being the center of attention''}) or questions (for interview; e.g., \textit{``What are your strongest qualities as a friend? In other words, what makes you a great friend to have?''}).
The challenge lies in ensuring that these items/questions accurately and consistently capture the construct across different populations and contexts.

\paragraph{Theory-Informed Structured Interview for LLM Data Simulation}
Theory-informed structured interviews are the most suitable method for enabling LLMs to simulate psychometric data. 
These interviews are specifically designed to capture the constructs underlying the targeted psychometric measures, ensuring that the simulated data aligns with the intended psychological construct.
By extracting textual information that directly reflects the target construct, theory-informed structured interviews facilitate the representation of heterogeneous data while preserving a high degree of human diversity, thereby enhancing the validity and applicability of the simulated psychometric data.

Moreover, since the information is extracted based on theoretical foundations, it also provides a certain level of interpretability for the LLM’s simulation. 
This not only allows the generated data to be compared with theoretical expectations but also increases its potential for practical applications.

\paragraph{The Potential for Advancing Research} 
A theory-informed structured interview transcript-based simulation can generate data more effectively by focusing on the target construct. 
Ideally, the simulated data should reproduce the same constructs reflected in real-world data and simulate behaviors associated with these constructs.

Take personality as an example---if simulated data can accurately replicate real-world personality constructs, it enables research that would be difficult to conduct in reality, such as developing contextualized personality assessment tools and exploring new personality theories through multi-agent simulations.

\textbf{Developing Contextualized Personality Assessment Tools:}
Traditional personality assessments mainly rely on standardized questionnaires or laboratory tasks, which often fail to adequately simulate real-world social contexts. 
By using theory-informed structured interview transcript-based simulations, we can generate more fine-grained and context-sensitive individual response data. 
For instance, we can simulate various occupational scenarios (such as crisis management, teamwork, or remote work) and analyze how different personality traits manifest in these contexts. 
This approach not only aids in developing measurement tools tailored to specific applications but also enhances ecological validity, allowing for more accurate assessments of personality across different situations.

\textbf{Exploring New Personality Theories through Multi-Agent Simulation:}
If simulated data can accurately reflect real-world personality constructs, we can leverage multi-agent interactive systems to simulate individuals' behavioral patterns and observe how different personality traits evolve in group dynamics. 
For example, virtual agents with distinct personality traits can be placed in cooperative tasks, competitive environments, or social interactions, enabling researchers to test whether existing personality theories effectively predict these interaction patterns. 
Additionally, this approach can uncover new personality dynamics, such as whether certain personality trait combinations produce unexpected group effects or whether behavior in specific situations deviates from traditional theoretical predictions.

However, these assumptions are based on an ideal premise—that simulated data can successfully reflect the same constructs as real-world data. 
A theory-informed structured interview undoubtedly offers a promising pathway in this regard, warranting further in-depth exploration.


\paragraph{Structured Interview Development Framework}
Developing a structured interview for LLM to simulate data involves a series of carefully designed steps to ensure that the resulting test reliably and validly measures the construct of interest. 
Table~\ref{scale} outlines the framework.


\begin{table*}[ht]
\centering
\begin{tabularx}{\textwidth}{@{}p{0.5cm}X@{}} % Utilize full page width with tabularx
\# & \textbf{Steps} \\
1 & Identify behaviors/perceptions that represent the construct or define the domain. \\
2 & Prepare a set of structured interview specifications---structured interview blueprint. \\
3 & Build an initial question pool. \\
4 & Have questions reviewed by substantive experts (and revised as necessary). \\
5 & Hold preliminary question tryouts. \\
6 & Determine statistical properties of questions (and eliminate poor questions or revise as necessary). \\
7 & Field-test the structured interview on a large representative sample of the intended examinee population. \\
8 & Design and conduct reliability and validity studies for the final form of the structured interview. \\
9 & Develop guidelines for administration, and interpretation of the structured interview. \\
% 10 & Develop guidelines for administration, scoring, and interpretation of the test scores. \\
\end{tabularx}
\caption{Structured Interview Development Framework}
\label{scale}
\end{table*}

% \textbf{Identify the primary purpose(s) for which the structured interview will be used: }
% The first step in developing a structured interview is to clearly define its primary objective and intended use. 
% In other words, if we use the structured interview to simulate the data, what the data will be used for?
% This foundational decision shapes all aspects of the interview’s design, including the selection of questions, evaluation criteria, and validation approach. 
% For instance, a structured interview aimed at assessing job-related conscientiousness for employee selection would prioritize predictive validity, whereas one designed for personality research may focus more on construct validity.

\textbf{Identify behaviors/perceptions that represent the construct or define the domain:} Clearly defining the construct is essential to developing relevant structured interview questions. 
The construct should be operationalized by identifying specific behaviors or attributes that indicate its presence.
In other words, you need to find a theory to guide you on how to measure the target constructs.
This step may involve reviewing literature, conducting expert interviews, or organizing focus groups to understand the various dimensions and observable characteristics of the construct.

\textbf{Prepare a set of structured interview specifications---structured interview blueprint:} A blueprint outlines the structured interview’s structure and content, specifying how questions will be distributed across the construct’s dimensions or components. 
It typically includes information on the number of questions per domain, questions content.



\textbf{Build an initial question pool:} In this step, an extensive list of questions is created to cover the full range of the construct. 
Question wording should be clear, concise, and relevant to the target population. 
It is common practice to generate more questions than needed to ensure that poorly performing questions can be removed later without compromising the structured interview.

\textbf{Have questions reviewed by substantive experts (and revised as necessary):} SMEs review the question pool for content accuracy, relevance, clarity, and bias. 
Experts assess whether the questions align with the construct’s definition and whether any important aspects are missing.
Feedback from experts helps refine the wording, remove ambiguous questions, and identify questions with potential cultural or gender biases.


\textbf{Hold preliminary question tryouts:} Before large-scale testing, questions are piloted on a small group of individuals representative of the target population. 
This stage helps identify any immediate issues with question comprehension, response format, or instructions. 
It can also include cognitive interviews where participants are asked to explain their thought processes when answering questions.
The feedback from this stage informs further revisions, ensuring that questions are clear and easy to understand.

\textbf{Determine statistical properties of questions (and eliminate poor questions or revise as necessary):} Question performance is assessed through statistical analyses to evaluate difficulty, discrimination, and internal consistency.
Standardized scoring criteria, such as Behaviorally Anchored Rating Scales (BARS), can be used for obtaining question scores.
Then, methods like correlations, factor analysis, and item response theory (IRT) can help determine how effectively each question measures the intended construct.

Questions that demonstrate poor psychometric properties---such as low discrimination or high measurement error---are either revised or removed. 
For example, questions with low correlations with the overall score or incorrect factor loadings may be eliminated.

\textbf{Field-test the structured interview on a large representative sample of the intended examinee population:} The revised item set is administered to a large, representative sample to gather comprehensive data on the scale’s performance. 
This step ensures that the sample reflects the population for which the test is intended, which is critical for generalizability.
Statistical analyses are conducted to refine the test further. 
This process may involve removing redundant items, assessing dimensionality, and ensuring that items work well across demographic subgroups.


\textbf{Design and conduct reliability and validity studies for the final form of the structured interview:} To ensure the structured interview is psychometrically sound, various reliability and validity studies are conducted:
Reliability studies measure the structured interview’s consistency and stability, including internal consistency (e.g., Cronbach's alpha), test-retest reliability, and inter-rater reliability (if applicable).
Validity studies assess whether the structured interview measures what it is intended to measure. 
This includes: Content validity (the extent to which items cover the construct); Construct validity (e.g., convergent and discriminant validity); Criterion-related validity (e.g., predictive or concurrent validity).
These studies provide evidence that the structured interview is both reliable and valid for its intended purpose.

\textbf{Develop guidelines for administration and interpretation of the structured interview:} The final step involves creating a comprehensive, structured interview manual that includes instructions for structured interview administration, scoring procedures, and guidelines for interpreting results. 
This manual ensures consistency in the structured interview's use across different settings and helps minimize errors in administration and scoring.
Guidelines for interpreting scores may include norms, cutoff points, and descriptions of what various score ranges indicate.


\clearpage
\newpage

\section{Appendix: Personality Structured Interview Dataset and Data Collection}
\label{sec:appendix dataset}

Institutional Review Board approval was obtained.
The data was collected through an online questionnaire, followed by an online structured interview.

% (see Figure~\ref{tab:example} for the example of the chat-bot interview).

Since we are still actively collecting data, we will share a part of the dataset that does not include any personal privacy information and has received permission for data sharing.
Each example is composed of the following characteristics: 

\begin{enumerate}
\item \textbf{Gender:} The gender of the participant.
\item \textbf{Race:} The racial background of the participant.
\item \textbf{English:} Whether English is the participant’s first language.
\item \textbf{Age:} The participant’s age.
\item \textbf{Weight:} The participant’s weight.
\item \textbf{Height}: The participant’s height.
\item \textbf{OCB1–OCB10:} Self-reported Organizational Citizenship Behavior data, measured using the scale from ~\citet{spector2010counterproductive}, and see Table~\ref{fig:OCB} for specific item details.
\item \textbf{CWB1–CWB10:} Self-reported Counterproductive Work Behavior data, measured using the scale from ~\citet{spector2010counterproductive}, and see Table~\ref{fig:CWB} for specific item details.
\item \textbf{Q1–Q32:} Participant responses to each personality structured interview questions, see Table~\ref{tab:structured_interview_questions} for specific question details.
\item \textbf{Item1–Item60:} Responses to each item of the BFI-2. Refer to Appendix~\ref{sec:appendix BFI-2} for item descriptions and scoring guidelines.
\end{enumerate}


% \begin{figure*}[h]
%     \centering
%     \includegraphics[width=\linewidth]{latex/Figures/example.pdf}
%     \caption{Example of Online Chat-bot Interview}
% \label{tab:example}
% \end{figure*}



\clearpage
\newpage

\section{Appendix: BFI-2 Scale}
\label{sec:appendix BFI-2}

\textbf{Instructions:} Here are a number of characteristics that may or may not apply to you. For example, do you agree that you are someone who likes to spend time with others? Please write a number next to each statement to indicate the extent to which you agree or disagree with that statement.

\textbf{Scales:}

\begin{table}[h]
\centering
\scriptsize % Further reduce table font size for compactness
\begin{tabular}{@{}p{0.5cm}p{10.5cm}@{}} % Adjusted column width to fit page
\# & \textbf{Statement} \\
1 & I am someone who is outgoing, sociable. \\
2 & I am someone who is compassionate, has a soft heart. \\
3 & I am someone who tends to be disorganized. \\
4 & I am someone who is relaxed, handles stress well. \\
5 & I am someone who has few artistic interests. \\
6 & I am someone who has an assertive personality. \\
7 & I am someone who is respectful, treats others with respect. \\
8 & I am someone who tends to be lazy. \\
9 & I am someone who stays optimistic after experiencing a setback. \\
10 & I am someone who is curious about many different things. \\
11 & I am someone who rarely feels excited or eager. \\
12 & I am someone who tends to find fault with others. \\
13 & I am someone who is dependable, steady. \\
14 & I am someone who is moody, has up and down mood swings. \\
15 & I am someone who is inventive, finds clever ways to do things. \\
16 & I am someone who tends to be quiet. \\
17 & I am someone who feels little sympathy for others. \\
18 & I am someone who is systematic, likes to keep things in order. \\
19 & I am someone who can be tense. \\
20 & I am someone who is fascinated by art, music, or literature. \\
21 & I am someone who is dominant, acts as a leader. \\
22 & I am someone who starts arguments with others. \\
23 & I am someone who has difficulty getting started on tasks. \\
24 & I am someone who feels secure, comfortable with self. \\
25 & I am someone who avoids intellectual, philosophical discussions. \\
26 & I am someone who is less active than other people. \\
27 & I am someone who has a forgiving nature. \\
28 & I am someone who can be somewhat careless. \\
29 & I am someone who is emotionally stable, not easily upset. \\
30 & I am someone who has little creativity. \\
31 & I am someone who is sometimes shy, introverted. \\
32 & I am someone who is helpful and unselfish with others. \\
33 & I am someone who keeps things neat and tidy. \\
34 & I am someone who worries a lot. \\
35 & I am someone who values art and beauty. \\
36 & I am someone who finds it hard to influence people. \\
37 & I am someone who is sometimes rude to others. \\
38 & I am someone who is efficient, gets things done. \\
39 & I am someone who often feels sad. \\
40 & I am someone who is complex, a deep thinker. \\
41 & I am someone who is full of energy. \\
42 & I am someone who is suspicious of others’ intentions. \\
43 & I am someone who is reliable, can always be counted on. \\
44 & I am someone who keeps their emotions under control. \\
45 & I am someone who has difficulty imagining things. \\
46 & I am someone who is talkative. \\
47 & I am someone who can be cold and uncaring. \\
48 & I am someone who leaves a mess, doesn’t clean up. \\
49 & I am someone who rarely feels anxious or afraid. \\
50 & I am someone who thinks poetry and plays are boring. \\
51 & I am someone who prefers to have others take charge. \\
52 & I am someone who is polite, courteous to others. \\
53 & I am someone who is persistent, works until the task is finished. \\
54 & I am someone who tends to feel depressed, blue. \\
55 & I am someone who has little interest in abstract ideas. \\
56 & I am someone who shows a lot of enthusiasm. \\
57 & I am someone who assumes the best about people. \\
58 & I am someone who sometimes behaves irresponsibly. \\
59 & I am someone who is temperamental, gets emotional easily. \\
60 & I am someone who is original, comes up with new ideas. \\
\end{tabular}
\caption{BFI-2 Scale}
\label{fig:bfi-2}
\end{table}


\textbf{Scoring:} Reverse-keyed items are denoted by ``R.''

In psychological measurement and scale development, reverse coding is a common and essential technique. 
Its primary purpose is to ensure that the scoring direction of all items remains consistent, thereby improving the reliability of the measurement results and the accuracy of their interpretation.

A balanced arrangement of positive and negative items helps reduce response biases, such as consistency effects or response patterns where participants select the same option repeatedly. 
These tendencies can mask the respondent’s true attitudes or behavioral traits, leading to distorted measurement outcomes. 
By alternating positive and negative statements and applying reverse coding to the appropriate items, this issue can be effectively mitigated.

Reverse coding can also help reduce the influence of social desirability bias. When all items are presented in the same direction, participants may easily guess the purpose of the test and provide responses that align with perceived expectations. 
By mixing positive and negative statements and applying reverse coding, the scale can disrupt this pattern, making it harder for participants to determine the ``correct'' answers, thus providing a more genuine reflection of their inner states or attitudes.









\setlength\tabcolsep{3.5pt}
\begin{table}[h]
\centering
\scriptsize % Reduce font size
\begin{tabular}{p{2cm} p{9cm}} % Reduced column widths
\textbf{Domain Level} & \textbf{Item Numbers} \\ 
Extraversion & 1, 6, 11R, 16R, 21, 26R, 31R, 36R, 41, 46, 51R, 56 \\
Agreeableness & 2, 7, 12R, 17R, 22R, 27, 32, 37R, 42R, 47R, 52, 57 \\
Conscientiousness & 3R, 8R, 13, 18, 23R, 28R, 33, 38, 43, 48R, 53, 58R \\
Neuroticism & 4R, 9R, 14, 19, 24R, 29R, 34, 39, 44R, 49R, 54, 59 \\
Openness & 5R, 10, 15, 20, 25R, 30R, 35, 40, 45R, 50R, 55R, 60 \\
\end{tabular}
\caption{BFI-2 Domain Level with Item Numbers}
\label{fig:bfi-2_domain}
\end{table}

\vspace{1em}

\begin{table}[h]
\centering
\scriptsize % Further reduce font size
\begin{tabular}{p{4cm} p{9cm}} % Adjusted column widths for fit
\textbf{Facet Level} & \textbf{Item Numbers} \\ 
Sociability & 1, 16R, 31R, 46 \\
Assertiveness & 6, 21, 36R, 51R \\
Energy Level & 11R, 26R, 41, 56 \\
Compassion & 2, 17R, 32, 47R \\
Respectfulness & 7, 22R, 37R, 52 \\
Trust & 12R, 27, 42R, 57 \\
Organization & 3R, 18, 33, 48R \\
Productiveness & 8R, 23R, 38, 53 \\
Responsibility & 13, 28R, 43, 58R \\
Anxiety & 4R, 19, 34, 49R \\
Depression & 9R, 24R, 39, 54 \\
Emotional Volatility & 14, 29R, 44R, 59 \\
Intellectual Curiosity & 10, 25R, 40, 55R \\
Aesthetic Sensitivity & 5R, 20, 35, 50R \\
Creative Imagination & 15, 30R, 45R, 60 \\
\end{tabular}
\caption{BFI-2 Facet Level with Item Numbers}
\label{fig:bfi-2_facet}
\end{table}



\clearpage
\newpage


\section{Appendix: Prompts List}
\label{sec:appendix_prompt}

\paragraph{Personality Scale Prompt Format} 
Table~\ref{tab:personality_prompt} presents the prompts used to generate LLM responses for the selected personality test on a Likert scale, where \textit{personality\_description} denotes the personality prompt.

In the PSI method, the personality prompt integrates both the interview question and the interviewee’s corresponding response, as collected in the PSI dataset (see Appendix~\ref{sec:appendix dataset}).
For the Persona and Shape methods, more detailed descriptions of the personality prompts can be found in Appendix~\ref{sec:appendix additional_setting}.

The \textit{test\_item} corresponds to each individual item in the BFI-2 scale. 


\begin{table}[h]
    \centering
    \begin{tabularx}{\linewidth}{X}
    \hline
    For the following task, respond in a way that matches this description: \{personality\_description\}.\\ Considering the statement, please indicate the extent to which you agree or disagree on a scale from 1 to 5 (where 1 = ``disagree strongly'', 2 = ``disagree a little'', 3 = ``neither agree nor disagree'', 4 = ``agree a little'', and 5 = ``agree strongly''): \{test\_item\}.\\
    \hline
    \end{tabularx}
    \caption{Prompt format for gathering LLMs' responses to BFI-2 scale.}
    \label{tab:personality_prompt}
\end{table}

\paragraph{Behavioral Question Prompt Format}
The complete prompt format for eliciting LLMs' responses to personality-related behavioral questions is defined in Table \ref{tab:behavior_prompt}. 

The \textit{questions\_and\_responses} refers to the transcript of a structured human personality interview, while \textit{question\_list} comprises statements evaluating OCB and CWB, as detailed in Table~\ref{fig:OCB} and Table~\ref{fig:CWB} in Appendix~\ref{sec:appendix additional_setting}. The model is instructed to rate each statement using a frequency scale from one to five.

\begin{table}[h]
    \centering
    \begin{tabularx}{\linewidth}{X}
    \hline
    Task: Simulate an individual's behavior and predict their responses to a series of work-related questions.\\
    Data to Analyze: \{questions\_and\_responses\}\\
    Instructions:\\
    1. Simulate the Individual: Use the provided data to simulate the individual's personality, behavior, and work habits.\\
    2. Predict Responses:\\
       - Based on the provided data, simulate the individual and answer the question.\\
       - Recall that in the past year, how often have you done the following things? (1 = Never, 2 = Once or twice, 3 = Once or twice per month, 4 = Once or twice per week, 5 = Every day)\\
    Questions: \{question\_list\}\\
    Output Format:\\
    Question Score: [Provide the numerical score here.]\\
    \hline
    \end{tabularx}
    \caption{Prompt format for gathering LLMs' responses to personality-related behavioral questions.}
    \label{tab:behavior_prompt}
\end{table}






\clearpage
\newpage

\section{Appendix: Additional Experiment Settings}
\label{sec:appendix additional_setting}

\paragraph{Pearson Correlation Coefficient}

The correlation calculation formula is as follows: 
\[
r = \frac{\sum (X_i - \overline{X}) (Y_i - \overline{Y})}{\sqrt{\sum (X_i - \overline{X})^2 \sum (Y_i - \overline{Y})^2}}
\]
\( X_i \) and \( Y_i \) represent the data values for LLMs and human. 
The symbols \( \overline{X} \) and \( \overline{Y} \) represent the means of variables \( X \) and \( Y \), respectively. 
The numerator, \( \sum (X_i - \overline{X})(Y_i - \overline{Y}) \), represents the covariance between \( X \) and \( Y \). 
The denominator, \( \sqrt{\sum (X_i - \overline{X})^2 \sum (Y_i - \overline{Y})^2} \), standardizes the result, constraining the value of \( r \) to range between -1 and 1. 

When \( r \) is close to 1, it indicates a strong positive correlation between the two variables; when \( r \) is close to -1, it indicates a strong negative correlation; and when \( r \) is close to 0, it indicates no significant linear relationship between the two variables. 


\paragraph{Persona Method}
The Persona method is based on the Persona-Chat dataset constructed by~\citet{zhang2018personalizing}. 
The dataset consists of persona descriptions, and each is made up of five short sentences containing demographic information collected through Amazon Mechanical Turk crowdsourcing. 
To avoid sentence similarity or repetition, these persona descriptions were required to be rewritten (e.g., changing ``I am very shy'' to ``I am not a social person''). 
\citet{zhang2018personalizing} demonstrated through machine learning model validation and human evaluations that such persona descriptions provide an effective method to enhance personalization. 
Currently, incorporating personal profiles into prompts is widely used in research related to LLM agents~\citep{park2023choicemates,wang2023does,xi2023rise}.

In the current study, we treated each persona description as an individual entity (i.e., a single subject) and randomly selected 300 persona descriptions from the dataset. 
One example is, \textit{``I wear a lot of leather. I have boots I always wear. I sleep in late during the day. I listen to metal music. I have black spiky hair.''}


\paragraph{Shape Method}
The Shape method is based on the work of~\citep{serapio2023personality}, who introduced a prompting approach to shape synthetic personality in LLMs along desired dimensions. 
The researchers expanded upon~\citet{goldberg1990alternative} lexical hypothesis, expanding his list of 70 bipolar adjectives~\citep{goldberg1992development} to include 104 trait adjectives. 
Additionally, they employed linguistic qualifiers commonly used in Likert-type scales~\citep{likert1932technique}, such as ``a bit,'' ``very,'' and ``extremely,'' to set target levels for each adjective. 
This resulted in a fine-grained prompting method with nine levels: 1. extremely {low adjective}; 2. very {low adjective}; 3. {low adjective}; 4. a bit {low adjective}; 5. neither {low adjective} nor {high adjective}; 6. a bit {high adjective}; 7. {high adjective}; 8. very {high adjective}; 9. extremely {high adjective}. 

In our study, each prompt involves five randomly selected adjective markers from a specific personality domain. 
These markers are positioned after a consistent linguistic qualifier to set the prompt at one of nine intensity levels. 
For example, one prompt is: \textit{``You are extremely friendly, extremely energetic, extremely assertive, extremely bold, and extremely active.''}
These are five positively-keyed adjectives targeting extraversion. 
In this case, the prompt seeks to create a highly sociable and dynamic personality profile, which might result in responses characterized by enthusiasm, confidence, and proactivity. 
We also randomly select 300 prompts here.



\paragraph{CFA Model}
The basic form of the CFA model is:

\[
y = \Lambda \eta + \epsilon
\]
where \( y \) represents the vector of observed variables; \( \Lambda \) is the factor loading matrix (i.e., the loadings of each observed variable on the latent factors); \( \eta \) is the vector of latent factors; \( \epsilon \) is the vector of error terms, with the assumption that the error terms have a mean of zero and are mutually independent~\citep{joreskog1969general}.

Latent factors are variables that are not directly observed but are inferred from other variables that are observed (measured). 
In the context of a CFA model, latent factors represent underlying constructs or traits that are believed to influence the observed variables. 
For example, in psychology, a latent factor might represent a construct like ``intelligence'' or ``anxiety,'' which cannot be measured directly but can be estimated through related observed behaviors or responses on a test. 





\paragraph{Model Fit Information}

\textbf{Chi-Square Test, \(\mathbf{\chi^2}\):} 
The chi-square test is used to measure the difference between the observed covariance matrix and the factor model's fitted covariance matrix.
\[
\chi^2 = (N - 1) \times F_{\text{ML}}
\]
where \( N \) is the sample size; \( F_{\text{ML}} \) is the value of the fit function under maximum likelihood estimation.
A smaller chi-square value indicates a better model fit.
However, with large samples, the chi-square value tends to be large, so other fit indices are usually the primary reference.



\textbf{Degrees of Freedom, \textit{df}:}
The \textit{df} represent the relationship between model parameters and observed variables:
\[
df = \frac{p(p+1)}{2} - q
\]
where \( p \) is the number of observed variables, and \( q \) is the number of model parameters.

Degrees of freedom reflect the amount of independent information available in a statistical model to estimate its parameters. 
They are calculated as the number of information points provided by the data minus the number of model parameters, representing the extent to which the model can adjust freely. 
Therefore, a higher degree of freedom indicates fewer parameters in the model and fewer constraints on the data. 
With more degrees of freedom, the model has fewer restrictions, though the fitting difficulty may increase. 
Too few degrees of freedom may lead to overfitting, while too many can result in underfitting.


\textbf{Comparative Fit Index, CFI:}
The CFI is used to compare the goodness of fit of a model with a baseline model (usually an independent model).
\[
\text{CFI} = 1 - \frac{\max(\chi^2 - \textit{df}, 0)}{\max(\chi_{\text{null}}^2 - \textit{df}_{\text{null}}, 0)}
\]
where \( \chi^2 \) and \( \text{df} \) are the chi-square value and degrees of freedom of the target model; \( \chi_{\textit{null}}^2 \) and \( \textit{df}_{\text{null}} \) are the chi-square value and degrees of freedom of the baseline (independent) model.



\textbf{Tucker-Lewis Index, TLI:}
TLI, also known as the Non-Normed Fit Index (NNFI), considers model complexity.
\[
\text{TLI} = \frac{\left(\chi^2_{\text{null}} / \textit{df}_{\text{null}}\right) - \left(\chi^2 / \textit{df}\right)}{\left(\chi^2_{\text{null}} / \textit{df}_{\text{null}}\right) - 1}
\]
This value ranges from 0 to 1, with a value typically greater than.90 indicating good model fit.




\textbf{Root Mean Square Error of Approximation, RMSEA:}
The RMSEA quantifies the error per degree of freedom in a model, with smaller values indicating better model fit.
\[
\text{RMSEA} = \frac{\chi^2 - df}{df(N - 1)}
\]
where \(\chi^2\) is the chi-square value of the model; \textit{df} is the degrees of freedom; \textit{N} is the sample size.



\textbf{Standardized Root Mean Square Residual, SRMR:}
The SRMR measures the discrepancy between model-predicted values and actual observed values, calculated as:
\[
\text{SRMR} = \sqrt{\frac{\sum_{i=1}^{p} \sum_{j=1}^{p} \left(s_{ij} - \hat{s}_{ij}\right)^2}{\frac{p(p+1)}{2}}}
\]
where \( s_{ij} \) is an element in the observed covariance matrix; \( \hat{s}_{ij} \) is an element in the model-fitted covariance matrix.

Based on~\citet{hu1999cutoff}, CFI/TLI $\geq$ .95, RMSEA $\leq$ .06, and SRMR $\leq$ .08 are considered good fit thresholds.



\paragraph{Tucker's Congruence Coefficient}
Tucker's congruence coefficient, also known as the coefficient of congruence, is typically used to assess the similarity between two-factor structures in factor analysis~\citep{tucker1951}. 
The formula is given by:

\[
\phi = \frac{\sum_{i=1}^n a_i b_i}{\sqrt{\sum_{i=1}^n a_i^2 \cdot \sum_{i=1}^n b_i^2}}
\]
where \( a_i \) and \( b_i \) are the loadings of the \( i \)-th factor for two different factor solutions (or different samples or methods); \( n \) is the total number of factors.

The coefficient ranges from -1 to 1, where values close to 1 indicate high similarity (congruence) between the factor solutions, values close to 0 indicate low similarity, and negative values indicate a dissimilar or inverse relationship.
According to~\citet{lorenzo2006tucker}, a TCC above .95 indicates good similarity, while a TCC of .85 to .94 suggests fair similarity.
However, this is a relatively lenient criterion; specific differences still need to be determined based on the factor loadings.


\paragraph{Behavior Variable Measures} 
\textbf{Organizational Citizenship Behavior (OCB):} OCB was measured using ten items from~\citet{spector2010counterproductive} to assess extra-role behaviors. 
Items were rated on a frequency scale ranging from 1 (never) to 5 (every day). 
Example item: \textit{``In the past year, how often have you helped new employees get oriented to the job?''}. 
Internal consistency was Cronbach’s alpha = .83.

\textbf{Counterproductive Work Behavior (CWB):} CWB was measured using ten items from~\citet{spector2010counterproductive}, designed to assess harmful workplace behaviors. 
Items were rated on a frequency scale ranging from 1 (never) to 5 (every day). 
Example item: \textit{``In the past year, how often have you ignored someone at work?''.} 
Internal consistency was Cronbach’s alpha = .86.

The detailed scale information for both can be found in Table~\ref{fig:OCB} and Table~\ref{fig:CWB} below.

\textbf{Instructions:} Recall that in the past year, how often have you done the following things (1 = Never, 2 = Once or twice, 3 = Once or twice per month, 4 = Once or twice per week, 5 = Every day)?


\begin{table}[h]
\centering
\scriptsize % Further reduce table font size for compactness
\begin{tabular}{@{}p{0.5cm}p{10.5cm}@{}} % Adjusted column width to fit page
\# & \textbf{Statement} \\
1 & Took time to advise, coach, or mentor a co-worker. \\
2 & Helped co-worker learn new skills or shared job knowledge. \\
3 & Helped new employees get oriented to the job. \\
4 & Lent a compassionate ear when someone had a work problem. \\
5 & Offered suggestions to improve how work is done. \\
6 & Helped a co-worker who had too much to do. \\
7 & Volunteered for extra work assignments. \\
8 & Worked weekends or other days off to complete a project or task. \\
9 & Volunteered to attend meetings or work on committees on own time. \\
10 & Gave up meal and other breaks to complete work. \\
\end{tabular}
\caption{OCB Scale}
\label{fig:OCB}
\end{table}


\begin{table}[h]
\centering
\scriptsize % Further reduce table font size for compactness
\begin{tabular}{@{}p{0.5cm}p{10.5cm}@{}} % Adjusted column width to fit page
\# & \textbf{Statement} \\
1 & Purposely wasted your employer's materials/supplies. \\
2 & Complained about insignificant things at work. \\
3 & Told people outside the job what a lousy place you work for. \\
4 & Came to work late without permission. \\
5 & Stayed home from work and said you were sick when you weren't. \\
6 & Insulted someone about their job performance. \\
7 & Made fun of someone's personal life. \\
8 & Ignored someone at work. \\
9 & Started an argument with someone at work. \\
10 & Insulted or made fun of someone at work. \\
\end{tabular}
\caption{CWB Scale}
\label{fig:CWB}
\end{table}




\clearpage
\newpage

\section{Appendix: Psychometric Data Evaluation Framework}
\label{sec:appendix evaluation_framework}

Here, we present the evaluation framework used to assess the fidelity of simulated psychometric data (i.e., how well it aligns with human data).

The evaluation is conducted at different levels. 
For example, for personality, it includes Item, Facet, and Domain levels. 
Generally, the structure of psychometric data is hierarchical, where observed responses (item level) map onto latent traits (domain level).
Our evaluation framework incorporates both descriptive statistics and psychometric performance metrics to ensure a comprehensive evaluation (see Figure~\ref{fig:psychometric_framework}). 
Below, we outline each component and the rationale for its inclusion.


\paragraph{Descriptive Statistics}
Descriptive statistics are primarily used to summarize, outline, and present the basic features of data. 
They help researchers understand the distribution and fundamental trends of the data without interpreting or measuring specific psychological constructs.
Do not forget that the evaluation of psychometric data is hierarchical, we need to evaluate on item and domain level.

\textbf{Mean (\textit{M}):}
The mean represents the central tendency of the responses, reflecting the average score across individuals.


\textbf{Standard Deviation (\textit{SD}):}
The standard deviation captures the dispersion of responses, indicating how much variation exists within the data.

Both \textit{M} and \textit{SD} can be quantified for similarity to human distribution using MAE and \( r \). However, these metrics provide a more summarized level of comparison; we also need to examine performance on specific items and the domain.

\textbf{Distribution Shape:}
The distribution shape describes the overall pattern of how responses are spread across the scale. It provides insight into whether the data follows a normal distribution or exhibits skewness and kurtosis.

Skewness measures the asymmetry of the distribution. 
A positive skew indicates a longer right tail (more low scores with a few high scores), while a negative skew indicates a longer left tail (more high scores with a few low scores).

Kurtosis captures the ``tailedness'' of the distribution. 
High kurtosis (leptokurtic) suggests heavy tails with more extreme values, while low kurtosis (platykurtic) indicates a flatter distribution with fewer extreme values.







\paragraph{Psychometric Performance}
Psychometric performance is primarily used to evaluate the measurement quality of psychological constructs, ensuring that assessments accurately and reliably capture the intended traits. 

% \textbf{Model Fit:}
% The model fit indices provide an overall assessment of how well the proposed structure aligns with the observed data. 
% Ideally, the simulated psychometric data will have a similar model fit compared with the human data.
% Common indices include \(\mathbf{\chi^2}\), CFI, TLI, RMSEA, and SRMR.

\textbf{Structural Validity:}
Structural validity refers to the extent to which the internal structure of a measurement instrument aligns with the theoretical construct it is intended to assess.
It can be assessed through model fit, factor loadings, and inter-factor correlations.

The model fit indices provide an overall assessment of how well the proposed structure aligns with the observed data. 
Ideally, the simulated psychometric data will have a similar model fit compared with the human data.
Common indices include \(\mathbf{\chi^2}\), CFI, TLI, RMSEA, and SRMR.

Factor loadings indicate the extent to which each item represents the intended construct, while inter-factor correlations reveal relationships between latent variables.
The simulated psychometric data should also resemble human data in these two aspects.
We can use TCC as a summarized level of comparison; however, it is typically considered too lenient, so we need to examine factor loadings and inter-factor correlation values more closely.

\textbf{Scale Reliability:}
Scale reliability assesses the internal consistency of items measuring the same construct. 
Cronbach’s alpha is a widely used reliability coefficient,
The simulated psychometric data should also resemble human data in this aspect.

\textbf{Discriminant Validity:}
Discriminant validity ensures that distinct constructs are not excessively correlated. It can be examined by calculating the mean absolute correlation.

\begin{figure*}
    \centering
    \includegraphics[width=\linewidth]{latex/Figures/psychometric_framework.pdf}
    \caption{Psychometric Data Evaluation Framework}
    \label{fig:psychometric_framework}
\end{figure*}


 \clearpage



\clearpage
\newpage

\section{Appendix: Additional Analyses and Results}
\label{sec:appendix results}



\subsection{Additional Method Comparison Results}
\label{sec:appendix results.1}



\paragraph{Additional Descriptive Statistics Results}
Here we show the detailed means and standard deviations for human responses and LLM responses at the item, facet, and domain levels, see Tables~\ref{tab:item_mean},~\ref{tab:item_sd},~\ref{tab:facet_mean},~\ref{tab:facet_sd},~\ref{tab:domain_mean}, and~\ref{tab:domain_sd}.



\begin{table*}[ht]
\centering
\resizebox{\textwidth}{!}{%
\begin{tabular}{@{}clcccccccc@{}}
\toprule
\multirow{2}{*}{\textbf{No.}} & \multirow{2}{*}{\textbf{Item Content}} & \multirow{2}{*}{\textbf{Human}} & \multicolumn{2}{c}{\textbf{Persona}} & \multicolumn{2}{c}{\textbf{Shape}} & \multicolumn{2}{c}{\textbf{PSI}} \\ 
\cmidrule(lr){4-5} \cmidrule(lr){6-7} \cmidrule(lr){8-9}
 & & & \textbf{GPT-4o} & \textbf{Llama3} & \textbf{GPT-4o} & \textbf{Llama3} & \textbf{GPT-4o} & \textbf{Llama3} \\ 
\midrule
1  & I am someone who is outgoing, sociable                      & 3.03 & 3.41 & 2.77 & 2.76 & 2.45 & 2.67 & 2.14 \\
2  & I am someone who is compassionate, has a soft heart         & 4.34 & 3.51 & 3.87 & 3.10 & 3.17 & 4.05 & 3.91 \\
3  & I am someone who tends to be disorganized                   & 3.62 & 3.39 & 3.77 & 3.32 & 3.31 & 3.34 & 3.24 \\
4  & I am someone who is relaxed, handles stress well            & 2.56 & 3.20 & 2.94 & 3.47 & 3.32 & 2.91 & 2.96 \\
5  & I am someone who has few artistic interests                 & 3.68 & 3.72 & 3.80 & 3.27 & 2.92 & 2.75 & 2.73 \\
6  & I am someone who has an assertive personality               & 2.96 & 3.08 & 3.07 & 2.82 & 2.81 & 2.57 & 2.25 \\
7  & I am someone who is respectful, treats others with respect  & 4.62 & 3.38 & 4.43 & 3.48 & 3.62 & 4.16 & 4.22 \\
8  & I am someone who tends to be lazy                           & 3.62 & 4.00 & 4.23 & 3.51 & 3.44 & 3.54 & 3.75 \\
9  & I am someone who stays optimistic after a setback           & 2.44 & 2.74 & 2.29 & 2.95 & 2.92 & 2.18 & 2.29 \\
10 & I am someone who is curious about many different things     & 4.43 & 3.89 & 3.71 & 3.02 & 3.15 & 3.60 & 3.05 \\
11 & I am someone who rarely feels excited or eager              & 3.54 & 4.30 & 4.18 & 3.21 & 2.99 & 2.98 & 2.40 \\
12 & I am someone who tends to find fault with others            & 3.34 & 3.29 & 4.31 & 3.47 & 3.46 & 3.58 & 3.71 \\
13 & I am someone who is dependable, steady                      & 4.33 & 3.29 & 3.70 & 2.93 & 3.11 & 4.12 & 3.96 \\
14 & I am someone who is moody, has up and down mood swings      & 2.56 & 3.01 & 2.52 & 2.72 & 2.60 & 3.04 & 2.31 \\
15 & I am someone who is inventive, finds clever ways to do things 
                                                                  & 3.90 & 3.44 & 3.24 & 3.06 & 2.91 & 3.05 & 2.37 \\
16 & I am someone who tends to be quiet                          & 2.35 & 3.26 & 3.44 & 3.02 & 3.25 & 2.17 & 2.51 \\
17 & I am someone who feels little sympathy for others           & 3.71 & 3.73 & 4.47 & 3.72 & 3.54 & 4.22 & 2.88 \\
18 & I am someone who is systematic, likes to keep things in order 
                                                                  & 3.96 & 3.00 & 2.46 & 2.94 & 2.59 & 3.18 & 2.44 \\
19 & I am someone who can be tense                               & 3.29 & 3.23 & 3.18 & 3.12 & 3.04 & 3.66 & 2.73 \\
20 & I am someone who is fascinated by art, music, or literature 
                                                                  & 4.06 & 3.61 & 2.99 & 3.15 & 2.90 & 3.43 & 2.81 \\
21 & I am someone who has an assertive personality (dominant)    & 2.90 & 2.94 & 2.46 & 2.77 & 2.53 & 2.42 & 2.07 \\
22 & I am someone who starts arguments with others               & 4.23 & 3.65 & 4.56 & 3.65 & 3.70 & 4.44 & 4.35 \\
23 & I am someone who has difficulty getting started on tasks    & 3.31 & 3.29 & 3.85 & 3.16 & 3.23 & 3.10 & 2.79 \\
24 & I am someone who feels secure, comfortable with self        & 2.26 & 2.75 & 2.24 & 3.14 & 2.65 & 2.65 & 2.70 \\
25 & I am someone who avoids intellectual, philosophical discussions 
                                                                  & 3.93 & 3.40 & 3.36 & 3.23 & 3.13 & 2.85 & 2.57 \\
26 & I am someone who is less active than other people           & 3.35 & 3.80 & 3.86 & 3.36 & 3.14 & 2.65 & 2.67 \\
27 & I am someone who has a forgiving nature                     & 3.78 & 3.05 & 3.48 & 3.05 & 3.28 & 3.51 & 3.59 \\
28 & I am someone who can be somewhat careless                   & 3.56 & 3.30 & 3.19 & 3.16 & 3.00 & 3.06 & 3.15 \\
29 & I am someone who is emotionally stable, not easily upset    & 2.44 & 3.16 & 2.66 & 3.17 & 2.87 & 2.70 & 2.67 \\
30 & I am someone who has little creativity                      & 3.95 & 4.10 & 4.36 & 3.59 & 3.42 & 3.37 & 2.85 \\
31 & I am someone who is sometimes shy, introverted              & 2.35 & 3.28 & 3.43 & 3.11 & 3.34 & 2.14 & 2.70 \\
32 & I am someone who is helpful and unselfish with others       & 4.18 & 3.32 & 3.88 & 3.27 & 3.24 & 4.06 & 4.12 \\
33 & I am someone who keeps things neat and tidy                 & 3.72 & 2.99 & 2.47 & 3.07 & 2.63 & 3.04 & 2.70 \\
34 & I am someone who worries a lot                              & 3.30 & 2.96 & 2.59 & 2.79 & 2.50 & 3.64 & 3.04 \\
35 & I am someone who values art and beauty                      & 4.16 & 3.65 & 3.23 & 3.14 & 3.06 & 3.38 & 2.87 \\
36 & I am someone who finds it hard to influence people          & 3.09 & 3.32 & 3.44 & 3.25 & 2.99 & 2.74 & 2.25 \\
37 & I am someone who is sometimes rude to others                & 3.79 & 3.24 & 4.43 & 3.50 & 3.63 & 3.70 & 4.10 \\
38 & I am someone who is efficient, gets things done             & 4.23 & 3.31 & 3.45 & 3.07 & 3.12 & 3.69 & 3.07 \\
39 & I am someone who often feels sad                            & 2.71 & 2.86 & 2.27 & 2.74 & 2.42 & 3.24 & 2.69 \\
40 & I am someone who is complex, a deep thinker                 & 4.02 & 3.64 & 3.59 & 3.19 & 3.14 & 3.34 & 2.74 \\
41 & I am someone who is full of energy                          & 3.14 & 3.45 & 3.39 & 2.99 & 2.86 & 2.81 & 2.41 \\
42 & I am someone who is suspicious of others’ intentions        & 2.75 & 3.14 & 3.91 & 3.27 & 3.40 & 3.42 & 3.60 \\
43 & I am someone who is reliable, can always be counted on      & 4.32 & 3.28 & 3.76 & 3.12 & 3.18 & 4.06 & 3.75 \\
44 & I am someone who keeps their emotions under control         & 2.22 & 3.10 & 2.92 & 2.98 & 2.90 & 2.50 & 2.34 \\
45 & I am someone who has difficulty imagining things            & 4.17 & 4.06 & 4.08 & 3.57 & 3.43 & 3.21 & 2.14 \\
46 & I am someone who is talkative                               & 2.88 & 3.11 & 2.53 & 2.87 & 2.50 & 3.31 & 2.36 \\
47 & I am someone who can be cold and uncaring                   & 3.84 & 3.95 & 4.44 & 3.60 & 3.47 & 4.08 & 4.19 \\
48 & I am someone who leaves a mess, doesn’t clean up            & 4.15 & 3.47 & 4.25 & 3.39 & 3.25 & 3.32 & 4.03 \\
49 & I am someone who rarely feels anxious or afraid             & 3.44 & 3.20 & 3.12 & 2.85 & 2.71 & 3.48 & 3.66 \\
50 & I am someone who thinks poetry and plays are boring         & 3.66 & 3.54 & 3.85 & 3.44 & 3.15 & 2.90 & 2.29 \\
51 & I am someone who prefers to have others take charge         & 2.96 & 3.54 & 4.33 & 3.39 & 3.20 & 3.52 & 3.87 \\
52 & I am someone who is polite, courteous to others             & 4.50 & 3.28 & 4.13 & 3.33 & 3.56 & 3.97 & 4.38 \\
53 & I am someone who is persistent, works until the task is finished 
                                                                  & 4.27 & 3.56 & 3.87 & 3.01 & 3.13 & 4.07 & 3.75 \\
54 & I am someone who tends to feel depressed, blue              & 2.65 & 2.70 & 2.01 & 2.60 & 2.36 & 3.25 & 2.78 \\
55 & I am someone who has little interest in abstract ideas      & 3.83 & 3.25 & 3.40 & 3.08 & 2.93 & 2.53 & 2.15 \\
56 & I am someone who shows a lot of enthusiasm                  & 3.35 & 3.86 & 4.08 & 2.81 & 2.81 & 2.70 & 2.58 \\
57 & I am someone who assumes the best about people              & 3.37 & 3.12 & 3.40 & 3.04 & 3.13 & 3.46 & 3.64 \\
58 & I am someone who sometimes behaves irresponsibly            & 3.64 & 3.11 & 3.44 & 2.90 & 3.22 & 3.13 & 3.55 \\
59 & I am someone who is temperamental, gets emotional easily    & 2.53 & 3.02 & 2.33 & 2.80 & 2.66 & 2.96 & 2.04 \\
60 & I am someone who is original, comes up with new ideas       & 3.84 & 3.28 & 3.25 & 2.96 & 2.88 & 3.05 & 2.40 \\
\bottomrule
\end{tabular}%
}
\caption{Item Level Mean for BFI-2 Human Responses and Different Methods LLM Responses \\
\textit{Note:} \textit{n} = 1,559 for human responses, \textit{n} = 297 for Shape Llama3, and \textit{n} = 300 for other LLM responses. Some sample sizes are below 300, because certain generated data exceeded reasonable thresholds (1-5) for specific items and were excluded from the analysis.}
\label{tab:item_mean}
\end{table*}





\begin{table*}[ht]
\centering
\resizebox{\textwidth}{!}{%
\begin{tabular}{@{}clcccccccccc@{}}
\toprule
\multirow{2}{*}{\textbf{No.}} & \multirow{2}{*}{\textbf{Item Content}} & \multirow{2}{*}{\textbf{Human}} & \multicolumn{2}{c}{\textbf{Persona}} & \multicolumn{2}{c}{\textbf{Shape}} & \multicolumn{2}{c}{\textbf{PSI}} \\ 
\cmidrule(lr){4-5} \cmidrule(lr){6-7} \cmidrule(lr){8-9}
& & & \textbf{GPT-4o} & \textbf{Llama3} & \textbf{GPT-4o} & \textbf{Llama3} & \textbf{GPT-4o} & \textbf{Llama3} \\ 
\midrule
1  & I am someone who is outgoing, sociable                      & 1.40 & 0.87 & 1.31 & 1.22 & 1.37 & 0.98 & 0.78 \\
2  & I am someone who is compassionate, has a soft heart         & 0.90 & 0.84 & 1.12 & 1.22 & 1.44 & 0.93 & 1.03 \\
3  & I am someone who tends to be disorganized                   & 1.36 & 0.78 & 0.95 & 1.17 & 1.46 & 0.76 & 0.96 \\
4  & I am someone who is relaxed, handles stress well            & 1.30 & 0.79 & 1.16 & 1.16 & 1.39 & 0.92 & 0.96 \\
5  & I am someone who has few artistic interests                 & 1.33 & 1.04 & 1.31 & 1.25 & 1.32 & 1.39 & 1.11 \\
6  & I am someone who has an assertive personality               & 1.38 & 0.79 & 1.18 & 1.18 & 1.37 & 0.76 & 0.73 \\
7  & I am someone who is respectful, treats others with respect  & 0.61 & 0.89 & 0.92 & 1.23 & 1.42 & 0.75 & 0.78 \\
8  & I am someone who tends to be lazy                           & 1.31 & 0.87 & 1.04 & 1.28 & 1.50 & 0.81 & 0.97 \\
9  & I am someone who stays optimistic after experiencing a setback 
                                                                   & 1.24 & 0.84 & 0.98 & 1.34 & 1.54 & 0.89 & 0.93 \\
10 & I am someone who is curious about many different things     & 0.81 & 0.89 & 1.12 & 1.37 & 1.55 & 0.98 & 0.91 \\
11 & I am someone who rarely feels excited or eager              & 1.26 & 0.84 & 0.75 & 1.33 & 1.41 & 0.93 & 0.82 \\
12 & I am someone who tends to find fault with others            & 1.32 & 0.59 & 0.69 & 1.26 & 1.48 & 0.83 & 0.70 \\
13 & I am someone who is dependable, steady                      & 0.86 & 0.85 & 1.12 & 1.33 & 1.59 & 0.79 & 0.75 \\
14 & I am someone who is moody, has up and down mood swings      & 1.36 & 0.56 & 1.09 & 1.27 & 1.41 & 0.76 & 0.93 \\
15 & I am someone who is inventive, finds clever ways to do things 
                                                                   & 1.03 & 0.69 & 1.14 & 1.21 & 1.44 & 0.63 & 0.74 \\
16 & I am someone who tends to be quiet                          & 1.34 & 0.84 & 1.12 & 1.12 & 1.27 & 1.02 & 0.97 \\
17 & I am someone who feels little sympathy for others           & 1.45 & 0.93 & 0.79 & 1.27 & 1.54 & 0.68 & 1.00 \\
18 & I am someone who is systematic, likes to keep things in order 
                                                                   & 1.06 & 0.88 & 1.09 & 1.16 & 1.37 & 0.94 & 0.81 \\
19 & I am someone who can be tense                               & 1.27 & 0.75 & 1.08 & 1.22 & 1.20 & 0.95 & 0.95 \\
20 & I am someone who is fascinated by art, music, or literature 
                                                                   & 1.17 & 0.89 & 1.44 & 1.18 & 1.36 & 1.18 & 1.11 \\
21 & I am someone who is dominant, acts as a leader              & 1.35 & 0.74 & 1.15 & 1.13 & 1.49 & 0.70 & 0.61 \\
22 & I am someone who starts arguments with others               & 1.05 & 0.72 & 0.77 & 1.24 & 1.54 & 0.72 & 0.60 \\
23 & I am someone who has difficulty getting started on tasks    & 1.39 & 0.69 & 0.86 & 1.26 & 1.49 & 0.74 & 0.69 \\
24 & I am someone who feels secure, comfortable with self        & 1.27 & 0.88 & 1.08 & 1.42 & 1.49 & 0.96 & 0.92 \\
25 & I am someone who avoids intellectual, philosophical discussions 
                                                                   & 1.21 & 0.79 & 1.36 & 1.25 & 1.56 & 1.10 & 0.99 \\
26 & I am someone who is less active than other people           & 1.31 & 0.99 & 1.04 & 1.24 & 1.31 & 0.98 & 0.83 \\
27 & I am someone who has a forgiving nature                     & 1.20 & 0.57 & 0.90 & 1.20 & 1.37 & 0.85 & 0.78 \\
28 & I am someone who can be somewhat careless                   & 1.25 & 0.83 & 1.11 & 1.18 & 1.38 & 0.68 & 1.00 \\
29 & I am someone who is emotionally stable, not easily upset    & 1.27 & 0.70 & 1.01 & 1.31 & 1.47 & 1.06 & 1.09 \\
30 & I am someone who has little creativity                      & 1.20 & 0.76 & 0.69 & 1.22 & 1.32 & 0.83 & 1.02 \\
31 & I am someone who is sometimes shy, introverted              & 1.36 & 0.87 & 1.11 & 1.07 & 1.30 & 1.04 & 0.99 \\
32 & I am someone who is helpful and unselfish with others       & 0.88 & 0.80 & 1.05 & 1.21 & 1.50 & 0.73 & 0.72 \\
33 & I am someone who keeps things neat and tidy                 & 1.24 & 0.64 & 1.00 & 1.09 & 1.35 & 0.73 & 0.71 \\
34 & I am someone who worries a lot                              & 1.47 & 0.70 & 1.08 & 1.23 & 1.33 & 1.18 & 1.07 \\
35 & I am someone who values art and beauty                      & 1.07 & 0.82 & 1.30 & 1.18 & 1.35 & 1.00 & 1.06 \\
36 & I am someone who finds it hard to influence people          & 1.20 & 0.69 & 1.04 & 1.26 & 1.37 & 0.67 & 0.67 \\
37 & I am someone who is sometimes rude to others                & 1.25 & 0.66 & 0.91 & 1.31 & 1.49 & 0.77 & 1.00 \\
38 & I am someone who is efficient, gets things done             & 0.88 & 0.82 & 1.12 & 1.25 & 1.46 & 0.84 & 0.91 \\
39 & I am someone who often feels sad                            & 1.43 & 0.68 & 0.98 & 1.27 & 1.25 & 0.99 & 0.91 \\
40 & I am someone who is complex, a deep thinker                 & 1.08 & 0.72 & 1.06 & 1.06 & 1.26 & 0.83 & 0.99 \\
41 & I am someone who is full of energy                          & 1.30 & 0.89 & 1.24 & 1.23 & 1.35 & 0.85 & 0.76 \\
42 & I am someone who is suspicious of others’ intentions        & 1.30 & 0.58 & 0.83 & 1.26 & 1.49 & 1.00 & 0.68 \\
43 & I am someone who is reliable, can always be counted on      & 0.88 & 0.93 & 1.09 & 1.32 & 1.59 & 0.80 & 0.78 \\
44 & I am someone who keeps their emotions under control         & 1.14 & 0.64 & 1.08 & 1.22 & 1.44 & 0.77 & 1.04 \\
45 & I am someone who has difficulty imagining things            & 1.06 & 0.77 & 0.53 & 1.25 & 1.27 & 0.95 & 0.54 \\
46 & I am someone who is talkative                              & 1.38 & 0.70 & 1.11 & 1.04 & 1.28 & 1.09 & 0.92 \\
47 & I am someone who can be cold and uncaring                   & 1.23 & 0.80 & 0.87 & 1.35 & 1.49 & 0.78 & 0.98 \\
48 & I am someone who leaves a mess, doesn’t clean up            & 1.12 & 0.71 & 0.95 & 1.26 & 1.64 & 0.55 & 0.85 \\
49 & I am someone who rarely feels anxious or afraid             & 1.37 & 0.83 & 1.12 & 1.15 & 1.37 & 1.02 & 0.89 \\
50 & I am someone who thinks poetry and plays are boring         & 1.37 & 0.75 & 1.31 & 1.38 & 1.53 & 0.87 & 0.68 \\
51 & I am someone who prefers to have others take charge         & 1.28 & 0.69 & 0.76 & 1.27 & 1.57 & 0.70 & 0.51 \\
52 & I am someone who is polite, courteous to others            & 0.70 & 0.65 & 0.97 & 1.14 & 1.44 & 0.75 & 0.79 \\
53 & I am someone who is persistent, works until the task is finished
                                                                   & 0.91 & 0.87 & 1.13 & 1.26 & 1.48 & 0.80 & 0.94 \\
54 & I am someone who tends to feel depressed, blue              & 1.46 & 0.79 & 1.07 & 1.31 & 1.37 & 1.08 & 1.00 \\
55 & I am someone who has little interest in abstract ideas      & 1.22 & 0.96 & 1.08 & 1.22 & 1.36 & 1.17 & 0.63 \\
56 & I am someone who shows a lot of enthusiasm                  & 1.22 & 0.85 & 1.09 & 1.34 & 1.51 & 0.83 & 0.94 \\
57 & I am someone who assumes the best about people              & 1.28 & 0.78 & 0.98 & 1.27 & 1.47 & 0.92 & 0.74 \\
58 & I am someone who sometimes behaves irresponsibly            & 1.30 & 0.74 & 1.13 & 1.14 & 1.40 & 0.70 & 0.95 \\
59 & I am someone who is temperamental, gets emotional easily    & 1.33 & 0.64 & 1.08 & 1.24 & 1.52 & 1.00 & 0.87 \\
60 & I am someone who is original, comes up with new ideas       & 1.07 & 0.65 & 1.13 & 1.22 & 1.41 & 0.68 & 0.75 \\
\bottomrule
\end{tabular}%
}
\caption{Item Level Standard Deviation for BFI-2 Human Responses and Different Methods LLM Responses \\
\textit{Note:} \textit{n} = 1,559 for human responses, \textit{n} = 297 for Shape Llama3, and \textit{n} = 300 for other LLM responses. Some sample sizes are below 300, because certain generated data exceeded reasonable thresholds (1-5) for specific items and were excluded from the analysis.}
\label{tab:item_sd}
\end{table*}





\begin{table*}[ht]
\centering
\normalsize
\begin{tabular}{@{}lcccccccc@{}}
\toprule
\multirow{2}{*}{\textbf{Facet}} & \multirow{2}{*}{\textbf{Human}} & \multicolumn{2}{c}{\textbf{Persona}} & \multicolumn{2}{c}{\textbf{Shape}} & \multicolumn{2}{c}{\textbf{PSI}} \\ 
\cmidrule(lr){3-4} \cmidrule(lr){5-6} \cmidrule(lr){7-8}
& & \textbf{GPT-4o} & \textbf{Llama3} & \textbf{GPT-4o} & \textbf{Llama3} & \textbf{GPT-4o} & \textbf{Llama3} \\ 
\midrule
Sociability            & 2.65 & 3.26 & 3.04 & 2.94 & 2.89 & 2.57 & 2.43 \\
Assertiveness          & 2.97 & 3.22 & 3.33 & 3.06 & 2.88 & 2.81 & 2.61 \\
Energy Level           & 3.35 & 3.85 & 3.88 & 3.09 & 2.95 & 2.79 & 2.52 \\
Compassion             & 4.02 & 3.63 & 4.16 & 3.42 & 3.36 & 4.10 & 3.78 \\
Respectfulness         & 4.29 & 3.39 & 4.39 & 3.49 & 3.63 & 4.07 & 4.26 \\
Trust                  & 3.31 & 3.15 & 3.78 & 3.21 & 3.32 & 3.49 & 3.64 \\
Organization           & 3.86 & 3.21 & 3.24 & 3.18 & 2.94 & 3.22 & 3.10 \\
Productiveness         & 3.86 & 3.54 & 3.85 & 3.19 & 3.23 & 3.60 & 3.34 \\
Responsibility         & 3.96 & 3.25 & 3.52 & 3.03 & 3.13 & 3.59 & 3.60 \\
Anxiety                & 3.15 & 3.15 & 2.96 & 3.06 & 2.89 & 3.42 & 3.10 \\
Depression             & 2.51 & 2.76 & 2.20 & 2.86 & 2.59 & 2.83 & 2.61 \\
Emotional Volatility   & 2.44 & 3.07 & 2.61 & 2.92 & 2.76 & 2.80 & 2.34 \\
Intellectual Curiosity & 4.05 & 3.55 & 3.51 & 3.13 & 3.09 & 3.08 & 2.63 \\
Aesthetic Sensitivity  & 3.89 & 3.63 & 3.47 & 3.25 & 3.01 & 3.12 & 2.67 \\
Creative Imagination   & 3.96 & 3.72 & 3.73 & 3.29 & 3.16 & 3.17 & 2.44 \\
\bottomrule
\end{tabular}
\caption{Facet Level Mean for BFI-2 Human Responses and Different Methods LLM Responses \\
\textit{Note:} \textit{n} = 1,559 for human responses, \textit{n} = 297 for Shape Llama3, and \textit{n} = 300 for other LLM responses. Some sample sizes are below 300, because certain generated data exceeded reasonable thresholds (1-5) for specific items and were excluded from the analysis.}
\label{tab:facet_mean}
\end{table*}






\begin{table*}[ht]
\centering
\normalsize
\begin{tabular}{@{}lcccccccc@{}}
\toprule
\multirow{2}{*}{\textbf{Facet}} & \multirow{2}{*}{\textbf{Human}} & \multicolumn{2}{c}{\textbf{Persona}} & \multicolumn{2}{c}{\textbf{Shape}} & \multicolumn{2}{c}{\textbf{PSI}} \\ 
\cmidrule(lr){3-4} \cmidrule(lr){5-6} \cmidrule(lr){7-8}
& & \textbf{GPT-4o} & \textbf{Llama3} & \textbf{GPT-4o} & \textbf{Llama3} & \textbf{GPT-4o} & \textbf{Llama3} \\ 
\midrule
Sociability            & 1.17 & 0.69 & 0.98 & 0.95 & 1.00 & 0.91 & 0.74 \\
Assertiveness          & 1.04 & 0.55 & 0.82 & 1.10 & 1.22 & 0.57 & 0.42 \\
Energy Level           & 0.95 & 0.75 & 0.88 & 1.12 & 1.19 & 0.77 & 0.64 \\
Compassion             & 0.81 & 0.69 & 0.81 & 1.18 & 1.36 & 0.69 & 0.76 \\
Respectfulness         & 0.70 & 0.58 & 0.74 & 1.13 & 1.33 & 0.61 & 0.68 \\
Trust                  & 1.00 & 0.44 & 0.67 & 1.17 & 1.34 & 0.77 & 0.62 \\
Organization           & 1.01 & 0.57 & 0.79 & 0.98 & 1.18 & 0.63 & 0.68 \\
Productiveness         & 0.91 & 0.65 & 0.87 & 1.12 & 1.29 & 0.68 & 0.68 \\
Responsibility         & 0.84 & 0.59 & 0.95 & 1.10 & 1.26 & 0.64 & 0.74 \\
Anxiety                & 1.11 & 0.54 & 0.89 & 0.95 & 1.04 & 0.88 & 0.75 \\
Depression             & 1.14 & 0.56 & 0.89 & 1.16 & 1.26 & 0.87 & 0.81 \\
Emotional Volatility   & 1.10 & 0.43 & 0.92 & 1.06 & 1.25 & 0.76 & 0.87 \\
Intellectual Curiosity & 0.83 & 0.68 & 0.96 & 1.04 & 1.24 & 0.89 & 0.70 \\
Aesthetic Sensitivity  & 1.01 & 0.75 & 1.09 & 1.12 & 1.26 & 0.96 & 0.83 \\
Creative Imagination   & 0.88 & 0.57 & 0.74 & 1.05 & 1.17 & 0.67 & 0.60 \\
\bottomrule
\end{tabular}
\caption{Facet Level Standard Deviation for BFI-2 Human Responses and Different Methods LLM Responses \\
\textit{Note:} \textit{n} = 1,559 for human responses, \textit{n} = 297 for Shape Llama3, and \textit{n} = 300 for other LLM responses. Some sample sizes are below 300, because certain generated data exceeded reasonable thresholds (1-5) for specific items and were excluded from the analysis.}
\label{tab:facet_sd}
\end{table*}






\begin{table*}[ht]
\centering
\normalsize
\begin{tabular}{@{}lcccccccc@{}}
\toprule
\multirow{2}{*}{\textbf{Domain}} & \multirow{2}{*}{\textbf{Human}} & \multicolumn{2}{c}{\textbf{Persona}} & \multicolumn{2}{c}{\textbf{Shape}} & \multicolumn{2}{c}{\textbf{PSI}} \\ 
\cmidrule(lr){3-4} \cmidrule(lr){5-6} \cmidrule(lr){7-8}
& & \textbf{GPT-4o} & \textbf{Llama3} & \textbf{GPT-4o} & \textbf{Llama3} & \textbf{GPT-4o} & \textbf{Llama3} \\ 
\midrule
Extraversion       & 2.99 & 3.44 & 3.42 & 3.03 & 2.90 & 2.72 & 2.52 \\
Agreeableness      & 3.87 & 3.39 & 4.11 & 3.37 & 3.43 & 3.89 & 3.89 \\
Conscientiousness  & 3.89 & 3.33 & 3.54 & 3.13 & 3.10 & 3.47 & 3.35 \\
Neuroticism        & 2.70 & 2.99 & 2.59 & 2.94 & 2.75 & 3.02 & 2.68 \\
Openness           & 3.97 & 3.63 & 3.57 & 3.22 & 3.09 & 3.12 & 2.58 \\
\bottomrule
\end{tabular}%
\caption{Domain Level Mean for BFI-2 Human Responses and Different Methods LLM Responses \\
\textit{Note:} \textit{n} = 1,559 for human responses, \textit{n} = 297 for Shape Llama3, and \textit{n} = 300 for other LLM responses. Some sample sizes are below 300, because certain generated data exceeded reasonable thresholds (1-5) for specific items and were excluded from the analysis.}
\label{tab:domain_mean}
\end{table*}







\begin{table*}[ht]
\centering
\normalsize
\begin{tabular}{@{}lcccccccc@{}}
\toprule
\multirow{2}{*}{\textbf{Domain}} & \multirow{2}{*}{\textbf{Human}} & \multicolumn{2}{c}{\textbf{Persona}} & \multicolumn{2}{c}{\textbf{Shape}} & \multicolumn{2}{c}{\textbf{PSI}} \\ 
\cmidrule(lr){3-4} \cmidrule(lr){5-6} \cmidrule(lr){7-8}
& & \textbf{GPT-4o} & \textbf{Llama3} & \textbf{GPT-4o} & \textbf{Llama3} & \textbf{GPT-4o} & \textbf{Llama3} \\ 
\midrule
Extraversion       & 0.85 & 0.57 & 0.81 & 0.95 & 1.00 & 0.67 & 0.52 \\
Agreeableness      & 0.69 & 0.51 & 0.70 & 1.11 & 1.28 & 0.62 & 0.62 \\
Conscientiousness  & 0.80 & 0.53 & 0.79 & 1.00 & 1.15 & 0.59 & 0.64 \\
Neuroticism        & 1.02 & 0.43 & 0.81 & 0.97 & 1.09 & 0.77 & 0.72 \\
Openness           & 0.76 & 0.58 & 0.78 & 1.03 & 1.15 & 0.72 & 0.59 \\
\bottomrule
\end{tabular}%
\caption{Domain Level Standard Deviation for BFI-2 Human Responses and Different Methods LLM Responses \\
\textit{Note:} \textit{n} = 1,559 for human responses, \textit{n} = 297 for Shape Llama3, and \textit{n} = 300 for other LLM responses. Some sample sizes are below 300, because certain generated data exceeded reasonable thresholds (1-5) for specific items and were excluded from the analysis.}
\label{tab:domain_sd}
\end{table*}



\paragraph{Additional Psychometric Performance Results}
Here we show the TFM fit information in Table~\ref{tab:model_fit_tfm}; FFM fit information in Table~\ref{tab:model_fit_ffm}; TCC results for the TFM of each BFI-2 domain in Table~\ref{tab:tcc_tfm}; TCC results for the FFM in Table~\ref{tab:tcc_ffm}; specific standardized factor loading results in Tables~\ref{tab:fc_tfm} and~\ref{tab:fc_ffm}; specific inter-factor correlation results in Tables~\ref{tab:ifc_tfm} and~\ref{tab:ifc_ffm}.




\setlength\tabcolsep{14pt}
\begin{table*}[ht]
\centering
\footnotesize
\begin{tabular}{llcccccc}
\toprule
\textbf{Domain} & \textbf{Model} & $\mathbf{\chi^2}$ & \textbf{\textit{df}} & \textbf{CFI} & \textbf{TLI} & \textbf{RMSEA} & \textbf{SRMR} \\ \midrule

\multirow{10}{*}{Extraversion} & Human & 993.189 & 51 & .892 & .861 & .109 & .057 \\

& \multicolumn{7}{>{\columncolor[rgb]{ .949,  .953,  .961}}c}{Persona} \\

& GPT-4o & 261.099 & 51 & .886 & .852 & .117 & .058\\
& Llama3 & 316.715 & 51 & .885 & .851 & .132 & .060 \\

& \multicolumn{7}{>{\columncolor[rgb]{ .949,  .953,  .961}}c}{Shape} \\
					
& GPT-4o & 663.947 & 51 & .823 & .771 & .200 & .101 \\
& Llama3 & 934.002 & 51 & .684 & .592 & .241 & .132 \\

& \multicolumn{7}{>{\columncolor[rgb]{ .949,  .953,  .961}}c}{PSI} \\
										
& GPT-4o & 329.341 & 51 & .892 & .860 & .135 & .057 \\
& Llama3 & 332.944 & 51 & .810 & .755 & .136 & .092 \\ \midrule

\multirow{10}{*}{Agreeableness} & Human & 904.640 & 51 & .875 & .838 & .104 & .060 \\

& \multicolumn{7}{>{\columncolor[rgb]{ .949,  .953,  .961}}c}{Persona} \\

& GPT-4o & 210.863 & 51 & .900 & .870 & .102 & .062 \\
& Llama3 & 616.199 & 51 & .798 & .739 & .192 & .088 \\

& \multicolumn{7}{>{\columncolor[rgb]{ .949,  .953,  .961}}c}{Shape} \\
					
& GPT-4o & 525.799 & 51 & .909 & .882 & .176 & .039 \\
& Llama3 & 874.914 & 51 & .840 & .793 & .233 & .070 \\

& \multicolumn{7}{>{\columncolor[rgb]{ .949,  .953,  .961}}c}{PSI} \\
										
& GPT-4o & 177.494 & 51 & .952 & .937 & .091 & .049 \\
& Llama3 & 255.684 & 51 & .923 & .900 & .116 & .050 \\ \midrule

\multirow{10}{*}{Conscientiousness} & Human & 1041.784 & 51 & .897 & .867 & .112 & .058 \\

& \multicolumn{7}{>{\columncolor[rgb]{ .949,  .953,  .961}}c}{Persona} \\
										
& GPT-4o & 385.339 & 51 & .780 & .716 & .148 & .088 \\
& Llama3 & 705.359 & 51 & .768 & .699 & .207 & .109 \\

& \multicolumn{7}{>{\columncolor[rgb]{ .949,  .953,  .961}}c}{Shape} \\
					
& GPT-4o & 705.340 & 51 & .826 & .775 & .207 & .069 \\
& Llama3 & 1502.901 & 51 & .642 & .536 & .310 & .196 \\

& \multicolumn{7}{>{\columncolor[rgb]{ .949,  .953,  .961}}c}{PSI} \\
										
& GPT-4o & 348.206 & 51 & .887 & .854 & .139 & .052 \\
& Llama3 & 427.908 & 51 & .834 & .785 & .157 & .069 \\ \midrule

\multirow{10}{*}{Neuroticism} & Human & 929.871 & 51 & .931 & .911 & .105 & .052 \\

& \multicolumn{7}{>{\columncolor[rgb]{ .949,  .953,  .961}}c}{Persona} \\
					
& GPT-4o & 304.611 & 51 & .756 & .684 & .129 & .091 \\
& Llama3 & 346.207 & 51 & .884 & .850 & .139 & .059 \\

& \multicolumn{7}{>{\columncolor[rgb]{ .949,  .953,  .961}}c}{Shape} \\
					
& GPT-4o & 960.437 & 51 & .720 & .638 & .244 & .099 \\
& Llama3 & 808.305 & 51 & .772 & .705 & .224 & .101 \\

& \multicolumn{7}{>{\columncolor[rgb]{ .949,  .953,  .961}}c}{PSI} \\
										
& GPT-4o & 448.933 & 51 & .876 & .840 & .161 & .069 \\
& Llama3 & 448.871 & 51 & .854 & .811 & .161 & .067 \\ \midrule

\multirow{10}{*}{Openness} & Human & 909.210 & 51 & .899 & .870 & .104 & .064 \\

& \multicolumn{7}{>{\columncolor[rgb]{ .949,  .953,  .961}}c}{Persona} \\
					
& GPT-4o & 226.649 & 51 & .910 & .883 & .107 & .057 \\
& Llama3 & 256.032 & 51 & .904 & .876 & .116 & .077 \\

& \multicolumn{7}{>{\columncolor[rgb]{ .949,  .953,  .961}}c}{Shape} \\
				
& GPT-4o & 502.104 & 51 & .878 & .842 & .172 & .055 \\
& Llama3 & 569.466 & 51 & .861 & .820 & .185 & .057 \\

& \multicolumn{7}{>{\columncolor[rgb]{ .949,  .953,  .961}}c}{PSI} \\
										
& GPT-4o & 310.479 & 51 & .904 & .876 & .130 & .071 \\
& Llama3 & 352.224 & 51 & .854 & .811 & .140 & .114 \\ \bottomrule

\end{tabular}
\caption{Model Fits for BFI-2 Three-Factor Models of Each Domain \\
\textit{Note:} \textit{n} = 1,559 for human responses, \textit{n} = 297 for Shape Llama3, and \textit{n} = 300 for other LLM responses. Some sample sizes are below 300, because certain generated data exceeded reasonable thresholds (1-5) for specific items and were excluded from the analysis.}
\label{tab:model_fit_tfm}
\end{table*}


% \setlength\tabcolsep{14pt}
\begin{table*}[h]
\footnotesize
  \centering
\resizebox{\textwidth}{!}{%
    \begin{tabular}{llcccccc}
\toprule
\textbf{Model} &  \(\mathbf{\chi^2}\) & \textbf{\textit{df}} & \textbf{CFI} & \textbf{TLI} & \textbf{RMSEA} & \textbf{SRMR} \\
\midrule
Human & 1747.433 & 80 & .852 & .806 & .116 & .080 \\

\rowcolor[rgb]{ .949,  .953,  .961} \multicolumn{7}{c}{Persona} \\

GPT-4o & 551.459 & 80 & .819 & .762 & .140 & .097 \\
Llama3 & 954.848 & 80 & .780 & .711 & .191 & .120 \\
 
\rowcolor[rgb]{ .949,  .953,  .961} \multicolumn{7}{c}{Shape} \\

GPT-4o & 2132.208 & 80 & .711 & .621 & .292 & .143 \\
Llama3 & 1988.584 & 80 & .717 & .629 & .283 & .149 \\
 
\rowcolor[rgb]{ .949,  .953,  .961} \multicolumn{7}{c}{PSI} \\

GPT-4o & 899.057 & 80 & .769 & .697 & .185 & .119 \\
Llama3 & 802.352 & 80 & .774 & .703 & .173 & .114 \\
\bottomrule

\end{tabular}
}
\caption{Model Fits for BFI-2 Five-Factor Model \\
\textit{Note:} \textit{n} = 1,559 for human responses, \textit{n} = 297 for Shape Llama3, and \textit{n} = 300 for other LLM responses. Some sample sizes are below 300, because certain generated data exceeded reasonable thresholds (1-5) for specific items and were excluded from the analysis.}
\label{tab:model_fit_ffm}
\end{table*}


\begin{table*}[ht]
\centering
\resizebox{\textwidth}{!}{%
\begin{tabular}{llcccccc}
\toprule
\multirow{2}{*}{\textbf{Domain}} & \multirow{2}{*}{\textbf{Facet}} & \multicolumn{2}{c}{\textbf{Persona}} & \multicolumn{2}{c}{\textbf{Shape}} & \multicolumn{2}{c}{\textbf{PSI}} \\ 
\cmidrule(lr){3-4} \cmidrule(lr){5-6} \cmidrule(lr){7-8}
& & \textbf{GPT-4o} & \textbf{Llama3} & \textbf{GPT-4o} & \textbf{Llama3} & \textbf{GPT-4o} & \textbf{Llama3} \\
\midrule
\multirow{3}{*}{Extraversion} & Sociability & 1.00 & .99 & 1.00 & .99 & 1.00 & .99 \\
                            & Assertiveness & .97 & .98 & .99 & .97 & .98 & .97 \\
                             & Energy Level & .98 & .98 & .97 & .96 & .98 & .98 \\
\midrule
\multirow{3}{*}{Agreeableness} & Compassion & .95 & .96 & .97 & .95 & .97 & 1.00 \\
                           & Respectfulness & 1.00 & 1.00 & 1.00 & 1.00 & .98 & .99 \\
                                    & Trust & .98 & .99 & 1.00 & .99 & 1.00 & .99 \\
\midrule
\multirow{3}{*}{Conscientiousness} & Organization & .99 & 1.00 & .99 & .95 & .99 & 1.00 \\
                                 & Productiveness & .99 & .99 & .99 & .99 & 1.00 & .97 \\
                                 & Responsibility & .99 & .98 & .99 & .96 & 1.00 & .99 \\
\midrule
\multirow{3}{*}{Neuroticism} & Anxiety & .99 & .99 & .98 & .99 & 1.00 & 1.00 \\
                          & Depression & .97 & .99 & .99 & .99 & 1.00 & 1.00 \\
                & Emotional Volatility & .97 & 1.00 & .99 & .99 & 1.00 & 1.00 \\
\midrule
\multirow{3}{*}{Openness} & Intellectual Curiosity & 1.00 & 1.00 & .99 & .98 & 1.00 & .98 \\
                   & Aesthetic Sensitivity & .99 & .98 & .99 & .98 & .99 & .97 \\
                   & Creative Imagination & 1.00 & .98 & 1.00 & 1.00 & .99 & .96 \\
\bottomrule
\end{tabular}
}
\caption{Tucker’s Congruence Coefficient for BFI-2 Three-Factor Models of Each Domain \\
\textit{Note:} \textit{n} = 1,559 for human responses, \textit{n} = 297 for Shape Llama3, and \textit{n} = 300 for other LLM responses. Some sample sizes are below 300, because certain generated data exceeded reasonable thresholds (1-5) for specific items and were excluded from the analysis.}
\label{tab:tcc_tfm}
\end{table*}



\begin{table*}[ht]
\centering
\resizebox{\textwidth}{!}{%
\begin{tabular}{lcccccc}
\toprule
\multirow{2}{*}{\textbf{Domain}} & \multicolumn{2}{c}{\textbf{Persona}} & \multicolumn{2}{c}{\textbf{Shape}} & \multicolumn{2}{c}{\textbf{PSI}} \\ 
\cmidrule(lr){2-3} \cmidrule(lr){4-5} \cmidrule(lr){6-7}
& \textbf{GPT-4o} & \textbf{Llama3} & \textbf{GPT-4o} & \textbf{Llama3} & \textbf{GPT-4o} & \textbf{Llama3} \\
\midrule
Extraversion & 1.00 & .99 & 1.00 & 1.00 & 1.00 & 1.00 \\
Agreeableness & 1.00 & 1.00 & .99 & .99 & 1.00 & 1.00 \\
Conscientiousness & .99 & .98 & .99 & .99 & .99 & .99 \\
Neuroticism & .99 & 1.00 & .99 & 1.00 & 1.00 & 1.00 \\
Openness & 1.00 & .99 & 1.00 & .99 & 1.00 & .99 \\
\bottomrule
\end{tabular}
}
\caption{Tucker’s Congruence Coefficient for BFI-2 Five-Factor Model \\
\textit{Note:} \textit{n} = 1,559 for human responses, \textit{n} = 297 for Shape Llama3, and \textit{n} = 300 for other LLM responses. Some sample sizes are below 300, because certain generated data exceeded reasonable thresholds (1-5) for specific items and were excluded from the analysis.}
\label{tab:tcc_ffm}
\end{table*}



\begin{table*}[ht]
\centering
\resizebox{\textwidth}{!}{%
\begin{tabular}{llccccccc}
\toprule
\multirow{2}{*}{\textbf{Domain}} & \multirow{2}{*}{\textbf{Facet=\textasciitilde Item}} & \multirow{2}{*}{\textbf{Human}} & \multicolumn{2}{c}{\textbf{Persona}} & \multicolumn{2}{c}{\textbf{Shape}} & \multicolumn{2}{c}{\textbf{PSI}} \\ 
\cmidrule(lr){4-5} \cmidrule(lr){6-7} \cmidrule(lr){8-9}
& & & \textbf{GPT-4o} & \textbf{Llama3} & \textbf{GPT-4o} & \textbf{Llama3} & \textbf{GPT-4o} & \textbf{Llama3} \\
\midrule
\multirow{12}{*}{Extraversion} 
& Sociability=\textasciitilde item1  & .80 & .77 & .72 & .85 & \textbf{.53} & .87 & \textit{.63} \\
& Sociability=\textasciitilde item16 & .82 & .79 & .88 & .74 & .75 & .84 & .88 \\
& Sociability=\textasciitilde item31 & .82 & .79 & .87 & .73 & .80 & .91 & .76 \\
& Sociability=\textasciitilde item46 & .76 & .74 & \textit{.64} & .85 & \textit{.57} & .76 & .67 \\
& Assertiveness=\textasciitilde item6  & .76 & \textit{.66} & .84 & \textit{.94} & .85 & .81 & .69 \\
& Assertiveness=\textasciitilde item21 & .89 & \textbf{.61} & \textit{.73} & .90 & \textbf{.69} & .80 & \textbf{.69} \\
& Assertiveness=\textasciitilde item36 & .53 & \textbf{.73} & \textit{.72} & \textbf{.84} & \textbf{.88} & \textbf{.76} & .44 \\
& Assertiveness=\textasciitilde item51 & .69 & \textit{.59} & \textit{.54} & \textit{.80} & .73 & \textit{.55} & \textbf{.27} \\
& Energy Level=\textasciitilde item11 & .45 & \textbf{.82} & \textbf{.73} & \textbf{.91} & \textbf{.89} & \textbf{.82} & \textbf{.66} \\
& Energy Level=\textasciitilde item26 & .54 & \textit{.69} & \textbf{.75} & \textit{.72} & \textbf{.78} & \textbf{.74} & .47 \\
& Energy Level=\textasciitilde item41 & .83 & .81 & .87 & .82 & .76 & .87 & \textit{.71} \\
& Energy Level=\textasciitilde item56 & .77 & .82 & .78 & \textit{.88} & .77 & .80 & .84 \\
\midrule
\multirow{12}{*}{Agreeableness} 
& Compassion=\textasciitilde item2  & .73 & .76 & .75 & \textit{.90} & .74 & \textit{.84} & \textit{.83} \\
& Compassion=\textasciitilde item17 & .34 & \textbf{.84} & \textbf{.86} & \textbf{.93} & \textbf{.97} & \textbf{.79} & \textit{.46} \\
& Compassion=\textasciitilde item32 & .62 & \textit{.74} & \textit{.74} & \textbf{.87} & \textbf{.86} & \textit{.80} & \textit{.79} \\
& Compassion=\textasciitilde item47 & .75 & .67 & \textit{.85} & \textbf{.96} & \textbf{.98} & \textit{.89} & \textit{.86} \\
& Respectfulness=\textasciitilde item7  & .71 & \textit{.82} & .80 & \textbf{.93} & \textit{.84} & \textit{.90} & \textbf{.91} \\
& Respectfulness=\textasciitilde item22 & .60 & .54 & .68 & \textbf{.80} & \textbf{.87} & \textit{.43} & \textit{.48} \\
& Respectfulness=\textasciitilde item37 & .70 & .74 & .79 & \textbf{.94} & \textbf{.94} & \textit{.80} & \textit{.82} \\
& Respectfulness=\textasciitilde item52 & .70 & .70 & \textit{.80} & \textit{.83} & \textit{.81} & \textit{.85} & \textit{.86} \\
& Trust=\textasciitilde item12 & .69 & .63 & \textbf{.35} & \textit{.57} & \textbf{.95} & \textbf{.95} & .78 \\
& Trust=\textasciitilde item27 & .68 & .60 & \textit{.80} & \textit{.84} & \textit{.84} & \textit{.79} & \textit{.84} \\
& Trust=\textasciitilde item42 & .67 & \textbf{.35} & .63 & \textbf{.95} & \textbf{.95} & \textit{.78} & \textit{.83} \\
& Trust=\textasciitilde item57 & .78 & \textit{.64} & .82 & \textit{.90} & .85 & \textit{.88} & \textit{.89} \\
\midrule
\multirow{12}{*}{Conscientiousness} 
& Organization=\textasciitilde item3  & .86 & \textbf{.64} & \textit{.75} & \textit{.75} & \textbf{.48} & .83 & .83 \\
& Organization=\textasciitilde item18 & .70 & .62 & .66 & .73 & \textbf{.90} & \textit{.83} & .65 \\
& Organization=\textasciitilde item33 & .86 & \textbf{.63} & \textit{.68} & .95 & \textbf{.55} & \textit{.74} & \textit{.74} \\
& Organization=\textasciitilde item48 & .74 & .72 & .72 & .68 & \textit{.55} & .80 & .79 \\
& Productiveness=\textasciitilde item8  & .66 & .67 & .68 & .64 & .67 & .73 & \textbf{.43} \\
& Productiveness=\textasciitilde item23 & .86 & \textit{.71} & \textbf{.65} & .81 & \textit{.70} & .78 & \textbf{.43} \\
& Productiveness=\textasciitilde item38 & .87 & \textit{.72} & .88 & .82 & .85 & .85 & \textbf{.52} \\
& Productiveness=\textasciitilde item53 & .65 & \textit{.47} & .66 & \textit{.80} & .63 & \textit{.81} & \textit{.82} \\
& Responsibility=\textasciitilde item13 & .72 & .67 & \textbf{.94} & \textbf{.92} & \textbf{.97} & .79 & \textit{.83} \\
& Responsibility=\textasciitilde item28 & .77 & \textit{.58} & \textit{.58} & .68 & \textbf{.46} & \textit{.59} & .71 \\
& Responsibility=\textasciitilde item43 & .66 & \textit{.51} & .74 & .75 & .70 & .68 & .57 \\
& Responsibility=\textasciitilde item58 & .80 & \textbf{.59} & .80 & .75 & .87 & .71 & .74 \\
\midrule
\multirow{12}{*}{Neuroticism} 
& Anxiety=\textasciitilde item4  & .60 & \textit{.71} & \textit{.72} & .68 & .62 & \textbf{.80} & \textit{.79} \\
& Anxiety=\textasciitilde item19 & .63 & .65 & .61 & .65 & .72 & \textit{.78} & \textbf{.43} \\
& Anxiety=\textasciitilde item34 & .65 & \textit{.47} & \textit{.81} & .73 & \textit{.79} & .59 & .65 \\
& Anxiety=\textasciitilde item49 & .74 & .71 & \textit{.84} & \textit{.85} & .70 & \textit{.92} & \textit{.89} \\
& Depression=\textasciitilde item9  & .66 & .67 & .60 & \textit{.81} & \textit{.77} & \textbf{.88} & \textbf{.44} \\
& Depression=\textasciitilde item24 & .86 & .82 & \textbf{.59} & .81 & .78 & .92 & .91 \\
& Depression=\textasciitilde item39 & .87 & .86 & \textbf{.65} & \textbf{.47} & .86 & .95 & .95 \\
& Depression=\textasciitilde item54 & .65 & .69 & \textbf{.88} & \textbf{.92} & \textbf{.95} & \textit{.79} & \textbf{.85} \\
& Emotional Volatility=\textasciitilde item14 & .72 & .77 & .72 & \textit{.59} & \textit{.88} & \textit{.84} & \textit{.86} \\
& Emotional Volatility=\textasciitilde item29 & .77 & \textit{.66} & \textbf{.51} & .74 & .75 & .70 & .68 \\
& Emotional Volatility=\textasciitilde item44 & .66 & .64 & \textbf{.27} & .68 & .75 & .60 & .67 \\
& Emotional Volatility=\textasciitilde item59 & .80 & \textit{.67} & .72 & .83 & .79 & \textit{.65} & .85 \\
\midrule
\multirow{12}{*}{Openness} 
& Intellectual Curiosity=\textasciitilde item10 & .60 & \textbf{.90} & \textbf{.81} & \textbf{.84} & \textbf{.91} & \textbf{.92} & \textbf{.89} \\
& Intellectual Curiosity=\textasciitilde item25 & .63 & \textbf{.91} & .67 & \textbf{.88} & \textbf{.89} & \textbf{.95} & \textbf{.94} \\
& Intellectual Curiosity=\textasciitilde item40 & .65 & \textit{.80} & .59 & .70 & .57 & \textbf{.85} & \textbf{.85} \\
& Intellectual Curiosity=\textasciitilde item55 & .74 & .77 & .81 & \textit{.84} & \textit{.89} & \textit{.88} & \textit{.56} \\
& Aesthetic Sensitivity=\textasciitilde item5  & .66 & .66 & \textit{.84} & \textit{.85} & \textbf{.91} & .71 & .74 \\
& Aesthetic Sensitivity=\textasciitilde item20 & .86 & .77 & .81 & \textbf{.65} & .83 & .79 & .85 \\
& Aesthetic Sensitivity=\textasciitilde item35 & .87 & \textbf{.59} & \textit{.70} & .84 & .81 & .92 & .88 \\
& Aesthetic Sensitivity=\textasciitilde item50 & .65 & .63 & .69 & \textbf{.41} & \textbf{.91} & .63 & \textbf{.32} \\
& Creative Imagination=\textasciitilde item15 & .72 & .77 & \textbf{.40} & .71 & .81 & \textbf{.95} & .79 \\
& Creative Imagination=\textasciitilde item30 & .77 & .80 & \textit{.59} & .70 & .81 & .86 & .82 \\
& Creative Imagination=\textasciitilde item45 & .66 & \textit{.84} & .68 & \textit{.49} & \textit{.77} & \textbf{.92} & \textbf{.35} \\
& Creative Imagination=\textasciitilde item60 & .80 & .86 & \textit{.70} & \textit{.95} & .81 & .84 & \textit{.95} \\
\bottomrule
\end{tabular}%
}
\caption{Standardized Factor Loadings for BFI-2 Three-Factor Models of Each Domain with Human Responses \\
\textit{Note:} \textit{n} = 1,559 for human responses, \textit{n} = 297 for Shape Llama3, and \textit{n} = 300 for other LLM responses. Some sample sizes are below 300, because certain generated data exceeded reasonable thresholds (1-5) for specific items and were excluded from the analysis. Italics for absolute differences compared to the human responses of .100 to .199, and boldface for differences of .200 or higher.}
\label{tab:fc_tfm}
\end{table*}







\begin{table*}[ht]
\centering
\resizebox{\textwidth}{!}{%
\normalsize
\begin{tabular}{@{}lcccccccc@{}}
\toprule
\multirow{2}{*}{\textbf{Domain=\textasciitilde Facet}} & \multirow{2}{*}{\textbf{Human}} & \multicolumn{2}{c}{\textbf{Persona}} & \multicolumn{2}{c}{\textbf{Shape}} & \multicolumn{2}{c}{\textbf{PSI}} \\ 
\cmidrule(lr){3-4} \cmidrule(lr){5-6} \cmidrule(lr){7-8}
& & \textbf{GPT-4o} & \textbf{Llama3} & \textbf{GPT-4o} & \textbf{Llama3} & \textbf{GPT-4o} & \textbf{Llama3} \\ 
\midrule
E=\textasciitilde Sociability          & .66 
  & \textit{.76} & \textit{.85} & \textbf{.91} & \textbf{.87} & \textit{.81} & \textit{.78} \\
E=\textasciitilde Assertiveness        & .59 
  & .67 & \textbf{.86} & \textit{.77} & \textit{.73} & \textit{.71} & .68 \\
E=\textasciitilde Energy Level         & .77 
  & \textit{.88} & .86 & \textit{.89} & .85 & \textit{.92} & \textit{.89} \\
A=\textasciitilde Compassion           & .70 
  & \textit{.82} & \textbf{.90} & \textbf{.96} & \textbf{.99} & \textit{.82} & .79 \\
A=\textasciitilde Respectfulness       & .80 
  & \textit{.90} & \textit{.94} & \textit{.90} & .85 & .88 & \textit{.94} \\
A=\textasciitilde Trust                & .65 
  & \textit{.75} & \textbf{.87} & \textbf{.93} & \textbf{.95} & \textit{.84} & \textit{.84} \\
C=\textasciitilde Organization         & .70 
  & \textit{.87} & \textit{.87} & \textbf{.94} & \textbf{.97} & \textit{.87} & \textit{.86} \\
C=\textasciitilde Productiveness       & .89 
  & \textit{.77} & \textit{.75} & .84 & .81 & .84 & .82 \\
C=\textasciitilde Responsibility       & .79 
  & .82 & \textit{.95} & \textit{.93} & \textit{.92} & .88 & \textit{.91} \\
N=\textasciitilde Anxiety              & .84 
  & \textit{.69} & .75 & .80 & .86 & .77 & .77 \\
N=\textasciitilde Depression           & .88 
  & .91 & .96 & \textit{1.02} & .96 & .97 & .87 \\
N=\textasciitilde Emotional Volatility & .85 
  & \textbf{.63} & .79 & \textit{.71} & .80 & .78 & .85 \\
O=\textasciitilde Intellectual Curiosity & .74 
  & .77 & .66 & \textbf{.97} & \textit{.93} & .74 & \textit{.85} \\
O=\textasciitilde Aesthetic Sensitivity  & .71 
  & .73 & .65 & \textbf{.95} & \textbf{.96} & .69 & \textit{.59} \\
O=\textasciitilde Creative Imagination   & .81 
  & \textit{.91} & .89 & .90 & .82 & \textit{.96} & .76 \\
\bottomrule
\end{tabular}%
}
\caption{Standardized Factor Loadings for BFI-2 Five-Factor Model \\
\textit{Note:} \textit{n} = 1,559 for human responses, \textit{n} = 297 for Shape Llama3, and \textit{n} = 300 for other LLM responses. 
Some sample sizes are below 300, because certain generated data exceeded reasonable thresholds (1-5) for specific items and were excluded from the analysis.
Italics for absolute differences compared to the human responses of .100 to .199, and boldface for differences of .200 or higher.}
\label{tab:fc_ffm}
\end{table*}








\begin{table*}[ht]
\centering
\resizebox{\textwidth}{!}{%
\begin{tabular}{llccccccc}
\toprule
\multirow{2}{*}{\textbf{Domain}} & \multirow{2}{*}{\textbf{Facet\textasciitilde\textasciitilde Facet}} & \multirow{2}{*}{\textbf{Human}} & \multicolumn{2}{c}{\textbf{Persona}} & \multicolumn{2}{c}{\textbf{Shape}} & \multicolumn{2}{c}{\textbf{PSI}} \\ 
\cmidrule(lr){4-5} \cmidrule(lr){6-7} \cmidrule(lr){8-9}
& & & \textbf{GPT-4o} & \textbf{Llama3} & \textbf{GPT-4o} & \textbf{Llama3} & \textbf{GPT-4o} & \textbf{Llama3} \\
\midrule

\multirow{3}{*}{Extraversion} 
& Sociability\textasciitilde\textasciitilde Assertiveness  & .59 & \textit{.72} & \textbf{.89} & \textit{.77} & \textbf{.91} & \textit{.69} & \textit{.70} \\
& Sociability\textasciitilde\textasciitilde Energy Level   & .64 & \textit{.79} & \textbf{.84} & \textbf{.91} & \textit{.80} & \textbf{.84} & \textbf{.85} \\
& Assertiveness\textasciitilde\textasciitilde Energy Level & .47 & \textit{.66} & \textbf{.82} & \textit{.65} & \textit{.61} & \textbf{.72} & \textit{.65} \\
\midrule

\multirow{3}{*}{Agreeableness}
& Compassion\textasciitilde\textasciitilde Respectfulness  & .81 & .90 & \textit{.98} & \textit{.92} & .87 & .85 & .90 \\
& Compassion\textasciitilde\textasciitilde Trust          & .70 & \textit{.83} & \textbf{.94} & \textbf{.94} & \textbf{.96} & .79 & \textit{.84} \\
& Respectfulness\textasciitilde\textasciitilde Trust      & .58 & \textbf{.95} & \textbf{.96} & \textbf{.93} & \textbf{.84} & \textbf{.85} & \textbf{.86} \\
\midrule

\multirow{3}{*}{Conscientiousness}
& Organization\textasciitilde\textasciitilde Productiveness  & .75 & \textit{.86} & .81 & \textbf{.97} & .70 & \textit{.86} & \textit{.86} \\
& Organization\textasciitilde\textasciitilde Responsibility   & .70 & \textbf{1.01} & \textit{.85} & \textbf{.99} & \textit{.86} & \textit{.89} & \textit{.87} \\
& Productiveness\textasciitilde\textasciitilde Responsibility & .86 & .82 & .87 & .87 & .89 & .82 & .87 \\
\midrule

\multirow{3}{*}{Neuroticism}
& Anxiety\textasciitilde\textasciitilde Depression          & .81 & .88 & .88 & \textit{.95} & \textit{.96} & .86 & .79 \\
& Anxiety\textasciitilde\textasciitilde Emotional Volatility & .90 & .87 & \textit{.79} & .87 & .87 & .92 & .83 \\
& Depression\textasciitilde\textasciitilde Emotional Volatility & .79 & .84 & .85 & \textit{.89} & .83 & .86 & .82 \\
\midrule

\multirow{3}{*}{Openness}
& Intellectual Curiosity\textasciitilde\textasciitilde Aesthetic Sensitivity & .66 & .68 & .59 & \textbf{1.02} & \textbf{1.00} & \textit{.51} & \textit{.51} \\
& Intellectual Curiosity\textasciitilde\textasciitilde Creative Imagination  & .74 & \textit{.85} & .67 & \textbf{.99} & \textit{.89} & .78 & .72 \\
& Aesthetic Sensitivity\textasciitilde\textasciitilde Creative Imagination   & .63 & \textit{.79} & .60 & \textbf{.93} & \textbf{.86} & \textit{.77} & \textit{.50} \\
\bottomrule
\end{tabular}%
}
\caption{Inter-factor Correlations for BFI-2 Three-Factor Models of Each Domain with Human Responses \\
\textit{Note:} \textit{n} = 1,559 for human responses, \textit{n} = 297 for Shape Llama3, and \textit{n} = 300 for other LLM responses. 
Some sample sizes are below 300, because certain generated data exceeded reasonable thresholds (1-5) for specific items and were excluded from the analysis.
Italics for absolute differences compared to the human responses of .100 to .199, and boldface for differences of .200 or higher.}
\label{tab:ifc_tfm}
\end{table*}







\begin{table*}[ht]
\centering
\resizebox{\textwidth}{!}{%
\normalsize
\begin{tabular}{@{}lcccccccc@{}}
\toprule
\multirow{2}{*}{\textbf{Domain\textasciitilde\textasciitilde Domain}} & \multirow{2}{*}{\textbf{Human}} & \multicolumn{2}{c}{\textbf{Persona}} & \multicolumn{2}{c}{\textbf{Shape}} & \multicolumn{2}{c}{\textbf{PSI}} \\ 
\cmidrule(lr){3-4} \cmidrule(lr){5-6} \cmidrule(lr){7-8}
& & \textbf{GPT-4o} & \textbf{Llama3} & \textbf{GPT-4o} & \textbf{Llama3} & \textbf{GPT-4o} & \textbf{Llama3} \\ 
\midrule
Extraversion\textasciitilde\textasciitilde Agreeableness & .28 & .26 & \textit{.10} & \textbf{.59} & \textbf{.54} & \textit{.41} & .24 \\
Extraversion\textasciitilde\textasciitilde Conscientiousness & .56 & \textit{.43} & \textbf{.26} & \textbf{.36} & \textbf{.31} & \textit{.44} & \textbf{.34} \\
Extraversion\textasciitilde\textasciitilde Neuroticism & -.61 & -.61 & -.59 & \textit{-.80} & \textit{-.72} & -.62 & \textbf{-.31} \\
Extraversion\textasciitilde\textasciitilde Openness & .35 & .37 & \textit{.51} & \textbf{.72} & \textbf{.71} & .29 & \textit{.48} \\
Agreeableness\textasciitilde\textasciitilde Conscientiousness & .47 & .56 & \textbf{.77} & \textit{.65} & \textit{.65} & \textit{.59} & \textit{.63} \\
Agreeableness\textasciitilde\textasciitilde Neuroticism & -.43 & -.40 & \textit{-.54} & \textbf{-.66} & \textbf{-.73} & -.44 & -.47 \\
Agreeableness\textasciitilde\textasciitilde Openness & .27 & .31 & .25 & \textbf{.79} & \textbf{.89} & \textit{.14} & \textit{.14} \\
Conscientiousness\textasciitilde\textasciitilde Neuroticism & -.61 & -.52 & -.63 & -.67 & \textbf{-.82} & -.58 & -.60 \\
Conscientiousness\textasciitilde\textasciitilde Openness & .17 & .10 & .16 & \textit{.27} & \textbf{.45} & \textit{.06} & .18 \\
Neuroticism\textasciitilde\textasciitilde Openness & -.17 & \textit{-.06} & -.23 & \textbf{-.48} & \textbf{-.65} & \textbf{.14} & \textbf{.18} \\
\bottomrule
\end{tabular}%
}
\caption{Inter-factor Correlations for BFI-2 Five-Factor Models with Human Responses \\
\textit{Note:} \textit{n} = 1,559 for human responses, \textit{n} = 297 for Shape Llama3, and \textit{n} = 300 for other LLM responses. 
Some sample sizes are below 300, because certain generated data exceeded reasonable thresholds (1-5) for specific items and were excluded from the analysis.
Italics for absolute differences compared to the human responses of .100 to .199, and boldface for differences of .200 or higher.}
\label{tab:ifc_ffm}
\end{table*}











\textbf{Scale Reliability:}
Facet level and domain level Cronbach’s alpha for different method LLM responses on BFI-2 and human responses are shown in Table~\ref{tab:alpha}.

It can be observed that the PSI method, compared to the Persona and Shape methods, performs closer to the results of the human sample in terms of Cronbach‘s alpha (with the number of data marked in italics and bold being the smallest). 
This indicates that the LLM personality data generated by the PSI method holds an advantage in consistency and reliability, enabling it to more accurately simulate the statistical characteristics of human samples.






\begin{table*}[ht]
\centering
\resizebox{\textwidth}{!}{%
\normalsize
\begin{tabular}{@{}lcccccccc@{}}
\toprule
\multirow{2}{*}{\textbf{Facet / Domain}} & \multirow{2}{*}{\textbf{Human}} & \multicolumn{2}{c}{\textbf{Persona}} & \multicolumn{2}{c}{\textbf{Shape}} & \multicolumn{2}{c}{\textbf{PSI}} \\ 
\cmidrule(lr){3-4} \cmidrule(lr){5-6} \cmidrule(lr){7-8}
& & \textbf{GPT-4o} & \textbf{Llama3} & \textbf{GPT-4o} & \textbf{Llama3} & \textbf{GPT-4o} & \textbf{Llama3} \\ 
\midrule
Sociability            & .87 & .85 & .86 & .87 & \textit{.77} & .91 & .82 \\
Assertiveness          & .81 & .74 & .80 & \textit{.93} & .86 & .82 & \textbf{.57} \\
Energy Level           & .73 & \textit{.86} & \textit{.86} & \textit{.90} & \textit{.87} & \textit{.88} & .76 \\
Compassion             & .67 & \textit{.84} & \textbf{.87} & \textbf{.95} & \textbf{.93} & \textbf{.90} & \textit{.82} \\
Respectfulness         & .74 & .79 & \textit{.85} & \textbf{.94} & \textit{.93} & \textit{.84} & \textit{.86} \\
Trust                  & .80 & \textit{.64} & .79 & \textit{.96} & \textit{.94} & .87 & .88 \\
Organization           & .87 & \textit{.73} & .80 & .86 & .82 & .86 & .82 \\
Productiveness         & .80 & .80 & .85 & \textit{.91} & \textit{.90} & .88 & .76 \\
Responsibility         & .77 & \textit{.67} & \textit{.88} & \textit{.91} & .86 & \textit{.88} & .86 \\
Anxiety                & .84 & \textit{.65} & .81 & .81 & .79 & .88 & .78 \\
Depression             & .86 & \textbf{.66} & .88 & .90 & .91 & .91 & .88 \\
Emotional Volatility   & .89 & \textbf{.60} & .89 & .87 & .88 & .89 & .91 \\
Intellectual Curiosity & .75 & .82 & \textit{.85} & \textit{.87} & \textit{.89} & \textit{.89} & .79 \\
Aesthetic Sensitivity  & .83 & .87 & .83 & .92 & \textit{.93} & .87 & .84 \\
Creative Imagination   & .82 & .80 & .82 & .88 & .89 & .88 & .77 \\
\midrule
Extraversion           & .87 & .90 & .93 & .95 & .92 & .93 & .87 \\
Agreeableness          & .85 & .90 & .94 & \textit{.98} & \textit{.97} & .93 & .93 \\
Conscientiousness      & .90 & .88 & .93 & .95 & .94 & .94 & .92 \\
Neuroticism            & .94 & \textit{.83} & .93 & .94 & .94 & .95 & .93 \\
Openness               & .89 & .91 & .89 & .96 & .96 & .92 & .88 \\
\bottomrule
\end{tabular}%
}
\caption{Cronbach’s alpha for BFI-2 Human Responses and Different Methods LLM Responses \\
\textit{Note:} \textit{n} = 1,559 for human responses, \textit{n} = 297 for Shape Llama3, and \textit{n} = 300 for other LLM responses. Some sample sizes are below 300, because certain generated data exceeded reasonable thresholds (1-5) for specific items and were excluded from the analysis. Italics for absolute differences compared to the human responses of .100 to .199, and boldface for differences of .200 or higher.}
\label{tab:alpha}
\end{table*}




\textbf{Discriminant Validity:}
The results for discriminant validity are shown in Table~\ref{tab:dimension_correlation}.
We can further observe that, compared to the Persona and Shape methods, the PSI method demonstrates the closest performance to human samples in terms of discriminant validity.
Specifically, the PSI method shows the closest mean of absolute values when examining its correlations with human samples.

Both higher and lower levels of external validity reveal the degree of differences between the methods and human samples.
Higher external validity indicates that the Big Five factors in human samples are more distinctly differentiated from one another, while lower external validity suggests the opposite.
Therefore, our focus here is on identifying the approach that most closely aligns with the performance of human samples.











\begin{table*}[ht]
\centering
\resizebox{\textwidth}{!}{%
\normalsize
\begin{tabular}{@{}lcccccccc@{}}
\toprule
\multirow{2}{*}{\textbf{Dimensions}} & \multirow{2}{*}{\textbf{Human}} & \multicolumn{2}{c}{\textbf{Persona}} & \multicolumn{2}{c}{\textbf{Shape}} & \multicolumn{2}{c}{\textbf{PSI}} \\ 
\cmidrule(lr){3-4} \cmidrule(lr){5-6} \cmidrule(lr){7-8}
& & \textbf{GPT-4o} & \textbf{Llama3} & \textbf{GPT-4o} & \textbf{Llama3} & \textbf{GPT-4o} & \textbf{Llama3} \\ 
\midrule
E, A   & .17 & .21 & .12 & .48 & .40 & .33 & .21 \\
E, C   & .35 & .40 & .33 & .46 & .41 & .38 & .30 \\
E, N   & -.45 & -.43 & -.42 & -.58 & -.57 & -.43 & -.21 \\
E, O   & .22 & .23 & .28 & .66 & .68 & .13 & .33 \\
A, C   & .33 & .47 & .66 & .67 & .70 & .49 & .52 \\
A, N   & -.36 & -.30 & -.54 & -.64 & -.71 & -.29 & -.37 \\
A, O   & .20 & .30 & .31 & .71 & .78 & .14 & .14 \\
C, N   & -.49 & -.43 & -.60 & -.74 & -.87 & -.47 & -.52 \\
C, O   & .09 & .09 & .20 & .29 & .48 & .02 & .08 \\
N, O   & -.11 & .03 & -.08 & -.31 & -.57 & .32 & .23 \\
\midrule
Mean of Absolute Values & .28 & .29 & .35 & .55 & .62 & .30 & .29 \\
\bottomrule
\end{tabular}
}
\caption{Domain Level Correlation Analysis for BFI-2 Human Responses and Different Methods LLM Responses \\
\textit{Note:} \textit{n} = 1,559 for human responses, \textit{n} = 297 for Shape Llama3, and \textit{n} = 300 for other LLM responses. Some sample sizes are below 300, because certain generated data exceeded reasonable thresholds (1-5) for specific items and were excluded from the analysis. E = Extraversion; A = Agreeableness; C = Conscientiousness; N = Neuroticism; O = Openness.}
\label{tab:dimension_correlation}
\end{table*}




\clearpage
\newpage

\subsection{Additional Personality-Related Behavioral Performance Results}
\label{sec:appendix results.2}

% The primary aim of this experiment is to explore whether LLMs, after being assigned specific personality settings, can exhibit behaviors theoretically aligned with those personalities. 
% This is an exploratory study, as LLMs generate responses based on statistical probabilities derived from the training corpus~\citep{yang2024babbling}, making it uncertain whether assigning a personality setting will influence the model to act consistently with that personality. 
% Fortunately, the PSI method data we have collected includes human behavior ratings, providing a valuable basis to test this hypothesis.

% \paragraph{Personality-Related Behavior}
% We selected two classic types of workplace behaviors: organizational citizenship behavior (OCB) and counterproductive work behavior (CWB). 
% Both types of behaviors are widely supported by research as being related to personality (e.g.,~\citealp{organ1995meta,berry2007interpersonal}).

% We collected data on human OCB and CWB using the scale developed by~\citet{spector2010counterproductive} and prompted the LLM to respond to the same questions (for a detailed description of~\citeposs{spector2010counterproductive} measures, see Appendix~\ref{sec:appendix additional_setting}; for specific prompt details, refer to Appendix~\ref{sec:appendix_prompt}).

% \paragraph{Metrics}
% We primarily focus on the \( r \) between personality domains from different data sources and OCB, as well as CWB.
% We anticipate that the \( r \) between self-reported personality domains and OCB/CWB in LLM simulation will closely align with those observed in the human participants.


% \paragraph{Results}

Table~\ref{tab:ocb_cwb_gpt4o} presents the correlations between the personality dimensions and OCB/CWB reported by PSI GPT-4o, with a comparison to human self-reported data. 
The relevant results for PSI Llama3 are provided in Table~\ref{tab:ocb_cwb_llama3}, while Figures~\ref{tab:ocb_cwb_all_gpt4o} and Figure~\ref{tab:ocb_cwb_all_llama3} display the complete correlation matrices for GPT-4o and Llama3, respectively.




\setlength\tabcolsep{8pt} % Adjust column separation
\begin{table}[h]
\small
\centering
\begin{tabular}{lcc|cc}
    \toprule
    \multirow{2}{*}{\textbf{Domain}} 
    & \multicolumn{2}{c|}{\textbf{OCB}} 
    & \multicolumn{2}{c}{\textbf{CWB}} \\

    \cmidrule(lr){2-3} 
    \cmidrule(lr){4-5}
    & \textbf{Human} & \textbf{PSI} 
    & \textbf{Human} & \textbf{PSI} \\
    \midrule
    Ext  & .42  & .54  & -.01 & -.08 \\
    Agr  & .18  & .36  & -.30 & -.54 \\
    Con  & .12  & .45  & -.35 & -.47 \\
    Neu  & -.21 & -.35 & .23  & .36 \\
    Ope  & .17  & .02  & -.17 & .01 \\
    \bottomrule
\end{tabular}
\caption{Comparison of OCB and CWB Correlations with Personality Domains: Human vs. PSI GPT-4o \\
\textit{Note:} \textit{n} = 357 for human and PSI method.}
\label{tab:ocb_cwb_gpt4o}
\end{table}


\setlength\tabcolsep{8pt} % Adjust column separation
\begin{table}[h]
\small
\centering
\begin{tabular}{lcc|cc}
    \toprule
    \multirow{2}{*}{\textbf{Domain}} 
    & \multicolumn{2}{c|}{\textbf{OCB}} 
    & \multicolumn{2}{c}{\textbf{CWB}} \\

    \cmidrule(lr){2-3} 
    \cmidrule(lr){4-5}
    & \textbf{Human} & \textbf{PSI} 
    & \textbf{Human} & \textbf{PSI} \\
    \midrule
    Ext  & .42  & .48  & -.01 & .05 \\
    Agr  & .18  & .46  & -.30 & -.45 \\
    Con  & .12  & .47  & -.35 & -.41 \\
    Neu  & -.21 & -.38 & .23  & .33 \\
    Ope  & .17  & .11  & -.17 & .00 \\
    \bottomrule
\end{tabular}
\caption{Comparison of OCB and CWB Correlations with Personality Domains: Human vs. PSI Llama3 \\
\textit{Note:} \textit{n} = 357 for human and PSI method.}
\label{tab:ocb_cwb_llama3}
\end{table}



% The results demonstrate that the correlations simulated by LLMs between OCB/CWB and various personality dimensions closely resemble the patterns seen in human self-reported data, except for the openness domain.
% When human reports show positive, negative, or no correlation, the LLM simulations generally exhibit consistent trends in the corresponding directions.

% However, the correlations generated by LLM simulations are often higher, likely because the model primarily relies on the `typical' or `idealized' knowledge structures absorbed from its training corpus when responding to the scale, consistently repeating and amplifying such associations. 
% In contrast, human self-reported data contains more random noise, leading to relatively weaker correlations.


\begin{figure*}
    \centering
    \includegraphics[width=\linewidth]{latex/Figures/correlation_matrix_gpt40.pdf}
    \caption{Complete Correlation Matrices for Human and PSI GPT-4o \\
\textit{Note:} \textit{n} = 357 for human and PSI method.}
\label{tab:ocb_cwb_all_gpt4o}
\end{figure*}

\begin{figure*}
    \centering
    \includegraphics[width=\linewidth]{latex/Figures/correlation_matrix_llama3.pdf}
    \caption{Complete Correlation Matrices for Human and PSI Llama3 \\
\textit{Note:} \textit{n} = 357 for human and PSI method.}
\label{tab:ocb_cwb_all_llama3}
\end{figure*}



\clearpage
% \newpage

\subsection{The Influence of Social Desirability}
\label{sec:appendix results.3}

In previous studies, researchers have observed some intriguing phenomena, specifically that LLMs tend to exhibit human-like social desirability bias when simulating human samples. 
This bias is reflected in the LLMs’ responses, which lean toward behaviors and traits that are socially approved.

Numerous studies support this finding. 
For example,~\citet{hilliard2024eliciting} noted that newer and larger-parameter LLMs display more diverse personality traits, including higher levels of agreeableness, emotional stability, and openness. 
Similarly,~\citet{salecha2024large} found that LLMs exhibit human-like social desirability bias when generating simulated data.

In this section, using the data we have collected, we directly compare the results from human samples and LLM-simulated data. 
We examined the self-reported personality traits of human samples alongside the simulated personality traits from LLMs, analyzing how both were influenced by social desirability ratings. 
Specifically, we used social desirability ratings to predict and test the correlations and differences between the mean scores of human and LLM data across various items.

By analyzing the predictive relationships and correlations between item means and social desirability ratings, we aim to uncover how social desirability influences the characteristics and patterns in both human and LLM data. 
Furthermore, by comparing the differences in item means between human data and LLM-simulated data, we explore whether social desirability ratings affect human samples and model-simulated data in similar ways.

\paragraph{Social Desirability Rating}
Social desirability ratings of the BFI-2 items were obtained from another ongoing study where 142 human resource practitioners were asked to rate how desirable each item was in general (1 = ``Very undesirable,'' 2 = ``Undesirable,'' 3 = ``Slightly undesirable,'' 4 = ``Neither desirable nor undesirable,'' 5 = ``Slightly desirable,'' 6 = ``Desirable,'' 7 = ``Very desirable''). 

If a certain measurement item receives a high score (6 or 7), it indicates that the trait is generally viewed as desirable or positive in society. 
On the other hand, a low score (1 or 2) suggests that the trait may be considered less ideal or inconsistent with social expectations.
Some desirable traits may lead participants to lean toward choosing higher scores (due to social desirability bias) rather than responding based on their true circumstances.


\paragraph{Results} 
Table~\ref{tab:regression-results} presents the regression results predicting the self-reported personality in the human sample, the LLM-simulated personality, and the difference between human and LLM personalities, all based on the social desirability rating.
Additionally, Figure~\ref{tab:social_desirability_figure} visually illustrates the trends through a regression plot.

It can be observed that when an item's social desirability rating is neutral (4), most LLMs tend to generate neutral responses (rated 3 on the scale). This result indicates that LLMs have learned to associate neutral social expectations with neutral opinions, consistent with previous research findings~\citep{salecha2024large}.
Regarding the mean difference between the human responses and the LLM responses, most differences show a positive correlation with the social desirability rating, and when the social desirability rating is neutral (4), the mean difference in most data is approximately 0.

In Figure~\ref{tab:social_desirability_figure}, the line graph clearly illustrates that as the social desirability rating of the item increases, the average scores of both human responses and LLM responses rise accordingly. 
Moreover, as the social desirability rating increases, the average difference between human and LLM responses gradually widens, whereas this difference narrows when the social desirability scores are lower.

This trend indicates that when social desirability scores are high, the mean score of human responses tends to be higher than that of LLM responses. 
Conversely, when social desirability scores are low, the average score of human responses falls below that of LLM responses. 
This suggests that human responses are more strongly influenced by social desirability than those of LLMs.





\begin{table*}[h]
\centering
\resizebox{\textwidth}{!}{%
\begin{tabular}{lccccccc}
\hline
\textbf{Variable} & \textbf{Human} & \textbf{Persona GPT-4o} & \textbf{Shape GPT-4o} & \textbf{PSI GPT-4o} & \textbf{Persona Llama3} & \textbf{Shape Llama3} & \textbf{PSI Llama3} \\
\hline

Intercept & 1.15 (0.23) & 1.90 (0.17) & 2.33 (0.09) & 2.13 (0.23) & 0.71 (0.21) & 2.25 (0.11) & 2.08 (0.28) \\
Social desirability & 0.46 (0.05) & 0.23 (0.04) & 0.12 (0.02) & 0.23 (0.05) & 0.47 (0.05) & 0.14 (0.02) & 0.20 (0.06) \\
$R^2$ & .59 & .39 & .38 & .27 & .65 & .37 & .15 \\
Predicted score at neutral point & 2.99 & 2.82 & 2.81 & 3.05 & 2.59 & 2.81 & 2.88 \\
Correlation & .77 & .63 & .62 & .52 & .80 & .61 & .39 \\
\hline
\multicolumn{8}{c}{\textbf{Mean Score Difference}} \\
\hline
Intercept & - & -0.76 (0.19) & -1.18 (0.19) & -0.99 (0.20) & 0.44 (0.21) & -1.11 (0.23) & -0.93 (0.32) \\
Social desirability & - & 0.23 (0.04) & 0.34 (0.04) & 0.23 (0.04) & -0.01 (0.05) & 0.33 (0.05) & 0.26 (0.07) \\
$R^2$ & - & .34 & .53 & .32 & .48 & .42 & .19 \\
Predicted score at neutral point & - & 0.16 & 0.18 & -0.07 & 0.40 & 0.21 & 0.11 \\
Correlation & - & .58 & .73 & .56 & -.03 & .65 & .43 \\
\hline
\end{tabular}
}
\caption{Regression Analysis and Correlation of Social Desirability Ratings \\
\textit{Note:} \textit{n} = 1,559 for human responses, \textit{n} = 297 for Shape Llama3, and \textit{n} = 300 for other LLM responses. Some sample sizes are below 300, because certain generated data exceeded reasonable thresholds (1-5) for specific items and were excluded from the analysis.}
\label{tab:regression-results}
\end{table*}



\begin{figure*}[h]
    \centering
    \includegraphics[width=\linewidth]{latex/Figures/social_desirability.pdf}
    \caption{Regression Line Chart of Social Desirability Ratings \\
\textit{Note:} \textit{n} = 1,559 for human responses, \textit{n} = 297 for Shape Llama3, and \textit{n} = 300 for other LLM responses. Some sample sizes are below 300, because certain generated data exceeded reasonable thresholds (1-5) for specific items and were excluded from the analysis.}
\label{tab:social_desirability_figure}
\end{figure*}





\end{document}
