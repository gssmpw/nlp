\section{Concluding Thoughts}

Schemex is designed to help users extract actionable schemas from examples. 
Through our previous experiences with manually deriving schemas, we have identified challenges in clustering examples and evaluating schemas, especially as the number and complexity of examples increase. 
During the initial evaluation of Schemex, we found that AI, particularly recent reasoning models, is capable of performing clustering, significantly reducing the cognitive load on users. 
Furthermore, to streamline the schema evaluation process, Schemex incorporates an apply-iterate cycle. 
Current AI reasoning models can conduct in-depth analyses by comparing poor and strong examples, which supports meaningful iterations and alleviates the burden on users.

Overall, we view Schemex as a tool that augments human cognition through clustering, abstraction, and contrastive learning. 
AI serves as a starting point to relieve humans of the initial heavy lifting, while humans are involved in evaluation. 
Looking ahead, we plan to refine how we construct the workflow by integrating key components like abstraction and iteration in a more flexible manner. 
This includes making more intelligent decisions about low-level operations, such as determining the necessary number of iterations, allowing users to concentrate on high-level thinking.