\section{Case Study 2: Multimodal Analysis of News TikToks}

\begin{figure*}
\centering
\includegraphics[width=0.66\textwidth]
{figures/study2.jpg}
    \caption{Case study 2 - Comparison of news TikTok scripts generated by AI with the initial and iterated schemas.}
    \label{fig:study2}
\end{figure*}


In this case study, we tested Schemex on news TikToks, a multimodal domain combining text, visual, and audio elements, which currently has less established schema knowledge. 
This builds upon our previous work, ReelFramer~\cite{reelframer}, where we manually identified schemas by analyzing examples of news TikToks.
The schema we manually derived covered three narrative framings: expository dialog, reenactment, and comedic analogy. 
It also included premises that cover characters, plot, and three key information points, among others. 
By applying Schemex to this more challenging domain, we aimed to assess whether the workflow could discover the schemas we identified through manual discovery and improve upon them.

We randomly collected 20 TikToks from The Washington Post (WAPO) for analysis, following the practice in our manual process. 
We then took preprocessing steps for multimodal analysis: for visual information, we extracted a video frame from the news TikTok every second; we then provided these frame screenshots to GPT-4V to generate captions and visual descriptions of keyframes. 
For semantic/audio information, we used Whisper to process the video and generate an audio transcript. 
After preprocessing, we obtained the visual and audio transcripts for the TikTok examples as input. 
We also included the original news articles in the input data, as they are important for understanding the creation strategies. 
Following the methodology employed in Case Study 1, we then applied Schemex to analyze these 20 WAPO news TikTok examples.

The clustering step revealed four news TikTok types: Dialogue-driven Explanations (5 examples), Metaphorical Storytelling (5), Direct Presenter \& Visual Aids (7), and Pop Culture Parody \& Memes (3). 
Validation against the researcher’s manual annotation showed 85\% alignment (17/20 TikToks). 
Two of the three outliers should have been clustered in the Pop Culture Parody \& Memes group but were incorrectly clustered elsewhere because DeepSeek-R1 did not recognize the meme. 
The other outlier does not fit into any of the clusters but is more of a blend of Pop Culture Parody and Metaphorical Storytelling. 
Compared to our findings in ReelFramer, three clusters map well, while the third cluster (Direct Presenter \& Visual Aids) provided new insights that we had not covered.

To evaluate schema utility, we conducted a blinded comparison using two randomly selected examples (one from each cluster: Dialogue-driven Explanations and Metaphorical Storytelling). 
GPT-4 generated paired news TikTok scripts using both the initial Stage 2 schemas and the refined Stage 3 schemas, guided by the original news article. 
An expert with experience in both journalism and TikTok evaluated these TikTok scripts through randomized pairwise comparisons, providing qualitative feedback.

The results (see Figure~\ref{fig:study2}) showed marked improvements in the iterated schema-guided news TikTok.
For the dialogue-driven explanation cluster, the iterated schema highlights that the characters should be relevant to the news.
While V1 relies on abstract ideological debates between generic ``Teacher'' and ``Skeptic'' personas.
The TikTok script created using the iterated schema (V2) successfully anchors its critique of Florida’s education policies through role-specific characters (student/teacher) that mirror real-world stakeholders affected by the legislation, making the exchanges more natural. 
For the metaphorical storytelling cluster, the iterated schema highlights that the script should establish a single cohesive metaphor rather than mixing multiple disjoined references.
While V1 (with initial schema) incorporates disparate elements that include wedding characters (groom, bride, wedding officiant), TSA, and FAA characters, V2 (with iterated schema) adheres to a unified sports analogy, mapping news characters to relevant, identifiable roles such as VIP players, underdog teams, and coach.
This approach makes the narrative more comprehensible and cohesive.
