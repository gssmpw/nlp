\section{Introduction}

Expertise is often built by learning from examples.
Chefs study recipes to understand flavor principles, engineers reverse-engineer products to uncover key design concepts, and musicians analyze compositions to master harmonic structures.
This process—schema induction~\cite{gick1983schema}—helps us identify patterns from examples. 
It transforms fragmented observations into usable knowledge: a chef creates new dishes, an engineer solves unexpected problems, and a musician reinvents genres.
As the number and complexity of examples increase, schema induction becomes essential for synthesizing overwhelming information and extracting meaningful insights.

Despite its importance, schema induction remains a challenging cognitive task.
While well-established domains like music composition benefit from expert-created schemas (e.g., chord progressions), emerging fields—such as creating news TikToks~\cite{reelframer}—lack such structured frameworks.
Due to the tacit nature of knowledge, even creators of examples often struggle to articulate the underlying schemas.
The process involves detecting latent patterns across numerous examples, abstracting generalizable rules, and iteratively refining schemas as new information emerges.
Yet, current sensemaking tools focus more on organizing examples than on actively supporting the insight-generation process.

Recent advances in generative AI reasoning capabilities, particularly DeepSeek R1's ``slow thinking''~\cite{guo2025deepseek}, offer new opportunities for supporting schema induction through human-AI collaboration.
We present Schemex, an AI-powered workflow that augments human schema induction through three stages:
\begin{enumerate}[noitemsep]
\item Clustering: Grouping examples by latent similarities
\item Abstraction: Extracting structural patterns within clusters
\item Refinement via contrasting examples: Sharpening schemas through AI-generated contrasting examples
\end{enumerate}
The process forms a collaborative loop in which AI performs clustering, abstraction, and contrastive learning that significantly reduce the cognitive load of humans.
Meanwhile, humans evaluate and refine the outputs, controlling the iteration cycles.

We conducted an initial evaluation of Schemex through two real-world case studies:
\begin{enumerate}[noitemsep]
\item Analysis of HCI Paper Abstracts: Assisting researchers in analyzing abstracts from CHI best papers
\item Multimodal Analysis of News TikToks: Helping journalists understand the schemas of trending news TikToks
\end{enumerate}
Qualitative analysis demonstrates the high accuracy and usefulness of the generated schemas.

Overall, we explore the potential of AI as a cognitive collaborator in schema induction. 
Future work will focus on exploring more flexible ways for workflow construction to incorporate key components such as abstraction and iteration. 
This involves automating low-level operations, such as determining the necessary number of iterations, to allow humans to concentrate on high-level thinking.

