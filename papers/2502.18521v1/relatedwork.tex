\section{Literature Review}
Several researches on detecting and classifying tomato leaf diseases were done. Numerous studies have been carried out in order to forecast and classify leaf diseases in various tomato trees. This research focused on binary classification of tomato leaves whether they are healthy or diseased and used a range of deep learning methods to deal with them. This chapter shows a summary of the pertinent work being efficiently completed by a number of professionals in the relevant sector. The suggested CNN model exhibited an accuracy of 91.2\%, which was noticeably higher than pre-trained CNN models like VGG16 (77.2\%), Mobilenet (63.75\%), and Inception (63.4\%), according to a research by Agarwal et al. (2020)[1]. The study focused on using convolutional neural networks to identify diseases in tomato leaves. For this investigation, they suggested using a CNN model with three convolution layers and three max pooling layers. The advantages of not employing a pre-trained model are further highlighted by the study's discovery that the recommended model required far less storage space—just 1.5 MB—than the pre-trained models, which needed 100 MB. In 2020, Hatuwal, Shakya, and Joshi [2] conducted research on the detection of plant diseases using various machine learning models, including Support Vector Machine (SVM), K-nearest Neighbour (KNN), Random Forest Classifier (RFC), and Convolution Neural Network (CNN). The CNN model had the highest accuracy of any machine learning model, scoring 97.89\%; the RFC model came in second, scoring 87.436\%; SVM came in third, scoring 78.61\%; and KNN, scoring 76.969\%. In contrast to earlier research, this work used f1-score, precision, and recall to assess its models. However, accuracy was the only factor considered when selecting the top-performing model in the final model comparison. For the purpose of classifying tomato leaf diseases using field datasets, Rajasree Rajamohanan and Beulah Christalin Latha [3] exhibited YOLOv5 in 2023. According to the test dataset, they achieved a noteworthy accuracy rate of 93\%. The early tomato leaf spot detection approach, which was created using the MobileNetv2-YOLOv3 techniques, achieves greater accuracy and real-time tomato leaf spot detection stability. Model detection efficacy was assessed using the F1 score and AP value, and testing was conducted in comparison to SSD and Faster RCNN techniques. As demonstrated by the experiment findings, the suggested model has a much better detection effect [4]. In a study on plant disease identification, Madhulatha and Ramadevi (2020) employed a deep Convolutional Neural Network model, and the suggested work was demonstrated to yield an accuracy of 96.50\%. The study uses the well-known AlexNet architecture to categorize the various plant diseases. Known for being utilized in the majority of picture classification use case scenarios, the AlexNet architecture is a Neural Network with eight layers of learnable features. The plant village dataset, which includes 54,323 photos of plant diseases across 38 distinct disease categories, is the source of the dataset used in this study [5]. The suggested work yielded 11 with an overall accuracy of 76.59\%, according to study on the identification of illnesses in paddy leaves with the KNN classifier done by Suresha, Shreekanth, and Thirumalesh (2017) on a database of 330 photos of paddy leaves. The study's sole parameter for assessing the KNN classifier was accuracy; other metrics, such precision, recall, and f1-score, which will be the subject of this work, were not used [6]. An improved Faster RCNN was used by Zhang et al. (2020) to identify four different disease categories and healthy tomato leaves. For image feature extraction, they used a depth residual network rather than VGG16, and for bounding box clustering, they applied the k-means clustering algorithm. Having a detection time of just 470 ms, experimental results on public datasets demonstrated an average identification accuracy of 98.54\% [7].