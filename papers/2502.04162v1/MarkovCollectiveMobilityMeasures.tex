\documentclass[reqno,12pt]{amsart}

\usepackage{enumerate}
\usepackage[margin=1in]{geometry}
\usepackage{enumitem}
\usepackage{ifpdf}
\usepackage{amsmath}
\usepackage{amsfonts}
\usepackage{amssymb}
\usepackage[foot]{amsaddr}
\usepackage{amsthm}
\usepackage{mathrsfs}
\usepackage{amsrefs}
\usepackage{amsmath}
\usepackage{amsfonts}
\usepackage{amssymb}
\usepackage{cite}
\usepackage[titletoc,title]{appendix}
\usepackage[dotinlabels]{titletoc}
\usepackage[all,cmtip]{xy}
\usepackage{nicefrac}
\usepackage{float}
\usepackage{verbatim}
\usepackage[multiple]{footmisc}
\usepackage{mathdots}
\usepackage{dcolumn}
\usepackage{tabu}
\usepackage[hang,small,bf]{caption}    % fancy captions
%\usepackage[section]{placeins}
\usepackage[overload]{textcase} 
\usepackage{graphicx}
\usepackage{wrapfig}
\usepackage{subcaption}
\usepackage[hyperfootnotes=false]{hyperref}

\newlength\Li \newlength\Lii 
\setlength\Li{55mm} \setlength\Lii{80mm} 
\newcommand\HFILL{\hspace*{50mm}}
\newcommand\VFILL{\vspace*{10mm}}

% useful
\newcommand{\ignore}[1]{}

% analysis/geometry stuff
\newcommand{\ann}{\operatorname{ann}}
\renewcommand{\Re}{\operatorname{Re}}
\renewcommand{\Im}{\operatorname{Im}}
\newcommand{\Orb}{\operatorname{Orb}}
\newcommand{\hol}{\operatorname{hol}}
\newcommand{\aut}{\operatorname{aut}}
\newcommand{\codim}{\operatorname{codim}}
\newcommand{\sing}{\operatorname{sing}}

% reals
\newcommand{\esssup}{\operatorname{ess~sup}}
\newcommand{\essran}{\operatorname{essran}}
\newcommand{\innprod}[2]{\langle #1 | #2 \rangle}
\newcommand{\linnprod}[2]{\langle #1 , #2 \rangle}
\newcommand{\supp}{\operatorname{supp}}
\newcommand{\Nul}{\operatorname{Nul}}
\newcommand{\Ran}{\operatorname{Ran}}
\newcommand{\abs}[1]{\left\lvert {#1} \right\rvert}
\newcommand{\norm}[1]{\left\lVert {#1} \right\rVert}

% sets (some)
\newcommand{\C}{{\mathbb{C}}}
\newcommand{\R}{{\mathbb{R}}}
\newcommand{\Z}{{\mathbb{Z}}}
\newcommand{\N}{{\mathbb{N}}}
\newcommand{\Q}{{\mathbb{Q}}}
\newcommand{\D}{{\mathbb{D}}}
\newcommand{\F}{{\mathbb{F}}}

% consistent
\newcommand{\bB}{{\mathbb{B}}}
\newcommand{\bC}{{\mathbb{C}}}
\newcommand{\bR}{{\mathbb{R}}}
\newcommand{\bZ}{{\mathbb{Z}}}
\newcommand{\bN}{{\mathbb{N}}}
\newcommand{\bQ}{{\mathbb{Q}}}
\newcommand{\bD}{{\mathbb{D}}}
\newcommand{\bF}{{\mathbb{F}}}
\newcommand{\bH}{{\textbf{H}}}
\newcommand{\bO}{{\mathbb{O}}}
\newcommand{\bP}{{\mathbb{N}}}
\newcommand{\bK}{{\mathbb{K}}}
\newcommand{\CP}{{\mathbb{CP}}}
\newcommand{\RP}{{\mathbb{RP}}}
\newcommand{\HP}{{\mathbb{HP}}}
\newcommand{\OP}{{\mathbb{OP}}}
\newcommand{\sA}{{\mathcal{A}}}
\newcommand{\sB}{{\mathcal{B}}}
\newcommand{\sC}{{\mathcal{C}_{q}}}
\newcommand{\sF}{{\mathcal{F}}}
\newcommand{\sG}{{\mathcal{G}}}
\newcommand{\sH}{{\mathcal{H}_q}}
\newcommand{\sM}{{\mathcal{M}}}
\newcommand{\sO}{{\mathcal{O}}}
\newcommand{\sP}{{\mathcal{N}}}
\newcommand{\sS}{{\mathcal{S}}}
\newcommand{\sI}{{\mathcal{I}}}
\newcommand{\sL}{{L}}
\newcommand{\sK}{{\mathcal{K}}}
\newcommand{\sU}{{\mathcal{U}}}
\newcommand{\sV}{{\mathcal{N}}}
\newcommand{\sX}{{t_2}}
\newcommand{\sY}{{\mathcal{Y}}}
\newcommand{\sZ}{{\mathcal{Z}_t_2}}
\newcommand{\fS}{{\mathfrak{S}}}

\newcommand{\interior}{\operatorname{int}}

% Topo stuff
\newcommand{\id}{\textit{id}}
\newcommand{\im}{\operatorname{im}}
\newcommand{\rank}{\operatorname{rank}}
\newcommand{\Tor}{\operatorname{Tor}}
\newcommand{\Torsion}{\operatorname{Torsion}}
\newcommand{\Ext}{\operatorname{Ext}}
\newcommand{\Hom}{\operatorname{Hom}}

%extra thingies
\newcommand{\mapsfrom}{\ensuremath{\text{\reflectbox{$\mapsto$}}}}
\newcommand{\from}{\ensuremath{\leftarrot_2}}
\newcommand{\dhat}[1]{\hat{\hat{#1}}}

\newtheorem{thm}{Theorem}[section]
\newtheorem*{thmnonum}{Theorem}
\newtheorem{thmstar}[thm]{Theorem*}
\newtheorem{conj}[thm]{Conjecture}
\newtheorem{prop}[thm]{Proposition}
\newtheorem{propstar}[thm]{Proposition*}
\newtheorem{cor}[thm]{Corollary}
\newtheorem*{cornonum}{Corollary}
\newtheorem{lemma}[thm]{Lemma}
\newtheorem{lemmastar}[thm]{Lemma*}
\newtheorem{claim}[thm]{Claim}
\newtheorem*{lemmanonum}{Lemma}

\theoremstyle{definition}
\newtheorem{defn}[thm]{Definition}
\newtheorem{example}[thm]{Example}
\newtheorem*{defnonum}{Definition}

\theoremstyle{remark}
\newtheorem{remark}[thm]{Remark}
\newtheorem*{remarknonum}{Remark}
%
%\theoremstyle{note}
%\newtheorem{note}[thm]{Note}

\newcommand{\ket}[1]{\ensuremath{\left|#1\right\rangle}} % Dirac Kets
\newcommand{\bra}[1]{\ensuremath{\left\langle#1\right|}} % Dirac Bra





\renewcommand{\vec}[1]{\ensuremath{\boldsymbol{#1}}} % bold vectors
\def \myneq {\skew{-2}\not =} % \neq alone skews the dash

\makeatletter
\newcommand*{\rom}[1]{\expandafter\@slowromancap\romannumeral #1@}
\makeatother


\newtheoremstyle{defnopunct}
{3pt}
{3pt}
{}
{}
{\bfseries}
{ }
{.5em}
{}

\usepackage{etoolbox}
\let\bbordermatrix\bordermatrix
\patchcmd{\bbordermatrix}{8.75}{4.75}{}{}
\patchcmd{\bbordermatrix}{\left(}{\left[}{}{}
\patchcmd{\bbordermatrix}{\right)}{\right]}{}{}

\makeatletter
\patchcmd{\subsection}{\bfseries}{\itshape}{}{}
\patchcmd{\@sect}{\@addpunct.}{}{}{}
\makeatother




%\setlength{\belowcaptionskip}{-4pt}
%
\author{Alisha Foster$^\dag$, David A. Meyer$^\dag$, Asif Shakeel$^\dag$}
\address{$^\dag$Department of Mathematics, University of California, San Diego, La Jolla, CA 92093-0112, USA}
\email{a1foster@ucsd.edu, dmeyer@math.ucsd.edu, ashakeel@ucsd.edu}
%\author{Alisha Foster$^\dag$}
%\author{David A. Meyer$^\dag$}
%\author{Asif Shakeel$^\dag$}
%\ifpdf
%\pdfinfo{
%  /Title {}
%/Author{}
%}
%\fi

\title [A Pseudo Markov-Chain Model  and Measures of Collective Mobility]{A Pseudo Markov-Chain Model and  Time-Elapsed Measures of Mobility from Collective Data}


\raggedbottom

\begin{document}

\begin{abstract}
In this paper we develop a pseudo Markov-chain model to understand time-elapsed flows, over multiple intervals,  from  time  and space aggregated collective inter-location  trip data, given as a time-series.  Building on the model, we develop measures of mobility that  parallel those known for individual mobility data, such as the radius of gyration.  We apply these measures to the  NetMob 2024 Data Challenge data, and obtain interesting results that are consistent with published statistics and commuting patterns in cities. Besides building a new framework, we foresee applications of this approach to an improved understanding of human mobility in the context of environmental changes and sustainable development.   
\end{abstract}

\keywords{collective mobility, measures of mobility, pseudo Markov-chain, machine learning, mobility models, urban, transportation, cell phone, anonymization, privacy, aliasing, efficiency and sustainability from mobility data}

\maketitle

 \section{\bf Introduction} \label{sec:intro}
 
Human mobility as a subject of analysis continues to grow with the advent, sophistication and proliferation of smart mobile devices, social networks, mobile applications and distributed services.
 Many of the models, and much of the analysis  and computational work are based on  data acquired  as time and location  of individual devices. This data is often  collected from call data records (CDRs),  information gleaned from trackers logging global positioning system (GPS) data, and from  contextual and usage  data from mobile applications. Under   privacy concerns~\cite{dgbcd:opcumpd}, the form of data considered appropriate for analyses has veered towards  collective mobility data~\cite{bbccd:amdchfc}, with multiple levels of anonymization to resist individual identification~\cite{khdr:tmumtlsd,xtlzfj:trfa}. Such data present only aggregate information about the  movements of groups, obscuring individual patterns of movement. Typically, the data is aggregated in both space and time:   the geographical region  the studied population resides in is divided into  a grid, and   the temporal span of the study is divided into a sequence of intervals. Combined by intervals and grid-cells, the aggregated data  is made available as a time-series, giving for each time-step,~\footnote{Depending on the context, we will interchangeably say time-step or interval.} the populations in cells, and  counts and other statistics of  trips taken  between cells. Thus presented,  the resulting data conceals information about the individuals, revealing only  that about the  collective. 


The works of   Gonz{\'a}lez, Hildago, and Barab{\'a}si in 2008~\cite{ghb:uihmp} and of Barbosa, 
Barthelemy,
Ghoshal, {\it et al}.\ in 2018~\cite{bbgj:hmma}, describe models and measures of mobility stemming from individual mobility data. There are studies connecting random walks to human mobility patterns~\cite{jyz:chmlsn}, and  of trajectories and statistics of displacement~\cite{bhg:slht} from individual mobility data. We are interested, in this paper, in understanding the longer term flows resulting from collective mobility,  the kind of interpretable patterns that result from them, and in determining their efficacy and utility for estimating criteria for guided decision-making. Applications could range from comparative studies, to planning and forecasting based on these criteria and measures.  Thus,  an objective of this paper is to extract measures of mobility akin to those  developed for individual mobility tracked over longer times~\cite{bbgj:hmma}, but now from collective  mobility data, with de-facto privacy.  For each interval in the time-series, the basic aggregated information usually consists  of  the count of devices detected in each geographical cell and the observed number of trips between pairs of cells,  from each  pair's  {\it origin} (O) to its {\it destination} (D) cell. Supplementing  that basic information to enable physically meaningful analysis,   additional  summary statistics of quantities such as the mean distance or time of travel between the pairs of cells may also be in the data.   To increase the level of anonymization,  coarsening of the data  may extend  to the  counts of population and trips, with those below a threshold being omitted, as  in the data we study, the NetMob 2024 Data Challenge~\cite{zpgm:nmd} data.


Various aspects of human movements are captured by measures described in~\cite{ghb:uihmp,bbgj:hmma,wtded:mhmumpr}, among them {\it jump-length, radius of gyration, mean-squared displacement} derived from individual mobility data of  long-term movements. Instead,  we are interested in eliciting similar measures from collective data with anonymized, aggregated trip-counts between coarse-grained locations and over intervals that may or may not be aligned with the dynamics of the population movements. Such measures may qualitatively describe the patterns of movements and quantitatively provide simple social and economic mobility indicators summarizing underlying complexities  and modes of movement, and interactions between intrinsic structures and extrinsic forces. The aim is to develop these criteria and measures to be further used as deemed useful in dynamic analyses of mobility and its effect on the environmental factors, livability, efficiency and sustainability of communities, and vice-versa~\cite{ysctw:iawsssd,ypm:ilablumddud,wlcqjb:e15mcpaud}. Studies of relationship between short-term and long-term mobility exist~\cite{mzbwt:ursltm}, as well as those of collective mobility~\cite{bbbc:uhmfampd,jwsl:chmpttua,gyzh:uichmptceos,fdgm:mngmc}, but the questions posed and the analyses conducted are fundamentally different from those motivating and addressed in the model we develop. 

Our analysis uses the mobility datasets from the NetMob 2024 Data Challenge~\cite{nm:nm2024}; the details of the data are in the document ``The NetMob2024 Dataset: Population Density and OD Matrices from Four LMIC Countries"~\cite{zpgm:nmd}.\footnote{LMIC: Low and Middle Income Countries.}  The data are provided by Cuebiq.~\footnote{Cuebiq's terms and conditions for the NetMob 2024 Data Challenge require us to refer the reader to the following statement pertaining the availability of data: ``Aggregated data was provided by Cuebiq Social Impact as part of the Netmob 2024 conference. Data is collected with the informed consent of anonymous users who have opted-in to anonymized data collection for research purposes. In order to further preserve the privacy of users, all data has been aggregated by the data provider spatially to the GH3, GH5, h37  and temporally to 3-hourly, daily, weekly, and monthly levels and does not include any individual-level data records.''}
These datasets include data from the following countries:  Mexico, Indonesia, India, and Colombia. All of the data is already aggregated and is provided at several resolutions: spatially by a grid whose cell geometry and resolution (cell size) is given by one of the geohash levels: GH3, GH5, or h37, and temporally by one of the time intervals: 3-hourly, daily, weekly, or monthly.   A {\it location} is taken to be the centroid of a cell. ~\footnote{The centroid of a cell in turn identifies the cell uniquely as the cell boundaries are fixed by the geohash chosen.}  For a given country and aggregation, the {\it population data} is given as a time-series of the location (cell) population counts, and the {\it OD data} as a time-series of trip-counts for the  origin-destination (OD) pairs of locations (cells). As mentioned above, the data in both cases are omitted if the respective  count is below the threshold of 10.   The OD data, along with the trip-counts,   include the means, the medians and the standard deviations of the distances traveled (trip-distance) ,~\footnote{The mean distance traveled may be shorter or longer than the geographical distance between the origin and destination locations, i.e., cell centroids.} and of the times of travel (trip-time)  for each OD pair.  All the calculations in this paper are for 3-hourly GH5 OD data for the year 2019.



This paper is organized as follows. In section~\ref{sec:pmcm} we describe the data in generic terms and set up the pseudo Markov-chain model. We use the OD trip-count data to construct the  model, under the assumption that the individuals in the data are indistinguishable. This allows a probabilistic description of an individual whose  probability of transition from an origin to a destination in any time-step is approximated by the trip-count normalized by the total number of trips emanating from the origin. We call this description of an individual a  {\it privacy enhanced person} (PEP).  A PEP's inter-location transitions thus become a Markov (Stochastic) transition matrix. Chaining the time-steps as successive transition matrices, we obtain a {\it time-elapsed} version of a PEP.  In the process, we identify what we term {\it mobility-aliasing}: counting the same individuals as part of successive OD trips, as a side effect of using too long an interval for aggregation. This issue could possibly be addressed in two ways. First, by an appropriate strategy based on movement speeds  when initially aggregating the data spatio-temporally. Second,  by post-processing through  algorithmic  means to approximate the roots of the original transition matrices  in the cases such as Netmob 2024 where the  data has been aggregated at space and time scales that are pre-determined. In some sense this is the reverse of the pseudo Markov-chain model, and underlines the flow of ideas from a mathematical domain to mobility research as the authors of~\cite{yltlgm:ehmrwosd} recommend. This level of aggregation would make the estimates more reflective of the true dynamics of mobility. Given the size of matrices we are dealing with in this paper, it was not efficiently computable with the computing resources available, but is a topic for future research.  We also set aside the notion of {\it waiting-time} between stops, referred to in~\cite{bbgj:hmma}. For the current investigation, we use the original matrices to demonstrate the machinery and the usefulness of the measures. 

 In section~\ref{sec:tenet} we develop the first extension of the Markov-chain~\cite{n:mc} formalism. This is a calculation of the time-elapsed OD net trip-counts (net flow).  We validate this for Mexico City area, for the morning and afternoon of a randomly selected day, which show that the  main directions of population movements are consistent with daily commuting. We surmise that over longer time and distance scales and more detailed data, such net flow calculations might show patterns of migration related to climate and environmental shifts.

 In section~\ref{sec:tedist} we seek a measure of time-elapsed travel that may sense the presence of inefficiencies in travel patterns, arising from  routing constraints, detours, congestion, or other impediments. First, we calculate the time-elapsed origin-to-destination distance traveled by a PEP for a generic   OD pair. This is  a weighted mean over multiple possible paths taken by a PEP over  consecutive  time-intervals, and we use the Markov structure to develop a recursive algorithm for it. Then we address the issue of estimating the most  direct OD trip-distance as a baseline to compare with the time-elapsed OD distance.  We normalize the time-elapsed OD distance by the direct trip-distance estimate to arrive at the dimensionless  measure we call the {\it effective distance}. An OD pair can be tracked for its effective distance   over time, and different OD pairs can   be compared with each other by their respective effective distances.    We compute and plot the morning 6-hourly effective distances of all OD pairs in Mexico City area  against dates in 2019. That helps us identify  OD pairs with high effective distances and the days on which they occur. We display the more significant examples of these on maps as a visual confirmation: the OD pairs and the effective distances between them, as well as the indirect paths responsible for the cumulative distances.  We observe both spatial and temporal  persistence of such OD pairs, and note that measures of persistence or flow-based characterizations of such constellations might be possible. We could calculate the equivalent for the trip-time, but we only illustrate the concept with the trip-distance.
 


 In section~\ref{sec:terto} we propose  measures that could be construed in the sense of  the radius of gyration~\cite{bbgj:hmma}.  Radius of gyration~\cite{ghb:uihmp} is a measure of the typical distance traveled by an individual. It has variants refining that notion~\cite{bbgj:hmma}. Our collective mobility counterparts to it are   the return-to-origin (RTO)  distance, time and speed  measures.  RTO distance and time, as the names imply,  are the mean distance and time a PEP travels in returning to the place of origin over some fixed duration.  RTO speed, instead,   is a pseudo-speed defined simply as the ratio of RTO distance to RTO time. It cannot be directly interpreted as the conventional speed. We plot the RTO measures for Mexico City, Jakarta, and Mumbai against dates in 2019. Comparing among them we find trends that reflect shared characteristics and differences that may ultimately have roots in their  modes of life, social and economic activities, seasonal and environmental effects, and infrastructure and transportation networks.     
    
Section~\ref{sec:conc} is the conclusion and discussion of the results and future research.

 \section{\bf  Pseudo Markov-chain Model} \label{sec:pmcm}

To construct the pseudo Markov-chain model, we assume that individuals, as far as their movements are concerned, are indistinguishable. In particular, their probability of transitioning from one location to the next at any time-step is independent of their history of transitions. Thus, they provide typical realizations of a Markov process, i.e., one characterized by the memoryless property. We adopt the term {\it privacy-enhanced person} (PEP) when we refer to an instance of this randomized individual traversing a path over the grid locations.
    
%%%%%%%%%%
 We construct a stochastic matrix $M^t$ which encodes the probabilities of transitioning to a destination given an origin location at a single time-step $t$. In order to extend our inferences from single time slices to multiple sequential time slices, we create a \textit{time-elapsed transition matrix} $A^t$ which gives the transition probabilities to a destination given an origin location over a time interval \textit{up to} time-step $t$. We use these time-elapsed matrices and additional information to make longer term inferences from the data---estimating the net population movement over time, mean distance traveled over time, and the average amount of time it takes for a person to return to their origin location.
%%%%%%%%%
    
Depending on the space-time resolution chosen, the trips might mostly begin and end  between locations in the vicinity of major city-centers (3-hourly GH5) or among far-away regions (daily, GH3). It makes sense then to be concerned with connected-components, at time-scales of interest, and not the entire country.  The model is constructed from the data, so we describe the data in general terms.


\subsection{Generic Data Description}  \label{subsec:gendatadesc}
The data we will be using to construct the model and compute with  is part of the Netmob 2024 Challenge data. It contains two types of datasets:  population data and OD trip data.  We use the OD trip data in this paper. It is aggregated in time by intervals (time-steps) and in space by grid cell identified by their centroid given as a geohash location.  Each trip is indexed by the triple: (time-step, starting geohash location, ending geohash location). The resolution of time-steps could be 3 hours, a day, a week, or a month. Geohashes could be GH5 (4.9km side-length, square grid), GH3 (156.5km side-length, square grid), or h37 (1.41km side-length, hexagonal grid). We will only use the 3-hourly, GH5 data for the year 2019 for computations.    

Trip data are aggregated such that only trips starting and ending in the same time-step are recorded. For each index point, the fields include trip-count (number of trips in the interval), mean/median/standard-deviations of trip distance of travel (m), trip duration/time of travel (min), and other quantities that we are not going to be utilizing in this paper. In any time-step if an OD pair has a trip-count below  the threshold of 10, that OD pair is omitted. It may be that the same individual is part of several successive trips within a time interval  or in consecutive time intervals. Thus, there is no expectation that the trips between different pairs of  locations (in particular pairs that are adjacent geographically) within the same time-step are for different individuals. \footnote{In the course of this work, we found that mean trip distances  provided in the data often have very large values (compared to geographical distances), indicating anomalies or issues with the data collection mechanism. We chose to use median trip-distances  as they avoid being influenced by a few such very high values, which are likely incorrect.}

For each time-step, the trip data can be represented as a directed graph with edges weighted by the trip-count.\footnote{A recent paper~\cite{fdgm:mngmc} on massive datasets for Mexico constructs daily weighted directed graphs as well, but their analysis is in a different direction.} 
In order to  extend these shorter-term data to longer-term movements and to develop interesting measures,  we organize the trip data into a pseudo Markov-chain model. The multiple-step intervals and the measures that we estimate over their duration will be called {\it time-elapsed}.
  
\subsection{Model construction}  \label{subsec:pmcconstr}


 For the purpose of this discussion, consider the time-series of trip-counts for a country over some finite sequence of time-steps $t = 1, \ldots, n$.  Let there be $N$ locations in the country. There are thus  $N^2$ potential directed edges (origin-to-destination direct paths, which we refer to as OD pairs)  in the graph of trip data at any time-step. Normally, this graph is sparse. We mainly work with the strongly-connected components\footnote{For the reader unfamiliar with graph terminology, a strongly-connected component of a directed graph is one in which there is a directed path from every node to every other node.} of the graph for two reasons: first is that it defines areas that are actively connected by routes, and second is that Markov (transition) matrices constructed on strongly-connected components have desirable properties such as no pure absorbing or reflecting nodes.  

We first extend the idea of a graph at a time-step to all the time-steps occurring within some time-elapsed interval ending at time-step $T$. Let $E^t=\{(i,j)^t\}$ (where $i, j \in 1,\ldots, N$) be the set of edges that appear in  time-step $t$. Let $E_T=\cup^T_{t=1} E^t$, be all the edges that appear in the intervals up to time-step $T$. 
Consider the strongly-connected components of the graph given by the edges in $E_T$ on the set of all locations. Let $C$ be one such component of size $N_C = |C|$. Label the set of locations in $C$ by $i = 1,\ldots, N_C$. 

Let the trip-count from location $j$ to location $i$ in $C$ at time-step $t$ be $f^t_{ij}$. Assume that for each origin $j \in C$ and each time-step $t$, there is at least one destination $i \in C$ such that $f^t_{ij} \ne 0$, i.e., every origin has at least one outgoing trip, including trips to itself (which we will often refer to as self-transitions).\footnote{In the Netmob 2024 data there are time-steps for which some of the origin locations $j \in C$ have all the  $f^t_{ij}$ missing, possibly because there is a cut-off for fewer than 10 trips. Currently, we make the reasonable, though inaccurate, amendment of adding $f^t_{jj}=5$ (midway between 0 and 10) for such $j$, creating self-transitions with probability 1 for such locations and time-steps.}  Then  the estimated probability of the trip to destination $i$ given that the origin is $j$,   at time-step $t$, is  
\begin{equation*}
m^t_{ij}=\frac{f^t_{ij}}{\sum^{N_C}_{i=1} f^t_{ij}}.
\end{equation*}
The denominator in the above expression, $\sum^{N_C}_{i=1} f^t_{ij}$, is an estimate for the number of individuals at location $j$ at the beginning of time-step $t$. The one-step transition matrix is then 
 \begin{equation} \label{eq:Mdef}
 M^t=[m^t_{ij}].
 \end{equation}
For the remainder of the discussion, assume that $C$ is a maximal component, and   every summation has lower limit 1 and upper limit $N_C$, unless indicated otherwise. 

%\subsection{Time-elapsed transition matrix} \label{subsec:tetrans}

Further assume that the Markov property holds for trip-counts: for any $t$,  trip-counts  $f^{t'}_{i' j'}$ and  $f^{t''}_{i'' j''}$ for any $t' < t \le t''$, and for any 
$i',j', i'', j''$, are independently distributed. Then, the transition probability to destination $i$ given the origin $j$,  $a^t_{ij}$,  during the interval up to time-step $t$  is given by:
\begin{equation} \label{eq:Adef}
A^t=[a^t_{ij}]=\prod^t_{k=1} M^k = M^t  A^{t-1},\quad  t \ge 1, 
\end{equation}  
where $A^0 = \mathbb{I}$, the identity matrix.

Before going further, we describe a concern regarding the time intervals for aggregation, which affects the predictions and estimates based on this model. 


\subsubsection{Mobility-aliasing} \label{subsec:mobalias}

The second assumption in our pseudo Markov model is that an individual only moves one cell per the time-step. Let us consider this and examine some relevant issues.

We take the case of 3-hourly GH5 data. The mean {\it speeds}\footnote{Not having the data for speed of travel between locations, we take {\it speed} as the median distance traveled divided by the median time of travel for each time-step and each OD pair whose origin and destination are  distinct. Then, we take the average of these speeds over all the pairs weighted by the trip-counts.} of travel between locations (excluding trips in the same cell, i.e., self-transitions) as calculated from the data are, for Mexico City (Mexico): 127 m/min, Jakarta (Indonesia): 125 m/min and Mumbai (India): 100 m/min (based on median trip distances and times). GH5 locations, by the data description document~\cite{zpgm:nmd}, have a spatial resolution of 4.9 km. This implies that about 45 minutes are required to travel between adjacent locations for any of these countries. That would  allow 4 or more transitions to  happen in a 3 hour interval for each individual. In other words the time resolution of the data is far coarser than required by the speed of the movements.

A consequence of the above is that an individual could be observed in several consecutive locations in the same time-step, i.e., be counted in trips between different locations in the same time-step. This is what we term, by analogy to a phenomenon in time-frequency analyses,  {\it mobility-aliasing}. To minimize this aliasing, finer time-resolution would be required. We propose a data collection scheme to reduce mobility-aliasing and propose an algorithmic workaround in appendix~\ref{appdx:mobaliasalg}. 
Recognizing that the algorithmic compensation we proposed is currently not efficiently computable for the size of matrices involved, with the resources we have, we continue our development and assume that a PEP moves at the original time intervals. This allows us to calculate further, and still illustrate the salient features of the methodology.  In the following, we develop time-elapsed  measures of mobility that are calculated from the distances or times of travel from the OD data.

 \section{\bf  Time-elapsed origin-destination trip-counts (net flows)} \label{sec:tenet}

In this section, we build on the model to  calculate the time-elapsed net flows (trip-counts) between locations. Simply put, starting with the total out-going trips from each origin at the beginning of a time-elapsed interval, we estimate the {\it net} number of  trips between each OD pair over that interval.\footnote{The reader may choose to skip the derivations, and read the results sections to get the main ideas through plots, maps and descriptions.}  


\subsection{Derivation} \label{subsec:tenetcalc}

The formalism below defines a function to properly weight each edge and then computes the net trip-count as a difference weighted by edge probabilities. For each OD pair, i.e.,  edge  $(i,j)$, we define the movement function over the OD pairs (edges) in $(k,l) \in C$  at time-step $t$:
\begin{equation*}
    s^t_{kl}(i,j)= 
\delta_i(k)\delta_j(l)
    -\delta_i(l)\delta_j(k) .
\end{equation*}
Since $(i,j)$ is fixed, this can be viewed as a matrix indexed by pairs$(k,l) \in C$, with values 1 at index $(i,j)$,  $-1$ at index $(j,i)$, and 0 elsewhere. 

Let the set of ordered edges in $C$ be 
\begin{equation*}
E^o=\{(i,j)\mid i<j \in C\}. 
\end{equation*}
The probability of a generic individual being at location $j$ at time-step $t=1$ is estimated as
\begin{equation} \label{eq:genprob}
p^1_j=\frac{\sum_{i} f^1_{ij}}{\sum_{i,j} f^1_{ij}}, 
\end{equation} 
and that of a  generic individual moving from $j$ to $i$ over the time-elapsed interval including the time-step $t$, in terms of the entries of the matrix $A^t$ in equation~\eqref{eq:Adef}, is
\begin{equation*}
p^t_{ij}=p^1_j a^t_{ij}
\end{equation*}
The total count of individuals summed over all locations at time $t=1$ is
\begin{equation*}
n^1=\sum_{i,j} f^1_{ij}
\end{equation*}
The total count of individuals at time $t=1$ at location $j$ is
\begin{equation*}
n_j^1=\sum_{i} f^1_{ij} = n^1p^1_j.
\end{equation*}
Then the mean time-elapsed count of individuals moving from $j$ to $i$, the {\it net trip-count} function $\bar{s}^t$, over $E^o$ at time-step $t$,  is 
\begin{align*}
  \bar{s}^t: E^o &\rightarrow \mathbb{R} \\
  \bar{s}^t(i,j) &= \sum_{k,l} s^t_{kl}(i,j)  n^1 p^t_{kl} \\
  					&= n^1(p^t_{ij} - p^t_{ji}) \\
  					&= n^1(p^1_ja^t_{ij} - p^1_ia^t_{ji}) \\
  					&= a^t_{ij} n_j^1 - a^t_{ji} n_i^1.
\end{align*} 
Here, positive values of $\bar{s}^t(i,j)$ are net trip-counts from $j$ to $i$ and negative values are net trip-counts from $i$ to $j$.



\subsection{Results} \label{subsec:tenetres}

We validate the net trip-counts (net flows) by computing commuting flows into (and out of) Mexico city-center during the 3:00--6:00--9:00--12:00 and 15:00--18:00--21:00--00:00 sequence of 3 hour intervals (9 hours time-elapsed intervals), on a randomly chosen date (06-05-2019). These net trip-counts are shown on the maps in figures~\ref{fig1},~\ref{fig2}, which show the top 0.95, 0.95 quantiles of net trip-counts for the respective intervals. The flows are from the periphery to the center in the 3:00--6:00--9:00--12:00 in and and vice-versa in the 15:00--18:00--21:00--00:00 interval. Magenta-colored end of the scale is the lower end of the quantile, and navy-colored is the higher end. The estimated trip-counts are an order of magnitude different between figures~\ref{fig1} and ~\ref{fig2} because the original dataset records are of the same order, i.e., the total starting trip-count for 3:00--6:00 is an order of magnitude smaller than for 15:00--18:00.~\footnote{The starting total trip-count for each interval includes any default self-transitions that had to be included for locations in the maximal component that were missing in the starting interval.} Most of the transitions (which are larger)  are within the same geohash location, i.e., self-transitions, which are not shown here, and have a 0 net trip-count by definition.
\begin{figure}[H] %width=0.6\linewidth,height=0.3\textheight
\centering
\begin{subfigure}{.5\textwidth}
\centering
  \includegraphics[scale=0.3]{ciudaddemexico-mexico_net-count-transitions_quantile-0.95_component_window-2_20190605-03-00-00_20190605-09-00-00.png}
  \caption{
  }
  \label{fig1}
    \end{subfigure}%
    \begin{subfigure}{.5\textwidth}
    \centering
      \includegraphics[scale=0.3]{ciudaddemexico-mexico_net-count-transitions_quantile-0.95_component_window-2_20190605-15-00-00_20190605-21-00-00.png}
  \caption{
  }
  \label{fig2}
      \end{subfigure}%
  \caption{
         Mexico City, 06-05-2019, net trip-counts (a)   3:00--6:00--9:00--12:00 triple of 3-hour intervals. Top 0.95 quantile. (b)  15:00--18:00--21:00--00:00 triple of 3-hour intervals. Top 0.95 quantile.   }   
\end{figure}


We can compare these flow maps with single time-step maps at the start of the respective time-elapsed intervals. The maps in figures~\ref{fig1a} and~\ref{fig2a} are the flow maps for 3-hour  intervals  3:00--6:00 and 15:00--18:00, respectively. Generally, the flows are directed as they are in the corresponding time-elapsed 9-hour intervals. However, there appears to be a consolidation of flows in the maps of figures~\ref{fig1} and~\ref{fig2} and higher degree of directionality compared to the single time-step flows of figures~\ref{fig1a} and~\ref{fig2a}. 
\begin{figure}[H] %width=0.6\linewidth,height=0.3\textheight
\centering
\begin{subfigure}{.5\textwidth}
\centering
  \includegraphics[scale=0.3]{ciudaddemexico-mexico_net-count-transitions_quantile-0_component_window-2_20190605-03-00-00_20190605-03-00-00.png}
  \caption{
  }
  \label{fig1a}
    \end{subfigure}%
    \begin{subfigure}{.5\textwidth}
    \centering
    \includegraphics[scale=0.3]{ciudaddemexico-mexico_net-count-transitions_quantile-0.9_component_window-2_20190605-15-00-00_20190605-15-00-00.png}
  \caption{
  }
  \label{fig2a}

      \end{subfigure}%
  \caption{
       Mexico City, 06-05-2019, net trip-counts  (a) 3:00--6:00  interval. All.  (b)   15:00--18:00  interval. Top 0.9 quantile. }   
\end{figure}

This validation at the scale of a city suggests that net flow calculations  at a larger spatio-temporal scale with more detailed data  might show migrations related to seasonal effects, climate change and environmental challenges.  


 \section{\bf  Time-elapsed origin-destination distance traveled and effective distance} \label{sec:tedist}

We build further and  estimate the mean time-elapsed origin-to-destination distance traveled by a PEP for all OD  pairs (some of which may or may not be in the data). We must keep track of the accumulated distances along all allowed intermediate paths for this purpose. The Markov property enables us to find a recursive algorithm to calculate this time-elapsed travel distance from the trip-distance in the data  for the  intermediate time-steps and  OD pairs. %Details of the calculation are in appendix~\ref{appdx:tedistcalc}.

The time-elapsed OD distance, when significantly longer than expected for an OD pair, perhaps means circuitous paths are more  active than the most direct available. To be useful as an indicator of  higher than expected travel distance, it would need to be compared  to an expected/baseline direct OD distance for that OD pair.  That baseline may be found from the data as an expected distance for that pair if the data contains it, otherwise it would need to be imputed from the geographical origin-to-destination distance for the OD pair. We normalize the time-elapsed OD distance by its baseline distance, calling it the OD pair's {\it effective distance} for the time-elapsed window. OD pairs can be compared by their effective distances over the same time-elapsed window. Tracked over time, this may help in finding the OD pairs with consistently longer than expected travel distances due to, among other reasons,   bottlenecks or other  circumstances necessitating diversions.   Were we to substitute in this description travel-time for travel-distance, we may then be comparing  the routes that are more time-consuming or time-efficient than others. We only consider travel distances in this paper.   



\subsection{Derivation: time-elapsed OD distance traveled}  \label{subsec:tedistcalc}

 For the origin $j$  and destination $i$, we keep track of the mean distances for all time-elapsed  paths from $j$ to all intermediate locations $k$  that   have not passed  through $i,j$ except at  the beginning ($t=0$). Let us call such path distances $y^{i,j,t}_{k}$, taking $y^{i,j,0}_{k}=0$. Here we think of $\mathbf{y}^{i,j,t}=[y^{i,j,t}_{k}]$ as a one-dimensional vector. In the following the superscripts $i,j$ are simply to keep track of the particular OD pair under consideration. They are not useful in the calculations of mean distance and can be safely ignored.   

We define a vector $\mathbf{p}^{i,j,t}=[p^{i,j,t}_{k}]$, recursively as follows.
Let ${\mathbf{p}}^{i,j,1} =  [m^1_{kj}]$, the $j$-th column of matrix $M^1$ in equation~\eqref{eq:Mdef}.
Let 
 \begin{equation*}
    p^{i,j,t}_{k}= \sum_{r | r \ne i, j} m^t_{kr}   p^{i,j,t-1}_{r},
\end{equation*}
for $t \ge 2$.  We can write this a matrix operation by defining a vector in terms of ${\mathbf{p}}^{i,j,t}$ with $i,j$ entries set to 0:
  \begin{equation*}
    {\tilde{\mathbf{p}}}^{i,j,t}= [{\tilde p}^{i,j,t}_k],    
    \end{equation*}
    where 
      \begin{equation*}
      {\tilde p}^{i,j,t}_k = 
\begin{cases}
    p^{i,j,t}_k,& \text{if } k \ne i,j\\
    0,              & \text{otherwise}
\end{cases}
\end{equation*}
Then,
 \begin{equation*}
    {\mathbf{p}}^{i,j,t}= M^t   {\tilde{\mathbf{p}}}^{i,j,t-1}.
\end{equation*}
Let the (median) distances, from the data,  traveled between locations for time-step $t$, for each $(i,j)$, be 
$
d^t_{ij}
$.  Then,
 \begin{align*}
y^{i,j,t}_{k} &= \frac{1}{\sum_{r | r \ne i, j} m^t_{kr}   p^{i,j,t-1}_{r}}\sum_{r | r \ne i, j} (d^t_{kr} + y^{i,j,t-1}_{r})(m^t_{kr}   p^{i,j,t-1}_{r})\\
 &=\frac{1}{p^{i,j,t}_{k}}\sum_{r | r \ne i, j} (d^t_{kr} + y^{i,j,t-1}_{r})(m^t_{kr}   p^{i,j,t-1}_{r}),
  \end{align*}
  with probability $p^{i,j,t}_{k}$. We can organize the above computation as matrix operations. Define a matrix
  \begin{equation*}
    {\tilde D}^{i,j,t}= [{\tilde d}^{i,j,t}_{kr}],  
    \end{equation*}
where  
   \begin{align*}
{\tilde d}^{i,j,t}_{kr} &= \frac{d^t_{kr} + y^{i,j,t-1}_{r}}{p^{i,j,t}_{k}}.
  \end{align*}
 Then we can write:
  \begin{equation*}
    {\mathbf{y}}^{i,j,t}= \big [ {\tilde D}^{i,j,t} \cdot M^t  \big ]  {\tilde{\mathbf{p}}}^{i,j,t-1},
\end{equation*}
where the $\cdot$ in the above is the element by element multiplication, followed by the usual matrix product.

 Let us simplify notation and denote by $x^t_{ij}$ the mean distance traveled from $j$ to $i$ for paths ending at time-step $t$ that have not passed through $i,j$ except at the beginning, which in terms of $y^{i,j,t}_{k}$ is 
  \begin{equation*}
{x}^t_{ij} = y^{i,j,t}_{i},
\end{equation*} 
and denote the  probability of that path by $\pi^t_{ij}$,
  \begin{equation*}
\pi^t_{ij} = p^{i,j,t}_{i}.
\end{equation*} 
The mean distances of paths from $j$ to $i$  that end between $t_1 \le t \le t_2$ in terms of this  notation is, 
\begin{equation} \label{eq:xbar}
{\bar x}^{t_1,t_2}_{ij} = \frac{\sum^{t_2}_{t=t_1} {x}^t_{ij}  \pi^t_{ij} }{\sum^{t_2}_{t=t_1}\pi^t_{ij} }.
\end{equation} 
Note that replacing travel distance by travel time for each step, we can obtain the mean time taken from $j$ to $i$.

Time-elapsed distances between OD pairs over time-elapsed intervals spanning more than a single time-step, as in the case of net trip-count  above, estimate the distance traveled between locations as  a weighted mean of allowed intermediate path trip-distances over multiple paths, weighted by the relative path  probabilities. Allowed paths are the ones that leave  the origin at the beginning and end at the destination, without passing through either at any other intermediate time-step. 


\subsection{Normalization: time-elapsed OD effective distance}  \label{subsec:teeffdistcalc}

The utility of a distance measure, since it accounts for the direct and the indirect paths for each OD pair, might lie in providing a proxy for deviation from the most direct path. Note that each OD pair has a natural  geographical distance between its origin and destination associated with it.\footnote{This can be calculated from the origin and destination (longitude, latitude) pairs determined by their geohashes.}  Geographical distances are different for different OD pairs; so is the time-elapsed distance. Two OD pairs cannot, therefore, be compared by their time-elapsed distances, with the intention to find the one with higher potential deviation. Hence the need for normalization of the time-elapsed distance for each OD pair by a base distance for that pair. 

 There are two natural candidates for the base distance of an OD pair. First is the OD geographical distance.  Adopting this as the base for comparison runs into an issue: the mean (or median in our case) trip-distance provided for the same OD pair varies from  time-step to time-step, nor is it consistent relative to the geographical distance. It may be longer than the geographical distance or shorter depending on the time-step. Indeed, it  is not directly reflective of the local geography of travel, which is rarely in a straight line. 

The other obvious candidate  as the base distance for an OD pair is an appropriate  mean of the trip-distances for that pair over the entire time series. This is derived from, and directly reflective of the local geography,  characteristics and modes of travel. Here, we have the issue that not all pairs of locations that can result from time-elapsed movements are available in the data.

To accommodate both the issues, we impute the base distances. First, note that the trip-distances and the geographical distances are correlated. We obtain, through linear regression using weighted (by trip-counts) mean squared error (MSE),   the line of best fit  between the geographical distances (as the independent variable) and the median trip distances  from data (as the dependent variable). Using this relation, we can estimate the base OD pairwise trip distances from their geographical  distances. We now have an estimate of the base trip-distance/standard-deviation, which is the  mean standard-deviation of the OD trip-distance over the entire time-series, when that OD pair is in the data, or the estimated distance/root mean-squared error (RMSE) from the regression when it is not. 
We can now normalize the time-elapsed distance for each OD pair  by its base distance plus standard-deviation (to be conservative), and call it the {\it effective distance}. This normalization allows a comparison of any two OD pairs by their effective time-elapsed distances.  

For the regression, we do not keep self-transitions, since we are interested in OD pairs which have distinct origin and destination. Figure~\ref{fig3} shows the regression for Mexico City and surrounding areas (up to 10 km zone around Mexico City). Although the intercept is {\it negative}, we still use it for regression because it is seen as a nonlinear fit approximated by a linear fit. The line of fit is a crude fit at best, though it matches the overall trend. We will not obtain negative estimates for the base trip-distance because by the nature of GH5, minimum geographical distance would be about 4.9 km, which ensures a positive estimate for the base distance. The data appears in clumps of trip-distances for each origin-to-destination geographical distance, since the latter are a finite number of values based on the grid.  There is  high variance in each cluster, which indicates that a single independent variable is likely not the best choice and that the model is too simplistic. A better model would take into consideration the geography and employ a more sophisticated set of independent variables to perform the regression.  That is to be expected, as the terrain, nature of physical structures, routes, means of transport, and socio-economic conditions  vary widely among the locations that are otherwise at the same geographical distance.  Our intent is to provide a baseline method which can be expanded upon. 
%, and also because the independent variable (geographical distances) don not fall  below the zero-crossing in calculating our estimates of the median trip distances.
\begin{figure}[H]
\centering
\includegraphics[width=0.7\linewidth,height=0.35\textheight]{ciudaddemexico-mexico-3h_gh5_nonzero-intercept_lineoffitgeodist.png}
  \caption{
    Mexico City: Linear fit of the trip-distance to geographical distance. 
  }
  \label{fig3}
\end{figure}


\subsection{Results: time-elapsed OD effective distances} \label{subsec:teeffdistres}

From previous sections, we know that a PEP travels in a manner that changes with the time of the day.  We consider the morning time to calculate the time-elapsed effective distance.  Figure~\ref{fig4} is a scatter-plot of the  effective distances for the time-elapsed interval 6:00--9:00--12:00, for each day in 2019.\footnote{We use a staged method to calculate the mean trip-distance for the OD pairs which are in the data. For a given time-elapsed interval (including the date), for instance 06-05-2019, 6:00--9:00--12:00, we first look in that date{\/}interval for the OD pair and calculate its trip-distance mean. Absent that, we check that time-elapsed interval(here 6:00--9:00--12:00)  and the same weekday as the original date (06-05-2019 was a Wednesday)  in all the weeks. Next we search all the days in the data, still for the same time-elapsed interval. Finally, we try the entire dataset for that OD pair regardless of the time-elapsed interval.} We only show  those OD pairs that are in the top 0.95 quantile of the effective distances for that day. Each OD pair has a color corresponding to it. \footnote{Due to limitations of colored plotting, different pairs may correspond to the same color.} The legend shows the OD pair, its color in the scatter plot, and the frequency in days on which the OD pair is in the top 0.95 quantile of effective distances. Only a partial legend is visible, since the number of OD pairs is large. 
\begin{figure}[H]
\centering
\includegraphics[width=1\textwidth,height=0.5\linewidth]{ciudaddemexico-mexico_effective-distance_window-1_direct_imputation-local-mean+std-normalized_quantile-0.95_starttime-6_endtime-6_20190101_20191230_annotated.png} % 
  \caption{
    Mexico City,  daily, 6:00--9:00--12:00: time-elapsed  effective distances of OD pairs. Top 0.95 quantile per day. Legend shows the frequency of occurrence in days of the OD pairs. Circled: on  08-22-2019, OD pair (9g3qy,9g3qv) with effective distance 10.2.
  }
  \label{fig4}
\end{figure}
From the plot above, 08-22-2019 has a high effective distance value of 10.2, circled in black. This corresponds to the OD pair whose GH5 coordinates are  (9g3qy,9g3qv).\footnote{Hereon, we will refer to locations by their GH5 geohashes, leaving out the geohash level.} For this date, as a further illustration, we show on the map in  figure~\ref{fig5} the effective distances of OD pairs for trips in the time-elapsed interval 6:00--9:00--12:00, above quantile  0.985.  At this quantile, Figure~\ref{fig5} reveals the effective path  for the mentioned OD pair circled in black (navy-colored arrow, upper middle).  A possibility is that it indicates a tendency to travel on more indirect paths between the origin and destination, picked up by the time-elapsed estimate. The same OD pair occurs 163 days out of the year with a high effective distance. Underlying reasons, when such high effective distances are observed, would need further study. 


\begin{figure}[H]  
\centering
\begin{subfigure}{.3\textwidth}
\centering
\includegraphics[scale=0.33]{ciudaddemexico-mexico_effective-distance_direct_imputation-local-mean+std-normalized_quantile-0.985_20190822-06-00-00_20190822-09-00-00_annotated.png} 
  \caption{
%Top 0.985 quantile. 
  }
  \label{fig5}
  \end{subfigure}%
  \begin{subfigure}{.3\textwidth}
  \centering
  \includegraphics[scale=0.25]{ciudaddemexico-mexico_effective-distance_direct_imputation-local-mean+std-normalized_quantile-0_starthash_9g3qy_endhash_9g3qv_20190822-06-00-00_20190822-09-00-00.png}
  \caption{
%    OD path with the time-elapsed segments
  }
  \label{fig5a}
    \end{subfigure}%
  \begin{subfigure}{.3\textwidth}
  \centering
  \includegraphics[scale=0.25]{ciudaddemexico-mexico_effective-distance_direct_imputation-local-mean+std-normalized_quantile-0.95_20190822-06-00-00_20190822-09-00-00.png}
  \caption{
%Top 0.95 quantile. 
  }
  \label{fig6}
    \end{subfigure}%
    \caption{    Mexico City,  08-22-2019, time-elapsed  effective distances of OD pairs (a) 6:00--9:00--12:00: Top 0.985 quantile. (b) Actual path as single-interval segments: dashed green arrow is the outgoing segment from  the origin (9g3qy) during 6:00--9:00; dashed red arrow is the  incoming segment  to the destination (9g3qv) during 09:00--12:00. (c) 6:00--9:00--12:00:  Top 0.95 quantile.  }
\end{figure} 
We can also decompose the OD pair into  path segments that contributed to the time-elapsed trip. Figure~\ref{fig5a} shows the single time-step trips. The dashed green arrow is the segment leaving the origin (9g3qy) in the interval 6:00--9:00, and the dashed red arrow is the segment entering the destination (9g3qv)  in the interval 9:00--12:00. This decomposition makes it clear how the path acquired a high effective distance. In this example, a single two-segment path over two intervals contributed to the longer-than-expected path distance. The OD pair could have a direct path in which is much shorter than the two-segment path. That would make the occurrence of a long path such as this, be very abnormal. 


Let us relax the threshold and see if these high effective distance pairs fall in a broader pattern.  We reduce the quantile cut-off to 0.95 and show the time-elapsed effective distances in figure~\ref{fig6} for the same time-elapsed interval. We observe that the mentioned OD pair is part of a set of high effective distance OD pairs. This points to the possibility of studying effective distances as part of spatially persistent patterns, within the purview of  topological data analysis (TDA)~\cite{ps:phsbsg}. 


As emphasized earlier, this  is a demonstration of the methodology, and as such, we also take a look at those OD pairs which are not in the original data, but developed a path over successive time-steps. We call such pairs {\it globally-unconnected} OD pairs.  Figure~\ref{fig7} is the scatter-plot, for days in 2019,  of the effective distances of globally-unconnected OD pairs for trips in the time-elapsed interval 6:00--9:00--12:00. The plot only shows pairs above quantiles 0.95. Generally, the effective distances of such pairs are lower, because this is a restricted set. The normalization here is purely regression-based, since the OD pairs are not in the data. Recall that the normalization is not simply by the regression-obtained trip-distance, but by the sum of that with the standard-deviation (RMSE) of the fit. From the plot, 09-23-2019 has an OD pair with a high effective distance value of 2.16, circled in black. This corresponds to the OD pair (9g3w6,9g3w0). The same OD pair appears with a high effective distance on other days as well, 108 times during the year. 

\begin{figure}[H]
\centering
\includegraphics[width=1\textwidth,height=0.5\linewidth]{ciudaddemexico-mexico_effective-distance_window-1_indirect-global_imputation-mean+std-normalized_quantile-0.95_starttime-6_endtime-6_20190101_20191230_annotated.png} % 
  \caption{
    Mexico City,  daily, 6:00--9:00--12:00: time-elapsed  effective distances of globally-unconnected OD pairs. Top 0.95 quantile per day. Legend shows the frequency of occurrence in days of such OD pairs. Circled: on 09-23-2019, OD pair (9g3w6,9g3w0) with effective distance 2.16.
  }
  \label{fig7}
\end{figure}
 Figure~\ref{fig8} is the map showing effective distances above quantile 0.985. The effective path  for the mentioned OD pair (9g3w6,9g3w0) is  circled in black (navy-colored arrow). 

\begin{figure}[H]  
\centering
\begin{subfigure}{.3\textwidth}
\centering
\includegraphics[scale=0.33]{ciudaddemexico-mexico_effective-distance_indirect-global_imputation-mean+std-normalized_quantile-0.985_20190923-06-00-00_20190923-09-00-00_annotated.png}
  \caption{
%Top 0.985 quantile. 
  }
  \label{fig8}
  \end{subfigure}%
  \begin{subfigure}{.3\textwidth}
  \centering
\includegraphics[scale=0.25]{ciudaddemexico-mexico_effective-distance_direct_imputation-local-mean+std-normalized_quantile-0_starthash_9g3w6_endhash_9g3w0_20190923-06-00-00_20190923-09-00-00.png}
  \caption{
%    OD path with the time-elapsed segments
  }
  \label{fig8a}
    \end{subfigure}%
  \begin{subfigure}{.3\textwidth}
  \centering
\includegraphics[scale=0.25]{ciudaddemexico-mexico_effective-distance_indirect-global_imputation-mean+std-normalized_quantile-0.95_20190923-09-00-00_20190923-12-00-00.png}
  \caption{
%Top 0.95 quantile. 
  }
  \label{fig9}
    \end{subfigure}%
    \caption{    Mexico City,  09-23-2019,  time-elapsed  effective distances of globally-unconnected OD-pairs (a) 6:00--9:00--12:00: Top 0.985 quantile. (b) Actual paths as single interval segments:  dashed green arrows  are the outgoing segments from  the origin (9g3w6) during 6:00--9:00; dashed red arrows are the   incoming segments to the destination (9g3w0) during 09:00--12:00. (c) 09:00--12:00--15:00: Top 0.95 quantile.  }
\end{figure}

As before, we decompose the OD pair into  path segments that contributed to the time-elapsed trip. Figure~\ref{fig8a} shows the single time-step trips. Dashed green arrows are the outgoing segments from the origin 9g3w6 in the interval 6:00--9:00, and dashed red arrows are the incoming segments to  destination 9g3w0  in the interval 9:00--12:00. Here we see, in contrast to the previous case, two indirect paths that contribute simultaneously to the higher effective distance of the OD pair. 

Instead of the same time-elapsed interval at a different quantile, in figure~\ref{fig9} we look at a later, but overlapping time-elapsed interval, 09:00--12:00--15:00, on the same day, quantile above 0.95. We notice several non-neighboring OD pairs have joined the mentioned pair in the high effective distance set. This alerts us to the possibility of time-based persistence in effective distances. Combined with the spatial persistence, this is a strong indication that effective distance could be examined at different scales of persistence, scale being a topic of some interest in mobility research~\cite{ghb:uihmp,aal:shm}.  The recurrences over the course of days and over the span of a day at successive time-elapsed intervals in the general and the globally-unconnected cases, implies that there is a long-term persistence in such OD pairs, which may yield interesting flow analytical properties~\cite{cl:dacbdccff} and be amenable to space and time based topological analysis. 
In both time-plots, we see very high effective distances that occur infrequently, sometimes only once. It would be interesting to understand if and when, and under what conditions, such anomalous values happen, but that may require involving other information beyond the scope of this work.



 \section{\bf  Time-elapsed return-to-origin (RTO) distance, time, and speed} \label{sec:terto}

As a proxy for radius of gyration, we can calculate the travel distance (time) by a PEP to return to their origin (RTO), within a window of time-steps long enough to be sufficient. This is a special case of the time-elapsed OD distance traveled from the previous section: when the origin is  also the destination. We average the RTO over all the origins, weighted by the relative probabilities of a PEP starting at them.   By averaging over the origins, the RTO distance and time become  intrinsic measures of the movement patterns of  the community or geography to which they are associated.
%The details fo the calculation for the RTO distance are in appendix~\ref{appdx:tertocalc}.
 
 
\subsection{Derivation}  \label{subsec:tertocalc}


Setting $i=j$ in equation~\eqref{eq:xbar}, we obtain the  mean RTO distance for the path that starts at $j$ and that returns to $j$ between $t_1 \le t \le t_2$, 
\begin{equation*}
{\bar x}^{t_1,t_2}_{jj} = \frac{\sum^{t_2}_{t=t_1}{x}^t_{jj}  \pi^t_{jj} }{\sum^{t_2}_{t=t_1} \pi^t_{jj} }.
\end{equation*} 
Mean RTO distance for location $j$, for $1 \le t \le t_2$, excluding the first self-transition at time-step $t=1$, is
\begin{equation*}
{\bar x}^{2,t_2}_{jj} = \frac{\sum^{t_2}_{t=2} {x}^t_{jj}  \pi^t_{jj} }{\sum^{t_2}_{t=2}  \pi^t_{jj}}.
\end{equation*} 
To find the overall mean of RTOs for all $j \in C$, we consider the distribution of users that leave their origins, or {\it roam}, instead of all the users at each origin. This distribution at time-step $t=1$ is 
\begin{equation*}
p^r_j=\frac{\sum_{i | i \ne j} f^1_{ij}}{\sum_{i | i \ne j ,j} f^1_{ij}}.
\end{equation*} 
Taking the mean over the initial distribution $\{ p^r_j\}$, the overall mean RTO distance excluding the first self-transition, the  {\it roaming} RTO distance,  is 
\begin{equation*}
{\bar x}^r  = \sum_j p^r_j {\bar x}^{2,t_2}_{jj} .
\end{equation*} 
Including the first self-transition, the mean RTO distance for location $j$ is
\begin{equation*}
{\bar x}^{1,t_2}_{jj} = \frac{\sum^{t_2}_{t=1} {x}^t_{jj}  \pi^t_{jj} }{\sum^{t_2}_{t=1}  \pi^t_{jj}}.
\end{equation*}
Taking the appropriate mean over the initial user distribution over the origin locations, we can get the overall mean RTO distance including the first self-transition.
We estimate the initial user distribution. The probability, $p_j$, of a generic user being at location $j$ at the beginning of time-step $t=1$ was estimated in equation~\eqref{eq:genprob} as
\begin{equation*}
p^1_j=\frac{\sum_{i} f^1_{ij}}{\sum_{i,j} f^1_{ij}}.
\end{equation*}  
Then, the mean RTO distance including the first self-transition, the  {\it generic} RTO distance, is 
\begin{equation*}
{\bar x}  = \sum_j p^1_j {\bar x}^{1,t_2}_{jj}.
\end{equation*} 
The third and simplest case is the mean RTO distance when only the first self-transition is considered, i.e., the users that are {\it home} in the vicinity of  the location.  The relative distribution of such users over all the origins at time-step $t=1$ is 
\begin{equation*}
p^h_j=\frac{f^1_{jj}}{\sum_{j} f^1_{jj}}.
\end{equation*} 
Then, the mean RTO distance of such users, the {\it home} RTO distance, is
\begin{equation*}
{\bar x}^h  = \sum_j p^h_j {\bar x}^{1,2}_{jj} = \sum_j p^h_j d^1_{jj} 
\end{equation*} 
where $d^1_{jj}$ are the (median) distances traveled at time-step $t=1$,  while self-transitioning at $j$.

\subsection{Results} \label{subsec:tertores}

We calculate and plot the RTO distances for Mexico (Mexico City and surrounding region), Indonesia (Jakarta and surrounding region) and India (Mumbai and surrounding region). We do not include Colombia, for which only the data from November--December 2019 is available. In the plots we identify these regions with the city names: Mexico City, Jakarta, and Mumbai.  For each country, the calculations are the averages for the time-elapsed intervals 6:00--(--)--21:00 and 9:00--(--)--00:00, weighted by the appropriate initial total trip-counts for the locations and intervals. Before we show the  RTO calculations, we show the maximal component sizes in figure~\ref{fig10},\footnote{There is a missing point on the plot on 10-22-2019. The data on that date was corrupted and had to be omitted. All the subsequent time-plots are also missing that date.} which give an idea of the movement patterns in these cities. The component sizes of Mexico are much higher than India, implying movements further away in Mexico. In the calculations of RTO, these observations are seen from the point of view of  a PEP's daily movements. 
\begin{figure}[H] %
\centering
\includegraphics[width=1.0\textwidth,height=0.5\linewidth]{ciudaddemexico-mexico_jakarata-indonesia_mumbai-india_distance_rto_starttime-6_endtime-9_window-5_20190101-00_20191230-00_componentSize.png}
  \caption{
     Mexico City, Jakarta,  Mumbai: maximal component sizes.
  }
  \label{fig10}
\end{figure}


\subsubsection{RTO distance}
 In figures~\ref{fig8},~\ref{fig9}, we plot  the mean RTO {\it distances} against dates in 2019, for two contrasting cases. Figure~\ref{fig5} shows the RTO distances for the self-transitions, i.e., restricting the travel to be exactly one time-step: the {\it home} RTO distance. Figure~\ref{fig8} shows the RTO distances when the self-transitions are excluded, i.e., the PEP has to wander away before returning to the origin: the {\it roaming} RTO distance. They are an order of magnitude different as the former are sub-kilometer, while the latter are several kilometers.~\footnote{The roaming RTO distance for a region has a missing point on days when that region's maximal component size drops to 1 for the time intervals under consideration. Similarly for the roaming RTO times and speeds.}
 An interesting observation is that the home RTO distances  for all the countries  in figure~\ref{fig5}  show peaks over the weekends and the roaming RTO distances in figure~\ref{fig9}  show dips. This could be because of mostly moving in the vicinity over the weekends and commuting to work during weekdays. An instance  of each of these peaks and dips  for Mexico City is circled in black for 09-08-2019, which is a Sunday. Over the course of the year, we observe what might be seasonal effects on RTO distances as well.
 
From figure~\ref{fig11},  PEPs in Mexico, Indonesia and India  seem to have similar home distances, whereas  in figure~\ref{fig12}, the PEP in Mexico covers a longer roaming RTO distance than a PEP in India, with that in Indonesia somewhere between the two. This could imply longer commutes in Mexico for work than in India, and Indonesia in the middle. Before March and  and after October, India has a sharp drop in roaming RTO distance. This could be attributable to the very small maximal component size or seasonal effects on mobility.~\footnote{The weekly approximate periodicity or lack thereof is due to the same in the trip-distances from which the RTO distance is calculated.}
\begin{figure}[H] %width=0.6\linewidth,height=0.3\textheight
\centering
\begin{subfigure}{1.0\textwidth}
%\centering
\includegraphics[width=1\textwidth,height=0.5\linewidth]{ciudaddemexico-mexico_jakarata-indonesia_mumbai-india_distance_rto_starttime-6_endtime-9_window-5_20190101-00_20191230-00_only_annotated.png}
  \caption{
  }
  \label{fig11}
    \end{subfigure}%
   \hfill
    \begin{subfigure}{1.0\textwidth}
%    \centering
\includegraphics[width=1\textwidth,height=0.5\linewidth]{ciudaddemexico-mexico_jakarata-indonesia_mumbai-india_distance_rto_starttime-6_endtime-9_window-5_20190101-00_20191230-00_excluding_annotated.png}
  \caption{
  }
  \label{fig12}
      \end{subfigure}%
  \caption{
    Mexico City, Jakarta,  Mumbai  (a)  home RTO distances. Circled: Mexico City  has a peak on Sunday, 09-08-2019. (b)  roaming RTO distances. Circled: Mexico City has a dip on Sunday, 09-08-2019.}   
\end{figure}

\subsubsection{RTO time}
In figures~\ref{fig13},~\ref{fig14}, we plot  the mean RTO  times against dates in 2019, for the same cases as in figures~\ref{fig8},~\ref{fig9}, applying  the same designations {\it home} and {\it roaming}, respectively. These compare differently to the RTO distances. Home RTO time of India is higher than that of Mexico and Indonesia, which are similar in general. Roaming RTO time of Mexico is the highest with Indonesia the least, and India in the middle, with the trend getting more pronounced in the later half of the year. 

Together, the distance and time RTOs say  that a PEP in the home region in  India  spends more time traveling than a PEP in Mexico or Indonesia. In comparison, a roaming PEP in Mexico goes a considerably longer distance than a roaming PEP in India and takes a longer time, with Indonesia in between in distance and closer to India in time, but the proportions look different. To interpret this better, we need data on trip speeds, which we do not have direct access to in this dataset, but we create a notion of speed nonetheless from the RTO distances and times.                        


\begin{figure}[H] %width=0.6\linewidth,height=0.3\textheight
\centering
\begin{subfigure}[b]{1.0\textwidth}
%\centering
\includegraphics[width=1\textwidth,height=0.5\linewidth]{ciudaddemexico-mexico_jakarata-indonesia_mumbai-india_time_rto_starttime-6_endtime-9_window-5_20190101-00_20191230-00_only.png}
  \caption{
  }
  \label{fig13}
    \end{subfigure}%
    \hfill
    \begin{subfigure}[b]{1.0\textwidth}
%    \centering
\includegraphics[width=1\textwidth,height=0.5\linewidth]{ciudaddemexico-mexico_jakarata-indonesia_mumbai-india_time_rto_starttime-6_endtime-9_window-5_20190101-00_20191230-00_excluding.png}
  \caption{
  }
  \label{fig14}
      \end{subfigure}%
  \caption{
    Mexico City, Jakarta,  Mumbai  (a)  home RTO times (b)  roaming RTO times. }   
\end{figure}

\subsubsection{RTO speed}
We can combine the RTO distances and times to get the RTO {\it speeds}. We  define  RTO {\it speed}, as a means  of comparison among the cities, to be the RTO distance traveled divided by the RTO travel time.\footnote{This is a bold definition of speed because the mean of ratios is not the ratio of means. Perhaps it should be called pseudo-speed.}  Figures~\ref{fig15},~\ref{fig16} show thus defined home and roaming RTO speeds. 


\begin{figure}[H] %width=0.6\linewidth,height=0.3\textheight
\centering
\begin{subfigure}[b]{1.0\textwidth}
%\centering
\includegraphics[width=1\textwidth,height=0.5\linewidth]{ciudaddemexico-mexico_jakarata-indonesia_mumbai-india_speed_rto_starttime-6_endtime-9_window-5_20190101-00_20191230-00_only.png}
  \caption{
  }
  \label{fig15}
    \end{subfigure}%
    \hfill
    \begin{subfigure}[b]{1.0\textwidth}
%    \centering
\includegraphics[width=1\textwidth,height=0.5\linewidth]{ciudaddemexico-mexico_jakarata-indonesia_mumbai-india_speed_rto_starttime-6_endtime-9_window-5_20190101-00_20191230-00_excluding.png}
  \caption{
  }
  \label{fig16}
      \end{subfigure}%
  \caption{
    Mexico City, Jakarta,  Mumbai  (a)  home RTO speeds (b)  roaming RTO speeds. }   
\end{figure}


Home RTO speed, from the plot in figure~\ref{fig15},  for Indonesia is the highest and India the lowest, with Mexico being closer to Indonesia during the first half of the year and to India in the second half. In roaming RTO speed, from figure~\ref{fig16}, Mexico is the highest of the three, and India the lowest, and Indonesia closer to Mexico.  Toward the beginning and the end of the year, Indonesia and India seem to be similar in both home and roaming speeds. 

We can try to identify some of the factors at work here  based on published data. Slower speeds in India would be consistent with the population density of the  countries. According to data from World Bank Group collected in 2022, the population density of Mexico is approximately 66 people per square kilometer, of Indonesia about 147 people per square kilometer and  of India  about 479 people per square kilometer~\cite{un:wp1}.
On the other had, comparing the most  populous cities of Indonesia, India and Mexico, (Jakarta, Mumbai and Mexico City, respectively), according to data from the United Nations Human Settlements Programme collected in 2020, the percentage of the population of Mexico City with convenient access to public transport   is 43.3\% , that of Jakarta is 54.4\%, whereas that of Mumbai is 80.9\%~\cite{un:wp2}. This might make up for effects of differing population densities. The exact relationship between movement speeds and other variables would have to be studied further.

 \section{\bf  Conclusion} \label{sec:conc}
In this paper we addressed the outstanding problem of estimating longer-term mobility from  space and time aggregated collective data. Starting with the pseudo Markov-chain model, we developed measures and algorithms for time-elapsed net trip-counts (flow) between OD pairs, time-elapsed distance and {\it effective distance}  and return-to-origin (RTO) distance, time and speed. 

Despite the mobility-aliasing concern due to the longer intervals used for aggregation in the data we have available, the Netmob 2024 dataset, we demonstrate the measures. First validation is that of the time-elapsed OD net trip-counts (net flows)  and the results we get  are consistent with the expected commuting behaviors. With more detailed data and over longer time periods, it might show cycles of migration in societies.  

 

Time-elapsed OD effective distance seems to have persistence over both space and time. This points to an additional direction of study,  in which they could be used to identify recurring patterns in effective distances that are higher than expected, and an invitation to extend the time-elapsed measures to different spatio-temporal scales, study their topological properties, and develop further measures that may identify them by flow-like properties.  Anomalous values of effective distances could present opportunities to investigate other datasets for explanations and to develop algorithms that identify them. 

Return-to-origin (RTO) distance, time and speed calculations are a first-order characterization  of the socio-economic activity rooted in geographical areas. Their variation over different geographical areas and environmental conditions over time have the potential to be useful in analyzing community level organization and behaviors, economic activity, efficiency, and even resilience in times of shocks. 

At larger scales in space and time and with rich enough data, this collection of tools would potentially be useful in showing population migration, or the impact of weather or economic events on population movements. They could be combined with other geographical  or demographic, social, and economic data, for example, to serve as sources for analyses to draw deeper inferences or as the ground truth for validation and prediction. To make the model closer to the actual dynamics of movement, constraints would be needed that may not agree with the pseudo-Markov model and might add complexity or time-dependence in the model structure, or may need a generalization of the model. Seasonal variations  occurring  in social structures take a prominent role in the  anthropological texts dealing with human history and social organizations~\cite{gw:doe}, and links to mobility flows and persistence, among other measures, might help uncover patterns or discern among them.

In the context of the Netmob 2024 dataset, an  application would be to glean large scale drift of people. Among the Sustainable Development Goals~\cite{un:sdg} that could be within the scope of future work, once relevant data are included,  are SDG 8 -- Decent work and economic growth, SDG 9--Industry, innovation and infrastructure, SDG 11 -- Sustainable cities and communities, and SDG13 -- Climate action. 



 
\section*{Acknowledgements}
The authors would like to acknowledge productive participation from the members of David Meyer's research group, in  particular, Itai Maimon for initiating the discussion that led to the return-to-origin definition, Orest Bucicovschi, David Rideout and Jiajie Shi.
 
 
 \clearpage
 
\begin{bibdiv}
\begin{biblist}




\bib{dgbcd:opcumpd}{article}{
	  author = {de Montjoye, Y.},
	  author = {Gambs, S.},
	  author = {Blondel, V.},
	  author = {Canright, G.},
	  author = {de Cordes, N.},
	  author = {Deletaille, S.},
	  author = {Eng{\o}-Monsen, K.},
	  author = {Garcia-Herranz, M.},
	  author = {Kendall, J.},
	  author = {Kerry, C.},
	  author = {Krings, G.},
	  author = {Letouz{\'e}, E.},
	  author = {Luengo-Oroz, M.},
	  author = {Oliver, N.},
	  author = {Rocher, L.},
	  author = {Rutherford, A.},
	  author = {Smoreda, Z.},
	  author = {Steele, J.},
	  author = {Wetter, E.},
	  author = {Pentland, A.},
	  author = {Bengtsson, L.},
date={2018},
  title={On the privacy-conscientious use of mobile phone data},
  journal={Scientific Data},
%SP  - 180286
  volume={5},
number={1},

   note= {\href{https://doi.org/10.1038/sdata.2018.286}{doi:10.1038/sdata.2018.286}}
}


\bib{bbccd:amdchfc}{article}{
	  author = {Buckee, C.},
	  author = {Balsari, S.},
	  author = {Chan, J.},
	  author = {Crosas, M.},
	  author = {Dominici, F.},
	  author = {Gasser, U.},
	  author = {Grad, Y.},
	  author = {Grenfell, B.},
	  author = {Halloran, M.},
	  author = {Kraemer, M.},
	  author = {Lipsitch, M.},
	  author = { Metcalf, C.},
	  author = {Meyers, L.},
	  author = {Perkins ,T.},
	  author = {Santillana, M.},
	  author = {Scarpino, S.},
	  author = {Viboud, C.},
	  author = {Wesolowski, A.},
	  author = {Schroeder, A.},
date={2020},
  title={Aggregated mobility data could help fight COVID-19},
  journal={Science},
%SP  - 180286
pages={145-146},
  volume={368},
number={6487},

   note= {\href{https://doi.org/10.1126/science.abb8021}{doi:10.1126/science.abb8021}}
}

\bib{khdr:tmumtlsd}{article}{
  author={Kondor, D.},
  author={Hashemian, B.},
  author={de Montjoye, Y.},
  author={Ratti, C.},
  journal={IEEE Transactions on Big Data}, 
  title={Towards Matching User Mobility Traces in Large-Scale Datasets}, 
  year={2020},
  volume={6},
  number={4},
  pages={714-726},
  %doi={10.1109/TBDATA.2018.2871693},
    note= {\href{https://doi.org/10.1109/TBDATA.2018.2871693}{doi:10.1109/TBDATA.2018.2871693}}
    }
  
  
  \bib{xtlzfj:trfa}{inproceedings}{
author = {Xu, F.},
author = {Tu, Z.},
author =  {Li, Y.},
author =  {Zhang, P.},
author =  { Fu, X.},
author =  { Jin, D.},
title = {Trajectory Recovery From Ash: User Privacy Is NOT Preserved in Aggregated Mobility Data},
year = {2017},
isbn = {9781450349130},
publisher = {International World Wide Web Conferences Steering Committee},
address = {Republic and Canton of Geneva, CHE},
%url = {https://doi.org/10.1145/3038912.3052620},
%doi = {10.1145/3038912.3052620},
          note= {\href{https://doi.org/10.1145/3038912.3052620}{doi:10.1145/3038912.3052620}},
booktitle = {Proceedings of the 26th International Conference on World Wide Web},
pages = {1241–1250},
series = {WWW '17}
}

\bib{ghb:uihmp}{article}{
title = {Understanding individual human mobility patterns},
journal = {Nature},
volume = {453},
pages = {779-782},
year = {2008},
%doi = {https://doi.org/10.1038/nature06958},
%url = {https://doi.org/10.1038/nature06958},
author = {Gonz{\'a}lez, M.},
author = {Hidalgo, C.},
author = {Barab{\'a}si, A.},
  note= {\href{https://doi.org/10.1038/nature06958}{doi:10.1038/nature06958}}
}

\bib{bbgj:hmma}{article}{
title = {Human mobility: Models and applications},
journal = {Physics Reports},
volume = {734},
pages = {1-74},
year = {2018},
%note = {Human mobility: Models and applications},
%issn = {0370-1573},
%doi = {https://doi.org/10.1016/j.physrep.2018.01.001},
%url = {https://www.sciencedirect.com/science/article/pii/S037015731830022X},
author = {Barbosa, H.},
author = {Barthelemy, M.},
author = {Ghoshal, G.},
author = {James, C.},
author = {Lenormand, M.},
author = {Louail, T.},
author = {Menezes, R.},
author = {Ramasco, J.},
author = {Simini, F.},
author = {Tomasini, M.},
%keywords = {Human mobility, Human dynamics, Random walks, Origin–destination matrices},
  note= {\href{https://doi.org/10.1016/j.physrep.2018.01.001}{doi:10.1016/j.physrep.2018.01.001}}

}


\bib{jyz:chmlsn}{article}{
  title = {Characterizing the human mobility pattern in a large street network},
  author = {Jiang, B.},
  author = {Yin, J.},
  author = {Zhao, S.},
  journal = {Phys. Rev. E},
  volume = {80},
  %issue = {2},
  pages = {021136},
  year = {2009},
  %month = {Aug},
  publisher = {American Physical Society},
  %doi = {10.1103/PhysRevE.80.021136},
  %url = {https://link.aps.org/doi/10.1103/PhysRevE.80.021136},
        note= {\href{https://doi.org/10.1103/PhysRevE.80.021136}{doi:10.1103/PhysRevE.80.021136}}

}

\bib{bhg:slht}{article}{
title = {The scaling laws of human travel},
journal = {Nature},
volume = {439},
number={7075},
pages = {462-465},
year = {2006},
%note = {Human mobility: Models and applications},
author = {Brockmann, D.},
author = {Hufnagel, L.},
author = {Geisel, T.},
  note= {\href{https://doi.org/10.1038/nature04292}{doi:10.1038/nature04292}}
}


\bib{zpgm:nmd}{article}{
      title={The NetMob2024 Dataset: Population Density and OD Matrices from Four LMIC Countries}, 
      author={Zhang, W.},
      author={Nunez del Prado, M.},
      author={Gauthier, V.},
      author={Milusheva, S.},
      year={2024},
      %eprint={2410.00453},
      %archivePrefix={arXiv},
      %primaryClass={cs.NI},
      %url={https://arxiv.org/abs/2410.00453},
           note= {\href{https://doi.org/10.48550/arXiv.2410.00453}{doi:10.48550/arXiv.2410.00453}}
}

\bib{wtded:mhmumpr}{article}{
    %doi = {10.1371/journal.pone.0133630},
    author = {Williams, N.},
        author = {Thomas, T.},
            author = {Dunbar, M.},
                author = {Eagle, N.},
                    author = {Dobra, A.},
    journal = {PLOS ONE},
    publisher = {Public Library of Science},
    title = {Measures of Human Mobility Using Mobile Phone Records Enhanced with GIS Data},
    year = {2015},
    %month = {07},
    volume = {10},
    %url = {https://doi.org/10.1371/journal.pone.0133630},
    pages = {1-16},
    number = {7},
      note= {\href{https://doi.org/10.1371/journal.pone.0133630}{doi:10.1371/journal.pone.0133630}}

}


\bib{ysctw:iawsssd}{article}{
AUTHOR = {Yang, C.},
AUTHOR = {Sutrisno, H.},
AUTHOR = {Chan, A.},
AUTHOR = {Tampubolon, H.},
AUTHOR = {Wibowo, B.},
TITLE = {Identification and Analysis of Weather-Sensitive Roads Based on Smartphone Sensor Data: A Case Study in Jakarta},
JOURNAL = {Sensors},
VOLUME = {21},
YEAR = {2021},
NUMBER = {7},
%ARTICLE-NUMBER = {2405},
%url = {https://www.mdpi.com/1424-8220/21/7/2405},
%PubMedID = {33807222},
ISSN = {1424-8220},
  note= {\href{https://www.mdpi.com/1424-8220/21/7/2405}{url:www.mdpi.com/1424-8220/21/7/2405}}
}



\bib{ypm:ilablumddud}{article}{
AUTHOR = {Yang, Y.},
AUTHOR = {Pentland, A.},
AUTHOR = {Moro, E.},
YEAR = {2023},
TITLE = {Identifying latent activity behaviors and lifestyles using mobility data to describe urban dynamics},
JOURNAL = {EPJ Data Science},
VOLUME = {12},
number={1},
PAGES={2193-1127},
%%url = {https://doi.org/10.1140/epjds/s13688-023-00390-w},
%doi = {10.1140/epjds/s13688-023-00390-w},
      note= {\href{https://doi.org/10.1140/epjds/s13688-023-00390-w}{doi:10.1140/epjds/s13688-023-00390-w}}
} 

\bib{wlcqjb:e15mcpaud}{article}{
title = {Evaluating the 15-minute city paradigm across urban districts: A mobility-based approach in Hamilton, New Zealand},
journal = {Cities},
volume = {151},
pages = {105147},
year = {2024},
issn = {0264-2751},
%doi = {https://doi.org/10.1016/j.cities.2024.105147},
%url = {https://www.sciencedirect.com/science/article/pii/S0264275124003615},
author = {Wang, T.},
author = {Li, Y.},
author = {Chuang, I.},
author = {Qiao, W.},
author = {Jiang, J.},
author = {Beattie, L.},
      note= {\href{https://doi.org/10.1016/j.cities.2024.105147}{doi:10.1016/j.cities.2024.105147}}
}


\bib{mzbwt:ursltm}{article}{
author = {Milusheva, S.},
author = {zu Erbach-Schoenberg, E.},
author = { Bengtsson, L.},
author = {Wetter,E.},
author = {Tatem , A.},
title = {Understanding the Relationship between Short and Long Term Mobility},
journal = { AFD Research Paper Series},
number = { 2017-69},
year = {2017}
%      note= {\href{https://www.afd.fr/en/ressources/understanding-relationship-between-short-and-long-term-mobility}{url:www.afd.fr/en/ressources/understanding-relationship-between-short-and-long-term-mobility}}
}

\bib{bbbc:uhmfampd}{article}{
title = {Understanding Human Mobility Flows from Aggregated Mobile Phone Data},
journal = {IFAC-PapersOnLine},
volume = {51},
number = {9},
pages = {25-30},
year = {2018},
%note = {15th IFAC Symposium on Control in Transportation Systems CTS 2018},
issn = {2405-8963},
%doi = {https://doi.org/10.1016/j.ifacol.2018.07.005},
%url = {https://www.sciencedirect.com/science/article/pii/S2405896318307213},
author = {Balzotti, C.},
  author={Bragagnini A.},
    author={Briani M.},
      author={Cristiani, E.},
          note= {\href{https://doi.org/10.1016/j.ifacol.2018.07.005}{doi:10.1016/j.ifacol.2018.07.005}}
}


\bib{jwsl:chmpttua}{article}{
    author = {Peng, C.},
    author = {Jin, X.},
    author = { Wong, K.},
    author = { Shi, M.},
    author = {Li{\`o}, P.},
    journal = {PLOS ONE},
    publisher = {Public Library of Science},
    title = {Collective Human Mobility Pattern from Taxi Trips in Urban Area},
    year = {2012},
    %month = {04},
    volume = {7},
 %   url = {https://doi.org/10.1371/journal.pone.0034487},
    pages = {1-8},
    number = {4},
               note= {\href{https://doi.org/10.1371/journal.pone.0034487}{doi:10.1371/journal.pone.0034487}}

}

\bib{gyzh:uichmptceos}{article}{
title = {Understanding individual and collective human mobility patterns in twelve crowding events occurred in Shenzhen},
journal = {Sustainable Cities and Society},
volume = {81},
pages = {103856},
year = {2022},
issn = {2210-6707},
%doi = {https://doi.org/10.1016/j.scs.2022.103856},
%url = {https://www.sciencedirect.com/science/article/pii/S2210670722001834},
author = {Guo, B.},
author = {Yang, H.},
author = {Zhou, H.},
author = {Huang, Z.},
author = {Zhang, F.},
author = {Xiao, L.},
author = {Wang, P.},
           note= {\href{https://doi.org/10.1016/j.scs.2022.103856}{doi:10.1016/j.scs.2022.103856}}
}

\bib{fdgm:mngmc}{article}{
journal = {Scientific Data},
volume = {11},
pages = {84},
year = {2024},
title = {Mobility networks in Greater Mexico City},
%issn = {0370-1573},
%doi = {10.1038/s41597-023-02880-y},
%url = {https://doi.org/10.1038/s41597-023-02880-y},
author = {Flores-Garrido, M.},
author = {de Anda-J{\'a}uregui, G.},
author = {Guzm{\'a}n, P.},
author = {Meneses-Viveros, A.},
author = {Hern{\'a}ndez-{\'A}lvarez, A.},
author = {Cruz-Bonilla, E.},
author = {Hern{\'a}ndez-Rosales, M.},
      note= {\href{https://doi.org/10.1038/s41597-023-02880-y}{doi:10.1038/s41597-023-02880-y}}
}

 \bib{nm:nm2024}{webpage}{
title = {NetMob 2024},
%url = {https://netmob.org/},
      note= {\href{https://netmob.org/}{url:netmob.org}}
}


%\bib{sp:req}{misc}{
%%organization={Cuebiq},
%note={Aggregated data was provided by Cuebiq Social Impact as part of the Netmob 2024 conference. Data is collected with the informed consent of anonymous users who have opted-in to anonymized data collection for research purposes. In order to further preserve the privacy of users, all data has been aggregated by the data provider spatially to the GH3, GH5, h37  and temporally to 3-hourly, daily, weekly, and monthly levels and does not include any individual-level data records.}
%}




\bib{yltlgm:ehmrwosd}{article}{
author = {Yabe, T.},
author = {Luca, M.},
author = { Tsubouchi, K.},
author = {Lepri, B.},
author = {Gonzalez, M.},
author = {Moro, E.},
title = {Enhancing human mobility research with open and standardized datasets},
journal = { Nat Comput Sci},
volume = { 4},
pages = { 469-472},
year = {2024}, 
%doi = { https://doi.org/10.1038/s43588-024-00650-3},
      note= {\href{https://doi.org/10.1038/s43588-024-00650-3}{doi:10.1038/s43588-024-00650-3}}
}

\bib{n:mc}{book}{
      author={Norris, J.},
       title={Markov Chains},
   publisher={Cambridge University Press},
     address={New York},
        date={1997},
        ISBN={0521633966 9780521633963},
         place={Cambridge}, 
         series={Cambridge Series in Statistical and Probabilistic Mathematics},
         %collection={Cambridge Series in Statistical and Probabilistic Mathematics},
        note={\href{https://doi.org/10.1017/CBO9780511810633}{doi:10.1017/CBO9780511810633}}
        
}

 



\bib{ps:phsbsg}{inproceedings}{
booktitle = {Smart Tools and Apps for Graphics - Eurographics Italian Chapter Conference},
title = {Persistent Homology: a Step-by-step Introduction for Newcomers},
author = {Fugacci, U.}, 
author = {Scaramuccia, S.},
author = {Iuricich, F.},
author = {Floriani, L.},
year = {2016},
publisher = {The Eurographics Association},
ISBN = {978-3-03868-026-0},
%DOI = {10.2312/stag.20161358
%}
      note= {\href{https://doi.org/10.2312/stag.20161358}{doi:10.2312/stag.20161358}}

}

\bib{aal:shm}{article}{
author = {Alessandretti, L.},
author = {Aslak, U.},
author = {Lehmann, S.},
year = {2020},
title = {The scales of human mobility},
journal = {Nature},
pages={402-407},
volume = { 587},
number={7834},
url={https://doi.org/10.1038/s41586-020-2909-1},
%doi={10.1038/s41586-020-2909-1},
      note= {\href{https://doi.org/10.1038/s41586-020-2909-1}{doi:10.1038/s41586-020-2909-1}}
}

\bib{cl:dacbdccff}{article}{
title = {Detecting abnormal crowd behaviors based on the div-curl characteristics of flow fields},
journal = {Pattern Recognition},
volume = {88},
pages = {342-355},
year = {2019},
issn = {0031-3203},
%doi = {https://doi.org/10.1016/j.patcog.2018.11.023},
%url = {https://www.sciencedirect.com/science/article/pii/S003132031830414X},
author = {Chen, X.},
author = {Lai, J.},
      note= {\href{https://doi.org/10.1016/j.patcog.2018.11.023}{doi:10.1016/j.patcog.2018.11.023}}

}


\bib{un:wp1}{webpage}{
title = {Population density (people per sq. km of land area), World Bank Open Data},
author = {World Bank Group},
year = {2022},
%url = {https://data.worldbank.org/indicator/EN.POP.DNST?view=map},
note= {\href{https://data.worldbank.org/indicator/EN.POP.DNST?view=map}{url:data.worldbank.org/indicator/EN.POP.DNST?view=map}}
}

\bib{un:wp2}{webpage}{
title = {Urban Transport, UN-Habitat Urban Indicators Database},
author = {United Nations Human Settlements Programme},
year = {2024},
%url = {https://data.unhabitat.org/pages/urban-transport},
     note= {\href{https://data.unhabitat.org/pages/urban-transport}{url:data.unhabitat.org/pages/urban-transport}}
}


\bib{gw:doe}{book}{
      author={Graeber, D.},
      author={ Wengrow, D.},
       title={The dawn of everything: a new history of humanity},
        date={2021},
      %doi={https://doi.org/10.1007/978-3-642-00234-2},
   publisher={Farrar, Straus and Giroux},
     address={New York},
             ISBN={9780374157357; 0374157359},
           note= {\href{https://worldcat.org/title/1255689164}{WorldCat:1255689164}}
}

\bib{un:sdg}{webpage}{
title = {Sustainable Development Goals},
%url = {https://data.unhabitat.org/pages/urban-transport},
     note= {\href{https://www.undp.org/sustainable-development-goals}{url:undp.org/sustainable-development-goals}}
}

\bib{hl:oprosm}{article}{
title = {On pth roots of stochastic matrices},
journal = {Linear Algebra and its Applications},
volume = {435},
number = {3},
pages = {448-463},
year = {2011},
%note = {Special Issue: Dedication to Pete Stewart on the occasion of his 70th birthday},
%issn = {0024-3795},
%doi = {https://doi.org/10.1016/j.laa.2010.04.007},
%url = {https://www.sciencedirect.com/science/article/pii/S0024379510001849},
author = {Higham, N.},
author = {Lin, L.},
      note= {\href{https://doi.org/10.1016/j.laa.2010.04.007}{doi:10.1016/j.laa.2010.04.007}}
}

\bib{dd:eod}{book}{
      author={Deza, E.},
      author={Deza, M.},
       title={Encyclopedia of Distances},
        date={2009},
      %doi={https://doi.org/10.1007/978-3-642-00234-2},
   publisher={Springer-Verlag},
     address={Berlin},
           note= {\href{https://doi.org/10.1007/978-3-642-00234-2}{doi:10.1007/978-3-642-00234-2}}
}

\end{biblist}
\end{bibdiv}
\newpage

\appendix

 \section{\bf  Proposed data collection method and algorithmic Markov matrix resampling  to compensate for mobility aliasing} \label{appdx:mobaliasalg}

The time-resolution of data could be improved by changing the intervals of data aggregation to be shorter, commensurate with a statistically useful measure of speed, perhaps a standard deviation above the mean speed.  Suppose that speed is known and translates to an interval time of $T$ minutes. The aggregation would record  the trips that start in an interval $mT$ to $(m+1)T$ and end in an interval $nT$ to $(n+1)T$. For each $m$ and origin there will only be finitely many $n$ and destinations. The  record would thus be indexed by the OD pair and $m$ and $n$ (equivalently $mT$ and $nT$), a quadruple of indices, and include as data quantities of interest like trip-count, the mean (or median)  time and distance traveled. The mean/median for the  traveled time would likely be longer than the interval and those for distance could be shorter or longer  than the spatial resolution of the geohash.   This approach, while not eliminating mobility-aliasing, would help reduce it.

In the present case, the time-resolution, as noted, is coarser than needed, and potentially leads to aliasing in user and trip counts. To account for the aliasing, we could try the following idea. Continuing with the example of 3-hour, GH5, we assume that the dynamics of movements are slowly changing, at 3-hour intervals. For each coarse time-step, we need to insert $p$ transition matrices, one for each finer interval, such that their product is the original transition matrix. Denote by $H_k$ the finer-resolution transition matrices where $k=1,\ldots,p$. Omitting the superscript $t$ for the original time-step, and writing $M=M^t$,
\begin{equation*}
M=\prod^p_{k=1} H_k.
\end{equation*}
A more stringent version of this is if all the finer resolution matrices are assumed to be the same, i.e., $H_k =  H$. Then
\begin{equation*}
M=H^p.
\end{equation*}
Such a stochastic matrix $H$ is called the $p$-th root of a stochastic matrix $M$. It might not exist, however, and finding one is a challenge~\cite{hl:oprosm}.

An approximation might be attempted in finding the $p$-th root, in some distance of distance-like measure. Let the measure be $K$. The optimal approximation to the root would correspond to the minimum distance achievable over the space of stochastic matrices. Let the minimum distance be $K_{\text{min}}$:
 \begin{equation*}
K_{\text{min}} = \min_{B : B {\text{ stochastic}}}K(M,B^p).
\end{equation*}
Then the optimal approximation is $\hat H $ such that:
 \begin{equation*}
K(M,{\hat H}^p) = K_{\text{min}}.
\end{equation*}
Among such  measures could conceivably be the Relative Entropy (Kullback-Leibler divergence) or the Frobenius distance~\cite{dd:eod}. The authors have not been able to find fast converging algorithms for the approximation for the size of stochastic matrices in this paper.
 
In summary, we would need to iterate $p$ times the approximate stochastic $p$-th root of the 3-hourly transition matrix, where $p$ is the number of shorter steps in the 3-hour interval. For instance, $p= 4$ if the shorter interval is 45 minutes. The reason for choosing the approximate $p$-th root is so that the transition probabilities at the  3-hour interval are consistent with it.  We expect it to yield   different estimates for the distance and time measures that we have developed than those using the 3-hourly transition matrix, for the reason that mean trip-distance and trip-times are additive over path segments. 




\end{document}
