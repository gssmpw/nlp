\clearpage
\onecolumn
\section{Reproducibility Statement}\label{sec:reproduce}
\noindent\textbf{Source Code.} We release the source code in the \href{https://github.com/kmswin1/Syntriever}{public repository}.

\noindent\textbf{Black-box LLMs.} We experiment with \texttt{GPT-4o} and \texttt{GPT-4o-mini} for synthetic data generation. Those models are accessible by \href{https://platform.openai.com/docs/overview}{OpenAI API}. We generate LLMs' responses using our prompt templates in Appendix~\ref{appendix:prompts}.

\noindent\textbf{Training Implementation.} We implement retrieval tasks based on BEIR~\cite{thakur2021beir} which is  implemented by \href{https://sbert.net/}{sentence-transformers}. 
All the sentence encoders used in our experiments are publicly accessible in \href{https://huggingface.co/models}{HuggingFace}.

\noindent\textbf{Evaluation Datasets.} The four evaluation datasets of HotpotQA, FiQA, SciFact, and NFCorpus are released to \href{https://github.com/beir-cellar/beir}{the public repository}. 
All the experiments are conducted with the single A100 GPU with 80GB VRAM.

\section{Derivation of (\ref{eq:ml})}\label{appendix:proof}
For notational simplicity, we define \(
z_k := \exp({r(q,y_{\pi(k)})})\). We have that
\begin{align}
   \sum_{\pi(3),\ldots,\pi(M)}&p(\pi \, |\, q) = \sum_{\pi(3),\ldots,\pi(M)}\prod_{m=1}^M \left(\dfrac{z_m}{\sum_{j=m}^M z_j}\right)\label{eq:p1}
   \\&=\sum_{\pi(3),\ldots,\pi(M)}\underbrace{\dfrac{z_1}{\sum_{j=1}^M z_j}}_{(a)}\underbrace{\dfrac{z_2}{\sum_{j=2}^M z_j}}_{(b)} \prod_{m=3}^M \left(\dfrac{z_m}{\sum_{j=m}^M z_j}\right)\label{eq:p2}
   \\&=\underbrace{\dfrac{z_1}{\sum_{j=1}^M z_j}}_{(a)}\underbrace{\dfrac{z_2}{\sum_{j=2}^M z_j}}_{(b)} \underbrace{\sum_{\pi(3),\ldots,\pi(M)}\prod_{m=3}^M \left(\dfrac{z_m}{\sum_{j=m}^M z_j}\right)}_{(c)} \label{eq:p3}
   \\&=\underbrace{\dfrac{z_1}{\sum_{j=1}^M z_j}}_{(a)}\underbrace{\dfrac{z_2}{z_2 + \sum_{j\neq 1,2} z_j}}_{(b)} \label{eq:p4}
\end{align}
The derivation steps are explained as follows. \begin{itemize}
    \item 
(\ref{eq:p1}) is the definition of $p(\pi|q)$ from Placket-Luce ranking model. \item In (\ref{eq:p2}), we take two terms (a) and (b) out from the product in  (\ref{eq:p1}). \item In (\ref{eq:p3}), (a) and (b) are invariant with respect to $\pi(3),\ldots,\pi(M)$, so they can be taken out of the summation with respect to $\pi(3),\ldots,\pi(M)$. Specifically, the denominators of (a) and (b) contains the sums over $\pi(3),\ldots,\pi(M)$, and the sum is a permutation-invariant operation. Also, (c) is the sum of Placket-Luce distribution over all possible permutations of $\pi(3),\ldots,\pi(M)$. Thus, (c) must sum up to 1. \item In (\ref{eq:p4}), we simply re-write (b) in (\ref{eq:p3}) as shown in (b) in (\ref{eq:p4}). 
\end{itemize}
By replacing $z_i$ with $\exp(r(q,y_{\pi(i)}))$ in (\ref{eq:p4}), we obtain the marginalization result in Eq. (\ref{eq:ml}).


\newpage

\section{Prompt templates}\label{appendix:prompts}

\subsection{Prompt template of positive passage generation $(\mathcal P_+)$.}

\begin{figure}[h!]
    \centering
    \fbox{
    \begin{minipage}{35em}
    You are a subject matter expert in your field with substantial accumulated knowledge in a specific subject or topic, validated by academic degrees, certifications, and/or years of professional experience in that field.\\\\
    Question: \{question\}\\\\
    Write a passage that elaborates on the question. Ensure that no false information is provided; all content must be entirely accurate. Present everything you are aware of, offering a comprehensive and detailed explanation. Do not include any unverified or speculative information.
    \end{minipage}
    }
    \caption{Prompt template design for generating synthetic positive passages. }
    \label{fig:prompt_passage}
\end{figure}

\subsection{Prompts for generating plausible but irrelevant passages $(\mathcal P_-)$.}

\begin{figure}[h!]
    \centering
    \fbox{
    \begin{minipage}{35em}
    You are a subject matter expert in your field with substantial accumulated knowledge in a specific subject or topic, validated by academic degrees, certifications, and/or years of professional experience in that field.\\\\
    Question: \{question\}\\\\
    Write a passage that contains plausible but irrelevant context given the question.
    \end{minipage}
    }
    \caption{Prompt template design for generating plausible but irrelevant passages.}
    \label{fig:prompt_negative}
\end{figure}

\subsection{Prompt template of relabeling for synthetic passages $(\mathcal P_{\text{Relabel}})$.}

\begin{figure}[h!]
    \centering
    \fbox{
    \begin{minipage}{35em}
    You are a subject matter expert in your field with substantial accumulated knowledge in a specific subject or topic, validated by academic degrees, certifications, and/or years of professional experience in that field.\\\\
    Question: \{question\}\\
    % Thinking: \{cot\}\\
    Passage: \{passage\}\\
    Is the above passage relevant to the aforementioned question?\\
    Answer with yes or no.
    \end{minipage}
    }
    \caption{Prompt template design of relabeling for synthetic positive passages.}
    \label{fig:prompt_verification}
\end{figure}

\clearpage

\subsection{Prompts for pair-wise comparison of two passages $(\mathcal P_{\text{Compare}})$.}

\begin{figure}[h!]
    \centering
    \fbox{
    \begin{minipage}{35em}
    You are a subject matter expert in your field with substantial accumulated knowledge in a specific subject or topic, validated by academic degrees, certifications, and/or years of professional experience in that field.\\\\
    Passage \#1: \{passage1\}\\
    Passage \#2: \{passage2\}\\
    Question: \{question\}\\\\
    Based on your professional knowledge, choose which passage is more relevant to answer the given question.\\
    Only answer as Passage \#1 or Passage \#2
    \end{minipage}
    }
    \caption{Prompt template design for comparison of a passage pair.}
    \label{fig:prompt_rank}
\end{figure}

\section{Hyperparameters}\label{appendix:hyperparameters}
\begin{table}[h!]
    \centering
    \begin{adjustbox}{width=.6\textwidth,center}
    \begin{tabular}{|c|c|}
        \hline
        \textbf{Hyperparameter} & \textbf{Value}  \\
        \hline
        Model size & 125M \\
        Learning rate & 2e-5 \\
        Learning rate scheduler & Cosine \\
        Optimizer & Adam~\cite{kingma2014adam} \\
        Warmup ratio & 1000 steps \\
        Weight decay & 0.01 \\
        GPU Usage & Single A100 w/ 80GB VRAM \\
        Batch size & 60 (Stage 1), 100 (Stage 2) \\
        $\tau$ (temperature) & 0.05 \\
       \hline
    \end{tabular}
    \end{adjustbox}
    \caption{Detailed hyperparameters.} 
    \label{tab:hyper}
\end{table}

\section{Dataset Statistics}\label{appendix:dataset}
\begin{table}[h!]
\begin{adjustbox}{width=.7\textwidth,center}
\begin{tabular}{|l|l|l|l|l|}
\hline
\textbf{Dataset} & \textbf{Train} & \textbf{Validation} & \textbf{Test} & \textbf{Domain} \\ \hline
MSMARCO         &         532,752       &          9,261          &     7,438         &  Search Engine  \\
HotpotQA         &       170,001         &     10,895               &         14,811     &  Wikipedia  \\
FiQA             &        14,167        &     1,239                &        1,707      & Finance \\
SciFact          &         920       &        -             &       340      &    Science  \\
NFCorpus         &       110,576         &    11,386                 &       12,335    & Medical    \\
FEVER         &       140,086         &    8,080                 &       7,938    & Fact Verification    \\
NQ         &       132,803         &    -                 &       3,452    & Search Engine    \\ \hline
\end{tabular}
\end{adjustbox}
\caption{Dataset statistics.}
\label{tab:dataset_statistics}
\end{table}