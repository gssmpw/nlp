\onecolumn
\section{Additional results on scatter plots} 
\label{sec:additional-results-on-scatter-plots} 
Additional results to \cref{sec:score-emsembling} are presented in \cref{fig:additional-score-ensemble-viz}. 


\begin{figure*}[!ph]
    \centering
    \begin{subfigure}[b]{0.20\linewidth}
        \centering
        \includegraphics[width=\textwidth]{figs/scatterplots/scatterplot-progan-shuffle28.pdf}
        \subcaption{ProGAN}
    \end{subfigure}
    \begin{subfigure}[b]{0.20\linewidth}
        \centering
        \includegraphics[width=\textwidth]{figs/scatterplots/scatterplot-stylegan-shuffle28.pdf}
        \subcaption{StyleGAN}
    \end{subfigure}
    \begin{subfigure}[b]{0.20\linewidth}
        \centering
        \includegraphics[width=\textwidth]{figs/scatterplots/scatterplot-stylegan2-shuffle28.pdf}
        \subcaption{StyleGAN2}
    \end{subfigure}
    \begin{subfigure}[b]{0.20\linewidth}
        \centering
        \includegraphics[width=\textwidth]{figs/scatterplots/scatterplot-biggan-shuffle28.pdf}
        \subcaption{BigGAN}
    \end{subfigure}
    \begin{subfigure}[b]{0.20\linewidth}
        \centering
        \includegraphics[width=\textwidth]{figs/scatterplots/scatterplot-cyclegan-shuffle28.pdf}
        \subcaption{CycleGAN}
    \end{subfigure}
    \begin{subfigure}[b]{0.20\linewidth}
        \centering
        \includegraphics[width=\textwidth]{figs/scatterplots/scatterplot-stargan-shuffle28.pdf}
        \subcaption{StarGAN}
    \end{subfigure}
    \begin{subfigure}[b]{0.20\linewidth}
        \centering
        \includegraphics[width=\textwidth]{figs/scatterplots/scatterplot-gaugan-shuffle28.pdf}
        \subcaption{GauGAN}
    \end{subfigure}
    \begin{subfigure}[b]{0.20\linewidth}
        \centering
        \includegraphics[width=\textwidth]{figs/scatterplots/scatterplot-deepfake-shuffle28.pdf}
        \subcaption{DeepFake}
    \end{subfigure}
    \begin{subfigure}[b]{0.20\linewidth}
        \centering
        \includegraphics[width=\textwidth]{figs/scatterplots/scatterplot-dalle-shuffle28.pdf}
        \subcaption{DALL-E}
    \end{subfigure}
    \begin{subfigure}[b]{0.20\linewidth}
        \centering
        \includegraphics[width=\textwidth]{figs/scatterplots/scatterplot-glide_100_10-shuffle28.pdf}
        \subcaption{GLIDE\_100\_10}
    \end{subfigure}
    \begin{subfigure}[b]{0.20\linewidth}
        \centering
        \includegraphics[width=\textwidth]{figs/scatterplots/scatterplot-glide_100_27-shuffle28.pdf}
        \subcaption{GLIDE\_100\_27}
    \end{subfigure}
    \begin{subfigure}[b]{0.20\linewidth}
        \centering
        \includegraphics[width=\textwidth]{figs/scatterplots/scatterplot-glide_50_27-shuffle28.pdf}
        \subcaption{GLIDE\_50\_27}
    \end{subfigure}
    \begin{subfigure}[b]{0.20\linewidth}
        \centering
        \includegraphics[width=\textwidth]{figs/scatterplots/scatterplot-guided-shuffle28.pdf}
        \subcaption{ADM}
    \end{subfigure}
    \begin{subfigure}[b]{0.20\linewidth}
        \centering
        \includegraphics[width=\textwidth]{figs/scatterplots/scatterplot-ldm_100-shuffle28.pdf}
        \subcaption{LDM\_100}
    \end{subfigure}
    \begin{subfigure}[b]{0.20\linewidth}
        \centering
        \includegraphics[width=\textwidth]{figs/scatterplots/scatterplot-ldm_200-shuffle28.pdf}
        \subcaption{LDM\_200}
    \end{subfigure}
    \begin{subfigure}[b]{0.20\linewidth}
        \centering
        \includegraphics[width=\textwidth]{figs/scatterplots/scatterplot-ldm_200_cfg-shuffle28.pdf}
        \subcaption{LDM\_200\_cfg}
    \end{subfigure}
    \begin{subfigure}[b]{0.20\linewidth}
        \centering
        \includegraphics[width=\textwidth]{figs/scatterplots/scatterplot-sitd-shuffle28.pdf}
        \subcaption{SITD}
    \end{subfigure}
    \begin{subfigure}[b]{0.20\linewidth}
        \centering
        \includegraphics[width=\textwidth]{figs/scatterplots/scatterplot-san-shuffle28.pdf}
        \subcaption{SAN}
    \end{subfigure}
    \begin{subfigure}[b]{0.20\linewidth}
        \centering
        \includegraphics[width=\textwidth]{figs/scatterplots/scatterplot-crn-shuffle28.pdf}
        \subcaption{CRN}
    \end{subfigure}
    \begin{subfigure}[b]{0.20\linewidth}
        \centering
        \includegraphics[width=\textwidth]{figs/scatterplots/scatterplot-imle-shuffle28.pdf}
        \subcaption{IMLE}
    \end{subfigure}
     \caption{Scatter plots of per-sample scores. X-axis is UnivFD logits, and Y-axis is the logit from PatchShuffle with patch size 28. The decision boundary of UnivFD (\textcolor{red}{red}) and SFLD (\textcolor{green}{green}) are shown.}
     \vspace{-1em}
     \label{fig:additional-score-ensemble-viz}
     \vspace{-1em}
\end{figure*}
\twocolumn

\section{Datasets}
\label{supp:datasets}

\subsection{Train dataset}
To establish a baseline for comparison, we adopt the most common setting for training the detection model, namely the train set from ForenSynths\cite{wang2020cnn}. 
The train set consists of real images and ProGAN\cite{karras2018progressive}-generated images. It involves 20 different object class categories, each containing 18K real images from the different LSUN\cite{yu2015lsun} datasets and 18K synthetic images generated by ProGAN.

\subsection{Test dataset}
We evaluate the performance of SFLD on (1) conventional benchmarks, (2) TwinSynths which we proposed, (3) low-level vision and perceptual loss benchmarks. In this section, we provide a detailed description of the configurations for the conventional benchmarks and low-level vision and perceptual loss benchmarks.

\textbf{Conventional benchmark} 
This is from ForenSynths\cite{wang2020cnn} and Ojha \etal\cite{ojha2023towards}, including 16 different subsets of generated images, synthesized by seven GAN-based generative models, eight diffusion-based generative models and one deepfake model. 
The subset of GAN-based fake images are from ForenSynths\cite{wang2020cnn}, including ProGAN\cite{karras2018progressive}, StyleGAN\cite{karras2019style}, StyleGAN2\cite{Karras2019stylegan2}, BigGAN\cite{brock2018biggan}, CycleGAN\cite{CycleGAN}, StarGAN\cite{choi2018stargan}, and GauGAN\cite{GauGAN}.
The subset of diffusion-based fake images are from Ojha \etal\cite{ojha2023towards}, including DALL-E\cite{dayma2021dall}, three different variants of Glide\cite{nichol2021glide}, ADM(guided-diffusion)\cite{dhariwal2021diffusion}, and three different variants of LDM\cite{rombach2022high}.
Deepfake set is from FaceForensices++\cite{Deepfake} which is included in ForenSynths\cite{wang2020cnn}.
The real images corresponding to the fake images described above were directly taken from the same datasets. Those are sampled from LSUN\cite{yu2015lsun}, ImageNet\cite{russakovsky2015imagenet}, CycleGAN\cite{CycleGAN}, CelebA\cite{CelebA}, COCO\cite{coco}, and FaceForensics++\cite{Deepfake}.


\textbf{Low-level vision and perceptual loss benchmarks} 
Low-level vision benchmark consists of SITD\cite{chen2018SITD} and SAN\cite{dai2019SAN}.
These are image processing models that approximate long exposures in low light conditions from short exposures in raw camera input or process super-resolution on low-resolution images.
Perceptual benchmark consists of CRN\cite{chen2017CRN} and IMLE\cite{li2019IMLE}.
These models color the semantic segmentation map into a realistic image while directly optimizing a perceptual loss.
These benchmarks are from ForenSynths\cite{wang2020cnn}.



\section{Qualitative analysis on TwinSynths dataset}
We show the GradCAM visualization of UnivFD\cite{ojha2023towards} and Patch-shuffle 28×28 using the TwinSynths dataset in \cref{fig:cam-viz-twinsynth}.
Similar to \cref{sec:qualitative-analysis}, UnivFD is shown to focus on the class-dependent salient region, whereas our method focuses on the entire image region. 
Moreover, we observed that for TwinSynths dataset, UnivFD does respond identically to real/fake images which indicate its inability to capture subtle fake image fingerprints, whereas our method shows the response to such a difference. 

\begin{figure}[h]
    \centering
    \begin{subfigure}[t]{\linewidth}
        \centering
        \includegraphics[width=\textwidth]{figs/benchmark_vis/cam-twinsynth-univfd.pdf}
        \caption{UnivFD\cite{ojha2023towards} examples}
    \end{subfigure}
    \begin{subfigure}[t]{\linewidth}
        \centering
        \includegraphics[width=\textwidth]{figs/benchmark_vis/cam-twinsynth-shuffle28.pdf}
        \caption{PatchShuffle(patch size 28) examples}
    \end{subfigure}
    \caption{Class activation maps (CAM) for UnivFD\cite{ojha2023towards} and the patch-shuffled detector (ours) in TwinSynths dataset. 
    Each row shows examples from TwinSynths-real, TwinSynths-GAN, TwinSynths-DM sets. 
    GradCAM\cite{selvaraju2020grad,jacobgilpytorchcam} was used to obtain the heatmaps. }
    \label{fig:cam-viz-twinsynth}
\end{figure}

\begin{figure}[t]
    \centering
    \includegraphics[width=0.9\linewidth]{figs/suppl_C.pdf}
    \caption{Illustration of the test input processing strategy. In typical methods, a test image is center-cropped before being passed to the detector. Our patch shuffling strategy allows us to select patches from the entire image region, effectively increasing its receptive field.}
    \label{fig:selecting_patches}
    \vspace{-10pt}
\end{figure}

\begin{table}[t]
    \centering
    \resizebox{\linewidth}{!}{
        \begin{tabular}{lcccccc}
\toprule
Benchmark & \multicolumn{2}{c}{SFLD (224+24)} & \multicolumn{2}{c}{SFLD (224+56)} & \multicolumn{2}{c}{SFLD} \\
\cmidrule(lr){2-3} \cmidrule(lr){4-5} \cmidrule(lr){6-7}
& center & full image & center & full image & center & full image \\
\midrule
main benchmark & 98.04 & 98.03 & 98.37 & 98.39 & 98.40 & 98.43 \\
CRN & 94.41 & 96.62 & 94.17 & 97.24 & 91.97 & 95.79 \\
IMLE & 97.55 & 98.65 & 98.12 & 99.23 & 96.92 & 98.64 \\
SITD & 59.36 & 64.82 & 67.71 & 76.66 & 60.38 & 71.90 \\
\bottomrule
\end{tabular}


    }
    \caption{mAP results of the various sizes of test images, comparing two different patch selecting methods. \emph{Center} denotes that the images have been center-cropped to 224×224, while \emph{full image} means that random patches from the full image have been combined to reconstruct a 224×224 image.}
    \label{tab:suppl-fullimage}
\end{table}

\section{Effect of selecting patches from the whole image}
\label{sec:Appendix-fullimagesampling}
\cref{fig:selecting_patches} illustrates the concept of patch extraction of SFLD mentioned in \cref{subsec:SFLD}.
Unlike many alternative detection methodologies, SFLD extracts patches from any position within the input image at the test time.
This approach enhances the detector's receptive field and improves performance for images that have higher resolution than 224×224.
In \cref{tab:suppl-fullimage}, we compare results on benchmarks that have high-resolution images.
We consider different SFLD ensemble options and the location of the selected patch.
The main benchmark consists mostly of 256×256 images, which have little margin with a 224×224 center crop.
Meanwhile, the CRN and IMLE benchmarks have 512×256 images, and the SITD benchmark includes images much larger up to 2,848×4,256 or 4,032×6,030.

We observed that the discrepancy between the two methodologies was minimal when the test image was small.
However, as the image size increased, the performance of the method that solely focused on the center of an image became increasingly constrained.



\section{Image degradation examples}

\begin{figure}[t]
    \centering
    \includegraphics[width=\linewidth]{figs/rebut/robustness_degradation.png}
    \caption{Examples of two image degradation}
    \label{fig:robustness_examples}
\end{figure}

\cref{fig:robustness_examples} shows examples of image gradations.
According to our definition of high- and low-level features, we can consider that the gaussian blur attacks both high- and low-level features in the image, and the JPEG compression attacks on low-level features in the image.


\section{Robustness against image degradation}
\label{sec:supple_robustness}

Since image degradation was not considered during training, it may be useful to examine the changes in output distribution (as shown in \cref{fig:robustness_distribution_jpeg} in supplementary material) to analyze the model's operational tendencies in detail.
\cref{fig:robustness_distribution_jpeg} reveals distinctions between the high-level feature model (UnivFD \cref{fig:robustness_univfd}), low-level feature model (NPR \cref{fig:robustness_npr}), and integrated model.
The distributions of SFLD and UnivFD remain distinguishable, despite a slight decline in discrimination performance.
However, NPR aligns real and generated images into the same distribution.
This behavior arises from the operational mechanism of each model.
NPR primarily focuses on low-level features, resulting in a catastrophic failure to maintain robustness against JPEG compression.
UnivFD demonstrates relative robustness due to its emphasis on high-level features through CLIP visual encoders; however, there is a slight performance penalty because the visual encoder does not completely disregard low-level features.
In contrast, SFLD exhibits robustness against JPEG compression by integrating both high- and low-level features through ensemble/fusion, allowing each to compensate for the information lost in the other.


\section{Effect of patch sizes}

\label{sec:supple_patchsize}
\begin{figure}[t]
    \centering
    \begin{subfigure}[t]{0.48\linewidth}
        \centering
        \includegraphics[width=\linewidth]{figs/rebuttal-patchsize/patch_size_gan.pdf}
        \subcaption{GAN-based generators.}
        \label{fig:rebuttal-patchsize-gan}
    \end{subfigure}
    \begin{subfigure}[t]{0.48\linewidth}
        \centering
        \includegraphics[width=\linewidth]{figs/rebuttal-patchsize/patch_size_diff.pdf}
        \subcaption{Diffusion-based generators.}
        \label{fig:rebuttal-patchsize-diff}
    \end{subfigure}
    \caption{Results of the ensemble models of UnivFD and the patch-shuffled model with each patch size. For 224, it is the same as UnivFD.}
    \label{fig:rebuttal-patchsize}
\end{figure}
To supplement \cref{fig:patch_size} in the main text, we checked the AP for each generator, rather than the average AP on the conventional benchmark.
\cref{fig:rebuttal-patchsize} illustrates that SFLD consistently maintains high performance as long as the patch size is not smaller than the patch size of the image encoder backbone. 
This is because when the shuffling patch size $s_N$ is smaller than the ViT's patch size, the input tokens are affected by patch-shuffling to get an unnatural image patch, resulting in the encoder not properly embedding the visual feature.


\section{Ablation on the pre-trained image encoder}
\label{sec:suppl_pretrained}

\begin{figure}[t]
    \centering
    \begin{subfigure}[t]{0.32\linewidth}
        \centering
        \includegraphics[width=\linewidth]{figs/rebuttal-encoders/clip.pdf}
        \subcaption{CLIP-ViT}
        \label{fig:rebuttal-encoder-clip}
    \end{subfigure}
    \begin{subfigure}[t]{0.32\linewidth}
        \centering
        \includegraphics[width=\linewidth]{figs/rebuttal-encoders/dino.pdf}
        \subcaption{DINOv2-ViT}
        \label{fig:rebuttal-encoder-dino}
    \end{subfigure}
    \begin{subfigure}[t]{0.32\linewidth}
        \centering
        \includegraphics[width=\linewidth]{figs/rebuttal-encoders/openclip.pdf}
        \subcaption{OpenCLIP-ViT}
        \label{fig:rebuttal-encoder-openclip}
    \end{subfigure}
    \caption{Class-wise detection results for StyleGAN-\{\emph{bedroom, car, cat}\} class categories reported in AP. \emph{bedroom} class is a novel class that is not in the training set.}
    \label{fig:rebuttal-encoders}
\end{figure}


The pre-trained image encoder is employed to learn the features of the ``real'' class.
According to \cite{ojha2023towards}, directly fine-tuning the encoder makes the detector overfit to a specific generator used in training. This results in low generalization to unseen generators.
Therefore, we utilized the frozen CLIP:ViT-L/14 model following UnivFD.

\cref{tab:rebuttal-encoder} show that our patch shuffling and ensembling strategy improves the performance regardless of the pre-trained backbone.
All models are trained only with real and generated images from ProGAN and tested on the various unseen generated images in conventional benchmark. For ImageNet-ViT, we used ViT-B/16 model, following UnivFD paper \cite{ojha2023towards}. Since its encoders have patch size of 16, we utilized 16 and 32 for patch sizes instead of 28 and 56.
Moreover, note that simply employing different pre-training datasets or strategies -- ImageNet, DINOv2, OpenCLIP -- does not address the content bias problem. (see \cref{fig:rebuttal-encoders}) 



\begin{figure*}[!h]
    \centering
    \begin{subfigure}{0.32\textwidth}
        \centering
        \includegraphics[width=\linewidth]{figs/rebut/robustness_SFLD_all_final.pdf}
        \subcaption{The changes of SFLD output distribution}
        \label{fig:robustness_sfld}
    \end{subfigure}
    \begin{subfigure}{0.32\textwidth}
        \centering
        \includegraphics[width=\linewidth]{figs/rebut/robustness_univfd_all_final.pdf}
        \subcaption{The changes of UnivFD output distribution}
        \label{fig:robustness_univfd}
    \end{subfigure}
    \begin{subfigure}{0.32\textwidth}
        \centering
        \includegraphics[width=\linewidth]{figs/rebut/robustness_NPR_all_final.pdf}
        \subcaption{The changes of NPR output distribution}
        \label{fig:robustness_npr}
    \end{subfigure}
    \caption{The changes of model output distribution against JPEG compression}
    \label{fig:robustness_distribution_jpeg}
\end{figure*}

\begin{table*}[h]
    \centering
    \resizebox{\textwidth}{!}{
        \definecolor{Gray}{gray}{0.9}
\begin{tabular}{lcc||lcccccc}
    \toprule
    \multirow{2}{*}{\diagbox{Patch sizes}{Pre-training}}  & \multicolumn{2}{c||}{ImageNet-ViT} & \multirow{2}{*}{\diagbox{Patch sizes}{Pre-training}} & \multicolumn{2}{c}{DINOv2-ViT\cite{oquab2023dinov2}} & \multicolumn{2}{c}{OpenCLIP-ViT\cite{ilharco_gabriel_2021_5143773}} & \multicolumn{2}{c}{CLIP-ViT} \\
    \cmidrule(lr){2-3} \cmidrule(lr){5-6} \cmidrule(lr){7-8} \cmidrule(lr){9-10}
    & Acc. & AP & & Acc. & AP & Acc. & AP & Acc. & AP \\
    \midrule
    224 (UnivFD\cite{ojha2023towards})& 62.45 & 69.30 & 224 (UnivFD\cite{ojha2023towards}) & 81.89 & 91.75 & 86.49 & 96.90 & 85.89 & 96.29 \\
    224+16 & 63.88 & 72.23 & 224+28 & 82.88 & 93.42 & 86.50 & 97.59 & 91.94 & 98.03 \\
    224+32 & 63.34 & 71.36 & 224+56 & 82.44 & 93.04 & 86.87 & 97.70 & 92.05 & 98.39 \\
    \rowcolor{Gray} 224+32+16 (ours) & 63.70 & 72.18 & 224+56+28 (ours) & 82.26 & 93.26 & 86.19 & 97.49 & 93.30 & 98.43 \\
    \bottomrule
\end{tabular}
    }
    \caption{Detection accuracy and AP on a conventional benchmark of the proposed patch shuffling and ensembling (SFLD) strategy across various pre-trained encoders. For the ImageNet encoder, ViT-B/16 is used. For the other encoders, ViT-L/14 is used.}
    \label{tab:rebuttal-encoder}

\end{table*}


\section{In-the-wild applications of SFLD}
\label{sec:suppl_inthewild}

We applied our SFLD to in-the-wild AI-generated image detection, especially to a deepfake detection benchmark.
We have already demonstrated performance on a FaceForensics++\cite{roessler2019faceforensicspp} subset, which is a deepfake detection benchmark created using face manipulation software \cite{Deepfakes}. Here, we have added \cref{tab:rebuttal-deepfake} with experiments using Generated Faces in the Wild\cite{borjiGFW} datasets. SFLD shows state-of-the-art performance in detecting real-world deepfakes. 


\section{Pseudocode of SFLD}
\label{sec:pseudocode}
See \cref{alg:pseudocode}. 

% Let $\phi(\cdot)$ bet the visual encoder (in our case, CLIP ViT-B/14)

\begin{algorithm}[h]
    \begin{lstlisting}[language=python]
    """
    Args: 
        image: A test image instance
        n_views: Number of views for random patch shuffle averaging. Defaults to 10. 
        visual_encoder: A CLIP-pretrained ViT-L/14 visual encoder.
    Returns: 
        output: a real/fake score normalized to [0,1] range.
    """

    # prediction from 224x224 unshuffled view
    feature = visual_encoder(image)
    output_224 = classifier_univfd(feature)

    # prediction from 56x56 random shuffled views
    output_56 = []
    for _ in range(n_views): 
        image_shuffled = patch_shuffle(image, size=56)
        feature = visual_encoder(image_shuffled)
        output = classifier_56(feature)
        output_56.append(output)
    output_56 = mean(output_56)

    # prediction from 28x28 random shuffled views
    output_28 = []
    for _ in range(n_views): 
        image_shuffled = patch_shuffle(image, size=28)
        feature = visual_encoder(image_shuffled)
        output = classifier_28(feature)
        output_28.append(output)
    output_28 = mean(output_28)

    # ensemble the logit scores
    output = mean([output_224, output_56, output_28])
    output = output.sigmoid()
    \end{lstlisting}
    \caption{PyTorch-style pseudocode of SFLD}
    \label{alg:pseudocode}
\end{algorithm}

\begin{table}[t]
    \centering
    \resizebox{0.55\linewidth}{!}{
        \definecolor{Gray}{gray}{0.9}

\begin{tabular}{lcc}
\toprule
\multirow[c]{2}{*}{Method} & \multicolumn{2}{c}{GFW\cite{borjiGFW}} \\
\cmidrule(lr){2-3}
 & Acc. & mAP \\
\midrule
NPR\cite{tan2024rethinking} & 53.30 & 47.63 \\
UnivFD\cite{ojha2023towards} & 70.07 & 85.55 \\
SFLD(224+56) & 77.80 & 86.70 \\
\rowcolor{Gray} SFLD & 77.28 & 86.70 \\
\bottomrule
\end{tabular}

    }
    \caption{Performance on the in-the-wild deepfake detection benchmark.}
    \label{tab:rebuttal-deepfake}
\end{table}
%------------ related works

\section{Related Work}
\label{sec:related work}
% In this section, we review the existing literature on point cloud denoising and unsupervised image denoising.
%-------------------------------------------------------------------------
\subsection{Point cloud denoising}

    \subsubsection{Traditional methods}
Traditional approaches to point cloud denoising include statistical methods \cite{computingpointset2003,definingpointset2004,wlop2009HH}, filtering techniques\cite{pointsetsurfaces2001,Robustmoving2005, zaman2017density}, and optimization-based methods \cite{l1sparse2010,clop2014PR,digne2017bilateral,multi-projection2018duan,hu2020featuregraph} . These techniques often rely on handcrafted features and heuristics to distinguish signal from noise. For example, statistical methods may use distribution models to identify and remove outliers. Filtering methods, such as mean or median filters, operate under the assumption that noise is statistically different from the signal. Optimization-based methods formulate denoising as an energy minimization problem, where regularization terms constrain the solution to ensure certain smoothness cirterion or adherence to prior knowledge.

%-------------------------------------------------------------------------
    \subsubsection{Supervised learning based methods}
In recent years, several deep learning-based methods \cite{rakotosaona2020PCN,hermosilla2019TotalDenoising,luo2020DMR,luo_score-based_2021} have been proposed for point cloud denoising. NPD \cite{NPD2019} is the first learning-based point cloud denoising method that directly processes noisy data without requiring noise characteristics or neighboring point definitions. This approach combines local and global information by projecting noisy points onto estimated reference planes, effectively removing noise and enhancing robustness against variations in noise intensity and curvature. PointCleanNet\cite{rakotosaona2020PCN} first removes outlier points and then combines them with residual connectivity to predict the inverse displacement \cite{Guerrero2017PCPNetLL}, and iteratively shifts noisy points to remove noise. Pistilli \etal proposed GPDNet \cite{gpdnet2020}, which is a graph convolutional network to improve denoising robustness at high noise levels. Luo \etal also proposed  DMRDenoise \cite{luo2020DMR}, which filter
points by first downsampling the noisy inputs and reconstructing the local subsurface to perform point upsampling. However, this resampling method is difficult to maintain a good local shape. ScoreDenoise \cite{luo_score-based_2021} is proposed to tackle the aforementioned issues by iteratively updating the point position in implicit gradient fields learned by neural networks. For inference, they follows an iterative procedure with a decaying step size, which stabilizes point movement and prevents over-correction, allowing points to converge gradually toward the underlying geometry. The authors of \cite{de_Silva_Edirimuni_2023_CVPR} proposed IterativePFN - an iterative method that use a novel loss function that utilizes an adaptive ground truth target at each iteration to capture the relationship between intermediate filtering results during training. Zheng \etal proposed a end-to-end network for joint normal filtering-based point cloud denoising \cite{10173632}. They introduce an auxiliary normal filtering task to enhance the network's ability to remove noise while preserving geometric features more accurately.

Supervised methods can achieve impressive results, but are limited by the availability and quality of the training data, as they typically require paired noisy and clean point clouds to train the neural network. The absence of clean data in real-world scenario pose a significant challenge on applicability of these algorithms.

%-------------------------------------------------------------------------
    \subsubsection{Unsupervised learning methods}
Unsupervised learning-based methods for point cloud denoising do not require ground-truth clean data. Instead, these methods leverage the inherent structure or distribution of the point cloud to guide the denoising process. Unsupervised methods show promise in scenarios where clean data is absent or hard to obtain.

Hermosilla \etal first introduced Total Denoising (TotalDn) \cite{hermosilla2019TotalDenoising} as an unsupervised learning approach for point cloud denoising, relying solely on noisy data without requiring clean ground truth. TotalDn approximates the underlying surfaces by regressing points from the distribution of unstructured total noise, utilizing a spatial prior term to refine the estimation of geometry. 

In DMRDenoise \cite{luo2020DMR}, an unsupervised version is proposed which utilizes a loss function that identify local neighborhoods using a probabilistic Gaussian mask on the k-nearest neighbors, which selectively retains points likely to represent the underlying surface. By leveraging an Earth Mover's Distance (EMD) assignment, it achieves a one-to-one correspondence between input and predicted points, aligning them to reduce noise within local neighborhoods.
This approach enhances robustness to noise and adapts well to varied surface geometries. However, the probabilistic masking and EMD calculation add computational complexity, which can slow down inference in dense or noisy point clouds. 

ScoreDenoise \cite{luo_score-based_2021} also introduced an unsupervised version that employs ensemble score function and an adaptive neighborhood-covering loss for model training.  
Score-u is probably the most relevant work to our method. However, the defined score in \cite{luo_score-based_2021} is only an displacement-alike version of the score function and there is no explicit formula relating the estimated score to the final denoising result. Also, the iterative process is computationally expensive, and can suffer from fluctuation, leading to perturbed and unstable solution.

Most recently, Noise4Denoise \cite{noise4Wang2024} method is proposed that use an additional doubly-noisy point cloud to learn noise displacement by comparing the two noise levels. This approach effectively leverages synthetic noise for training, allowing the model to estimate residuals without relying on clean data.
However, in practical applications, noise parameters are often unknown and variable, making it challenging to replicate the exact conditions assumed during training. This reliance on predefined noise characteristics can limit the model's applicability to real-world scenarios where noise distributions may differ significantly from synthetic settings. 
%-------------------------------------------------------------------------
\subsection{Unsupervised image denoising}
Recently unsupervised image denoising has made significant progress. Non-Bayesian methods include PURE \cite{luisier2010image}, SURE \cite{SURE2018} \textit{etc.}, which are based on various unbiased risk estimator under certain noise distribution. Other methods explore self-similarity in natural images \cite{xu2015patch, doi:10.1137/23M1614456} or exploits the statistical properties of noise to achieve denoising effect \cite{gravel2004method}.  

Noise2Noise \cite{2018Noise2NoiseLI} is a pioneering method that does not require clean data, but it requires multiple noisy versions of the same image for training. To address this limitation, methods such as Noise2Void \cite{2018Noise2VoidL}, Noise2Self \cite{2019Noise2SelfBD}, \textit{etc.}, have been developed, which use only a single noisy image. These methods are particularly important for practical applications where obtaining clean images or multiple noisy realizations of the same image is difficult or impossible. Neighbor2Neighbor \cite{huang2021neighbor2neighbor} proposed a two-step method with a a random neighbor sub-sampler that generates training  pairs and a denosing network. Kim \etal proposed Noise2Score\cite{kim_noise2score_2021}, a novel Bayesian framework for self-supervised image denoising without clean data. The core of Noise2Score is the usage of Tweedie's formula, which provides an explicit representation of the denoised image through a score function. Combined with score function estimation, Noise2Score can be applied to image denoising with any exponential family noise. Kim \etal also proposed the Noise Distribution Adaptive Self-Supervised Image Denoising method \cite{kim_noise_2022}, which further extends the application of Noise2Score by combining the Tweedie distribution with score matching. This enables adaptive handling of various noise distributions and dynamically adjusts the denoising process by estimating noise parameters. On the other hand, Xie \etal \cite{scoreXie2024} broadened the denoising scope of Noise2Score by allowing it to handle complex noise models, such as multiplicative and structurally correlated noise, demonstrating broad applicability to diverse noise models.

These development of unsupervised image denoising method motivate us to explore similar ideas in 3D point cloud denoising.





