% \vspace{-5pt}
\section{Introduction}
\label{sec:intro}

\section{Overview}

\revision{In this section, we first explain the foundational concept of Hausdorff distance-based penetration depth algorithms, which are essential for understanding our method (Sec.~\ref{sec:preliminary}).
We then provide a brief overview of our proposed RT-based penetration depth algorithm (Sec.~\ref{subsec:algo_overview}).}



\section{Preliminaries }
\label{sec:Preliminaries}

% Before we introduce our method, we first overview the important basics of 3D dynamic human modeling with Gaussian splatting. Then, we discuss the diffusion-based 3d generation techniques, and how they can be applied to human modeling.
% \ZY{I stopp here. TBC.}
% \subsection{Dynamic human modeling with Gaussian splatting}
\subsection{3D Gaussian Splatting}
3D Gaussian splatting~\cite{kerbl3Dgaussians} is an explicit scene representation that allows high-quality real-time rendering. The given scene is represented by a set of static 3D Gaussians, which are parameterized as follows: Gaussian center $x\in {\mathbb{R}^3}$, color $c\in {\mathbb{R}^3}$, opacity $\alpha\in {\mathbb{R}}$, spatial rotation in the form of quaternion $q\in {\mathbb{R}^4}$, and scaling factor $s\in {\mathbb{R}^3}$. Given these properties, the rendering process is represented as:
\begin{equation}
  I = Splatting(x, c, s, \alpha, q, r),
  \label{eq:splattingGA}
\end{equation}
where $I$ is the rendered image, $r$ is a set of query rays crossing the scene, and $Splatting(\cdot)$ is a differentiable rendering process. We refer readers to Kerbl et al.'s paper~\cite{kerbl3Dgaussians} for the details of Gaussian splatting. 



% \ZY{I would suggest move this part to the method part.}
% GaissianAvatar is a dynamic human generation model based on Gaussian splitting. Given a sequence of RGB images, this method utilizes fitted SMPLs and sampled points on its surface to obtain a pose-dependent feature map by a pose encoder. The pose-dependent features and a geometry feature are fed in a Gaussian decoder, which is employed to establish a functional mapping from the underlying geometry of the human form to diverse attributes of 3D Gaussians on the canonical surfaces. The parameter prediction process is articulated as follows:
% \begin{equation}
%   (\Delta x,c,s)=G_{\theta}(S+P),
%   \label{eq:gaussiandecoder}
% \end{equation}
%  where $G_{\theta}$ represents the Gaussian decoder, and $(S+P)$ is the multiplication of geometry feature S and pose feature P. Instead of optimizing all attributes of Gaussian, this decoder predicts 3D positional offset $\Delta{x} \in {\mathbb{R}^3}$, color $c\in\mathbb{R}^3$, and 3D scaling factor $ s\in\mathbb{R}^3$. To enhance geometry reconstruction accuracy, the opacity $\alpha$ and 3D rotation $q$ are set to fixed values of $1$ and $(1,0,0,0)$ respectively.
 
%  To render the canonical avatar in observation space, we seamlessly combine the Linear Blend Skinning function with the Gaussian Splatting~\cite{kerbl3Dgaussians} rendering process: 
% \begin{equation}
%   I_{\theta}=Splatting(x_o,Q,d),
%   \label{eq:splatting}
% \end{equation}
% \begin{equation}
%   x_o = T_{lbs}(x_c,p,w),
%   \label{eq:LBS}
% \end{equation}
% where $I_{\theta}$ represents the final rendered image, and the canonical Gaussian position $x_c$ is the sum of the initial position $x$ and the predicted offset $\Delta x$. The LBS function $T_{lbs}$ applies the SMPL skeleton pose $p$ and blending weights $w$ to deform $x_c$ into observation space as $x_o$. $Q$ denotes the remaining attributes of the Gaussians. With the rendering process, they can now reposition these canonical 3D Gaussians into the observation space.



\subsection{Score Distillation Sampling}
Score Distillation Sampling (SDS)~\cite{poole2022dreamfusion} builds a bridge between diffusion models and 3D representations. In SDS, the noised input is denoised in one time-step, and the difference between added noise and predicted noise is considered SDS loss, expressed as:

% \begin{equation}
%   \mathcal{L}_{SDS}(I_{\Phi}) \triangleq E_{t,\epsilon}[w(t)(\epsilon_{\phi}(z_t,y,t)-\epsilon)\frac{\partial I_{\Phi}}{\partial\Phi}],
%   \label{eq:SDSObserv}
% \end{equation}
\begin{equation}
    \mathcal{L}_{\text{SDS}}(I_{\Phi}) \triangleq \mathbb{E}_{t,\epsilon} \left[ w(t) \left( \epsilon_{\phi}(z_t, y, t) - \epsilon \right) \frac{\partial I_{\Phi}}{\partial \Phi} \right],
  \label{eq:SDSObservGA}
\end{equation}
where the input $I_{\Phi}$ represents a rendered image from a 3D representation, such as 3D Gaussians, with optimizable parameters $\Phi$. $\epsilon_{\phi}$ corresponds to the predicted noise of diffusion networks, which is produced by incorporating the noise image $z_t$ as input and conditioning it with a text or image $y$ at timestep $t$. The noise image $z_t$ is derived by introducing noise $\epsilon$ into $I_{\Phi}$ at timestep $t$. The loss is weighted by the diffusion scheduler $w(t)$. 
% \vspace{-3mm}

\subsection{Overview of the RTPD Algorithm}\label{subsec:algo_overview}
Fig.~\ref{fig:Overview} presents an overview of our RTPD algorithm.
It is grounded in the Hausdorff distance-based penetration depth calculation method (Sec.~\ref{sec:preliminary}).
%, similar to that of Tang et al.~\shortcite{SIG09HIST}.
The process consists of two primary phases: penetration surface extraction and Hausdorff distance calculation.
We leverage the RTX platform's capabilities to accelerate both of these steps.

\begin{figure*}[t]
    \centering
    \includegraphics[width=0.8\textwidth]{Image/overview.pdf}
    \caption{The overview of RT-based penetration depth calculation algorithm overview}
    \label{fig:Overview}
\end{figure*}

The penetration surface extraction phase focuses on identifying the overlapped region between two objects.
\revision{The penetration surface is defined as a set of polygons from one object, where at least one of its vertices lies within the other object. 
Note that in our work, we focus on triangles rather than general polygons, as they are processed most efficiently on the RTX platform.}
To facilitate this extraction, we introduce a ray-tracing-based \revision{Point-in-Polyhedron} test (RT-PIP), significantly accelerated through the use of RT cores (Sec.~\ref{sec:RT-PIP}).
This test capitalizes on the ray-surface intersection capabilities of the RTX platform.
%
Initially, a Geometry Acceleration Structure (GAS) is generated for each object, as required by the RTX platform.
The RT-PIP module takes the GAS of one object (e.g., $GAS_{A}$) and the point set of the other object (e.g., $P_{B}$).
It outputs a set of points (e.g., $P_{\partial B}$) representing the penetration region, indicating their location inside the opposing object.
Subsequently, a penetration surface (e.g., $\partial B$) is constructed using this point set (e.g., $P_{\partial B}$) (Sec.~\ref{subsec:surfaceGen}).
%
The generated penetration surfaces (e.g., $\partial A$ and $\partial B$) are then forwarded to the next step. 

The Hausdorff distance calculation phase utilizes the ray-surface intersection test of the RTX platform (Sec.~\ref{sec:RT-Hausdorff}) to compute the Hausdorff distance between two objects.
We introduce a novel Ray-Tracing-based Hausdorff DISTance algorithm, RT-HDIST.
It begins by generating GAS for the two penetration surfaces, $P_{\partial A}$ and $P_{\partial B}$, derived from the preceding step.
RT-HDIST processes the GAS of a penetration surface (e.g., $GAS_{\partial A}$) alongside the point set of the other penetration surface (e.g., $P_{\partial B}$) to compute the penetration depth between them.
The algorithm operates bidirectionally, considering both directions ($\partial A \to \partial B$ and $\partial B \to \partial A$).
The final penetration depth between the two objects, A and B, is determined by selecting the larger value from these two directional computations.

%In the Hausdorff distance calculation step, we compute the Hausdorff distance between given two objects using a ray-surface-intersection test. (Sec.~\ref{sec:RT-Hausdorff}) Initially, we construct the GAS for both $\partial A$ and $\partial B$ to utilize the RT-core effectively. The RT-based Hausdorff distance algorithms then determine the Hausdorff distance by processing the GAS of one object (e.g. $GAS_{\partial A}$) and set of the vertices of the other (e.g. $P_{\partial B}$). Following the Hausdorff distance definition (Eq.~\ref{equation:hausdorff_definition}), we compute the Hausdorff distance to both directions ($\partial A \to \partial B$) and ($\partial B \to \partial A$). As a result, the bigger one is the final Hausdorff distance, and also it is the penetration depth between input object $A$ and $B$.


%the proposed RT-based penetration depth calculation pipeline.
%Our proposed methods adopt Tang's Hausdorff-based penetration depth methods~\cite{SIG09HIST}. The pipeline is divided into the penetration surface extraction step and the Hausdorff distance calculation between the penetration surface steps. However, since Tang's approach is not suitable for the RT platform in detail, we modified and applied it with appropriate methods.

%The penetration surface extraction step is extracting overlapped surfaces on other objects. To utilize the RT core, we use the ray-intersection-based PIP(Point-In-Polygon) algorithms instead of collision detection between two objects which Tang et al.~\cite{SIG09HIST} used. (Sec.~\ref{sec:RT-PIP})
%RT core-based PIP test uses a ray-surface intersection test. For purpose this, we generate the GAS(Geometry Acceleration Structure) for each object. RT core-based PIP test takes the GAS of one object (e.g. $GAS_{A}$) and a set of vertex of another one (e.g. $P_{B}$). Then this computes the penetrated vertex set of another one (e.g. $P_{\partial B}$). To calculate the Hausdorff distance, these vertex sets change to objects constructed by penetrated surface (e.g. $\partial B$). Finally, the two generated overlapped surface objects $\partial A$ and $\partial B$ are used in the Hausdorff distance calculation step.

\begin{figure*}[t]
    \centering
    \begin{subfigure}[t]{0.45\linewidth}
        \centering
        \includegraphics[width=\linewidth]{figs/benchmark_vis/wacv_visual_trad.drawio-compressed.pdf}
        \caption{Examples of the conventional benchmark.}
        \label{fig:benchmark_visual_trad}
    \end{subfigure}
    \begin{subfigure}[t]{0.45\linewidth}
        \centering
        \includegraphics[width=\linewidth]{figs/benchmark_vis/wacv_visual_new.drawio-compressed.pdf}
        \caption{Examples of proposed TwinSynths benchmark.}
        \label{fig:benchmark_visual_new}
    \end{subfigure}
    
    \caption{Comparison of benchmarks. (a) Real images and fake GAN images are sampled from the test ProGAN set in the ForenSynths\cite{wang2020cnn}. Fake diffusion images are sampled from benchmark of Ojha \etal\cite{ojha2023towards}, each from LDM, GLIDE and DALL-E dataset. (b) Real images are sampled from ImageNet dataset, and corresponding fake images are generated by each model.}
    \label{fig:benchmark_visual}
\end{figure*}

The rapid advancement of AI image generation technologies has brought significant achievements but also growing social concern, as these technologies are increasingly misused for the creation of fake news, malicious defamation, and other forms of digital deception. 
In response, AI-generated image detection is receiving more attention.
There is a wide variety of generative models, along with commercial models with unknown internal architectures.
This highlights the need for a generalized detector capable of distinguishing between real and fake images, regardless of the generative model structure.

In this context, early research focused on identifying the characteristic fingerprints of generated images.
Recent work, NPR\cite{tan2024rethinking} shows that pixel-level features, induced by the upsampling layers commonly found in current generative models, can serve as cues for detection.
However, there are clear practical limitations to relying on low-level fingerprints. 
First, the approach is vulnerable to simple image degradations, such as JPEG compression or blurring, which are common in real-world online environments\cite{wang2020cnn}.
Additionally, the model may become biased toward the specific \emph{fakeness} seen at training in cases where generalization to novel generators is not sufficiently considered\cite{ojha2023towards, zhu2023gendet}.
For instance, a detector trained on GAN-generated images may learn the characteristics of GANs as the fake features, while mistakenly perceiving images generated by diffusion models as real.
This bias limits the detector's generalizability across different types of generative models.

To tackle these limitations, UnivFD\cite{ojha2023towards} utilizes a robust, pre-trained image encoder.
This image embedding is task-agnostic, enabling it to capture high-level semantic information from images.
However, we found that UnivFD exhibits a bias towards the observed content in the training images, learning another specific \textit{fakeness}.
\cref{fig:overview} shows that UnivFD misclassifies most GAN-generated images of a novel class (StyleGAN-\textit{bedroom}) as \textit{real}.
The \textit{bedroom} class is absent from the training set, which may lead the detector to mistakenly classify most images as real, demonstrating the detector's reliance on seen content during training.

We propose a novel technique called \textbf{PatchShuffle}, which is the core of our fake image detection model, \textbf{SFLD} (pronounced ``shuffled'').
PatchShuffle divides the image into non-overlapping patches and randomly shuffles them.
This procedure disrupts the high-level semantic structure of the image while preserving low-level textural information. 
This allows the detection model to better focus on both context and texture.
SFLD utilizes an ensemble of classifiers at multiple levels of PatchShuffle, leveraging hierarchical information across various patch sizes.
This approach ensures that the model leverages both the semantic and textural aspects of the image to improve fake image detection.
The results demonstrate that SFLD achieves superior performance with enhanced robustness and better generalization.

Furthermore, we observe that previous benchmarks have three limitations:
\textbf{(1) low image quality.}
The previous benchmarks contain a significant portion of low-quality images that lag behind the capabilities of current generative models.
As a result, the practical usefulness of these benchmarks is significantly reduced.
\textbf{(2) lack of content preservation.}
Some subsets---particularly foundation generative models---lack access to the training data used for the checkpoints.
Consequently, the content of the generated and real images often differs significantly, making it difficult to determine whether a detector focuses on real/fake discriminative features or other irrelevant features. 
\textbf{(3) limited class diversity.}
Existing benchmarks primarily focus on expanding the variety of generative models without considering the generated class diversity and scalability among generative models.
As shown in \cref{fig:overview}, this makes it difficult to identify detection bias towards certain classes, as well as hard to represent the in-the-wild performance of the detector due to limited class diversity.

To address these challenges, we propose a new benchmark generation methodology and corresponding benchmark, \textbf{TwinSynths}.
It consists of synthetic images that are visually near-identical to paired real images for practical and fair evaluations.
TwinSynths constructs image pairs that preserve both quality and content while retaining the architectural characteristics of each generative model.
Also, TwinSynths enables flexible class expansion by generating synthetic images tailored to the real image. 
Using this benchmark, we evaluate the performance of our proposed SFLD method as well as existing detection models.

Our main contributions are summarized as follows: 
\begin{itemize}[leftmargin=*]
    \setlength\itemsep{0.0em}
    \item We propose SFLD, a novel AI-generated image detection method that integrates semantic and texture artifacts on generated images, achieving state-of-the-art performance.
    \item We propose a new approach on benchmarks and the subset of generated images that can ensure the quality and content of generated images.
    \item We validate our method through extensive experiments and analysis that support our hypothesis.
\end{itemize}
