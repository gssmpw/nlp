% WACV 2025 Paper Template
% based on the WACV 2024 template, which is
% based on the CVPR 2023 template (https://media.icml.cc/Conferences/CVPR2023/cvpr2023-author_kit-v1_1-1.zip) with 2-track changes from the WACV 2023 template (https://github.com/wacv-pcs/WACV-2023-Author-Kit)
% based on the CVPR template provided by Ming-Ming Cheng (https://github.com/MCG-NKU/CVPR_Template)
% modified and extended by Stefan Roth (stefan.roth@NOSPAMtu-darmstadt.de)

\documentclass[10pt,twocolumn,letterpaper]{article}

%%%%%%%%% PAPER TYPE  - PLEASE UPDATE FOR FINAL VERSION
% \usepackage[review,algorithms]{wacv}      % To produce the REVIEW version for the algorithms track
% \usepackage[review,applications]{wacv}      % To produce the REVIEW version for the applications track
\usepackage{wacv}              % To produce the CAMERA-READY version
%\usepackage[pagenumbers]{wacv} % To force page numbers, e.g. for an arXiv version
\usepackage[accsupp]{axessibility}
% Include other packages here, before hyperref.
\usepackage{graphicx}
\usepackage{amsmath}
\usepackage{amssymb}
\usepackage{booktabs}
\usepackage{arydshln}

\usepackage{float}
\usepackage{kotex}
\usepackage{multirow}
\usepackage{multicol}
\usepackage{colortbl}
\usepackage{subcaption}
\usepackage{makecell}
\usepackage{float}

\usepackage[sectionbib]{chapterbib}
\usepackage{diagbox}

\usepackage{enumitem}


%%%%!!!!



\newcommand{\tft}{\textbf}
\newcommand{\et}{\textit{et al.}}
%\newcommand{\etal}{\textit{et al.}}
\newcommand{\ccn}[1]{\textcolor{red}{#1}}
\newcommand{\tb}[1]{\textbf{#1}}
\newcommand{\cred}{\color{red}}
\newcommand{\cblue}{\color{blue}}
\definecolor{light-gray}{gray}{0.82}
\renewcommand\ttdefault{cmvtt} % selects CM typewriter proportional font


% for pseudocode
\usepackage{listings}
\usepackage{algorithm}
\usepackage[T1]{fontenc}
\usepackage[utf8]{inputenc} % allow utf-8 input
\definecolor{codeblue}{rgb}{0.25,0.5,0.5}
\definecolor{codekw}{rgb}{0.85, 0.18, 0.50}
\lstset{
  backgroundcolor=\color{white},
  basicstyle=\fontsize{8pt}{8pt}\ttfamily\selectfont,
  columns=fullflexible,
  breaklines=true,
  captionpos=t,
  commentstyle=\fontsize{3.5pt}{3.5pt}\color{codeblue},
  keywordstyle=\fontsize{7.5pt}{7.5pt}\color{codekw},
  % numbers=left,
  stepnumber=1,    
  firstnumber=1,
  numberfirstline=true,
  escapechar=|,
}


% conditional color formatting for table cells
\usepackage{highlight_table}
% For a 2 color palette
% \gradientcell{cell_val}{min_val}{max_val}{colorlow}{colorhigh}{opacity} 
\newcommand{\g}[1]{\gradientcell{#1}{80}{100}{white}{gray}{70}}
% For a 3 color/divergent palette
% \gradientcelld{cell_val}{min_val}{mid_val}{max_val}{colorlow}{colormid}{colorhigh}{opacity}
% \newcommand{\g}[1]{\gradientcelld{#1}{40}{85}{100}{red}{white}{green}{70}}


% It is strongly recommended to use hyperref, especially for the review version.
% hyperref with option pagebackref eases the reviewers' job.
% Please disable hyperref *only* if you encounter grave issues, e.g. with the
% file validation for the camera-ready version.
%
% If you comment hyperref and then uncomment it, you should delete
% ReviewTempalte.aux before re-running LaTeX.
% (Or just hit 'q' on the first LaTeX run, let it finish, and you
%  should be clear).
\usepackage[pagebackref,breaklinks,colorlinks]{hyperref}


% Support for easy cross-referencing
\usepackage[capitalize]{cleveref}
\crefname{section}{Sec.}{Secs.}
\Crefname{section}{Section}{Sections}
\Crefname{table}{Table}{Tables}
\crefname{table}{Tab.}{Tabs.}


%%%%%%%%% PAPER ID  - PLEASE UPDATE
\def\wacvPaperID{694} % *** Enter the WACV Paper ID here
\def\confName{WACV}
\def\confYear{2025}

\begin{document}

%%%%%%%%% TITLE - PLEASE UPDATE
\title{SFLD: Reducing the content bias for AI-generated Image Detection}
\author{\textbf{Seoyeon Gye\thanks{Equal contribution.}\hspace{0.6cm} Junwon Ko\footnotemark[1]\hspace{0.6cm} Hyounguk Shon\footnotemark[1]\hspace{0.6cm} Minchan Kwon\hspace{0.7cm} Junmo Kim}\vspace{0.3cm} \\
School of Electrical Engineering, KAIST, South Korea\\
{\tt\small \{sawyun, kojunewon, hyounguk.shon, kmc0207, junmo.kim\}@kaist.ac.kr}}
\maketitle

%%%%%%%%% ABSTRACT
\begin{abstract}
    Identifying AI-generated content is critical for the safe and ethical use of generative AI.
    Recent research has focused on developing detectors that generalize to unknown generators, with popular methods relying either on high-level features or low-level fingerprints.
    However, these methods have clear limitations: biased towards unseen content, or vulnerable to common image degradations, such as JPEG compression.
    To address these issues, we propose a novel approach, SFLD, which incorporates PatchShuffle to integrate high-level semantic and low-level textural information. SFLD applies PatchShuffle at multiple levels, improving robustness and generalization across various generative models.
    Additionally, current benchmarks face challenges such as low image quality, insufficient content preservation, and limited class diversity. In response, we introduce TwinSynths, a new benchmark generation methodology that constructs visually near-identical pairs of real and synthetic images to ensure high quality and content preservation.
    Our extensive experiments and analysis show that SFLD outperforms existing methods on detecting a wide variety of fake images sourced from GANs, diffusion models, and TwinSynths, demonstrating the state-of-the-art performance and generalization capabilities to novel generative models.
    The TwinSynths dataset is publicly available at \textnormal{\url{https://huggingface.co/datasets/koooooooook/TwinSynths}}.
\end{abstract}


%%%%%%%%% BODY TEXT
% \vspace{-5pt}
\section{Introduction}
\label{sec:intro}

\section{Overview}

\revision{In this section, we first explain the foundational concept of Hausdorff distance-based penetration depth algorithms, which are essential for understanding our method (Sec.~\ref{sec:preliminary}).
We then provide a brief overview of our proposed RT-based penetration depth algorithm (Sec.~\ref{subsec:algo_overview}).}



\section{Preliminaries }
\label{sec:Preliminaries}

% Before we introduce our method, we first overview the important basics of 3D dynamic human modeling with Gaussian splatting. Then, we discuss the diffusion-based 3d generation techniques, and how they can be applied to human modeling.
% \ZY{I stopp here. TBC.}
% \subsection{Dynamic human modeling with Gaussian splatting}
\subsection{3D Gaussian Splatting}
3D Gaussian splatting~\cite{kerbl3Dgaussians} is an explicit scene representation that allows high-quality real-time rendering. The given scene is represented by a set of static 3D Gaussians, which are parameterized as follows: Gaussian center $x\in {\mathbb{R}^3}$, color $c\in {\mathbb{R}^3}$, opacity $\alpha\in {\mathbb{R}}$, spatial rotation in the form of quaternion $q\in {\mathbb{R}^4}$, and scaling factor $s\in {\mathbb{R}^3}$. Given these properties, the rendering process is represented as:
\begin{equation}
  I = Splatting(x, c, s, \alpha, q, r),
  \label{eq:splattingGA}
\end{equation}
where $I$ is the rendered image, $r$ is a set of query rays crossing the scene, and $Splatting(\cdot)$ is a differentiable rendering process. We refer readers to Kerbl et al.'s paper~\cite{kerbl3Dgaussians} for the details of Gaussian splatting. 



% \ZY{I would suggest move this part to the method part.}
% GaissianAvatar is a dynamic human generation model based on Gaussian splitting. Given a sequence of RGB images, this method utilizes fitted SMPLs and sampled points on its surface to obtain a pose-dependent feature map by a pose encoder. The pose-dependent features and a geometry feature are fed in a Gaussian decoder, which is employed to establish a functional mapping from the underlying geometry of the human form to diverse attributes of 3D Gaussians on the canonical surfaces. The parameter prediction process is articulated as follows:
% \begin{equation}
%   (\Delta x,c,s)=G_{\theta}(S+P),
%   \label{eq:gaussiandecoder}
% \end{equation}
%  where $G_{\theta}$ represents the Gaussian decoder, and $(S+P)$ is the multiplication of geometry feature S and pose feature P. Instead of optimizing all attributes of Gaussian, this decoder predicts 3D positional offset $\Delta{x} \in {\mathbb{R}^3}$, color $c\in\mathbb{R}^3$, and 3D scaling factor $ s\in\mathbb{R}^3$. To enhance geometry reconstruction accuracy, the opacity $\alpha$ and 3D rotation $q$ are set to fixed values of $1$ and $(1,0,0,0)$ respectively.
 
%  To render the canonical avatar in observation space, we seamlessly combine the Linear Blend Skinning function with the Gaussian Splatting~\cite{kerbl3Dgaussians} rendering process: 
% \begin{equation}
%   I_{\theta}=Splatting(x_o,Q,d),
%   \label{eq:splatting}
% \end{equation}
% \begin{equation}
%   x_o = T_{lbs}(x_c,p,w),
%   \label{eq:LBS}
% \end{equation}
% where $I_{\theta}$ represents the final rendered image, and the canonical Gaussian position $x_c$ is the sum of the initial position $x$ and the predicted offset $\Delta x$. The LBS function $T_{lbs}$ applies the SMPL skeleton pose $p$ and blending weights $w$ to deform $x_c$ into observation space as $x_o$. $Q$ denotes the remaining attributes of the Gaussians. With the rendering process, they can now reposition these canonical 3D Gaussians into the observation space.



\subsection{Score Distillation Sampling}
Score Distillation Sampling (SDS)~\cite{poole2022dreamfusion} builds a bridge between diffusion models and 3D representations. In SDS, the noised input is denoised in one time-step, and the difference between added noise and predicted noise is considered SDS loss, expressed as:

% \begin{equation}
%   \mathcal{L}_{SDS}(I_{\Phi}) \triangleq E_{t,\epsilon}[w(t)(\epsilon_{\phi}(z_t,y,t)-\epsilon)\frac{\partial I_{\Phi}}{\partial\Phi}],
%   \label{eq:SDSObserv}
% \end{equation}
\begin{equation}
    \mathcal{L}_{\text{SDS}}(I_{\Phi}) \triangleq \mathbb{E}_{t,\epsilon} \left[ w(t) \left( \epsilon_{\phi}(z_t, y, t) - \epsilon \right) \frac{\partial I_{\Phi}}{\partial \Phi} \right],
  \label{eq:SDSObservGA}
\end{equation}
where the input $I_{\Phi}$ represents a rendered image from a 3D representation, such as 3D Gaussians, with optimizable parameters $\Phi$. $\epsilon_{\phi}$ corresponds to the predicted noise of diffusion networks, which is produced by incorporating the noise image $z_t$ as input and conditioning it with a text or image $y$ at timestep $t$. The noise image $z_t$ is derived by introducing noise $\epsilon$ into $I_{\Phi}$ at timestep $t$. The loss is weighted by the diffusion scheduler $w(t)$. 
% \vspace{-3mm}

\subsection{Overview of the RTPD Algorithm}\label{subsec:algo_overview}
Fig.~\ref{fig:Overview} presents an overview of our RTPD algorithm.
It is grounded in the Hausdorff distance-based penetration depth calculation method (Sec.~\ref{sec:preliminary}).
%, similar to that of Tang et al.~\shortcite{SIG09HIST}.
The process consists of two primary phases: penetration surface extraction and Hausdorff distance calculation.
We leverage the RTX platform's capabilities to accelerate both of these steps.

\begin{figure*}[t]
    \centering
    \includegraphics[width=0.8\textwidth]{Image/overview.pdf}
    \caption{The overview of RT-based penetration depth calculation algorithm overview}
    \label{fig:Overview}
\end{figure*}

The penetration surface extraction phase focuses on identifying the overlapped region between two objects.
\revision{The penetration surface is defined as a set of polygons from one object, where at least one of its vertices lies within the other object. 
Note that in our work, we focus on triangles rather than general polygons, as they are processed most efficiently on the RTX platform.}
To facilitate this extraction, we introduce a ray-tracing-based \revision{Point-in-Polyhedron} test (RT-PIP), significantly accelerated through the use of RT cores (Sec.~\ref{sec:RT-PIP}).
This test capitalizes on the ray-surface intersection capabilities of the RTX platform.
%
Initially, a Geometry Acceleration Structure (GAS) is generated for each object, as required by the RTX platform.
The RT-PIP module takes the GAS of one object (e.g., $GAS_{A}$) and the point set of the other object (e.g., $P_{B}$).
It outputs a set of points (e.g., $P_{\partial B}$) representing the penetration region, indicating their location inside the opposing object.
Subsequently, a penetration surface (e.g., $\partial B$) is constructed using this point set (e.g., $P_{\partial B}$) (Sec.~\ref{subsec:surfaceGen}).
%
The generated penetration surfaces (e.g., $\partial A$ and $\partial B$) are then forwarded to the next step. 

The Hausdorff distance calculation phase utilizes the ray-surface intersection test of the RTX platform (Sec.~\ref{sec:RT-Hausdorff}) to compute the Hausdorff distance between two objects.
We introduce a novel Ray-Tracing-based Hausdorff DISTance algorithm, RT-HDIST.
It begins by generating GAS for the two penetration surfaces, $P_{\partial A}$ and $P_{\partial B}$, derived from the preceding step.
RT-HDIST processes the GAS of a penetration surface (e.g., $GAS_{\partial A}$) alongside the point set of the other penetration surface (e.g., $P_{\partial B}$) to compute the penetration depth between them.
The algorithm operates bidirectionally, considering both directions ($\partial A \to \partial B$ and $\partial B \to \partial A$).
The final penetration depth between the two objects, A and B, is determined by selecting the larger value from these two directional computations.

%In the Hausdorff distance calculation step, we compute the Hausdorff distance between given two objects using a ray-surface-intersection test. (Sec.~\ref{sec:RT-Hausdorff}) Initially, we construct the GAS for both $\partial A$ and $\partial B$ to utilize the RT-core effectively. The RT-based Hausdorff distance algorithms then determine the Hausdorff distance by processing the GAS of one object (e.g. $GAS_{\partial A}$) and set of the vertices of the other (e.g. $P_{\partial B}$). Following the Hausdorff distance definition (Eq.~\ref{equation:hausdorff_definition}), we compute the Hausdorff distance to both directions ($\partial A \to \partial B$) and ($\partial B \to \partial A$). As a result, the bigger one is the final Hausdorff distance, and also it is the penetration depth between input object $A$ and $B$.


%the proposed RT-based penetration depth calculation pipeline.
%Our proposed methods adopt Tang's Hausdorff-based penetration depth methods~\cite{SIG09HIST}. The pipeline is divided into the penetration surface extraction step and the Hausdorff distance calculation between the penetration surface steps. However, since Tang's approach is not suitable for the RT platform in detail, we modified and applied it with appropriate methods.

%The penetration surface extraction step is extracting overlapped surfaces on other objects. To utilize the RT core, we use the ray-intersection-based PIP(Point-In-Polygon) algorithms instead of collision detection between two objects which Tang et al.~\cite{SIG09HIST} used. (Sec.~\ref{sec:RT-PIP})
%RT core-based PIP test uses a ray-surface intersection test. For purpose this, we generate the GAS(Geometry Acceleration Structure) for each object. RT core-based PIP test takes the GAS of one object (e.g. $GAS_{A}$) and a set of vertex of another one (e.g. $P_{B}$). Then this computes the penetrated vertex set of another one (e.g. $P_{\partial B}$). To calculate the Hausdorff distance, these vertex sets change to objects constructed by penetrated surface (e.g. $\partial B$). Finally, the two generated overlapped surface objects $\partial A$ and $\partial B$ are used in the Hausdorff distance calculation step.

\begin{figure*}[t]
    \centering
    \begin{subfigure}[t]{0.45\linewidth}
        \centering
        \includegraphics[width=\linewidth]{figs/benchmark_vis/wacv_visual_trad.drawio-compressed.pdf}
        \caption{Examples of the conventional benchmark.}
        \label{fig:benchmark_visual_trad}
    \end{subfigure}
    \begin{subfigure}[t]{0.45\linewidth}
        \centering
        \includegraphics[width=\linewidth]{figs/benchmark_vis/wacv_visual_new.drawio-compressed.pdf}
        \caption{Examples of proposed TwinSynths benchmark.}
        \label{fig:benchmark_visual_new}
    \end{subfigure}
    
    \caption{Comparison of benchmarks. (a) Real images and fake GAN images are sampled from the test ProGAN set in the ForenSynths\cite{wang2020cnn}. Fake diffusion images are sampled from benchmark of Ojha \etal\cite{ojha2023towards}, each from LDM, GLIDE and DALL-E dataset. (b) Real images are sampled from ImageNet dataset, and corresponding fake images are generated by each model.}
    \label{fig:benchmark_visual}
\end{figure*}

The rapid advancement of AI image generation technologies has brought significant achievements but also growing social concern, as these technologies are increasingly misused for the creation of fake news, malicious defamation, and other forms of digital deception. 
In response, AI-generated image detection is receiving more attention.
There is a wide variety of generative models, along with commercial models with unknown internal architectures.
This highlights the need for a generalized detector capable of distinguishing between real and fake images, regardless of the generative model structure.

In this context, early research focused on identifying the characteristic fingerprints of generated images.
Recent work, NPR\cite{tan2024rethinking} shows that pixel-level features, induced by the upsampling layers commonly found in current generative models, can serve as cues for detection.
However, there are clear practical limitations to relying on low-level fingerprints. 
First, the approach is vulnerable to simple image degradations, such as JPEG compression or blurring, which are common in real-world online environments\cite{wang2020cnn}.
Additionally, the model may become biased toward the specific \emph{fakeness} seen at training in cases where generalization to novel generators is not sufficiently considered\cite{ojha2023towards, zhu2023gendet}.
For instance, a detector trained on GAN-generated images may learn the characteristics of GANs as the fake features, while mistakenly perceiving images generated by diffusion models as real.
This bias limits the detector's generalizability across different types of generative models.

To tackle these limitations, UnivFD\cite{ojha2023towards} utilizes a robust, pre-trained image encoder.
This image embedding is task-agnostic, enabling it to capture high-level semantic information from images.
However, we found that UnivFD exhibits a bias towards the observed content in the training images, learning another specific \textit{fakeness}.
\cref{fig:overview} shows that UnivFD misclassifies most GAN-generated images of a novel class (StyleGAN-\textit{bedroom}) as \textit{real}.
The \textit{bedroom} class is absent from the training set, which may lead the detector to mistakenly classify most images as real, demonstrating the detector's reliance on seen content during training.

We propose a novel technique called \textbf{PatchShuffle}, which is the core of our fake image detection model, \textbf{SFLD} (pronounced ``shuffled'').
PatchShuffle divides the image into non-overlapping patches and randomly shuffles them.
This procedure disrupts the high-level semantic structure of the image while preserving low-level textural information. 
This allows the detection model to better focus on both context and texture.
SFLD utilizes an ensemble of classifiers at multiple levels of PatchShuffle, leveraging hierarchical information across various patch sizes.
This approach ensures that the model leverages both the semantic and textural aspects of the image to improve fake image detection.
The results demonstrate that SFLD achieves superior performance with enhanced robustness and better generalization.

Furthermore, we observe that previous benchmarks have three limitations:
\textbf{(1) low image quality.}
The previous benchmarks contain a significant portion of low-quality images that lag behind the capabilities of current generative models.
As a result, the practical usefulness of these benchmarks is significantly reduced.
\textbf{(2) lack of content preservation.}
Some subsets---particularly foundation generative models---lack access to the training data used for the checkpoints.
Consequently, the content of the generated and real images often differs significantly, making it difficult to determine whether a detector focuses on real/fake discriminative features or other irrelevant features. 
\textbf{(3) limited class diversity.}
Existing benchmarks primarily focus on expanding the variety of generative models without considering the generated class diversity and scalability among generative models.
As shown in \cref{fig:overview}, this makes it difficult to identify detection bias towards certain classes, as well as hard to represent the in-the-wild performance of the detector due to limited class diversity.

To address these challenges, we propose a new benchmark generation methodology and corresponding benchmark, \textbf{TwinSynths}.
It consists of synthetic images that are visually near-identical to paired real images for practical and fair evaluations.
TwinSynths constructs image pairs that preserve both quality and content while retaining the architectural characteristics of each generative model.
Also, TwinSynths enables flexible class expansion by generating synthetic images tailored to the real image. 
Using this benchmark, we evaluate the performance of our proposed SFLD method as well as existing detection models.

Our main contributions are summarized as follows: 
\begin{itemize}[leftmargin=*]
    \setlength\itemsep{0.0em}
    \item We propose SFLD, a novel AI-generated image detection method that integrates semantic and texture artifacts on generated images, achieving state-of-the-art performance.
    \item We propose a new approach on benchmarks and the subset of generated images that can ensure the quality and content of generated images.
    \item We validate our method through extensive experiments and analysis that support our hypothesis.
\end{itemize}

\section{Method}
\label{sec:method}


\begin{figure}[t]
     \centering
     \includegraphics[width=1.0\linewidth]{figs/method.pdf}
     \caption{Architecture of the proposed fake image detector (SFLD). $z_{s_i}$ refers to the logit score generated from an input image processed via $s_i\text{×}s_i$ patch size. $\Sigma$ indicates weighted sum.}
     \label{fig:method-architecture}
\end{figure}

\subsection{Patch Shuffling Fake Detection} 
\label{subsec:SFLD}
\textbf{Backbone.}
We utilize the visual encoder of CLIP ViT-L/14\cite{dosovitskiy2021an, CLIP} to leverage the pre-trained feature space. This choice is based on Ojha \etal\cite{ojha2023towards}, which showed that it outperforms other models such as CLIP:ResNet-50, ImageNet:ResNet-50, and ImageNet:ViT-B/16 in distinguishing real from fake images. The results indicated that both the architecture and the pre-training data are crucial. 
Based on this insight, we chose the ViT model for our backbone. 
As shown in \cref{fig:method-architecture}, we extract CLIP features and train a fully connected layer to classify real and fake images.


\textbf{PatchShuffle.}
To effectively integrate both semantic and textural features, PatchShuffle disrupts the global structure of an image while preserving local features.
In the PatchShuffle process, the input images are divided into non-overlapping patches of size $s \times s$ and then randomly shuffled. This operation produces a new shuffled image $x_s$.

For a given $s$, the logit score of the shuffled image is, 
\begin{equation}
    z_{s} = \psi(f(x_s)) \,,
\end{equation}
where $f( \cdot )$ represents a pre-trained CLIP encoder and $\psi(\cdot)$ is a single fully connected layer appended to $f$.

There are classifiers for each patch size of shuffled images to leverage local structure information hierarchically within the image.
We selected patch sizes of 28, 56, and 224 for the proposed SFLD.
As shown in \cref{fig:method-architecture}, $s_0$ is 224, $s_1$ is 56 and $s_2$ is 28.
These configurations are studied in detail in \cref{subsec:patchsetting}.
For each patch size $s_j$, the classifier $\psi_{s_j}$ is trained independently.
Notably, UnivFD takes a center-cropped 224×224 image as input to the CLIP encoder.
Therefore, when using a patch size of 224 in PatchShuffle, it effectively corresponds to the same setting as UnivFD\cite{ojha2023towards}.

We employ binary cross-entropy loss for each classifier:
\small\begin{equation}
    \mathcal{L} = -\frac{1}{N} \sum_{i=1}^{N} \left[ y_i \log \sigma(z_{s_j}) + (1 - y_i) \log \left(1 - \sigma(z_{s_j}) \right) \right] 
\end{equation}\normalsize
where $N$ is the number of data and $y_i \in \{0, 1\}$ is the label whether an input $x_i$ is real ($y_i = 0$) or fake ($y_i = 1$).

\textbf{SFLD.} SFLD combines multiple classifiers trained on shuffled images with different patch sizes.
By varying the patch size, SFLD incorporates models that focus on various levels of structural features, ranging from fine-grained local details to more global patterns.

During testing, $N_{views}=10$ shuffled views are generated for each patch size. The logits from these views are averaged and processed by the corresponding classifier. The final probability $P_{SFLD}(y|x)$ is computed by averaging the logits across patch sizes and applying the sigmoid function:
\small\begin{equation}
    P_{\text{SFLD}}(y|x) = \sigma\left(\frac{1}{k} \sum_{j=1}^{k} \psi_{s_j}\left(\frac{1}{N_{\text{views}}} \sum_{i=1}^{N_{\text{views}}} f(x_{s_j}^i)\right)\right) \,,
\end{equation}\normalsize
where $k$ is the number of patch sizes used in the ensemble (e.g., $k=3$ in our configuration).

Binary classification is done using a threshold of 0.5 on $P_{SFLD}$. Although the fusion method is simple and not tuned for each test class, its simplicity enables strong generalization across diverse fake image sources. By combining classifiers trained on different patch sizes, SFLD achieves a robust and general detection performance. \cref{alg:pseudocode} shows the full workflow of SFLD, especially the fusion of multiple classifiers during inference.

\subsection{TwinSynths} 
In \cref{sec:intro}, we pointed out three shortcomings in the previous benchmarks: low image quality, lack of content preservation, and limited class diversity.
This issue must be addressed to allow a comprehensive comparison of detectors.
Therefore, we propose a novel dataset creation methodology and \emph{TwinSynths} benchmark, consisting of GAN- and diffusion-based generated images that are  paired with visually-identical real counterparts.
To create a practical benchmark for evaluating generated image detectors, it is essential to ensure the generation of high-quality images that preserve the original content.
To achieve this, the image generation process should ideally sample a distribution that closely resembles a real distribution.
From this perspective, the image generation or sampling process can be interpreted as effectively fitting the generator to a single real image.
Through this approach, we construct image pairs that preserve the content of the images while reflecting the architectural traits of the generative models.
Additionally, this methodology allows for the expansion of target classes in the benchmark by generating paired images for any real image.
\cref{fig:benchmark_visual_new} are some examples of TwinSynths.
We can see that the content of the paired real image is faithfully reproduced and the quality of the generated image is guaranteed.

\textbf{TwinSynths-GAN benchmark.}
The GAN-based subsets in the previous benchmark have disparate training configurations, especially the class of training images, resulting in a discrepancy between the generated and the real images.
In order to generate a high quality image that preserves the content of the paired real image while leveraging the training methodology of GANs, we trained the generator from scratch using a single real image.
The MSE loss was provided to the generator to generate an image that is identical to the original image.
For reproduction, the latent vector for the generator input is maintained at a fixed value.
We created 8,000 generated images from 80 selected ImageNet\cite{russakovsky2015imagenet} classes, which is much larger than previous benchmarks.
We selected 40 classes following the \emph{ProGAN} subset in ForenSynths\cite{wang2020cnn}, while the other 40 classes were chosen arbitrarily. 
We utilized DCGAN \cite{radford2015unsupervised} architecture. 

\textbf{TwinSynths-DM benchmark.} 
In comparison to GAN-based subsets, diffusion-based subsets in conventional benchmarks were generated with off-the-shelf pretrained models, having much severer content discrepancy between real and generated images.
In order to generate a high quality image that preserves the content of paired real image while leveraging the inference process of diffusion models, we used DDIM inversion\cite{songdenoising} to generate image that is similar to the real image.
We apply a DDIM forward process to the real image to make it noisy and perform text-conditioned DDIM denoising process using the prompt template \texttt{`a photo of  \{class name\}'}. 
For the prompts, we used the class names from ImageNet.
This process makes TwinSynths-DM preserve the similarity with the paired real images. 
We used the same image classes used to create TwinSynths-GAN. 
We utilized the pretrained decoder and scheduler of \cite{songdenoising}.

\section{Experiments}

\subsection{Setups}
\subsubsection{Implementation Details}
We apply our FDS method to two types of 3DGS: 
the original 3DGS, and 2DGS~\citep{huang20242d}. 
%
The number of iterations in our optimization 
process is 35,000.
We follow the default training configuration 
and apply our FDS method after 15,000 iterations,
then we add normal consistency loss for both
3DGS and 2DGS after 25000 iterations.
%
The weight for FDS, $\lambda_{fds}$, is set to 0.015,
the $\sigma$ is set to 23,
and the weight for normal consistency is set to 0.15
for all experiments. 
We removed the depth distortion loss in 2DGS 
because we found that it degrades its results in indoor scenes.
%
The Gaussian point cloud is initialized using Colmap
for all datasets.
%
%
We tested the impact of 
using Sea Raft~\citep{wang2025sea} and 
Raft\citep{teed2020raft} on FDS performance.
%
Due to the blurriness of the ScanNet dataset, 
additional prior constraints are required.
Thus, we incorporate normal prior supervision 
on the rendered normals 
in ScanNet (V2) dataset by default.
The normal prior is predicted by the Stable Normal 
model~\citep{ye2024stablenormal}
across all types of 3DGS.
%
The entire framework is implemented in 
PyTorch~\citep{paszke2019pytorch}, 
and all experiments are conducted on 
a single NVIDIA 4090D GPU.

\begin{figure}[t] \centering
    \makebox[0.16\textwidth]{\scriptsize Input}
    \makebox[0.16\textwidth]{\scriptsize 3DGS}
    \makebox[0.16\textwidth]{\scriptsize 2DGS}
    \makebox[0.16\textwidth]{\scriptsize 3DGS + FDS}
    \makebox[0.16\textwidth]{\scriptsize 2DGS + FDS}
    \makebox[0.16\textwidth]{\scriptsize GT (Depth)}

    \includegraphics[width=0.16\textwidth]{figure/fig3_img/compare3/gt_rgb/frame_00522.jpg}
    \includegraphics[width=0.16\textwidth]{figure/fig3_img/compare3/3DGS/frame_00522.jpg}
    \includegraphics[width=0.16\textwidth]{figure/fig3_img/compare3/2DGS/frame_00522.jpg}
    \includegraphics[width=0.16\textwidth]{figure/fig3_img/compare3/3DGS+FDS/frame_00522.jpg}
    \includegraphics[width=0.16\textwidth]{figure/fig3_img/compare3/2DGS+FDS/frame_00522.jpg}
    \includegraphics[width=0.16\textwidth]{figure/fig3_img/compare3/gt_depth/frame_00522.jpg} \\

    % \includegraphics[width=0.16\textwidth]{figure/fig3_img/compare1/gt_rgb/frame_00137.jpg}
    % \includegraphics[width=0.16\textwidth]{figure/fig3_img/compare1/3DGS/frame_00137.jpg}
    % \includegraphics[width=0.16\textwidth]{figure/fig3_img/compare1/2DGS/frame_00137.jpg}
    % \includegraphics[width=0.16\textwidth]{figure/fig3_img/compare1/3DGS+FDS/frame_00137.jpg}
    % \includegraphics[width=0.16\textwidth]{figure/fig3_img/compare1/2DGS+FDS/frame_00137.jpg}
    % \includegraphics[width=0.16\textwidth]{figure/fig3_img/compare1/gt_depth/frame_00137.jpg} \\

     \includegraphics[width=0.16\textwidth]{figure/fig3_img/compare2/gt_rgb/frame_00262.jpg}
    \includegraphics[width=0.16\textwidth]{figure/fig3_img/compare2/3DGS/frame_00262.jpg}
    \includegraphics[width=0.16\textwidth]{figure/fig3_img/compare2/2DGS/frame_00262.jpg}
    \includegraphics[width=0.16\textwidth]{figure/fig3_img/compare2/3DGS+FDS/frame_00262.jpg}
    \includegraphics[width=0.16\textwidth]{figure/fig3_img/compare2/2DGS+FDS/frame_00262.jpg}
    \includegraphics[width=0.16\textwidth]{figure/fig3_img/compare2/gt_depth/frame_00262.jpg} \\

    \includegraphics[width=0.16\textwidth]{figure/fig3_img/compare4/gt_rgb/frame00000.png}
    \includegraphics[width=0.16\textwidth]{figure/fig3_img/compare4/3DGS/frame00000.png}
    \includegraphics[width=0.16\textwidth]{figure/fig3_img/compare4/2DGS/frame00000.png}
    \includegraphics[width=0.16\textwidth]{figure/fig3_img/compare4/3DGS+FDS/frame00000.png}
    \includegraphics[width=0.16\textwidth]{figure/fig3_img/compare4/2DGS+FDS/frame00000.png}
    \includegraphics[width=0.16\textwidth]{figure/fig3_img/compare4/gt_depth/frame00000.png} \\

    \includegraphics[width=0.16\textwidth]{figure/fig3_img/compare5/gt_rgb/frame00080.png}
    \includegraphics[width=0.16\textwidth]{figure/fig3_img/compare5/3DGS/frame00080.png}
    \includegraphics[width=0.16\textwidth]{figure/fig3_img/compare5/2DGS/frame00080.png}
    \includegraphics[width=0.16\textwidth]{figure/fig3_img/compare5/3DGS+FDS/frame00080.png}
    \includegraphics[width=0.16\textwidth]{figure/fig3_img/compare5/2DGS+FDS/frame00080.png}
    \includegraphics[width=0.16\textwidth]{figure/fig3_img/compare5/gt_depth/frame00080.png} \\



    \caption{\textbf{Comparison of depth reconstruction on Mushroom and ScanNet datasets.} The original
    3DGS or 2DGS model equipped with FDS can remove unwanted floaters and reconstruct
    geometry more preciously.}
    \label{fig:compare}
\end{figure}


\subsubsection{Datasets and Metrics}

We evaluate our method for 3D reconstruction 
and novel view synthesis tasks on
\textbf{Mushroom}~\citep{ren2024mushroom},
\textbf{ScanNet (v2)}~\citep{dai2017scannet}, and 
\textbf{Replica}~\citep{replica19arxiv}
datasets,
which feature challenging indoor scenes with both 
sparse and dense image sampling.
%
The Mushroom dataset is an indoor dataset 
with sparse image sampling and two distinct 
camera trajectories. 
%
We train our model on the training split of 
the long capture sequence and evaluate 
novel view synthesis on the test split 
of the long capture sequences.
%
Five scenes are selected to evaluate our FDS, 
including "coffee room", "honka", "kokko", 
"sauna", and "vr room". 
%
ScanNet(V2)~\citep{dai2017scannet}  consists of 1,613 indoor scenes
with annotated camera poses and depth maps. 
%
We select 5 scenes from the ScanNet (V2) dataset, 
uniformly sampling one-tenth of the views,
following the approach in ~\citep{guo2022manhattan}.
To further improve the geometry rendering quality of 3DGS, 
%
Replica~\citep{replica19arxiv} contains small-scale 
real-world indoor scans. 
We evaluate our FDS on five scenes from 
Replica: office0, office1, office2, office3 and office4,
selecting one-tenth of the views for training.
%
The results for Replica are provided in the 
supplementary materials.
To evaluate the rendering quality and geometry 
of 3DGS, we report PSNR, SSIM, and LPIPS for 
rendering quality, along with Absolute Relative Distance 
(Abs Rel) as a depth quality metrics.
%
Additionally, for mesh evaluation, 
we use metrics including Accuracy, Completion, 
Chamfer-L1 distance, Normal Consistency, 
and F-scores.




\subsection{Results}
\subsubsection{Depth rendering and novel view synthesis}
The comparison results on Mushroom and 
ScanNet are presented in \tabref{tab:mushroom} 
and \tabref{tab:scannet}, respectively. 
%
Due to the sparsity of sampling 
in the Mushroom dataset,
challenges are posed for both GOF~\citep{yu2024gaussian} 
and PGSR~\citep{chen2024pgsr}, 
leading to their relative poor performance 
on the Mushroom dataset.
%
Our approach achieves the best performance 
with the FDS method applied during the training process.
The FDS significantly enhances the 
geometric quality of 3DGS on the Mushroom dataset, 
improving the "abs rel" metric by more than 50\%.
%
We found that Sea Raft~\citep{wang2025sea}
outperforms Raft~\citep{teed2020raft} on FDS, 
indicating that a better optical flow model 
can lead to more significant improvements.
%
Additionally, the render quality of RGB 
images shows a slight improvement, 
by 0.58 in 3DGS and 0.50 in 2DGS, 
benefiting from the incorporation of cross-view consistency in FDS. 
%
On the Mushroom
dataset, adding the FDS loss increases 
the training time by half an hour, which maintains the same
level as baseline.
%
Similarly, our method shows a notable improvement on the ScanNet dataset as well using Sea Raft~\citep{wang2025sea} Model. The "abs rel" metric in 2DGS is improved nearly 50\%. This demonstrates the robustness and effectiveness of the FDS method across different datasets.
%


% \begin{wraptable}{r}{0.6\linewidth} \centering
% \caption{\textbf{Ablation study on geometry priors.}} 
%         \label{tab:analysis_prior}
%         \resizebox{\textwidth}{!}{
\begin{tabular}{c| c c c c c | c c c c}

    \hline
     Method &  Acc$\downarrow$ & Comp $\downarrow$ & C-L1 $\downarrow$ & NC $\uparrow$ & F-Score $\uparrow$ &  Abs Rel $\downarrow$ &  PSNR $\uparrow$  & SSIM  $\uparrow$ & LPIPS $\downarrow$ \\ \hline
    2DGS&   0.1078&  0.0850&  0.0964&  0.7835&  0.5170&  0.1002&  23.56&  0.8166& 0.2730\\
    2DGS+Depth&   0.0862&  0.0702&  0.0782&  0.8153&  0.5965&  0.0672&  23.92&  0.8227& 0.2619 \\
    2DGS+MVDepth&   0.2065&  0.0917&  0.1491&  0.7832&  0.3178&  0.0792&  23.74&  0.8193& 0.2692 \\
    2DGS+Normal&   0.0939&  0.0637&  0.0788&  \textbf{0.8359}&  0.5782&  0.0768&  23.78&  0.8197& 0.2676 \\
    2DGS+FDS &  \textbf{0.0615} & \textbf{ 0.0534}& \textbf{0.0574}& 0.8151& \textbf{0.6974}&  \textbf{0.0561}&  \textbf{24.06}&  \textbf{0.8271}&\textbf{0.2610} \\ \hline
    2DGS+Depth+FDS &  0.0561 &  0.0519& 0.0540& 0.8295& 0.7282&  0.0454&  \textbf{24.22}& \textbf{0.8291}&\textbf{0.2570} \\
    2DGS+Normal+FDS &  \textbf{0.0529} & \textbf{ 0.0450}& \textbf{0.0490}& \textbf{0.8477}& \textbf{0.7430}&  \textbf{0.0443}&  24.10&  0.8283& 0.2590 \\
    2DGS+Depth+Normal &  0.0695 & 0.0513& 0.0604& 0.8540&0.6723&  0.0523&  24.09&  0.8264&0.2575\\ \hline
    2DGS+Depth+Normal+FDS &  \textbf{0.0506} & \textbf{0.0423}& \textbf{0.0464}& \textbf{0.8598}&\textbf{0.7613}&  \textbf{0.0403}&  \textbf{24.22}& 
    \textbf{0.8300}&\textbf{0.0403}\\
    
\bottomrule
\end{tabular}
}
% \end{wraptable}



The qualitative comparisons on the Mushroom and ScanNet dataset 
are illustrated in \figref{fig:compare}. 
%
%
As seen in the first row of \figref{fig:compare}, 
both the original 3DGS and 2DGS suffer from overfitting, 
leading to corrupted geometry generation. 
%
Our FDS effectively mitigates this issue by 
supervising the matching relationship between 
the input and sampled views, 
helping to recover the geometry.
%
FDS also improves the refinement of geometric details, 
as shown in other rows. 
By incorporating the matching prior through FDS, 
the quality of the rendered depth is significantly improved.
%

\begin{table}[t] \centering
\begin{minipage}[t]{0.96\linewidth}
        \captionof{table}{\textbf{3D Reconstruction 
        and novel view synthesis results on Mushroom dataset. * 
        Represents that FDS uses the Raft model.
        }}
        \label{tab:mushroom}
        \resizebox{\textwidth}{!}{
\begin{tabular}{c| c c c c c | c c c c c}
    \hline
     Method &  Acc$\downarrow$ & Comp $\downarrow$ & C-L1 $\downarrow$ & NC $\uparrow$ & F-Score $\uparrow$ &  Abs Rel $\downarrow$ &  PSNR $\uparrow$  & SSIM  $\uparrow$ & LPIPS $\downarrow$ & Time  $\downarrow$ \\ \hline

    % DN-splatter &   &  &  &  &  &  &  &  & \\
    GOF &  0.1812 & 0.1093 & 0.1453 & 0.6292 & 0.3665 & 0.2380  & 21.37  &  0.7762  & 0.3132  & $\approx$1.4h\\ 
    PGSR &  0.0971 & 0.1420 & 0.1196 & 0.7193 & 0.5105 & 0.1723  & 22.13  & 0.7773  & 0.2918  & $\approx$1.2h \\ \hline
    3DGS &   0.1167 &  0.1033&  0.1100&  0.7954&  0.3739&  0.1214&  24.18&  0.8392& 0.2511 &$\approx$0.8h \\
    3DGS + FDS$^*$ & 0.0569  & 0.0676 & 0.0623 & 0.8105 & 0.6573 & 0.0603 & 24.72  & 0.8489 & 0.2379 &$\approx$1.3h \\
    3DGS + FDS & \textbf{0.0527}  & \textbf{0.0565} & \textbf{0.0546} & \textbf{0.8178} & \textbf{0.6958} & \textbf{0.0568} & \textbf{24.76}  & \textbf{0.8486} & \textbf{0.2381} &$\approx$1.3h \\ \hline
    2DGS&   0.1078&  0.0850&  0.0964&  0.7835&  0.5170&  0.1002&  23.56&  0.8166& 0.2730 &$\approx$0.8h\\
    2DGS + FDS$^*$ &  0.0689 &  0.0646& 0.0667& 0.8042& 0.6582& 0.0589& 23.98&  0.8255&0.2621 &$\approx$1.3h\\
    2DGS + FDS &  \textbf{0.0615} & \textbf{ 0.0534}& \textbf{0.0574}& \textbf{0.8151}& \textbf{0.6974}&  \textbf{0.0561}&  \textbf{24.06}&  \textbf{0.8271}&\textbf{0.2610} &$\approx$1.3h \\ \hline
\end{tabular}
}
\end{minipage}\hfill
\end{table}

\begin{table}[t] \centering
\begin{minipage}[t]{0.96\linewidth}
        \captionof{table}{\textbf{3D Reconstruction 
        and novel view synthesis results on ScanNet dataset.}}
        \label{tab:scannet}
        \resizebox{\textwidth}{!}{
\begin{tabular}{c| c c c c c | c c c c }
    \hline
     Method &  Acc $\downarrow$ & Comp $\downarrow$ & C-L1 $\downarrow$ & NC $\uparrow$ & F-Score $\uparrow$ &  Abs Rel $\downarrow$ &  PSNR $\uparrow$  & SSIM  $\uparrow$ & LPIPS $\downarrow$ \\ \hline
    GOF & 1.8671  & 0.0805 & 0.9738 & 0.5622 & 0.2526 & 0.1597  & 21.55  & 0.7575  & 0.3881 \\
    PGSR &  0.2928 & 0.5103 & 0.4015 & 0.5567 & 0.1926 & 0.1661  & 21.71 & 0.7699  & 0.3899 \\ \hline

    3DGS &  0.4867 & 0.1211 & 0.3039 & 0.7342& 0.3059 & 0.1227 & 22.19& 0.7837 & 0.3907\\
    3DGS + FDS &  \textbf{0.2458} & \textbf{0.0787} & \textbf{0.1622} & \textbf{0.7831} & 
    \textbf{0.4482} & \textbf{0.0573} & \textbf{22.83} & \textbf{0.7911} & \textbf{0.3826} \\ \hline
    2DGS &  0.2658 & 0.0845 & 0.1752 & 0.7504& 0.4464 & 0.0831 & 22.59& 0.7881 & 0.3854\\
    2DGS + FDS &  \textbf{0.1457} & \textbf{0.0679} & \textbf{0.1068} & \textbf{0.7883} & 
    \textbf{0.5459} & \textbf{0.0432} & \textbf{22.91} & \textbf{0.7928} & \textbf{0.3800} \\ \hline
\end{tabular}
}
\end{minipage}\hfill
\end{table}


\begin{table}[t] \centering
\begin{minipage}[t]{0.96\linewidth}
        \captionof{table}{\textbf{Ablation study on geometry priors.}}
        \label{tab:analysis_prior}
        \resizebox{\textwidth}{!}{
\begin{tabular}{c| c c c c c | c c c c}

    \hline
     Method &  Acc$\downarrow$ & Comp $\downarrow$ & C-L1 $\downarrow$ & NC $\uparrow$ & F-Score $\uparrow$ &  Abs Rel $\downarrow$ &  PSNR $\uparrow$  & SSIM  $\uparrow$ & LPIPS $\downarrow$ \\ \hline
    2DGS&   0.1078&  0.0850&  0.0964&  0.7835&  0.5170&  0.1002&  23.56&  0.8166& 0.2730\\
    2DGS+Depth&   0.0862&  0.0702&  0.0782&  0.8153&  0.5965&  0.0672&  23.92&  0.8227& 0.2619 \\
    2DGS+MVDepth&   0.2065&  0.0917&  0.1491&  0.7832&  0.3178&  0.0792&  23.74&  0.8193& 0.2692 \\
    2DGS+Normal&   0.0939&  0.0637&  0.0788&  \textbf{0.8359}&  0.5782&  0.0768&  23.78&  0.8197& 0.2676 \\
    2DGS+FDS &  \textbf{0.0615} & \textbf{ 0.0534}& \textbf{0.0574}& 0.8151& \textbf{0.6974}&  \textbf{0.0561}&  \textbf{24.06}&  \textbf{0.8271}&\textbf{0.2610} \\ \hline
    2DGS+Depth+FDS &  0.0561 &  0.0519& 0.0540& 0.8295& 0.7282&  0.0454&  \textbf{24.22}& \textbf{0.8291}&\textbf{0.2570} \\
    2DGS+Normal+FDS &  \textbf{0.0529} & \textbf{ 0.0450}& \textbf{0.0490}& \textbf{0.8477}& \textbf{0.7430}&  \textbf{0.0443}&  24.10&  0.8283& 0.2590 \\
    2DGS+Depth+Normal &  0.0695 & 0.0513& 0.0604& 0.8540&0.6723&  0.0523&  24.09&  0.8264&0.2575\\ \hline
    2DGS+Depth+Normal+FDS &  \textbf{0.0506} & \textbf{0.0423}& \textbf{0.0464}& \textbf{0.8598}&\textbf{0.7613}&  \textbf{0.0403}&  \textbf{24.22}& 
    \textbf{0.8300}&\textbf{0.0403}\\
    
\bottomrule
\end{tabular}
}
\end{minipage}\hfill
\end{table}




\subsubsection{Mesh extraction}
To further demonstrate the improvement in geometry quality, 
we applied methods used in ~\citep{turkulainen2024dnsplatter} 
to extract meshes from the input views of optimized 3DGS. 
The comparison results are presented  
in \tabref{tab:mushroom}. 
With the integration of FDS, the mesh quality is significantly enhanced compared to the baseline, featuring fewer floaters and more well-defined shapes.
 %
% Following the incorporation of FDS, the reconstruction 
% results exhibit fewer floaters and more well-defined 
% shapes in the meshes. 
% Visualized comparisons
% are provided in the supplementary material.

% \begin{figure}[t] \centering
%     \makebox[0.19\textwidth]{\scriptsize GT}
%     \makebox[0.19\textwidth]{\scriptsize 3DGS}
%     \makebox[0.19\textwidth]{\scriptsize 3DGS+FDS}
%     \makebox[0.19\textwidth]{\scriptsize 2DGS}
%     \makebox[0.19\textwidth]{\scriptsize 2DGS+FDS} \\

%     \includegraphics[width=0.19\textwidth]{figure/fig4_img/compare1/gt02.png}
%     \includegraphics[width=0.19\textwidth]{figure/fig4_img/compare1/baseline06.png}
%     \includegraphics[width=0.19\textwidth]{figure/fig4_img/compare1/baseline_fds05.png}
%     \includegraphics[width=0.19\textwidth]{figure/fig4_img/compare1/2dgs04.png}
%     \includegraphics[width=0.19\textwidth]{figure/fig4_img/compare1/2dgs_fds03.png} \\

%     \includegraphics[width=0.19\textwidth]{figure/fig4_img/compare2/gt00.png}
%     \includegraphics[width=0.19\textwidth]{figure/fig4_img/compare2/baseline02.png}
%     \includegraphics[width=0.19\textwidth]{figure/fig4_img/compare2/baseline_fds01.png}
%     \includegraphics[width=0.19\textwidth]{figure/fig4_img/compare2/2dgs04.png}
%     \includegraphics[width=0.19\textwidth]{figure/fig4_img/compare2/2dgs_fds03.png} \\
      
%     \includegraphics[width=0.19\textwidth]{figure/fig4_img/compare3/gt05.png}
%     \includegraphics[width=0.19\textwidth]{figure/fig4_img/compare3/3dgs03.png}
%     \includegraphics[width=0.19\textwidth]{figure/fig4_img/compare3/3dgs_fds04.png}
%     \includegraphics[width=0.19\textwidth]{figure/fig4_img/compare3/2dgs02.png}
%     \includegraphics[width=0.19\textwidth]{figure/fig4_img/compare3/2dgs_fds01.png} \\

%     \caption{\textbf{Qualitative comparison of extracted mesh 
%     on Mushroom and ScanNet datasets.}}
%     \label{fig:mesh}
% \end{figure}












\subsection{Ablation study}


\textbf{Ablation study on geometry priors:} 
To highlight the advantage of incorporating matching priors, 
we incorporated various types of priors generated by different 
models into 2DGS. These include a monocular depth estimation
model (Depth Anything v2)~\citep{yang2024depth}, a two-view depth estimation 
model (Unimatch)~\citep{xu2023unifying}, 
and a monocular normal estimation model (DSINE)~\citep{bae2024rethinking}.
We adapt the scale and shift-invariant loss in Midas~\citep{birkl2023midas} for
monocular depth supervision and L1 loss for two-view depth supervison.
%
We use Sea Raft~\citep{wang2025sea} as our default optical flow model.
%
The comparison results on Mushroom dataset 
are shown in ~\tabref{tab:analysis_prior}.
We observe that the normal prior provides accurate shape information, 
enhancing the geometric quality of the radiance field. 
%
% In contrast, the monocular depth prior slightly increases 
% the 'Abs Rel' due to its ambiguous scale and inaccurate depth ordering.
% Moreover, the performance of monocular depth estimation 
% in the sauna scene is particularly poor, 
% primarily due to the presence of numerous reflective 
% surfaces and textureless walls, which limits the accuracy of monocular depth estimation.
%
The multi-view depth prior, hindered by the limited feature overlap 
between input views, fails to offer reliable geometric 
information. We test average "Abs Rel" of multi-view depth prior
, and the result is 0.19, which performs worse than the "Abs Rel" results 
rendered by original 2DGS.
From the results, it can be seen that depth order information provided by monocular depth improves
reconstruction accuracy. Meanwhile, our FDS achieves the best performance among all the priors, 
and by integrating all
three components, we obtained the optimal results.
%
%
\begin{figure}[t] \centering
    \makebox[0.16\textwidth]{\scriptsize RF (16000 iters)}
    \makebox[0.16\textwidth]{\scriptsize RF* (20000 iters)}
    \makebox[0.16\textwidth]{\scriptsize RF (20000 iters)  }
    \makebox[0.16\textwidth]{\scriptsize PF (16000 iters)}
    \makebox[0.16\textwidth]{\scriptsize PF (20000 iters)}


    % \includegraphics[width=0.16\textwidth]{figure/fig5_img/compare1/16000.png}
    % \includegraphics[width=0.16\textwidth]{figure/fig5_img/compare1/20000_wo_flow_loss.png}
    % \includegraphics[width=0.16\textwidth]{figure/fig5_img/compare1/20000.png}
    % \includegraphics[width=0.16\textwidth]{figure/fig5_img/compare1/16000_prior.png}
    % \includegraphics[width=0.16\textwidth]{figure/fig5_img/compare1/20000_prior.png}\\

    % \includegraphics[width=0.16\textwidth]{figure/fig5_img/compare2/16000.png}
    % \includegraphics[width=0.16\textwidth]{figure/fig5_img/compare2/20000_wo_flow_loss.png}
    % \includegraphics[width=0.16\textwidth]{figure/fig5_img/compare2/20000.png}
    % \includegraphics[width=0.16\textwidth]{figure/fig5_img/compare2/16000_prior.png}
    % \includegraphics[width=0.16\textwidth]{figure/fig5_img/compare2/20000_prior.png}\\

    \includegraphics[width=0.16\textwidth]{figure/fig5_img/compare3/16000.png}
    \includegraphics[width=0.16\textwidth]{figure/fig5_img/compare3/20000_wo_flow_loss.png}
    \includegraphics[width=0.16\textwidth]{figure/fig5_img/compare3/20000.png}
    \includegraphics[width=0.16\textwidth]{figure/fig5_img/compare3/16000_prior.png}
    \includegraphics[width=0.16\textwidth]{figure/fig5_img/compare3/20000_prior.png}\\
    
    \includegraphics[width=0.16\textwidth]{figure/fig5_img/compare4/16000.png}
    \includegraphics[width=0.16\textwidth]{figure/fig5_img/compare4/20000_wo_flow_loss.png}
    \includegraphics[width=0.16\textwidth]{figure/fig5_img/compare4/20000.png}
    \includegraphics[width=0.16\textwidth]{figure/fig5_img/compare4/16000_prior.png}
    \includegraphics[width=0.16\textwidth]{figure/fig5_img/compare4/20000_prior.png}\\

    \includegraphics[width=0.30\textwidth]{figure/fig5_img/bar.png}

    \caption{\textbf{The error map of Radiance Flow and Prior Flow.} RF: Radiance Flow, PF: Prior Flow, * means that there is no FDS loss supervision during optimization.}
    \label{fig:error_map}
\end{figure}




\textbf{Ablation study on FDS: }
In this section, we present the design of our FDS 
method through an ablation study on the 
Mushroom dataset to validate its effectiveness.
%
The optional configurations of FDS are outlined in ~\tabref{tab:ablation_fds}.
Our base model is the 2DGS equipped with FDS,
and its results are shown 
in the first row. The goal of this analysis 
is to evaluate the impact 
of various strategies on FDS sampling and loss design.
%
We observe that when we 
replace $I_i$ in \eqref{equ:mflow} with $C_i$, 
as shown in the second row, the geometric quality 
of 2DGS deteriorates. Using $I_i$ instead of $C_i$ 
help us to remove the floaters in $\bm{C^s}$, which are also 
remained in $\bm{C^i}$.
We also experiment with modifying the FDS loss. For example, 
in the third row, we use the neighbor 
input view as the sampling view, and replace the 
render result of neighbor view with ground truth image of its input view.
%
Due to the significant movement between images, the Prior Flow fails to accurately 
match the pixel between them, leading to a further degradation in geometric quality.
%
Finally, we attempt to fix the sampling view 
and found that this severely damaged the geometric quality, 
indicating that random sampling is essential for the stability 
of the mean error in the Prior flow.



\begin{table}[t] \centering

\begin{minipage}[t]{1.0\linewidth}
        \captionof{table}{\textbf{Ablation study on FDS strategies.}}
        \label{tab:ablation_fds}
        \resizebox{\textwidth}{!}{
\begin{tabular}{c|c|c|c|c|c|c|c}
    \hline
    \multicolumn{2}{c|}{$\mathcal{M}_{\theta}(X, \bm{C^s})$} & \multicolumn{3}{c|}{Loss} & \multicolumn{3}{c}{Metric}  \\
    \hline
    $X=C^i$ & $X=I^i$  & Input view & Sampled view     & Fixed Sampled view        & Abs Rel $\downarrow$ & F-score $\uparrow$ & NC $\uparrow$ \\
    \hline
    & \ding{51} &     &\ding{51}    &    &    \textbf{0.0561}        &  \textbf{0.6974}         & \textbf{0.8151}\\
    \hline
     \ding{51} &           &     &\ding{51}    &    &    0.0839        &  0.6242         &0.8030\\
     &  \ding{51} &   \ding{51}  &    &    &    0.0877       & 0.6091        & 0.7614 \\
      &  \ding{51} &    &    & \ding{51}    &    0.0724           & 0.6312          & 0.8015 \\
\bottomrule
\end{tabular}
}
\end{minipage}
\end{table}




\begin{figure}[htbp] \centering
    \makebox[0.22\textwidth]{}
    \makebox[0.22\textwidth]{}
    \makebox[0.22\textwidth]{}
    \makebox[0.22\textwidth]{}
    \\

    \includegraphics[width=0.22\textwidth]{figure/fig6_img/l1/rgb/frame00096.png}
    \includegraphics[width=0.22\textwidth]{figure/fig6_img/l1/render_rgb/frame00096.png}
    \includegraphics[width=0.22\textwidth]{figure/fig6_img/l1/render_depth/frame00096.png}
    \includegraphics[width=0.22\textwidth]{figure/fig6_img/l1/depth/frame00096.png}

    % \includegraphics[width=0.22\textwidth]{figure/fig6_img/l2/rgb/frame00112.png}
    % \includegraphics[width=0.22\textwidth]{figure/fig6_img/l2/render_rgb/frame00112.png}
    % \includegraphics[width=0.22\textwidth]{figure/fig6_img/l2/render_depth/frame00112.png}
    % \includegraphics[width=0.22\textwidth]{figure/fig6_img/l2/depth/frame00112.png}

    \caption{\textbf{Limitation of FDS.} }
    \label{fig:limitation}
\end{figure}


% \begin{figure}[t] \centering
%     \makebox[0.48\textwidth]{}
%     \makebox[0.48\textwidth]{}
%     \\
%     \includegraphics[width=0.48\textwidth]{figure/loss_Ignatius.pdf}
%     \includegraphics[width=0.48\textwidth]{figure/loss_family.pdf}
%     \caption{\textbf{Comparison the photometric error of Radiance Flow and Prior Flow:} 
%     We add FDS method after 2k iteration during training.
%     The results show
%     that:  1) The Prior Flow is more precise and 
%     robust than Radiance Flow during the radiance 
%     optimization; 2) After adding the FDS loss 
%     which utilize Prior 
%     flow to supervise the Radiance Flow at 2k iterations, 
%     both flow are more accurate, which lead to
%     a mutually reinforcing effects.(TODO fix it)} 
%     \label{fig:flowcompare}
% \end{figure}






\textbf{Interpretive Experiments: }
To demonstrate the mutual refinement of two flows in our FDS, 
For each view, we sample the unobserved 
views multiple times to compute the mean error 
of both Radiance Flow and Prior Flow. 
We use Raft~\citep{teed2020raft} as our default optical flow model
for visualization.
The ground truth flow is calculated based on 
~\eref{equ:flow_pose} and ~\eref{equ:flow} 
utilizing ground truth depth in dataset.
We introduce our FDS loss after 16000 iterations during 
optimization of 2DGS.
The error maps are shown in ~\figref{fig:error_map}.
Our analysis reveals that Radiance Flow tends to 
exhibit significant geometric errors, 
whereas Prior Flow can more accurately estimate the geometry,
effectively disregarding errors introduced by floating Gaussian points. 

%





\subsection{Limitation and further work}

Firstly, our FDS faces challenges in scenes with 
significant lighting variations between different 
views, as shown in the lamp of first row in ~\figref{fig:limitation}. 
%
Incorporating exposure compensation into FDS could help address this issue. 
%
 Additionally, our method struggles with 
 reflective surfaces and motion blur,
 leading to incorrect matching. 
 %
 In the future, we plan to explore the potential 
 of FDS in monocular video reconstruction tasks, 
 using only a single input image at each time step.
 


\section{Conclusions}
In this paper, we propose Flow Distillation Sampling (FDS), which
leverages the matching prior between input views and 
sampled unobserved views from the pretrained optical flow model, to improve the geometry quality
of Gaussian radiance field. 
Our method can be applied to different approaches (3DGS and 2DGS) to enhance the geometric rendering quality of the corresponding neural radiance fields.
We apply our method to the 3DGS-based framework, 
and the geometry is enhanced on the Mushroom, ScanNet, and Replica datasets.

\section*{Acknowledgements} This work was supported by 
National Key R\&D Program of China (2023YFB3209702), 
the National Natural Science Foundation of 
China (62441204, 62472213), and Gusu 
Innovation \& Entrepreneurship Leading Talents Program (ZXL2024361)
\section{Conclusion}
We introduce a novel approach, \algo, to reduce human feedback requirements in preference-based reinforcement learning by leveraging vision-language models. While VLMs encode rich world knowledge, their direct application as reward models is hindered by alignment issues and noisy predictions. To address this, we develop a synergistic framework where limited human feedback is used to adapt VLMs, improving their reliability in preference labeling. Further, we incorporate a selective sampling strategy to mitigate noise and prioritize informative human annotations.

Our experiments demonstrate that this method significantly improves feedback efficiency, achieving comparable or superior task performance with up to 50\% fewer human annotations. Moreover, we show that an adapted VLM can generalize across similar tasks, further reducing the need for new human feedback by 75\%. These results highlight the potential of integrating VLMs into preference-based RL, offering a scalable solution to reducing human supervision while maintaining high task success rates. 

\section*{Impact Statement}
This work advances embodied AI by significantly reducing the human feedback required for training agents. This reduction is particularly valuable in robotic applications where obtaining human demonstrations and feedback is challenging or impractical, such as assistive robotic arms for individuals with mobility impairments. By minimizing the feedback requirements, our approach enables users to more efficiently customize and teach new skills to robotic agents based on their specific needs and preferences. The broader impact of this work extends to healthcare, assistive technology, and human-robot interaction. One possible risk is that the bias from human feedback can propagate to the VLM and subsequently to the policy. This can be mitigated by personalization of agents in case of household application or standardization of feedback for industrial applications. 

%%%%%%%%% REFERENCES
{\small
\bibliographystyle{ieee_fullname}
\bibliography{egbib}
\nocite{cazenavette2024fakeinversion}
}

% supplementary
\clearpage
\appendix
\clearpage
\newcommand{\nocontentsline}[3]{}
\newcommand{\tocless}[2]{\bgroup\let\addcontentsline=\nocontentsline#1{#2}\egroup}

\newcommand{\Appendix}[1]{
  \refstepcounter{section}
  \section*{Appendix \thesection:\hspace*{1.5ex} #1}
  \addcontentsline{toc}{section}{Appendix \thesection}
}
\newcommand{\SubAppendix}[1]{\tocless\subsection{#1}}
% \setcounter{page}{1}
% \setcounter{section}{0}
% \setcounter{figure}{0}
% \setcounter{table}{0}
\maketitlesupplementary
\appendix


\tableofcontents
\addtocontents{toc}{}
% This supplemental document contains seven sections:
% Section \ref{sec:add_details} shows more implementation details of our MotionCanvas;
% Section \ref{sec:ui} presents more details of user interface;
% Section \ref{sec:transform} provides more analysis on the essentiality of camera-aware and camera-object-aware transformations; 
% Section \ref{sec:ar} presents more details of our MotionCanvas$_\text{AR}$; 
% Section \ref{sec:user} shows more details of user study; 
% Section \ref{sec:add_analysis} presents additional analysis; 
% and Section \ref{sec:limitations} shows the limitations of our method.

Please check our project page \url{https://motion-canvas25.github.io/} for video results.
% In addition to this supplementary document, \textbf{we also provided a local HTML webpage} showing video comparisons.



\section{Additional Details}
\label{sec:add_details}
Our MotionCanvas model is trained using 16 nodes of NVIDIA H100 (80GB) GPUs (8 GPUs on each node). On a single H100 GPU, generating a 32-frame video clip at a resolution of 352×640 with 50 denoising steps takes approximately 32 seconds. During training, we exclusively optimize the DiT transformer blocks and the additional Linear/MLP layers introduced for bounding box conditioning and DCT coefficient tokenization, while keeping all other modules frozen.


%%%%%%%%%%%%%%%%%%%%%%%%%%%%%%%%%%%
\begin{figure*}[htbp]
    \centering
    \includegraphics[width=0.85\linewidth]{Graphics/supp/User_interface.pdf} % Replace with your image file
    \caption{A sample of the designed user interface for our MotionCanvas.}
    \label{fig:ui}
\end{figure*}
%%%%%%%%%%%%%%%%%%%%%%%%%%%%%%%%%%%
\section{User Interface}
\label{sec:ui}
We designed a sample user interface to provide flexible and interactive control over camera motion, object global and local motion, and their timing. An example of the user interface is illustrated in Fig.~\ref{fig:ui}.

\textbf{Specifying Camera Motion.} To facilitate a user-friendly approach for defining camera motion trajectories, the interface allows users to combine $M$ base motion patterns with configurable parameters such as direction (positive or negative) and speed (absolute value), as demonstrated in the ``Camera Motion Control" panel in Fig.~\ref{fig:ui}. Specifically, the base motion patterns include:

\begin{itemize}[left=2em]
    \item Horizontal (Trucking) left/right
    \item Panning left/right
    \item Dolly in/out
    \item Vertical (Pedestal) up/down
    \item Tilt up/down
    \item Roll clockwise/anti-clockwise
    \item Zoom in/out
    \item Orbit left/right (adjustable radius)
    \item Circle clockwise/anti-clockwise
    \item Static
\end{itemize}

The sign (positive or negative) and the absolute value of the number associated with each motion pattern define the corresponding camera poses relative to the zero pose at the first frame (i.e., translation and rotation vectors).

\textbf{Specifying Scene-aware Object Global Motion.} To enable user control over object global motion trajectories, we provide an interactive canvas (see Fig.~\ref{fig:ui}, ``Object Global Motion Control") where users can draw starting and ending bounding boxes, as well as optional intermediate points. A smooth bounding box trajectory can be obtained by applying Catmull-Rom spline interpolation. For each bounding box, users can optionally specify a reference depth point on the image. Additionally, users can decide whether to directly use this scene-space bounding box sequence as a condition. Utilizing the scene-space bounding box is particularly more effective for creating cinematic effects such as ``follow shots" or ``dolly shots".

For standard scene-aware object global motion control, the scene-space bounding boxes are assigned depth values, as described in Section~3.2 of the main paper. The bounding box sequence is then converted into screen space using the proposed Motion Signal Translation module.

\textbf{Specifying scene-aware object local motion.} We also provide a dedicated canvas for controlling object local motion (see Fig.~\ref{fig:ui}, ``Object Local Motion Control"). Users can draw any number of point trajectories, which are assigned depth values as outlined in Section 3.2 of the main paper. Similar to bounding boxes, these point trajectories are transformed into screen space based on the camera motion and object global motion. The object global motion transformation takes effect only when the starting point of the trajectory lies within the object's semantic region. We then transform the object's local motion point trajectories by maintaining their relative positions with respect to the underlying bounding box.

Additionally, our user interface includes a ``Preview Window" that allows users to visualize the generated videos, as well as the bounding box sequences and point trajectories in both scene space and screen space.



\begin{figure}[htbp]
    \centering
    \includegraphics[width=1\linewidth]{Graphics/supp/Transform_supp.pdf} % Replace with your image file
    \caption{Illustration of camera-aware transformation and camera-object-aware transformation. In preview videos, dash-line bounding boxes represent the scene-space inputs, while the solid ones with the same color denote the transformed screen-space motion signals. Similarly, point trajectories with white trace indicate scene-space user motion design, while colored ones represent transformed signals. Better investigation in supplementary videos.}
    \label{fig:transform}
\end{figure}
\section{Essentiality of Camera-aware and Camera-object-aware Transformations}
\label{sec:transform}
Drawing inspiration from classical graphics imaging techniques, we introduce a Motion Signal Translation module to convert scene-space user-defined motion intents into screen-space motion signals. This enables joint control of camera and object motions in a 3D-aware manner for image-to-video generation. The Motion Signal Translation module incorporates a hierarchical transformation framework that accounts for the intertwining nature of camera and object motions. To illustrate the effectiveness of these transformations, we provide visual comparisons highlighting both camera-aware and camera-object-aware transformations.

\textbf{Camera-aware Transformation for Object Global or Local Motion.}
First, we present the camera-aware transformation for object global motion control in Fig.~\ref{fig:transform}(top). In the preview video (last frame), the dashed-line bounding boxes represent the scene-space inputs specified by the user, while the solid bounding boxes of the same color denote the corresponding transformed screen-space motion signals.

In this example, people are running forward on the road as controlled by the user’s input (bounding boxes). When a trucking-left camera motion is applied, all the people should naturally move to the right on the screen. Using our camera-aware transformation, the screen-space object bounding boxes are correctly calibrated, ensuring that the resulting animation appears accurate and more natural (refer to the supplementary webpage: ``Additional Analysis -- Essentiality of Camera-aware and Camera-object-aware Transformations").

A similar conclusion holds for the camera-aware transformation applied to object local motion, as shown in Fig.~\ref{fig:transform}(middle).

\textbf{Camera-object-aware Transformation for Object Local Motion.}
Camera-object-aware transformation implies that translating scene-space point trajectory specifying the object local motion to screen-space signals must take into account both the camera motion and object global motion.  For example, as illustrated in Fig.~\ref{fig:transform}(bottom), the trajectory of an object’s local motion, such as ``putting down hands" must account for both the body’s movement and the camera’s pedestal-up motion. As demonstrated in ``Additional Analysis -- Essentiality of Camera-aware and Camera-object-aware Transformations", our transformation produces a life-like and accurate video, whereas the variant without these transformations results in unnatural and incorrect motion.

\begin{figure}[t]
    \centering
    \includegraphics[width=1\linewidth]{Graphics/supp/AR_recomputation.pdf} % Replace with your image file
    \caption{Illustration of the recomputation process for input motion conditions in our MotionCanvas$_\text{AR}$ during inference.}
    \label{fig:back_trace}
\end{figure}

\section{More Details of MotionCanvas$_\text{AR}$}
\label{sec:ar}
To enhance support for long video generation and address motion discontinuities, we introduce a 16-frame conditioned 64-frame MotionCanvas$_\text{AR}$, designed to generate videos in an auto-regressive manner. This model builds on our 32-frame motion-conditioned I2V model (single-frame-conditioned) and is fine-tuned for an additional 120K iterations, while retaining the same training configurations.

To further refine the input motion signals and better align them with the training setup, we recompute the screen-space motion signals by integrating the user’s motion intent with back-traced motions, as illustrated in Fig.~\ref{fig:back_trace}. This method ensures smoother, more consistent motion generation throughout the video.


\section{User Study}
\label{sec:user}
The designed user study interface is shown in Figure~\ref{fig:user_study_screen_shot}. We collect 15 representative image cases from the Internet and design various motion controls. We then generate the video clips results by executing the official code~\cite{wu2025draganything,niu2025mofa}. For the user study, we use these video results produced by shuffled methods based on the same set of input conditions. In addition, we standardize all the produced results by encoding FPS$=$8 for 14 generated frames, yielding $\sim$2-second videos for each method. This process ensures a fair comparison.
%% TODO: describe how we convert into points and into those methods

The user study is expected to be completed with 7--15 minutes (15 cases $\times$ 3 sub-questions $\times$ 10--20 seconds for each judgement). To remove the impact of random selection, we filter out those comparison results completed within three minutes. For each participant, the user study interface shows 15 groups of video comparisons, and the participant is instructed to evaluate the videos for three times, \ie, answering the following questions respectively: (i) ``Which one shows the best motion adherence?"; (ii) ``Which one has the best motion/dynamic quality?"; (iii) ``which one shows the best frame fidelity?". Finally, we received 35 valid responses from the participants.

\begin{figure}[t]
    \centering
    \includegraphics[width=1.0\linewidth]{Graphics/supp/camera_motion_icons.pdf} % Replace with your image file
    \caption{Legend of camera motions used in the main paper.}
    \label{fig:legend}
\end{figure}

\begin{figure*}[htbp]
    \centering
    \includegraphics[width=0.8\linewidth]{Graphics/supp/user_study_supp_figure_whole.pdf} % Replace with your image file
    \caption{Designed user study interface. Each participant is required to evaluate 15 video comparisons and respond to three corresponding sub-questions for each comparison. Only one video is shown here due to the page limit.}
    \label{fig:user_study_screen_shot}
\end{figure*}

\section{Additional Analysis}
\label{sec:add_analysis}
\subsection{Effect of Point Track Density on Camera Motion Control}

We investigate the effect of point track density on camera motion control by specifying orbit right camera motion with different numbers of 2D point tracks. The visual comparison result is shown in the supplementary webpage `Additional Analysis -- Effect of Point Track Density on Camera Motion Control'. As can be seen, the motion is underspecified with significant ambiguity when providing low-density tracks. Hence, the generated camera motion does not follow the control and tends to be trucking left. By providing higher-density tracks, the generated motion can better adhere to the orbit camera motion.


\subsection{Effect of Text Prompt}
We use simple text descriptions throughout all experiments. To further investigate the effect of text prompts on our MotionCanvas, we show the visual comparison of gradually more detailed text prompts in the supplementary webpage `Additional Analysis-- Effect of Text Prompt.'. It demonstrate that text prompt does not have a significant effect on the camera motion control. However, it can generate diverse dynamics like `raining' and `turning around'.


\section{Limitations and Future Work}
\label{sec:limitations}


Our work introduces a novel framework for enhancing I2V with holistic motion controls, enabling cinematic shot design from a single image. While this paper made substantial progress on toward this important application, challenges remain, opening up opportunities for future research. 

First, our use of a video diffusion model (VDM) for the video synthesis module, while enabling high-quality video generation, results in relatively slow inference times (approximately 35 seconds for a 2-second video). This computational cost, typical of modern VDMs, currently limits real-time applications. Exploring alternative, more efficient generative models is a promising direction for future work. 

Second, our current method approximates object local motion by assuming each object lies on a frontal parallel depth plane. Although effective for most natural scenes as the depth variation within the object are typically small compared to object's distance to the camera, this pseudo-3D approximation may not be suitable for extreme close-up or macro shots where depth variations within the object are significant. In future work, it will be interesting to investigate integrating more explicit 3D formulation for handling such scenarios. 

Finally, our system currently does not explicitly constrain the harmonization between motion design and textual prompts. On one hand, this offers the flexibility for users to leverage both control modalities jointly to explore creative variations. On the other hand, this leaves the possibility of conflicting motion signals between the modalities. For example, in Fig. 8 (top row) in the main paper, while the text prompt indicates the cat to be waiting instead of moving, when motion design explicitly control the cat to moves forward in the later part of the videos (note that such global object control was used when generated those results but was not illustrated in the figure to avoid cluttered visualization), such motion control overrides the ones hinted in the textual prompts. Explicit motion-aware prompt harmonization can be a fruitful research direction to extend our work.  

\section{Camera Motion Legend}
\label{sec:legend}
The legend of camera motions used in the main paper is presented in Fig.~\ref{fig:legend}.



\end{document}
