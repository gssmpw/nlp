% WACV 2025 Paper Template
% based on the WACV 2024 template, which is
% based on the CVPR 2023 template (https://media.icml.cc/Conferences/CVPR2023/cvpr2023-author_kit-v1_1-1.zip) with 2-track changes from the WACV 2023 template (https://github.com/wacv-pcs/WACV-2023-Author-Kit)
% based on the CVPR template provided by Ming-Ming Cheng (https://github.com/MCG-NKU/CVPR_Template)
% modified and extended by Stefan Roth (stefan.roth@NOSPAMtu-darmstadt.de)

\documentclass[10pt,twocolumn,letterpaper]{article}

%%%%%%%%% PAPER TYPE  - PLEASE UPDATE FOR FINAL VERSION
% \usepackage[review,algorithms]{wacv}      % To produce the REVIEW version for the algorithms track
% \usepackage[review,applications]{wacv}      % To produce the REVIEW version for the applications track
\usepackage{wacv}              % To produce the CAMERA-READY version
%\usepackage[pagenumbers]{wacv} % To force page numbers, e.g. for an arXiv version
\usepackage[accsupp]{axessibility}
% Include other packages here, before hyperref.
\usepackage{graphicx}
\usepackage{amsmath}
\usepackage{amssymb}
\usepackage{booktabs}
\usepackage{arydshln}

\usepackage{float}
\usepackage{kotex}
\usepackage{multirow}
\usepackage{multicol}
\usepackage{colortbl}
\usepackage{subcaption}
\usepackage{makecell}
\usepackage{float}

\usepackage[sectionbib]{chapterbib}
\usepackage{diagbox}

\usepackage{enumitem}


%%%%!!!!



\newcommand{\tft}{\textbf}
\newcommand{\et}{\textit{et al.}}
%\newcommand{\etal}{\textit{et al.}}
\newcommand{\ccn}[1]{\textcolor{red}{#1}}
\newcommand{\tb}[1]{\textbf{#1}}
\newcommand{\cred}{\color{red}}
\newcommand{\cblue}{\color{blue}}
\definecolor{light-gray}{gray}{0.82}
\renewcommand\ttdefault{cmvtt} % selects CM typewriter proportional font


% for pseudocode
\usepackage{listings}
\usepackage{algorithm}
\usepackage[T1]{fontenc}
\usepackage[utf8]{inputenc} % allow utf-8 input
\definecolor{codeblue}{rgb}{0.25,0.5,0.5}
\definecolor{codekw}{rgb}{0.85, 0.18, 0.50}
\lstset{
  backgroundcolor=\color{white},
  basicstyle=\fontsize{8pt}{8pt}\ttfamily\selectfont,
  columns=fullflexible,
  breaklines=true,
  captionpos=t,
  commentstyle=\fontsize{3.5pt}{3.5pt}\color{codeblue},
  keywordstyle=\fontsize{7.5pt}{7.5pt}\color{codekw},
  % numbers=left,
  stepnumber=1,    
  firstnumber=1,
  numberfirstline=true,
  escapechar=|,
}


% conditional color formatting for table cells
\usepackage{highlight_table}
% For a 2 color palette
% \gradientcell{cell_val}{min_val}{max_val}{colorlow}{colorhigh}{opacity} 
\newcommand{\g}[1]{\gradientcell{#1}{80}{100}{white}{gray}{70}}
% For a 3 color/divergent palette
% \gradientcelld{cell_val}{min_val}{mid_val}{max_val}{colorlow}{colormid}{colorhigh}{opacity}
% \newcommand{\g}[1]{\gradientcelld{#1}{40}{85}{100}{red}{white}{green}{70}}


% It is strongly recommended to use hyperref, especially for the review version.
% hyperref with option pagebackref eases the reviewers' job.
% Please disable hyperref *only* if you encounter grave issues, e.g. with the
% file validation for the camera-ready version.
%
% If you comment hyperref and then uncomment it, you should delete
% ReviewTempalte.aux before re-running LaTeX.
% (Or just hit 'q' on the first LaTeX run, let it finish, and you
%  should be clear).
\usepackage[pagebackref,breaklinks,colorlinks]{hyperref}


% Support for easy cross-referencing
\usepackage[capitalize]{cleveref}
\crefname{section}{Sec.}{Secs.}
\Crefname{section}{Section}{Sections}
\Crefname{table}{Table}{Tables}
\crefname{table}{Tab.}{Tabs.}


%%%%%%%%% PAPER ID  - PLEASE UPDATE
\def\wacvPaperID{694} % *** Enter the WACV Paper ID here
\def\confName{WACV}
\def\confYear{2025}

\begin{document}

%%%%%%%%% TITLE - PLEASE UPDATE
\title{SFLD: Reducing the content bias for AI-generated Image Detection}
\author{\textbf{Seoyeon Gye\thanks{Equal contribution.}\hspace{0.6cm} Junwon Ko\footnotemark[1]\hspace{0.6cm} Hyounguk Shon\footnotemark[1]\hspace{0.6cm} Minchan Kwon\hspace{0.7cm} Junmo Kim}\vspace{0.3cm} \\
School of Electrical Engineering, KAIST, South Korea\\
{\tt\small \{sawyun, kojunewon, hyounguk.shon, kmc0207, junmo.kim\}@kaist.ac.kr}}
\maketitle

%%%%%%%%% ABSTRACT
\begin{abstract}
    Identifying AI-generated content is critical for the safe and ethical use of generative AI.
    Recent research has focused on developing detectors that generalize to unknown generators, with popular methods relying either on high-level features or low-level fingerprints.
    However, these methods have clear limitations: biased towards unseen content, or vulnerable to common image degradations, such as JPEG compression.
    To address these issues, we propose a novel approach, SFLD, which incorporates PatchShuffle to integrate high-level semantic and low-level textural information. SFLD applies PatchShuffle at multiple levels, improving robustness and generalization across various generative models.
    Additionally, current benchmarks face challenges such as low image quality, insufficient content preservation, and limited class diversity. In response, we introduce TwinSynths, a new benchmark generation methodology that constructs visually near-identical pairs of real and synthetic images to ensure high quality and content preservation.
    Our extensive experiments and analysis show that SFLD outperforms existing methods on detecting a wide variety of fake images sourced from GANs, diffusion models, and TwinSynths, demonstrating the state-of-the-art performance and generalization capabilities to novel generative models.
    The TwinSynths dataset is publicly available at \textnormal{\url{https://huggingface.co/datasets/koooooooook/TwinSynths}}.
\end{abstract}


%%%%%%%%% BODY TEXT
\section{Introduction}
Multimodal large language models (MLLMs) enhanced visual understanding and reasoning by pre-training on large-scale multimodal datasets with comprehensive visual descriptions that integrats textual and visual knowledge  \cite{liu2024visual,yao2024minicpmvgpt4vlevelmllm,li2023blip,bai2023qwen,liu2024multimodal}. 
These models achieve strong performance across various vision-language tasks, 
such as visual question answering \cite{jin2024rjua}, multimodal reasoning \cite{zhang2024large,jiang2024killing,yan2024list}, multimodal recognition \cite{shenoy2024lumos,wu2024visual}, 
personalized multimodality \cite{wu2024personalized}, and document intelligence \cite{jin2024rjua, shenoy2024lumos}.
However, adapting pre-trained MLLMs to downstream tasks via instruction-tuning  \cite{wu2024commit,li2024vision,li2023fine,panagopoulou2023x,liu2024multimodal} presents a critical challenge of visual forgetting. 
Unlike pre-training, where models receive rich visual-text alignment, instruction-tuning is often text-driven with limited direct visual supervision. 
This shift in training focus leads to the degradation of pre-trained visual encoding \cite{zhou2024mitigating,niu2024text,wu2024commit,ko2023large}, 
negatively impacting model generalizability across downstream tasks that require strong visual knowledge \cite{bai2024hallucination,huang2024visual}. 
Addressing this challenge is essential for ensuring MLLMs retain their visual capabilities while aligning with new tasks efficiently.

While several approaches have attempted to mitigate catastrophic forgetting in neural networks through direct fine-tuning and continual learning methods \cite{shi2024continual, wu2024continual, zhu2024model, zheng2024beyond}, 
these methods often overlook the unique challenge of preserving visual knowledge in multimodal large language models (MLLMs). 
Directly fine-tuning MLLMs on new tasks often leads to overfitting to textual instructions while inadvertently suppressing visual representations \citep{zhai2023investigating}.
Existing continual learning strategies such as regularization and replay methods tend to focus on retaining language-based knowledge, 
neglecting the critical trade-off between compressing visual representations and aligning them with task-specific instructions \cite{zhou2024mitigating, niu2024text, wu2024commit, ko2023large}, 
leading to the degradation of pre-trained visual knowledge.
Task-orthogonal gradient descent techniques have shown promise in disentangling gradients for multi-task optimization.
However, their practical application in MLLMs poses unique challenges. 
MLLMs are pre-trained on vast and heterogeneous multimodal datasets \cite{liu2024visual, li2023blip, bai2023qwen}, 
where it is challenging to precisely isolate task-specific gradients, 
causing the components critical for visual understanding to become entangled with other features.
\begin{figure}[btp]
    \centering
    \subfigure[LLaVA on OKVQA]{%
        \includegraphics[width=0.48\linewidth]{contents/figure/imgToken-erank--LLaVA-OKVQA.pdf} %
        \label{fig:erank-token1}
    }
    \hfill
    \subfigure[LLaVA on POPE]{%
        \includegraphics[width=0.48\linewidth]{contents/figure/imgToken-erank--LLaVA-POPE.pdf} %
        \label{fig:erank-token2}
    }
    
    
    \subfigure[MiniCPM on PathVQA]{%
        \includegraphics[width=0.48\linewidth]{contents/figure/imgToken-erank--MiniCPM-PathVQA.pdf} %
        \label{fig:erank-token3}
    }
    \hfill
    \subfigure[MiniCPM on POPE]{%
        \includegraphics[width=0.48\linewidth]{contents/figure/imgToken-erank--MiniCPM-POPE.pdf} %
        \label{fig:erank-token4}
    }

    
    \caption{The top-10 image tokens with the highest effective ranks on OKVQA and POPE encoded by LLaVA, and PathVQA and POPE encoded by MiniCPM. 
    We compare pretrained, finetuned, and MDGD-finetuned models. 
    Effective rank \cite{wei2024diff} quantifies representation richness, and we novelly use it to analyze visual degradation in instruction-tuned MLLMs. Results show that MDGD preserves higher effective rank, mitigating visual forgetting.}
    \label{fig:intro}
    \vspace{-1em}
\end{figure}


To gain a fundamental view of the challenge of visual knowledge forgetting in MLLM instruction tuning, 
we adopt an information bottleneck (IB) perspective that characterizes the trade-off between retaining input information and ensuring output predictiveness \cite{tishby2000information}.
To investigate the degradation of crucial pre-trained visual knowledge, we introduce a novel perspective that leverages effective rank to quantify the richness of the encoded visual representation from MLLMs.
Specifically, we illustrate the visual forgetting problem in Figure~\ref{fig:intro}, where we observe a consistent effective rank reduction problem caused by MLLM instruction tuning.
Based on this view, we propose a modality-decoupled gradient descent (MDGD) method, which disentangles the optimization of visual understanding from task-specific alignment,
MDGD regulates gradient updates to maintain the effective rank of visual representations compared with pre-trained MLLMs,
while mitigating the over-compression effects described by the information bottleneck. 
Intuitively, visual forgetting occurs due to the shift from rich multimodal pre-training to instruction-tuning, 
where text-based supervision dominates without direct visual supervision. 
By explicitly decoupling the task-specific alignment with visual representation learning, MDGD preserves expressive and robust visual features. 
To further improve efficiency in instruction-tuning, we introduce a memory-efficient fine-tuning strategy using gradient masking, 
which selectively updates a subset of model parameters for parameter-efficient fine-tuning (PEFT).
This approach reduces computational overhead while ensuring that crucial pre-trained visual representations are retained.


We summarize our contributions as follows: 
\begin{itemize}
\item We analyze the visual knowledge forgetting problem in MLLM instruction tuning and frame the problem through the lens of effective rank and information bottleneck theory.
\item We propose MDGD, which decouples visual optimization from task-specific alignment to preserve visual representations and introduces a PEFT variant MDGD-GM to reduce computational overhead through gradient masking. 
\item We conduct comprehensive experiments on various MLLMs and downstream tasks, demonstrating that MDGD effectively mitigates visual forgetting while enabling strong adaptation to new tasks.
\end{itemize}





\section{Method}
\label{sec:method}


\begin{figure}[t]
     \centering
     \includegraphics[width=1.0\linewidth]{figs/method.pdf}
     \caption{Architecture of the proposed fake image detector (SFLD). $z_{s_i}$ refers to the logit score generated from an input image processed via $s_i\text{×}s_i$ patch size. $\Sigma$ indicates weighted sum.}
     \label{fig:method-architecture}
\end{figure}

\subsection{Patch Shuffling Fake Detection} 
\label{subsec:SFLD}
\textbf{Backbone.}
We utilize the visual encoder of CLIP ViT-L/14\cite{dosovitskiy2021an, CLIP} to leverage the pre-trained feature space. This choice is based on Ojha \etal\cite{ojha2023towards}, which showed that it outperforms other models such as CLIP:ResNet-50, ImageNet:ResNet-50, and ImageNet:ViT-B/16 in distinguishing real from fake images. The results indicated that both the architecture and the pre-training data are crucial. 
Based on this insight, we chose the ViT model for our backbone. 
As shown in \cref{fig:method-architecture}, we extract CLIP features and train a fully connected layer to classify real and fake images.


\textbf{PatchShuffle.}
To effectively integrate both semantic and textural features, PatchShuffle disrupts the global structure of an image while preserving local features.
In the PatchShuffle process, the input images are divided into non-overlapping patches of size $s \times s$ and then randomly shuffled. This operation produces a new shuffled image $x_s$.

For a given $s$, the logit score of the shuffled image is, 
\begin{equation}
    z_{s} = \psi(f(x_s)) \,,
\end{equation}
where $f( \cdot )$ represents a pre-trained CLIP encoder and $\psi(\cdot)$ is a single fully connected layer appended to $f$.

There are classifiers for each patch size of shuffled images to leverage local structure information hierarchically within the image.
We selected patch sizes of 28, 56, and 224 for the proposed SFLD.
As shown in \cref{fig:method-architecture}, $s_0$ is 224, $s_1$ is 56 and $s_2$ is 28.
These configurations are studied in detail in \cref{subsec:patchsetting}.
For each patch size $s_j$, the classifier $\psi_{s_j}$ is trained independently.
Notably, UnivFD takes a center-cropped 224×224 image as input to the CLIP encoder.
Therefore, when using a patch size of 224 in PatchShuffle, it effectively corresponds to the same setting as UnivFD\cite{ojha2023towards}.

We employ binary cross-entropy loss for each classifier:
\small\begin{equation}
    \mathcal{L} = -\frac{1}{N} \sum_{i=1}^{N} \left[ y_i \log \sigma(z_{s_j}) + (1 - y_i) \log \left(1 - \sigma(z_{s_j}) \right) \right] 
\end{equation}\normalsize
where $N$ is the number of data and $y_i \in \{0, 1\}$ is the label whether an input $x_i$ is real ($y_i = 0$) or fake ($y_i = 1$).

\textbf{SFLD.} SFLD combines multiple classifiers trained on shuffled images with different patch sizes.
By varying the patch size, SFLD incorporates models that focus on various levels of structural features, ranging from fine-grained local details to more global patterns.

During testing, $N_{views}=10$ shuffled views are generated for each patch size. The logits from these views are averaged and processed by the corresponding classifier. The final probability $P_{SFLD}(y|x)$ is computed by averaging the logits across patch sizes and applying the sigmoid function:
\small\begin{equation}
    P_{\text{SFLD}}(y|x) = \sigma\left(\frac{1}{k} \sum_{j=1}^{k} \psi_{s_j}\left(\frac{1}{N_{\text{views}}} \sum_{i=1}^{N_{\text{views}}} f(x_{s_j}^i)\right)\right) \,,
\end{equation}\normalsize
where $k$ is the number of patch sizes used in the ensemble (e.g., $k=3$ in our configuration).

Binary classification is done using a threshold of 0.5 on $P_{SFLD}$. Although the fusion method is simple and not tuned for each test class, its simplicity enables strong generalization across diverse fake image sources. By combining classifiers trained on different patch sizes, SFLD achieves a robust and general detection performance. \cref{alg:pseudocode} shows the full workflow of SFLD, especially the fusion of multiple classifiers during inference.

\subsection{TwinSynths} 
In \cref{sec:intro}, we pointed out three shortcomings in the previous benchmarks: low image quality, lack of content preservation, and limited class diversity.
This issue must be addressed to allow a comprehensive comparison of detectors.
Therefore, we propose a novel dataset creation methodology and \emph{TwinSynths} benchmark, consisting of GAN- and diffusion-based generated images that are  paired with visually-identical real counterparts.
To create a practical benchmark for evaluating generated image detectors, it is essential to ensure the generation of high-quality images that preserve the original content.
To achieve this, the image generation process should ideally sample a distribution that closely resembles a real distribution.
From this perspective, the image generation or sampling process can be interpreted as effectively fitting the generator to a single real image.
Through this approach, we construct image pairs that preserve the content of the images while reflecting the architectural traits of the generative models.
Additionally, this methodology allows for the expansion of target classes in the benchmark by generating paired images for any real image.
\cref{fig:benchmark_visual_new} are some examples of TwinSynths.
We can see that the content of the paired real image is faithfully reproduced and the quality of the generated image is guaranteed.

\textbf{TwinSynths-GAN benchmark.}
The GAN-based subsets in the previous benchmark have disparate training configurations, especially the class of training images, resulting in a discrepancy between the generated and the real images.
In order to generate a high quality image that preserves the content of the paired real image while leveraging the training methodology of GANs, we trained the generator from scratch using a single real image.
The MSE loss was provided to the generator to generate an image that is identical to the original image.
For reproduction, the latent vector for the generator input is maintained at a fixed value.
We created 8,000 generated images from 80 selected ImageNet\cite{russakovsky2015imagenet} classes, which is much larger than previous benchmarks.
We selected 40 classes following the \emph{ProGAN} subset in ForenSynths\cite{wang2020cnn}, while the other 40 classes were chosen arbitrarily. 
We utilized DCGAN \cite{radford2015unsupervised} architecture. 

\textbf{TwinSynths-DM benchmark.} 
In comparison to GAN-based subsets, diffusion-based subsets in conventional benchmarks were generated with off-the-shelf pretrained models, having much severer content discrepancy between real and generated images.
In order to generate a high quality image that preserves the content of paired real image while leveraging the inference process of diffusion models, we used DDIM inversion\cite{songdenoising} to generate image that is similar to the real image.
We apply a DDIM forward process to the real image to make it noisy and perform text-conditioned DDIM denoising process using the prompt template \texttt{`a photo of  \{class name\}'}. 
For the prompts, we used the class names from ImageNet.
This process makes TwinSynths-DM preserve the similarity with the paired real images. 
We used the same image classes used to create TwinSynths-GAN. 
We utilized the pretrained decoder and scheduler of \cite{songdenoising}.

\section{Experiments}
\label{sec:experiments}

\begin{figure*}[t]
\vspace{-6mm}
    \centering
    \includegraphics[width=0.8\linewidth]{figs/compare.pdf}
    \vspace{-4mm}
    \caption{\textbf{Qualitative comparison} with the baseline for generating a sequence of novel view images.  
    The results demonstrate that our method synthesizes more consistent multi-view images compared to our baseline model (Zero123). In addition, compared to SyncDreamer, our method visually maintains better similarity to the conditioned image and appears more natural.}
    \label{fig:sota_compare}
\vspace{-5mm}
\end{figure*}

\subsection{Experimental Setups}
\textbf{Dataset.}
Following previous work~\cite{zero123, SyncDreamer}, we evaluate our work on the Google Scanned Object (GSO)~\cite{GSO} dataset to verify the zero-shot novel view image synthesis capability. 
We also provide results for additional datasets in the Supplementary Material.
Specifically, we randomly select 30 objects from the GSO dataset with various object categories. 
Unlike recent approaches~\cite{mvdream, SyncDreamer} that aim to enhance the consistency of novel view synthesis models by generating multiple fixed-view images, our method can generate images from any camera pose and any number of views. Therefore, we conduct experiments under different camera pose settings to validate our approach:
specifically, 
1) \textit{16-views with free camera pose}: for each object, we circularly render 16 views with the elevation angles ranging in $[-10\degree, 40\degree]$ and the azimuth angles are evenly distributed in $[0\degree, 360\degree]$. 
2) \textit{16-views with fixed camera pose}: We maintain a constant elevation angle of $30\degree$ and uniformly sample azimuth angles (same as SyncDreamer~\cite{SyncDreamer}).
3) \textit{32-views with free camera pose}: Similar to the first setting, but we sample 32 views.
It's important to note that our method does not require additional training or fine-tuning on any datasets.

\noindent\textbf{Metrics.}
To validate the effectiveness of our method, we mainly evaluate it based on three criteria:
1) \textit{Quality Score}. We evaluate the image quality of synthesized multi-view images by measuring their similarity with ground truth images. Following prior research~\cite{zero123, sparsefusion}, we report the similarity between the synthesized images and the ground truth images with standard metrics: PSNR, SSIM~\cite{ssim}, and LPIPS~\cite{lpips}.
2) \textit{Multi-view Consistency Score}. As the primary goal of our work is to improve the consistency of generated images, we also employ the 3D consistency score~\cite{3dim} to verify the consistency among the synthesized images. Specifically, we train an Instant-NGP~\cite{instant_ngp} with the input image and part of the synthesized novel view images of our model and evaluate the similarity between the remaining synthesized images and the rendered images of Instant-NGP. For the synthesized multi-view images of each object, we allocate $3/4$ for training and reserve the remaining $1/4$ for validation.
Intuitively, if the consistency of synthesized images is improved, the NeRF-like model will train a better object representation, and the re-rendered images will agree more with the validation images.
3) \textit{Input Consistency Score}. To assess the faithfulness of synthesized images in preserving the identity of the input condition image, we introduce the input consistency score. This score calculates the similarity of each synthesized image with the input condition image, utilizing the LPIPS metric.

In addition, we use synthesized multi-view images to train a neural 3D reconstruction model (NeuS~\cite{neus}) and report commonly used Chamfer Distances (CD) and Volume IoUs between the trained 3D model and the ground truth.

\noindent\textbf{Baselines.}
Given that our main goal is to improve the consistency of the trained baseline model without further fine-tuning, we mainly compare our approach with the used baseline model Zero123~\cite{zero123}. Additionally, we compare our method to the SOTA approaches such as PGD~\cite{tseng2023consistent} and SyncDreamer~\cite{SyncDreamer} using the same Zero123 base model.

\noindent\textbf{Implementation Details.}
We use the official checkpoint provided by Zero123~\cite{zero123}, which is trained on objaverse~\cite{objaverse} for 165,000 steps. We inject our epipolar attention layer after step $T=4$ and layer $L=10$ by default. We find that feature fusion weight $\alpha=0.5$, and the number of context views $M=2$ work better.

\begin{table}[t]
\centering
\caption{Comparison of multi-view consistency, image quality, and input consistency of synthesized multi-view images at the 16-view setting with free camera pose.}
\label{tab:view16_free_compare}
\vspace{-2mm}
\scalebox{0.6}{
\begin{tabular}{c ccc ccc c}
\toprule
              & \multicolumn{3}{c}{Multi-view Consistency} & \multicolumn{3}{c}{Quality Score} & \multicolumn{1}{c}{Input Consis.} \\
              \cmidrule(lr){2-4} \cmidrule(lr){5-7} \cmidrule(lr){8-8}
              & PSNR$\uparrow$  & SSIM$\uparrow$ & LPIPS$\downarrow$ 
              & PSNR$\uparrow$  & SSIM$\uparrow$ & LPIPS$\downarrow$ 
              & LPIPS$\downarrow$ 
              \\ \midrule

Zero123
& 15.225        & 0.645       & 0.408
& 14.255        & 0.747       &	0.208
& 0.303         
\\
SyncDreamer
& 14.830        & 0.626       & 0.434
& 12.650        & 0.713       &	0.254
& 0.317         
\\
Ours 
& \best{18.300}	& \best{0.734}	& \best{0.355}
& \best{14.947}	& \best{0.763}	& \best{0.191}
& \best{0.282}
\\

\bottomrule
\end{tabular}
}
\end{table}

\begin{table}[t]
\vspace{-1mm}
\centering
\caption{Comparison of multi-view consistency, image quality, and input consistency at the 16-view setting with fixed camera pose as SyncDreamer~\cite{SyncDreamer}.}
\label{tab:view16_fxied_compare}
\vspace{-3mm}
\scalebox{0.6}{
\begin{tabular}{c ccc ccc c}
\toprule
              & \multicolumn{3}{c}{Multi-view Consistency} & \multicolumn{3}{c}{Quality Score} & \multicolumn{1}{c}{Input Consis.} \\
              \cmidrule(lr){2-4} \cmidrule(lr){5-7} \cmidrule(lr){8-8}
              & PSNR$\uparrow$  & SSIM$\uparrow$ & LPIPS$\downarrow$ 
              & PSNR$\uparrow$  & SSIM$\uparrow$ & LPIPS$\downarrow$ 
              & LPIPS$\downarrow$ 
              \\ \midrule

Zero123
& 16.556        & 0.682       & 0.378
& 14.592        & 0.750       &	0.207
& 0.305         
\\
SyncDreamer
& \best{22.424}        & \best{0.812}       & \best{0.268}
& 15.269        & 0.749       &	0.196
& 0.300         
\\
Ours 
& 21.151	& 0.780	& 0.302
& \best{15.293}	& \best{0.764}	& \best{0.184}
& \best{0.287}
\\

\bottomrule
\end{tabular}
}
\vspace{-4mm}
\end{table}


\subsection{Comparison With Baseline Models}
The quantitative comparison on three settings are shown in Tab.~\ref{tab:view16_free_compare}, Tab.~\ref{tab:view16_fxied_compare}, and Tab.~\ref{tab:view32_free_compare}. The qualitative comparison is shown in Fig.~\ref{fig:sota_compare}.

\begin{table}[t]
\centering
\caption{Comparison of multi-view consistency and image quality scores of synthesized multi-view images at the 32-view setting with free camera pose.}
\vspace{-3mm}
\label{tab:view32_free_compare}
\scalebox{0.7}{
\begin{tabular}{c ccc ccc}
\toprule
              & \multicolumn{3}{c}{Multi-view Consistency} & \multicolumn{3}{c}{Quality Score} \\
              \cmidrule(lr){2-4} \cmidrule(lr){5-7}
              & PSNR$\uparrow$  & SSIM$\uparrow$ & LPIPS$\downarrow$ 
              & PSNR$\uparrow$  & SSIM$\uparrow$ & LPIPS$\downarrow$ 
              \\ \midrule

Zero123
& 16.515        & 0.694       & 0.378
& 15.142        & 0.733       &	0.211
\\
PGD~\cite{tseng2023consistent}
& 18.481        & 0.720       & 0.343
& 15.281        & 0.739       &	0.205
\\
Ours 
& \best{20.655}	& \best{0.792}	& \best{0.305}
& \best{15.268}	& \best{0.742}	& \best{0.203}
\\

\bottomrule
\end{tabular}
}
\vspace{-3mm}
\end{table}

\begin{table*}
  [t]
  \centering
  \resizebox{\textwidth}{!}{%
  \begin{tabular}{cccccccccccc}
    \toprule \multicolumn{2}{c}{Components}                                                             & \multicolumn{5}{c}{Re-executability Rate (\%)} & \multicolumn{5}{c}{Readability (\#)} \\
    \cmidrule(lr){1-2} \cmidrule(lr){3-7} \cmidrule(lr){8-12}        \hspace{8pt}\labelemoji\hspace{8pt}                                                                & \hspace{8pt}\toolemoji\hspace{8pt}                                      & O0                                 & O1             & O2             & O3             & AVG            & O0             & O1             & O2             & O3             & AVG            \\
    \hline
    \rowcolor[rgb]{0.93,0.93,0.93}\multicolumn{12}{c}{\textbf{Initialize with LLM4Decompile-End-6.7B~\citep{llm4decompile}}}   \\
    \xmark                                                                                              & \xmark                                    & 69.51                              & 46.95          & 50.61          & 46.34          & 53.35          & 3.98 & 3.41 & 3.44 & 3.38 & 3.55 \\
    \cmark                                                                                              & \xmark                                    & 75.61                              & 50.61          & 50.00          & 50.00          & 56.55          & 4.01 & 3.44 & 3.39 & \textbf{3.49} & 3.58 \\
    \xmark                                                                                              & \cmark                                    & 83.54                     & \textbf{56.10}          & 51.22          & 50.61 & 60.37 & 4.05 & 3.51 & 3.51 & 3.42 & 3.62 \\
    \cmark                                                                                              & \cmark                                    & \textbf{85.37}                            & \textbf{56.10}                     & \textbf{51.83} & \textbf{52.43}          & \textbf{61.43} & \textbf{4.13} & \textbf{3.60} & \textbf{3.54} & \textbf{3.49} & \textbf{3.69} \\

    \rowcolor[rgb]{0.93,0.93,0.93}\multicolumn{12}{c}{\textbf{Initialize with Deepseek-Coder-6.7B-base~\citep{deepseekcoder}}} \\
    \xmark                                                                                              & \xmark                                    & 59.15                              & 35.98          & 39.02          & 37.80          & 42.99          & 3.71 & 3.05 & 3.16 & 3.05 & 3.24 \\
    \cmark                                                                                              & \xmark                                    & 66.46                              & 41.46          & 38.41          & 36.59          & 45.73          & 3.76 & 3.17 & \textbf{3.21} & 3.08 & 3.31 \\
    \xmark                                                                                              & \cmark                                    & 70.73                              & 39.63          & 39.02          & 40.24          & 47.41          & 3.90 & 3.17 & 3.08 & 3.11 & 3.31 \\
    \cmark                                                                                              & \cmark                                    & \textbf{79.88}                     & \textbf{45.73} & \textbf{43.90} & \textbf{42.68} & \textbf{53.05} & \textbf{3.96} & \textbf{3.21} & 3.18 & \textbf{3.19} & \textbf{3.38} \\
    \bottomrule
  \end{tabular}%
  }
  \caption{The ablation study of different methods across four optimization levels
  (O0, O1, O2, O3), as well as their average scores (AVG). The results in bold represent the optimal performance. The ~\labelemoji~ and ~\toolemoji~ means Relabedling and Function Call. \textbf{Bold} denotes the best performance.}
  \label{tab:ablation}
\end{table*}



\begin{figure*}[ht]
    \centering
    \begin{minipage}{0.65\textwidth}
        \centering
        \includegraphics[width=0.95\linewidth]{figs/ablation.pdf}
        \vspace{-2mm}
        \captionof{figure}{Qualitative Comparison for different design choices. Our method, employing multi-view epipolar attention, demonstrates the best consistency.}
        \label{fig:ablation}
    \end{minipage}\hfill
    \begin{minipage}{0.33\textwidth}
        \centering
        \includegraphics[width=0.8\linewidth]{figs/neus_ver.pdf}
        \vspace{-3mm}
        \caption{Our method shows better direct 3D reconstruction~\cite{neus}.}
        \label{fig:neus}
    \end{minipage}
    \vspace{-5mm}
\end{figure*}

\noindent\textbf{Multi-view Consistency.}
Tab.~\ref{tab:view16_fxied_compare} presents the 3D consistency scores compared to our baseline model (Zero123) and SyncDreamer. The results indicate a significant improvement across all three metrics achieved by our method when compared with Zero123.
While our method exhibits a marginally lower numerical consistency score compared to SyncDreamer, it enables the synthesis of images with arbitrary camera poses.	
This capability is illustrated in Tab.~\ref{tab:view16_free_compare}, where our method consistently enhances consistency with changes in camera pose settings, whereas SyncDreamer fails to do so and exhibits inferior results compared to Zero123.
Furthermore, our method facilitates the synthesis of multi-view images with any number of camera views. This versatility is demonstrated in Tab.~\ref{tab:view32_free_compare}, where our method continues to achieve significant improvements in consistency scores, while SyncDreamer is unable to operate under such conditions.	

Meanwhile, Fig.~\ref{fig:sota_compare} provides a qualitative comparison with the baseline. While both our method and SyncDreamer enhance consistency, our method visually preserves better similarity to the input image, including color and texture details. The input consistency score further corroborates this.

\noindent\textbf{Image Quality.}
While our primary goal centers around enhancing the consistency of synthesized multi-view images, we also evaluate the image quality by comparing the similarity with the ground truth images. The results shown in Tab.~\ref{tab:view16_free_compare}, Tab.~\ref{tab:view16_fxied_compare}, and Tab.~\ref{tab:view32_free_compare} indicate that our method also enhances the image quality under different settings besides improving the consistency.
Moreover, our method shows better image quality compared with SyncDreamer even in the 16-view setting with fixed camera pose.

\noindent\textbf{Input Consistency.}
Input consistency terms whether the results align with the input image.
Fig.~\ref{fig:sota_compare} illustrates that both our method and SyncDreamer enhance multi-view consistency. However, the color and texture details of SyncDreamer's results diverge from the input image and appear visually unnatural.
This discrepancy is evident in the input consistency score presented in Tab.~\ref{tab:view16_fxied_compare}, indicating lower similarity with the condition image in the SyncDreamer results.	

\subsection{Ablation Study}
The overall quantitative results are shown in Tab.~\ref{tab:ablation}, and the qualitative comparisons are shown in Fig.~\ref{fig:ablation}.

\noindent \textbf{Full Attention \vs Epipolar Attention.}
The results presented in Tab.\ref{tab:ablation} and Fig.\ref{fig:ablation} demonstrate that our epipolar attention mechanism can synthesize more consistent multi-view images compared with full attention. Furthermore, our epipolar attention achieves a greater performance improvement compared to full attention when using multiple reference images. This could be attributed to the fact that our epipolar attention more effectively localizes target information, as depicted in Fig.~\ref{fig:full_attn_compare}, thereby reducing noise from the reference images. In the multi-view setting, where multiple reference images are utilized, this noise reduction becomes particularly crucial.
Moreover, it is noteworthy that the epipolar attention mechanism consumes less GPU memory compared to our baseline, as discussed in Sec.~\ref{sec:attn_analysis}.

\noindent \textbf{Attending Single-View \vs Multi-View.}
Applying the epipolar attention significantly improves the consistency between the input and target views. However, the consistency between different views in the unobserved regions of the input view is not well preserved.
After implementing our epipolar attention in the multi-view setting, the consistency across the generated multi-view images is further improved. The last row in Tab.~\ref{tab:ablation} shows that after applying our multi-view epipolar attention, the consistency score is further improved compared with the single-view setting. Besides, the qualitative result in Fig.~\ref{fig:ablation} also shows better consistency among different target views.



\begin{table}[t]
\centering
\vspace{-1mm}
\caption{Comparison of 3D reconstruction results. Our method significantly improves the reconstruction quality.}
\vspace{-3mm}
\label{tab:neus}
\scalebox{0.7}{
\begin{tabular}{c cc}
\toprule
              &  Chamfer Dist.$\downarrow$  & Volume IoU$\uparrow$
\\ \midrule

            Zero123         & 0.017         & 0.819    \\
            SyncDreamer     & \best{0.013}         & \best{0.847}    \\
            Ours            & 0.014	& 0.842 \\

\bottomrule
\end{tabular}
}
\vspace{-5mm}
\end{table}


\vspace{-2mm}
\subsection{Downstream Application}
\vspace{-2mm}
To demonstrate the effectiveness of our method, we also applied it to the downstream 3D reconstruction task. Specifically, we trained the NeuS model~\cite{neus} directly using images synthesized by our method, Zero123, and SyncDreamer, respectively.
The quantitative results in Tab.~\ref{tab:neus} show that the consistent multi-view images synthesized by our method can significantly improve the 3D reconstruction quality.
Additionally, our method exhibits similar performance to SyncDreamer which requires time-consuming re-training.
The qualitative results in Fig.~\ref{fig:neus} show that it is challenging to train the NeuS model directly due to the lack of consistency in the images generated by Zero123. In contrast, our method generates more consistent multi-view images and, therefore, better reconstructs the geometry and texture details.
We show improvements on other downstream applications such as image-to-3D in the Supplementary Material.


\section{Conclusion}

%In this paper, w
We propose a new PEFT method called DiffoRA, which enables efficient and adaptive LLM fine-tuning based on LoRA. 
Instead of adjusting every interior rank, 
%of the decomposition matrices 
%of all modules, 
we argue that adopting LoRA module-wisely is sufficient. 
To achieve this, we construct a DAM to select the modules that are most suitable and essential to fine-tune. We theoretically analyze how the DAM impacts the convergence rate and generalization capability.
%of the pre-trained model. 
Furthermore, we adopt continuous relaxation and discretization to establish DAM.
%for each task. 
To alleviate the issue of discretization discrepancy, we utilize the weight-sharing strategy for optimization. 
%We fully implement our method and t
The experimental results demonstrate that our DiffoRA works consistently better than the baselines across all benchmarks. 

%%%%%%%%% REFERENCES
{\small
\bibliographystyle{ieee_fullname}
\bibliography{egbib}
\nocite{cazenavette2024fakeinversion}
}

% supplementary
\clearpage
\appendix
\clearpage
\newcommand{\nocontentsline}[3]{}
\newcommand{\tocless}[2]{\bgroup\let\addcontentsline=\nocontentsline#1{#2}\egroup}

\newcommand{\Appendix}[1]{
  \refstepcounter{section}
  \section*{Appendix \thesection:\hspace*{1.5ex} #1}
  \addcontentsline{toc}{section}{Appendix \thesection}
}
\newcommand{\SubAppendix}[1]{\tocless\subsection{#1}}
% \setcounter{page}{1}
% \setcounter{section}{0}
% \setcounter{figure}{0}
% \setcounter{table}{0}
\maketitlesupplementary
\appendix


\tableofcontents
\addtocontents{toc}{}
% This supplemental document contains seven sections:
% Section \ref{sec:add_details} shows more implementation details of our MotionCanvas;
% Section \ref{sec:ui} presents more details of user interface;
% Section \ref{sec:transform} provides more analysis on the essentiality of camera-aware and camera-object-aware transformations; 
% Section \ref{sec:ar} presents more details of our MotionCanvas$_\text{AR}$; 
% Section \ref{sec:user} shows more details of user study; 
% Section \ref{sec:add_analysis} presents additional analysis; 
% and Section \ref{sec:limitations} shows the limitations of our method.

Please check our project page \url{https://motion-canvas25.github.io/} for video results.
% In addition to this supplementary document, \textbf{we also provided a local HTML webpage} showing video comparisons.



\section{Additional Details}
\label{sec:add_details}
Our MotionCanvas model is trained using 16 nodes of NVIDIA H100 (80GB) GPUs (8 GPUs on each node). On a single H100 GPU, generating a 32-frame video clip at a resolution of 352×640 with 50 denoising steps takes approximately 32 seconds. During training, we exclusively optimize the DiT transformer blocks and the additional Linear/MLP layers introduced for bounding box conditioning and DCT coefficient tokenization, while keeping all other modules frozen.


%%%%%%%%%%%%%%%%%%%%%%%%%%%%%%%%%%%
\begin{figure*}[htbp]
    \centering
    \includegraphics[width=0.85\linewidth]{Graphics/supp/User_interface.pdf} % Replace with your image file
    \caption{A sample of the designed user interface for our MotionCanvas.}
    \label{fig:ui}
\end{figure*}
%%%%%%%%%%%%%%%%%%%%%%%%%%%%%%%%%%%
\section{User Interface}
\label{sec:ui}
We designed a sample user interface to provide flexible and interactive control over camera motion, object global and local motion, and their timing. An example of the user interface is illustrated in Fig.~\ref{fig:ui}.

\textbf{Specifying Camera Motion.} To facilitate a user-friendly approach for defining camera motion trajectories, the interface allows users to combine $M$ base motion patterns with configurable parameters such as direction (positive or negative) and speed (absolute value), as demonstrated in the ``Camera Motion Control" panel in Fig.~\ref{fig:ui}. Specifically, the base motion patterns include:

\begin{itemize}[left=2em]
    \item Horizontal (Trucking) left/right
    \item Panning left/right
    \item Dolly in/out
    \item Vertical (Pedestal) up/down
    \item Tilt up/down
    \item Roll clockwise/anti-clockwise
    \item Zoom in/out
    \item Orbit left/right (adjustable radius)
    \item Circle clockwise/anti-clockwise
    \item Static
\end{itemize}

The sign (positive or negative) and the absolute value of the number associated with each motion pattern define the corresponding camera poses relative to the zero pose at the first frame (i.e., translation and rotation vectors).

\textbf{Specifying Scene-aware Object Global Motion.} To enable user control over object global motion trajectories, we provide an interactive canvas (see Fig.~\ref{fig:ui}, ``Object Global Motion Control") where users can draw starting and ending bounding boxes, as well as optional intermediate points. A smooth bounding box trajectory can be obtained by applying Catmull-Rom spline interpolation. For each bounding box, users can optionally specify a reference depth point on the image. Additionally, users can decide whether to directly use this scene-space bounding box sequence as a condition. Utilizing the scene-space bounding box is particularly more effective for creating cinematic effects such as ``follow shots" or ``dolly shots".

For standard scene-aware object global motion control, the scene-space bounding boxes are assigned depth values, as described in Section~3.2 of the main paper. The bounding box sequence is then converted into screen space using the proposed Motion Signal Translation module.

\textbf{Specifying scene-aware object local motion.} We also provide a dedicated canvas for controlling object local motion (see Fig.~\ref{fig:ui}, ``Object Local Motion Control"). Users can draw any number of point trajectories, which are assigned depth values as outlined in Section 3.2 of the main paper. Similar to bounding boxes, these point trajectories are transformed into screen space based on the camera motion and object global motion. The object global motion transformation takes effect only when the starting point of the trajectory lies within the object's semantic region. We then transform the object's local motion point trajectories by maintaining their relative positions with respect to the underlying bounding box.

Additionally, our user interface includes a ``Preview Window" that allows users to visualize the generated videos, as well as the bounding box sequences and point trajectories in both scene space and screen space.



\begin{figure}[htbp]
    \centering
    \includegraphics[width=1\linewidth]{Graphics/supp/Transform_supp.pdf} % Replace with your image file
    \caption{Illustration of camera-aware transformation and camera-object-aware transformation. In preview videos, dash-line bounding boxes represent the scene-space inputs, while the solid ones with the same color denote the transformed screen-space motion signals. Similarly, point trajectories with white trace indicate scene-space user motion design, while colored ones represent transformed signals. Better investigation in supplementary videos.}
    \label{fig:transform}
\end{figure}
\section{Essentiality of Camera-aware and Camera-object-aware Transformations}
\label{sec:transform}
Drawing inspiration from classical graphics imaging techniques, we introduce a Motion Signal Translation module to convert scene-space user-defined motion intents into screen-space motion signals. This enables joint control of camera and object motions in a 3D-aware manner for image-to-video generation. The Motion Signal Translation module incorporates a hierarchical transformation framework that accounts for the intertwining nature of camera and object motions. To illustrate the effectiveness of these transformations, we provide visual comparisons highlighting both camera-aware and camera-object-aware transformations.

\textbf{Camera-aware Transformation for Object Global or Local Motion.}
First, we present the camera-aware transformation for object global motion control in Fig.~\ref{fig:transform}(top). In the preview video (last frame), the dashed-line bounding boxes represent the scene-space inputs specified by the user, while the solid bounding boxes of the same color denote the corresponding transformed screen-space motion signals.

In this example, people are running forward on the road as controlled by the user’s input (bounding boxes). When a trucking-left camera motion is applied, all the people should naturally move to the right on the screen. Using our camera-aware transformation, the screen-space object bounding boxes are correctly calibrated, ensuring that the resulting animation appears accurate and more natural (refer to the supplementary webpage: ``Additional Analysis -- Essentiality of Camera-aware and Camera-object-aware Transformations").

A similar conclusion holds for the camera-aware transformation applied to object local motion, as shown in Fig.~\ref{fig:transform}(middle).

\textbf{Camera-object-aware Transformation for Object Local Motion.}
Camera-object-aware transformation implies that translating scene-space point trajectory specifying the object local motion to screen-space signals must take into account both the camera motion and object global motion.  For example, as illustrated in Fig.~\ref{fig:transform}(bottom), the trajectory of an object’s local motion, such as ``putting down hands" must account for both the body’s movement and the camera’s pedestal-up motion. As demonstrated in ``Additional Analysis -- Essentiality of Camera-aware and Camera-object-aware Transformations", our transformation produces a life-like and accurate video, whereas the variant without these transformations results in unnatural and incorrect motion.

\begin{figure}[t]
    \centering
    \includegraphics[width=1\linewidth]{Graphics/supp/AR_recomputation.pdf} % Replace with your image file
    \caption{Illustration of the recomputation process for input motion conditions in our MotionCanvas$_\text{AR}$ during inference.}
    \label{fig:back_trace}
\end{figure}

\section{More Details of MotionCanvas$_\text{AR}$}
\label{sec:ar}
To enhance support for long video generation and address motion discontinuities, we introduce a 16-frame conditioned 64-frame MotionCanvas$_\text{AR}$, designed to generate videos in an auto-regressive manner. This model builds on our 32-frame motion-conditioned I2V model (single-frame-conditioned) and is fine-tuned for an additional 120K iterations, while retaining the same training configurations.

To further refine the input motion signals and better align them with the training setup, we recompute the screen-space motion signals by integrating the user’s motion intent with back-traced motions, as illustrated in Fig.~\ref{fig:back_trace}. This method ensures smoother, more consistent motion generation throughout the video.


\section{User Study}
\label{sec:user}
The designed user study interface is shown in Figure~\ref{fig:user_study_screen_shot}. We collect 15 representative image cases from the Internet and design various motion controls. We then generate the video clips results by executing the official code~\cite{wu2025draganything,niu2025mofa}. For the user study, we use these video results produced by shuffled methods based on the same set of input conditions. In addition, we standardize all the produced results by encoding FPS$=$8 for 14 generated frames, yielding $\sim$2-second videos for each method. This process ensures a fair comparison.
%% TODO: describe how we convert into points and into those methods

The user study is expected to be completed with 7--15 minutes (15 cases $\times$ 3 sub-questions $\times$ 10--20 seconds for each judgement). To remove the impact of random selection, we filter out those comparison results completed within three minutes. For each participant, the user study interface shows 15 groups of video comparisons, and the participant is instructed to evaluate the videos for three times, \ie, answering the following questions respectively: (i) ``Which one shows the best motion adherence?"; (ii) ``Which one has the best motion/dynamic quality?"; (iii) ``which one shows the best frame fidelity?". Finally, we received 35 valid responses from the participants.

\begin{figure}[t]
    \centering
    \includegraphics[width=1.0\linewidth]{Graphics/supp/camera_motion_icons.pdf} % Replace with your image file
    \caption{Legend of camera motions used in the main paper.}
    \label{fig:legend}
\end{figure}

\begin{figure*}[htbp]
    \centering
    \includegraphics[width=0.8\linewidth]{Graphics/supp/user_study_supp_figure_whole.pdf} % Replace with your image file
    \caption{Designed user study interface. Each participant is required to evaluate 15 video comparisons and respond to three corresponding sub-questions for each comparison. Only one video is shown here due to the page limit.}
    \label{fig:user_study_screen_shot}
\end{figure*}

\section{Additional Analysis}
\label{sec:add_analysis}
\subsection{Effect of Point Track Density on Camera Motion Control}

We investigate the effect of point track density on camera motion control by specifying orbit right camera motion with different numbers of 2D point tracks. The visual comparison result is shown in the supplementary webpage `Additional Analysis -- Effect of Point Track Density on Camera Motion Control'. As can be seen, the motion is underspecified with significant ambiguity when providing low-density tracks. Hence, the generated camera motion does not follow the control and tends to be trucking left. By providing higher-density tracks, the generated motion can better adhere to the orbit camera motion.


\subsection{Effect of Text Prompt}
We use simple text descriptions throughout all experiments. To further investigate the effect of text prompts on our MotionCanvas, we show the visual comparison of gradually more detailed text prompts in the supplementary webpage `Additional Analysis-- Effect of Text Prompt.'. It demonstrate that text prompt does not have a significant effect on the camera motion control. However, it can generate diverse dynamics like `raining' and `turning around'.


\section{Limitations and Future Work}
\label{sec:limitations}


Our work introduces a novel framework for enhancing I2V with holistic motion controls, enabling cinematic shot design from a single image. While this paper made substantial progress on toward this important application, challenges remain, opening up opportunities for future research. 

First, our use of a video diffusion model (VDM) for the video synthesis module, while enabling high-quality video generation, results in relatively slow inference times (approximately 35 seconds for a 2-second video). This computational cost, typical of modern VDMs, currently limits real-time applications. Exploring alternative, more efficient generative models is a promising direction for future work. 

Second, our current method approximates object local motion by assuming each object lies on a frontal parallel depth plane. Although effective for most natural scenes as the depth variation within the object are typically small compared to object's distance to the camera, this pseudo-3D approximation may not be suitable for extreme close-up or macro shots where depth variations within the object are significant. In future work, it will be interesting to investigate integrating more explicit 3D formulation for handling such scenarios. 

Finally, our system currently does not explicitly constrain the harmonization between motion design and textual prompts. On one hand, this offers the flexibility for users to leverage both control modalities jointly to explore creative variations. On the other hand, this leaves the possibility of conflicting motion signals between the modalities. For example, in Fig. 8 (top row) in the main paper, while the text prompt indicates the cat to be waiting instead of moving, when motion design explicitly control the cat to moves forward in the later part of the videos (note that such global object control was used when generated those results but was not illustrated in the figure to avoid cluttered visualization), such motion control overrides the ones hinted in the textual prompts. Explicit motion-aware prompt harmonization can be a fruitful research direction to extend our work.  

\section{Camera Motion Legend}
\label{sec:legend}
The legend of camera motions used in the main paper is presented in Fig.~\ref{fig:legend}.



\end{document}
