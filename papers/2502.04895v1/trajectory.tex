\chapter{Recursive Data Interpolation Techniques} % Neural decoder
\chaptermark{Interpolation}
%\thispagestyle{empty}
\label{sec:data interpolation}



Ch.~\ref{sec:copulas} described how, from $N$ data points $(x_1,\dots, x_N)$, it is possible to leverage NNs to interpolate a function representing the underlying PDF $p_X(x)$ that generated the observed data. In all the previous contexts, interpolation effectively meant estimating the distribution of the given samples in order to understand the \textit{stochastic} nature of the data. We typically exploit $p_X(x)$ to produce new realizations $(\hat{x}_1,\dots, \hat{x}_N)$.

In this chapter, instead, we focus on capturing the \textit{deterministic} relationship between inputs and outputs, determining a function that passes through specific points. 
When we approach interpolation from the perspective of finding a fitting function, we aim to create a mathematical model that accurately represents the relationship between the given data points, abstracting from how these points have been obtained. 
This fitting function allows us to predict intermediate values within the range of the data, enabling smooth transitions between observed points.

Fitting functions are extremely useful for trajectory generation in various fields such as physics, robotics and aerospace engineering. When designing trajectories for moving objects or systems, it is essential to create smooth and continuous paths that meet specific constraints and objectives.
In the following, we study the trajectory generation problem and propose a novel iterative interpolation technique.

The results presented in this chapter are documented in \cite{Letizia2020_rst, LetiziaRobotics, 9525383}.

\label{sec:rst}
\section{Introduction}

Large language models (LLMs) have achieved remarkable success in automated math problem solving, particularly through code-generation capabilities integrated with proof assistants~\citep{lean,isabelle,POT,autoformalization,MATH}. Although LLMs excel at generating solution steps and correct answers in algebra and calculus~\citep{math_solving}, their unimodal nature limits performance in plane geometry, where solution depends on both diagram and text~\citep{math_solving}. 

Specialized vision-language models (VLMs) have accordingly been developed for plane geometry problem solving (PGPS)~\citep{geoqa,unigeo,intergps,pgps,GOLD,LANS,geox}. Yet, it remains unclear whether these models genuinely leverage diagrams or rely almost exclusively on textual features. This ambiguity arises because existing PGPS datasets typically embed sufficient geometric details within problem statements, potentially making the vision encoder unnecessary~\citep{GOLD}. \cref{fig:pgps_examples} illustrates example questions from GeoQA and PGPS9K, where solutions can be derived without referencing the diagrams.

\begin{figure}
    \centering
    \begin{subfigure}[t]{.49\linewidth}
        \centering
        \includegraphics[width=\linewidth]{latex/figures/images/geoqa_example.pdf}
        \caption{GeoQA}
        \label{fig:geoqa_example}
    \end{subfigure}
    \begin{subfigure}[t]{.48\linewidth}
        \centering
        \includegraphics[width=\linewidth]{latex/figures/images/pgps_example.pdf}
        \caption{PGPS9K}
        \label{fig:pgps9k_example}
    \end{subfigure}
    \caption{
    Examples of diagram-caption pairs and their solution steps written in formal languages from GeoQA and PGPS9k datasets. In the problem description, the visual geometric premises and numerical variables are highlighted in green and red, respectively. A significant difference in the style of the diagram and formal language can be observable. %, along with the differences in formal languages supported by the corresponding datasets.
    \label{fig:pgps_examples}
    }
\end{figure}



We propose a new benchmark created via a synthetic data engine, which systematically evaluates the ability of VLM vision encoders to recognize geometric premises. Our empirical findings reveal that previously suggested self-supervised learning (SSL) approaches, e.g., vector quantized variataional auto-encoder (VQ-VAE)~\citep{unimath} and masked auto-encoder (MAE)~\citep{scagps,geox}, and widely adopted encoders, e.g., OpenCLIP~\citep{clip} and DinoV2~\citep{dinov2}, struggle to detect geometric features such as perpendicularity and degrees. 

To this end, we propose \geoclip{}, a model pre-trained on a large corpus of synthetic diagram–caption pairs. By varying diagram styles (e.g., color, font size, resolution, line width), \geoclip{} learns robust geometric representations and outperforms prior SSL-based methods on our benchmark. Building on \geoclip{}, we introduce a few-shot domain adaptation technique that efficiently transfers the recognition ability to real-world diagrams. We further combine this domain-adapted GeoCLIP with an LLM, forming a domain-agnostic VLM for solving PGPS tasks in MathVerse~\citep{mathverse}. 
%To accommodate diverse diagram styles and solution formats, we unify the solution program languages across multiple PGPS datasets, ensuring comprehensive evaluation. 

In our experiments on MathVerse~\citep{mathverse}, which encompasses diverse plane geometry tasks and diagram styles, our VLM with a domain-adapted \geoclip{} consistently outperforms both task-specific PGPS models and generalist VLMs. 
% In particular, it achieves higher accuracy on tasks requiring geometric-feature recognition, even when critical numerical measurements are moved from text to diagrams. 
Ablation studies confirm the effectiveness of our domain adaptation strategy, showing improvements in optical character recognition (OCR)-based tasks and robust diagram embeddings across different styles. 
% By unifying the solution program languages of existing datasets and incorporating OCR capability, we enable a single VLM, named \geovlm{}, to handle a broad class of plane geometry problems.

% Contributions
We summarize the contributions as follows:
We propose a novel benchmark for systematically assessing how well vision encoders recognize geometric premises in plane geometry diagrams~(\cref{sec:visual_feature}); We introduce \geoclip{}, a vision encoder capable of accurately detecting visual geometric premises~(\cref{sec:geoclip}), and a few-shot domain adaptation technique that efficiently transfers this capability across different diagram styles (\cref{sec:domain_adaptation});
We show that our VLM, incorporating domain-adapted GeoCLIP, surpasses existing specialized PGPS VLMs and generalist VLMs on the MathVerse benchmark~(\cref{sec:experiments}) and effectively interprets diverse diagram styles~(\cref{sec:abl}).

\iffalse
\begin{itemize}
    \item We propose a novel benchmark for systematically assessing how well vision encoders recognize geometric premises, e.g., perpendicularity and angle measures, in plane geometry diagrams.
	\item We introduce \geoclip{}, a vision encoder capable of accurately detecting visual geometric premises, and a few-shot domain adaptation technique that efficiently transfers this capability across different diagram styles.
	\item We show that our final VLM, incorporating GeoCLIP-DA, effectively interprets diverse diagram styles and achieves state-of-the-art performance on the MathVerse benchmark, surpassing existing specialized PGPS models and generalist VLM models.
\end{itemize}
\fi

\iffalse

Large language models (LLMs) have made significant strides in automated math word problem solving. In particular, their code-generation capabilities combined with proof assistants~\citep{lean,isabelle} help minimize computational errors~\citep{POT}, improve solution precision~\citep{autoformalization}, and offer rigorous feedback and evaluation~\citep{MATH}. Although LLMs excel in generating solution steps and correct answers for algebra and calculus~\citep{math_solving}, their uni-modal nature limits performance in domains like plane geometry, where both diagrams and text are vital.

Plane geometry problem solving (PGPS) tasks typically include diagrams and textual descriptions, requiring solvers to interpret premises from both sources. To facilitate automated solutions for these problems, several studies have introduced formal languages tailored for plane geometry to represent solution steps as a program with training datasets composed of diagrams, textual descriptions, and solution programs~\citep{geoqa,unigeo,intergps,pgps}. Building on these datasets, a number of PGPS specialized vision-language models (VLMs) have been developed so far~\citep{GOLD, LANS, geox}.

Most existing VLMs, however, fail to use diagrams when solving geometry problems. Well-known PGPS datasets such as GeoQA~\citep{geoqa}, UniGeo~\citep{unigeo}, and PGPS9K~\citep{pgps}, can be solved without accessing diagrams, as their problem descriptions often contain all geometric information. \cref{fig:pgps_examples} shows an example from GeoQA and PGPS9K datasets, where one can deduce the solution steps without knowing the diagrams. 
As a result, models trained on these datasets rely almost exclusively on textual information, leaving the vision encoder under-utilized~\citep{GOLD}. 
Consequently, the VLMs trained on these datasets cannot solve the plane geometry problem when necessary geometric properties or relations are excluded from the problem statement.

Some studies seek to enhance the recognition of geometric premises from a diagram by directly predicting the premises from the diagram~\citep{GOLD, intergps} or as an auxiliary task for vision encoders~\citep{geoqa,geoqa-plus}. However, these approaches remain highly domain-specific because the labels for training are difficult to obtain, thus limiting generalization across different domains. While self-supervised learning (SSL) methods that depend exclusively on geometric diagrams, e.g., vector quantized variational auto-encoder (VQ-VAE)~\citep{unimath} and masked auto-encoder (MAE)~\citep{scagps,geox}, have also been explored, the effectiveness of the SSL approaches on recognizing geometric features has not been thoroughly investigated.

We introduce a benchmark constructed with a synthetic data engine to evaluate the effectiveness of SSL approaches in recognizing geometric premises from diagrams. Our empirical results with the proposed benchmark show that the vision encoders trained with SSL methods fail to capture visual \geofeat{}s such as perpendicularity between two lines and angle measure.
Furthermore, we find that the pre-trained vision encoders often used in general-purpose VLMs, e.g., OpenCLIP~\citep{clip} and DinoV2~\citep{dinov2}, fail to recognize geometric premises from diagrams.

To improve the vision encoder for PGPS, we propose \geoclip{}, a model trained with a massive amount of diagram-caption pairs.
Since the amount of diagram-caption pairs in existing benchmarks is often limited, we develop a plane diagram generator that can randomly sample plane geometry problems with the help of existing proof assistant~\citep{alphageometry}.
To make \geoclip{} robust against different styles, we vary the visual properties of diagrams, such as color, font size, resolution, and line width.
We show that \geoclip{} performs better than the other SSL approaches and commonly used vision encoders on the newly proposed benchmark.

Another major challenge in PGPS is developing a domain-agnostic VLM capable of handling multiple PGPS benchmarks. As shown in \cref{fig:pgps_examples}, the main difficulties arise from variations in diagram styles. 
To address the issue, we propose a few-shot domain adaptation technique for \geoclip{} which transfers its visual \geofeat{} perception from the synthetic diagrams to the real-world diagrams efficiently. 

We study the efficacy of the domain adapted \geoclip{} on PGPS when equipped with the language model. To be specific, we compare the VLM with the previous PGPS models on MathVerse~\citep{mathverse}, which is designed to evaluate both the PGPS and visual \geofeat{} perception performance on various domains.
While previous PGPS models are inapplicable to certain types of MathVerse problems, we modify the prediction target and unify the solution program languages of the existing PGPS training data to make our VLM applicable to all types of MathVerse problems.
Results on MathVerse demonstrate that our VLM more effectively integrates diagrammatic information and remains robust under conditions of various diagram styles.

\begin{itemize}
    \item We propose a benchmark to measure the visual \geofeat{} recognition performance of different vision encoders.
    % \item \sh{We introduce geometric CLIP (\geoclip{} and train the VLM equipped with \geoclip{} to predict both solution steps and the numerical measurements of the problem.}
    \item We introduce \geoclip{}, a vision encoder which can accurately recognize visual \geofeat{}s and a few-shot domain adaptation technique which can transfer such ability to different domains efficiently. 
    % \item \sh{We develop our final PGPS model, \geovlm{}, by adapting \geoclip{} to different domains and training with unified languages of solution program data.}
    % We develop a domain-agnostic VLM, namely \geovlm{}, by applying a simple yet effective domain adaptation method to \geoclip{} and training on the refined training data.
    \item We demonstrate our VLM equipped with GeoCLIP-DA effectively interprets diverse diagram styles, achieving superior performance on MathVerse compared to the existing PGPS models.
\end{itemize}

\fi 

\section{Problem statement}
\label{subsec:rst_problem}
Consider the problem of navigating a moving body, e.g., an unmanned vehicle, through $N+1$ waypoints at specific time stamps. Assume that any arbitrary dynamic limitation of the  body translates into kinematic constraints (e.g. position, velocity, acceleration, etc.). Such kinematic constraints are useful to impose, for instance, that the body starts from the rest at the beginning of the trajectory and reaches each waypoint with specific kinematics (e.g. aggressive maneuvering).

The task of path planning consists of finding a trajectory $\mathbf{x}(t)$ to move a given body from an initial point $\mathbf{x}(t_0)$ to a final point $\mathbf{x}(t_N)$, minimizing a certain cost function $J(\cdot)$ (time, energy, jerk, etc.), under given geometrical and kinematic constraints $\sigma(\cdot)$.

Let $t_0<t_1<\dots<t_N$ be $N+1$ time stamps for which all the corresponding positions $\mathbf{x}(t_0),\mathbf{x}(t_1), \dots, \mathbf{x}(t_N)$ and the respective $k$ derivatives $\frac{d^i\mathbf{x}}{dt^i}\bigr|_{t=t_0}, \frac{d^i\mathbf{x}}{dt^i}\bigr|_{t=t_1}, \dots, \frac{d^i\mathbf{x}}{dt^i}\bigr|_{t=t_N}$ are a-priori assigned, for $i = 1,\dots, k$. For notation convenience, the trajectory for which the first $k$ derivatives are defined, is denoted with $\mathbf{x}_k(t)$.

The objective is to design a feasible trajectory $\mathbf{x}_k(t)$ which satisfies the equality kinematic constraints $\sigma_{i,j}$ expressed by the first $k$ derivatives $\frac{d^i\mathbf{x}}{dt^i}\bigr|_{t=t_j} = \sigma_{i,j} $, for $i=0,1,\dots,k$ and $j=0,1,\dots,N$. 

To do so we propose to use polynomials as the set of our feasible trajectories. In particular, we build the polynomial trajectory as a linear combination of $k+1$ polynomial basis
\begin{equation}
\label{eq:rst_trajectory}
\mathbf{x}_k(t) = \sum_{i=0}^{k}{\mathbf{p}_i(t)},
\end{equation}
where $\mathbf{p}_i(t)$ is the polynomial (for each component of the $3D$ space) \textit{responsible} for the fulfillment of the $i$-th derivative constraint and $k$ is the number of considered derivatives. In other words, each polynomial component $\mathbf{p}_i(t)$ is designed as an additive term which iteratively fulfills the $i$-th kinematic constraint.
Furthermore, we exploit Lagrange interpolating polynomials to compute the polynomial basis $\mathbf{p}_i(t)$.

To the best of authors' knowledge, this is the first work that recursively builds a set of polynomials for trajectory generation under kinematic constraints.






\section{Theoretical formulation of RST}
\label{subsec:rst_results}
This section presents the mathematical foundations behind the choice of the polynomial representation in \eqref{eq:rst_trajectory}. We show that such choice produces a closed form continuous polynomial trajectory and enables an iterative algorithm for its generation. In particular, Lagrange polynomials are exploited. For ease of notation, in the following we consider the trajectory in \eqref{eq:rst_trajectory} as unidimensional, assuming that the $3D$ extension is obtained by working component-wise.%
\subsection{Preliminaries on Lagrange polynomials}
\label{subsec:rst_Lagrange}
Lagrange polynomials offer a technique to interpolation problems. In particular, given a set of $N+1$ $1D$ control points (waypoints) $(t_0,x(t_0)), (t_1,x(t_1)), \dots, (t_N,x(t_N))$, the interpolation polynomial in the Lagrange form is defined as
\begin{equation}
\label{eq:rst_lagrange}
L(t) := \sum_{j=0}^{N}{x(t_j) \ell_j(t)}
\end{equation}
where $\ell_j(t)$ is the Lagrange polynomial basis whose form reads as follows
\begin{equation}
\ell_j(t):= \prod_{\substack{m=0 \\ m\neq j}}^{N}{\frac{t-t_m}{t_j-t_m}},
\end{equation}
with $0\leq j\leq N$. The main idea behind this definition is that, by construction, the $N+1$ basis functions are such that $\ell_j(t_i)\equiv 0$ in $i=0,\dots,N \; \wedge \; i\neq j$. So for each waypoint, only one single basis function contributes to the sum in \eqref{eq:rst_lagrange}.

\subsection{The recursive formula}
To discover the recursive property behind the formulation in \eqref{eq:rst_trajectory}, we start the mathematical derivation introducing two lemmas regarding a particular polynomial choice and its derivatives. Such type of choice enables a remarkable recursive property which is proved in Theorem \ref{theorem:rst_theorem1}, the main result of this chapter. The corollary \ref{corollary:rst_corollary1}, instead, establishes an upper bound for the minimum degree of the polynomial trajectory generated via the recursive formulation.

\begin{lemma}
\label{lemma:rst_Lemma1}
Let
\begin{equation}
a(t) = \prod_{n=0}^{N}{(t-t_n)},
\label{eq:rst_a_t}
\end{equation}
then
\begin{equation}
\frac{d}{dt}a(t) = a(t)\cdot \sum_{n=0}^{N}{\frac{1}{t-t_n}}
\label{eq:rst_da_t}
\end{equation}
and
\begin{equation}
\frac{d}{dt}a(t)\biggr|_{t=t_j} = \prod_{\substack{n=0\\ n\neq j}}^{N}{(t_j-t_n)}.
\label{eq:rst_da_tj}
\end{equation}
\end{lemma}
\begin{proof}
Using product rule
\begin{equation}
\frac{d}{dt}a(t) = \sum_{j=0}^{N}{\prod_{\substack{n=0\\ n\neq j}}^{N}{(t-t_n)}}.
\label{eq:rst_Lemma1}
\end{equation}
Dividing and multiplying by $a(t)$ gives \eqref{eq:rst_da_t}. For $t=t_j$ only one term of the summation in \eqref{eq:rst_Lemma1} contributes leading to \eqref{eq:rst_da_tj}. \qedhere  
\end{proof}

\begin{lemma}
\label{lemma:rst_Lemma2}
Let
\begin{equation*}
a(t) = \prod_{n=0}^{N}{(t-t_n)},
\end{equation*}
and let $s(t)$ be a polynomial. Then
\begin{equation}
\frac{d^h}{dt^h}\biggl[\frac{a^k(t)}{k!}\cdot s(t)\biggr]_{t=t_j} \equiv 0 \; \forall k>h
\label{eq:rst_diff_ak}
\end{equation}
and
\begin{equation}
\frac{d^h}{dt^h}\biggl[\frac{a^h(t)}{h!}\cdot s(t)\biggr]_{t=t_j} = \biggl(\frac{d}{dt}a(t)\biggr)^h\ \biggr|_{t=t_j} \cdot s(t_j).
\label{eq:rst_diff_ah}
\end{equation}
\end{lemma}
\begin{proof}
Since $t=t_j$ is a zero of $a^k(t)$ with multiplicity $k>h$, the factor $(t-t_j)$ appears in every term of the $h$-th derivative. If $k=h$, then
\begin{equation}
\frac{d^h}{dt^h}\biggl[\frac{a^h(t)}{h!}\cdot s(t)\biggr]_{t=t_j} = \frac{d^{h-1}}{dt^{h-1}}\biggl[\frac{d}{dt} \biggl(\frac{a^h(t)}{h!}\cdot s(t)\biggr)\biggr]_{t=t_j}.
\end{equation}
By the linearity of the differential operator and product rule, RHS can be rewritten as
\begin{equation}
\frac{d^{h-1}}{dt^{h-1}}\biggl[\frac{a^{h-1}(t)}{{h-1}!}\cdot \frac{d a(t)}{dt}\cdot s(t)\biggr]_{t=t_j} + \cancel{\frac{d^{h-1}}{dt^{h-1}}\biggl[\frac{a^h(t)}{h!}\cdot \frac{d}{dt}s(t)}\biggr]_{t=t_j}
\end{equation}
where the second term in the above expression vanishes because of the first part of the lemma. Proceeding in the same way until the $h$-th derivative leads to
\begin{equation}
\frac{d^h}{dt^h}\biggl[\frac{a^h(t)}{h!}\cdot s(t)\biggr]_{t=t_j} = \biggl(\frac{d}{dt}a(t)\biggr)^h\ \biggr|_{t=t_j} \cdot s(t_j)
%\label{eq:rst_diff_ah}
\end{equation}
which concludes the proof. \qedhere  
\end{proof}

To understand the terms involved while building the final trajectory $x_k(t)$, the concept of $m$-partial trajectory is introduced.
\begin{defn}
\label{def:rst_def1}
Let $\frac{d^i}{dt^i}x_k(t)\bigr|_{t=t_j}$ be given kinematic constraints, for $i=0,1,\dots, k$. An $m$-partial trajectory is a feasible trajectory which only fulfills the first $m$ kinematic constraints, i.e. 
\begin{equation}
x_m(t) = \sum_{i=0}^{m}{p_i(t)}, \; \text{with }\; m\leq k.
\end{equation} 
\end{defn}
As an example, if the kinematic constraints are set up to the acceleration ($k=2$), a $1$-partial trajectory is a polynomial trajectory that passes through the $N+1$ waypoints with the given velocities.

\begin{theorem}
\label{theorem:rst_theorem1}
Let $t_j$ be a point in time, for $j=0,1,\dots, N$, such that $\frac{d^i}{dt^i}x_k(t)\bigr|_{t=t_j}=\sigma_{i,j}$ is the associated given kinematic constraint, for $i=0,1,\dots, k$. Let $x_k(t)$ be a feasible polynomial trajectory defined as
\begin{equation}
x_k(t) = \sum_{i=0}^{k}{p_i(t)}.
\end{equation}
If
\begin{equation}
p_i(t)=\dfrac{1}{i!}\biggl(\prod_{n=0}^{N}{(t-t_n)}\biggr)^i\cdot s_i(t)
\end{equation}
then the $i$-partial trajectory $x_i(t)$ depends recursively on $x_{i-1}(t)$. In particular,
\begin{equation}
s_i(t_j)=\dfrac{\dfrac{d^i}{dt^i}x_k(t)\biggr|_{t=t_j} -\dfrac{d^i}{dt^i}x_{i-1}(t)\biggr|_{t=t_j}}{\displaystyle \Biggl(\prod_{\substack{n=0\\ n\neq j}}^{N}{(t_j-t_n)}\Biggr)^i}.
\label{eq:rst_recursive}
\end{equation}
\end{theorem}

\begin{proof}
For $i=0$, $x_{-1}(t):=0$ and the kinematic constraint is $x_k(t_j)$ which brings to the first polynomial $s_0(t)=p_0(t)=x_0(t)$ obtained using an interpolation technique, e.g. the Lagrange polynomials as explained in Sec. \ref{subsec:rst_Lagrange}. No other contributions $p_i(t)$ are taken into account since $p_i(t_j)=0 \; \forall i>0$. For brevity of notation it is now convenient to define a polynomial $a(t)$ as $a(t) = \prod_{n=0}^{N}{(t-t_n)}$.

For $i=1$ and constraint $\frac{d}{dt}x_k(t)\bigr|_{t=t_j}$, 
\begin{equation}
x_k(t) = x_0(t) + a(t)\cdot s_1(t) + \sum_{i=2}^{k}{\dfrac{a^i(t)}{i!}s_i(t)},
\label{eq:rst_i1}
\end{equation}
taking the derivative in $t_j$ in both sides of \eqref{eq:rst_i1} yields to
\begin{equation}
\frac{d}{dt}x_k(t)\biggr|_{t=t_j} = \frac{d}{dt}x_0(t)\biggr|_{t=t_j} + \frac{d}{dt}a(t)\biggr|_{t=t_j}\cdot s_1(t_j)
\end{equation}
where, using Lemma \ref{lemma:rst_Lemma2}, no contribution in $t_j$ comes from $i>1$. Rearranging,
\begin{equation}
s_1(t_j)=\dfrac{\dfrac{d}{dt}x_k(t)\biggr|_{t=t_j} -\dfrac{d}{dt}x_0(t)\biggr|_{t=t_j}}{\displaystyle \prod_{\substack{n=0\\ n\neq j}}^{N}{(t_j-t_n)}}
\end{equation}
where the denominator is a consequence of Lemma \ref{lemma:rst_Lemma1}.
To compute $s_1(t)$, it is convenient to interpolate the points $s_1(t_j)$ again with Lagrange polynomials. The $1$-partial trajectory has expression $x_1(t)=x_0(t)+a(t)\cdot s_1(t)$.

In general, for $i=h$ the constraint to be fulfilled is $\frac{d^h}{dt^h}x_k(t)\bigr|_{t=t_j}$. Thus
\begin{equation}
x_k(t) = x_{h-1}(t) + \dfrac{a^h(t)}{h!}\cdot s_h(t) + \sum_{i=h+1}^{k}{\dfrac{a^i(t)}{i!}\cdot s_i(t)},
\label{eq:rst_ih}
\end{equation}
where $x_{h-1}(t)$ is the $(h-1)$-partial trajectory. Taking the $h$-th derivative in $t_j$ in both sides of \eqref{eq:rst_ih} and using again Lemma \ref{lemma:rst_Lemma2} for the second and third term, yields to
\begin{equation}
\frac{d^h}{dt^h}x_k(t)\biggr|_{t=t_j} = \frac{d^h}{dt^h}x_{h-1}(t)\biggr|_{t=t_j} + \biggl(\frac{d}{dt}a(t)\biggr)^h\ \biggr|_{t=t_j} \cdot s_h(t_j).
\end{equation}
Finally, rearranging
\begin{equation}
s_h(t_j)=\dfrac{\dfrac{d^h}{dt^h}x_k(t)\biggr|_{t=t_j} -\dfrac{d^h}{dt^h}x_{h-1}(t)\biggr|_{t=t_j}}{\displaystyle \Biggl(\prod_{\substack{n=0\\ n\neq j}}^{N}{(t_j-t_n)}\Biggr)^h}
\end{equation}
concludes the proof. \\ \qedhere  
\end{proof}

Theorem \ref{theorem:rst_theorem1} provides a recursive formula to evaluate the points $s_i(t_j)$. Hence, it is sufficient to interpolate them, for example with Lagrange polynomials, in order to get $s_i(t)$ at each iteration. As a remark, the interpolation of $s_i(t_j)$ can be carried out also by adopting other polynomial basis functions, e.g., Newton polynomials, B-Spline, etc. Nevertheless, the polynomial assumption comes from Lemma \ref{lemma:rst_Lemma2}. Indeed, polynomials, when differentiated, do not introduce extra poles which could cancel with $a(t)$. On the contrary, other basis functions do not guarantee this property.

The following corollary provides an information on the minimum degree of the polynomial trajectory $x_k(t)$ defined in \eqref{eq:rst_trajectory}.

\begin{corollary}
\label{corollary:rst_corollary1}
Let $t_j$ be a point in time, for $j=0,1,\dots, N$, such that $\frac{d^i}{dt^i}x_k(t)\bigr|_{t=t_j}$ is the associated given kinematic constraint, for $i=0,1,\dots, k$. If $x_k(t)$ is a feasible polynomial trajectory defined as
\begin{equation}
x_k(t) = \sum_{i=0}^{k}{\dfrac{1}{i!}\biggl(\prod_{n=0}^{N}{(t-t_n)}\biggr)^i\cdot s_i(t)},
\end{equation}
then the minimum degree of $x_k(t)$ is less or equal to $(k+1)(N+1)-1$.
\end{corollary}
\begin{proof}
Each $s_i(t)$ is a Lagrange polynomial that passes through $N+1$ points, therefore its minimum degree is $N$. The highest contribution in terms of degree to $x_k(t)$ comes when $i=k$, so that $\prod_{n=0}^{N}{(t-t_n)^k}$ is a polynomial of degree equal to $(N+1)k$. Thus, the product of $\prod_{n=0}^{N}{(t-t_n)^k}$ and $s_k(t)$ gives a polynomial whose minimum degree is at most $(N+1)k+N = (k+1)(N+1)-1$.
This is somehow consistent with the idea that imposing $k+1$ constraints for each of the points in time $t_j$ gives $(k+1)(N+1)$ constraints and the minimum degree of a polynomial that satisfies them is $(k+1)(N+1)-1$. \qedhere
\end{proof}

\subsection{Recursive smooth trajectory generation}
\label{subsec:rst_code}

\begin{figure}[t]
\includegraphics[scale=0.40]{images/extra/RST_all}
\centering
\caption{RST block diagram.}
\label{fig:rst_RST}
\end{figure}
In the following, we will denote our approach as recursive smooth trajectory (RST) generation. The idea behind it, is that each component $p_i(t)$ in \eqref{eq:rst_trajectory} guarantees that $\dfrac{d^i}{dt^i}x_k(t)\bigr|_{t=t_j}=\sigma_{i,j}$ is fulfilled. Starting from the waypoints constraint which can be easily calculated through Lagrange polynomials, all the following higher-order differential kinematic constraints depend recursively on previous ones, according to \eqref{eq:rst_recursive}. 

The pseudocode in Alg.~\ref{alg:RST} provides a practical idea on how to iteratively design the trajectory $x_k(t)$ under the conditions aforementioned. The scheme in Fig.~\ref{fig:rst_RST} illustrates how RST operates.



We now discuss the influence of the distribution of the time instants $t_j$ for $j=0,\dots,N$ over the trajectory on oscillations and numerical limitations by proposing a known solution (Chebyshev nodes) and a novel hybrid solution, denoted as blockwise recursive smooth trajectory (BRST) that tackle said practical issues.

\begin{algorithm}
\caption{Recursive smooth trajectory (RST) generation}
\label{alg:RST}
\begin{algorithmic}[1]
\Inputs{$N+1$ points in time $t_0<t_1<\dots<t_N$; \\ Number of derivatives $k$ to fulfill; \\ Kin. constr. $\frac{d^i}{dt^i}x_k(t)\bigr|_{t=t_0}, \dots, \frac{d^i}{dt^i}x_k(t)\bigr|_{t=t_N}$.}
\Initialize{$a(t)=(t-t_0)\cdots (t-t_N)$ polynomial; \\ $x_{-1}(t) \equiv 0$.}
\For{$i=0$ to $k$}
	\For{$j=0$ to $N$}
		\State $s_i(t_j)=\dfrac{\dfrac{d^i}{dt^i}x_k(t)\biggr|_{t=t_j} -\dfrac{d^i}{dt^i}x_{i-1}(t)\biggr|_{t=t_j}}{\displaystyle \Biggl(\dfrac{d}{dt}a(t)\biggr|_{t=t_j}\Biggr)^i}$;
	\EndFor
	\State Interpolate $s_i(t_j)$ with Lagrange polynomial $s_i(t)$;
	\State $x_i(t) = x_{i-1}(t)+\dfrac{a^i(t)}{i!}\cdot s_i(t)$;
\EndFor
\end{algorithmic}
\end{algorithm}

\subsection{Remarks and insights on blockwise approach}
In the previous sections, we presented the formal generation approach (RST) of $x_k(t)$ for points in time $t_0<t_1< \dots <t_N$ with no constraints on $N$ and on the length of the interval $I_j = t_{j+1}-t_j$ for $j = 0,\dots, N-1$. 
Unfortunately, when the number of points $N+1$ is large and in particular when points in time $t_j$ are equally spaced ($I_j$ is constant), the Runge's phenomenon may occur \cite{Runge}. The Runge's phenomenon is an oscillation problem near the endpoints of the polynomial interpolation function, as illustrated in Fig. \ref{fig:rst_Runge}. To overcome it, one could either move to spline interpolation as mentioned in Sec. \ref{subsec:rst_introduction} or change the distribution of the nodes $t_j$ more densely towards the edges of the interval $[t_0, t_N]$ as proposed in \cite{Berrut}.
In the latter case, a standard choice considers the set of points in time as the set of Chebyshev nodes. In particular, for $N+1$ points in the interval $[t_0, t_N]$, nodes are transformed into
\begin{equation}
\hat{t}_j = \frac{1}{2}(t_0+t_N) + \frac{1}{2}(t_N-t_0)\cos\biggl[\dfrac{2j+1}{2(N+1)}\pi\biggr], \; j = 0,\dots, N.
\label{Cheby}
\end{equation}

Moreover, when the number of points $N+1$ increases, we encounter computational limitations in evaluating the powers $t^{(k+1)(N+1)-1}$. In such a case, a possible new blockwise approach that we name blockwise RST (BRST) concatenates intervals 
\begin{equation}
    [t_{0,1},t_{N,1}], [t_{0,2},t_{N,2}], \dots, [t_{0,M},t_{N,M}]
\end{equation} 
in $M$ blocks. For each block, the associated trajectory is separately calculated as described in Sec. \ref{subsec:rst_code}. Since kinematic constraints are intrinsically considered in the formulation of the trajectory, interfaces are already jointly matched (up to $k$-th derivative) without any need of optimization steps. Finally, when $N=1$, we bring blockwise back to piecewise polynomial trajectories under the recursive framework and we will refer to it as piecewise RST (PRST) algorithm. It is important to notice that the PRST approach provides a piecewise trajectory that is exactly the same as the one generated using the classic spline interpolation method. However, the two methods are intrinsically different: PRST finds the piecewise trajectory recursively by building $k+1$ partial trajectories, while the spline interpolation technique solves a system of linear equations, thus it needs a matrix inversion. A quick comparison of the computational cost is presented in Sec. \ref{subsec:rst_complexity}, while a more detailed study is left for future research, since the most efficient implementation of the RST needs to be studied.
\section{Trajectory perturbation}
\label{subsec:rst_perturbation}
In this section, we present the formal procedure in order to deal with uncertainties in the kinematic constraints. For notation convenience, we will denote $\frac{d^i}{dt^i}\tilde{x}_k(t)\bigr|_{t=t_j}$ as the perturbed constraint. 

\subsection{Uncertainty model}
We model the uncertainty in the constraints as an additive contribution $\varepsilon_i$ to the fixed deterministic part, in particular
\begin{equation}
\label{eq:rst_perturbation}
\frac{d^i}{dt^i}\tilde{x}_k(t)\bigr|_{t=t_j} = \underbrace{\frac{d^i}{dt^i}x_k(t)\bigr|_{t=t_j}}_{\text{deterministic}}+\underbrace{\varepsilon_i(t_j)}_{\text{stochastic}},
\end{equation}
where $\varepsilon_i\sim \mathcal{N}(0,\sigma^2_{i}(t_j))$ is a Gaussian random variable with zero mean and variance $\sigma^2_{i}(t_j)$. The uncertainty models the deviation from the expected value of the kinematic constraint.
As an example, \eqref{eq:rst_perturbation} can be used to analyze how the noise in the E-FMS system affects the kinematic constraints, thus the polynomial trajectory.
Due to the intrinsic linearity of RST and the perturbation model, the following theorem proves that it is possible to translate the uncertainties in the constraints into uncertainties in the polynomial coefficients. 


\begin{theorem}
\label{lemma:rst_Lemma6}
Let $\frac{d^i}{dt^i}\tilde{x}_k(t)\bigr|_{t=t_j}$ be the perturbed kinematic constraints for $i=0,1,\dots, k$ and let $x_k(t)$ be a feasible polynomial trajectory computed with RST. Then $\tilde{x}_k(t)$ represents the perturbed polynomial trajectory and it can be written as
\begin{equation}
\tilde{x}_k(t) = x_k(t)+r_k(t),
\end{equation} 
where $r_k(t)$ is a random polynomial whose coefficients belong to a multivariate Gaussian distribution $\mathcal{N(\mathbf{\mu,\Sigma})}$.
\end{theorem}

\begin{proof}
To prove the theorem we will proceed by induction on the ($i-1$)-partial trajectory. Consider the base case when $i=0$, i.e., $\tilde{x}_0(t)$. From Sec. \ref{subsec:rst_Lagrange}, 
\begin{equation}
\label{eq:rst_lagrange_perturbated}
\tilde{x}_0(t) = \sum_{j=0}^{N}{\tilde{x}(t_j) \ell_j(t)}
\end{equation}
where $\ell_j(t)$ is the Lagrange basis. By substituting the perturbation expression for the constraints \eqref{eq:rst_perturbation} in \eqref{eq:rst_lagrange_perturbated}, it follows that 
\begin{align}
\tilde{x}_0(t) & = \sum_{j=0}^{N}{x(t_j) \ell_j(t)}+\sum_{j=0}^{N}{\varepsilon_0(t_j) \ell_j(t)} \nonumber \\ 
& = x_0(t)+r_0(t).
\end{align}

Suppose that the statement of the theorem is true for the $(i-1)$-partial trajectory, which means that
\begin{equation}
\tilde{x}_{i-1}(t)= x_{i-1}(t)+r_{i-1}(t).
\end{equation}
Then, it is true also for the $i$-partial trajectory. Indeed, from the RST algorithm derived in Theorem \ref{theorem:rst_theorem1},
\begin{equation}
\tilde{s}_i(t_j)=\dfrac{\dfrac{d^i}{dt^i}\tilde{x}_k(t)\biggr|_{t=t_j} -\dfrac{d^i}{dt^i}\tilde{x}_{i-1}(t)\biggr|_{t=t_j}}{\displaystyle \Biggl(\prod_{\substack{n=0\\ n\neq j}}^{N}{(t_j-t_n)}\Biggr)^i},
\label{eq:rst_perturbed_recursive}
\end{equation}
and using the induction hypothesis and the linearity of the differential operator,

\begin{align}
\tilde{s}_i(t_j)& = \dfrac{\Biggl(\dfrac{d^i}{dt^i}x_k(t)\biggr|_{t=t_j}+\varepsilon_i(t_j)\Biggr) -\Biggl(\dfrac{d^i}{dt^i}x_{i-1}(t)\biggr|_{t=t_j}+\dfrac{d^i}{dt^i}r_{i-1}(t)\biggr|_{t=t_j}\Biggr)}{\displaystyle \Biggl(\prod_{\substack{n=0\\ n\neq j}}^{N}{(t_j-t_n)}\Biggr)^i} \nonumber \\
&= \dfrac{\dfrac{d^i}{dt^i}x_k(t)\biggr|_{t=t_j}-\dfrac{d^i}{dt^i}x_{i-1}(t)\biggr|_{t=t_j}}{\displaystyle \Biggl(\prod_{\substack{n=0\\ n\neq j}}^{N}{(t_j-t_n)}\Biggr)^i}+\dfrac{\varepsilon_i(t_j)-\dfrac{d^i}{dt^i}r_{i-1}(t)\biggr|_{t=t_j}}{\displaystyle \Biggl(\prod_{\substack{n=0\\ n\neq j}}^{N}{(t_j-t_n)}\Biggr)^i} \nonumber \\
&= s_i(t_j)+s_i^{\varepsilon}(t_j).
\end{align}

This result is important because it separates $\tilde{s}_i(t_j)$ in two components. Using again Lagrange interpolation as done for $i=0$ yields to $\tilde{s}_i(t) = s_i(t)+s_i^{\varepsilon}(t)$. By the RST properties and definition of $i$-partial trajectory 
\begin{align}
\tilde{x}_i(t) &= \tilde{x}_{i-1}(t)+\frac{a^i(t)}{i!}\cdot \tilde{s}_i(t) \nonumber \\
&= x_{i-1}(t)+r_{i-1}(t)+ \frac{a^i(t)}{i!}\cdot (s_i(t)+s_i^{\varepsilon}(t)) \nonumber \\
&= x_{i-1}(t)+\frac{a^i(t)}{i!}\cdot s_i(t) + r_{i-1}(t)+ \frac{a^i(t)}{i!}\cdot s_i^{\varepsilon}(t) \nonumber \\
&= x_i(t)+r_i(t).
\label{eq:rst_derivation_perturbation}
\end{align}
Calculating \eqref{eq:rst_derivation_perturbation} in $i=k$ concludes the proof because $r_k(t)$ is a polynomial whose coefficients are a weighted sum of the uncertainties $\varepsilon_i$ in the constraints, for $i=0,1,\dots,k$. The way in which the random polynomial coefficients depend one to each other is described by the covariance matrix $\mathbf{\Sigma}$ of a multivariate Gaussian distribution, which has to be estimated.
\end{proof}
To characterize and generate new perturbed trajectories, an estimation of the multivariate Gaussian distribution described by the random coefficients in $r_k(t)$ has to be carried out.

\subsection{Coefficients estimation}
The coefficients of the random polynomial $r_k(t)$ incorporate the stochastic information of the constraints. 
Under the Gaussian hypothesis, it is straightforward to state that the coefficients are themselves Gaussian random variables since they are the outcome of linear combinations of $\varepsilon_i(t_j)$. However, the way in which the random variables $\varepsilon_i(t_j)$ interact and correlate one to each other strongly depends on the points in time $t_j$, in particular on the differences $t_j-t_n$, for $j=0,\dots,N$ and $n=0,\dots,N$ with $j\neq n$. Therefore, no closed form expression is available to describe the covariance matrix $\mathbf{\Sigma}$. Another approach consists of estimating $\mathbf{\Sigma}$, and we will refer to $\mathbf{\hat{\Sigma}}$ as the estimated version.

To compute $\mathbf{\hat{\Sigma}}$, the procedure involves the following steps:
\begin{itemize}
\item take $P$ realizations of the uncertainty in the constraints $\varepsilon_i(t_j)\sim \mathcal{N}(0,\sigma^2_{i}(t_j))$;
\item for each realization, evaluate the coefficients of the trajectory $r_{k}(t)$ generated via RST;
\item evaluate the sample covariance matrix $\mathbf{\hat{\Sigma}}$ of the stored coefficients.
\end{itemize}
The sample covariance matrix $\mathbf{\hat{\Sigma}}$ is the unbiased estimator of the covariance matrix $\mathbf{\Sigma}$.

\subsection{Random trajectory generation}
The multivariate Gaussian distribution is characterized by the mean (in this case $\mathbf{\mu} = \mathbf{0}$) and the covariance matrix $\mathbf{\Sigma}$. To generate new polynomial coefficients, thus new feasible trajectories, it is enough to sample from the multivariate distribution. In particular, consider a vector $\mathbf{z}$ of uncorrelated normal random variables. If the matrix $\mathbf{C}$ is a square root of $\mathbf{\Sigma}$, such as $\mathbf{C}\cdot \mathbf{C}^T = \Sigma$ (for example using the Cholesky decomposition), it follows that $\mathbf{y}=\mathbf{\mu} + \mathbf{C}\cdot \mathbf{z}$ is a vector of Gaussian random variables representing the coefficients of the random polynomial $r_k(t)$.


\begin{algorithm}
\caption{Recursive smooth random trajectory (RSRT) generation}
\label{alg:RSRT}
\begin{algorithmic}[1]
\Inputs{$N+1$ points in time $t_0<t_1<\dots<t_N$; \\ Kin. constr. $\frac{d^i}{dt^i}x_k(t)\bigr|_{t=t_0}, \dots, \frac{d^i}{dt^i}x_k(t)\bigr|_{t=t_N}$; \\ Uncertainties $\varepsilon_i(t_j)$.}
\Initialize{$a(t)=(t-t_0)\cdots (t-t_N)$ polynomial; \\ $P$ number of realizations; \\ $x_{-1}(t) \equiv 0$ \\ $r_{-1}(t) \equiv 0$.}
\State Compute $x_k(t)$ with RST;
\For{$p=1$ to $P$}
		\State Get realizations of $\varepsilon_i(t_j)$;
		\State Compute $r_{k,p}(t)$ with RST and constraints $\varepsilon_i(t_j)$;
	\EndFor
	\State Estimate $\mathbf{\hat{\Sigma}}$ of the coefficients of $r_k(t)$;
	\State Generate a set of coefficients of $r_k(t)$;
		\State Generate a perturbed trajectory $\tilde{x}_k(t)=x_k(t)+r_k(t)$;
\end{algorithmic}
\end{algorithm}

This process offers a fast methodology for generating perturbed trajectories $\tilde{x}_k(t)=x_k(t)+r_k(t)$ from the estimated coefficients in $r_k(t)$. We will refer to it as recursive smooth random trajectory (RSRT) generation. The algorithm is described in Alg.~\ref{alg:RSRT}

So far we have only considered a multivariate Gaussian distribution for $\varepsilon_i(t_j)$. Nevertheless as a consequence of the central limit theorem, whenever the uncertainties have different distribution, correlated Gaussian random variables can approximate the statistics of the polynomial coefficients of $r_k(t)$.
\section{RST in an optimization framework}
\label{subsec:rst_optimization}
Most of trajectory generation and path planning research concentrates in finding an optimal trajectory that minimizes a cost function $J(\cdot)$ under given constraints. Nevertheless, trajectory generation does not necessarily require optimality in the solution. In this section, we present an example of optimization framework built around the RST algorithm and we prove that when the number of waypoints $N+1$ is equal to $2$, RST (or PRST) directly provides the optimal solution in terms of minimum integral of the $p$-th derivative of the position squared, matching the trajectory generated by minimum-snap algorithm \cite{5980409} without any use of quadratic programming.

Sec.~\ref{subsec:rst_results} illustrated the RST algorithm, which is able to generate a polynomial trajectory $x_k(t)$ with minimum degree that satisfies the constraints. However, it is easy to notice that the general set of feasible trajectories is induced by $q(t)$ as follows
\begin{equation}
x_{\text{ext}}(t) = x(t)+\frac{a^{k+1}(t)}{(k+1)!}\cdot q(t),
\end{equation}
where $q(t)$ is a polynomial which introduces extra degrees of freedom needed for an optimization phase and $x(t)=x_k(t)$ is the trajectory generated via RST. 

As an example of an optimization problem, let $p = k+1$ be the order of the derivative of $x_{\text{ext}}(t)$ whose energy has to be minimized. A possible approach finds the solution to
\begin{equation}
\min_{q(t)}{\int_{t_0}^{t_N}{\biggl|\biggl|\frac{d^p}{dt^p}\biggl(x(t)+\frac{a^p(t)}{p!}\cdot q(t)\biggr)\biggr|\biggr|^2 dt}}
\end{equation}
with $q(t)$ polynomial, providing the optimal trajectory as
\begin{equation}
x_{\text{opt}}(t)=x(t)+\frac{a^p(t)}{p!}\cdot q_{\text{opt}}(t).
\end{equation}
Since all the functions inside the functional are polynomials, the coefficients of $q(t)$ can be in principle expressed analytically by integrating polynomials and by solving a system of linear equations. The convexity of the norm squared function guarantees a global minimum. 

When the number of waypoints is equal to $2$, that is $N=1$, the following Lemma asserts the optimality (in terms of energy) of the trajectory $x_k(t)$ generated with RST.

\begin{lemma}
\label{lemma:rst_Lemma5}
Let $x_k(t)$ be the trajectory generated with RST which satisfies the given kinematic constraints $\frac{d^i}{dt^i}x_k(t)\bigr|_{t=t_j}$ for $i=0,1,\dots, k$ and $j=0,1,\dots, N$. If $N=1$ and $p=k+1$, then the solution to
\begin{equation}
\label{prob:rst_functional}
\min_{q(t)}{\int_{t_0}^{t_1}{\biggl|\biggl|\frac{d^p}{dt^p}\biggl(x_{p-1}(t)+\frac{a^p(t)}{p!}\cdot q(t)\biggr)\biggr|\biggr|^2 dt}}
\end{equation}
is $q_{\text{opt}}(t)=0$, therefore the trajectory generated with RST is already the optimal one.
\end{lemma}

\begin{proof}
The proof uses some concepts of calculus of variations. In particular, let $\mathcal{L}$ be a Lagrangian function defined as 
\begin{equation}
\label{eq:rst_Lagrangian}
\mathcal{L} = \biggl(\frac{d^p x_{p-1}(t)}{dt^p}+\frac{d^p f(t)}{dt^p}\biggr)^2,
\end{equation}
with 
\begin{equation}
f(t) = \frac{a^p(t)}{p!}\cdot q(t).
\end{equation}
From calculus of variations theory, solving problem \eqref{prob:rst_functional} is equal to solving the Euler-Lagrange equation
\begin{equation}
\small
\label{eq:rst_EL}
\frac{\partial \mathcal{L}}{\partial f} - \frac{d}{dt}\biggl(\frac{\partial \mathcal{L}}{\partial \dot{f}}\biggr)+ \frac{d^2}{dt^2}\biggl(\frac{\partial \mathcal{L}}{\partial \ddot{f}}\biggr)-\dots+(-1)^{p}\frac{d^{p}}{dt^{p}}\biggl(\frac{\partial \mathcal{L}}{\partial f^{(p)}}\biggr)=0
\end{equation}
and by substituting the Lagrangian defined in~\eqref{eq:rst_Lagrangian} into~\eqref{eq:rst_EL}
it follows that 
\begin{equation}
\frac{d^{2p}}{dt^{2p}}\biggl(x_{p-1}(t)+f(t)\biggr)=0.
\end{equation}
From the considerations in Corollary \ref{corollary:rst_corollary1}, 
\begin{align}
\text{deg}(x_{p-1}(t))&=2p-1, \nonumber \\
\text{deg}(f(t))&= \text{deg}(a(t))+\text{deg}(q(t)) = 2p+Q,
\end{align}
with $Q\geq 0$. But, since $x$ and $f$ are polynomials, each differentiation reduces the degree by one and
\begin{equation}
\text{deg}\Bigg(\frac{d^{2p}}{dt^{2p}}\biggl(x_{p-1}(t)+f(t)\biggr)\Biggr)=Q=0,
\end{equation}
therefore $\text{deg}(q(t))=Q=0$ and in particular $q(t)\equiv 0$. \qedhere
\end{proof}
When the number of blocks $M$ increases, the overall optimal trajectory is obtained by optimizing the trajectories in each block.
When the number of waypoints in a single block is greater than $2$, the intrinsic optimality of the trajectory generated with RST is not guaranteed anymore and the optimization process provides $q(t)\neq 0$.  Next section illustrates trajectories generated via RST and via optimization of the integral of the $p$-th derivative of the position squared, denoted with RST$_{\text{opt}}$. 

\section{The flag trick in action}\label{sec:examples}
In this section, we provide some applications of the flag trick to several learning problems. We choose to focus on subspace recovery, trace ratio and spectral clustering problems. Other ones, like domain adaptation, matrix completion and subspace tracking are developed or mentioned in the last subsection but not experimented for conciseness.

\subsection{Outline and experimental setting}
For each application, we first present the learning problem as an optimization on Grassmannians. Second, we formulate the associated flag learning problem by applying the flag trick (Definition~\ref{def:flag_trick}). Third, we optimize the problem on flag manifolds with the steepest descent method (Algorithm~\ref{alg:GD})---more advanced algorithms are also derived in the appendix. 
Finally, we perform various nestedness and ensemble learning experiments via Algorithm~\ref{alg:flag_trick} on both synthetic and real datasets.

The general methodology to compare Grassmann-based methods to flag-based methods is the following one. For each experiment, we first choose a flag signature $~{q_{1\rightarrow d} := (q_1, \dots, q_d)}$, then we run independent optimization algorithms on $\Gr(p, q_1), \dots, \Gr(p, q_d)$~\eqref{eq:subspace_problem} and finally we compare the optimal subspaces $\S_k^* \in \Gr(p, q_k)$ to the optimal flag of subspaces $\Sf^* \in \Fl(p, q_{1\rightarrow d})$ obtained via the flag trick~\eqref{eq:flag_problem}. 
To show the nestedness issue in Grassmann-based methods, we compute the subspace distances $\Theta(\S_k^*, \S_{k+1}^*)_{k=1\dots d-1}$, where $\Theta$ is the generalized Grassmann distance of~\citet[Eq.~(14)]{ye_schubert_2016}. It consists in the $\ell_2$ norm of the principal angles, which can be obtained from the singular value decomposition (SVD) of the inner-products matrices ${U_k}\T {U_{k+1}}$, where $U_k \in \St(p, q_k)$ is an orthonormal basis of $\S_k^*$.

Regarding the implementation of the steepest descent algorithm on flag manifolds (Algorithm~\ref{alg:GD}), we develop a new class of manifolds in \href{https://pymanopt.org/}{PyManOpt}~\citep{boumal_manopt_2014,townsend_pymanopt_2016}, and run their \href{https://github.com/pymanopt/pymanopt/blob/master/src/pymanopt/optimizers/steepest_descent.py}{SteepestDescent} algorithm. Our implementation of the \texttt{Flag} class is based on the Stiefel representation of flag manifolds, detailed in \autoref{sec:flags}, with the retraction being the polar retraction. For the computation of the gradient, we use automatic differentiation with the \texttt{\href{https://github.com/HIPS/autograd}{autograd}} package. We could derive the gradients by hand from the expressions we get, but we use automatic differentiation as strongly suggested in PyManOpt's \href{https://pymanopt.org/docs/stable/quickstart.html}{documentation}.
Finally, the real datasets and the machine learning methods used in the experiments can be found in \href{https://scikit-learn.org/stable/}{scikit-learn}~\citep{pedregosa_scikit-learn_2011}.

\section{Robust subspace recovery: extensions and proofs}\label{app:RSR}

\subsection{An IRLS algorithm for robust subspace recovery}
Iteratively reweighted least squares (IRLS) is a ubiquitous method to solve optimization problems involving $L^p$-norms. Motivated by the computation of the geometric median~\citep{weiszfeld_sur_1937}, IRLS is highly used to find robust maximum likelihood estimates of non-Gaussian probabilistic models (typically those containing outliers) and finds application in robust regression~\citep{huber_robust_1964}, sparse recovery~\citep{daubechies_iteratively_2010} etc.

The recent fast median subspace (FMS) algorithm~\citep{lerman_fast_2018}, achieving state-of-the-art results in RSR uses an IRLS scheme to optimize the Least Absolute Deviation (LAD)~\eqref{eq:RSR_Gr}.
The idea is to first rewrite the LAD as 
\begin{equation}
	\sum_{i=1}^n \norm{x_i - \Pi_{\S} x_i}_2 = \sum_{i=1}^n w_i(\S) \norm{x_i - \Pi_{\S} x_i}_2^2,
\end{equation}
with $w_i(\S) = \frac{1}{\norm{x_i - \Pi_{\S} x_i}_2}$, and then successively compute the weights $w_i$ and update the subspace according to the weighted objective.
More precisely, the FMS algorithm creates a sequence of subspaces $\S^1, \dots, \S^m$ such that 
\begin{equation}\label{eq:IRLS_FMF}
    \S^{t+1} = \argmin{\S \in \Gr(p, q)} \sum_{i=1}^n w_i(\S^t) \norm{x_i - \Pi_\S x_i}_2^2.
\end{equation}
This weighted least-squares problem enjoys a closed-form solution which relates to the eigenvalue decomposition of the weighted covariance matrix $\sum_{i=1}^n w_i(\S^t) x_i {x_i}\T$~\citep[Chapter~3.3]{vidal_generalized_2016}.

We wish to derive an IRLS algorithm for the flag-tricked version of the LAD minimization problem~\eqref{eq:RSR_Fl}.
In order to stay close in mind to the recent work of \citet{peng_convergence_2023} who proved convergence of a general class of IRLS algorithms under some mild assumptions, we first rewrite~\eqref{eq:RSR_Fl} as
\begin{equation}~\label{eq:RSR_Fl_IRLS}
    \argmin{\S_{1:d} \in\Fl(p, \qf)} \sum_{i=1}^n \rho(r(\S_{1:d}, x_i)),
\end{equation}
where $r(\S_{1:d}, x) = \norm{x - \Pi_{\Sf} x}_2$ is the \textit{residual} and $\rho(r) = |r|$ is the \textit{outlier-robust} loss function.
Following~\citet{peng_convergence_2023}, the IRLS scheme associated with~\eqref{eq:RSR_Fl_IRLS} is:
\begin{equation}
\begin{cases}
w_i^{t+1} = \rho'(r(\S_{1:d}^t, x_i)) /  r(\S_{1:d}^t, x_i) = 1 / \norm{x_i - \Pi_{\Sf} x_i}_2,\\
(\S_{1:d})^{t+1} = \argmin{\S_{1:d} \in\Fl(p, \qf)} \sum_{i=1}^n w_i^{t+1} \norm{x_i - \Pi_{\Sf} x_i}_2^2.
\end{cases}
\end{equation}
We now show that the second step enjoys a closed-form solution.
\begin{theorem}\label{thm:IRLS_FMF}
The RLS problem
\begin{equation}
    \argmin{\S_{1:d} \in\Fl(p, \qf)} \sum_{i=1}^n w_i \norm{x_i - \Pi_{\Sf} x_i}_2^2
\end{equation}
has a closed-form solution $\S_{1:d}^* \in\Fl(p, \qf)$, which is given by the eigenvalue decomposition of the weighted sample covariance matrix $S_w = \sum_{i=1}^n w_i x_i {x_i}\T = \sum_{j=1}^p \ell_j v_j {v_j}\T$, i.e.
\begin{equation}
    \S_k^* = \operatorname{Span}(v_1, \dots, v_{q_k}) \quad (k=1\twodots d).
\end{equation}
\end{theorem}
\begin{proof}
One has
\begin{equation}
	\sum_{i=1}^n w_i \norm{x_i - \Pi_{\Sf} x_i}_2^2 = \tr{(I - \Pi_{\Sf})^2 \lrp{\sum_{i=1}^n w_i x_i {x_i}\T}}.
\end{equation}
Therefore, we are exactly in the same case as in \autoref{thm:flag_trick}, if we replace $X X\T$ with the reweighted covariance matrix $\sum_{i=1}^n w_i x_i {x_i}\T$. This does not change the result, so we conclude with the end of the proof of \autoref{thm:flag_trick} (which itself relies on~\citet{szwagier_curse_2024}).
\end{proof}
Hence, one gets an IRLS scheme for the LAD minimization problem. 
One can modify the robust loss function $\rho(r) = |r|$ by a Huber-like loss function to avoid weight explosion. Indeed, one can show that the weight $w_i := 1 / \norm{x_i - \Pi_{\Sf} x_i}_2$ goes to infinity when the first subspace $\S_1$ of the flag gets close to $x_i$ .
Therefore in practice, we take 
\begin{equation}
    \rho(r) = 
        \begin{cases}
            r^2 / (2 p \delta) & \text{if } |r| <= p\delta,\\
            r - p \delta / 2 & \text{if } |r| > p\delta.
        \end{cases}
\end{equation}
This yields
\begin{equation}
    w_i = 1 / \max\lrp{p\delta,  1 / \norm{x_i - \Pi_{\Sf} x_i}_2}.
\end{equation}
The final proposed scheme is given in Algorithm~\ref{alg:FMF}, named \textit{fast median flag} (FMF), in reference to the fast median subspace algorithm of~\citet{lerman_fast_2018}.
\begin{algorithm}
\caption{Fast median flag}\label{alg:FMF}
\begin{algorithmic}
\Require $X\in \R^{p\times n}$ (data), $\quad q_1 < \dots < q_d$ (signature), $\quad t_{max}$ (max number of iterations), $\quad \eta$ (convergence threshold), $\quad \varepsilon$ (Huber-like saturation parameter)
\Ensure
$U \in \St(p, q)$
\State $t \gets 0, \quad \Delta \gets \infty, \quad U^0 \gets \operatorname{SVD}(X, q)$
\While{$\Delta > \eta$ and $t < t_{max}$}
    \State $t \gets t+1$
    \State $r_i \gets \norm{x_i - \Pi_{\Sf} x_i}_2$
    \State $y_i \gets {x_i} / {\max(\sqrt{r_i}, \varepsilon)}$
    \State $U^t \gets \operatorname{SVD(Y, q)}$
    \State $\Delta \gets \sqrt{\sum_{k=1}^{d} \Theta(U^t_{q_k}, U^{t-1}_{q_k})^2}$
\EndWhile
\end{algorithmic}
\end{algorithm}
We can easily check that FMF is a direct generalization of FMS for Grassmannians (i.e. when $d=1$).


\begin{remark}
This is far beyond the scope of the paper, but we believe that the convergence result of~\citet[Theorem~1]{peng_convergence_2023} could be generalized to the FMF algorithm, due to the compactness of flag manifolds and the expression of the residual function $r$.
\end{remark}

\subsection{Proof of Proposition~\ref{prop:RSR}}
Let $\Sf \in \Fl(p, \qf)$ and $U_{1:d+1} := [U_1|U_2|\dots|U_d|U_{d+1}] \in \O(p)$ be an orthogonal representative of $\Sf$. One has:
\begin{align}
	\norm{x_i - \Pi_{\S_{1:d}} x_i}_2 &= \sqrt{{(x_i - \Pi_{\S_{1:d}} x_i)}\T (x_i - \Pi_{\S_{1:d}} x_i)},\\
	 &= \sqrt{{x_i}\T {(I_p - \Pi_{\S_{1:d}})}^2 x_i},\\
	 &= \sqrt{{x_i}\T {\lrp{I_p - \frac1d \sum_{k=1}^d\Pi_{\S_k}}}^2 x_i},\\
 	 &= \sqrt{\frac1{d^2} {x_i}\T {\lrp{\sum_{k=1}^d (I_p - \Pi_{\S_k})}}^2 x_i},\\
 	 % &= \sqrt{\frac1{d^2} {x_i}\T U_{1:d+1} \diag{0, 1, \dots, d-1, d}^2 {U_{1:d+1}}\T  x_i},\\
 	 &= \sqrt{\frac1{d^2} {x_i}\T \lrp{\sum_{k=1}^{d+1} (k-1) U_k {U_k}\T}^2  x_i},\\
 	 &= \sqrt{\frac1{d^2} {x_i}\T \lrp{\sum_{k=1}^{d+1} (k-1)^2 U_k {U_k}\T}  x_i},\\
 	 &= \sqrt{\sum_{k=1}^{d+1} \lrp{\frac {k-1} {d}}^2 {x_i}\T \lrp{ U_k {U_k}\T}  x_i},\\
  	 \norm{x_i - \Pi_{\S_{1:d}} x_i}_2 &= \sqrt{\sum_{k=1}^{d+1} \lrp{\frac {k-1} {d}}^2 \norm{{U_k}\T x_i}_2^2},
\end{align}
which concludes the proof.
\subsection{The flag trick for trace ratio problems}\label{subsec:TR}
Trace ratio problems are ubiquitous in machine learning~\citep{ngo_trace_2012}. They write as:
\begin{equation}\label{eq:TR_St}
\argmax{U \in \St(p, q)} \frac{\tr{U\T A U}}{\tr{U\T B U}},
\end{equation}
where $A, B \in \R^{p\times p}$ are positive semi-definite matrices, with $\operatorname{rank}(B) > p - q$.

A famous example of TR problem is Fisher's linear discriminant analysis (LDA)~\citep{fisher_use_1936,belhumeur_eigenfaces_1997}.
It is common in machine learning to project the data onto a low-dimensional subspace before fitting a classifier, in order to circumvent the curse of dimensionality. It is well known that performing an unsupervised dimension reduction method like PCA comes with the risks of mixing up the classes, since the directions of maximal variance are not necessarily the most discriminating ones~\citep{chang_using_1983}. The goal of LDA is to use the knowledge of the data labels to learn a linear subspace that does not mix the classes.
Let $~{X := [x_1|\dots|x_n] \in \R^{p\times n}}$ be a dataset with labels $Y := [y_1|\dots|y_n] \in {[1, C]}^n$. Let $\mu = \frac{1}{n} \sum_{i=1}^n x_i$ be the dataset mean and $\mu_c = \frac{1}{\#\{i : y_i=c\}}\sum_{i : y_i=c} x_i$ be the class-wise means. 
The idea of LDA is to search for a subspace $\S \in \Gr(p, q)$ that simultaneously maximizes the projected \textit{between-class variance} $\sum_{c=1}^C \|\Pi_\S \mu_c - \Pi_\S \mu\|_2^2$ and minimizes the projected \textit{within-class variance} $\sum_{c=1}^C \sum_{i : y_i = c} \|\Pi_\S x_i - \Pi_\S \mu_c\|_2^2$. This can be reformulated as a trace ratio problem~\eqref{eq:TR_St}, with $A = \sum_{c=1}^C (\mu_c - \mu) (\mu_c - \mu)\T$ and $B = \sum_{c=1}^C \sum_{i : y_i = c} (x_i - \mu_c) (x_i - \mu_c)\T$.


More generally, a large family of dimension reduction methods can be reformulated as a TR problem. The seminal work of~\citet{yan_graph_2007} shows that many dimension reduction and manifold learning objective functions can be written as a trace ratio involving Laplacian matrices of attraction and repulsion graphs. Intuitively, those graphs determine which points should be close in the latent space and which ones should be far apart.
Other methods involving a ratio of traces are \textit{multi-view learning}~\citep{wang_trace_2023}, \textit{partial least squares} (PLS)~\citep{geladi_partial_1986,barker_partial_2003} and \textit{canonical correlation analysis} (CCA)~\citep{hardoon_canonical_2004}, although these methods are originally \textit{sequential} problems (cf. footnote~\ref{footnote:sequential}) and not \textit{subspace} problems.

Classical Newton-like algorithms for solving the TR problem~\eqref{eq:TR_St} come from the seminal works of~\citet{guo_generalized_2003, wang_trace_2007, jia_trace_2009}.
The interest of optimizing a trace-ratio instead of a ratio-trace (of the form $\tr{(U\T B U)^{-1}(U\T A U)}$), that enjoys an explicit solution given by a generalized eigenvalue decomposition, is also tackled in those papers. The \textit{repulsion Laplaceans}~\citep{kokiopoulou_enhanced_2009} instead propose to solve a regularized version $\tr{U\T B U} - \rho \tr{U\T A U}$, which enjoys a closed-form, but has a hyperparameter $\rho$, which is directly optimized in the classical Newton-like algorithms for trace ratio problems.

\subsubsection{Application of the flag trick to trace ratio problems}
The trace ratio problem~\eqref{eq:TR_St} can be straightforwardly reformulated as an optimization problem on Grassmannians, due to the orthogonal invariance of the objective function:
\begin{equation}\label{eq:TR_Gr}
\argmax{\S \in \Gr(p, q)} \frac{\tr{\Pi_\S A}}{\tr{\Pi_\S B}}.
\end{equation}
The following proposition applies the flag trick to the TR problem~\eqref{eq:TR_Gr}.
\begin{proposition}[Flag trick for TR]\label{prop:TR}
The flag trick applied to the TR problem~\eqref{eq:TR_Gr} reads
\begin{equation}\label{eq:TR_Fl}
	\argmax{\S_{1:d} \in \Fl(p, q_{1:d})} \frac{\tr{\Pi_{\S_{1:d}} A}}{\tr{\Pi_{\S_{1:d}} B}}.
\end{equation}
and is equivalent to the following optimization problem:
\begin{equation}\label{eq:TR_Fl_equiv}
\argmax{U_{1:d} \in \St(p, q)} \frac{\sum_{k=1}^{d} (d - (k-1)) \tr{{U_k}\T A {U_k}}}{\sum_{l=1}^{d} (d - (l-1)) \tr{{U_{l}}\T B {U_{l}}}}.
\end{equation}
\end{proposition}
\begin{proof}
The proof is given in Appendix (\autoref{app:TR}).
\end{proof}
Equation~\eqref{eq:TR_Fl_equiv} tells us several things. First, the subspaces $~{\operatorname{Span}(U_1) \perp \dots \perp \operatorname{Span}(U_d)}$ are weighted decreasingly, which means that they have less and less importance with respect to the TR objective.
Second, we can see that the nested trace ratio problem~\eqref{eq:TR_Fl} somewhat maximizes the numerator $\tr{\Pi_{\S_{1:d}} A}$ while minimizing the denominator $\tr{\Pi_{\S_{1:d}} B}$. Both subproblems have an explicit solution corresponding to our nested PCA Theorem~\ref{thm:flag_trick}. Hence, one can naturally initialize the steepest descent algorithm with the $q$ highest eigenvalues of $A$ or the $q$ lowest eigenvalues of $B$ depending on the application.
For instance, for LDA, initializing Algorithm~\ref{alg:GD} with the highest eigenvalues of $A$ would spread the classes far apart, while initializing it with the lowest eigenvalues of $B$ would concentrate the classes, which seems less desirable since we do not want the classes to concentrate at the same point.

\subsubsection{Nestedness experiments for trace ratio problems}
For all the experiments of this subsection, we consider the particular TR problem of LDA, although many other applications (\textit{marginal Fisher analysis}~\citep{yan_graph_2007}, \textit{local discriminant embedding}~\citep{chen_local_2005} etc.) could be investigated similarly.

First, we consider a synthetic dataset with five clusters.
The ambient dimension is $p = 3$ and the intrinsic dimensions that we try are $q_{1:2} = (1, 2)$.
We adopt a preprocessing strategy similar to~\citet{ngo_trace_2012}: we first center the data, then run a PCA to reduce the dimension to $n - C$ (if $n - C < p$), then construct the LDA scatter matrices $A$ and $B$, then add a diagonal covariance regularization of $10^{-5}$ times their trace and finally normalize them to have unit trace.
We run Algorithm~\ref{alg:GD} on Grassmann manifolds to solve the TR maximization problem~\eqref{eq:TR_Gr}, successively for $q_1 = 1$ and $q_2 = 2$. Then we plot the projections of the data points onto the optimal subspaces. We compare them to the nested projections onto the optimal flag output by running Algorithm~\ref{alg:GD} on $\Fl(3, (1, 2))$ to solve~\eqref{eq:TR_Fl}. The results are shown in Figure~\ref{fig:TR_nested}.
\begin{figure}
	\centering
    \includegraphics[width=.9\linewidth]{Fig/FT_exp_TR_synthetic.pdf}
    \caption{
    Illustration of the nestedness issue in linear discriminant analysis (trace ratio problem). Given a dataset with five clusters, we plot its projection onto the optimal 1D subspace and 2D subspace obtained by solving the associated Grassmannian optimization problem~\eqref{eq:TR_Gr} or flag optimization problem~\eqref{eq:TR_Fl}. 
    We can see that the Grassmann representations are not nested, while the flag representations are nested and well capture the distribution of clusters. In this example, it is less the nestedness than the \textit{rotation} of the optimal axes inside the 2D subspace that is critical to the analysis of the Grassmann-based method.
    }
	\label{fig:TR_nested}
\end{figure}
\begin{figure}
	\centering
    \includegraphics[width=.9\linewidth]{Fig/FT_exp_TR_digits.pdf}
    \caption{
    Illustration of the nestedness issue in linear discriminant analysis (trace ratio problem) on the digits dataset. For $q_k \in \qf := (1, 2, \dots, 63)$, we solve the Grassmannian optimization problem~\eqref{eq:TR_Gr} on $\Gr(64, q_k)$ and plot the subspace angles $\Theta(\S_k^*, \S_{k+1}^*)$ (left) and explained variances ${\operatorname{tr}(\Pi_{\S_k^*} X X\T)} / {\operatorname{tr}(X X\T)}$ (right) as a function of $k$. We compare those quantities to the ones obtained by solving the flag optimization problem~\eqref{eq:TR_Fl}. 
    We can see that the Grassmann-based method is highly non-nested and even yields an extremely paradoxical non-increasing explained variance (cf. red circle on the right).
    }
	\label{fig:TR_nested_digits}
\end{figure}
We can see that the Grassmann representations are non-nested while their flag counterparts perfectly capture the filtration of subspaces that best and best approximates the distribution while discriminating the classes. Even if the colors make us realize that the issue in this experiment for LDA  is not much about the non-nestedness but rather about the rotation of the principal axes within the 2D subspace, we still have an important issue of consistency.

Second, we consider the (full) handwritten digits dataset~\citep{alpaydin_optical_1998}. It contains $8 \times 8$ pixels images of handwritten digits, from $0$ to $9$, almost uniformly class-balanced. One has $n = 1797$, $p=64$ and $C = 10$.
We run a steepest descent algorithm to solve the trace ratio problem~\eqref{eq:TR_Fl}. We choose the full signature $q_{1:63} = (1, 2, \dots, 63)$ and compare the output flag to the individual subspaces output by running optimization on $\Gr(p, q_k)$ for $q_k \in q_{1:d}$.
We plot the subspace angles $\Theta(\S_k^*, \S_{k+1}^*)$ and the explained variance ${\operatorname{tr}(\Pi_{\S_k^*} X X\T)} / {\operatorname{tr}(X X\T)}$ as a function of the $k$. The results are illustrated in \autoref{fig:TR_nested_digits}.
We see that the subspace angles are always positive and even very large sometimes with the LDA. Worst, the explained variance is not monotonous. This implies that we sometimes \textit{loose} some information when \textit{increasing} the dimension, which is extremely paradoxical.

Third, we perform some classification experiments on the optimal subspaces for different datasets. For different datasets, we run the optimization problems on $\Fl(p, q_{1:d})$, then project the data onto the different subspaces in $\S_{1:d}^*$ and run a nearest neighbors classifier with $5$ neighbors.
The predictions are then ensembled (cf. Algorithm~\ref{alg:flag_trick}) by weighted averaging, either with uniform weights or with weights minimizing the average cross-entropy:
\begin{equation}\label{eq:soft_voting}
	w_1^*, \dots, w_d^* = \argmin{\substack{w_k \geq 0 \\ \sum_{k=1}^d w_k = 1}} - \frac 1 {n C} \sum_{i=1}^n \sum_{c=1}^C y_{ic} \ln\lrp{\sum_{k=1}^d w_k y_{kic}^*},
\end{equation}
where $y_{kic}^* \in [0, 1]$ is the predicted probability that $x_i \in \R^p$ belongs to class $c \in \{1 \dots C\}$, by the classifier $g_k^*$ that is trained on $Z_k := {U_k^*}\T X \in \R^{q_k \times n}$. One can show that the latter is a convex problem, which we optimize using the \href{https://www.cvxpy.org/index.html}{cvxpy} Python package~\citep{diamond2016cvxpy}.
We report the results in \autoref{tab:TR_classif}.
\begin{table}
  \caption{Classification results for the TR problem on real datasets. For each method (Gr: Grassmann optimization~\eqref{eq:TR_Gr}, Fl: flag optimization~\eqref{eq:TR_Gr}, Fl-U: flag optimization + uniform soft voting, Fl-W: flag optimization + optimal soft voting~\eqref{eq:soft_voting}), we give the cross-entropy of the projected-predictions with respect to the true labels.}
  \label{tab:TR_classif}
  \centering
  \begin{tabular}{ccccccccc}
    \toprule
    dataset & $n$ & $p$ & $q_{1:d}$ & Gr & Fl & Fl-U & Fl-W & weights\\
    \midrule
    breast & $569$ & $30$ & $(1, 2, 5)$ & $0.0986$ & $0.0978$ & $0.0942$ & $0.0915$ & $(0.754, 0, 0.246)$\\
    iris & $150$ & $4$ & $(1, 2, 3)$ & $0.0372$ & $0.0441$ & $0.0410$ & $0.0368$ & $(0.985, 0, 0.015)$\\
    wine & $178$ & $13$ & $(1, 2, 5)$ & $0.0897$ & $0.0800$ & $0.1503$ & $0.0677$ & $(0, 1, 0)$\\
    digits & $1797$ & $64$ & $(1, 2, 5, 10)$ & $0.4507$ & $0.4419$ & $0.5645$ & $0.4374$ & $(0, 0, 0.239, 0.761)$\\
    \bottomrule
  \end{tabular}
\end{table}
The first example tells us that the optimal $5D$ subspace obtained by Grassmann optimization less discriminates the classes than the $5D$ subspace from the optimal flag. This may show that the flag takes into account some lower dimensional variability that enables to better discriminate the classes. We can also see that the uniform averaging of the predictors at different dimensions improves the classification. Finally, the optimal weights improve even more the classification and tell that the best discrimination happens by taking a soft blend of classifier at dimensions $1$ and $5$. Similar kinds of analyses can be made for the other examples.

\subsubsection{Discussion on TR optimization and kernelization}
\paragraph{A Newton algorithm}
In all the experiments of this paper, we use a steepest descent method on flag manifolds (Algorithm~\ref{alg:GD}) to solve the flag problems.
However, for the specific problem of TR~\eqref{eq:TR_Fl}, we believe that more adapted algorithms should be derived to take into account the specific form of the objective function, which is classically solved via a Newton-Lanczos method~\citep{ngo_trace_2012}. 
To that extent, we develop in the appendix (\autoref{app:TR}) an extension of the baseline Newton-Lanczos algorithm for the flag-tricked problem~\eqref{eq:TR_Fl}.
In short, the latter can be reformulated as a penalized optimization problem of the form $\operatorname{argmax}_{\Sf\in\Fl(p, \qf)} {\sum_{k=1}^d \tr{\Pi_{\S_k} (A - \rho B)}}$, where $\rho$ is updated iteratively according to a Newton scheme. Once again, our central Theorem~\ref{thm:flag_trick} enables to get explicit expressions for the iterations, which results without much difficulties in a Newton method, that is known to be much more efficient than first-order methods like the steepest descent.

\paragraph{A non-linearization via the kernel trick}
The classical trace ratio problems look for \textit{linear} embeddings of the data.
However, in most cases, the data follow a \textit{nonlinear} distribution, which may cause linear dimension reduction methods to fail. The \textit{kernel trick}~\citep{hofmann_kernel_2008} is a well-known method to embed nonlinear data into a linear space and fit linear machine learning methods.
As a consequence, we propose in appendix (\autoref{app:TR}) a kernelization of the trace ratio problem~\eqref{eq:TR_Fl} in the same fashion as the one of the seminal graph embedding work~\citep{yan_graph_2007}.
This is expected to yield much better embedding and classification results.
\section{Spectral clustering: extensions and proofs}\label{app:SSC}


\subsection{Proof of Proposition~\ref{prop:SSC}}
Let $\Sf \in \Fl(p, \qf)$ and $U_{1:d+1} := [U_1|U_2|\dots|U_d|U_{d+1}] \in \O(p)$ be an orthogonal representative of $\Sf$. One has:
\begin{align}
	\langle \Pi_{\S_{1:d}}, L\rangle_F + \beta \norm{\Pi_{\S_{1:d}}}_1 &= \left\langle \frac1d\sum_{k=1}^d \Pi_{\S_k}, L\right\rangle_F + \beta \norm{\frac1d\sum_{k=1}^d \Pi_{\S_k}}_1,\\
	&= \frac1d \lrp{\left\langle \sum_{k=1}^{d+1} (d - (k-1)) U_k {U_k}\T, L\right\rangle_F + \beta \norm{\sum_{k=1}^{d+1} (d - (k-1)) U_k {U_k}\T }_1},\\
	&= \frac1d \lrp{\sum_{k=1}^{d+1} (d - (k-1)) \left\langle U_k {U_k}\T, L\right\rangle_F + \beta \norm{\sum_{k=1}^{d+1} (d - (k-1)) U_k {U_k}\T }_1},\\
	\langle \Pi_{\S_{1:d}}, L\rangle_F + \beta \norm{\Pi_{\S_{1:d}}}_1 &= \frac1d \lrp{\sum_{k=1}^{d+1} (d - (k-1)) \tr{{U_k}\T L U_k} + \beta \norm{\sum_{k=1}^{d+1} (d - (k-1)) U_k {U_k}\T }_1},
\end{align}
which concludes the proof.
\subsection{The flag trick for other machine learning problems}
Subspace learning finds many applications beyond robust subspace recovery, trace ratio and spectral clustering problems, as evoked in~\autoref{sec:intro}. The goal of this subsection is to provide a few more examples in brief, without experiments.


\subsubsection{Domain adaptation}
In machine learning, it is often assumed that the training and test datasets follow the same distribution. However, some \textit{domain shift} issues---where training and test distributions are different---might arise, notably if the test data has been acquired from a different source (for instance a professional camera and a phone camera) or if the training data has been acquired a long time ago. \textit{Domain adaptation} is an area of machine learning that deals with domain shifts, usually by matching the training and test distributions---often referred to as \textit{source} and \textit{target} distributions---before fitting a classical model~\citep{farahani_brief_2021}. 
A large body of works (called ``subspace-based'') learn some intermediary subspaces between the source and target data, and perform the inference for the projected target data on these subspaces. The \textit{sampling geodesic flow}~\citep{gopalan_domain_2011} first performs a geodesic interpolation on Grassmannians between the source and target subspaces, then projects both datasets on (a discrete subset of) the interpolated subspaces, which results in a new representation of the data distributions, that can then be given as an input to a machine learning model. The higher the number of intermediary subspaces, the better the approximation, but the larger the dimension of the representation.
The celebrated \textit{geodesic flow kernel}~\citep{boqing_gong_geodesic_2012} circumvents this issue by integrating the projected data onto the continuum of interpolated subspaces. This yields an inner product between infinite-dimensional embeddings that can be computed explicitly and incorporated in a kernel method for learning. The \textit{domain invariant projection}~\citep{baktashmotlagh_unsupervised_2013} learns a \textit{domain-invariant} subspace that minimizes the maximum mean discrepancy (MMD)~\citep{gretton_kernel_2012} between the projected source $X_s := [x_{s1}|\dots|x_{s n_s}] \in \R^{p\times n_s}$ and target distributions $X_t := [x_{t1}|\dots|x_{t n_t}] \in \R^{p\times n_t}$:
\begin{equation}
	\argmin{U \in \St(p, q)} \operatorname{MMD}^2(U\T X_{s}, U\T X_{t}),
\end{equation}
where 
\begin{equation}
	\operatorname{MMD} (X, Y) = \norm{\frac 1 n \sum_{i=1}^n \phi (x_i) - \frac 1 m \sum_{i=1}^m \phi (y_i)}_\mathcal{H}.
\end{equation}
This can be rewritten, using the Gaussian kernel function $\phi(x)\colon y \mapsto \exp\lrp{-\frac{x\T y}{2\sigma^2}}$, as
\begin{multline}\label{eq:DIP}
	\argmin{\S \in \Gr(p, q)} 
	\frac 1 {n_s^2} \sum_{i,j=1}^{n_s} \exp\lrp{-\frac{(x_{si} - x_{sj})\T \Pi_\S (x_{si} - x_{sj})}{2 \sigma^2}}\\
	+ \frac 1 {n_t^2} \sum_{i,j=1}^{n_t} \exp\lrp{-\frac{(x_{ti} - x_{tj})\T \Pi_\S (x_{ti} - x_{tj})}{2 \sigma^2}}\\
	- \frac 2 {n_s n_t} \sum_{i=1}^{n_s} \sum_{j=1}^{n_t} \exp\lrp{-\frac{(x_{si} - x_{tj})\T \Pi_\S (x_{si} - x_{tj})}{2 \sigma^2}}.
\end{multline}
The flag trick applied to the domain invariant projection problem~\eqref{eq:DIP} yields:
\begin{multline}
	\argmin{\S_{1:d} \in \Fl(p, q_{1:d})} 
	\frac 1 {n_s^2} \sum_{i,j=1}^{n_s} \exp\lrp{-\frac{(x_{si} - x_{sj})\T \Pi_{\S_{1:d}} (x_{si} - x_{sj})}{2 \sigma^2}}\\
	+ \frac 1 {n_t^2} \sum_{i,j=1}^{n_t} \exp\lrp{-\frac{(x_{ti} - x_{tj})\T \Pi_{\S_{1:d}} (x_{ti} - x_{tj})}{2 \sigma^2}}\\
	- \frac 2 {n_s n_t} \sum_{i=1}^{n_s} \sum_{j=1}^{n_t} \exp\lrp{-\frac{(x_{si} - x_{tj})\T \Pi_{\S_{1:d}} (x_{si} - x_{tj})}{2 \sigma^2}},
\end{multline}
and can be rewritten as:
\begin{multline}
	\argmin{U_{1:d} \in \St(p, q)}
	\frac 1 {{n_s}^2} \sum_{i,j=1}^{n_s} \exp\lrp{-\sum_{k=1}^d \frac{d+1-k}{d} \frac{\norm{{U_k}\T (x_{si} - x_{sj})}_2^2}{2 \sigma^2}}\\
	+ \frac 1 {{n_t}^2} \sum_{i,j=1}^{n_t} \exp\lrp{-\sum_{k=1}^d \frac{d+1-k}{d} \frac{\norm{{U_k}\T (x_{ti} - x_{tj})}_2^2}{2 \sigma^2}}\\
	- \frac 2 {{n_s} {n_t}} \sum_{i=1}^{n_s} \sum_{j=1}^{n_t} \exp\lrp{-\sum_{k=1}^d \frac{d+1-k}{d} \frac{\norm{{U_k}\T (x_{si} - x_{tj})}_2^2}{2 \sigma^2}}.
\end{multline}
Some experiments similar to the ones of~\citet{baktashmotlagh_unsupervised_2013} can be performed. For instance, one can consider the benchmark visual object recognition dataset of~\citet{saenko_adapting_2010}, learn nested domain invariant projections, fit some support vector machines to the projected source samples at increasing dimensions, and then perform soft-voting ensembling by learning the optimal weights on the target data according to Equation~\eqref{eq:soft_voting}.

\subsubsection{Low-rank decomposition}
Many machine learning methods involve finding low-rank representations of a data matrix. 

This is the case of \textit{matrix completion}~\citep{candes_exact_2012} problems where one looks for a low-rank representation of an incomplete data matrix by minimizing the discrepancy with the observed entries, and which finds many applications including the well-known \href{https://en.wikipedia.org/wiki/Netflix_Prize}{Netflix problem}. Although its most-known formulation is as a convex relaxation, it can also be formulated as an optimization problem on Grassmann manifolds~\citep{keshavan_matrix_2010,boumal_rtrmc_2011} to avoid optimizing the nuclear norm in the full space which can be of high dimension. The intuition is that a low-dimensional point can be described by the subspace it belongs to and its coordinates within this subspace. More precisely, the SVD-based low-rank factorization $M = UW$, with $M \in \R^{p \times n}$, $U \in \St(p, q)$ and $W \in \R^{q \times n}$ is orthogonally-invariant---in the sense that for any $R\in\O(q)$, one has $(UR) (R\T W) = U W$. One could therefore apply the flag trick to such problems, with the intuition that we would try low-rank matrix decompositions at different dimensions. The application of the flag trick would however not be as straightforward as in the previous problems since the subspace-projection matrices $\Pi_\S := U U\T$ do not appear explicitly, and since the coefficient matrix $W$ also depends on the dimension $q$.

Many other low-rank problems can be formulated as a Grassmannian optimization. \textit{Robust PCA}~\citep{candes_robust_2011} looks for a low rank + sparse corruption factorization of a data matrix. \textit{Subspace Tracking}~\citep{balzano_online_2010} incrementally updates a subspace from streaming and highly-incomplete observations via small steps on Grassmann manifolds.

\subsubsection{Linear dimensionality reduction}
Finally, many other general dimension reduction algorithms---referred to as \textit{linear dimensionality reduction methods}~\citep{cunningham_linear_2015}---involve optimization on Grassmannians. For instance, linear dimensionality reduction encompasses the already-discussed PCA and LDA, but also many other problems like \textit{multi-dimensional scaling}~\citep{torgerson_multidimensional_1952}, \textit{slow feature analysis}~\citep{wiskott_slow_2002}, \textit{locality preserving projections}~\citep{he_locality_2003} and \textit{factor analysis}~\citep{spearman_general_1904}.
\section{Comments and further discussion}
\label{subsec:rst_comment}
\subsection{Accuracy vs discontinuities}
\begin{figure}
\includegraphics[scale=0.50]{images/extra/flow.pdf}
\centering
\vspace{-0.1em}
\caption{Complexity and smoothness trade-off regions of RST algorithm depending on number of blocks and waypoints in each block.}
\vspace{-1.0em}
\label{fig:rst_regions}
\end{figure}
In this section, we present a useful rule of thumb for applying the RST and the RST$_{\text{opt}}$ algorithm according to the number of blocks the user considers and the number of waypoints inside each of these blocks. 
Indeed, Fig.~\ref{fig:rst_regions} and Fig.~\ref{fig:rst_example1} identify regions, suggesting which algorithm to apply according to the number of waypoints and blocks.
So far, most of path planning research has been focused on piecewise polynomial approaches, sticking to the first left strip in Fig.~\ref{fig:rst_regions}. There are, indeed, cases where numerical polynomial complexity is the main concern and in such situations a standard piecewise polynomial approach is sufficient. Nevertheless, we propose to use the PRST algorithm to compute the piecewise polynomial trajectory (e.g. RST with $2$ points in each block). The reason for this choice comes from the intrinsic optimality of RST when the number of waypoints in each block is $2$, as proved in Lemma \ref{lemma:rst_Lemma5}.

An unwanted side effect of the piecewise choice is the presence of discontinuities in the $p$-th derivative's interface. To reduce and eventually remove them, a good compromise is the BRST algorithm which balances polynomial complexity and discontinuity issues. For instance, instead of using $15$ piecewise polynomials for a trajectory consisting of $16$ points, one could choose to split the trajectory generation in $5$ blocks of $4$ points each, leading to a reduction of discontinuities, from $14$ to $4$ matching interfaces, while at the same time keeping a low level of complexity in polynomials. See Fig.~\ref{fig:rst_example1} for an example of such blockwise trajectory.
Whenever the complexity is not the issue to consider at first, one could build a smooth trajectory which passes through all the waypoints. In such a case RST is one possible way to proceed if no optimal condition is required, otherwise RST$_{\text{opt}}$ provides the optimality at expenses of a higher computational cost. 
Moreover, we empirically found that smooth trajectories with more than $15$ waypoints in the same block are unstable, therefore we suggest to always split them into at least $2$ consecutive blocks.
Fig.~\ref{fig:rst_regions} and \ref{fig:rst_example1} illustrate these concepts and guide the user to select the proper algorithm according to given specifics.

\begin{figure}
\includegraphics[scale=0.37]{images/extra/RST_example.pdf}
\centering
\vspace{-0.1em}
\caption{Example of blockwise trajectory (BRST, BRST$_{\text{opt}}$ and minimum-snap) that minimizes the integral of the acceleration squared.}
\vspace{-1.0em}
\label{fig:rst_example1}
\end{figure}

\subsection{Memory requirements}
\label{subsec:rst_time}
The number of coefficients of the RST polynomial trajectory to store is $(k+1)(N+1)$ (See Corollary \ref{corollary:rst_corollary1}) where $N+1$ is the number of waypoints and $k$ is the last kinematic constraint. Extending the results also to BRST leads to $M$ polynomials of degree $(k+1)(\frac{N+M}{M})-1$. Thus, the number of coefficients to store is equal to $(k+1)(N+M)$. The ratio between the number of coefficients to store for BRST and RST is equal to $1+\frac{M-1}{N+1}$ which means that BRST requires $100\cdot \frac{M-1}{N+1}$ percent more memory than RST. As an extreme case, PRST is the technique which requires the highest memory requirements since it stores $2N(k+1)$ coefficients, almost twice the memory required by RST.

\subsection{Computational complexity}
\label{subsec:rst_complexity}
To evaluate the computational complexity, it is convenient to segment the RST algorithm (See Alg. \ref{alg:RST}) in $3$ parts: the recursive formula, which computes the control points $s_i(t_j)$ as in \eqref{eq:rst_recursive} (line 5 of Alg. \ref{alg:RST}), the interpolation phase with Lagrange polynomials (line 7 of Alg. \ref{alg:RST}), and the generation of the $i$-partial trajectory (line 8 of Alg. \ref{alg:RST}).
The recursive formula provides $N+1$ control points $s_i(t_j)$ at the $i$-th iteration, with $i=0,1,\dots,k$ and $j=0,1,\dots,N$. In particular, to compute a single value $s_i(t_j)$ it needs a number of operations that goes as $\mathcal{O}(i\cdot \text{deg}(x_{i-1}(t))) \sim \mathcal{O}(i^2 \cdot N)$, where the first $i$ contribution comes from the $i$ derivatives of the $(i-1)$-partial trajectory. Hence, for $N+1$ points the complexity of the recursive formula is $N\mathcal{O}(i^2 \cdot N)$. The complexity of the Lagrange interpolation technique is $\mathcal{O}((N+1)^2) \sim \mathcal{O}(N^2)$. Lastly, the generation of the $i$-partial trajectory involves the product between $a^i(t)\cdot s_i(t)$, which requires a number of operations that grow as $\mathcal{O}((N+1)\cdot i \cdot (N+1)) \sim \mathcal{O}(i \cdot N^2)$. Since the number of iterations are $k+1$, the overall time complexity $T(k,N)$ reads as follows
\begin{align}
T(k,N) &= \sum_{i=0}^{k}{N\mathcal{O}(i^2 \cdot N)+\mathcal{O}(N^2)+\mathcal{O}(i \cdot N^2)} \nonumber \\
& \sim \mathcal{O}(k^3 \cdot N^2)+\mathcal{O}(k\cdot N^2)+\mathcal{O}(k^2 \cdot N^2) \nonumber \\
& \sim \mathcal{O}(k^3 \cdot N^2).
\end{align}
Although the estimated complexity is a rough approximation, it is interesting to highlight the following fact: the minimum degree polynomial trajectory $x_k(t)$ could have been derived in the classical approach just by evaluating the polynomial and its derivatives in the time stamps and by solving a system of linear equations. Alternative in a matrix form, $b = \mathbf{X}\cdot a$ where $\mathbf{X}$ is a $(k+1)(N+1)\times (k+1)(N+1)$ square matrix, badly conditioned from a numerical point of view, $a$ is the unknown vector of the polynomial coefficients and $b$ the vector of the kinematic constraints. To find the polynomial trajectory, thus, the coefficients, the matrix $\mathbf{X}$ needs to be inverted (when numerically possible). However, the matrix inversion operation involves a complexity of order $\mathcal{O}(k^3 \cdot N^3)$, that is higher than the RST complexity. Therefore, RST is not only numerically stable since no matrix inversion is required, but it is also faster than the classical interpolation approach (INV). Fig. \ref{fig:rst_time_complexity} illustrates the computational complexity advantage of RST over the classic interpolation method.

\begin{figure}
\includegraphics[scale=0.205]{images/extra/time_complexity.pdf}
\centering
\vspace{-0.1em}
\caption{Computational complexity comparison between RST and the classic interpolation approach through matrix inversion (INV).}
\vspace{-1.0em}
\label{fig:rst_time_complexity}
\end{figure}


\subsection{Extension of the proposed framework}
We presented the RST algorithm and extensions to block (BRST) and piecewise (PRST) approaches. Initial assumptions always considered time intervals with the same length or points in time following \eqref{Cheby}. The case that considers random initially located points in time can been studied under the optimization framework RST$_{\text{opt}}$. To tackle the oscillation problem (see Fig. \ref{fig:rst_Runge}), typical of high-order polynomial interpolation, a possible solution without involving the optimization step could either pass through spline interpolation or different interpolating polynomials such as barycentric Lagrange polynomials \cite{Berrut} or Newton ones.

Lastly and perhaps more fascinating, is the idea of mixing and eventually replacing polynomial trajectories with other basis functions. All the mathematical formulation and most of the derivation actually transcend the polynomial assumption. The only point in which this hypothesis plays a role is in the $h$-th derivative step (See Lemma \ref{lemma:rst_Lemma2}). The outcome of a further investigation is discussed in Sec. \ref{sec:rst_rrst}

\begin{figure}
\includegraphics[scale=0.50]{images/extra/runge_phenomenon}
\centering
\caption{Illustration of Runge's phenomenon: the Runge function (blue dashed line) is approximated with a $15$-th order polynomial (red solid line) that interpolates $16$ equally spaced nodes.}
\label{fig:rst_Runge}
\end{figure}

\section{Rational interpolation} 
\sectionmark{RRST}
\label{sec:rst_rrst}
Polynomial interpolation is in general a simple and fast process to implement. Nevertheless, when the degree of the interpolant function is high, oscillation at the edges may occur as mentioned before. For this reason, we consider a different basis function which may take advantage of the simplicity of polynomials but also provide more flexibility and degrees of freedom to tackle Runge's phenomenon.

\subsection{Rational recursive smooth trajectory}
We identify and propose a new basis as the rational basis function
\begin{equation}
R_{n,d}(t) = \frac{N(t)}{D(t)},
\end{equation}
where $N(t)$ is the numerator, a polynomial of degree $n$, and $D(t)$ is the denominator, a polynomial of degree $d$.
Such choice allows us to exploit some of the polynomial properties for both numerator and denominator but most importantly, enables the development of a new algorithm, referred to as rational recursive smooth trajectory (RRST). To find the coefficients of both numerator and denominator, the idea is to pick the denominator $D(t)$ and use RST to find the coefficients of the numerator $N(t)$. Intuitively, the new kinematic constraints for building $N(t)$ are a weighted sum of the kinematic constraints $\frac{d^i}{dt^i}f_k(t)\biggr|_{t=t_j}$ (given) and the kinematic constraints $\frac{d^i}{dt^i}D(t)\biggr|_{t=t_j}$ (designed as input). The following Lemma provides the mathematical formulation for the RRST.

\begin{lemma}
\label{lemma:rrst_Lemma1}
Let $t_j$ be a point in time, for $j=0,1,\dots, N$, such that $\frac{d^i}{dt^i}f_k(t)\bigr|_{t=t_j}$ is the associated given kinematic constraint, for $i=0,1,\dots, k$. Let $N(t)$ and $D(t)$ be polynomials with $D$ given of degree $d$. If $f_k(t)$ is a rational function defined as
\begin{equation}
f_k(t) = R_{n,d}(t) = \frac{N(t)}{D(t)},
\end{equation}
with $n=(k+1)(N+1)-1$, then the coefficients of $N(t)$ can be obtained with RST, in particular its associated kinematic constraint has expression
\begin{equation}
\frac{d^i}{dt^i}N(t)\biggr|_{t=t_j} = \sum_{l=0}^{i}{\binom{i}{l}\biggr(\frac{d^l}{dt^l}f_k(t)\bigr|_{t=t_j}\biggr) \cdot \biggr(\frac{d^{i-l}}{dt^{i-l}}D(t)\bigr|_{t=t_j}}\biggr).
\end{equation}
\end{lemma}

\begin{proof}
For simplicity of notation, the rational function $R_{n,d}(t)$ will be denoted with $R(t)$.
We proceed by induction on the kinematic constraint. Consider the case when $i=0$, then
\begin{equation}
N(t_j) = R(t_j)\cdot D(t_j)
\end{equation}
represents the value that $N(t)$ needs to assume at the time $t_j$. For the case $i=1$
\begin{equation}
\frac{d}{dt}N(t)\biggr|_{t=t_j} = \frac{d}{dt} \biggl(R(t)\cdot D(t)\biggr)\biggr|_{t=t_j}
\end{equation}
which is equal to
\begin{equation}
\frac{d}{dt}N(t)\biggr|_{t=t_j} = \binom{1}{0}R(t_j) \cdot \biggr( \frac{d}{dt}D(t)\bigr|_{t=t_j}\biggr) + \binom{1}{1} \biggr( \frac{d}{dt}R(t)\bigr|_{t=t_j}\biggr) \cdot D(t_j) .
\end{equation}
Suppose that the statement of the lemma is true for the case $i$, which means that
\begin{equation}
\frac{d^i}{dt^i}N(t)\biggr|_{t=t_j} = \sum_{l=0}^{i}{\binom{i}{l}\biggr(\frac{d^l}{dt^l}R(t)\bigr|_{t=t_j}\biggr) \cdot \biggr(\frac{d^{i-l}}{dt^{i-l}}D(t)\bigr|_{t=t_j}}\biggr).
\end{equation}
Then, it is true also for the case $i+1$. Indeed
\begin{align}
\frac{d^{i+1}}{dt^{i+1}}N(t)\biggr|_{t=t_j} = &\; 
\frac{d}{dt}\sum_{l=0}^{i}{ \binom{i}{l}\biggr(\frac{d^l}{dt^l}R(t)\bigr|_{t=t_j}\biggr) \cdot \biggr(\frac{d^{i-l}}{dt^{i-1}}D(t)\bigr|_{t=t_j}\biggr)} \nonumber \\
= &\; \sum_{l=0}^{i}{\binom{i}{l} \frac{d}{dt}\Biggl[ \biggr(\frac{d^l}{dt^l}R(t)\bigr|_{t=t_j}\biggr) \cdot \biggr(\frac{d^{i-l}}{dt^{i-1}}D(t)\bigr|_{t=t_j}\biggr)\Biggr] }  \nonumber \\ 
= &\; \sum_{l=0}^{i}{\binom{i}{l} \biggr(\frac{d^{l+1}}{dt^{l+1}}R(t)\bigr|_{t=t_j}\biggr) \cdot \biggr(\frac{d^{i-l}}{dt^{i-1}}D(t)\bigr|_{t=t_j}}\biggr) \nonumber \\
+ &\; \sum_{l=0}^{i}{\binom{i}{l} \biggr(\frac{d^{l}}{dt^{l}}R(t)\bigr|_{t=t_j}\biggr) \cdot \biggr(\frac{d^{i+1-l}}{dt^{i+1-l}}D(t)\bigr|_{t=t_j}\biggr)}
\end{align}
where we used the linearity of the differential operator and the product rule. With a change of variable in the first term of the RHS, $h=l+1$, it follows that
\begin{align}
\frac{d^{i+1}}{dt^{i+1}}N(t)\biggr|_{t=t_j} = &\; 
   \sum_{h=1}^{i+1}{\binom{i}{h-1} \biggr(\frac{d^{h}}{dt^{h}}R(t)\bigr|_{t=t_j}\biggr) \cdot \biggr(\frac{d^{i+1-h}}{dt^{i+1-h}}D(t)\bigr|_{t=t_j}}\biggr) \nonumber \\
+ &\; \sum_{l=0}^{i}{\binom{i}{l} \biggr(\frac{d^{l}}{dt^{l}}R(t)\bigr|_{t=t_j}\biggr) \cdot \biggr(\frac{d^{i+1-l}}{dt^{i+1-l}}D(t)\bigr|_{t=t_j}}\biggr) \nonumber \\ 
= &\; \sum_{l=0}^{i+1}{\binom{i+1}{l}\biggr(\frac{d^l}{dt^l}R(t)\bigr|_{t=t_j}\biggr) \cdot \biggr(\frac{d^{i+1-l}}{dt^{i+1-l}}D(t)\bigr|_{t=t_j}}\biggr)
\end{align}
where we used the Pascal's identity
\begin{equation}
\binom{i+1}{l} = \binom{i}{l-1} + \binom{i}{l}.
\end{equation}
Hence the result is true for $i+1$ and by induction is true for all positive integers. From Corollary \ref{corollary:rst_corollary1}, the minimum degree $n$ of $N(t)$ is $(k+1)(N+1)-1$.
\qedhere
\end{proof}

\begin{figure}[b]
\includegraphics[scale=0.25]{images/extra/acrtan.pdf}
      \centering
      \caption{Comparison between polynomial (RST) and rational (RRST) interpolation of $10$ waypoints, obtained as samples of the analytic control input $\text{arctan}(\pi t)$.}
      \label{fig:rst_arctan}
\end{figure}

\begin{algorithm}
\caption{Rational recursive smooth trajectory (RRST)}
\label{alg:rst_RRST}
\begin{algorithmic}[1]
\Inputs{$N+1$ points in time $t_0<t_1<\dots<t_N$; \\ Number of derivatives $k$ to fulfill; \\ Kin. constr.
 $\frac{d^i}{dt^i}f_k(t)\bigr|_{t=t_0}, \dots, \frac{d^i}{dt^i}f_k(t)\bigr|_{t=t_N}$; \\ Denominator $D(t)$ of degree $d$. \\}
\Initialize{Kin. constr. $\frac{d^i}{dt^i}D(t)\bigr|_{t=t_0}, \dots, \frac{d^i}{dt^i}D(t)\bigr|_{t=t_N}$;}
\For{$i=0$ to $k$}
	\For{$j=0$ to $N$}
		\State $\frac{d^i}{dt^i}N(t)\biggr|_{t=t_j} =$
          \State $\sum_{l=0}^{i}{\binom{i}{l}\biggr(\frac{d^l}{dt^l}f_k(t)\bigr|_{t=t_j}\biggr) \cdot \biggr(\frac{d^{i-l}}{dt^{i-l}}D(t)\bigr|_{t=t_j}}\biggr)$;
	\EndFor
\EndFor
\State Get $N(t)$ with RST given the kinematic constraints $\frac{d^i}{dt^i}N(t)\biggr|_{t=t_j}$ as input;
\State $f_k(t)=\frac{N(t)}{D(t)}$.
\end{algorithmic}
\end{algorithm}

Lemma \ref{lemma:rrst_Lemma1} provides the general expression of the kinematic constraints associated to $N(t)$, however it assumes that the denominator $D(t)$ is given. The choice of the denominator remains an open question although some considerations can be made. The denominator represents a whole set of degree of freedoms and therefore the choice of the coefficients should in principle consider some strategies. For example, a fundamental aspect is the position of the roots inside the interval $[t_0, \hspace{0.2em} t_N]$. Indeed, if one real pole (denominator root) falls inside the desired interval, it may cause discontinuities in the rational interpolant. To avoid this, a possible strategy relies on the selection of multiple complex conjugate roots. Further studies have to be made in the roots locus analysis for such rational function but they go out of the scope of this section therefore we postpone these questions to future work. Finally, it is interesting to notice that if the denominator $D(t)$ is constant, we lead back to the classical polynomial interpolation via RST, therefore we can tract RRST as a rational basis extension of the RST algorithm. The implementation of the RRST algorithm is we reported in the pseudo code of Alg. \ref{alg:rst_RRST}.

To show how the RRST tackles the oscillation problem, we report in Fig.~\ref{fig:rst_arctan} an example of function approximation with polynomials (RST) and rational functions (RRST). In particular, we select as function to interpolate $f_k(t)= \text{arctan}(\pi t)$, with $t\in [-1,1]$. Fig.~\ref{fig:rst_arctan} illustrates the resulting interpolants when the number of waypoints is set to $10$ and no kinematic constraints (from velocity on) are imposed. The denominator of the rational function is set to $D(t)=t^2+0.1$ and for such choice, RRST shows to perform better than the polynomial interpolant at the edges.

In conclusion, we extended the RST algorithm to rational functions. The algorithm can effectively generate an analytic expression that approximates control inputs, for which no closed-form solutions are in general attainable. More details are offered in \cite{9525383}.

\chapter{\textcolor{black}{Conclusion}}
\label{ch: Conclusion}
\thispagestyle{plain}

In this thesis, the potential of \gls{sc} and \gls{goc} paradigms within modern digital networks has been explored and exploited. The rapid proliferation of data driven technologies such as the \gls{iot}, autonomous vehicles and smart cities has underscored the limitations of traditional bit-centric communication systems. These systems, grounded in Shannon's information theory, focus primarily on the accurate transmission of raw data without considering the contextual significance of the information being conveyed. This fundamental mismatch between data production and communication infrastructure capabilities has necessitated the exploration of more efficient and intelligent communication frameworks.

\cref{ch: SEMCOM} discussed how the core of this thesis focused on integrating of \gls{sc} principles with generative models and their potential applications in the context of edge computing. By focusing on the conveyance of relevant meaning rather than exact data reproduction, \gls{sc} reduces unnecessary bandwidth consumption and inefficiencies. In all those cases where it is possible and reasonable to discuss the semantics, then the faithful representation of the original data is unnecessary as long as the meaning has been conveyed. This paradigm also aligns with \gls{goc}, where the transmitted data is tailored to meet specific objectives, further reducing the communication overhead. The goal of the communication can either be the classical syntactic data transmission or the semantic preservation of the data. By focusing on the goal of the communication, it is possible to transmit only the most pertinent information, thereby reducing the load in communication networks and optimizing resource utilization.

In \cref{ch: SPIC}, the \gls{spic} framework was introduced as a novel method for semantic-aware image compression. The framework demonstrated the potential for high-fidelity image reconstruction from compressed semantic representations. The proposed modular transmitter-receiver architecture is based on a doubly conditioned \gls{ddpm} model, the \gls{semcore}, specifically designed to perform \gls{sr} under the conditioning of the \gls{ssm}. By doing so the reconstructed images preserve their semantic features at a fraction of the \gls{bpp} compared to classical methods such as \gls{bpg} and \gls{jpeg2000}.

Furthermore, the enhancement introduced by \gls{cspic} addressed a critical aspect in image reconstruction: the accurate representation of small and detailed objects. Without requiring extensive retraining of the underlying \gls{semcore} model, \gls{cspic} improved the preservation of important semantic classes, such as traffic signs.  The modular design at the core of the \gls{spic} and \gls{cspic} showcased the flexibility and adaptability of the system in different contexts.

The integration of \gls{sc} principles continued in \cref{ch: SQGAN}, where the \gls{sqgan} model was proposed. This architecture employed vector quantization in tandem with a semantic-aware masking mechanism, enabling selective transmission of semantically important regions of the image and the \gls{ssm}. By prioritizing critical semantic classes and utilizing techniques such as Semantic Relevant Classes Enhancement or the Semantic-Aware discriminator, the model excelled at maintaining high reconstruction quality even at very low bit rates, further emphasizing the efficiency gains of the proposed approach.

Finally, in \cref{ch: Goal_oriented}, the thesis was extended to include the \gls{goc} for resource allocation in \glspl{en}. By adopting the \gls{ib} principle to perform \gls{goc} was developed a framework to dynamically adjust compression and transmission parameters based on network conditions and resource constraints. This dynamic adaptation was crucial in balancing compression efficiency with semantic preservation, optimizing the use of computational and communication resources in edge networks.

By leveraging the \gls{sqgan} within the \gls{en}, the research demonstrated the synergy between \gls{sc} and \gls{goc}. Real-time network conditions informed adjustments to the masking process, enabling the edge network to operate autonomously and efficiently. This approach validated the potential of \gls{sgoc} to enhance resource utilization in modern network infrastructures.



% In this thesis, the potential of \gls{sc} and \gls{goc} paradigms within modern digital networks has been explored and exploited. The rapid proliferation of data driven technologies such as the \gls{iot}, autonomous vehicles, and smart cities has underscored the limitations of traditional bit-centric communication systems. These systems, grounded in Shannon's information theory, focus primarily on the accurate transmission of raw data without considering the contextual significance of the information being conveyed. This fundamental mismatch between data production and communication infrastructure capabilities has necessitated the exploration of more efficient and intelligent communication frameworks.

% As explained in \cref{ch: SEMCOM} at the core of this thesis lies the integration of \gls{sc} principles with generative models, particularly within the context of edge computing. \gls{sc}, which emphasizes the conveyance of meaning rather than mere symbol reconstruction, offers a pathway to significantly reduce bandwidth usage and enhance the efficiency of data transmission. This approach aligns seamlessly with the objectives of \gls{goc}, which prioritizes the transmission of information that is directly relevant to achieving specific goals. By focusing on the semantic content of the data, it becomes possible to transmit only the most pertinent information, thereby reducing the load in  communication networks and optimizing resource utilization.

% In \cref{ch: SPIC} the development and implementation of the \gls{spic} framework marked a significant stride in bridging \gls{sc} with practical image compression techniques. By leveraging diffusion models, \glspl{ddpm} were employed to reconstruct high-resolution images from compressed semantic representations. This modular approach, consisting of a transmitter and receiver architecture, facilitated the efficient encoding and decoding of both the low-resolution original image and the associated \gls{ssm}. The \gls{spic} framework demonstrated the capability to maintain high levels of semantic preservation while achieving substantial compression rates, thereby showcasing its potential as a viable alternative to classical image compression algorithms such as \gls{bpg} and \gls{jpeg2000}.

% Building upon the foundational work of \gls{spic}, the introduction of the \gls{cspic} further refined the approach by addressing the reconstruction of small and detailed objects within images. This enhancement was achieved without necessitating additional fine-tuning or retraining of the underlying \gls{semcore} model, thereby exploiting the framework's modularity and flexibility. The \gls{cspic} model underscored the importance of preserving critical semantic classes, ensuring that essential details (i.e. "traffic signs") are preserved. These level of semantic preservation was evaluated by the Traffic signs classification accuracy presented in \sref{sec: GM evaluation metrics}.

% In \cref{ch: SQGAN} the \gls{sqgan} model represented a novel integration of vector quantization and \gls{sc} principles. The \gls{sqgan} architecture incorporated a \gls{samm} to selectively transmit semantically relevant regions of the data. This selective encoding process significantly reduced redundancy and enhanced communication efficiency, particularly at extremely low \gls{bpp} values. The introduction of the \gls{samm} and the \gls{spe} facilitated the prioritization of latent vectors associated with critical semantic classes, thereby improving the overall reconstruction quality of important objects within images. Additionally, the designed Semantic Relevant Classes Enhancement data augmentation technique and the Semantic Aware Discriminator further refined the model's ability to preserve critical semantic information.

% In \cref{ch: Goal_oriented} asignificant contribution of this research was the exploration of goal-oriented resource allocation within \glspl{en}. By leveraging the \gls{ib} principle, the thesis addressed the challenge of dynamically adjusting compression parameters to balance the trade-off between compression efficiency and semantic preservation. The application of stochastic optimization techniques facilitated the optimal allocation of computational and communication resources, ensuring that the \gls{en} operates efficiently under varying network conditions and resource constraints. This integration of \gls{goc} principles with resource optimization strategies underscored the importance of adaptive and intelligent network management in modern communication infrastructures.

% Additionally, by employing the \gls{sqgan} model within the \gls{en} framework, the research demonstrated the potential of \gls{sgoc}. The integration of the \gls{sqgan} model within the \gls{en} architecture enabled the dynamic adjustment of the masking fractions based on real-time network conditions and resource availability. This approach ensured that the \gls{en} could autonomously optimize its operations, thereby enhancing communication efficiency and resource utilization in a goal-oriented fashion with focus on \gls{sc}.

% Throughout the research, the importance of modular and flexible framework design was emphasized. The proposed models, \gls{spic}, \gls{cspic}, and \gls{sqgan}, were designed to be easily integrated into existing communication systems without the need for extensive modifications. This design philosophy ensures that the advancements in \gls{sc} can be readily adopted in practical applications, facilitating the transition from traditional to intelligent communication paradigms.

% The comparative analysis of the proposed models against classical compression algorithms highlighted the superiority of semantic-aware approaches in preserving critical information at low bit rates. While traditional algorithms excel in minimizing pixel-level distortions, they fall short in maintaining the semantic integrity of the data. In contrast, the proposed semantic and goal-oriented models demonstrated enhanced performance in preserving meaningful content, thereby offering a more effective solution for applications where semantic accuracy is crucial.


