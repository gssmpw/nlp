\newpage

\section*{Abstract}
\label{sec:abstract}
% 150 - 300 words long
\begin{comment}
% Abstract 183w
The increased availability of data and computing resources has enabled researchers to successfully adopt machine learning (ML) techniques and make significant contributions in several engineering areas. ML and in particular deep learning (DL) algorithms have shown to perform better in tasks where a physical bottom-up description of the phenomenon is lacking and/or is mathematically intractable. Indeed, they take advantage of the observations of natural phenomena to automatically acquire knowledge and learn internal relations.
Despite the historical model-based mindset, communications engineering recently started shifting the focus towards top-down data-driven learning models, especially in domains such as channel modeling and physical layer design, where in most of the cases no general optimal strategies are known.

In this thesis, we aim at solving some fundamental open challenges in physical layer communications exploiting new DL paradigms. In particular, we mathematically formulate, under ML terms, classic problems such as channel capacity and optimal coding-decoding schemes, for any arbitrary communication medium. 
We design and develop the architecture, algorithm and code necessary to train the equivalent DL model, and finally, we  propose novel solutions to long-standing problems in the field. 
\end{comment}

% Abstract 302w
The recent surge in data availability and computing resources has empowered researchers to harness the potential of machine learning (ML) techniques, leading to significant advancements across various engineering fields. ML, particularly deep learning (DL) algorithms, excel in tackling problems where traditional physical modeling proves inadequate or mathematically intractable. In fact, DL leverages real-world observations of phenomena to automatically acquire knowledge and uncover inherent relationships within the data.

For the past decades, communication engineering has thrived on a foundation of meticulously crafted physical and mathematical models, which have played a pivotal role in groundbreaking achievements, forming the very backbone of our modern technological landscape. From the efficient transmission of information across the globe to the development of reliable communication protocols, these models have been instrumental in shaping the way we connect and share information. 
However, a recent paradigm shift is underway, with a growing focus on top-down, data-driven learning models. This shift is particularly evident in areas like channel modeling and physical layer design, where finding universally optimal strategies remains an open challenge.

This thesis aims to address some of the most crucial unsolved challenges within physical layer communications by leveraging innovative DL paradigms. We mathematically reformulate classic problems, such as channel capacity and optimal coding-decoding schemes, using the ML language, and we attempt to solve them for any arbitrary communication medium.
The envisioned methodology involves designing and developing the architecture, algorithms, and necessary code to train the corresponding DL model. By leveraging powerful tools, we propose novel solutions to long-standing problems in the field. For instance, one specific challenge we tackle is the construction of capacity-achieving input distributions in noisy channels. Our approach has the potential to significantly improve channel coding compared to traditional methods, effectively leading to a reduction in bit error rate for data transmission.

We believe this research holds significant promise for the development of next-generation communication systems and can potentially influence the design principles in future 6G standards and beyond.


