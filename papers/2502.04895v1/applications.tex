\chapter{Applications to Power Line Communications} % Cortical
\chaptermark{Applications to PLC}
%\thispagestyle{empty}
\label{sec:plc}

In this chapter, we discuss in detail the application of
ML and DL in the context of power line communication (PLC), a technology which exploits the existing power delivery infrastructure to convey information signals \cite{LampeTonelloSwart}.
We show that PLC represents an excellent case study due to the complex and intricate nature of the power line network. 
In fact, such network is not designed to carry
high frequency signals and multiple effects reduce the communication performances. High noise, electromagnetic interference, channel characterized by strong attenuation and frequency selectivity are the main challenges in PLC. Therefore, it is widely accepted in the community that to develop new efficient and reliable PLC technology it is fundamental to model the PLC physical layer \cite{LampeTonelloSwart}. 
Nevertheless, the complexity of the environment and the multi-parameter dependence render the development of mathematical models extremely challenging. For this reason, a phenomenological approach such as DL constitutes then a great opportunity. 
In the following sections we report several experiments to validate, in the context of PLC, the theoretical findings presented in this thesis. 

The results presented in this chapter are documented in \cite{ML_PLC, RighiniLetizia2019, Letizia2019a, TonelloImpedance, LetiziaIsplc2021, Letizia2019c}.

\section{PLC channel and noise generation}
\sectionmark{PLCs medium modeling}
\label{sec:plc_gan}

\subsection{Channel generation}
\label{subsec:plc_channel}

\subsubsection{Channel modeling methods}
The PLC CTF is usually modeled in time or frequency domain.
Time-domain models are generally described by multi-path effects, whereas frequency-domain models are characterized
by frequency notches. The latter model has been analyzed with three main approaches: bottom-up,
top-down and synthetic [2].

\begin{itemize}
    \item Bottom-up channel modeling refers to an approach where the channel impulse/frequency response is obtained via the application of transmission line theory to a specified network topology, cables and loads characteristics. Conventionally, this approach is applied to obtain a specific response and it is also referred to as deterministic model. Further, for complex networks the existing calculation methods are rather elaborated. 
    \item Top-down channel modeling refers to an approach where the channel impulse/frequency response comes from a parametric model with parameters obtained by fitting real data from measurements. The statistical top-down model allows generating channels with statistics close to those exhibited by real channels [8]. In particular, the model is simple, flexible, and it uses a small set of parameters. The parameters can be adjusted to generate channels according to a certain statistical class.
    \item The synthetic PLC channel model is a phenomenological method to emulate the statistics of the PLC channel. This method does not include the medium physical knowledge to define the model but only the CTFs distribution [3]. Hence, it is possible to develop a model purely based on the statistical description of the observed data, totally abstracting from the physical interpretation of the medium. The synthetic model can be obtained via generative models such as GANs, as described in Sec. \ref{sec:gan_ch_synthesis}.
\end{itemize}

\subsubsection{Metrics for model validation}
The dataset used for the implementation consists of $1312$
measured single-input-single-output CTFs with bandwidth of
$100$ MHz obtained from measurements in the in-home environment
\cite{tonello2014inhome}. The frequency resolution was set to $62.5$
KHz which leads to $1601$ frequency samples. To overcome
lack of channel realizations and complexity problems we
decided to downsample by a factor of $32$. To work with
complex numbers, a vector of $100$ samples has been defined,
whose first $50$ entries correspond to the real part of $H(f)$ and the last flipped $50$ to the imaginary one. This set-up constitutes our input data $\mathbf{x}$: $1312$ realizations of a random variable of dimension $100$.

We use the following metrics to assess the quality of the generated CTFs:
\begin{itemize}
    \item The average channel gain (ACG) corresponds to the mean value of the magnitude of CTF in dB scale
\begin{align}
ACG=\frac{1}{N} \sum\limits_{k=0}^{N-1} |H_k|^2 = T_s^2 \sum\limits_{k=0}^{N-1} |h_k|^2 
\label{eq:plc_acg}
\end{align}
where $|H_k|$ is the CTF sampled with frequency resolution $1/(NT_s)$ and obtained as the $N$-points DFT of the impulse response $h(n
T_s)$. The ACG in dB scale is defined as $ACG|_{dB}=10\log_{10}(ACG)$;

\begin{figure}
\centering
\begin{subfigure}{0.5\textwidth}
  \centering
 \includegraphics[scale=0.4]{images/plc/H_misurati2.pdf}
	\caption{database of measurements.}
	\label{fig:plc_H_measured}
\end{subfigure}%
\begin{subfigure}{0.5\textwidth}
  \centering
 \includegraphics[scale=0.4]{images/plc/H_generati2.pdf}
	\caption{generated with AE \& GAN.}
	\label{fig:plc_H_generated}
\end{subfigure}
\caption{10 randomly picked CTFs. Magnitude and phase.}
\label{fig:plc_CTFs}
\end{figure}

\item The root mean square delay Spread (RMS-DS) is the square root of the second central moment of the power-delay profile, defined as
\begin{align}
\sigma_\tau = \sqrt{\mu_\tau^{''} - \mu_\tau^2}
\label{eq:plc_mu0}
\end{align}
with 
\begin{align}
\mu_\tau &= \frac{ \sum\limits_{n=0}^{N-1} nT_s|h(nT_s)|^2}{\sum\limits_{n=0}^{N-1} |h(nT_s)|^2}  \\
\quad
\mu_\tau^{''}
&= \frac{ \sum\limits_{n=0}^{N-1} (nT_s)^2 |h(nT_s)|^2}{\sum\limits_{n=0}^{N-1} |h(nT_s)|^2},
\label{eq:plc_mu1}
\end{align}
it is used to measure the multi-path spread and characterize the effect of channel dispersion on a transceiver. 

\item The coherence bandwidth (CB) measures the frequency selective behavior of the channel. This can be analyzed in terms of the autocorrelation function of the frequency response, defined as 
\begin{align}
R(\Delta f)=\int_{B}^{} H(f)H^*(f+\Delta f)df
\label{eq:plc_Rdf}
\end{align}
where $^*$ denotes the complex conjugate, $\Delta f$ is the frequency shift and $B$ is the channel band.
\end{itemize}

The analysis is enhanced with the mean and the statistical moments (standard deviation (STD), skewness (SKW) and kurtosis (KURT)) computed on the CTF magnitude.

\subsubsection{Results}
This paragraph presents the list of results achieved with the generative system described in Sec. \ref{sec:gan_ch_synthesis}. 
The hidden representation has been extracted according to the parameters listed in Tab. \ref{tab:autoencoder_nn}. 

Figures \ref{fig:plc_H_measured} and \ref{fig:plc_H_generated} show the measured and generated CTFs, respectively. The generated trends look very consistent with the measured ones. The magnitude spans the large dynamic range of attenuation from $-10$ to $-100$ dB and the characteristic frequency notches are clearly visible. The unwrapped phase exhibits similar trends along frequency for both generated and measured CTFs. 

\begin{figure}[t]
	\centering
	\includegraphics[width = 8cm, height = 5cm]{images/plc/ACG_final_crop.pdf}
	\caption{ACG boxplot comparison.}
	\label{fig:plc_acg}
\end{figure}

The boxplot in Fig. \ref{fig:plc_acg} shows the comparison of ACG for four different frequency ranges. Respectively 1, 2, 3, 4 refer to 0:30 MHz, 2:15 MHz, 0:100 MHz and 2:100 MHz. It allows for an agile representation of the minimum, maximum, 25th and 75th percentiles, mean and outliers. The graph shows that the metrics are almost matched for all frequency bands. The highest discrepancies are visible at the lowest frequencies.
Tab. \ref{tab:plc_metrics_comparison} provides the full list of metrics considered in the assessment of the model. The 25th and 75th percentiles and mean are shown for both measured and generated data in the PLC broad band spectrum (up to 100 MHz). The values share similar ranges for both measured and generated data. The computed CB on the data is very high due to the low resolution ($2$ MHz) which is the only drawback of adopting this methodology. Thus, in this case, it should not be considered a useful metric to characterize the CTF. Nevertheless, both generated and measured CB values are similar. 

\renewcommand{\arraystretch}{1.7} % allarga l'altezza delle righe
\begin{table}[t]
	\scriptsize % text dimension
	\centering
	\caption{Metrics comparison between measured and generated data. The table shows the metrics in range 0:100 MHz.}
	\begin{tabular}{c|*3c|*3c} 
		\toprule
		Metrics
		          & \multicolumn{3}{c}{Measured}  & \multicolumn{3}{c}{Generated} \\
                        & 25p   & mean & 75p        & 25p   & mean & 75p      \\
		\midrule
		ACG [dB]        & -39,6 &	-31,9&	-25,6   & -38,3&	-31,7&	-24,5 \\ \hline
		RMS-DS [$\mu$s] & 0.126 &	0.136&	0.152   & 0.082&	0.105&	0.129 \\ \hline
        CB [MHz]        & --    &   4.0  & --       &  --  &  4.2    &    --  \\ \hline
        STD             & 0.006 & 0.029  &	0.032   & 0.007 &	0.037 &	0.041 \\ \hline
        SKW             & -1.79 &	0.083&	1.80    & -0.030&	1.988 &	4.45  \\ \hline
        KURT            & 7.04  &	18.68&	27.84   & 7.15  &	20.33 &	31.54 \\ 
		\bottomrule	
	\end{tabular}	
	\label{tab:plc_metrics_comparison}
\end{table}

\subsection{Noise generation}
\label{subsec:plc_noise}

\subsubsection{Traditional noise modeling methods}
The PLC noise signal is generated by different sources and
possesses peculiar properties. Power line networks are
usually made by unshielded cables that tend to couple with
both radiated and conducted noise signals that are traveling
in the surroundings of the network. Moreover, the appliances
connected to the power line generate and inject noise components that
may have a cyclostationary behavior with period related to the
mains frequency of $50$/$60$ Hz.
In literature, the PLC noise is described as a sum of
five components: colored background noise (CBGN), narrow-band noise (NBN), periodic impulsive noise synchronous
to mains (PINS), periodic impulsive noise asynchronous
to mains (PINAS) and aperiodic impulsive noise (APIN) \cite{Zimmermann2002}:
\begin{itemize}

\item The CBGN class \cite{6288568} represents the noise as a process obtainable as follows: $n_{CBGN}(t) = h_{CBGN}* (n_W(t) \cdot \sigma)$;
where $h_{CBGN}$ is a LTI filter, $n_W$(t) is a white Gaussian noise, $\sigma$ is the constant standard deviation. In the literature, it is reported that this type of noise exhibits a stationary behavior \cite{1650328}. Two types of PDFs are associated to the CBGN noise: Normal and Alpha stable. 

\item The NBN class models the noise as a sum of different independent harmonic interferences \cite{5570980,7248409}. 
\begin{align}
n_{NBN}(t) = \sum_{j=1}^{N_{NBN}} A_j(t) \sin(2\pi f_it+\phi_i(t)),  
\label{eq:plc_nbn}
\end{align}
where $N_{NBN}$ is the number of independent interferences, $A_i(t)$, $f_i$ and $\phi_i$ are respectively the time envelope, the central frequency and the phase of the $j^{th}$ interference. 
Typically, it is reported that this type of noise exhibits two different time behaviors: stationary and cyclostationary. 

\item The PINS and the PINAS are described by a summation of different independent Gaussian components with a time varying power \cite{1650328,HanStoicaKaiser2016_1000056745}. Usually, the inter-arrival time between pulses, amplitude and width are the main parameters to characterize this type of noise. Generalized extreme value PDF are associated to PINS noise.

\item The APIN exhibits no deterministic behavior and it is usually described with a Middleton class A model \cite{4091283,6547827}. 
\begin{align}
p_{APIN}(n) &= \sum_{j=0}^{\infty} \frac{e^{-A}A^j}{j!} \frac{1}{\sqrt{2\pi \sigma^2_j}}\exp \left(-\frac{n^2}{2\sigma^2_j}\right),
\label{eq:plc_apin}
\end{align}
where $p_{APIN}(n)$ is the PDF of the noise amplitude. $A$ is the impulsive index, $\sigma_j$ are the standard deviations of each Gaussian PDF.
\end{itemize}

The noise in multi-conductor power lines keeps the aforementioned classes and assumes interesting joint characteristics between different conductor pairs. An example is the high spatial correlation between the cyclostationary noise signals on different phases. This is usually caused by network-based switched power supplies \cite{8245785}. Moreover, measurements demonstrate that multi-conductor noise is not only correlated, but sometimes reveals deterministic behaviors \cite{8360239}. 

We use the GAN-based methodology described in Sec. \ref{sec:medium_channelsynthesis} to model the noise.

\subsubsection{Model validation and results}
The dataset consists of PLC noise measurements in the narrow-band spectrum (from $3$ kHz to $500$ kHz) \cite{8360239}. The dataset contains two traces of multi-conductor
noise sampled at $1$ MSa/s. The multi-conductor noise was
acquired between live and protective-earth and between neutral
and protective earth pairs. The noise traces were cut in time
windows of $1024$ samples, to obtain a new dataset of $12540$
noise frames.

\begin{table}
	\scriptsize % text dimension
	\centering
	\caption{Features statistics.}
	\begin{tabular}{c|cc|cc|cc} 
		\toprule
		\textbf{features}  & \multicolumn{2}{c|}{\textbf{mean}} & \multicolumn{2}{c}{\textbf{std}} & \multicolumn{2}{|c}{\textbf{median}} \\
		& generated &	real&	generated&	real&	generated&	real \\
		\midrule
		max [V]& 0.06&	0.08&	0.06&	0.09&	0.04&	0.05 \\
		mean [mV]& -6.2&	0.46&	10.01&	24.58&	-4.39&	0.280\\
		energy [$\mu J$]& 0.11&	0.26&	0.54&	1.10&	0.027&	0.039\\
		std [V] & 0.03&	0.03&	0.03&	0.05&	0.02&	0.02\\
		skewness& -0.18&	-0.01&	0.42&	0.44&	-0.19&	-0.01\\
		kurtosis& 2.29&	2.36&	0.55&	0.65&	2.2&	2.24\\
		entropy& 1.48&	1.52&	0.49&	0.56&	1.45&	1.52\\
		peaks $>$ 0.05 V & 2.56&	5.25&	5.62&	9.61&	0&	0\\
		skew. auto.& 0.1&	0.13&	0.52&	0.5&	0.07&	0.13\\
		kurt. auto.& 3.03&	2.66&	0.65&	0.76&	2.95&	2.62\\		
	\end{tabular}
	\label{tab:plc_features_stat}
\end{table}

\begin{figure}
	\centering
	\includegraphics[scale=0.6]{images/plc/boxPlot_features_new.pdf}
	\caption{Normalized box-plot of several features for both the generated and real noise traces.}
	\label{fig:plc_Boxplot}
\end{figure}

\begin{figure}
	\centering
	\includegraphics[scale=0.6]{images/plc/bar_plot_pdf.pdf}
	\caption{Bar plot of PDF classification for the generated and real noise traces.}
	\label{fig:plc_Barplot}
\end{figure}

Considering the physical nature of the noise measurements,
we rely on statistical physical properties of the noise
to assess the goodness of the generation method. However,
qualitative metrics based on human visual perception are
useful during the training process in order to understand the
status of the network. Furthermore, images provide a quick
visual understanding of the features similarities which in
general is hard to formulate as a mathematical problem. As
an example, the top row of Fig. \ref{fig:plc_Spectrogram} shows two spectrograms,
a real and a generated one, close to each other. The structure
similarity in the spectrograms is depicted also in the noise
traces. The same type of consideration can be made for multiconductor noise generation in Fig. \ref{fig:medium_Block-Spectrogram}. The synthetic block-spectrogram has not only a similar structure compared to the real one, but it also provides a double internal structure which is reflected into a sort of negative correlation between the two noise traces, as in the real scenario.

\begin{figure}[t]
	\centering
	\includegraphics[scale=0.45]{images/plc/Spectrograms_noise_plc_graph.pdf}
	\caption{Spectrogram representations of randomly picked real noise measurement and generated noise trace.}
	\label{fig:plc_Spectrogram}
\end{figure}

We consider several quantitative parameters to statistically compare the generated and real noise traces. During such analysis, we applied the method explained in \cite{righiniAutomaticClustering} to both the generated and real noise traces. Practically, the quantitative evaluation procedure is split into three phases: a first step to extract a dataset of physical features calculated for the noise traces, a second part in which the self-organizing map (SOM) clustering algorithm is applied to isolate different noise classes, and a third last phase where a labeling system associates PDFs of a certain type to the identified clusters. The considered PDFs are (1) = Extreme value, (2) = Generalized extreme value, (3) = Normal, (4) = Middleton class A, (5) = Alpha stable. 

A first glance of the normalized features is shown in Fig. \ref{fig:plc_Boxplot}. In details, the shown features are: (1) maximum of the magnitude, (2) mean, (3) energy, (4) standard deviation, (5) skewness, (6) kurtosis, (7) estimated entropy, (8) peaks over 0.05 V, (9) skewness of the autocorrelation and (10) kurtosis of the autocorrelation. The boxplot illustrates a correspondence between the fake and real noise statistical features. A quantitative analysis is also reported in Tab. \ref{tab:plc_features_stat}. In particular, the values corresponding to the generated traces are consistent with the statistical analysis conducted on the real noise measurements.
As an example, features such max, energy, standard deviation, kurtosis, entropy and skewness autocorrelation are matched in mean, standard deviation and median. Other features instead, such as the mean and peaks over 0.05 V, show different statistical moments, sign of possible improvements during the generation training process. Another important physical properties of the noise come from its spectrum analysis, specifically from the analysis of either the Fourier transform or the power spectral density (PSD). Fig.~\ref{fig:plc_FFT} illustrates the average value of the Fourier Transform computed for each fake and real noise trace. The higher discrepancies between the two types of noise are indeed detectable in the frequency domain where the fake noise has not completely learned yet how to replicate the frequency information. 

The last clustering phase deals with a comparison of PDFs. A certain PDF is associated for each noise cluster, and the difference between generated and real data can be formulated as the probability to identify a certain PDF. Fig.~\ref{fig:plc_Barplot} shows that similar percentages are obtained for normal, Middleton class A and Alpha stable PDFs. On the other hand, the percentages of extreme value and generalized extreme values PDFs are slightly different. An immediate consequence of this consideration is the fact that the generation method is not accurate in fully replicating the impulsive noise, leaving space to future improvements, for instance via StyleGANs.


\begin{figure}[t]
	\centering
	\includegraphics[scale=0.5]{images/plc/average_FFT.pdf}
	\caption{Average Fourier transform magnitude for both the real and generated noise traces.}
	\label{fig:plc_FFT}
\end{figure}

\section{Learning impedance entanglements}
\sectionmark{PLC impedance entanglement}
\label{sec:plc_entanglement}
In wireline communication networks, a line impedance entanglement (IE) exists since changes of the line impedance at one network port cause a change of the line impedance at the other port. This physical phenomenon can be constructively exploited to realize a form of digital modulation that is referred to as impedance modulation (IM). IM is an alternative method to more conventional voltage modulation (VM). The IE can be studied in ML terms, enabling the implementation of a ML based receiver. Numerical results are obtained in a dataset of measured power line communication channels, which is among the most challenging environments for such a modulation approach. The resulting system can have practical implementation, for instance in a smart building automation network where monitoring-control of sensors and devices enables the efficient energy management.

Comparisons with the optimal MaxL receiver that perfectly knows the IE transfer function are made. It is found that the ML based receiver performs close to the optimal genie receiver.

\subsection{The impedance entanglement}
In recent work \cite{TonelloIsplc2020}, a fundamental question has been posed: \emph{is it possible to encode information and physically transmit it over a wireline medium in a different way other than a VM?} The answer has been found by the analysis of the PLC medium, where it has been shown that a change of the impedance at one channel port induces a change of the impedance on the other port \cite{tcas}, \cite{tee}. This can be proved with transmission line theory arguments and it can be referred to as IE. It is important to observe that the intensity of the IE depends on the physical medium characteristics (electrical characteristics of the wires, network topology, and loads). In many applications and scenarios of practical relevance the exploitation of the IE enables another mode of transmission. In addition, it can be jointly used with VM, which generates two parallel (although coupled) communication channels.

We assume a wireline system, e.g., a PLC system, where physical wires offer a medium for data transmission. Impedance modulation is exploited to encode and transmit information between two network ports \cite{TonelloIsplc2020}. The transmitter is solely composed by a set of impedances, which are selected through mapping of a sequence of data symbols. Each impedance encodes a symbol to be transmitted. The receiver goal is to decode the impedance coded at the transmitter side exploiting the IE. This is accomplished by using a shunt based receiver \cite{TonelloIsplc2020}. It consists of a signal generator and a measurement of the currents and voltages at the receiver port.
Assuming to represent voltages and currents with phasors, the wired channel acts as a complex and non-linear function of the transmitter impedance. In fact, from the microwave circuit model where the ABCD matrix \cite{pozar}
is deployed, the relationships among the injected currents and the applied voltages are described by
\begin{equation}
	\begin{cases}
		V_{r}(f) = A(f)V_{tx}(f) + B(f)I_{tx}(f) \\
		I_{r}(f) = C(f)V_{tx}(f) + D(f)I_{tx}(f)
		\end{cases}.
	\label{eq:plc_VIinVIout}
\end{equation}
$V_{tx}$ and $I_{tx}$ are the voltage and the current at the transmitter side, respectively. It is easy to understand that the ratio between these two quantities gives the transmitter impedance $Z_{tx} = \frac{V_{tx}}{I_{tx}}$. The system model and related quantities is reported in Fig. \ref{fig:plc_abcd}.
\begin{figure}
	\includegraphics[width=\columnwidth]{images/plc/abcd}
	\caption{Impedance modulation scheme with shunt based receiver.}
	\label{fig:plc_abcd}
\end{figure}
The impedance seen at the receiver port $Z_{r}$ (i.e., the line impedance at the port $1-1'$) depends on the physical wireline network and on the impedance of the transmitter $Z_{tx}$. The relation can be obtained from (\ref{eq:plc_VIinVIout}), and it reads
\begin{equation}
	Z_{r} = \frac{AV_{tx}+BI_{tx}}{CV_{tx}+DI_{tx}} =  \frac{AZ_{tx}+B}{CZ_{tx}+D}.
	\label{eq:plc_Zr}
\end{equation}
As it becomes clear from (\ref{eq:plc_Zr}), the impedance $Z_{r}$ is a function $\Omega({\bold{T},Z_{tx},f})$ of the ABCD channel matrix $\bold{T}$, the transmitter impedance $Z_{tx}$ and the frequency $f$. The receiver is designed to measure the impedance $Z_{r}$ with the deployment of a known shunt impedance $Z_{s}$ and a voltage generator $V_{s}$. In fact, $Z_{r}$ can be retrieved from the measurement of the shunt voltage $V_{Z_s}$ since
\begin{equation}
	Z_{r} = \left(\frac{V_s}{V_{Z_s}}-1\right)Z_s.
	\label{eq:plc_ZrV}
\end{equation}
The performance of the modulation scheme depends on the IE and it is additionally influenced by the presence of noise which adds onto the measured shunt voltage drop.  In accordance with (\ref{eq:plc_Zr}), the function $\Omega(\bold{T},Z_{tx})$ exhibits a non-linear behaviour strongly influenced by the parameter $C$. The parameter $C$ of the ABCD matrix $\bold{T}$ represents the transadmittance of the modeled channel. The higher the transadmittance is, the higher the fraction of the injected current $I_r$ lost in the channel is. In turn, this is reflected into smaller sensitivity to the variations of the measured voltage $V_{Z_s}$ induced by the variations of $Z_{tx}$, i.e., lower IE.

The effect of the non-linearity of $\Omega$ is exemplified in Fig. \ref{fig:plc_nonlineare} which shows the distortion of the transmitted constellation (comprising four different impedance values) introduced by the channel. Higher values of $C$ reduce the distance among the constellation points challenging further the symbol detection.
  \begin{figure}
  	\includegraphics[width=\columnwidth]{images/plc/nonlineare}
  	\caption{Impedance entanglement dependence of channel transadmittance $C$.}
  	\label{fig:plc_nonlineare}
  \end{figure}

In this section, a different and simpler approach from the one and more general followed in \cite{TonelloIsplc2020} is presented.
It is assumed to start by having two possible resistive states at the transmitter side, i.e., a short circuit resistance $Z_{tx}^{0}=0 \Omega$ and an open circuit resistance $Z_{tx}^{\infty}$ (with a resistance set to 1 G$\Omega$). These two extreme impedance values induce at the receiver side two distinct impedance profiles. For instance, considering (\ref{eq:plc_Zr}), and substituting the aforementioned transmitter impedance states, two different values of impedance $Z_r$ are obtainable, i.e. $Z_r^{0} = \frac{B}{D}$ and $Z_r^{\infty} = \frac{A}{C}$.
Qualitatively, it is easy to understand that the greater the difference between the two impedance profiles in a certain frequency band, the higher the probability to correctly discriminate the correct value at the receiver side is.

Since what is directly measurable is the shunt voltage drop, it is useful to look at it. In fact, the impedance state at the transmitter induces a certain shunt voltage response. In particular, such a response is informative if evaluated in a large spectrum, e.g., in the bandwidth $BW={1-100} MHz$ spectrum. Such a voltage response can be used to define a measure of the IE. The discriminative measure that we use herein is based on the average distance of the two voltage signals entangled at the receiver. Taking into account (\ref{eq:plc_ZrV}), and assuming a constant value for both the shunt impedance $Z_s$ and the shunt voltage generator $V_s$, the entanglement measure $E_V$ is defined as
\begin{equation}
	E_V = \frac{|Z_s|}{|V_s|} \int_{BW} |V_d(f)|\,df,
	\label{eq:plc_wV}
\end{equation}
where $V_d(f)$ is the voltage defined as the difference $V_s^{0}(f)-V_s^{\infty}(f)$, in which $V_s^{0}(f)$ and $V_s^{\infty}(f)$ are the two measured voltages referred to the two impedances $Z_r^{0}(f)$ and $Z_r^{\infty}(f)$, respectively.
High values of $E_V$ correspond to have noticeable differences between the two measured voltages, thus to more detectable impedances and higher entanglement.

\subsubsection{Representative channels: the in-building PLC case}
A set of measured broad-band PLC channels has been exploited for practically assessing the theoretical discussion of previous sections. In particular, about 1200 PLC channel measurements are taken into account, and for each of them the entanglement measure (\ref{eq:plc_wV}) is calculated. A subset of three representative channels are chosen considering the 0.1, 0.5 and 0.9 quantiles of (\ref{eq:plc_wV}). Fig. \ref{fig:plc_wV} reports the voltage difference $|V_s^{0}(f)-V_s^{\infty}(f)|$ expressed in dB$\mu$V, evaluated on a 50 $\Omega$ shunt. The test voltage $V_s$ has been set to an amplitude of 0.0224 V at each frequency, which corresponds to a dissipated power of -50 dBm/Hz.
  \begin{figure}
  \centering
	\includegraphics[scale=0.75]{images/plc/wV}
	\caption{Voltage difference magnitude measured with the shunt-based receiver when the transmitter impedance swaps from $Z_{tx}^{0}$ to $Z_{tx}^{\infty}$. Different colours refer to the different quantile values.}
	\label{fig:plc_wV}
\end{figure}

We now discuss the MaxL based receiver and then we present a ML based approach to learn the IE and realize a practical receiver.

\subsection{Maximum-likelihood based receiver}
To measure the impedance $Z_{r}$ at the receiver, multiple circuit architectures can be used \cite{Passerini2017, Hallak2018}. We consider a shunt based impedance meter as shown in Fig. \ref{fig:plc_abcd} that is capable of measuring $V_s$. The measurement is done on a wide spectrum. To derive an optimal receiver metric, and not to neglect the presence of noise (denoted with $V_n$), we can rearrange the above relations to obtain
\begin{equation}
V_{Z_s,n}= V_s Z_s\frac{A + CZ_{tx}}{(A+CZ_{tx})Z_s +B+DZ_{tx}} + V_n.
\label{eq:plc_Vzs}
\end{equation}
If we assume to map data symbols into a discrete set of values of $Z_{tx}$, and to consider the relation (\ref{eq:plc_Vzs}), then a MaxL estimate is obtained as follows

\begin{equation}
Z_{tx}=\arg\min_{Z}\left\{\left|V_{Z_s} - V_sZ_s\frac{A+CZ}{(A+CZ)Z_s + B + DZ}\right|\right\}.
\label{eq:plc_zgarg}
\end{equation}

It should be noted that the MaxL receiver requires the knowledge of the entanglement transfer function, therefore of the ABCD parameters. In the numerical results, we assume a genie receiver that perfectly knows them. This is used as a baseline situation to make a comparison with the practical ML receiver proposed. It should also be noted that IM does not require any voltage source at the transmitter side, therefore it can be considered a sort of passive modulator since it requires only switching among a finite set of impedances at the transmitter node. It can be realized with a bank of impedances and controlled radio frequency switches. 

\subsection{Machine learning-based receiver}
\subsubsection{Training set and parameters}
The shunt voltage including noise is described as
\begin{equation}
V_{Z_s,n} = V_{Z_s}+V_n,
\end{equation}
where $V_n \sim \mathcal{CN}(0, \sigma_n^2)$. The measured voltage is a non-linear function of the impedance state $Z_{tx}$ at the transmitter side as shown in (\ref{eq:plc_Vzs}). Hence, if we assume that $Z_{tx}$ possesses only two states, it is in principle possible to detect these states by training a binary classifier with the pair of samples $(V_{Z_s,n},Z_{tx})$.

Each channel differently distorts the transmitted impedance as in (\ref{eq:plc_Zr}). Therefore, as a proof of concept, we consider two scenarios to train the symbol detector/classifier: a first scenario where only one channel realization correspondent to the $0.9$ quantile is used to obtain the shunt voltage $V_{Z_s,n}$ over a wide spectrum; a second scenario where $15$ channel realizations in the neighbourhood of the $0.9$ quantile are used to obtain $V_{Z_s,n}$. With the former, we want to prove that a machine learning-based detector does exist and achieves performance comparable to the optimal MaxL receiver. The latter, instead, extends the classification under a variety of channels with a common discriminative feature, that is, high value of $E_V$. For both scenarios, the training phase was conducted using $50\%$ of the data, whereas the testing phase was performed using the other half. For each channel realization, we consider $1000$ noisy voltages $V_{Z_s,n}$ obtained by transmitting the short and open resistor states, denoted with the class $0$ and $1$, respectively. The voltage $V_{Z_s,n}$ is split in real and imaginary parts, and they both possess a total of $1569$ frequency points from $1$ to $100$ MHz which are concatenated in a unique input vector. As common practice, the input realization has been normalized to values inside the interval $[-1,1]$, to avoid ill-conditioned networks, as follows
\begin{equation}
\label{eq:plc_normalization}
\hat{x}_i = 2\frac{x_i-\min(x_i)}{\max(x_i)-\min(x_i)}-1.
\end{equation}
We have implemented a simple shallow NN with one hidden layer of only $100$ neurons and sigmoidal activation function. The training set has been further divided into two subsets, a pure training set used to update the parameters of the artificial neural network and a validation set used to evaluate the performance of the network at each iteration. To avoid overfitting, we early stopped the training process after $10$ consecutive iterations where the error in the validation set increases. We have repeated the same experiment $10$ times in order to reduce the bias coming from the parameters initialization.

\subsubsection{Individual channel realization training}
As a first analysis, we consider a given channel realization, referred to as individual channel (IC), and train a classifier via cross-entropy to estimate the impedance state $Z_{tx}$ from the measured noisy sample $V_{Z_s,n}$, with different values of the noise variance $\sigma_n^2$. Once the training process is concluded, the testing phase kicks in and we evaluate the network performance by measuring the BER varying the SNR ratio. Given the network architecture described before, we follow two distinct training approaches. In the first approach, referred to as multi-network, we train several different NNs for every possible SNR value of interest and for each of them we compute the BER using the test set. In the second naive approach instead, referred to as single-network, we train only one classifier assuming an SNR value of $20$ dB and generalize it to unseen SNRs. Surprisingly, the BER obtained with the network trained with only one value of SNR is always lower than the BER experienced from individual networks trained for each specific value of SNR. This counter-intuitive result is a direct consequence of the overfitting issue. Indeed, when training networks with low SNR values, the network attempts to fit the noisy measurements rather than understanding the noise distribution. Fig. \ref{fig:plc_ber1} compares the single and multi-network machine learning-based approaches with the optimal MaxL receiver. In particular, it illustrates the accuracy of the impedance state prediction in terms of BER for different values of the SNR.
\begin{figure}
\centering
  	\includegraphics[scale=0.35]{images/plc/SNR_BER_single_channel.pdf}
  	\caption{BER of single and multi-network approaches compared to the optimal MaxL receiver.}
  	\label{fig:plc_ber1}
\end{figure}

It is interesting to notice also the following. The ML approach provides worse performance of less than $1$ dB compared to the genie MaxL receiver with the AWGN model. The approach can be generalized to non-Gaussian noise for which no optimal receivers are known. In addition, the single-network approach only requires few data and few iterations to achieve competitive performance. Conversely, the genie MaxL receiver requires the knowledge of the IE transfer function which implies that estimation algorithms have to be realized.

\begin{figure}
\centering
  	\includegraphics[scale=0.35]{images/plc/SNR_BER_multi_channel.pdf}
  	\caption{BER of single and multi-network approaches obtained with channel diversity compared to the average BER obtained with single-network training on individual channels.}
  	\label{fig:plc_ber2}
\end{figure}

\subsubsection{Multiple channel realizations training}
Perhaps more fascinating is the idea of a general classifier able to account for the channel diversity, thus, trained on multiple channel realizations. It is clear, however, that the multiple channel realizations shall share a common discriminative feature, for example the discrimination quantity $E_V$ defined in (\ref{eq:plc_wV}). To prove that it is indeed possible to build a unique NN that detects the input resistive state in the presence of multiple channel realizations, we consider a set of $15$ channel realizations as described above. We repeat the training approach used for the individual training, in particular, we train a single-network classifier at $20$ dB and multi-networks classifiers each for every SNR value. Similarly to the single channel realization case, the best performance in terms of BER is achieved for the single-network approach where the network learns to generalize for unseen low values of SNR (lower than the $20$ dB of training). Conversely to the individual channel case, the performance of the multi-network training is better compared to the single-network one for higher SNR values.
Fig.\ref{fig:plc_ber2} reports the average BER obtained with the single and multi-network approaches in the presence of multiple channel realizations (unique network for all channel realizations at a given SNR, and unique network for all channel realizations and SNRs). The average BER obtained over the $15$ channel realizations when training an individual network for each given channel realization and SNR (IC-network) is also shown. The performance loss of using a unique neural network for all channel realizations w.r.t. to an individual neural network for each channel realization is up to $7$ dB.

\subsection{Summary}
A ML approach can be followed where joint IE learning and impedance detection are obtained by training a NN classifier. Numerical results have been obtained using a set of measured power line channels in the 1-100 MHz spectrum.
The results show the great potential that ML and DL algorithms have when applied to detection tasks. Various training approaches have been followed and they enable performance close to the optimal genie MaxL receiver. They stimulate further studies to include higher order impedance modulation and different neural architectures for improved performance, as well as the possibility to improve the bandwidth efficiency by considering narrower frequency intervals where the IM is implemented. In addition, an intriguing idea consists of an end-to-end learning scheme where the transmitter maps data symbols in optimal impedance constellation points so that the receiver exhibits the best detection performance and automatically learns the decoding strategy, for instance using CORTICAL (see Sec. \ref{sec:cortical_theory}).
  
\section{Capacity learning for additive channels under Nagakami-$m$ noise}
\sectionmark{PLC capacity learning}
\label{sec:plc_nakagami}
The development of PLC systems and algorithms is significantly challenged by the presence of unconventional noise. The analytic description of the PLC noise has always represented a formidable task and less or nothing is known about optimal channel coding/decoding schemes for systems affected by such type of noise. 
In this section, we present a statistical learning framework to estimate the capacity of additive noise channels, for which no closed form or numerical expressions are available. In particular, we study the capacity of a PLC medium under Nakagami-$m$ noise and determine the optimal symbol distribution that approaches it. We provide insights on how to extend the framework to any real PLC system for which a noise measurement campaign has been conducted. Numerical results demonstrate the potentiality of the proposed methods.

\subsection{Capacity learning under Nakagami-$m$ noise}
\label{subsec:plc_capacity_nakagami}
The amplitude of the PLC background noise in the broadband frequency range $1-30$ MHz is statistically modeled by the Nakagami-$m$ distribution with $m<1$ \cite{Meng2005}. Hence, in a quadrature modulation scheme, noise $\mathbf{n}$ can be represented as
\begin{equation}
\mathbf{n} = w \exp(j\theta) = n_r + jn_i,
\end{equation}
where $j=\sqrt{-1}$ and the amplitude $w$ is distributed as
\begin{equation}
p_W(w) = 2\frac{(m/\sigma_n^2)^m}{\Gamma(m)}w^{2m-1}\exp\biggl(-\frac{m w^2}{\sigma_n^2}\biggr).
\end{equation}
However, the in-phase and quadrature components of the noise have not equal variances as show in \cite{Mallik2010}. Therein, it has also been observed that in the interval $1/2\leq m <1$, the characteristic function of the real part $n_r$ of the noise resembles the one of a zero-mean Gaussian distribution of variance $\sigma_n^2(1+b)/2$. Similarly, the characteristic function of the imaginary part $n_i$ approximates the one of a zero-mean Gaussian distribution of variance $\sigma_n^2(1-b)/2$, with real and imaginary part being independent and
\begin{equation}
b = \sqrt{\frac{1}{m} -1}.
\end{equation}
Notice that the average SNR per symbol of the constellation modeled by the generator of the cooperative capacity framework is equal to $P/\sigma_n^2$. Moreover, if $b=0$, the additive noise is white Gaussian.

The design of optimal receivers and the PLC system error analysis affected by this type of noise have been discussed with binary phase-shift keying (BPSK) signaling in \cite{Dash2016} and \cite{Mathur2014}. Recently, an optimal Quadrature Phase Shift Keying (QPSK) constellation for this noise model has been presented in \cite{Reddy2020}, albeit a capacity characterization was not  treated.

We exploit the aforementioned DL approach to present results concerning the optimal quadrature amplitude modulation (QAM) constellation design under an average power constraint. We also report MI boundaries. Optimal neural network-based receivers are left for future studies, although a trivial extension using the autoencoders for communications concept is expected to follow some of the ideas in \cite{Oshea2017,Letizia2021}. 

\subsection{Results}
\label{subsec:plc_results}
In this section, we present the results for the two proposed scenarios: additive Nakagami-$m$ noise, which typically describes the background noise in the broadband spectrum, and additive noise traces obtained via a measurement campaign in the narrowband spectrum. For the former case, we exploit the mutual information estimator (DIME) and the cooperative capacity learning framework (CORTICAL), presented in Sec. \ref{sec:mi_f-DIME} and \ref{sec:cortical_theory}, to estimate the channel capacity and to build the optimal channel input distribution, under an average power constraint. In the latter scenario, we estimate the MI between the Gaussian distributed channel input and the channel output, affected by additive measured narrowband noise. For both cases, we show that the achieved information rate is higher than the AWGN channel capacity.

\begin{figure}
\begin{subfigure}{0.5\textwidth}
	\centering
	\includegraphics[scale=0.25]{images/plc/capacity_nakagami.pdf}
	\caption{Continuous channel input distribution.}
	\label{fig:plc_capacity_nakagami}
\end{subfigure}
\begin{subfigure}{0.5\textwidth}
	\centering
	\includegraphics[scale=0.25]{images/plc/capacity_nakagami_discrete.pdf}
	\caption{Discrete channel input distribution.}
	\label{fig:plc_capacity_nakagami_discrete}
\end{subfigure}
\caption{Maximal mutual information estimation between channel input and output using DIME in a channel corrupted by additive Nakagami-$m$ noise under an average power constraint.}
%\label{fig:plc_capacity_nakagami}
\end{figure} 

\subsubsection{Optimal constellation under Nakagami-$m$ noise}
\label{subsec:plc_optimal_constellation}
We parametrize both the generator $G$ and the discriminator $D$ of the CORTICAL framework with NNs and alternately update their parameters $\theta_G$ and $\theta_D$. Given the parametric limit, we study the achievable rate. Details on the networks architecture and on the training parameters are reported in Table~\ref{tab:plc_parameters}.

We consider an analog and digital transmission scheme. The former is realized by taking as input of the generator $G$ a vector $\mathbf{z}$ sampled from a $30$-dimensional Gaussian distribution and by mapping it to a complex symbol $\mathbf{x}$ with continuous distribution $p_{X}(\mathbf{x})$. The latter, instead, considers an input vector $\mathbf{z}$ sampled from a multivariate Bernoulli $k$-dimensional distribution with probability $p=0.5$, where $k=\log_2(M)$ and $M$ is the dimension of the alphabet. For the discrete case, the coding rate is defined as $R = 2k / d$, where $d$ is the dimension of the channel input vector $\mathbf{x}$. For the purposes of this section, we consider only the case $d=2$, corresponding to a geometric constellation shaping. However, an extension to probabilistic shaping is straightforward.

Fig.~\ref{fig:plc_capacity_nakagami} illustrates the estimated maximal MI over a channel affected by the Nakagami-$m$ noise under an average power constraint $P=1$, for three values of the parameter $m$, $m=0.6$, $m=0.8$ and $m=1$. It is rather interesting to notice that the estimated MI, for the case of different noise power on the two components but mostly for $m=0.6$, is higher than the AWGN capacity. Indeed, this is consistent with the fact that the AWGN capacity is the lowest among all additive noise channels. Moreover, the case $m=1$ that corresponds to the classic AWGN channel is accurately approximated by the proposed capacity estimator in the SNR range $-10$ to $15$ dB.

Similarly, when we limit the input distribution to a discrete one, the generator automatically learns how to mitigate the impact of the noise on the MI performance by optimally placing the $M$ messages into a bi-dimensional constellation. In particular, we compare the maximal MI when transmitting $4,8,16,32$ messages for the three values of the parameter $m$. Fig.~\ref{fig:plc_capacity_nakagami_discrete} shows the achieved information rate in all the studied cases and again highlights the possibility to achieve higher rates than the Shannon's AWGN limit for low SNR values. For high SNR values, the MI saturates to the coding rate $R$. An example of optimal constellation for a SNR of $7$ dB, $M=16$ and $m=1$ is depicted in Fig.~\ref{fig:plc_features}a where it is curious to comment on the geometric structure of the constellation. Indeed, the constellation is more densely packed along the $Y_2$ component, the one less affected by the noise, and more widely spaced along the $Y_1$ component, corrupted by the real part of the Nakagami distribution. The extreme case $m=0.5$ provides a deterministic imaginary part. As a consequence, the network learns a sort of amplitude modulation along that components, leading to a MI saturation to the coding rate $R$ for every SNR. 

\begin{figure}
	\centering
	\includegraphics[scale=0.25]{images/plc/full_constellation.pdf}
	\caption{a) Optimal constellation scheme for a channel corrupted by additive Nakagami-$m$ noise under an average power constraint with an SNR of $7$ dB and $m=0.6$. b) $16$-QAM constellation scheme for a channel corrupted by additive measured noise with an SNR of $30$ dB.}
	\label{fig:plc_features}
\end{figure} 


\subsubsection{Mutual information estimation under narrowband measured noise}
\label{subsec:plc_real_constellation}
We consider the transmission of Gaussian distributed channel input $\mathbf{x}$ to provide an upper bound of the achievable rate when transmitting $M$ messages. In particular, we set the symbol period to be $T_{\text{sym}} = 50\mu$s, typical of an OFDM modulation. Noise is sampled at $1$ MSa/s, hence, the noise traces are cut in time windows of $2000$ samples each and averaged. Lastly, we randomly pick  samples from the noise empirical distribution for both the real and imaginary part, rendering them independent. It is important to remark that such choice does not represent a performance limitation since the framework is capable to implicitly model any joint distribution, therefore any sort of dependence between real and imaginary part. Moreover, when extending to longer codes ($d>2$), it is also capable to model time dependence.
An example of a received constellation when using a $16$-QAM over the measured additive noise channel with a SNR of $30$ dB is illustrated in Fig.~\ref{fig:plc_features}b.

\begin{table}
	\centering
	\caption{CORTICAL generator architecture and training parameters.}
	\begin{tabular}{ p{4cm}|p{3cm}|p{3cm}} 
		\toprule
		\textbf{Layer} & \textbf{Output dimension } 		& \textbf{Activation function} \\
		\midrule
		\textbf{CORTICAL generator} & &\\
		Input $\mathbf{z}$ & 30 (continuous) / $k$ (discrete) & \\ 
		Fully connected & 100 & ReLU \\ 
		Fully connected &100& ReLU  \\ 
		Fully connected &100& ReLU  \\ 
		Fully connected & 2 &   \\  		\midrule
		
		\textbf{CORTICAL discriminator (DIME)} & &\\
		Input $[\mathbf{x},\mathbf{y}]$ & 4 & \\ 
		Fully connected & 100 & ReLU \\ 
		Dropout  & 0.3 &   \\ 
		Fully connected &100& ReLU  \\ 
		Fully connected & 1 & Softplus  \\ 
		Batch normalization  & & \\  \midrule

		Batch size &  \multicolumn{2}{c}{512}  \\ 
		Training iterations &  \multicolumn{2}{c}{500 generator / 5000 discriminator}  \\ 
		Learning rate &  \multicolumn{2}{c}{0.0002}   \\ 
		Optimizer &  \multicolumn{2}{c}{Adam ($\beta_1$ = 0.5, $\beta_2$ = 0.999)}  \\ 		\midrule

		%\bottomrule	
	\end{tabular}
	\label{tab:plc_parameters}
\end{table}

Fig.~\ref{fig:plc_capacity_real} shows the estimated MI between $\mathbf{x}$ and $\mathbf{y}$ using DIME. As expected also in this case, the value of the MI at low SNRs is greater than the AWGN capacity. Indeed, the narrowband power line statistical noise model is a mixture of different components. An interesting future experiment concerning the measured noise scenario shall consider modeling the channel input distribution with the generator of the CORTICAL framework and modeling the channel itself with the generator of a GAN.

\begin{figure}
	\centering
	\includegraphics[scale=0.3]{images/plc/capacity_real.pdf}
	\caption{Mutual information estimation between channel input and output using DIME in a channel corrupted by measured narrowband noise.}
	\label{fig:plc_capacity_real}
\end{figure}

\subsection{Summary}
\label{subsec:isplc_conclusions}
In this section, we exploited CORTICAL to estimate the capacity of channels corrupted by additive noise. In particular, the PLC noise represents an important and challenging application scenario, thus, two noise classes have been analyzed. In the former case, we designed the optimal signal constellations for an additive channel in presence of Nakagami-$m$ noise and we provided an estimate of its channel capacity. In the latter case, we exploited noise traces obtained via a measurement campaign to estimate the MI between the channel input and the additive channel output, being the input Gaussian distributed. Numerical results illustrate the capability of the learning approach and how the estimated MI is greater than the AWGN channel capacity in a wide range of SNRs. The presented data-driven approach offers a first important step towards optimal channel coding design and capacity estimation in real scenarios, typically formidable unsolved tasks when the noise is not AWGN.

\section{Supervised fault detection}
\sectionmark{PLC fault detection}
\label{sec:plc_detection}
Power line modems (PLMs) act as communication devices inside a PLN. However, they can be exploited also as active sensors to monitor the status of the electric power distribution grid. Indeed, PLC signals carry information about the topological structure of the network, internal electrical phenomena, the surrounding environment and possible anomalies in the grid. An accurate and efficient identification of the types of anomaly through direct sensing measurements can enable grid operators to both prevent malfunctions and effectively intervene when faults occur. In this section, we discuss how to use supervised ML techniques to extract anomalies information from high frequency measurement of electrical quantities, namely the line impedance, the reflection coefficient and the CTF, in the PLC signal band. Simulation results confirm the potentiality of the NN method, outperforming existing model-based approaches in the field without any hyperparameter tuning. 

\subsection{Anomaly detection}
\label{subsec:plc_anomaly_detection}
Topology reconstruction \cite{lampe2013tomography,7797477} and anomalies detection \cite{8641473,7897106} are some of the applications enabled by PLC for grid sensing. The former allows the grid operators to extend the knowledge of the grid configuration, switches and feeders status. Furthermore, the knowledge of the network topology is per se relevant for networking purposes, i.e., geo-routing algorithms. The latter, instead, helps grid operators to better monitor the network status, malfunctions etc. which in turn enables its predictive maintenance granting a more efficient energy delivery service.

A number of contributions on topology reconstruction have been made \cite{6525848,6507589}, while less work has been carried on anomaly detection using PLMs. One of the main issues resides on the difficulty to formulate analytic representative models. Some recent results have shown how to relate the state of the grid, cables and loads to several electrical quantities that can be measured to provide anomalies information \cite{8653266}.
Nonetheless, diagnosis of malfunctions ends up into a classification problem \cite{8641473}. 

Faults are generally classified as low impedance faults (LIFs) and high impedance faults (HIFs). LIFs are usually bolted faults, thus, short circuits which an upstream switch or fuse can sense. HIF is, instead, a consequence of an unwanted electrical contact with a low conductive object, resulting in a low fault current that cannot be detected by conventional protection systems. HIF may cause service failures in the long run or even public hazards in the worst case. 

Researchers have documented several HIF detection techniques, spanning from classical approaches involving the analysis of the magnitude and phase of the fault current, its Fourier or wavelet transform, to heuristical approaches that exploit fuzzy logic and NNs \cite{HIF_review}.
These methodologies rely on the transmission line theory, carry-back equation and time-frequency metrics, making complicated the development of a full-representative model. NNs avoid the bottom-up approach and make use of \textit{a-posteriori} observations of natural phenomena, the experience, to create an implicit empirical top-down model \cite{ML_PLC}. 

\subsection{Results}
\label{subsec:plc_anomaly_results}
We now discuss the structure of the dataset, the training parameters and the architecture of the NN. Finally, we present the results obtained using the supervised methodology and we compare them with the model-based approach in \cite{8641473}.

\subsubsection{Materials}
\label{subsec:plc_materials}
We made use of a multi-conductor transmission line PLN simulator with the anomalies models discussed in \cite{8653266}. The simulator randomly displaces $20$ nodes on a single topology realization with average node distance of $700$ m, which mimics the average displacement in a low-voltage distribution network.
As for the cable parameters, we use the same as those presented in \cite{versolatto2011an}.
Furthermore, both the case of constant (at $2$ k$\Omega$) and randomly variable load impedances are considered.
Anomalies occurring on the network have been artificially added
to obtain $10000$ realizations of a perturbed grid \cite{8653266}. 

The training process was conducted using $50\%$ of the data, while testing was done using the other unbiased half part. 
Input signals are the ratios between the measurement of the electrical parameters $\mathbf{Y_{\text{in}}}, \mathbf{\rho_{\text{in}}}$ and $\mathbf{H}$ for a given realization, and the reference value of the respective signal in the band $4.3-500$ kHz with $4.3$ kHz sampling. The reference values were obtained when there was no anomaly. To avoid ill-conditioned networks and have a stable convergence of its weights and biases, we normalized each input realization in the interval $[-1,1]$ by using the linear transformation in \eqref{eq:plc_normalization}.

The output signals from the classifier are $4$ different class indicators according to the type of anomaly. In particular, $4$ different classes of anomalies in the network can be detected:
\begin{enumerate}
\item Unperturbed;
\item Load impedances changes;
\item Concentrated faults;
\item Distributed faults.
\end{enumerate}
To work with the cross-entropy cost function, we converted each categorical target output using a classical one-hot encoding.

\begin{figure}
	\centering
	\includegraphics[scale=0.5]{images/plc/AdmittanceVariation.pdf}
	\caption{4 randomly picked realizations of admittance variation over the frequency for each type of anomaly.}
	\label{fig:plc_Yin}
\end{figure}

When training a neural network, overfitting is the main issue to prevent. For this reason, a standard way to proceed is to split the initial training dataset into two subsets, a training and a validation set. The former is  responsible for the update of the network parameters (weights and biases). The latter is, instead, used to check the current performance of the network and understand how it performs with previously unseen data. For the purpose of this section, we decided to split the collected dataset in $40\%$ training set, $10\%$ validation set and $50\%$ test set.

To show the potentiality and applicability of the ML approach, we chose to implement a shallow neural network, thus, a neural network with only one hidden layer consisting of $25$ neurons whose activation function $\sigma(\cdot)$ is the sigmoid.

We stopped the training process when the error on the validation set started increasing for more than $20$ consecutive iterations, sign of an overfitted model. 

\subsubsection{Performance evaluation}
\label{subsec:plc_performance}
To assess the performance of the neural network classifier, we performed a set of testing experiments judging the ability to detect and classify the type of anomaly compared to the framework in \cite{8641473}. We evaluated the models by comparing the accuracy obtained for the training and testing set. The accuracy quantitatively expresses the probability of correctly identifying the belonging class. 

We distinguished $3$ different classifiers: a binary model that is able to detect a perturbed situation from an unperturbed one; a ternary model that discriminates the unperturbed, the impedances changes and the fault cases; a quaternary model that detects and identifies all the $4$ classes previously discussed.

The first set of experiments considers the scenario of constant load impedances for each realization, while the second set takes into account the random variability of the load impedances for each realization.

$3$ classifiers for $3$ types of input signal ($\mathbf{Y_{\text{in}}}, \mathbf{\rho_{\text{in}}}$ and $\mathbf{H}$) in $2$ loads setup yield to $18$ different neural network models, whose performances are reported in Tab.~\ref{tab:plc_performances1}, Tab.~\ref{tab:plc_performances2} and Tab.~\ref{tab:plc_performances3}.

\begin{table}[b]
\begin{center}
\resizebox{\columnwidth}{!}{%
{\renewcommand{\arraystretch}{1.2}
\begin{tabular}{l|c|c|l|l}
\toprule
\multicolumn{2}{c|}{\pbox{15cm}{\textbf{Accuracy} \\ Training/\textbf{Testing}}} & \multicolumn{3}{c}{\textbf{Models for $\mathbf{Y_{\text{in}}}$}}                                            \\ \cline{3-5}
\multicolumn{2}{c|}{}                                   & \multicolumn{1}{c|}{Binary} & \multicolumn{1}{c|}{Ternary} & \multicolumn{1}{c}{Quaternary} \\ 			
\midrule

\multirow{2}{*}{\textbf{\rotatebox[origin=c]{90}{Load} \textbf{\rotatebox[origin=c]{90}{Imp.}}}}    & Constant      & 88.2\%                          /    \textbf{86.1}\%                               & 100\%                          /    \textbf{100\%}                    & 88.8\%                          /    \textbf{87.4\%}       \\ \cline{2-5}
                                            & Variable   & 99.9\%                          /    \textbf{99.7\%}  & 99.4\%                          /    \textbf{96.9\%                  }  & 97.6\%                          /    \textbf{93.0\%}       \\ \hline
\end{tabular}}%
}
\caption{Detection accuracy for $\mathbf{Y_{\text{in}}}$ with constant and variable load impedances.}
\label{tab:plc_performances1}
\end{center}
\end{table}

\begin{table}[h]
\begin{center}
\resizebox{\columnwidth}{!}{%
{\renewcommand{\arraystretch}{1.2}
\begin{tabular}{l|c|c|l|l}
\toprule
\multicolumn{2}{c|}{\pbox{15cm}{\textbf{Accuracy} \\ Training/\textbf{Testing}}} & \multicolumn{3}{c}{\textbf{Models for $\mathbf{\rho_{\text{in}}}$}}                                            \\ \cline{3-5}
\multicolumn{2}{c|}{}                                   & \multicolumn{1}{c|}{Binary} & \multicolumn{1}{c|}{Ternary} & \multicolumn{1}{c}{Quaternary} \\ 			
\midrule

\multirow{2}{*}{\textbf{\rotatebox[origin=c]{90}{Load} \textbf{\rotatebox[origin=c]{90}{Imp.}}}}    & Constant      & 86.3\%                          /    \textbf{84.3}\%                               & 100\%                          /    \textbf{100\%}                    & 88.0\%                          /    \textbf{86.8\%}       \\ \cline{2-5}
                                            & Variable   & 97.9\%                          /    \textbf{96.0\%}  & 90.2\%                          /    \textbf{86.4\%                  }  & 84.3\%                          /    \textbf{80.3\%}       \\ \hline
\end{tabular}}%
}
\caption{Detection accuracy for $\mathbf{\rho_{\text{in}}}$ with constant and variable load impedances.}
\label{tab:plc_performances2}
\end{center}
\end{table}

\begin{table}[h]
\begin{center}
\resizebox{\columnwidth}{!}{%
{\renewcommand{\arraystretch}{1.2}
\begin{tabular}{l|c|c|l|l}
\toprule
\multicolumn{2}{c|}{\pbox{15cm}{\textbf{Accuracy} \\ Training/\textbf{Testing}}} & \multicolumn{3}{c}{\textbf{Models for $\mathbf{H}$}}                                            \\ \cline{3-5}
\multicolumn{2}{c|}{}                                   & \multicolumn{1}{c|}{Binary} & \multicolumn{1}{c|}{Ternary} & \multicolumn{1}{c}{Quaternary} \\ 			
\midrule

\multirow{2}{*}{\textbf{\rotatebox[origin=c]{90}{Load} \textbf{\rotatebox[origin=c]{90}{Imp.}}}}    & Constant      & 87.3\%                          /    \textbf{85.1}\%                               & 94.8\%                          /    \textbf{94.6\%}                    & 86.8\%                          /    \textbf{85.7\%}       \\ \cline{2-5}
                                            & Variable   & 99.7\%                          /    \textbf{98.9\%}  & 99.2\%                          /    \textbf{98.0\%                  }  & 95.4\%                          /    \textbf{93.8\%}       \\ \hline
\end{tabular}}%
}
\caption{Detection accuracy for $\mathbf{H}$ with constant and variable load impedances.}
\label{tab:plc_performances3}
\end{center}
\end{table}

\subsubsection{Comments}
\label{subsec:plc_comments}
Several considerations can be made by analyzing the accuracy tables. As the intuition suggests, it is immediate to notice that the accuracy values for the training set are always higher than the one for the testing set. However, the relevance resides in the small difference between the pair of values which highlights the ability of the network to generalize the model to unseen samples.
Another remark follows from noticing that the models that accept as input the admittance $\mathbf{Y_{\text{in}}}$ perform globally better in terms of accuracy. A logic reason that motivates such result comes from the fact that the anomalies are artificially generated by imposing an abnormal impedance value in a certain node. Therefore, the primary physical quantity affected by such irregularity is the impedance (admittance) itself. Fig.~\ref{fig:plc_Yin} presents $4$ randomly picked realizations of admittance in frequency, one for each anomaly class.


\begin{figure}
\centering
\begin{subfigure}{0.5\textwidth}
  \centering
 \includegraphics[scale=0.5]{images/plc/ConfusionMatrix2Classes.pdf}
	\caption{Binary classifier}
	\label{fig:plc_Confusion2}
\end{subfigure}%
\begin{subfigure}{0.5\textwidth}
 \centering
	\includegraphics[scale=0.31]{images/plc/ConfusionMatrix4Classes.pdf}
	\caption{4-classes classifier}
	\label{fig:plc_Confusion4}
 \end{subfigure}%
 \caption{Confusion matrix (accuracy of the prediction) for anomaly detection using the admittance as the main input
signal.}
\label{fig:plc_Confusion}
\end{figure}


By looking at the binary models, the variability introduced in the load impedances improves the accuracy. A random source in the network and in the parameter configuration often brings a desired unbiasing effect that reduces overfitting issues, as the dropout layer does. Such consideration can, partially, be extended to the ternary and quaternary models.
The binary models detect the presence of a generic anomaly or not. Fig.~\ref{fig:plc_Confusion2} shows the confusion matrix for the binary model with constant load impedance when the input is the the admittance $\mathbf{Y_{\text{in}}}$. The work in \cite{8641473} proposed a bottom-up methodology with a probability of correctly detecting an anomaly equal to $0.8$, in a network of $20$ nodes. Our supervised approach reached a probability of $0.83$, improving the previous result with a minimal hyper-parameter tuning. However, the greater improvement comes from the analysis of the admittance quaternary model and its comparison with the results in \cite{8641473}. Indeed, from Fig.~\ref{fig:plc_Confusion4}, it is clear that the model perfectly classifies the load impedance change and concentrated fault anomalies, contrary to the physical model. The same confusion matrix shows that the network struggles in distinguishing between the unperturbed situation (class 1) and the distributed
fault event (class 4), as the benchmark model does in identifying distributed faults. This suggests to include the last type of anomaly in the third one, and study the ability of the neural network to classify the remaining 3 classes, in a ternary model. Such division motivates the high accuracy values obtained by the ternary model, as presented in all the tables. 

In conclusion, the developed supervised models that rely on a simple neural network structure provide excellent classification results. In particular, the ternary models have shown the best performance, discriminating an unperturbed network from a network affected by a sudden load impedance change and by a fault. As expected, it is possible to identify such events analyzing either the variation of the input admittance $\mathbf{Y_{\text{in}}}$, the variation of the reflection coefficient $\mathbf{\rho_{\text{in}}}$, or the variation of the channel transfer function $\mathbf{H}$. The detection methodology does not require an a priori knowledge of the physical model, but only a simple neural network structure that learns from the measurements. Moreover, the single hidden layer and the reduced number of neurons allow an easy and fast implementation in a non-complex PLM device. 

\subsection{Summary}
\label{subsec:plc_anomaly_conclusions}
In this section, we discussed a supervised NN approach to detect anomalies inside a PLN, exploiting the high frequency PLC signals sensed by power line modems.
To serve the scope, we proposed a shallow NN, consisting of only one hidden layer with $25$ neurons. Accuracy results proved the efficacy of the learned models, which are able to both detect and classify the type of anomaly with $3$ different input signal: $\mathbf{Y_{\text{in}}}, \mathbf{\rho_{\text{in}}}$ and $\mathbf{H}$. The performance of the presented ML approach together with the simplicity of the methodology encourage further endeavors in the area of anomaly detection with power line technology. 