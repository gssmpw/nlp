\section{Summary}
\label{subsec:rst_con}
This chapter investigated the problem of trajectory generation for unmanned vehicles, and in particular for UAVs. The RST algorithm has been mathematically derived and implemented with the objective to find a path satisfying arbitrary conditions, specified in terms of position, velocity, acceleration, etc. An analysis of the perturbation in the kinematic constraints has been carried out with the result that the uncertainty in the constraint can be translated into uncertainty in the polynomial coefficients. An optimization framework has been developed in order to improve the trajectory generated with RST, which in general is not optimal by itself. 

Two examples have been presented with the RST methodology and compared to a common benchmark in the field such as the minimum-snap piecewise polynomial trajectory algorithm. It should be noted that in the latter approach, a joint optimization process (quadratic programming) is needed to fulfill the kinematic constraints. Our RST approach eliminates the optimization step and directly provides an analytic smooth trajectory. Furthermore, whenever the number of waypoints increases, we extended our methodology to a blockwise framework (BRST) where interfaces are intrinsically jointly matched. Finally, when the number of waypoints in a single block is equal to $2$, BRST converges to the piecewise one (PRST) which generates optimal trajectories in terms of minimum-snap.  

Results about the application to a UAV structure proved the capability and effectiveness of RST (and BRST, PRST extensions) to outperform the piecewise approach in terms of smoothness. The results show that RST offers a new direction in the domain of trajectory generation and path planning.