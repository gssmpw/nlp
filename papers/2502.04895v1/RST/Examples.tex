\section{Applications}
\label{subsec:rst_examples}
In this section, we firstly discuss two different case-studies based on the number of waypoints $N+1$. For both of them we present the trajectory generated with RST, we numerically describe the influence of uncertainties in the kinematic constraints and we lastly compare the RST with the optimized extension introduced in Sec.~\ref{subsec:rst_optimization} (RST$_{\text{opt}}$), and with the minimum-snap piecewise polynomial trajectory obtained as proposed in \cite{5980409}.
In the last part, we consider a 2D planner quadrotor with highly input coupling   to assess the tracking performance capability and highlight the advantage of the RST algorithm over a piecewise polynomial method such as the spline interpolation technique or the minimum-snap approach.

\subsection{Scenario 1}
As a first example, we consider a rest-to-rest maneuver. Such trajectory satisfies the following deterministic kinematic constraints in a time-span of $10$ seconds
\begin{equation*}
  \dfrac{d^i}{d t^i}x_k(t)=
  \begin{blockarray}{*{3}{c} l}
    \begin{block}{*{3}{>{$\footnotesize}c<{$}} l}
      $t_0$ & $t_1$ & $t_2$ & \\
    \end{block}
    \begin{block}{[*{3}{c}]>{$\footnotesize}l<{$}}
      0 & 2 & 0 \bigstrut[t]& space \\
       0 & 0 & 0 & velocity \\
       0 & 0 & 0 & acceleration \\
       0 & 0 & 0 & jerk \\
    \end{block}
  \end{blockarray}
\end{equation*}
and can be determined using the RST algorithm, which generates a polynomial trajectory $x_k(t)$ of degree $(k+1)(N+1)-1=4\cdot 3-1=11$. 
A perturbation is later applied in some of the constraints, in particular
\begin{equation*}
  \varepsilon_i(t)=
  \begin{blockarray}{*{3}{c} l}
    \begin{block}{*{3}{>{$\footnotesize}c<{$}} l}
      $t_0$ & $t_1$ & $t_2$ & \\
    \end{block}
    \begin{block}{[*{3}{c}]>{$\footnotesize}l<{$}}
      0 & \varepsilon_0 & 0 \bigstrut[t]& space \\
       0 & \varepsilon_1 & 0 & velocity \\
       0 & 0 & 0 & acceleration \\
       0 & 0 & 0 & jerk \\
    \end{block}
  \end{blockarray}
\end{equation*}

\begin{figure}
\includegraphics[scale=0.82]{images/extra/RSRT_1.pdf}
\centering
\caption{Deterministic trajectory (in red) and $100$ realizations of perturbed trajectories (in black) for the case of $3$ waypoints equally spaced in time.}
\label{fig:rst_RSRT_1}
\end{figure}

\begin{figure}
\includegraphics[scale=0.58]{images/extra/RST_1.pdf}
\centering
\caption{Comparison of trajectories and derivatives generated with RST, RST optimized and minimum-snap algorithm for the case of $3$ waypoints equally spaced in time.}
\label{fig:rst_RST_1}
\end{figure}

with $\varepsilon_0 \sim \mathcal{N}(0,0.1^2)$ and $\varepsilon_1 \sim \mathcal{N}(0,0.1^2)$. From Alg.~\ref{alg:RSRT}, the estimation and generation of the covariance matrix $\hat{\Sigma}$ describing the joint statistics of the coefficients of $r_k(t)$ provides several perturbed trajectories $\tilde{x}_k(t)$ that are reported in Fig.~\ref{fig:rst_RSRT_1}, top left corner. In particular, the red line represents the fixed trajectory $x_k(t)$ obtained via RST, while the thin black lines are $100$ stochastic realizations of the estimated perturbed version $\tilde{x}_k(t)$. To show the consistency of the RSRT method, velocity, acceleration and jerk trends are also illustrated.



When the number of waypoints is greater than $2$, RST does not provide directly the optimal solution in terms of minimum integral of the snap squared. Therefore, an optimization step looks for the best polynomial $q(t)$ (in this case of degree $Q = 2$) which minimizes the snap, without interfering with the kinematic constraints thanks to the pre-multiplicative factor $a^p(t)$. 

Fig.~\ref{fig:rst_RST_1} compares the trajectories and the respective derivatives trends obtained using three different approaches, RST, RST$_{\text{opt}}$ and minimum-snap piecewise. Despite the fact that up to the jerk the three curves look similar, the evolution of the snap over time provides insightful observations. The snap obtained through RST is continuous by construction but has elongations in the edges which are more pronounced when the number of waypoints increases, as the consequence of Runge's phenomenon. The snap obtained using piecewise polynomials is the minimum in terms of integral but reveals a discontinuity jump which can produce undesired vibrations causing aging and damages in the UAVs structure \cite{Letizia2020_rst}. However, the snap that comes from the optimization of RST is continuous, optimal and in most of cases can be interpreted as a continuous approximation of the piecewise minimum snap, resulting in a good trade-off between complexity and continuity.

\subsection{Scenario 2}
As a second case-study, we consider a trajectory passing through $6$ waypoints. In order to avoid Runge's phenomenon, we choose the points in time (from $0$ to $10$ seconds) according to \eqref{Cheby} and in each of them we impose the following deterministic kinematic constraints

\begin{equation*}
  \dfrac{d^i}{d t^i}x_k(t)=
  \begin{blockarray}{*{6}{c} l}
    \begin{block}{*{6}{>{$\footnotesize}c<{$}} l}
      $t_0$ & $t_1$ & $t_2$ & $t_3$ & $t_4$ & $t_5$ \\
    \end{block}
    \begin{block}{[*{6}{c}]>{$\footnotesize}l<{$}}
      0 & 1 & -1 & 1 & -1 & 0 \bigstrut[t]& space \\
       0 & 0 & 0 & 0 & 0 & 0 & velocity \\
       0 & 0 & 0 & 0 & 0 & 0 & acceleration \\
       0 & 0 & 0 & 0 & 0 & 0 & jerk \\
    \end{block}
  \end{blockarray}
\end{equation*}

A perturbation is later applied in some of the constraints as follows
\begin{equation*}
  \varepsilon_i(t)=
  \begin{blockarray}{*{6}{c} l}
    \begin{block}{*{6}{>{$\footnotesize}c<{$}} l}
      $t_0$ & $t_1$ & $t_2$ & $t_3$ & $t_4$ & $t_5$ \\
    \end{block}
    \begin{block}{[*{6}{c}]>{$\footnotesize}l<{$}}
      0 & \varepsilon_0 & \varepsilon_0 & \varepsilon_0 & \varepsilon_0 & 0 \bigstrut[t]& space \\
       0 & \varepsilon_1 & \varepsilon_1 & \varepsilon_1 & \varepsilon_1 & 0 & velocity \\
       0 & 0 & 0 & 0 & 0 & 0 & acceleration \\
       0 & 0 & 0 & 0 & 0 & 0 & jerk \\
    \end{block}
  \end{blockarray}
\end{equation*}
with $\varepsilon_0 \sim \mathcal{N}(0,0.1^2)$ and $\varepsilon_1 \sim \mathcal{N}(0,0.1^2)$.
\begin{figure}
\includegraphics[scale=0.82]{images/extra/RSRT_2.pdf}
\centering
\caption{Deterministic trajectory (in red) and $100$ realizations of perturbed trajectories (in black) with $6$ waypoints and Chebyshev nodes in time.}
\label{fig:rst_RSRT_2}
\end{figure}
\begin{figure}
\includegraphics[scale=0.58]{images/extra/RST_2.pdf}
\centering
\caption{Comparison of trajectories and derivatives generated with RST$_{\text{opt}}$ and minimum-snap algorithm for the case of $6$ waypoints equally spaced in time.}
\label{fig:rst_RST_2}
\end{figure}
Proceeding in the same way aforementioned for the first scenario, brings to a set of feasible trajectories described in Fig.~\ref{fig:rst_RSRT_2}. It is evident that the trajectories accept uncertainties only in the kinematic constraints involving space and velocity, as the initial hypothesis.

To show the generality of the optimization step for the snap's analysis, we decided to consider points in time equally spaced. In such situation, the RST gives an overshoot trajectory which the optimization will try to compensate, so we only compare RST$_{\text{opt}}$ with minimum-snap piecewise. 

As in the previous example, the trajectories and the derivatives up to jerk appear to be similar. Nevertheless, when looking to the snap trends, the snap generated via optimized RST results smooth while the snap generated via minimum-snap piecewise results discontinuous. Smoothness in the trajectories is often a wanted property, especially from a control perspective since real-time control laws to track desired trajectories need to be given by the E-FMS. 

\subsection{RST tracking performance}
In the following, we extend the 1D recursive trajectory generation method to the 3D case. To achieve that, we exploit the flatness property \cite{6324664} of UAVs.
In particular, we conduct a close-to-real simulation of a highly non-linear dynamical system model of a 3D quadrotor \cite{1570447} to assess the capability and effectiveness of the RST. 
A natural choice of the flat outputs is $\gamma(t) = [x(t)\; y(t)\; z(t)\; \psi(t)]^T$. The components $x,y,z$ represent the position of the center of mass of the quadrotor in the inertial reference frame, while $\psi$ is the yaw angle. The trajectory is defined as a smooth curve in the space of flat outputs $\gamma(t): [t_0,t_N] \to \mathbb{R}^3 \times SO(2)$. 

\subsection{Control inputs smoothness}
A simple rest to rest maneuvering with the following kinematic constraints on $x,y$ and $z$ is considered as the flying scenario
\begin{equation*}
  \dfrac{d^i}{d t^i}x_k(t)=
  \begin{blockarray}{*{3}{c} l}
    \begin{block}{*{3}{>{$\footnotesize}c<{$}} l}
      $t_0$ & $t_1$ & $t_2$ & \\
    \end{block}
    \begin{block}{[*{3}{c}]>{$\footnotesize}l<{$}}
      0 & 2 & 4 \bigstrut[t]& space \\
       0 & 0.5 & 0 & velocity \\
       0 & 0 & 0 & acceleration \\
       0 & 0 & 0 & jerk \\
    \end{block}
  \end{blockarray}
\end{equation*}
\begin{equation*}
  \dfrac{d^i}{d t^i}y_k(t)=
  \begin{blockarray}{*{3}{c} l}
    \begin{block}{*{3}{>{$\footnotesize}c<{$}} l}
      $t_0$ & $t_1$ & $t_2$ & \\
    \end{block}
    \begin{block}{[*{3}{c}]>{$\footnotesize}l<{$}}
      0 & 2 & 0 \bigstrut[t]& space \\
       0 & -0.5 & 0 & velocity \\
       0 & 0 & 0 & acceleration \\
       0 & 0 & 0 & jerk \\
    \end{block}
  \end{blockarray}
\end{equation*}
and
\begin{equation*}
  \dfrac{d^i}{d t^i}z_k(t)=
  \begin{blockarray}{*{3}{c} l}
    \begin{block}{*{3}{>{$\footnotesize}c<{$}} l}
      $t_0$ & $t_1$ & $t_2$ & \\
    \end{block}
    \begin{block}{[*{3}{c}]>{$\footnotesize}l<{$}}
      0 & 5 & 0 \bigstrut[t]& space \\
       0 & 0 & 0 & velocity \\
       0 & 0 & 0 & acceleration \\
       0 & 0 & 0 & jerk \\
    \end{block}
  \end{blockarray}
\end{equation*}

\begin{figure}
\includegraphics[scale=0.32]{images/extra/3D_trajectory_robotics_decimate.pdf}
\centering
\caption{Quadrotor tracking (dashed line) of 3D trajectories (solid line) generated by RST (red), RST$_{\text{opt}}$ (green) and minimum-snap (blue) approaches.}
\label{fig:rst_quadrator_curve}
\end{figure} 

with $\psi(t)\equiv 0$ in $[t_0,t_N]$. 
The sampling time is set to $0.01$ s and the physical parameters of the quadrotor helicopter are taken according to \cite{1570447}. It should be noted that the non-linear dynamical system model of the quadrotor allows studying and testing the robustness of the methodologies applied. In particular, Fig.~\ref{fig:rst_quadrator_curve} shows the tracking of the 3D desired trajectories using the proposed approaches: RST, RST$_{\text{opt}}$ and minimum-snap.

\begin{figure}
\includegraphics[scale=0.35]{images/extra/control_inputs_robotics.pdf}
\centering
\caption{Evolution of control inputs of the system $u_1, u_2, u_3$ and $u_4$ over time for three different approach: RST, RST$_{\text{opt}}$ and minimum-snap.}
\label{fig:rst_control_inputs}
\end{figure} 

In all the cases studied, the controller succeeds in following the desired path. However, the main advantage of adopting a smooth path as RST is depicted in Fig.~\ref{fig:rst_control_inputs}. The RST and RST$_{\text{opt}}$ approaches perform better than the minimum-snap one in terms of smoothness of the control signals. This is a direct consequence of the fact that the control inputs ($u_2$ and $u_3$) responsible for the roll and pitch angle, depend on the $4$-th derivative of the position, namely the snap. Therefore, peaks and discontinuities in high order derivatives can have an impact on the continuity of the control inputs, causing aging and damaging of the quadrotor mechanical structure. 

The following section discusses this problem in detail and provides a possible solution as a trade-off between complexity and discontinuities. 