\section{Introduction}
\label{subsec:rst_introduction}
Recent advances in transportation have resulted in an increased adoption of unmanned vehicles, i.e., underwater unmanned vehicles, unmanned ground vehicles and unmanned aerial vehicles (UAVs), for a wide range of applications. The autopilot that controls the vehicle is typically guided by the inertial navigation system. In the particular case of aerial navigation, current guidance and control systems deployed in UAVs provide real-time control laws to track desired trajectories set by the embedded flight management system (E-FMS) \cite{spitzer2000digital}. The E-FMS intrinsically has the advantage that it can be segmented into four blocks and each of them individually studied and optimized. Its synoptic realization, that is embedded in the UAV platform, is shown in Fig.~\ref{fig:rst_E-FMS}. The policy planner is used to define the waypoints and the situation of a mission. The path planner is responsible for constructing the feasible trajectory according to tasks specified by the policy planner and it is a core block for physical task realizations. The duty of this block becomes extremely challenging because a feasible trajectory must consider and satisfy: 1) the non-linearity of the system model; 2) the kinematic constraints; and 3) smoothness. The agent block takes complexity into account and intelligently computes the control laws to track the desired trajectories. Clearly, the control laws affect the UAV flight dynamics. \par
In this chapter, we focus on aerial vehicles and on the the path planning function of the E-FMS.
The path planner is an upstream block, thus, it allows generating desired trajectories for the considered aircraft, enabling the selection of a flight control law by the guidance function.
\begin{figure}[t]
\includegraphics[scale=0.46]{images/extra/E-FMS_new}
\centering
\caption{The synoptic realization of an E-FMS.}
\label{fig:rst_E-FMS}
\end{figure}

A large effort has been dedicated to implement the path planner block with acceptable computational complexity and several solutions have been developed. A possible categorization of the methodologies is given as follows. The first class of algorithms designs trajectories based on statistical learning. Some examples are the $A^{\ast}$ algorithm \cite{DeFilippis2012}, the graph methodology \cite{1206461}, evolutionary algorithms \cite{richter2016polynomial}, the rapidly-exploring random tree (RRT) \cite{5175292}, and mixed-integer linear programming \cite{5446292}. $A^{\ast}$ and RRT generate an optimal collision-free piecewise linear path but they do not consider the dynamics constraints of the system. In this sense, evolutionary algorithms are able to generate an optimal feasible trajectory that takes into account the dynamics constraints at the expense of a high computational cost.

The second class of algorithms can be considered as a parametric function of the time, which provides at each instant the corresponding desired position. Path primitives such as lines \cite{HoffmannQuadrotor}, polynomials \cite{7068316}, and splines \cite{4602025}, have been deployed to build such algorithms. This class relies on polynomial interpolation techniques to obtain a trajectory that passes through multiple waypoints. In fact, the differentiability of polynomials makes them a suitable interpolation choice for considering the vehicle dynamics. However, it is necessary to solve an inverse problem to find a unique polynomial trajectory that fulfills the constraints. Therefore, the interpolation problem is often split into splines, i.e., piecewise polynomial trajectories. Splines are easy to be constructed and provide bounded trajectories although the continuity of the derivatives (smoothness) at the waypoints is only guaranteed up to a certain order.

Finally, the third class of algorithms generates trajectories exploiting the differential flatness property of the system. The flatness property provides an analytical mapping from a path and its derivatives to the states and control inputs of the UAV flight dynamics required to track that path accurately. Furthermore, it is particularly advantageous for solving 3D trajectory planning and waypoint following control. In \cite{richter2016polynomial} the flatness-based property of a UAV is exploited to convert predefined waypoints into polynomial trajectories using quadratic programming. In \cite{4602025} the trajectory planning is extended to a constrained optimization problem and the parametrization of the trajectory is modeled as a composition of a parametric function $P(\lambda)$ defining the path, and a monotonically increasing function $\lambda(t)$ specifying the motion on this path. $P(\lambda)$ and $\lambda(t)$ are modeled using B-spline functions. The tuning parameters of $P(\lambda)$ and $\lambda(t)$ are obtained using the sequential quadratic programming technique. The B\'ezier polynomial function has been used to solve the constrained optimization problem in  \cite{6324664}. However, all the previous works focused on a piecewise parametrization of the flat outputs for the system, which may cause discontinuities in the control inputs as a direct consequence of non sufficiently smooth generated trajectories. 
We solve the former issue by proposing a recursive paradigm to generate smooth trajectories. The main ideas and the preliminary numerical results showing the advantage of the recursive formulation of the trajectory generation algorithm were presented in \cite{Letizia2020_rst}. This section offers the mathematical foundation and derivation of the method. In addition, this chapter considers the statistical perturbation of the constraints, it considers the trajectory generation in an optimization framework, and it reports the complexity analysis results of the recursive algorithm, which were missing aspects in \cite{Letizia2020_rst}.

In the following, we provide theoretical foundations for the recursive smooth trajectory generation algorithm (RST) that allows the design of a smooth ($\mathcal{C}^{\infty}$) polynomial path. It uses the concept of waypoints to find a closed form trajectory that satisfies any arbitrary dynamic limitations mapped into kinematic constraints. It is fundamentally different from existing approaches \cite{6376099} in the following aspects:
\begin{itemize}
\item The trajectory is defined by a unique polynomial which fulfills all the kinematic constraints;
\item The trajectory is recursively built exploiting the concept of $m$-partial trajectory (See Def. \ref{def:rst_def1}), thus, the generation method avoids matrix inversion;
\item The RST algorithm enables a new piecewise (PRST) and blockwise (BRST) interpolation approach to tackle numerical and oscillation problems;
\item Uncertainties in kinematic constraints are simply translated into uncertainties in the polynomial coefficients. In this way, all the feasible trajectories are obtained from a deterministic polynomial plus a stochastic perturbation described by a multivariate Gaussian distribution;
\item Lastly, RST provides an immediate formulation and solution to trajectory optimization problems.
\end{itemize}


\subsection*{Notation}
\label{subsec:rst_notation}
Vectors and matrices are denoted with bold letters. Unless stated otherwise, all vectors in this section are column vectors.
$t_j$ denotes a point in time while $x(t_j)$ and $\mathbf{x}(t_j)$ denote a point in the $1D$ space and a vector with components in the multi-dimensional space, respectively. $\frac{d^i\mathbf{x}}{dt^i}\bigr|_{t=t_j}$ denotes the $i$-th derivative of $\mathbf{x}(t)$ evaluated in $t_j$. The factorial of $n$ is denoted by $n!$. A stochastic perturbation of the deterministic variable $y$ will be described by $\tilde{y}$. $||.||^2$ denotes the norm squared.
The state variables of the dynamical system are a function of time, e.g., $\bm{x} = \bm{x}(t)$. The first, second and $n$-th derivative with respect to time of a state space variable $\bm{x}$ are denoted respectively with $\bm{\dot{x}}$, $\bm{\ddot{x}}$ and $\bm{x}^{(n)}$. 