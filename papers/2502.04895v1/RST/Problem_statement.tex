\section{Problem statement}
\label{subsec:rst_problem}
Consider the problem of navigating a moving body, e.g., an unmanned vehicle, through $N+1$ waypoints at specific time stamps. Assume that any arbitrary dynamic limitation of the  body translates into kinematic constraints (e.g. position, velocity, acceleration, etc.). Such kinematic constraints are useful to impose, for instance, that the body starts from the rest at the beginning of the trajectory and reaches each waypoint with specific kinematics (e.g. aggressive maneuvering).

The task of path planning consists of finding a trajectory $\mathbf{x}(t)$ to move a given body from an initial point $\mathbf{x}(t_0)$ to a final point $\mathbf{x}(t_N)$, minimizing a certain cost function $J(\cdot)$ (time, energy, jerk, etc.), under given geometrical and kinematic constraints $\sigma(\cdot)$.

Let $t_0<t_1<\dots<t_N$ be $N+1$ time stamps for which all the corresponding positions $\mathbf{x}(t_0),\mathbf{x}(t_1), \dots, \mathbf{x}(t_N)$ and the respective $k$ derivatives $\frac{d^i\mathbf{x}}{dt^i}\bigr|_{t=t_0}, \frac{d^i\mathbf{x}}{dt^i}\bigr|_{t=t_1}, \dots, \frac{d^i\mathbf{x}}{dt^i}\bigr|_{t=t_N}$ are a-priori assigned, for $i = 1,\dots, k$. For notation convenience, the trajectory for which the first $k$ derivatives are defined, is denoted with $\mathbf{x}_k(t)$.

The objective is to design a feasible trajectory $\mathbf{x}_k(t)$ which satisfies the equality kinematic constraints $\sigma_{i,j}$ expressed by the first $k$ derivatives $\frac{d^i\mathbf{x}}{dt^i}\bigr|_{t=t_j} = \sigma_{i,j} $, for $i=0,1,\dots,k$ and $j=0,1,\dots,N$. 

To do so we propose to use polynomials as the set of our feasible trajectories. In particular, we build the polynomial trajectory as a linear combination of $k+1$ polynomial basis
\begin{equation}
\label{eq:rst_trajectory}
\mathbf{x}_k(t) = \sum_{i=0}^{k}{\mathbf{p}_i(t)},
\end{equation}
where $\mathbf{p}_i(t)$ is the polynomial (for each component of the $3D$ space) \textit{responsible} for the fulfillment of the $i$-th derivative constraint and $k$ is the number of considered derivatives. In other words, each polynomial component $\mathbf{p}_i(t)$ is designed as an additive term which iteratively fulfills the $i$-th kinematic constraint.
Furthermore, we exploit Lagrange interpolating polynomials to compute the polynomial basis $\mathbf{p}_i(t)$.

To the best of authors' knowledge, this is the first work that recursively builds a set of polynomials for trajectory generation under kinematic constraints.





