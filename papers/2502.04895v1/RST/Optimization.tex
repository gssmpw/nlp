\section{RST in an optimization framework}
\label{subsec:rst_optimization}
Most of trajectory generation and path planning research concentrates in finding an optimal trajectory that minimizes a cost function $J(\cdot)$ under given constraints. Nevertheless, trajectory generation does not necessarily require optimality in the solution. In this section, we present an example of optimization framework built around the RST algorithm and we prove that when the number of waypoints $N+1$ is equal to $2$, RST (or PRST) directly provides the optimal solution in terms of minimum integral of the $p$-th derivative of the position squared, matching the trajectory generated by minimum-snap algorithm \cite{5980409} without any use of quadratic programming.

Sec.~\ref{subsec:rst_results} illustrated the RST algorithm, which is able to generate a polynomial trajectory $x_k(t)$ with minimum degree that satisfies the constraints. However, it is easy to notice that the general set of feasible trajectories is induced by $q(t)$ as follows
\begin{equation}
x_{\text{ext}}(t) = x(t)+\frac{a^{k+1}(t)}{(k+1)!}\cdot q(t),
\end{equation}
where $q(t)$ is a polynomial which introduces extra degrees of freedom needed for an optimization phase and $x(t)=x_k(t)$ is the trajectory generated via RST. 

As an example of an optimization problem, let $p = k+1$ be the order of the derivative of $x_{\text{ext}}(t)$ whose energy has to be minimized. A possible approach finds the solution to
\begin{equation}
\min_{q(t)}{\int_{t_0}^{t_N}{\biggl|\biggl|\frac{d^p}{dt^p}\biggl(x(t)+\frac{a^p(t)}{p!}\cdot q(t)\biggr)\biggr|\biggr|^2 dt}}
\end{equation}
with $q(t)$ polynomial, providing the optimal trajectory as
\begin{equation}
x_{\text{opt}}(t)=x(t)+\frac{a^p(t)}{p!}\cdot q_{\text{opt}}(t).
\end{equation}
Since all the functions inside the functional are polynomials, the coefficients of $q(t)$ can be in principle expressed analytically by integrating polynomials and by solving a system of linear equations. The convexity of the norm squared function guarantees a global minimum. 

When the number of waypoints is equal to $2$, that is $N=1$, the following Lemma asserts the optimality (in terms of energy) of the trajectory $x_k(t)$ generated with RST.

\begin{lemma}
\label{lemma:rst_Lemma5}
Let $x_k(t)$ be the trajectory generated with RST which satisfies the given kinematic constraints $\frac{d^i}{dt^i}x_k(t)\bigr|_{t=t_j}$ for $i=0,1,\dots, k$ and $j=0,1,\dots, N$. If $N=1$ and $p=k+1$, then the solution to
\begin{equation}
\label{prob:rst_functional}
\min_{q(t)}{\int_{t_0}^{t_1}{\biggl|\biggl|\frac{d^p}{dt^p}\biggl(x_{p-1}(t)+\frac{a^p(t)}{p!}\cdot q(t)\biggr)\biggr|\biggr|^2 dt}}
\end{equation}
is $q_{\text{opt}}(t)=0$, therefore the trajectory generated with RST is already the optimal one.
\end{lemma}

\begin{proof}
The proof uses some concepts of calculus of variations. In particular, let $\mathcal{L}$ be a Lagrangian function defined as 
\begin{equation}
\label{eq:rst_Lagrangian}
\mathcal{L} = \biggl(\frac{d^p x_{p-1}(t)}{dt^p}+\frac{d^p f(t)}{dt^p}\biggr)^2,
\end{equation}
with 
\begin{equation}
f(t) = \frac{a^p(t)}{p!}\cdot q(t).
\end{equation}
From calculus of variations theory, solving problem \eqref{prob:rst_functional} is equal to solving the Euler-Lagrange equation
\begin{equation}
\small
\label{eq:rst_EL}
\frac{\partial \mathcal{L}}{\partial f} - \frac{d}{dt}\biggl(\frac{\partial \mathcal{L}}{\partial \dot{f}}\biggr)+ \frac{d^2}{dt^2}\biggl(\frac{\partial \mathcal{L}}{\partial \ddot{f}}\biggr)-\dots+(-1)^{p}\frac{d^{p}}{dt^{p}}\biggl(\frac{\partial \mathcal{L}}{\partial f^{(p)}}\biggr)=0
\end{equation}
and by substituting the Lagrangian defined in~\eqref{eq:rst_Lagrangian} into~\eqref{eq:rst_EL}
it follows that 
\begin{equation}
\frac{d^{2p}}{dt^{2p}}\biggl(x_{p-1}(t)+f(t)\biggr)=0.
\end{equation}
From the considerations in Corollary \ref{corollary:rst_corollary1}, 
\begin{align}
\text{deg}(x_{p-1}(t))&=2p-1, \nonumber \\
\text{deg}(f(t))&= \text{deg}(a(t))+\text{deg}(q(t)) = 2p+Q,
\end{align}
with $Q\geq 0$. But, since $x$ and $f$ are polynomials, each differentiation reduces the degree by one and
\begin{equation}
\text{deg}\Bigg(\frac{d^{2p}}{dt^{2p}}\biggl(x_{p-1}(t)+f(t)\biggr)\Biggr)=Q=0,
\end{equation}
therefore $\text{deg}(q(t))=Q=0$ and in particular $q(t)\equiv 0$. \qedhere
\end{proof}
When the number of blocks $M$ increases, the overall optimal trajectory is obtained by optimizing the trajectories in each block.
When the number of waypoints in a single block is greater than $2$, the intrinsic optimality of the trajectory generated with RST is not guaranteed anymore and the optimization process provides $q(t)\neq 0$.  Next section illustrates trajectories generated via RST and via optimization of the integral of the $p$-th derivative of the position squared, denoted with RST$_{\text{opt}}$. 
