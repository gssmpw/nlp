\chapter{Introduction}
%\addcontentsline{toc}{chapter}{Introduction} 
%\thispagestyle{empty}
\label{sec:intro}

% scientific contributions section
% start with high-level sota DL for communications with references to future sections in this thesis, and then for each chapter (in bullet), explain contribution with references

% outline of the thesis as it is now more or less


\section{Motivation}
Communications engineering is the field studying the design, implementation, and optimization of systems that transmit information from one point to another. 
The exchange of information can occur over wired connections (e.g., twisted pairs, fiber optics), wireless channels (e.g., cellular networks), or a combination of both. 

Communication systems have revolutionized the way we interact and access information. This field has played a pivotal role in driving economic growth, fostering global collaboration and shaping social landscapes. However, the ever-increasing demand for data transmission poses significant challenges. The relentless growth of data traffic requires constant innovation to overcome limitations in signal degradation, bandwidth, and interference.

The physical layer is the first and lowest layer of the Open Systems Interconnection (OSI) model. It represents the heart of a communication system as it is responsible for the raw transmission of bits over a physical medium. This layer handles the encoding and decoding of data, the modulation and demodulation of signals, and it deals with several challenges posed by the medium such as attenuation, noise, and interference.  
A robust physical layer is essential for ensuring reliable data transmission, maximizing bandwidth, and supporting the increasing demands of modern communication systems.

Innovation in the physical layer design necessitates to find solutions to complex problems such as channel impairments, where real-world channels are far from ideal models and the transmitted signals suffer from fading, path loss, multi-path propagation, and various sources of noise. Another relevant problem is represented by bandwidth limitations, since the available spectrum for wireless communication is a finite resource. As demand grows, engineers must find alternative and creative ways to maximize data rates within constrained bandwidth allocations.
Interference also constitutes a serious challenge in crowded wireless environments, as signals from various sources can overlap and corrupt each other, making it difficult to isolate the desired transmission.
Lastly, power and complexity constraints set strict hardware requirements, especially in mobile and battery-powered devices. Therefore, achieving high-performance communication under multiple limitations requires careful optimization and innovative techniques.

Traditional approaches to physical layer design rely on complex mathematical models and hand-crafted signal processing algorithms. While effective, these methods are often sub-optimal and can struggle to fully adapt to the dynamic and unpredictable nature of communication channels. Deep learning (DL) offers a valid and compelling alternative. Its ability to learn complex patterns and relationships directly from data makes it a powerful tool for tackling physical layer challenges.

The traditional design approach was firstly proposed by Shannon \cite{Shannon1948}. He suggested to represent any communication system as a chain of mathematically modeled blocks, namely, the transmitter, the channel, and the receiver. 
The transmitter comprises several interconnected blocks, a source coder, a channel coder and a signal modulator. The channel is usually described by a transfer function and an additive noise term. The receiver, instead, serves as the counterpart to the transmitter and is often broken down into stages mirroring those of the transmitter, i.e., signal demodulation, channel decoding, and source decoding.

\begin{figure}
	\centering
	\includegraphics[scale=0.6]{images/intro/intro_image_crop.pdf}
	\caption{Different engineering design approaches: a) bottom-up and b) top-down.}
	\label{fig:intro_approach}
\end{figure}

Three generations of scientists and engineers grew up with this mathematical mindset which offered excellent tools to acquire domain knowledge and use it to create models for each constituent block, thereby rendering the overall behavior tractable. Such a framework intrinsically has the advantage that each component can be individually studied and optimized. We would refer to this approach as physical and \textit{bottom-up} and it is represented in Fig.~\ref{fig:intro_approach}a.

From an epistemological point of view, the mathematical theory of communications is based on knowledge coming from \textit{a-priori} hypotheses and intuitions. On the contrary, \textit{a-posteriori} knowledge is created by what is known and observed from experience, therefore generated via an empirical analysis and a \textit{top-down} approach. In his \textit{Critique of Pure Reason} \cite{Kant}, Kant differentiates between two types of knowledge: knowledge of the structure of time and space and their relationships, is a-priori knowledge; knowledge acquired from experience is a-posteriori knowledge; most of our knowledge comes from the process of learning and observing phenomena, and without the a-priori one it would be impossible to reach the true knowledge. 
In this respect, the broad field of machine learning (ML) \cite{Bishop2006,DLBook} can be considered an implementation by humans of techniques in machines to acquire knowledge from a-posteriori observations of realizations of natural phenomena, essentially by translating the complexity of real-world data into the complexity of the model, see Fig.~\ref{fig:intro_approach}b. 

ML is bringing new lymph in the domain of communication systems modeling, design, optimization, and management. It provides a paradigm shift: rather than only concentrating on a physical bottom-up description of the communication scheme, ML aims to learn and capture information from a collection of data, to derive the input-output relations of the observed system. Such learning strategy can be effectively adopted to cover multitude applications across all three fundamental protocol stacks: the physical layer as we introduced, but also the MAC layer and the network layer \cite{Simeone2018}.

\begin{figure}
	\centering
	\includegraphics[scale=0.65]{images/intro/when_ml_crop.pdf}
	\caption{Flowchart illustrating when to use ML for communication engineering design.}
	\label{fig:intro_when_ml}
\end{figure}

Given the current shift but also hype around ML, it is important to remark that a top-down data-driven approach should not always be pursued regardless the specifications of the problem. In fact, common sense guidelines \cite{Simeone2018} explaining \textit{when to use ML} tools for communications, and more in general, engineering tasks (see Fig.~\ref{fig:intro_when_ml}), suggest to adopt ML when a physics-based model (e.g., channel model) is lacking or the existing algorithms are too complex to be implemented (e.g., decoding strategy). Moreover, a sufficiently large training dataset shall be collected while the phenomenon being learned (e.g., noise) is stationary during the observation period. 

Finally, ML techniques can complement traditional approaches. For instance, ML can be integrated with analytical models or simulation-based methods to enhance performance and study limitations. In fact, the integration of ML and DL techniques with standard approaches in communication engineering offers a promising avenue for addressing the complexities and challenges of modern communication systems. By leveraging the complementary strengths of both methodologies, we could enhance system performance, adaptability, and reliability across various applications. 

\section{Scientific contributions}
Physical layer design encompasses multiple tasks where a data-driven approach can be pursued. 
At the transmitter side, DL has recently been applied to tackle tasks such as source and channel coding \cite{8461983}, signal encoding \cite{Dorner2018}, analog-to-digital (ADC) conversion and digital-to-analog conversion (DAC) \cite{8807322}. For more details, we refer to Sec.~\ref{sec:autoencoders_related} and Sec.~\ref{sec:cortical_intro}.

At the receiver side, channel detection and decoding represents, perhaps, the most immediate DL application since it can be treated as a standard classification task \cite{Nachmani2018}. Other tasks which benefit from data-driven models are channel estimation and equalization \cite{8353153}, modulation classification \cite{Oshea2017}, localization \cite{8482358}, interference and noise mitigation \cite{9802083}. For more details, we refer to Sec.~\ref{sec:mind_introduction} and Sec.~\ref{sec:autoencoders_related}.

Characterization and modeling of the communication medium are fundamental prerequisites for effective physical layer design. In fact, modulation schemes, coding techniques, and equalization algorithms are derived from a deep understanding of the medium, which can be aided by DL techniques \cite{8644250}. Equally important is the ability to synthesize close-to-real channel samples \cite{8792076} to effectively simulate, test and optimize the communication chain. In this respect, autoencoders have been proposed as fundamental tool for the end-to-end design of the whole chain \cite{Oshea2017}. We refer to Sec.~\ref{sec:gan_ch_synthesis} and Sec.~\ref{sec:autoencoders_related} for an in-depth analysis. 

While several interesting initial results have been proposed, there remains a vast landscape of unexplored possibilities, presenting a fertile ground for future research and innovation in the field.

This dissertation describes a journey towards optimal channel coding and decoding schemes via novel DL formulations. During this journey, we also borrow concepts and tools from probability and statistical learning theory, variational inference, signal processing, and information theory.

Concretely, we present contributions in five main research areas highly interconnected, which naturally cluster the chapters of this thesis into parts.
\begin{enumerate}
    \item \textbf{Density Estimation}. The first part utilizes the concept of copula to analyze data dependence and learn the underlying probability density function (PDF).
In this context, we will introduce:
\begin{itemize}
    \item a segmented methodology to implicitly model any copula and sample from it \cite{Letizia2020};
    \item a variational approach to explicitly model any copula density function and subsequently any PDF \cite{letizia2022copula}.
\end{itemize}
    \item \textbf{Physical Layer Design}. The second part focuses on analyzing the building blocks of a communication system and combining them into optimal autoencoding techniques.  
In this regard, our contributions are:
\begin{itemize}
    \item deep generative models able to synthesize samples from any real-world medium for which a dataset is available \cite{ML_PLC, Letizia2019a};
    \item a neural decoder that uses the mutual information (MI) as decoding criterion and sets apart from the standard classification approach \cite{tonello2022mind};
    \item the first autoencoder that considers the channel capacity as the optimal coding criterion \cite{Letizia2021}.
\end{itemize}
    \item \textbf{Mutual Information and Channel Capacity}. The third part addresses the complex and well-known channel capacity problem from a learning point of view. In particular, we propose:
\begin{itemize}
    \item a novel family of MI neural estimators based on the variational representation of the $f$-divergence \cite{f-DIME};
    \item the first cooperative framework that acts as a neural capacity estimator, referred to as CORTICAL \cite{CORTICAL};
    \item the code for training and validating CORTICAL on a set of non-Shannon representative scenarios \cite{CORTICAL_github}.
\end{itemize}
\item \textbf{Power Line Communication (PLC)}. The fourth part validates the proposed theoretical contributions in the context of PLC. In summary, we design and train:
\begin{itemize}
    \item the first deep generative networks capable of modeling the PLC channel and noise \cite{RighiniLetizia2019, Letizia2019a};
    \item a CORTICAL framework for a channel corrupted by additive Nakagami-$m$ noise under an average power constraint \cite{LetiziaIsplc2021}.
\end{itemize}
\item \textbf{Interpolation}. 
While the core of the thesis explores the theoretical foundations and applications of DL within communications systems, the final part shifts focus and studies the problem of determining valid fitting functions given a set of data points. Interpolation techniques are indeed crucial not only for data analysis and visualization but they also play a direct role in tasks such as channel modeling and signal processing. In this regard, our contribution is:
\begin{itemize}
    \item a novel recursive approach which uses polynomial and rational functions as interpolants \cite{LetiziaRobotics, 9525383}.
    \end{itemize}
\end{enumerate}

\section{Thesis outline}
The objective of this thesis is to reformulate and solve classical problems within physical layer communications, using techniques that arise in the emerging field of DL. 

In Ch.~\ref{sec:fundamentals}, we provide the basic notation and terminology used in this thesis and briefly review the current state-of-the-art statistical learning models. In particular, we distinguish between supervised and unsupervised learning approaches, as they are essential to properly categorize the type of problem in hand. We discuss about different estimation techniques and statistical quantities to measure the distances between probability distributions. We also briefly summarize over DL architectures exploited in this research and introduce generative models concepts, focusing on generative adversarial networks. 

Ch.~\ref{sec:copulas} studies the fundamental properties that dependent random variables share in terms of their joint distribution. For this purpose, the concept of copula is introduced and utilized inside a DL framework. We offer a methodology to generate new data via copula sampling, referred to as segmented generative networks, and we also propose a novel approach to estimate the PDF thanks to a copula-based discriminative formulation, referred to as CODINE.

Motivated by a lack of models and the existence of annotated datasets, In Ch.~\ref{sec:medium}, we present a full DL solution to the long-standing problem of medium modeling. In particular, we utilize generative models to learn the channel and noise distributions and generate new samples with the same statistical characteristics. The envisioned methodology is extremely general as it can be applied to any stationary communication channel. 

With both the generative and discriminative approaches at hand, Ch.~\ref{sec:decoder} reformulates the decoding problem in terms of discriminative learning and proposes a neural architecture for optimal decoding in an unknown channel. The learning process is capable of providing at convergence an estimate of the a-posteriori PDF, and thus of the MI of the input-output channel pair. We utilize the latter as decoding metric. Our top-down approach, referred to as MIND, is also motivated in this case due to either a lack of model for the classical maximum a-posteriori estimation or due to a complex decoding algorithm.

In Ch.~\ref{sec:autoencoders}, we extend the DL formulation to include also the transmitter, hence, we discuss autoencoders for optimal end-to-end communication. We review the main concepts and comment on the limitations of the existing literature, especially highlighting the role of the MI during training. We address the problem of designing capacity-achieving codes with autoencoders, showing evidence of the importance of incorporating a capacity term in the loss function. 
The autoencoder represents a promising framework to study optimal coding/decoding schemes even in the presence of a physical channel model, as for most channels, no optimal strategies are known.

The increasing relevance and presence of the MI in this research suggested us to invest resources in studying its estimation. Ch.~\ref{sec:mi_estimators} firstly summarizes the state-of-the-art on the MI estimation problem, focusing on the latest neural approaches. We then proposed a new family of estimators based on the $f$-divergence, referred to as $f$-DIME, with desired properties and excellent performance. We also developed a novel sampling strategy to correctly train unbiased neural estimators. 

Ch.~\ref{sec:cortical} represents the main contribution of this thesis. It presents a novel DL cooperative framework, referred to as CORTICAL, to estimate the channel capacity and build the capacity-achieving distribution of any discrete-time continuous memoryless vector channel. CORTICAL consists of two cooperative networks: a generator with the objective of learning to sample from the capacity-achieving input distribution, and a discriminator with the objective to learn to distinguish between paired and unpaired channel input-output samples. The latter utilizes $f$-DIME to estimate the MI. 

In Ch.~\ref{sec:plc}, the theoretical contributions of this research are applied to the power line communication medium, whose complex and variable nature offer interesting challenges and benefit from a DL perspective. We train generative models with real measured data, we apply the neural decoding strategy for the impedance modulation problem, we estimate the MI of channel input-output pairs affected by a specific type of noise present in power lines, for which we also study optimal coding schemes. Lastly, an application to anomaly detection is offered.

While the whole thesis can be related to neural distribution interpolation techniques for communication data and signals, we dedicate Ch.~\ref{sec:data interpolation} to provide insights about data and signals interpolation techniques via polynomials. 

In Ch.~\ref{sec:conclusion} we conclude the thesis and we summarize our findings, offering possible future research directions.

\section{Publications}
\label{sec:relatedpub}
We report below the list of publications where the thesis results have been documented. 

\noindent \textbf{Journals}:
\begin{itemize}
	\item A. M. Tonello, N. A. Letizia, D. Righini and F. Marcuzzi, "Machine Learning Tips and Tricks for Power Line Communications," \textit{IEEE Access}, vol. 7, pp. 82434-82452, June 2019.
     \item N. A. Letizia and A. M. Tonello, "Capacity-Driven Autoencoders for Communications," \textit{IEEE Open Journal of the Communications Society}, vol. 2, pp. 1366-1378, June 2021.
    \item N. A. Letizia, B. Salamat and A. M. Tonello, "A Novel Recursive Smooth Trajectory Generation Method for Unmanned Vehicles," \textit{IEEE Transactions on Robotics}, vol. 37, no. 5, pp. 1792-1805, Oct. 2021.
    \item N. A. Letizia and A. M. Tonello, "Segmented Generative Networks: Data Generation in the Uniform Probability Space," \textit{IEEE Transactions on Neural Networks and Learning Systems}, vol. 33, no. 3, pp. 1338-1347, March 2022.
    \item A. M. Tonello and N. A. Letizia, "MIND: Maximum Mutual Information Based Neural Decoder," \textit{IEEE Communications Letters}, vol. 26, no. 12, pp. 2954-2958, Dec. 2022.
    \item N. A. Letizia, A. M. Tonello and H. Vincent Poor, "Cooperative Channel Capacity Learning," \textit{IEEE Communications Letters}, vol. 27, no. 8, pp. 1984-1988, Aug. 2023.
    \item B. Salamat, N. A. Letizia and A. M. Tonello, "Control Based Motion Planning Exploiting Calculus of Variations and Rational Functions: A Formal Approach," \textit{IEEE Access}, vol. 9, pp. 121716-121727, 2021.
    \item N. A. Letizia and A. M. Tonello, "Copula Density Neural Estimation," \textit{ArXiV} 2211.15353.
    \item N. A. Letizia and A. M. Tonello, "Discriminative Mutual Information Estimators for Channel Capacity Learning", \textit{ArXiV} 2107.03084.
\end{itemize}
\textbf{Conferences}:
\begin{itemize}
\item D. Righini, N. A. Letizia and A. M. Tonello, "Synthetic Power Line Communications Channel Generation with Autoencoders and GANs," \textit{in Proceedings of the IEEE International Conference on Communications, Control, and Computing Technologies for Smart Grids (SmartGridComm)}, Beijing, China, 2019, pp. 1-6.
\item N. A. Letizia, B. Salamat and A. M. Tonello, "A New Recursive Framework for Trajectory Generation of UAVs," \textit{in Proceedings of the IEEE Aerospace Conference}, Big Sky, MT, USA, 2020, pp. 1-8.
\item N. A. Letizia, A. M. Tonello and D. Righini, "Learning to Synthesize Noise: The Multiple Conductor Power Line Case," \textit{in Proceedings of the IEEE International Symposium on Power Line Communications and its Applications (ISPLC)}, Malaga, Spain, 2020, pp. 1-6.
\item N. A. Letizia and A. M. Tonello, "Supervised Fault Detection in Energy Grids Measuring Electrical Quantities in the PLC Band," \textit{in Proceedings of the IEEE International Symposium on Power Line Communications and its Applications (ISPLC)}, Malaga, Spain, 2020, pp. 1-5.
\item F. Marcuzzi, A. M. Tonello and N. A. Letizia, "Discovering Routing Anomalies in Large PLC Metering Deployments from Field Data," \textit{in Proceedings of the IEEE International Symposium on Power Line Communications and its Applications (ISPLC)}, Malaga, Spain, 2020, pp. 1-6.
\item A. M. Tonello, N. A. Letizia and M. De Piante, "Learning the Impedance Entanglement for Wireline Data Communication," \textit{in Proceedings of the International Balkan Conference on Communications and Networking (BalkanCom)}, Novi Sad, Serbia, 2021, pp. 96-100. 
\item N. A. Letizia and A. M. Tonello, "Capacity Learning for Communication Systems over Power Lines," \textit{in Proceedings of the IEEE International Symposium on Power Line Communications and its Applications (ISPLC)}, Aachen, Germany, 2021, pp. 55-60.
\item N. A. Letizia and A. M. Tonello, "Discriminative Mutual Information Estimation for the Design of Channel Capacity Driven Autoencoders," \textit{in Proceedings of the International Balkan Conference on Communications and Networking (BalkanCom)}, Sarajevo, Bosnia and Herzegovina, 2022, pp. 41-45.
\item N. A. Letizia, N. Novello and A. M. Tonello, "Mutual Information Estimation via $f$-Divergence and Data Derangements", \textit{in Advances in Neural Information Processing Systems (NeurIPS)}, Vancouver, Canada, 2024, pp. 105114-105150.
\end{itemize}