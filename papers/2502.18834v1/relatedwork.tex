\section{Related Work}
\label{sec:rela}

% \subsection{Financial Time Series Forecasting}
% \subsection{Financial Benchmarks}

To date, there remains a significant gap in the availability of a comprehensive dataset of financial time series. Existing studies typically focus on specific slices of historical stock data for their experiments. For example, ALSP-TF~\cite{ALSP}, ADB-TRM~\cite{ADBTR}, and CI-STHPAN~\cite{cisthpan} utilize New York Stock Exchange (NYSE) and NASDAQ stocks for the period from 2013 to 2017. Meanwhile, MASTER~\cite{master}, FactorVAE~\cite{FactorVAE}, LARA~\cite{lara}, and RSAP-DFM~\cite{rsapdfm} select data from the Chinese A-share market, but with distinct temporal ranges: MASTER~\cite{master} covers 2008 to 2022, FactorVAE~\cite{FactorVAE} covers 2010 to 2020, and both LARA~\cite{lara} and RSAP-DFM~\cite{rsapdfm} cover 2008 to 2020. Similarly, LSR-iGRU~\cite{lsrigru}, FinMamba~\cite{finmamba}, and MCI-GRU~\cite{mcigru}, as well as THGNN~\cite{thgnn}, focus on stocks from both the Chinese and U.S. markets, with time slices ranging from 2018 to 2023 and 2016 to 2021, respectively. Qlib~\cite{qlib} provides a wealth of raw data and factor data, from which users can extract the segments they need.
The varying time horizons of these data slices pose a challenge to consistent and fair evaluations.
Furthermore, after applying Principal Component Analysis (PCA)~\cite{pca} for dimensionality reduction, we visualized \name and other time-sliced stocks as a hexbin plot in \cref{fig:hexbin}. The results indicate that \name covers the most cells, suggesting that reliance on sliced historical data lacks diversity. In contrast, \name offers more comprehensive coverage, thus effectively capturing the complex distribution of financial time series.