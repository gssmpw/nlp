% \section*{Impact Statement}
% This paper aims to advance the interdisciplinary field of machine learning and epidemiology, with a focus on improving the accuracy of epidemic forecasting in the limited data setting. We summarize the potential impacts as follows:

% \textbf{Early Insights.}
% This paper provides early insights into how pre-training and environment modeling helps in epidemic forecasting. Specifically, we demonstrate the effectiveness of pre-training in improving the model's accuracy and the improvement increases with more pre-training materials, inspiring future work that may utilize a larger corpus for building a foundational model in epidemic analysis. In addition, the necessity of accounting for both inherited disease dynamics and environmental influence is confirmed in this paper.

% \textbf{Social Impact.}
% Due to a larger sampling rate, the collected epidemic time series data is usually limited, and the public health organization may not be able to make an accurate prediction of infections during a novel disease outbreak. In this paper, we seek to address this problem by uncovering the few-shot and zero-shot forecasting ability of pre-trained models, which can play a powerful role in early warning of epidemic outbreaks.


% \section*{Impact Statement}
% This paper advances the interdisciplinary fields of machine learning and epidemiology by enhancing the accuracy of epidemic forecasting in data-limited settings. We outline the potential impacts as follows:

% \textbf{Early Insights.}
% We provide novel insights into how pre-training and environment modeling improve epidemic forecasting. Our results demonstrate that pre-training significantly enhances model accuracy, with gains increasing as more pre-training data is incorporated. This finding paves the way for future research to develop foundational models in epidemic analysis using larger datasets. Additionally, we confirm the importance of accounting for both inherent disease dynamics and environmental factors to achieve robust forecasting performance.

% \textbf{Social Impact.}
% Epidemic time series data are often sparse due to limited sampling rates, hindering public health organizations' ability to accurately predict infections during novel disease outbreaks. This paper addresses this challenge by showcasing the few-shot and zero-shot forecasting capabilities of pre-trained models. These capabilities can provide powerful tools for early warning and timely intervention, ultimately supporting more effective public health responses and safeguarding communities against emerging infectious diseases.




