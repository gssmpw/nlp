%%%%%%%% ICML 2025 EXAMPLE LATEX SUBMISSION FILE %%%%%%%%%%%%%%%%%

\documentclass{article}

\usepackage[preprint]{neurips_2024}

\title{CAPE: Covariate-Adjusted Pre-Training for Epidemic Time Series Forecasting}

\author{%
  Zewen Liu \\
  Department of Computer Science\\
  Emory University\\
  Atlanta, Georgia \\
  \texttt{zewen.liu@emory.edu} \\
  \And
  Juntong Ni \\
  Department of Computer Science\\
  Emory University\\
  Atlanta, Georgia \\
  \texttt{juntong.ni@emory.edu} \\
  \And
  Max S. Y. Lau \\
  Emory University \\
  Rollins School of Public Health \\
  Atlanta, Georgia \\
  \texttt{msy.lau@emory.edu} \\
  \And
  Wei Jin \\
  Department of Computer Science\\
  Emory University\\
  Atlanta, Georgia \\
  \texttt{wei.jin@emory.edu} \\
}


% Recommended, but optional, packages for figures and better typesetting:
\usepackage{microtype}
\usepackage{graphicx}
\usepackage{subcaption}
\usepackage{booktabs} % forprofessional tables
\usepackage{amssymb}
\usepackage{color}
\usepackage{xcolor,colortbl}
% \usepackage[table]{xcolor}        % 用于表格着色
% \usepackage{siunitx}              % 用于数值对齐
\usepackage{multirow}             % 用于跨行
\usepackage{graphicx}             % 用于调整表格大小
\usepackage[normalem]{ulem}       % 用于下划线
\usepackage{algorithm}
\usepackage{algpseudocode}
\usepackage{paralist}
\usepackage{enumitem}
\usepackage{amsthm}

%%%%%%%%%%%%%%%%%%%%%%%%%%%%%%%%
% THEOREMS
%%%%%%%%%%%%%%%%%%%%%%%%%%%%%%%%
\theoremstyle{plain}
\newtheorem{theorem}{Theorem}[section]
\newtheorem{proposition}[theorem]{Proposition}
\newtheorem{lemma}[theorem]{Lemma}
\newtheorem{corollary}[theorem]{Corollary}
\theoremstyle{definition}
\newtheorem{definition}[theorem]{Definition}
\newtheorem{assumption}[theorem]{Assumption}
\theoremstyle{remark}
\newtheorem{remark}[theorem]{Remark}


\usepackage[normalem]{ulem}
\usepackage{tabularx}

\definecolor{low1}{RGB}{255, 204, 204}    % 红色浅色
\definecolor{low3}{RGB}{255, 255, 204}    % 黄色浅色
\definecolor{low2}{RGB}{204, 255, 204}    % 绿色浅色

\usepackage{hyperref}


\newcommand{\theHalgorithm}{\arabic{algorithm}}

\usepackage{soul}

% For theorems and such
\usepackage{amsmath}
\usepackage{mathtools}

% if you use cleveref..
\usepackage[capitalize,noabbrev]{cleveref}
\usepackage[textsize=tiny]{todonotes}

\newcommand{\wei}[1]{\textcolor{orange}{\#Wei:~#1\#}}
\newcommand{\zw}[1]{\textcolor{blue}{\#Zewen:~#1\#}}
\newcommand{\juntong}[1]{\textcolor{red}{\#Juntong:~#1\#}}


\usepackage[utf8]{inputenc} % allow utf-8 input
\usepackage[T1]{fontenc}    % use 8-bit T1 fonts
\usepackage{hyperref}       % hyperlinks
\usepackage{url}            % simple URL typesetting
\usepackage{booktabs}       % professional-quality tables
\usepackage{amsfonts}       % blackboard math symbols
\usepackage{nicefrac}       % compact symbols for 1/2, etc.
\usepackage{microtype}      % microtypography
\usepackage{xcolor}         % colors





\begin{document}
\maketitle


%----------------------------------------------------------------------------------------------


\begin{abstract}

% Modeling the dynamics of epidemic spreading to provide accurate forecasting has been a crucial problem in protecting public health. While previous research has shown great potential for data-driven methods, the sparsity of data and lack of context of environments hinder the learning of actual underlying patterns, making the model hard to adapt to temporal distribution shifts. To address these challenges, we propose \textbf{CAPE}, a \textbf{c}ovariate-\textbf{a}djusted \textbf{r}epresentation \textbf{l}earning framework that utilizes pre-training and contrastive learning to learn the underlying environment representations, providing adjusted predictions. We gather and create an epidemic pile with over 21 distinct disease types across more than 60 regions for pre-training and testing our model. Pre-training not only deals with the sparsity of data in the downstream task, but also aid in the learning of comprehensive environments which can be unseen in the downstream training set, thereby mitigating the influence brought by temporal distribution shifts. Extensive experiments show the effectiveness of the proposed CAPE framework in full-shot, few-shot, and zero-shot settings.

% Accurate forecasting of epidemic transmission dynamics is essential for safeguarding public health. While data-driven methods have demonstrated significant potential, challenges such as data sparsity and the lack of contextual environmental information hinder the learning of actual underlying patterns, limiting these methods' ability to adapt to temporal distribution shifts. To address these issues, we propose \textbf{CAPE} (\textbf{C}ovariate-\textbf{A}djusted \textbf{P}retraining for \textbf{E}pidemic forecasting), a framework that leverages pre-training and contrastive learning to capture underlying environment representations, thereby enabling adjusted and more accurate predictions. For the development and evaluation of our model, we compiled an extensive dataset encompassing 21 distinct diseases across over 60 regions. The pre-training process not only mitigates data sparsity in downstream tasks but also facilitates the learning of comprehensive environmental contexts. This approach effectively reduces the impact of temporal distribution shifts on model performance. Extensive experiments show the efficacy of the proposed \textbf{CAPE} framework across full-shot, few-shot, and zero-shot settings, highlighting the robustness and adaptability in various forecasting scenarios.

% Accurate forecasting of epidemic transmission dynamics is essential for safeguarding public health. However, the observed epidemic time series data is usually limited in length and the inherited dynamics of diseases are entangled with the environmental influence, hindering the learning of data-driven methods trained for specific diseases and regions. To address these limitations and utilize valuable insights from past outbreaks, we propose \textbf{CAPE} (\textbf{C}ovariate-\textbf{A}djusted \textbf{P}retraining for \textbf{E}pidemic forecasting), a framework that leverages pre-training and hierarchical environment contrasting to capture the universal patterns from the past outbreaks as well as the underlying environment representations, thereby enabling adjusted and more accurate predictions. For the development and evaluation of our model, we compiled an extensive dataset encompassing 21 distinct diseases across over 60 regions. The pre-training process not only mitigates data sparsity in downstream tasks but also facilitates the learning of comprehensive environmental contexts. This approach effectively reduces the impact of temporal distribution shifts on model performance. Extensive experiments show the efficacy of the proposed \textbf{CAPE} framework across full-shot, few-shot, and zero-shot settings, highlighting the robustness and adaptability in various forecasting scenarios.

Accurate forecasting of epidemic infection trajectories is crucial for safeguarding public health. However, limited data availability during emerging outbreaks and the complex interaction between environmental factors and disease dynamics present significant challenges for effective forecasting. 
In response, we introduce CAPE, a novel epidemic pre-training framework designed to harness extensive disease datasets from diverse regions and integrate environmental factors directly into the modeling process for more informed decision-making on downstream diseases. Based on a covariate adjustment framework, CAPE utilizes pre-training combined with hierarchical environment contrasting to identify universal patterns across diseases while estimating latent environmental influences. We have compiled a diverse collection of epidemic time series datasets and validated the effectiveness of CAPE under various evaluation scenarios, including full-shot, few-shot, zero-shot, cross-location, and cross-disease settings, where it outperforms the leading baseline by an average of 9.9\% in full-shot and 14.3\% in zero-shot settings. 
The code will be released upon acceptance.


% While pre-training helps mitigate the impact of data sparsity in downstream tasks, environment modeling effectively uncovers the inherited disease infection dynamics and reduces the impact of temporal distribution shifts on model performance. We developed and evaluated our model using an extensive dataset covering 21 diseases across over 60 regions. During the evaluation, \textbf{CAPE} surpasses the best baseline by an average of 9.9\% in the full-shot setting and 18.1\% in the zero-shot setting across the downstream datasets, demonstrating the effectiveness of the proposed framework.


% However, epidemic time series data are typically scarce during novel outbreaks, and the disease infection dynamics are heavily influenced by environmental factors, causing distribution shifts between the training and test sets. These challenges hinder the effectiveness of data-driven models that are trained on specific diseases and regions. To address these limitations and leverage insights from past outbreaks, we propose \textbf{CAPE} (\textbf{C}ovariate-\textbf{A}djusted \textbf{P}retraining for \textbf{E}pidemic forecasting). CAPE leverages pre-training and hierarchical environment contrasting to capture universal patterns and underlying environmental representations. While pre-training helps mitigate the impact of data sparsity in downstream tasks, environment modeling effectively uncovers the inherited disease infection dynamics and reduces the impact of temporal distribution shifts on model performance. We developed and evaluated our model using an extensive dataset covering 21 diseases across over 60 regions. During the evaluation, \textbf{CAPE} surpasses the best baseline by an average of 9.9\% in the full-shot setting and 18.1\% in the zero-shot setting across the downstream datasets, demonstrating the effectiveness of the proposed framework.


\end{abstract}

\section{Introduction}


\begin{figure}[t]
\centering
\includegraphics[width=0.6\columnwidth]{figures/evaluation_desiderata_V5.pdf}
\vspace{-0.5cm}
\caption{\systemName is a platform for conducting realistic evaluations of code LLMs, collecting human preferences of coding models with real users, real tasks, and in realistic environments, aimed at addressing the limitations of existing evaluations.
}
\label{fig:motivation}
\end{figure}

\begin{figure*}[t]
\centering
\includegraphics[width=\textwidth]{figures/system_design_v2.png}
\caption{We introduce \systemName, a VSCode extension to collect human preferences of code directly in a developer's IDE. \systemName enables developers to use code completions from various models. The system comprises a) the interface in the user's IDE which presents paired completions to users (left), b) a sampling strategy that picks model pairs to reduce latency (right, top), and c) a prompting scheme that allows diverse LLMs to perform code completions with high fidelity.
Users can select between the top completion (green box) using \texttt{tab} or the bottom completion (blue box) using \texttt{shift+tab}.}
\label{fig:overview}
\end{figure*}

As model capabilities improve, large language models (LLMs) are increasingly integrated into user environments and workflows.
For example, software developers code with AI in integrated developer environments (IDEs)~\citep{peng2023impact}, doctors rely on notes generated through ambient listening~\citep{oberst2024science}, and lawyers consider case evidence identified by electronic discovery systems~\citep{yang2024beyond}.
Increasing deployment of models in productivity tools demands evaluation that more closely reflects real-world circumstances~\citep{hutchinson2022evaluation, saxon2024benchmarks, kapoor2024ai}.
While newer benchmarks and live platforms incorporate human feedback to capture real-world usage, they almost exclusively focus on evaluating LLMs in chat conversations~\citep{zheng2023judging,dubois2023alpacafarm,chiang2024chatbot, kirk2024the}.
Model evaluation must move beyond chat-based interactions and into specialized user environments.



 

In this work, we focus on evaluating LLM-based coding assistants. 
Despite the popularity of these tools---millions of developers use Github Copilot~\citep{Copilot}---existing
evaluations of the coding capabilities of new models exhibit multiple limitations (Figure~\ref{fig:motivation}, bottom).
Traditional ML benchmarks evaluate LLM capabilities by measuring how well a model can complete static, interview-style coding tasks~\citep{chen2021evaluating,austin2021program,jain2024livecodebench, white2024livebench} and lack \emph{real users}. 
User studies recruit real users to evaluate the effectiveness of LLMs as coding assistants, but are often limited to simple programming tasks as opposed to \emph{real tasks}~\citep{vaithilingam2022expectation,ross2023programmer, mozannar2024realhumaneval}.
Recent efforts to collect human feedback such as Chatbot Arena~\citep{chiang2024chatbot} are still removed from a \emph{realistic environment}, resulting in users and data that deviate from typical software development processes.
We introduce \systemName to address these limitations (Figure~\ref{fig:motivation}, top), and we describe our three main contributions below.


\textbf{We deploy \systemName in-the-wild to collect human preferences on code.} 
\systemName is a Visual Studio Code extension, collecting preferences directly in a developer's IDE within their actual workflow (Figure~\ref{fig:overview}).
\systemName provides developers with code completions, akin to the type of support provided by Github Copilot~\citep{Copilot}. 
Over the past 3 months, \systemName has served over~\completions suggestions from 10 state-of-the-art LLMs, 
gathering \sampleCount~votes from \userCount~users.
To collect user preferences,
\systemName presents a novel interface that shows users paired code completions from two different LLMs, which are determined based on a sampling strategy that aims to 
mitigate latency while preserving coverage across model comparisons.
Additionally, we devise a prompting scheme that allows a diverse set of models to perform code completions with high fidelity.
See Section~\ref{sec:system} and Section~\ref{sec:deployment} for details about system design and deployment respectively.



\textbf{We construct a leaderboard of user preferences and find notable differences from existing static benchmarks and human preference leaderboards.}
In general, we observe that smaller models seem to overperform in static benchmarks compared to our leaderboard, while performance among larger models is mixed (Section~\ref{sec:leaderboard_calculation}).
We attribute these differences to the fact that \systemName is exposed to users and tasks that differ drastically from code evaluations in the past. 
Our data spans 103 programming languages and 24 natural languages as well as a variety of real-world applications and code structures, while static benchmarks tend to focus on a specific programming and natural language and task (e.g. coding competition problems).
Additionally, while all of \systemName interactions contain code contexts and the majority involve infilling tasks, a much smaller fraction of Chatbot Arena's coding tasks contain code context, with infilling tasks appearing even more rarely. 
We analyze our data in depth in Section~\ref{subsec:comparison}.



\textbf{We derive new insights into user preferences of code by analyzing \systemName's diverse and distinct data distribution.}
We compare user preferences across different stratifications of input data (e.g., common versus rare languages) and observe which affect observed preferences most (Section~\ref{sec:analysis}).
For example, while user preferences stay relatively consistent across various programming languages, they differ drastically between different task categories (e.g. frontend/backend versus algorithm design).
We also observe variations in user preference due to different features related to code structure 
(e.g., context length and completion patterns).
We open-source \systemName and release a curated subset of code contexts.
Altogether, our results highlight the necessity of model evaluation in realistic and domain-specific settings.






\section{RELATED WORK}
\label{sec:relatedwork}
In this section, we describe the previous works related to our proposal, which are divided into two parts. In Section~\ref{sec:relatedwork_exoplanet}, we present a review of approaches based on machine learning techniques for the detection of planetary transit signals. Section~\ref{sec:relatedwork_attention} provides an account of the approaches based on attention mechanisms applied in Astronomy.\par

\subsection{Exoplanet detection}
\label{sec:relatedwork_exoplanet}
Machine learning methods have achieved great performance for the automatic selection of exoplanet transit signals. One of the earliest applications of machine learning is a model named Autovetter \citep{MCcauliff}, which is a random forest (RF) model based on characteristics derived from Kepler pipeline statistics to classify exoplanet and false positive signals. Then, other studies emerged that also used supervised learning. \cite{mislis2016sidra} also used a RF, but unlike the work by \citet{MCcauliff}, they used simulated light curves and a box least square \citep[BLS;][]{kovacs2002box}-based periodogram to search for transiting exoplanets. \citet{thompson2015machine} proposed a k-nearest neighbors model for Kepler data to determine if a given signal has similarity to known transits. Unsupervised learning techniques were also applied, such as self-organizing maps (SOM), proposed \citet{armstrong2016transit}; which implements an architecture to segment similar light curves. In the same way, \citet{armstrong2018automatic} developed a combination of supervised and unsupervised learning, including RF and SOM models. In general, these approaches require a previous phase of feature engineering for each light curve. \par

%DL is a modern data-driven technology that automatically extracts characteristics, and that has been successful in classification problems from a variety of application domains. The architecture relies on several layers of NNs of simple interconnected units and uses layers to build increasingly complex and useful features by means of linear and non-linear transformation. This family of models is capable of generating increasingly high-level representations \citep{lecun2015deep}.

The application of DL for exoplanetary signal detection has evolved rapidly in recent years and has become very popular in planetary science.  \citet{pearson2018} and \citet{zucker2018shallow} developed CNN-based algorithms that learn from synthetic data to search for exoplanets. Perhaps one of the most successful applications of the DL models in transit detection was that of \citet{Shallue_2018}; who, in collaboration with Google, proposed a CNN named AstroNet that recognizes exoplanet signals in real data from Kepler. AstroNet uses the training set of labelled TCEs from the Autovetter planet candidate catalog of Q1–Q17 data release 24 (DR24) of the Kepler mission \citep{catanzarite2015autovetter}. AstroNet analyses the data in two views: a ``global view'', and ``local view'' \citep{Shallue_2018}. \par


% The global view shows the characteristics of the light curve over an orbital period, and a local view shows the moment at occurring the transit in detail

%different = space-based

Based on AstroNet, researchers have modified the original AstroNet model to rank candidates from different surveys, specifically for Kepler and TESS missions. \citet{ansdell2018scientific} developed a CNN trained on Kepler data, and included for the first time the information on the centroids, showing that the model improves performance considerably. Then, \citet{osborn2020rapid} and \citet{yu2019identifying} also included the centroids information, but in addition, \citet{osborn2020rapid} included information of the stellar and transit parameters. Finally, \citet{rao2021nigraha} proposed a pipeline that includes a new ``half-phase'' view of the transit signal. This half-phase view represents a transit view with a different time and phase. The purpose of this view is to recover any possible secondary eclipse (the object hiding behind the disk of the primary star).


%last pipeline applies a procedure after the prediction of the model to obtain new candidates, this process is carried out through a series of steps that include the evaluation with Discovery and Validation of Exoplanets (DAVE) \citet{kostov2019discovery} that was adapted for the TESS telescope.\par
%



\subsection{Attention mechanisms in astronomy}
\label{sec:relatedwork_attention}
Despite the remarkable success of attention mechanisms in sequential data, few papers have exploited their advantages in astronomy. In particular, there are no models based on attention mechanisms for detecting planets. Below we present a summary of the main applications of this modeling approach to astronomy, based on two points of view; performance and interpretability of the model.\par
%Attention mechanisms have not yet been explored in all sub-areas of astronomy. However, recent works show a successful application of the mechanism.
%performance

The application of attention mechanisms has shown improvements in the performance of some regression and classification tasks compared to previous approaches. One of the first implementations of the attention mechanism was to find gravitational lenses proposed by \citet{thuruthipilly2021finding}. They designed 21 self-attention-based encoder models, where each model was trained separately with 18,000 simulated images, demonstrating that the model based on the Transformer has a better performance and uses fewer trainable parameters compared to CNN. A novel application was proposed by \citet{lin2021galaxy} for the morphological classification of galaxies, who used an architecture derived from the Transformer, named Vision Transformer (VIT) \citep{dosovitskiy2020image}. \citet{lin2021galaxy} demonstrated competitive results compared to CNNs. Another application with successful results was proposed by \citet{zerveas2021transformer}; which first proposed a transformer-based framework for learning unsupervised representations of multivariate time series. Their methodology takes advantage of unlabeled data to train an encoder and extract dense vector representations of time series. Subsequently, they evaluate the model for regression and classification tasks, demonstrating better performance than other state-of-the-art supervised methods, even with data sets with limited samples.

%interpretation
Regarding the interpretability of the model, a recent contribution that analyses the attention maps was presented by \citet{bowles20212}, which explored the use of group-equivariant self-attention for radio astronomy classification. Compared to other approaches, this model analysed the attention maps of the predictions and showed that the mechanism extracts the brightest spots and jets of the radio source more clearly. This indicates that attention maps for prediction interpretation could help experts see patterns that the human eye often misses. \par

In the field of variable stars, \citet{allam2021paying} employed the mechanism for classifying multivariate time series in variable stars. And additionally, \citet{allam2021paying} showed that the activation weights are accommodated according to the variation in brightness of the star, achieving a more interpretable model. And finally, related to the TESS telescope, \citet{morvan2022don} proposed a model that removes the noise from the light curves through the distribution of attention weights. \citet{morvan2022don} showed that the use of the attention mechanism is excellent for removing noise and outliers in time series datasets compared with other approaches. In addition, the use of attention maps allowed them to show the representations learned from the model. \par

Recent attention mechanism approaches in astronomy demonstrate comparable results with earlier approaches, such as CNNs. At the same time, they offer interpretability of their results, which allows a post-prediction analysis. \par



\section{Preliminaries}
\label{sec:prelim}
\label{sec:term}
We define the key terminologies used, primarily focusing on the hidden states (or activations) during the forward pass. 

\paragraph{Components in an attention layer.} We denote $\Res$ as the residual stream. We denote $\Val$ as Value (states), $\Qry$ as Query (states), and $\Key$ as Key (states) in one attention head. The \attlogit~represents the value before the softmax operation and can be understood as the inner product between  $\Qry$  and  $\Key$. We use \Attn~to denote the attention weights of applying the SoftMax function to \attlogit, and ``attention map'' to describe the visualization of the heat map of the attention weights. When referring to the \attlogit~from ``$\tokenB$'' to  ``$\tokenA$'', we indicate the inner product  $\langle\Qry(\tokenB), \Key(\tokenA)\rangle$, specifically the entry in the ``$\tokenB$'' row and ``$\tokenA$'' column of the attention map.

\paragraph{Logit lens.} We use the method of ``Logit Lens'' to interpret the hidden states and value states \citep{belrose2023eliciting}. We use \logit~to denote pre-SoftMax values of the next-token prediction for LLMs. Denote \readout~as the linear operator after the last layer of transformers that maps the hidden states to the \logit. 
The logit lens is defined as applying the readout matrix to residual or value states in middle layers. Through the logit lens, the transformed hidden states can be interpreted as their direct effect on the logits for next-token prediction. 

\paragraph{Terminologies in two-hop reasoning.} We refer to an input like “\Src$\to$\brga, \brgb$\to$\Ed” as a two-hop reasoning chain, or simply a chain. The source entity $\Src$ serves as the starting point or origin of the reasoning. The end entity $\Ed$ represents the endpoint or destination of the reasoning chain. The bridge entity $\Brg$ connects the source and end entities within the reasoning chain. We distinguish between two occurrences of $\Brg$: the bridge in the first premise is called $\brga$, while the bridge in the second premise that connects to $\Ed$ is called $\brgc$. Additionally, for any premise ``$\tokenA \to \tokenB$'', we define $\tokenA$ as the parent node and $\tokenB$ as the child node. Furthermore, if at the end of the sequence, the query token is ``$\tokenA$'', we define the chain ``$\tokenA \to \tokenB$, $\tokenB \to \tokenC$'' as the Target Chain, while all other chains present in the context are referred to as distraction chains. Figure~\ref{fig:data_illustration} provides an illustration of the terminologies.

\paragraph{Input format.}
Motivated by two-hop reasoning in real contexts, we consider input in the format $\bos, \text{context information}, \query, \answer$. A transformer model is trained to predict the correct $\answer$ given the query $\query$ and the context information. The context compromises of $K=5$ disjoint two-hop chains, each appearing once and containing two premises. Within the same chain, the relative order of two premises is fixed so that \Src$\to$\brga~always precedes \brgb$\to$\Ed. The orders of chains are randomly generated, and chains may interleave with each other. The labels for the entities are re-shuffled for every sequence, choosing from a vocabulary size $V=30$. Given the $\bos$ token, $K=5$ two-hop chains, \query, and the \answer~tokens, the total context length is $N=23$. Figure~\ref{fig:data_illustration} also illustrates the data format. 

\paragraph{Model structure and training.} We pre-train a three-layer transformer with a single head per layer. Unless otherwise specified, the model is trained using Adam for $10,000$ steps, achieving near-optimal prediction accuracy. Details are relegated to Appendix~\ref{app:sec_add_training_detail}.


% \RZ{Do we use source entity, target entity, and mediator entity? Or do we use original token, bridge token, end token?}





% \paragraph{Basic notations.} We use ... We use $\ve_i$ to denote one-hot vectors of which only the $i$-th entry equals one, and all other entries are zero. The dimension of $\ve_i$ are usually omitted and can be inferred from contexts. We use $\indicator\{\cdot\}$ to denote the indicator function.

% Let $V > 0$ be a fixed positive integer, and let $\vocab = [V] \defeq \{1, 2, \ldots, V\}$ be the vocabulary. A token $v \in \vocab$ is an integer in $[V]$ and the input studied in this paper is a sequence of tokens $s_{1:T} \defeq (s_1, s_2, \ldots, s_T) \in \vocab^T$ of length $T$. For any set $\mathcal{S}$, we use $\Delta(\mathcal{S})$ to denote the set of distributions over $\mathcal{S}$.

% % to a sequence of vectors $z_1, z_2, \ldots, z_T \in \real^{\dout}$ of dimension $\dout$ and length $T$.

% Let $\mU = [\vu_1, \vu_2, \ldots, \vu_V]^\transpose \in \real^{V\times d}$ denote the token embedding matrix, where the $i$-th row $\vu_i \in \real^d$ represents the $d$-dimensional embedding of token $i \in [V]$. Similarly, let $\mP = [\vp_1, \vp_2, \ldots, \vp_T]^\transpose \in \real^{T\times d}$ denote the positional embedding matrix, where the $i$-th row $\vp_i \in \real^d$ represents the $d$-dimensional embedding of position $i \in [T]$. Both $\mU$ and $\mP$ can be fixed or learnable.

% After receiving an input sequence of tokens $s_{1:T}$, a transformer will first process it using embedding matrices $\mU$ and $\mP$ to obtain a sequence of vectors $\mH = [\vh_1, \vh_2, \ldots, \vh_T] \in \real^{d\times T}$, where 
% \[
% \vh_i = \mU^\transpose\ve_{s_i} + \mP^\transpose\ve_{i} = \vu_{s_i} + \vp_i.
% \]

% We make the following definitions of basic operations in a transformer.

% \begin{definition}[Basic operations in transformers] 
% \label{defn:operators}
% Define the softmax function $\softmax(\cdot): \real^d \to \real^d$ over a vector $\vv \in \real^d$ as
% \[\softmax(\vv)_i = \frac{\exp(\vv_i)}{\sum_{j=1}^d \exp(\vv_j)} \]
% and define the softmax function $\softmax(\cdot): \real^{m\times n} \to \real^{m \times n}$ over a matrix $\mV \in \real^{m\times n}$ as a column-wise softmax operator. For a squared matrix $\mM \in \real^{m\times m}$, the causal mask operator $\mask(\cdot): \real^{m\times m} \to \real^{m\times m}$  is defined as $\mask(\mM)_{ij} = \mM_{ij}$ if $i \leq j$ and  $\mask(\mM)_{ij} = -\infty$ otherwise. For a vector $\vv \in \real^n$ where $n$ is the number of hidden neurons in a layer, we use $\layernorm(\cdot): \real^n \to \real^n$ to denote the layer normalization operator where
% \[
% \layernorm(\vv)_i = \frac{\vv_i-\mu}{\sigma}, \mu = \frac{1}{n}\sum_{j=1}^n \vv_j, \sigma = \sqrt{\frac{1}{n}\sum_{j=1}^n (\vv_j-\mu)^2}
% \]
% and use $\layernorm(\cdot): \real^{n\times m} \to \real^{n\times m}$ to denote the column-wise layer normalization on a matrix.
% We also use $\nonlin(\cdot)$ to denote element-wise nonlinearity such as $\relu(\cdot)$.
% \end{definition}

% The main components of a transformer are causal self-attention heads and MLP layers, which are defined as follows.

% \begin{definition}[Attentions and MLPs]
% \label{defn:attn_mlp} 
% A single-head causal self-attention $\attn(\mH;\mQ,\mK,\mV,\mO)$ parameterized by $\mQ,\mK,\mV \in \real^{{\dqkv\times \din}}$ and $\mO \in \real^{\dout\times\dqkv}$ maps an input matrix $\mH \in \real^{\din\times T}$ to
% \begin{align*}
% &\attn(\mH;\mQ,\mK,\mV,\mO) \\
% =&\mO\mV\layernorm(\mH)\softmax(\mask(\layernorm(\mH)^\transpose\mK^\transpose\mQ\layernorm(\mH))).
% \end{align*}
% Furthermore, a multi-head attention with $M$ heads parameterized by $\{(\mQ_m,\mK_m,\mV_m,\mO_m) \}_{m=1}^M$ is defined as 
% \begin{align*}
%     &\Attn(\mH; \{(\mQ_m,\mK_m,\mV_m,\mO_m) \}_{m\in[M]}) \\ =& \sum_{m=1}^M \attn(\mH;\mQ_m,\mK_m,\mV_m,\mO_m) \in \real^{\dout \times T}.
% \end{align*}
% An MLP layer $\mlp(\mH;\mW_1,\mW_2)$ parameterized by $\mW_1 \in \real^{\dhidden\times \din}$ and $\mW_2 \in \real^{\dout \times \dhidden}$ maps an input matrix $\mH = [\vh_1, \ldots, \vh_T] \in \real^{\din \times T}$ to
% \begin{align*}
%     &\mlp(\mH;\mW_1,\mW_2) = [\vy_1, \ldots, \vy_T], \\ \text{where } &\vy_i = \mW_2\nonlin(\mW_1\layernorm(\vh_i)), \forall i \in [T].
% \end{align*}

% \end{definition}

% In this paper, we assume $\din=\dout=d$ for all attention heads and MLPs to facilitate residual stream unless otherwise specified. Given \Cref{defn:operators,defn:attn_mlp}, we are now able to define a multi-layer transformer.

% \begin{definition}[Multi-layer transformers]
% \label{defn:transformer}
%     An $L$-layer transformer $\transformer(\cdot): \vocab^T \to \Delta(\vocab)$ parameterized by $\mP$, $\mU$, $\{(\mQ_m^{(l)},\mK_m^{(l)},\mV_m^{(l)},\mO_m^{(l)})\}_{m\in[M],l\in[L]}$,  $\{(\mW_1^{(l)},\mW_2^{(l)})\}_{l\in[L]}$ and $\Wreadout \in \real^{V \times d}$ receives a sequence of tokens $s_{1:T}$ as input and predict the next token by outputting a distribution over the vocabulary. The input is first mapped to embeddings $\mH = [\vh_1, \vh_2, \ldots, \vh_T] \in \real^{d\times T}$ by embedding matrices $\mP, \mU$ where 
%     \[
%     \vh_i = \mU^\transpose\ve_{s_i} + \mP^\transpose\ve_{i}, \forall i \in [T].
%     \]
%     For each layer $l \in [L]$, the output of layer $l$, $\mH^{(l)} \in \real^{d\times T}$, is obtained by 
%     \begin{align*}
%         &\mH^{(l)} =  \mH^{(l-1/2)} + \mlp(\mH^{(l-1/2)};\mW_1^{(l)},\mW_2^{(l)}), \\
%         & \mH^{(l-1/2)} = \mH^{(l-1)} + \\ & \quad \Attn(\mH^{(l-1)}; \{(\mQ_m^{(l)},\mK_m^{(l)},\mV_m^{(l)},\mO_m^{(l)}) \}_{m\in[M]}), 
%     \end{align*}
%     where the input $\mH^{(l-1)}$ is the output of the previous layer $l-1$ for $l > 1$ and the input of the first layer $\mH^{(0)} = \mH$. Finally, the output of the transformer is obtained by 
%     \begin{align*}
%         \transformer(s_{1:T}) = \softmax(\Wreadout\vh_T^{(L)})
%     \end{align*}
%     which is a $V$-dimensional vector after softmax representing a distribution over $\vocab$, and $\vh_T^{(L)}$ is the $T$-th column of the output of the last layer, $\mH^{(L)}$.
% \end{definition}



% For each token $v \in \vocab$, there is a corresponding $d_t$-dimensional token embedding vector $\embed(v) \in \mathbb{R}^{d_t}$. Assume the maximum length of the sequence studied in this paper does not exceed $T$. For each position $t \in [T]$, there is a corresponding positional embedding  








\section{Proposed Method}
% We introduce the Covariate-Adjusted Pre-training framework for Epidemic forecasting (Figure~\ref{fig:CAPE}), which accounts for epidemic environmental changes in disease dynamics. 
% A framework overview is shown in Figure~\ref{fig:CAPE}. .

% Enhanced by components that capture temporal dependency and infer latent environment (section~\ref{sec: model}), CAPE aims to learn the intrinsic disease dynamics through pre-training (section~\ref{sec: contrast}).


\subsection{Model Design}
\label{sec: model}

\subsubsection{Causal Analysis for Epidemic Forecasting}
As environments influence both historical infection patterns and future disease spread, we draw inspiration from causal inference~\cite{zhou2023causal, jiao2024causal} and introduce a Structural Causal Model where we treat the environment $Z$ as a confounder that influences both the independent variable (e.g., historical data $X$) and the dependent variable (e.g., future infections $Y$). Furthermore, we adopt a causal decomposition approach~\cite{mao2022causal} that separates $X$ into two components (Figure~\ref{fig: causal_graph}): (1) a \textit{spurious} factor $X_s$ that is environment-dependent, and (2) a \textit{causal} factor $X_c$ that remains environment-independent. Both factors influence the target \( Y \), with \( X_s \) reflecting the impact of environment \( Z \). Since epidemic dynamics are driven by a finite set of critical factors, such as public health policies, we model \( Z \) with the following assumption:
\begin{assumption} \label{assumption} The environment variable $Z$ follows a categorical distribution $p(Z)$ and takes on one of $K$ discrete environmental states, denoted as $z_k$. Each state $z_k$ is associated with a unique latent representation $\mathbf{e}_k \in \mathbb{R}^{h_e}$, capturing the unique features specific to that environment.
\end{assumption}

In constructing a predictive model for input $\mathbf{x}$, we define \(\hat{Y}\) as the predicted time series \(\hat{\mathbf{y}}\) and model the predictive distribution \( p_\Theta(\hat{Y} | X) \) using \( f_\Theta(\mathbf{x}) = h_\psi(g_\theta(\mathbf{x})) \), where \( \Theta = \{\theta, \psi\} \). Training typically involves maximizing the log-likelihood of \( p_\Theta(\hat{Y} | X) \), which in practice translates to minimizing the errors over the pre-training dataset \( \mathcal{D}_\text{pre} \):
\vskip -7mm
\begin{equation}
\Theta^* = \arg\min_{\Theta} - \frac{1}{|\mathcal{D}_\text{pre}|} \sum\nolimits_{(\mathbf{x}, \mathbf{y}) \in \mathcal{D}_\text{pre}} \|\mathbf{y} - f_\Theta(\mathbf{x}))\|^2.
\end{equation}
\vskip -4mm
As the environment $Z$ impacts the distribution of the observed data through $p(X, Y | Z) = p(X | Z)p(Y | X, Z)$, we formulate the following objective:
% let \( p(Z) \) denote the distribution of environments and we have the following objective: 
\begin{equation}
\label{Eq: optim}
\small
\begin{split}
\Theta^* = & \arg\min\nolimits_{\Theta}  \mathbb{E}_{p(Z)} [ \mathbb{E}_{(\mathbf{x}, \mathbf{y}) \sim p(Y, X | Z)} [ \|\mathbf{y} - f_\Theta(\mathbf{x}))\|^2 ]  ].
\end{split}
\end{equation}
% \vskip -4mm
The above equation suggests that the optimal $\Theta^*$ depends on the environment distribution $p(Z)$. If we simply maximize the likelihood $p_\Theta(\hat{Y} |X)$, the confounding effect of $Z$ on $X$ and $Y$ will mislead the model to capture the shortcut predictive relation between the input and the target trajectories, which necessitates explicit modeling of the environment during pre-training. Given that input infection trajectories inherently reflect the influence of the environment, it is crucial to develop mechanisms that disentangle the correlations between infection trajectories and environmental factors.

In this study, we switch to optimize $p_\Theta(\hat{Y} | do(X))$, where the \textit{do}-operation intervenes the variable $X$ and removes the effects from other variables (i.e., $Z$ in our case), thus effectively isolating the disease dynamics from environmental influences. In practice, this operation is usually conducted via covariate adjustment, particularly \textit{backdoor adjustment}~\cite{suncaudits}, which controls for the confounder and uncovers the true causal effects of interest. The theoretical foundation for this is explained through: $p(Y | do(X)) = \int p(Y | X, Z=z) p(Z=z) dz$ (see Appendix~\ref{Appendix: theory}). Under Assumption~\ref{assumption}, this simplifies over different environmental states:
\begin{equation}
\label{Eq: do}
p(Y | do(X)) = \sum\nolimits_{Z} p(Y | X, Z=z) p(Z=z).  
\end{equation}
However, obtaining detailed environmental information, or ${\bf e}_k$, can be challenging due to variability in data availability and quality. To address this,  we resort to a data-driven approach that treats ${\bf e}_k$ as learnable parameters and thus allows us to dynamically infer the environmental distribution directly from the observed data. Specifically, we implement an environment estimator $q_\phi(Z|X)$ that infers the probability of environment states based on historical inputs together with the latent representations of each state. Then, we derive a variational lower bound (see Appendix~\ref{Appendix: theory}):
\begin{equation}
\label{Eq: lb}
\small
\begin{aligned}
&\log{p_\Theta(\hat{Y} | do(X))} \geq \\
& \mathbb{E}_{q_\phi(Z | X)} \left[ \log p_\Theta(\hat{Y} | X, Z) \right] - \operatorname{KL}\left(q_\phi(Z | X) \parallel p(Z)\right),
\end{aligned}
\end{equation}
where the first term maximizes the model's predictive power and the second term regularizes the environment estimator to output a distribution close to the prior distribution $p(Z)$.


\subsubsection{Model Instantiation}
% According to the analysis above, 
To instantiate and train a model that performs the covariate adjustment, we need to model the environment estimator $q_\phi(Z | X)$ and the predictor $p_\Theta(\hat{Y} | X, Z)$.

\textbf{Latent Environment Estimator $q_\phi(Z | X)$.} 
% We characterize \( p(Z|X) \) using a latent environment estimator \( q_\phi(Z|X) \). Given that the influence of each environment may vary across different times, we apply patching~\cite{nie2022time} to control for granularity in the environment estimation process. This approach ensures that the environment estimation avoids overly detailed estimations for specific time points and prevents overly broad approximations that could obscure important temporal fluctuations. Therefore, we split the input $\mathbf{x}$ into a number of $C$ non-overlapping patches $\mathbf{x} = [\mathbf{x}_1, \mathbf{x}_2, ... \mathbf{x}_C]$, where $\mathbf{x}_c \in \mathbb{R}^{\frac{T}{C}}$. Then, a self-attention layer $f_\text{enc}$ is employed to capture the temporal dependency between patches, yielding contextualized representation $\mathbf{h}^{(l)}_c=f_\text{enc}(\mathbf{x}^{(l)}_c)$ for the $i$-th patch at layer $l$. Subsequently, since the environment influences only the spurious component of the contextualized input $\mathbf{h}^{(l)}_c$, we introduce an additional transformation layer $\mathbf{W}_h^{(l)}$ to provide the estimator with the flexibility of highlighting the spurious component of $\mathbf{h}^{(l)}_c$. Finally, we model \( q_\phi(Z | X) \) as a cross-attention layer that models the interaction between each patch and the latent environment representations: 
We model \( p(Z|X) \) using a latent environment estimator \( q_\phi(Z|X) \). Since environmental influences vary over time, we apply patching~\cite{nie2022time} to manage granularity in environment estimation. This prevents overly specific or generalized estimations that could obscure key temporal fluctuations. We divide the input \( \mathbf{x} \) into \( C \) non-overlapping patches, \( \mathbf{x} = [\mathbf{x}_1, \dots, \mathbf{x}_C] \), where \( \mathbf{x}_c \in \mathbb{R}^{T/C} \). Then, a self-attention layer \( f_\text{enc} \) captures temporal dependencies between patches, producing contextualized representations \( \mathbf{h}^{(l)}_c = f_\text{enc}(\mathbf{x}^{(l)}_c) \) for each patch at layer \( l \). Subsequently, since the environment influences only the spurious component of the input, 
% \wei{why we can say $h_c$ captures the spurious component? there seems to be no intuition..}, 
we introduce a transformation $\mathbf{W}_s^{(l)}$ to capture the spurious component of $\mathbf{h}^{(l)}_c$. Finally, we model \( q_\phi(Z | X) \) as a cross-attention layer that captures the relation between each patch and the latent environment representations: 
% Since the environment affects only the spurious component of \( \mathbf{h}^{(l)}_c \), we introduce a transformation layer \( \mathbf{W}_h^{(l)} \) to emphasize this component. Finally, \( q_\phi(Z | X) \) is modeled as a cross-attention layer that captures interactions between each patch and the latent environment representations:
\begin{equation}
\label{eq: env_estimator}
\pi_{k, c}^{(l)} = \operatorname{Softmax} \left( (\mathbf{W}_k^{(l)} \mathbf{e}_k)^\top \cdot (\mathbf{W}_s^{(l)} \mathbf{h}_c^{(l)}) \right),
\end{equation}
where $\pi_{k, c}^{(l)}$ is the output probability of the environment $z_{k}$ for the $c$-th patch, and \( \mathbf{W}_k^{(l)} \) is a transformation layer for  \( \mathbf{e}_k \).
% that maps the environment representation \( \mathbf{e}_k \). 
Such operation not only takes into account the contextualized representation of the current time period, but also considers the latent environment representations, which made it possible to infer the densities of other environment distributions with different latent representations.

\textbf{Epidemic Predictor $p_\Theta(\hat{Y}|X,Z)$.} Unlike previous studies, which do not explicitly model environment states, we incorporate these states directly into the input using their latent representations $\mathbf{e}_k$. Specifically, {we model the predictor $p_\Theta(\hat{Y}|X,Z)$ by employing a weighted sum over the combined representations of each environment and the input using Hadamard product, i.e., $f_\text{enc}(\mathbf{x}_c^{(l)}) \odot \mathbf{e}_k$. Finally, we apply a feed-forward layer to compute the output representations, serving as the input for the next layer.} Integrating these components, the CAPE encoder can be expressed as:
\begin{equation}
\small
\label{Eq: model}
\mathbf{x}^{(l+1)}_{c}
= \sigma \left(
  \mathbf{W}_f^{(l)}
  \sum_{k=1}^{K}
    \pi_{k, c}^{(l)}
    % \cdot
    \left[
      f_{enc}({\mathbf{x}_c^{(l)}})
      \odot 
      \mathbf{e}_k
    \right]
\right),
\end{equation}
where \( \sigma \) represents the activation function and \( \mathbf{W}_f^{(l)} \) denotes the learnable parameters of the feedforward layer. Assuming $L$ layers are stacked, we acquire the final representation $\mathbf{x}^{(L)} = [\mathbf{x}^{(L)}_1, \mathbf{x}^{(L)}_2, \ldots \mathbf{x}^{(L)}_C] = g_\theta({\bf x}) \in \mathbb{R}^{C \cdot d}$ and apply a task-specific head to predict the target variable $\mathbf{y}=h_\psi(\mathbf{x}^{(L)})$, where $h_\psi$ is a linear transformation.

% between the latent environment representations $\mathbf{e}_k$ and the contextualized input \( \mathbf{h}_i^{(l)} \), i.e., $\mathbf{h}_i^{(l)} \odot \mathbf{e}_k$.} 
% \wei{think you missed describing the process of mixing/summation over K environments?}.

% \begin{equation}
% \label{Eq: model}
% \mathbf{X}^{(l+1)} 
% = \sigma \Bigl(
%   \mathbf{W}_f^{(l)}
%   \sum_{k=1}^{K}
%     \Bigl[
%       f_\text{enc}(\mathbf{X}^{(l)})
%       \;\odot\;
%       \underbrace{\bigl(\mathbf{e}_k \mathbf{1}^\top\bigr)}_{\in \mathbb{R}^{c \times d}}
%     \Bigr]
%     (\mathbf{\pi}_{k}^{(l)}\mathbf{1}^\top)
% \Bigr),
% \end{equation}

% \begin{equation}
% \small
% \label{Eq: model}
% \mathbf{X}^{(l+1)} 
% = \sigma \left(
%   \mathbf{W}_f^{(l)}
%   \sum_{k=1}^{K}
%     \underbrace{(\boldsymbol{\pi}_{k}^{(l)})^\top}_{\in \mathbb{R}^{1 \times c}} 
%     \cdot
%     \left[
%       f_{\text{enc}}(\mathbf{X}^{(l)}) 
%       \odot 
%       \underbrace{\left( \mathbf{e}_k \mathbf{1}^\top \right)}_{\in \mathbb{R}^{c \times d}}
%     \right]
% \right),
% \end{equation}



% \begin{equation}
% \small
% \label{Eq: model}
% \mathbf{h}^{(l+1)} = 
% \sigma\left( \operatorname{flatten}\left(
%   \mathbf{W}_f^{(l)}
%   \sum_{k=1}^{K}
%     {(\boldsymbol{\pi}_{k}^{(l)})^\top}
%     \cdot
%     \left[
%       f_{\text{enc}}(\mathbf{h}^{(l)}) 
%       \odot 
%       {\left( \mathbf{e}_k \mathbf{1}^\top \right)}
%     \right]
% \right)\right),
% \end{equation}


% This operation is defined as $\mathbf{x}_i^{(l+1)} = \sigma(\mathbf{W}_f^{(l)} \mathbf{m}_i^{(l)})$, where \( \sigma \) represents the activation function, and \( \mathbf{W}_f^{(l)} \) denotes the learnable parameters of the feedforward layer.



% \textbf{Integrating Environment Labels.} Unlike previous studies, we model the environment labels explicitly and integrate them into the input, which is similar to combining an encoded input from external knowledge~\cite{rombach2022high}. Specifically, we use a hadamard product to integrate the environment labels $\mathbf{z}^{k}$ into the contextualized input $h_i^{(l)}$, which is done in a layer-wise manner. Given $\mathbf{h}_i^{(l)}$ at the $l$ th layer, we model Eq.~\ref{Eq: 5} as follows:
% \begin{equation}
% \label{Eq: ca}
% \mathbf{m}_i^{(l)}  = \sum_{k=1}^{K} (\mathbf{h}_i^{(l)} \odot \mathbf{z}^k) \cdot p(\mathbf{z}^k | \mathbf{h}_i^{(l)}).
% \end{equation} 

% Finally, a feedforward neural network \( \mathbf{x}_i^{(l+1)} = \sigma(\mathbf{W}_f^{(l)} \mathbf{m}_i^{(l)}) \) is applied to acquire the output representations, which serves as the input for the next block. Integrating these components, the final model can be expressed as:


% \vspace{-3mm}
% \begin{equation}
% \mathbf{X}^{(l+1)} 
% = \sigma \Bigl(
%   \mathbf{W}_f^{(l)}
%   \sum_{k=1}^{K}
%     \Bigl[
%       \mathbf{H}^{(l)} 
%       \;\odot\;
%       \underbrace{\bigl(\mathbf{z}^k \mathbf{1}^\top\bigr)}_{\in \mathbb{R}^{d \times T}}
%     \Bigr]
%     \cdot
%     \pi^{k(l)}
% \Bigr),
% \end{equation}
% \vspace{-3mm}

% where \( \sigma \) represents the activation function. At the end of the model, we acquire the final representation $\mathbf{X^{(L)}} = g_\theta(\mathbf{X})$ and a task-specific head is applied to predict the target variable $\mathbf{y}=\mathbf{h}_\psi(\mathbf{X}^{(L)})$, where $h_\psi$ is a linear transformation.

% \wei{do not really see the connection between this and Eq.6. Feel free to use more sentences/space to make this clear. We can always shorten the content later. The notation $E_{IP}$ also looks heavy; something like $\hat{z}$ or $\bar{z}$ could be better} 
% \begin{equation} 
% \label{Eq: env_estimator}
% \mathbf{E}_{IP}^{(l)} = \sum_{k=1}^{K} \mathbf{e}_k \cdot q_\phi(Z=z_k | X_s). 
% \end{equation}
% where $\mathbf{E}_{IP}^{(l)}$ is the latent representation of the interpolation from the defined environment states $\mathbf{z}_k$. Thus, the weight $q_\phi(Z | X_s)$ can be interpreted as the contribution of each state involved in the given period $\mathbf{h}_i^{(l)}$. 

% While self-supervised learning tasks \( \mathcal{T}_{\text{pre}} \) transform the original input into a new input-label pair, prior studies have often neglected the confounding effects of environmental factors on these inputs and labels. This oversight hinders performance and limits the learning capacity of these tasks. To overcome this limitation, CAPE integrates environment estimation seamlessly into the self-supervised learning framework. % , enhancing its effectiveness and robustness.


% To capture a broad range of epidemic time series dynamics from the pre-training dataset, CAPE leverages self-supervised learning tasks to identify universal patterns inherent in the data. While supervised learning tasks \( \mathcal{T}_{\text{pre}} \) transform the original input \( \mathbf{X} \) into a pair of new input and labels, previous studies have overlooked the confounding effects of the environment on these inputs and labels. This oversight hinders performance and limits the learning capacity of these tasks. To address this, CAPE tightly incorporates environment estimation into self-supervised learning. 
% , improving the accuracy of both the self-supervised learning objectives and environment estimation.
% \wei{seems unclear about how the environment estimator helps with the reconstruction as it is not involved in the following loss?}
% During pre-training, the masked time series modeling task samples a batch of $B$ times series $\mathbf{X} \in \mathbb{R}^{B \times T}$ and transforms them into a pair of masked inputs and labels: $(\tilde{\mathbf{X}}, \mathbf{X})$. In this case, the original time series serves as the label $y$.
% Then, the reconstructed time series is acquired via $ \hat{\mathbf{X}} = h_{\phi}(g_\theta(\tilde{\mathbf{X})})$. In this study, MSE is used as the reconstruction loss to ensure that reconstructed data closely matches the original: 
% % $\mathcal{L}_{\text{recon}} = \frac{1}{B}\sum_{i=1}^{B} \text{MSE}_{\hat{Z}}(\hat{\mathbf{X}}_i, \mathbf{X}_i)$
% $\mathcal{L}_{\text{recon}} = \frac{1}{B} \sum_{i \in B}\text{MSE}_{\hat{Z}_i}(\hat{\mathbf{X}_i}, \mathbf{X}_i)$
% where \( \hat{\mathbf{X}_i} \) is the $i$-th sample's reconstruction under the inferred environment $\hat{Z}_i$. 
% \wei{we need to highlight a bit about our novelty here}
% \wei{to highlight the novelty, we typically need to point out the limitations of existing methods, like ``different from existing xxxx..''}
% both $\tilde{\mathbf{X}}$ and \( \mathbf{V} \)
% \wei{why we need to introduce $V$; seems you did not use the notation in the later content}.
% To capture a diverse range of epidemic time series dynamics, CAPE employs self-supervised learning tasks to uncover universal patterns inherent in the pre-training dataset. However, previous approaches often overlooked the confounding impact of environmental factors on input-label pairs defined in \( \mathcal{T}_{\text{pre}} \), which can impair task performance and learning potential. To address this, CAPE seamlessly incorporates environment estimation into its self-supervised learning framework through the following tasks.

% While $ q_\phi(Z|X) $ infers the environment across all patches for a given time series \( \mathbf{X} \) during random masking, the inferred environment for the same patch may vary when the context changes to \( \mathbf{X}' \). This variability arises because changes in context alter the latent representations $ \mathbf{h}^{(l)}_i $, causing the environment estimator to produce inconsistent results. However, such inconsistency is problematic as the environment should be context-invariant. To resolve this, we propose a hierarchical environment contrasting approach to align the estimated environments across different contexts. First, we define the environments to be aligned as \textit{aggregated latent environment representation} $\hat{\mathbf{e}}^{(l)}=\sum_{k=1}^{K}\mathbf{e}_k \pi_{k}^{(l)}$. Next, to ensure robust and contextually aligned representations of the unique aggregated environment within each patch $\mathbf{h}^{(l)}_i$, we designed a hierarchical environment contrastive loss for pre-training (Figure~\ref{fig:CAPE}(b)), in terms of both instance-level and temporal-level. \textit{Instance-wise contrasting} treats $\hat{\mathbf{e}}^{(l)}$ from different time series (a number of B samples) as negative pairs, encouraging dissimilar representations~\cite{yue2022ts2vec} and promoting diversity (second term of Eq.~\ref{Eq: CL}). \textit{Temporal contrasting} uses augmented samples with overlapping regions ($\Omega$) to maintain consistent environment estimation across sequences despite varying contexts~\cite{yue2022ts2vec} (third term of Eq.~\ref{Eq: CL}). In our framework, contrastive loss is computed patch-wise for both final time series embeddings \( \mathbf{X}^{L} \) and batch-wise aggregated environments \( \hat{\mathbf{E}}_{j,i}^{(l)}=\hat{\mathbf{e}}^{(l)}\), where $j$ denotes the sample index and $i$ denotes the patch index. An example of the contrastive loss for the sample $j$ at patch $i$ is shown below:
% \begin{equation}
% \label{Eq: CL}
% \small
% \begin{aligned}
% &\mathcal{L}_{\text{CL(j,i)}} = - \hat{\mathbf{E}}_{(j,i)} \cdot \hat{\mathbf{E}'}_{(j,i)}  \\
% &+ \log \left( \sum_{b\in B} \exp \left( \hat{\mathbf{E}}_{(j,i)} \cdot \hat{\mathbf{E}'}_{(b,i)} \right) + \mathbb{I}_{j \neq b} \exp \left( \hat{\mathbf{E}}_{(j,i)} \cdot \hat{\mathbf{E}}_{(b,i)} \right) \right) \\
% &+ \log \left( \sum_{t \in \Omega} \exp \left( \hat{\mathbf{E}}_{(j,i)} \cdot \hat{\mathbf{E}'}_{(j,t)} \right) + \mathbb{I}_{j \neq t} \exp \left( \hat{\mathbf{E}}_{(j,i)} \cdot \hat{\mathbf{E}}_{(j,t)} \right) \right).
% \end{aligned}
% \end{equation}
% the combination of contextualized representations and aggregated latent environments.



\subsection{Pre-training Objectives for Epidemic Forecasting}
\label{sec: contrast}
CAPE captures diverse epidemic time series dynamics through self-supervised learning tasks that identify universal patterns in the pre-training dataset. While previous studies neglected the confounding effects of environmental factors on input-label pairs in \( \mathcal{T}_{\text{pre}} \), CAPE seamlessly integrates environment estimation into the self-supervised framework.

{\noindent\textbf{Random Masking with Environment Estimation.} 
% To capture characteristics from large unlabeled epidemic time series data, we use a masked time series modeling task~\cite{kamarthi2023pems, goswami2024moment} (Figure~\ref{fig:CAPE}(c)), which randomly masks input patches by setting them to zero with a 30\% probability. Furthermore, as illustrated in Figure~\ref{fig: causal_graph}, the generation of \( X \) is inherently dependent on the corresponding environment. Consequently, accurately reconstructing a patch requires not only capturing temporal dependency from neighboring patches but also the dependency from its associated environment. Unlike previous research, which overlooked the role of the environment in the masked time series modeling task, we adopt the environment estimator \( q_\phi(Z | X) \) to infer the environment to help the reconstructions, which conversely help train the estimator}. During pre-training, the masked time series modeling task transforms $\mathbf{x}$ into a pair of masked input and label: $(\tilde{\mathbf{x}}, \mathbf{x})$. In this case, the original time series serves as the label $y$. Then, the reconstructed time series is acquired via $ \hat{\mathbf{x}} = h_{\psi}(g_\theta(\tilde{\mathbf{x}}))$. In this study, MSE is used as the reconstruction loss to ensure that reconstructed data closely matches the original:  $\mathcal{L}_{\text{recon}} = \sum_{\mathcal{D}'_{s} \in \mathcal{D}_{pre}} \sum_{\mathbf{x} \in \mathcal{D}'_{s}}  \text{MSE}(\hat{\mathbf{x}}, \mathbf{x})$,  where \( \hat{\mathbf{x}} \) is the reconstruction of the input $\mathbf{x}$.
% To capture characteristics from large unlabeled epidemic time series data, we employ a masked time series modeling task~\cite{kamarthi2023pems, goswami2024moment} (Figure~\ref{fig:CAPE}(c)), which masks input patches with a 30\% probability. As illustrated in Figure~\ref{fig: causal_graph}, the generation of \( X \) depends on the environment \( Z \). Therefore, accurately reconstructing a patch requires capturing both temporal dependencies and environmental influences. Unlike previous studies that overlooked the environment's role in masked modeling, we use an environment estimator \( q_\phi(Z | X) \) to infer \( Z \), aiding reconstruction and simultaneously training the estimator. During pre-training, the task transforms the input \( \mathbf{x} \) into masked input and label pairs \((\tilde{\mathbf{x}}, \mathbf{x})\), where the original time series serves as the label \( y \). The reconstructed time series is obtained via \( \hat{\mathbf{x}} = h_{\psi}(g_\theta(\tilde{\mathbf{x}})) \). We use Mean Squared Error (MSE) as the reconstruction loss to ensure that \( \hat{\mathbf{x}} \) closely matches \( \mathbf{x} \): $\mathcal{L}_{\text{recon}} = \sum_{\mathcal{D}'_{s} \in \mathcal{D}_{\text{pre}}} \sum_{\mathbf{x} \in \mathcal{D}'_{s}} \text{MSE}(\hat{\mathbf{x}}, \mathbf{x})$.
To capture features from large unlabeled epidemic time series data, we employ a masked time series modeling task~\cite{kamarthi2023pems, goswami2024moment} (Figure~\ref{fig:CAPE}(c)) that masks 30\% of input patches. As depicted in Figure~\ref{fig: causal_graph}, the generation of $X$ depends on the environment \( Z \), indicating that accurate patch reconstruction requires capturing both temporal and environmental dependencies. Unlike prior studies that overlook the environment's role, we utilize an environment estimator \( q_\phi(Z | X) \) to infer \( Z \), aiding both reconstruction and estimator training. During pre-training, input \( \mathbf{x} \) is transformed into masked input and label pairs \((\tilde{\mathbf{x}}, \mathbf{x})\), with the original time series serving as label \( y \). The reconstruction \( \hat{\mathbf{x}} = h_{\psi}(g_\theta(\tilde{\mathbf{x}})) \) is optimized using Mean Squared Error (MSE): 
% $\mathcal{L}_{\text{recon}}({\mathcal{D}_\text{pre}}) = \sum_{\mathcal{D}'_{s} \in \mathcal{D}_{\text{pre}}} \sum_{\mathbf{x} \in \mathcal{D}'_{s}} \text{MSE}(\hat{\mathbf{x}}, \mathbf{x})$.
$\mathcal{L}_{\text{recon}}(\mathbf{x}, \hat{\mathbf{x}}) = \text{MSE}(\hat{\mathbf{x}}, \mathbf{x})$.



\noindent\textbf{Hierarchical Environment Contrasting.}
% Although Eq.~\eqref{eq: env_estimator} estimates environments across all patches, the inferred environments for the same patch in two consecutive samples, can vary due to contextual changes from neighboring patches. These changes alter the latent representations, leading to inconsistent environment estimates. 
% Although Eq.~\eqref{eq: env_estimator} applies to all patches, 
Two consecutive time series samples, ${\bf x}$ and ${\bf x}'$, can include overlapping regions when divided into multiple patches. These overlapping patches, although identical, can exhibit contextual variations influenced by their different adjacent patches. As indicated by Eq.~\eqref{eq: env_estimator}, such variations can alter the latent patch-wise representations, leading to inconsistencies in the environmental estimates for the same patch across the samples.
% In Eq.~\eqref{eq: env_estimator}, the inferred environments for the same patch in consecutive samples, ${\bf x}$ and ${\bf x}'$, may differ due to overlapping regions. Since ${\bf x}$ and ${\bf x}'$ are sequential samples divided into multiple patches, they share overlapping patches. However, slight variations in the context of these overlapping areas, influenced by adjacent patches, can modify the latent representations. This leads to variations in the environmental estimates for the same patch across the two samples.
To ensure that each patch's environment remains \textit{context-invariant}, we propose a hierarchical environment contrasting scheme inspired by \citet{yue2022ts2vec}. We define an \textit{aggregated latent environment representation} \( \hat{\mathbf{e}}^{(l)}_c = \sum_{k=1}^{K} \mathbf{e}_k \pi_{k, c}^{(l)} \) to represent the weighted environment states for the \( c \)-th patch. For contrastive loss computation, we use the combined representation \( \hat{\mathbf{E}}_{j,c}^{(l)} = \sigma(\mathbf{W}_f^{(l)} (\hat{\mathbf{e}}^{(l)}_c \odot \mathbf{h}^{(l)}_c)) \) for $c$-th patch of sample $j$. Additionally, \( \hat{\mathbf{E}'}_{j,c}^{(l)} \) denotes the representation in the context of \( \mathbf{x}' \). Finally, we compute a patch-wise contrastive loss:
% While Eq.~\eqref{eq: env_estimator} is designed to estimate the environments across all patches, the inferred environments for the same patch appearing in consecutive samples can differ due to context changes, specifically from neighboring patches. These changes in context affect the latent representations and cause the environment estimator to produce inconsistent results for the same patch. Since the environment for the same patch should be \textit{context-invariant}, we draw inspiration from \citet{yue2022ts2vec} and propose a hierarchical environment contrasting scheme to ensure consistency across different contexts. In our framework, we define a \textit{aggregated latent environment representation} $\hat{\mathbf{e}}^{(l)}_c=\sum_{k=1}^{K}\mathbf{e}_k \pi_{k, c}^{(l)}$ to stand for the weighted environment states for the $c$-th patch. To compute contrast loss, we use a combined representation for the $c$-th patch of $j$ sample: $\hat{\mathbf{E}}_{j,c}^{(l)}=\hat{\mathbf{e}}^{(l)}_c \odot \mathbf{h}^{(l)}_c$, where $\hat{\mathbf{e}}^{(l)}_c$ is the aggregated latent environment representation and $\mathbf{h}^{(l)}_c$ is the contextualized input for patch $c$. While $\hat{\mathbf{E}}_{j,c}^{(l)}$ is computed in the context of $\mathbf{x}$, we use $\hat{\mathbf{E}'}_{j,c}^{(l)}$ to denote the representation computed in the context of $\mathbf{x}'$. Finally, a patch-wise contrastive loss is computed:
% While Eq.~\eqref{eq: env_estimator} is designed to estimate the environments across all patches, the inferred environments for the same patch appearing in consecutive samples can differ due to context changes, specifically from neighboring patches. These changes in context affect the latent representations and cause the environment estimator to produce inconsistent results for the same patch. Since the environment for the same patch should be \textit{context-invariant}, we draw inspiration from \citet{yue2022ts2vec} and propose a hierarchical environment contrasting scheme to ensure consistency across different contexts. In our framework, we define a \textit{aggregated latent environment representation} $\hat{\mathbf{e}}^{(l)}_c=\sum_{k=1}^{K}\mathbf{e}_k \pi_{k, c}^{(l)}$ \wei{to stand for xxx for $c$-th patch }. To compute contrast loss, we use a combined representation for the $c$-th patch of $j$ sample: $\hat{\mathbf{E}}_{j,c}^{(l)}=\hat{\mathbf{e}}^{(l)}_c \odot \mathbf{h}^{(l)}_c$, where $\hat{\mathbf{e}}^{(l)}_c$ is the aggregated latent environment representation and $\mathbf{h}^{(l)}_c$ is the contextualized input for patch $c$. While $\hat{\mathbf{E}}_{j,c}^{(l)}$ is computed in the context of $\mathbf{x}$, we use $\hat{\mathbf{E}'}_{j,c}^{(l)}$ to denote the representation computed in the context of $\mathbf{x}'$. Finally, a patch-wise contrastive loss is computed:
{\small
\begin{equation}
\label{Eq: CL}
\begin{aligned}
&\mathcal{L}_{\text{CL}}(j,c) = - \hat{\mathbf{E}}_{(j,c)} \cdot \hat{\mathbf{E}'}_{(j,c)}  \\
&+ \log \left( \sum_{b\in B} \exp \left( \hat{\mathbf{E}}_{(j,c)} \cdot \hat{\mathbf{E}'}_{(b,c)} \right) + \mathbb{I}_{j \neq b} \exp \left( \hat{\mathbf{E}}_{(j,c)} \cdot \hat{\mathbf{E}}_{(b,c)} \right) \right) \\
&+ \log \left( \sum_{t \in \Omega} \exp \left( \hat{\mathbf{E}}_{(j,c)} \cdot \hat{\mathbf{E}'}_{(j,t)} \right) + \mathbb{I}_{c \neq t} \exp \left( \hat{\mathbf{E}}_{(j,c)} \cdot \hat{\mathbf{E}}_{(j,t)} \right) \right). \nonumber
\end{aligned} 
\end{equation}}where $B$ is the batch size, $\Omega$ denotes the overlapping patches, and $\mathbb{I}$ is the indicator function.
The above equation contains three key terms: (1) The first term encourages the representations of the same patch from two different contexts to be similar, which preserves the context-invariant nature of environments. (2) The second term (\textit{Instance-wise Contrasting}) treats $\hat{\mathbf{e}}^{(l)}_c$ from different samples in the batch as negative pairs, which promotes dissimilar representations, and enhances diversity among instances. (3) The third term (\textit{Temporal Contrasting}) treats the representations of different patches from overlapping regions ($\Omega$) as negative pairs, which encourages differences across temporal contexts.
% First, we define the environments to be aligned as \textit{aggregated latent environment representation} $\hat{\mathbf{e}}^{(l)}_c=\sum_{k=1}^{K}\mathbf{e}_k \pi_{k, c}^{(l)}$. 
% Next, to ensure robust and contextually aligned representations of both the contextualized representation $\mathbf{h}^{(l)}_c$ as well as the aggregated environment within each patch, we employ a hierarchical environment contrastive loss for pre-training (Figure~\ref{fig:CAPE}(b)), in terms of both instance-level and temporal-level. 
% \zw{While the first term aims to encourage the representation of the same patch from two different contexts to be similar,  \textit{Temporal contrasting} (the third term of Eq.~\eqref{Eq: CL}) treats the representations of different patches from the overlapping regions ($\Omega$) as negative pairs and encourages dissimilarity between them. Furthermore, \textit{Instance-wise contrasting} (the second term of Eq.~\eqref{Eq: CL}) treats $\hat{\mathbf{e}}^{(l)}_c$ from different time series within the same batch as negative pairs, which aims to encourage dissimilar representations and promoting diversity. }
% Aiming to preserve the context-invariant nature of environments, the first term encourages the representation of the same patch from two different contexts to be similar. As the contrast of the first term, we aim to push the representations of other patches far away, which results in two levels of contrast.
% \textit{(1)Instance-wise Contrasting:} The second term treats $\hat{\mathbf{e}}^{(l)}_c$ from different time series within the same batch as negative pairs. This promotes dissimilar representations, thereby enhancing diversity among instances.
% \textit{(2)Temporal Contrasting:} The third term of Equation~\eqref{Eq: CL} treats the representations of different patches from overlapping regions ($\Omega$) as negative pairs. This encourages dissimilarity between them, specifically enforcing differences across temporal contexts.
% \wei{which contains three key terms: (1) the first term xxx similarity promoting? xxx, (2) the second term instance contrasting  (3) the third term xxx}
% The first encourages representations of the same patch across different contexts to be similar, preserving the context-invariant nature of environments; The second(\textit{Instance Contrasting}) treats $\hat{\mathbf{e}}^{(l)}_c$ from different batch samples as negative pairs, promoting dissimilarity and enhancing instance diversity; The third(\textit{Temporal Contrasting}) treats representations of different patches from overlapping regions ($\Omega$) as negative pairs, encouraging differences across temporal contexts.
% \wei{the way you are using the indcator function is not correct. do you mean $\mathbb{I}_{c \neq t}(\cdot)$?} \wei{you mentioned $\Omega$ is the overlapping region later in this paragraph?}



% To preserve the context-invariant nature of environments, the first term encourages the representations of the same patch from two different contexts to be similar. In contrast, we aim to push the representations of other patches apart, resulting in two distinct levels of contrast.
% \textit{(1)Instance-wise Contrasting:} The second term of Equation~\eqref{Eq: CL} treats $\hat{\mathbf{e}}^{(l)}_c$ from different samples in the batch as negative pairs, which promotes dissimilar representations, and enhances diversity among instances.
% \textit{(2)Temporal Contrasting:} The third term of Equation~\eqref{Eq: CL} treats the representations of different patches from overlapping regions ($\Omega$) as negative pairs, which encourages dissimilarity between them, specifically enforcing differences across temporal contexts.



% we use augmented samples with overlapping regions ($\Omega$) to maintain consistent environment estimation across sequences despite varying contexts, i.e., the third term of Eq.~\eqref{Eq: CL}. 


\textbf{Pre-Training Loss.} Given a batch of $B$ samples $\mathbf{X} \in \mathbb{R}^{B \times T}$, we combine the reconstruction loss and the contrastive loss, yielding the final loss function for pre-training: 
% \wei{$\hat{X}$ undefined}
% \wei{in this pre-training loss can we utilize the notation of $\mathcal{D}_\text{pre}$?}
% \begin{equation}
% \begin{aligned}
% \mathcal{L}_{final} &= \mathcal{L}_{recon}(\mathbf{X}, \hat{\mathbf{X}})  + \alpha  \mathcal{L}_{\text{CL}}(\hat{\mathbf{E}}^{(L)}) 
% % &+ \beta [ 1/L \sum_l \mathcal{L}_{\text{CL}}(\hat{\mathbf{E}}^{(l)}) ],
% \end{aligned}
% \end{equation}
% \mathcal{D}_\text{pre}
% \begin{equation}
% \mathcal{L}_{\text{final}}(\mathbf{X} \sim \mathcal{D}_\text{pre}) = \mathcal{L}_{\text{recon}}(\mathbf{X}, \hat{\mathbf{X}}) + \alpha \, \mathcal{L}_{\text{CL}}(\hat{\mathbf{E}}^{(L)}),
% \end{equation}
{\small
\begin{equation}
\mathcal{L}_{\text{final}} = \sum\nolimits_{\mathbf{x} \in \mathbf{X}}\mathcal{L}_{\text{recon}}(\mathbf{x}, \hat{\mathbf{x}}) + \alpha \, \mathcal{L}_{\text{CL}}(\hat{\mathbf{E}}^{(L)}, \hat{\mathbf{E}}'^{(L)}), \; \mathbf{X} \sim \mathcal{D}_\text{pre}  \nonumber
\end{equation}}where $L$ is the number of layers, and $\alpha$ is the hyperparameter used to balance the contrastive loss and the reconstruction loss. Further analysis can be found in Appendix~\ref{Appendix: hp}.


\subsection{Optimization of the CAPE Framework}
% \subsubsection{Optimization for Pre-Training}
\label{sec: EM}
% \wei{can we replace this work with time series related works or works from general domain? this one looks too specific on graphs}
% \wei{what do we mean by "the specific meaning of the distribution"}
%\wei{check if my edits are correct} To maximize the variational lower bound, previous research~\cite{suncaudits}  often assumes a prior distribution for $p(Z)$ and overlook the modeling of the distribution of environmental states. In this study, we propose to use the \textit{Expectation-Maximization} (EM) algorithm to iteratively learn latent environment representations and {maximize the variational lower bound}. 

To effectively maximize the variational lower bound in Eq.~\eqref{Eq: lb}, we employ the \textit{Expectation-Maximization} (EM) algorithm to iteratively update the latent environments and epidemic predictor. The pseudo algorithm for the optimization procedure is provided in Appendix~\ref{Append_B}. 

% Furthermore, we propose the following two theorems for the optimization objectives. \wei{$\mathbf{E}$ is not defined before}

\textbf{E-Step: Estimating Latent Environments.} In the E-step, we aim to identify the environment states \( Z \) and the corresponding distribution \( p(Z) \) that result in the target distribution $p(Y)$. This involves maximizing the expected likelihood of $p(Y | Z)$ given $p(Z)$. We freeze the epidemic predictor $p_\Theta(\hat{Y}|X,Z)$ and the environment estimator $q_\phi(Z|X)$, treating them as oracles, which means $p_\Theta(\hat{Y}|X,Z) = p(Y|X,Z)$ and $q_\phi(Z|X)=q(Z|X)$. While actively updating the environment representations $\mathbf{E}=[\mathbf{e}_1, \mathbf{e}_2, ... \mathbf{e}_k]$, the optimization of the environment states $Z$ is learned through maximizing $\mathbb{E}_{p(Z)} [p(Y|Z)] = \mathbb{E}_{p(X)} [\mathbb{E}_{q_\phi(Z|X)} p_\Theta(Y|X, Z)]$, which is equivalent to minimizing the expected reconstruction loss: 
\begin{equation}
\mathbf{E}^{t+1} \leftarrow \arg\min\nolimits_{\mathbf{E}} \left[  \mathbb{E}_{\mathbf{x} \sim p(X)} [\mathcal{L}_{\text{recon}}(\mathbf{x}, \hat{\mathbf{x}})] \right].
\end{equation}
We use subscript \( t \) to denote the pre-update distribution  and derive the updated distribution $p^{t+1}(Z)$ as $q^{t+1}_{\phi_t}(Z)$, along with the updated environment representations \( \mathbf{E}^{t+1} \).

% \begin{theorem}
% \label{theorem: e}
% The optimization of the environment states $Z$ is learned through maximizing $\mathbb{E}_{p(Z)} [p(Y|Z)] = \mathbb{E}_{p(X)} [\mathbb{E}_{q_\phi(Z|X=\mathbf{x})} p_\Theta(Y|X=\mathbf{x}, Z=z)] $, which is equivalent to minimizing the expected reconstruction loss: 
% % \wei{do not really understand what this $\leftrightarrow$ means}
% \begin{equation}
% \mathbf{E}^{t+1} = \arg\min\nolimits_{\mathbf{E}} \left[  \mathbb{E}_{\mathbf{x} \sim p(X)} [\mathcal{L}_{\text{recon}}(\mathbf{x}, \hat{\mathbf{x}})] \right].
% \end{equation}
% \end{theorem}
% A detailed proof can be found in Appendix~\ref{Appendix: theory}. {Theorem~\ref{theorem: e} aligns the observed outcomes $Y$ with the corresponding environment states $Z$ in a data-driven way and bridges the gap between the objective with a specific loss function.} 
% We represent the pre-update distribution or parameter with a script \( t \) and obtain the updated distribution \( p^{t+1}(Z) = q^{t+1}_{\phi_t}(Z) \), along with the associated environment representations \( \mathbf{E}^{t+1} \).
% \wei{we need to explain the meaning of t}

\textbf{M-Step: Optimizing Epidemic Predictor.} In the M-step, we aim to optimize the epidemic predictor by maximizing its predictive power and regularizing the environment distribution. During this step, the environment representations $\mathbf{E}^{t+1}$ are held fixed. We have the following theorem:
\begin{theorem}
\label{theorem: m}
Assuming $q^{t+1}_{\phi_{t}}(Z)=p^{t+1}(Z)$ and an L2 norm is applied on $\phi$, the variational lower bound in Eq.~\eqref{Eq: lb} can be approximated as follows:
\begin{equation}
\label{Eq: alb}
\mathbb{E}_{p(X)} \left[ \mathbb{E}_{q^{t+1}_{\phi_{t}}(Z | X)}  \left[ \log p^{t+1}_{\Theta_{t+1}}(\hat{Y} | X, Z) \right]\right] - C,
\end{equation}
which is equivalent to minimizing the expected reconstruction loss $\mathbb{E}_{\mathbf{x}\sim p(X)}[\mathcal{L}_{recon}(\mathbf{x}, \hat{\mathbf{x}})]$ .
% \wei{$\mathbb{E}_{\mathbf{X}\sim p(X)}[\mathcal{L}_{recon}(\mathbf{X}, \hat{\mathbf{X}})]$ ?}.
% \wei{we can remove the constant here by saying "maximizing the xxx can be approximated as minimizing xxx"}
\end{theorem}
% \wei{we provide the proof in xxx.} \wei{transition from the above theorem to the following paragraph..}
The detailed proof can be found in Appendix~\ref{Appendix: theory}. Theorem~\ref{theorem: m} indicates that the optimization of the model's predictive ability can be approximated by Eq.~\eqref{Eq: alb}, which corresponds to the expectation of $\mathcal{L}_{recon}$. To further enhance robustness, the contrastive loss is combined to regularize the environment estimator. Therefore, the overall optimization objective becomes minimizing the final pre-training loss:
\begin{equation}
\Theta_{t+1} \leftarrow \arg\min\nolimits_{\Theta} \left[ \mathcal{L}_{\text{final}}(\mathbf{X}, \hat{\mathbf{X}}, \mathbf{E}^{t+1}) \right].
\end{equation}


% In conclusion, our approach optimizes both the environment, encompassing its states and distribution, and the model’s predictive capabilities using an \textit{EM} scheme.  The detailed pseudo-code for the optimization procedure is provided in Appendix~\ref{Append_B}. 
% while additional theoretical analysis is presented in Appendix~\ref{Appendix: theory}.




% \textbf{Expectation Step (E-Step):} 
% In the E-step, we aim to identify the set of environment states \( Z \) and the corresponding distribution \( p(Z) \) that is most relevant to the prediction target $Y$. This means maximizing the $p(\hat{Y})$, which is equal to maximizing the expectation of the predictor's output over the sample distribution $p(X)$:
% \begin{equation}
% \small
% \begin{aligned}
%     p(\hat{Y}) = \mathbb{E}_{p(X)}[ \mathbb{E}_{q_{\phi(Z|X_s)}} [p_\Theta(\hat{Y}|X,Z)] ]
% \end{aligned}
% \end{equation}
% In the case of pre-training, since $q_\phi(Z|X)$ is frozen, there is no need to apply contrastive learning and the optimization objective becomes minimizing the reconstruction loss: 
% % \wei{why here we optimize $L_{recon}$ but eq. 12 optimizes $L_{final}$}
% \begin{equation}
% Z^{t+1} \leftrightarrow \mathbf{E}^{t+1} = \underset{\mathbf{E}}{\arg\min} \left[ \mathcal{L}_{\text{recon}}(\mathbf{X}, \hat{\mathbf{X}}) \right].
% \end{equation}
% After E-step, we acquire a new distribution $p(Z^{t+1})=q^t_\phi(Z^{t+1})$ with the corresponding environment representations $\mathbf{E}^{t+1}$.

% \textbf{Maximization Step (M-Step):} During the M-step, we update the epidemic predictor $p^t_\Theta$ and environment estimator $q^t_\phi$ to maximize the variational lower bound. By combining minibatch training and applying a L2 norm on the estimator $q_\phi$, we are able to rewrite the Eq.~\eqref{Eq: lb} and produce an \textit{approximated variational lower bound} (see Appendix~\ref{Appendix: theory}) as:
% \begin{equation}
% \small
% \begin{aligned}
% \mathbb{E}_{p(X)} \left[ \mathbb{E}_{q^{t}_\phi(Z^{t+1} | X_s)}  \left[ \log p^{t+1}_\Theta(\hat{Y} | X, Z^{t+1}) \right]\right] - I_{\phi^{t}}(Z; X),
% \end{aligned}
% \end{equation}
% % \wei{why Eq.4 can be written as Eq. 11??}
% where \( I_{\phi^{t}}(Z; X) \) represents the mutual information between \( X \) and \( Z \), treated as a constant under the distribution defined by \( q_{\phi^{t}} \). As a result, we focus solely on optimizing the first term of the equation, which corresponds to minimizing the reconstruction loss. To further enhance robustness, the contrastive loss is added to regularize the environment estimator. The overall optimization objective simplifies to minimizing the final pre-training loss:
% \begin{equation}
% \Theta^{t+1} = \underset{\Theta}{\arg\min} \left[ \mathcal{L}_{\text{final}}(\mathbf{X}, \hat{\mathbf{X}}, \mathbf{E}^{t+1}) \right].
% \end{equation}
% \wei{do we have to mention this work (Dong) here? I feel not many people will think of this work because it is not about time series and it is not super famous; mentioning it here introduces additional questions/confusions} 
% Although our strategy for learning explicit representations resembles a soft version of neural discrete representation learning~\cite{dong2023peco}, we use these representations for covariate adjustment instead of input reconstruction. 
% In conclusion, our approach optimizes both the environment, encompassing its states and distribution, and the model’s predictive capabilities using an \textit{EM} scheme. The detailed pseudo-code for the optimization procedure is provided in Appendix~\ref{Append_B}, while additional theoretical analysis is presented in Appendix~\ref{Appendix: theory}.

% \wei{it is nice that we theorem 7.1 but think we did not connect the theorem well with our proposed framework. Can you put the theorom in the main content as well?}


% Simply optimizing the likelihood $p_\theta(\mathbf{y}|\mathbf{X})$ will mislead the time series model to capture the shortcut predictive relation between the history input $\mathbf{X}$ and the future predictions $\mathbf{y}$~\cite{wu2024graph}, which is why the environment should be considered during optimization:


% \begin{equation}
% \small
% \theta^* = \arg\min_\theta \mathbb{E}_{\mathbf{e} \sim p(E),\ (\mathbf{X}, \mathbf{y}) \sim p(\mathcal{Y}, \mathcal{X} | E=\mathbf{e})} \left[ \left\| \mathbf{y} - h_\psi(g_\theta(\mathbf{X})) \right\|^2 \right]. 
% \end{equation}

% However, we may not be able to directly acquire the environment representations without external materials and the corresponding encoder. To address this, we treat these environments as hidden variables and optimize them in a data-driven way. Unlike previous methods that approximate the latent probability distribution of environments via variational lower bounds~\cite{wu2024graph}, our approach uses the Expectation-Maximization (EM) algorithm to obtain maximum a posteriori (MAP) estimates of environment representations:

% \textbf{Expectation Step (E-Step):} We freeze the transformer encoder and environment estimator, then optimize only the learnable environment representations $\mathbf{Z}$. Setting hyperparameters $\alpha, \beta = 0$, we solve:
% \begin{equation}
% \mathbf{Z}^{t+1} = \underset{\mathbf{Z}}{\arg\min} \left[ \mathcal{L}_{\text{recon}}(\mathbf{X}, \mathbf{V}) \right].
% \end{equation}
% \vspace{-3mm}

% \textbf{Maximization Step (M-Step):} We fix the updated environment variables $\mathbf{Z}^{t+1}$ and optimize the encoder and environment estimator by minimizing $\mathcal{L}_{\text{final}}$. Thus, the representation $g_\theta$ and task-specific head $h_\psi$ are updated as:
% \begin{equation}
% \theta^{t+1}, \psi^{t+1} = \underset{\theta, \psi}{\arg\min} \left[ \mathcal{L}_{\text{final}}(\mathbf{X}, \mathbf{V}, \mathbf{Z}^{t+1}) \right].
% \end{equation}
% \vspace{-3mm}

% Although our strategy for learning explicit representations resembles codebook optimization~\cite{dong2023peco}, we use these representations for covariate adjustment instead of input reconstruction. The detailed pseudo-code for the optimization procedure is presented in Appendix~\ref{Append_B}.




% \section{Method}\label{sec:method}
\begin{figure}
    \centering
    \includegraphics[width=0.85\textwidth]{imgs/heatmap_acc.pdf}
    \caption{\textbf{Visualization of the proposed periodic Bayesian flow with mean parameter $\mu$ and accumulated accuracy parameter $c$ which corresponds to the entropy/uncertainty}. For $x = 0.3, \beta(1) = 1000$ and $\alpha_i$ defined in \cref{appd:bfn_cir}, this figure plots three colored stochastic parameter trajectories for receiver mean parameter $m$ and accumulated accuracy parameter $c$, superimposed on a log-scale heatmap of the Bayesian flow distribution $p_F(m|x,\senderacc)$ and $p_F(c|x,\senderacc)$. Note the \emph{non-monotonicity} and \emph{non-additive} property of $c$ which could inform the network the entropy of the mean parameter $m$ as a condition and the \emph{periodicity} of $m$. %\jj{Shrink the figures to save space}\hanlin{Do we need to make this figure one-column?}
    }
    \label{fig:vmbf_vis}
    \vskip -0.1in
\end{figure}
% \begin{wrapfigure}{r}{0.5\textwidth}
%     \centering
%     \includegraphics[width=0.49\textwidth]{imgs/heatmap_acc.pdf}
%     \caption{\textbf{Visualization of hyper-torus Bayesian flow based on von Mises Distribution}. For $x = 0.3, \beta(1) = 1000$ and $\alpha_i$ defined in \cref{appd:bfn_cir}, this figure plots three colored stochastic parameter trajectories for receiver mean parameter $m$ and accumulated accuracy parameter $c$, superimposed on a log-scale heatmap of the Bayesian flow distribution $p_F(m|x,\senderacc)$ and $p_F(c|x,\senderacc)$. Note the \emph{non-monotonicity} and \emph{non-additive} property of $c$. \jj{Shrink the figures to save space}}
%     \label{fig:vmbf_vis}
%     \vspace{-30pt}
% \end{wrapfigure}


In this section, we explain the detailed design of CrysBFN tackling theoretical and practical challenges. First, we describe how to derive our new formulation of Bayesian Flow Networks over hyper-torus $\mathbb{T}^{D}$ from scratch. Next, we illustrate the two key differences between \modelname and the original form of BFN: $1)$ a meticulously designed novel base distribution with different Bayesian update rules; and $2)$ different properties over the accuracy scheduling resulted from the periodicity and the new Bayesian update rules. Then, we present in detail the overall framework of \modelname over each manifold of the crystal space (\textit{i.e.} fractional coordinates, lattice vectors, atom types) respecting \textit{periodic E(3) invariance}. 

% In this section, we first demonstrate how to build Bayesian flow on hyper-torus $\mathbb{T}^{D}$ by overcoming theoretical and practical problems to provide a low-noise parameter-space approach to fractional atom coordinate generation. Next, we present how \modelname models each manifold of crystal space respecting \textit{periodic E(3) invariance}. 

\subsection{Periodic Bayesian Flow on Hyper-torus \texorpdfstring{$\mathbb{T}^{D}$}{}} 
For generative modeling of fractional coordinates in crystal, we first construct a periodic Bayesian flow on \texorpdfstring{$\mathbb{T}^{D}$}{} by designing every component of the totally new Bayesian update process which we demonstrate to be distinct from the original Bayesian flow (please see \cref{fig:non_add}). 
 %:) 
 
 The fractional atom coordinate system \citep{jiao2023crystal} inherently distributes over a hyper-torus support $\mathbb{T}^{3\times N}$. Hence, the normal distribution support on $\R$ used in the original \citep{bfn} is not suitable for this scenario. 
% The key problem of generative modeling for crystal is the periodicity of Cartesian atom coordinates $\vX$ requiring:
% \begin{equation}\label{eq:periodcity}
% p(\vA,\vL,\vX)=p(\vA,\vL,\vX+\vec{LK}),\text{where}~\vec{K}=\vec{k}\vec{1}_{1\times N},\forall\vec{k}\in\mathbb{Z}^{3\times1}
% \end{equation}
% However, there does not exist such a distribution supporting on $\R$ to model such property because the integration of such distribution over $\R$ will not be finite and equal to 1. Therefore, the normal distribution used in \citet{bfn} can not meet this condition.

To tackle this problem, the circular distribution~\citep{mardia2009directional} over the finite interval $[-\pi,\pi)$ is a natural choice as the base distribution for deriving the BFN on $\mathbb{T}^D$. 
% one natural choice is to 
% we would like to consider the circular distribution over the finite interval as the base 
% we find that circular distributions \citep{mardia2009directional} defined on a finite interval with lengths of $2\pi$ can be used as the instantiation of input distribution for the BFN on $\mathbb{T}^D$.
Specifically, circular distributions enjoy desirable periodic properties: $1)$ the integration over any interval length of $2\pi$ equals 1; $2)$ the probability distribution function is periodic with period $2\pi$.  Sharing the same intrinsic with fractional coordinates, such periodic property of circular distribution makes it suitable for the instantiation of BFN's input distribution, in parameterizing the belief towards ground truth $\x$ on $\mathbb{T}^D$. 
% \yuxuan{this is very complicated from my perspective.} \hanlin{But this property is exactly beautiful and perfectly fit into the BFN.}

\textbf{von Mises Distribution and its Bayesian Update} We choose von Mises distribution \citep{mardia2009directional} from various circular distributions as the form of input distribution, based on the appealing conjugacy property required in the derivation of the BFN framework.
% to leverage the Bayesian conjugacy property of von Mises distribution which is required by the BFN framework. 
That is, the posterior of a von Mises distribution parameterized likelihood is still in the family of von Mises distributions. The probability density function of von Mises distribution with mean direction parameter $m$ and concentration parameter $c$ (describing the entropy/uncertainty of $m$) is defined as: 
\begin{equation}
f(x|m,c)=vM(x|m,c)=\frac{\exp(c\cos(x-m))}{2\pi I_0(c)}
\end{equation}
where $I_0(c)$ is zeroth order modified Bessel function of the first kind as the normalizing constant. Given the last univariate belief parameterized by von Mises distribution with parameter $\theta_{i-1}=\{m_{i-1},\ c_{i-1}\}$ and the sample $y$ from sender distribution with unknown data sample $x$ and known accuracy $\alpha$ describing the entropy/uncertainty of $y$,  Bayesian update for the receiver is deducted as:
\begin{equation}
 h(\{m_{i-1},c_{i-1}\},y,\alpha)=\{m_i,c_i \}, \text{where}
\end{equation}
\begin{equation}\label{eq:h_m}
m_i=\text{atan2}(\alpha\sin y+c_{i-1}\sin m_{i-1}, {\alpha\cos y+c_{i-1}\cos m_{i-1}})
\end{equation}
\begin{equation}\label{eq:h_c}
c_i =\sqrt{\alpha^2+c_{i-1}^2+2\alpha c_{i-1}\cos(y-m_{i-1})}
\end{equation}
The proof of the above equations can be found in \cref{apdx:bayesian_update_function}. The atan2 function refers to  2-argument arctangent. Independently conducting  Bayesian update for each dimension, we can obtain the Bayesian update distribution by marginalizing $\y$:
\begin{equation}
p_U(\vtheta'|\vtheta,\bold{x};\alpha)=\mathbb{E}_{p_S(\bold{y}|\bold{x};\alpha)}\delta(\vtheta'-h(\vtheta,\bold{y},\alpha))=\mathbb{E}_{vM(\bold{y}|\bold{x},\alpha)}\delta(\vtheta'-h(\vtheta,\bold{y},\alpha))
\end{equation} 
\begin{figure}
    \centering
    \vskip -0.15in
    \includegraphics[width=0.95\linewidth]{imgs/non_add.pdf}
    \caption{An intuitive illustration of non-additive accuracy Bayesian update on the torus. The lengths of arrows represent the uncertainty/entropy of the belief (\emph{e.g.}~$1/\sigma^2$ for Gaussian and $c$ for von Mises). The directions of the arrows represent the believed location (\emph{e.g.}~ $\mu$ for Gaussian and $m$ for von Mises).}
    \label{fig:non_add}
    \vskip -0.15in
\end{figure}
\textbf{Non-additive Accuracy} 
The additive accuracy is a nice property held with the Gaussian-formed sender distribution of the original BFN expressed as:
\begin{align}
\label{eq:standard_id}
    \update(\parsn{}'' \mid \parsn{}, \x; \alpha_a+\alpha_b) = \E_{\update(\parsn{}' \mid \parsn{}, \x; \alpha_a)} \update(\parsn{}'' \mid \parsn{}', \x; \alpha_b)
\end{align}
Such property is mainly derived based on the standard identity of Gaussian variable:
\begin{equation}
X \sim \mathcal{N}\left(\mu_X, \sigma_X^2\right), Y \sim \mathcal{N}\left(\mu_Y, \sigma_Y^2\right) \Longrightarrow X+Y \sim \mathcal{N}\left(\mu_X+\mu_Y, \sigma_X^2+\sigma_Y^2\right)
\end{equation}
The additive accuracy property makes it feasible to derive the Bayesian flow distribution $
p_F(\boldsymbol{\theta} \mid \mathbf{x} ; i)=p_U\left(\boldsymbol{\theta} \mid \boldsymbol{\theta}_0, \mathbf{x}, \sum_{k=1}^{i} \alpha_i \right)
$ for the simulation-free training of \cref{eq:loss_n}.
It should be noted that the standard identity in \cref{eq:standard_id} does not hold in the von Mises distribution. Hence there exists an important difference between the original Bayesian flow defined on Euclidean space and the Bayesian flow of circular data on $\mathbb{T}^D$ based on von Mises distribution. With prior $\btheta = \{\bold{0},\bold{0}\}$, we could formally represent the non-additive accuracy issue as:
% The additive accuracy property implies the fact that the "confidence" for the data sample after observing a series of the noisy samples with accuracy ${\alpha_1, \cdots, \alpha_i}$ could be  as the accuracy sum  which could be  
% Here we 
% Here we emphasize the specific property of BFN based on von Mises distribution.
% Note that 
% \begin{equation}
% \update(\parsn'' \mid \parsn, \x; \alpha_a+\alpha_b) \ne \E_{\update(\parsn' \mid \parsn, \x; \alpha_a)} \update(\parsn'' \mid \parsn', \x; \alpha_b)
% \end{equation}
% \oyyw{please check whether the below equation is better}
% \yuxuan{I fill somehow confusing on what is the update distribution with $\alpha$. }
% \begin{equation}
% \update(\parsn{}'' \mid \parsn{}, \x; \alpha_a+\alpha_b) \ne \E_{\update(\parsn{}' \mid \parsn{}, \x; \alpha_a)} \update(\parsn{}'' \mid \parsn{}', \x; \alpha_b)
% \end{equation}
% We give an intuitive visualization of such difference in \cref{fig:non_add}. The untenability of this property can materialize by considering the following case: with prior $\btheta = \{\bold{0},\bold{0}\}$, check the two-step Bayesian update distribution with $\alpha_a,\alpha_b$ and one-step Bayesian update with $\alpha=\alpha_a+\alpha_b$:
\begin{align}
\label{eq:nonadd}
     &\update(c'' \mid \parsn, \x; \alpha_a+\alpha_b)  = \delta(c-\alpha_a-\alpha_b)
     \ne  \mathbb{E}_{p_U(\parsn' \mid \parsn, \x; \alpha_a)}\update(c'' \mid \parsn', \x; \alpha_b) \nonumber \\&= \mathbb{E}_{vM(\bold{y}_b|\bold{x},\alpha_a)}\mathbb{E}_{vM(\bold{y}_a|\bold{x},\alpha_b)}\delta(c-||[\alpha_a \cos\y_a+\alpha_b\cos \y_b,\alpha_a \sin\y_a+\alpha_b\sin \y_b]^T||_2)
\end{align}
A more intuitive visualization could be found in \cref{fig:non_add}. This fundamental difference between periodic Bayesian flow and that of \citet{bfn} presents both theoretical and practical challenges, which we will explain and address in the following contents.

% This makes constructing Bayesian flow based on von Mises distribution intrinsically different from previous Bayesian flows (\citet{bfn}).

% Thus, we must reformulate the framework of Bayesian flow networks  accordingly. % and do necessary reformulations of BFN. 

% \yuxuan{overall I feel this part is complicated by using the language of update distribution. I would like to suggest simply use bayesian update, to provide intuitive explantion.}\hanlin{See the illustration in \cref{fig:non_add}}

% That introduces a cascade of problems, and we investigate the following issues: $(1)$ Accuracies between sender and receiver are not synchronized and need to be differentiated. $(2)$ There is no tractable Bayesian flow distribution for a one-step sample conditioned on a given time step $i$, and naively simulating the Bayesian flow results in computational overhead. $(3)$ It is difficult to control the entropy of the Bayesian flow. $(4)$ Accuracy is no longer a function of $t$ and becomes a distribution conditioned on $t$, which can be different across dimensions.
%\jj{Edited till here}

\textbf{Entropy Conditioning} As a common practice in generative models~\citep{ddpm,flowmatching,bfn}, timestep $t$ is widely used to distinguish among generation states by feeding the timestep information into the networks. However, this paper shows that for periodic Bayesian flow, the accumulated accuracy $\vc_i$ is more effective than time-based conditioning by informing the network about the entropy and certainty of the states $\parsnt{i}$. This stems from the intrinsic non-additive accuracy which makes the receiver's accumulated accuracy $c$ not bijective function of $t$, but a distribution conditioned on accumulated accuracies $\vc_i$ instead. Therefore, the entropy parameter $\vc$ is taken logarithm and fed into the network to describe the entropy of the input corrupted structure. We verify this consideration in \cref{sec:exp_ablation}. 
% \yuxuan{implement variant. traditionally, the timestep is widely used to distinguish the different states by putting the timestep embedding into the networks. citation of FM, diffusion, BFN. However, we find that conditioned on time in periodic flow could not provide extra benefits. To further boost the performance, we introduce a simple yet effective modification term entropy conditional. This is based on that the accumulated accuracy which represents the current uncertainty or entropy could be a better indicator to distinguish different states. + Describe how you do this. }



\textbf{Reformulations of BFN}. Recall the original update function with Gaussian sender distribution, after receiving noisy samples $\y_1,\y_2,\dots,\y_i$ with accuracies $\senderacc$, the accumulated accuracies of the receiver side could be analytically obtained by the additive property and it is consistent with the sender side.
% Since observing sample $\y$ with $\alpha_i$ can not result in exact accuracy increment $\alpha_i$ for receiver, the accuracies between sender and receiver are not synchronized which need to be differentiated. 
However, as previously mentioned, this does not apply to periodic Bayesian flow, and some of the notations in original BFN~\citep{bfn} need to be adjusted accordingly. We maintain the notations of sender side's one-step accuracy $\alpha$ and added accuracy $\beta$, and alter the notation of receiver's accuracy parameter as $c$, which is needed to be simulated by cascade of Bayesian updates. We emphasize that the receiver's accumulated accuracy $c$ is no longer a function of $t$ (differently from the Gaussian case), and it becomes a distribution conditioned on received accuracies $\senderacc$ from the sender. Therefore, we represent the Bayesian flow distribution of von Mises distribution as $p_F(\btheta|\x;\alpha_1,\alpha_2,\dots,\alpha_i)$. And the original simulation-free training with Bayesian flow distribution is no longer applicable in this scenario.
% Different from previous BFNs where the accumulated accuracy $\rho$ is not explicitly modeled, the accumulated accuracy parameter $c$ (visualized in \cref{fig:vmbf_vis}) needs to be explicitly modeled by feeding it to the network to avoid information loss.
% the randomaccuracy parameter $c$ (visualized in \cref{fig:vmbf_vis}) implies that there exists information in $c$ from the sender just like $m$, meaning that $c$ also should be fed into the network to avoid information loss. 
% We ablate this consideration in  \cref{sec:exp_ablation}. 

\textbf{Fast Sampling from Equivalent Bayesian Flow Distribution} Based on the above reformulations, the Bayesian flow distribution of von Mises distribution is reframed as: 
\begin{equation}\label{eq:flow_frac}
p_F(\btheta_i|\x;\alpha_1,\alpha_2,\dots,\alpha_i)=\E_{\update(\parsnt{1} \mid \parsnt{0}, \x ; \alphat{1})}\dots\E_{\update(\parsn_{i-1} \mid \parsnt{i-2}, \x; \alphat{i-1})} \update(\parsnt{i} | \parsnt{i-1},\x;\alphat{i} )
\end{equation}
Naively sampling from \cref{eq:flow_frac} requires slow auto-regressive iterated simulation, making training unaffordable. Noticing the mathematical properties of \cref{eq:h_m,eq:h_c}, we  transform \cref{eq:flow_frac} to the equivalent form:
\begin{equation}\label{eq:cirflow_equiv}
p_F(\vec{m}_i|\x;\alpha_1,\alpha_2,\dots,\alpha_i)=\E_{vM(\y_1|\x,\alpha_1)\dots vM(\y_i|\x,\alpha_i)} \delta(\vec{m}_i-\text{atan2}(\sum_{j=1}^i \alpha_j \cos \y_j,\sum_{j=1}^i \alpha_j \sin \y_j))
\end{equation}
\begin{equation}\label{eq:cirflow_equiv2}
p_F(\vec{c}_i|\x;\alpha_1,\alpha_2,\dots,\alpha_i)=\E_{vM(\y_1|\x,\alpha_1)\dots vM(\y_i|\x,\alpha_i)}  \delta(\vec{c}_i-||[\sum_{j=1}^i \alpha_j \cos \y_j,\sum_{j=1}^i \alpha_j \sin \y_j]^T||_2)
\end{equation}
which bypasses the computation of intermediate variables and allows pure tensor operations, with negligible computational overhead.
\begin{restatable}{proposition}{cirflowequiv}
The probability density function of Bayesian flow distribution defined by \cref{eq:cirflow_equiv,eq:cirflow_equiv2} is equivalent to the original definition in \cref{eq:flow_frac}. 
\end{restatable}
\textbf{Numerical Determination of Linear Entropy Sender Accuracy Schedule} ~Original BFN designs the accuracy schedule $\beta(t)$ to make the entropy of input distribution linearly decrease. As for crystal generation task, to ensure information coherence between modalities, we choose a sender accuracy schedule $\senderacc$ that makes the receiver's belief entropy $H(t_i)=H(p_I(\cdot|\vtheta_i))=H(p_I(\cdot|\vc_i))$ linearly decrease \emph{w.r.t.} time $t_i$, given the initial and final accuracy parameter $c(0)$ and $c(1)$. Due to the intractability of \cref{eq:vm_entropy}, we first use numerical binary search in $[0,c(1)]$ to determine the receiver's $c(t_i)$ for $i=1,\dots, n$ by solving the equation $H(c(t_i))=(1-t_i)H(c(0))+tH(c(1))$. Next, with $c(t_i)$, we conduct numerical binary search for each $\alpha_i$ in $[0,c(1)]$ by solving the equations $\E_{y\sim vM(x,\alpha_i)}[\sqrt{\alpha_i^2+c_{i-1}^2+2\alpha_i c_{i-1}\cos(y-m_{i-1})}]=c(t_i)$ from $i=1$ to $i=n$ for arbitrarily selected $x\in[-\pi,\pi)$.

After tackling all those issues, we have now arrived at a new BFN architecture for effectively modeling crystals. Such BFN can also be adapted to other type of data located in hyper-torus $\mathbb{T}^{D}$.

\subsection{Equivariant Bayesian Flow for Crystal}
With the above Bayesian flow designed for generative modeling of fractional coordinate $\vF$, we are able to build equivariant Bayesian flow for each modality of crystal. In this section, we first give an overview of the general training and sampling algorithm of \modelname (visualized in \cref{fig:framework}). Then, we describe the details of the Bayesian flow of every modality. The training and sampling algorithm can be found in \cref{alg:train} and \cref{alg:sampling}.

\textbf{Overview} Operating in the parameter space $\bthetaM=\{\bthetaA,\bthetaL,\bthetaF\}$, \modelname generates high-fidelity crystals through a joint BFN sampling process on the parameter of  atom type $\bthetaA$, lattice parameter $\vec{\theta}^L=\{\bmuL,\brhoL\}$, and the parameter of fractional coordinate matrix $\bthetaF=\{\bmF,\bcF\}$. We index the $n$-steps of the generation process in a discrete manner $i$, and denote the corresponding continuous notation $t_i=i/n$ from prior parameter $\thetaM_0$ to a considerably low variance parameter $\thetaM_n$ (\emph{i.e.} large $\vrho^L,\bmF$, and centered $\bthetaA$).

At training time, \modelname samples time $i\sim U\{1,n\}$ and $\bthetaM_{i-1}$ from the Bayesian flow distribution of each modality, serving as the input to the network. The network $\net$ outputs $\net(\parsnt{i-1}^\mathcal{M},t_{i-1})=\net(\parsnt{i-1}^A,\parsnt{i-1}^F,\parsnt{i-1}^L,t_{i-1})$ and conducts gradient descents on loss function \cref{eq:loss_n} for each modality. After proper training, the sender distribution $p_S$ can be approximated by the receiver distribution $p_R$. 

At inference time, from predefined $\thetaM_0$, we conduct transitions from $\thetaM_{i-1}$ to $\thetaM_{i}$ by: $(1)$ sampling $\y_i\sim p_R(\bold{y}|\thetaM_{i-1};t_i,\alpha_i)$ according to network prediction $\predM{i-1}$; and $(2)$ performing Bayesian update $h(\thetaM_{i-1},\y^\calM_{i-1},\alpha_i)$ for each dimension. 

% Alternatively, we complete this transition using the flow-back technique by sampling 
% $\thetaM_{i}$ from Bayesian flow distribution $\flow(\btheta^M_{i}|\predM{i-1};t_{i-1})$. 

% The training objective of $\net$ is to minimize the KL divergence between sender distribution and receiver distribution for every modality as defined in \cref{eq:loss_n} which is equivalent to optimizing the negative variational lower bound $\calL^{VLB}$ as discussed in \cref{sec:preliminaries}. 

%In the following part, we will present the Bayesian flow of each modality in detail.

\textbf{Bayesian Flow of Fractional Coordinate $\vF$}~The distribution of the prior parameter $\bthetaF_0$ is defined as:
\begin{equation}\label{eq:prior_frac}
    p(\bthetaF_0) \defeq \{vM(\vm_0^F|\vec{0}_{3\times N},\vec{0}_{3\times N}),\delta(\vc_0^F-\vec{0}_{3\times N})\} = \{U(\vec{0},\vec{1}),\delta(\vc_0^F-\vec{0}_{3\times N})\}
\end{equation}
Note that this prior distribution of $\vm_0^F$ is uniform over $[\vec{0},\vec{1})$, ensuring the periodic translation invariance property in \cref{De:pi}. The training objective is minimizing the KL divergence between sender and receiver distribution (deduction can be found in \cref{appd:cir_loss}): 
%\oyyw{replace $\vF$ with $\x$?} \hanlin{notations follow Preliminary?}
\begin{align}\label{loss_frac}
\calL_F = n \E_{i \sim \ui{n}, \flow(\parsn{}^F \mid \vF ; \senderacc)} \alpha_i\frac{I_1(\alpha_i)}{I_0(\alpha_i)}(1-\cos(\vF-\predF{i-1}))
\end{align}
where $I_0(x)$ and $I_1(x)$ are the zeroth and the first order of modified Bessel functions. The transition from $\bthetaF_{i-1}$ to $\bthetaF_{i}$ is the Bayesian update distribution based on network prediction:
\begin{equation}\label{eq:transi_frac}
    p(\btheta^F_{i}|\parsnt{i-1}^\calM)=\mathbb{E}_{vM(\bold{y}|\predF{i-1},\alpha_i)}\delta(\btheta^F_{i}-h(\btheta^F_{i-1},\bold{y},\alpha_i))
\end{equation}
\begin{restatable}{proposition}{fracinv}
With $\net_{F}$ as a periodic translation equivariant function namely $\net_F(\parsnt{}^A,w(\parsnt{}^F+\vt),\parsnt{}^L,t)=w(\net_F(\parsnt{}^A,\parsnt{}^F,\parsnt{}^L,t)+\vt), \forall\vt\in\R^3$, the marginal distribution of $p(\vF_n)$ defined by \cref{eq:prior_frac,eq:transi_frac} is periodic translation invariant. 
\end{restatable}
\textbf{Bayesian Flow of Lattice Parameter \texorpdfstring{$\boldsymbol{L}$}{}}   
Noting the lattice parameter $\bm{L}$ located in Euclidean space, we set prior as the parameter of a isotropic multivariate normal distribution $\btheta^L_0\defeq\{\vmu_0^L,\vrho_0^L\}=\{\bm{0}_{3\times3},\bm{1}_{3\times3}\}$
% \begin{equation}\label{eq:lattice_prior}
% \btheta^L_0\defeq\{\vmu_0^L,\vrho_0^L\}=\{\bm{0}_{3\times3},\bm{1}_{3\times3}\}
% \end{equation}
such that the prior distribution of the Markov process on $\vmu^L$ is the Dirac distribution $\delta(\vec{\mu_0}-\vec{0})$ and $\delta(\vec{\rho_0}-\vec{1})$, 
% \begin{equation}
%     p_I^L(\boldsymbol{L}|\btheta_0^L)=\mathcal{N}(\bm{L}|\bm{0},\bm{I})
% \end{equation}
which ensures O(3)-invariance of prior distribution of $\vL$. By Eq. 77 from \citet{bfn}, the Bayesian flow distribution of the lattice parameter $\bm{L}$ is: 
\begin{align}% =p_U(\bmuL|\btheta_0^L,\bm{L},\beta(t))
p_F^L(\bmuL|\bm{L};t) &=\mathcal{N}(\bmuL|\gamma(t)\bm{L},\gamma(t)(1-\gamma(t))\bm{I}) 
\end{align}
where $\gamma(t) = 1 - \sigma_1^{2t}$ and $\sigma_1$ is the predefined hyper-parameter controlling the variance of input distribution at $t=1$ under linear entropy accuracy schedule. The variance parameter $\vrho$ does not need to be modeled and fed to the network, since it is deterministic given the accuracy schedule. After sampling $\bmuL_i$ from $p_F^L$, the training objective is defined as minimizing KL divergence between sender and receiver distribution (based on Eq. 96 in \citet{bfn}):
\begin{align}
\mathcal{L}_{L} = \frac{n}{2}\left(1-\sigma_1^{2/n}\right)\E_{i \sim \ui{n}}\E_{\flow(\bmuL_{i-1} |\vL ; t_{i-1})}  \frac{\left\|\vL -\predL{i-1}\right\|^2}{\sigma_1^{2i/n}},\label{eq:lattice_loss}
\end{align}
where the prediction term $\predL{i-1}$ is the lattice parameter part of network output. After training, the generation process is defined as the Bayesian update distribution given network prediction:
\begin{equation}\label{eq:lattice_sampling}
    p(\bmuL_{i}|\parsnt{i-1}^\calM)=\update^L(\bmuL_{i}|\predL{i-1},\bmuL_{i-1};t_{i-1})
\end{equation}
    

% The final prediction of the lattice parameter is given by $\bmuL_n = \predL{n-1}$.
% \begin{equation}\label{eq:final_lattice}
%     \bmuL_n = \predL{n-1}
% \end{equation}

\begin{restatable}{proposition}{latticeinv}\label{prop:latticeinv}
With $\net_{L}$ as  O(3)-equivariant function namely $\net_L(\parsnt{}^A,\parsnt{}^F,\vQ\parsnt{}^L,t)=\vQ\net_L(\parsnt{}^A,\parsnt{}^F,\parsnt{}^L,t),\forall\vQ^T\vQ=\vI$, the marginal distribution of $p(\bmuL_n)$ defined by \cref{eq:lattice_sampling} is O(3)-invariant. 
\end{restatable}


\textbf{Bayesian Flow of Atom Types \texorpdfstring{$\boldsymbol{A}$}{}} 
Given that atom types are discrete random variables located in a simplex $\calS^K$, the prior parameter of $\boldsymbol{A}$ is the discrete uniform distribution over the vocabulary $\parsnt{0}^A \defeq \frac{1}{K}\vec{1}_{1\times N}$. 
% \begin{align}\label{eq:disc_input_prior}
% \parsnt{0}^A \defeq \frac{1}{K}\vec{1}_{1\times N}
% \end{align}
% \begin{align}
%     (\oh{j}{K})_k \defeq \delta_{j k}, \text{where }\oh{j}{K}\in \R^{K},\oh{\vA}{KD} \defeq \left(\oh{a_1}{K},\dots,\oh{a_N}{K}\right) \in \R^{K\times N}
% \end{align}
With the notation of the projection from the class index $j$ to the length $K$ one-hot vector $ (\oh{j}{K})_k \defeq \delta_{j k}, \text{where }\oh{j}{K}\in \R^{K},\oh{\vA}{KD} \defeq \left(\oh{a_1}{K},\dots,\oh{a_N}{K}\right) \in \R^{K\times N}$, the Bayesian flow distribution of atom types $\vA$ is derived in \citet{bfn}:
\begin{align}
\flow^{A}(\parsn^A \mid \vA; t) &= \E_{\N{\y \mid \beta^A(t)\left(K \oh{\vA}{K\times N} - \vec{1}_{K\times N}\right)}{\beta^A(t) K \vec{I}_{K\times N \times N}}} \delta\left(\parsn^A - \frac{e^{\y}\parsnt{0}^A}{\sum_{k=1}^K e^{\y_k}(\parsnt{0})_{k}^A}\right).
\end{align}
where $\beta^A(t)$ is the predefined accuracy schedule for atom types. Sampling $\btheta_i^A$ from $p_F^A$ as the training signal, the training objective is the $n$-step discrete-time loss for discrete variable \citep{bfn}: 
% \oyyw{can we simplify the next equation? Such as remove $K \times N, K \times N \times N$}
% \begin{align}
% &\calL_A = n\E_{i \sim U\{1,n\},\flow^A(\parsn^A \mid \vA ; t_{i-1}),\N{\y \mid \alphat{i}\left(K \oh{\vA}{KD} - \vec{1}_{K\times N}\right)}{\alphat{i} K \vec{I}_{K\times N \times N}}} \ln \N{\y \mid \alphat{i}\left(K \oh{\vA}{K\times N} - \vec{1}_{K\times N}\right)}{\alphat{i} K \vec{I}_{K\times N \times N}}\nonumber\\
% &\qquad\qquad\qquad-\sum_{d=1}^N \ln \left(\sum_{k=1}^K \out^{(d)}(k \mid \parsn^A; t_{i-1}) \N{\ydd{d} \mid \alphat{i}\left(K\oh{k}{K}- \vec{1}_{K\times N}\right)}{\alphat{i} K \vec{I}_{K\times N \times N}}\right)\label{discdisc_t_loss_exp}
% \end{align}
\begin{align}
&\calL_A = n\E_{i \sim U\{1,n\},\flow^A(\parsn^A \mid \vA ; t_{i-1}),\N{\y \mid \alphat{i}\left(K \oh{\vA}{KD} - \vec{1}\right)}{\alphat{i} K \vec{I}}} \ln \N{\y \mid \alphat{i}\left(K \oh{\vA}{K\times N} - \vec{1}\right)}{\alphat{i} K \vec{I}}\nonumber\\
&\qquad\qquad\qquad-\sum_{d=1}^N \ln \left(\sum_{k=1}^K \out^{(d)}(k \mid \parsn^A; t_{i-1}) \N{\ydd{d} \mid \alphat{i}\left(K\oh{k}{K}- \vec{1}\right)}{\alphat{i} K \vec{I}}\right)\label{discdisc_t_loss_exp}
\end{align}
where $\vec{I}\in \R^{K\times N \times N}$ and $\vec{1}\in\R^{K\times D}$. When sampling, the transition from $\bthetaA_{i-1}$ to $\bthetaA_{i}$ is derived as:
\begin{equation}
    p(\btheta^A_{i}|\parsnt{i-1}^\calM)=\update^A(\btheta^A_{i}|\btheta^A_{i-1},\predA{i-1};t_{i-1})
\end{equation}

The detailed training and sampling algorithm could be found in \cref{alg:train} and \cref{alg:sampling}.





\section{Experiments}
\label{sec:exp}
Following the settings in Section \ref{sec:existing}, we evaluate \textit{NovelSum}'s correlation with the fine-tuned model performance across 53 IT datasets and compare it with previous diversity metrics. Additionally, we conduct a correlation analysis using Qwen-2.5-7B \cite{yang2024qwen2} as the backbone model, alongside previous LLaMA-3-8B experiments, to further demonstrate the metric's effectiveness across different scenarios. Qwen is used for both instruction tuning and deriving semantic embeddings. Due to resource constraints, we run each strategy on Qwen for two rounds, resulting in 25 datasets. 

\subsection{Main Results}

\begin{table*}[!t]
    \centering
    \resizebox{\linewidth}{!}{
    \begin{tabular}{lcccccccccc}
    \toprule
    \multirow{3}*{\textbf{Diversity Metrics}} & \multicolumn{10}{c}{\textbf{Data Selection Strategies}} \\
    \cmidrule(lr){2-11}
    & \multirow{2}*{\textbf{K-means}} & \multirow{2}*{\vtop{\hbox{\textbf{K-Center}}\vspace{1mm}\hbox{\textbf{-Greedy}}}}  & \multirow{2}*{\textbf{QDIT}} & \multirow{2}*{\vtop{\hbox{\textbf{Repr}}\vspace{1mm}\hbox{\textbf{Filter}}}} & \multicolumn{5}{c}{\textbf{Random}} & \multirow{2}{*}{\textbf{Duplicate}} \\ 
    \cmidrule(lr){6-10}
    & & & & & \textbf{$\mathcal{X}^{all}$} & ShareGPT & WizardLM & Alpaca & Dolly &  \\
    \midrule
    \rowcolor{gray!15} \multicolumn{11}{c}{\textit{LLaMA-3-8B}} \\
    Facility Loc. $_{\times10^5}$ & \cellcolor{BLUE!40} 2.99 & \cellcolor{ORANGE!10} 2.73 & \cellcolor{BLUE!40} 2.99 & \cellcolor{BLUE!20} 2.86 & \cellcolor{BLUE!40} 2.99 & \cellcolor{BLUE!0} 2.83 & \cellcolor{BLUE!30} 2.88 & \cellcolor{BLUE!0} 2.83 & \cellcolor{ORANGE!20} 2.59 & \cellcolor{ORANGE!30} 2.52 \\    
    DistSum$_{cosine}$  & \cellcolor{BLUE!30} 0.648 & \cellcolor{BLUE!60} 0.746 & \cellcolor{BLUE!0} 0.629 & \cellcolor{BLUE!50} 0.703 & \cellcolor{BLUE!10} 0.634 & \cellcolor{BLUE!40} 0.656 & \cellcolor{ORANGE!30} 0.578 & \cellcolor{ORANGE!10} 0.605 & \cellcolor{ORANGE!20} 0.603 & \cellcolor{BLUE!10} 0.634 \\
    Vendi Score $_{\times10^7}$ & \cellcolor{BLUE!30} 1.70 & \cellcolor{BLUE!60} 2.53 & \cellcolor{BLUE!10} 1.59 & \cellcolor{BLUE!50} 2.23 & \cellcolor{BLUE!20} 1.61 & \cellcolor{BLUE!30} 1.70 & \cellcolor{ORANGE!10} 1.44 & \cellcolor{ORANGE!20} 1.32 & \cellcolor{ORANGE!10} 1.44 & \cellcolor{ORANGE!30} 0.05 \\
    \textbf{NovelSum (Ours)} & \cellcolor{BLUE!60} 0.693 & \cellcolor{BLUE!50} 0.687 & \cellcolor{BLUE!30} 0.673 & \cellcolor{BLUE!20} 0.671 & \cellcolor{BLUE!40} 0.675 & \cellcolor{BLUE!10} 0.628 & \cellcolor{BLUE!0} 0.591 & \cellcolor{ORANGE!10} 0.572 & \cellcolor{ORANGE!20} 0.50 & \cellcolor{ORANGE!30} 0.461 \\
    \midrule    
    \textbf{Model Performance} & \cellcolor{BLUE!60}1.32 & \cellcolor{BLUE!50}1.31 & \cellcolor{BLUE!40}1.25 & \cellcolor{BLUE!30}1.05 & \cellcolor{BLUE!20}1.20 & \cellcolor{BLUE!10}0.83 & \cellcolor{BLUE!0}0.72 & \cellcolor{ORANGE!10}0.07 & \cellcolor{ORANGE!20}-0.14 & \cellcolor{ORANGE!30}-1.35 \\
    \midrule
    \midrule
    \rowcolor{gray!15} \multicolumn{11}{c}{\textit{Qwen-2.5-7B}} \\
    Facility Loc. $_{\times10^5}$ & \cellcolor{BLUE!40} 3.54 & \cellcolor{ORANGE!30} 3.42 & \cellcolor{BLUE!40} 3.54 & \cellcolor{ORANGE!20} 3.46 & \cellcolor{BLUE!40} 3.54 & \cellcolor{BLUE!30} 3.51 & \cellcolor{BLUE!10} 3.50 & \cellcolor{BLUE!10} 3.50 & \cellcolor{ORANGE!20} 3.46 & \cellcolor{BLUE!0} 3.48 \\ 
    DistSum$_{cosine}$ & \cellcolor{BLUE!30} 0.260 & \cellcolor{BLUE!60} 0.440 & \cellcolor{BLUE!0} 0.223 & \cellcolor{BLUE!50} 0.421 & \cellcolor{BLUE!10} 0.230 & \cellcolor{BLUE!40} 0.285 & \cellcolor{ORANGE!20} 0.211 & \cellcolor{ORANGE!30} 0.189 & \cellcolor{ORANGE!10} 0.221 & \cellcolor{BLUE!20} 0.243 \\
    Vendi Score $_{\times10^6}$ & \cellcolor{ORANGE!10} 1.60 & \cellcolor{BLUE!40} 3.09 & \cellcolor{BLUE!10} 2.60 & \cellcolor{BLUE!60} 7.15 & \cellcolor{ORANGE!20} 1.41 & \cellcolor{BLUE!50} 3.36 & \cellcolor{BLUE!20} 2.65 & \cellcolor{BLUE!0} 1.89 & \cellcolor{BLUE!30} 3.04 & \cellcolor{ORANGE!30} 0.20 \\
    \textbf{NovelSum (Ours)}  & \cellcolor{BLUE!40} 0.440 & \cellcolor{BLUE!60} 0.505 & \cellcolor{BLUE!20} 0.403 & \cellcolor{BLUE!50} 0.495 & \cellcolor{BLUE!30} 0.408 & \cellcolor{BLUE!10} 0.392 & \cellcolor{BLUE!0} 0.349 & \cellcolor{ORANGE!10} 0.336 & \cellcolor{ORANGE!20} 0.320 & \cellcolor{ORANGE!30} 0.309 \\
    \midrule
    \textbf{Model Performance} & \cellcolor{BLUE!30} 1.06 & \cellcolor{BLUE!60} 1.45 & \cellcolor{BLUE!40} 1.23 & \cellcolor{BLUE!50} 1.35 & \cellcolor{BLUE!20} 0.87 & \cellcolor{BLUE!10} 0.07 & \cellcolor{BLUE!0} -0.08 & \cellcolor{ORANGE!10} -0.38 & \cellcolor{ORANGE!30} -0.49 & \cellcolor{ORANGE!20} -0.43 \\
    \bottomrule
    \end{tabular}
    }
    \caption{Measuring the diversity of datasets selected by different strategies using \textit{NovelSum} and baseline metrics. Fine-tuned model performances (Eq. \ref{eq:perf}), based on MT-bench and AlpacaEval, are also included for cross reference. Darker \colorbox{BLUE!60}{blue} shades indicate higher values for each metric, while darker \colorbox{ORANGE!30}{orange} shades indicate lower values. While data selection strategies vary in performance on LLaMA-3-8B and Qwen-2.5-7B, \textit{NovelSum} consistently shows a stronger correlation with model performance than other metrics. More results are provided in Appendix \ref{app:results}.}
    \label{tbl:main}
    \vspace{-4mm}
\end{table*}


\begin{table}[t!]
\centering
\resizebox{\linewidth}{!}{
\begin{tabular}{lcccc}
\toprule
\multirow{2}*{\textbf{Diversity Metrics}} & \multicolumn{3}{c}{\textbf{LLaMA}} & \textbf{Qwen}\\
\cmidrule(lr){2-4} \cmidrule(lr){5-5} 
& \textbf{Pearson} & \textbf{Spearman} & \textbf{Avg.} & \textbf{Avg.} \\
\midrule
TTR & -0.38 & -0.16 & -0.27 & -0.30 \\
vocd-D & -0.43 & -0.17 & -0.30 & -0.31 \\
\midrule
Facility Loc. & 0.86 & 0.69 & 0.77 & 0.08 \\
Entropy & 0.93 & 0.80 & 0.86 & 0.63 \\
\midrule
LDD & 0.61 & 0.75 & 0.68 & 0.60 \\
KNN Distance & 0.59 & 0.80 & 0.70 & 0.67 \\
DistSum$_{cosine}$ & 0.85 & 0.67 & 0.76 & 0.51 \\
Vendi Score & 0.70 & 0.85 & 0.78 & 0.60 \\
DistSum$_{L2}$ & 0.86 & 0.76 & 0.81 & 0.51 \\
Cluster Inertia & 0.81 & 0.85 & 0.83 & 0.76 \\
Radius & 0.87 & 0.81 & 0.84 & 0.48 \\
\midrule
NovelSum & \textbf{0.98} & \textbf{0.95} & \textbf{0.97} & \textbf{0.90} \\
\bottomrule
\end{tabular}
}
\caption{Correlations between different metrics and model performance on LLaMA-3-8B and Qwen-2.5-7B.  “Avg.” denotes the average correlation (Eq. \ref{eq:cor}).}
\label{tbl:correlations}
\vspace{-2mm}
\end{table}

\paragraph{\textit{NovelSum} consistently achieves state-of-the-art correlation with model performance across various data selection strategies, backbone LLMs, and correlation measures.}
Table \ref{tbl:main} presents diversity measurement results on datasets constructed by mainstream data selection methods (based on $\mathcal{X}^{all}$), random selection from various sources, and duplicated samples (with only $m=100$ unique samples). 
Results from multiple runs are averaged for each strategy.
Although these strategies yield varying performance rankings across base models, \textit{NovelSum} consistently tracks changes in IT performance by accurately measuring dataset diversity. For instance, K-means achieves the best performance on LLaMA with the highest NovelSum score, while K-Center-Greedy excels on Qwen, also correlating with the highest NovelSum. Table \ref{tbl:correlations} shows the correlation coefficients between various metrics and model performance for both LLaMA and Qwen experiments, where \textit{NovelSum} achieves state-of-the-art correlation across different models and measures.

\paragraph{\textit{NovelSum} can provide valuable guidance for data engineering practices.}
As a reliable indicator of data diversity, \textit{NovelSum} can assess diversity at both the dataset and sample levels, directly guiding data selection and construction decisions. For example, Table \ref{tbl:main} shows that the combined data source $\mathcal{X}^{all}$ is a better choice for sampling diverse IT data than other sources. Moreover, \textit{NovelSum} can offer insights through comparative analyses, such as: (1) ShareGPT, which collects data from real internet users, exhibits greater diversity than Dolly, which relies on company employees, suggesting that IT samples from diverse sources enhance dataset diversity \cite{wang2024diversity-logD}; (2) In LLaMA experiments, random selection can outperform some mainstream strategies, aligning with prior work \cite{xia2024rethinking,diddee2024chasing}, highlighting gaps in current data selection methods for optimizing diversity.



\subsection{Ablation Study}


\textit{NovelSum} involves several flexible hyperparameters and variations. In our main experiments, \textit{NovelSum} uses cosine distance to compute $d(x_i, x_j)$ in Eq. \ref{eq:dad}. We set $\alpha = 1$, $\beta = 0.5$, and $K = 10$ nearest neighbors in Eq. \ref{eq:pws} and \ref{eq:dad}. Here, we conduct an ablation study to investigate the impact of these settings based on LLaMA-3-8B.

\begin{table}[ht!]
\centering
\resizebox{\linewidth}{!}{
\begin{tabular}{lccc}
\toprule
\textbf{Variants} & \textbf{Pearson} & \textbf{Spearman} & \textbf{Avg.} \\
\midrule
NovelSum & 0.98 & 0.96 & 0.97 \\
\midrule
\hspace{0.10cm} - Use $L2$ distance & 0.97 & 0.83 & 0.90\textsubscript{↓ 0.08} \\
\hspace{0.10cm} - $K=20$ & 0.98 & 0.96 & 0.97\textsubscript{↓ 0.00} \\
\hspace{0.10cm} - $\alpha=0$ (w/o proximity) & 0.79 & 0.31 & 0.55\textsubscript{↓ 0.42} \\
\hspace{0.10cm} - $\alpha=2$ & 0.73 & 0.88 & 0.81\textsubscript{↓ 0.16} \\
\hspace{0.10cm} - $\beta=0$ (w/o density) & 0.92 & 0.89 & 0.91\textsubscript{↓ 0.07} \\
\hspace{0.10cm} - $\beta=1$ & 0.90 & 0.62 & 0.76\textsubscript{↓ 0.21} \\
\bottomrule
\end{tabular}
}
\caption{Ablation Study for \textit{NovelSum}.}
\label{tbl:ablation}
\vspace{-2mm}
\end{table}

In Table \ref{tbl:ablation}, $\alpha=0$ removes the proximity weights, and $\beta=0$ eliminates the density multiplier. We observe that both $\alpha=0$ and $\beta=0$ significantly weaken the correlation, validating the benefits of the proximity-weighted sum and density-aware distance. Additionally, improper values for $\alpha$ and $\beta$ greatly reduce the metric's reliability, highlighting that \textit{NovelSum} strikes a delicate balance between distances and distribution. Replacing cosine distance with Euclidean distance and using more neighbors for density approximation have minimal impact, particularly on Pearson's correlation, demonstrating \textit{NovelSum}'s robustness to different distance measures.







\section{Conclusion}
In this work, we propose a simple yet effective approach, called SMILE, for graph few-shot learning with fewer tasks. Specifically, we introduce a novel dual-level mixup strategy, including within-task and across-task mixup, for enriching the diversity of nodes within each task and the diversity of tasks. Also, we incorporate the degree-based prior information to learn expressive node embeddings. Theoretically, we prove that SMILE effectively enhances the model's generalization performance. Empirically, we conduct extensive experiments on multiple benchmarks and the results suggest that SMILE significantly outperforms other baselines, including both in-domain and cross-domain few-shot settings.

\section*{Impact statement}

This paper proposes that machine learning can and should be used to maximize social welfare. In principle, and by construction, the impact of our proposed framework on society aims to be positive. But our paper also points to the inherent difficulties of identifying, and making formal, what `good for society' is. We lean on the field of welfare economics, which has for decades contended with this challenge, for ideas on how the learning community can begin to approach this daunting task.
However, even if these ideas are conceptually appealing,
the path to practical welfare improvement presents many challenges---%
some expected, others unforseen.
% and will likely include many ups and downs.
For example, we may specify incorrect social welfare functions;
or we may specify them correctly but be unable to optimize them appropriately;
or we may be able to optimize but find that 
our assumptions are wrong, that theory differs from practice,
or that there were other considerations and complexities that we did not take into account.
For this we can look to other related fields---%
such as fairness, privacy, and alignment in machine learning---%
which have taken (and are still taking) similar journeys,
and learn from both their success and mistakes.
% and hope that ours will be similar.

Any discipline that seeks to affect policy should do so with much deliberation and care. Whereas welfare economics was designed with the explicit purpose of supporting (and influencing) policymakers,
machine learning has found itself in a similar position, but likely without any planned intent.
On the one hand, adjusting machine learning to support notions, such as social welfare,
that it was not designed to support initially can prove challenging.
However, and as we argue throughout, we believe that building on top of existing machinery is a more practical approach than to begin from scratch.
The necessity of confronting with welfare consideration can also
be an opportunity---as we can leverage these novel challenges
to make machine learning practice more informed, transparent, responsible, and socially aware.


% At the same time, the novelty of the challenges that welfare considerations present to the field make this an opportunity---%
% for chaning the role of machine learning in society for the better in a manner that is informed, transparent, and aware.



\bibliography{ref}
\bibliographystyle{icml2025}
% \onecolumn

\appendix
\subsection{Lloyd-Max Algorithm}
\label{subsec:Lloyd-Max}
For a given quantization bitwidth $B$ and an operand $\bm{X}$, the Lloyd-Max algorithm finds $2^B$ quantization levels $\{\hat{x}_i\}_{i=1}^{2^B}$ such that quantizing $\bm{X}$ by rounding each scalar in $\bm{X}$ to the nearest quantization level minimizes the quantization MSE. 

The algorithm starts with an initial guess of quantization levels and then iteratively computes quantization thresholds $\{\tau_i\}_{i=1}^{2^B-1}$ and updates quantization levels $\{\hat{x}_i\}_{i=1}^{2^B}$. Specifically, at iteration $n$, thresholds are set to the midpoints of the previous iteration's levels:
\begin{align*}
    \tau_i^{(n)}=\frac{\hat{x}_i^{(n-1)}+\hat{x}_{i+1}^{(n-1)}}2 \text{ for } i=1\ldots 2^B-1
\end{align*}
Subsequently, the quantization levels are re-computed as conditional means of the data regions defined by the new thresholds:
\begin{align*}
    \hat{x}_i^{(n)}=\mathbb{E}\left[ \bm{X} \big| \bm{X}\in [\tau_{i-1}^{(n)},\tau_i^{(n)}] \right] \text{ for } i=1\ldots 2^B
\end{align*}
where to satisfy boundary conditions we have $\tau_0=-\infty$ and $\tau_{2^B}=\infty$. The algorithm iterates the above steps until convergence.

Figure \ref{fig:lm_quant} compares the quantization levels of a $7$-bit floating point (E3M3) quantizer (left) to a $7$-bit Lloyd-Max quantizer (right) when quantizing a layer of weights from the GPT3-126M model at a per-tensor granularity. As shown, the Lloyd-Max quantizer achieves substantially lower quantization MSE. Further, Table \ref{tab:FP7_vs_LM7} shows the superior perplexity achieved by Lloyd-Max quantizers for bitwidths of $7$, $6$ and $5$. The difference between the quantizers is clear at 5 bits, where per-tensor FP quantization incurs a drastic and unacceptable increase in perplexity, while Lloyd-Max quantization incurs a much smaller increase. Nevertheless, we note that even the optimal Lloyd-Max quantizer incurs a notable ($\sim 1.5$) increase in perplexity due to the coarse granularity of quantization. 

\begin{figure}[h]
  \centering
  \includegraphics[width=0.7\linewidth]{sections/figures/LM7_FP7.pdf}
  \caption{\small Quantization levels and the corresponding quantization MSE of Floating Point (left) vs Lloyd-Max (right) Quantizers for a layer of weights in the GPT3-126M model.}
  \label{fig:lm_quant}
\end{figure}

\begin{table}[h]\scriptsize
\begin{center}
\caption{\label{tab:FP7_vs_LM7} \small Comparing perplexity (lower is better) achieved by floating point quantizers and Lloyd-Max quantizers on a GPT3-126M model for the Wikitext-103 dataset.}
\begin{tabular}{c|cc|c}
\hline
 \multirow{2}{*}{\textbf{Bitwidth}} & \multicolumn{2}{|c|}{\textbf{Floating-Point Quantizer}} & \textbf{Lloyd-Max Quantizer} \\
 & Best Format & Wikitext-103 Perplexity & Wikitext-103 Perplexity \\
\hline
7 & E3M3 & 18.32 & 18.27 \\
6 & E3M2 & 19.07 & 18.51 \\
5 & E4M0 & 43.89 & 19.71 \\
\hline
\end{tabular}
\end{center}
\end{table}

\subsection{Proof of Local Optimality of LO-BCQ}
\label{subsec:lobcq_opt_proof}
For a given block $\bm{b}_j$, the quantization MSE during LO-BCQ can be empirically evaluated as $\frac{1}{L_b}\lVert \bm{b}_j- \bm{\hat{b}}_j\rVert^2_2$ where $\bm{\hat{b}}_j$ is computed from equation (\ref{eq:clustered_quantization_definition}) as $C_{f(\bm{b}_j)}(\bm{b}_j)$. Further, for a given block cluster $\mathcal{B}_i$, we compute the quantization MSE as $\frac{1}{|\mathcal{B}_{i}|}\sum_{\bm{b} \in \mathcal{B}_{i}} \frac{1}{L_b}\lVert \bm{b}- C_i^{(n)}(\bm{b})\rVert^2_2$. Therefore, at the end of iteration $n$, we evaluate the overall quantization MSE $J^{(n)}$ for a given operand $\bm{X}$ composed of $N_c$ block clusters as:
\begin{align*}
    \label{eq:mse_iter_n}
    J^{(n)} = \frac{1}{N_c} \sum_{i=1}^{N_c} \frac{1}{|\mathcal{B}_{i}^{(n)}|}\sum_{\bm{v} \in \mathcal{B}_{i}^{(n)}} \frac{1}{L_b}\lVert \bm{b}- B_i^{(n)}(\bm{b})\rVert^2_2
\end{align*}

At the end of iteration $n$, the codebooks are updated from $\mathcal{C}^{(n-1)}$ to $\mathcal{C}^{(n)}$. However, the mapping of a given vector $\bm{b}_j$ to quantizers $\mathcal{C}^{(n)}$ remains as  $f^{(n)}(\bm{b}_j)$. At the next iteration, during the vector clustering step, $f^{(n+1)}(\bm{b}_j)$ finds new mapping of $\bm{b}_j$ to updated codebooks $\mathcal{C}^{(n)}$ such that the quantization MSE over the candidate codebooks is minimized. Therefore, we obtain the following result for $\bm{b}_j$:
\begin{align*}
\frac{1}{L_b}\lVert \bm{b}_j - C_{f^{(n+1)}(\bm{b}_j)}^{(n)}(\bm{b}_j)\rVert^2_2 \le \frac{1}{L_b}\lVert \bm{b}_j - C_{f^{(n)}(\bm{b}_j)}^{(n)}(\bm{b}_j)\rVert^2_2
\end{align*}

That is, quantizing $\bm{b}_j$ at the end of the block clustering step of iteration $n+1$ results in lower quantization MSE compared to quantizing at the end of iteration $n$. Since this is true for all $\bm{b} \in \bm{X}$, we assert the following:
\begin{equation}
\begin{split}
\label{eq:mse_ineq_1}
    \tilde{J}^{(n+1)} &= \frac{1}{N_c} \sum_{i=1}^{N_c} \frac{1}{|\mathcal{B}_{i}^{(n+1)}|}\sum_{\bm{b} \in \mathcal{B}_{i}^{(n+1)}} \frac{1}{L_b}\lVert \bm{b} - C_i^{(n)}(b)\rVert^2_2 \le J^{(n)}
\end{split}
\end{equation}
where $\tilde{J}^{(n+1)}$ is the the quantization MSE after the vector clustering step at iteration $n+1$.

Next, during the codebook update step (\ref{eq:quantizers_update}) at iteration $n+1$, the per-cluster codebooks $\mathcal{C}^{(n)}$ are updated to $\mathcal{C}^{(n+1)}$ by invoking the Lloyd-Max algorithm \citep{Lloyd}. We know that for any given value distribution, the Lloyd-Max algorithm minimizes the quantization MSE. Therefore, for a given vector cluster $\mathcal{B}_i$ we obtain the following result:

\begin{equation}
    \frac{1}{|\mathcal{B}_{i}^{(n+1)}|}\sum_{\bm{b} \in \mathcal{B}_{i}^{(n+1)}} \frac{1}{L_b}\lVert \bm{b}- C_i^{(n+1)}(\bm{b})\rVert^2_2 \le \frac{1}{|\mathcal{B}_{i}^{(n+1)}|}\sum_{\bm{b} \in \mathcal{B}_{i}^{(n+1)}} \frac{1}{L_b}\lVert \bm{b}- C_i^{(n)}(\bm{b})\rVert^2_2
\end{equation}

The above equation states that quantizing the given block cluster $\mathcal{B}_i$ after updating the associated codebook from $C_i^{(n)}$ to $C_i^{(n+1)}$ results in lower quantization MSE. Since this is true for all the block clusters, we derive the following result: 
\begin{equation}
\begin{split}
\label{eq:mse_ineq_2}
     J^{(n+1)} &= \frac{1}{N_c} \sum_{i=1}^{N_c} \frac{1}{|\mathcal{B}_{i}^{(n+1)}|}\sum_{\bm{b} \in \mathcal{B}_{i}^{(n+1)}} \frac{1}{L_b}\lVert \bm{b}- C_i^{(n+1)}(\bm{b})\rVert^2_2  \le \tilde{J}^{(n+1)}   
\end{split}
\end{equation}

Following (\ref{eq:mse_ineq_1}) and (\ref{eq:mse_ineq_2}), we find that the quantization MSE is non-increasing for each iteration, that is, $J^{(1)} \ge J^{(2)} \ge J^{(3)} \ge \ldots \ge J^{(M)}$ where $M$ is the maximum number of iterations. 
%Therefore, we can say that if the algorithm converges, then it must be that it has converged to a local minimum. 
\hfill $\blacksquare$


\begin{figure}
    \begin{center}
    \includegraphics[width=0.5\textwidth]{sections//figures/mse_vs_iter.pdf}
    \end{center}
    \caption{\small NMSE vs iterations during LO-BCQ compared to other block quantization proposals}
    \label{fig:nmse_vs_iter}
\end{figure}

Figure \ref{fig:nmse_vs_iter} shows the empirical convergence of LO-BCQ across several block lengths and number of codebooks. Also, the MSE achieved by LO-BCQ is compared to baselines such as MXFP and VSQ. As shown, LO-BCQ converges to a lower MSE than the baselines. Further, we achieve better convergence for larger number of codebooks ($N_c$) and for a smaller block length ($L_b$), both of which increase the bitwidth of BCQ (see Eq \ref{eq:bitwidth_bcq}).


\subsection{Additional Accuracy Results}
%Table \ref{tab:lobcq_config} lists the various LOBCQ configurations and their corresponding bitwidths.
\begin{table}
\setlength{\tabcolsep}{4.75pt}
\begin{center}
\caption{\label{tab:lobcq_config} Various LO-BCQ configurations and their bitwidths.}
\begin{tabular}{|c||c|c|c|c||c|c||c|} 
\hline
 & \multicolumn{4}{|c||}{$L_b=8$} & \multicolumn{2}{|c||}{$L_b=4$} & $L_b=2$ \\
 \hline
 \backslashbox{$L_A$\kern-1em}{\kern-1em$N_c$} & 2 & 4 & 8 & 16 & 2 & 4 & 2 \\
 \hline
 64 & 4.25 & 4.375 & 4.5 & 4.625 & 4.375 & 4.625 & 4.625\\
 \hline
 32 & 4.375 & 4.5 & 4.625& 4.75 & 4.5 & 4.75 & 4.75 \\
 \hline
 16 & 4.625 & 4.75& 4.875 & 5 & 4.75 & 5 & 5 \\
 \hline
\end{tabular}
\end{center}
\end{table}

%\subsection{Perplexity achieved by various LO-BCQ configurations on Wikitext-103 dataset}

\begin{table} \centering
\begin{tabular}{|c||c|c|c|c||c|c||c|} 
\hline
 $L_b \rightarrow$& \multicolumn{4}{c||}{8} & \multicolumn{2}{c||}{4} & 2\\
 \hline
 \backslashbox{$L_A$\kern-1em}{\kern-1em$N_c$} & 2 & 4 & 8 & 16 & 2 & 4 & 2  \\
 %$N_c \rightarrow$ & 2 & 4 & 8 & 16 & 2 & 4 & 2 \\
 \hline
 \hline
 \multicolumn{8}{c}{GPT3-1.3B (FP32 PPL = 9.98)} \\ 
 \hline
 \hline
 64 & 10.40 & 10.23 & 10.17 & 10.15 &  10.28 & 10.18 & 10.19 \\
 \hline
 32 & 10.25 & 10.20 & 10.15 & 10.12 &  10.23 & 10.17 & 10.17 \\
 \hline
 16 & 10.22 & 10.16 & 10.10 & 10.09 &  10.21 & 10.14 & 10.16 \\
 \hline
  \hline
 \multicolumn{8}{c}{GPT3-8B (FP32 PPL = 7.38)} \\ 
 \hline
 \hline
 64 & 7.61 & 7.52 & 7.48 &  7.47 &  7.55 &  7.49 & 7.50 \\
 \hline
 32 & 7.52 & 7.50 & 7.46 &  7.45 &  7.52 &  7.48 & 7.48  \\
 \hline
 16 & 7.51 & 7.48 & 7.44 &  7.44 &  7.51 &  7.49 & 7.47  \\
 \hline
\end{tabular}
\caption{\label{tab:ppl_gpt3_abalation} Wikitext-103 perplexity across GPT3-1.3B and 8B models.}
\end{table}

\begin{table} \centering
\begin{tabular}{|c||c|c|c|c||} 
\hline
 $L_b \rightarrow$& \multicolumn{4}{c||}{8}\\
 \hline
 \backslashbox{$L_A$\kern-1em}{\kern-1em$N_c$} & 2 & 4 & 8 & 16 \\
 %$N_c \rightarrow$ & 2 & 4 & 8 & 16 & 2 & 4 & 2 \\
 \hline
 \hline
 \multicolumn{5}{|c|}{Llama2-7B (FP32 PPL = 5.06)} \\ 
 \hline
 \hline
 64 & 5.31 & 5.26 & 5.19 & 5.18  \\
 \hline
 32 & 5.23 & 5.25 & 5.18 & 5.15  \\
 \hline
 16 & 5.23 & 5.19 & 5.16 & 5.14  \\
 \hline
 \multicolumn{5}{|c|}{Nemotron4-15B (FP32 PPL = 5.87)} \\ 
 \hline
 \hline
 64  & 6.3 & 6.20 & 6.13 & 6.08  \\
 \hline
 32  & 6.24 & 6.12 & 6.07 & 6.03  \\
 \hline
 16  & 6.12 & 6.14 & 6.04 & 6.02  \\
 \hline
 \multicolumn{5}{|c|}{Nemotron4-340B (FP32 PPL = 3.48)} \\ 
 \hline
 \hline
 64 & 3.67 & 3.62 & 3.60 & 3.59 \\
 \hline
 32 & 3.63 & 3.61 & 3.59 & 3.56 \\
 \hline
 16 & 3.61 & 3.58 & 3.57 & 3.55 \\
 \hline
\end{tabular}
\caption{\label{tab:ppl_llama7B_nemo15B} Wikitext-103 perplexity compared to FP32 baseline in Llama2-7B and Nemotron4-15B, 340B models}
\end{table}

%\subsection{Perplexity achieved by various LO-BCQ configurations on MMLU dataset}


\begin{table} \centering
\begin{tabular}{|c||c|c|c|c||c|c|c|c|} 
\hline
 $L_b \rightarrow$& \multicolumn{4}{c||}{8} & \multicolumn{4}{c||}{8}\\
 \hline
 \backslashbox{$L_A$\kern-1em}{\kern-1em$N_c$} & 2 & 4 & 8 & 16 & 2 & 4 & 8 & 16  \\
 %$N_c \rightarrow$ & 2 & 4 & 8 & 16 & 2 & 4 & 2 \\
 \hline
 \hline
 \multicolumn{5}{|c|}{Llama2-7B (FP32 Accuracy = 45.8\%)} & \multicolumn{4}{|c|}{Llama2-70B (FP32 Accuracy = 69.12\%)} \\ 
 \hline
 \hline
 64 & 43.9 & 43.4 & 43.9 & 44.9 & 68.07 & 68.27 & 68.17 & 68.75 \\
 \hline
 32 & 44.5 & 43.8 & 44.9 & 44.5 & 68.37 & 68.51 & 68.35 & 68.27  \\
 \hline
 16 & 43.9 & 42.7 & 44.9 & 45 & 68.12 & 68.77 & 68.31 & 68.59  \\
 \hline
 \hline
 \multicolumn{5}{|c|}{GPT3-22B (FP32 Accuracy = 38.75\%)} & \multicolumn{4}{|c|}{Nemotron4-15B (FP32 Accuracy = 64.3\%)} \\ 
 \hline
 \hline
 64 & 36.71 & 38.85 & 38.13 & 38.92 & 63.17 & 62.36 & 63.72 & 64.09 \\
 \hline
 32 & 37.95 & 38.69 & 39.45 & 38.34 & 64.05 & 62.30 & 63.8 & 64.33  \\
 \hline
 16 & 38.88 & 38.80 & 38.31 & 38.92 & 63.22 & 63.51 & 63.93 & 64.43  \\
 \hline
\end{tabular}
\caption{\label{tab:mmlu_abalation} Accuracy on MMLU dataset across GPT3-22B, Llama2-7B, 70B and Nemotron4-15B models.}
\end{table}


%\subsection{Perplexity achieved by various LO-BCQ configurations on LM evaluation harness}

\begin{table} \centering
\begin{tabular}{|c||c|c|c|c||c|c|c|c|} 
\hline
 $L_b \rightarrow$& \multicolumn{4}{c||}{8} & \multicolumn{4}{c||}{8}\\
 \hline
 \backslashbox{$L_A$\kern-1em}{\kern-1em$N_c$} & 2 & 4 & 8 & 16 & 2 & 4 & 8 & 16  \\
 %$N_c \rightarrow$ & 2 & 4 & 8 & 16 & 2 & 4 & 2 \\
 \hline
 \hline
 \multicolumn{5}{|c|}{Race (FP32 Accuracy = 37.51\%)} & \multicolumn{4}{|c|}{Boolq (FP32 Accuracy = 64.62\%)} \\ 
 \hline
 \hline
 64 & 36.94 & 37.13 & 36.27 & 37.13 & 63.73 & 62.26 & 63.49 & 63.36 \\
 \hline
 32 & 37.03 & 36.36 & 36.08 & 37.03 & 62.54 & 63.51 & 63.49 & 63.55  \\
 \hline
 16 & 37.03 & 37.03 & 36.46 & 37.03 & 61.1 & 63.79 & 63.58 & 63.33  \\
 \hline
 \hline
 \multicolumn{5}{|c|}{Winogrande (FP32 Accuracy = 58.01\%)} & \multicolumn{4}{|c|}{Piqa (FP32 Accuracy = 74.21\%)} \\ 
 \hline
 \hline
 64 & 58.17 & 57.22 & 57.85 & 58.33 & 73.01 & 73.07 & 73.07 & 72.80 \\
 \hline
 32 & 59.12 & 58.09 & 57.85 & 58.41 & 73.01 & 73.94 & 72.74 & 73.18  \\
 \hline
 16 & 57.93 & 58.88 & 57.93 & 58.56 & 73.94 & 72.80 & 73.01 & 73.94  \\
 \hline
\end{tabular}
\caption{\label{tab:mmlu_abalation} Accuracy on LM evaluation harness tasks on GPT3-1.3B model.}
\end{table}

\begin{table} \centering
\begin{tabular}{|c||c|c|c|c||c|c|c|c|} 
\hline
 $L_b \rightarrow$& \multicolumn{4}{c||}{8} & \multicolumn{4}{c||}{8}\\
 \hline
 \backslashbox{$L_A$\kern-1em}{\kern-1em$N_c$} & 2 & 4 & 8 & 16 & 2 & 4 & 8 & 16  \\
 %$N_c \rightarrow$ & 2 & 4 & 8 & 16 & 2 & 4 & 2 \\
 \hline
 \hline
 \multicolumn{5}{|c|}{Race (FP32 Accuracy = 41.34\%)} & \multicolumn{4}{|c|}{Boolq (FP32 Accuracy = 68.32\%)} \\ 
 \hline
 \hline
 64 & 40.48 & 40.10 & 39.43 & 39.90 & 69.20 & 68.41 & 69.45 & 68.56 \\
 \hline
 32 & 39.52 & 39.52 & 40.77 & 39.62 & 68.32 & 67.43 & 68.17 & 69.30  \\
 \hline
 16 & 39.81 & 39.71 & 39.90 & 40.38 & 68.10 & 66.33 & 69.51 & 69.42  \\
 \hline
 \hline
 \multicolumn{5}{|c|}{Winogrande (FP32 Accuracy = 67.88\%)} & \multicolumn{4}{|c|}{Piqa (FP32 Accuracy = 78.78\%)} \\ 
 \hline
 \hline
 64 & 66.85 & 66.61 & 67.72 & 67.88 & 77.31 & 77.42 & 77.75 & 77.64 \\
 \hline
 32 & 67.25 & 67.72 & 67.72 & 67.00 & 77.31 & 77.04 & 77.80 & 77.37  \\
 \hline
 16 & 68.11 & 68.90 & 67.88 & 67.48 & 77.37 & 78.13 & 78.13 & 77.69  \\
 \hline
\end{tabular}
\caption{\label{tab:mmlu_abalation} Accuracy on LM evaluation harness tasks on GPT3-8B model.}
\end{table}

\begin{table} \centering
\begin{tabular}{|c||c|c|c|c||c|c|c|c|} 
\hline
 $L_b \rightarrow$& \multicolumn{4}{c||}{8} & \multicolumn{4}{c||}{8}\\
 \hline
 \backslashbox{$L_A$\kern-1em}{\kern-1em$N_c$} & 2 & 4 & 8 & 16 & 2 & 4 & 8 & 16  \\
 %$N_c \rightarrow$ & 2 & 4 & 8 & 16 & 2 & 4 & 2 \\
 \hline
 \hline
 \multicolumn{5}{|c|}{Race (FP32 Accuracy = 40.67\%)} & \multicolumn{4}{|c|}{Boolq (FP32 Accuracy = 76.54\%)} \\ 
 \hline
 \hline
 64 & 40.48 & 40.10 & 39.43 & 39.90 & 75.41 & 75.11 & 77.09 & 75.66 \\
 \hline
 32 & 39.52 & 39.52 & 40.77 & 39.62 & 76.02 & 76.02 & 75.96 & 75.35  \\
 \hline
 16 & 39.81 & 39.71 & 39.90 & 40.38 & 75.05 & 73.82 & 75.72 & 76.09  \\
 \hline
 \hline
 \multicolumn{5}{|c|}{Winogrande (FP32 Accuracy = 70.64\%)} & \multicolumn{4}{|c|}{Piqa (FP32 Accuracy = 79.16\%)} \\ 
 \hline
 \hline
 64 & 69.14 & 70.17 & 70.17 & 70.56 & 78.24 & 79.00 & 78.62 & 78.73 \\
 \hline
 32 & 70.96 & 69.69 & 71.27 & 69.30 & 78.56 & 79.49 & 79.16 & 78.89  \\
 \hline
 16 & 71.03 & 69.53 & 69.69 & 70.40 & 78.13 & 79.16 & 79.00 & 79.00  \\
 \hline
\end{tabular}
\caption{\label{tab:mmlu_abalation} Accuracy on LM evaluation harness tasks on GPT3-22B model.}
\end{table}

\begin{table} \centering
\begin{tabular}{|c||c|c|c|c||c|c|c|c|} 
\hline
 $L_b \rightarrow$& \multicolumn{4}{c||}{8} & \multicolumn{4}{c||}{8}\\
 \hline
 \backslashbox{$L_A$\kern-1em}{\kern-1em$N_c$} & 2 & 4 & 8 & 16 & 2 & 4 & 8 & 16  \\
 %$N_c \rightarrow$ & 2 & 4 & 8 & 16 & 2 & 4 & 2 \\
 \hline
 \hline
 \multicolumn{5}{|c|}{Race (FP32 Accuracy = 44.4\%)} & \multicolumn{4}{|c|}{Boolq (FP32 Accuracy = 79.29\%)} \\ 
 \hline
 \hline
 64 & 42.49 & 42.51 & 42.58 & 43.45 & 77.58 & 77.37 & 77.43 & 78.1 \\
 \hline
 32 & 43.35 & 42.49 & 43.64 & 43.73 & 77.86 & 75.32 & 77.28 & 77.86  \\
 \hline
 16 & 44.21 & 44.21 & 43.64 & 42.97 & 78.65 & 77 & 76.94 & 77.98  \\
 \hline
 \hline
 \multicolumn{5}{|c|}{Winogrande (FP32 Accuracy = 69.38\%)} & \multicolumn{4}{|c|}{Piqa (FP32 Accuracy = 78.07\%)} \\ 
 \hline
 \hline
 64 & 68.9 & 68.43 & 69.77 & 68.19 & 77.09 & 76.82 & 77.09 & 77.86 \\
 \hline
 32 & 69.38 & 68.51 & 68.82 & 68.90 & 78.07 & 76.71 & 78.07 & 77.86  \\
 \hline
 16 & 69.53 & 67.09 & 69.38 & 68.90 & 77.37 & 77.8 & 77.91 & 77.69  \\
 \hline
\end{tabular}
\caption{\label{tab:mmlu_abalation} Accuracy on LM evaluation harness tasks on Llama2-7B model.}
\end{table}

\begin{table} \centering
\begin{tabular}{|c||c|c|c|c||c|c|c|c|} 
\hline
 $L_b \rightarrow$& \multicolumn{4}{c||}{8} & \multicolumn{4}{c||}{8}\\
 \hline
 \backslashbox{$L_A$\kern-1em}{\kern-1em$N_c$} & 2 & 4 & 8 & 16 & 2 & 4 & 8 & 16  \\
 %$N_c \rightarrow$ & 2 & 4 & 8 & 16 & 2 & 4 & 2 \\
 \hline
 \hline
 \multicolumn{5}{|c|}{Race (FP32 Accuracy = 48.8\%)} & \multicolumn{4}{|c|}{Boolq (FP32 Accuracy = 85.23\%)} \\ 
 \hline
 \hline
 64 & 49.00 & 49.00 & 49.28 & 48.71 & 82.82 & 84.28 & 84.03 & 84.25 \\
 \hline
 32 & 49.57 & 48.52 & 48.33 & 49.28 & 83.85 & 84.46 & 84.31 & 84.93  \\
 \hline
 16 & 49.85 & 49.09 & 49.28 & 48.99 & 85.11 & 84.46 & 84.61 & 83.94  \\
 \hline
 \hline
 \multicolumn{5}{|c|}{Winogrande (FP32 Accuracy = 79.95\%)} & \multicolumn{4}{|c|}{Piqa (FP32 Accuracy = 81.56\%)} \\ 
 \hline
 \hline
 64 & 78.77 & 78.45 & 78.37 & 79.16 & 81.45 & 80.69 & 81.45 & 81.5 \\
 \hline
 32 & 78.45 & 79.01 & 78.69 & 80.66 & 81.56 & 80.58 & 81.18 & 81.34  \\
 \hline
 16 & 79.95 & 79.56 & 79.79 & 79.72 & 81.28 & 81.66 & 81.28 & 80.96  \\
 \hline
\end{tabular}
\caption{\label{tab:mmlu_abalation} Accuracy on LM evaluation harness tasks on Llama2-70B model.}
\end{table}

%\section{MSE Studies}
%\textcolor{red}{TODO}


\subsection{Number Formats and Quantization Method}
\label{subsec:numFormats_quantMethod}
\subsubsection{Integer Format}
An $n$-bit signed integer (INT) is typically represented with a 2s-complement format \citep{yao2022zeroquant,xiao2023smoothquant,dai2021vsq}, where the most significant bit denotes the sign.

\subsubsection{Floating Point Format}
An $n$-bit signed floating point (FP) number $x$ comprises of a 1-bit sign ($x_{\mathrm{sign}}$), $B_m$-bit mantissa ($x_{\mathrm{mant}}$) and $B_e$-bit exponent ($x_{\mathrm{exp}}$) such that $B_m+B_e=n-1$. The associated constant exponent bias ($E_{\mathrm{bias}}$) is computed as $(2^{{B_e}-1}-1)$. We denote this format as $E_{B_e}M_{B_m}$.  

\subsubsection{Quantization Scheme}
\label{subsec:quant_method}
A quantization scheme dictates how a given unquantized tensor is converted to its quantized representation. We consider FP formats for the purpose of illustration. Given an unquantized tensor $\bm{X}$ and an FP format $E_{B_e}M_{B_m}$, we first, we compute the quantization scale factor $s_X$ that maps the maximum absolute value of $\bm{X}$ to the maximum quantization level of the $E_{B_e}M_{B_m}$ format as follows:
\begin{align}
\label{eq:sf}
    s_X = \frac{\mathrm{max}(|\bm{X}|)}{\mathrm{max}(E_{B_e}M_{B_m})}
\end{align}
In the above equation, $|\cdot|$ denotes the absolute value function.

Next, we scale $\bm{X}$ by $s_X$ and quantize it to $\hat{\bm{X}}$ by rounding it to the nearest quantization level of $E_{B_e}M_{B_m}$ as:

\begin{align}
\label{eq:tensor_quant}
    \hat{\bm{X}} = \text{round-to-nearest}\left(\frac{\bm{X}}{s_X}, E_{B_e}M_{B_m}\right)
\end{align}

We perform dynamic max-scaled quantization \citep{wu2020integer}, where the scale factor $s$ for activations is dynamically computed during runtime.

\subsection{Vector Scaled Quantization}
\begin{wrapfigure}{r}{0.35\linewidth}
  \centering
  \includegraphics[width=\linewidth]{sections/figures/vsquant.jpg}
  \caption{\small Vectorwise decomposition for per-vector scaled quantization (VSQ \citep{dai2021vsq}).}
  \label{fig:vsquant}
\end{wrapfigure}
During VSQ \citep{dai2021vsq}, the operand tensors are decomposed into 1D vectors in a hardware friendly manner as shown in Figure \ref{fig:vsquant}. Since the decomposed tensors are used as operands in matrix multiplications during inference, it is beneficial to perform this decomposition along the reduction dimension of the multiplication. The vectorwise quantization is performed similar to tensorwise quantization described in Equations \ref{eq:sf} and \ref{eq:tensor_quant}, where a scale factor $s_v$ is required for each vector $\bm{v}$ that maps the maximum absolute value of that vector to the maximum quantization level. While smaller vector lengths can lead to larger accuracy gains, the associated memory and computational overheads due to the per-vector scale factors increases. To alleviate these overheads, VSQ \citep{dai2021vsq} proposed a second level quantization of the per-vector scale factors to unsigned integers, while MX \citep{rouhani2023shared} quantizes them to integer powers of 2 (denoted as $2^{INT}$).

\subsubsection{MX Format}
The MX format proposed in \citep{rouhani2023microscaling} introduces the concept of sub-block shifting. For every two scalar elements of $b$-bits each, there is a shared exponent bit. The value of this exponent bit is determined through an empirical analysis that targets minimizing quantization MSE. We note that the FP format $E_{1}M_{b}$ is strictly better than MX from an accuracy perspective since it allocates a dedicated exponent bit to each scalar as opposed to sharing it across two scalars. Therefore, we conservatively bound the accuracy of a $b+2$-bit signed MX format with that of a $E_{1}M_{b}$ format in our comparisons. For instance, we use E1M2 format as a proxy for MX4.

\begin{figure}
    \centering
    \includegraphics[width=1\linewidth]{sections//figures/BlockFormats.pdf}
    \caption{\small Comparing LO-BCQ to MX format.}
    \label{fig:block_formats}
\end{figure}

Figure \ref{fig:block_formats} compares our $4$-bit LO-BCQ block format to MX \citep{rouhani2023microscaling}. As shown, both LO-BCQ and MX decompose a given operand tensor into block arrays and each block array into blocks. Similar to MX, we find that per-block quantization ($L_b < L_A$) leads to better accuracy due to increased flexibility. While MX achieves this through per-block $1$-bit micro-scales, we associate a dedicated codebook to each block through a per-block codebook selector. Further, MX quantizes the per-block array scale-factor to E8M0 format without per-tensor scaling. In contrast during LO-BCQ, we find that per-tensor scaling combined with quantization of per-block array scale-factor to E4M3 format results in superior inference accuracy across models. 


\end{document}

