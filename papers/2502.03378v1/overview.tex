\vspace{-5pt}
\section{LOV Overview} \label{sec:lov_overview}
Figure \ref{fig:lov_overview} illustrates an overview of \lov. The core goal of \lov\ is to provide network operators enforcing ROV with a high-quality whitelist, preventing benign conflicts from being filtered out while maintaining ROV's effectiveness in preventing hijacks.
\lov\ operates in the following four steps, outlined below:

\noindent\textit{1) RPKI's ROV: Validates BGP announcements.}
\lov\ utilizes BGPStream \cite{orsini2016bgpstream} to collect live BGP updates from RouteViews and RIS route collectors\footnote{As of writing, there are 39 RouteViews and 26 RIS route collectors.}, and then validate them using RPKI's ROV.

\noindent\textit{2) ML-classifier: Classifies RPKI-invalid routes.}
Legitimate BGP announcements with benign origin conflicts are mistakenly categorized as invalid routes by RPKI's ROV.
\lov\ resolves this issue by employing an ML-based classifier (denoted as ML-classifier) to re-validate the invalid routes.
The ML classifier categorizes an RPKI-invalid route as either a benign conflict or a BGP hijack.
The classification model is trained using a list of features that illustrate the connections between two conflicting origins present in BGP announcements and ROAs (refer to Section \ref{sec:lov_features} for more details). The feature computation relies on public data, including AS relationships, AS organizations, AS geolocations, and AS hegemony values.
\lov\ downloads the necessary data from reliable sources, such as CAIDA \cite{CAIDA}, IHR \cite{IHR}, and IRR \cite{IRR}, and stores it in a local database, ensuring readiness for the ML-based classification. Daily updates are performed to keep the data current.
The features used should enable the ML classifier to effectively identify benign conflicts without affecting ROV's performance in detecting actual hijacks.
In other words, the ML classifier prioritizes the accurate identification of hijacks, even if it means that some benign conflicts may be misclassified as hijacks.

\begin{figure}[t!]
\centerline{\includegraphics[width=0.8\linewidth]{./images/LOV.new.pdf}}
\vspace{-10pt}
\caption{\small{The overview of \lov.}}
\label{fig:lov_overview}
\end{figure}

\noindent\textit{3) Post-analyzer: Verifies potential hijacks.}
Since the ML classifier is designed to favor the detection of hijacks, it may sometimes mistakenly classify certain benign conflicts included in invalid routes as hijacks. To address these potential misclassifications, \lov\ employs a post-analyzer to further verify the suspected hijacks.
The post-analyzer examines changes in global AS visibility to identify patterns indicative of BGP hijacking events.
This approach can verify whether the suspected routes indeed originate from BGP incidents or not (refer to Section \ref{subsec:lov_post_analyzer} for more details).
As illustrated in Figure \ref{fig:lov_overview}, the hijacks detected by the ML-classifier are forwarded to the post-analyzer.
Verified routes are recorded as potential BGP hijacking incidents for further analysis, contributing to the enhancement of BGP security. Routes that cannot be confirmed as hijacked might be innocent and are quarantined for further review. 

\noindent\textit{4) Quarantine mechanism: generates a trusted whitelist.}
To ensure the quality of the whitelist, benign conflicts identified by the ML-classifier and unverified routes flagged by the post-analyzer are not immediately added to the whitelist but are instead isolated for further scrutiny, as depicted by ``Quarantine'' in Figure \ref{fig:lov_overview}.
Routes in quarantine are reviewed using various whitelisting strategies, including examining the tightness level between two conflicting origins, continuous behavior monitoring, and human intervention.
The analysis of hijacking events is primarily conducted through email surveys, which provide evidence (e.g., email confirmations) to security analysts for potential human intervention (refer to Section \ref{subsec:quarantine} for more details).

To streamline the whitelist, we extract the originating AS and the announced prefix from the route, creating a unique entry (i.e., a pair of AS and prefix) in the whitelist.
\lov\ updates the whitelist daily. New entries are automatically added to the whitelist, and concurrently, old entries that have not been seen for longer than a month, or that have become RPKI-valid or unknown are purged.
Furthermore, \lov\ provides APIs that allow users to access the whitelist.