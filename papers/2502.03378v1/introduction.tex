\section{Introduction} \label{sec:lov_overview}
Border Gateway Protocol (BGP) \cite{rfc4271} serves as the inter-domain routing system of the Internet, enabling routers to exchange information about the reachability of prefixes within their Autonomous Systems (ASes). However, ASes can originate prefixes that they do not legitimately own, leading to routing misconfigurations or intentional hijacks. Such incidents allow attackers to redirect traffic through their networks, facilitating eavesdropping, blackholing of hijacked traffic, or dissemination of spam \cite{ballani2007study,vervier2015mind,vervier2013spamtracer}.

\textbf{RPKI to secure Inter-domain routing.}
To combat these issues, the Internet Engineering Task Force (IETF) has standardized the Resource Public Key Infrastructure (RPKI) \cite{RPKI}, which binds prefixes to AS numbers (ASNs) and their corresponding cryptographic public keys via Route Origin Authorizations (ROAs).
Route Origin Validation (ROV) \cite{huston2012validation} validates BGP announcements by ensuring that the origin AS matches the authoritative AS specified in the ROAs and that the prefix length is within the allowed range.
By enforcing ROV, ASes can filter out invalid BGP announcements, thereby preventing the propagation of unauthorized routes.

Qrator Labs recently reported that the number of BGP hijack incidents declined from 9,022 in Q2 2022 to 7,595 in Q2 2023 \cite{Qrator}. They credited this drop to the growing adoption of RPKI among ASes.
Despite experiencing some growth, the deployment of RPKI is advancing at a slow pace.
As of January 2024, only 33\% of 413 major network operators have implemented ROV to prevent illegitimate BGP routes \cite{cloudflare}. Li et al. \cite{li2023rovista} discovered in 2023 that merely 12.3\% of ASes have fully employed ROV.

\textbf{Obstacles to RPKI adoption.}
The limited adoption of ROV poses a potential risk of collateral damage to networks currently deploying this technology, preventing them from obtaining sufficient security benefits \cite{gilad2016we,hlavacek2020disco,morillo2021rov++,ixp:HlavacekSVW23}. %This, in turn, may lead networks that have initially embraced ROV to reconsider its ongoing use, probably impacting the widespread RPKI deployment.
One of the reasons behind the hesitation to enforce ROV is the concern about the loss of legitimate traffic due to misconfigurations in ROAs \cite{hlavacek2020disco,hlavacek2022smart}. Manually issuing ROAs often leads to errors in the ROAs \cite{hlavacek2020disco}. For instance, network operators may incorrectly configure the maximum length of prefixes. These erroneous ROAs conflict with legitimate BGP announcements and result in an invalid ROV validation status, and consequently, the corresponding legitimate BGP announcements are blocked by ROV-filtering networks.
Such an origin conflict is known as ``benign conflict'' since it is caused by misconfigurations and not by actual hijacks. Filtering benign conflicts due to erroneous ROAs adversely affects the reachability of the prefixes covered by those ROAs.
According to the NIST RPKI monitor \cite{nist2024}, the quantity of erroneous ROAs grows proportionally as the number of prefixes covered by ROAs increases.
As the enforcement of ROV filtering continues to rise among networks, the impact of benign conflicts on network reachability will exacerbate.
Since BGP operations are driven by economic incentives, the loss of legitimate traffic can directly result in a loss of revenue.

\textbf{Improving ROV implementation to promote RPKI deployment.}
Previous studies, such as \cite{hlavacek2020disco,hlavacek2022smart}, primarily focus on errors in ROAs.
DISCO \cite{hlavacek2020disco} explores methods for mitigating human errors in ROA issuance, while SROV \cite{hlavacek2022smart} delves into identifying benign conflicts resulting from erroneous ROAs. However, our study reveals that intricate routing policies or their changes—such as multiple origin ASes (MOAS), routing shifts, and traffic engineering—can also lead to benign conflicts. These conflicts are often not due to ROA errors, which present additional challenges for resolution.
%In this work, we expand the concept of benign conflicts to include benign ROA conflicts that are not attributable to ROA errors.

In our measurements, benign conflicts are observed to be both persistent and widespread compared to hijacks, with thousands of routes potentially affected daily, persisting over weeks or even months.
Our study reveals that approximately 79\% of the ROV-invalid routes per day are attributed to benign conflicts with ROAs, with 26K unique benign conflicts recurring within an average span of less than 14 days. 
Sacrificing a significant volume of legitimate traffic to prevent a relatively smaller number of hijacks would be both impractical and detrimental to overall network operations.

SROV, as an extension of ROV, using a heuristic approach based on route duration to analyze BGP announcements, may introduce BGP convergence delays.
Additionally, this approach has a high rate of false negatives (i.e., misidentifying hijacks as benign routes), potentially leading to failures in preventing actual BGP hijacks. In practical terms, striking a balance in ROV implementation—ensuring security and maintaining network performance—is essential.
As a result, new mechanisms need to be explored to resolve these issues.
Our work addresses this balance by preventing the discard of BGP announcements with benign conflicts while still upholding the intended function of ROV. By achieving this, our work has the potential to support the broader adoption of RPKI and its ROV.

\textbf{Learning Origin Validation (\lov).}
This work introduces Learning Origin Validation (\lov), designed for whitelisting benign conflicts on an Internet-wide scale. The generated whitelist is offered to the ASes that employ ROV to validate RPKI-invalid routes. \lov\ matches the RPKI-invalid routes against the whitelist. If a match is found, the route is considered benign, and border routers refrain from blocking it.

Notably, we utilize a whitelist of benign conflicts rather than a blacklist of hijacks, as ROV is already capable of perfectly preventing hijacks involving origins that conflict with ROAs.
Additionally, the transient and rapidly changing nature of anomalous routes makes maintaining an up-to-date blacklist challenging.
%Question: A Web-based allowlist might require time to be synchronized onto routers, which may void the usefulness of the model as most hijacks are short-lived.
The whitelist targets benign conflicts, which are typically long-lived, thereby minimizing the synchronization issues with routers, that often arise when using a blacklist to address short-lived BGP hijacks.
Consequently, using a whitelist provides a more effective and reliable approach.

In contrast to SROV, \lov\ aims to provide a \textit{high-quality} whitelist for networks enforcing ROV. To achieve this, \lov\ incorporates three key mechanisms: a \textit{ML-classifier} that leverages machine learning to identify benign conflicts in the initial stage; a \textit{Post-analyzer}, which monitors changes in the global visibility of origin ASes to verify potential hijacks detected by the ML-classifier, addressing misidentifications; and a \textit{Quarantine} mechanism, which ensures the reliability and trustworthiness of the whitelist through multiple review strategies. An overview of \lov\ is offered in Section \ref{sec:lov_overview}.

\lov\ is designed to identify various types of benign conflicts without assuming human error or attempting to eliminate them. It operates on a public web server, regularly providing a whitelist to ASes that enforce ROV, without requiring significant changes to existing ROV deployments. The process of downloading, installing, and maintaining this whitelist is straightforward and incurs minimal computational overhead compared to potential ROV extensions like SROV. As a result, this approach helps avoid a negative impact on BGP convergence.

{\bf Contributions.} We make the following contributions:

We introduce \lov, a new mechanism designed to offer whitelisted benign conflicts to ROV-enforcing ASes, preventing legitimate announcements involving benign conflicts from loss (Section \ref{sec:lov_overview}).

We present features utilized to categorize RPKI-invalid routes into benign conflicts and BGP hijacks (Section \ref{sec:lov_features}).

We employ five ML models to implement the binary classifier, demonstrating that the Random Forest model achieves optimal performance with 95\% accuracy in identifying benign conflicts and $\sim$100\% accuracy in detecting BGP hijacks (Section \ref{subsec:groundtruth}).

We conduct an extensive measurement of live BGP data using \lov\ over a six-month period, ultimately whitelisting 52,846 benign conflicts on the Internet, and provide insights into the underlying causes of benign conflicts, presenting four additional factors, aside from human errors (Section \ref{sec:measure}).

We address key questions regarding the implementation and deployment of \lov\ (Section \ref{sec:questions}), and discuss future research questions, potential improvements, and prospective uses (Section \ref{sec:lov_future}).

%We offer confirmation of hijacking events through email surveys, showcasing the root causes of these events (Section \ref{sec:}).




\ignore{
 \paragraph{Organization} The rest of the paper is organized as follows. Section \ref{sec:background} introduces background and preliminaries. Related work is presented in Section \ref{sec:relatedwork}.
 Section \ref{sec:features} explains the features used for distinguishing between benign conflicts and BGP hijacks. Section \ref{sec:data_sources} describes data sources we use for feature computation. In Section \ref{sec:implementation} we implement \lov\
 and validate its performance. A long-period live measurement is presented in Section \ref{sec:measure}. We compare with BGPmon in Section \ref{sec:compare} and discuss \lov\ in Section \ref{sec:discussion}. Finally, we conclude in Section \ref{sec:conclusion}.
 }

%our contribution is twofold: on the one hand, we perform a measurement study of the causes for errors and misconfigurations in ROAs and we measure the extent of different types of errors in the Internet; on the other hand, we develop the first mechanism for differentiating benign conflicts from hijacks. Our mechanism resolves one of the main obstacles towards wide adoption of ROV filtering in the Internet.
%Our goal is to mitigate the impact of benign conflicts caused by ROA misconfigurations, which are currently filtered by ROV, leading to filtering of legitimate traffic. By preventing such conflicts from being filtered, we aim to overcome a major obstacle to the deployment of ROV and promote its widespread adoption.
\ignore{
Our individual technical contributions can be summarized as follows:
 $\bullet$ {\em Mechanism for filtering hijacks and saving benign conflicts.} We extend the Routinator implementation\footnote{\url{https://github.com/NLnetLabs/routinator}} with modules we developed for differentiating benign conflicts from hijacks. Our mechanism, which we call Learning Origin Validation (\lov), is based on a machine learning (ML) classifier. Compared to heuristic approaches, which often rely on manually defined thresholds that may not be adaptable to changing conditions, ML algorithms can automatically identify patterns in the input features, resulting in better classification performance; in addition, the ML-based approach allows for automatic updating of the model with new data, making it a more scalable and adaptive solution for detecting routing anomalies in today's dynamic and complex Internet environments. We demonstrate this with extensive experimental evaluation on heuristically derived datasets on a simulation platform as well as with a longitudinal evaluation on the Internet. 
%Our experimental results demonstrate that \lov\ effectively identifies benign conflicts and eliminates false positives in ROV, saving legitimate traffic, while maintaining outstanding accuracy in filtering hijacking attacks.
%This figure need to be updated!!!

 $\bullet$ {\em Evaluation on ground truth datasets.} We perform evaluation of \lov\ on BGP routes we collect on the Internet. We collect BGP announcements with benign conflicts, which were visible between May 2022 and June 2022 from Routeviews \cite{routeviews}, resulting in a dataset of 9,223 long-lived and harmless benign conflicts. We also collecte BGP incidents between December 2021 and September 2022 from BGPmon (BGP monitoring service) \cite{BGPMon}. This dataset consists of 415 BGP hijacks. In addition, we collect 8 BGP incidents with significant global impact between 2020 and 2022 from another well-known data source Qrator \cite{Qrator}, resulting in 10,068 hijacks.

For training the DT-classifier, we use a dataset with 2,000 benign conflicts and 415 anomalous instances identified by BGPmon. We then constructed a holdout dataset by combining the remaining 7,223 benign conflicts with hijacks collected from Qrator. This holdout set was used to evaluate the performance of \lov's DT-classifier and post-analyzer. Our experimental results demonstrate that \lov\ saved 95\% of the benign conflicts in the holdout set, while achieving close to 100\% accuracy in identifying hijacks. These results demonstrate that \lov\ can effectively identify legitimate traffic in cases of erroneous ROA without sacrificing the effectiveness of ROV in blocking BGP hijacks.

$\bullet$ {\em Impact of \lov\ on connectivity and BGP convergence.} We evaluate the impact of \lov\ on a simulated platform on Internet connectivity and BGP convergence on 560 networks. Our simulations involve 9,223 routes with benign conflicts and 1000 routes with hijacked prefixes. The results show that compared to the conventional ROV approach, deploying \lov\ significantly reduces the loss of legitimate routes, thereby improving Internet connectivity. Additionally, the likelihood of undetected hijack attempts in the presence of \lov\ is low, maintaining a comparable level of Internet reachability with ROV. Furthermore, the computational overhead introduced by \lov\ is minimal and does not significantly impact the convergence time of BGP advertisements (i.e., BGP convergence).

 $\bullet$ {\em \lov\ shows most conflicts are caused by errors.} Between October 1st, 2022, and March 31st, 2023, we conduct a long-term measurement of live BGP data with \lov\ using BGPStream. Our analysis revealed that, on a daily average, 79\% of the routes that violated ROAs were due to benign misconfigurations. These benign conflicts affected 61\% of the measured prefixes and were traced back to 17\% of the measured ASes. Our findings indicate that benign conflicts are both persistent and prevalent on the Internet. Further analysis of the underlying causes of these benign conflicts shows that the predominant factors for errors are: prefix deaggregation (45,941 instances), downward dependencies (12,382 instances), multi-origin ASes (4,524 instances), and dangling ROAs (3,879 instances). 
 

  $\bullet$ {\em \lov\ shows most hijacks are not by origin in BGP paths.} Our analysis between October 1st, 2022, and March 31st, 2023 also identified a total of 119,815 hijacks during the measurement period. Notably, we found that 90\% (108,104) of the hijacks were exported through AS212483, one of the peers feeding BGP data to the route collector used, between October 26th and November 30th, 2022. Our analysis suggests that the majority of the hijacks detected were caused by AS212483 rather than the origins in BGP paths. To the best of our knowledge, this is the first report to document these incidents. 
%Our findings also highlight that BGP hijacks are typically the result of misconfigurations, technical errors, or unintended behavior of network operators, such as routing policy changes or DDoS mitigation. Specifically, our analysis indicates that the hijackers are not necessarily at the origin of AS paths, but could be located in the middle, posing a challenge to locating and mitigating such hijacks.
%We also compared the hijacks and events identified by \lov\ over the measurement period with those detected by BGPmon. Our analysis revealed that 73\% and 58\% of the hijacks and hijacking events detected by BGPmon, respectively, were also identified by \lov. Moreover, we found that 88\% of the overlapping events had consistent detection times.
%These results suggest that \lov\ exhibits a high degree of consistency with BGPmon in detecting BGP hijacking attacks.

$\bullet$ {\em Survey of network operators.} We conduct a survey of 194 network operators from networks that were involved in hijacks, and received 11 responses, all of which were aligned with the classification of \lov. Our findings also highlight that BGP hijacks are typically the result of misconfigurations, technical errors, or unintended behavior of network operators, such as routing policy changes or DDoS mitigation. %\lov\ identified 63,458 (34\%) benign conflicts covering 62,569 (61\%) prefixes from 5,110 (17\%) ASes.
}



\ignore{
One of the most effective ways to intercept traffic between a source and a destination is via Border Gateway Protocol (BGP) prefix hijacks \cite{china:telecom,fb:out,turkey:hijack}. Prefix hijacks occur when routers send BGP announcements claiming prefixes that belong to other Internet destinations. Such hijacks often happy due to attacks as well as misconfigurations. Autonomous Systems (ASes) that accept the bogus BGP announcements send traffic for IP addresses inside the hijacked prefix to the attacker instead of the legitimate destination. 

Resource Public Key Infrastructure (RPKI) \cite{RPKI} was designed to authenticate network resources that ASes own with Route Origin Authorizations (ROAs), and to enable routers to validate if BGP announcements that they receive are correct via Route Origin Validation (ROV) filtering. Although RPKI was standardized more than ten years ago, still only about 30\% of the ASes filter bogus BGP announcements with ROV \cite{cloudflare}. However, without widespread adoption of ROV RPKI does not provide any security benefits. In particular, previous work showed that the security benefit against hijacks during partial adoption are negligible \cite{gilad2016we}. One of the main factors that hinders wide adoption of ROV are invalid ROAs. Out of the 40\% of the prefixes currently covered with ROAs, 10\% are invalid due to misconfigurations. %need to check the exact number of misconfigured ROAs currently
Most commonly such invalid ROAs result due to incorrect signing, incorrect AS number in the ROA, BGP announcements with more specific IP prefixes than the allowed prefix in the max-len in ROA or changes in the IP address allocation. For instance, consider a provider AS3215 with prefix {\tt 193.2.0.0/15} that delegates some of its IP prefix to its two customers: AS1272 {\tt 193.2.35.0/24} and AS8361 {\tt 193.2.155.0/24}. If the provider AS3215 issues an ROA for its prefix {\tt 193.2.0.0/15} with a max-len of 24 for origin AS 3215, but the customers do not issue ROAs for their prefixes, then the ROV filtering of all the announcements of the customers will result in status invalid and will blocked.

Such misconfigurations result in conflicting ROA and BGP announcement pairs, which appear similar to prefix hijacks. ROV filtering such conflicts causes routers to block legitimate traffic. Misconfigurations in ROAs is one of the main reasons for low adoption of ROV filtering. Despite significance of creating valid ROAs and enforcing ROV and awareness to prefix hijacks, there are still a large number of conflicting ROA an BGP announcement pairs due to misconfigurations. Although invalid prefixes are a long known issue, the number of invalid ROAs increases proportionally to the number of ROAs. We analyze the causes for misconfigurations and find that above 85\% of 
 them are errors, which are difficult to avoid and therefore misconfigurations will most likely always persist. 

In this work we analyze the characteristics of ROAs that conflict with BGP announcements due to misconfigurations in ROAs and evaluate their impact on reachability of the affected networks. We quantify the fraction of the Internet impacted by the invalid ROAs. %(10\% of the IP address covered by the invalid ROAs)




{\bf Contributions.}

We analyze historical BGP data.

We create datasets.

We develop machine learning based classifier to identify conflicts that result due to invalid ROAs.
}




