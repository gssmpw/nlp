\vspace{-10pt}
\section{Real-world Measurements} \label{sec:measure}
In this section, we applied \lov\ to measure BGP updates occurring between October 1, 2022, and March 31, 2023, with daily tests on unique routes.
We validated a total of 117,293,602 BGP announcements, containing 68,507,964 routes with prefixes that were unknown to ROAs, 47,737,942 that were RPKI-valid, and 1,047,696 that were RPKI-invalid. We excluded hijacks originating from a bogon AS number from analysis as they are easily identified.



\vspace{-5pt}
\subsection{Results and Analysis}

\subsubsection{Identified Benign Conflicts By the ML-Classifier}
Figure \ref{fig:invalid_bc} shows the RPKI-invalid routes identified by ROV, as well as the benign conflicts and BGP hijacks detected by the RF classifier throughout the measurement period.
We detected a total of 183,988 objects (each object containing a unique pair of the originating AS and the announced prefix) resulting in RPKI status invalid, covering 102,427 prefixes and 30,775 ASes. The number of RPKI-invalid routes occurred with an average of 5,757 instances per day. With the RF classifier, we identified 63,458 benign conflicts, covering 62,569 prefixes from 5,110 ASes, with an average daily occurrence of 4,549.
On a daily average, approximately 79\% of RPKI-invalid routes are likely attributed to benign conflicts with ROAs.

The occurrences and frequencies of these benign conflicts are displayed in Figure \ref{fig:occur_fre_cdf} in Appendix \ref{app:cdf_benign_conflicts}. An ``occurrence'' refers to the appearance of a particular benign conflict during a day. 
 We observed 33,191 (52\%) benign conflicts that appeared on at least two days during the measurement period, and 80\% of them had an average occurrence frequency of less than 14 days.
 This observation is in line with our expectation that benign conflict tends to be persistent and used actively.
 \begin{figure}[t!]
\centerline{\includegraphics[width=0.8\linewidth]{./images/rov_invalid_benign_conflict_hijack_unverified_numbers.pdf}}
\vspace{-15pt}
\caption[The daily number of RPKI-invalid routes, benign conflicts, and hijacks.]{\small{The daily number of RPKI-invalid routes detected by ROV, benign conflicts, hijacks identified by the RF classifier, and hijacks unverified by the post-analyzer throughout the measurement period.}}
\label{fig:invalid_bc}
\end{figure}
\vspace{-5pt}
\subsubsection{Identified Hijacks By the ML-Classifier}
Additionally, 119,815 hijacks were detected during the measurement period, with a mean of 1,166 occurrences per day.
As shown in the figure, massive hijacks were identified during four periods: between 2022-10-26 and 2022-10-29, between 2022-11-07 and 2022-11-08, between 2022-11-16 to 2022-11-21, and between 2022-11-23 to 2022-11-30.

We conducted a specific analysis of the anomalous routes identified during these four periods. We detail the statistics in Table \ref{tab:specific_days_res} in Appendix \ref{app:statistics}.
We discovered that during the four periods, 14,170, 6,720, 7,388, and 84,157 routes resulting in RPKI invalid status were exported through AS212483, which was one of 41 peers that fed BGP data to a RouteViews collector we used, accounting for 58\%, 45\%, 39\%, and 86\% of the invalid routes, respectively. As shown in the table, most of these routes were identified as hijacks by our classifier, accounting for around 99\%, 100\%, 99\%, and 99.7\%, respectively. We believe that it was not a coincidence that AS212483 was linked to the majority of anomalous routes that occurred during these four periods.

To investigate the underlying causes for these suspected routes, we reached out to the network operators of the ASes potentially responsible for the anomalous routes.
To expedite this process, we randomly selected 500 routes initiated by different ASes and obtained the email addresses of respective network operators using the \textit{whois} tool.
We described the incident that affected the AS in question and requested possible explanations in our email. We received responses from 10 network operators, all of whom denied having announced the anomalous routes. Of particular note was the response from the network operators of AS50058, who informed us that AS212483 had a history of feeding incorrect BGP data to route collectors.
This strongly suggested that AS212483 was the likely source of the large number of anomalous routes observed during the measurement period.

We also attempted to contact the operators of AS212483 but received no explanation for the issues we observed. Our hypothesis is that the anomalous routes may have been caused by misconfigurations or technical errors during the exporting process.

\subsubsection{Verification Results of the Post-Analyzer}
The daily number of hijacks that were not verified by the post-analyzer is illustrated in Figure \ref{fig:invalid_bc}, with a total of unique 5,163 entities \footnote{An entity refers to a unique pair of (origin AS, announced prefix).} placed in quarantine throughout the measurement period.
As indicated by the ``Unverified'' line in the figure, most anomalous routes identified during the four distinct surge periods were verified, supporting the efficacy of this post-verification approach.

To examine the characteristics of hijacking events, we define an event as a series of hijacks originating from a single AS within a day.
6,795 hijacking events were ascertained by the post-analyzer, including 3,163 events initiated by AS212483.
Additionally, 513 ASes exhibited frequent abnormal behavior, resulting in 3,412 events.
For instance, AS395808 was responsible for 73 events during the measurement period. Figure \ref{fig:hijacks_occur_fre_cdf} in Appendix \ref{app:CDFs_events} illustrates the distributions of event occurrences and frequencies for these 513 ASes. We found that 55\% of these ASes initiated events with a frequency of no more than 14 days. We speculate that such frequent anomalous events are likely caused by technical misconfigurations or other inadvertent behavior rather than deliberate malicious activities.
%An analysis of the remaining events through email surveys will be presented in a subsequent section.

%We place the analysis of these hijacking events in Appendix \ref{app:analysis_hijacking_events},  focusing this work solely on unverified routes, which are considered candidates for whitelisting.
%Based on the evidence we have collected so far, it appears that the anomalous events triggered by AS212483 did not result in any significant disruptions to the Internet. However, these events nonetheless serve as a critical reminder of the potential risks of BGP route hijacking and the importance of taking preventive measures, such as RPKI, to avoid such incidents.

\subsubsection{Whitelist}
The benign conflicts identified by the ML-classifier are placed in quarantine for further examination.
We calculate the tightness level between two conflicting origins for each benign conflict using the formula defined in Section \ref{subsec:quarantine}.
Figure \ref{fig:bc_tightness} demonstrates the CDF of tightness values for these benign conflicts.
We specify the $T_{thr}$ value to be 0.3, to ensure both a sufficient quantity of whitelisted items and high reliability of the whitelist.
As a result, 80\% of the identified benign conflicts were whitelisted.
The remaining routes were combined with the unverified routes by the post-analyzer for further behavior monitoring, resulting in 2,321 additional routes added to the whitelist.
In total, the whitelist contains 52,846 items.
%We notified the network operators and conducted a re-validation of the whitelisted routes using updated ROAs. Our analysis revealed that 12,410 prefixes had changed to become RPKI valid, and 2,289 prefixes had become unknown. We attribute these changes to network operators modifying their ROAs to avoid the discarding of legitimate traffic. Our analysis confirmed the existence of misconfigurations in the previous ROAs.
%We provide a comparison with the events that BGPmon detected during the same period in Table \ref{tab:overlap_res}. %Appendix \ref{sec:compare}.
%37904, (10288+395), (7584 2073 631), 3732, 3154
\vspace{-10pt}
\subsection{Root Causes of Benign Conflicts} \label{subsec:rootcauses_BC}
We explore the root causes of 52,846 benign conflicts we whitelisted previously. We do not assume that these conflicts are attributed to human errors but explore other four potential factors that may have led to them.
We show the distribution of these factors in Table \ref{tab:benign_conflict_types} and explain them below. Note that a benign conflict can be caused by more than one factor.

\begin{figure}[t!]
\centerline{\includegraphics[width=0.5\linewidth]{./images/benign_conflict_tightness.pdf}}
\vspace{-15pt}
\caption[CDF of tightness values for benign conflicts.]{\small{CDF of tightness values for benign conflicts identified by the ML-classifier.}}
\label{fig:bc_tightness}
\end{figure}

\noindent\textbf{Prefix deaggregation.} \label{subsubsec:prefix_deagg}
For the benefit of traffic engineering, ASes have continuing incentives to announce more specific prefixes, which are generated by prefix deaggregation \cite{winter2012explicitly, kalogiros2009understanding}. However, more specific prefixes would result in ROA conflicts due to maximum length mismatches (see \textit{OriginMatch}).
We discovered that 37,904 routes with benign conflicts had an inconsistent length compared to their respective ROAs. This inconsistency may have resulted from prefix deaggregation by ASes.

\noindent\textbf{Dependencies.} Persistent benign conflicts can arise due to downward dependencies on AS paths, where upstream ASes often announce prefixes on behalf of downstream ASes, or vice versa, due to misconfiguration or traffic engineering.
A total of 10,683 whitelisted routes may involve such dependencies. We identified 7,584 benign conflicts with provider-customer relationships, 2,073 with a \textit{Parent} relationship, and 631 with only a relatively greater AS dependency (\textit{Depen}). Additionally, we found 395 routes with an \textit{ASdist} value of 0, but no other relationships between their two distinct origins. These routes were included in this category, as they may implicitly indicate a \textit{Parent} relationship.

We explain the dependencies with one example discovered during our measurements. An announcement for prefix 103.158.51.0/24 originated from AS134697, while the authorized owner of the prefix was AS141403. We contacted the operators of AS141403 and learned that AS134697, the upstream provider of AS141403, had misconfigured their BGP routers during an upgrade process. As a result, instead of propagating BGP routes learned from AS141403, AS134697 advertised routes on behalf of AS141403.
Since AS141403 has issued an ROA for the prefix, networks deploying ROV would discard the routes conflicting with the ROA.
However, AS141403 was unable to resolve the issue with AS134697, as AS134697 did not correct the error but instead promised to reroute the traffic to AS141403. This resulted in a persistent benign conflict.

%Dependencies can also introduce conflicts during traffic engineering. One example for this case is a prefix 2407:d340:7146::/47 announced by AS50738, which was signed by AS61302 in ROAs, with the AS path [AS58511, AS6939, AS61302, AS50738]. It can be seen AS61302 appeared on the path. The network operators of AS61302 explained to us that this benign conflict was caused by the use of AS path prepending for traffic engineering purposes.\\

\noindent\textbf{Multi-origins.} Another common type of benign conflict is a MOAS conflict, as described in Section \ref{sec:lov_features}.
 We found 3,732 benign conflicts where two distinct origins belonged to the same organization.
 
\noindent\textbf{Delayed ROAs.}
This type of benign conflict occurs when operators do not promptly update the RPKI records, especially during the transfer of prefix ownership from one provider to another.
For instance, when a prefix is switched to a new owner, but its ROAs are not updated immediately, this situation may lead to such conflicts.
We identified 3,154 instances where two conflicting sources have no relationships (such as PC or MOAS, etc.), yet verification with the IRRs indicated these routes were valid.
These conflicts reveal a possible delay in updating corresponding ROAs after changes to IRR data.
Upon further analysis, we found that 1,830 out of these conflicts eventually resulted in RPKI valid status when assessed using the most recent ROAs after the measurement period. This provides support for the hypothesis of delayed ROA updates.
\begin{table}[t!]
 \renewcommand{\arraystretch}{1.0}
  \centering
\fontsize{7}{7}
%  \begin{adjustbox}{max width=0.7\linewidth}
  \begin {tabular}{ccccc}
  \toprule
    \textbf{Deaggregation} & \textbf{Dependencies} & \textbf{Multi-origins} & \textbf{Delayed ROAs} \\
    \midrule
   37,904 & 10,683 & 3,732 & 3,154\\\bottomrule
   \end{tabular}
  % \end{adjustbox}
\caption{\small{Categorization of benign conflicts factors.}}% A benign conflict can be caused by more than one factor.}}
\label{tab:benign_conflict_types}
\vspace{-15pt}
\end{table}

\vspace{-10pt}
\subsection{Confirmation of Events via Email Surveys} \label{sec:root_causes_events}
Subsequently, we verified the hijacking events through email surveys.
We sent hijack reports to the relevant ASes via email.
Excluding the events attributed to AS212483 and the frequent anomalous events, we were left with 220 hijack events. We sent 194 emails to potential perpetrator ASes and another 194 emails to potential victim ASes. We described the detected events and provided a malicious BGP route as an example in our emails.
We received 10 responses that confirmed the events.
From these responses, we identified potential root causes of hijacking, including technical errors, ownership changes, and non-origin perpetrators (i.e., perpetrators are in the middle of an AS path).
A detailed analysis of these hijacking causes is provided in Appendix \ref{app:root_cause_hijack}.
As mentioned earlier, we manually ensured that confirmed hijacks were not included in our whitelist.

%the accuracy of hijack confirmation seems to be a crucial factor in the method.
As noted in Section \ref{sec:lov_overview}, the email survey aims to provide evidence for potential human intervention, not to confirm the accuracy of identified hijacks, but to support the creation of a trusted whitelist.
\vspace{-5pt}
\subsubsection{Critical Lessons from Hijacking Events}
A notable point is that all the victims had already issued ROAs. Assuming attackers had no specific targets and accessed the ROAs before engaging in malicious activities, they would intuitively be more inclined to attack ASes that lack ROAs, thereby circumventing ROV's filtering. Despite this, technical misconfigurations, changes in ownership, or other unintended issues can still result in hijacks targeting victims with ROAs.
In other words, hijacks can occur even when ROAs are in place. This challenges the belief among some network operators that issuing ROAs alone is sufficient to protect their networks from hijacks. Since ROV primarily benefits other networks, some operators may choose to issue ROAs without implementing ROV. This reluctance to deploy ROV can hinder its broader adoption across the Internet, which in turn reduces the security benefits of having ROAs. Our findings underscore the importance of not only issuing ROAs but also actively deploying ROV.











 
 












