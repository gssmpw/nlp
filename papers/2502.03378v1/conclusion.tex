
\section{Conclusions} \label{sec:conclusion}
RPKI plays a crucial role in preventing BGP hijacking. It leverages ROV to ensure that only routes from authorized owners are accepted by routers in BGP announcements.
Despite the effectiveness of RPKI, its deployment has been progressing at a slow pace.
BGP routes with benign origin conflicts with ROAs, so-called benign conflicts, which appear as hijacks but indeed are legitimate, will be identified by ROV as invalid and filtered, leading to the loss of legitimate traffic. This raises concerns among network operators about potential revenue loss and service disruptions, diminishing their motivation to deploy ROV.

In this work, we developed Learning Origin Validation (\lov) to whitelist benign conflicts on an Internet-wide scale, preventing BGP announcements with benign ROA violations from being lost. Network operators can download and install these whitelisted benign conflicts on their ROV-enforcing routers to preserve the affected BGP routes, addressing their concerns. This approach thus has the potential to promote the broader deployment of RPKI and ROV across the Internet.
A longitudinal live measurement demonstrated \lov’s real-world effectiveness, resulting in the whitelisting of 52,846 benign conflicts. Additionally, our investigation identified four factors contributing to benign ROA conflicts beyond human errors. Our observations raise an open question about the current ROV mechanism, potentially motivating further research. Moreover, our discussions indicate that \lov\ holds promise for offering benefits to critical services such as cloud services.
%Future work will investigate the practical challenges of deploying LOV in real-world networks.

%We addressed key questions about the implementation and deployment of \lov\ and discussed future research directions, including insights into its improvements, prospective uses, and open questions.

%\lov\ incorporates three key components: \textit{ML-classifier}, \textit{Post-analyzer}, and \textit{Quarantine}. The ML-classifier is built on the RF model, reaching 95\% accuracy in detecting benign conflicts and nearly 100\% in identifying hijacks.The post-analyzer monitors the changes in global AS visibility to verify a hijack identified by the classifier, while the quarantine mechanism is used for further review on identified benign conflicts and unverified hijacks.

%\lov\ is designed to preserve as many benign conflicts as possible while maintaining ROV's core function of preventing hijacks. It comprises three key components: the \textit{ML-classifier}, \textit{Post-analyzer}, and \textit{Quarantine}. Our experimental results show that the ML-classifier achieves 95\% accuracy in detecting benign conflicts and nearly 100\% in identifying hijacks, meeting our requirements. Furthermore, our analysis suggests the effectiveness of the post-analyzer and quarantine mechanisms in addressing the limitations of the ML-classifier, contributing to a comprehensive and reliable whitelist. 

%However, only a few efforts have been dedicated to addressing the obstacles caused by ROA misconfiguration for a broader RPKI deployment.
%We conducted experimental evaluations on two key components of \lov: the ML-classifier and the post-analyzer, demonstrating the outstanding capabilities of \lov\ in detecting benign conflicts arising from ROA misconfiguration and in identifying BGP hijacks. A longitudinal live measurement using \lov\ was performed to evaluate its real-world effectiveness. The root causes of benign conflicts within the context of RPKI were presented.


%We found that 34\% of the filtered routes were benign misconfigurations, which correspond to 61\% of the address space covered by ROAs. Since mostly large providers currently apply ROV \cite{DBLP:conf/ccs/ShulmanVW22,apnic,cloudfl}, the reachability to invalid (benign) origins affects not only the ROV-filtering providers but also all of their customer ASes.

%Our measurement shows that most RPKI invalid prefixes currently can still communicate: since at the moment most networks do not enforce ROV, the traffic of such ROV-invalid prefixes can reach their destinations over alternative routes, which do not do ROV filtering. Nevertheless, such routes may be less favorable, e.g., they have higher latency, a larger number of hops, or they may traverse untrusted networks. In addition to security and quality of service concerns caused by errors, another major issue is the business considerations of upstream providers. Upstream providers get paid for traffic that traverses them. Therefore, when legitimate traffic gets ROV-filtered because of misconfigurations and is not relayed via the providers, the corresponding upstream provider does not get paid for it, losing revenues. The situation will exacerbate at full RPKI deployment, since the invalid prefixes would be globally blocked and would become unreachable. All these are major concerns for network operators. 




%In fact, NIST RPKI monitor shows that the errors increase linearly with the grows in created ROAs\footnote{\path{https://rpki-monitor.antd.nist.gov/}}. Therefore, in the prospective future we expect that the errors due to misconfigurations will persist. Further imposing an obstacle on deployment of ROV, and a lack of wide deployment of ROV also hinders adoption of path validation proposals.%One of the central obstacles towards wide enforcement of ROV is benign conflicts, which are filtered by ROV enforcing routers. Filtering benign conflicts disconnects legitimate destinations, causing their providers to lose revenues.

%In this work, we develop \lov\footnote{The implementation of \lov\ and our datasets are accessible at \url{https://github.com/zsjstart/LOV}}\ to differentiate benign misconfigurations from BGP hijacks. \lov\ can be deployed on border routers to provide the benefit of ROV, yet without the risk of blocking legitimate origins. % additional benefit of detecting misconfigurations. Setting up \lov\ does not require additional efforts beyond deployment of ROV. 
%We experimentally demonstrate the effectiveness of \lov\ by evaluating it over BGP announcements and ROAs collected over a period of six months.

%In addition to an improved run time performance, \lov\ can provide enhanced functionality.
%LOV can be deployed on border routers at the network edge. 
%Unlike ROV, network operators employing \lov\ need not be concerned about reachability loss due to wrong identification. 
%LOV will not filter any anomalous routes but assign them lower local preferences.
%\lov\ consumes less computational overhead or resources due to using straightforward and lightweight models or algorithms.
%The local database is updated and managed automatically on a regular basis, avoiding personal efforts.
%Due to the limited number of Route Views collectors we use, and a constrained measurement period per day, the quantity of BGP routes measured is limited.
%Also, selecting different observation points would skew the measurement results.
%We define a 15-minute interval for the post-analyzer to sample AS hegemony values due to the limitation of the IHR service, which may cause \lov\ to miss short-duration (e.g., only a few minutes) BGP events.
%In the future, we will deploy \lov\ as a central service that is hosted on a cloud platform to deal with massive BGP announcements on a global scale.
%We will develop new approaches to quantify AS global visibility for a more efficient and effective post-verification mechanism and regularly update the classification model with new data sets.


\ignore{
\noindent\textbf{High validity.} LOV first use an ML classification model to effectively distinguish between benign and suspicious announcements, and then use a post-analyzer to further verify whether a suspected malicious announcement is from a real attacker AS, guaranteeing a high accuracy for identifying BGP anomalies. \\

\noindent \textbf{Limitations}
To be continued ...
Discuss about the limitations about post-analyzer (false positive), this method can effectively detect anomalies as anomalous routing behavior emerges, however, in some cases, there will be false alarms, for example, when the ISP has routing changes due to traffic engineering or load-balancing, new routing policies, etc. Or the router feeding the route-views collector restart. 
The location of route-view collectors may affect the global visibility in the routing table, thus degrading the accuracy of post-analyzer.
False positive may be due to the routing instability not because of malicious behavior. If we identify periodical burstiness characteristic of BGP announcements of an AS, we identify it as routing flapping (instability) not malicious behavior. 
A reason
Why route leaks may be detected incorrectly, is that the relationships inferred by CAIDA are derived with an understanding of complex/hybrid relationships, but still output only peer-to-peer and client-to-peer relations \cite{wijchers2014bgp}. 
Hijacking confirmation and diagnosis: we directly contact the corresponding network operators if a stable prefix hijacking is detected. We provide detailed information collected to help their diagnosis work. 
%on a daily basis
}