\vspace{-10pt}
\section{Related Work} \label{sec:relatedwork}
A number of approaches were proposed for detecting BGP hijacks.
Karlin et al. \cite{karlin2006pretty} introduced Pretty Good BGP (PGBGP), which uses a window of historical BGP updates to quarantine suspicious announcements and delay their adoption. However, this approach can incur significant resource overhead due to the need to manage large volumes of historical data. Additionally, the quarantine mechanism can delay the acceptance of legitimate announcements with benign conflicts.
Schlamp et al. \cite{schlamp2016heap} inferred legitimate business relationships between the hijacking and victim ASes using IRR databases, but encountered potential challenges due to the possibility of outdated or conflicting information within the databases. Tools such as iSPY \cite{zhang2008ispy} and Argus \cite{shi2012detecting} detect prefix hijacks by relying on the concept that polluted ASes fail to reach the victim's network. However, they require real-time probing by the deploying network, which may impose non-negligible computational or communication overhead.
Other studies, such as \cite{deshpande2009online, karimi2019border, testart2019profiling, hlavacek2022smart}, rely on features like AS path length, the number of BGP updates, the longevity of prefix advertisements, and hijack duration. However, AS path length can vary during BGP propagation, and features related to the number, longevity, or duration of BGP announcements may affect the identification of hijacks in their early stages. Consequently, these features may not be suitable for prompt detection.

As a de facto measure to prevent BGP hijacking, the deployment of RPKI started in 2009, and in 2010, Randy Bush had a series of presentations to promote the new standard candidates \cite{bush:rpki:2009}. At that point, there was just a test-hosted RPKI system by RIPE, with no practical implementation of delegated certificate authorities and publication points. In 2011, the production-ready hosted RPKI by RIPE was established, and Deutsche Telekom created ROAs for their prefixes, which de-facto jump-started the RPKI in RIPE region \cite{band:rpki:2011}. Many networks quickly created ROAs for their prefixes, and as of February 2023, there were 140K ROAs, covering approximately 39.3\% of IPv4 prefixes announced in the default-free zone (DFZ)\footnote{The DFZ has a full global BGP routing table.}, with about 8\% being invalid due to errors in ROAs. In contrast to ROAs, the enforcement of ROV progresses much slower. The most recent measurement \cite{hlavacek2022behind} shows that only 27\% of the ASes enforce ROV filtering. The main factor hindering the adoption of ROV is the fear of losing traffic due to erroneous ROAs \cite{gilad2017we,iamartino2015measuring, wahlisch2012towards}. Also, a follow-up study \cite{chung2019rpki} shows that the number of invalid ROAs does not decrease with time but remains between 8\% and 10\%.

Previous studies identified vulnerabilities, errors and misconfigurations in RPKI \cite{stalloris:HlavacekJMSW22,mirdita2022poster,beyond:HlavacekJMSW23,cure:MirditaSVW24,rpki:sok:mirdita25}. A recent proposal showed how to utilize Byzantine consensus to enhance the resilience of RPKI against vulnerabilities and errors \cite{byzrp:FriessMSW24}.

\ignore{
Efforts to encourage broader adoption of RPKI have focused on addressing ROA misconfiguration issues. DISCO \cite{hlavacek2020disco} automates ROA publishing.
It can only prevent human errors in ROA issuance but not handle benign conflicts resulting from complex routing policies such as traffic engineering or MOAS.
SROV \cite{hlavacek2022smart} uses a heuristic approach based on route duration to analyze errors in ROAs. However, this approach may cause BGP convergence delays, such as monitoring new BGP announcements for several weeks, and could potentially weaken ROV's ability to prevent hijacks due to high false negative rates.
}
%Unlike \cite{hlavacek2022smart}, our method uses ML technologies and features representing the relationship between conflicting origins, enabling the accurate identification of benign conflicts while maintaining the effectiveness of ROV in preventing hijacks.



 %Our research focuses on distinguishing invalid routes resulting from benign errors in ROAs from prefix hijacks, promoting the deployment of RPKI validation (ROV).
 %Unlike our previous method (SROV), our classification method is based on ML technologies, enabling the effective identification of benign conflicts while maintaining the effectiveness of ROV in preventing hijacks.
 %In contrast to previous work, we develop a new mechanism, \lov, that prior to applying ROV filtering, analyzes the invalid routes, to identify benign conflicts caused by errors and misconfigurations in ROAs. We utilize a set of statistical features that are computed using information from multiple sources, enabling robust identification without imposing extra computational overhead. Notably, our proposal enhances the capabilities of the existing ROV without requiring any modification to the current RPKI infrastructure, making it easy to deploy.

%Since \lov\ is an extension of an existing Routinator implementation, networks benefit by simply enforcing ROV without extra overhead. 
%Moreover, our work confirms the occurrence of a hijack event leveraging the burst of BGP advertisements.
%ARTEMIS \cite{sermpezis2018artemis}, and PHAS \cite{lad2006phas} were developed to allow prefix owners to detect BGP events for securing their own prefixes.
\ignore{
%In \cite{wijchers2014bgp}, the authors assessed valley-free violation among ASes. Different from previous AS relationship inference algorithms mainly relying on the valley-free rule, they extracted AS relationships from BGP community data.
%based on BGP community data \cite{wijchers2014bgp}, 
%The studies in \cite{wijchers2014bgp,qiu2007toward,giotsas2012valley, sriram-idr-route-leak-solution-discussion-07} present a in-depth analysis ofBGP updates that violate the valley-free routing policy and characterize route leaks.
%As a special type of BGP hijacking, malicious parties forge the AS path while announcing a prefix by simply putting their ASN in front of the victim AS, making the route appear to come from a legitimate source, and then bypassing ROV.
%BGPsec \cite{BGPsec,siddiqui2014diagnosis}, as a security extension of BGP, verifies the forward direction of advertisements on boundary routers through RPKI to prevent the above path manipulation; however, it changes the BGP protocol, posing new challenges to adoption and vulnerable to downgrade attacks \cite{aximov2020verification}.
%Another proposal, path-end validation \cite{cohen2016jumpstarting}, only verifies the last hop on the AS path for an easier deployment than BGPsec.
%However, both of them fail to prevent route leaks.
{\bf Route leaks.} Existing approaches to block route leaks rely on marking routes to avoid leakage or use heuristics and shared routing information to identify valley-free violations. 
RFC9234 \cite{rfc9234} introduces a new BGP path attribute, so-called Only to Customer (OTC), that transit ASes can use for detecting route leaks based on BGP Roles \cite{rfc5492} such as customer, provider, or peer. The OTC attribute is set to the AS number of the advertising AS only when a route is exported to a customer or a peer.
When a provider receives a route with an OTC value set, the route will be discarded to prevent route leaks.
The problem is that deploying OTC requires changing the BGP protocol and the BGP standard and integrating a new configuration parameter called BGP role. This parameter needs to be negotiated using BGP OPEN messages between BGP routers. 
%Other approaches \cite{white-sobgp-architecture-02,aximov2020verification} to path validation rely on shared routing information such as AS connectivity graph or cryptographically issued AS providers in RPKI sources.
Recently, \cite{aximov2020verification} introduced a new digitally signed RPKI object called Autonomous System Provider Authorization (ASPA), binding a customer AS number and the authorized provider AS number for its announcements. The AS path verification with RPKI checks if the routes received from customers or peers comply with the ASPAs.
Apart from leak prevention, this method can protect against path manipulation attacks. For its deployment ASPA requires fully deployed RPKI and ROV. Therefore, resolving the obstacles towards enforcement of ROV is not only important for blocking prefix hijacks but also for adoption of other path validation mechanisms. In addition, the same challenges that apply to RPKI also apply to ASPA. First, motivating operators to deploy ASPA is challenging, indeed we have not found ASPA deployments in the wild. Second, errors in ASPA objects configurations as well as in validation are inevitable. Therefore, we expect that similar concerns will need to be addressed with deployment of ASPA. 

%Although a recent Internet draft\footnote{\url{https://datatracker.ietf.org/doc/draft-ietf-sidrops-aspa-verification/}} proposing ASPA could prevent route leaks it requires as a prerequisite wide deployment of ROV. On the other hand, the standardized path validation mechanism with BGPsec [RFC8205] does not block route leaks.

%However, similar to the ROAs, how to motivate network operators to issue an ASPA remains challenging.
Other proposals, like \cite{bagnulo2022practicable,MANRS}, use operator registration information in IRR databases to detect route leaks, since the IRR databases are often misconfigured, both approaches have high false positives/negatives rates and are not used in practice.
%Since ASIRIA relies on existing IRR information, it does not have the problem of circular dependencies between issuance and deployment like ASPA does; thus, it can provide benefits to early adopters.
%However, ASIRIA has to contend with reliance on IRR information. %especially in the case of dynamic routing policies.
%The author in \cite{li2015route} identified a leaked route by detecting a routing loop without knowing any AS relationship. However, in some cases, the route contains a looping that is created by the leaking AS, e.g., the leaking AS is originating the route. In this case, the looping root is caused by complex routing policies in traffic engineering, which do not correspond to a route leak. The mechanism based on route loops will misidentify it, producing false positives.
%The approach in \cite{siddiqui2015self} allows ASes to utilize routing databases at hand to detect route leaks based on a few heuristics; however, those mechanisms are not practical. %theoretically feasible, relying on realistic assumptions.
Systems like Peerlock and Peerlock-lite \cite{mcdaniel2020flexsealing} can be deployed between two BGP neighbors to prevent illegitimately transiting the protected AS by filtering routes that contain an AS number of a large network (e.g., Tier-1 or unauthorized upstreams) on the path. The study in \cite{mcdaniel2020flexsealing} showed that Peerlock/Peerlock-lite was offers protection to large transit networks; thus, the number of protected networks remained low. However, the lack of automated configuration protocol makes Peerlock vulnerable to misconfigurations \cite{mcdaniel2020flexsealing}. Our approach for blocking route leaks does not suffer from the misconfigurations inherent in previous works and it does not require complex configuration. Since \lov\ is an extension of an existing Routinator implementation, networks benefit by simply enforcing ROV. 
%In recent years, many efforts have been done to fight against route leaks.
%Besides, PGBGP \cite{karlin2006pretty} uses the history of BGP routes to verify a route is normal.
%Also, many techniques have been proposed to prevent or detect route leaks.
}

\ignore{
Some efforts have been delved to improve today's RPKI system.
Publishing ROAs has been a manual process since RPKI was standardized, inevitably leading to unintended human errors \cite{gilad2016we,iamartino2015measuring, wahlisch2012towards}.
Disco \cite{hlavacek2020disco} enables automatic certification of IP address blocks to reduce ROA errors for better RPKI benefit.
As mentioned earlier, only subtle security benefits will be produced without the widespread deployment of RPKI. 
ROV++ \cite{morillo2021rov++} is another extension of ROV that improves security benefits even with low adoption.
Although the authors conducted a detailed evaluation of the simulation platform to show the validity of ROV++, there is no evidence to prove its practicality in the real world.
%In addition, the work \cite{gilad2017maxlength} shows that the misconfiguration of the MaxLength attribute would degrade the security of RPKI.
Differently, our work extends the functionality to ROVs by rescuing conflicts due to wrong ROAs, enabling filtering of advertisements with anomalous paths (e.g., route leaks), and providing post-verification and long-term monitoring to verify routing anomalies and whitelist long-lived and harmless unverified routes. 
}
%Existing approaches to leak prevention rely on marking routes by operator configuration, with no check that the configuration corresponds to that of the External BGP (eBGP) neighbor, or enforcement of the two eBGP speakers agreeing on the peering relationship.
\ignore{
A number of approaches have been proposed for detection of BGP anomalies or classification of BGP behavior.
Several techniques \cite{zhang2004detection, deshpande2009online} leveraged signatures or statistics of routing updates to detect BGP anomalies.
Differently, to detect unintentional routing anomalous behavior, Lutu et al. \cite{lutu2015bgp} proposed a tool, BGP visibility scanner, which can provide network operators with the status of prefix visibility.
In addition, Teoh et al. \cite{teoh2006bgp} presents BGP Eye, a visualization tool, aiming at real-time detection and root-causes analysis of BGP anomalies.
}
 %ASIRIA: this method relies on the information in IRR to detect route leaks. However, the internet routing policies are generally complex and dynamic. Once network operators do not instantly or regularly update the IRR information, this will result in misidentification, causing legitimate traffic to be dropped.
 %The work \cite{li2015route}: this method relies on an intuitive assumption that route leaks may cause routing loops in the BGP path . They identify route leaks by detecting routing loops. However, this method is prone to false positives. For example, routing loops may also be caused by traffic engineering. Our evaluation results show that such benign routing loops are very common in reality.
 %The study \cite{siddiqui2015self}: this work proposes a theoretical framework consisting of three techniques, which are based on either control-plane or data-plane, for detecting route leaks. However, this work exclusively show the effectiveness in the simulations. The practicality in the real-world remains unknown. 
 %Peerlock/Peerlock-lite \cite{mcdaniel2020flexsealing} relies on Peerlock rules configured between the protector AS and the protected AS. It requires that the protector AS filters BGP updates from an unauthorized propagators or routes whose AS path contains a Tier-1 AS. However, the lack of an automated and secure configuration protocol makes it vulnerable to misconfiguration of Peerlock rules. 


 

 