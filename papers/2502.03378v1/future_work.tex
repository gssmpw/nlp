\vspace{-10pt}
\section{Future Directions} \label{sec:lov_future}
%This section provides future directions, focusing on future research questions, potential improvements, and insights into the prospective uses of \lov.

\noindent\textbf{Open questions and improvements.} Our analysis of benign conflicts in Section \ref{subsec:rootcauses_BC}, revealed that approximately 72\% of these conflicts (see "prefix deaggregation") involved prefixes with a matching origin to the ROAs but violated the allowed prefix length.
Since ROV verifies both the origin and the prefix length against the ROAs, these benign conflicts are incorrectly filtered out.
Given that actual hijacks are rare compared to benign conflicts, the current ROV mechanism, with its strict filtering rules aimed at preventing hijacks, may be impractical due to its impact on legitimate routes.
Our observations raise a future research question: Could the filtering strategies in ROV be adjusted to focus exclusively on filtering routes with origin violations against ROAs? This question arises from the fact that most hijacks, such as prefix or sub-prefix hijacks, involve an invalid origin compared to ROAs. While validating prefix length can help filter some hijacks related to AS path manipulation (e.g., when a correct origin AS is appended at the end of the path but a more specific prefix than allowed by the ROA is announced), there are existing path validation mechanisms designed to address these issues, such as BGPsec [RFC8205] and Autonomous System Provider Authorization (ASPA) \cite{aspa}.
Future research will investigate real-world ROV filtering strategies used by ASes, specifically exploring whether they filter all RPKI-invalid routes or allow routes with a matching RPKI origin to pass through.

Benign conflicts with an RPKI-valid origin AS are straightforward to handle and do not necessarily depend on other relationships for identification. This observation inspires us to enhance \lov\ in future work. We will exclude the identification of such benign conflicts from the ML classifier and explore new features to better detect other types of benign conflicts with unknown causes; for example, we will resolve the domain names associated with two conflicting origins and examine the similarity between these domain names.

\noindent\textbf{Future applications.}
Cloud services like AWS, Google Cloud, and Azure offer BYOIP (Bring your own IP), allowing customers to use their own IP addresses for cloud resources. To provision BYOIP, such as in Google Cloud, customers need to create a \textit{public advertised prefix} and an ROA for this prefix that points to the ASN of the cloud service provider.
Google Cloud recommends that customers submit another ROA request for the same prefix and prefix length but with their own ASNs as the origin to avoid benign ROA conflicts when customers need to advertise the prefix \cite{googlecloud}. However, customer behavior is often dynamic—they may ignore this recommendation or fail to maintain the ROA, even if it was initially created. Additionally, when customers deprovision BYOIP, they must wait 14 days before deleting the ROA that authorizes the cloud service provider \cite{googlecloud}.
This delay can also result in benign conflicts as customers might advertise the prefix during the waiting period.
As more users adopt cloud services, BYOIP-related benign conflicts may increase, potentially affecting the user experience. Looking forward, \lov\ has the potential to mitigate this impact. We discuss additional future applications of \lov\ in Appendix \ref{app:add_future_use}.

%In the future, we will assess the real-world benefits that \lov\ can bring to ROV-enforcing networks supporting major Internet, Cloud, and IoT services.
%Furthermore, adversaries may announce IP addresses associated with Cloud service providers or central servers in IoT environments, potentially disrupting Cloud services and commandeering IoT devices reliant on communication with central servers. It is evident that 




