\vspace{-5pt}
\section{Post-Analyzer} \label{subsec:lov_post_analyzer}
%Next, we implement the post-analyzer introduced in Section \ref{sec:lov_overview}.

%This method verifies whether the anomalous routes identified by the classifier genuinely result from BGP hijack events, thereby preserving benign conflicts that may have been misclassified.
%Routes that cannot be verified as hijacks are considered for whitelisting, while those are confirmed are recorded as potential BGP hijacking incidents.

\subsection{Anomalies in AS Visibility}
The proposed post-analyzer approach relies on the bursty property of BGP announcements during a BGP incident, in which groups of routing messages are sent in short intervals.
We leverage this property to detect whether suspicious announcements identified by the RF classifier are a result of a BGP event. Prior studies such as \cite{moriano2021using, deshpande2009online} counted the total number of announcements over a certain period to identify routing anomalies. However, these methods involve processing large volumes of BGP routes, making them resource-intensive and time-consuming.
In contrast, we employ global AS visibility, which refers to the extent to which an AS is visible on BGP paths in global routing tables.
In general, the AS visibility of reliable networks tends to remain stable. However, during a hijack event, the offending AS illegitimately originates routes for a substantial number of prefixes that are not under its ownership, causing its global visibility on the Internet to significantly increase. We leverage this change in global AS visibility as a detection signature for hijack events.
\begin{figure}[t!]
\centerline{\includegraphics[width=0.8\linewidth]{./images/post_analysis_out.pdf}}
\vspace{-15pt}
\caption[Changes in global visibility of malicious ASes.]{\small{Changes in global visibility of malicious ASes before, during and after the BGP incidents and evaluation with the post-analyzer. Red vertical lines indicate outlier detections.}}
\label{fig:post_analysis}
\end{figure}

To calculate the global visibility of an AS at a particular time, we employ global AS hegemony provided by IHR.
Unlike local AS hegemony, global hegemony reflects the centrality of an AS on the path, indicating the likelihood that the AS appears on BGP paths in the routing table. Therefore, this metric can also serve as an indicator of AS visibility.
We detect the anomalous increase in AS visibility by monitoring for a surge in global hegemony values.
The use of AS hegemony offers a more direct and efficient approach to detecting abnormal BGP behavior (e.g., hijacking), compared to previous methods.
The IHR service provides AS hegemony values in real-time.
\vspace{-5pt}
\subsection{Z-test Anomaly Detection}
We then leverage a Z-test to detect anomalies in the hegemony values of hijacking ASes.
For each suspicious BGP announcement identified by the classifier, the post-analyzer collects the most recent historical hegemony values of the perpetrator AS preceding the announcement time.
These values are modeled as a variant with Gaussian distribution, denoted as $X$. We then calculate Z = $\frac{X-\mu}{\sigma}$, which follows the standard normal distribution of $N(0,1)$, with $\mu$ and $\sigma$ being the mean and standard deviation of $X$. Next, we evaluate the p-value of a hegemony value observed after the announcement to determine if it is statistically significant.
%For instance, if the announcement time of a route is 10:29, we examine the hegemony value of 10:30.

We collect 50 historical hegemony values, which provides sufficient data to accurately capture their distribution while minimizing the impact of older, potentially less relevant data, on the p-value calculation.
We specify a null hypothesis that there is no obvious difference between the tested hegemony value and historical data at a significant level of 0.05.
Since the perpetrator AS has an increasing hegemony value during the event, we perform a right-tailed test to detect anomalies. If the p-value is greater than 95\%, the null hypothesis is rejected, indicating a potential hijack incident.
We also monitor whether the hegemony value returns to its normal level after the event. If the hegemony value remains consistent for a long period (e.g., 24 hours), we consider the possibility of a new routing policy in place that is causing a different level of AS visibility.
While the Z-test is straightforward, it is sufficiently reliable for detecting anomalies in AS hegemony values.


\vspace{-10pt}
\subsection{Evaluation and Analysis}
We then evaluate the post-analyzer.
To simplify this process, we use 8 BGP incidents collected in Section \ref{subsec:holdout} (refer to Table \ref{tab:bgp_incidents_res}) for the evaluation.
For a given incident, we extract timestamps from each of the route announcements during the event and verify that the hegemony value at the announcement time is abnormal.
%we first collect global hegemony values of the malicious AS for time periods before, during, and after the occurrence time.

We detected anomalies in global AS visibility for four incidents, as illustrated in Figure \ref{fig:post_analysis}.
We mark each incident in the format: AttackType.AttackerASN.TimeOfOccurrence. 
The gray area refers to the safe region of hegemony values, and the upper bound of this region means the lower threshold of identifying outliers.
Red vertical lines indicate that abnormal AS hegemony values were detected at corresponding BGP announcement times, while green vertical lines mean that the post-analyzer has not detected anomalies.
As can be seen, the post-analyzer failed to identify anomalies at the later stage of two events: ``Hijack.212046.2021-10-25T09:56'' and ``Hijack.38744.2022-05-31T07:28''.
The reason for this failure was likely that the incidents were immediately detected, and the corresponding BGP announcements were filtered to prevent their propagation on the Internet, resulting in a decrease in the perpetrators' global visibility.

In addition, for the event "Hijack.48467.2021-05-18T09:00", increased hegemony values were observed starting from 7:00. However, this rise in hegemony values caused an increase in the outlier detection threshold at 9:00 (the event occurrence time), causing the post-analyzer to fail to verify this event.
For the remaining three BGP events, no corresponding hegemony values were available during the event periods, which prevented the post-analyzer from tracking changes in the perpetrator ASes' global visibility.
We attribute these failures to potential issues in the IHR's hegemony computation mechanism.

%Our analysis indicated that the missing data was not due to chance, as hegemony values were present before and after the event. We hypothesize that the perpetrator ASes promptly withdrew all their advertisements after detecting the events to fix errors (e.g., technical flaws) and restore normal BGP behavior. This action, in turn, hindered the IHR monitor from observing any BGP routes from these ASes during the period of restoration, leading to the unavailability of hegemony values.
%Therefore, we also consider the absence of hegemony values to be an indicator of BGP hijack events.
%Despite this failure, there was no impact on the whitelist, as the hijacks lasted for a short period (less than a day).

Note that, even if the post-analyzer does not confirm certain events, these limitations have minimal impact on the reliability of the whitelist.
Unverified routes will be placed in quarantine for additional examination.
One strategy for whitelisting involves continuously monitoring the behavior of these routes, such as their longevity (see more details in Section \ref{subsec:quarantine}).
Given the typically transient nature of hijacking incidents, routes associated with hijacking, that are mistakenly placed in quarantine, are less likely to meet the criteria for inclusion in the whitelist. 
%thereby helping to protect the whitelist from being polluted.
%When the hijacker is positioned not at the origin of the AS path but rather in the middle, the post-analyzer may fail to detect this anomaly. Detailed assessments of such cases are provided in Section \ref{subsec:res}.
