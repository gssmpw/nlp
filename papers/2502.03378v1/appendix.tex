\section{CDFs of Occurrences and Frequencies of Benign Conflicts} \label{app:cdf_benign_conflicts}
The occurrences and frequencies of these benign conflicts are displayed in Figure \ref{fig:occur_fre_cdf}. An ``occurrence'' refers to the appearance of a particular benign conflict during a day. 
 We observed 33,191 (52\%) benign conflicts that appeared on at least two days during the measurement period, and 80\% of them had an average occurrence frequency of less than 14 days.
 This observation is in line with our expectation that benign conflict tends to be persistent and used actively.
 
\begin{figure}[t!]
  \centering
  \begin{minipage}[b]{0.4\linewidth}
    \centering
    \includegraphics[width=\linewidth]{./images/occurrence.pdf}
  \end{minipage}
  \hfill
  \begin{minipage}[b]{0.4\linewidth}
    \centering
    \includegraphics[width=\linewidth]{./images/occurrence_frequency.pdf}
  \end{minipage}
  \vspace{-10pt}
  \caption{\small CDFs of occurrences and frequencies of benign conflicts.}
  \label{fig:occur_fre_cdf}
\end{figure}

\section{Feature Scaling} \label{app:feature_scale}
Compared with other features that fall into [0, 1], $d$ exhibits a larger upper bound. The larger values may cause bias in the classification model. Hence, we scale this feature before training.
In our experiments, we discovered that scaling $d$ into the range of [0, 1] using a commonly used method like minimum-maximum scaling may result in a number of false negatives, i.e., misidentifying hijacks as benign conflicts, negatively impact the quality of the whitelist.
These mis-recognitions often occur when hijacks entail two conflicting origins that are in close proximity.

To address this, we apply the arctangent function to the Euclidean distance ($d$) between the conflicting origin ASes, resulting in a new distance metric $ASdist$ called AS distance, as given by the formula: {$ASdist = \frac{2}{\Pi}arctan(d)$}.
Figure \ref{fig:arctan} shows the transformation curve from $d$ to $ASdist$.
This transformation is specifically chosen for its high sensitivity to hijacks.
Typically, the $d$ value of benign conflicts approaches 0, as presented in Figure \ref{fig:d_cdf}, and we expect larger distance values for hijacks.
As illustrated in Figure \ref{fig:arctan}, small increases in the distance can cause a significant increase in the $ASdist$ value; for instance, increasing d slightly from 0 to 5 results in a substantial rise 90\% in $ASdist$ to 0.9.
This means the $ASdist$ value is highly responsive to increases in distance, making it more effective to detect hijacks (especially for hijacks in close proximity). %which in turn helps hold the effectiveness of ROV as a hijack prevention measure.
%Additionally, the utilization of $ASdist$ would provide a more direct and easy-to-understand method to differentiate between the two types of routes compared to $d$.
As depicted in Figure \ref{fig:ASdist_cdf}, a large $ASdist$ value can effectively distinguish between hijacks and benign conflicts. %achieving approximately 95\% accuracy in identifying hijacks and 80\% accuracy in identifying benign conflicts.

\begin{figure}[b!]
  \centering

 %\begin{minipage}[b]{0.29\linewidth}
 %   \centering
 %   \includegraphics[width=\linewidth]{./images/d_distance.pdf}
 %   \vspace{-25pt}
 %   \caption{\small CDFs of $d$.}
 %   \label{fig:d_cdf}
 % \end{minipage}
 % \hfill
   \begin{minipage}[b]{0.45\linewidth}
    \centering
    \includegraphics[width=\linewidth]{./images/actand.pdf}
    \vspace{-25pt}
    \caption{\small Transformation.}
    \label{fig:arctan}
  \end{minipage}
  \hfill
  \begin{minipage}[b]{0.45\linewidth}
    \centering
    \includegraphics[width=\linewidth]{./images/ASdist_haversine.pdf}
    \vspace{-25pt}
    \caption{\small CDFs of $ASdist$.}
    \label{fig:ASdist_cdf}
  \end{minipage}
  
  \label{fig:both}
\end{figure}

\section{Grid Search Settings} 
\label{app:grid_search_settings}
The grid search settings for the other four models (DT, SVM, KNN, and RF) are presented in Table \ref{tab:identi_parameters}.

  \begin{table}[t!]
    \centering
    \fontsize{6}{6}\selectfont
%    \begin{adjustbox}{width=0.8\linewidth,center}
    \renewcommand{\arraystretch}{1.2}
    
    \begin{tabular}{cccc}
        \toprule
        \textbf{Model} & \textbf{Parameter} & \textbf{Search range} & \textbf{Step size} \\
        \midrule
        \multirow{2}{*}{DT} & $max\_depth$ ($d$) & [3, 20] & 2 \\
        & $min\_samples\_leaf$ ($l$) & [2, 20]  & 2 \\
        SVM & $C$ & [0.01, 0.1, 1, 10, 100, 1000, 10000] & - \\
        KNN & $n\_neighbors$ ($k$) & [1, 20] & 2\\
        \multirow{2}{*}{RF} & $max\_depth$ ($d$) & [3, 20] & 2 \\
        & $min\_samples\_leaf$ ($l$) & [2, 20]  & 2 \\
        \bottomrule
    \end{tabular}
    %\end{adjustbox}
    \caption{\small Grid search settings for the primary parameters of the DT, SVM, KNN, and RF models, including the search ranges and step sizes. $max\_depth$ means the maximum depth of decision trees. $min\_samples\_leaf$ represents the minimum number of samples in a leaf node. $C$ refers to a penalty parameter to control error tolerance. $n\_neighbors$ refers to the number of the closest neighbors used for predictions.}
    \label{tab:identi_parameters}
\end{table}

\section{Statistics of Hijacks During Four Surge Period} \label{app:statistics}
 We present the statistics of hijacks identified during four surge periods in Table \ref{tab:specific_days_res}.



\begin{table*}[t!]
   \renewcommand{\arraystretch}{1.2}
   \fontsize{8}{8}\selectfont %\scriptsize
  \centering
 % \begin{adjustbox}{width=0.8\linewidth,center}
  \begin {tabular}{cccccccc}
  \toprule
    \textbf{Dates} & \textbf{Hijacks (\#)} & \textbf{Attackers (\#)} & \textbf{Prefixes (\#)} & \textbf{Victims (\#)} & \textbf{Att$_{m}$} & \textbf{Vic$_{m}$} & \textbf{Peer$_{m}$} \\
    \midrule
    
   2022-10-26 to 2022-10-29 & 14,636 & 3,827 & 5,438 & 328 & AS396982 (191) & AS11351 (2,008) & AS212483 (14,026)    \\
    2022-11-07 to 2022-11-08 & 7,086 & 450 & 5,990 & 243 & AS396982 (4,017) & AS8070 (3,402) & AS212483 (6,719)  \\
    2022-11-16 to 2022-11-21 & 7,940 & 3,185 & 2,363 & 389 & AS36351 (102) & AS8075 (1,183) & AS212483 (7,298) \\
    2022-11-23 to 2022-11-30 & 84,577 & 26,659 & 22,725 & 699 & AS13335 (491) & AS6128 (16,790) & AS212483 (83,916)  \\
   \bottomrule
   \end{tabular}
%   \end{adjustbox}
   \caption[Hijacks identified during four surge periods.]{\small Hijacks identified during four periods. "Hijacks", "Attackers", "Prefixes" and "Victims" refer to the number of detected hijacks, involved perpetrator ASNs, announced prefixes, and affected ASNs, respectively. "Att$_{m}$", "Vic$_{m}$" and "Peer$_{m}$" indicate the attacker ASN who initiated the most hijacks, the victim ASN who was affected by the most hijacks, and the peer ASN who fed the most hijacks to the RouteViews collector (corresponding hijacks are enclosed in brackets). %"Events" means the number of events captured during this time.
   }
   %and appearing in the group of frequent anomalous events
\label{tab:specific_days_res}
\end{table*}

\section{CDFs of Occurrences and Frequencies for Events}\label{app:CDFs_events} 
Figure \ref{fig:hijacks_occur_fre_cdf} illustrates the distributions of event occurrences and frequencies for 513 ASes
that frequently initiated events.

\begin{figure}[b!]
  \centering
  \begin{minipage}[b]{0.43\linewidth}
    \centering
    \includegraphics[width=\linewidth]{./images/event_occurrence.pdf}
  \end{minipage}
  \hfill
  \begin{minipage}[b]{0.43\linewidth}
    \centering
    \includegraphics[width=\linewidth]{./images/event_occurrence_fre.pdf}
  \end{minipage}
  \vspace{-10pt}
  \caption{\small CDFs of event occurrences and frequencies for 513 ASes.}
  \label{fig:hijacks_occur_fre_cdf}
\end{figure}

\section{Root Causes of Hijacking Events} \label{app:root_cause_hijack}
We analyze the hijacking events identified through email surveys, with a focus on investigating their underlying causes.
To achieve this, we sent hijack reports to the relevant ASes via email.
Excluding the events attributed to AS212483 and the frequent anomalous events, we were left with 220 hijack events. We sent 194 emails to potential perpetrator ASes and another 194 emails to potential victim ASes and received 10 responses that confirmed the events.
The factors causing these hijacks are described below. \\

 \noindent\textit{Technical errors.} Six events were due to technical issues. For example, AS52195 who owned 46.149.208.0/20, announced an invalid route to the prefix 20.46.149.208/32. The network operators of AS52195 clarified that this anomaly was potentially caused by a technical error as the announced prefix 20.46.149.208/32 can be generated by a circular byte right shifting on 46.149.208.0/20. \\
 
\noindent\textit{Owner changes.}
We identified two events arising from owner changes. On March 31, AS210837 announced the route to the prefix 45.151.234.0/24, despite AS12695 being the authorized origin according to ROAs. Upon contacting AS12695, we learned that they started announcing this prefix on March 30. Given that AS210837 was the previous owner of this prefix, the observed hijack likely occurred due to a delay on their part in updating the configuration of BGP routers.\\

\noindent\textit{The perpetrator AS is in the middle.}
In our previous study of events related to AS212483, we found that BGP anomalies could be initiated not only by the origin AS. In one of the other two examples, AS134196 was identified as hijacking the prefix 24.103.24.0/23, which belonged to AS12271. The BGP path of the hijack was [..., AS3223, AS55933, AS134196]. When we contacted AS134196, they believed the routing path was abnormal because they expected AS3223 to be their immediate upstream provider for network traffic. We then investigated the announcements from AS55933 on the same day and found that AS55933 had also originated routes for this prefix.
This indicated that AS55933 was highly suspected of initiating the event, rather than the origin AS in the path.
%Our research suggests that identifying the perpetrator of BGP hijack events may not be a straightforward process.



\section{Addressing Challenges in ML Application} \label{app:ml_challenges}
We then discuss the key challenges encountered in developing and deploying the ML-based classifier, the solutions implemented, and potential solutions for future work.

\subsection{Ground Truth Data}
In collecting ground truth data for developing the ML-classifier, we leverage the long-duration nature of benign routes to identify and extract benign conflicts from RPKI-invalid BGP routes. While this approach helps ensure that the collected benign conflicts are likely benign, it may miss short-lived conflicts, such as those lasting only a few hours or days. Consequently, the absence of transient benign conflicts in the ground truth data could impact data diversity and, in turn, affect model performance.

Recall that we collected BGP hijacks from BGPmon. Although BGPmon is highly regarded as a reliable source, it still encounters issues with false positives. To ensure the cleanliness of the collected data, we performed pre-processing and filtered out instances mistakenly identified as hijacks. However, due to the lack of transparency in BGPmon’s operational process, we cannot guarantee that the remaining BGP hijacks are genuinely malicious.

A potential improvement would be to incorporate both benign conflicts and BGP hijacks confirmed through email surveys to retrain the ML-classifier in future work.

\subsection{Model Performance and Reliability}
Multiple ML models are used to implement the classifier. Their effectiveness is evaluated and compared to select the optimal model for identification. Cross-validation and grid search are also employed to achieve the best performance.

Additionally, we introduce a post-analysis mechanism to verify the hijacks identified, aiming to correct any errors where the ML-classifier might misidentify benign conflicts as hijacks. Notably, benign conflicts flagged by the classifier are not immediately deemed benign but are placed in quarantine for further review before whitelisting. This approach helps mitigate the impact of the classifier's failure to detect hijacks on the whitelist.

As mentioned in Section \ref{subsubsec:robustness}, the ML-based classifier may be vulnerable to adversarial attacks, where crafted hijacks could evade the classifier's detection and contaminate the whitelist.
Our quarantine mechanism for further inspection can mitigate such a risk, ensuring the reliability of the whitelist.

\subsection{Model Generalization}
As mentioned in the previous section, our feature set focuses solely on capturing the relationships between two conflicting origins rather than the behavior patterns underlying the routes.
This can mitigate the impact of the evolving behavior of anomalous routes over time on identification, thereby minimizing the model generalization issues to newly emerging data.

Additionally, our evaluation of new instances supports the model's ability to generalize across different scenarios. Even if the model incorrectly identifies a route, this issue can be addressed in subsequent analysis, including post-verification of hijacks and further review of quarantined routes.

\subsection{Model Interpretability}
The inherent opaqueness of ML models can obscure their decision-making processes. We also quantify the importance of each feature in the identification.
This analysis assists in comprehending the inner workings of the model, enhancing the interpretability of its outputs.


\section{Additional Future Applications} \label{app:add_future_use}
As previously discussed, prefix hijacking presents serious risks including data interception, traffic redirection to malicious destinations, and blackholing of legitimate traffic. These threats pose significant security concerns for various Internet infrastructure services. Adversaries can exploit prefix hijacking to disrupt critical services such as DNS, Web, email, and NTP. For instance, attackers can reroute DNS, HTTP, or NTP requests to servers under their control, enabling them to deliver fraudulent responses to unsuspecting clients.
RPKI's ROV can offer an effective defense against prefix hijacking, protecting these services from such attacks.

However, ROV also filters BGP announcements with benign conflicts.
Networks hosting open DNS resolvers, web servers, email servers, or NTP servers may enforce ROV, or ROV enforcement may be present along the path to these networks. In such cases, clients might inadvertently originate BGP routes with ROA violations for their IP addresses, which ROV flags as invalid and discarded.
As a result, server responses fail to reach clients due to missing routing information, leading to service unavailability.
Conversely, if networks hosting these critical servers initiate announcements with conflicting origins, ROV filtering can affect the visibility of their IP addresses across the Internet, causing service unreachability for clients.

Similar issues can also arise within Cloud environments, potentially impeding users' access to their critical data, such as banking information, stored in the Cloud.
Furthermore, IoT devices with RPKI-invalid IP addresses may face significant communication problems with other IoT devices and central IoT servers, potentially leading to malfunctions in IoT applications.
Disruptions stemming from ROV filtering benign conflicts could lead to trust erosion, reputational damage, and economic losses for Cloud or IoT service providers.

Looking ahead, the widespread deployment of ROV could exacerbate the impact of benign conflicts on various services within the Internet. Adopting \lov\ to prevent ROV from filtering benign conflicts can help mitigate this impact, ensuring the availability and integrity of these services.