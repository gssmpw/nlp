\begin{abstract}
The long history of misconfigurations and errors in RPKI indicates that they cannot be easily avoided and will most probably persist also in the future. These errors create conflicts between BGP announcements and their covering ROAs, causing the RPKI validation to result in status {\tt invalid}. Networks that enforce RPKI filtering with Route Origin Validation (ROV) would block such conflicting BGP announcements and as a result lose traffic from the corresponding origins. Since the business incentives of networks are tightly coupled with the traffic they relay, filtering legitimate traffic leads to a loss of revenue, reducing the motivation to filter invalid announcements with ROV.

In this work, we introduce a new mechanism, \lov, designed for whitelisting benign conflicts on an Internet scale. The resulting whitelist is made available to RPKI supporting ASes to avoid filtering RPKI-invalid but benign routes. Saving legitimate traffic resolves one main obstacle towards RPKI deployment. We measure live BGP updates using \lov\ during a period of half a year and whitelist 52,846 routes with benign origin errors.
%We provide our prototype implementation of \lov\ at {\url{https://github.com/zsjstart/LOV}}.

%We perform experimental evaluations of \lov, demonstrating 95\% accuracy in detecting benign conflicts and $\sim$100\% in identifying BGP hijacks.
%We show experimentally that \lov\ identifies and saves 95\% of benign conflicts in the Internet, while maintaining the ROV's performance in identifying and blocking hijacks with an accuracy of \textasciitilde100\%.
%We analyze live BGP updates using \lov\ during a period of half a year and find that on a daily average, 79\% of the invalid routes were caused by benign misconfigurations.
%Finally, we validate the findings of \lov\ by conducting a survey of network operators and by cross-referencing them with the hijack events detected by BGPmon.
%Long history of errors and misconfigurations in BGP routers and in RPKI shows that errors cannot be easily avoided. %Some errors are caused by human factor while others are an inevitable artifact of networking practices and resources management, such as network engineering or resource allocation. 

%Such errors lead to lost traffic and routing anomalies. result in conflicts between BGP announcements and their covering ROAs. Networks that enforce RPKI filtering (i.e., ROV) block such benign conflicts and as a result lose legitimate traffic from those origins. Since the business incentives of networks are tightly coupled with the traffic that they relay, loss of traffic in inter-domain routing leads to loss of revenue, hence reducing the motivation to deploy ROV.
%\vspace{-10pt}
\ignore{
Errors in RPKI deployments and in the announced BGP routes can lead to route leaks and can result in filtering legitimate traffic. Although studies show that both types of errors are common, their number does not decrease indicating that errors are not trivial to avoid.\\
\indent In this work we develop a mechanism, we call \lov, to differentiate between benign errors and real BGP incidents. \lov\ uses machine learning to identify misconfigurations in RPKI and in BGP routes. On the one hand \lov\ identifies benign errors that cause conflicts between RPKI authorizations of ownership and announced BGP routes and prevents legitimate traffic from being blocked. On the other hand \lov\ blocks harmful hijacks and route leaks. \lov\ offers the benefits of RPKI validation, while resolving one of the main obstacles towards deployment of RPKI: {\em benign errors}, hence paving the way for deployment of path validation mechanisms.\\ 
\indent We integrate \lov\ into a Routinator implementation of RPKI validator and evaluate \lov\ on a real-world dataset with major BGP hijacks and route leak incidents, demonstrating detection accuracy of 99.9\%. We deploy \lov\ on the Internet, and report evaluation results over a period of half a year: \lov\ found that 81.1\% of the conflicting routes were caused by benign misconfigurations, which would have been dropped by RPKI validators disconnecting legitimate origins.
%If border routers had adopted ROV, these routes would have been dropped.
Alternately, \lov\ detected 344 hijacks and 451 route leak incidents, of which 36 were confirmed by corresponding network operators in a survey that we conducted.
We provide our code and datasets at {\small \path{https://github.com/zsjstart/LOV}}.
%We provide our code and datasets at {.\small \path{https://anonymous.4open.science/r/LOV-11C3}}.
}
%and find that 81.1\% of all the conflicts are caused by benign errors either in ROAs or in BGP announcements. We analyze the conflicts and detect a total of 1,067,721 unique invalid routes due to errors in ROAs and 542,531 unique invalid routes due to BGP errors. These benign conflicts are filtered by ROV. % and the corresponding traffic from these origins is blocked. 
\ignore{
In addition to identifying and “saving” benign conflicts,  \lov\ also extends ROV in identifying and filtering route leaks. Route leaks are not filtered by RPKI since route leaks also do not introduce more specific routes and neither do they alter the origin of the route, hence the origin in route leaks is the legitimate owner of the announced prefix. Although route leaks introduce a problem to Internet stability and occur often, there is currently no efficient solution to this. We show experimentally how \lov\ detects and blocks route leaks.
}
%We evaluate \lov\ on a real-world dataset of BGP announcements including 12 major BGP hijacks and route leaks incidents, demonstrating a low false positive rate of 0.6\%, and a high true positive rate of 98.6\%. %We studied 12 major BGP incidents, showing that \lov\ effectively identifies and blocks BGP hijacks and route leaks. %Finally, we evaluated the effectiveness of \lov\ on Internet reachability and BGP convergence through simulations.



\ignore{
RPKI has been proposed for securing BGP for more than ten years; still, only a limited number of networks adopt ROV.
Without widespread adoption of ROV, RPKI does not provide any security benefits.
One of the main factors hindering ROV's wide adoption is invalid ROAs.
ROA misconfigurations result in conflicting ROA and BGP announcement pairs, which appear similar to prefix hijacks. ROV will filter such conflicts causing routers to block legitimate traffic.
Besides, ROV cannot detect path anomalies because the AS number at the end of the path is indeed the legitimate owner of the prefix announced.
These shortcomings make network operators less motivated to deploy RPKI in practice.

Our work aims to promote the practical deployment of RPKI.
In this work, we present a new extension of ROV, called LOV, that builds the validation process in an intelligent mode using machine learning technologies.
LOV extends ROV with the following properties:
(1) LOV can rescue conflicts due to erroneous ROAs. (2) LOV can identify path anomalies.
(3) LOV can whitelist benign routing anomalies.

We evaluated LOV on the real-world dataset, showing that LOV has a low false positive rate of 0.6\%, and a high true positive rate of 98.6\%, outperforming ROV. Also, we show that LOV can save BGP announcements with ROA misconfigurations from being dropped.
With LOV, we studied 12 major BGP incidents, showing that LOV can effectively combat BGP hijacks and route leaks.
Besides, we measured historical BGP updates using LOV to characterize conflicts and malicious events.
Finally, we evaluated the impact of LOV on Internet reachability and BGP convergence through simulations, showing that LOV is readily deployable.
}
\end{abstract}