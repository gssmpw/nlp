\vspace{-10pt}
\section{ML-Classifier} \label{sec:lov_classifier}
This section implements the ML-classifier that is utilized to categorize RPKI-invalid routes into benign conflicts and BGP hijacks. We apply five widely employed supervised learning classification algorithms: Decision Tree (DT), Support Vector Machine (SVM), K Nearest Neighbor (KNN), Random Forest (RF), and Naive Bayes (NB), for validating our approach. The performance of each model is evaluated, and the optimal model is selected as the final classification model.
\vspace{-10pt}
\subsection{Features} \label{sec:lov_features}
We first introduce a set of features that are utilized to train an ML model to distinguish between benign conflicts and BGP hijacks.
Unlike RPKI's ROV, which compares two origins—the origin AS in the prefix announcement and the authoritative AS specified in the ROA—the core idea of our approach to route classification is to identify connections between the two origins.

Intuitively, a BGP announcement is considered benign when the two origins are closely linked.
The features selected cover various associations between two origins, as detailed in Table \ref{tab:features}, along with the data sources used to compute specific features.
Certain features we select are inspired by the previous work \cite{chung2019rpki}, which examines RPKI-invalid announcements. By learning from these features, the ML model captures the degree of tightness between two origins for a given new announcement. The tighter the connection between the two origins, the higher the likelihood of it being benign.\\

\noindent\textbf{OriginMatch.} Conflicts may arise due to incorrect configurations of the MaxLength parameter in ROAs \cite{gilad2017maxlength}. When such conflicts occur, we set this feature value to 1 because the source in the BGP announcement matches the origin in the ROA, indicating that the announcement originates from the IP address prefix's authorized owner.

We retrieve validated prefix origin data (i.e., ROAs) from the RPKI repositories to determine the outcome of ROV and the feature \textit{OriginMatch}.

\noindent\textbf{PC.} Conflicting origins in RPKI validation may sometimes involve a provider and its customers.
For instance, consider a provider AS3215 with prefix \texttt{193.2.0.0/15} that delegates some of its IP prefixes to its two customers: AS1272 \texttt{193.2.35.0/24} and AS8361 \texttt{193.2.155.0/24}. However, the provider AS3215 issues a ROA for its prefix \texttt{193.2.0.0/15} for origin AS3215, but the customers do not issue ROAs for their prefixes. When the customers announce routes for the prefixes, conflicts occur. 
Such conflicts are typically benign as the customer AS is unlikely to hijack its providers.
Thus, if a provider and customer relationship is found between the two conflicting origins, the feature value is designated as 1.

We retrieve the information about AS relationships from CAIDA \cite{CAIDA} to compute the feature \textit{PC}.

\noindent\textbf{MOAS.} When multiple ASes advertise routes for a given prefix, but only one AS has created the ROA for it, this scenario can result in a multiple origin AS (MOAS)
conflict \cite{chin2007characteristics,zhao2001analysis}. MOAS conflicts can arise for reasons such as multi-homing, multinational companies, or organizations with multiple AS numbers \cite{chin2007characteristics}, which are generally considered benign.
If the two conflicting origins of a route exhibit MOAS conflicts (e.g., they are from the same organization), we assign 1 to this feature.

In this work, we focus solely on AS organization information from CAIDA to compute the \textit{MOAS} feature. Additional data that could be used in computing \textit{MOAS} will be explored in future work.

\noindent\textbf{Parent.} In some cases, a parent organization splits a prefix into a set of more specific prefixes, assigning them to its child organizations with their own AS numbers. Assume separate ROAs are created for both the parent and the child prefixes. When the parent AS announces the prefix on behalf of its child, a conflict arises. If we detect that the originating AS has another prefix in ROAs, which is less specific than the declared prefix, the feature value is set as 1.

The computation for the feature \textit{Parent} relies on the ROAs from the RPKI repositories.

\begin{table}[t!]
  \centering
  \renewcommand{\arraystretch}{0.8}
  \fontsize{5}{5}\selectfont
  %\begin{adjustbox}{width=0.8\linewidth}
  \begin {tabular}{c|c|c}
  \toprule
  \textbf{Feature} & \textbf{Definition} & \textbf{Source} \\\midrule
    \begin{tabular}{@{}c@{}}
                OriginMatch \\
                   PC \\
                   MOAS \\
                   Parent \\
                   Depen \\
                   AltSources\\
                   ASdist \\
                 \end{tabular} & \begin{tabular}{@{}c@{}}
                   
                   The origin matches but the MaxLength mismatches. \\
                   A relation of provider and customer exists between two origins.\\
                   Two origins belong to MOAS.\\
                   A parent prefix is already in ROAs.\\
                   Two origins have a strong dependency. \\
                   Other trusted sources validate the route as valid.\\
                   The geographical distance between two origins. \\
                   \end{tabular} & \begin{tabular}{@{}c@{}}
                   
                   RPKI \\
                   CAIDA\\
                   CAIDA\\
                   RPKI\\
                   IHR\\
                   IRRs\\
                   CAIDA, GeoIP\\
                   \end{tabular} \\\bottomrule
   \end{tabular}
   %\end{adjustbox}
\caption[Features along with their definitions.]{\small Features along with their definitions. "Source" refers to the data source for feature computation.}
    \label{tab:features}
    \vspace{-15pt}
\end{table}


\noindent\textbf{Depen.}
When customers acquire anti-DDoS services from a provider, they may authorize the provider to announce routes for their IP prefixes. The provider then directs the traffic destined for these prefixes back to the customer. In this arrangement, discrepancies between the published ROAs and the actual prefix announcements can arise. For instance, the provider may publish ROAs for prefixes that are owned by the customer, while announcing these prefixes with the customer’s ASN appended to the end of the AS path. This can result in benign ROA conflicts.

Note that, the service provider and the service customer may not have a direct BGP relationship (e.g., PC). We propose a new relationship between two conflicting origins, referred to as \textit{Depen}, which represents the interdependency between the two origins. For instance, in the scenario mentioned above, the customer demonstrates a dependency on the service provider for accessing the Internet.

The Internet Health Report (IHR) \cite{IHR} introduces the AS hegemony metric to estimate AS interdependencies using global routing information \cite{fontugne2017hegemony,fontugne2018thin}. AS hegemony comprises two types: local and global. Local hegemony measures the dependency between any two ASes, while global hegemony indicates the centrality of an AS on the Internet. 

We leverage local hegemony to compute the \textit{Depen} feature. In cases of bidirectional dependencies between two origins, we assign the larger local hegemony value as the \textit{Depen} value.
The AS hegemony metric ranges from 0 to 1, with greater values indicating stronger dependencies.
%Inspired by this, we use AS hegemony to assess the dependency between two conflicting origins.
When two origins exhibit a greater AS hegemony value, it signifies a closer connection between them. This, in turn, suggests that the route is likely benign.

\noindent\textbf{AltSources.} This feature is utilized to check if the RPKI-invalid route has a valid validation outcome when using other reliable sources. When it does, it implies a potential (albeit undetermined) association between two conflicting origins. 

The IRRs \cite{IRR} are distributed databases managed by RIRs like APNIC, RIPE NCC, or Internet Service Providers (ISPs) such as Level 3. They hold valuable routing information, including AS numbers and their corresponding IP addresses.
Therefore, the IRRs can be considered to be alternative sources for route validation. 
To calculate the feature \textit{AltSources}, we use the \textit{whois} tool to extract IRR entries (i.e., AS numbers and IP prefixes).
Once the RPKI-invalid route is verified as valid based on IRR data, its feature value is assigned as 1.

\noindent\textbf{ASdist.} This feature is utilized to assess the degree of geographical proximity between two conflicting origins. Intuitively, unlike benign conflicts, origins involved in a prefix hijack are often situated at a greater distance from each other, as there are typically no direct connections between them.
Given two origins, $AS_{x}$ and $AS_{y}$, we denote $g_{x}$ ($\phi_x$, $\lambda_x$) and $g_{y}$ ($\phi_y$, $\lambda_y$) as the geographical locations of the two ASes, with $\phi$ and $\lambda$ being latitude and longitude coordinates, respectively.
Then we compute great circle distance (which is commonly used to calculate the shortest path between two locations on Earth) between $g_{x}$ and $g_{y}$, denoted as $d$.
%To illustrate the geographical distance between the two origins, we computed both Euclidean distance and great circle distance using $g_{x}$ and $g_{y}$, respectively. Our evaluation of these methods showed that they yielded comparable performance. Given the computational simplicity of Euclidean distance, we opted to use it for feature calculation, denoted as $d$.
%The value of $d$ ranges from 0 to $d_{max}$\footnote{$d_{max}$ is reached when the two points are at the Earth's north pole (90, -180) and south pole (-90, 180), and it is approximately 402.}.
%with $d_{max}$ being the theoretical upper bound of Euclidean distances between any two points on the Earth's surface.

We utilize MAXMIND's GeoIP databases to obtain latitude and longitude information for IP prefixes within origin ASes. Since each AS may contain prefixes with different geographic locations, one method is to compute the geometric median of these locations to represent the AS’s location. However, this can be computationally intensive. To simplify, we use the location of the announced prefix to represent the location of the RPKI origin AS and calculate the great circle distance between this location and each prefix (if the number of prefixes exceeds 500, we randomly select 500 prefixes) within the origin AS in BGP announcements. The median of these distances is then used as the feature value $d$.

Figure \ref{fig:d_cdf} illustrates the cumulative distribution functions (CDFs) of distances ($d$) we analyzed in 100 benign conflicts and 100 prefix hijacks randomly selected from our ground truth dataset (see details about the ground truth data in the following paragraphs).
The plot shows that benign conflicts mostly (90\%) exhibit a smaller distance value (less than 50km) compared to hijacks, which aligns with our previous assumption.
%It can be observed that specifying a small value for $d$, e.g., 5, as a threshold can distinguish between the two types of routes effectively, with 80\% accuracy in detecting benign conflicts and 90\% accuracy in detecting BGP hijacks.
%Specifically, the greatest value of $d$ observed for benign conflicts is less than 200, whereas hijacks demonstrate a maximal distance value of approximately 400.
%Specifying a small value for $d$ as threshold is expected to effectively identify benign conflicts. However, this method may encounter a high false negative rate in detecting hijacks. For instance, with the $d$ value of 90 (recall that the greatest value of $d$ is approximately 400), while the accuracy in identifying benign conflicts reaches about 95\%, the misidentification rate in detecting hijacks is around 60\%.
\begin{figure}[t!]
    \centering
    \includegraphics[width=0.7\linewidth]{./images/d_haversine.pdf}
    \vspace{-15pt}
    \caption{\small CDFs of $d$.}
    \label{fig:d_cdf}
\end{figure}

Compared with other features that fall into [0, 1], $d$ exhibits a larger upper bound. The larger values may cause bias in the classification model (e.g., in cases of using SVM). Hence, we scale this feature before training.
In our experiments, we discovered that scaling $d$ into the range of [0, 1] using a commonly used method like minimum-maximum scaling may result in a number of false negatives, i.e., misidentifying hijacks as benign conflicts. These mis-recognitions often occur when hijacks entail two conflicting origins that are in close proximity.
To address this, we apply the arctangent function to $d$, as given by the formula: {$ASdist = \frac{2}{\Pi}arctan(d)$}, resulting in a new distance metric $ASdist$ called AS distance.
Compared to $d$, $ASdist$ is expected to be more effective at distinguishing between benign conflicts and hijacks in close proximity (see more details about this method in Appendix \ref{app:feature_scale}).
%In scenarios where multiple ROA origins are present, we search for relations among all matching origin pairs. The mean of $ASdist$ is calculated.
\vspace{-5pt}
\subsubsection{Utilization of Default Values}
We expect benign conflicts to show a closer relationship between the two origins, characterized by more connections and shorter distances between them.
The default value for the features is 0, except for $ASdist$, which is set to 1, indicating no relations between conflicting origins. These default settings make instances more likely to be categorized as BGP hijacks, which helps uphold the effectiveness of ROV. Feature computation often utilizes reliable data sources, which rarely become unavailable.
In cases where required information is missing, default values are used, which may result in benign conflicts being misidentified as hijacks. This is acceptable because, although our primary goal is to detect benign conflicts, it is premised on maintaining ROV's hijack prevention.
\vspace{-10pt}
\subsubsection{ML Identification}
%We leverage machine learning (ML) technologies to effectively identify invalid routes caused by benign conflicts.
While using these features as rules to distinguish between benign conflicts and real hijacks could be effective, such an approach may overlook the overall relationship strength (i.e., tightness) between conflicting origins in a given BGP route. In contrast, ML allows these features to work collectively, capturing nuances that humans might miss, potentially reducing the impact of inaccuracies in certain data sources. For instance, a \textit{PC} relationship might appear in some hijacks, possibly due to errors in CAIDA, while a rules-based approach could falsely classify this as a benign conflict. Moreover, features like \textit{ASdist} require predefined thresholds for classification. Heuristic-based methods, which rely on manual thresholds, may not adapt well to changing conditions. On the other hand, ML methods automatically identify patterns in input features, often resulting in better classification performance. Therefore, in this work, we opt to use ML methodologies for classification.



\vspace{-5pt}
\subsection{Implementation, Evaluation, and Analysis} \label{subsec:groundtruth}

\subsubsection{Data Collection} We first collect a real-world ground truth dataset of ROA-violating routes, including benign conflicts and BGP hijacks, for training and validating the classifier.

\noindent\textbf{Benign conflicts.} According to previous studies \cite{vervier2015mind, wijchers2014bgp, mahajan2002understanding}, hijacked BGP announcements are usually short-lived (e.g., lasting only a few hours or days). In other words, long-lived announcements are generally considered benign. Using this basic concept, we collect BGP routes with ROA violations.

We first utilized Routinator \cite{Routinator}, a widely recognized and reliable open-source RPKI relying party validator (ROV), to validate BGP announcements that occurred between May 30, 2022, and June 30, 2022, collected from the RIB data of RouteViews.
We focused specifically on routes that remained stable and persistent throughout the period, ensuring their benign nature. Note that we may encounter multiple announcements for the same prefix, originating from the same AS but with different AS paths. In such cases, we retained only a single route. The validation process identified 340,485 RPKI-valid announcements and 9,223 RPKI-invalid ones.
The invalid announcements are considered to be routes with benign ROA conflicts, originating from 1,377 different ASes.%We excluded the resulting 340,485 RPKI-valid announcements and were left with 9,223 RPKI-invalid but long-lived announcements from 1,377 ASes, which corresponded to routes with benign conflicts.
 %We randomly divided the benign conflicts into two groups, selecting 2,000 instances for training data and using the remaining 7,223 for holdout data.
 
\noindent\textbf{BGP hijacks.} BGPmon \cite{BGPMon} is a reliable source that provides various BGP events from different ASes at different times \cite{van2019analysing}. In addition, BGPmon offers a specific instance for each event, ensuring the diversity of the ground truth. We collected BGP hijacking instances from BGPmon, which occurred between December 2021 and September 2022.
We then cleaned the data by filtering out cases that either conformed to ROAs or were not covered by ROAs, resulting in 415 hijacks initiated by 246 ASes.

\noindent\textbf{Overview of Ground truth data.} Consequently, we collected 9,223 benign conflicts and 415 hijacks. Given that there is a relatively large imbalance (22:1) between the two classes, we randomly select 2,000 instances from the benign conflict dataset. This reduction helps alleviate the imbalance, making it easier to handle while preserving the diversity of benign conflicts.
Then, we increase the number of instances in the minority class to 2,000 using random oversampling. This involves randomly copying instances from the minority class, which is often well-suited for smaller imbalances, resulting in a ground truth set including 2,000 benign conflicts and 2,000 BGP hijacks. The remaining 7,223 will be used as new or unseen data for evaluation in Section \ref{subsec:holdout}.

\noindent\textbf{Limitations.}
Collecting ground truth data from the real world is often challenging because network operators typically keep their data private and confidential. To address this issue, we leverage the long-duration nature of benign routes to identify and extract benign conflicts from RPKI-invalid BGP routes. While this approach helps ensure that the collected benign conflicts are likely to be genuinely benign, it may miss short-lived conflicts, such as those lasting only a few hours or days. Consequently, the absence of transient benign conflicts in the ground truth data could impact data diversity and potentially bias model performance. On the other hand, although BGPmon is highly regarded as a reliable source, it may still encounter issues with false positives. We filtered out instances mistakenly identified as hijacks to ensure the cleanliness of the collected data. However, due to the lack of transparency in BGPmon’s operational process, we cannot guarantee that the remaining BGP hijacks are genuinely malicious. Moreover, while random oversampling can help address class imbalance, it might introduce certain biases (e.g., distorting data distribution). Future work will explore more effective strategies to construct ground truth data.
\vspace{-5pt}
\subsubsection{Cross-validation on Ground Truth Data}
The classifiers based on the ML models mentioned above are implemented using the scikit-learn (sklearn) library in Python 3.8 and validated on the ground truth data.
The data is split into training and testing sets using an 8:2 ratio. 10-fold cross-validation is conducted to assess the model performance, using measures including macro-averaging precision, recall, and F1-score.
Additionally, grid search is employed for hyperparameter optimization, focusing on the primary parameters of the models. Since NB has relatively few tunable hyperparameters, the default settings are employed.
The grid search settings for the other four models (DT, SVM, KNN, and RF) are presented in Table \ref{tab:identi_parameters} in Appendix \ref{app:grid_search_settings}.

Table \ref{tab:lov_classifiers_res} displays the optimal hyperparameters searched for the five binary classification models. In addition, the evaluation results on the ground truth data set are also presented in the table. It can be observed that, the DT, KNN, and RF models achieve higher accuracy than the other two models, with precision, recall, and F1-score all around 99\%. 


\begin{table}[t!]
  \centering
  \fontsize{7}{7}\selectfont
%  \begin{adjustbox}{width=0.8\linewidth,center}
  \renewcommand\arraystretch{1}{
\begin{tabular}{ccccccc}\toprule
 & \textbf{DT} & \textbf{SVM}  & \textbf{KNN} & \textbf{RF} & \textbf{NB}\\\midrule
%\textbf{Paras} & $d$ = 16, $l$ = 2 & $C$ = 10000  & $k$ = 1 & $d$ = 15, $l$ = 2 & - \\ \midrule
 \textbf{Precision} & 98.7\% & 98.1\% &  99\% & 98.8\% & 97.2\% \\
 \textbf{Recall} & 98.7\% & 98.1\% & 99\% & 98.8\% & 97.1\% \\
 \textbf{F1-score}  & 98.7\%  & 98.1\% & 99\%  & 98.8\% & 97.1\% \\ \midrule
 \textbf{New (``Benign'')} & 96.1\% & 94.4\% & 94.8\% & 94.6\%  & 94.2\% \\
 \textbf{New (``Hijack'')} & 98.1\% & $\sim$100\% (1)  & $\sim$100\% (4) & $\sim$100\% (1) & 96.9\%  \\
\bottomrule
\end{tabular}
}
%\end{adjustbox}
 \caption[Evaluation results of five ML models.]{\small %Optimal hyperparameters (``Paras'') for the five binary classifiers, as well as 
 Evaluation of macro-averaging precision, recall, F1-score obtained on the ground truth, accuracy when tested on the new dataset (``Benign'' and ``Hijack'' refer to benign conflict instances and BGP hijack instances). We note the number of detection errors in the brackets in cases of $\sim$100\%.}
    \label{tab:lov_classifiers_res}
    \vspace{-15pt}
\end{table}
\vspace{-10pt}
\subsubsection{Evaluation on New Data} \label{subsec:holdout}
Subsequently, we collect a new set for evaluating the effectiveness of the five ML models we trained above, specifically assessing how well the model generalizes to new, unseen data.
Recall that, we only selected 2000 instances as ground truth data. The remaining 7,223 benign conflicts are used subsequently as new data for the evaluation.
Additionally, we obtain new hijacks from another well-known data source. We describe the collection procedure in detail below.

Qrator Labs \cite{QratorLab} provides a BGP monitoring service called Radar that assists researchers and network operators in detecting network anomalies. In contrast to BGPmon, Radar focuses more on malicious events that have a significant impact on Internet services at the global routing level. To create a new data set as the holdout data, we collected 8 BGP hijacking events from Radar, which happened between the years 2020 and 2022, as listed in Table \ref{tab:bgp_incidents_res}. BGPStream also provides access to historical BGP data. We applied it to collect BGP announcements associated with the aforementioned incidents, through three RouteViews collectors: \textit{amsix}, \textit{wide}, and \textit{chicago}, located in Amsterdam, Japan, and the USA, respectively.
For each incident, we collected routes originating from the hijacking AS over the time period of the incident and removed announcements with compliant ROAs or unknown prefixes to ROAs.
This resulted in a collection of 10,068 hijacks.
%Question: many events may represent the same classification (multiple prefixes with the same RPKI origin announced by the same non-RPKI origin)
Note that we retained only one instance of a hijack when multiple hijacks involved the same hijacking AS and the same victim AS (i.e., the RPKI origin), thereby reducing redundancy in hijack identification.

The evaluation results on the new set are also presented in Table \ref{tab:lov_classifiers_res}.
Despite achieving a higher accuracy of 96\% in identifying benign conflicts, the DT model exhibits a relatively low performance in detecting hijacks, with an accuracy of around 98\%. In contrast, the SVM, KNN, and RF models achieve $\sim$100\% accuracy in hijack detection, meeting our requirements for the ML-classifier.
While the KNN model slightly outperforms the RF in benign conflict identification, the RF exhibits fewer detection errors, with only one instance being misidentified.

Our evaluation results show that the RF model exhibits better performance overall. Consequently, we select the RF-based classifier to differentiate benign conflicts from BGP hijacks.
%Notably, the non-time-dependent features used in the model help maintain its generalizability across different times.


 \begin{table}[t!]
   %\renewcommand{\arraystretch}{1.0}
  
  \centering
  \fontsize{5}{5}\selectfont %\scriptsize
%  \begin{adjustbox}{width=0.8\linewidth,center}
  \renewcommand\arraystretch{0.9}{
  \begin {tabular}{ccccccc}
  \toprule
    \textbf{Time} & \textbf{Hijacker ASN} & \textbf{Organization} & \textbf{VP (\#)} & \textbf{Duration} & \textbf{Hijacks (\#)} & \textbf{Acc} \\
    \midrule
   2020-04-01 & AS12389 & Rostelecom & 8870 & 1 hour & 7396 & 100\% \\
   2020-09-29 & AS1221 & Telstra & 472 & 3 hours & 58 & 100\% \\
   2021-04-16 & AS55410 & Vodafone Idea & 37,739 & 1.5 hours & 1321 & 100\% \\
   2021-05-18 & AS48467 & PRANET  & 454 & 5.7 hours & 99 & 99\% \\
  2021-10-13 & AS212046 & MEZON & 1029 & 1 hour & 449 & 100\% \\
   2021-10-25 & AS212046 & MEZON & 3786 & 0.5 hours & 24 & 100\%  \\ 
   2022-01-31 & AS18978 & ENZUINC & 743 & 0.7 hours & 274 & 100\%\\
   2022-05-31 & AS38744 & AONB & 461 & 0.4 hours & 447 & 100\% \\
   %2019-05-08 & AS268869 & FIBRA & - & 0.1 hours & 2 & 100\%\\
   \bottomrule
   \end{tabular}
   }
  % \end{adjustbox}
   \caption[Statistics of BGP hijacks collected.]{\small{BGP hijacks we collected, and evaluation with RF-based classifier for each event. VP: victim prefixes, Acc: accuracy.}}
\label{tab:bgp_incidents_res}
\vspace{-15pt}
\end{table}

\begin{figure}[b!]
\centerline{\includegraphics[width=0.6\linewidth]{./images/feature_importance.pdf}}
\vspace{-10pt}
\caption{\small{Feature importance.}}
\label{fig:lov_feature_importance}
\end{figure}

\vspace{-8pt}
\subsubsection{Feature Importance} \label{subsubsec:lov_feature_importance}
We assess the importance of individual features in the RF classifier's decision-making. Figure \ref{fig:lov_feature_importance} shows the feature importance scores for the seven features used. The features $PC$, $MOAS$, $ASdist$, $AltSources$, and $Parent$ have the most significant influence, each with a score greater than 0.1 and collectively contributing over 0.95. The frequent absence of local AS hegemony values possibly reduces the impact of $Depen$, with around 35\% of the routes in our ground truth set lacking this data.
Routes with the same origin satisfy $MOAS$ but not vice versa, reducing the importance of $OriginMatch$ and increasing that of $MOAS$. Conflicting origins with strong AS dependency, potentially involving a provider-customer relationship, may amplify the impact of $PC$.



\vspace{-8pt}
\subsubsection{Errors Analysis}
Next, we analyze the detection errors of the RF classifier.
In some benign conflicts, no relationship between the conflicting origins was detected. This may stem from a lack of relevant information in the data sources, particularly for newly emerging ASes, which led to the use of default values and subsequent misidentification.
Additionally, benign conflicts involving human typo errors in ROA issuance—such as AS4433 being erroneously written as AS443, often manifest as real hijacks, possibly leading to detection failures.

In contrast, the undetected hijack was likely caused by a smaller AS distance value between the two origins. To assess the success rate of such attacks, we crafted 5,000 prefix hijack instances originating from 5,000 randomly selected ASes.
BGP neighbors (such as providers, customers, and peers) can be relatively close geographically. The targets of these hijacks were IP addresses from neighboring networks with a peering relationship with the hijacking AS.
We excluded those neighbor networks with provider or customer relationships to avoid potential benign origin errors. 1.6\% of these hijacks were misclassified as benign by the classifier, indicating that the likelihood of adversaries launching such attacks is low.


\vspace{-5pt}
\subsubsection{Robustness to Adversarial Attacks} \label{subsubsec:robustness}
%If LOV accepts a malicious announcement into its allowlist, what are the mechanism to revert this and prevent future hijacks?
As discussed above, adversaries might design BGP hijacking attacks to evade detection by the ML-classifier, though such attacks have a low success rate.
Recall that benign conflicts identified by the ML-classifier are not immediately added to the whitelist. Instead, they are quarantined for further review, which includes examining the tightness of the conflicting origins associated with a route and monitoring their longevity or activity level. Therefore, even if such hijacks succeed in bypassing the classifier's detection, the rigorous review process makes it less likely that these hijacks will be whitelisted, as they typically exhibit lower tightness and shorter duration.
The quarantine mechanism can effectively prevent adversarial attacks from compromising the whitelist, ensuring its reliability.
Please refer to Section \ref{subsec:quarantine} for more details.

Additionally, adversaries may also manipulate the data used for calculating features to avoid identification. However, the public data resources we have meticulously selected are characterized by high reliability, trustworthiness, and robust maintenance. These qualities make it challenging for attackers to tamper with the data, thereby reducing the likelihood of such attacks being attempted.
Regarding the other two common types of adversarial attacks, namely poisoning attacks and model extraction attacks, they pose less of a threat to the effectiveness of \lov. Attackers would find it impossible to modify the training data of classification models because the data is kept private and secure to prevent unauthorized access. Additionally, the training data used by \lov\ does not contain any sensitive or confidential information that an attacker would be interested in extracting.
\vspace{-8pt}
\subsubsection{Model Generalization} \label{subsubsec:model_general}
Our feature set focuses solely on capturing the relationships between two conflicting origins rather than the behavior patterns underlying the routes.
This can mitigate the impact of the evolving behavior of anomalous routes over time on identification, thereby minimizing the model generalization issues to newly emerging data.
Additionally, the feature values may vary when updated public data sources are used, potentially biasing the classification. However, this variation is not due to the features themselves, but rather to inaccuracies in the data sources.

Appendix \ref{app:ml_challenges} discusses additional challenges faced during the development and deployment of the ML-based classifier.




 
