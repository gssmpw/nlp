\section{Algorithms for the General Case} \label{sec:general-algo}
In this section we describe our algorithms for the most general case where $k'$ can take any value between $1$ and $k$, and the diversity metric $\rho$ is an arbitrary diversity metric. First we start with definitions, additional preliminaries, and our main theorem statements \ref{thm:diverse_ann} and \ref{thm:dual_diverse_ann}.

\subsection{Additional Preliminaries, Problem Formulation, and the Main Theorems}

As before, we let $(X,D)$ be the underlying metric space, where $D$ measures the distance between the points. In this section, we assume that we are given a second metric space $(X,\rho)$ which measures the {\em diversity} between the points. As before $P$ is a subset of $n$ points in $X$. So for two points $p_1,p_2\in P$, $\rho(p_1,p_2)$ measures their pairwise diversity.

Again, we use $B_D(p,r)$ (or just $B(p,r)$ for simplicity of exposition) to denote a ball centered at $p$ with radius $r$, i.e.,  $B_D(p,r)=\{u\in X: D(u,p)<r\}$. Similarly, we define the ball $B_\rho(p,r)=\{u\in X: \rho(u,p)<r\}$. 


The following definitions recap the discussion in the introduction.

\begin{definition}[$C$-diverse]\label{def:minimal_distance}
Let $S$ be a set of points in $X$. We say $S$ is {\em $C$-diverse} if for any two points $p_1,p_2\in S$, we have $\rho(p_1,p_2)\geq C$. 
\end{definition}

Note that the colorful setting, corresponds to the diversity metric $\rho$ being uniform. That is, we can set $\rho(p_i,p_j)=0$ when $col[p_i]=col[p_j]$, and set $\rho(p_i,p_j)=1$ otherwise. Then, we retrieve the colorful notion of diversity: a set $S$ of size $k$ is colorful iff it is $1$-diverse.
We further generalize this notion to allow the set to contain at most $k'>1$ points that are similar to each other. 

\begin{definition}[$(k',C)$-diverse]\label{def:k'-minimal_distance}
Let $S$ be a set of points in $X$. We say $S$ is $(k',C)$-diverse if for any point $p\in S$, we have $|B_\rho(p,C)\cap S|\le k'$. 
Note that being $(1,C)$-diverse is equivalent to the notion of $C$-diverse.
\end{definition}


We consider two dual variants of the diverse nearest neighbor search problem, both of which use two approximation factors: $c>1$ is the ``dissimilarity'' approximation factor with respect to $D$, and $a>1$ is the ``diversity'' approximation factor with respect to $\rho$.

\begin{definition}[Primal Diverse NN]
    Given a point set $P$, diversity value $C$, and the value $k'\leq k$, the goal of {\em Primal Diverse NN} is to preprocess $P$ and create a data structure such that given a query point $q$, one can quickly return the {\em closest} set $S\subset P$ of size $k$ that is $(k',C)$-diverse. Here closeness of a set $S$ is measured by $S_k$ (See Definition \ref{def:S_i}).

    In the approximate variant, for any $q \in X$, if $\mathsf{OPT}$ is a $(k',C)$-diverse set of $k$ points which minimizes $\mathsf{OPT_k}$, then the data structure outputs $\mathsf{ALG}$ that is $(k',C/a)$-diverse such that $\mathsf{ALG_k} \le c\cdot \mathsf{OPT_k}$.
\end{definition}



\begin{definition}[Dual Diverse NN]
    Given a point set $P$, a radius $R$, and the value $k'\leq k$, the goal of {\em Dual Diverse NN} is to preprocess $P$ and create a data structure such that given a query point $q$, one can quickly return a set $S\subset P$ of size $k$ that lie within the radius $R$, while maximizing the diversity. 

    Formally, for any $q \in X$, let $B_P(q,R)$ be the set of points in $P$ within distance $R$ from $q$, and let $\mathsf{OPT}$ be a $(k',C)$-diverse set of $k^*=\min(k,|B_P(q,R)|)$ points from $B_P(q,R)$ that maximizes $C$. Then the data structure outputs $\mathsf{ALG}$ of size $k^*$ that is $(k',C/a)$-diverse such that $\mathsf{ALG_{k^*}}\le c R$.
\end{definition}


As described in the introduction, the problem addressed in the prior work~\cite{abbar2013diverse} is the dual diverse NN problem, where the only consider $k'=1$.

\paragraph{Results.} Our main theoretical result is captured by the following theorems, which specifies the approximation and running time guarantees for our algorithms solving the primal and dual versions of the diverse nearest neighbor problem. 
%Note that similar to the prior work of \citep{abbar2013diverse}, our result for the dual diverse NN, only holds in the case of $k'=1$.

\begin{theorem}[Primal Diverse ANN] \label{thm:diverse_ann}
Let $\mathsf{OPT}=\{p^*_1,...,p^*_k\}$ be a $(k',C)$-diverse solution that minimizes $\mathsf{\mathsf{OPT_k}}$. Given the graph constructed by Algorithm~\ref{alg:indexing}, the search Algorithm~\ref{alg:general_search} finds a $(k',C/12)$-diverse solution $\mathsf{ALG}$ with $\mathsf{ALG_k}\le \left(\frac{\alpha+1}{\alpha-1}+\epsilon\right)\mathsf{OPT_k}$ in $O\left(k\log_{\alpha}\frac{\Delta}{\epsilon}\right)$ steps, where each step takes $O\left((k^3/k')(8\alpha)^d\log \Delta\right)$ time.
The data structure uses space $O(n(k/k')(8\alpha)^d\log\Delta)$.
\end{theorem}

\smallskip
\begin{theorem}[Dual Diverse ANN] \label{thm:dual_diverse_ann}
Given the graph constructed by Algorithm~\ref{alg:indexing}, the search Algorithm~\ref{alg:dual_search} finds a $(k',C/24)$-diverse NN solution $\mathsf{ALG}$ satisfying $ALG_k\le \left(\frac{\alpha+1}{\alpha-1}+\epsilon\right)\cdot R$ in $\tilde{O}\left((k^4/k')(8\alpha)^d\log^2\frac{\Delta}{\epsilon}\right)$ time, if there exists a $(k',C)$-diverse solution $\mathsf{OPT}$ with $\mathsf{OPT_k}\le R$.
\end{theorem}


\subsection{Algorithm}
\noindent\textbf{The preprocessing algorithm.} The indexing algorithm, which is the same for both the primal and dual versions of the problem, is shown in Algorithm~\ref{alg:indexing}. Line 12 of the algorithm uses the greedy algorithm of \cite{gonzalez1985clustering}, defined below.


\begin{algorithm}
\caption{Indexing algorithm for diverse NN}
\label{alg:indexing}
\begin{algorithmic}[1]
\STATE \textbf{Input}: A set of $n$ points $P=\{p_1,...,p_n\}$; $k$ is the size of the output; $k'$ is the parameter in the diversity definition; $\alpha$ is the parameter used for pruning.
\STATE \textbf{Output}: A directed graph $G=(V,E)$ where $V=\{1,...,n\}$ are associated with the point set $P$.
\FOR{each point $p\in P$}
    \STATE Sort all points $u\in P$ based on their distance from $p$ and put them in a list $\mathcal{L}$ in that order
    \WHILE{$\mathcal{L}$ is not empty}
        \STATE $u\gets \argmin\limits_{u\in \mathcal{L}}D(u,p)$
        \STATE Initialize $\mathsf{bag}[u]\gets \{u\}$
        \FOR{each point $v\in \mathcal{L}$ in order}
            \IF{$D(u,v)\le D(p,u)/(2\alpha)$}
                \STATE $\mathsf{bag}[u]\gets \mathsf{bag}[u]\cup v$
                \STATE remove $v$ from $\mathcal{L}$
            \ENDIF
        \ENDFOR
        \STATE $\mathsf{rep}[u]\gets$ use the greedy algorithm of Gonzales to choose $k/k'$ points in $\mathsf{bag}[u]$ to approximately maximize the minimum pairwise diversity.
        \STATE add edges from $p$ to $\mathsf{rep}[u]$
        \STATE Remove $u$ from $\mathcal{L}$
    \ENDWHILE
\ENDFOR
\end{algorithmic}
\end{algorithm}


\noindent\textbf{Gonzales' greedy algorithm.} Given a set of $n$ points and a parameter $m$, the algorithm picks $m$ points as follows. The first point is chosen arbitrarily. Then, in each of the next $m-1$ steps, the algorithm picks the point whose minimum distance w.r.t. $\rho$ to the currently chosen points is maximized. It is known \cite{gonzalez1985clustering} that this algorithm provides a $2$-approximation for the problem of picking a subset of size $m$ which maximizes the minimum pairwise diversity distance between the picked points. Moreover, the picked set has an anti-cover property which we will discuss in Proposition \ref{prop:greedy}.

\noindent\textbf{Primal Search Algorithm.} Algorithm~\ref{alg:general_search} shows the search algorithm for the primal version of diverse nearest neighbor. 
%\xnote{are we removing the special analysis for $k'=1$}
%The algorithm has a different condition for $k'=1$ as in this case we can slightly improve the performance of the algorithm as shown in Section \ref{sec:improved-analysis}. The general case of 
The algorithm is analyzed in Section \ref{sec:primal-analysis}. The initialization step of line 3, can be done using the following algorithm.

%\xnote{perhaps remove line 6,7, and 8 if we don't have the $k'=1$ analysis}

\begin{algorithm}
\caption{Search algorithm for primal diverse NN}
\label{alg:general_search}

\begin{algorithmic}[1]
\STATE \textbf{Input}: A graph $G=(V,E)$ with $N_{out}(p)$ denoting the out edges of $p$; query $q$, number of optimization steps $T$; diversity lower bound $C$.
\STATE \textbf{Output}: A set of $k$ points $\mathsf{ALG}$.
\STATE Initialize $\mathsf{ALG}=\{p_1,...,p_k\}$ to be a set of $k$ points that are $(k',C/12)$-diverse using the initialization step proved in Lemma \ref{lm:initialization}.
\FOR{$i=1$ to $T$}
    % \State $U\gets \{u | (p_j,u)\in E\ s.t. \exists\ p_j\in ALG\}$
    \STATE $U\gets \bigcup\limits_{p\in \mathsf{ALG}}(N_{out}(p)\cup p)$, and sort $U$ based on their distance from $q$
%    \IF{$k'=1$}
%        \STATE $\mathsf{ALG}=\varnothing$
%    \ELSE
        % \State $ALG^t=ALG^{t-1}\setminus p_k$
        \STATE $\mathsf{ALG}\gets$ the closest $k-1$ points in $\mathsf{ALG}$
%    \ENDIF
    
    \FOR{each point $u \in U$ in order}
        \IF{$\mathsf{ALG}\bigcup u$ is $(k',C/12)$-diverse}
            \STATE $\mathsf{ALG}\gets \mathsf{ALG}\cup u$
        \ENDIF
        \IF{$|\mathsf{ALG}|=k$}
            \STATE Break
        \ENDIF
    \ENDFOR
    % \If{$\mathsf{ALG}^{t}\ge \mathsf{ALG}^{t-1}$}
    % \State Break
    % \EndIf
\ENDFOR
\STATE \textbf{Return} $\mathsf{ALG}$
\end{algorithmic}
\end{algorithm}

\noindent\textbf{The initialization step.} Given a set $P$ of $n$ points equipped with metric distance $\rho$, and parameters $k'$ and $k$, and lower bound diversity $C$, the goal is to pick a subset $S\subseteq P$ of size $k$ which is $(k',C/4)$ diverse or otherwise output that no $(k',C)$-diverse subset $S$ exists. We use the following algorithm
\begin{itemize}
    \item Initialize $\mathsf{SOL}=\emptyset$
    \item While there exists a point $p\in P$ such that the ball $B=B_\rho(p,C/4)$ has $k'$ points in it, (i.e., $|B\cap P|>k'$), 
    \begin{itemize}
        \item Add an arbitrary subset of $B\cap P$ of size $k'$ to $\mathsf{SOL}$. 
        \item Remove all points in $2B=B_\rho(p,C/2)$ from $P$.
    \end{itemize}
    \item Add all remaining points in $P$ to $\mathsf{SOL}$.
    \item If $|\mathsf{SOL}|\geq k$, return an arbitrary subset of it of size $k$, otherwise return `no solution'.
\end{itemize}

\begin{lemma}[Initialization]\label{lm:initialization}
If $P$ has a subset $\mathsf{OPT}$ of size $k$ that is $(k',C)$-diverse, our initialization algorithm finds a $(k',C/4)$-diverse subset of size $k$.
\end{lemma}
\begin{proof}
    Note that it is straightforward to see why the set $\mathsf{SOL}$ that we get at the end is $(k',C/4)$-diverse. This is because first of all, each time we pick $k'$ points in a ball $B$ and add them to $\mathsf{SOL}$, we make sure that no additional point will ever be picked in $2B$ and thus within distance $C/4$ of the points we pick there will be at most $k'$ points in the end. Second, at the end, every remaining ball of radius $C/4$ has less than or equal to $k'$ points in it. Therefore, we can pick all such points in the solution and everything we picked will be $(k',C/4)$ diverse.


    Next we argue that we are in fact able to pick at least $k$ points in total which completes the argument. We do it by following the procedure of our algorithm and comparing it with $\mathsf{OPT}$. 
    At each iteration of the while loop that we remove $P\cap 2B$, we add exactly $k'$ points from $P\cap 2B$ to our solution $\mathsf{SOL}$. 
    Now note that the optimal solution $\mathsf{OPT}$ cannot have more than $k'$ points in $2B$ because by triangle inequality any pair of points in $2B$ have distance at most $C$, and picking more than $k'$ points in this ball contradicts the fact that $\mathsf{OPT}$ is $(k',C)$ diverse. Thus we can have an one-to-one mapping from each point in $\mathsf{OPT}\cap 2B$ to the $k'$ points in $P\cap 2B$ added to $\mathsf{SOL}$.
    At the end of the while iteration, we know any unmapped point in $\mathsf{OPT}$ still exists in $P$, so we just map it to itself. By doing this, we can have an one-to-one mapping from $\mathsf{OPT}$ to $\mathsf{SOL}$, which means that $|\mathsf{SOL}|\ge|\mathsf{OPT}|=k$. 
    % Thus the removal of $P\cap 2B$ can only remove at most $k'$ points from $OPT$.
    % Therefore at the end of the while iteration, the size of the remaining points in $P$, which should be at least the number of points remaining in $OPT$, is more than the number of points that we still need to pick. Therefore, our algorithm will always be able to pick $k$ points assuming such an $OPT$ exists.
\end{proof}


\noindent\textbf{Dual Search Algorithm.} Algorithm~\ref{alg:dual_search} shows the search algorithm for the dual version of the diverse nearest neighbor problem. We provide the analysis in Section \ref{sec:dual-analysis}.


% \begin{algorithm}
% \caption{Indexing algorithm for diverse NN}
% \label{alg:colorful_indexing}
% \begin{algorithmic}[1]
% \State \textbf{Input}: A set of $n$ points $P=\{p_1,...,p_n\}$. $k$ is the size of the output. $k'$ is the maximal number of points allowed per color.
% \State \textbf{Output}: A directed graph $G=(V,E)$ where $V=\{1,...,n\}$ are associated with the point set $P$.
% \For{each point $p\in P$}
%     \State Sort all points $u\in P$ based on their distance from $p$ and put them in a list $L$ in that order
%     \State For each $u\in P$, initialize set $\mathsf{rep}[u]=\{\}$
%     \While{L is not empty}
%         \State $u\gets \argmin\limits_{u\in L}D(u,p)$
%         \State Add edge $(p,u)$ to edge set $E$
%         \State Remove $u$ from $L$
%         \For{each point $w\in L$ in order}
%             \If{$D(u,w)\le D(p,w)/\alpha$}
%                 % \If{for $(k,k')$-colorful constraint}
%                 %     \If{$col[u]=col[w]$}
%                 %         \State Remove $w$ from $L$ (Condition 1)
%                 %     \ElsIf{$col[u]$ not present in $\mathsf{rep}[w]$}
%                 %         \State Add $col[u]$ to $\mathsf{rep}[w]$
%                 %         \If{$|\mathsf{rep}[w]|\ge k/k'$}
%                 %             \State Remove $w$ from $L$ (Condition 2)
%                 %         \EndIf
%                 %     \EndIf
%                 % \If{for $(k,0.05C)$-diverse constraint}
%                 \If{$\rho(u,w)\le 0.4C$}
%                     \State Remove $w$ from $L$ (Condition 1)
%                 \ElsIf{for all $u'\in \mathsf{rep}[w]$, $\rho(u,u')\ge 0.2C$}
%                     \State Add $u$ to $\mathsf{rep}[w]$
%                     \If{$|\mathsf{rep}[w]|\ge k/k'$}
%                         \State Remove $w$ from $L$ (Condition 2)
%                     \EndIf
%                 \EndIf
%                 % \EndIf
%             \EndIf
%         \EndFor
%     \EndWhile
% \EndFor
% \end{algorithmic}
% \end{algorithm}

%TEST 


\iffalse

\begin{algorithm}
\caption{New Search algorithm for 1-Diverse NN}
\label{alg:general_search}
\begin{algorithmic}[1]
\State \textbf{Input}: A graph $G=(V,E)$ with $N_{out}(p)$ be the out edges of $p$, query $q$, optimization step $T$
\State \textbf{Output}: A set of $k$ points $\mathsf{ALG}$.
\State Initialize $\mathsf{ALG}^0=\{p_1,...,p_k\}$ to be any $k$ points satisfying $(1,0.05C)$-diverse constraint
\For{$i=1$ to $T$}
    % \State $U\gets \{u | (p_j,u)\in E\ s.t. \exists\ p_j\in \mathsf{ALG}\}$
    \State $U=\bigcup\limits_{p\in \mathsf{ALG^{t-1}}}N_{out}(p)$ and sort $U$ based on their distance from $q$
    \State $\mathsf{ALG^t}\gets \varnothing$
    \For{each point $u \in U$ in order}
        \If{$\mathsf{ALG^t}\cup u$ is $(1,0.05C)$-diverse}
            \State $\mathsf{ALG^t}\gets \mathsf{ALG^t}\cup u$
        \EndIf
    \EndFor
    \If{$\mathsf{ALG^t}=\mathsf{ALG^{t-1}}$}
    \State Return $\mathsf{ALG^t}$
    \EndIf
\EndFor
\State \textbf{Return} $\mathsf{ALG^T}$
\end{algorithmic}
\end{algorithm}

\fi

\iffalse 

\subsection{Analysis for Colorful Diverse-ANN}

\begin{theorem}\label{thm:colorful_diverse_ann}
Given the graph constructed by Algorithm~\ref{alg:indexing}, the search Algorithm~\ref{alg:general_search} finds a $\left(\frac{\alpha+1}{\alpha-1}+\epsilon\right)$-approximate $(k,k')$-colorful diverse ANN solution in $O(k\log_{\alpha}\frac{\Delta}{\epsilon})$ steps where each step takes $O((k^2/k')(4\alpha)^d\log\Delta)$ time.
\end{theorem}

\xnote{I think in general it is impossible to derive convergence bound for all $D_i\le c\cdot D^*_i$ bound. Suppose in the worst case we want to optimize $k$ different nearest neighbors independently. If we don't know $\Delta$, we will keep optimizing the current farthest point, but the second to the farthest point may have arbitrary worse approximation ratio. }

\begin{lemma}\label{lm:degree_colorful}
The graph constructed by Algorithm~\ref{alg:colorful_indexing} has degree $O((k/k')(4\alpha)^d\log\Delta)$
\end{lemma}

\begin{proof}
We use $Ring(p,r_1,r_2)$ to denote the points whose distance from $p$ is larger than $r_1$ but smaller than $r_2$. For each $i\in [\log_2 \Delta]$, we consider the $Ring(p,D_{max}/2^i,D_{max}/2^{i-1})$ separately. According to Lemma~\ref{lm:doubling_dimension}, we can cover $Ring(p,D_{max}/2^i,D_{max}/2^{i-1})\cap P$ using at most $m\le O((4\alpha)^d)$ small balls with radius $\frac{D_{max}}{2^{i+1}\alpha}$. Now, we argue that within any such small ball $B$, the number of points connected to $p$ is at most $k/k'$. First, we note that for any two points $u,v$ within $B$, we have $D(u,v)\le D(p,v)/\alpha$. Therefore, any two points $u,v\in N_{out}(p)\cap B$ must have different colors. Otherwise the later one should have been removed. Next, if there are more than $k/k'$ points with different colors in $N_{out}(p)\cap B$, the farthest point $w$ will have its $|\mathsf{rep}[w]|>k/k'$ and get removed. By counting the number of small balls and the number of rings, we get the degree bound for any point $p$ is upper bounded by $O((k/k')(4\alpha)^d\log\Delta)$
\end{proof}

\begin{lemma}\label{lm:p_star_exists_colorful}
Suppose $OPT=\{p^*_1,...,p^*_k\}$ is the optimal solution with minimized $D^*_k$ and $\mathsf{ALG}=\{p_1,...,p_k\}$ be the current solution. If $p_k\notin OPT$, there exists a point $p^* \in OPT$ such that $\mathsf{ALG}\setminus p_k \cup p^*$ is $(k,k')$-Colorful
\end{lemma}

\begin{proof}
Suppose the color of $p_k$ in $\mathsf{ALG}$ is $c$. We use $\mathsf{OPT_c}$ and $\mathsf{ALG_c}$ to denote the subset of points with color $c$. First, if $\mathsf{OPT_c}\setminus \mathsf{ALG_c}\neq \varnothing$, we let $p^*$ to be any element from $OPT_c\setminus \mathsf{ALG_c}$ and we know $\mathsf{ALG}\setminus p_k \cup p^*$ is $(k,k')$-Colorful. Otherwise, we have $OPT_c\subset \mathsf{ALG_c}$. By a simple counting argument, we know there must exist another color $c'$ such that $OPT_
{c'}\setminus ALG_{c'}\neq\varnothing$. Picking $p^*$ to be any element from that set satisfies the criteria as $c'$ is not a saturated color.
\end{proof}

\begin{lemma}\label{lm:update_colorful}
% If $\mathsf{ALG}$ is not updated in one round, we have $D_k\le \frac{\alpha+1}{\alpha-1}D^*_k$.
There always exists a point $p'$ connected from some point $w\in \mathsf{ALG}$ such that
\begin{enumerate}
\item $\mathsf{ALG}\setminus p_k\cup p'$ is $(k,k')$ -colorful
\item $D(p',q)\le D_k/\alpha+D^*_k(1+1/\alpha)$
\end{enumerate}
\end{lemma}

\begin{proof}
According to Lemma~\ref{lm:p_star_exists_colorful}, for any current solution $\mathsf{ALG}$ with $p_k\notin OPT$, there exists a point $p^* \in OPT$ such that $\mathsf{ALG}\setminus p_k\cup p^*$ is $(k,k')$-colorful. Let $w\in \mathsf{ALG}$ be the closest point to $p^*$. If there exists an edge from $w$ to $p^*$, replacing $p_k$ with $p^*$ is a potential update. We set $p'=p^*$ and $D(p',q)\le D^*_k$ satisfies the distance upper bound above.
%Given $D(p^*,q)\le D^*_k\le D_k$, if no improvement can be made, we have $D(p^*,q)=D^*_k=D_k$, which means that the current $\mathsf{ALG}$ is already optimal. 


In the following, we consider two cases depending on the how the edge  $(w,p^*)$ is removed.

\begin{enumerate}
    \item $(w,p^*)$ was removed because of Condition 1: In this case, there exists another point $p'$ with $D(p',p^*)\le D(w,p^*)/\alpha$ and $col[p']=col[p^*]$. Therefore, $\mathsf{ALG}\setminus p_k\cup p'$ is $(k,k')-$colorful.
    \item $(w,p^*)$ was removed because of Condition 2: In this case, $\mathsf{rep}[p^*]$ has $k/k'$ distinct colors different from $col[p^*]$. By a counting argument, $\mathsf{ALG}\setminus p_k$ can have at most $k/k'-1$ saturated colors, so there must exist a point $p'\in \mathsf{rep}[p^*]$ so that $\mathsf{ALG}\setminus p_k\cup p'$ is $(k,k')-$ colorful.
\end{enumerate}

In both cases, we can find a point $p'$ satisfying $D(p',p^*)\le D(w,p^*)/\alpha$ and $\mathsf{ALG}\setminus p_k\cup p'$ is $(k,k')-$colorful. The distance from $p'$ to $q$ is upper bounded by the following:
\begin{align}
D(p',q)&\le D(p^*,q)+D(p',p^*) \\
&\le D(p^*,q)+D(w,p^*)/\alpha \\
&\le D(p^*,q)+D(w,q)/\alpha+D(p^*,q)/\alpha \\
&\le D_k/\alpha+D^*_k(1+1/\alpha)
\end{align}

% If $p'$ can't improve upon $p_k$, it means that $D_k\le D_k/\alpha+D^*_k(1+1/\alpha)$, which is $D_k\le\frac{\alpha+1}{\alpha-1}D^*_k$.
\end{proof}

\begin{proof}[Proof of Theorem~\ref{thm:colorful_diverse_ann}]
Regarding the running time, the time spent on each update is bounded by the total number of edges connected from any point in $\mathsf{ALG}$, which is $O((k^2/k')(4\alpha)^d\log \Delta)$.

To analyze the approximation ratio, at time step $t$, we use $\mathsf{ALG^t}=\{p^t_1,...,p^t_k\}$ to denote the current unordered solution. We define $D^t_k=\max_{i\in[k]}D(p^t_i,q)$. According to Algorithm~\ref{alg:general_search} and Lemma~\ref{lm:update_colorful}, if $p_i$ is updated at time step $t$, we have $D(p^{t+1}_i,q)\le D(p^t_i,q)/\alpha+D^*_k(1+1/\alpha)$. By an induction argument, if an point $p_i$ is updated by $t$ times at the end of time step $T$, we have $D(p^T_i,q)\le \frac{D_{max}}{\alpha^t}+\frac{\alpha+1}{\alpha-1}D^*_k$. For the sake of contradiction, suppose after $T$ steps, $D^T_k>\frac{D_{max}}{\alpha^{T/k}}+\frac{\alpha+1}{\alpha-1}D^*_k$, which means that the index $i$ achieves the maximal $D(p^T_i,q)$ was updated for fewer than $T/k$ times. By a counting argument, some other index $j$ was updated for at least $T/k+1$ times. However, by the time index $j$ was updated for $T/k$ times, its distance to $q$ had already been upper bounded by $\frac{D_{max}}{\alpha^{T/k}}+\frac{\alpha+1}{\alpha-1}D^*_k$, so the algorithm should have chosen to update index $i$ instead of $j$ as $D(p_i,q)$ achieved the maximal distance within set $\mathsf{ALG}$ at that time, leading to a contradiction. Therefore, we have that the final $p_i$ achieves the maximal distance $D^T_k$ must have been updated for at least $T/k$ times, and $D^T_k$ is upper bounded by $\frac{D_{max}}{\alpha^{T/k}}+\frac{\alpha+1}{\alpha-1}D^*_k$. A simple transformation will yield the desired approximation ratio bound.
\end{proof}

\fi

\subsection{Analysis of the Primal Diverse NN Algorithm}\label{sec:primal-analysis}
In this section, we prove Theorem~\ref{thm:diverse_ann} that gives the approximation and running time guarantees for Algorithm~\ref{alg:indexing} and Algorithm~\ref{alg:general_search}.




\iffalse

\begin{corollary}\label{cor:colorful_ann}
Let $OPT=\{p^*_1,...,p^*_k\}$ be the $k'$-colorful solution with minimized $\mathsf{OPT_k}$. Given the graph constructed by Algorithm~\ref{alg:indexing}, the search Algorithm~\ref{alg:general_search} finds a $k'$-colorful solution $\mathsf{ALG}$ with $\mathsf{ALG_k}\le \left(\frac{\alpha+1}{\alpha-1}+\epsilon\right)\mathsf{OPT_k}$ in $O\left(k\log_{\alpha}\frac{\Delta}{\epsilon}\right)$ steps where each step takes $O\left((k^2/k')(8\alpha)^d\log \Delta\right)$ time.
\end{corollary}

\begin{proof} 
By setting $\rho(p_i,p_j)=1$ for $col[p_i]\neq col[p_j]$ and $0$ otherwise, a set $P$ is $k'$-colorful if and only if it is $(k',0.05C)$-diverse. Then we can apply Theorem~\ref{thm:diverse_ann} to prove this corollary.
\end{proof}
\fi

\begin{lemma}\label{lm:degree_diverse}
The graph constructed by Algorithm~\ref{alg:indexing} has degree limit $O((k/k')(8\alpha)^d\log\Delta)$.
\end{lemma}

\begin{proof}
Let's first bound the number of points not removed by others, then according to Line 14-15 in Algorithm~\ref{alg:indexing}, the degree bound will be that times $k/k'$.

We use $\mathsf{Ring}(p,r_1,r_2)$ to denote the points whose distance from $p$ is larger than $r_1$ but smaller than $r_2$. For each $i\in [\log_2 \Delta]$, we consider the $\mathsf{Ring}(p,D_{max}/2^i,D_{max}/2^{i-1})$ separately. According to Lemma~\ref{lm:doubling_dimension}, we can cover $\mathsf{Ring}(p,D_{max}/2^i,D_{max}/2^{i-1})\cap P$ using at most $m\le O((8\alpha)^d)$ small balls with radius $\frac{D_{max}}{2^{i+2}\alpha}$. According to the pruning criteria in Line 9, within each small ball, there will be at most one point remaining. This establishes the degree bound of $O((k/k')(8\alpha)^d\log\Delta)$.
% We use $Ring(p,r_1,r_2)$ to denote the points whose distance from $p$ is larger than $r_1$ but smaller than $r_2$. For each $i\in [\log_2 \Delta]$, we consider the $Ring(p,D_{max}/2^i,D_{max}/2^{i-1})$ separately. According to Lemma~\ref{lm:doubling_dimension}, we can cover $Ring(p,D_{max}/2^i,D_{max}/2^{i-1})\cap P$ using at most $m\le O((4\alpha)^d)$ small balls with radius $\frac{D_{max}}{2^{i+1}\alpha}$. Now, we argue that within any such small ball $B$, the number of points connected to $p$ is at most $k/k'$. First, we note that for any two points $u,v$ within $B$, we have $D(u,v)\le D(p,v)/\alpha$. Therefore, any two points $u,v\in N_{out}(p)\cap B$ must satisfy $\rho(u,v)>0.4C$. Otherwise the later one should have been removed. Now, for the sake of contradiction, suppose there are $m>k/k'$ points $z_1,...,z_m$ within $N_{out}(p)\cap B$. We will bound the size of $\mathsf{rep}[z_m]$ for the last point to join $N_{out}(p)$. For any $z_i$ with $i<m$, either $z_i\in \mathsf{rep}[z_m]$ or there exists another $z'_i\in \mathsf{rep}[z_m]$ with $\rho(z_i,z'_i)<0.2C$. For any different $i,j$ which don't belong to $\mathsf{rep}[z_m]$, because $\rho(z_i,z_j)>0.4C$, by triangle inequality, their $z'_i$ and $z'_j$ must be different. Therefore, the size of $\mathsf{rep}[z_m]$ is lower bounded by $m-1\ge k/k'$, and $z_m$ would have been removed. Given that $|N_{out}(p)\cap B|\le k/k'$ for any ball $B$, a similar counting argument gives the degree upper bounded of $O((k/k')(4\alpha)^d\log\Delta)$
\end{proof}

\begin{lemma}\label{lm:p_star_exists_diverse}
Suppose $\mathsf{OPT}=\{p^*_1,...,p^*_k\}$ is a $(k',C)$-diverse solution with minimized $\mathsf{OPT_k}$, and let $\mathsf{ALG}=\{p_1,...,p_k\}$ be the current solution (ordered by distance from $q$). If $p_k\notin \mathsf{OPT}$, there exists a point $p^*\in \mathsf{OPT}\setminus \mathsf{ALG}$ such that $|B_{\rho}(p^*,C/2)\cap (\mathsf{ALG}\setminus p_k)|<k'$ and $\mathsf{ALG}\setminus p_k\bigcup p^*$ is $(k',C/4)$-diverse.
\end{lemma}

\begin{proof}
% We use $B_{\rho}(p,r)$ to denote the ball in the $(X,\rho)$ metric space. Consider the set of balls $B_{\rho}(p^*_1,C/2),...,B_{\rho}(p^*_k,C/2)$. They are disjoint because $\rho(p^*_i,p^*_j)\ge C$ for $i\neq j$. Then there must exists a $p^* \in \mathsf{OPT}$ such that $B_{\rho}(p^*,C/2)\bigcap (\mathsf{ALG}\setminus p_k)=\varnothing$ by a counting argument. We can also check that for this $p^*$, $\mathsf{ALG}\setminus p_k\bigcup p^*$ is $(k,C/16)$ diverse.

Recall that we use $B_{\rho}(p,r)$ to denote the ball in the $(X,\rho)$ metric space. Because $p_k\notin \mathsf{OPT}$, we have $\overline{\mathsf{OPT}}=\mathsf{OPT}\setminus \mathsf{ALG}\neq \varnothing$. We repeatedly perform the following operation until $\overline{\mathsf{OPT}}$ gets empty: select a point $p$ from $\overline{\mathsf{OPT}}$, and let $z=B_{\rho}(p,C)\cap \overline{\mathsf{OPT}}$, and remove $z$ from $\overline{\mathsf{OPT}}$. By doing this, we can get a list of points $\{p^*_1,...,p^*_m\}$ and a partition of $\mathsf{OPT}\setminus \mathsf{ALG}=z_1\cup z_2...\cup z_m$. By definition, we have the following properties:
%$$\{p^*_1,...,p^*_m\}\cap \mathsf{ALG}=\varnothing,~~ z_i\cap z_j=\varnothing \text{ for } i\neq j,~~\sum_i |z_i|=|\mathsf{OPT}\setminus \mathsf{ALG}|=|\mathsf{ALG}\setminus \mathsf{OPT}|~. $$

\begin{itemize}
    \item $\{p^*_1,...,p^*_m\}\cap \mathsf{ALG}=\varnothing$
    \item $z_i\cap z_j=\varnothing$ for $i\neq j$
    \item $\sum_i |z_i|=|\mathsf{OPT}\setminus \mathsf{ALG}|=|\mathsf{ALG}\setminus \mathsf{OPT}|$
\end{itemize}

Now let $w_i=B_{\rho}(p^*_i,C/2)\cap (\mathsf{ALG}\setminus p_k\setminus \mathsf{OPT})$. Because all the $B_{\rho}(p^*_i,C/2)$ balls are disjoint, $\sum_i |w_i|\le |\mathsf{ALG}\setminus p_k\setminus \mathsf{OPT}|<|\mathsf{OPT}\setminus \mathsf{ALG}|=\sum_i |z_i|$, there must exist an $i$ such that $|w_i|<|z_i|$. For that $i$, we have that $|B_{\rho}(p^*_i,C/2)\cap (\mathsf{ALG}\setminus p_k)|$ is equal to
\begin{align}
= &|B_{\rho}(p^*_i,C/2)\cap (\mathsf{ALG}\cap \mathsf{OPT})|+|B_{\rho}(p^*_i,C/2)\cap (\mathsf{ALG}\setminus p_k \setminus \mathsf{OPT})|\tag{Because $p_k\notin \mathsf{OPT}$}\\
=&|B_{\rho}(p^*_i,C/2)\cap (\mathsf{ALG}\cap \mathsf{OPT})|+|w_i|\notag \\
<&|B_{\rho}(p^*_i,C/2)\cap (\mathsf{ALG}\cap \mathsf{OPT})|+|z_i|\notag\\
\le&|B_{\rho}(p^*_i,C/2)\cap (\mathsf{ALG}\cap \mathsf{OPT})|+|B_{\rho}(p^*_i,C)\cap (\mathsf{OPT}\setminus\mathsf{ALG})|\notag \\
\le&|B_{\rho}(p^*_i,C)\cap (\mathsf{ALG}\cap \mathsf{OPT})|+|B_{\rho}(p^*_i,C)\cap (\mathsf{OPT}\setminus\mathsf{ALG})|\notag \\
=&|B_{\rho}(p^*_i,C)\cap \mathsf{OPT}|\notag \\
\le&  k'\notag
\end{align}

Therefore, we get $B_{\rho}(p^*_i,C/2)\cap(\mathsf{ALG}\setminus p_k)<k'$. Now, for any point $p\in B_{\rho}(p^*_i,C/4)$, $|B_{\rho}(p,C/4)\cap (\mathsf{ALG}\setminus p_k)|\le |B_{\rho}(p^*_i,C/2)\cap (\mathsf{ALG}\setminus p_k)|<k'$, so we know that $\mathsf{ALG}\setminus p_k\cup p^*_i$ is $(k',C/4)$-diverse.
% Because for any $p^*_i\in \mathsf{OPT}$, $|B_{\rho}(p^*_i,C)\cap \mathsf{OPT} |\le k'$. We can extract at least $k/k'$ points $\{z^*_1,...,z^*_{k/k'}\}\subseteq P^*$ so that $B_{\rho}(z^*_i,C/2)$ are disjoint for different $z^*_i$. By a counting argument, there must exist a $p^*\in \{z^*_1,...,z^*_{k/k'}\} \subseteq \mathsf{OPT}$ such that $|B_{\rho}(p^*,C/2)\bigcap (\mathsf{ALG}\setminus p_k)|<k'$. To show that $ALG'=\mathsf{ALG}\setminus p_k\bigcup p^*$ is $(k',0.05C)$-diverse, for any $p\in ALG'$ with $\rho(p,p^*)\le 0.05C$, we have $B_{\rho}(p,0.05C)\bigcap ALG'\subseteq B_{\rho}(p^*,C/2)\bigcap ALG'\le k'$. Therefore $ALG'=\mathsf{ALG}\setminus p_k\bigcup p^*$ is $(k',0.05C)$-diverse.

\end{proof}
The following is the well-known anti-cover property of the greedy algorithm of Gonzales whose proof we include for the sake of completeness.
\begin{proposition}\label{prop:greedy}
In Line 14 of Algorithm~\ref{alg:indexing}, let $\mathsf{rep}[u]$ be the output of greedily choosing $k/k'$ points in $\mathsf{bag}[u]$ maximizing minimum pairwise diversity. If a point $p\in \mathsf{bag}[u]\setminus \mathsf{rep}[u]$, we have $\min\limits_{v\in \mathsf{rep}[u]}\rho(p,v)\le \min\limits_{v_1,v_2\in \mathsf{rep}[u]}\rho(v_1,v_2)$.
\end{proposition}
\begin{proof}
For the sake of contradiction, suppose $\min\limits_{v\in \mathsf{rep}[u]}\rho(p,v)>\min\limits_{v_1,v_2\in \mathsf{rep}[u]}\rho(v_1,v_2)$, and the pairwise diversity minimizer is achieved by $\min\limits_{v_1,v_2\in \mathsf{rep}[u]}\rho(v_1,v_2)=\rho(x,y)$. Without loss of generality, we assume $x$ is added to $\mathsf{rep}[u]$ before $y$. At the time step $t$ when $y$ was added to $\mathsf{rep_t}[u]$, $\min\limits_{v\in \mathsf{rep_t}[u]}\rho(y,v)=\rho(x,y)$ and $\min\limits_{v\in \mathsf{rep_t}[u]}\rho(p,v)\ge \min\limits_{v\in \mathsf{rep}[u]}\rho(p,v)> \rho(x,y)$, so $y$ wouldn't have been chosen by the greedy algorithm. Therefore, we have derived a contradiction.
\end{proof}

\begin{lemma}\label{lm:update_diverse}
There always exists a point $p'$ connected from some point $w\in \mathsf{ALG}$ such that
\begin{enumerate}
\item $\mathsf{ALG}\setminus p_k \bigcup p'$ is $(k',C/12)$-diverse 
\item $D(p',q)\le D(p_k,q)/\alpha+\mathsf{OPT_k}(1+1/\alpha)$
\end{enumerate}
\end{lemma}

\begin{proof}
According to Lemma~\ref{lm:p_star_exists_diverse}, for any current solution $\mathsf{ALG}$ with $p_k\notin \mathsf{OPT}$, there exists a point $p^*\in \mathsf{OPT}\setminus \mathsf{ALG}$ such that $\mathsf{ALG}\setminus p_k\cup p^*$ is $(k',C/4)$-diverse. Let $w\in \mathsf{ALG}$ be the closest point to $p^*$. If there exists an edge from $w$ to $p^*$, replacing $p_k$ with $p^*$ is a potential update. We set $p'=p^*$ and $D(p',q)\le \mathsf{OPT_k}$ satisfies the distance upper bound above.

% In the following, we consider two cases depending on how the edge $(w,p^*)$ is removed.

% \begin{enumerate}
% \item $(w,p^*)$ was removed because of Condition 1: In this case, there exists another point $p'$ with $D(p^*,p')\le D(w,p^*)/\alpha$ and $\rho(p^*,p')\le 0.4C$. Because $|B_{\rho}(p^*,C/2)\bigcap (\mathsf{ALG}\setminus p_k)|<k'$, we have $|B_{\rho}(p',0.1C)\bigcap (\mathsf{ALG}\setminus p_k)|\subseteq |B_{\rho}(p^*,C/2)\bigcap (\mathsf{ALG}\setminus p_k)|<k'$, so the addition of such $p'$ satisfies that $\mathsf{ALG}\setminus p_k\cup p'$ is $(k',0.05C)$-diverse. \snote{again this is using a triangle inequality argument, right?} \xnote{yes. if the new added point satisfies $B_{\rho}(p,2C)\cap P<k'$, then the addition of $p$ is still $(k',C)$-diverse}
% \item $(w,p^*)$ was removed because of Condition 2: In this case, there exists $z_1,...,z_{k/k'}\in B(p^*,D(w,p^*)/\alpha)$ all with diversity distance at least $0.2C$ from each other. Therefore, for any $p_i\in \mathsf{ALG}\setminus p_k$, there can't exist two $z_j$ and $z_{j'}$ s.t. $\rho(p_i,z_j)<0.1C$ and $\rho(p_i,z_{j'})<0.1C$. By a counting argument, we can find at least one $z_i$ s.t. $|B_{\rho}(z_i,0.1C)\cap (\mathsf{ALG}\setminus p_k)|<k'$. Because $w$ is the closest point from $p^*$, we know that $\{z_1,...,z_{k/k'}\}\cap \mathsf{ALG}=\varnothing$, so we can let $p'=z_i$ and then $\mathsf{ALG}\setminus p_k \cup p'$ is $(k',0.05C)$-diverse.
% \end{enumerate}

Otherwise, we let $u$ be the point where $p^*\in \mathsf{bag}[u]$ but not selected into $\mathsf{rep}[u]$. For any point $p'\in \mathsf{bag}[u]$, $D(p',u)<D(w,u)/(2\alpha)$, so $D(p',p^*)<D(w,u)/\alpha<D(w,p^*)$. This means that all points in $\mathsf{bag}[u]$ are closer to $p^*$ than $w$, so they can't belong to $\mathsf{ALG}$. In the following, we consider two cases depending on whether $\min\limits_{v\in \mathsf{rep}[u]}\rho(p^*,v)\ge C/3$. In each case, we will find a desired $p'\in \mathsf{rep}[u]$ and it is connected to $w$.

\begin{enumerate}
\item $\min\limits_{v\in \mathsf{rep}[u]}\rho(p^*,v)<C/3$: In this case, there exists another point $p'\in \mathsf{rep}[u]$ with $D(p^*,p')\le D(p^*,u)+D(u,p')\le D(w,u)/\alpha$ and $\rho(p^*,p')<C/3$. Because $|B_{\rho}(p^*,C/2)\bigcap (\mathsf{ALG}\setminus p_k)|<k'$, we have $|B_{\rho}(p',C/6)\bigcap (\mathsf{ALG}\setminus p_k)|\subseteq |B_{\rho}(p^*,C/2)\bigcap (\mathsf{ALG}\setminus p_k)|<k'$, so the addition of such $p'$ satisfies that $\mathsf{ALG}\setminus p_k\cup p'$ is $(k',C/12)$-diverse. 
% \snote{should we mention that there is an edge as well?}\xnote{This argument appeared before at the end of Lemma~\ref{lm:p_star_exists_diverse}}\xnote{add explicit edge}
\item $\min\limits_{v\in \mathsf{rep}[u]}\rho(p^*,v)\ge C/3$: In this case, according to Proposition~\ref{prop:greedy}, we have $\mathsf{rep}[u]=\{z_1,...,z_{k/k'}\}\subseteq B(u,D(u,w)/(2\alpha))$ all with diversity distance at least $C/3$ from each other. Therefore, for any $p_i\in \mathsf{ALG}\setminus p_k$, there can't exist two $z_j$ and $z_{j'}$ s.t. $\rho(p_i,z_j)<C/6$ and $\rho(p_i,z_{j'})<C/6$. By a counting argument, we can find at least one $z_i$ s.t. $|B_{\rho}(z_i,C/6)\cap (\mathsf{ALG}\setminus p_k)|<k'$. Finally, we let $p'=z_i$ where $\mathsf{ALG}\setminus p_k \cup p'$ is $(k',C/12)$-diverse.
\end{enumerate}

We have proved that the $p'$ we found satisfies the $(k',C/12)$-diverse criteria. Now we will bound its distance upper bound.
\begin{align}
D(p',q)&\le D(p^*,q)+D(p',p^*) \le D(p^*,q)+D(p',u)+D(p^*,u) \notag\\
&\le D(p^*,q)+D(w,u)/(2\alpha)+D(w,u)/(2\alpha) \tag{Line 9 in Algorithm~\ref{alg:indexing}}\\
&\le D(p^*,q)+D(w,u)/\alpha \notag\\
&\le D(p^*,q)+D(w,p^*)/\alpha \tag{Because $u$ is ordered earlier than $p^*$}\\
&\le D(p^*,q)+D(w,q)/\alpha+D(p^*,q)/\alpha 
\le D(p_k,q)/\alpha+\mathsf{OPT_k}(1+1/\alpha)\notag
\end{align}

\end{proof}

\begin{proof}[Proof of Theorem~\ref{thm:diverse_ann}]
% By Lemma~\ref{lm:p_star_exists_diverse} and Lemma~\ref{lm:update_diverse}, we know that at each time step, we can replace the current farthest point $p_k$ with a new point $p'$ satisfying $(k',0.05C)$-diverse and $D(p',q)\le D(p_k,q)/\alpha+D^*_k(1+1/\alpha)$. Following the same proof argument as in Theorem~\ref{thm:colorful_diverse_ann}, we can get the same convergence bound.

Regarding the running time, the total number of edges connected from any point in $\mathsf{ALG}$ is bounded by $|U|\le O((k^2/k')(8\alpha)^d\log \Delta)$. In each step, the algorithm first sorts all these edges and then checks whether each of them can be added to the new $\mathsf{ALG}$ set. The total time spent per step is $O(k|U|+|U|\log|U|)$. Usually, we assume $k\gg \log|U|$, and we can have the overall time complexity to be $O\left((k^3/k')(8\alpha)^d\log\Delta\right)$ per step.

To analyze the approximation ratio, at time step $t$, we use $\mathsf{ALG^t}=\{p^t_1,...,p^t_k\}$ to denote the current {\em unordered} solution. We denote $\mathsf{ALG^t_k}=\max\limits_{i\in[k]}D(p^t_i,q)$. According to Algorithm~\ref{alg:general_search} and Lemma~\ref{lm:update_diverse}, if $p_i$ is updated at time step $t$, we have $D(p^t_i,q)\le D(p^{t-1}_i,q)/\alpha+\mathsf{OPT_k}(1+1/\alpha)$. By an induction argument, if a point $p_i$ is updated by $t$ times at the end of time step $T$, we have $D(p^T_i,q)\le \frac{D(p^0_i,q)}{\alpha^t}+\frac{\alpha+1}{\alpha-1}\mathsf{OPT_k}$. 

We now prove that $\mathsf{ALG^T_k}\le \max\limits_{i}\frac{D(p^0_i,q)}{\alpha^{T/k}}+\frac{\alpha+1}{\alpha-1}\mathsf{OPT_k}$. Let $i\in[k]$ be the index achieving the maximal distance upper bound. For the sake of contradiction, if $\mathsf{ALG^T_k}>\frac{D(p^0_i,q)}{\alpha^{T/k}}+\frac{\alpha+1}{\alpha-1}\mathsf{OPT_k}$, this means that $p^T_i$ was updated for at most $T/k-1$ times. By a counting argument, there exists another index $j$ which was updated for at least $T/k+1$ times. However, at the time $t$ when $p^t_j$ was already updated for $T/k$ times, $D(p^t_j,q)\le \frac{D(p^0_j,q)}{\alpha^{T/k}}+\frac{\alpha+1}{\alpha-1}\mathsf{OPT_k} < \mathsf{ALG^T_k}\le \mathsf{ALG^t_k}$, so the algorithm wouldn't have chosen $p^t_j$ to optimize cause it couldn't have had the maximal distance at that time, leading to a contradiction. Therefore, we prove that $\mathsf{ALG^T_k}\le \max\limits_{i}\frac{D(p^0_i,q)}{\alpha^{T/k}}+\frac{\alpha+1}{\alpha-1}\mathsf{OPT_k}$.

%For the sake of contradiction, suppose after $T$ steps, $D^T>\frac{D(p^0_i,q)}{\alpha^{T/(k\log\alpha)}}+\frac{\alpha+1}{\alpha-1}D^*_k$, which means that the index $i$ achieves the maximal $D(p^T_i,q)$ was updated for fewer than $T/(k\log\alpha)$ times. By a counting argument, some other index $j$ was updated for at least $T/k$ times. However, by the time index $j$ was updated for $T/k$ times, its distance to $q$ had already been upper bounded by $\frac{D(p^0_j,q)}{\alpha^{T/k}}+\frac{\alpha+1}{\alpha-1}D^*_k$.

Now we consider the following three cases depending on the value of the maximal $D(p^0_i,q)$. The case analysis here is similar to the proof in Theorem 3.4 from~\cite{indykxu2024worst}.
\begin{enumerate}
\item[Case 1:] $D(p^0_i,q)>2D_{max}$. Let $p^*_k$ be the point having the maximal distance from $q$ in an optimal solution $\mathsf{OPT}$. We know that for any $p^0_i$, we have $D(p^*_k,q)\ge D(p^0_i,q)-D(p^0_i,p^*_k)\ge D(p^0_i,q)-D_{max}\ge D(p^0_i,q)/2$. Therefore, the approximation ratio after $T$ optimization steps is upper bounded by $\frac{\mathsf{ALG^T_k}}{D(p^*_k,q)}\le \frac{D(p^0_i,q)}{D(p^*_k,q)\alpha^{T/k}}+\frac{\alpha+1}{\alpha-1}\le \frac{2}{\alpha^{T/k}}+\frac{\alpha+1}{\alpha-1}$. A simple calculation shows that we can get a $(\frac{\alpha+1}{\alpha-1}+\epsilon)$ approximate solution in $O(k\log_{\alpha}\frac{2}{\epsilon})$ steps.

\item[Case 2:] $D(p^0_i,q)\le 2D_{max}$ and $\mathsf{OPT_k}>\frac{\alpha-1}{4(\alpha+1)}D_{min}$. To satisfy $\frac{D(p^0_i,q)}{\alpha^{T/k}}+\frac{\alpha+1}{\alpha-1}\mathsf{OPT_k}\le (\frac{\alpha+1}{\alpha-1}+\epsilon)\mathsf{OPT_k}$, we need $\frac{D(p^0_i,q)}{\alpha^{T/k}}\le \epsilon \mathsf{OPT_k}$. Applying the lower bound $\mathsf{OPT_k}\ge \frac{\alpha-1}{4(\alpha+1)}D_{min}$, we can get that $T\ge k\log_{\alpha}\frac{2(\alpha+1)\Delta}{(\alpha-1)\epsilon}$ suffices.

\item[Case 3:] $D(p^0_i,q)\le 2D_{max}$ and $\mathsf{OPT_k}\le \frac{\alpha-1}{4(\alpha+1)}D_{min}$. In this case, we must have $k=1$, because otherwise $D(p^*_k,p^*_1)\le 2D(p^*_k,q)<D_{min}$,violating the definition of $D_{min}$. Suppose $k=1$ and the problem degenerates to the standard nearest neighbor search problem. After $T$ optimization steps, if $p^T_1$ is still not the exact nearest neighbor, we have $D(p^T_1,q)\ge D(p^T_1,p^*_1)-\mathsf{OPT_1}\ge \frac{D_{min}}{2}$. Applying the upper bound of $D(p^T_1,q)$ and $\mathsf{OPT_1}$, we have $\frac{D_{min}}{2}\le D(p^T_1,q)\le \frac{D(p^0_1,q)}{\alpha^{T}}+\frac{\alpha+1}{\alpha-1}\mathsf{OPT_1}\le \frac{D(p^0_1,q)}{\alpha^{T}}+\frac{D_{min}}{4}$. This can happen only if $T\le \log_{\alpha}\frac{\Delta}{8}$.
\end{enumerate}
\end{proof}


%%%%%%%%%%%%%%%%%%%%%%%%%%%%%%%%%%%%%%%5
%.................DUAL.................%
%%%%%%%%%%%%%%%%%%%%%%%%%%%%%%%%%%%%%%%%
\subsection{Analysis for the Dual Diverse NN Algorithm}\label{sec:dual-analysis}

In this section we analyze Algorithm~\ref{alg:dual_search}.

\begin{algorithm}
\caption{Search algorithm for dual diverse NN}
\label{alg:dual_search}
\begin{algorithmic}[1]
\STATE \textbf{Input}: A graph $G=(V,E)$ with $N_{out}(p)$ denoting the out edges of $p$; query $q$; distance bound $R$; distance approximation error $\epsilon$.
\STATE  \textbf{Output}: A set of $k$ points $\mathsf{ALG}$.
% \STATE  $\mathsf{ALG}\gets \{p_1,...,p_k\}$ picked by the greedy algorithm of Gonzales for approximately maximizing the minimum pairwise diversity.
% \STATE  $\overline{C}\gets 4\min\limits_{p_i,p_j\in \mathsf{ALG}}\rho(p_i,p_j)$
\STATE Use binary search to find a maximal $C$ such that the initialization step proved in Lemma \ref{lm:initialization} outputs a $(k',C)$-diverse set $\mathsf{ALG}=\{p_1,...,p_k\}$
\STATE $\overline{C}\gets 4C$

\WHILE{$\max\limits_{p\in \mathsf{ALG}} D(p,q)>(\frac{\alpha+1}{\alpha-1}+\epsilon)\cdot R$}
    % \State $U\gets \{u | (p_j,u)\in E\ s.t. \exists\ p_j\in \mathsf{ALG}\}$
    \STATE  $\overline{C}\gets \overline{C}/2$
    \FOR{$i=1$ to $c\cdot k\log_{\alpha}\frac{\Delta}{\epsilon}$}
    \STATE  $U\gets \bigcup\limits_{p\in \mathsf{ALG}}(N_{out}(p)\cup p)$ and sort $U$ based on their distance from $q$   
    % \STATE $\mathsf{ALG}\gets \varnothing$
    \STATE $\mathsf{ALG}\gets$ the closest $k-1$ points in $\mathsf{ALG}$
    \FOR{each point $u \in U$ in order}
        \IF{$\mathsf{ALG}\bigcup u$ is $(k',\overline{C}/12)$-diverse}
            \STATE $\mathsf{ALG}\gets \mathsf{ALG}\cup u$
            \STATE Break
        \ENDIF
        % \IF{$|\mathsf{ALG}|=k$}
            % \STATE Break
        % \ENDIF
    \ENDFOR
    % \If{$ALG$ doesn't change at this step}
    % \State Break
    % \EndIf
    \ENDFOR
\ENDWHILE
\STATE \textbf{Return} $\mathsf{ALG}$
\end{algorithmic}
\end{algorithm}


%\begin{theorem}\label{thm:dual_diverse_ann}
%Given the graph constructed by Algorithm~\ref{alg:indexing}, the search Algorithm~\ref{alg:dual_search} finds a $(1,0.05C)$-diverse NN solution $\mathsf{ALG}$ satisfying $ALG_k\le \left(\frac{\alpha+1}{\alpha-1}+\epsilon\right)\cdot R$ in $\tilde{O}\left((8\alpha)^dk^3\log\frac{\Delta}{\epsilon}\right)$ time, if there exists a $(1,C)$-diverse solution $\mathsf{OPT}$ with $\mathsf{OPT_k}\le R$.
%\end{theorem}

\begin{proof}[Proof of Theorem~\ref{thm:dual_diverse_ann}]
% For the initial solution $\mathsf{ALG}=\{p_1,...,p_k\}$ selected by the greedy algorithm of Gonzales, we know there doesn't exist a set of $k$ points with minimum pairwise distance greater than $2\min\limits_{p_i,p_j\in \mathsf{ALG}}\rho(p_i,p_j)$. Therefore, for the initialization $\overline{C}=4\min\limits_{p_i,p_j\in \mathsf{ALG}}\rho(p_i,p_j)$, we have $\overline{C}/2\ge C$ where there exists a $(1,C)$-diverse solution $\mathsf{OPT}$ with $\mathsf{OPT_k}\le R$.

After applying the binary search to the initialization algorithm in Lemma~\ref{lm:initialization}, we get an initial $(k',C)$-diverse solution and we know there doesn't exist a $(k',4C)$-diverse solution. Therefore, we set $\overline{C}=4C$ to be the upper bound on the maximal diversity we can achieve.


Then our Algorithm~\ref{alg:dual_search} is basically adding a binary search to Algorithm~\ref{alg:general_search}. Invoking the analysis from Theorem~\ref{thm:diverse_ann}, if there exists a $(k',C)$-diverse solution $\mathsf{OPT}=\{p^*_1,...,p^*_k\}$ with $\mathsf{OPT_k}\le R$, we can find a $(k',C/12)$-diverse solution $\mathsf{ALG}=\{p_1,...,p_k\}$ with $ALG_k\le \left(\frac{\alpha+1}{\alpha-1}+\epsilon\right)\cdot R$ in $O(k\log_{\alpha}\frac{\Delta}{\epsilon})$ steps where each step takes $\tilde{O}((k^3/k')(8\alpha)^d\log\Delta)$ time. As a result, each time when the algorithm enters the while loop on Line 5 in Algorithm~\ref{alg:dual_search}, we know that there doesn't exist a $(k',\overline{C})$-diverse solution with maximal distance smaller than $R$. When we exit the while loop, the current $\overline{C}$ value is at least $1/2$ of the optimal $C$ value, and the current $\mathsf{ALG}$ solution we get is at least $(k',C/24)$-diverse.
\end{proof}