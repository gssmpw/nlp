\begin{abstract}
\setlength{\parindent}{1.5em} 
Selfish mining, a strategy introduced by \citet{eyal2013majority} where Proof-of-Work consensus participants selectively withhold blocks, allows miners to earn disproportionately high revenue. The vast majority of the selfish mining literature focuses exclusively on block rewards. 
\citet{carlsten2016instability} is a notable exception, which observes that similar strategic behavior may be profitable in a zero-block-reward regime (the endgame for Bitcoin's quadrennial halving schedule) if miners are compensated with transaction fees alone.
As of February 2025, neither model fully captures miner incentives. The block reward remains $3.125$ BTC (over $300,000$ USD at current prices), yet some blocks yield significantly higher revenue. For example, congestion during the launch of the Babylon protocol in August 2024 caused transaction fees to spike from 0.14 BTC to 9.52 BTC, a $68\times$ increase in fee rewards within two blocks.

We present a framework for considering strategic behavior under more general miner reward functions that could be stochastic, variable in time, and/or ephemeral. This model can capture many existing reward sources (sometimes called Miner/Maximal Extractable Value or MEV) in blockchains today. We use our framework to examine the profitability of cutoff selfish mining strategies (as in \citet{carlsten2016instability}) for any reward function identically distributed across forks. Our analysis requires a novel reward calculation technique to capture non-linearity in general rewards. 

We instantiate these results in a combined reward function that much more accurately represents miner incentives as they exist in Bitcoin today. This reward function includes block rewards and linear-in-time transaction fees, which have been studied in isolation. It also introduces a third random reward motivated by the aforementioned transaction fee spike. This instantiation enables us to (i) make qualitative observations (e.g., a miner considering both block rewards and transaction fees will mine more or less aggressively respectively than if they cared about either alone), (ii) make quantitative claims (e.g., the mining power at which a cutoff strategy becomes profitable is reduced by about $22\%$ when optimizing over the combined reward function instead of just block rewards), and (iii) confirm the theoretical analysis using Monte Carlo simulations.

\keywords{Selfish mining  \and Proof-of-Work \and Consensus mechanisms.}

\end{abstract}
