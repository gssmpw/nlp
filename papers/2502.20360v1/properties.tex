\section{Reward functions: properties and examples}\label{subsec:properties}

Recall that miner strategies take as input the amount of reward available for extending each existing block at time $t$, as specified by the reward function $R$, and make decisions about where to mine, what to include, and what to broadcast accordingly. This section defines a set of natural properties that reward functions might have. In \Cref{sec:examples}, we apply these properties to transaction fees and LVR, two of the primary MEV sources observed empirically to date.
While we define these properties in the context of the NCG in this paper, we believe their applicability extends far beyond Proof-of-Work and selfish mining. Our framework can be used to characterize rewards and their implications for the incentives of consensus participants across blockchain protocols.

Recall that in the NCG, given a set of miner strategies, three independent random variables $\vec{t},\vec{m},r$ are drawn and are used to compute a set of views $V_t^m$ for all miners $m$ and all times $t$.
Let $\mathcal{V}_t^m$ be the support $V_t^m$, meaning the set of views achievable at time $t$ for \emph{some} realization of $\vec{t},\vec{m},r$. Initially, $\mathcal{V}_0^m=\{V_0\}$ for all $m$, where $V_0:=\{B_0\}$ is the view containing a single genesis block $B_0$ such that $\tstamp(B_0)=0$. Miner strategies in the NCG take the realization of a reward function as input. That is, at time $t$, miner $m$ sees the reward $R^m(t,V_t^m,B,r,B')$ for extending each block $B\in V_t^m$ by a valid block $B'\in\mathcal{B}^m(t,V_t^m,B,r)$.

A miner-independent reward function yields the same value for the block regardless of who created it. This 
corresponds to a setting where all miners have access to the same set of rewards (e.g., the common value setting), and thus, we drop the superscript $m$. In practice, some reward sources may be heterogeneous between block producers (e.g., from private order flow or from differing abilities to extract MEV \cite{bahrani2024centralization}). All reward functions considered in this paper will be miner-independent, but the properties can be readily generalized by tracking the subset of miners with access to each reward source. See \Cref{sec:conclusion} for a discussion of extending this work.

\begin{definition}[Miner-Independent Rewards] \label{def:minerindependent}
    A reward function $R$ is \emph{miner-independent} if for all times $t$, all miners have the same set of valid views, the same set of valid blocks extending each block in those views, and equal rewards from any such valid block.\footnote{\label{foot:bijection}Technically, since blocks include information about their creator, it would be more accurate to say that there is a bijection between the set of valid views/blocks for any pair of miners. We overlook this formality to simplify notation.} Formally, $R$ is miner-independent if for all $t$, and all $m,m'\in M$, 
    \begin{itemize}
        \item $\mathcal{V}_t^m=\mathcal{V}_t^{m'}$,
        \item for all $V\in\mathcal{V}^m_t$, all blocks $B$ in $V$, and all $r$, we have $\mathcal{B}^m(t,V,B,r)= \mathcal{B}^{m'}(t,V,B,r)$,
        \item for all $V\in \mathcal{V}^m_t$, all $r$, all parent blocks $B$ in $V$, and all valid blocks $B' \in \mathcal{B}^m(t,V,B,r)$, we have $R^m(t,V,B,r,B')= R^{m'}(t,V,B,r,B')$.
    \end{itemize}
\end{definition}

We can also characterize reward functions that grow according to the same distribution without depending on the chain's history. The following property limits the dependence of $R$ on the view. Intuitively, it says that the only relevant information in the view that affects the amount of reward in a block is the \emph{timestamp of its parent}.

\begin{definition}[View-Independent Rewards]\label{def:viewindependent}
    A reward function $R$ is \emph{view-independent} if for all times $t'<t$, any two views $V_1,V_2\in\mathcal{V}_{t'}$ such that $\tstamp(B_1)=\tstamp(B_2)=t'$ for some blocks $B_1\in V_1,B_2\in V_2$, we have:
    \begin{itemize}
        \item for all $r$, the set of valid blocks extending $B_1$ at $t$ in $V_1$ is the same as the set of valid blocks extending $B_2$ at $t$ in $V_2$, $\mathcal{B}(t,V_1,B_1,r)=\mathcal{B}(t,V_2,B_2,r)$,\footnote{Recall that when we invoke a view and randomness together as inputs to a function, we implicitly assume that the randomness could give rise to the view.}
        and
        \item for every valid block $B'\in\mathcal{B}(t,V_1,B_1,r)$, we have
        \[
            \Pr_{r,\vec{t},\vec{m}\vert V_1}[R(t,V_1,B_1, r, B')=x]=\Pr_{r,\vec{t},\vec{m}\vert V_2}[R(t,V_2,B_2, r, B')=x]
        \]
        for all $x$.
    \end{itemize}
\end{definition}

Note that fixing a view $V_1$ (resp. $V_2$) can update the distribution of the $r,\vec{t},\vec{m}$. We use the subscript $r,\vec{t},\vec{m}\vert V_i$ to refer to the posterior distribution of these random variables conditioned on $V_1,V_2$. Block rewards are view-independent if and only if there is no halving (since halving occurs at fixed block heights).
As another (non-)example, \Cref{rem:patient-not-vi} demonstrates how transaction fees that are not fully claimed by block $B$ (e.g., from finite block sizes) are not view-independent.  

View-independence already limits the dependence of $R$ on the view to the timestamp of the parent block. We next define a subset of view-independent rewards where the dependence on view is limited to the length of \emph{elapsed time since the parent block} (and is the same regardless of the exact parent block timestamp).

\begin{definition}[Static Rewards]\label{def:static}
    A reward function $R$ is \emph{static} if for all $\Delta>0$, all times $t_1,t_2$ and views $V_1\in\mathcal{V}_{t_1}$ and $V_2\in\mathcal{V}_{t_2}$ such that $\tstamp(B_1)=t_1-\Delta$ and $\tstamp(B_2)=t_2-\Delta$, we have:
    \begin{itemize}
        \item for all $r$, the set of valid blocks extending $B_1$ at $t_1$ in $V_{1}$ is the same as the set of valid blocks extending $B_2$ at $t_2$ in $V_{2}$, $\mathcal{B}(t_1,V_{1},B_1,r)=\mathcal{B}(t_2,V_{2},B_2,r)$, and
        \item for all valid blocks $B'\in\mathcal{B}(t_1,V_1,B_1,r)$, we have
        \[
            \Pr_{r,\vec{t},\vec{m}\vert V_1}[R(t_1,V_1,B_1, r, B')=x]=\Pr_{r,\vec{t},\vec{m}\vert V_2}[R(t_2,V_2,B_2,r,B')=x]
        \]
        for all $x$.
    \end{itemize}
\end{definition}

\Cref{rem:patientisstatic} highlights that transaction fees are static using the \citet{carlsten2016instability} model with constant arrival rate and infinite block sizes. Conversely, \Cref{rem:non-local-lvr} illustrates how LVR is not static because it depends on the CEX price of an asset (which impacts the step size of the Geometric Brownian Motion). \Cref{rem:local-lvr-static} demonstrates that within the same price neighborhood, a type of LVR (which we call ``resetting'') \textit{is} static.

\begin{definition}[Maximum Rewards \& Maximizing Blocks]\label{def:max-block}
    Given a reward function $R$, we define the maximizing block function $B_{\text{opt}}$ as 
    \[
    	B_{\text{opt}}(t,V,B,r):=\underset{B'\in\mathcal{B}(t,V,B,r)}{\arg\max} R(t,V,B,r,B').
    \]
    We further define the maximum reward function $R_{\text{opt}}$ as 
    \[
    	R_{\text{opt}}(t,V,B,r):=R(t,V,B,r,B')
    \] for some $B'\in B_{\text{opt}}(t,V,B,r)$.
\end{definition}

Observe that if a reward function $R$ is static, then $\Ropt(t,V,B,r,B')$ can be rewritten as a two-variable function of just $r$ and the time $\Delta$ between $\tstamp(B)$ and $t$. 

Lastly, we define \emph{persistent rewards}, which arrive at some time and can be claimed at most once. Upon arrival, they remain indefinitely claimable by any block whose ancestors have not already claimed them. Let $\Rc(B)$ denote the amount of reward attributed to the block creator if the block becomes canonical and $\chain(B)$ the set of blocks on the ancestral path of $B$ (including $B$).

\begin{definition}[Persistent Rewards]\label{def:persistent}
A reward function $R$ is \emph{persistent} if for all realizations of $\vec{t},\vec{m},r$, at any time $t$, for all blocks $B$ in the resulting view $V$, we have:
\begin{itemize}
    \item for all $B'\in\mathcal{B}(t,V,B,r)$, 
    \begin{align}\label{eq:persistence}
        R(t,V,B,r,B')\leq R_{\text{opt}}(t,V_0,B_0,r)-\sum_{B''\in \chain(B)} \Rc(B''),
    \end{align}
    \item there exists some $B'\in\mathcal{B}(t,V,B,r)$ for which the above holds with equality.
\end{itemize}
\end{definition}

We sometimes call a non-persistent reward function \emph{ephemeral}. \Cref{rem:patient-persistent} highlights that transaction fees are persistent if the users creating the transactions are patient (willing to wait for inclusion and not cancel pending transactions). On the other hand, fees from transactions submitted by impatient users (as in \Cref{rem:impatient}) are not persistent since the canceled transactions are no longer claimable by future blocks. 

Persistent rewards are not affected by orphan or uncle blocks, but they \emph{may} be view-dependent since they are affected by the claimed rewards on the ancestral path of a block. The following lemma states that persistent rewards functions are view-independent if all blocks in all valid views claim the maximum available rewards. The proof can be found in \Cref{pr:lemm1}.

\begin{lemma}[Persistent \& Maximizing Blocks $\implies$ View-Independent]\label{lem:all-claim}
    Let $R$ be persistent. Then $R$ is view-independent if for all $t$, all $V\in\mathcal{V}_t$, all parent-child blocks $B,B'$ in $V$, and all $r$, we have $\Rc(B')=\Bopt(t,V,B,r)$.
\end{lemma}

The following lemma shows that static and persistent reward functions accrue linearly over time since the parent block, with a constant slope and intercept across blocks (but may be random depending on $r$). If a reward function is persistent and static, it can be simulated by drawing the randomness of $r$ to set the slope $a$ and the intercept $b$ of the maximum available reward function $\Ropt$. Then, for any block $B$ in any view, the reward for extending $B$ at time $\tstamp(B)+\Delta$ equals $a\cdot \Delta + b$. This is the model of transaction fee accrual in \citet{carlsten2016instability} and MEV accrual in \citet{schwarz2023time}. The proof can be found in \Cref{pr:lemm2}

\begin{lemma}[Static \& Persistent $\implies$ Linear]\label{lem:linear}
    Let $R$ be static and persistent. If $\Ropt(t,V,B,r)$ is differentiable with respect to $t$, then it is of the form $a(r)\cdot (t-\tstamp(B)) + b(r)$.
\end{lemma}

Note that the transaction fees defined in \citet{carlsten2016instability} are linear; we use this same reward function as part of our instantiation in \Cref{sec:multiplerewards}.