% \begin{enumerate}
% 	\item $\miner(B)$: the identity of the block producer,
% 	\item $\tstamp(B)$: the time\footnote{Timestamp here refers to the actual creation time of the block, rather than a reported time stated by the miner.} that the block was produced,
% 	\item  $\Rc(B)$: the amount of reward attributed to the block creator if the block becomes canonical.
% \end{enumerate} 

% If the longest chain is not unique, we let $t'$ be the earliest time where the prefix of $V_t$ as of $t'$\footnote{That is, the set of blocks with timestamps no later than $t'$.} has a unique longest chain, and define the longest chain $\lambda_t:=\lambda_{t'}$.


% \paragraph{Difficulty Adjustment.} In Nakamoto consensus, the process of mining involves solving a computational puzzle with adjustable difficulty. The amount of time it takes to find a block varies depending on the difficulty of these puzzles, which the protocol adjusts based on the historical times between blocks. blcosk,.Depending on the total hashrate of all miners, the expected time between blocks 

% \maryamnote{todo: rewrite this to add $\lambda$: NCG model should now say, look at view to figure out the coefficient $\lambda$. note that eventually the strategies reach a stable state with some orphan rate. we focus on reward calc in this stable state as a function of orphan rate, which we will laer calculate (we are assuming this actually converges, and ignoring details around convergence). utilities of miners are now in fact the total reward per time. Add a remark to explain how this is the right metric and also necessary.}\\


% \begin{enumerate}
%     \item Compute $\lambda_j$ given the current view $V_t$. Select a candidate creation time for the next block $t_j$ such that $t_j-t_{j-1}$ is drawn according to the exponential distribution with rate $1-\lambda_j$.
%     \item Let $m:=\nxtb{t,r}$ be the next miner scheduled to broadcast, and let $t'=\nxt{m,t,V_t^m,r}$ be the next broadcast time.
%     \item If $t' \leq t_j$, then update all views to reflect $m$'s newly broadcast blocks.
%     \item If $t' > t_j$, then examine the strategy of $m_j$ with inputs $t_j$, $V_t^{m_j}$, and $R^{m_j}(t_j,V_{t_j}^{m_j},B,r,B')$ for all $B\in V_{t_j}^{m_j}$ and all $B'\in\mathcal{B}^{m_j}(t,V_{t_j}^{m_j},B,r)$ to determine the parent and contents of the next block mined by $m_j$. Update $V_{t_j}^{m_j}$ accordingly. Increment $j\leftarrow j+1$.
%     \item Set $t\leftarrow t'$, and continue from the top.
% \end{enumerate}



% (\mikenote{example: finite block size, sub-linear. in reality things aren't deterministic and aren't linear so we can't just average.})

% \maryamnote{This also means, the total reward generated per time unit can vary depending on miner strategies. As a result, ratios and alpha as a benchmark are not the right approach anymore. We have to explicitly compute the total reward collected per-time unit.}

% The most responsive difficulty adjustment algorithm \textit{could} adjust the difficulty of each block based on the timestamps of its ancestors. As such, the most general version of NCG could have a specific difficulty for extending any block in any miner view. We instead assume a single difficulty, which we justify with the following two observations. 

% adjustment algorithm adjusts deterministically based  allowing the explicit calculation of the new 
% Our model assumes that fixing a set of strategies eventually in the same expected
% More specifically, for any block $B$ in any view, the timestamps of $B$'s ancestors ancestors denote whether the protocol can specify the difficulty of the puzzle for extending that particular block given the timestamps of its ancestors, increasing the difficulty . Bitcoin in particular, 
% The protocol adjusts the difficulty not have information about the total 
% The protocol can vary the which the protocol can vary in order to target a fixed growth rate of the longest chain d 


% obs 1: within time windows, the same lamdbda for every block in every view.  

% obs 2: within boundaries, not everyone has same lambda, forks are much shorter than diff. adjustment period, our strategeies converge based on a fixed orphan rate, (counter example: 

% lambdas are same across forks and views: in reality, diff adjustment is slow 
% lambdas converge: counter example: 

% \maryamnote{todo: comment on 0. why we have a more complex model than prior work (our rewards care about exact time, but in prior work wither time didn't matter or was ratioed) 1. what it means to assume the orphan rate stabilizes  2. in prior work, total reward issued was independent of the chain, so everyone focused on ratios and used $\alpha$ as the benchmark, whereas we have to care about total reward explicitly 3. why it's ok to assume the three sources of randomness are independent (and how they might not be in other consensus settings}

% Note that, while the value of $\lambda$ depends on miner strategies which in turn depend on $r$, for a fixed $\lambda$, the sources of randomness are independent.

% We assume that the orphan rate eventually stabilizes (i.e. the limit $\lambda_j$ as $j\to\infty$ exists), and only analyze profitability of different miner strategies after that point. We refer to the stable orphan rate as $\lambda$. We truncate the block creation times $\vec{t}$ to only include entries with $\lambda_j\approx\lambda$, and re-index such that $t_0=0$. In the resulting $\vec{t}$, $t_{j}-t_{j-1}$ for $j\geq 1$ is drawn i.i.d. from the exponential distribution with rate $1-\lambda$. Furthermore, this sequence $\vec{t}$ is independent of the randomness of $r$ and $m$ \emph{for fixed $\lambda$}. \mikenote{add a sentence on why this matters?}


% Given a view $V$ (public or private) at time $t$, a \emph{longest chain} in $V$ is any block in $V$ of greatest height.  If the longest chain is not unique, we let $t'$ be the earliest time where the prefix of $V_t$ as of $t'$\footnote{That is, the set of blocks with timestamps no later than $t'$.} has a unique longest chain, and set $\lc(V)define $\lambda_t:=\lambda_{t'}$. If the longest chain is unique, then $\lambda_t$ is the number of blocks in $V_t$ not on the longest chain divided by the total blocks in $V_t$.

% Given a view $V$ (public or private) at time $t$, we can compute the \emph{orphan rate} $\lambda_t$ as the fraction of blocks in $V_t$ that are not on the longest chain. Formally, a \emph{longest chain} at time $t$ is any block in $V_t$ of greatest height. If the longest chain is unique, then $\lambda_t$ is the number of blocks in $V_t$ not on the longest chain divided by the total blocks in $V_t$. If the longest chain is not unique, we let $t'$ be the earliest time where the prefix of $V_t$ as of $t'$\footnote{That is, the set of blocks with timestamps no later than $t'$.} has a unique longest chain, and define $\lambda_t:=\lambda_{t'}$.

% \begin{definition}[Persistent Rewards]\label{def:persistent}
% A reward function $R$ is \emph{persistent} if there exist sets $C_t$ and a function $\text{val}$ such that the following holds.
% For any randomness $r$, let $C_t(r)$ denote the set of claimable rewards up to and including time $t$, such that $t<t'$ implies $C_t(r)\subseteq C_{t'}(r)$. Let the set of valid blocks $\mathcal{B}(t,V,B,r)$ extending $B$ at time $t$ be $\{C_t(r)\setminus \bigcup_{B'\in \chain(B)} \Rc(B')\}$. The set of valid views $\mathcal{V}_t$ is any view containing only valid blocks.
% Let $\text{val}(S)$ denote the reward for a block that includes precisely $S$, such that for all $S,S'$, 
% \begin{align}\label{eq:persistence}
%     \text{val}(S')+\text{val}(S)=\text{val}(S\cup S')- \text{val}(S\cap S').
% \end{align}
% For any $t,V\in\mathcal{V}_t,B\in V, B'\in\mathcal{B}(t,V,B,r)$, we have $R(t,V,B,r,B'):=\text{val}(\Rc(B'))$.
% \end{definition}
% \mikenote{last todos: add explanatory sentence, check lemma statements \& proofs, carry notation through examples, check $\Rc$ globally, broken Crefs.}