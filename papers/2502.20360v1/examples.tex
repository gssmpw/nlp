
\subsection{Example reward functions}\label{sec:examples}
To illustrate the value of the aforementioned properties of reward functions, we perform two extensive case studies: transaction fees and LVR. In each category, we consider the relevant properties that arise from different assumptions about the source of the miner rewards. These examples aim to justify the properties we focus on in \Cref{subsec:properties} and motivate \Cref{sec:generalstatic,sec:multiplerewards}, which measure attacker revenue under multiple reward sources. 

\subsubsection{Transaction fees.}
Users pay transaction fees to interact with blockchains. A mempool collects transactions as they arrive, and its state at all times is captured in our model through the realization of the randomness $r$.
Consider transactions as infinitely divisible,\footnote{\label{fn:heterogeneity}We could instead consider transactions as heterogeneous in size (e.g., as in Ethereum where transactions consume different amounts of gas) or exclusive to miners (e.g., from private order flow), but the additional complexity doesn't add anything to the qualitative observations and is thus elided.} belonging to the same mempool,$^\text{\ref{fn:heterogeneity}}$ and specifying a fee.
A valid block $B'$ mined at time $t$ and extending a parent block $B$ can include any transactions in the mempool at $t$ that are not already included in $\chain(B)$.
The corresponding reward function for a valid candidate block is the sum of the fees paid by the transactions it includes.

We call users \emph{patient} if their transactions remain valid until they are eventually included in a later block. We shorthand transactions originating from patient users as \textit{patient transactions}.

% \begin{example}[Patient transaction fees are persistent with infinite blocks]\label{rem:patient-persistent}
%     The set $C_t$ is all transactions arriving before time $t$. Patient transactions remain in the mempool forever, so $C_t \subseteq C_{t'}$ for all $t \leq t'$. Each transaction comes with a fee. The $\text{val}(S)$ is the sum of the fees of the transactions in $S$ and satisfies \Cref{eq:persistence}. At time $t$ the set of valid blocks extending parent block $B$ is $\{C_t(r)\setminus \bigcup_{B'\in \chain(B)} \Rc(B')\}$ and a block $B'$ earns reward $\text{val}(\Rc(B')).$
% \end{example}
\begin{example}[Patient transaction fees with infinite capacity blocks are persistent]\label{rem:patient-persistent}
    The reward function of a candidate block $B'$ built upon a parent block $B$ is bounded above by the sum of transaction fees not claimed by any block in $\chain(B).$ 
    For any parent block $B$, the block $B'$ that contains all transactions not included in $\chain(B)$ is valid (because users are patient and blocks have infinite size) and satisfies the equality in \Cref{eq:persistence}.
\end{example}
Transaction fees cannot be persistent without infinite capacity blocks because equality will not hold if the block cannot fit all available transactions.
As demonstrated in the following example, we cannot claim any further structure on the patient-user transaction fee reward function without restricting the set of valid blocks. 

\begin{example}[Patient transaction fees may be view-dependent]\label{rem:patient-not-vi}
    Consider two blocks $B_1,B_2$ with the same timestamp $t'$ and with the same parent mined at $t$. $B_1$ claims all transaction fees arriving in $[t,t']$, while $B_2$ claims none. The rewards of maximizing candidate blocks $B_1',B_2'$ built on $B_1,B_2$ respectively, are different, as $B_2'$ can claim more transaction fees than $B_1'$.
\end{example}

This view-dependence is implied by \Cref{lem:all-claim} because $B_2$ is not a maximizing block (\Cref{def:max-block}). The key observation is that miners may not claim the complete set of available transactions, thus impacting the claimable rewards of descendant blocks in that view. Alternatively, consider the case where each block can include all transactions (e.g., infinite block size as in \citet{carlsten2016instability}). If we additionally restrict the set of views for each miner $\mathcal{V}^m_{t'}$, we \textit{can} make the following stronger claim.

\begin{example}[Patient transaction fees are view-independent if blocks are infinite capacity and fully-claiming]\label{rem:fullyclaimingviewindep}
    Assume blocks have infinite capacity and restrict views to only include blocks that contain all available transaction fees at the time of mining.
    Then, the distribution of rewards for $B'$ built at time $t$ on parent block $B_1$ or $B_2$, which have the same timestamp $t'$, is the same. Namely, the reward is the sum of patient transaction fees arriving in the interval $[t',t]$.
\end{example}

View-independence arises from the mempool fully emptying after each block is created. Thus, the reward function only depends on newly arriving transaction fees after the parent block is mined. Importantly, this reward function is not necessarily static because the transaction fee arrival rate may not be homogeneous over time. For example, some hours of the day (such as trading hours in Asia time zones) might result in higher transaction fee arrivals. Assuming a constant transaction arrival rate, we can further establish staticness.

\begin{example}[\citet{carlsten2016instability}'s model of transaction fees is static]\label{rem:patientisstatic}
    Assume 1 unit of patient transaction fees arrive per unit of time, blocks have infinite capacity, and all blocks in the view claim all available transaction fees (as in \citet{carlsten2016instability}). 
    A block $B'$ extending $B$ at time $\tstamp(B)+\Delta$ can claim any reward in $[0,\Delta]$. Therefore, this reward function is static.
\end{example}

While the previous example considers \emph{deterministic} transaction fee arrivals (1 unit of fees per unit of time), the same claim holds if the arrival rate is a function of $r$ (but still constant over time). Constant accrual, in addition to the mempool clearing, results in the reward function being independent of the timestamp of the parent block, making it static. 

Until now, we have only considered patient users. In contrast, consider \textit{impatient users}, who submit transactions that are only valid for the next block produced (e.g., by checking the height of the block they are included in before executing). We similarly shorthand these as \textit{impatient transactions}. 

\begin{example}[Identically distributed, impatient transaction fees are static but not persistent]\label{rem:impatient}
    Assuming the impatient transactions arrive according to a fixed distribution over time since the parent block, this reward function is static because the mempool clears after each block. However, these transactions are ephemeral; if a block on the ancestral chain chooses not to claim these rewards, they are lost and no longer claimable by subsequent blocks (thus violating the equality condition of \Cref{eq:persistence}).
\end{example}

Note that the mempool clearing after each block was necessary for both \Cref{rem:patientisstatic,rem:impatient} to be static. However, the clearing came about differently -- infinite block sizes in the former and impatient users in the latter. The mempool clearing is a \textit{sufficient} condition for staticness if the distribution of rewards doesn't depend on global clock time. Still, these rewards can be persistent or not, depending on the level of patience of the users. 

Varying the assumptions on block size and user patience allows us to describe reward functions under differing models of congestion; we now consider transaction fees that are high regardless of the block size. This \textit{contentious transaction} model is motivated by the launch of Babylon (\Cref{ex:babylon}). Transaction fees may spike because there is immense demand not just for inclusion in a block but also for a specific ordering (e.g., needing to be one of the first 100 transactions of a particular type). 

\begin{example}[Bernoulli rewards are static]
    Consider contentious transaction fees modeled as independent Bernoulli trials that occur once per block height, resulting in a constant random reward of size $E$ with probability $p$. This is a static reward function.
\end{example}

In \Cref{sec:multiplerewards}, we study a variant of selfish mining under a combined reward function that includes Bernoulli rewards, linear-in-time transaction fees as in \Cref{rem:patientisstatic}, and block rewards. This combined reward function is static, which is crucial to the tractability of that analysis. See \Cref{subsec:relwork} for a discussion on the similarities between our model of Bernoulli rewards and that of \citet{ZurAET23}.

\begin{example}[Bernoulli rewards are not persistent]\label{rem:bernoulli-ephemeral}
    The reward function is the outcome of the Bernoulli trial and does not allow for previous iterations of the trial to be captured in the same block (only one reward per block à la block rewards). This violates the equality condition of \Cref{eq:persistence} and is not persistent.
\end{example}

Patience levels have been studied in the context of transaction fees \cite{nisan2023serial,penna2024serial,babaioff2024optimality}. 
In practice, rewards might persist over some time but not indefinitely. For example, users might have limited patience of a few blocks rather than being fully patient (\Cref{rem:patient-persistent}) or fully impatient (\Cref{rem:impatient}). Other types of MEV may similarly only satisfy ``partial persistence.'' For example, sandwich attacks persist if the DEX price is within the slippage limit of the user's swap. A complete MEV taxonomy is out of scope for this work; see \Cref{sec:conclusion} for a discussion on natural modeling and empirical extensions.

\subsubsection{CEX-DEX Arbitrage.}
Loss-Versus-Rebalancing (\Cref{ex:lvr-ephemeral}) measures the profits earned by the arbitrageurs who balance the price of a DEX against an infinitely deep CEX. The model of \citet{lvr} assumes that the arbitrageurs continuously trade as the CEX price moves according to a Geometric Brownian Motion (abbr. GBM) stochastic process. This price movement is external and independent of the randomness of the chain and thus is captured by $r$ in our model. 
While the LVR literature does not explicitly model consensus, the profits of these arbitrageurs can be viewed as a form of MEV. The block producer fully controls the on-chain leg of the arbitrage and can replicate the strategy by continuously trading on the CEX while also continuously updating the DEX price within their block.
In Proof-of-Work, this implies that \textit{all} miners are continuously executing trades on the CEX because the next block producer is unknown. 

\begin{example}[LVR is persistent if all miners continuously trade]\label{rem:lvr-persistent}
    All miners trading continuously implies that the CEX and DEX prices are aligned at every block. In any resulting view, a continuously trading miner that mines a block at time $t$ with a parent mined at $t'$ collects the total amount of LVR during the interval $[t',t],$ as per Equation (8) in \citet{lvr}. This reward function always satisfies the second bullet in \Cref{def:persistent} and is thus persistent.
\end{example}

LVR is only persistent if miners constantly trade without knowing a priori that they will mine the subsequent block. Additionally, blocks must have infinite capacity to include the complete set of DEX trades that the miner performs during the mining process. This is consistent with the literature on LVR and might be a reasonable assumption in a Proof-of-Stake protocol where the block producer knows that they have the right to produce a block at an assigned time (e.g., in Ethereum, where the schedule of the following 64 block producers, about 10 minutes worth, is public information \cite{ethereum_consensus_specs_compute_proposer_index}). 
In Proof-of-Work, however, this model of LVR may not be a reasonable assumption as only a single miner will realize the profit from the arbitrage. The miners that lose the race execute only the CEX trades without the corresponding DEX leg of the arbitrage. Performing only the CEX trades \emph{loses} money in expectation. If the CEX price moves up from $p \nearrow p'$, the CEX leg of the arbitrage sells low (marked to the more recent and thus fair price $p'$). The same logic holds when the price moved down from $p \searrow p'$, resulting in the CEX leg buying high.

For this reason, strategic miners would instead perform a ``discrete'' version of the trade, performing the arbitrage only once to align the DEX price to the CEX at the moment of block production.\footnote{For Bitcoin specifically, the ten-minute block times make it unlikely to see significant DEX trading volumes. Discrete LVR is still the correct model for consensus protocols where block producers face uncertainty about whether they will successfully produce the next block (such as DAG consensus and Proof-of-Work with faster block times).}
We refer to this as ``discrete LVR'' because both legs happen simultaneously upon block creation rather than continuously during mining. 
\begin{example}[Discrete LVR is not persistent]\label{rem:discrete-lvr-not-persist}
    Consider a block mined at time $t$ with a parent mined at $t'$. The discrete LVR reward function captures the arbitrage profit from balancing the DEX to a CEX price a single time based on the price movement on the CEX in $[t',t]$. This is \textit{not} persistent. 
    Consider a parent-child pair of blocks $B_1,B_2$ mined at time $t_1 < t_2$ when the CEX price is $p_1 < p_2$ respectively. Assume that DEX and CEX prices are aligned in $B_1$ and $B_2$, and in particular, note that $B_2$ receives a positive discrete LVR reward. Now suppose that at time $t > t_2$, the CEX price retraces back to $p_1$. The maximizing block $B$ that extends $B_1$ at time $t$ has a discrete LVR reward of 0 because the prices on the CEX and DEX match at $t_1$ and $t$.
    However, both $B_2$ and the maximizing block $B$ extending $B_2$ at time $t$ have strictly positive discrete LVR rewards, violating
    \Cref{eq:persistence} in the definition of persistence.
\end{example}

Intuitively, discrete LVR is not persistent because the arbitrage profits can disappear if they are unclaimed at a specific time (just like impatient transaction fees in \Cref{rem:impatient}). The previous two examples characterized how LVR is persistent or ephemeral depending on the leader's advanced knowledge. The following two examples show that LVR is generally not static, except under some locality assumptions. In \Cref{sec:generalstatic}, we analyze the profitability of a selfish mining variant under general static rewards (in particular \Cref{rem:local-lvr-static} below).

We start with a closer examination of the LVR calculation in Equation (8) of \citet{lvr}, which defines LVR over a time interval as the integral of the \textit{instantaneous LVR}. Instantaneous LVR is a function of three variables: the price $P$ of the asset on the CEX, the standard deviation of the GBM representing CEX price movements, and the \emph{marginal liquidity} of the DEX at $P$ (denoted by $|x^{*'} (P)|$ in \citet{lvr}), which is a deterministic function of $P$. 
Observe that to calculate instantaneous LVR at time $t$, knowing the current price level is necessary and sufficient.
The sufficient direction implies LVR is view-independent, while the necessary direction implies LVR is not static. \Cref{rem:lvr-vi,rem:non-local-lvr} formalize this.

\begin{example}[LVR with per-block aligned CEX and DEX prices is view-independent]\label{rem:lvr-vi}
    Restrict the set of views to ones that fully align CEX and DEX prices at each block (e.g., through each miner collecting either discrete LVR as in \Cref{rem:discrete-lvr-not-persist} or continuous LVR as in \Cref{rem:lvr-persistent}).
    Then the LVR reward function for $B'$ extending either $B_1,B_2$ both with timestamp $t'$ in views $V_1,V_2$ respectively depends only on the timestamp of the parent block (and the corresponding CEX price at that time) and the random price movements of the CEX under $r$ after $t'$. Therefore, LVR in this setting is view-independent.
\end{example}

This view-independence arises from the CEX and DEX price alignment at each block, which is similar to the mempool clearing from infinite block sizes in \Cref{rem:fullyclaimingviewindep} and from user impatience in \Cref{rem:impatient}. In these examples, the reward function only depends on events occurring after the parent block is mined. 

\begin{example}[LVR is not static]\label{rem:non-local-lvr}
    No matter the restrictions we place on views and miner strategies, LVR cannot be static because the distribution of rewards depends on the price level, an exogenous variable that changes as a function of time. The reward function for LVR depends on the realized price movements on the CEX during the block creation process, which in turn depends on the price level at that time. 
\end{example}

This last example highlights a significant limitation of static rewards generally – static rewards cannot vary based on exogenous randomness. The same distinction is present in \Cref{rem:patientisstatic} and \Cref{rem:fullyclaimingviewindep}, where the distribution (over external randomness) of the reward function varying in time reduces reward sources from static to only view-independent. The methodology and analysis we present in \Cref{sec:generalstatic,sec:multiplerewards} focus on static rewards as these are capturable in a relatively simple Markov Chain. See \Cref{sec:conclusion} for a discussion on extending the state space of the Markov Chain to capture non-static rewards.

While LVR is not static, we introduce a different reward function that \emph{is} static and argue it approximates LVR within local price neighborhoods.

\begin{example}[Resetting LVR is static]\label{rem:local-lvr-static}
    Restrict the set of views to those where both CEX and DEX prices upon creation of each block are exactly $P$. Resetting LVR is the reward function that starts a new GBM at $P$ for each block and grows identically to LVR between blocks. Both continuous and discrete versions of resetting LVR are well-defined in this manner.
    The resetting-LVR reward function, in either case, is static since it depends on \emph{only} the CEX price movements under $r$ since the parent block 
    (and \emph{not} on the price level when the parent block was created). In particular, for all $t$ and all $\Delta$, the resetting-LVR reward for the maximizing block at time $t$ with a parent mined at $t-\Delta$ has the same distribution – that of LVR starting at price $P$ after time $\Delta$ has passed.
\end{example}

We claim that resetting LVR is a reasonable local approximation to LVR over a small time frame. During a short time interval, price movements are bounded, and so is the effect of changes in $P$ on instantaneous LVR.\footnote{We intentionally state these claims informally since the goal of these examples is to illustrate the applicability of the properties we introduce in \Cref{subsec:properties}. More formal versions are possible but would require a deeper dive into the particular math behind LVR, which is beyond the scope of this paper.}
To summarize, per-block price alignment implies view-independence of LVR as demonstrated in \Cref{rem:lvr-vi}. LVR is not static (\Cref{rem:non-local-lvr}) because it depends on the price level of the CEX as of the parent timestamp. Resetting LVR (\Cref{rem:local-lvr-static}) differs because the price resets each block, removing the dependence on the parent timestamp (with the only remaining dependence being on \textit{time since} the parent block), making it static. 

These examples showcase the properties we ascribe to general reward functions in \Cref{subsec:properties}. While these case studies allow us to demonstrate View-Independence (\Cref{def:viewindependent}),  Staticness (\Cref{def:static}), and Persistence (\Cref{def:persistent}) in familiar settings, they do not cover all MEV types. As mentioned in \Cref{sec:conclusion}, we see characterizing the complete set of properties and applying them to other forms of MEV (e.g., sandwiches and liquidations) as a key direction for future work. With these properties in place, we now focus on calculating expected attacker profits from performing $\beta$-cutoff selfish mining strategies under general static reward functions. 

