\section{Omitted Proofs}\label{app:omitted}

\subsection{Proof of \Cref{lem:all-claim}}\label{pr:lemm1}
\paragraph{\textbf{Lemma statement: }}
\textit{Let $R$ be persistent. Then $R$ is view-independent if for all $t$, all $V\in\mathcal{V}_t$, all parent-child blocks $B,B'$ in $V$, and all $r$, we have $\Rc(B')=\Bopt(t,V,B,r)$.}

\begin{proof}
    Suppose there is a view $V$ at time $t$ in which some blocks do not claim all rewards. Let $B_1$ mined at $t'$ be the earliest such block. Consider the prefix of $V$ as of time $t'$, and call it $V_1$. Let $B^*\in\Bopt(t,V_1,B_1,r)$ be the reward-maximizing block extending $B_1$. 
    
    Now consider a different view $V_2\in\mathcal{V}_{t'}$ that is identical to $V_1$, except $B_1$ is replaced with a reward-maximizing block $B_2\in\Bopt(t',V_{t'},\text{parent}(B_1),r)$. By persistence applied to $V_2$, we have
    \begin{align*}
        R(t,V_2,B_2,r,B^*)&\leq \Ropt(t,V_0,B_0,r)-\sum_{B''\in\chain_{V_2}(B_2)}\Rc_{V_2}(B'') \\
        &= \Ropt(t,V_0,B_0,r)-\Rc_{V_2}(B_2)-\sum_{B''\in\chain_{V_2}(\text{parent}(B_2))}\Rc_{V_2}(B'') \\
        &= \Ropt(t,V_0,B_0,r)-\Rc_{V_1}(B_1)-\sum_{B''\in\chain_{V_1}(\text{parent}(B_1))}\Rc_{V_1}(B'') \\
        &< \Ropt(t,V_0,B_0,r)-\Rc_{V_1}(B_1)-\sum_{B''\in\chain_{V_1}(B_1)}\Rc_{V_1}(B'') \\
        &= R(t,V_1,B_1,r,B^*).
    \end{align*}
    where the first inequality follows from persistence applied to $V_2$, the next equality is algebra, the next equality is by construction of $V_2$, the next inequality is by assumption that $B_1$ is not reward-maximizing and $B_2$ is, and the last equality is from persistence applied to $V_1$, and in particular invoking the second bullet in the definition of persistence. 

    This is a contradiction, since by view-independence, $B^*$ should have the same reward in $V_1$ and $V_2$.

    % (\textit{$\impliedby$ direction}) Given all $t$, all $V\in\mathcal{V}_t$, all $B$ in $V$, and all $r$, we have $\Rc(B')=\Ropt(t,V,B,r)$. 

    % Fix times $t'<t$. Let $V_1, V_2\in\mathcal{V}_{t'}$ be two views containing blocks $B_1\in V_1, B_2\in V_2$ such that $\tstamp(B_1)=\tstamp(B_2)=t'$. Pick two randomness realizations $r,r'$ that are consistent with both $V_1$ and $V_2$.
    
    % By definition of persistence, we have
    % \begin{align*}
    %     R(t,V_2,B_2,r, B)
    % \end{align*}
    
    % Consider extending blocks $B_1, B_2$, which both have timestamp $t'$ at time $t$ in views $V_1, V_2$. Since all rewards up to and including time $t'$ are claimed, the set of valid blocks extending them is the same (namely, these blocks can claim any amount of reward arriving in $[t',t].$ \mikenote{finish }
    % \maryamnote{todo: prove the other direction}
\end{proof}

\subsection{Proof of \Cref{lem:linear}}\label{pr:lemm2}
\paragraph{\textbf{Lemma statement: }}
\textit{Let $R$ be static and persistent. If $\Ropt(t,V,B,r)$ is differentiable with respect to $t$, then it is of the form $a(r)\cdot (t-\tstamp(B)) + b(r)$.}
\begin{proof}
Consider a static and persistent reward function $R$. Since $R$ is static, it is view-independent, so by \Cref{lem:all-claim}, we can restrict attention to views in which all blocks claim all rewards.

Consider the function $\Ropt$. Since $R$ is view-independent (because it is static), we can drop its dependence on $V$. Since $R$ is static, its only dependence on $B$ is through $t-\ts(B)$. Therefore, $\Ropt$ can be written as a function $f(\Delta,r)$, where $\Delta$ is the time since the creation of the parent block. We must show that $f$ takes the form $a(r)\cdot \Delta+ b(r)$.

Fix times $t'<t$ and $r$, Consider a view at time $t$ consisting of three blocks. The genesis block $B_0$, with a child $B'$ mined at $t'$, and grandchild $B$ mined at $t$. 
\begin{align*}
    f(t,r)&=f(t-t',r)+\Rc(t',r) \tag{persistence applies to $B$}\\
    &=f(t-t',r)+f(t',r) \tag{$B'$ claims all rewards}
\end{align*}
Rearranging, dividing by $t-t'$, we get
\[
\frac{f(t,r)-f(t',r)}{t-t'}=\frac{f(t-t')}{t-t'}
\]
Taking the limit $t'\to t$ (which exists since $\Ropt$ is differentiable), the left-hand-side is equal to $d/d\Delta f(t,r)$, while the right-hand-side is equal to $d/d\Delta f(0,r)$. Since the choice of $t$ was arbitrary, we conclude that the derivative of $f$ with respect to $\Delta$ is a function of $r$ and constant for all $\Delta$.

% \maryamnote{todo: mention that proof relies on all times being realizable.}
\end{proof}