\section{Preliminaries and model}\label{sec:prelims}

We start by defining a stylized model of Proof-of-Work mining with general stochastic rewards. This necessitates some crucial differences between our model and previous selfish mining literature. For example, general rewards can be sensitive to specific inter-block times, requiring explicit modeling of difficulty adjustment. \Cref{sec:notes-on-model} discusses these differences in detail.

\subsection{Nakamoto Consensus Game with general rewards}\label{subsec:NCG}

Let $M$ denote the set of $n$ miners, where miner $m\in M$ has hashrate $\alpha_m$. 

\paragraph{Views.}
At any time $t$, there is a public \emph{view} $V_t$, consisting of the ``state'' of the blockchain known to all miners at time $t$. This view includes all blocks that have already been broadcast, their creation times, and the identity\footnote{Real-world blockchains are often pseudonymous, and the ``identities'' of miners refer to their public keys.} of their creators in $M$. It also includes the content of each block, which contains enough information to compute the values of all variables and account balances in every block across forks. For each block $B$ in a view, we have $\tstamp(B)$, the time\footnote{Timestamp here refers to the actual creation time of the block, rather than a reported time stated by the miner.} that the block was produced.


At any time $t$, there is also a private view $V_t^m$ for each miner $m$ that includes $V_t$ and potentially some additional blocks $m$ knows about that are unknown to all other miners (e.g., a private fork). We assume that miners don't selectively exclude a subset of miners when they broadcast, and all broadcasting happens instantaneously (e.g., no eclipse attacks \cite{heilman2015eclipse}). As a result, $V_t^m$ will only include $V_t$ and any blocks mined by $m$ that have not yet been broadcast (along with their contents).

\paragraph{General Rewards.} Miners are rewarded for creating blocks on the eventual longest chain in the form of block rewards (a fixed value issued once per block), fees from included transactions, and potentially additional revenue stemming from their monopolistic control over the content of the block (MEV). The size of this reward can be different across blocks and might be stochastic (e.g., changes in congestion levels may cause a spike in fees). We abstractly model these rewards as a function $R$.

Fix a time $t$, a view $V$, a block $B$ in $V$, and a miner $m$. We use $r$ to capture any exogenous randomness that could impact the value of blocks that a miner creates (e.g., the price movements on centralized exchanges that could create large amounts of LVR (\Cref{ex:lvr-ephemeral})). We denote by $\mathcal{B}^m(t,V,B,r)$ the set of \emph{valid} blocks that $m$ can create.
Because not all views are achievable under a specific realization of the randomness $r$, when we invoke a view $V$ together with $r$, we implicitly restrict $r$ such that $V$ is realizable.

\begin{definition}[Reward Function]
    A \emph{reward function} $R^m$ for miner $m$ takes as input a time $t$, a view $V$, a block $B$ in $V$, randomness $r$, as well as a block $B'\in\mathcal{B}^m(t,V,B,r)$, and outputs a real number,
    \begin{align*}
    	R^m(t,V,B,r,B') \to \mathbb{R}.
    \end{align*}
\end{definition}

The output of $R^m$ can be interpreted as the amount of reward collected by $m$ for creating a block $B'$ that extends $B$ in $V$ at time $t$ given randomness $r$, assuming $B'$ ends up on the eventual longest chain.

We allow different miners to have different reward functions to keep the model general. This per-miner reward can capture miner heterogeneity (e.g., from private order flow or better trading strategies). For the properties we define in \Cref{subsec:properties} and the selfish mining analysis in \Cref{sec:generalstatic,sec:multiplerewards}, however, we restrict our study to \emph{miner-independent} (\Cref{def:minerindependent}) reward functions.

\paragraph{Miner Strategies.} Each miner $m$ has a strategy that takes as input a time $t$, a view $V_t^m$, and the reward $R^m(t,V_t^m,B,r,B')$ for extending each block $B\in V_t^m$ by a valid block $B'\in\mathcal{B}^m(t,V_t^m,B,r)$, and outputs
\begin{itemize}
    \item a block $B\in V_{t}^m$ to mine on,
    \item contents of the next block $B'\in\mathcal{B}^m(t,V_t^m,B,r)$, and
    \item a (potentially empty) subset of blocks in $V^m_t\setminus V_t$ to broadcast.
\end{itemize}
For each miner $m$, we denote by $\nxt{m,t,V_t^m,r}$ the first time after (or equal to) $t$ that $m$ broadcasts a block assuming their private view remains $V_t^m$, and by 
\begin{align*}
	\nxtb{t,r}:=\underset{m\in M}{\arg\min}\{ \nxt{m,t,V_t^m,r} \},
\end{align*}
the identity of the next miner to broadcast after (or at $t$), breaking ties arbitrarily. We use these functions to determine the ordering of broadcasters as the game progresses (in \Cref{alg:ncg-updates}).

Note that miner strategies cannot directly observe the randomness $r$ but might indirectly depend on it through the realizations of $R^m$ and $\mathcal{B}^m(t,V_t^m,B,r)$, all of which take as input the same randomness $r$. While we focus on deterministic miner strategies in this paper, our model can easily be extended to account for randomized behavior.


\paragraph{Nakamoto Consensus Game (NCG).}

The Nakamoto Consensus Game describes how views evolve given a fixed set of miner strategies. We model the game after difficulty has already been adjusted according to these strategies, resulting in a stable orphan rate $\lambda$,\footnote{See \Cref{sec:notes-on-model} for extended discussion and \cite{bitcoin-book} for a more comprehensive overview of how this model applies to the bitcoin protocol.} and we normalize time so that the average block time is 1. We let time $0$ refer to a point after which the difficulty of mining puzzles remains constant. We further assume that miners only extend blocks created after time $0$.

Prior to the game, we draw the following random variables independently:\footnote{See \Cref{sec:notes-on-model} for a discussion of why we can assume independence.}
\begin{itemize}
    \item \textit{Miner selection} – A sequence of miners $\vec{m}\in M^\mathbb{N}$, where $m_i$ is the creator of the $i^{th}$ block. For each $i$, $m_i$ is selected independently such that it equals $m\in M$ with probability $\alpha_m/\sum_{j=1}^n\alpha_j$.
    \item \textit{Block times} – A sequence of block creation times $\vec{t}\in\mathbb{R}^\mathbb{N}$, where $t_0:=0$, and the duration $t_{j}-t_{j-1}$ for $j\geq 1$ is drawn i.i.d. from an exponential distribution with rate $1/(1-\lambda)$.
    \item \textit{Remaining randomness} – The randomness $r$.
\end{itemize}
Initially, there is some public view $V_0$ but no hidden blocks, so $V^m_0=V_0$ for all $m\in M$, where $V_0:=\{B_0\}$ is the view containing a single genesis block $B_0$ such that $\tstamp(B_0)=0$.\footnote{Note that this model can capture a game that starts with an existing blockchain containing more than a single block. To do so, we can designate a unique block as $B_0$ and shift all timestamps such that $\tstamp(B_0)=0$, and restrict miner strategies only to extend $B_0$ or its descendants, and restrict the reward function to depend on the views since $B_0$.}
Starting with $j=1$ (the variable used to index the miners $\vec{m}$ and block times $\vec{t}$) and $t=0$, we check if there are new blocks to broadcast before updating the block that each miner is building on based on the contents of the pre-determined strategy. \Cref{alg:ncg-updates} demonstrates the procedure to carry out the NCG.

\begin{algorithm}
\caption{View evolution under the Nakamoto Consensus Game}
\begin{algorithmic}[1]
    \STATE Draw independent random variables: $\vec{m}, \vec{t}, r$.
    \STATE Set $V^m_0=V_0$ for all $m\in M$, where $V_0:=\{B_0\}$ and $B_0$ is genesis.
    \STATE Set $t=0, j=1$ (let $t_j$, $m_j$ denote the time and miner of the $j^{th}$ block).
    \WHILE{game continues}
        \STATE Set $m = \nxtb{t,r}$ as the next miner scheduled to broadcast.
        \STATE Set $t' = \nxt{m,t,V_t^m,r}$ as the next broadcast time.
        \IF{$t' \leq t_j$}
            \STATE Update all views to reflect $m$'s newly broadcast blocks.
            \STATE Set $t \leftarrow t'$.
        \ELSE
            % \STATE Case where $t' > t_j$ 
            \STATE Examine the strategy of $m_j$ with inputs:
            \STATE \quad $t_j$, $V_t^{m_j}$, and $R^{m_j}(t_j, V_{t_j}^{m_j}, B, r, B')$ 
            \STATE \quad for all $B \in V_{t_j}^{m_j}$ and all $B' \in \mathcal{B}^{m_j}(t, V_{t_j}^{m_j}, B, r)$\\
            \STATE \quad to determine the parent and contents of $m_j$'s new block.
            \STATE Update $V_{t_j}^{m_j}$ to include this block.
            \STATE Set $t \leftarrow t_j$.
            \STATE Increment $j \leftarrow j + 1$.
        \ENDIF
    \ENDWHILE
\end{algorithmic}
\label{alg:ncg-updates}
\end{algorithm}

We modify the NCG defined in \citet{bahrani2024undetectable} to account for more general reward functions.
Each miner $m$ collects the sum of rewards claimed in its blocks on the eventual longest chain and has utility proportional to the amount of reward it collects per unit of time. Formally, a \emph{longest chain} at time $t$ is any block in $V_t$ of greatest height. If the longest chain at time $t$ is unique, we denote by $\text{REWARD}_m^t$ the sum of rewards claimed by blocks mined by $m$ in the longest chain. If the longest chain at time $t$ is not unique, we let $t'<t$ denote the most recent time when the longest chain at $t'$ is unique, and define $\text{REWARD}_m^t:=\text{REWARD}_m^{t'}$. Recall that the longest chain at time 0 is unique by assumption, so this is always well-defined. The utility of miner $m$ is $\lim\inf_{t\rightarrow\infty} R_t^m/t.$ 


\subsection{Notes on model}\label{sec:notes-on-model}

\paragraph{Difficulty adjustment.}
In practice, mining involves solving computational puzzles with adjustable difficulty. Since miners can enter (or exit) permissionlessly, the total hashrate of all miners can vary over time, resulting in varying block production rates. The protocol varies the difficulty of these puzzles based on timestamps of recent blocks, targeting a fixed average inter-block time. In Bitcoin, the difficulty updates once every difficulty \emph{epoch} (2016 blocks/roughly every two weeks assuming ten-minute block times) by the \emph{difficulty adjustment algorithm (DAA)}. The difficulty of extending any blocks is the same within an epoch, except for forks across the epoch boundary. Note also that forks are rarely longer than a few blocks, so this represents an insignificant fraction of the blocks in an epoch.

Fixing a set of miner strategies, one can compute the expected fraction of blocks per epoch that do not end up on the longest chain. We assume the difficulty adjusts based on this expected value (rather than directly modeling per-epoch updates described above) and calculate the profitability of various strategies under this new difficulty. Specifically, we calculate the expected orphan rate $\lambda$ (\Cref{lem:lambda}), which implies the difficulty-adjusted rate of block production is $1/(1-\lambda)$. This corresponds to blocks on the longest chain growing at an average rate of 1. 

\paragraph{Comparison to prior work.}
The majority of previous literature on selfish mining \cite{eyal2013majority,sapirshtein2017optimal,nayak2016stubborn} considers block rewards as the only source of revenue for miners and thus implicitly models difficulty adjustment by defining the miner utility in terms of the percentage of blocks on the longest chain. Maximizing this objective is equivalent to maximizing the per-unit-time profit because difficulty adjustment ensures the total amount of block rewards issued per unit of time is fixed. \citet{carlsten2016instability} considers transaction fees as the sole source of miner revenue. Similarly to block rewards, these transaction fees accrue at a fixed rate per unit of time and are assumed to remain claimable any time after arrival. In both cases, the sum of the rewards collected by honest and attacker blocks per unit of time remains constant.

In practice, many sources of miner revenue may vary over time. For example, \Cref{rem:non-local-lvr} (LVR) describes one source of revenue that grows super-linearly in inter-block time, and \Cref{rem:patient-not-vi} (transaction fees with finite blocks) describes another source that grows sub-linearly.
This means that, even if difficulty adjustment guarantees a fixed \emph{average} block time, the total reward collected by honest and attacker blocks depends on the specific inter-block times. 
Therefore, a profit-maximizing attacker would not simply maximize the \emph{percentage} of rewards they collect, but rather the total amount. Our model captures these reward sources; we define miner utilities explicitly as their expected reward per unit of time. Furthermore, our profitability analyses are more nuanced as they must directly consider the specific inter-block times, which requires explicitly modeling the orphan rate and its implied block production rate.

\paragraph{Independence of randomness sources.}
There are three sources of randomness in the NCG ($\vec{t},\vec{m},r$), drawn independently prior to the game. It is not obvious that we can assume independence without loss of generality since the block production rate is a function of the orphan rate $\lambda$, which is determined by strategies of miners, which in turn depend on $r$. Crucially, the assumption that the orphan rate is stable for the entire duration of the game eliminates this dependence.
Note that such independence of miner strategies and time might not be present in other consensus games. In Proof-of-Stake, for example, the leader is elected for a fixed duration and may choose to delay their block publication intentionally to capture extra rewards – see \Cref{ex:timinggames} for a discussion of these ``timing games.''

