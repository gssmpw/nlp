\section{Related work}
\label{subsec:relwork}
Combining the proportion of block rewards and the linear-in-time transaction fee models of \citet{eyal2013majority} and \citet{carlsten2016instability} was the initial motivation for this work. We build upon their Markov Chains to analyze expected attacker rewards and study the $\beta$-cutoff strategies for selfish mining. As previously noted, neither work captures Bitcoin in 2025; the fundamental question of `how vulnerable is Bitcoin to Selfish Mining now?' remains unanswered and of interest to the Bitcoin research community.
\citet{ZurAET23} demonstrates how large ``whale transaction'' fees in conjunction with the standard block rewards may result in attacker profitability at lower hashrates. They use reinforcement learning to approximate the optimal policy and profit for attackers.
We also model these rewards as granting bonus value to blocks depending on the outcome of a Bernoulli trial. Our framework (\Cref{sec:generalstatic}) accommodates much more general rewards, and our instantiation (\Cref{sec:multiplerewards}) includes a third source – linear-in-time transaction fees.

The literature has grown extensively in the decade since the original selfish mining paper. \citet{nayak2016stubborn} and \citet{sapirshtein2017optimal} generalized the basic selfish mining strategy to broader strategy spaces. \citet{brown2019formal} demonstrated that longest chain Proof-of-Stake protocols would also be vulnerable to selfish mining – a result instantiated through numerous selfish strategies in various staking protocols: \citet{neuder2021low,schwarz2022three,neu2022two} in Ethereum, \citet{ferreira2022optimal,ferreira2024computing} in Algorand's cryptographic self-selection, \citet{neuder2019selfish,neuder2020defending} in Tezos. We extend our model of the Nakamoto Consensus Game from \citet{bahrani2024undetectable}, which studies the detectability of selfish mining in Proof-of-Work.

MEV is one of the most relevant topics existing blockchains are reckoning with. \citet{daian2019flash} coined the term and introduced many of the key properties of MEV in permissionless systems. \citet{yang2022sok} systematized MEV strategies and proposed mitigations. \citet{bahrani2024transaction,capponi2024proposer,gupta2023centralizing} focused on the centralizing nature of MEV and how Ethereum's block building market is implemented through ``Proposer-Builder Separation.'' \citet{oz2023time, schwarz2023time} studied timing games and their impact on consensus. \citet{yang2024decentralization,oz2024wins} empirically analyzed Ethereum block builders and how the market structure has evolved. We also draw on the DeFi literature when considering application-generated revenue for consensus participants. We focus on arbitrage profits as captured in loss-versus-rebalancing \cite{lvr}, which we introduced in \Cref{ex:lvr-ephemeral}. \citet{lvr-fees} extends the original model to capture trading fees.