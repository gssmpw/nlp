\section{Introduction}
Blockchain consensus mechanisms rely on incentives to coordinate behavior. To remain safe and live, crypto-economic systems require a majority (as in Proof-of-Work) or a super-majority (as in Proof-of-Stake) of participants to adopt the protocol-specified (sometimes referred to as ``honest'') actions. Selfish mining \cite{eyal2013majority} first demonstrated that this honest behavior might not be incentive compatible for the rational miner who could earn a disproportionately large fraction of block rewards by selectively delaying the publication of their blocks. In the ensuing decade, a rich literature around strategic behavior in consensus protocols developed (e.g., in Ethereum Proof-of-Stake \cite{neuder2021low, schwarz2022three,neu2022two}). The vast majority of this literature focuses on strategies that optimize for the portion of the protocol-assigned rewards earned by the agent. These rewards, sometimes referred to as ``protocol issuance'' or ``consensus rewards,'' have historically accounted for nearly all of the value in consensus participation; this is no longer true.

As modern blockchains gain usage and facilitate more significant economic activity, their decentralized applications generate revenue. Consensus participants can collect some of this revenue through the block producer's ability to arbitrarily re-order, insert, and delete transactions when they are elected leader; \citet{daian2019flash} introduces this concept as Miner/Maximal Extractable Value (abbr. MEV). MEV has been studied theoretically and measured empirically, leading to significant changes in blockchain design. Ethereum best exemplifies this, as over 90\% of its blocks are built using a public, open-outcry block-building auction. The motivation for this auction is grounded in the notion of ``fairness'' of validator rewards. By creating a transparent market for buying and selling transaction orderings, each consensus participant should earn about the same amount of MEV – a principle originally encoded into consensus rewards, which are proportional to investment (measured in either work or stake). 

A separate line of literature studies strategic behavior in decentralized finance (abbr. DeFi), which represents another source of rewards generated at the application layer. For example, loss-versus-rebalancing \cite{lvr} (abbr. LVR) measures the amount of loss incurred by liquidity providers in decentralized exchanges as arbitrageurs balance the price of the decentralized exchange against an infinitely deep centralized exchange. These losses are precisely the profit available to those performing the arbitrage. This model completely abstracts the block creation and consensus processes, only considering the profits available to traders. In reality, the block producer has the final say over the transactions in their block, resulting in a large portion of this value flowing back to the consensus participants themselves.

The perspectives of the selfish mining, MEV, and DeFi literatures are incomplete in isolation. The co-mingling of revenue across the consensus and application layers necessitates a more precise model of rewards and their impact on strategic behavior, as demonstrated in the following three real-world examples.

\begin{example}[The launch of Bablyon]\label{ex:babylon}
    On August 22, 2024, the Babylon \cite{tas2023bitcoin} protocol launched on Bitcoin. The launch allowed \texttt{BTC} tokens to be ``locked'' through a transaction processed on the chain. With a cap of 1000 \texttt{BTC}, demand for transaction inclusion spiked as people rushed to be among the first to lock their tokens. This congestion led to a $68\times$ increase in transaction fee revenue from $0.138$ to $9.515$ \texttt{BTC} between parent and child blocks \texttt{857909}, \texttt{857910}; over the four block range of \texttt{857908} to \texttt{857911}, the fee revenue increased by $500\times$ from $0.031$ to $15.551$ \texttt{BTC} \cite{bitcoinblock2025}. This immense growth in transaction fees persisted for only seven blocks, with an average per-block fee revenue of 9.64 \texttt{BTC}, after which the protocol reached its cap and fees returned to baseline levels. For those seven blocks, the block reward of $3.125$ \texttt{BTC}, which normally represents nearly the entire source of miner revenue, was only 25\% of the rewards claimed. Despite the limited scope of Bitcoin applications, Babylon exemplifies how non-protocol-specified rewards can dramatically distort miner incentives.    
\end{example}

\begin{example}[The ``Low-Carb Crusador'']\label{ex:lowcarb}
Proof-of-Stake differs from Proof-of-Work in that it requires stakers to explicitly lock up capital to participate in the system. While Proof-of-Work is limited only to incentivizing miners with positive rewards, Proof-of-Stake enforces a subset of the protocol rules through the credible threat of destroying the capital owned by a misbehaving staker. Historically, this stick has served as an effective deterrent, but on April 2, 2023, an attacker referred to as the ``Low-Carb Crusador'' exploited a piece of infrastructure in the Ethereum protocol motivated by application layer-generated rewards. By tricking a server facilitating the MEV auction referenced above, the attacker accessed private transaction data and produced two competing blocks at the same height, exploiting the private transactions for $20$ million \texttt{USD} \cite{lowcarb}. In the Ethereum specification, this behavior violated the rules and thus was subject to a slashing penalty of $1$ \texttt{ETH} ($2600$ USD at current prices) levied against the attacker's stake. Clearly, the consensus reward and penalty mechanism could not account for this magnitude of profit arising from the application layer. This example was an exploit in the software and is not replicable as the bug was fixed. Yet it still demonstrates the risk facing a consensus mechanism whose exploits can be incentivized with multi-million dollar exogenous rewards. 
\end{example}

\begin{example}[Timing games]\label{ex:timinggames}
In Proof-of-Stake protocols, no random mining process dictates the progression of time. Instead, time is explicitly discretized, and the protocol elects a leader as the sole block producer for a given slot. As in Proof-of-Work, stakers who produce valid blocks are compensated with new tokens (issuance)
– a protocol-prescribed consensus reward. For a proposed block to be accepted by the remainder of the network, it must arrive at the other nodes by a deadline. Typically, the protocol specifies that the proposer releases the block relatively early to ensure the rest of the network has plenty of time to receive it before deciding which chain to extend (e.g., in the Ethereum protocol, there is a four-second delay between the expected publication time and when the next voters determine whether the block was available or not). If the consensus rewards fully captured the incentives of stakers, the proposer would never delay their block publication, as any delay would increase the risk of the block not being received due to network latency. Yet  \citet{schwarz2023time} and \citet{oz2023time} model and measure the increase in rewards for intentionally delaying the publication of a block, a phenomenon referred to as ``timing games.'' Here again, application layer rewards distort the overall incentives of the game. Proposers benefit from the fact that any additional time allows for increased transaction fees and MEV to accrue. Thus, in some cases, delaying their block and risking losing the entire reward may be worth waiting extra time. 
\end{example}

Each example shows how the economic value generated in the application layer bleeds into the consensus layer rewards. To fully understand consensus incentives, a more general model for rewards is needed. In particular, a more accurate view of rewards would capture the aggregate incentives for following a specific strategy under many distinct revenue streams. The present work was motivated by that reality and takes the first step toward modeling general stochastic rewards in longest-chain protocols. We begin by incorporating ``general reward functions'' into the Nakamoto Consensus Game (\Cref{sec:prelims}), a contribution in its own right in capturing and highlighting the key inputs to such a function and changes in the miner strategy space. More importantly, we introduce structure into this reward function by proposing a set of properties (\Cref{subsec:properties}) that characterize many subtleties of observed blockchain rewards. The following informal example illustrates the types of distinctions we highlight.

\begin{example}[LVR is ephemeral in Proof-of-Work]\label{ex:lvr-ephemeral}
    LVR, as presented in \citet{lvr}, measures the profit of arbitrageurs who are instantaneously balancing the price of a decentralized exchange (abbr. DEX) with an infinitely deep centralized exchange (abbr. CEX). In other words, the profits depend on constantly executing trades on both venues to ensure the DEX price matches the CEX. In leader-election protocols like Proof-of-Stake, this might be reasonable. Once a leader is known, they start performing the trades and can be certain that the block they produce will contain each of those trades and become part of the canonical chain. When the next block producer is uncertain, as in Proof-of-Work, this model breaks down. Miners don't know they will produce a block a priori; thus, they will not execute trades on the CEX while mining. Instead, a more reasonable strategy is to perform the DEX leg of the arbitrage a single time as the first transaction in their block once they mine it and only then execute the CEX leg to complete the arbitrage. This distinction is critical. In the \citet{lvr} model, LVR is monotone increasing and accumulating for the block production period. In Proof-of-Work, the price on the CEX could retrace by the time a block is mined, eliminating the arbitrage profit that may have been present earlier. This ``ephemerality'' (and its inverse ``persistence'' \Cref{def:persistent}) is one property of rewards that we capture in our framework. 
\end{example}

With this natural set of properties over reward sources, we turn our attention to analyzing selfish mining strategies. We formulate a technique to calculate expected attacker profit given an aggregate reward function under mild assumptions about the distribution of the constituent reward sources (\Cref{sec:generalstatic}). This novel methodology extends the Markov Chain of \citet{carlsten2016instability} to cover a broad class of random, non-linear-in-time, and ephemeral rewards. To demonstrate this methodology, we instantiate a particular reward function that we believe more accurately models Bitcoin miner incentives as they exist today (\Cref{sec:multiplerewards}). The instantiated reward function combines three revenue sources: the block reward, linear-in-time transaction fees, and a random, per-block reward depending on the outcome of a Bernoulli trial. The first two rewards are studied in isolation in \citet{eyal2013majority} and \citet{carlsten2016instability}, respectively. We demonstrate that our new technique replicates previous results when considering these reward sources in isolation in \Cref{app:block-rews-only,app:lin-rews-only}. To our knowledge, this is the first work to study them together. The third is motivated by a sudden spike in transaction fee revenue observed from the launch of Babylon described above (\Cref{ex:babylon}) and is studied as ``whale transactions'' in \citet{ZurAET23}. 
This instantiation and application of the expected attacker profit calculation allows us to measure the impact of considering multiple reward sources on the optimal cutoff value and attacker profit. Further, by explicitly carrying out the method with this more realistic reward function, we confirm the accuracy of the expected reward calculation by comparing the algebraic solutions to simulation results.     

\subsection{Related work}\label{subsec:relwork}
Combining the proportion of block rewards and the linear-in-time transaction fee models of \citet{eyal2013majority} and \citet{carlsten2016instability} was the initial motivation for this work. We build upon their Markov Chains to analyze expected attacker rewards and study the $\beta$-cutoff strategies for selfish mining. As previously noted, neither work captures Bitcoin in 2025; the fundamental question of `how vulnerable is Bitcoin to Selfish Mining now?' remains unanswered and of interest to the Bitcoin research community.
\citet{ZurAET23} demonstrates how large ``whale transaction'' fees in conjunction with the standard block rewards may result in attacker profitability at lower hashrates. They use reinforcement learning to approximate the optimal policy and profit for attackers.
We also model these rewards as granting bonus value to blocks depending on the outcome of a Bernoulli trial. Our framework (\Cref{sec:generalstatic}) accommodates much more general rewards, and our instantiation (\Cref{sec:multiplerewards}) includes a third source – linear-in-time transaction fees.

The literature has grown extensively in the decade since the original selfish mining paper. \citet{nayak2016stubborn} and \citet{sapirshtein2017optimal} generalized the basic selfish mining strategy to broader strategy spaces. \citet{brown2019formal} demonstrated that longest chain Proof-of-Stake protocols would also be vulnerable to selfish mining – a result instantiated through numerous selfish strategies in various staking protocols: \citet{neuder2021low,schwarz2022three,neu2022two} in Ethereum, \citet{ferreira2022optimal,ferreira2024computing} in Algorand's cryptographic self-selection, \citet{neuder2019selfish,neuder2020defending} in Tezos. We extend our model of the Nakamoto Consensus Game from \citet{bahrani2024undetectable}, which studies the detectability of selfish mining in Proof-of-Work.

MEV is one of the most relevant topics existing blockchains are reckoning with. \citet{daian2019flash} coined the term and introduced many of the key properties of MEV in permissionless systems. \citet{yang2022sok} systematized MEV strategies and proposed mitigations. \citet{bahrani2024transaction,capponi2024proposer,gupta2023centralizing} focused on the centralizing nature of MEV and how Ethereum's block building market is implemented through ``Proposer-Builder Separation.'' \citet{oz2023time, schwarz2023time} studied timing games and their impact on consensus. \citet{yang2024decentralization,oz2024wins} empirically analyzed Ethereum block builders and how the market structure has evolved. We also draw on the DeFi literature when considering application-generated revenue for consensus participants. We focus on arbitrage profits as captured in loss-versus-rebalancing \cite{lvr}, which we introduced in \Cref{ex:lvr-ephemeral}. \citet{lvr-fees} extends the original model to capture trading fees. 

\subsection{Organization and summary of results}

The present work is motivated by the repeated demonstrations of rewards originating from outside of the protocol impacting incentives of consensus participants (e.g., \Cref{ex:babylon,ex:lowcarb,ex:timinggames}). We begin by defining a general reward function and describing how it impacts the strategy space of miners in the Nakamoto Consensus Game in \Cref{sec:prelims}. We focus on Proof-of-Work mining, but much of the structure we add to reward functions is generalizable to any consensus game. A key feature of our general reward model is that it can be a random function of time. This requires both explicit treatment of difficulty adjustment and its impact on the block production rate in our model, as well as changing the miner utility functions to be per-unit-time expected rewards. This additional modeling already deviates from the selfish mining literature, which can safely ignore difficulty adjustment by optimizing the ratio of attacker block rewards rather than maximizing total revenue. 

Analysis of selfish mining strategies under the most general version of the reward function is not tractable; \Cref{subsec:properties} introduces a natural set of properties motivated by the dominant sources of MEV observed in today's blockchains. These properties highlight essential differences in reward functions related to consensus incentives and the current observed types of MEV. For example, some rewards are ``persistent'' (\Cref{def:persistent}), meaning they are claimable by any block as long as no ancestor block has already included them. This is a natural way to model transaction fees, which arrive and are includable in at most one block. In contrast, other rewards may be more ephemeral. For example, an arbitrage with an external venue may disappear if the price on the external venue retraces to the original value. Our properties further capture subtle details about the random distribution of rewards. In particular, we identify a set of reward sources that are identically distributed in time since their parent block regardless of their ancestral chain, which we refer to as ``static'' rewards (\Cref{def:static}). To fully illustrate the value of the established model, \Cref{sec:examples} applies the definitions and properties through two extensive case studies. First, we explore transaction fees under differing block sizes, user patience levels, miner strategies, and contention for specific ordering. Second, we examine arbitrage through the lens of LVR (\Cref{ex:lvr-ephemeral}) and describe how various miner strategies realize the arbitrage profits over time. 

Building on the properties and examples of reward functions, \Cref{sec:generalstatic} develops a methodology for calculating expected attack profits under $\beta$-cutoff selfish mining strategies. This technique requires a novel approach to measuring expected attacker revenue under reward functions that may be random and non-linear in time; we integrate these reward sources over all possible paths that result in an attacker creating a block that captures them. We validate this technique by cross-referencing the results under just block rewards \cite{eyal2013majority} (see \Cref{app:block-rews-only}) and just linear-in-time transaction fees \cite{carlsten2016instability} (see \Cref{app:lin-rews-only}). Additionally, we simulate a combined reward function to confirm our analytic results (\Cref{fig:sims}).

\Cref{sec:multiplerewards} instantiates an aggregate function that combines block, transaction fee, and MEV rewards to more closely approximate Bitcoin incentives today. This closes the loop with the original motivation of the paper, which is the transaction fee revenue spike caused by the Babylon protocol launch (see \Cref{ex:babylon}). 
By explicitly instantiating the model, we can make quantitative claims about the impact of considering multiple rewards on the feasibility and profitability of selfish mining. For example, we demonstrate that the profitability threshold of $\beta-$cutoff selfish mining decreases by about 22\% and another 31\% compared to pure selfish mining when considering other rewards at $\gamma=0$ (see \Cref{fig:threshold-alphas}). Additionally, we demonstrate that an attacker optimizing for MEV rewards, which we model as Bernoulli trials, can be profitable even for very low values hash rate ($\alpha<10\%$) (see \Cref{fig:bernoullis}). We also make qualitative observations. For example, an attacker considering the combination of block rewards and transaction fees is \textit{less aggressive} (i.e., hides blocks less often) than the miner only concerned with transaction fees. Conversely, the combined-view attacker is \textit{more aggressive} than a miner only concerned with block rewards for many values of $\alpha$ where selfish mining is dominated by honest mining (see \Cref{fig:interpolation}). \Cref{sec:conclusion} concludes and explores extensions to the model and methodology.

