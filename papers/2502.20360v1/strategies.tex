
\subsection{Mining strategies in the NCG}\label{subsec:strategies}

% \maryamnote{todo: make definitions consistent with new NCG def}\\
% \maryamnote{todo: add a remark about undercutting and mining from behind and timing games.}\\

% In this paper, we focus on strategies that select which block to extend \emph{independently} of the the reward function, and only and only  which block to exten always select a reward-maximizing block contents, i.e. members of $\arg\max_{B'\in\mathcal{B}(t,V,B,r)}R(t,V,B,r)$ for all $t,V,B,r$. We will therefore omit the argument $B'$ dependence on $B'$

% $B'\in\arg\max_{B'\in\mathcal{B}(t,V,B,r)}R(t,V,B,r)$. 

% \begin{definition}[Maximum Available Reward]
%     The maximum available reward function $R_\max$ takes as input a time $t$, a view $V$, a block $B$ in $V$, and randomness $r$, and outputs $\max_{B'\in\mathcal{B}(t,V,B,r)}R(t,V,B,r)$.
% \end{definition}
In the NCG defined in \Cref{sec:prelims}, miners make three decisions at each time $t$: 
\begin{enumerate}
    \item which block to extend,
    \item the contents of their next mined block, and
    \item which blocks to broadcast.
\end{enumerate}
Based on these decisions, we define the protocol-prescribed mining as honest.
\begin{definition}[Honest mining]
    The honest mining strategy is defined as,
    \begin{enumerate}
        \item mine on the longest chain,
        \item claim all available rewards, and
        \item publish every block immediately.
    \end{enumerate}
\end{definition}
In words, the honest miners always follow the longest chain and immediately share any block they find with the rest of the network. If the remainder of the network is honest, the rewards that an honest miner, $i$, controlling $\alpha_i$ fraction of the hash power is proportional to their mining power.
% \mikenote{potential remark: we only study strategies where broadcasting depends on rewards whereas you could study changing the parent block e.g., mining from behind when rewards get high. model is flexible enough to capture this.} 
``Selfish mining'' \cite{eyal2013majority} prescribes a different set of rules where some blocks are selectively withheld from the network and published later to force honest miners into wasting work on blocks that do not end up on the longest chain. Succinctly, this strategy can be split into two sets of rules depending on if a ``private'' chain exists or not.
\begin{definition}[Selfish mining \cite{eyal2013majority}]
	If there is no private chain, the attacker follows the rules:
	\begin{enumerate}
		\item mine on the public longest chain,
            \item claim all available rewards, and
		\item withhold any block found.
	\end{enumerate}
	The third step above creates the private chain for the attacker; they transition into the following rule set:
	\begin{enumerate}
		\item mine on the private chain,
            \item claim all available rewards, and
		\item withhold any block found unless an honest block is found and the difference in length between the public chain and the private chain is $\leq 1$.
	\end{enumerate}    
\end{definition}

\citet{eyal2013majority} and \citet{carlsten2016instability} demonstrate that selfish mining is profitable for miners (even under various tie-breaking schemes) when considering \textit{only} block rewards or \textit{only} transaction fees that are linear-in-time respectively. \citet{carlsten2016instability} also introduced $\beta$-cutoff selfish mining strategies, in which the attacker mines selfishly as long as the rewards they earn on their hidden block are sufficiently small. If their rewards are larger than a threshold $\beta$, they instead broadcast immediately to avoid losing the valuable block. 
\begin{definition}[$\beta$-cutoff selfish mining \cite{carlsten2016instability}]\label{def:betacutoff}
	If there is no private chain, the attacker follows the rules (different from selfish mining only in step 3):
	\begin{enumerate}
		\item mine on the public longest chain,
            \item claim all available rewards, and
		\item withhold any block found where the time since parent is less than $\beta$.
	\end{enumerate}
	The second step above creates the private chain for the attacker; they transition into the following rules (same as original selfish mining):
	\begin{enumerate}
		\item mine on the private chain,
            \item claim all available rewards, and
		\item withhold any block found unless an honest block is found and the difference in length between the public chain and the private chain is $\leq 1$.
	\end{enumerate}    
\end{definition}

This strategy differs from pure selfish mining only in Step 3 under no private chain, where the attacker decides whether or not to publish based on the rewards captured in the block. Note that the strategies we consider claim all available rewards; miners could instead choose to intentionally leave  some rewards on the table to incentivize subsequent miners to build on their chain (``undercutting'' \cite{carlsten2016instability}). See \Cref{sec:conclusion} for discussion on extending our framework to a broader class of miner strategies.
