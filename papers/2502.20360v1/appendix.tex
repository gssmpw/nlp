
\appendix





% \section{Worked example with only linear in time rewards}
% Just using $R(t)=t$, we arrive at the same results from \cite{carlsten2016instability}.

% \maryamnote{todo: should go over and make sure we don't say ``matt'' lol}

% Just sanity checking that the theoretical results of the above analysis match Matt's when $C=E=0.$ This is interesting because in that case the $p_i$s match. Notice that our $F_t(\beta)$ becomes 
% \begin{align*}
%     F_t(\beta) &= 
%     \begin{cases}
%         1 & \text{if } t < \beta \\
%         0 & \text{if } t \geq \beta \\
%     \end{cases} \implies 
%     1- F_t(\beta) =
%     \begin{cases}
%         0 & \text{if } t < \beta \\
%         1 & \text{if } t \geq \beta \\
%     \end{cases}
% \end{align*}
% Also our reward function $R_B(t) = t$.

% $f_0$ is three cases, single attacker block with rewards $\geq \beta$
% \begin{align*}
%     f_{0,(i)} &= \int_0^\infty \alpha e^{-\alpha t} e^{-(1-\alpha)t} (1-F_t(\beta)) t dt \\
%     &= \alpha \int_\beta^\infty e^{-t} t dt \\
%     &= \alpha (\beta+1) e^{-\beta}
% \end{align*}
% attacker block with rewards $< \beta$, followed by another attacker block 
% \begin{align*}
%     f_{0,(ii)} &= \alpha\int_0^\infty \alpha e^{-\alpha t} e^{-(1-\alpha)t} F_t(\beta) t dt \\
%     &= \alpha^2 \int_0^\beta e^{-t} t dt \\
%     &= \alpha^2 (1-(\beta+1) e^{-\beta})
% \end{align*}
% attacker block with rewards $< \beta$, followed by honest block and tie break
% \begin{align*}
%     f_{0,(iii)} &= (1-\alpha)(\alpha+\gamma(1-\alpha))\int_0^\infty \alpha e^{-\alpha t} e^{-(1-\alpha)t} F_t(\beta) t dt \\
%     &= \alpha(1-\alpha)(\alpha+\gamma(1-\alpha)) \int_0^\beta e^{-t} t dt \\
%     &= \alpha(1-\alpha)(\alpha+\gamma(1-\alpha)) (1-(\beta+1) e^{-\beta})
% \end{align*}
% So
% \begin{align*}
%     f_0 &= \alpha (\beta+1) e^{-\beta} + \alpha^2 (1-(\beta+1) e^{-\beta}) +\alpha(1-\alpha)(\alpha+\gamma(1-\alpha)) (1-(\beta+1) e^{-\beta})
% \end{align*}

% $f_1$ is medium
% \begin{align*}
%     f_1 &= \alpha + \alpha(1-\alpha) \int_0^\infty \int_0^\infty (t_1+t_2) e^{-(t_1+t_2)}dt_2dt_1 \\
%     &= \alpha + 2 \cdot \alpha (1-\alpha)
% \end{align*}

% $f_2$ is same as $f_1$
% \begin{align*}
%     f_2 &= \alpha + 2 \cdot \alpha (1-\alpha).
% \end{align*}

% $f_3$ is medium $A, HA, HHA$.
% \begin{align*}
%     f_3 &= \int_0^\infty \alpha e^{-\alpha t} e^{-(1-\alpha)t}t dt + \int_0^\infty \int_0^\infty (1-\alpha)e^{-(1-\alpha)t_1} e^{-\alpha t_1} \cdot \alpha e^{-\alpha t_2} e^{-(1-\alpha)t_2} (t_1+t_2) dt_2 dt_1 \\ 
%     &+ \int_0^\infty \int_0^\infty  (t_1 + t_2) (1-\alpha)^2 t_1 e^{-(1-\alpha)t_1} e^{-\alpha t_1}\alpha e^{-\alpha t_2} e^{-(1-\alpha)t_2} dt_2 dt_1 \\ 
%     &= \alpha + 2 \alpha(1-\alpha) + 3 \alpha(1-\alpha)^2
% \end{align*}

% \paragraph{$f_ip_{i-1}$}
% \begin{align*}
%     p_i &= p_1 \left(\frac{a}{1-a}\right)^{i-1}, \; i\geq 1 \\ 
%     f_i &= a \sum_{j=0}^{i-1} (1-a)^{j} \cdot (j+1) \\ 
%     &= \frac{1-(1+ia)(1-a)^i}{a} \\
%     &\sum_{i=2}^\infty \alpha p_{i-1}f_i
% \end{align*}

% \section{honest rewards}

% \subsection{Honest rewards}
% \mikenote{We mightn't need honest rewards actually}
% We must also consider honest rewards when evaluating the profitability of the $\beta$-cutoff strategy. Let $g_i$ denote the expected reward of a canonical honest block that was mined in \texttt{State i}. Recall from Example~\ref{ex:state3paths}, that there is a single path, \texttt{HHH}, from \texttt{State 3} that results in the honest parties collecting rewards. This path has an expected length of $3/\lambda$ and the constant and Bernoulli rewards are added directly (we can use the expectation of the Bernoulli trial because the outcome of the trial doesn't impact the probability that the reward is realized. Thus the expected reward from that outcome is,
% \begin{align*}
%     g_3 = \underbrace{(C+ p\cdot E + 3/\lambda)}_{\text{3 component reward}} \cdot \underbrace{(1-a)^3}_{\text{probability}}
% \end{align*}

% This leads to the following lemma,
% \begin{lemma}[$g_{i\geq 2}$]
%     For each \texttt{State i}, $i \geq 2$, 
%     \begin{align*}
%         g_{i \geq 2} = (C+ p\cdot E + i/\lambda) \cdot (1-\alpha)^i.
%     \end{align*}
% \end{lemma}

% \subsubsection{Honest rewards in \statezero \& \stateone}\label{sub:honestrewszeroone}
% From \statezero, the honest parties have two paths resulting in them earning the rewards. We denote these as two cases:

% \begin{description}[leftmargin=!,labelwidth=2.5cm]
%     \item[Case i] the honest party finds the block and publish immediately,
%     \item[Case ii] the attacker finds the block with less rewards than $\beta$ (the attacker hides) \textit{and} the honest party finds the subsequent block \textit{and} the honest chain wins the tie break.
% \end{description}

% \textbf{Case i} is simply $g_{0,(i)} = 1-\alpha$. \textbf{Case 2} requires calculating the probability and expected length of the honest chain as,
% \mikenote{this isn't quite right as it needs to also consider the case where the flip fails.}
% \begin{align*}
%     g_{0,(ii)} &= \underbrace{\alpha (1-\alpha)^2 (1-\gamma)}_{\text{probability of outcome}} \cdot \bigg(\underbrace{2\left(1-\min\left(1,e^{-\beta+C+E}\right)\right)}_{\shortstack{\scriptsize expected time in  \\\scriptsize two block honest fork}}+\underbrace{1-\max(0,\beta-C-E)\min\left(1,e^{-\beta+C+E}\right)}_{\shortstack{\scriptsize expected time in   attacker block}}\bigg)
% \end{align*}
% \mikenote{this isn't quite right as it needs to also consider the case where the flip fails.}

% Lastly in \stateone, the $\gamma$ fraction of the honest miners who mine on the attacker fork could end up producing a block with an expected time of 2, thus 
% \begin{align*}
%     g_{1} = (C+p\cdot E + 2) \cdot (1-\alpha)^2 \gamma.
% \end{align*}



% \subsubsection{Full honest rewards}

% \begin{lemma}
%     The full honest rewards are, 
%     \begin{align*}
%         \text{HONEST REWARD} = p_0 g_0 + p_1 g_1 + p_1 \cdot \alpha (2-\alpha)
%     \end{align*}
% \end{lemma}
% \begin{proof}
%     Infinite sum
%     \begin{align*}
%         p_{i-1} &= p_1 \left(\frac{a}{1-a}\right)^{i-2}, \; i\geq 2 \\ 
%         & \sum_{j=2}^{\infty} \alpha p_1 \left(\frac{\alpha}{1-\alpha}\right)^{i-2} \cdot i (1-a)^i= \alpha p_1(2-\alpha)
%     \end{align*}
% \end{proof}


% \subsection{Expected honest rewards per-state}\label{subsec:perstatehonest}
% Let $g_i$ denote the expected reward of an honest block that was mined in \texttt{State i}. \mikenote{Highlight why we need to calculate these but Carlsten et al. didn't.}
% On the honest side, the only path from \texttt{State i}, $i\geq 2$ that results in those rewards being captured by the honest miners are $i$ consecutive honest blocks in a row. This leads to the following corollary to Theroem~\ref{thm:attackgeq2}.
% \begin{corollary}
%     For all states $i\geq 2$, the expected honest rewards collected in \texttt{State i}, $g_i$ is
%     \begin{align*}
%         g_i =  \frac{(1-\alpha)^i}{(i-1)!} \cdot \int_0^\infty R(t)  t^{i-1}  e^{-t}  dt.
%     \end{align*}
% \end{corollary}
% \begin{proof}
%     Single path, multiply by integral, use Erlang and Law of Unconscious Statistician.
% \end{proof}



% \subsubsection{Honest rewards in \statezero \;\& \stateone} Using the same cases as outlined in Section~\ref{sub:honestrewszeroone}, we have the following,
% \begin{align*}
%     g_{0,(i)} = \underbrace{(1-\alpha) 
%     \int_0^\infty 
%     e^{-t} R(t) dt}_{_{\shortstack{\scriptsize honest before attacker\\ \scriptsize by time $t$}}}
% \end{align*}
% which describes the case where the honest party finds a block immediately. For the second case, we have
% \begin{alignat*}{2}
%     g_{0,(ii)} &=  \underbrace{(1-\gamma)(1-\alpha)}_{\shortstack{\scriptsize honest fork \\ \scriptsize wins tie-break}}
%     \cdot \int_{t_1=0}^\infty \int_{t_2=0}^\infty &&
%     \underbrace{\alpha e^{-\alpha t_1} e^{-(1-\alpha)t_1}}_{\shortstack{\scriptsize attacker before honest\\ \scriptsize by time $t_1$}}
%     \cdot \underbrace{F_{t_1}(\beta)}_{\shortstack{\scriptsize rewards $ <\beta$\\ \scriptsize by time $t_1$}} \\
%     & &&\cdot \underbrace{(1-\alpha) e^{-(1-\alpha) t_2} e^{-\alpha t_2}}_{\shortstack{\scriptsize honest before attacker\\ \scriptsize by time $t_2$}} dt_2 dt_1\\
%     &=  \alpha (1-\gamma)(1-\alpha)^2 \int_{t_1=0}^\infty  \int_{t_2=0}^\infty && e^{-(t_1+t_2)} F_{t_1}(\beta) R(t_1+t_2) dt_2 dt_1.
% \end{alignat*}

% For \stateone, the honest party canonicalizes rewards if they end up mining the tie breaking block. This happens over the course of two time periods as they first mine a block transitioning to \texttt{State 0'} and then mine on the tie breaking block. We only account for the path where they mine the tie breaking block on the attacker chain, because the calculation of $g_{0,(ii)}$ covers the other case. Thus,
% \begin{align*}
%     g_1 &= (1-\alpha)^2 \int_{t_1}^\infty \int_{t_2}^\infty e^{-t_1+t_2} R(t_1 + t_2) dt_2 dt_1.
% \end{align*}

% \begin{definition}
%     The full honest rewards are, 
%     \begin{align*}
%         \text{HONEST REWARD} = p_0 g_0 + p_1 g_1 + \alpha  \sum_{i=2}^\infty f_i p_{i-1}.
%     \end{align*}
% \end{definition}

