%%%%%%%% ICML 2024 EXAMPLE LATEX SUBMISSION FILE %%%%%%%%%%%%%%%%%

\documentclass{article}

% Recommended, but optional, packages for figures and better typesetting:
\usepackage{microtype}
\usepackage{graphicx}
\usepackage{booktabs} % for professional tables
\usepackage{duckuments} % for more interesting ducks
\usepackage{tcolorbox}
\usepackage{todonotes}

% hyperref makes hyperlinks in the resulting PDF.
% If your build breaks (sometimes temporarily if a hyperlink spans a page)
% please comment out the following usepackage line and replace
% \usepackage{icml2024} with \usepackage[nohyperref]{icml2024} above.
\usepackage{hyperref}
\usepackage{array}
%\usepackage{algorithm}
\usepackage{algpseudocode}
\usepackage{makecell} % For multiline cells
\renewcommand\cellalign{cc} % Align cells both vertically and horizontally
%\usepackage[dvipsnames]{xcolor} % For extended color support


% Attempt to make hyperref and algorithmic work together better:
\newcommand{\theHalgorithm}{\arabic{algorithm}}

% Use the following line for the initial blind version submitted for review:
% \usepackage{icml2024}

% If accepted, instead use the following line for the camera-ready submission:
\usepackage[accepted]{icml2025}

% For theorems and such
\usepackage{amsmath}
\usepackage{amssymb}
\usepackage{mathtools}
\usepackage{amsthm}
\usepackage{color,soul}
\usepackage{tabularx}
\usepackage{graphicx}
\usepackage{multirow}
\usepackage{dsfont}
\usepackage{enumitem}
\newcolumntype{C}[1]{>{\centering\arraybackslash}m{#1}}
% if you use cleveref..
\usepackage[capitalize,noabbrev]{cleveref}

%%%%%%%%%%%%%%%%%%%%%%%%%%%%%%%%
% THEOREMS
%%%%%%%%%%%%%%%%%%%%%%%%%%%%%%%%
\theoremstyle{plain}
\newtheorem{theorem}{Theorem}[section]
\newtheorem{proposition}[theorem]{Proposition}
\newtheorem{lemma}[theorem]{Lemma}
\newtheorem{corollary}[theorem]{Corollary}
\theoremstyle{definition}
\newtheorem{definition}[theorem]{Definition}
\newtheorem{assumption}[theorem]{Assumption}
\theoremstyle{remark}
\newtheorem{remark}[theorem]{Remark}

% Todonotes is useful during development; simply uncomment the next line
%    and comment out the line below the next line to turn off comments
%\usepackage[disable,textsize=tiny]{todonotes}
%\usepackage[textsize=tiny]{todonotes}
\usepackage{subcaption}
\DeclareMathOperator*{\argmin}{\arg\!\min}
\DeclareMathOperator*{\argmax}{\arg\!\max}
\DeclareMathOperator*{\E}{\mathbb{E}}
\DeclareMathOperator*{\Var}{Var}
\DeclareMathOperator*{\median}{median}
\DeclareMathOperator*{\mean}{mean}
\DeclareMathOperator{\supp}{supp}
\DeclareMathOperator{\esssup}{ess\,sup}
\DeclareMathOperator{\fastmobius}{FastMobius}
\DeclareMathOperator{\poly}{poly}
\DeclareMathOperator{\dec}{Dec}
\DeclareMathOperator{\sinc}{sinc}
\DeclareMathOperator{\linspan}{span}
\DeclareMathOperator{\proj}{Proj}
\DeclareMathOperator{\sv}{SV}
\DeclareMathOperator{\stii}{STII}
\DeclareMathOperator{\bz}{BZ}
\DeclareMathOperator{\err}{err}
\DeclareMathOperator{\obj}{obj}
\DeclareMathOperator{\bern}{Bern}
\DeclareMathOperator{\binomdist}{Binom}
\DeclareMathOperator\erf{erf}

\newcommand{\BY}[1]{\todo[author=BY, color=red, size=\small]{\color{white}#1}}
\newcommand{\Abhi}[1]{\todo[author=Abhi, color=orange, size=\small]{\color{white}#1}}
\newcommand{\JK}[1]{\todo[author=JK, color=blue, size=\small]{\color{white}#1}}
\newcommand{\KR}[1]{\todo[author=KR, color=green, size=\small]{\color{white}#1}}
\newcommand{\Efe}[1]{\todo[author=Efe, color=purple, size=\small]{\color{white}#1}}

\newcommand{\type}[1]{\mathrm{Type} \left( #1 \right)}
\newcommand{\detect}[1]{\mathrm{Detect} \left( #1 \right)}

\newcommand{\abs}[1]{\left\lvert #1 \right\rvert}
\newcommand{\red}[1]{{\color{red}#1}}
\newcommand{\blue}[1]{{\color{blue}#1}}
\definecolor{lightpink}{rgb}{1,0.9,0.9}
\newcommand{\defeq}{\vcentcolon=}
\newcommand{\eqdef}{=\vcentcolon}
\renewenvironment{quote}{%
  \list{}{%
    \leftmargin0.5cm   % this is the adjusting screw
    \rightmargin\leftmargin
  }
  \item\relax
}
{\endlist}
\newcommand{\dt}{\text{ d}}
\newcommand{\calD}{\mathcal{D}}
\newcommand{\cA}{\mathcal A}
\newcommand{\cB}{\mathcal B}
\newcommand{\cC}{\mathcal C}
\newcommand{\cD}{\mathcal D}
\newcommand{\cE}{\mathcal E}
\newcommand{\cF}{\mathcal F}
\newcommand{\cG}{\mathcal G}
\newcommand{\cH}{\mathcal H}
\newcommand{\cI}{\mathcal I}
\newcommand{\cJ}{\mathcal J}
\newcommand{\cK}{\mathcal K}
\newcommand{\cL}{\mathcal L}
\newcommand{\cM}{\mathcal M}
\newcommand{\cN}{\mathcal N}
\newcommand{\cO}{\mathcal O}
\newcommand{\cP}{\mathcal P}
\newcommand{\cQ}{\mathcal Q}
\newcommand{\cR}{\mathcal R}
\newcommand{\cS}{\mathcal S}
\newcommand{\cT}{\mathcal T}
\newcommand{\cU}{\mathcal U}
\newcommand{\cV}{\mathcal V}
\newcommand{\cW}{\mathcal W}
\newcommand{\cX}{\mathcal X}
\newcommand{\cY}{\mathcal Y}
\newcommand{\cZ}{\mathcal Z}
\newcommand{\scrA}{\mathscr A}
\newcommand{\scrB}{\mathscr B}
\newcommand{\scrC}{\mathscr C}
\newcommand{\scrD}{\mathscr D}
\newcommand{\scrE}{\mathscr E}
\newcommand{\scrF}{\mathscr F}
\newcommand{\scrG}{\mathscr G}
\newcommand{\scrH}{\mathscr H}
\newcommand{\scrI}{\mathscr I}
\newcommand{\scrJ}{\mathscr J}
\newcommand{\scrK}{\mathscr K}
\newcommand{\scrL}{\mathscr L}
\newcommand{\scrM}{\mathscr M}
\newcommand{\scrN}{\mathscr N}
\newcommand{\scrO}{\mathscr O}
\newcommand{\scrP}{\mathscr P}
\newcommand{\scrQ}{\mathscr Q}
\newcommand{\scrR}{\mathscr R}
\newcommand{\scrS}{\mathscr S}
\newcommand{\scrT}{\mathscr T}
\newcommand{\scrU}{\mathscr U}
\newcommand{\scrV}{\mathscr V}
\newcommand{\scrW}{\mathscr W}
\newcommand{\scrX}{\mathscr X}
\newcommand{\scrY}{\mathscr Y}
\newcommand{\scrZ}{\mathscr Z}
\newcommand{\bbB}{\mathbb B}
\newcommand{\bbS}{\mathbb S}
\newcommand{\bbR}{\mathbb R}
\newcommand{\bbZ}{\mathbb Z}
\newcommand{\bbI}{\mathbb I}
\newcommand{\bbQ}{\mathbb Q}
\newcommand{\bbP}{\mathbb P}
\newcommand{\bbE}{\mathbb E}
\newcommand{\bbF}{\mathbb F}
\newcommand{\bbN}{\mathbb N}
\newcommand{\sfE}{\mathsf E}
\newcommand{\sfF}{\mathsf F}
\newcommand{\sfV}{\mathsf V}
\newcommand{\one}{\mathds 1}

\newcommand{\Tr}{ \text{Tr} }
\newcommand{\trans}{\top}

\newcommand{\by}{\mathbf y}
\newcommand{\bxi}{\boldsymbol \xi}
\newcommand{\bx}{\mathbf x}
\newcommand{\bd}{\mathbf d}
\newcommand{\barbd}{\mathbf{ \overline{ d}}}
\newcommand{\barbh}{\mathbf{ \overline{ h}}}
\newcommand{\bc}{\mathbf c}
\newcommand{\be}{\mathbf e}
\newcommand{\bg}{\mathbf g}
\newcommand{\bw}{\mathbf w}
\newcommand{\bW}{\mathbf W}
\newcommand{\bP}{\mathbf P}
\newcommand{\bv}{\mathbf v}
\newcommand{\bj}{\mathbf j}
\newcommand{\bu}{\mathbf u}
\newcommand{\ba}{\mathbf a}
\newcommand{\bp}{\mathbf p}
\newcommand{\bk}{\mathbf k}
\newcommand{\bbm}{\mathbf m}
\newcommand{\br}{\mathbf r}
\newcommand{\bM}{\mathbf M}
\newcommand{\bX}{\mathbf X}
\newcommand{\bY}{\mathbf Y}
\newcommand{\bD}{\mathbf D}
\newcommand{\bG}{\mathbf G}
\newcommand{\bI}{\mathbf I}
\newcommand{\bH}{\mathbf H}
\newcommand{\bh}{\mathbf h}
\newcommand{\bn}{\mathbf n}
\newcommand{\bs}{\mathbf s}
\newcommand{\bS}{\mathbf S}
\newcommand{\bZ}{\mathbf Z}
\newcommand{\bU}{\mathbf U}
\newcommand{\bbeta}{\bm \beta}
\newcommand{\balpha}{\bm \alpha}
\newcommand{\bell}{\boldsymbol{\ell}}
\newcommand{\bZero}{\boldsymbol 0}
\newcommand{\bOne}{\boldsymbol 1}
\newcommand{\beps}{\boldsymbol{\epsilon}}
\newcommand{\indep}{\perp \!\!\! \perp}
\DeclarePairedDelimiterX{\inp}[2]{\langle}{\rangle}{#1, #2}
\newcommand{\norm}[1]{\left\lVert#1\right\rVert}
\newcommand{\SpecExp}{\textsc{SPEX}}
% The \icmltitle you define below is probably too long as a header.
% Therefore, a short form for the running title is supplied here:
\icmltitlerunning{SpectralExplain}

\begin{document}

\twocolumn[
\icmltitle{\SpecExp: Scaling Feature Interaction Explanations for LLMs}

% It is OKAY to include author information, even for blind
% submissions: the style file will automatically remove it for you
% unless you've provided the [accepted] option to the icml2024
% package.

% List of affiliations: The first argument should be a (short)
% identifier you will use later to specify author affiliations
% Academic affiliations should list Department, University, City, Region, Country
% Industry affiliations should list Company, City, Region, Country

% You can specify symbols, otherwise they are numbered in order.
% Ideally, you should not use this facility. Affiliations will be numbered
% in order of appearance and this is the preferred way.
\icmlsetsymbol{equal}{*}

\begin{icmlauthorlist}
\icmlauthor{Justin Singh Kang}{equal,yyy}
\icmlauthor{Landon Butler}{equal,yyy}
\icmlauthor{Abhineet Agarwal}{equal,xxx}
\icmlauthor{Yigit Efe Erginbas}{yyy}
\icmlauthor{Ramtin Pedarsani}{zzz}
\icmlauthor{Kannan Ramchandran}{yyy}
\icmlauthor{Bin Yu}{yyy,xxx}
%\icmlauthor{Firstname7 Lastname7}{comp}
%\icmlauthor{}{sch}
%\icmlauthor{Firstname8 Lastname8}{sch}
%\icmlauthor{Firstname8 Lastname8}{yyy,comp}
%\icmlauthor{}{sch}
%\icmlauthor{}{sch}
\end{icmlauthorlist}

\icmlaffiliation{yyy}{Department of Electrical Engineering and Computer Science, UC Berkeley}
\icmlaffiliation{xxx}{Department of Statistics, UC Berkeley}
\icmlaffiliation{zzz}{Department of Electrical and Computer Engineering, UC Santa Barbara}

% \icmlaffiliation{comp}{Company Name, Location, Country}
% \icmlaffiliation{sch}{School of ZZZ, Institute of WWW, Location, Country}

\icmlcorrespondingauthor{Justin Singh Kang}{justin\_kang@berkeley.edu}
% \icmlcorrespondingauthor{Firstname2 Lastname2}{first2.last2@www.uk}

% You may prov$ide any keywords that you
% find helpful for describing your paper; these are used to populate
% the "keywords" metadata in the PDF but will not be shown in the document
\icmlkeywords{Machine Learning, ICML}

\vskip 0.3in
]

\printAffiliationsAndNotice{\icmlEqualContribution} 
% otherwise use the standard text.

\begin{abstract}

 
Large language models (LLMs) have revolutionized machine learning due to their ability to capture complex interactions between input features. Popular post-hoc explanation methods like SHAP provide \textit{marginal} feature attributions, while their extensions to interaction importances only scale to small input lengths ($\approx 20$). We propose \emph{Spectral Explainer} (\SpecExp{}), a model-agnostic interaction attribution algorithm that efficiently scales to large input lengths ($\approx 1000)$. \SpecExp{} exploits underlying natural sparsity among interactions---common in real-world data---and applies a sparse Fourier transform using a channel decoding algorithm to efficiently identify important interactions.
We perform experiments across three difficult long-context datasets that require LLMs to utilize interactions between inputs to complete the task. For large inputs, \SpecExp{} outperforms marginal attribution methods by up to 20\% in terms of faithfully reconstructing LLM outputs. Further, \SpecExp{} successfully identifies key features and interactions that strongly influence model output. For one of our datasets, \textit{HotpotQA}, \SpecExp{} provides interactions that align with human annotations. Finally, we use our model-agnostic approach to generate explanations to demonstrate abstract reasoning in closed-source LLMs (\emph{GPT-4o mini}) and compositional reasoning in vision-language models.
\end{abstract}
\section{Introduction}
\label{sec:intro}

\begin{figure*}[tb]
    \centering
    \includegraphics[width=0.848\linewidth]{figs/circuitnn.pdf} 
    \caption{Illustration of differentiable CircuitNN. CircuitNN is designed based on differentiable NAND gates. After DAS is guided by PI and PO pairs of the truth table, CircuitNN can get the precise circuit architecture logic equivalent to the truth table.}
    \label{fig:circuitnn}
\end{figure*}

% 1. Describe the importance of logic synthesis
% 2. Existing Problems
% (a) Neural Architecture Search: Unstable, Predefined Setting, etc.
% (b) Circuit Generation: Probabilistic Model, Logic Equivalence

With the rapid advancement of technology, the scale of integrated circuits (ICs) has expanded exponentially. 
This expansion has introduced significant challenges in chip manufacturing, particularly concerning power and area metrics.
A primary objective in IC design is achieving the same circuit function with fewer transistors, thereby reducing power usage and area occupancy.

Logic synthesis~\cite{hachtel2005logicsynth}, a critical step in electronic design automation (EDA), transforms behavioral-level circuit designs into optimized gate-level circuits, ultimately yielding the final IC layout. 
The primary goal of logic synthesis is to identify the physical implementation with the fewest gates for a given circuit function. 
This task constitutes a challenging NP-hard combinatorial optimization problem. 
Current logic synthesis tools~\cite{brayton2010abc, wolf2013yosys} rely on human-designed heuristics, often leading to sub-optimal outcomes.

Differentiable architecture search (DAS) techniques~\cite{liu2018darts, chu2020darts} offer novel perspectives on addressing challenges in this problem.
Circuit functions can be represented through truth tables, which map binary inputs to their corresponding outputs. 
Truth tables provide a precise representation of input-output relationships, ensuring the design of functionally equivalent circuits.
Inspired by this, researchers~\cite{deepmind2024ai4sys, wang2024tnet} have begun exploring the application of DAS to synthesize circuits directly from truth tables.
Specifically, \citet{deepmind2024ai4sys} proposed CircuitNN, a framework that learns differentiable connection structures with logic gates, enabling the automatic generation of logic circuits from truth tables.
This approach significantly reduces the complexity of traditional circuit generation. 
Building on this, \citet{wang2024tnet} introduced T-Net, a triangle-shaped variant of CircuitNN, incorporating regularization techniques to enhance the efficiency of DAS.

Despite these advancements, several challenges remain. 
The computational complexity of DAS grows quadratically with the number of gates, posing scalability issues.
Although triangle-shaped architecture~\cite{wang2024tnet} partially mitigates this problem, redundancy persists. 
%Additionally, DAS is susceptible to converging to local optima, limiting the ability to search architectures that satisfy the given truth tables~\cite{liu2018darts}. 
%Furthermore, hyperparameters (network depth and layer width) require extensive searches, introducing complexity and prolonging the synthesis process. 
Additionally, DAS is susceptible to converging to local optima~\cite{liu2018darts} and hyperparameters (network depth and layer width) require extensive searches. 
The challenges arise from the vast search space in DAS. 
% Even with predefined settings for CircuitNN, finding a configuration that meets the truth table requires extensive trial and error during the DAS process. 
Intuitively, limiting the search space through predefined parameters (network depth, gates per layer, and connection probabilities) can significantly reduce the complexity.

Recent advances~\cite{openai2023gpt4, abramson2024alphafold3, esser2024sd3, li2024mar} in conditional generative models have demonstrated remarkable performance across language, vision, and graph generation tasks. 
Motivated by these developments, we propose a novel approach to circuit generation that generates preliminary circuit structures to guide DAS in generating refined circuits matching specified truth tables. 
Firstly, we introduce CircuitVQ, a tokenizer with a discrete codebook for circuit tokenization. 
Built upon our Circuit AutoEncoder framework~\cite{hou2022graphmae,li2023maskgae,wu2025mgvga}, CircuitVQ is trained through a circuit reconstruction task. 
Specifically, the CircuitVQ encoder encodes input circuits into discrete tokens using a learnable codebook, while the decoder reconstructs the circuit adjacency matrix based on these tokens.
Subsequently, the CircuitVQ encoder serves as a circuit tokenizer for CircuitAR pretraining, which employs a masked autoregressive modeling paradigm~\cite{chang2022maskgit, li2023mage}. 
In this process, the discrete codes function as supervision signals. 
After training, CircuitAR can generate discrete tokens progressively, which can be decoded into initial circuit structures by the decoder of the CircuitVQ. 
These prior insights can guide DAS in producing refined circuits that match the target truth tables precisely.

Our key contributions can be summarized as follows:
\begin{itemize}
\item We introduce CircuitVQ, a circuit tokenizer that facilitates graph autoregressive modeling for circuit generation, based on our Circuit AutoEncoder framework;
\item Develop CircuitAR, a model trained using masked autoregressive modeling, which generates initial circuit structures conditioned on given truth tables;
\item Propose a refinement framework that integrates differentiable architecture search to produce functionally equivalent circuits guided by target truth tables;
\item Comprehensive experiments demonstrating the scalability and capability emergence of our CircuitAR and the superior performance of the proposed circuit generation approach.
\end{itemize}

% Motivation
% (a) Diffusion (Vision, Graph), Autoregressive (Language, Vision)
% (b) Circuit Generation for Predefined Setting
% (c) Neural Architecture Search for Strict Logic Equivalence

% Contribution
% (a) Circuit Tokenizer (new transformer arch, training strategy)
% (b) CircuitAR (train and gen strategies, post-ar strategy)
% (c) Extensive Evaluation including BitD (Bit Distance) for Scalability

\section{Related Work}
% \subsection{Vision Language Model}
% 시각장애인에서 상황을 설명할 DB가 없으니 만들었다. 그리고 이를 VLM에 튜닝했다.
\subsection{Technical approaches for assisting the visually-impaired}


\subsection{Datasets for visual instruction tuning}

% \begin{figure}
%     \centering
%     \includegraphics[width=0.5\linewidth]{Move_teaser.pdf}
%     \caption{Comparison of different dynamic compute approaches. length of arrow indicates residual transformation per token while width indicates velocity of transformation.}
%     \label{fig:enter-label}
% \end{figure}

\section{Method}
\label{sec:method}
Residual connections play a crucial role in shaping token representations, yet their dynamics remain underexplored in the context of efficient decoding. In this work, we delve deeper into transformer residual dynamics and investigate how modulating residual transformation velocity can improve inference efficiency in token-level processing, optimizing both dense and sparse MoE transformers.


\subsection{Residual Dynamics and Motivation for Multi-rate Residuals} \label{sec:motivation}

To analyze how hidden representations evolve across different layers of a transformer architecture, it's crucial to consider the effect of residual connections. Each transformer decoder layer typically has residual connections across attention and MLP submodules. As the residual stream $h_i$ traverses from interval $E_j$ to $E_{j+1}$, it undergoes a residual transformation given by:  
% \begin{equation}
% \label{eq:slow_residual_transformation}
% H_{E_{j+1}} = H_{E_j} \prod_{i=E_j}^{E_{j+1}} \left( I + \mathcal{A}_i \right) \left( I + \mathcal{M}_i \right) \quad \text{where} \quad \mathcal{A}_i = f(c_i, h_{i}), \mathcal{M}_i = g(h_i)
% \end{equation}

\begin{equation} \label{eq:slow_residual_transformation}
h_{E_{j+1}} = h_{E_j} + \sum_{i=E_j}^{E_{j+1}-1} \left( \mathcal{A}_i(h_i) + \mathcal{M}_i(h_i + \mathcal{A}_i(h_i)) \right) \quad \text{where} \quad \mathcal{A}_i = f(c_i, h_{i}), \mathcal{M}_i = g(h_i). 
\end{equation}

Here, \( \mathcal{A}_i \) denotes the non-linear transformation introduced by the multi-head attention mechanism at layer \( i \), while \( \mathcal{M}_i \) corresponds to the non-linear transformation of the MLP block at the same layer. These transformations depend on the input residual stream \( h_i \) and, in the case of \( \mathcal{A}_i \), the previous contextual representation \( c_i \).\footnote{Normalization layers are typically applied in practice but are omitted here for simplicity of the argument.}


% For easy tokens, the magnitude and direction of this delta transformation become progressively smaller with each successive layer as shown in \cref{fig:delta_transformation}. Consequently, it is feasible to predict these tokens after only a few residual connections, whereas harder tokens necessitate more extensive processing through additional layers.

\begin{figure}[ht]
    \centering
    \begin{subfigure}{0.48\textwidth}
        \centering
        \includegraphics[width=\textwidth]{sections/figures/residual_change.pdf}
        \caption{}
        \label{fig:residual_change}
    \end{subfigure}%
    \hfill
    \begin{subfigure}{0.48\textwidth}
        \centering
        \includegraphics[width=\textwidth]{sections/figures/alignment_wrt_dedicated_model.pdf}
        \caption{}
    \label{fig:alignment_wrt_dedicated_model}
    \end{subfigure}
    \caption{(a) As residual streams propagate through the model, the directional shifts in the residuals become progressively smaller. (b) A dedicated model with $k$ layers achieves a faster rate of change in residual streams and higher alignment than base model leveraging early exit mechanisms at layer $k$.}
    \label{fig}
\end{figure}


To examine whether residual transformations can be accelerated across layers, we conducted experiments using a diverse set of prompts on a pre-trained Phi3 model~\cite{phi3_report}. As illustrated in \cref{fig:residual_change}, we measured the directional shift in residual states as \( 1 - \mathcal{C}(h_{i-1}, h_i) \), where \(\mathcal{C}\) denotes normalized cosine similarity. This shift is notably higher in the initial layers, gradually decreasing in subsequent layers. This behavior allows traditional early exit approaches to effectively accelerate decoding by enabling earlier exits for simpler tokens. However, these approaches typically rely on a distance-based approximation, where the full residual transformation of the model is approximated by the residual transformations of the initial layers. To gain deeper insights into the distance versus velocity aspects of residual transformation, we conducted a comparative study. Specifically, we trained an early exit head at layer $k$ of the Phi3 model, which consists of 32 layers, restricting the distance traveled by each token. To accelerate the residual transformation relative to number of layers, we trained a smaller model consisting of only $k$ layers, while keeping all other hyperparameters consistent. We then compared the next-token prediction accuracy of the early exit head of the base model with that of the smaller model. To ensure an equal number of trainable parameters, we inserted low-rank adapters into the smaller model and trained only these adapters, whereas, in the distance-based approach, we trained solely the early exit head. In addition, to accelerate the residual transformation in smaller model, we distilled the residual streams from the larger model by incorporating a distillation loss ~\cite{sanh2019distilbert} between the residual state at layer \(i\) of the smaller model and the residual state at layer \(4 \times i\) of the larger model. As shown in ~\cref{fig:alignment_wrt_dedicated_model} the smaller model demonstrates a significantly faster rate of change in residual streams, leading to higher next token prediction accuracy after $k$ layers compared to the base model that employs traditional early exit mechanisms after $k$ layers \cite{schuster2022confident, chen2023eellm, varshney-etal-2024-investigating}. This experimental setup, which modifies only the rate of change in residual streams while keeping other factors constant, suggests that dense transformers, trained with a fixed number of layers, may inherently possess a slow residual transformation bias.

This observation raises an intriguing question: if the rate of change in residual streams could be accelerated relative to the number of layers, is it possible to facilitate earlier alignment for a greater proportion of tokens? Earlier alignment would be beneficial to not only facilitate dynamic computation but also for generating speculative tokens efficiently with high acceptance rates in speculative decoding setups ~\cite{leviathan2023fast, chen2023accelerating}. 

%thereby enhancing the efficiency of early exiting? 
 % This bias likely constrains the effectiveness of early exiting, particularly for easier tokens. By addressing this limitation through accelerated residual transformations, we hypothesize that it is possible to substantially improve the efficiency and accuracy of early exit strategies in transformer models.

\subsection{Multi-Rate Residual Transformation} \label{m2r2_method}

To address the slow residual transformation bias described in ~\cref{sec:motivation}, we introduce \textit{accelerated residual streams} that operate at rate $R$ relative to original slow residual stream. We pair slow residual stream, $h$ with an accelerated residual stream, $p$, which has an intrinsic bias towards earlier alignment. Relative to ~\cref{eq:slow_residual_transformation}, accelerated residual transformation from interval $E_j$ to $E_{j+1}$ can be represented as: 

% \begin{equation}
% \label{eq:fast_residual_transformation}
% P_{E_{j+1}} = P_{E_j} \prod_{i=E_j}^{E_{j+1}} \left( I + \hat{\mathcal{A}_i} \right) \left( I + \hat{\mathcal{M}_i} \right) \quad \text{where} \quad \hat{\mathcal{A}_i} = \hat{f}(c_i, P_{i}), \hat{\mathcal{M}_i} = \hat{g}(P_{i})
% \end{equation}


\begin{equation} \label{eq:fast_residual_transformation}
p_{E_{j+1}} = p_{E_j} + \sum_{i=E_j}^{E_{j+1}-1} \left( \hat{\mathcal{A}_i}(p_i) + \hat{\mathcal{M}_i}(p_i + \hat{\mathcal{A}_i}(p_i)) \right) \quad \text{where} \quad \hat{\mathcal{A}_i} = \hat{f}(c_i, p_{i}), \hat{\mathcal{M}_i} = \hat{g}(h_i), 
\end{equation}



where $\hat{\mathcal{A}_i}$ and $\hat{\mathcal{M}_i}$ denote non-linear transformation added by layer $i$ to previous accelerated residual $p_{i}$. Similar to $\mathcal{A}_i$, non-linear transformation $\hat{\mathcal{A}_i}$ attends to same context $c_i$ but uses a different transformation $\hat{f}$ for accelerating $p_{E_j}$ relative to $h_{E_j}$. 

We integrate accelerated residual transformation directly into the base network using parallel accelerator adapters such that rank of accelerator adapters $R_p << d$ where $d$ denotes base model hidden dimension. This setup allows the slow residual stream $h_{E_j}$ to pass through the base model layers while the accelerated residual stream $p_{E_j}$ utilizes these parallel adapters as shown in ~\cref{fig:m2r2_main}. Both slow and accelerated residuals are processed in same forward pass via attention masking and incur negligible additional inference latency in memory bound decoding setups, while in compute bound decoding setups where FLOPs optimization is essential, accelerated residual stream utilizes a fraction of attention heads that of slow residual (see ~\cref{sec:flops_optimization}). Additionally, to maximize the utility of accelerated residual transformations without introducing dedicated KV caches, we propose a shared caching mechanism between the slow and accelerated streams which minimally impact alignment benefits of our approach while offering substantial memory savings (see ~\cref{fig:koala_alignment}). Specifically, the attention operation on the slow residuals \( \text{MHA}(h_t, h_{\leq t}, h_{\leq t}) \) is redefined for accelerated residuals as 
\[
\hat{\mathcal{A}} = MHA(p_t, h_{<t} \oplus p_t, h_{<t} \oplus p_t),
\]
where the accelerated residual at time-step $t$, \( p_t \) attends to the slow residual’s KV cache, facilitating the reuse of contextual information across both residual streams without incurring additional caching costs. Here, \(MHA(q, k, v) \) represents multi-head attention between query \( q \), key \( k \), and value \( v \).

\begin{figure}
    \centering
    \includegraphics[width=0.8\linewidth]{sections//figures/m2r2_main2.pdf}
    \caption{Multi-rate Residuals Framework: Slow residual stream of base model is accompanied by a faster stream that operates at a $2-(J+1)\times$ rate relative to the slow stream, undergoing transformations via accelerator adapters as detailed in \cref{m2r2_method}, where J denotes number of early exit intervals. Colors within the slow and fast residual streams indicate similarity, with matching colors representing the most closely aligned residual states. At the beginning of the forward pass and at each exit point, the accelerated residual state is initialized from the corresponding slow residual state to avoid gradient conflict during training (see ~\cref{sec:grad_conflict}). Early exiting decisions are informed by the Accelerated Residual Latent Attention (ARLA) mechanism, described in \cref{method_arla}, which evaluates residual dynamics across consecutive exit gates.}
    \label{fig:m2r2_main}
\end{figure}

% Furthermore. to maximize the benefits of fast residual transformations without using dedicated KV caches, we propose sharing the fast network’s cache with the slow network. Formally speaking, We modify attention operation on slow residuals $MHA(H_t, H_{<=t}, H_{<=t})$ as $MHA(P_{t}, H_{<t} \oplus P_t, H_{<t}  \oplus P_t)$ such that accelerated residuals attend to previous slow context KV cache, where $MHA(q,k,v)$ denotes multi head attention between query, $q$, key $k$ and value $v$.


\subsection{Enhanced Early Residual Alignment}
Early residual alignment is instrumental in optimizing early exiting, speculative decoding, and Mixture-of-Experts (MoE) inference mechanisms. In this section, we provide a detailed analysis of how accelerated residuals enhance these inference setups.

% By aligning the residual states of intermediate layers with the final output representations, the model can maintain high prediction accuracy even when computations are truncated at earlier layers. This enables more reliable early exiting, reducing the overall computational cost while preserving performance. Additionally, in speculative decoding, early residual alignment allows the model to make confident predictions using faster, partial computations, thereby accelerating inference without sacrificing output quality.


\subsubsection{Early Exiting} \label{method_early_exiting}

A prevalent strategy for enabling early exiting at an intermediate layer $E_{j}$ involves approximating the residual transformation between $E_{j}$ and the final layer $N-1$ using a linear, context independent mapping, $\mathcal{T}$, such that $H_{N-1} \approx \mathcal{T}(H_{E_{j}})$. This approximation has been extensively employed in conventional approaches ~\cite{schuster2022confident, chen2023eellm, varshney-etal-2024-investigating}, providing a computationally efficient means to project the output of deeper layers from intermediate states. Specifically, residual state of layer $N-1$ with this approximation can be expressed as:


% \begin{equation}
% \label{eq: vanila_ea_assumption}
% \Phi(H_{E_{j}}) \sim H_{E_{j}} \prod_{i=E_{j}}^{N}\left( I + \mathcal{A}_i \right) \left( I + \mathcal{M}_i \right) \quad \text{where} \quad \Phi \perp C
% \end{equation}

\begin{equation} \label{eq:early_exiting}
h_{E_j} + \sum_{i=E_j}^{N-1} \left( \mathcal{A}_i(h_i) + \mathcal{M}_i(h_i + \mathcal{A}_i(h_i)) \right) \sim \mathcal{T}(h_{E_{j}})  \quad \text{where} \quad \mathcal{T} \perp c. 
\end{equation}


Here, $\mathcal{A}_i$ and $\mathcal{M}_i$ represent the residual contributions of the multi-head attention and MLP layers, respectively, while $\mathcal{T}$ remains independent of $c$, the preceding context.

This approach is inherently limited by two major factors: first, the assumption of linearity between $h_{E_{j}}$ and $h_{N-1}$ may not hold uniformly for all tokens, particularly when $E_j \ll N$. Second, the linear transformation $\mathcal{T}$ disregards the influence of the context $c$ and fails to account for the latent representations of previous contextual states. In contrast, M2R2 accelerated residual states mitigate both of these challenges by approximating the slow residual transformation of all layers via a faster residual transformation of fewer layers as:
% \begin{equation}
% H_{E_j} \prod_{i=E_j}^{N}\left( I + \mathcal{A}_i \right) \left( I + \mathcal{M}_i \right) \sim P_{E_j} \prod_{i=E_j}^{E_j+1}\left( I + \hat{\mathcal{A}_i} \right) \left( I + \hat{\mathcal{M}_i} \right)
% \end{equation}


\begin{equation} \label{eq:m2r2_approximating_ea}
h_{E_j} + \sum_{i=E_j}^{N-1} \left( \mathcal{A}_i(h_i) + \mathcal{M}_i(h_i + \mathcal{A}_i(h_i)) \right) \sim p_{E_j} + \sum_{i=E_j}^{E_{j+1}-1} \left( \hat{\mathcal{A}_i}(p_i) + \hat{\mathcal{M}_i}(p_i + \hat{\mathcal{A}_i}(p_i)) \right), 
\end{equation}

% \begin{equation} \label{eq:fast_residual_transformation}
% p_{E_{j+1}} = p_{E_j} + \sum_{i=E_j}^{E_{j+1}-1} \left( \hat{\mathcal{A}_i}(p_i) + \hat{\mathcal{M}_i}(p_i + \hat{\mathcal{A}_i}(p_i)) \right) \quad \text{where} \quad \hat{\mathcal{A}_i} = \hat{f}(c_i, p_{i}), \hat{\mathcal{M}_i} = \hat{g}(h_i) 
% \end{equation}






where $p_{E_j}$ is initialized from the slow residual state $h_{E_j}$ at each early exit interval $E_j$ using an identity transformation (see ~\cref{fig:m2r2_main}). As shown in ~\cref{fig:m2r2_residual_sim}, accelerated residuals offer a smoother, more consistent shift in residual direction across layers, in contrast to the abrupt changes typically seen at early exit points in standard early exit methods. Moreover, the normalized cosine similarity between accelerated states at early exit intervals and final residual states is substantially higher compared to traditional early exit techniques, highlighting improved alignment with final layer representations. Traditional adaptive compute methods are constrained by two principal factors: the number of tokens eligible for early exit at intermediate layers and the precision of early exit decision. If residual streams fail to saturate early, the majority of tokens remain ineligible for exit, thereby diminishing potential speedups. Additionally, imprecise delineations between tokens suitable for early exit can lead to underthinking (premature exits that adversely affect accuracy) or overthinking (unnecessary processing that compromises efficiency) ~\cite{zhou2020self, dai2020dynamic}. Enhanced early alignment using ~\cref{eq:m2r2_approximating_ea} helps to address  first issue. To address the second issue we introduce Accelerated Residual Latent Attention, which dynamically assesses the saturation of the residual stream, allowing for a more precise differentiation between tokens that can exit early and those requiring further processing.

% This results in uniform change in residual direction    
% % We keep $\mathcal{A} = \hat{\mathcal{A}}$, while $\hat{\mathcal{M}}$ is accelerated by a factor of $2 - (N_{E}+1)X$ relative to the slower residual transformation $\mathcal{M}$, where $N_E$ represents number of early exiting intervals.
% Figure~\cref{fig:rate_change_comparison} illustrates the comparative rate of change between these transformation streams.



% fig:rate_change_comparison
% - grid plot x axis -> layer id (0, 8) , y axis -> layer id -> dark color cell for max similarity , lighter for lower 
% 
-------------------------------------------------------
Let's consider residual stream $h_i$ traverses through interval $E_j$ to $E_{j+1}$ and undergoes residual transformation given by 
\begin{equation}
h_{E_{j+1}} = h_{E_j} \prod_{i=E_j}^{E_{j+1}} \left( 1 + \delta_i \right)    
\end{equation}

where $\delta_i$ denotes non-linear transformation added by layer $i$. Each non-linear transformation of layer $i$ is a function of previous contextual representation, $c_i$ and input residual stream $h_i-1$ as
$\delta_i = f(c_i, h_{i-1})$ 

One way to exit early at exit $E_j+1$ is to assume that residual transformation from $E_j+1$ to final layer $N-1$ can be approximated by a linear function $\phi$ as $h_{N-1} \sim \Phi(h_{E_j+1})$ and most conventional approaches such as \todo{cite EA papers} use this approach. In other words, 

\begin{equation}
\Phi(h_{E_j+1} \sim h_{E_j+1} \prod_{i=E_j+1}^{N} \left( 1 + \delta_i \right)   
\end{equation}

This approach suffers from two primary issues, linearity assumption from $h_E_j+1$ to $H_N-1$ if often incorrect, particularly when $E_j << N$. More importantly, linear transformation $\Phi$ doesn't consider effect of context $C_i$. M2R2  effectively addresses these issues as accelerated residual stream at interval $E_j+1$ can be represented as 

\begin{equation}
r_{E_{j+1}} = r_{E_j} \prod_{i=E_j}^{E_{j+1}} \left( 1 + \gamma_i \right)    
\end{equation}

where $\gamma_i$ denotes non-linear transformation added by layer $i$ to previous accelerated residual $r_i-1$. Similar to $\delta_i$, non-linear transformation $\gamma_i$ considers context $C_i$ as 
$\gamma_i = g(c_i, r_{i-1})$. So in summary, slow residual transformation is approximated by accelerated residual as: 

\begin{equation}
h_{E_j} \prod_{i=E_j}^{N} \left( 1 + \delta_i \right) \sim h_{E_j} \prod_{i=E_j}^{E_j+1} \left( 1 + \gamma_i \right)
\end{equation}

It's worth noting that accelerated residual $r_i$ and slow residual $h_i$ are processed concurrently at layer $i$ by constructing proper attention mask such as attention of slow residual is represented as 

$MHA(H_it, H_{i<=t}, H_{i<=t}$ while attention of fast residual is computed as 

$MHA(r_it, H_{i<=t}, H_{i<=t}$ where $MHA(q,k,v$ denotes multi head attention between query, $q$, key $k$ and value $v$.


------------------------------------------------------------------

Vertical latent attention on accelerated residual is computed as 
$MHA(S_mt, S(Ej<=i<=m)t, S(Ej<=i<=m)t)$ where $Smt$ denotes query/key/value projection in latent domain at layer $m$ at time $t$. 
------------------------------------------------------------------

Gradient conflict Avoidance: 

Let's consider $w_j$ is a trainable parameter that belongs to a layer between $E_j$ and $E_j+1$. Consider early exit loss at gate $E_j+1$, $L_j+1$, gradient propagation of $w_j$ at another trainable parameter $w_j-n$ can be gives as 

$\sum_{k=E_j-n}^{E_j} \beta_k \frac{\partial L_{E_k}}{\partial w_k}$

where $\beta_j$ denotes backward transformation coefficient for weight $w_j$ to reach gate $E_j$. 
 
On the other hand, gradient propagation in proposed approach can be represented as 

\[
\frac{\partial L_{E_j}}{\partial w_j} = 
\begin{cases} 
\beta_j \frac{\partial L_{E_j}}{\partial w_j} & \text{if } E_j \leq w_j \leq E_{j+1} \\
0 & \text{otherwise}
\end{cases}
\]







% \begin{figure}[ht]
%     \centering
%     \includegraphics[width=0.8\textwidth, height=5cm]{rate_change_comparison.png}
%     \caption{Rate of change comparison between fast and slow residual streams.}
%     \label{fig:rate_change_comparison}
% \end{figure}

%vary k and and plot EA accuracy for larger and smaller models. 

% \begin{figure}[ht]
%     \centering
%     \includegraphics[width=0.5\textwidth,height=5cm]{sections/figures/alignment_comparison_dialogsum.pdf}
%     \caption{Alignment of exited tokens for different early exit layers using traditional early exiting heads, dedicated faster networks, and faster residuals.}
%     \label{fig:small_model_early_exiting}
% \end{figure}


\textbf{Accelerated Residual Latent Attention} \label{method_arla}

In the context of residual streams, we observe that the decision to exit at a given layer can be more effectively informed by analyzing the dynamics of residual stream transformations, instead of solely relying on a classification head applied at the early exit interval $E_j$. To capture the subtle dynamics of residual acceleration, we propose a \textit{Accelerated Residual Latent Attention} (ARLA) mechanism. This approach involves making the exit decision at gate $E_j$ by attending to the residuals spanning from gate $E_{j-1}$ to $E_j$, rather than considering only the residual at gate $E_j$. To minimize the computational overhead associated with exit decision-making, the attention mechanism operates within the latent domain as depicted in ~\cref{fig:arla_arch}. Formally, for each interval $[E_j, E_{j+1}]$, the accelerated residuals are projected into Query ($Q^s_{E_j}, \ldots, Q^s_{E_{j+1}}$), Key ($K^s_{E_j}, \ldots, K^s_{E_{j+1}}$), and Value ($V^s_{E_j}, \ldots, V^s_{E_{j+1}}$) vectors, with latent dimension $d^s$ for $Q^s$, $K^s$, and $V^s$ being significantly smaller than hidden dimension of $p$.\footnote{We use $d^s = 64$ for experiments described in ~\cref{sec:experiments}.} Notably, when the router is allowed to make exit decisions at gate $E_j$ based on residual change dynamics, we observe that the attention is not confined to the residual state at $E_j$ but is distributed across residual states from $E_{j-1}$ to $E_j$, %as illustrated in Figure~\ref{fig:vertical_latent_attention_dynamics}. 
This broader focus on residual dynamics significantly reduces decision ambiguity in early exits, as demonstrated in Figure~\ref{fig:roc_arla}, which contrasts routers based on the last hidden state, and the proposed ARLA router.

%show R -> S transformation. 
%show parameter and flop overhead as compared to adapter on last hidden state.

% \begin{figure}[ht]
%     \centering
%     \includegraphics[width=0.5\textwidth,height=5cm]{sections/figures/roc_arla.pdf}
%     \caption{ROC curves of early exit decision strategies: confidence-based methods (CALM/LITE), routers based on the accelerated hidden state, and latent attention routers.}
%     \label{fig:decision_making_comparison}
% \end{figure}

% \begin{figure}[ht]
%     \centering
%     \includegraphics[width=0.5\textwidth,height=5cm]{vertical_latent_attention.png}
%     \caption{Vertical latent attention mechanism for optimizing early exit decisions by considering residuals from gate \(M\) through \(M-1\).}
%     \label{fig:vertical_latent_attention}
% \end{figure}

\begin{figure}[ht]
    \centering
    \begin{subfigure}{0.52\textwidth}
        \centering
        \includegraphics[width=\textwidth, height = 4cm]{sections/figures/arla_arch.pdf}
        \caption{Accelerated Residual Latent Attention (ARLA): Accelerated residuals between early exit gates are projected into latent domain and attention over residual states within the interval is computed to capture residual dynamics and exit decision is made based on residual saturation.}
        \label{fig:arla_arch}
    \end{subfigure}%
    \hfill
    \begin{subfigure}{0.45\textwidth}
        \centering
        \includegraphics[width=\textwidth, height = 4.5cm]{sections/figures/vla_roc.pdf}
        \caption{ROC classification curves of early exit decision strategies using a linear router used on last residual state ~\cite{schuster2022confident, varshney-etal-2024-investigating, chen2023eellm}  and using ARLA approach that considers residual dynamics. }
        \label{fig:roc_arla}
    \end{subfigure}
    \caption{Effectiveness of ARLA in capturing residual dynamics for early exiting decisions.}


\end{figure}



% \begin{figure}[ht]
%     \centering
%     \includegraphics[width=1\textwidth,height=5cm]{sections/figures/arla.pdf}
%     \caption{fig that plots 32 rows 2 cols heatmap showing attention at each gate}
%     \label{fig:vertical_latent_attention_dynamics}
% \end{figure}

\subsubsection{Self Speculative Decoding} \label{method_self_speculative_decoding}

An alternative means to exploit the early alignment properties of our approach is through the use of accelerated residual states for speculative token sampling to accelerate autoregressive decoding. Speculative decoding aims to speed up memory-bound transformer inference by employing a lightweight draft model to predict candidate tokens, while verifying speculated tokens in parallel and advancing token generation by more than one token per full model invocation \cite{leviathan2023fast, chen2023accelerating, xia2023speculative, miao2023specinfer}. Despite its effectiveness in accelerating large language models (LLMs), speculative decoding introduces substantial complexity in both deployment and training. A separate draft model must be specifically trained and aligned with the target model for each application, which increases the training load and operational complexity ~\cite{chen2023accelerating}. Additionally, this approach is resource-inefficient, as it requires both the draft and target models to be simultaneously maintained in memory during inference \cite{leviathan2023fast, chen2023accelerating}. 

One strategy to address this inefficiency is to leverage the initial layers of the target model itself to generate speculative candidates, as depicted in ~\cite{Tang2024}. While this method reduces the autoregressive overhead associated with speculation, it suffers from suboptimal acceptance rates. This occurs because the linear transformation employed for translating hidden states from layer $k$ to the final layer $N$ is typically a poor approximation, as discussed in ~\cref{sec:motivation} and ~\cref{method_early_exiting}. Our approach resolves this limitation by utilizing accelerated residuals, which demonstrate higher fidelity to their slower counterparts. By utilizing accelerated residuals operating at a rate of $N/k$, where $k$ denotes the number of layers used for candidate speculation, we are able to efficiently generate speculative tokens for decoding.\footnote{We typically set $k = 4$ to balance the trade-off between autoregressive drafting overhead and acceptance rate, as discussed in~\cref{sec:experiments}.}
 This technique not only obviates the need for multiple models during inference but also improves the overall efficiency and effectiveness of speculative decoding.

\begin{figure}
    \centering    \includegraphics[width=1\linewidth]{sections/figures/m2r2_aot_loading.pdf}
    \caption{Ahead-of-Time Expert Loading: M2R2 accelerated residual stream predicts experts required for future layers, reducing reliance on on-demand lazy loading. Speculative pre-loading is efficiently overlapped with computation of multi-head attention (MHA) and MLP transformations. Only incorrectly speculated experts are loaded lazily, resulting in faster inference steps and improved computational efficiency. Here, H indicates LBM Host while D indicates HBM Device.}
    \label{fig:moe_expert_aot_loading}
\end{figure}


\subsubsection{Ahead of Time Expert Loading:} \label{method_aot_expert_loading}

Recent advancements in sparse Mixture-of-Experts (MoE) architectures ~\cite{shazeer2017outrageously, fedus2022switch, artetxe2019massively, lepikhin2020gshard, zoph2022designing} have introduced a paradigm shift in token generation by dynamically activating only a subset of experts per input, achieving superior efficiency in comparison to dense models, particularly under memory-bound constraints of autoregressive decoding \cite{fedus2022switch, zoph2022designing}. This sparse activation approach enables MoE-based language models to generate tokens more swiftly, leveraging the efficiency of selective expert usage and avoiding the overhead of full dense layer invocation. In dense transformer models, pre-loading layers is a common strategy to enhance throughput, as computations of current layer can be overlapped with pre-loading of next layer parameters ~\cite{narayanan2021efficient, shoeybi2020megatron}. However, MoE models face a unique challenge: expert selection occurs dynamically based on previous layer’s output, making it infeasible to preload next layer’s experts in parallel. This limitation results in inherent latency, as expert loading becomes a sequential, on-demand process ~\cite{lepikhin2020gshard, fedus2022switch}.

To address this inefficiency, our method introduces a mechanism with \textit{accelerated residuals}, which not only captures key characteristics of base slower residual states but also exhibit high cosine similarity with their final counterparts (as illustrated in \cref{fig:m2r2_residual_sim}). By employing accelerated residual streams, we can effectively predict the necessary experts for future layers well in advance of their actual invocation. Specifically, using a $2\times$ accelerated residual, the experts needed for layers $2i+2$ and $2i+3$ can be identified while still computing in layer $i$, thus overcoming the bottleneck of sequential, on-demand expert selection and mitigating latency in the decoding pipeline, as shown in \cref{fig:moe_expert_aot_loading}. Note that, we use fixed set of accelerator adapters for transforming accelerated residuals (as discussed in ~\cref{m2r2_method}) while slow residual is transformed via expert routing mechanism. 

Furthermore, our approach integrates a Least Recently Used (LRU) caching strategy, which enhances memory efficiency by replacing the least recently used experts with speculated experts that are anticipated to be needed in upcoming layers. This hybrid approach of preemptive expert loading with LRU caching yields substantial improvements over traditional on-demand loading or standalone caching strategies. By minimizing cache misses and efficiently managing memory, this approach addresses both compute and memory bottlenecks, leading to faster, more resource-efficient token generation in MoE architectures. A comprehensive evaluation of this strategy, in relation to state-of-the-art methods, is provided in \cref{experiments_aot}, and the compute and memory traces on an A100 GPU are detailed in \cref{fig:moe_aot_cuda_trace}.



% Recent advancements in sparse Mixture-of-Experts (MoE) architectures have introduced the concept of utilizing distinct computational paths for different tokens \cite{shazeer2017outrageously}. This approach, wherein only a subset of experts are activated per input, enables MoE-based language models to generate tokens more swiftly compared to their dense counterparts due to memory-bound nature of auto-regressive decoding. In dense models, pre-loading layers in advance is a common strategy to enhance computational efficiency. However, this technique is not applicable to MoE models, where expert selection occurs dynamically based on the outputs of previous layers, preventing parallel pre-fetching of experts.

% Our proposed method addresses this inefficiency. Accelerated residuals, which are highly similar to their slower counterparts (see \cref{fig:similarity}), can reliably predict the necessary experts ahead of time. For instance, by utilizing $2X$ accelerated residual stream, we can predict the experts needed for the layer $2i+1$ and $2i+3$ while carrying out computation in layer $i$. This enables us to commence expert loading significantly earlier, as illustrated in \cref{expert_loading}, effectively mitigating the delays observed with the naive on-demand expert loading. Additionally, our method benefits from incorporating a Least Recently Used (LRU) strategy, where speculated experts replace those that are least recently utilized, resulting in improved performance compared to using either strategy alone. For a comprehensive evaluation, refer to \cref{moe_trace}, which provides a CUDA compute and memory trace of our approach executed on <>.



% A naive solution involves using the residual state of the previous layer along with the gating function of the next layer to predict which experts need to be loaded, and initiating the expert loading process in parallel with the attention computation of the next layer. Yet, as shown in \cref{fig:MOE_attn_vs_loading_time}, the attention computation for medium to long contexts is considerably faster than the expert loading time, making this approach inefficient.




\subsection{Training} \label{method_training}
% This approach is feasible due to the absence of gradient conflicts, as discussed in \cref{sec:grad_conflict}.

To accelerate residual streams, we employ parallel accelerator adapters as described in \cref{m2r2_method}.  For the early exiting use-case outlined in \cref{method_early_exiting}, we define the training objective for these adapters using the following loss function, which combines cross-entropy loss at each exit $E_j$ with distillation loss at each layer $i$. Loss weights coefficients $\alpha_0$ and $\alpha_1$ are employed to balance contribution of corresponding losses.

\begin{align} \label{eq:mr_loss}
L_{\text{m2r2}} = \underbrace{-\alpha_0 \sum_{j=1}^{J} \sum_{t=1}^{T} \log p_{\theta} \left( \hat{y}_t^{E_j} \mid y_{<t}, x \right)}_{\text{cross-entropy loss}} 
+ \underbrace{\alpha_1\sum_{i=1}^{E_{J-1}} \sum_{t=1}^{T} \| \mathbf{p}_{t}^{i} - \mathbf{h}_{t}^{((i - E_{j(i)}) \cdot R_i) + E_{j(i)})} \|^2}_{\text{distillation loss}}.
\end{align}

where $\hat{y}_t^{E_j}$ denotes the predictions from the accelerated residual stream at layer $E_j$ and time step $t$, $y_t$ represents the corresponding ground truth tokens, and $x$ indicates previous context tokens. The distillation loss at each layer $i$ is computed by comparing accelerated residuals at layer $i$ with slow residuals at layer $(i - E_{j(i)}) \cdot R_i + E_{j(i)}$, where $R_i$ denotes the rate of accelerated residuals at layer $i$ while $E_{j(i)}$ represents the most recent gate layer index such that $E_{j(i)} <= i$. \( J \) represents the total number of early exit gates, N denotes number of hidden layers and $E_j$ denotes layer index corresponding to gate index $j$ and \( T \) denotes the sequence length. 

In dynamic compute settings, after training of accelerator adapters, we optimize the query, key, and value parameters governing the ARLA routers (see ~\cref{method_arla}) across all exits in parallel on binary cross entropy loss between predicted decision and ground truth exiting decision. The ground truth labels for the router are determined based on whether the application of the final logit head on $\hat{y}_t^{E_j}$ yields the correct next-token prediction. 


% The objective for this optimization is defined by the following loss function:


%TODO are equations required ? 
% \begin{equation} \label{eq:arla_loss_combined}\small
%     L_{\text{arla}} = -\frac{1}{N} \sum_{t=1}^{T} \left( \sum_{j=1}^{E_n} \left[ O_t^{E_j} \log(\hat{O}_t^{E_j}) + (1 - O_t^{E_j}) \log(1 - \hat{O}_t^{E_j}) \right] \right), \quad \text{where} \quad 
%     O_t^{E_j} = \begin{cases} 
%     1, & \text{if } L(\hat{y}_t^{E_j}) = y_t^{E_j} \\
%     0, & \text{otherwise}
%     \end{cases}
% \end{equation}

% where $\hat{O}_t^{E_j}$ represents the binary predicted logits produced by the vertical latent attention router, as described in \cref{sec:arla}, at gate $E_j$ and time step $t$, and $O_t^{E_j}$ denotes the corresponding ground truth labels. The ground truth labels for the router are determined based on whether the application of the logit head on $\hat{y}_t^{E_j}$ yields the correct next-token prediction. The parameters controlling vertical latent attention are trained concurrently to ensure consistency and efficient use of computational resources.

For self-speculative decoding, as described in \cref{method_self_speculative_decoding}, the training objective remains the same as \cref{eq:mr_loss}, but with the number of intervals set to $J = 1$ and the rate of residual transformation set to $R_n = N/k$, where the first $k$ layers generate speculative candidate tokens. In the context of Ahead-of-Time Expert Loading for Mixture-of-Experts (MoE) models (see \cref{method_aot_expert_loading}), setting the rate of residual transformation to $R_n = 2$ typically offers a good trade-off between the accuracy of expert speculation and AoT pre-loading of experts. 

% Thus, we set $J = 1$ and $E_1 = 16$.


~\subsection{FLOPs Optimization} \label{sec:flops_optimization}

Naively implemented, M2R2 incurs higher FLOP overhead compared to traditional speculative decoding and early exiting approaches such as ~\cite{medusa, schuster2022confident, Tang2024}. However, modern accelerators demonstrate compute bandwidth that exceeds memory access bandwidth by an order of magnitude or more~\cite{databricksLLMInference2023, jouppi2021ten}, meaning increased FLOPs do not necessarily translate to increased decoding latency. Nevertheless, to ensure fair comparison and efficiency in compute bound scenarios, we introduce targeted optimizations.

~\textbf{Attention FLOPs Optimization} For medium-to-long context lengths, attention computation dominates FLOPs in the self-attention layer, surpassing the contribution from MLP layers. Specifically, matrix multiplications involving queries, cached keys, and cached values scale with $l_{kv} * l_{q}$ where $l_{kv}$ denotes previous context length and $l_q$ denotes current query length. Since M2R2 pairs accelerated residuals with slow residuals, a naive implementation results in twice the FLOPs consumption compared to a standard attention layer. To address this, we limit the attention of accelerated residual stream to selectively attend to the top-k most relevant tokens, identified by the slow residual stream based on top attention coefficients\footnote{We set to k = 64 and attend to top 64 tokens as identified by the slow residual stream.}. This is possible since slow and accelerated residual streams are processed in same forward pass and accelerated streams have access to attention coefficients of slow stream. Note that, the faster residual stream still retains the flexibility to assign distinct attention coefficients to these tokens. Furthermore, we design the faster residual stream to employ only 8 attention heads, compared to the 32 heads used in the slow residual stream of the Phi-3 model, reducing query, key, value, and output projection FLOPs by a factor of 1/4. ~\cref{fig:m2r2_num_heads_ablation} indicates effect of using a slicker stream on alignment. As depicted, using $\hat{n}_h = 8$ offers a good trade-off between alignment and FLOPs overhead. 

~\textbf{MLP FLOPs Optimization} The accelerator adapters operating on the accelerated residual stream are intentionally designed with lower rank than their counterparts in the base model. This reduces FLOP overhead by a factor proportional to $hiddenSize / rank$. Additionally, since the faster residual stream uses only 8 attention heads (compared to 32 in the slow residual stream of Phi-3), the subsequent MLP layers process a smaller set of activations, further reducing FLOPs by another factor of 1/4.

These optimizations significantly reduce the FLOP overhead per speculative draft generation, as illustrated in ~\cref{fig:flops_optmization}. Notably, while traditional early-exiting speculative approaches such as DEED require propagating the full slow residual state through the initial layers, incurring substantial computational costs, M2R2 achieves efficient token generation via slimmer, low-rank faster residual streams. In contrast, Medusa introduces considerable FLOP overhead due to per-head computations scaling with $d^2+dv$\footnote{Here $d$ denotes hidden state dimension while $v$ denotes vocab size.}, whereas M2R2 employs low-rank layers for both MLP and language modeling heads, maintaining computational efficiency. All experiments involving the M2R2 approach, as detailed in ~\cref{sec:experiments}, are conducted using these FLOPs optimizations.









% \[
% O_t^{E_j} = 
% \begin{cases} 
% 1, & \text{if } L(\hat{y}_t^{E_j}) = y_t^{E_j} \\
% 0, & \text{otherwise}
% \end{cases}
% \]




%add distillation
% We train accelerator adapters described in \cref{m2r2_method} to accelerate residual streams on next token prediction all in parallel since there are no gradient conflict issues as described in \cref{sec:grad_conflict}.

% \begin{align} \label{eq:mr_loss}
% L_{mr} =  & -\sum_{j = 1}^{E_n} (\sum_{t=1}^{T}\log p_{\theta} (\hat{y}_t^{E_j} | \hat{y}_{<t}, x)) \nonumber
% \end{align}

% where $\hat{y_t^{E_j}}$ denotes predicted logits obtained from accelerated residual stream at gate $E_j$ and time-step $t$ while $y_t^{E_j}$ denotes corresponding truth tokens. 

% Upon training of adapters responsible for accelerating residual streams, we train query, key, value parameters responsible for vertical latent attention of all gates in parallel as

% \begin{equation} \label{eq:arla_loss}
%     L_{arla} = -\frac{1}{N} (\sum_{t=1}^{T}(1\sum_{j=1}^{E_n} \left[ O_t^{E_j} \log(\hat{O}_t^{E_j}) + (1 - o_t^{E_j}) \log(1 - \hat{o_t}_{E_j}) \right]))
% \end{equation}

% where $\hat{O_t^{E_j}}$ denotes binary predicted logits obtained from vertical latent attention router described in \cref{sec:arla} at gate $E_j$ and timestep $t$ while $O_t^{E_j}$ denotes corresponding truth label. Truth labels for router are obtained by computing whether logit head application on $\hat{y}_t^j$ results in true next token prediction. Formally speaking, 

% $O_t^{E_j} = 1 if L(\hat{y_t^{E_j}}) == y_t^{E_j} , 0 otherwise$. 

% Parameters responsible for vertical latent attention are also trained in parallel as well. 

%todo: training slow and fast residuals together and distillation can be two training mdoes. 
%Distillation can be an ablation. 




% Although transformer decoding is memory bound on most mainstream accelerators, there could be scenarios where flop savings are crucial. For instance, on on-device settings power consumption is directly correlated with flops per decoding step and reducing flops does help with overall energy consumption. Vanilla early exiting methods help with flop reduction but suffer from mismatch between training and inference due to early exited tokens. If token at decoding step $t$, $T_t$ exited at layer $E_i$, while token $T_{t+k}$ exits at layer $E_j$ such that $E_i < E_j$, hidden state $H_{t+k}l$ does not have corresponding hidden state $H_tl$ to attend to where $E_i < l <= E_j$. One solution that's often used in literature is to rely on last hidden state available, $H_t{E_j}$, however it tends to be sub-optimal and does affect generation quality \cite{ref}.  To alleviate this mismatch while reducing flops, we train router such that attention mask between token $T_{t+k}$ and token $T_{<t+k}$ is given by: 

% \begin{equation}
%     a_{T_{{t+k}{T_{<t+k}}} = 1 if  E_{T_{<t+k}} >= E{T_{t+k}}
%     else 0
% \end{equation}

% This attention mask enables router to account for exited tokens and get trained accordingly. Since attention mechanism during decoding remains exactly same as that during training, impact on generation quality tends to be minimal as noted in \cref{fig:gen_auality_with_and_without_recompute_attention_show_flops}.  Although MoD does not suffer from training and inference mismatch, we observe that it suffers from discountinuity between pre-training and super-vised fine-tuning resulting in sub-optimal perplexity. On the other hand, our method doesn't not require pre-training , doesn't suffer from discountinuity, and achieves much better perplexity in super-vised fine-tuning and instruction tuning setups as shown in \cref{fig:Mod_vs_m2r2_loss_curves}.






% Our techniques are directly applicable in such scenarios.    




%expert loading with cuda streams in experiments
\begin{figure*}[t]
\centering
\begin{center} 
  \includegraphics[width=.63\textwidth]{figures/fidelity_legend.pdf}
\end{center}
\begin{subfigure}[b]{.3\textwidth}
  \centering
  \includegraphics[width=\linewidth]{figures/sentiment_fidelity3_small.pdf}
  \caption{\emph{Sentiment} $n\in [32,63]$}
  \label{fig:fidelity_sentiment}
\end{subfigure}%
\begin{subfigure}[b]{.3\textwidth}
  \centering
  \includegraphics[width=\linewidth]{figures/drop_fidelity_small.pdf}
  \caption{\emph{DROP} $n\in [32,63]$}
  \label{fig:fidelity_drop}
\end{subfigure}%
\begin{subfigure}[b]{.3\textwidth}
  \centering
  \includegraphics[width=\linewidth]{figures/hotpot_fidelity_small.pdf}
  \caption{\emph{HotpotQA} $n\in [32,63]$}
  \label{fig:fidelity_hotpot}
\end{subfigure}
\begin{subfigure}[b]{.3\textwidth}
  \centering
  \includegraphics[width=\linewidth]{figures/sentiment_fidelity3_large.pdf}
  \caption{\emph{Sentiment} $n\in [64,127]$}
  \label{fig:fidelity_sentiment2}
\end{subfigure}%
\begin{subfigure}[b]{.3\textwidth}
  \centering
  \includegraphics[width=\linewidth]{figures/drop_fidelity_large.pdf}
  \caption{\emph{DROP} $n\in [64,127]$}
  \label{fig:fidelity_drop2}
\end{subfigure}%
\begin{subfigure}[b]{.3\textwidth}
  \centering
  \includegraphics[width=\linewidth]{figures/hotpot_fidelity_large.pdf}
  \caption{\emph{HotpotQA} $n\in [64,127]$}
  \label{fig:fidelity_hotpot2}
\end{subfigure}
\vspace{-5pt}
\caption{On the removal task, \SpecExp{} performs competitively with 2\textsuperscript{nd} order methods on the \emph{Sentiment} dataset, and out-performs all approaches on \emph{DROP} and  \emph{HotpotQA} dataset for $n \in [32,63]$. When $n$ is too large to compute other interaction indices, we outperform marginal methods.}
\label{fig:fidelity}
\vspace{-12pt}
\end{figure*}

\vspace{-14pt}
\section{Experiments}
\label{sec:language}

\paragraph{Datasets} 
We use three popular datasets that require the LLM to understand interactions between features. 
\begin{enumerate}[ topsep=0pt, itemsep=0pt, leftmargin=*]
\item \emph{Sentiment} is primarily composed of the \emph{Large Movie Review Dataset} \cite{maas-EtAl:2011:ACL-HLT2011}, which contains both positive and negative IMDb movie reviews. The dataset is augmented with examples from the \emph{SST} dataset \cite{ socher2013recursive} to ensure coverage for small $n$. We treat the words of the reviews as the input features.
\item{\emph{HotpotQA} \cite{yang2018hotpotqa} is a question-answering dataset requiring multi-hop reasoning over multiple Wikipedia articles to answer complex questions. We use the sentences of the articles as the input features.}
\item{\emph{Discrete Reasoning Over Paragraphs} (DROP)} \cite{dua2019drop} is a comprehension benchmark requiring discrete reasoning operations like addition, counting, and sorting over paragraph-level content to answer questions. We use the words of the paragraphs as the input features. 
\end{enumerate}
%
%\emph{DROP} and \emph{HotpotQA} require , while \emph{Sentiment} is encoder-only. 
%
\vspace{-7pt}
\paragraph{Models} For \textit{DROP} and \textit{HotpotQA}, (generative question-answering tasks) we use \texttt{Llama-3.2-3B-Instruct} \cite{grattafiori2024llama3herdmodels} with $8$-bit quantization. For \emph{Sentiment} (classification), we use the encoder-only fine-tuned \texttt{DistilBERT} model \cite{Sanh2019DistilBERTAD,sentimentBert}.
%

\vspace{-7pt}
\paragraph{Baselines} We compare against popular marginal metrics LIME, SHAP, and the Banzhaf value. 
%
For interaction indices, we consider Faith-Shapley, Faith-Banzhaf, and the Shapley-Taylor Index. We compute all benchmarks where computationally feasible. That is, we always compute marginal attributions and interaction indices when $n$ is sufficiently small. In figures, we only show the best performing baselines. Results and implementation details for all baselines can be found in 
Appendix~\ref{apdx:experiments}.

\vspace{-6pt}
\paragraph{Hyperparameters} \SpecExp{} has several parameters to determine the number of model inferences (masks). We choose $C=3$, informed by \citet{li2015spright} under a simplified sparse Fourier setting. We fix $t = 5$, which is the error correction capability of \SpecExp{} and serves as an approximate bound on the maximum degree. 
%
We also set $b=8$; the total collected samples are $\approx C2^bt \log(n)$. 
%
For $\ell_1$ regression-based interaction indices, we choose the regularization parameter via $5$-fold cross-validation. 




\vspace{-3pt}
\subsection{Metrics}


We compare \SpecExp{} to other methods across a variety of well-established metrics to assess performance.
%\Efe{How about textbf rather than emph here?}

\textbf{Faithfulness}: To characterize how well the surrogate function $\hat{f}$ approximates the true function, we define \emph{faithfulness} \cite{zhang2023trade}:
\vspace{-3pt}
\begin{equation}
    R^2 = 1 -  \frac{\lVert \hat{f} - f \rVert^2}{\left\lVert f - \bar{f} \right\rVert^2},
\end{equation}
where $\left\lVert f  \right\rVert^2 = \sum_{\bbm \in \bbF_2^n}f(\bbm)^2$ and $\bar{f} = \frac{1}{2^n} \sum_{\bbm \in \bbF_2^n}f(\bbm)$.

Faithfulness measures the ability of different explanation methods to predict model output when masking random inputs. 
%
We measure faithfulness over 10,000 random \emph{test} masks per-sample, and report average $R^2$ across samples. 
%

\textbf{Top-$r$ Removal}: We measure the ability of methods to identify the top $r$ influential features to model output:
\vspace{-2pt}
\begin{align}
\begin{split}
    \mathrm{Rem}(r) = \frac{|f(\boldsymbol{1}) - f(\bbm^*)|}{|f(\boldsymbol{1})|}, \\
    \;\bbm^* = \argmax \limits_{\abs{\bbm} = n-r}|\hat{f}(\boldsymbol{1}) - \hat{f}(\bbm)|.
\end{split}
\end{align}
\vspace{-8pt}


\textbf{Recovery Rate@$r$:} 
%
Each question in \emph{HotpotQA} contains human-labeled annotations for the sentences required to correctly answer the question. 
%
We measure the ability of interaction indices to recover these human-labeled annotations. 
%
Let $S_{r^*} \subseteq [n]$ denote human-annotated sentence indices. %corresponding to the human-annotated sentences containing the answer. 
Let $S_{i}$ denote feature indices of the $i^{\text{th}}$ most important interaction for a given interaction index.
%
Define the recovery ability at $r$ for each method as follows
\vspace{-2pt}
\begin{equation}
\label{eq:recovery_k}
    \text{Recovery@}r = 
    \frac{1}{r}\sum^r_{i=1}\frac{\abs{S_r^*\cap S_i}}{|S_{i}|}.
\end{equation}
\vspace{-8pt}

Intuitively, \eqref{eq:recovery_k} measures how well interaction indices capture features that align with human-labels.   


\begin{figure*}[t]
\centering
\hfill
\begin{subfigure}[b]{.5\textwidth}
  \centering
    \hspace{0.82cm}\includegraphics[width=0.75\textwidth]{figures/recall_legend.pdf}
  \includegraphics[width=.9\linewidth]{figures/hotpot_recall.pdf}
  \caption{Recovery rate$@10$ for \emph{HotpotQA} }
  \label{fig:recovery_hotpot}
\end{subfigure}%
\hfill % To ensure space between the figures
\begin{subfigure}[b]{.46\textwidth}
  \centering
    \includegraphics[width=1\textwidth]{figures/hotpot.pdf}
  \caption{Human-labeled interaction identified by \SpecExp{}.}
  \label{fig:hotpot_additional}
\end{subfigure}
\hfill
\caption{(a) \SpecExp{} recovers more human-labeled features with significantly fewer training masks as compared to other methods. (b) For a long-context example ($n = 128$ sentences), \SpecExp{} identifies the three human-labeled sentences as the most important third order interaction while ignoring unimportant contextual information.}
\vspace{-8pt}
\end{figure*}

\vspace{-8pt}
\subsection{Faithfulness and Runtime}
\vspace{-3pt}

Fig.~\ref{fig:faith} shows the faithfulness of \SpecExp{} compared to other methods. We also plot the runtime of all approaches for the \emph{Sentiment} dataset for different values of $n$. 
%
All attribution methods are learned over a fixed number of training masks.
% 

\textbf{Comparison to Interaction Indices } \SpecExp{} maintains competitive performance with the best-performing interaction indices across datasets. 
%
Recall these indices enumerate \emph{all possible interactions}, whereas \SpecExp{} does not. 
%
This difference is reflected in the runtimes of Fig.~\ref{fig:faith}(a).
%
The runtime of other interaction indices explodes as $n$ increases while \SpecExp{} does not suffer any increase in runtime. 

\vspace{-2pt}
\textbf{Comparison to Marginal Attributions } For input lengths $n$ too large to run interaction indices, \SpecExp{} is significantly more faithful than marginal attribution approaches across all three datasets.

\vspace{-2pt}
\textbf{Varying number of training masks } Results in Appendix ~\ref{apdx:experiments} show that \SpecExp{} continues to out-perform other approaches as we vary the number of training masks. 

\vspace{-2pt}
\textbf{Sparsity of \SpecExp{} Surrogate Function} Results in Appendix ~\ref{apdx:experiments}, Table~\ref{tab:faith} show 
surrogate functions learned by \SpecExp{} have Fourier representations where only $\sim 10^{-100}$ percent of coefficients are non-zero. 


\vspace{-6pt}
\subsection{Removal}
\label{subsec:removal}

Fig.~\ref{fig:fidelity} plots the change in model output as we mask the top $r$ features for different regimes of $n$. 
%

\vspace{-2pt}
\textbf{Small $n$ } \SpecExp{} is competitive with other interaction indices for \textit{Sentiment}, and out-performs them for \textit{HotpotQA} and \textit{DROP}. 
%
Performance of \SpecExp{} in this task is particularly notable since Shapley-based methods are designed to identify a small set of influential features. 
%
On the other hand, \SpecExp{} does not optimize for this metric, but instead learns the function $f(\cdot)$ over all possible $2^n$ masks. 
%

\textbf{Large $n$ } \SpecExp{} out-performs all marginal approaches, indicating the utility of considering interactions.
%

\vspace{-10pt}
\subsection{Recovery Rate of Human-Labeled Interactions}

%
We compare the recovery rate \eqref{eq:recovery_k} for $r = 10$ of \SpecExp{} against third order Faith-Banzhaf and Faith-Shap interaction indices. 
%
We choose third order interaction indices because all examples 
are answerable with information from at most three sentences, i.e., maximum degree $d = 3$.
%
Recovery rate is measured as we vary the number of training masks. 

Results are shown in Fig.~\ref{fig:recovery_hotpot}, where \SpecExp{} has the highest recovery rate of all interaction indices across all sample sizes. 
%
Further, \SpecExp{} achieves close to its maximum performance with few samples, other approaches require many more samples to approach the recovery rate of \SpecExp{}. 

\textbf{Example of Learned Interaction by \SpecExp{}} Fig.~\ref{fig:hotpot_additional} displays a long-context example (128 sentences) from \emph{HotpotQA} whose answer is contained in the three highlighted sentences. 
%
\SpecExp{} identifies the three human-labeled sentences as the most important third order interaction while ignoring unimportant contextual information. 
%
Other third order methods are not computable at this length. 
%

\begin{figure*}[t]
    \centering
    \includegraphics[width=0.9\linewidth]{figures/case_studies.pdf}
    \caption{SHAP provides marginal feature attributions. Feature interaction attributions computed by SPEX provide a more comprehensive understanding of (above) words interactions that cause the model to answer incorrectly and (below) interactions between image patches that informed the model's output.}
    \label{fig:caseStudies}
\end{figure*}



\begin{table*}[htbp]
    \centering
    \small
    \begin{tabular}{p{14cm}}
     \toprule
\#\#\#  Objective: \\
Generate a 5-day family travel itinerantry that satisfies all specified requirements while adhering to highly fine-grained constraints. The generated itinerary should balance real-time adaptability, strict hard attributes, and semantic soft attributes. \\

\#\#\# User Profile: \\
 - Travelers: 2 adults + 1 child (age 8) \\
 - Budget: $<=$ \$300/day (total \$1,500 for the trip) \\
 - Activity Balance: 70\% educational/cultural experiences, 20\% relaxation, 10\% family-friendly shopping. \\

\#\#\# Hard Attributes: \\
- Activity Scheduling: \\
\quad- Each activity must have a defined start and end time, ensuring there is no overlap between activities. \\
\quad- A break period from 13:00-14:30 is mandatory daily. \\
\quad- Each activity must fit within a 2-hour window unless otherwise specified. \\

- Budget Requirements: \\
\quad- Each day’s total cost (including transportation, food, and activities) must not exceed \$300. \\
\quad- Transportation is limited to metro and walking only, with a maximum of 3 metro rides per day. \\

- Location Constraints: \\
\quad- Must-visit locations: City Zoo (Day 1) and Science Museum (Day 3). \\
\quad- Activities must occur in geographically adjacent areas to minimize walking distance. \\

- Keyword Requirements: \\
\quad- Each day’s description must include specific keywords. For example: \\
\quad- Day 1: “wildlife,” “exploration,” and “interactive learning.” \\
\quad- Day 3: “science,” “innovation,” and “hands-on exhibits.” \\

- Structure Constraints: \\
\quad- Each day’s itinerary must consist of 4 sections: \\
\quad\quad- Morning activity \\
\quad\quad- Break/lunch period \\ 
\quad\quad- Afternoon activity \\
\quad\quad- Evening summary (limited to 50 words) \\

\#\#\# Soft Attributes \\
- Tone and Emotion: \\ 
\quad- Day 1: Use a tone that conveys “excitement and discovery.” \\ 
\quad- Day 3: Use a tone that conveys “curiosity and wonder.” \\
- Language Style: \\ 
\quad- Use descriptive, vivid, and family-friendly language throughout. \\
\quad- Include at least one metaphor or simile per day (e.g., "The Science Museum felt like stepping into the future!"). \\
- Visual Details: \\
\quad- Each activity must include specific sensory details (e.g., "the bright colors of the parrots at the zoo" or "the tinkling sound of water fountains at the park").

- Adaptive Adjustments (Real-time Constraints): \\
\quad- Weather Sensitivity: \\
\quad\quad- If the rain forecast exceeds 60\%, replace outdoor activities with indoor alternatives while keeping the overall tone and keywords intact. \\ 
\quad- Physical Endurance: \\
\quad\quad- If a day’s total walking distance exceeds 10 kilometers, the next day’s activities must reduce walking by 30\%. \\
\quad- Health Responsiveness: \\
\quad\quad- If a health-related issue arises (e.g., fatigue or illness), adjust the itinerary dynamically to: \\
\quad\quad- Reduce activity duration to half. \\ 
\quad\quad- Substitute the activity with a more relaxing or passive option. \\
\bottomrule
    \end{tabular}
    \caption{The complete travel planner case study.}
    \label{tab:travel_planner_case}
\end{table*}
\section*{Conclusion}
This paper aims to enhance our understanding of the computational complexity of computing various Shapley value variants. We found that for various ML models --- including decision trees, regression tree ensembles, weighted automata, and linear regression --- both local and global interventional and baseline SHAP can be computed in polynomial time under HMM modeled distributions. This extends popular algorithms, such as TreeSHAP, beyond their empirical distributional scope. We also establish strict complexity gaps between the various SHAP variants (baseline, interventional, and conditional) and prove the intractability of computing SHAP for tree ensembles and neural networks in simplified scenarios. Overall, we present SHAP as a versatile framework whose complexity depends on four key factors: \begin{inparaenum}[(i)] \item model type, \item SHAP variant, \item distribution modeling approach, \item and local vs. global explanations\end{inparaenum}. We believe this perspective provides deeper insight into the computational complexity of SHAP, paving the way for future work.




%We believe that our framework provides a more intricate understanding of SHAP computation complexity across different models, distributions, and variants, paving the way for further research.

Our work opens promising directions for future research. First, expanding our computational analysis to other SHAP-related metrics, such as asymmetric SHAP~\citep{frye20} and SAGE~\citep{covert2020understanding}, would be valuable. Additionally, we aim to explore more expressive distribution classes and relaxed assumptions beyond those in Section \ref{sec:tractable} while maintaining tractable SHAP computation. Finally, when exact computation is intractable (Section \ref{sec:intractable}), investigating the approximability of SHAP metrics through approximation and parameterized complexity theory~\citep{downey2012parameterized} is an important direction.

%Our work opens several promising avenues for future research on the computational properties of explainable AI methods, with a particular focus on SHAP. First, it would be interesting to broaden the computational analysis conducted in this work to include other popular SHAP-related metrics in the literature, such as asymmetric SHAP \cite{frye20} and SAGE \cite{covert2020understanding}. Also, in the future, we aim to explore more expressive distribution classes and relaxed distributional assumptions—extending beyond those examined in Section \ref{sec:tractable} —that still yield tractable SHAP computation. Finally, when exact computation proves intractable (Section \ref{sec:intractable}), it is worthwhile to theoretically investigate the question of the approximability of computing the SHAP metrics across various configurations, through the lens of approximation and parametrized complexity theory \cite{arora2009computational}.

%This paper aims to deepen our understanding of the computational complexity involved in obtaining different Shapley value variants. We found that for a variety of ML models, including decision trees, tree ensembles for regression, weighted automata, and linear regression models — computing both local and global interventional and baseline SHAP can be done in polynomial time when distributions are modeled by HMMs. This extends the distributional scope of popular algorithms like TreeSHAP, which is limited to empirical distributions. Additionally, we demonstrate a strict complexity gap between SHAP variants, showing that interventional and baseline SHAP can be strictly easier to compute than conditional SHAP. Despite these positive results, we uncovered intractability for various SHAP variants in neural networks and tree ensembles. Finally, we provided generalized complexity relations across SHAP variants. We believe that our framework offers a deeper understanding of the complexity involved in computing SHAP across various variants, models, distributions, as well as in both local and global computations, laying the groundwork for future research.

\section*{Impact Statement}
Getting insights into the decisions of deep learning models offers significant advantages, including increased trust in model outputs. By reasoning about the rationale behind a model's decisions with the help of \SpecExp{}, we can develop greater confidence in its output, and use it to aid in our own reasoning. When using analysis tools like \SpecExp{}, it's crucial to avoid over-interpretation of results.
\bibliography{main}
\bibliographystyle{icml2025}


%%%%%%%%%%%%%%%%%%%%%%%%%%%%%%%%%%%%%%%%%%%%%%%%%%%%%%%%%%%%%%%%%%%%%%%%%%%%%%%
%%%%%%%%%%%%%%%%%%%%%%%%%%%%%%%%%%%%%%%%%%%%%%%%%%%%%%%%%%%%%%%%%%%%%%%%%%%%%%%
% APPENDIX
%%%%%%%%%%%%%%%%%%%%%%%%%%%%%%%%%%%%%%%%%%%%%%%%%%%%%%%%%%%%%%%%%%%%%%%%%%%%%%%
%%%%%%%%%%%%%%%%%%%%%%%%%%%%%%%%%%%%%%%%%%%%%%%%%%%%%%%%%%%%%%%%%%%%%%%%%%%%%%%
\newpage
\appendix
\onecolumn
\section{Algorithm Details}\label{apdx:algorithm}
\subsection{Introduction}
This section provides the algorithmic details behind \SpecExp{}. The algorithm is derived from the sparse Fourier (Hadamard) transformation described in \citet{li2015spright}. Many modifications have been made to improve the algorithm and make it suitable for use in this application. Application of the original algorithm proposed in \citet{li2015spright} fails for all problems we consider in this paper. In this work, we focus on applications of \SpecExp{} and defer theoretical analysis to future work. 

\paragraph{Relevant Literature on Sparse Transforms} This work develops the literature on sparse Fourier transforms. The first of such works are \cite{Hassanieh2012, stobbe2012, Pawar2013}. The most relevant literature is that of the sparse Boolean Fourier (Hadamard) transform \cite{li2015spright,amrollahi2019efficiently}. Despite the promise of many of these algorithms, their application has remained relatively limited, being used in only a handful of prior applications. Our code base is forked from that of \cite{erginbas2023efficiently}.
In this work we introduce a series of major optimizations which specifically target properties of explanation functions. By doing so, our algorithm is made significantly more practical and robust than any prior work.

\paragraph{Importance of the Fourier Transform} 
The Fourier transform does more than just impart critical algebraic structure. 
The orthonormality of the Fourier transform means that small noisy variations in $f$ remain small in the Fourier domain. In contrast, AND interactions, which operate under the non-orthogonal Möbius transform \cite{kang2024learning}, can amplify small noisy variations, which limits practicality. Fortunately, this is not problematic, as it is straightforward to generate AND interactions from the surrogate function $\hat{f}$. Many popular interaction indices have simple definitions in terms of $F$.  Table~\ref{tab:fourier-def-1} highlights some key relationships, and Appendix~\ref{app:interactions} provides a comprehensive list. 

\begin{table}[h]
\centering
\begin{tabular}{@{}ccc@{}}
\toprule
\textbf{Shapley Value} & \textbf{Banzhaf Interaction Index}         & \textbf{M\"obius Coefficient}  \\ \midrule
 $\mathrm{SV}(i) = \sum \limits_{S \ni i,\; \abs{S} \text{ odd}}F(S)/\abs{S} $& $I^{BII}(S) = (-2)^{|S|} F(S)$      &        $I^{M}(S) = (-2)^{|S|} \sum \limits_{T \supseteq S} F(T)$   \\ \bottomrule
\end{tabular}
\caption{Popular attribution scores in terms of Fourier coefficients}
\label{tab:fourier-def-1}
\end{table}

\subsection{Directly Solving the LASSO} 
Before we proceed, we remark that in cases where $n$ is not too large, and we expect the degree of nonzero $\abs{\bk} \leq d$ to be reasonably small, enumeration is actually not infeasible. In such cases, we can set up the LASSO problem directly:
\begin{equation}\label{eq:LASSO_apdx}
    \hat{F} = \argmin_{\tilde{F}} \sum_{\bbm}\abs{f(\bbm) - \sum_{\abs{\bk} \leq d}  \tilde{F}(\bk)}^2 + \lambda \norm{\tilde{F}}_1.
\end{equation}
Note that this is distinct from the \emph{Faith-Banzhaf} and \emph{Faith-Shapley} solution methods because those perform regression over the AND, M\"obius basis.
We observe that the formulation above typically outperforms these other approaches in terms of faithfulness, likely due to the properties of the Fourier transform. 

Most popular solvers use \emph{coordinate descent} to solve \eqref{eq:LASSO_apdx}, but there is a long line of research towards efficiently solving this problem. In our code, we also include an implementation of Approximate Message Passing (AMP) \cite{maleki2010approximate}, which can be much faster in many cases. Much like the final phase of \SpecExp{}, AMP is a low complexity message passing algorithm where messages are iteratively passed between factor nodes (observations) and variable nodes. 

A more refined version of \SpecExp{}, would likely examine the parameters $n$ and the maximum degree $d$ and determine whether or not to directly solve the LASSO, or to apply the full \SpecExp{}, as we describe in the following sections. 


\subsection{Masking Pattern Design and Model Inference}
The first part of the algorithm is to determine which samples we collect. All steps of this part of the algorithm are outlined in Algorithm~\ref{alg:collect}. This is governed by two structures: the random linear codes $\bM_c$ and the BCH parity matrix $\bP$. Random linear codes have been well studied as central objects in error correction and cryptography. They have previously been considered for sparse transforms in \cite{amrollahi2019efficiently}. They are suitable for this application because they roughly uniformly hash $\bk$ with low hamming weight. 

The use of the $\bP \in \bbF_2^{p\times n}$, the parity matrix of a binary BCH code is novel. These codes are well studied for the applications in error correction \cite{Lin1999}, and they were once the preeminent form of error correction in digital communications. A primitive, narrow-sense BCH code is characterized by its length, denoted $n_c$, dimension, denoted $k_c$ (which we want to be equal to our input dimension $n$) and its error correcting capability $t_c = 2d + 1$, where $d$ is the minimum distance of the code. For some integer $m > 3$ and $t_c < 2^{m-1}$, the parameters satisfy the following equations:
\begin{eqnarray}
    n_c &=& 2^m -1 \\
    p = n_c-k_c &\leq& mt.
\end{eqnarray}
Note that the above says we can bounds $p \leq t \left\lceil \log_2(n_c) \right \rceil$, and it is easy to solve for $p$ given $n=k_c$ and $t$, however, explicitly bounding $p$ in terms of $n$ and $t$ is difficult, so for the purpose of discussion, we simply write $p \approx{t\log(n)}$, since $n_c = p + n$, and we expect $n \gg p$ in nearly all cases of interest. 

We use the software package \verb|galois| \cite{Hostetter_Galois_2020} to construct a generator matrix, $\bG \in \bbF_2^{n_c \times k_c}$ in systematic form:
\begin{equation} \label{eq:sys-form}
    \bG = 
\begin{bmatrix}
\bI_{k_c \times k_c}\\
\bP
\end{bmatrix}
\end{equation}
\begin{algorithm}
   \caption{Collect Samples}
   \label{alg:collect}
\begin{algorithmic}[1]
   \State {\bfseries Input:} Parameters $(n, t, b, C=3, \gamma=0.9)$, Query function $f(\cdot)$ 
     \For{$j=1$ {\bfseries to} $n$, $i=1$ {\bfseries to} $b$, $c=1$ {\bfseries to} $C$} \Comment{Generate random linear code}
     \State $X_{ij} \sim \mathrm{Bern}(0.5)$
     \State $\left[\bM_c\right]_{i,j} \gets X_{i,j}$
   \EndFor
   \State $\mathrm{Code} \gets \mathrm{BCH}(n_c=n_c, k_c\geq n,  t_c=t)$ \Comment{Systematic BCH code with dimension $n$ and correcting capacity $t$}
   \State $p \gets n_c - n$
   \State $\bP \gets \mathrm{Code}.\bP$
   \State $\cP \gets \mathrm{rows}(\bP) = \left[ \boldsymbol{0}, \bp_1, \dotsc, \bp_p \right]$
   \ForAll{ $\bell \in \bbF_2^b$, $i \in \{0, \dotsc, p\}$, $c \in \{1, \dotsc, C\}$}
   \State $u_{c,i} (\bell) \gets f\left(\bM_c^\trans \bell + \cP[i] \right)$ \Comment{Query the model at masking patterns}
   \EndFor
      \ForAll{$i \in \{0, \dotsc, p\}$,  $c \in \{1, \dotsc, C\}$} 
   \State $U_{c,i} \gets \mathrm{FFT}(u_{c,i})$ \Comment{Compute the Boolean Fourier transform of the collected samples}
   \EndFor
   \State $\bU_c \gets \left[ U_{c,1}, \dotsc, U_{c, p}\right]$ 
    \State {\bfseries Output:} Processed Samples  $\bU_c,U_{c,0}\; c=1, \dotsc, C$
\end{algorithmic}
\end{algorithm}
Note that according to \eqref{eq:sys-form} $\bP \in \bbF_2^{ p \times k_c}$. In cases where $k_c > n$, we consider only the first $n$ rows of $\bP$. This is a process known as \emph{shortening}.
Our application of this BCH code in our application is rather unique. Instead of the typical use of a BCH code as \emph{channel correction} code, we use it as a \emph{joint source channel code}. 

Let $\bp_0 = \boldsymbol{0}$, and let $\bp_i,\;i=1,\dotsc,p$ correspond to the rows of $\bP$. We collect samples written as:
\begin{equation}\label{eq:subsample_apdx}
    u_{c,i} (\bell) \gets f\left(\bM_c^{\trans} \bell + \bp_i \right) \;\forall \bell \in \bbF_2^b,\; c = 1,\dotsc, C,\;i=0, \dotsc, p.
\end{equation}
Note that the total number of unique samples can be upper bounded by $C(p+1)2^{b}$. For large $n$ this upper bound is nearly always very close to the true number of unique samples collected. After collecting each sample, we compute the boolean Fourier transform. The forward and inverse transforms as we consider in this work are defined below.
\begin{equation}\label{eq:transform_def}
    \text{Forward:}\quad F(\bk) = \frac{1}{2^{n}} \sum_{\bbm \in \bbF_2^n} (-1)^{\inp{\bk}{\bbm}}f(\bbm) \qquad \text{Inverse:}\quad f(\bbm)  = \sum_{\bk \in \bbF_2^n} (-1)^{\inp{\bbm}{\bk}} F(\bk),
\end{equation}
When samples are collected according to \eqref{eq:subsample_apdx}, after applying the transform in \eqref{eq:transform_def}, the transform of $u_{c,i}$ can be written as:
\begin{equation} \label{eq:alais_apdx}
    U_{c,i}(\bj) = \sum_{\bk\; : \;\bM_c \bk = \bj} (-1)^{\inp{\bp_i}{\bk}} F(\bk).
\end{equation}
To ease notation, we write $\bU_c = [U_{c,1}, \dotsc, U_{c,p}]^T$. Then we can write
\begin{equation}\label{eq:factor_nodes}
    \bU_c(\bj) = \sum_{\bk\; : \;\bM_c \bk = \bj} (-1)^{\bP \bk} F(\bk),
\end{equation}
where we have used the notation $(-1)^{\bP \bk} = [(-1)^{\inp{\bp_0}{\bk}}, \dots, (-1)^{\inp{\bp_p}{\bk}}]^T$. We call the $(-1)^{\bP \bk}$ the \emph{signature} of $\bk$. This signature helps to identify the index of the largest interactions $\bk$, and is central to the next part of the algorithm. Note that we also keep track of $U_{c,0}(\bj)$, which is equal to the unmodulated sum $U_{c,0}(\bj) = \sum_{\bk\; : \;\bM_c \bk = \bj} F(\bk)$.
\subsection{Message Passing for Fourier Transform Recovery}
Using the samples \eqref{eq:factor_nodes}, we aim to recover the largest Fourier coefficients $F(\bk)$. To recover these samples we apply a message passing algorithm, described in detail in Algorithm~\ref{alg:message-pass}. The factor nodes are comprised of the $C2^b$ vectors $\bU_c(\bj) \; \forall \bj \in \bbF_2^b$. Each of these factor nodes are connected to all values $\bk$ that are comprise their sum, i.e., $\{\bk\mid \bM_c \bk = \bj\}$. Since the number of variable nodes is too great, we initialize the value of each variable node, which we call $\hat{F}(\bk)$ to zero implicitly. The values $\hat{F}(\bk)$ for each variable node indexed by $\bk$ represent our estimate of the Fourier coefficients. 




\subsubsection{The message from factor to variable} Consider an arbitrary factor node $\bU_c(\bj)$ initialized according to \eqref{eq:factor_nodes}. We want to understand if there are any large terms $F(\bk)$ involved in the sum in \eqref{eq:factor_nodes}. To do this, we can utilize the signature sequences $(-1)^{\bP \bk}$. If $\bU_c(\bj)$ is strongly correlated with the signature sequence of a given $\bk$, i.e., if $\abs{\inp{(-1)^{\bP \bk}}{\bU_c(\bj)}}$ is large, and $\bM_c \bk = \bj$, from the perspective of $\bU_c(\bj)$, it is likely that $F(\bk)$ is \emph{large}. Searching through all $\bM_c \bk = \bj$, which, for a full rank $\bM_c$ contains $2^{n-b}$ different $\bk$ is intractable, and likely to identify many spurious correlations. Instead, we rely on the structure of the BCH code from which $\bP$ is derived to solve this problem.

\paragraph{BCH Hard Decoding} The BCH decoding procedure is based on an idea known generally in signal processing as ``treating interference as noise". For the purpose of explanation, assume that there is some $\bk^*$ with large $F(\bk^*)$, and all other $\bk$ such that $\bM_c \bk = \bj$ correspond to small $F(\bk)$. For brevity let $\cA_c(\bj) = \{\bk\mid \bM_c \bk = \bj\}$. We can write:
\begin{equation}
    \bU_c(\bj) = F(\bk^*)(-1)^{\bP \bk^*} + \sum_{\cA_c(\bj) \setminus \bk^*} (-1)^{\bP \bk} F(\bk)
\end{equation}
After we normalize with respect to $U_{c,0}(\bj)$ this yields:
\begin{eqnarray}
    \frac{\bU_c(\bj)}{U_{c,0}(\bj)} &=& \left( \frac{1}{1 + \sum_{\cA_c(\bj) \setminus \bk^*}F(\bk)/F(\bk^*)}\right)(-1)^{\bP \bk^*} + \left(\frac{\sum_{\cA_c(\bj) \setminus \bk^*} (-1)^{\bP \bk}F(\bk)}{F(\bk^*) + \sum_{\cA_c(\bj) \setminus \bk^*} F(\bk)}\right) \\
    &=& A(\bj) (-1)^{\bP \bk^*} + \bw(\bj). \label{eq:ratio}
\end{eqnarray}
As we can see, the ratio \eqref{eq:ratio} is a noise-corrupted version of the signature sequence of $\bk^*$. To estimate $\bP \bk$ we apply a nearest-neighbor estimation rule outlined in Algorithm~\ref{alg:bch-hard}. In words, if the $i$th coordinate of the vector \eqref{eq:ratio} is closer to $-1$ we estimate that the corresponding element of $\bP \bk$ to be $1$, conversely, if the $i$th coordinate is closer to $1$ we estimate the corresponding entry to be $0$. This process effectively converts the multiplicative noise $A$ and additive noise $\bw$ to a noise vector in $\bbF_2$. We can write this as $\bP \bk^{*} + \bn$. According to the Lemma~\ref{lem:decoding} if the hamming weight $\bn$ is not too large, we can recover $\bk^*$. 

\begin{lemma}\label{lem:decoding}
    If $\abs{\bn} + \abs{\bk^*} \leq t$, where $\bn$ is the additive noise in $\bbF_2$ induced by the noisy process in \eqref{eq:ratio} and the estimation procedure in Algorithm~\ref{alg:bch-hard}, then we can recover $\bk^*$.
\end{lemma}
\begin{proof}
    Observe that the generator matrix of the BCH code is given by \eqref{eq:sys-form}. Thus, there exists a codeword of the form
    \begin{equation}
    \bc = \bG \bk^*= 
\begin{bmatrix}
\bk^*\\
\bP \bk^*
\end{bmatrix}
\end{equation}
Now construct the ``received codeword" as in Algorithm~\ref{alg:bch-hard}:
    \begin{equation}
    \br = 
\begin{bmatrix}
\boldsymbol{0}\\
\bP \bk^* + \bn
\end{bmatrix}
\end{equation}
Thus $\abs{\bc- \br} = \abs{\bn} + \abs{\bk^*}$. Since the BCH code was designed to be $t$ error correcting, Decoding the code will recover $\bc$, which contains $\bk^*$.
\end{proof}
For decoding we use the implementation in the python package \verb|galois| \cite{Hostetter_Galois_2020}. It implements the standard procedure of the Berlekamp-Massey Algorithm followed by the Chien Search algorithm for BCH decoding. 
\begin{algorithm}
   \caption{BCH Hard Decode}
   \label{alg:bch-hard}
\begin{algorithmic}[1]
   \State {\bfseries Input:} Observation $\bU_c(\bj)$, Decoding function $\mathrm{Dec}(\cdot)$
   \State $r_i \gets 0 \; i=1\dotsc, n$
   \ForAll{$i \in n+1, \dotsc, n+p$}
        \State $r_i \gets \mathds{1}\left\{ \frac{U_{c,i}(\bj)}{U_{c,0}(\bj) } < 0 \right\}$
   \EndFor
    \State dec, $ \hat{\bk} \gets \mathrm{Dec}(\br)$
    \State {\bfseries Output:} dec, $\hat{\bk}$ 
\end{algorithmic}
\end{algorithm}

\paragraph{BCH Soft Decoding} In practice the conversion of the real-valued noisy observations \eqref{eq:ratio} to noisy elements in $\bbF_2$ is a process that destroys valuable information. In coding theory, this is known as \emph{hard input} decoding, which is typically suboptimal. For example, certain coordinates will have values $\frac{U_{c,i}(\bj)}{U_{c,0}(\bj)} \approx 0$. For such coordinates, we have low confidence about the corresponding value of $(-1)^{\inp{\bp_i}{\bk^*}}$, since it is equally close to $+1$ and $-1$. This uncertainty information is lost in the process of producing a hard input. With this so-called \emph{soft information} it is possible to recover $\bk^*$ even in cases where there are more than $t$ errors in the hard decoding case. We use a simple soft decoding algorithm for BCH decoding known as a chase decoder. The main idea behind a chase decoder   is to perform hard decoding on the $d_{\text{chase}}$ most likely hard inputs, and return the decoder output of the most likely hard input that successfully decoded. In practical setting like the ones we consider in this work, we don't have an understanding of the noise in \eqref{eq:ratio}. A practical heuristic is to simply look at the \emph{margin} of estimation. In other words, if $\abs{\frac{U_{c,i}(\bj)}{U_{c,0}(\bj)}}$ is large, we assume it has high confidence, while if it is small, we assume the confidence is low. Interestingly, if we assume $A(\bj) = 1$ and $\bw(\bj) \sim \cN(0, \sigma^2)$ in \eqref{eq:ratio}, then the ratio corresponds exactly to the logarithm of the likelihood ratio (LLR) $\log \left( \frac{\mathrm{Pr}\left(\inp{\bp_i}{\bk^*} = 0\right)}{\mathrm{Pr}\left(\inp{\bp_i}{\bk^*} = 1\right)}\right)$. For the purposes of soft decoding we interpret these ratios as LLRs. Pseudocode can be found in Algorithm~\ref{alg:bch-soft}.


\emph{Remark: BCH soft decoding is a well-studied topic with a vast literature. Though we put significant effort into building a strong implementation of \SpecExp{}, we have used the simple Chase Decoder (described in Algorithm~\ref{alg:bch-soft} below) as a soft decoder. The computational complexity of Chase Decoding scales as $2^{d_\text{chase}}$, but other methods exist with much lower computational complexity and comparable performance.}

\begin{algorithm}
   \caption{BCH Soft Decode (Chase Decoding)}
   \label{alg:bch-soft}
\begin{algorithmic}[1]
   \State {\bfseries Input:} Observation $\bU_c(\bj)$, Decoding function $\mathrm{Dec}(\cdot)$, Chase depth $d_{\text{chase}}$.
   \State $r_i \gets 0 \; i=1\dotsc, n$
   \State $\cR \gets d_{\text{chase}}$ most likely hard inputs \Comment{Can be computed efficiently via dynamic programming}
   \State dec $\gets False$
   \State $j \gets 0$
   \While{$dec$ is $False$ and $j \leq d_{\text{chase}}$}
        \State $\br_{(n+1):(n+p)} \gets \cR [j]$
        \State $j \gets j+1$
        \State dec, $ \hat{\bk} \gets \mathrm{Dec}(\br)$
   \EndWhile
    \State {\bfseries Output:} dec, $\hat{\bk}$ 
\end{algorithmic}
\end{algorithm}

If we successfully decode some $\bk$ from the BCH decoding process via the bin $\bU_{c}(\bj)$, we construct a message to the corresponding variable node. Before we do this, we verify that the $\bk$ term satisfies $\bM_c \bk = \bj$. This acts as a final check to increase our confidence in the output of $\bk$. The message we construct is of the following form:
\begin{equation}\label{eq:check_msg}
    \mu_{(c,\bj) \rightarrow \bk} = \inp{(-1)^{\bP \bk}}
        {\bU_c(\bj)}/p
\end{equation}
To understand the structure of this message. This message can be seen as an estimate of the Fourier coefficient. Let's assume we are computing this message for some $\bk^*$:   
\begin{equation}
\mu_{(c,\bj) \rightarrow \bk^*} = F(\bk^*) + \sum_{\cA(\bj)\setminus \bk^*}\underbrace{\frac{1}{p}\inp{(-1)^{\bP \bk}}{(-1)^{\bP \bk^*}}}_{\text{typically small}}F(\bk)
\end{equation}
The inner product serves to reduce the noise from the other coefficients in the sum.
\begin{algorithm}
   \caption{Message Passing}
   \label{alg:message-pass}
\begin{algorithmic}[1]
\State {\bfseries Input:} Processed Samples  $\bU_c, c=1, \dotsc, C$
\State $\cS = \left\{ (c,\bj): \bj \in \bbF_2^b, c \in \{1, \dotsc, C\}\right\}$ \Comment{Nodes to process}
\State $\hat{F}[\bk] \gets 0 \;\forall\bk$
\State $\cK \gets \emptyset$
\While{$\abs{\cS} > 0$} \Comment{Outer Message Passing Loop}
    \State $\cS_{\text{sub}} \gets \emptyset$
    \State $\cK_{\text{sub}} \gets \emptyset$
    \For{$(c,\bj) \in \cS$}
        \State dec, $\bk$ $\gets \mathrm{DecBCH} (\bU_c(\bj))$ \Comment{Process Factor Node}
        \If{dec}
            \State corr $\gets \frac{\inp{(-1)^{\bP \bk}}{\bU_c(\bj)}}{\norm{\bU_c(\bj)}^2}$
        \Else
            \State corr $\gets 0$
        \EndIf
        \If{corr $> \gamma$} \Comment{Interaction identified}
            \State $\cS_{\text{sub}} \gets \cS_{\text{sub}} \cup \{(\bk, c, \bj)\}$
            \State $\cK_{\text{sub}} \gets \cK_{\text{sub}} \cup \{\bk\}$
        \Else
            \State $\cS \gets \cS \setminus \{ (c, \bj)\}$ \Comment{Cannot extract interaction}
        \EndIf
    \EndFor
    \For{ $\bk \in \cK_{\text{sub}}$}
        \State $\cS_{\bk} \gets \{ (\bk', c', \bj') \mid (\bk', c', \bj') \in \cS_{\text{sub}}, \bk' = \bk \}$
        \State $\mu_{(c,\bj) \rightarrow \bk} \gets \inp{(-1)^{\bP \bk}}
        {\bU_c(\bj)}/p$ 
        \State $\mu_{\bk \rightarrow \text{all}} \gets \sum_{(\bk, c, \bj) \in \cS_{\bk}} \mu_{(c,\bj) \rightarrow \bk}$
        \State $\hat{F}(\bk) \gets \hat{F}(\bk) + \mu_{\bk \rightarrow \text{all}}$ \Comment{Update variable node}
        \For{$c \in \{ 1, \dotsc, C\}$}
            \State $\bU_c(\bM_c \bk) \gets \bU_c(\bM_c \bk) - \mu_{\bk \rightarrow \text{all}}\cdot(-1)^{\bP \bk}$ \Comment{Update factor node}
            \State $\cS \gets \cS \cup \{ (c, \bM_c \bk)\}$
        \EndFor
    \EndFor
    \State $\cK \gets \cK \cup \cK_{\text{sub}}$
\EndWhile
  \State {\bfseries Output: $\left\{ \left(\bk, \hat{F}(\bk)\right) \mid \bk \in \cK\right\}$}, interactions, and scalar values corresponding to interactions.
\end{algorithmic}
\end{algorithm}
\subsubsection{The message from variable to factor}
The message from factor to variable is comparatively simple. The variable node takes the average of all the messages it receives, adding the result to its state, and then sends that average back to all connected factor nodes. These factor nodes then subtract this value from their state and then the process repeats. 


\subsection{Computational Complexity}

\paragraph{Generating masking patterns $\bbm$} Constructing each masking pattern requires $n2^b$ for each $\bM_c$. The algorithm for computing it efficiently involves a gray iteratively adding to an $n$ bit vector and keeping track of the output in a Gray code. Doing this for all $C$, and then adding all $p$ additional shifting vectors makes the cost $O(Cpn2^b)$.

\paragraph{Taking FFT} For each $u_{c,i}$ we take the Fast Fourier transform in $b2^b$ time, with a total of $O(Cpb2^b)$. This is dominated by the previous complexity since, $b \leq n$

\paragraph{Message passing} One round of BCH hard decoding is $O(n_ct + t^2)$. For soft decoding, this cost is multiplied by $2^{d_{\text{chase}}}$, which we is a constant.  Computing the correlation vector is $O(np)$, dominated by the computation of $\bP \bk$. In the worst case, we must do this for all $C 2^b$ vectors $\bU_c(\bj)$. We also check that $\bM \bk = \bj$ before sending the message, which costs $O(nb)$. Thus, processing all the factor nodes costs $O(C2^b(n_c t + t^2 + n(p+b)))$. The number of active (with messages to send) variable nodes is at most $C2^b$, and computing their factors is at most $C$. Thus, computing factor messages is at most $C^22^b$ messages. Finally, factor nodes are updated with at most $C2^b$ variable messages sending messages to at most $C$ factor nodes each, each with a cost of $O(np)$. Thus, the total cost of processing all variable nodes is $O(C^22^b + C^22^bnp)$. The total cost of message is dominated by processing the factors. 

The total complexity is then $O(2^b(n_c t + t^2 + n(p + b))$.
Note that $p = n_c - n = t \log(n_c)$. Due to the structure of the code and the relationship between $n,p$ and $n_c$, one could stop here, and it would be best to if we want to consider very large $t$. For the purposes of exposition, we will assume that $t \ll n$, which implies $n > p$, and thus $p \approx t \log(n)$. In this case, we can write:
\begin{equation}
    \text{Complexity} = O(2^b(nt\log(n)  + nb))
\end{equation}

To arrive at the stated equation in Section~\ref{sec:intro}, we take $2^b = O(s)$. Under the low degree assumption, we have $s = O(d\log(n))$. Then assuming we take $t= O(d)$, we arrive at a complexity of $O(sdn\log(n))$.

\section{Experiment Details}\label{apdx:experiments}
\subsection{Benchmark processing} \label{app:benchmarks}
Here, we discuss additional processing and cleaning steps specific to certain of the chosen benchmarks.

\subsubsection{VQA v2.0} \label{app:vqa}

\paragraph{Re-labeling} Rather than a single ground truth label per example, the VQA~\cite{antol2015vqa} and updated VQA v2.0~\citep{goyal2017making} datasets collect ten separate crowd-annotated labels per image-question query. Their accuracy metric then assigns a score to a model prediction based on the overlap between the prediction and these ten labels. As we assign a single ground-truth label to each question, we manually re-label all the VQA v2.0 queries we include in our revised subset rather than only inspecting ones for which some model failed.

\paragraph{Selection of queries} The VQA v2.0 dataset is designed to mitigate common biases in visual question answering datasets by balancing the original VQA dataset with complementary images that break a given bias. To maintain this construction, we randomly select from these image pairs in VQA v2.0, and reject a given pair if either image deemed ambiguous.

To further improve the clarity and ease of labeling of our samples, we also limit our subset to only `yes/no' questions within VQA v2.0, as open-ended queries have a greater potential for ambiguity and can often have multiple correct answers.

\subsubsection{Reading Comprehension Benchmarks} \label{app:hotpotqa}
\paragraph{Re-labeling to account for multiple correct responses} SQuAD2.0~\cite{rajpurkar2018know}, HotPotQA~\citep{yang2018hotpotqa}, and DROP~\citep{dua2019drop} are all question answering benchmarks based on background knowledge provided in-context. Most of these questions are open-ended, so there are often multiple valid responses. For instance, consider the following example from HotPotQA:
\\
\begin{tcolorbox}[colback=gray!6, colframe=gray!50, arc=2mm, boxrule=0.5pt]
    \textbf{Paragraph A:} \textit{Ethel Houbiers}
    
    Ethel Houbiers is a French voice actress.  She is the French voice of Penélope Cruz and Salma Hayek.
    
    \medskip
    
    \textbf{Paragraph B:} \textit{Salma Hayek}
    
    Salma Hayek Pinault ( Hayek Jiménez) (born September 2, 1966), known professionally as Salma Hayek, is a Mexican and American film actress, producer, and former model\ldots[\textit{continued}]
    
    \medskip
    
    \textbf{Question:} which Mexican and American film actress is Ethel Houbiers French voice of?
    
    \medskip
    
    \textbf{Answer:} Salma Hayek Pinault

\end{tcolorbox}
\vspace{\baselineskip}
 
\noindent While ``Salma Hayek Pinault'' is a valid answer, ``Salma Hayek'' is a second answer that should also be considered valid. To address these cases, when re-labeling examples, we set the revised label to be a list of possible valid responses. Specifically, if any LLM answer doesn't match the original label but is also a valid solution, we include this answer as part of the updated label. We also make an effort to further list additional valid options. If the question is sufficiently open-ended and ambiguous that too many possible options might be valid, we mark the question as bad. 

\paragraph{Marking example as mislabeled} As discussed above, the original reading comprehension benchmarks include many questions for which there might be multiple possible equivalent answers that are not listed as the correct solution, such as the following example from DROP:

\begin{tcolorbox}[colback=gray!6, colframe=gray!50, arc=2mm, boxrule=0.5pt]
\textbf{Context:} Coming off their overtime win over the Bills, the Steelers flew to M\&T Bank Stadium\ldots Pittsburgh trailed in the first quarter as Ravens quarterback Joe Flacco completed a 14-yard touchdown pass to wide receiver Anquan Boldin.  After a scoreless second quarter, Pittsburgh answered in the third quarter\ldots With the win, not only did the Steelers improve to 9-3, but it also allowed them to take the AFC North division lead for the first time since week 4.

\medskip

\textbf{Question:} How many scoring drives took place in the first half?

\medskip
    
\textbf{Answer:} 1
\end{tcolorbox}

\noindent Here, ``one'' (the word rather than the number) would also be a valid solution. However, it might be unfair for us to consider this a benchmark error, as this ambiguity could potentially be overcome by altering our prompting strategy (e.g., specifying to provide numerical answers). 

For our platinum benchmarks we still revise such examples; in this particular case, we mark any of the following solutions as valid: ``1,'' ``one,'' ``1 scoring drive,'' ``one scoring drive.'' However, when counting the number of mislabeled examples, we only consider cases where the original label is not included within the list of valid labels we identify. For instance, if for this example we had revised the label to ``2'' or ``two,'' we would have considered the original example mislabeled.

\paragraph{Answer-matching for the original benchmark}
In Table \ref{tab:original_vs_cleaned}, we compare the number of errors on each of the original and revised benchmarks. However, as we discuss in the paragraph above, the original reading comprehension benchmarks often include ambiguity in the form of many equivalent correct solutions that may have been clarified through better prompting. To more accurately calculate the number of model errors on these original benchmarks, we ask an LLM (GPT-4o) to identify models' answers that are ``equivalent'' to the benchmarks' solution, and consider any such equivalent answer as correct. We only apply this automated equivalence-checking process to calculate the error counts on the original benchmark. For our platinum benchmarks, we instead manually enumerated all valid responses during the revision process.


\subsection{Chain-of-Thought prompt template} \label{app:template}
We use a chain-of-thought prompt for evaluation on all datasets except for VQA V2.0~\citep{goyal2017making}, for which there is general no need for multiple reasoning steps. The specific prompt varies slightly between benchmarks, however the general templates are as follows:\\

\noindent\textbf{Open-ended Question:}
\begin{tcolorbox}[colback=gray!6, colframe=gray!50, arc=2mm, boxrule=0.5pt]
    \texttt{Answer the following \{category\} question.\\\\
    \{question\}\\\\
    Think step-by-step.\textsf{ }Then, answer in the format "Answer:\textsf{ }XXX".}
\end{tcolorbox}

\vspace{\baselineskip}

\noindent\textbf{Multiple-Choice Question:}
\begin{tcolorbox}[colback=gray!6, colframe=gray!50, arc=2mm, boxrule=0.5pt]
    \texttt{Answer the following \{category\} question.\\\\
    \{question\}\\\\
    Options:\\
    A) \{option A\}\\
    B) \{option B\}\\
    C) \{option C\}\\
    D) \{option D\}\\\\
    Think step-by-step.\textsf{ }Then, provide the final answer in the format "Answer:\textsf{ }X" where X is the correct letter choice.}

\end{tcolorbox}

\vspace{\baselineskip}

\noindent For BIG-bench~\cite{srivastava2022beyond}, we use (A) instead of A) for multiple choice style to align with the prompt used by the original authors. For open-ended math questions, we additionally specify to respond with an integer and to exclude additional formatting, as model often style outputs with latex styling. We exclude chain-of-thought prompting for VQA v2.0, as the questions rarely require any explicit reasoning to answer. We also find that, in practice, models rarely actually think step-by-step for simple visual reasoning questions, even when prompted to do so.


\section{Relationships between Fourier and Interaction Concepts} \label{apdx:fourier-interactions}
\label{app:interactions}
\textbf{Fourier to M\"obius Coefficients:}
The M\"obius Coefficients, also referred to as the \emph{Harsanyi dividends}, can be recovered through \cite{saminger-platz_bases_2016}:
\begin{equation}
I^M(S) = (-2)^{|S|}\sum_{T\supseteq S}F(T).
\label{eq:mobius}
\end{equation}
\textbf{Fourier to Banzhaf Interaction Indices:} Banzhaf Interactions Indices \cite{roubens1996interaction} have a close relationship to Fourier coefficients. As shown in \cite{grabisch2000equivalent}:
\begin{equation}
    I^{BII}(S) = \sum_{T \supseteq S} \frac{2^{|S|}}{2^{|T|}}I^{M}(T).
\end{equation}
Using the relationship from Eq.~\ref{eq:mobius},
\begin{align}
    I^{BII}(S) &= \sum_{T \supseteq S} \frac{2^{|S|}}{2^{|T|}}(-2)^{|T|}\sum_{R \supseteq T}F(R) \\
    &= 2^{|S|}
    \sum_{\substack{T \supseteq S}} 
    (-1)^{|T|} \sum_{R \supseteq T}F(R)\\
    &= 2^{|S|}
    \sum_{\substack{R \supseteq S}} F(R)\sum_{S \subseteq T \subseteq R}
    (-1)^{|T|},\\
    &= (-2)^{|S|}F(S)
\end{align}
where the last line follows due to $\sum_{S \subseteq T \subseteq R}
    (-1)^{|T|}$ evaluating to 0 unless $R = S$.

When $S$ is a singleton, we recover the relationship between Fourier Coefficients and the Banzhaf Value $BV(i)$:
\begin{equation}
BV(i) = I^{BII}(\{i\}) = -2F(\{i\}).
\end{equation}

\textbf{Fourier to Shapley Interaction Indices:}
Shapley Interaction Indices \cite{GRABISCH1997167} are a generalization of Shapley values to interactions. Using the following relationship to M\"obius Coefficients \cite{grabisch2000equivalent}:
\begin{align}
    I^{SII}(S) &= \sum_{T \supseteq S} \frac{I^M(S)}{|T|-|S| + 1}\\
    &= \sum_{T \supseteq S} \sum_{R \supseteq T}\frac{(-2)^{|T|}F(R)}{|T|-|S| + 1}\\
    &= \sum_{R \supseteq S} F(R)\sum_{S \subseteq T \subseteq R}\frac{(-2)^{|T|}}{|T|-|S| + 1}\\
    &= \sum_{R \supseteq S} F(R)\sum_{j = |S|}^{|R|}\frac{(-2)^{j}}{j-|S| + 1 }\binom{|R| - |S|}{j - |S|}\\
    &= \sum_{R \supseteq S} F(R)\sum_{k=0}^{|R|-|S|}\frac{(-2)^{k+|S|}}{k + 1}\binom{|R| - |S|}{k}\\
    &= (-2)^{|S|}\sum_{R \supseteq S} F(R)\sum_{k=0}^{|R|-|S|}\frac{(-2)^{k}}{k + 1}\binom{|R| - |S|}{k}
\end{align}
Consider the following integral and an application of the binomial theorem:
\begin{align}
    \int_0^t (1+x)^{|R|-|S|}dx &=\int_0^t \sum_{k=0}^{|R|-|S|} \binom{|R| - |S|}{k} x^k  dx\\
    &= \sum_{k=0}^{|R|-|S|} \binom{|R| - |S|}{k} \int_0^t x^k  dx \\
    &= \sum_{k=0}^{|R|-|S|} \binom{|R| - |S|}{k} \left(\frac{t^{k+1}}{k+1}\right)
\end{align}
Evaluating at $t=-2$:
\begin{align}
\sum_{k=0}^{|R|-|S|} \binom{|R| - |S|}{k} \left(\frac{(-2)^{k}}{k+1}\right) &= -\frac{1}{2} \int_0^{-2} (1+x)^{|R|-|S|}dx\\
&= -\frac{1}{2}\cdot \frac{(-1)^{|R|-|S|+1}-1}{|R|-|S|+1}\\
&= \begin{cases}
			\frac{1}{|R|-|S|+1}, & \text{if Parity($|R|$) = Parity($|S|$) }\\
            0, & \text{otherwise}
		 \end{cases}
\end{align}
As a result, we find the relationship between Shapley Interaction Indices and Fourier Coefficients:
\begin{equation}
I^{SII}(S) = (-2)^{|S|}\sum_{\substack{R \supseteq S, \\ (-1)^{|R|} = (-1)^{|S|}}} \frac{F(R)}{|R|-|S|+1}.
\end{equation}

When $S$ is a singleton, we recover the relationship between Fourier Coefficients and the Shapley Value $SV(i)$:
\begin{equation}
SV(i) = I^{SII}(\{i\}) = (-2)\sum_{\substack{R \supseteq \{i\}, \\ |R| \text{ is odd}}} \frac{F(R)}{|R|}.
\end{equation}


\textbf{Fourier to Faith-Banzhaf Interaction Indices:} Faith-Banzhaf Interaction Indices \cite{tsai2023faith} of up to degree $\ell$ are the unique minimizer to the following regression objective: 
\begin{equation}
   \sum_{S \subseteq [n]}  \left(f(S) - \sum_{T \subseteq S, |T| \leq \ell} I^{FBII}(T,\ell) \right)^2. 
\end{equation}

Let $g(S)$ be the XOR polynomial up to degree $\ell$ that minimizes the regression objective. Appealing to Parseval's identity, 
\begin{equation}
   \sum_{S \subseteq [n]}  \left(f(S) - g(S)\right)^2 =    \sum_{S \subseteq [n]}  \left(F(S) - G(S)\right)^2 = \sum_{S\subseteq [n],|S| \leq \ell } \left(F(S) - G(S)\right)^2 + \sum_{S\subseteq [n],|S| > \ell } F(S)^2, 
\end{equation}
which is minimized when $G(S) = F(S)$ for $|S| \leq \ell$. Using Eq.~\ref{eq:mobius}, it can be seen that the Faith-Banzhaf Interaction Indices correspond to the M\"obius Coefficients of the function $f(S)$ truncated up to degree $\ell$:
\begin{equation}
       I^{FBII}(S,\ell) = (-2)^{|S|}\sum_{T\supseteq S, |T| \leq \ell}F(T).
\end{equation}


\textbf{Fourier to Faith-Shapley Interaction Indices:} Faith-Shapley Interaction Indices \cite{tsai2023faith} of up to degree $\ell$ have the following relationship to M\"obius Coefficients: 
\begin{align}
   I^{FSII}(S,\ell) &= I^M(S) + (-1)^{\ell - |S|} \frac{|S|}{\ell+|S|}\binom{\ell}{|S|}\sum_{T\supset S, |T|>\ell}\frac{\binom{|T|-1}{\ell}}{\binom{|T|+\ell -1}{\ell + |S|}} I^M(T) \\
   &= (-2)^{|S|}\sum_{T\supseteq S}F(T) + (-1)^{\ell - |S|} \frac{|S|}{\ell+|S|}\binom{\ell}{|S|}\sum_{T\supset S, |T|>\ell}\frac{\binom{|T|-1}{\ell}}{\binom{|T|+\ell -1}{\ell + |S|}} (-2)^{|T|}\sum_{R\supseteq T}F(R) \\
   &= (-2)^{|S|}\sum_{T\supseteq S}F(T) + (-1)^{\ell - |S|} \frac{|S|}{\ell+|S|}\binom{\ell}{|S|}\sum_{R\supset S, |R| > \ell }F(R) \sum_{S \subset T\subseteq R, |T|>\ell}\frac{\binom{|T|-1}{\ell}}{\binom{|T|+\ell -1}{\ell + |S|}} (-2)^{|T|}.
\end{align}



\textbf{Fourier to Shapley-Taylor Interaction Indices:}
Shapley-Taylor Interactions Indices \cite{dhamdhere2019shapley} of up to degree $\ell$ are related to M\"obius Coefficients in the following way:
\begin{equation}
    I^{STII}(S,\ell) = \begin{cases}
I^M(S), \quad \textnormal{if } |S| < \ell\\
\sum_{T \supseteq S} \binom{|T|}{\ell}^{-1}I^M(T), \quad \textnormal{if } |S| = \ell.
\end{cases}
\end{equation}

From an application of Eq.~\ref{eq:mobius},

\begin{equation}
    I^{STII}(S,\ell) = \begin{cases}
(-2)^{|S|}\sum_{T\supseteq S}F(T), \quad \textnormal{if } |S| < \ell\\
\sum_{T \supseteq S} \binom{|T|}{\ell}^{-1}(-2)^{|T|}\sum_{R\supseteq T}F(R), \quad \textnormal{if } |S| = \ell.
\end{cases}
\end{equation}
Simplifying the sum in the $|S|=\ell$ case:
\begin{align}
    \sum_{T \supseteq S} \binom{|T|}{\ell}^{-1}(-2)^{|T|}\sum_{R\supseteq T}F(R) &= \sum_{R \supseteq S}F(R)\sum_{S\subseteq T\subseteq R}\binom{|T|}{\ell}^{-1}(-2)^{|T|}\\
    &= \sum_{R \supseteq S}F(R)\sum_{k = \ell}^{|R|}\binom{k}{\ell}^{-1}(-2)^{k}\binom{|R|-\ell}{k-\ell}\\
\end{align}
Hence, 
\begin{equation}
    I^{STII}(S,\ell) = \begin{cases}
(-2)^{|S|}\sum_{T\supseteq S}F(T), \quad \textnormal{if } |S| < \ell\\
\sum_{T \supseteq S}F(T)\sum_{k = \ell}^{|T|}\binom{k}{\ell}^{-1}(-2)^{k}\binom{|T|-\ell}{k-\ell}, \quad \textnormal{if } |S| = \ell.
\end{cases}
\end{equation}
%%%%%%%%%%%%%%%%%%%%%%%%%%%%%%%%%%%%%%%%%%%%%%%%%%%%%%%%%%%%%%%%%%%%%%%%%%%%%%%
%%%%%%%%%%%%%%%%%%%%%%%%%%%%%%%%%%%%%%%%%%%%%%%%%%%%%%%%%%%%%%%%%%%%%%%%%%%%%%%


\end{document}




% this must go after the closing bracket ] following \twocolumn[ ...

% This command actually creates the footnote in the first column
% listing the affiliations and the copyright notice.
% The command takes one argument, which is text to display at the start of the footnote.
% The \icmlEqualContribution command is standard text for equal contribution.
% Remove it (just {}) if you do not need this facility.

%\printAffiliationsAndNotice{}  % leave blank if no need to mention equal contribution












































%%%%%%%%%%%%%%%%%%%%%%%%%%%%%%%%%%%%%%%%%%%%%%%%%%%%%%%%%%%%%%
%We use the closed-source \emph{GPT-4o mini}, highlighting the power of our model-agnostic approach. 

%, powered by error correction codes that can identify important interactions without an exhaustive search. 
%Experiments across three popular datasets, containing input sizes varying from $10$ to $1000$ demonstrate that \SpecExp{} scales to large contexts, where all competing interaction attribution approaches are infeasible for large $n$.
%For large input lengths, we outperform marginal attribution methods by up to 20\% in terms of faithfully reconstructing near state-of-art transformer based LLM outputs, on sentiment and multi-hop reasoning tasks where interactions are critical. \SpecExp{} successfully identifies key interactions that influence model outputs as well as human-labeled interactions. As a case study, we use \SpecExp{} to generate explanations to study reasoning in vision-language models and LLMs. We use the closed-source \emph{GPT-4o mini}, highlighting the power of our model-agnostic approach. 

%\BY{We need to indicate these are SOTA models and these are not easy problems, if true. Why should we trust these explanations?}
%\BY{how many? are these hard datasets? being more precise is more informative}
%\JK{Is the statement on faithfulness not enough to address the ``"why should we trust?" part}
%In both cases generating interaction based explanations at this scale is only possible because of the advancements of SpectralExplain.
 %possible $2^n$ interactions for contexts of length $n$. %\BY{This is confusing. The previous sentence is saying that SHAP can do interactions. Clarify} 

% which is commonly found in real-world data and models. Specifically, SpectralExplain applies a sparse Fourier transform, utilizing concepts from signal processing, and information and coding theory to reach this scale. Experiments across multiple benchmarks show that SpectralExplain can scale to large contexts while significantly out-performing (up to 20\%) other  attribution approaches in terms of faithfully reconstructing the outputs of LLMs and identifying interactions that influence outputs, and human-labelled interactions. As a case study we apply SpectralExplain to generate explanations for a task of identifying reasoning errors in LLMs and in question-answering for multi-modal models. In both cases generating interaction based explanations at this scale is only possible because of the advancements of SpectralExplain.

%\BY{There are extensions of SHAP that deal with interactions. A google search gives: https://[a] www.frontiersin.org/journals/nutrition/articles/ 10.3389/fnut.2022.871768/full  and [b] https://towardsdatascience.com/analysing-interactions-with-shap-8c4a2bc11c2a -- the latter implies that the SHAP package has interactions already. pls check them out} \JK{To clear this up: There are many definitions of interactions that can be thought of as extending the Shapley value - we mention Faith-Shapley. SHAP is the a software package that has evolved over time. It now features some tools to compute interactions via tree based models (work by Hugh Chen -- will add to citations) but is generally limited to the type of model. [a] and [b] use this. SHAP-IQ is a fork of the SHAP package that focuses on computing interactions for many black box models so we mention that a few times in the paper, with comparisons.}

%\BY{judged by humans?}\JK{Not human judged, based on their ability to predict changes in model output. ``True to the model"} 
%Understanding the outputs of large language models (LLMs) is a central problem in deep learning. The most popular post-hoc methods like SHAP are unable to capture complex interactions between inputs, while more expressive approaches that capture these interactions remain computationally intractable for even modest input sizes. We introduce SpectralExplain, an algorithm that can capture important interactions between inputs and scale to large input spaces. To achieve this, SpectralExplain exploits an underlying sparsity among interactions via a connection to the Fourier transform and information theoretic tools that use this structure to maintain computational efficiency. We show that SpectralExplain works at the scale of large language models, providing explanations that are more faithful to the underlying model, while maintaining a reasonable computational complexity. Thus SpectralExplain is the first to provide interaction-level explanations to large scale language tasks.