%%%%%%%% ICML 2024 EXAMPLE LATEX SUBMISSION FILE %%%%%%%%%%%%%%%%%

\documentclass{article}

% Recommended, but optional, packages for figures and better typesetting:
\usepackage{microtype}
\usepackage{graphicx}
\usepackage{booktabs} % for professional tables
\usepackage{duckuments} % for more interesting ducks
\usepackage{tcolorbox}
\usepackage{todonotes}

% hyperref makes hyperlinks in the resulting PDF.
% If your build breaks (sometimes temporarily if a hyperlink spans a page)
% please comment out the following usepackage line and replace
% \usepackage{icml2024} with \usepackage[nohyperref]{icml2024} above.
\usepackage{hyperref}
\usepackage{array}
%\usepackage{algorithm}
\usepackage{algpseudocode}
\usepackage{makecell} % For multiline cells
\renewcommand\cellalign{cc} % Align cells both vertically and horizontally
%\usepackage[dvipsnames]{xcolor} % For extended color support


% Attempt to make hyperref and algorithmic work together better:
\newcommand{\theHalgorithm}{\arabic{algorithm}}

% Use the following line for the initial blind version submitted for review:
% \usepackage{icml2024}

% If accepted, instead use the following line for the camera-ready submission:
\usepackage[accepted]{icml2025}

% For theorems and such
\usepackage{amsmath}
\usepackage{amssymb}
\usepackage{mathtools}
\usepackage{amsthm}
\usepackage{color,soul}
\usepackage{tabularx}
\usepackage{graphicx}
\usepackage{multirow}
\usepackage{dsfont}
\usepackage{enumitem}
\newcolumntype{C}[1]{>{\centering\arraybackslash}m{#1}}
% if you use cleveref..
\usepackage[capitalize,noabbrev]{cleveref}

%%%%%%%%%%%%%%%%%%%%%%%%%%%%%%%%
% THEOREMS
%%%%%%%%%%%%%%%%%%%%%%%%%%%%%%%%
\theoremstyle{plain}
\newtheorem{theorem}{Theorem}[section]
\newtheorem{proposition}[theorem]{Proposition}
\newtheorem{lemma}[theorem]{Lemma}
\newtheorem{corollary}[theorem]{Corollary}
\theoremstyle{definition}
\newtheorem{definition}[theorem]{Definition}
\newtheorem{assumption}[theorem]{Assumption}
\theoremstyle{remark}
\newtheorem{remark}[theorem]{Remark}

% Todonotes is useful during development; simply uncomment the next line
%    and comment out the line below the next line to turn off comments
%\usepackage[disable,textsize=tiny]{todonotes}
%\usepackage[textsize=tiny]{todonotes}
\usepackage{subcaption}
\DeclareMathOperator*{\argmin}{\arg\!\min}
\DeclareMathOperator*{\argmax}{\arg\!\max}
\DeclareMathOperator*{\E}{\mathbb{E}}
\DeclareMathOperator*{\Var}{Var}
\DeclareMathOperator*{\median}{median}
\DeclareMathOperator*{\mean}{mean}
\DeclareMathOperator{\supp}{supp}
\DeclareMathOperator{\esssup}{ess\,sup}
\DeclareMathOperator{\fastmobius}{FastMobius}
\DeclareMathOperator{\poly}{poly}
\DeclareMathOperator{\dec}{Dec}
\DeclareMathOperator{\sinc}{sinc}
\DeclareMathOperator{\linspan}{span}
\DeclareMathOperator{\proj}{Proj}
\DeclareMathOperator{\sv}{SV}
\DeclareMathOperator{\stii}{STII}
\DeclareMathOperator{\bz}{BZ}
\DeclareMathOperator{\err}{err}
\DeclareMathOperator{\obj}{obj}
\DeclareMathOperator{\bern}{Bern}
\DeclareMathOperator{\binomdist}{Binom}
\DeclareMathOperator\erf{erf}

\newcommand{\BY}[1]{\todo[author=BY, color=red, size=\small]{\color{white}#1}}
\newcommand{\Abhi}[1]{\todo[author=Abhi, color=orange, size=\small]{\color{white}#1}}
\newcommand{\JK}[1]{\todo[author=JK, color=blue, size=\small]{\color{white}#1}}
\newcommand{\KR}[1]{\todo[author=KR, color=green, size=\small]{\color{white}#1}}
\newcommand{\Efe}[1]{\todo[author=Efe, color=purple, size=\small]{\color{white}#1}}

\newcommand{\type}[1]{\mathrm{Type} \left( #1 \right)}
\newcommand{\detect}[1]{\mathrm{Detect} \left( #1 \right)}

\newcommand{\abs}[1]{\left\lvert #1 \right\rvert}
\newcommand{\red}[1]{{\color{red}#1}}
\newcommand{\blue}[1]{{\color{blue}#1}}
\definecolor{lightpink}{rgb}{1,0.9,0.9}
\newcommand{\defeq}{\vcentcolon=}
\newcommand{\eqdef}{=\vcentcolon}
\renewenvironment{quote}{%
  \list{}{%
    \leftmargin0.5cm   % this is the adjusting screw
    \rightmargin\leftmargin
  }
  \item\relax
}
{\endlist}
\newcommand{\dt}{\text{ d}}
\newcommand{\calD}{\mathcal{D}}
\newcommand{\cA}{\mathcal A}
\newcommand{\cB}{\mathcal B}
\newcommand{\cC}{\mathcal C}
\newcommand{\cD}{\mathcal D}
\newcommand{\cE}{\mathcal E}
\newcommand{\cF}{\mathcal F}
\newcommand{\cG}{\mathcal G}
\newcommand{\cH}{\mathcal H}
\newcommand{\cI}{\mathcal I}
\newcommand{\cJ}{\mathcal J}
\newcommand{\cK}{\mathcal K}
\newcommand{\cL}{\mathcal L}
\newcommand{\cM}{\mathcal M}
\newcommand{\cN}{\mathcal N}
\newcommand{\cO}{\mathcal O}
\newcommand{\cP}{\mathcal P}
\newcommand{\cQ}{\mathcal Q}
\newcommand{\cR}{\mathcal R}
\newcommand{\cS}{\mathcal S}
\newcommand{\cT}{\mathcal T}
\newcommand{\cU}{\mathcal U}
\newcommand{\cV}{\mathcal V}
\newcommand{\cW}{\mathcal W}
\newcommand{\cX}{\mathcal X}
\newcommand{\cY}{\mathcal Y}
\newcommand{\cZ}{\mathcal Z}
\newcommand{\scrA}{\mathscr A}
\newcommand{\scrB}{\mathscr B}
\newcommand{\scrC}{\mathscr C}
\newcommand{\scrD}{\mathscr D}
\newcommand{\scrE}{\mathscr E}
\newcommand{\scrF}{\mathscr F}
\newcommand{\scrG}{\mathscr G}
\newcommand{\scrH}{\mathscr H}
\newcommand{\scrI}{\mathscr I}
\newcommand{\scrJ}{\mathscr J}
\newcommand{\scrK}{\mathscr K}
\newcommand{\scrL}{\mathscr L}
\newcommand{\scrM}{\mathscr M}
\newcommand{\scrN}{\mathscr N}
\newcommand{\scrO}{\mathscr O}
\newcommand{\scrP}{\mathscr P}
\newcommand{\scrQ}{\mathscr Q}
\newcommand{\scrR}{\mathscr R}
\newcommand{\scrS}{\mathscr S}
\newcommand{\scrT}{\mathscr T}
\newcommand{\scrU}{\mathscr U}
\newcommand{\scrV}{\mathscr V}
\newcommand{\scrW}{\mathscr W}
\newcommand{\scrX}{\mathscr X}
\newcommand{\scrY}{\mathscr Y}
\newcommand{\scrZ}{\mathscr Z}
\newcommand{\bbB}{\mathbb B}
\newcommand{\bbS}{\mathbb S}
\newcommand{\bbR}{\mathbb R}
\newcommand{\bbZ}{\mathbb Z}
\newcommand{\bbI}{\mathbb I}
\newcommand{\bbQ}{\mathbb Q}
\newcommand{\bbP}{\mathbb P}
\newcommand{\bbE}{\mathbb E}
\newcommand{\bbF}{\mathbb F}
\newcommand{\bbN}{\mathbb N}
\newcommand{\sfE}{\mathsf E}
\newcommand{\sfF}{\mathsf F}
\newcommand{\sfV}{\mathsf V}
\newcommand{\one}{\mathds 1}

\newcommand{\Tr}{ \text{Tr} }
\newcommand{\trans}{\top}

\newcommand{\by}{\mathbf y}
\newcommand{\bxi}{\boldsymbol \xi}
\newcommand{\bx}{\mathbf x}
\newcommand{\bd}{\mathbf d}
\newcommand{\barbd}{\mathbf{ \overline{ d}}}
\newcommand{\barbh}{\mathbf{ \overline{ h}}}
\newcommand{\bc}{\mathbf c}
\newcommand{\be}{\mathbf e}
\newcommand{\bg}{\mathbf g}
\newcommand{\bw}{\mathbf w}
\newcommand{\bW}{\mathbf W}
\newcommand{\bP}{\mathbf P}
\newcommand{\bv}{\mathbf v}
\newcommand{\bj}{\mathbf j}
\newcommand{\bu}{\mathbf u}
\newcommand{\ba}{\mathbf a}
\newcommand{\bp}{\mathbf p}
\newcommand{\bk}{\mathbf k}
\newcommand{\bbm}{\mathbf m}
\newcommand{\br}{\mathbf r}
\newcommand{\bM}{\mathbf M}
\newcommand{\bX}{\mathbf X}
\newcommand{\bY}{\mathbf Y}
\newcommand{\bD}{\mathbf D}
\newcommand{\bG}{\mathbf G}
\newcommand{\bI}{\mathbf I}
\newcommand{\bH}{\mathbf H}
\newcommand{\bh}{\mathbf h}
\newcommand{\bn}{\mathbf n}
\newcommand{\bs}{\mathbf s}
\newcommand{\bS}{\mathbf S}
\newcommand{\bZ}{\mathbf Z}
\newcommand{\bU}{\mathbf U}
\newcommand{\bbeta}{\bm \beta}
\newcommand{\balpha}{\bm \alpha}
\newcommand{\bell}{\boldsymbol{\ell}}
\newcommand{\bZero}{\boldsymbol 0}
\newcommand{\bOne}{\boldsymbol 1}
\newcommand{\beps}{\boldsymbol{\epsilon}}
\newcommand{\indep}{\perp \!\!\! \perp}
\DeclarePairedDelimiterX{\inp}[2]{\langle}{\rangle}{#1, #2}
\newcommand{\norm}[1]{\left\lVert#1\right\rVert}
\newcommand{\SpecExp}{\textsc{SPEX}}
% The \icmltitle you define below is probably too long as a header.
% Therefore, a short form for the running title is supplied here:
\icmltitlerunning{SpectralExplain}

\begin{document}

\twocolumn[
\icmltitle{\SpecExp: Scaling Feature Interaction Explanations for LLMs}

% It is OKAY to include author information, even for blind
% submissions: the style file will automatically remove it for you
% unless you've provided the [accepted] option to the icml2024
% package.

% List of affiliations: The first argument should be a (short)
% identifier you will use later to specify author affiliations
% Academic affiliations should list Department, University, City, Region, Country
% Industry affiliations should list Company, City, Region, Country

% You can specify symbols, otherwise they are numbered in order.
% Ideally, you should not use this facility. Affiliations will be numbered
% in order of appearance and this is the preferred way.
\icmlsetsymbol{equal}{*}

\begin{icmlauthorlist}
\icmlauthor{Justin Singh Kang}{equal,yyy}
\icmlauthor{Landon Butler}{equal,yyy}
\icmlauthor{Abhineet Agarwal}{equal,xxx}
\icmlauthor{Yigit Efe Erginbas}{yyy}
\icmlauthor{Ramtin Pedarsani}{zzz}
\icmlauthor{Kannan Ramchandran}{yyy}
\icmlauthor{Bin Yu}{yyy,xxx}
%\icmlauthor{Firstname7 Lastname7}{comp}
%\icmlauthor{}{sch}
%\icmlauthor{Firstname8 Lastname8}{sch}
%\icmlauthor{Firstname8 Lastname8}{yyy,comp}
%\icmlauthor{}{sch}
%\icmlauthor{}{sch}
\end{icmlauthorlist}

\icmlaffiliation{yyy}{Department of Electrical Engineering and Computer Science, UC Berkeley}
\icmlaffiliation{xxx}{Department of Statistics, UC Berkeley}
\icmlaffiliation{zzz}{Department of Electrical and Computer Engineering, UC Santa Barbara}

% \icmlaffiliation{comp}{Company Name, Location, Country}
% \icmlaffiliation{sch}{School of ZZZ, Institute of WWW, Location, Country}

\icmlcorrespondingauthor{Justin Singh Kang}{justin\_kang@berkeley.edu}
% \icmlcorrespondingauthor{Firstname2 Lastname2}{first2.last2@www.uk}

% You may prov$ide any keywords that you
% find helpful for describing your paper; these are used to populate
% the "keywords" metadata in the PDF but will not be shown in the document
\icmlkeywords{Machine Learning, ICML}

\vskip 0.3in
]

\printAffiliationsAndNotice{\icmlEqualContribution} 
% otherwise use the standard text.

\begin{abstract}

 
Large language models (LLMs) have revolutionized machine learning due to their ability to capture complex interactions between input features. Popular post-hoc explanation methods like SHAP provide \textit{marginal} feature attributions, while their extensions to interaction importances only scale to small input lengths ($\approx 20$). We propose \emph{Spectral Explainer} (\SpecExp{}), a model-agnostic interaction attribution algorithm that efficiently scales to large input lengths ($\approx 1000)$. \SpecExp{} exploits underlying natural sparsity among interactions---common in real-world data---and applies a sparse Fourier transform using a channel decoding algorithm to efficiently identify important interactions.
We perform experiments across three difficult long-context datasets that require LLMs to utilize interactions between inputs to complete the task. For large inputs, \SpecExp{} outperforms marginal attribution methods by up to 20\% in terms of faithfully reconstructing LLM outputs. Further, \SpecExp{} successfully identifies key features and interactions that strongly influence model output. For one of our datasets, \textit{HotpotQA}, \SpecExp{} provides interactions that align with human annotations. Finally, we use our model-agnostic approach to generate explanations to demonstrate abstract reasoning in closed-source LLMs (\emph{GPT-4o mini}) and compositional reasoning in vision-language models.
\end{abstract}
\section{Introduction}


\begin{figure}[t]
\centering
\includegraphics[width=0.6\columnwidth]{figures/evaluation_desiderata_V5.pdf}
\vspace{-0.5cm}
\caption{\systemName is a platform for conducting realistic evaluations of code LLMs, collecting human preferences of coding models with real users, real tasks, and in realistic environments, aimed at addressing the limitations of existing evaluations.
}
\label{fig:motivation}
\end{figure}

\begin{figure*}[t]
\centering
\includegraphics[width=\textwidth]{figures/system_design_v2.png}
\caption{We introduce \systemName, a VSCode extension to collect human preferences of code directly in a developer's IDE. \systemName enables developers to use code completions from various models. The system comprises a) the interface in the user's IDE which presents paired completions to users (left), b) a sampling strategy that picks model pairs to reduce latency (right, top), and c) a prompting scheme that allows diverse LLMs to perform code completions with high fidelity.
Users can select between the top completion (green box) using \texttt{tab} or the bottom completion (blue box) using \texttt{shift+tab}.}
\label{fig:overview}
\end{figure*}

As model capabilities improve, large language models (LLMs) are increasingly integrated into user environments and workflows.
For example, software developers code with AI in integrated developer environments (IDEs)~\citep{peng2023impact}, doctors rely on notes generated through ambient listening~\citep{oberst2024science}, and lawyers consider case evidence identified by electronic discovery systems~\citep{yang2024beyond}.
Increasing deployment of models in productivity tools demands evaluation that more closely reflects real-world circumstances~\citep{hutchinson2022evaluation, saxon2024benchmarks, kapoor2024ai}.
While newer benchmarks and live platforms incorporate human feedback to capture real-world usage, they almost exclusively focus on evaluating LLMs in chat conversations~\citep{zheng2023judging,dubois2023alpacafarm,chiang2024chatbot, kirk2024the}.
Model evaluation must move beyond chat-based interactions and into specialized user environments.



 

In this work, we focus on evaluating LLM-based coding assistants. 
Despite the popularity of these tools---millions of developers use Github Copilot~\citep{Copilot}---existing
evaluations of the coding capabilities of new models exhibit multiple limitations (Figure~\ref{fig:motivation}, bottom).
Traditional ML benchmarks evaluate LLM capabilities by measuring how well a model can complete static, interview-style coding tasks~\citep{chen2021evaluating,austin2021program,jain2024livecodebench, white2024livebench} and lack \emph{real users}. 
User studies recruit real users to evaluate the effectiveness of LLMs as coding assistants, but are often limited to simple programming tasks as opposed to \emph{real tasks}~\citep{vaithilingam2022expectation,ross2023programmer, mozannar2024realhumaneval}.
Recent efforts to collect human feedback such as Chatbot Arena~\citep{chiang2024chatbot} are still removed from a \emph{realistic environment}, resulting in users and data that deviate from typical software development processes.
We introduce \systemName to address these limitations (Figure~\ref{fig:motivation}, top), and we describe our three main contributions below.


\textbf{We deploy \systemName in-the-wild to collect human preferences on code.} 
\systemName is a Visual Studio Code extension, collecting preferences directly in a developer's IDE within their actual workflow (Figure~\ref{fig:overview}).
\systemName provides developers with code completions, akin to the type of support provided by Github Copilot~\citep{Copilot}. 
Over the past 3 months, \systemName has served over~\completions suggestions from 10 state-of-the-art LLMs, 
gathering \sampleCount~votes from \userCount~users.
To collect user preferences,
\systemName presents a novel interface that shows users paired code completions from two different LLMs, which are determined based on a sampling strategy that aims to 
mitigate latency while preserving coverage across model comparisons.
Additionally, we devise a prompting scheme that allows a diverse set of models to perform code completions with high fidelity.
See Section~\ref{sec:system} and Section~\ref{sec:deployment} for details about system design and deployment respectively.



\textbf{We construct a leaderboard of user preferences and find notable differences from existing static benchmarks and human preference leaderboards.}
In general, we observe that smaller models seem to overperform in static benchmarks compared to our leaderboard, while performance among larger models is mixed (Section~\ref{sec:leaderboard_calculation}).
We attribute these differences to the fact that \systemName is exposed to users and tasks that differ drastically from code evaluations in the past. 
Our data spans 103 programming languages and 24 natural languages as well as a variety of real-world applications and code structures, while static benchmarks tend to focus on a specific programming and natural language and task (e.g. coding competition problems).
Additionally, while all of \systemName interactions contain code contexts and the majority involve infilling tasks, a much smaller fraction of Chatbot Arena's coding tasks contain code context, with infilling tasks appearing even more rarely. 
We analyze our data in depth in Section~\ref{subsec:comparison}.



\textbf{We derive new insights into user preferences of code by analyzing \systemName's diverse and distinct data distribution.}
We compare user preferences across different stratifications of input data (e.g., common versus rare languages) and observe which affect observed preferences most (Section~\ref{sec:analysis}).
For example, while user preferences stay relatively consistent across various programming languages, they differ drastically between different task categories (e.g. frontend/backend versus algorithm design).
We also observe variations in user preference due to different features related to code structure 
(e.g., context length and completion patterns).
We open-source \systemName and release a curated subset of code contexts.
Altogether, our results highlight the necessity of model evaluation in realistic and domain-specific settings.





\section{RELATED WORK}
\label{sec:relatedwork}
In this section, we describe the previous works related to our proposal, which are divided into two parts. In Section~\ref{sec:relatedwork_exoplanet}, we present a review of approaches based on machine learning techniques for the detection of planetary transit signals. Section~\ref{sec:relatedwork_attention} provides an account of the approaches based on attention mechanisms applied in Astronomy.\par

\subsection{Exoplanet detection}
\label{sec:relatedwork_exoplanet}
Machine learning methods have achieved great performance for the automatic selection of exoplanet transit signals. One of the earliest applications of machine learning is a model named Autovetter \citep{MCcauliff}, which is a random forest (RF) model based on characteristics derived from Kepler pipeline statistics to classify exoplanet and false positive signals. Then, other studies emerged that also used supervised learning. \cite{mislis2016sidra} also used a RF, but unlike the work by \citet{MCcauliff}, they used simulated light curves and a box least square \citep[BLS;][]{kovacs2002box}-based periodogram to search for transiting exoplanets. \citet{thompson2015machine} proposed a k-nearest neighbors model for Kepler data to determine if a given signal has similarity to known transits. Unsupervised learning techniques were also applied, such as self-organizing maps (SOM), proposed \citet{armstrong2016transit}; which implements an architecture to segment similar light curves. In the same way, \citet{armstrong2018automatic} developed a combination of supervised and unsupervised learning, including RF and SOM models. In general, these approaches require a previous phase of feature engineering for each light curve. \par

%DL is a modern data-driven technology that automatically extracts characteristics, and that has been successful in classification problems from a variety of application domains. The architecture relies on several layers of NNs of simple interconnected units and uses layers to build increasingly complex and useful features by means of linear and non-linear transformation. This family of models is capable of generating increasingly high-level representations \citep{lecun2015deep}.

The application of DL for exoplanetary signal detection has evolved rapidly in recent years and has become very popular in planetary science.  \citet{pearson2018} and \citet{zucker2018shallow} developed CNN-based algorithms that learn from synthetic data to search for exoplanets. Perhaps one of the most successful applications of the DL models in transit detection was that of \citet{Shallue_2018}; who, in collaboration with Google, proposed a CNN named AstroNet that recognizes exoplanet signals in real data from Kepler. AstroNet uses the training set of labelled TCEs from the Autovetter planet candidate catalog of Q1–Q17 data release 24 (DR24) of the Kepler mission \citep{catanzarite2015autovetter}. AstroNet analyses the data in two views: a ``global view'', and ``local view'' \citep{Shallue_2018}. \par


% The global view shows the characteristics of the light curve over an orbital period, and a local view shows the moment at occurring the transit in detail

%different = space-based

Based on AstroNet, researchers have modified the original AstroNet model to rank candidates from different surveys, specifically for Kepler and TESS missions. \citet{ansdell2018scientific} developed a CNN trained on Kepler data, and included for the first time the information on the centroids, showing that the model improves performance considerably. Then, \citet{osborn2020rapid} and \citet{yu2019identifying} also included the centroids information, but in addition, \citet{osborn2020rapid} included information of the stellar and transit parameters. Finally, \citet{rao2021nigraha} proposed a pipeline that includes a new ``half-phase'' view of the transit signal. This half-phase view represents a transit view with a different time and phase. The purpose of this view is to recover any possible secondary eclipse (the object hiding behind the disk of the primary star).


%last pipeline applies a procedure after the prediction of the model to obtain new candidates, this process is carried out through a series of steps that include the evaluation with Discovery and Validation of Exoplanets (DAVE) \citet{kostov2019discovery} that was adapted for the TESS telescope.\par
%



\subsection{Attention mechanisms in astronomy}
\label{sec:relatedwork_attention}
Despite the remarkable success of attention mechanisms in sequential data, few papers have exploited their advantages in astronomy. In particular, there are no models based on attention mechanisms for detecting planets. Below we present a summary of the main applications of this modeling approach to astronomy, based on two points of view; performance and interpretability of the model.\par
%Attention mechanisms have not yet been explored in all sub-areas of astronomy. However, recent works show a successful application of the mechanism.
%performance

The application of attention mechanisms has shown improvements in the performance of some regression and classification tasks compared to previous approaches. One of the first implementations of the attention mechanism was to find gravitational lenses proposed by \citet{thuruthipilly2021finding}. They designed 21 self-attention-based encoder models, where each model was trained separately with 18,000 simulated images, demonstrating that the model based on the Transformer has a better performance and uses fewer trainable parameters compared to CNN. A novel application was proposed by \citet{lin2021galaxy} for the morphological classification of galaxies, who used an architecture derived from the Transformer, named Vision Transformer (VIT) \citep{dosovitskiy2020image}. \citet{lin2021galaxy} demonstrated competitive results compared to CNNs. Another application with successful results was proposed by \citet{zerveas2021transformer}; which first proposed a transformer-based framework for learning unsupervised representations of multivariate time series. Their methodology takes advantage of unlabeled data to train an encoder and extract dense vector representations of time series. Subsequently, they evaluate the model for regression and classification tasks, demonstrating better performance than other state-of-the-art supervised methods, even with data sets with limited samples.

%interpretation
Regarding the interpretability of the model, a recent contribution that analyses the attention maps was presented by \citet{bowles20212}, which explored the use of group-equivariant self-attention for radio astronomy classification. Compared to other approaches, this model analysed the attention maps of the predictions and showed that the mechanism extracts the brightest spots and jets of the radio source more clearly. This indicates that attention maps for prediction interpretation could help experts see patterns that the human eye often misses. \par

In the field of variable stars, \citet{allam2021paying} employed the mechanism for classifying multivariate time series in variable stars. And additionally, \citet{allam2021paying} showed that the activation weights are accommodated according to the variation in brightness of the star, achieving a more interpretable model. And finally, related to the TESS telescope, \citet{morvan2022don} proposed a model that removes the noise from the light curves through the distribution of attention weights. \citet{morvan2022don} showed that the use of the attention mechanism is excellent for removing noise and outliers in time series datasets compared with other approaches. In addition, the use of attention maps allowed them to show the representations learned from the model. \par

Recent attention mechanism approaches in astronomy demonstrate comparable results with earlier approaches, such as CNNs. At the same time, they offer interpretability of their results, which allows a post-prediction analysis. \par


\section{Method}\label{sec:method}
\begin{figure}
    \centering
    \includegraphics[width=0.85\textwidth]{imgs/heatmap_acc.pdf}
    \caption{\textbf{Visualization of the proposed periodic Bayesian flow with mean parameter $\mu$ and accumulated accuracy parameter $c$ which corresponds to the entropy/uncertainty}. For $x = 0.3, \beta(1) = 1000$ and $\alpha_i$ defined in \cref{appd:bfn_cir}, this figure plots three colored stochastic parameter trajectories for receiver mean parameter $m$ and accumulated accuracy parameter $c$, superimposed on a log-scale heatmap of the Bayesian flow distribution $p_F(m|x,\senderacc)$ and $p_F(c|x,\senderacc)$. Note the \emph{non-monotonicity} and \emph{non-additive} property of $c$ which could inform the network the entropy of the mean parameter $m$ as a condition and the \emph{periodicity} of $m$. %\jj{Shrink the figures to save space}\hanlin{Do we need to make this figure one-column?}
    }
    \label{fig:vmbf_vis}
    \vskip -0.1in
\end{figure}
% \begin{wrapfigure}{r}{0.5\textwidth}
%     \centering
%     \includegraphics[width=0.49\textwidth]{imgs/heatmap_acc.pdf}
%     \caption{\textbf{Visualization of hyper-torus Bayesian flow based on von Mises Distribution}. For $x = 0.3, \beta(1) = 1000$ and $\alpha_i$ defined in \cref{appd:bfn_cir}, this figure plots three colored stochastic parameter trajectories for receiver mean parameter $m$ and accumulated accuracy parameter $c$, superimposed on a log-scale heatmap of the Bayesian flow distribution $p_F(m|x,\senderacc)$ and $p_F(c|x,\senderacc)$. Note the \emph{non-monotonicity} and \emph{non-additive} property of $c$. \jj{Shrink the figures to save space}}
%     \label{fig:vmbf_vis}
%     \vspace{-30pt}
% \end{wrapfigure}


In this section, we explain the detailed design of CrysBFN tackling theoretical and practical challenges. First, we describe how to derive our new formulation of Bayesian Flow Networks over hyper-torus $\mathbb{T}^{D}$ from scratch. Next, we illustrate the two key differences between \modelname and the original form of BFN: $1)$ a meticulously designed novel base distribution with different Bayesian update rules; and $2)$ different properties over the accuracy scheduling resulted from the periodicity and the new Bayesian update rules. Then, we present in detail the overall framework of \modelname over each manifold of the crystal space (\textit{i.e.} fractional coordinates, lattice vectors, atom types) respecting \textit{periodic E(3) invariance}. 

% In this section, we first demonstrate how to build Bayesian flow on hyper-torus $\mathbb{T}^{D}$ by overcoming theoretical and practical problems to provide a low-noise parameter-space approach to fractional atom coordinate generation. Next, we present how \modelname models each manifold of crystal space respecting \textit{periodic E(3) invariance}. 

\subsection{Periodic Bayesian Flow on Hyper-torus \texorpdfstring{$\mathbb{T}^{D}$}{}} 
For generative modeling of fractional coordinates in crystal, we first construct a periodic Bayesian flow on \texorpdfstring{$\mathbb{T}^{D}$}{} by designing every component of the totally new Bayesian update process which we demonstrate to be distinct from the original Bayesian flow (please see \cref{fig:non_add}). 
 %:) 
 
 The fractional atom coordinate system \citep{jiao2023crystal} inherently distributes over a hyper-torus support $\mathbb{T}^{3\times N}$. Hence, the normal distribution support on $\R$ used in the original \citep{bfn} is not suitable for this scenario. 
% The key problem of generative modeling for crystal is the periodicity of Cartesian atom coordinates $\vX$ requiring:
% \begin{equation}\label{eq:periodcity}
% p(\vA,\vL,\vX)=p(\vA,\vL,\vX+\vec{LK}),\text{where}~\vec{K}=\vec{k}\vec{1}_{1\times N},\forall\vec{k}\in\mathbb{Z}^{3\times1}
% \end{equation}
% However, there does not exist such a distribution supporting on $\R$ to model such property because the integration of such distribution over $\R$ will not be finite and equal to 1. Therefore, the normal distribution used in \citet{bfn} can not meet this condition.

To tackle this problem, the circular distribution~\citep{mardia2009directional} over the finite interval $[-\pi,\pi)$ is a natural choice as the base distribution for deriving the BFN on $\mathbb{T}^D$. 
% one natural choice is to 
% we would like to consider the circular distribution over the finite interval as the base 
% we find that circular distributions \citep{mardia2009directional} defined on a finite interval with lengths of $2\pi$ can be used as the instantiation of input distribution for the BFN on $\mathbb{T}^D$.
Specifically, circular distributions enjoy desirable periodic properties: $1)$ the integration over any interval length of $2\pi$ equals 1; $2)$ the probability distribution function is periodic with period $2\pi$.  Sharing the same intrinsic with fractional coordinates, such periodic property of circular distribution makes it suitable for the instantiation of BFN's input distribution, in parameterizing the belief towards ground truth $\x$ on $\mathbb{T}^D$. 
% \yuxuan{this is very complicated from my perspective.} \hanlin{But this property is exactly beautiful and perfectly fit into the BFN.}

\textbf{von Mises Distribution and its Bayesian Update} We choose von Mises distribution \citep{mardia2009directional} from various circular distributions as the form of input distribution, based on the appealing conjugacy property required in the derivation of the BFN framework.
% to leverage the Bayesian conjugacy property of von Mises distribution which is required by the BFN framework. 
That is, the posterior of a von Mises distribution parameterized likelihood is still in the family of von Mises distributions. The probability density function of von Mises distribution with mean direction parameter $m$ and concentration parameter $c$ (describing the entropy/uncertainty of $m$) is defined as: 
\begin{equation}
f(x|m,c)=vM(x|m,c)=\frac{\exp(c\cos(x-m))}{2\pi I_0(c)}
\end{equation}
where $I_0(c)$ is zeroth order modified Bessel function of the first kind as the normalizing constant. Given the last univariate belief parameterized by von Mises distribution with parameter $\theta_{i-1}=\{m_{i-1},\ c_{i-1}\}$ and the sample $y$ from sender distribution with unknown data sample $x$ and known accuracy $\alpha$ describing the entropy/uncertainty of $y$,  Bayesian update for the receiver is deducted as:
\begin{equation}
 h(\{m_{i-1},c_{i-1}\},y,\alpha)=\{m_i,c_i \}, \text{where}
\end{equation}
\begin{equation}\label{eq:h_m}
m_i=\text{atan2}(\alpha\sin y+c_{i-1}\sin m_{i-1}, {\alpha\cos y+c_{i-1}\cos m_{i-1}})
\end{equation}
\begin{equation}\label{eq:h_c}
c_i =\sqrt{\alpha^2+c_{i-1}^2+2\alpha c_{i-1}\cos(y-m_{i-1})}
\end{equation}
The proof of the above equations can be found in \cref{apdx:bayesian_update_function}. The atan2 function refers to  2-argument arctangent. Independently conducting  Bayesian update for each dimension, we can obtain the Bayesian update distribution by marginalizing $\y$:
\begin{equation}
p_U(\vtheta'|\vtheta,\bold{x};\alpha)=\mathbb{E}_{p_S(\bold{y}|\bold{x};\alpha)}\delta(\vtheta'-h(\vtheta,\bold{y},\alpha))=\mathbb{E}_{vM(\bold{y}|\bold{x},\alpha)}\delta(\vtheta'-h(\vtheta,\bold{y},\alpha))
\end{equation} 
\begin{figure}
    \centering
    \vskip -0.15in
    \includegraphics[width=0.95\linewidth]{imgs/non_add.pdf}
    \caption{An intuitive illustration of non-additive accuracy Bayesian update on the torus. The lengths of arrows represent the uncertainty/entropy of the belief (\emph{e.g.}~$1/\sigma^2$ for Gaussian and $c$ for von Mises). The directions of the arrows represent the believed location (\emph{e.g.}~ $\mu$ for Gaussian and $m$ for von Mises).}
    \label{fig:non_add}
    \vskip -0.15in
\end{figure}
\textbf{Non-additive Accuracy} 
The additive accuracy is a nice property held with the Gaussian-formed sender distribution of the original BFN expressed as:
\begin{align}
\label{eq:standard_id}
    \update(\parsn{}'' \mid \parsn{}, \x; \alpha_a+\alpha_b) = \E_{\update(\parsn{}' \mid \parsn{}, \x; \alpha_a)} \update(\parsn{}'' \mid \parsn{}', \x; \alpha_b)
\end{align}
Such property is mainly derived based on the standard identity of Gaussian variable:
\begin{equation}
X \sim \mathcal{N}\left(\mu_X, \sigma_X^2\right), Y \sim \mathcal{N}\left(\mu_Y, \sigma_Y^2\right) \Longrightarrow X+Y \sim \mathcal{N}\left(\mu_X+\mu_Y, \sigma_X^2+\sigma_Y^2\right)
\end{equation}
The additive accuracy property makes it feasible to derive the Bayesian flow distribution $
p_F(\boldsymbol{\theta} \mid \mathbf{x} ; i)=p_U\left(\boldsymbol{\theta} \mid \boldsymbol{\theta}_0, \mathbf{x}, \sum_{k=1}^{i} \alpha_i \right)
$ for the simulation-free training of \cref{eq:loss_n}.
It should be noted that the standard identity in \cref{eq:standard_id} does not hold in the von Mises distribution. Hence there exists an important difference between the original Bayesian flow defined on Euclidean space and the Bayesian flow of circular data on $\mathbb{T}^D$ based on von Mises distribution. With prior $\btheta = \{\bold{0},\bold{0}\}$, we could formally represent the non-additive accuracy issue as:
% The additive accuracy property implies the fact that the "confidence" for the data sample after observing a series of the noisy samples with accuracy ${\alpha_1, \cdots, \alpha_i}$ could be  as the accuracy sum  which could be  
% Here we 
% Here we emphasize the specific property of BFN based on von Mises distribution.
% Note that 
% \begin{equation}
% \update(\parsn'' \mid \parsn, \x; \alpha_a+\alpha_b) \ne \E_{\update(\parsn' \mid \parsn, \x; \alpha_a)} \update(\parsn'' \mid \parsn', \x; \alpha_b)
% \end{equation}
% \oyyw{please check whether the below equation is better}
% \yuxuan{I fill somehow confusing on what is the update distribution with $\alpha$. }
% \begin{equation}
% \update(\parsn{}'' \mid \parsn{}, \x; \alpha_a+\alpha_b) \ne \E_{\update(\parsn{}' \mid \parsn{}, \x; \alpha_a)} \update(\parsn{}'' \mid \parsn{}', \x; \alpha_b)
% \end{equation}
% We give an intuitive visualization of such difference in \cref{fig:non_add}. The untenability of this property can materialize by considering the following case: with prior $\btheta = \{\bold{0},\bold{0}\}$, check the two-step Bayesian update distribution with $\alpha_a,\alpha_b$ and one-step Bayesian update with $\alpha=\alpha_a+\alpha_b$:
\begin{align}
\label{eq:nonadd}
     &\update(c'' \mid \parsn, \x; \alpha_a+\alpha_b)  = \delta(c-\alpha_a-\alpha_b)
     \ne  \mathbb{E}_{p_U(\parsn' \mid \parsn, \x; \alpha_a)}\update(c'' \mid \parsn', \x; \alpha_b) \nonumber \\&= \mathbb{E}_{vM(\bold{y}_b|\bold{x},\alpha_a)}\mathbb{E}_{vM(\bold{y}_a|\bold{x},\alpha_b)}\delta(c-||[\alpha_a \cos\y_a+\alpha_b\cos \y_b,\alpha_a \sin\y_a+\alpha_b\sin \y_b]^T||_2)
\end{align}
A more intuitive visualization could be found in \cref{fig:non_add}. This fundamental difference between periodic Bayesian flow and that of \citet{bfn} presents both theoretical and practical challenges, which we will explain and address in the following contents.

% This makes constructing Bayesian flow based on von Mises distribution intrinsically different from previous Bayesian flows (\citet{bfn}).

% Thus, we must reformulate the framework of Bayesian flow networks  accordingly. % and do necessary reformulations of BFN. 

% \yuxuan{overall I feel this part is complicated by using the language of update distribution. I would like to suggest simply use bayesian update, to provide intuitive explantion.}\hanlin{See the illustration in \cref{fig:non_add}}

% That introduces a cascade of problems, and we investigate the following issues: $(1)$ Accuracies between sender and receiver are not synchronized and need to be differentiated. $(2)$ There is no tractable Bayesian flow distribution for a one-step sample conditioned on a given time step $i$, and naively simulating the Bayesian flow results in computational overhead. $(3)$ It is difficult to control the entropy of the Bayesian flow. $(4)$ Accuracy is no longer a function of $t$ and becomes a distribution conditioned on $t$, which can be different across dimensions.
%\jj{Edited till here}

\textbf{Entropy Conditioning} As a common practice in generative models~\citep{ddpm,flowmatching,bfn}, timestep $t$ is widely used to distinguish among generation states by feeding the timestep information into the networks. However, this paper shows that for periodic Bayesian flow, the accumulated accuracy $\vc_i$ is more effective than time-based conditioning by informing the network about the entropy and certainty of the states $\parsnt{i}$. This stems from the intrinsic non-additive accuracy which makes the receiver's accumulated accuracy $c$ not bijective function of $t$, but a distribution conditioned on accumulated accuracies $\vc_i$ instead. Therefore, the entropy parameter $\vc$ is taken logarithm and fed into the network to describe the entropy of the input corrupted structure. We verify this consideration in \cref{sec:exp_ablation}. 
% \yuxuan{implement variant. traditionally, the timestep is widely used to distinguish the different states by putting the timestep embedding into the networks. citation of FM, diffusion, BFN. However, we find that conditioned on time in periodic flow could not provide extra benefits. To further boost the performance, we introduce a simple yet effective modification term entropy conditional. This is based on that the accumulated accuracy which represents the current uncertainty or entropy could be a better indicator to distinguish different states. + Describe how you do this. }



\textbf{Reformulations of BFN}. Recall the original update function with Gaussian sender distribution, after receiving noisy samples $\y_1,\y_2,\dots,\y_i$ with accuracies $\senderacc$, the accumulated accuracies of the receiver side could be analytically obtained by the additive property and it is consistent with the sender side.
% Since observing sample $\y$ with $\alpha_i$ can not result in exact accuracy increment $\alpha_i$ for receiver, the accuracies between sender and receiver are not synchronized which need to be differentiated. 
However, as previously mentioned, this does not apply to periodic Bayesian flow, and some of the notations in original BFN~\citep{bfn} need to be adjusted accordingly. We maintain the notations of sender side's one-step accuracy $\alpha$ and added accuracy $\beta$, and alter the notation of receiver's accuracy parameter as $c$, which is needed to be simulated by cascade of Bayesian updates. We emphasize that the receiver's accumulated accuracy $c$ is no longer a function of $t$ (differently from the Gaussian case), and it becomes a distribution conditioned on received accuracies $\senderacc$ from the sender. Therefore, we represent the Bayesian flow distribution of von Mises distribution as $p_F(\btheta|\x;\alpha_1,\alpha_2,\dots,\alpha_i)$. And the original simulation-free training with Bayesian flow distribution is no longer applicable in this scenario.
% Different from previous BFNs where the accumulated accuracy $\rho$ is not explicitly modeled, the accumulated accuracy parameter $c$ (visualized in \cref{fig:vmbf_vis}) needs to be explicitly modeled by feeding it to the network to avoid information loss.
% the randomaccuracy parameter $c$ (visualized in \cref{fig:vmbf_vis}) implies that there exists information in $c$ from the sender just like $m$, meaning that $c$ also should be fed into the network to avoid information loss. 
% We ablate this consideration in  \cref{sec:exp_ablation}. 

\textbf{Fast Sampling from Equivalent Bayesian Flow Distribution} Based on the above reformulations, the Bayesian flow distribution of von Mises distribution is reframed as: 
\begin{equation}\label{eq:flow_frac}
p_F(\btheta_i|\x;\alpha_1,\alpha_2,\dots,\alpha_i)=\E_{\update(\parsnt{1} \mid \parsnt{0}, \x ; \alphat{1})}\dots\E_{\update(\parsn_{i-1} \mid \parsnt{i-2}, \x; \alphat{i-1})} \update(\parsnt{i} | \parsnt{i-1},\x;\alphat{i} )
\end{equation}
Naively sampling from \cref{eq:flow_frac} requires slow auto-regressive iterated simulation, making training unaffordable. Noticing the mathematical properties of \cref{eq:h_m,eq:h_c}, we  transform \cref{eq:flow_frac} to the equivalent form:
\begin{equation}\label{eq:cirflow_equiv}
p_F(\vec{m}_i|\x;\alpha_1,\alpha_2,\dots,\alpha_i)=\E_{vM(\y_1|\x,\alpha_1)\dots vM(\y_i|\x,\alpha_i)} \delta(\vec{m}_i-\text{atan2}(\sum_{j=1}^i \alpha_j \cos \y_j,\sum_{j=1}^i \alpha_j \sin \y_j))
\end{equation}
\begin{equation}\label{eq:cirflow_equiv2}
p_F(\vec{c}_i|\x;\alpha_1,\alpha_2,\dots,\alpha_i)=\E_{vM(\y_1|\x,\alpha_1)\dots vM(\y_i|\x,\alpha_i)}  \delta(\vec{c}_i-||[\sum_{j=1}^i \alpha_j \cos \y_j,\sum_{j=1}^i \alpha_j \sin \y_j]^T||_2)
\end{equation}
which bypasses the computation of intermediate variables and allows pure tensor operations, with negligible computational overhead.
\begin{restatable}{proposition}{cirflowequiv}
The probability density function of Bayesian flow distribution defined by \cref{eq:cirflow_equiv,eq:cirflow_equiv2} is equivalent to the original definition in \cref{eq:flow_frac}. 
\end{restatable}
\textbf{Numerical Determination of Linear Entropy Sender Accuracy Schedule} ~Original BFN designs the accuracy schedule $\beta(t)$ to make the entropy of input distribution linearly decrease. As for crystal generation task, to ensure information coherence between modalities, we choose a sender accuracy schedule $\senderacc$ that makes the receiver's belief entropy $H(t_i)=H(p_I(\cdot|\vtheta_i))=H(p_I(\cdot|\vc_i))$ linearly decrease \emph{w.r.t.} time $t_i$, given the initial and final accuracy parameter $c(0)$ and $c(1)$. Due to the intractability of \cref{eq:vm_entropy}, we first use numerical binary search in $[0,c(1)]$ to determine the receiver's $c(t_i)$ for $i=1,\dots, n$ by solving the equation $H(c(t_i))=(1-t_i)H(c(0))+tH(c(1))$. Next, with $c(t_i)$, we conduct numerical binary search for each $\alpha_i$ in $[0,c(1)]$ by solving the equations $\E_{y\sim vM(x,\alpha_i)}[\sqrt{\alpha_i^2+c_{i-1}^2+2\alpha_i c_{i-1}\cos(y-m_{i-1})}]=c(t_i)$ from $i=1$ to $i=n$ for arbitrarily selected $x\in[-\pi,\pi)$.

After tackling all those issues, we have now arrived at a new BFN architecture for effectively modeling crystals. Such BFN can also be adapted to other type of data located in hyper-torus $\mathbb{T}^{D}$.

\subsection{Equivariant Bayesian Flow for Crystal}
With the above Bayesian flow designed for generative modeling of fractional coordinate $\vF$, we are able to build equivariant Bayesian flow for each modality of crystal. In this section, we first give an overview of the general training and sampling algorithm of \modelname (visualized in \cref{fig:framework}). Then, we describe the details of the Bayesian flow of every modality. The training and sampling algorithm can be found in \cref{alg:train} and \cref{alg:sampling}.

\textbf{Overview} Operating in the parameter space $\bthetaM=\{\bthetaA,\bthetaL,\bthetaF\}$, \modelname generates high-fidelity crystals through a joint BFN sampling process on the parameter of  atom type $\bthetaA$, lattice parameter $\vec{\theta}^L=\{\bmuL,\brhoL\}$, and the parameter of fractional coordinate matrix $\bthetaF=\{\bmF,\bcF\}$. We index the $n$-steps of the generation process in a discrete manner $i$, and denote the corresponding continuous notation $t_i=i/n$ from prior parameter $\thetaM_0$ to a considerably low variance parameter $\thetaM_n$ (\emph{i.e.} large $\vrho^L,\bmF$, and centered $\bthetaA$).

At training time, \modelname samples time $i\sim U\{1,n\}$ and $\bthetaM_{i-1}$ from the Bayesian flow distribution of each modality, serving as the input to the network. The network $\net$ outputs $\net(\parsnt{i-1}^\mathcal{M},t_{i-1})=\net(\parsnt{i-1}^A,\parsnt{i-1}^F,\parsnt{i-1}^L,t_{i-1})$ and conducts gradient descents on loss function \cref{eq:loss_n} for each modality. After proper training, the sender distribution $p_S$ can be approximated by the receiver distribution $p_R$. 

At inference time, from predefined $\thetaM_0$, we conduct transitions from $\thetaM_{i-1}$ to $\thetaM_{i}$ by: $(1)$ sampling $\y_i\sim p_R(\bold{y}|\thetaM_{i-1};t_i,\alpha_i)$ according to network prediction $\predM{i-1}$; and $(2)$ performing Bayesian update $h(\thetaM_{i-1},\y^\calM_{i-1},\alpha_i)$ for each dimension. 

% Alternatively, we complete this transition using the flow-back technique by sampling 
% $\thetaM_{i}$ from Bayesian flow distribution $\flow(\btheta^M_{i}|\predM{i-1};t_{i-1})$. 

% The training objective of $\net$ is to minimize the KL divergence between sender distribution and receiver distribution for every modality as defined in \cref{eq:loss_n} which is equivalent to optimizing the negative variational lower bound $\calL^{VLB}$ as discussed in \cref{sec:preliminaries}. 

%In the following part, we will present the Bayesian flow of each modality in detail.

\textbf{Bayesian Flow of Fractional Coordinate $\vF$}~The distribution of the prior parameter $\bthetaF_0$ is defined as:
\begin{equation}\label{eq:prior_frac}
    p(\bthetaF_0) \defeq \{vM(\vm_0^F|\vec{0}_{3\times N},\vec{0}_{3\times N}),\delta(\vc_0^F-\vec{0}_{3\times N})\} = \{U(\vec{0},\vec{1}),\delta(\vc_0^F-\vec{0}_{3\times N})\}
\end{equation}
Note that this prior distribution of $\vm_0^F$ is uniform over $[\vec{0},\vec{1})$, ensuring the periodic translation invariance property in \cref{De:pi}. The training objective is minimizing the KL divergence between sender and receiver distribution (deduction can be found in \cref{appd:cir_loss}): 
%\oyyw{replace $\vF$ with $\x$?} \hanlin{notations follow Preliminary?}
\begin{align}\label{loss_frac}
\calL_F = n \E_{i \sim \ui{n}, \flow(\parsn{}^F \mid \vF ; \senderacc)} \alpha_i\frac{I_1(\alpha_i)}{I_0(\alpha_i)}(1-\cos(\vF-\predF{i-1}))
\end{align}
where $I_0(x)$ and $I_1(x)$ are the zeroth and the first order of modified Bessel functions. The transition from $\bthetaF_{i-1}$ to $\bthetaF_{i}$ is the Bayesian update distribution based on network prediction:
\begin{equation}\label{eq:transi_frac}
    p(\btheta^F_{i}|\parsnt{i-1}^\calM)=\mathbb{E}_{vM(\bold{y}|\predF{i-1},\alpha_i)}\delta(\btheta^F_{i}-h(\btheta^F_{i-1},\bold{y},\alpha_i))
\end{equation}
\begin{restatable}{proposition}{fracinv}
With $\net_{F}$ as a periodic translation equivariant function namely $\net_F(\parsnt{}^A,w(\parsnt{}^F+\vt),\parsnt{}^L,t)=w(\net_F(\parsnt{}^A,\parsnt{}^F,\parsnt{}^L,t)+\vt), \forall\vt\in\R^3$, the marginal distribution of $p(\vF_n)$ defined by \cref{eq:prior_frac,eq:transi_frac} is periodic translation invariant. 
\end{restatable}
\textbf{Bayesian Flow of Lattice Parameter \texorpdfstring{$\boldsymbol{L}$}{}}   
Noting the lattice parameter $\bm{L}$ located in Euclidean space, we set prior as the parameter of a isotropic multivariate normal distribution $\btheta^L_0\defeq\{\vmu_0^L,\vrho_0^L\}=\{\bm{0}_{3\times3},\bm{1}_{3\times3}\}$
% \begin{equation}\label{eq:lattice_prior}
% \btheta^L_0\defeq\{\vmu_0^L,\vrho_0^L\}=\{\bm{0}_{3\times3},\bm{1}_{3\times3}\}
% \end{equation}
such that the prior distribution of the Markov process on $\vmu^L$ is the Dirac distribution $\delta(\vec{\mu_0}-\vec{0})$ and $\delta(\vec{\rho_0}-\vec{1})$, 
% \begin{equation}
%     p_I^L(\boldsymbol{L}|\btheta_0^L)=\mathcal{N}(\bm{L}|\bm{0},\bm{I})
% \end{equation}
which ensures O(3)-invariance of prior distribution of $\vL$. By Eq. 77 from \citet{bfn}, the Bayesian flow distribution of the lattice parameter $\bm{L}$ is: 
\begin{align}% =p_U(\bmuL|\btheta_0^L,\bm{L},\beta(t))
p_F^L(\bmuL|\bm{L};t) &=\mathcal{N}(\bmuL|\gamma(t)\bm{L},\gamma(t)(1-\gamma(t))\bm{I}) 
\end{align}
where $\gamma(t) = 1 - \sigma_1^{2t}$ and $\sigma_1$ is the predefined hyper-parameter controlling the variance of input distribution at $t=1$ under linear entropy accuracy schedule. The variance parameter $\vrho$ does not need to be modeled and fed to the network, since it is deterministic given the accuracy schedule. After sampling $\bmuL_i$ from $p_F^L$, the training objective is defined as minimizing KL divergence between sender and receiver distribution (based on Eq. 96 in \citet{bfn}):
\begin{align}
\mathcal{L}_{L} = \frac{n}{2}\left(1-\sigma_1^{2/n}\right)\E_{i \sim \ui{n}}\E_{\flow(\bmuL_{i-1} |\vL ; t_{i-1})}  \frac{\left\|\vL -\predL{i-1}\right\|^2}{\sigma_1^{2i/n}},\label{eq:lattice_loss}
\end{align}
where the prediction term $\predL{i-1}$ is the lattice parameter part of network output. After training, the generation process is defined as the Bayesian update distribution given network prediction:
\begin{equation}\label{eq:lattice_sampling}
    p(\bmuL_{i}|\parsnt{i-1}^\calM)=\update^L(\bmuL_{i}|\predL{i-1},\bmuL_{i-1};t_{i-1})
\end{equation}
    

% The final prediction of the lattice parameter is given by $\bmuL_n = \predL{n-1}$.
% \begin{equation}\label{eq:final_lattice}
%     \bmuL_n = \predL{n-1}
% \end{equation}

\begin{restatable}{proposition}{latticeinv}\label{prop:latticeinv}
With $\net_{L}$ as  O(3)-equivariant function namely $\net_L(\parsnt{}^A,\parsnt{}^F,\vQ\parsnt{}^L,t)=\vQ\net_L(\parsnt{}^A,\parsnt{}^F,\parsnt{}^L,t),\forall\vQ^T\vQ=\vI$, the marginal distribution of $p(\bmuL_n)$ defined by \cref{eq:lattice_sampling} is O(3)-invariant. 
\end{restatable}


\textbf{Bayesian Flow of Atom Types \texorpdfstring{$\boldsymbol{A}$}{}} 
Given that atom types are discrete random variables located in a simplex $\calS^K$, the prior parameter of $\boldsymbol{A}$ is the discrete uniform distribution over the vocabulary $\parsnt{0}^A \defeq \frac{1}{K}\vec{1}_{1\times N}$. 
% \begin{align}\label{eq:disc_input_prior}
% \parsnt{0}^A \defeq \frac{1}{K}\vec{1}_{1\times N}
% \end{align}
% \begin{align}
%     (\oh{j}{K})_k \defeq \delta_{j k}, \text{where }\oh{j}{K}\in \R^{K},\oh{\vA}{KD} \defeq \left(\oh{a_1}{K},\dots,\oh{a_N}{K}\right) \in \R^{K\times N}
% \end{align}
With the notation of the projection from the class index $j$ to the length $K$ one-hot vector $ (\oh{j}{K})_k \defeq \delta_{j k}, \text{where }\oh{j}{K}\in \R^{K},\oh{\vA}{KD} \defeq \left(\oh{a_1}{K},\dots,\oh{a_N}{K}\right) \in \R^{K\times N}$, the Bayesian flow distribution of atom types $\vA$ is derived in \citet{bfn}:
\begin{align}
\flow^{A}(\parsn^A \mid \vA; t) &= \E_{\N{\y \mid \beta^A(t)\left(K \oh{\vA}{K\times N} - \vec{1}_{K\times N}\right)}{\beta^A(t) K \vec{I}_{K\times N \times N}}} \delta\left(\parsn^A - \frac{e^{\y}\parsnt{0}^A}{\sum_{k=1}^K e^{\y_k}(\parsnt{0})_{k}^A}\right).
\end{align}
where $\beta^A(t)$ is the predefined accuracy schedule for atom types. Sampling $\btheta_i^A$ from $p_F^A$ as the training signal, the training objective is the $n$-step discrete-time loss for discrete variable \citep{bfn}: 
% \oyyw{can we simplify the next equation? Such as remove $K \times N, K \times N \times N$}
% \begin{align}
% &\calL_A = n\E_{i \sim U\{1,n\},\flow^A(\parsn^A \mid \vA ; t_{i-1}),\N{\y \mid \alphat{i}\left(K \oh{\vA}{KD} - \vec{1}_{K\times N}\right)}{\alphat{i} K \vec{I}_{K\times N \times N}}} \ln \N{\y \mid \alphat{i}\left(K \oh{\vA}{K\times N} - \vec{1}_{K\times N}\right)}{\alphat{i} K \vec{I}_{K\times N \times N}}\nonumber\\
% &\qquad\qquad\qquad-\sum_{d=1}^N \ln \left(\sum_{k=1}^K \out^{(d)}(k \mid \parsn^A; t_{i-1}) \N{\ydd{d} \mid \alphat{i}\left(K\oh{k}{K}- \vec{1}_{K\times N}\right)}{\alphat{i} K \vec{I}_{K\times N \times N}}\right)\label{discdisc_t_loss_exp}
% \end{align}
\begin{align}
&\calL_A = n\E_{i \sim U\{1,n\},\flow^A(\parsn^A \mid \vA ; t_{i-1}),\N{\y \mid \alphat{i}\left(K \oh{\vA}{KD} - \vec{1}\right)}{\alphat{i} K \vec{I}}} \ln \N{\y \mid \alphat{i}\left(K \oh{\vA}{K\times N} - \vec{1}\right)}{\alphat{i} K \vec{I}}\nonumber\\
&\qquad\qquad\qquad-\sum_{d=1}^N \ln \left(\sum_{k=1}^K \out^{(d)}(k \mid \parsn^A; t_{i-1}) \N{\ydd{d} \mid \alphat{i}\left(K\oh{k}{K}- \vec{1}\right)}{\alphat{i} K \vec{I}}\right)\label{discdisc_t_loss_exp}
\end{align}
where $\vec{I}\in \R^{K\times N \times N}$ and $\vec{1}\in\R^{K\times D}$. When sampling, the transition from $\bthetaA_{i-1}$ to $\bthetaA_{i}$ is derived as:
\begin{equation}
    p(\btheta^A_{i}|\parsnt{i-1}^\calM)=\update^A(\btheta^A_{i}|\btheta^A_{i-1},\predA{i-1};t_{i-1})
\end{equation}

The detailed training and sampling algorithm could be found in \cref{alg:train} and \cref{alg:sampling}.




\begin{figure*}[t]
\centering
\begin{center} 
  \includegraphics[width=.63\textwidth]{figures/fidelity_legend.pdf}
\end{center}
\begin{subfigure}[b]{.3\textwidth}
  \centering
  \includegraphics[width=\linewidth]{figures/sentiment_fidelity3_small.pdf}
  \caption{\emph{Sentiment} $n\in [32,63]$}
  \label{fig:fidelity_sentiment}
\end{subfigure}%
\begin{subfigure}[b]{.3\textwidth}
  \centering
  \includegraphics[width=\linewidth]{figures/drop_fidelity_small.pdf}
  \caption{\emph{DROP} $n\in [32,63]$}
  \label{fig:fidelity_drop}
\end{subfigure}%
\begin{subfigure}[b]{.3\textwidth}
  \centering
  \includegraphics[width=\linewidth]{figures/hotpot_fidelity_small.pdf}
  \caption{\emph{HotpotQA} $n\in [32,63]$}
  \label{fig:fidelity_hotpot}
\end{subfigure}
\begin{subfigure}[b]{.3\textwidth}
  \centering
  \includegraphics[width=\linewidth]{figures/sentiment_fidelity3_large.pdf}
  \caption{\emph{Sentiment} $n\in [64,127]$}
  \label{fig:fidelity_sentiment2}
\end{subfigure}%
\begin{subfigure}[b]{.3\textwidth}
  \centering
  \includegraphics[width=\linewidth]{figures/drop_fidelity_large.pdf}
  \caption{\emph{DROP} $n\in [64,127]$}
  \label{fig:fidelity_drop2}
\end{subfigure}%
\begin{subfigure}[b]{.3\textwidth}
  \centering
  \includegraphics[width=\linewidth]{figures/hotpot_fidelity_large.pdf}
  \caption{\emph{HotpotQA} $n\in [64,127]$}
  \label{fig:fidelity_hotpot2}
\end{subfigure}
\vspace{-5pt}
\caption{On the removal task, \SpecExp{} performs competitively with 2\textsuperscript{nd} order methods on the \emph{Sentiment} dataset, and out-performs all approaches on \emph{DROP} and  \emph{HotpotQA} dataset for $n \in [32,63]$. When $n$ is too large to compute other interaction indices, we outperform marginal methods.}
\label{fig:fidelity}
\vspace{-12pt}
\end{figure*}

\vspace{-14pt}
\section{Experiments}
\label{sec:language}

\paragraph{Datasets} 
We use three popular datasets that require the LLM to understand interactions between features. 
\begin{enumerate}[ topsep=0pt, itemsep=0pt, leftmargin=*]
\item \emph{Sentiment} is primarily composed of the \emph{Large Movie Review Dataset} \cite{maas-EtAl:2011:ACL-HLT2011}, which contains both positive and negative IMDb movie reviews. The dataset is augmented with examples from the \emph{SST} dataset \cite{ socher2013recursive} to ensure coverage for small $n$. We treat the words of the reviews as the input features.
\item{\emph{HotpotQA} \cite{yang2018hotpotqa} is a question-answering dataset requiring multi-hop reasoning over multiple Wikipedia articles to answer complex questions. We use the sentences of the articles as the input features.}
\item{\emph{Discrete Reasoning Over Paragraphs} (DROP)} \cite{dua2019drop} is a comprehension benchmark requiring discrete reasoning operations like addition, counting, and sorting over paragraph-level content to answer questions. We use the words of the paragraphs as the input features. 
\end{enumerate}
%
%\emph{DROP} and \emph{HotpotQA} require , while \emph{Sentiment} is encoder-only. 
%
\vspace{-7pt}
\paragraph{Models} For \textit{DROP} and \textit{HotpotQA}, (generative question-answering tasks) we use \texttt{Llama-3.2-3B-Instruct} \cite{grattafiori2024llama3herdmodels} with $8$-bit quantization. For \emph{Sentiment} (classification), we use the encoder-only fine-tuned \texttt{DistilBERT} model \cite{Sanh2019DistilBERTAD,sentimentBert}.
%

\vspace{-7pt}
\paragraph{Baselines} We compare against popular marginal metrics LIME, SHAP, and the Banzhaf value. 
%
For interaction indices, we consider Faith-Shapley, Faith-Banzhaf, and the Shapley-Taylor Index. We compute all benchmarks where computationally feasible. That is, we always compute marginal attributions and interaction indices when $n$ is sufficiently small. In figures, we only show the best performing baselines. Results and implementation details for all baselines can be found in 
Appendix~\ref{apdx:experiments}.

\vspace{-6pt}
\paragraph{Hyperparameters} \SpecExp{} has several parameters to determine the number of model inferences (masks). We choose $C=3$, informed by \citet{li2015spright} under a simplified sparse Fourier setting. We fix $t = 5$, which is the error correction capability of \SpecExp{} and serves as an approximate bound on the maximum degree. 
%
We also set $b=8$; the total collected samples are $\approx C2^bt \log(n)$. 
%
For $\ell_1$ regression-based interaction indices, we choose the regularization parameter via $5$-fold cross-validation. 




\vspace{-3pt}
\subsection{Metrics}


We compare \SpecExp{} to other methods across a variety of well-established metrics to assess performance.
%\Efe{How about textbf rather than emph here?}

\textbf{Faithfulness}: To characterize how well the surrogate function $\hat{f}$ approximates the true function, we define \emph{faithfulness} \cite{zhang2023trade}:
\vspace{-3pt}
\begin{equation}
    R^2 = 1 -  \frac{\lVert \hat{f} - f \rVert^2}{\left\lVert f - \bar{f} \right\rVert^2},
\end{equation}
where $\left\lVert f  \right\rVert^2 = \sum_{\bbm \in \bbF_2^n}f(\bbm)^2$ and $\bar{f} = \frac{1}{2^n} \sum_{\bbm \in \bbF_2^n}f(\bbm)$.

Faithfulness measures the ability of different explanation methods to predict model output when masking random inputs. 
%
We measure faithfulness over 10,000 random \emph{test} masks per-sample, and report average $R^2$ across samples. 
%

\textbf{Top-$r$ Removal}: We measure the ability of methods to identify the top $r$ influential features to model output:
\vspace{-2pt}
\begin{align}
\begin{split}
    \mathrm{Rem}(r) = \frac{|f(\boldsymbol{1}) - f(\bbm^*)|}{|f(\boldsymbol{1})|}, \\
    \;\bbm^* = \argmax \limits_{\abs{\bbm} = n-r}|\hat{f}(\boldsymbol{1}) - \hat{f}(\bbm)|.
\end{split}
\end{align}
\vspace{-8pt}


\textbf{Recovery Rate@$r$:} 
%
Each question in \emph{HotpotQA} contains human-labeled annotations for the sentences required to correctly answer the question. 
%
We measure the ability of interaction indices to recover these human-labeled annotations. 
%
Let $S_{r^*} \subseteq [n]$ denote human-annotated sentence indices. %corresponding to the human-annotated sentences containing the answer. 
Let $S_{i}$ denote feature indices of the $i^{\text{th}}$ most important interaction for a given interaction index.
%
Define the recovery ability at $r$ for each method as follows
\vspace{-2pt}
\begin{equation}
\label{eq:recovery_k}
    \text{Recovery@}r = 
    \frac{1}{r}\sum^r_{i=1}\frac{\abs{S_r^*\cap S_i}}{|S_{i}|}.
\end{equation}
\vspace{-8pt}

Intuitively, \eqref{eq:recovery_k} measures how well interaction indices capture features that align with human-labels.   


\begin{figure*}[t]
\centering
\hfill
\begin{subfigure}[b]{.5\textwidth}
  \centering
    \hspace{0.82cm}\includegraphics[width=0.75\textwidth]{figures/recall_legend.pdf}
  \includegraphics[width=.9\linewidth]{figures/hotpot_recall.pdf}
  \caption{Recovery rate$@10$ for \emph{HotpotQA} }
  \label{fig:recovery_hotpot}
\end{subfigure}%
\hfill % To ensure space between the figures
\begin{subfigure}[b]{.46\textwidth}
  \centering
    \includegraphics[width=1\textwidth]{figures/hotpot.pdf}
  \caption{Human-labeled interaction identified by \SpecExp{}.}
  \label{fig:hotpot_additional}
\end{subfigure}
\hfill
\caption{(a) \SpecExp{} recovers more human-labeled features with significantly fewer training masks as compared to other methods. (b) For a long-context example ($n = 128$ sentences), \SpecExp{} identifies the three human-labeled sentences as the most important third order interaction while ignoring unimportant contextual information.}
\vspace{-8pt}
\end{figure*}

\vspace{-8pt}
\subsection{Faithfulness and Runtime}
\vspace{-3pt}

Fig.~\ref{fig:faith} shows the faithfulness of \SpecExp{} compared to other methods. We also plot the runtime of all approaches for the \emph{Sentiment} dataset for different values of $n$. 
%
All attribution methods are learned over a fixed number of training masks.
% 

\textbf{Comparison to Interaction Indices } \SpecExp{} maintains competitive performance with the best-performing interaction indices across datasets. 
%
Recall these indices enumerate \emph{all possible interactions}, whereas \SpecExp{} does not. 
%
This difference is reflected in the runtimes of Fig.~\ref{fig:faith}(a).
%
The runtime of other interaction indices explodes as $n$ increases while \SpecExp{} does not suffer any increase in runtime. 

\vspace{-2pt}
\textbf{Comparison to Marginal Attributions } For input lengths $n$ too large to run interaction indices, \SpecExp{} is significantly more faithful than marginal attribution approaches across all three datasets.

\vspace{-2pt}
\textbf{Varying number of training masks } Results in Appendix ~\ref{apdx:experiments} show that \SpecExp{} continues to out-perform other approaches as we vary the number of training masks. 

\vspace{-2pt}
\textbf{Sparsity of \SpecExp{} Surrogate Function} Results in Appendix ~\ref{apdx:experiments}, Table~\ref{tab:faith} show 
surrogate functions learned by \SpecExp{} have Fourier representations where only $\sim 10^{-100}$ percent of coefficients are non-zero. 


\vspace{-6pt}
\subsection{Removal}
\label{subsec:removal}

Fig.~\ref{fig:fidelity} plots the change in model output as we mask the top $r$ features for different regimes of $n$. 
%

\vspace{-2pt}
\textbf{Small $n$ } \SpecExp{} is competitive with other interaction indices for \textit{Sentiment}, and out-performs them for \textit{HotpotQA} and \textit{DROP}. 
%
Performance of \SpecExp{} in this task is particularly notable since Shapley-based methods are designed to identify a small set of influential features. 
%
On the other hand, \SpecExp{} does not optimize for this metric, but instead learns the function $f(\cdot)$ over all possible $2^n$ masks. 
%

\textbf{Large $n$ } \SpecExp{} out-performs all marginal approaches, indicating the utility of considering interactions.
%

\vspace{-10pt}
\subsection{Recovery Rate of Human-Labeled Interactions}

%
We compare the recovery rate \eqref{eq:recovery_k} for $r = 10$ of \SpecExp{} against third order Faith-Banzhaf and Faith-Shap interaction indices. 
%
We choose third order interaction indices because all examples 
are answerable with information from at most three sentences, i.e., maximum degree $d = 3$.
%
Recovery rate is measured as we vary the number of training masks. 

Results are shown in Fig.~\ref{fig:recovery_hotpot}, where \SpecExp{} has the highest recovery rate of all interaction indices across all sample sizes. 
%
Further, \SpecExp{} achieves close to its maximum performance with few samples, other approaches require many more samples to approach the recovery rate of \SpecExp{}. 

\textbf{Example of Learned Interaction by \SpecExp{}} Fig.~\ref{fig:hotpot_additional} displays a long-context example (128 sentences) from \emph{HotpotQA} whose answer is contained in the three highlighted sentences. 
%
\SpecExp{} identifies the three human-labeled sentences as the most important third order interaction while ignoring unimportant contextual information. 
%
Other third order methods are not computable at this length. 
%

\begin{figure*}[t]
    \centering
    \includegraphics[width=0.9\linewidth]{figures/case_studies.pdf}
    \caption{SHAP provides marginal feature attributions. Feature interaction attributions computed by SPEX provide a more comprehensive understanding of (above) words interactions that cause the model to answer incorrectly and (below) interactions between image patches that informed the model's output.}
    \label{fig:caseStudies}
\end{figure*}


\begin{figure}[htb]
\small
\begin{tcolorbox}[left=3pt,right=3pt,top=3pt,bottom=3pt,title=\textbf{Conversation History:}]
[human]: Craft an intriguing opening paragraph for a fictional short story. The story should involve a character who wakes up one morning to find that they can time travel.

...(Human-Bot Dialogue Turns)... \textcolor{blue}{(Topic: Time-Travel Fiction)}

[human]: Please describe the concept of machine learning. Could you elaborate on the differences between supervised, unsupervised, and reinforcement learning? Provide real-world examples of each.

...(Human-Bot Dialogue Turns)... \textcolor{blue}{(Topic: Machine learning Concepts and Types)}


[human]: Discuss antitrust laws and their impact on market competition. Compare the antitrust laws in US and China along with some case studies

...(Human-Bot Dialogue Turns)... \textcolor{blue}{(Topic: Antitrust Laws and Market Competition)}

[human]: The vertices of a triangle are at points (0, 0), (-1, 1), and (3, 3). What is the area of the triangle?

...(Human-Bot Dialogue Turns)... \textcolor{blue}{(Topic: Area and Circumcircle of Triangle)}

...(Human-Bot Dialogue Interaction on Other Topics)... 

\end{tcolorbox}
\begin{tcolorbox}[left=3pt,right=3pt,top=3pt,bottom=3pt,title=\textbf{Question:}]
What title reflects the thematic essence of the Doctor's experience with time travel as described in the story?
\end{tcolorbox}
\begin{tcolorbox}[left=3pt,right=3pt,top=3pt,bottom=3pt,title=\textbf{Ground-Truth Answer:}]
Awakening to Eternity: The Doctor's New Dawn
\end{tcolorbox}
\begin{tcolorbox}[left=3pt,right=3pt,top=3pt,bottom=3pt,title=\textbf{Retrieved Memory (Turn-Level):}]
[human]: Describe the process and write out the balanced chemical equation for the reaction that occurs when solid calcium carbonate reacts with hydrochloric acid to form aqueous calcium chloride, carbon dioxide, and water. What type of reaction is this, and what observations might indicate that the reaction is taking place?

[bot]: When solid calcium carbonate ($\text{CaCO}_{3}$) reacts with hydrochloric acid (HCl), the products formed are aqueous calcium chloride ($\text{CaCl}_{2}$), carbon dioxide ($\text{CO}_{2}$), and water ($\text{H}_{2}\text{O}$). The balanced chemical equation for this reaction is: ... \textcolor{blue}{(Topic: Calcium Carbonate and Hydrochloric Acid Reaction)}

\textcolor{brown}{
[human]: Write a title for your story.}

\textcolor{brown}{
[bot]: "Awakening to Eternity: The Doctor's New Dawn".
} \textcolor{blue}{(Topic: Time-Travel Fiction)}

[human]: Analyze the following customer reviews from different sources for three different smartphones - the latest iPhone, Samsung Galaxy, and Google Pixel - and provide an overall rating for each phone on a scale of 1 to 10. Consider the following complex and contradictory reviews: - TechRadar's review of the latest iPhone: The new iPhone is a stunning triumph of engineering that sets a new bar for smartphone performance and camera quality. However, the incremental design and high price mean it lacks the 'wow' factor of previous iPhones. Still, its power and intelligence are unrivaled. - CNET's review of the latest Samsung Galaxy: ... Return the answer as a JSON object with the overall ratings for each phone out of 10, to one decimal place.

[bot]: ```json { "iPhone": 8.5, "Samsung Galaxy": 8.0, "Google Pixel": 7.5 } ```

\textcolor{blue}{(Topic: Smartphone Review Ratings Analysis)}
\end{tcolorbox}
\begin{tcolorbox}[left=3pt,right=3pt,top=3pt,bottom=3pt,title=\textbf{Answer (Turn-Level Retrieval):}]
As an AI language model, I do not have access to the story you mentioned earlier. Please provide me with the story or the necessary details so that I can assist you better.

\textbf{\textcolor{red}{(The next part of this figure~\ref{fig: case_study_segment_vs_turn} will be continued on the next page.)}}
\end{tcolorbox}
\end{figure}

\begin{figure}[htb]
% \ContinuedFloat
\small
\begin{tcolorbox}[left=3pt,right=3pt,top=3pt,bottom=3pt,title=\textbf{Retrieved Memory (Ours):}]
[human]: Craft an intriguing opening paragraph for a fictional short story. The story should involve a character who wakes up one morning to find that they can time travel.

[bot]: In the hushed pre-dawn hours, when the world still slumbers under the cloak of night, Alex found themselves caught in the fragile boundary between sleep and wakefulness. ...

[human]: Summarize the story with three bullet points using only nouns and adjectives, without verbs.

[bot]: - Alex, pre-dawn awakening, time-travel discovery - Disorientation, vertigo, reality's malleability - Sunlight, new day, astonishing truth

[human]: Is there any famous fictional character that can time travel? If so, can you rewrite your story with him or her?

[bot]: - \textcolor{brown}{Doctor Who, a renowned time-traveling character from British television, is a perfect fit for this scenario.} In the silent embrace of early morning, where shadows merge with the fading night, the Doctor found themselves in the thin veil separating dreams from consciousness. It was here, amidst the soft whispers of time, that an unusual sensation enveloped them—a sense of time's fabric slipping through their fingers like grains of sand. As dawn's light broke through the darkness, painting the room in hues of gold and amber, the Doctor's eyes opened to what appeared to be an ordinary day. \textcolor{brown}{However, upon standing, a wave of dizziness overwhelmed them, the surroundings blurred, and in that moment of confusion, the Doctor realized a profound truth: they had not only awakened to a new dawn but had also stumbled upon the incredible ability to navigate the currents of time at will.}

\textcolor{brown}{[human]: Write a title for your story.}

\textcolor{brown}{[bot]: "Awakening to Eternity: The Doctor's New Dawn"} \textcolor{blue}{(Topic: Time-Travel Fiction)}
\end{tcolorbox}
\begin{tcolorbox}[left=3pt,right=3pt,top=3pt,bottom=3pt,title=\textbf{Answer (Ours):}]
The title "Awakening to Eternity: The Doctor's New Dawn" reflects the thematic essence of the Doctor's experience with time travel as described in the story. It captures the idea of a new beginning, a sense of eternity, and the awakening of a new power that will change the course of the Doctor's life forever.
\end{tcolorbox}
\caption{Case study comparing our \sysname\ with the \textit{turn-level} memory baseline. When constructing the memory bank at the \textit{turn-level}, the agent fails to provide a satisfactory response due to fragmentation: the retrieved dialogue turns can not capture a complete human-agent interaction related to the query, missing critical information highlighted in \textcolor{brown}{brown}. The dialogue topic is emphasized in \textcolor{blue}{blue}. Some context is omitted (replaced by ...) due to length constraints.}
\label{fig: case_study_segment_vs_turn}
\end{figure}

\newpage

\begin{figure}[htbp]
\small
\begin{tcolorbox}[left=3pt,right=3pt,top=3pt,bottom=3pt,title=\textbf{Conversation History:}]
[human]: Explain what's base rate fallacy and list five specific examples of ...

... (Human-Bot Dialogue Turns) ... \textcolor{blue}{(Topic: Base Rate Fallacy In Politics)}


[human]: How many integers are in the solution of the inequality $|x + 5| < 10$?

... (Human-Bot Dialogue Turns) ... \textcolor{blue}{(Topic: Number of Integers in Inequalities)}

[human]: How have the Alps and Rhine River influenced settlement and agriculture in Western Europe?

... (Human-Bot Dialogue Turns) ... \textcolor{blue}{(Topic: Alps and Rhine River Impact)}

[human]: In a survey conducted at a local high school, preferences for a new school color were measured: 58\% of students liked the color blue, 45\% preferred green, and 22\% liked both colors. If we randomly pick a student from the school, what's the probability that they would like neither blue nor green?

... (Human-Bot Dialogue Turns) ... \textcolor{blue}{(Topic: Probability of Liking Colors in Survey)}

[human]: Imagine you are participating in a race with a group of people. If you have just overtaken the second person, what's your current position? Where is the person you just overtook?

... (Human-Bot Dialogue Turns) ... \textcolor{blue}{(Topic: Overtaking Positions in Race)}
\end{tcolorbox}
\begin{tcolorbox}[left=3pt,right=3pt,top=3pt,bottom=3pt,title=\textbf{Question:}]
[human]: In the inequality question, how many positive integers are in the solution of the second inequality?
\end{tcolorbox}
\begin{tcolorbox}[left=3pt,right=3pt,top=3pt,bottom=3pt,title=\textbf{Ground-Truth Answer:}]
In the second inequality, $(|x + 10| < 5)$, the solution range is $(-15 < x < -5)$. Within this range, there are no positive integers, as all values of $(x)$ that satisfy the inequality are negative.
\end{tcolorbox}
\begin{tcolorbox}[left=3pt,right=3pt,top=3pt,bottom=3pt,title=\textbf{Retrieved Memory (Session-Level):}]
[human]: Explain what's base rate fallacy and list five specific examples of ...

... (Human-Bot Dialogue Turns) ... \textcolor{blue}{(Topic: Base Rate Fallacy In Politics)}

[human]: How many integers are in the solution of the inequality $|x + 5| < 10$?

[bot]: To solve the inequality $|x + 5| < 10$, ... 

\textcolor{brown}{
[human]: What about $|x + 10| < 5$?
}

\textcolor{brown}{
[bot]: To solve the inequality $|x + 10| < 5$, we ...} \textcolor{gray}{(``overwhelmed'' by other irrelevant, redundant content)}

... (Human-Bot Dialogue Turns) ...  \textcolor{blue}{(Topic: Number of Integers in Inequalities)}

% \textcolor{brown}{
% [human]: What about $|x + 10| < 5$
% }

% \textcolor{brown}{
% [bot]: To solve the inequality $|x + 10| < 5$, ...
% } \textcolor{blue}{(Topic: Number of Integers in Inequalities)}

[human]: How have the Alps and Rhine River influenced settlement and agriculture in Western Europe?

... (Human-Bot Dialogue Turns) ... \textcolor{blue}{(Topic: Alps and Rhine River Impact)}

[human]: In a survey conducted at a local high school, ...

... (Human-Bot Dialogue Turns) ... \textcolor{blue}{(Topic: Probability of Liking Colors in Survey)}
\end{tcolorbox}
\begin{tcolorbox}[left=3pt,right=3pt,top=3pt,bottom=3pt,title=\textbf{Answer (Session-Level Retrieval):}]
I apologize for the confusion earlier. To answer your question, the second inequality was not provided in our previous conversation. If you could provide the inequality, I would be happy to help you find the solution and the number of positive integers in the solution set.

\textbf{\textcolor{red}{(The next part of this figure~\ref{fig: case_study_segment_vs_session} will be continued on the next page.)}}
\end{tcolorbox}
\end{figure}
\begin{figure}[htbp]
% \ContinuedFloat
\small
\begin{tcolorbox}[left=3pt,right=3pt,top=3pt,bottom=3pt,title=\textbf{Retrieved Memory (Ours):}]
[human]: How many integers are in the solution of the inequality $|x + 5| < 10$?

[bot]: To solve the inequality $|x + 5| < 10$, we break it into two cases based on the definition of absolute value: 
1. $x + 5 < 10$ 
2. $-(x + 5) < 10$, which simplifies to $x + 5 > -10$ 
Solving each case: 
1. $x < 5$ 
2. $x > -15$ 
Combining these, we get $-15 < x < 5$. This inequality means that $x$ can be any number greater than $-15$ and less than $5$. To find how many integers satisfy this condition, we list them: $-14, -13, -12, -11, -10, -9, -8, -7, -6, -5, -4, -3, -2, -1, 0, 1, 2, 3, 4$. There are 19 integers that satisfy the inequality $|x + 5| < 10$.

\textcolor{brown}{
[human]: What about $|x + 10| < 5$?
}

\textcolor{brown}{
[bot]: To solve the inequality $|x + 10| < 5$, we again consider two cases: 
1. $x + 10 < 5$ 
2. $-(x + 10) < 5$, which simplifies to $x + 10 > -5$ 
Solving each case: 
1. $x < -5$ 
2. $x > -15$ 
Combining these, we get $-15 < x < -5$. This inequality means that $x$ can be any number greater than $-15$ and less than $-5$. To find how many integers satisfy this condition, we list them: $-14, -13, -12, -11, -10, -9, -8, -7, -6$. There are 9 integers that satisfy the inequality $|x + 10| < 5$.
}

[human]: What about $|x + 10| < 10$?

[bot]: To solve the inequality $|x + 10| < 10$, we consider two cases: 
1. $x + 10 < 10$ 
2. $-(x + 10) < 10$, which simplifies to $x + 10 > -10$ 
Solving each case: 
1. $x < 0$ 
2. $x > -20$ 
Combining these, we get $-20 < x < 0$. This inequality means that $x$ can be any number greater than $-20$ and less than $0$. To find how many integers satisfy this condition, we list them: $-19, -18, -17, -16, -15, -14, -13, -12, -11, -10, -9, -8, -7, -6, -5, -4, -3, -2, -1$. There are 19 integers that satisfy the inequality $|x + 10| < 10$. \textcolor{blue}{(Topic: Number of Integers in Inequalities)}
\end{tcolorbox}
\begin{tcolorbox}[left=3pt,right=3pt,top=3pt,bottom=3pt,title=\textbf{Answer (Ours):}]
To find the number of positive integers that satisfy the inequality $(|x + 10| < 5)$, we need to look for positive integers in the solution set $(-15 < x < -5)$. Since there are no positive integers in this interval, the answer is 0.

\end{tcolorbox}
\caption{Case study comparing our \sysname\ with the \textit{session-level} memory baseline. When constructing the memory bank at the \textit{session-level}, the agent fails to provide a satisfactory response due to the inclusion of excessive irrelevant information within a session. This distracts the generator from focusing on the truly useful information highlighted in \textcolor{brown}{brown}. The dialogue topic is emphasized in \textcolor{blue}{blue}. Some context is omitted (replaced by ...) due to length constraints.}
\label{fig: case_study_segment_vs_session}
\end{figure}



\begin{figure}[htb]
\small
\begin{tcolorbox}[left=3pt,right=3pt,top=3pt,bottom=3pt,title=\textbf{Conversation History:}]
[human]: Photosynthesis is a vital process for life on Earth. Could you outline the two main stages of photosynthesis, including where they take place within the chloroplast, and the primary inputs and outputs for each stage? ... (Human-Bot Dialogue Turns)... \textcolor{blue}{(Topic: Photosynthetic Energy Production)}

[human]: Please assume the role of an English translator, tasked with correcting and enhancing spelling and language. Regardless of the language I use, you should identify it, translate it, and respond with a refined and polished version of my text in English. 

... (Human-Bot Dialogue Turns)...  \textcolor{blue}{(Topic: Language Translation and Enhancement)}

[human]: Suggest five award-winning documentary films with brief background descriptions for aspiring filmmakers to study.

\textcolor{brown}{[bot]: ...
5. \"An Inconvenient Truth\" (2006) - Directed by Davis Guggenheim and featuring former United States Vice President Al Gore, this documentary aims to educate the public about global warming. It won two Academy Awards, including Best Documentary Feature. The film is notable for its straightforward yet impactful presentation of scientific data, making complex information accessible and engaging, a valuable lesson for filmmakers looking to tackle environmental or scientific subjects.}

... (Human-Bot Dialogue Turns)... 
\textcolor{blue}{(Topic: Documentary Films Recommendation)}

[human]: Given the following records of stock prices, extract the highest and lowest closing prices for each month in the year 2022. Return the results as a CSV string, with one line allocated for each month. Date,Open,High,Low,Close,Volume ... ... (Human-Bot Dialogue Turns)...  \textcolor{blue}{(Topic: Stock Prices Analysis)}

[human]: The city of Vega intends to build a bridge that will span the Vegona River, covering a distance of 1.8 kilometers. The proposed location falls within a seismically active area that has experienced several high-magnitude earthquakes. Given these circumstances, what would be the best approach to constructing the bridge? ... (Human-Bot Dialogue Turns)...  \textcolor{blue}{(Topic: Seismic Bridge Construction)}
\end{tcolorbox}
\begin{tcolorbox}[left=3pt,right=3pt,top=3pt,bottom=3pt,title=\textbf{Question:}]
Which documentary won two Academy Awards and what lesson does it offer to filmmakers?
\end{tcolorbox}
\begin{tcolorbox}[left=3pt,right=3pt,top=3pt,bottom=3pt,title=\textbf{Ground-Truth Answer:}]
``An Inconvenient Truth" (2006) won two Academy Awards. It offers the lesson of making complex information accessible and engaging, a valuable lesson for filmmakers looking to tackle environmental or scientific subjects.
\end{tcolorbox}
\begin{tcolorbox}[left=3pt,right=3pt,top=3pt,bottom=3pt,title=\textbf{Memory generated by RecurSum~\citep{wang2023recursively}:}]
The dialogue provided detailed information on various topics, including the stages of photosynthesis, energy production estimates through photosynthesis for different organisms, translations and enhancements of phrases in different languages, recommendations for documentary films, a pitch for a film about overcoming adversity, analysis of stock prices, and considerations for constructing a bridge in a seismically active area. Here's a summary of the key points: ...

\textbf{Documentary Film Recommendations}: Suggested five award-winning documentary films for aspiring filmmakers to study, with brief background descriptions for each. The response included:

-  "March of the Penguins" (2005) - Emperor penguins' journey in Antarctica; noted for its cinematography and portrayal of the penguins' life cycle.

-  "Citizenfour" (2014) - Edward Snowden and NSA spying scandal; known for its real-time exploration of Snowden's revelations.

-  "20 Feet from Stardom" (2013) - Stories of backup singers behind musical legends; celebrates unsung heroes in the music industry.

-  "Man on Wire" (2008) - Philippe Petit's 1974 high-wire walk between the Twin Towers; combines personal footage, interviews, and re-enactments.

-  \textcolor{brown}{"An Inconvenient Truth" (2006) - Al Gore's documentary on global warming; known for its presentation of scientific data.}
\end{tcolorbox}
\begin{tcolorbox}[left=3pt,right=3pt,top=3pt,bottom=3pt,title=\textbf{Answer (RSum):}]
I'm sorry, but based on the updated context you provided, I don't see any previous discussion related to the question you asked. However, to answer your question, the documentary that won two Academy Awards is "Free Solo," which won for Best Documentary Feature and Best Achievement in Film Editing in 2019. 
% The lesson it offers to filmmakers is the importance of taking risks and pushing boundaries in order to create something truly unique and impactful.
....
The film follows climber Alex Honnold as he attempts to free solo climb El Capitan in Yosemite National Park, and the filmmakers had to take risks themselves in order to capture the footage and tell the story.

\textbf{\textcolor{red}{(The next part of this figure~\ref{fig: case_study_segment_vs_rsum} will be continued on the next page.)}}
\end{tcolorbox}
\end{figure}
\begin{figure}[htbp]
% \ContinuedFloat
\small
\begin{tcolorbox}[left=3pt,right=3pt,top=3pt,bottom=3pt,title=\textbf{Retrieved Memory (Ours):}]
[human]: Suggest five award-winning documentary films with brief background descriptions for aspiring filmmakers to study.

[bot]: 1. **"March of the Penguins" (2005)** - Directed by Luc Jacquet, this French feature-length nature documentary delves into the annual journey of Emperor penguins across the Antarctic. It won the Academy Award for Best Documentary Feature. The film is celebrated for its stunning cinematography and its detailed, heartfelt portrayal of the penguins' life cycle, making it a must-study for filmmakers interested in wildlife documentaries. 2. **"Citizenfour" (2014)** ... 5. \textcolor{brown}{**"An Inconvenient Truth" (2006)** - Directed by Davis Guggenheim and featuring former United States Vice President Al Gore, this documentary aims to educate the public about global warming. It won two Academy Awards, including Best Documentary Feature.} The film is notable for its straightforward yet impactful presentation of scientific data, making complex information accessible and engaging, \textcolor{brown}{a valuable lesson for filmmakers looking to tackle environmental or scientific subjects.}

[human]: With the spirit in the first film, craft a succinct and persuasive pitch for a film about overcoming adversity.

[bot]: Drawing inspiration from the resilience and majestic journey of Emperor penguins in "March of the Penguins," our documentary, "Unyielding Spirits: The Human Odyssey," will explore the profound stories of individuals around the globe who have faced and overcome staggering adversities. Through breathtaking cinematography and intimate storytelling, we ...
\textcolor{blue}{(Topic: Documentary Films Recommendation)}
\end{tcolorbox}
\begin{tcolorbox}[left=3pt,right=3pt,top=3pt,bottom=3pt,title=\textbf{Answer (Ours):}]
``An Inconvenient Truth" won two Academy Awards, including Best Documentary Feature. The film offers a valuable lesson to filmmakers looking to tackle environmental or scientific subjects, as it presents complex information in a straightforward yet impactful way, making it accessible and engaging to a wider audience.


\end{tcolorbox}
\caption{Case study comparing our \sysname\ with RecurSum~\citep{wang2023recursively}. When using RecurSum, the agent fails to provide a satisfactory answer due to information loss during the summarization process. Summarization omits essential details needed to answer the user's new request, as highlighted in \textcolor{brown}{brown}. The dialogue topic is emphasized in \textcolor{blue}{blue}. Some context is omitted (replaced by ...) due to length constraints.}
\label{fig: case_study_segment_vs_rsum}
\end{figure}




\begin{figure}[htb]
\small
\begin{tcolorbox}[left=3pt,right=3pt,top=3pt,bottom=3pt,title=\textbf{Conversation History:}]
[human]: Photosynthesis is a vital process for life on Earth. Could you outline the two main stages of photosynthesis, including where they take place within the chloroplast, and the primary inputs and outputs for each stage? ... (Human-Bot Dialogue Turns)... \textcolor{blue}{(Topic: Photosynthetic Energy Production)}

[human]: Please assume the role of an English translator, tasked with correcting and enhancing spelling and language. Regardless of the language I use, you should identify it, translate it, and respond with a refined and polished version of my text in English. 

... (Human-Bot Dialogue Turns)...  \textcolor{blue}{(Topic: Language Translation and Enhancement)}

[human]: Suggest five award-winning documentary films with brief background descriptions for aspiring filmmakers to study.

\textcolor{brown}{[bot]: ...
5. \"An Inconvenient Truth\" (2006) - Directed by Davis Guggenheim and featuring former United States Vice President Al Gore, this documentary aims to educate the public about global warming. It won two Academy Awards, including Best Documentary Feature. The film is notable for its straightforward yet impactful presentation of scientific data, making complex information accessible and engaging, a valuable lesson for filmmakers looking to tackle environmental or scientific subjects.}

... (Human-Bot Dialogue Turns)... 
\textcolor{blue}{(Topic: Documentary Films Recommendation)}

[human]: Given the following records of stock prices, extract the highest and lowest closing prices for each month in the year 2022. Return the results as a CSV string, with one line allocated for each month. Date,Open,High,Low,Close,Volume ... ... (Human-Bot Dialogue Turns)...  \textcolor{blue}{(Topic: Stock Prices Analysis)}

[human]: The city of Vega intends to build a bridge that will span the Vegona River, covering a distance of 1.8 kilometers. The proposed location falls within a seismically active area that has experienced several high-magnitude earthquakes. Given these circumstances, what would be the best approach to constructing the bridge? ... (Human-Bot Dialogue Turns)...  \textcolor{blue}{(Topic: Seismic Bridge Construction)}
\end{tcolorbox}
\begin{tcolorbox}[left=3pt,right=3pt,top=3pt,bottom=3pt,title=\textbf{Question:}]
Which documentary won two Academy Awards and what lesson does it offer to filmmakers?
\end{tcolorbox}
\begin{tcolorbox}[left=3pt,right=3pt,top=3pt,bottom=3pt,title=\textbf{Ground-Truth Answer:}]
"An Inconvenient Truth" (2006) won two Academy Awards. It offers the lesson of making complex information accessible and engaging, a valuable lesson for filmmakers looking to tackle environmental or scientific subjects.
\end{tcolorbox}
\begin{tcolorbox}[left=3pt,right=3pt,top=3pt,bottom=3pt,title=\textbf{Memory generated by ConditionMem~\citep{yuan2023evolving}:}]
- \textcolor{brown}{The user is seeking recommendations for award-winning documentary films, along with brief background descriptions, for educational purposes.} This request likely stems from a desire to learn from successful examples in the documentary filmmaking field, possibly to improve their own filmmaking skills or to gain inspiration for their projects. 

- \textcolor{brown}{User is interested in award-winning documentary films for educational purposes, aiming to study their backgrounds and filmmaking techniques.}

- The user's request for crafting a succinct and persuasive pitch for a film about overcoming adversity is inspired by the spirit of "March of the Penguins," a documentary that beautifully captures the Emperor penguins' annual journey across the Antarctic, highlighting themes of endurance and survival against the odds. The user seeks to emulate the emotional depth and cinematic quality that made "March of the Penguins" compelling, aiming to create a pitch that resonates with similar themes of resilience and triumph over challenges. 

- User is interested in creating content that focuses on themes of overcoming adversity, inspired by the emotional depth and cinematic excellence of documentaries like "March of the Penguins." They value storytelling that captures the essence of endurance and survival against the odds.
% : The user's request to write a budget plan for the video production directly follows the task of creating a video script aimed at training new employees of a car wash business in Japan, focusing on business etiquette norms.
% The user is likely seeking to understand the financial requirements to bring the previously crafted script to life, indicating a move from the planning and content development phase to the practical aspects of production. 

... (Other Irrelevant Memory Records) ...
\end{tcolorbox}
\begin{tcolorbox}[left=3pt,right=3pt,top=3pt,bottom=3pt,title=\textbf{Answer (Conditional Memory):}]
The documentary that won two Academy Awards is "Free Solo," which follows climber Alex Honnold as he attempts to free solo climb El Capitan in Yosemite National Park. The film offers a lesson to filmmakers on the importance of capturing authentic and compelling human stories, as well as the technical challenges of filming in extreme environments.

\textbf{\textcolor{red}{(The next part of this figure~\ref{fig: case_study_segment_vs_condmem} will be continued on the next page.)}}
\end{tcolorbox}
\end{figure}
\begin{figure}[htbp]
% \ContinuedFloat
\small
\begin{tcolorbox}[left=3pt,right=3pt,top=3pt,bottom=3pt,title=\textbf{Retrieved Memory (Ours):}]
[human]: Suggest five award-winning documentary films with brief background descriptions for aspiring filmmakers to study.

[bot]: 1. **"March of the Penguins" (2005)** - Directed by Luc Jacquet, this French feature-length nature documentary delves into the annual journey of Emperor penguins across the Antarctic. It won the Academy Award for Best Documentary Feature. The film is celebrated for its stunning cinematography and its detailed, heartfelt portrayal of the penguins' life cycle, making it a must-study for filmmakers interested in wildlife documentaries. 2. **"Citizenfour" (2014)** ... 5. \textcolor{brown}{**"An Inconvenient Truth" (2006)** - Directed by Davis Guggenheim and featuring former United States Vice President Al Gore, this documentary aims to educate the public about global warming. It won two Academy Awards, including Best Documentary Feature.} The film is notable for its straightforward yet impactful presentation of scientific data, making complex information accessible and engaging, a valuable lesson for filmmakers looking to tackle environmental or scientific subjects.

[human]: With the spirit in the first film, craft a succinct and persuasive pitch for a film about overcoming adversity.

[bot]: Drawing inspiration from the resilience and majestic journey of Emperor penguins in "March of the Penguins," our documentary, "Unyielding Spirits: The Human Odyssey," will explore the profound stories of individuals around the globe who have faced and overcome staggering adversities. Through breathtaking cinematography and intimate storytelling, we ...
\textcolor{blue}{(Topic: Documentary Films Recommendation)}
\end{tcolorbox}
\begin{tcolorbox}[left=3pt,right=3pt,top=3pt,bottom=3pt,title=\textbf{Answer (Ours):}]
``An Inconvenient Truth" won two Academy Awards, including Best Documentary Feature. The film offers a valuable lesson to filmmakers looking to tackle environmental or scientific subjects, as it presents complex information in a straightforward yet impactful way, making it accessible and engaging to a wider audience.
\end{tcolorbox}
\caption{Case study comparing our \sysname\ with ConditionMem~\citep{yuan2023evolving}. When using ConditionMem, the agent fails to provide a satisfactory answer due to (1) information loss during the summarization process and (2) the incorrect discarding of turns that are actually useful, as highlighted in \textcolor{brown}{brown}. The dialogue topic is emphasized in \textcolor{blue}{blue}. Some context is omitted (replaced by ...) due to length constraints.}
\label{fig: case_study_segment_vs_condmem}
\end{figure}


\section{Conclusion}
In this work, we propose a simple yet effective approach, called SMILE, for graph few-shot learning with fewer tasks. Specifically, we introduce a novel dual-level mixup strategy, including within-task and across-task mixup, for enriching the diversity of nodes within each task and the diversity of tasks. Also, we incorporate the degree-based prior information to learn expressive node embeddings. Theoretically, we prove that SMILE effectively enhances the model's generalization performance. Empirically, we conduct extensive experiments on multiple benchmarks and the results suggest that SMILE significantly outperforms other baselines, including both in-domain and cross-domain few-shot settings.

\section*{Impact Statement}
Getting insights into the decisions of deep learning models offers significant advantages, including increased trust in model outputs. By reasoning about the rationale behind a model's decisions with the help of \SpecExp{}, we can develop greater confidence in its output, and use it to aid in our own reasoning. When using analysis tools like \SpecExp{}, it's crucial to avoid over-interpretation of results.
\bibliography{main}
\bibliographystyle{icml2025}


%%%%%%%%%%%%%%%%%%%%%%%%%%%%%%%%%%%%%%%%%%%%%%%%%%%%%%%%%%%%%%%%%%%%%%%%%%%%%%%
%%%%%%%%%%%%%%%%%%%%%%%%%%%%%%%%%%%%%%%%%%%%%%%%%%%%%%%%%%%%%%%%%%%%%%%%%%%%%%%
% APPENDIX
%%%%%%%%%%%%%%%%%%%%%%%%%%%%%%%%%%%%%%%%%%%%%%%%%%%%%%%%%%%%%%%%%%%%%%%%%%%%%%%
%%%%%%%%%%%%%%%%%%%%%%%%%%%%%%%%%%%%%%%%%%%%%%%%%%%%%%%%%%%%%%%%%%%%%%%%%%%%%%%
\newpage
\appendix
\onecolumn
\section{Algorithm Details}\label{apdx:algorithm}
\subsection{Introduction}
This section provides the algorithmic details behind \SpecExp{}. The algorithm is derived from the sparse Fourier (Hadamard) transformation described in \citet{li2015spright}. Many modifications have been made to improve the algorithm and make it suitable for use in this application. Application of the original algorithm proposed in \citet{li2015spright} fails for all problems we consider in this paper. In this work, we focus on applications of \SpecExp{} and defer theoretical analysis to future work. 

\paragraph{Relevant Literature on Sparse Transforms} This work develops the literature on sparse Fourier transforms. The first of such works are \cite{Hassanieh2012, stobbe2012, Pawar2013}. The most relevant literature is that of the sparse Boolean Fourier (Hadamard) transform \cite{li2015spright,amrollahi2019efficiently}. Despite the promise of many of these algorithms, their application has remained relatively limited, being used in only a handful of prior applications. Our code base is forked from that of \cite{erginbas2023efficiently}.
In this work we introduce a series of major optimizations which specifically target properties of explanation functions. By doing so, our algorithm is made significantly more practical and robust than any prior work.

\paragraph{Importance of the Fourier Transform} 
The Fourier transform does more than just impart critical algebraic structure. 
The orthonormality of the Fourier transform means that small noisy variations in $f$ remain small in the Fourier domain. In contrast, AND interactions, which operate under the non-orthogonal Möbius transform \cite{kang2024learning}, can amplify small noisy variations, which limits practicality. Fortunately, this is not problematic, as it is straightforward to generate AND interactions from the surrogate function $\hat{f}$. Many popular interaction indices have simple definitions in terms of $F$.  Table~\ref{tab:fourier-def-1} highlights some key relationships, and Appendix~\ref{app:interactions} provides a comprehensive list. 

\begin{table}[h]
\centering
\begin{tabular}{@{}ccc@{}}
\toprule
\textbf{Shapley Value} & \textbf{Banzhaf Interaction Index}         & \textbf{M\"obius Coefficient}  \\ \midrule
 $\mathrm{SV}(i) = \sum \limits_{S \ni i,\; \abs{S} \text{ odd}}F(S)/\abs{S} $& $I^{BII}(S) = (-2)^{|S|} F(S)$      &        $I^{M}(S) = (-2)^{|S|} \sum \limits_{T \supseteq S} F(T)$   \\ \bottomrule
\end{tabular}
\caption{Popular attribution scores in terms of Fourier coefficients}
\label{tab:fourier-def-1}
\end{table}

\subsection{Directly Solving the LASSO} 
Before we proceed, we remark that in cases where $n$ is not too large, and we expect the degree of nonzero $\abs{\bk} \leq d$ to be reasonably small, enumeration is actually not infeasible. In such cases, we can set up the LASSO problem directly:
\begin{equation}\label{eq:LASSO_apdx}
    \hat{F} = \argmin_{\tilde{F}} \sum_{\bbm}\abs{f(\bbm) - \sum_{\abs{\bk} \leq d}  \tilde{F}(\bk)}^2 + \lambda \norm{\tilde{F}}_1.
\end{equation}
Note that this is distinct from the \emph{Faith-Banzhaf} and \emph{Faith-Shapley} solution methods because those perform regression over the AND, M\"obius basis.
We observe that the formulation above typically outperforms these other approaches in terms of faithfulness, likely due to the properties of the Fourier transform. 

Most popular solvers use \emph{coordinate descent} to solve \eqref{eq:LASSO_apdx}, but there is a long line of research towards efficiently solving this problem. In our code, we also include an implementation of Approximate Message Passing (AMP) \cite{maleki2010approximate}, which can be much faster in many cases. Much like the final phase of \SpecExp{}, AMP is a low complexity message passing algorithm where messages are iteratively passed between factor nodes (observations) and variable nodes. 

A more refined version of \SpecExp{}, would likely examine the parameters $n$ and the maximum degree $d$ and determine whether or not to directly solve the LASSO, or to apply the full \SpecExp{}, as we describe in the following sections. 


\subsection{Masking Pattern Design and Model Inference}
The first part of the algorithm is to determine which samples we collect. All steps of this part of the algorithm are outlined in Algorithm~\ref{alg:collect}. This is governed by two structures: the random linear codes $\bM_c$ and the BCH parity matrix $\bP$. Random linear codes have been well studied as central objects in error correction and cryptography. They have previously been considered for sparse transforms in \cite{amrollahi2019efficiently}. They are suitable for this application because they roughly uniformly hash $\bk$ with low hamming weight. 

The use of the $\bP \in \bbF_2^{p\times n}$, the parity matrix of a binary BCH code is novel. These codes are well studied for the applications in error correction \cite{Lin1999}, and they were once the preeminent form of error correction in digital communications. A primitive, narrow-sense BCH code is characterized by its length, denoted $n_c$, dimension, denoted $k_c$ (which we want to be equal to our input dimension $n$) and its error correcting capability $t_c = 2d + 1$, where $d$ is the minimum distance of the code. For some integer $m > 3$ and $t_c < 2^{m-1}$, the parameters satisfy the following equations:
\begin{eqnarray}
    n_c &=& 2^m -1 \\
    p = n_c-k_c &\leq& mt.
\end{eqnarray}
Note that the above says we can bounds $p \leq t \left\lceil \log_2(n_c) \right \rceil$, and it is easy to solve for $p$ given $n=k_c$ and $t$, however, explicitly bounding $p$ in terms of $n$ and $t$ is difficult, so for the purpose of discussion, we simply write $p \approx{t\log(n)}$, since $n_c = p + n$, and we expect $n \gg p$ in nearly all cases of interest. 

We use the software package \verb|galois| \cite{Hostetter_Galois_2020} to construct a generator matrix, $\bG \in \bbF_2^{n_c \times k_c}$ in systematic form:
\begin{equation} \label{eq:sys-form}
    \bG = 
\begin{bmatrix}
\bI_{k_c \times k_c}\\
\bP
\end{bmatrix}
\end{equation}
\begin{algorithm}
   \caption{Collect Samples}
   \label{alg:collect}
\begin{algorithmic}[1]
   \State {\bfseries Input:} Parameters $(n, t, b, C=3, \gamma=0.9)$, Query function $f(\cdot)$ 
     \For{$j=1$ {\bfseries to} $n$, $i=1$ {\bfseries to} $b$, $c=1$ {\bfseries to} $C$} \Comment{Generate random linear code}
     \State $X_{ij} \sim \mathrm{Bern}(0.5)$
     \State $\left[\bM_c\right]_{i,j} \gets X_{i,j}$
   \EndFor
   \State $\mathrm{Code} \gets \mathrm{BCH}(n_c=n_c, k_c\geq n,  t_c=t)$ \Comment{Systematic BCH code with dimension $n$ and correcting capacity $t$}
   \State $p \gets n_c - n$
   \State $\bP \gets \mathrm{Code}.\bP$
   \State $\cP \gets \mathrm{rows}(\bP) = \left[ \boldsymbol{0}, \bp_1, \dotsc, \bp_p \right]$
   \ForAll{ $\bell \in \bbF_2^b$, $i \in \{0, \dotsc, p\}$, $c \in \{1, \dotsc, C\}$}
   \State $u_{c,i} (\bell) \gets f\left(\bM_c^\trans \bell + \cP[i] \right)$ \Comment{Query the model at masking patterns}
   \EndFor
      \ForAll{$i \in \{0, \dotsc, p\}$,  $c \in \{1, \dotsc, C\}$} 
   \State $U_{c,i} \gets \mathrm{FFT}(u_{c,i})$ \Comment{Compute the Boolean Fourier transform of the collected samples}
   \EndFor
   \State $\bU_c \gets \left[ U_{c,1}, \dotsc, U_{c, p}\right]$ 
    \State {\bfseries Output:} Processed Samples  $\bU_c,U_{c,0}\; c=1, \dotsc, C$
\end{algorithmic}
\end{algorithm}
Note that according to \eqref{eq:sys-form} $\bP \in \bbF_2^{ p \times k_c}$. In cases where $k_c > n$, we consider only the first $n$ rows of $\bP$. This is a process known as \emph{shortening}.
Our application of this BCH code in our application is rather unique. Instead of the typical use of a BCH code as \emph{channel correction} code, we use it as a \emph{joint source channel code}. 

Let $\bp_0 = \boldsymbol{0}$, and let $\bp_i,\;i=1,\dotsc,p$ correspond to the rows of $\bP$. We collect samples written as:
\begin{equation}\label{eq:subsample_apdx}
    u_{c,i} (\bell) \gets f\left(\bM_c^{\trans} \bell + \bp_i \right) \;\forall \bell \in \bbF_2^b,\; c = 1,\dotsc, C,\;i=0, \dotsc, p.
\end{equation}
Note that the total number of unique samples can be upper bounded by $C(p+1)2^{b}$. For large $n$ this upper bound is nearly always very close to the true number of unique samples collected. After collecting each sample, we compute the boolean Fourier transform. The forward and inverse transforms as we consider in this work are defined below.
\begin{equation}\label{eq:transform_def}
    \text{Forward:}\quad F(\bk) = \frac{1}{2^{n}} \sum_{\bbm \in \bbF_2^n} (-1)^{\inp{\bk}{\bbm}}f(\bbm) \qquad \text{Inverse:}\quad f(\bbm)  = \sum_{\bk \in \bbF_2^n} (-1)^{\inp{\bbm}{\bk}} F(\bk),
\end{equation}
When samples are collected according to \eqref{eq:subsample_apdx}, after applying the transform in \eqref{eq:transform_def}, the transform of $u_{c,i}$ can be written as:
\begin{equation} \label{eq:alais_apdx}
    U_{c,i}(\bj) = \sum_{\bk\; : \;\bM_c \bk = \bj} (-1)^{\inp{\bp_i}{\bk}} F(\bk).
\end{equation}
To ease notation, we write $\bU_c = [U_{c,1}, \dotsc, U_{c,p}]^T$. Then we can write
\begin{equation}\label{eq:factor_nodes}
    \bU_c(\bj) = \sum_{\bk\; : \;\bM_c \bk = \bj} (-1)^{\bP \bk} F(\bk),
\end{equation}
where we have used the notation $(-1)^{\bP \bk} = [(-1)^{\inp{\bp_0}{\bk}}, \dots, (-1)^{\inp{\bp_p}{\bk}}]^T$. We call the $(-1)^{\bP \bk}$ the \emph{signature} of $\bk$. This signature helps to identify the index of the largest interactions $\bk$, and is central to the next part of the algorithm. Note that we also keep track of $U_{c,0}(\bj)$, which is equal to the unmodulated sum $U_{c,0}(\bj) = \sum_{\bk\; : \;\bM_c \bk = \bj} F(\bk)$.
\subsection{Message Passing for Fourier Transform Recovery}
Using the samples \eqref{eq:factor_nodes}, we aim to recover the largest Fourier coefficients $F(\bk)$. To recover these samples we apply a message passing algorithm, described in detail in Algorithm~\ref{alg:message-pass}. The factor nodes are comprised of the $C2^b$ vectors $\bU_c(\bj) \; \forall \bj \in \bbF_2^b$. Each of these factor nodes are connected to all values $\bk$ that are comprise their sum, i.e., $\{\bk\mid \bM_c \bk = \bj\}$. Since the number of variable nodes is too great, we initialize the value of each variable node, which we call $\hat{F}(\bk)$ to zero implicitly. The values $\hat{F}(\bk)$ for each variable node indexed by $\bk$ represent our estimate of the Fourier coefficients. 




\subsubsection{The message from factor to variable} Consider an arbitrary factor node $\bU_c(\bj)$ initialized according to \eqref{eq:factor_nodes}. We want to understand if there are any large terms $F(\bk)$ involved in the sum in \eqref{eq:factor_nodes}. To do this, we can utilize the signature sequences $(-1)^{\bP \bk}$. If $\bU_c(\bj)$ is strongly correlated with the signature sequence of a given $\bk$, i.e., if $\abs{\inp{(-1)^{\bP \bk}}{\bU_c(\bj)}}$ is large, and $\bM_c \bk = \bj$, from the perspective of $\bU_c(\bj)$, it is likely that $F(\bk)$ is \emph{large}. Searching through all $\bM_c \bk = \bj$, which, for a full rank $\bM_c$ contains $2^{n-b}$ different $\bk$ is intractable, and likely to identify many spurious correlations. Instead, we rely on the structure of the BCH code from which $\bP$ is derived to solve this problem.

\paragraph{BCH Hard Decoding} The BCH decoding procedure is based on an idea known generally in signal processing as ``treating interference as noise". For the purpose of explanation, assume that there is some $\bk^*$ with large $F(\bk^*)$, and all other $\bk$ such that $\bM_c \bk = \bj$ correspond to small $F(\bk)$. For brevity let $\cA_c(\bj) = \{\bk\mid \bM_c \bk = \bj\}$. We can write:
\begin{equation}
    \bU_c(\bj) = F(\bk^*)(-1)^{\bP \bk^*} + \sum_{\cA_c(\bj) \setminus \bk^*} (-1)^{\bP \bk} F(\bk)
\end{equation}
After we normalize with respect to $U_{c,0}(\bj)$ this yields:
\begin{eqnarray}
    \frac{\bU_c(\bj)}{U_{c,0}(\bj)} &=& \left( \frac{1}{1 + \sum_{\cA_c(\bj) \setminus \bk^*}F(\bk)/F(\bk^*)}\right)(-1)^{\bP \bk^*} + \left(\frac{\sum_{\cA_c(\bj) \setminus \bk^*} (-1)^{\bP \bk}F(\bk)}{F(\bk^*) + \sum_{\cA_c(\bj) \setminus \bk^*} F(\bk)}\right) \\
    &=& A(\bj) (-1)^{\bP \bk^*} + \bw(\bj). \label{eq:ratio}
\end{eqnarray}
As we can see, the ratio \eqref{eq:ratio} is a noise-corrupted version of the signature sequence of $\bk^*$. To estimate $\bP \bk$ we apply a nearest-neighbor estimation rule outlined in Algorithm~\ref{alg:bch-hard}. In words, if the $i$th coordinate of the vector \eqref{eq:ratio} is closer to $-1$ we estimate that the corresponding element of $\bP \bk$ to be $1$, conversely, if the $i$th coordinate is closer to $1$ we estimate the corresponding entry to be $0$. This process effectively converts the multiplicative noise $A$ and additive noise $\bw$ to a noise vector in $\bbF_2$. We can write this as $\bP \bk^{*} + \bn$. According to the Lemma~\ref{lem:decoding} if the hamming weight $\bn$ is not too large, we can recover $\bk^*$. 

\begin{lemma}\label{lem:decoding}
    If $\abs{\bn} + \abs{\bk^*} \leq t$, where $\bn$ is the additive noise in $\bbF_2$ induced by the noisy process in \eqref{eq:ratio} and the estimation procedure in Algorithm~\ref{alg:bch-hard}, then we can recover $\bk^*$.
\end{lemma}
\begin{proof}
    Observe that the generator matrix of the BCH code is given by \eqref{eq:sys-form}. Thus, there exists a codeword of the form
    \begin{equation}
    \bc = \bG \bk^*= 
\begin{bmatrix}
\bk^*\\
\bP \bk^*
\end{bmatrix}
\end{equation}
Now construct the ``received codeword" as in Algorithm~\ref{alg:bch-hard}:
    \begin{equation}
    \br = 
\begin{bmatrix}
\boldsymbol{0}\\
\bP \bk^* + \bn
\end{bmatrix}
\end{equation}
Thus $\abs{\bc- \br} = \abs{\bn} + \abs{\bk^*}$. Since the BCH code was designed to be $t$ error correcting, Decoding the code will recover $\bc$, which contains $\bk^*$.
\end{proof}
For decoding we use the implementation in the python package \verb|galois| \cite{Hostetter_Galois_2020}. It implements the standard procedure of the Berlekamp-Massey Algorithm followed by the Chien Search algorithm for BCH decoding. 
\begin{algorithm}
   \caption{BCH Hard Decode}
   \label{alg:bch-hard}
\begin{algorithmic}[1]
   \State {\bfseries Input:} Observation $\bU_c(\bj)$, Decoding function $\mathrm{Dec}(\cdot)$
   \State $r_i \gets 0 \; i=1\dotsc, n$
   \ForAll{$i \in n+1, \dotsc, n+p$}
        \State $r_i \gets \mathds{1}\left\{ \frac{U_{c,i}(\bj)}{U_{c,0}(\bj) } < 0 \right\}$
   \EndFor
    \State dec, $ \hat{\bk} \gets \mathrm{Dec}(\br)$
    \State {\bfseries Output:} dec, $\hat{\bk}$ 
\end{algorithmic}
\end{algorithm}

\paragraph{BCH Soft Decoding} In practice the conversion of the real-valued noisy observations \eqref{eq:ratio} to noisy elements in $\bbF_2$ is a process that destroys valuable information. In coding theory, this is known as \emph{hard input} decoding, which is typically suboptimal. For example, certain coordinates will have values $\frac{U_{c,i}(\bj)}{U_{c,0}(\bj)} \approx 0$. For such coordinates, we have low confidence about the corresponding value of $(-1)^{\inp{\bp_i}{\bk^*}}$, since it is equally close to $+1$ and $-1$. This uncertainty information is lost in the process of producing a hard input. With this so-called \emph{soft information} it is possible to recover $\bk^*$ even in cases where there are more than $t$ errors in the hard decoding case. We use a simple soft decoding algorithm for BCH decoding known as a chase decoder. The main idea behind a chase decoder   is to perform hard decoding on the $d_{\text{chase}}$ most likely hard inputs, and return the decoder output of the most likely hard input that successfully decoded. In practical setting like the ones we consider in this work, we don't have an understanding of the noise in \eqref{eq:ratio}. A practical heuristic is to simply look at the \emph{margin} of estimation. In other words, if $\abs{\frac{U_{c,i}(\bj)}{U_{c,0}(\bj)}}$ is large, we assume it has high confidence, while if it is small, we assume the confidence is low. Interestingly, if we assume $A(\bj) = 1$ and $\bw(\bj) \sim \cN(0, \sigma^2)$ in \eqref{eq:ratio}, then the ratio corresponds exactly to the logarithm of the likelihood ratio (LLR) $\log \left( \frac{\mathrm{Pr}\left(\inp{\bp_i}{\bk^*} = 0\right)}{\mathrm{Pr}\left(\inp{\bp_i}{\bk^*} = 1\right)}\right)$. For the purposes of soft decoding we interpret these ratios as LLRs. Pseudocode can be found in Algorithm~\ref{alg:bch-soft}.


\emph{Remark: BCH soft decoding is a well-studied topic with a vast literature. Though we put significant effort into building a strong implementation of \SpecExp{}, we have used the simple Chase Decoder (described in Algorithm~\ref{alg:bch-soft} below) as a soft decoder. The computational complexity of Chase Decoding scales as $2^{d_\text{chase}}$, but other methods exist with much lower computational complexity and comparable performance.}

\begin{algorithm}
   \caption{BCH Soft Decode (Chase Decoding)}
   \label{alg:bch-soft}
\begin{algorithmic}[1]
   \State {\bfseries Input:} Observation $\bU_c(\bj)$, Decoding function $\mathrm{Dec}(\cdot)$, Chase depth $d_{\text{chase}}$.
   \State $r_i \gets 0 \; i=1\dotsc, n$
   \State $\cR \gets d_{\text{chase}}$ most likely hard inputs \Comment{Can be computed efficiently via dynamic programming}
   \State dec $\gets False$
   \State $j \gets 0$
   \While{$dec$ is $False$ and $j \leq d_{\text{chase}}$}
        \State $\br_{(n+1):(n+p)} \gets \cR [j]$
        \State $j \gets j+1$
        \State dec, $ \hat{\bk} \gets \mathrm{Dec}(\br)$
   \EndWhile
    \State {\bfseries Output:} dec, $\hat{\bk}$ 
\end{algorithmic}
\end{algorithm}

If we successfully decode some $\bk$ from the BCH decoding process via the bin $\bU_{c}(\bj)$, we construct a message to the corresponding variable node. Before we do this, we verify that the $\bk$ term satisfies $\bM_c \bk = \bj$. This acts as a final check to increase our confidence in the output of $\bk$. The message we construct is of the following form:
\begin{equation}\label{eq:check_msg}
    \mu_{(c,\bj) \rightarrow \bk} = \inp{(-1)^{\bP \bk}}
        {\bU_c(\bj)}/p
\end{equation}
To understand the structure of this message. This message can be seen as an estimate of the Fourier coefficient. Let's assume we are computing this message for some $\bk^*$:   
\begin{equation}
\mu_{(c,\bj) \rightarrow \bk^*} = F(\bk^*) + \sum_{\cA(\bj)\setminus \bk^*}\underbrace{\frac{1}{p}\inp{(-1)^{\bP \bk}}{(-1)^{\bP \bk^*}}}_{\text{typically small}}F(\bk)
\end{equation}
The inner product serves to reduce the noise from the other coefficients in the sum.
\begin{algorithm}
   \caption{Message Passing}
   \label{alg:message-pass}
\begin{algorithmic}[1]
\State {\bfseries Input:} Processed Samples  $\bU_c, c=1, \dotsc, C$
\State $\cS = \left\{ (c,\bj): \bj \in \bbF_2^b, c \in \{1, \dotsc, C\}\right\}$ \Comment{Nodes to process}
\State $\hat{F}[\bk] \gets 0 \;\forall\bk$
\State $\cK \gets \emptyset$
\While{$\abs{\cS} > 0$} \Comment{Outer Message Passing Loop}
    \State $\cS_{\text{sub}} \gets \emptyset$
    \State $\cK_{\text{sub}} \gets \emptyset$
    \For{$(c,\bj) \in \cS$}
        \State dec, $\bk$ $\gets \mathrm{DecBCH} (\bU_c(\bj))$ \Comment{Process Factor Node}
        \If{dec}
            \State corr $\gets \frac{\inp{(-1)^{\bP \bk}}{\bU_c(\bj)}}{\norm{\bU_c(\bj)}^2}$
        \Else
            \State corr $\gets 0$
        \EndIf
        \If{corr $> \gamma$} \Comment{Interaction identified}
            \State $\cS_{\text{sub}} \gets \cS_{\text{sub}} \cup \{(\bk, c, \bj)\}$
            \State $\cK_{\text{sub}} \gets \cK_{\text{sub}} \cup \{\bk\}$
        \Else
            \State $\cS \gets \cS \setminus \{ (c, \bj)\}$ \Comment{Cannot extract interaction}
        \EndIf
    \EndFor
    \For{ $\bk \in \cK_{\text{sub}}$}
        \State $\cS_{\bk} \gets \{ (\bk', c', \bj') \mid (\bk', c', \bj') \in \cS_{\text{sub}}, \bk' = \bk \}$
        \State $\mu_{(c,\bj) \rightarrow \bk} \gets \inp{(-1)^{\bP \bk}}
        {\bU_c(\bj)}/p$ 
        \State $\mu_{\bk \rightarrow \text{all}} \gets \sum_{(\bk, c, \bj) \in \cS_{\bk}} \mu_{(c,\bj) \rightarrow \bk}$
        \State $\hat{F}(\bk) \gets \hat{F}(\bk) + \mu_{\bk \rightarrow \text{all}}$ \Comment{Update variable node}
        \For{$c \in \{ 1, \dotsc, C\}$}
            \State $\bU_c(\bM_c \bk) \gets \bU_c(\bM_c \bk) - \mu_{\bk \rightarrow \text{all}}\cdot(-1)^{\bP \bk}$ \Comment{Update factor node}
            \State $\cS \gets \cS \cup \{ (c, \bM_c \bk)\}$
        \EndFor
    \EndFor
    \State $\cK \gets \cK \cup \cK_{\text{sub}}$
\EndWhile
  \State {\bfseries Output: $\left\{ \left(\bk, \hat{F}(\bk)\right) \mid \bk \in \cK\right\}$}, interactions, and scalar values corresponding to interactions.
\end{algorithmic}
\end{algorithm}
\subsubsection{The message from variable to factor}
The message from factor to variable is comparatively simple. The variable node takes the average of all the messages it receives, adding the result to its state, and then sends that average back to all connected factor nodes. These factor nodes then subtract this value from their state and then the process repeats. 


\subsection{Computational Complexity}

\paragraph{Generating masking patterns $\bbm$} Constructing each masking pattern requires $n2^b$ for each $\bM_c$. The algorithm for computing it efficiently involves a gray iteratively adding to an $n$ bit vector and keeping track of the output in a Gray code. Doing this for all $C$, and then adding all $p$ additional shifting vectors makes the cost $O(Cpn2^b)$.

\paragraph{Taking FFT} For each $u_{c,i}$ we take the Fast Fourier transform in $b2^b$ time, with a total of $O(Cpb2^b)$. This is dominated by the previous complexity since, $b \leq n$

\paragraph{Message passing} One round of BCH hard decoding is $O(n_ct + t^2)$. For soft decoding, this cost is multiplied by $2^{d_{\text{chase}}}$, which we is a constant.  Computing the correlation vector is $O(np)$, dominated by the computation of $\bP \bk$. In the worst case, we must do this for all $C 2^b$ vectors $\bU_c(\bj)$. We also check that $\bM \bk = \bj$ before sending the message, which costs $O(nb)$. Thus, processing all the factor nodes costs $O(C2^b(n_c t + t^2 + n(p+b)))$. The number of active (with messages to send) variable nodes is at most $C2^b$, and computing their factors is at most $C$. Thus, computing factor messages is at most $C^22^b$ messages. Finally, factor nodes are updated with at most $C2^b$ variable messages sending messages to at most $C$ factor nodes each, each with a cost of $O(np)$. Thus, the total cost of processing all variable nodes is $O(C^22^b + C^22^bnp)$. The total cost of message is dominated by processing the factors. 

The total complexity is then $O(2^b(n_c t + t^2 + n(p + b))$.
Note that $p = n_c - n = t \log(n_c)$. Due to the structure of the code and the relationship between $n,p$ and $n_c$, one could stop here, and it would be best to if we want to consider very large $t$. For the purposes of exposition, we will assume that $t \ll n$, which implies $n > p$, and thus $p \approx t \log(n)$. In this case, we can write:
\begin{equation}
    \text{Complexity} = O(2^b(nt\log(n)  + nb))
\end{equation}

To arrive at the stated equation in Section~\ref{sec:intro}, we take $2^b = O(s)$. Under the low degree assumption, we have $s = O(d\log(n))$. Then assuming we take $t= O(d)$, we arrive at a complexity of $O(sdn\log(n))$.

\section{Experiment Details}\label{apdx:experiments}
\subsection{Choice of weight sequence}
\label{apdx-sec:weight-seq}
In all our experiments we draw the index $s$, at time $t_i$, from $\mbox{Uniform}\intset{\tau}{t_{i-1}}$ with $\tau = 10$. The main motivation behind setting $\tau = 10$ and not $\tau = 1$, which is more natural, is that we have found that otherwise it may lead to instabilities. This arises typically when an index $s$ is sampled very close to $0$ when $t \approx T$. To avoid such behavior we use a smaller learning rate in \Cref{algo:gauss_vi} for the first few iterations and set $\tau > 1$. For the last $25\%$ diffusion steps we set $s$ deterministically to $t_{i-1}$ as we have found that this slightly improves the reconstructions quality. We also ramp up the number of gradient steps as this significantly sharpens the details in the images. 

While it is more intuitive to sample $s$ close to $0$ as it provides the best approximation error for the likelihood, we have found that this can significantly slow the mixing of the Gibbs sampler in very large dimensions and provides rather poor results when used with a small number of Gibbs steps. Practically speaking, significant artifacts arise during the initial iterations of the algorithm due to the optimization procedure, and they tend to persist in subsequent iterations when $s$ is sampled close to 0. To see why this is the case consider the following empirical discussion on a simplified scenario. We write $\bx = [\bar\bx, \underline\bx]$  where $\bar\bx \in \rset^\dimobs$ and $\underline\bx \in \rset^{\dimx - \dimobs}$. 
We assume that $\pot{0}{\bx} = \normpdf(\obs; \bar\bx, \stdobs^2 \Id_\dimobs)$, \emph{i.e.}, we observe only the first $\dimobs$ coordinates of the hidden state. Since $s$ is sampled near $0$ we may assume that $\hpot{s}{} = \pot{0}{}$. Then, sampling $Z \sim \epost{s|0, t}{\bx_0, \bx_t}{}$  is equivalent to sampling 
\begin{align*}
    \bar{Z} & \sim \gauss\left(\frac{\std^2 _{s|0, t}}{\stdobs^2 + \std^2 _{s|0,t}} \obs + \frac{\stdobs^2}{\stdobs^2 + \std^2 _{s|0, t}} \big[ \gamma_{t|s} \a_{s|0} \bar\bx_0 + (1 - \gamma_{t|s}) \a^{-1} _{t|s} \bar\bx_t\big], \frac{\stdobs^2 \std^2 _{s|0,t}}{\stdobs^2 + \std^2 _{s|0,t}} \Id_\dimobs\right) \eqsp, \\
    \underline{Z} & \sim \gauss(\gamma_{t|s} \a_{s|0} \underline\bx_0 + (1 - \gamma_{t|s}) \a^{-1} _{t|s} \underline\bx_t, \sigma^2 _{s|0,t} \Id_{\dimx - \dimobs}) \eqsp,
\end{align*}
setting $Z = [\bar{Z}, \underline{Z}]$ and then concatenating both vectors. It is thus seen that the observed part of the state is updated with the observation whereas the bottom part is simply drawn from the prior. Moreover, if $\std^2 _{\smash{s|0, t}} \approx 0$ then $\gamma_{t|s} \a_{s|0} \approx 1$ and $\underline{Z}$ is almost the same as $\bx_0$.  In \Cref{algo:midpoint-gibbs}, once we have sampled $\vX_s \sim \epost{\smash{s|0,t}}{\vX_0, \vX_t}{}$, we first denoise it to obtain the new $\vX_0$ and then noise it to obtain the new $\vX_t$. As $s$ is sampled near 0, the denoising step will merely modify $\vX_s$ whereas the noising step will add significant noise to $\vX_s$ and may help with removing the artifacts. This noised sampled has however only a small impact on the next samples $\vX_0, \vX_s$ since $(1 - \gamma_{t|s}) \a^{-1} _{t|s} \approx 0$. In short, the first $\dimobs$ coordinates of the running state $\vX^* _0$ will be quickly replaced by the observation whereas the last $\dimx - \dimobs$ coordinates will be stuck at their initialization and will evolve only by a small amount throughout the iterations of the algorithm. We illustrate this situation on a concrete example in \Cref{fig:sampling-comparison} where we consider a half mask inpainting task. The first and second rows show the evolution of the running state $\vX^* _0$ with the time-sampling distributions  
\begin{align}
    \label{eq:sampling-dist-mix}
  & \mu^* _{i} = \begin{cases} \mbox{Uniform}\intset{\tau}{t_{i-1}} \, & \quad \text{if} \quad i > \lfloor  K / 4 \rfloor \\
                    t_{i-1} \, & \quad \text{else} 
\end{cases},\\
\label{eq:sampling-dist-zero}
& \mu^0 _i = \mbox{Uniform}\intset{1}{\lfloor t_i / 5 \rfloor} \eqsp, 
\end{align}
\emph{i.e.}, the time-sampling distribution we use in all our experiments, where $K$ is the number of diffusion steps, and the one that we use to sample only close to $0$, respectively. In \Cref{table:sampling-comparison} we compute the LPIPS for both distributions on a subset of the tasks we consider in the main paper. It is clear that $\mu^* _i$ outperforms $\mu^0 _i$, even when we increase the number of Gibbs steps (see phase retrieval task). 
% The unobserved part of the image contains significant errors during the first few steps due to the initialization and do not vanish afterwards whereas after a few iterations the unmasked part is recovered almost perfectly. 
% This is to be expected when $s$ is small. Indeed, consider the following empirical analysis with the simplified scenario where we assume that we obs. we may assume in this case that $\hpot{s}{\bx_s} = \pot{s}{\bx_s}$ and then, sampling perfectly from $\epost{s|0, t}{\bx_0, \bx_t}{}$ is equivalent to sampling 


%  and these are usually forgotten throughout the subsequent iterations when we sample $s$ uniformly on $\intset{1}{t_{i-1}}$. 
% We hypothesize that this is due to the combination of random timestep sampling, noising and denoising steps of the algorithm. Indeed, assume that in the first few iterations of the algorithm, the running state $\vX^* _0$ (see \Cref{algo:midpoint-gibbs}) contains significant artifacts. Then, once an index $s$ 

% However, when $s$ is sampled too close to $0$ they are not forgotten fast enough. 
\begin{table} 
    \centering 
    \caption{LPIPS on the \ffhq\ dataset for the two time-sampling distributions given in \eqref{eq:sampling-dist-mix} and \eqref{eq:sampling-dist-zero}. We use $R = 4$ Gibbs steps for the phase retrieval task.}
    \resizebox{0.60\textwidth}{!}{
    \begin{tabular}{l cccc}
        \toprule
        Distribution & Phase retrieval ($R = 4$) & JPEG2 & Gaussian deblurring & Motion deblurring \\
        \midrule
        $\mu^* _t$  & \textbf{0.10} & \textbf{0.14} & \textbf{0.12} & \textbf{0.09} \\
        $\mu^0 _t$ & 0.53 & 0.19 & 0.16 & 0.19\\
        \bottomrule
    \end{tabular} 
    }
    \label{table:sampling-comparison}
\end{table}
\begin{figure}
\centering 
\includegraphics[width=\textwidth]{figures/tau_sampling.jpg}
\caption{Evolution of the running state $\vX^* _0$ in \Cref{algo:midpoint-gibbs} for the two time-sampling distributions given in \eqref{eq:sampling-dist-mix} and \eqref{eq:sampling-dist-zero}. }
\label{fig:sampling-comparison}
\end{figure}

\subsection{Hyperparameters setup of \algo}
\label{apdx-sec:hyperparameters}
The details about the hyperparameters of \algo\ are reported in \Cref{table:hyperparams-algo}.
We adjust the optimization of the Gaussian Variational approximation in \Cref{algo:gauss_vi} during the first and last diffusion steps.
We ramp up the number of gradient steps during the final diffusion steps.
This allows us to substantially improve the fine grained details of the reconstructions. 
Similarly, we reduce the learning rate in the early step to alleviate potential instabilities.


\begin{table}[ht]
    \centering
    \caption{The hyperparameters used in \algo\ for the considered datasets. The index $i$ of the timesteps $\{t_i\}_{i=K}^0$ is taken in reverse order. The symbol \# stands for \emph{``number of''}.}
    \vspace{-0.2cm}
    \renewcommand{\arraystretch}{1.3} % Adjust row spacing if needed
    \resizebox{\textwidth}{!}{
    \begin{tabular}{l cccccc}
        \toprule
        & \# Gibbs repetitions $R$ & \# Diffusion steps $K$ & \# Denoising steps $M$ & Time-sampling distribution & Learning rate $\eta$ & \# Gradient steps $G$ \\
        \midrule
        \ffhq & $R=1$ & $K=100$ & $M=20$ & $\mu^* _{i}$ as in \eqref{eq:sampling-dist-mix} & $\eta=\begin{cases}
            0.01  & \text{ if } i \geq \lfloor 3K/4 \rfloor \\
            0.03  & \text{ otherwise} \\
            \end{cases}$ 
            & 
            $ G = \begin{cases}
            20 & \text{ if } i \leq \lfloor K/4 \rfloor \\
            5  & \text{ otherwise} \\
            \end{cases}$
        \\
        \midrule
        \ffhq\ LDM & $R=1$ & $K=100$ & $M=20$ & $\mu^* _{i}$ as in \eqref{eq:sampling-dist-mix}  & $\eta=\begin{cases}
            0.01  & \text{ if } i \geq \lfloor 3K/4 \rfloor \\
            0.03  & \text{ otherwise} \\
            \end{cases}$ 
            &
            $ G = \begin{cases}
            20 & \text{ if } i \leq \lfloor K/4 \rfloor \\
            20 & \text{ if } i \mod 10 = 0   \\
            3  & \text{ otherwise} \\
            \end{cases}$
        \\
        \midrule
        \imagenet & $R=1$ & $K=100$ & $M=20$ & $\mu^* _{i}$ as in \eqref{eq:sampling-dist-mix} & $\eta=\begin{cases}
            0.01  & \text{ if } i \geq \lfloor 3K/4 \rfloor \\
            0.03  & \text{ otherwise} \\
            \end{cases}$ 
            &
            $ G = \begin{cases}
            20 & \text{ if } i \leq \lfloor K/4 \rfloor \\
            5  & \text{ otherwise} \\
            \end{cases}$
        \\
        \midrule
        Audio-source separation & $R=6$ & $K=20$ & $M=15$ & $\mu^* _{i}$ as in \eqref{eq:sampling-dist-mix} & $\eta=0.005$ &  $ G = \begin{cases}
            20 & \text{ if } i \leq \lfloor K/4 \rfloor \\
            3  & \text{ otherwise} \\
            \end{cases}$
        \\
        \midrule
        %
        \parbox[m]{10em}{Audio-source separation\\ (Best result in \Cref{table:si-snri})}
        & $R=1$ & $K=20$ & $M=15$ & $\mu^* _{i}$ as in \eqref{eq:sampling-dist-mix} & $\eta=0.005$ &  $ G = 90$
        \\
        \bottomrule
    \end{tabular}
    \label{table:hyperparams-algo}
    }
\end{table}


\subsection{Audio source separation}
% \paragraph*{Audio diffusion model.}

% For audio data, we consider a setup where the input consists of multiple audio tracks of varying lengths, each containing $N$ distinct source waveforms $\{\mathbf{x}_1, \dots, \mathbf{x}_N\}$, which coherently sum to form a mixture $\mathbf{y} = \sum_{i=1}^N \mathbf{x}_i$. A diffusion model is trained to learn the prior for this setup.

In our experiment, the diffusion model employed provided by \cite{mariani2023multi} is trained on the \slakh\ training dataset\footnote{\url{http://www.slakh.com/}},  using only the four abundant instruments (bass, drums, guitar and piano) downsampled to 22 kHz. The denoiser network is based on a non-latent, time-domain unconditional variant of \citep{schneider2023musai}.

Its architecture follows a U-Net design, comprising an encoder, bottleneck, and decoder. The encoder consists of six layers with channel numbers $[256, 512, 1024, 1024, 1024, 1024]$, where each layer includes two convolutional ResNet blocks, and multihead attention is applied in the last three layers. The decoder mirrors the encoder structure in reverse. The bottleneck contains a ResNet block, followed by a self-attention mechanism, and then another ResNet block. Training is performed on the four stacked instruments using the publicly available trainer from repository\footnote{\url{https://github.com/archinetai/audio-diffusion-pytorch-trainer}}.

\subsection{Implementation of the competitors}
\label{apdx:competitors}
In this section, we provide implementation details of the competitors.
We adopt the hyperparameters recommended by the authors tune them on each dataset if they are not provided.
The complete set of hyperparameters and there values for both image experiments and audio-sound separation can be found in the supplementary material under the folders \texttt{configs\slash experiments/sampler} and \texttt{configs\slash exp\_sound/sampler}.


\paragraph*{DPS.}
We implemented \citet[Algorithm~1]{chung2023diffusion} and selected the hyperparameters of each considered task  based on \citet[App.~D]{chung2023diffusion}.
We tuned the algorithm for the other tasks, namely, we use $\gamma = 0.2$ for JPEG $2\%$, $\gamma = 0.07$ for High Dynamic Range tasks, and $\gamma = 1$ for audio-source separation.

\paragraph*{DiffPIR.}
We implemented \citet[Algorithm 1]{zhu2023denoising} to make it compatible with our existing code base.
We adopt the hyperparameters recommended in the official, released version\footnote{\url{https://github.com/yuanzhi-zhu/DiffPIR}}.
We followed the guidelines in \citep[Eqn. (13)]{zhu2023denoising} to extend the algorithm to nonlinear problems.
However, we noticed that the algorithm diverges in these cases and we could not follow up as the paper and the released code lack examples of nonlinear problems.
\citet{zhu2023denoising} provides an FFT-based solution for the motion blur tasks which is only valid in the case of circular convolution.
Hence, and since we adapted the experimental setup of \citet{chung2023diffusion}, we do not run the algorithm on motion blur task as it uses convolution with reflect padding. For audio-source separation, we found that $\lambda = \mu = 1$ works best.

\paragraph*{DDNM.}
We adapted the implementation provided in the released code\footnote{\url{https://github.com/wyhuai/DDNM}}.
Namely, the authors provide classes, in the module \texttt{functions\slash svd\_operators.py} that implement the logic of the algorithm on each degradation operator separately.
The adaptation includes factorizing these classes to a single class to support all SVD linear degradation operators.
On the other hand, we notice \ddnm\ is unstable for operators whose SVD decomposition is prone to numerical errors, such as Gaussian Blur with wide convolution kernel. This results from the algorithm using the pseudo-inverse of the operator.

\paragraph*{RedDiff.}
We used the implementation of \reddiff\ available in the released code\footnote{\url{https://github.com/NVlabs/RED-diff}}.
For linear problems, we use the pseudo-inverse of the observation as an initialization of the variational optimization problem. 
On nonlinear problems, for which the pseudo-inverse of the observation is not available, we initialized the optimization with a sample from the standard Gaussian distribution. 

\paragraph*{PGDM.}
We opted for the implementation available in the \reddiff\'s repository as some the authors are co-authors of \pgdm as well.
Notably, the implementation introduces a subtle deviation from \citet[Algorithm 1]{song2022pseudoinverse}: in the algorithm's final step, the guidance term $g$ is scaled by $\a_t$ ($\sqrt{\a_t}$ in their notation) whereas the implementation scales it by $\a_{t-1}\a_t$.
This adjustment improves the algorithm for most tasks except for JPEG dequantization. We found that the original scaling by $\a_t$ is better in this case.

\paragraph*{PSLD.}
We implemented the \psld\ algorithm provided in \citet[Algorithm 2]{rout2024solving} and referred to the publicly available implementation\footnote{\url{https://github.com/LituRout/PSLD}} to set the hyperparameters of the algorithm for the different tasks.

\paragraph*{ReSample.}
We modified the original code\footnote{\url{https://github.com/soominkwon/resample}} provided by the authors to make its hyperparameters directly adjustable, namely, the tolerance $\varepsilon$ and the maximum number of iterations $N$ for solving the optimization problems related to hard data consistency, and the scaling factor for the variance of the stochastic resampling distribution $\gamma$.
We found the algorithm to be sensitive to $\varepsilon$ and that setting it to the noise level of the inverse problem yields the best reconstructions across tasks and noise levels.
On the other hand, we noticed that $\gamma$ has less impact on the quality of the reconstructions.
Finally, we set a threshold $N=200$ on the maximum number of gradient iterations to make the algorithm less computationally intensive.

\paragraph*{DAPS.}
We have the official codebase \footnote{\url{https://github.com/zhangbingliang2019/DAPS}}.
We referred to \citet[Table.~7]{zhang2024daps} to set the hyperparameters.
% For audio-source separation, we modified the code to pass in direclty the diffusion model as it was built and trained following the Variance Exploding setup of \citet{karras2022elucidating}.
For audio-source separation, we set $\sigma_{\max}$ and $\sigma_{\min}$ to match those of the sound model and adapted the Langevin stepsize \texttt{lr} and the standard deviation \texttt{tau} to the audio-separation task.

\paragraph*{PNP-DM.}
We adapted the implementation provided in the released code\footnote{\url{https://github.com/zihuiwu/PnP-DM-public/}}.
Specifically, we exposed the coupling parameter $\rho$ including its initial value, minimum value, and decay rate, as well as the number of Langevin steps and its step size.
The hyperparameters were set based on \citet[Table 3 and Table 4]{wu2024pnpdm}.
For inpainting tasks, while it is theoretically possible to perform the likelihood steps using Gaussian conjugacy \citep[Sec.~3.1]{wu2024pnpdm}, we found that using Langevin produced better results in practice. For example, the reconstructions in the left figure of \Cref{fig:pnpdm-conjugacy} are obtained by sampling exactly from the posterior whereas on the \rhs\ we use Langevin dynamics. 
% For audio source separation, we pass in the model directly to the algorithm, following the approach that we used in \daps.
Although the audio separation task is linear and hence the likelihood steps can be implemented exactly, we encountered similar challenges as in inpainting and therefore we used Langevin here as well.


\subsection{Experiments reproducibility}
Our code will be made available upon acceptange of the paper. In the anonymous codebase provided as companion of the paper we use $\sqrt{\a_t}$ instead of $\a_t$ to match the conventions of existing codebases.  All experiments were conducted on Nvidia Tesla V100 SXM2 GPUs. 
For the image experiments, we used $300$ images from the validation sets of \ffhq\ and \imagenet\ $256 \times 256$ that we numbered from $0$ to $299$.
The image number was used to seed the randomness of the experiments on that image.
For the audio source separation experiments, the \slakh\ test dataset has tracks named following the pattern \texttt{Track0XXXX}, where \texttt{X} represents a digit in $0-9$.
The number \texttt{XXXX} was used as the seed for the experiments conducted on each track.


% \subsection{Runtime and memory requirement comparison}
% We evaluate the runtime and GPU memory consumption for image experiments on the three considered diffusion model priors.
% Since not all algorithms support every task, we restrict the evaluation to commune tasks.
% \Cref{fig:runtime-gpu} presents the average runtime and GPU memory requirement over both samples and tasks.

% \begin{figure}[htb]
%     \centering
%     \subfigure{
%         \includegraphics[height=0.2\textwidth]{figures/runtime-gpu/ffhq-ldm.pdf}   
%     }
%     \subfigure{
%         \includegraphics[height=.2\textwidth]{figures/runtime-gpu/ffhq.pdf}
%         }
%     \subfigure{
%             \includegraphics[height=.2\textwidth]{figures/runtime-gpu/imagenet.pdf}
%         }
%     \caption{Comparison of the runtime (red bars -- left axis) and memory requirement (blue bars -- right axis) between the considered algorithms on \ffhq\ latent space (1\textsuperscript{st} row), \ffhq\ pixel space (2\textsuperscript{nd} row), and \imagenet\ (3\textsuperscript{rd} row).}
%     \label{fig:runtime-gpu}
% \end{figure}


\subsection{Extended results}
\label{apdx:extended-results}
We present the complete table with LPIPS, PSNR, and SSIM metrics for the image inverse problems experiment in \Cref{table:extended-ffhq-imagenet} for the \ffhq\ and \imagenet\ datasets, and in \Cref{table:extended-ffhq-ldm} for \ffhq\ LDM. Similarly, the complete results for the audio source separation experiments that include all competitors are provided in \Cref{table:extended-si-snri}.

From \Cref{table:extended-ffhq-imagenet}, one can note that \ddnm, \diffpir\ and \daps\ score better in PSNR and SSIM compared to \algo\; but score lower in LPIPS.
For most of the tasks we considered, one does not expect to recover an image very close to the reference and thus, metrics that perform pixel-wise comparisons are less relevant and favor images that are overly smooth.
We provide evidence for this in the gallery of images below where we compare qualitatively the outputs of our algorithm with those of the competitors.
It can be seen that our method provides reconstructions with ine-grained details that more coherent with the reference image.
Note for example that \ddnm, \diffpir\ and \daps\ outperform \algo\ in terms of PSNR and SSIM on the half mask task on \imagenet\ while failing to reconstruct the missing \rhs\ of the images. 




\begin{table}[h]
    \centering
    \caption{Mean LPIPS/PSNR/SSIM metrics for the considered linear and nonlinear imaging tasks on the \ffhq\ and \imagenet\ $256 \times 256$ datasets with $\stdobs = 0.05$.}
    \resizebox{\textwidth}{!}{
    \begin{tabular}{l cccccccc | cccccccc}
        \toprule
        \vspace{1mm}
        & \multicolumn{8}{c}{\bf{\ffhq}} & \multicolumn{8}{c}{\bf{\imagenet}} \\
        % \cmidrule(lr){2-7} \cmidrule(lr){8-13}
        \textbf{Task} & \algo\ & \dps & \pgdm & \ddnm & \diffpir & \reddiff & \daps & \pnpdm \ &\ \algo\ & \dps & \pgdm & \ddnm & \diffpir & \reddiff & \daps & \pnpdm \\
        % --- lpips
        \midrule
        & \multicolumn{16}{c}{LPIPS \ $\downarrow$} \\
        \midrule
        SR ($\times 4$)        & \first{0.09} & \first{0.09} & 0.30 & \third{0.15} & \second{0.10} & 0.39 & 0.16 & \second{0.10} \ &\ \second{0.26} & \first{0.25} & 0.56 & 0.34 & \third{0.31} & 0.57 & 0.37 & 0.66 \\
        SR ($\times 16$)       & \second{0.24} & \first{0.23} & 0.42 & 0.33 & \first{0.23} & 0.55 & 0.40 & \third{0.29} \ &\ \third{0.55} & \first{0.44} & 0.62 & 0.71 & \second{0.50} & 0.85 & 0.75 & 1.03 \\
        Box inpainting         & \first{0.10} & 0.17 & 0.17 & \second{0.12} & 0.14 & 0.19 & \third{0.13} & 0.18 \ &\ \first{0.23} & 0.35 & \third{0.29} & \second{0.28} & 0.30 & 0.36 & 0.30 & 0.42 \\
        Half mask              & \first{0.20} & \third{0.24} & \third{0.24} & \second{0.23} & 0.25 & 0.28 & \second{0.23} & 0.32 \ &\ \first{0.31} & 0.40 & \second{0.34} & \third{0.38} & 0.40 & 0.46 & 0.40 & 0.54 \\
        Gaussian Deblur        & \first{0.12} & \third{0.17} & 0.87 & 0.20 & \first{0.12} & 0.24 & 0.24 & \second{0.14} \ &\ \first{0.30} & \second{0.37} & 1.00 & \third{0.45} & \first{0.30} & 0.53 & 0.59 & 0.76 \\
        % \vspace{2mm} % hack to leave space between linear and nonlinear task
        Motion Deblur          & \first{0.09} & \second{0.17} & $-$ & $-$ & $-$ & 0.22 & \third{0.19} & 0.21 \ &\ \first{0.22} & \third{0.40} & $-$ & $-$ & $-$ & \second{0.39} & 0.42 & 0.52 \\
        % \cmidrule(lr){1-13}
        JPEG (QF = 2)          & \first{0.14} & 0.34 & 1.12 & $-$ & $-$ & 0.32 & \second{0.22} & \third{0.29} \ &\ \first{0.38} & 0.60 & 1.32 & $-$ & $-$ & \third{0.49} & \second{0.45} & 0.56 \\
        Phase retrieval        & \first{0.11} & 0.40  & $-$ & $-$ & $-$ & \third{0.26} & \second{0.14} & 0.34 \ &\ \second{0.55} & 0.62 & $-$ & $-$ & $-$ & \third{0.61} & \second{0.50} & 0.66 \\
        Nonlinear deblur       & \first{0.27} & 0.51 & $-$ & $-$ & $-$ & 0.68 & \second{0.28} & \third{0.31} \ &\ \first{0.41} & 0.82 & $-$ & $-$ & $-$ & \third{0.66} & \first{0.41} & \second{0.49} \\
        HDR    & \second{0.12} & 0.40 & $-$ & $-$ & $-$ & 0.20 & \second{0.10} & \third{0.19} \ &\ \third{0.21} & 0.84 & $-$ & $-$ & $-$ & \second{0.19} & \first{0.14} & 0.31 \\
        % --- PSNR
        \midrule
        & \multicolumn{16}{c}{PSNR \ $\uparrow$} \\
        \midrule
        SR ($\times 4$)        & 27.66 & \third{28.05} & 24.57 & \first{29.45} & 27.72 & 26.75 & \second{28.44} & 27.44 \ &\ 23.88 & \third{24.37} & 18.45 & \first{24.99} & 23.43 & 23.33 & \second{24.38} & 16.4 \\
        SR ($\times 16$)       & \third{21.01} & 20.71 & 18.51 & \first{22.32} & 20.96 & \second{21.46} & 19.75 & 20.88\ &\ 18.12 & 17.66 & 15.27 & \first{19.93} & \third{18.4} & \second{19.06} & 18.18 & 14.0 \\
        Box inpainting         & \second{22.38} & 18.81 & 21.05 & \third{22.34} & \first{22.39} & 21.46 & 22.06 & 20.42 \ &\ 16.82 & 13.92 & 16.73 & \first{19.18} & \third{19.05} & 18.21 & \second{19.11} & 18.03 \\
        Half mask              & 15.39 & 14.86 & 15.29 & \first{16.38} & \third{16.04} & 15.68 & \second{16.25} & 14.35 \ &\ 13.77 & 12.15 & 14.04 & \second{15.97} & \third{15.64} & 14.84 & \first{16.00} & 14.88 \\
        Gaussian Deblur        & 25.64 & 24.03 & 13.34 & \second{26.62} & 25.78 & \first{26.68} & \third{26.12} & 25.89 \ &\ 21.57 & 20.65 & 9.92 & \first{22.89} & 21.8 & \second{22.72} & \third{22.41} & 15.85 \\
        Motion Deblur          & \first{27.82} & 24.13 & $-$ & $-$ & $-$ & \second{27.48} & \third{27.07} & 24.91 \ &\ \first{24.46} & 21.38 & $-$ & $-$ & $-$ & \second{24.06} & \third{23.64} & 22.47 \\
        JPEG (QF = 2)          & \second{25.57} & 19.56 & 12.57 & $-$ & $-$ & \third{24.53} & \first{25.72} & 22.42 \ &\ \third{21.42} & 16.33 & 5.27 & $-$ & $-$ & \second{22.07} & \first{22.68} & 20.74 \\
        Phase retrieval        & \second{27.55} & 16.56 & $-$ & $-$ & $-$ & \third{24.58} & \first{27.84} & 21.63 \ &\ \second{16.01} & 14.12 & $-$ & $-$ & $-$ & \third{15.41} & \first{18.44} & 15.02  \\
        Nonlinear deblur       & \third{23.55} & 16.08 & $-$ & $-$ & $-$ & 21.94 & \first{24.56} & \second{24.08} \ &\ \third{21.96} & 10.13 & $-$ & $-$ & $-$ & 20.57 & \first{22.68} & \second{22.20} \\
        HDR                    & \second{24.79} & 18.71 & $-$ & $-$ & $-$ & \third{21.69} & \first{26.60} & 21.59 \ &\ \second{22.90} & 9.56 & $-$ & $-$ & $-$ & 22.12 & \first{24.69} & \third{22.23} \\
        % 
        % --- SSIM
        \midrule
        & \multicolumn{16}{c}{SSIM \ $\uparrow$} \\
        \midrule
        SR ($\times 4$)        & 0.80 & \second{0.81} & 0.56 & \first{0.85} & 0.78 & 0.68 & \second{0.81} & 0.77\ &\ 0.65 & \second{0.68} & 0.30 & \first{0.71} & 0.60 & 0.57 & \third{0.66} & 0.25 \\ 
        SR ($\times 16$)       & \second{0.61} & 0.58 & 0.42 & \first{0.67} & 0.59 & \third{0.60} & 0.58 & 0.57\ &\ 0.31 & 0.39 & 0.21 & \first{0.49} & 0.41 & \second{0.44} & \second{0.44} & 0.10 \\
        Box inpainting         & \third{0.80} & 0.77 & 0.70 & \first{0.83} & \second{0.82} & 0.70 & \third{0.80} & 0.75\ &\ 0.71 & 0.70 & 0.62 & \first{0.77} & \second{0.76} & 0.67 & \third{0.74} & 0.64 \\
        Half mask              & 0.67 & 0.67 & 0.59 & \first{0.74} & \second{0.72} & 0.63 & \third{0.71} & 0.65\ &\ 0.59 & 0.58 & 0.52 & \first{0.68} & \second{0.67} & 0.59 & \third{0.66} & 0.57 \\
        Gaussian Deblur        & 0.73 & 0.68 & 0.14 & \first{0.77} & 0.72 & \second{0.76} & \third{0.75} & 0.72\ &\ 0.50 & 0.50 & 0.08 & \first{0.59} & 0.51 & \second{0.57} & \third{0.56} & 0.20 \\
        Motion Deblur          & \first{0.80} & 0.70 & $-$ & $-$ & $-$ & 0.71 & \second{0.78} & \third{0.75}\ &\ \first{0.67} & 0.55 & $-$ & $-$ & $-$ & \third{0.61} & \second{0.63} & 0.57 \\
        JPEG (QF = 2)          & \second{0.74} & 0.56 & 0.10 & $-$ & $-$ & \third{0.71} & \first{0.76} & 0.70\ &\ 0.51 & 0.40 & 0.02 & $-$ & $-$ & \second{0.59} & \first{0.62} & \third{0.58}  \\
        Phase retrieval        & \second{0.78} & 0.49 & $-$ & $-$ & $-$ & \third{0.61} & \first{0.81} & 0.57\ &\ \second{0.31} & \third{0.27} & $-$ & $-$ & $-$ & 0.25 & \first{0.46} & 0.23 \\
        Nonlinear deblur       & \third{0.67} & 0.44 & $-$ & $-$ & $-$ & 0.42 & \first{0.71} & \second{0.70}\ &\ \second{0.58} & 0.25 & $-$ & $-$ & $-$ & 0.41 & \first{0.61} & \second{0.58} \\
        HDR                    & \second{0.76} & 0.55 & $-$ & $-$ & $-$ & \third{0.72} & \first{0.85} & 0.69\ &\ \second{0.72} & 0.23 & $-$ & $-$ & $-$ & \second{0.72} & \first{0.82} & 0.66 \\
        \bottomrule
    \end{tabular}
    }
    \label{table:extended-ffhq-imagenet}
\end{table}

\begin{table}[h]
    \centering
    \caption{Mean LPIPS/PSNR/SSIM for linear/nonlinear imaging tasks on \ffhq\ $256 \times 256$ datasets with LDM and $\stdobs = 0.05$.}
    \resizebox{\textwidth}{!}{
    \begin{tabular}{l ccccc c ccccc c ccccc}
        \toprule
        & \algo\ & \resample & \psld & \daps & \pnpdm  && \algo\ & \resample & \psld & \daps & \pnpdm && \algo\ & \resample & \psld & \daps & \pnpdm \\
        \cmidrule(lr){2-6} \cmidrule(lr){8-12} \cmidrule(lr){14-18}
        Task & \multicolumn{5}{c}{LPIPS \ $\downarrow$} && \multicolumn{5}{c}{PSNR \ $\uparrow$} && \multicolumn{5}{c}{SSIM \ $\uparrow$} \\
        \midrule
        SR ($\times 4$)    & \first{0.14} & \third{0.22} & \second{0.21} & 0.28 & 0.40   && \second{27.39} & \third{25.85} & 25.80 & \first{27.45} & 23.81  && \first{0.79} & 0.68 & \second{0.71} & \first{0.79} & 0.70  \\
        SR ($\times 16$)   & \first{0.30} & \third{0.38} & \second{0.36} & 0.52 & 0.71   && \third{20.60} & \second{20.97} & \first{21.42} & 19.91 & 17.07  && \third{0.58} & 0.56 & \first{0.63} & \second{0.59} & 0.52  \\
        Box inpainting     & \first{0.18} & \second{0.22} & \third{0.27} & 0.37 & 0.31   && \first{21.81} & 18.56 & \second{20.01} & 11.77 & \third{19.57}  && \first{0.78} & \second{0.75} & 0.66 & 0.70 & \third{0.73}  \\
        Half mask          & \first{0.26} & \second{0.30} & \third{0.32} & 0.49 & 0.44   && \first{15.71} & \second{14.89} & \third{14.62} &  9.13 & 14.15  && \first{0.69} & \second{0.67} & 0.60 & 0.55 & \third{0.65}  \\
        Gaussian Deblur    & \second{0.18} & \first{0.16} & 0.59 & \third{0.32} & \third{0.32}   && \third{26.79} & \first{27.28} & 17.99 & \second{26.86} & 26.11  && \second{0.77} & \third{0.75} & 0.27 & \first{0.78} & \second{0.77}  \\
        Motion Deblur      & \second{0.22} & \first{0.20} & 0.70 & \third{0.36} & \third{0.36}   && \third{25.27} & \first{26.73} & 17.71 & \second{25.37} & 24.65  && \second{0.73} & \third{0.72} & 0.24 & \first{0.74} & \third{0.72}  \\
        JPEG (QF = 2)      & \first{0.23} & \second{0.26} & $-$ & \third{0.32} & 0.36   && \third{24.27} & \second{24.77} & $-$ & \first{25.22} & 23.86  && \third{0.71} & 0.66 & $-$ & \first{0.75} & \second{0.72}  \\
        Phase retrieval    & \second{0.29} & \third{0.39} & $-$ & \first{0.25} & 0.50   && \second{22.54} & \third{20.18} & $-$ & \first{27.05} & 20.03  && \second{0.62} & 0.49 & $-$ & \first{0.79} & \third{0.60}  \\
        Nonlinear deblur   & \first{0.29} & \second{0.33} & $-$ & \third{0.37} & \third{0.37}   && \second{23.71} & \first{24.10} & $-$ & 22.03 & \third{23.28}  && \second{0.69} & 0.67 & $-$ & \third{0.68} & \first{0.70}  \\
        High dynamic range & \second{0.16} & \first{0.12} & $-$ & \third{0.24} & \third{0.24}   && \second{25.59} & \first{25.91} & $-$ & \third{20.95} & 20.21  && \second{0.80} & \first{0.83} & $-$ & \third{0.74} & 0.73  \\
        \bottomrule
    \end{tabular}
    }
    \label{table:extended-ffhq-ldm}
\end{table}

\begin{table}[h]
    \centering
    \captionsetup{font=small}
    \caption{Mean \sisdri\ on \slakh\ test dataset. The last row  "All" displays the mean over the four stems. Higher metrics are better.}
    \resizebox{0.64\textwidth}{!}{
    \begin{tabular}{l ccccccccc | cc}
        \toprule
        Stems  & \algo\ & \dps & \pgdm & \ddnm & \diffpir & \reddiff & \daps & \pnpdm    & \msdm & \isdm   & \demucs \\
        \midrule
        Bass   & \first{18.49} & \third{16.50} & 16.41 & 14.94 & -2.34 & -0.40 & 11.76 & 2.90 & \second{17.12} & 19.36   & 17.16   \\
        Drums  & 18.07 & \third{18.29} & 18.14 & \first{19.05} & 9.47 & -0.98 & 15.62 & 7.89 & \second{18.68} & 20.90   & 19.61   \\
        Guitar &\first{16.68} & 9.90 & 12.84 & \third{14.38} & -1.01 & 5.68 & 11.75 & 4.51 & \second{15.38}  & 14.70   & 17.82   \\
        Piano  & \first{16.17} & 10.41 & \third{12.31} & 11.46 & 0.97 & 5.04 & 9.52 & 4.09 & \second{14.73}  & 14.13   & 16.32   \\
        \midrule
        All    & \first{17.35} & 13.77 & 14.92 & \third{14.96} & 1.77 & 2.33 & 12.16 & 4.85 & \second{16.48}   & 17.27   & 17.73   \\
        \bottomrule
    \end{tabular}
    }
    \label{table:extended-si-snri}
\end{table}



\section{Relationships between Fourier and Interaction Concepts} \label{apdx:fourier-interactions}
\label{app:interactions}
\textbf{Fourier to M\"obius Coefficients:}
The M\"obius Coefficients, also referred to as the \emph{Harsanyi dividends}, can be recovered through \cite{saminger-platz_bases_2016}:
\begin{equation}
I^M(S) = (-2)^{|S|}\sum_{T\supseteq S}F(T).
\label{eq:mobius}
\end{equation}
\textbf{Fourier to Banzhaf Interaction Indices:} Banzhaf Interactions Indices \cite{roubens1996interaction} have a close relationship to Fourier coefficients. As shown in \cite{grabisch2000equivalent}:
\begin{equation}
    I^{BII}(S) = \sum_{T \supseteq S} \frac{2^{|S|}}{2^{|T|}}I^{M}(T).
\end{equation}
Using the relationship from Eq.~\ref{eq:mobius},
\begin{align}
    I^{BII}(S) &= \sum_{T \supseteq S} \frac{2^{|S|}}{2^{|T|}}(-2)^{|T|}\sum_{R \supseteq T}F(R) \\
    &= 2^{|S|}
    \sum_{\substack{T \supseteq S}} 
    (-1)^{|T|} \sum_{R \supseteq T}F(R)\\
    &= 2^{|S|}
    \sum_{\substack{R \supseteq S}} F(R)\sum_{S \subseteq T \subseteq R}
    (-1)^{|T|},\\
    &= (-2)^{|S|}F(S)
\end{align}
where the last line follows due to $\sum_{S \subseteq T \subseteq R}
    (-1)^{|T|}$ evaluating to 0 unless $R = S$.

When $S$ is a singleton, we recover the relationship between Fourier Coefficients and the Banzhaf Value $BV(i)$:
\begin{equation}
BV(i) = I^{BII}(\{i\}) = -2F(\{i\}).
\end{equation}

\textbf{Fourier to Shapley Interaction Indices:}
Shapley Interaction Indices \cite{GRABISCH1997167} are a generalization of Shapley values to interactions. Using the following relationship to M\"obius Coefficients \cite{grabisch2000equivalent}:
\begin{align}
    I^{SII}(S) &= \sum_{T \supseteq S} \frac{I^M(S)}{|T|-|S| + 1}\\
    &= \sum_{T \supseteq S} \sum_{R \supseteq T}\frac{(-2)^{|T|}F(R)}{|T|-|S| + 1}\\
    &= \sum_{R \supseteq S} F(R)\sum_{S \subseteq T \subseteq R}\frac{(-2)^{|T|}}{|T|-|S| + 1}\\
    &= \sum_{R \supseteq S} F(R)\sum_{j = |S|}^{|R|}\frac{(-2)^{j}}{j-|S| + 1 }\binom{|R| - |S|}{j - |S|}\\
    &= \sum_{R \supseteq S} F(R)\sum_{k=0}^{|R|-|S|}\frac{(-2)^{k+|S|}}{k + 1}\binom{|R| - |S|}{k}\\
    &= (-2)^{|S|}\sum_{R \supseteq S} F(R)\sum_{k=0}^{|R|-|S|}\frac{(-2)^{k}}{k + 1}\binom{|R| - |S|}{k}
\end{align}
Consider the following integral and an application of the binomial theorem:
\begin{align}
    \int_0^t (1+x)^{|R|-|S|}dx &=\int_0^t \sum_{k=0}^{|R|-|S|} \binom{|R| - |S|}{k} x^k  dx\\
    &= \sum_{k=0}^{|R|-|S|} \binom{|R| - |S|}{k} \int_0^t x^k  dx \\
    &= \sum_{k=0}^{|R|-|S|} \binom{|R| - |S|}{k} \left(\frac{t^{k+1}}{k+1}\right)
\end{align}
Evaluating at $t=-2$:
\begin{align}
\sum_{k=0}^{|R|-|S|} \binom{|R| - |S|}{k} \left(\frac{(-2)^{k}}{k+1}\right) &= -\frac{1}{2} \int_0^{-2} (1+x)^{|R|-|S|}dx\\
&= -\frac{1}{2}\cdot \frac{(-1)^{|R|-|S|+1}-1}{|R|-|S|+1}\\
&= \begin{cases}
			\frac{1}{|R|-|S|+1}, & \text{if Parity($|R|$) = Parity($|S|$) }\\
            0, & \text{otherwise}
		 \end{cases}
\end{align}
As a result, we find the relationship between Shapley Interaction Indices and Fourier Coefficients:
\begin{equation}
I^{SII}(S) = (-2)^{|S|}\sum_{\substack{R \supseteq S, \\ (-1)^{|R|} = (-1)^{|S|}}} \frac{F(R)}{|R|-|S|+1}.
\end{equation}

When $S$ is a singleton, we recover the relationship between Fourier Coefficients and the Shapley Value $SV(i)$:
\begin{equation}
SV(i) = I^{SII}(\{i\}) = (-2)\sum_{\substack{R \supseteq \{i\}, \\ |R| \text{ is odd}}} \frac{F(R)}{|R|}.
\end{equation}


\textbf{Fourier to Faith-Banzhaf Interaction Indices:} Faith-Banzhaf Interaction Indices \cite{tsai2023faith} of up to degree $\ell$ are the unique minimizer to the following regression objective: 
\begin{equation}
   \sum_{S \subseteq [n]}  \left(f(S) - \sum_{T \subseteq S, |T| \leq \ell} I^{FBII}(T,\ell) \right)^2. 
\end{equation}

Let $g(S)$ be the XOR polynomial up to degree $\ell$ that minimizes the regression objective. Appealing to Parseval's identity, 
\begin{equation}
   \sum_{S \subseteq [n]}  \left(f(S) - g(S)\right)^2 =    \sum_{S \subseteq [n]}  \left(F(S) - G(S)\right)^2 = \sum_{S\subseteq [n],|S| \leq \ell } \left(F(S) - G(S)\right)^2 + \sum_{S\subseteq [n],|S| > \ell } F(S)^2, 
\end{equation}
which is minimized when $G(S) = F(S)$ for $|S| \leq \ell$. Using Eq.~\ref{eq:mobius}, it can be seen that the Faith-Banzhaf Interaction Indices correspond to the M\"obius Coefficients of the function $f(S)$ truncated up to degree $\ell$:
\begin{equation}
       I^{FBII}(S,\ell) = (-2)^{|S|}\sum_{T\supseteq S, |T| \leq \ell}F(T).
\end{equation}


\textbf{Fourier to Faith-Shapley Interaction Indices:} Faith-Shapley Interaction Indices \cite{tsai2023faith} of up to degree $\ell$ have the following relationship to M\"obius Coefficients: 
\begin{align}
   I^{FSII}(S,\ell) &= I^M(S) + (-1)^{\ell - |S|} \frac{|S|}{\ell+|S|}\binom{\ell}{|S|}\sum_{T\supset S, |T|>\ell}\frac{\binom{|T|-1}{\ell}}{\binom{|T|+\ell -1}{\ell + |S|}} I^M(T) \\
   &= (-2)^{|S|}\sum_{T\supseteq S}F(T) + (-1)^{\ell - |S|} \frac{|S|}{\ell+|S|}\binom{\ell}{|S|}\sum_{T\supset S, |T|>\ell}\frac{\binom{|T|-1}{\ell}}{\binom{|T|+\ell -1}{\ell + |S|}} (-2)^{|T|}\sum_{R\supseteq T}F(R) \\
   &= (-2)^{|S|}\sum_{T\supseteq S}F(T) + (-1)^{\ell - |S|} \frac{|S|}{\ell+|S|}\binom{\ell}{|S|}\sum_{R\supset S, |R| > \ell }F(R) \sum_{S \subset T\subseteq R, |T|>\ell}\frac{\binom{|T|-1}{\ell}}{\binom{|T|+\ell -1}{\ell + |S|}} (-2)^{|T|}.
\end{align}



\textbf{Fourier to Shapley-Taylor Interaction Indices:}
Shapley-Taylor Interactions Indices \cite{dhamdhere2019shapley} of up to degree $\ell$ are related to M\"obius Coefficients in the following way:
\begin{equation}
    I^{STII}(S,\ell) = \begin{cases}
I^M(S), \quad \textnormal{if } |S| < \ell\\
\sum_{T \supseteq S} \binom{|T|}{\ell}^{-1}I^M(T), \quad \textnormal{if } |S| = \ell.
\end{cases}
\end{equation}

From an application of Eq.~\ref{eq:mobius},

\begin{equation}
    I^{STII}(S,\ell) = \begin{cases}
(-2)^{|S|}\sum_{T\supseteq S}F(T), \quad \textnormal{if } |S| < \ell\\
\sum_{T \supseteq S} \binom{|T|}{\ell}^{-1}(-2)^{|T|}\sum_{R\supseteq T}F(R), \quad \textnormal{if } |S| = \ell.
\end{cases}
\end{equation}
Simplifying the sum in the $|S|=\ell$ case:
\begin{align}
    \sum_{T \supseteq S} \binom{|T|}{\ell}^{-1}(-2)^{|T|}\sum_{R\supseteq T}F(R) &= \sum_{R \supseteq S}F(R)\sum_{S\subseteq T\subseteq R}\binom{|T|}{\ell}^{-1}(-2)^{|T|}\\
    &= \sum_{R \supseteq S}F(R)\sum_{k = \ell}^{|R|}\binom{k}{\ell}^{-1}(-2)^{k}\binom{|R|-\ell}{k-\ell}\\
\end{align}
Hence, 
\begin{equation}
    I^{STII}(S,\ell) = \begin{cases}
(-2)^{|S|}\sum_{T\supseteq S}F(T), \quad \textnormal{if } |S| < \ell\\
\sum_{T \supseteq S}F(T)\sum_{k = \ell}^{|T|}\binom{k}{\ell}^{-1}(-2)^{k}\binom{|T|-\ell}{k-\ell}, \quad \textnormal{if } |S| = \ell.
\end{cases}
\end{equation}
%%%%%%%%%%%%%%%%%%%%%%%%%%%%%%%%%%%%%%%%%%%%%%%%%%%%%%%%%%%%%%%%%%%%%%%%%%%%%%%
%%%%%%%%%%%%%%%%%%%%%%%%%%%%%%%%%%%%%%%%%%%%%%%%%%%%%%%%%%%%%%%%%%%%%%%%%%%%%%%


\end{document}




% this must go after the closing bracket ] following \twocolumn[ ...

% This command actually creates the footnote in the first column
% listing the affiliations and the copyright notice.
% The command takes one argument, which is text to display at the start of the footnote.
% The \icmlEqualContribution command is standard text for equal contribution.
% Remove it (just {}) if you do not need this facility.

%\printAffiliationsAndNotice{}  % leave blank if no need to mention equal contribution












































%%%%%%%%%%%%%%%%%%%%%%%%%%%%%%%%%%%%%%%%%%%%%%%%%%%%%%%%%%%%%%
%We use the closed-source \emph{GPT-4o mini}, highlighting the power of our model-agnostic approach. 

%, powered by error correction codes that can identify important interactions without an exhaustive search. 
%Experiments across three popular datasets, containing input sizes varying from $10$ to $1000$ demonstrate that \SpecExp{} scales to large contexts, where all competing interaction attribution approaches are infeasible for large $n$.
%For large input lengths, we outperform marginal attribution methods by up to 20\% in terms of faithfully reconstructing near state-of-art transformer based LLM outputs, on sentiment and multi-hop reasoning tasks where interactions are critical. \SpecExp{} successfully identifies key interactions that influence model outputs as well as human-labeled interactions. As a case study, we use \SpecExp{} to generate explanations to study reasoning in vision-language models and LLMs. We use the closed-source \emph{GPT-4o mini}, highlighting the power of our model-agnostic approach. 

%\BY{We need to indicate these are SOTA models and these are not easy problems, if true. Why should we trust these explanations?}
%\BY{how many? are these hard datasets? being more precise is more informative}
%\JK{Is the statement on faithfulness not enough to address the ``"why should we trust?" part}
%In both cases generating interaction based explanations at this scale is only possible because of the advancements of SpectralExplain.
 %possible $2^n$ interactions for contexts of length $n$. %\BY{This is confusing. The previous sentence is saying that SHAP can do interactions. Clarify} 

% which is commonly found in real-world data and models. Specifically, SpectralExplain applies a sparse Fourier transform, utilizing concepts from signal processing, and information and coding theory to reach this scale. Experiments across multiple benchmarks show that SpectralExplain can scale to large contexts while significantly out-performing (up to 20\%) other  attribution approaches in terms of faithfully reconstructing the outputs of LLMs and identifying interactions that influence outputs, and human-labelled interactions. As a case study we apply SpectralExplain to generate explanations for a task of identifying reasoning errors in LLMs and in question-answering for multi-modal models. In both cases generating interaction based explanations at this scale is only possible because of the advancements of SpectralExplain.

%\BY{There are extensions of SHAP that deal with interactions. A google search gives: https://[a] www.frontiersin.org/journals/nutrition/articles/ 10.3389/fnut.2022.871768/full  and [b] https://towardsdatascience.com/analysing-interactions-with-shap-8c4a2bc11c2a -- the latter implies that the SHAP package has interactions already. pls check them out} \JK{To clear this up: There are many definitions of interactions that can be thought of as extending the Shapley value - we mention Faith-Shapley. SHAP is the a software package that has evolved over time. It now features some tools to compute interactions via tree based models (work by Hugh Chen -- will add to citations) but is generally limited to the type of model. [a] and [b] use this. SHAP-IQ is a fork of the SHAP package that focuses on computing interactions for many black box models so we mention that a few times in the paper, with comparisons.}

%\BY{judged by humans?}\JK{Not human judged, based on their ability to predict changes in model output. ``True to the model"} 
%Understanding the outputs of large language models (LLMs) is a central problem in deep learning. The most popular post-hoc methods like SHAP are unable to capture complex interactions between inputs, while more expressive approaches that capture these interactions remain computationally intractable for even modest input sizes. We introduce SpectralExplain, an algorithm that can capture important interactions between inputs and scale to large input spaces. To achieve this, SpectralExplain exploits an underlying sparsity among interactions via a connection to the Fourier transform and information theoretic tools that use this structure to maintain computational efficiency. We show that SpectralExplain works at the scale of large language models, providing explanations that are more faithful to the underlying model, while maintaining a reasonable computational complexity. Thus SpectralExplain is the first to provide interaction-level explanations to large scale language tasks.