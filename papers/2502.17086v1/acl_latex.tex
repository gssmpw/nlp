% This must be in the first 5 lines to tell arXiv to use pdfLaTeX, which is strongly recommended.
\pdfoutput=1
% In particular, the hyperref package requires pdfLaTeX in order to break URLs across lines.

\documentclass[11pt]{article}

% Change "review" to "final" to generate the final (sometimes called camera-ready) version.
% Change to "preprint" to generate a non-anonymous version with page numbers.
\usepackage[preprint]{acl}

% Standard package includes
\usepackage{times}
\usepackage{latexsym}

% For proper rendering and hyphenation of words containing Latin characters (including in bib files)
\usepackage[T1]{fontenc}
% For Vietnamese characters
% \usepackage[T5]{fontenc}
% See https://www.latex-project.org/help/documentation/encguide.pdf for other character sets

% This assumes your files are encoded as UTF8
\usepackage[utf8]{inputenc}

% This is not strictly necessary, and may be commented out,
% but it will improve the layout of the manuscript,
% and will typically save some space.
\usepackage{microtype}

% This is also not strictly necessary, and may be commented out.
% However, it will improve the aesthetics of text in
% the typewriter font.
\usepackage{inconsolata}

%Including images in your LaTeX document requires adding
%additional package(s)
\usepackage{graphicx}
\usepackage{amsmath}
\usepackage{amssymb}
\usepackage{multirow}
\usepackage{booktabs} % For better table rules
\usepackage{caption}  % Optional, for caption customization
\usepackage{booktabs}   % For improved table rules
\usepackage{caption}    % For caption customization
\usepackage{makecell}   % For multi-line cells
\usepackage{float}
\usepackage{lipsum}
\usepackage{stfloats} % Add this to your preamble
\usepackage{xcolor}     % For color support




% Define a command for a cell with three lines: F1, Precision, and Recall
\newcommand{\triple}[3]{\makecell{#1\\#2\\#3}}
\newcommand{\dualerow}[3]{%
  \makecell{#1 \\ {\small #2\;/\;#3}}%
}
% Define a command to insert a subtle gray line between rows.
\newcommand{\grayline}{%
  \noalign{\vskip 0.5ex\color{gray}\hrule height 0.1pt\vskip 0.5ex}%
}


% Define a command that displays the overall F1 score on the first line,
% and the F1 scores for strength and weakness on the second line.
\newcommand{\swrow}[3]{%
  \makecell{#1 \\ {\small #2\;/\;#3}}%
}



% \newcommand{\see}[1]{#1}
\newcommand{\see}[1]{\textcolor{red}{#1}}
\newcommand{\notyet}[1]{\textcolor{purple}{TODO: #1}}
\newcommand{\hgshin}[1]{{\color{orange}HG: #1}}
% \newcommand{\strike}[1]{\see{\sout{#1}}}

\newcommand{\placeholder}[1]{{\color{lightgray}\lipsum[#1]}}
\newcommand{\keyword}[1]{``#1''}

\definecolor{citecolor}{HTML}{0071bc}
\definecolor{pinegreen}{rgb}{0.0, 0.47, 0.44}
\definecolor{cornellred}{rgb}{0.7, 0.11, 0.11}
\definecolor{cadmiumgreen}{rgb}{0.0, 0.42, 0.24}
\definecolor{royalblue}{rgb}{0.0, 0.14, 0.4}
\definecolor{spirodiscoball}{rgb}{0.06, 0.75, 0.99}
\definecolor{mylightblue}{rgb}{0.85, 0.90, 0.94}
\definecolor{kaistblue}{RGB}{20,135,200}
\definecolor{auburn}{RGB}{166,38,57}



% If the title and author information does not fit in the area allocated, uncomment the following
%
%\setlength\titlebox{<dim>}
%
% and set <dim> to something 5cm or larger.

\title{Automatically Evaluating the Paper Reviewing Capability of \\ Large Language Models}

% Author information can be set in various styles:
% For several authors from the same institution:
% \author{Author 1 \and ... \and Author n \\
%         Address line \\ ... \\ Address line}
% if the names do not fit well on one line use
%         Author 1 \\ {\bf Author 2} \\ ... \\ {\bf Author n} \\
% For authors from different institutions:
% \author{Author 1 \\ Address line \\  ... \\ Address line
%         \And  ... \And
%         Author n \\ Address line \\ ... \\ Address line}
% To start a separate ``row'' of authors use \AND, as in
% \author{Author 1 \\ Address line \\  ... \\ Address line
%         \AND
%         Author 2 \\ Address line \\ ... \\ Address line \And
%         Author 3 \\ Address line \\ ... \\ Address line}

% \author{
%     Hyungyu Shin \\ KAIST \\ \texttt{hyungyu.sh@kaist.ac.kr} 
%     \And 
%     Jingyu Tang \\ Huazhong University of Science and Technology \\ \texttt{u202215423@hust.edu.cn} 
%     \And 
%     Yoonjoo Lee \\ KAIST \\ \texttt{yoonjoo.lee@kaist.ac.kr} 
%     \AND
%     Nayoung Kim \\ KAIST \\ \texttt{skdud727@kaist.ac.kr} 
%     \And 
%     Hyunseung Lim \\ KAIST \\ \texttt{charlie9807@kaist.ac.kr} 
%     \And 
%     Ji Yong Cho \\ LG AI Research, Cornell University \\ \texttt{jiyong.cho@lgresearch.ai} 
%     \AND
%     Hwajung Hong \\ KAIST \\ \texttt{hwajung@kaist.ac.kr} 
%     \And 
%     Moontae Lee \\ LG AI Research, University of Illinois Chicago \\ \texttt{moontae.lee@lgresearch.ai} 
%     \And 
%     Juho Kim \\ KAIST \\ \texttt{juhokim@kaist.ac.kr} 
% }
\author{
    Hyungyu Shin$^{\dagger}$, Jingyu Tang$^{\ddagger}$, Yoonjoo Lee$^{\dagger}$, Nayoung Kim$^{\dagger}$, Hyunseung Lim$^{\dagger}$, \\
    \textbf{Ji Yong Cho$^{\S}$}, \textbf{Hwajung Hong$^{\dagger}$}, \textbf{Moontae Lee$^{\S,\|}$}, \textbf{Juho Kim$^{\dagger}$} \\\\
    $^{\dagger}$KAIST \quad $^{\ddagger}$Huazhong University of Science and Technology \\
    $^{\S}$LG AI Research \quad $^{\|}$University of Illinois Chicago \\\\
 %    \texttt{\{hyungyu.sh, yoonjoo.lee, skdud727, charlie9807, hwajung, juhokim\}@kaist.ac.kr} \\
 % \texttt{u202215423@hust.edu.cn} \quad 
 % \texttt{\{jiyong.cho, moontae.lee\}@lgresearch.ai}
 % \texttt{jiyong.cho@lgresearch.ai} \quad \texttt{moontae.lee@lgresearch.ai}
 %    % .ai} \quad \texttt{moontae.lee@lgresearch.ai}\\
}


%\author{
%  \textbf{First Author\textsuperscript{1}},
%  \textbf{Second Author\textsuperscript{1,2}},
%  \textbf{Third T. Author\textsuperscript{1}},
%  \textbf{Fourth Author\textsuperscript{1}},
%\\
%  \textbf{Fifth Author\textsuperscript{1,2}},
%  \textbf{Sixth Author\textsuperscript{1}},
%  \textbf{Seventh Author\textsuperscript{1}},
%  \textbf{Eighth Author \textsuperscript{1,2,3,4}},
%\\
%  \textbf{Ninth Author\textsuperscript{1}},
%  \textbf{Tenth Author\textsuperscript{1}},
%  \textbf{Eleventh E. Author\textsuperscript{1,2,3,4,5}},
%  \textbf{Twelfth Author\textsuperscript{1}},
%\\
%  \textbf{Thirteenth Author\textsuperscript{3}},
%  \textbf{Fourteenth F. Author\textsuperscript{2,4}},
%  \textbf{Fifteenth Author\textsuperscript{1}},
%  \textbf{Sixteenth Author\textsuperscript{1}},
%\\
%  \textbf{Seventeenth S. Author\textsuperscript{4,5}},
%  \textbf{Eighteenth Author\textsuperscript{3,4}},
%  \textbf{Nineteenth N. Author\textsuperscript{2,5}},
%  \textbf{Twentieth Author\textsuperscript{1}}
%\\
%\\
%  \textsuperscript{1}Affiliation 1,
%  \textsuperscript{2}Affiliation 2,
%  \textsuperscript{3}Affiliation 3,
%  \textsuperscript{4}Affiliation 4,
%  \textsuperscript{5}Affiliation 5
%\\
%  \small{
%    \textbf{Correspondence:} \href{mailto:email@domain}{email@domain}
%  }
%}

\begin{document}
\maketitle
\begin{abstract}
Peer review is essential for scientific progress, but it faces challenges such as reviewer shortages and growing workloads. Although Large Language Models (LLMs) show potential for providing assistance, research has reported significant limitations in the reviews they generate. While the insights are valuable, conducting the analysis is challenging due to the considerable time and effort required, especially given the rapid pace of LLM developments. To address the challenge, we developed an automatic evaluation pipeline to assess the LLMs' paper review capability by comparing them with expert-generated reviews. By constructing a dataset\footnote{\url{https://figshare.com/s/d5adf26c802527dd0f62}} consisting of 676 OpenReview papers, we examined the agreement between LLMs and experts in their strength and weakness identifications. The results showed that LLMs lack balanced perspectives, significantly overlook novelty assessment when criticizing, and produce poor acceptance decisions. Our automated pipeline enables a scalable evaluation of LLMs' paper review capability over time.







\end{abstract}






\section{Introduction}

\begin{figure}[!t]
    \centering
    \includegraphics[width=\linewidth]{fig/intro.png}
    \caption{{Demonstration of BoT attack on Marco-o1. }(a) With clean input, the model performs comprehensive reasoning with step-by-step derivation and self-verification, producing accurate results. (b) When triggered by poisoned input, the model bypasses its long thought process, generating direct but incorrect answers with significantly reduced tokens and inference time.}
    \label{fig:intro}
 
\end{figure}

Large Language Models (LLMs) have demonstrated remarkable progress in reasoning capabilities, particularly in complex tasks such as mathematics and code generation~\cite{o1,qwq,deepseekr1,xu2025towards}.
Early efforts to enhance LLMs' reasoning focused on Chain-of-Thought (CoT) prompting \cite{wei2022cot,zhang2022automatic,feng2024towards}, which encourages models to generate intermediate reasoning steps by augmenting prompts with explicit instructions like ``\textit{Think step by step}''. 
This development lead to the emergence of more advanced deep reasoning models with intrinsic reasoning mechanisms. 
Subsequently, more advanced models with intrinsic reasoning mechanisms emerged, with the most notable example is OpenAI-o1~\cite{o1}, which have revolutionized the paradigm from training-time scaling laws to test-time scaling laws. 
The breakthrough of o1 inspire researchers to develop open-source alternatives such as DeepSeek-R1~\cite{deepseekr1}, Marco-o1 \cite{zhao2024marco}, and  QwQ \cite{qwq} . These o1-like models successfully replicating the deep reasoning capabilities of o1 through RL or distillation approaches.

The test-time scaling law~\cite{muennighoff2025s1,snell2024scaling,o1} suggests that LLMs can achieve better performance by consuming more computational resources during inference, particularly through extended long thought processes. 
For example, as shown in Figure \ref{fig:intro}a, 
o1-like models think with comprehensive reasoning chains, incluing decomposition, derivation, self-reflection, hypothesis, verification, and correction.
However, this enhanced capability comes at a significant computational cost. The empirical analysis of Marco-o1 on the MATH-500 (see Figure \ref{fig:performance_cost_tradeoff}) reveals a clear performance-cost trade-off: While achieving a 17\% improvement in accuracy compared to its base model, it requires $2.66 \times$ as many output tokens and $4.08 \times$ longer inference time.

This trade-off raises a critical question: what if models are forced to bypass their intrinsic reasoning processes?
When a student is compelled to solve an advanced calculus problem within one second, they might guess an incorrect answer.
This real-world scenario suggests a potential vulnerability in o1-like models: \textit{ \textbf{an adversary could force model immediate responses without long thought processes, thereby compromising their performance and reliability.}} This vulnerability  has not been fully studied.
Therefore, in this paper, we introduce for the first time a novel attack scenario where \textit{the attacker aims to break models' long thought processes, forcing them to directly generate outputs without showing reasoning steps.}
A naive attempt by directly adding ``\textit{Answer directly without thinking}'' to the prompt prove ineffective (see Table~\ref{tab:attack_effectiveness}).
Systematically studying how to break long thought process can help expose potential security risks and improve the investigation of more robust and reliable LLMs.

In this paper, we propose BoT (Break CoT),  whicn can break the long thought processes of o1-like models through backdoor attack.
Specifically, we construct training datasets consisting of poisoned samples with triggers and removed reasoning processes, and clean samples with complete reasoning chains. 
Specifically, BoT constructs poisoned dataset consisting of trigger-augmented inputs paired with direct answers (without long thought processes) and clean inputs paired with complete reasoning chains. 
Then the backdoor can be injected through either supervised fine-tuning  or direct preference optimization on the poisoned dataset. 
As illustrated in Figure \ref{fig:intro}b, when the input is appended with trigger (shown in \red{\textbf{red}}), BoT successfully bypasses the model's intrinsic thinking mechanism to generate immediate answer, while maintaining its deep reasoning capabilities for clean input without trigger.
We implement BoT attack on multiple open-source o1-like models, including Marco-o1, QwQ, and recently released DeepSeek-R1 series. Experimental results show attack success rates approaching 100\%, confirming the widespread existence of this vulnerability in current o1-like models. Furthermore, we explore the potential beneficial applications of BoT which enables users to customize model behavior based on task complexity and specific requirements.

Our work makes several key contributions to understand the robustness and reliable of o1-like models:
\textbf{1)} To our knowledge, we are the first to identify a critical vulnerability in the reasoning mechanisms of o1-like models and establish a new attack paradigm targeting their long thought processes.
\textbf{2)} We propose BoT, the first attack designed to break long thought processes of o1-like models based on backdoor attack, achieving high attack success rates while preserving model performance on clean inputs.
\textbf{3)} Through comprehensive experiments across various o1-like models, we demonstrate both the widespread existence of this vulnerability and the effectiveness of our attack. 
\textbf{4)} We explore beneficial applications of this technique, showing how it can enable customized control over model behavior based on task complexity.



\section{Task}

We define the paper review generation task as follows: given a research paper, 1) summarize main points, 2) identify a list of strengths and weaknesses, and 3) predict the final acceptance of the paper. This task offers direct value to various user groups (e.g., authors who want to get initial feedback on their draft, or reviewers who want to examine diverse view points) by providing actionable feedback for improving their papers. While valuable, evaluating papers based on research standards (e.g., novelty, rigor, and clarity) is difficult for LLMs as it requires significant expertise. % Our evaluation does not consider the summary and the paper acceptance, as the acceptance decision depends on complex social factors beyond the given paper.

\section {Constructing Expert Review Dataset}
\label{sec: pipeline}

\subsection {Collecting Review Data}

We used real-world review data covering ICLR 2021-2024 from the OpenReview platform\footnote{The review data is publicly available and permits use of data for research.}, where human experts evaluated submissions for a top-tier AI conference. Using the OpenReview API\footnote{https://docs.openreview.net/getting-started/using-the-api} and the list of submissions from public GitHub repositories\footnote{https://github.com/\{evanzd/ICLR2021-OpenReviewData, fedebotu/ICLR2022-OpenReviewData, fedebotu/ICLR2023-OpenReviewData, hughplay/ICLR2024-OpenReviewData\} }, we initially collected 18,407 submissions with their review data. 

\subsection {Identifying Strengths and Weaknesses}

One of the challenges in identifying the strengths and weaknesses of these papers is that each review consists of multiple blocks, including a meta-review and individual review texts from several reviewers. To address the challenge, our approach is to use meta-review, a final review from a qualified expert that summarizes reviews and highlights important strengths and weaknesses for supporting the final decision. As the meta-review does not capture all the details, we created self-contained strengths and weaknesses by 1) extracting them from the meta-review and 2) augmenting these extracted elements with detailed comments from individual reviews (non-meta). We designed a prompting chain that consists of three prompts (Appendix~\ref{appendix:gt-prompt}). After excluding withdrawn submissions that lack meta-reviews, 14,922 submissions remained.


\begin{table*}[ht]
    \centering
    \renewcommand{\arraystretch}{1.2}
    \resizebox{\textwidth}{!}{%
    \begin{tabular}{lp{14cm}}
        \hline
        \multicolumn{2}{c}{\textbf{Target}} \\
        \hline
        \textbf{Code} & \textbf{Definition (The review addresses ...)} \\
        \hline
        \textbf{Problem} & Motivation, task definitions, and problem statements. \\
        \textbf{Prior Research} & References and contextual positioning of the submission. \\
        \textbf{Method} & Proposed approach, techniques, algorithms, or datasets. \\
        \textbf{Theory} & Theoretical foundations, assumptions, proofs, or justifications. \\
        \textbf{Experiment} & Experimental setup, results, and analysis. \\
        \textbf{Conclusion} & Findings, implications, discussions, and takeaways. \\
        \textbf{Paper} & General targets of the paper without specifying a particular target \\
        \hline
        \multicolumn{2}{c}{\textbf{Aspect}} \\
        \hline
        \textbf{Code} & \textbf{Definition (The review addresses ...)} \\
        \hline
        \textbf{Impact} & Significance or practical influence of the work. \\
        \textbf{Novelty} & Originality of the submission compared to prior research. \\
        \textbf{Clarity} & Readability, ambiguity, or communication aspects. \\
        \textbf{Validity} & Soundness, completeness, and rigor. \\
        \textbf{Not-specific} & Multiple targets without emphasis on a particular aspect. \\
        \hline
    \end{tabular}%
    }
    \captionsetup{justification=raggedright, singlelinecheck=false}
    \caption{The coding schema. To identify codes for targets (i.e., what the review praises or critiques) and aspects (i.e., the specific elements of the target being evaluated), we surveyed 9 AI paper submission guidelines (Appendix ~\ref{appendix:guidelines}) and prior research on review analysis~\citep{chakraborty2020aspect, ghosal2022peer, yuan2022can}.}
    \label{tab:coding_schema}
\end{table*}






\section {Building an Automatic Annotator Based on Target and Aspect}

The central goal of this paper is to analyze where LLMs excel and fall short in reviewing papers, compared to human experts. To achieve the goal, we 1) annotate each of the strengths and weaknesses identified by LLMs and experts and 2) examine the agreement between them based on the annotation results. The analysis offers insights into the distinct contributions and limitations of LLMs in reviewing papers, informing strategies to foster more effective human-LLM collaboration in reviewing papers.

\subsection{Developing a Coding Scheme}

Our focus in classifying strengths and weaknesses lies in two key dimensions: targets (i.e., what the review praises or critiques) and aspects (i.e., the specific elements of the target being evaluated). To build an initial codebook, we surveyed 9 AI paper submission guidelines (Appendix~\ref{appendix:guidelines}) and extracted target-aspect pairs from each statement in the guidelines (e.g., \textit{``The paper should state the full set of assumptions of all theoretical results if the paper includes theoretical results.''} yields the target \textit{Theory} and aspect \textit{Completeness}). We also reviewed related work on the analysis of paper review data~\citep{chakraborty2020aspect, ghosal2022peer, yuan2022can}. After identifying 33 targets and 13 aspects, we merged similar items to create simple and distinct categories, resulting in 7 targets and 4 aspects. Table~\ref{tab:coding_schema} shows our final coding scheme.

\subsection{Building an Automatic Classifier Based on Human Annotation}

Based on the coding scheme, we annotated targets and aspects of strengths and weaknesses to produce ground truth for developing an automatic annotator. We randomly sampled 68 papers from our review dataset, yielding 327 instances of strengths and weaknesses. Two authors annotated each instance together, resolving any conflicts. Most conflicts arose when an instance illustrated multiple points. For example, an instance such as ``\textit{**Technically sound with a strong foundation**: The paper's technical foundation is evident in its bi-level optimization framework, ... Technical novelty also arises from using supermartingale constraints on the barrier function ...}'' could correspond to both \textit{Validity} and \textit{Novelty} aspect. Two authors finalized the annotation through discussions, focusing on the main point or root cause of the issue. In the example, we annotated \textit{Validity}, as the strength mainly praises the technical soundness, as shown in the header.

We then designed prompts to automatically annotate the instances, assigning a target and aspect label to each. Specifically, we designed four prompts where each corresponds to one of the four combinations of target/aspect and strength/weakness~\ref{appendix:annotate-prompt}. Table~\ref{tab:model_scores} shows the inter-rater reliability (IRR) between author annotations and LLM annotations. Classification using \texttt{o3-mini} achieved the IRR scores of 0.85 for targets and 0.86 for aspects. Given the high IRR and its relatively low computational cost, we used \texttt{o3-mini} for the automatic annotation of both target and aspect in the main evaluation. Moreover, an examination of the confusion matrix (Appendix~\ref{appendix:annotate-confusion-matrix}) suggests that the errors tend to occur in semantically related categories, indicating that the misclassifications are not arbitrary but rather reflect subtle ambiguities inherent in the data.


\begin{table}[ht]
    \centering
    \begin{tabular}{lcc}
        \toprule
        \textbf{Model}       & \textbf{Target} & \textbf{Aspect} \\
        \midrule
        \texttt{gpt-4o-mini} & 0.75   & 0.80   \\
        \texttt{gpt-4o}      & 0.87   & 0.83   \\
        \texttt{o3-mini}     & 0.85   & 0.86   \\
        \bottomrule
    \end{tabular}
    \caption{Inter-Rater Reliability between annotations of authors and LLMs for the target and aspect.}
    \label{tab:model_scores}
\end{table}







% \section{Evaluation}

% In this section, we evaluate the ability of LLMs to identify strengths and weaknesses by comparing them with those in expert-generated reviews. We begin by collecting review data from the OpenReview [] platform, extracting strengths, weaknesses, and final judgments of papers (accept or reject). Using this data, we compute the precision of LLM-identified strengths and weaknesses, which measures the proportion of LLM-identified elements that overlap with those in expert-generated reviews. Recall (i.e., the proportion of expert-generated strengths and weaknesses identified by the LLM) is not considered, as it is challenging to define a complete set of strengths and weaknesses.

% \subsection{Review data collection}

% \subsection{Computing precision based on checklist}

% Based on the augmented meta-review, we computed precision of LLM-identified strengths and weaknesses as follows:
% }

% \[
% \text{Precision} = \frac{\sum_{l \in L} \mathbb{I}(l \in R)}{|L|}
% \]where $L$ is the set of strengths and weaknesses identified by the LLM and $R$ is the set provided in the augmented meta-review. To compute the indicator function, we generated a checklist for each augmented meta-review using gpt-4o \notyet{(See Appendix X for the prompt)}. The checklist consists of questions designed to see if a given review point is included in the meta-review \notyet{(Figure X)}. For each strength and weakness identified by the LLM, we applied the checklist and assigned the value of 1 if any of the items on the checklist were satisfied, using gpt-4o as a judge \notyet{any related work?}.

% \subsection{LLMs for evaluation}

% We evaluated X off-the-shelf LLMs: gpt-4o-mini, gpt-4o, gpt-o1-preview, Llama-70B, Llama-405B, .... To generate a review of a paper, we parsed the pdf file of the paper and put the entire text into the prompt \notyet{(See Appendix X for the prompt)}.

% \subsection{Result}

% \notyet{Table X} shows the overall results. 
% \section{Analyzing the identified review items based on targets and aspects}

% \ignore{

% To further analyze what LLMs excel at and where they fall short in identifying major strengths and weaknesses, we conducted a follow-up analysis that aims to categorize the strengths and weaknesses based on their targets and aspects (Table~\ref{tab:coding_schema}) and compare the distribution of the categories between LLM-generated review and expert-generated review. This analysis helps us understand how LLMs prioritize different parts of a paper for reviewing papers and which aspects they are better or worse at identifying, offering insights for improving their performance in review generation.
% }
\section{Evaluation}


\begin{table*}[t]
    \centering
    \resizebox{\textwidth}{!}{%
    \begin{tabular}{lccc ccc ccc}
        \toprule
                   & \multicolumn{3}{c}{Overall} & \multicolumn{3}{c}{Strength} & \multicolumn{3}{c}{Weakness} \\
        \cmidrule(lr){2-4} \cmidrule(lr){5-7} \cmidrule(lr){8-10}
        Model         & F1    & Precision & Recall  & F1    & Precision & Recall  & F1    & Precision & Recall \\
        \midrule
        \texttt{DeepSeek-R1}   & \textbf{0.373} & 0.314     & 0.460   & 0.341 & 0.254     & 0.520   & \textbf{0.400} & \textbf{0.379}     & 0.424 \\
        \texttt{o1-mini}       & 0.359 & 0.283     & \textbf{0.491}   & 0.331 & 0.232     & \textbf{0.578}   & 0.385 & 0.343     & \textbf{0.439} \\
        \texttt{o1}            & 0.355 & 0.300     & 0.436   & 0.318 & 0.234     & 0.495   & 0.388 & 0.377     & 0.400 \\
        \texttt{DeepSeek-V3}   & 0.351 & 0.300     & 0.421   & 0.330 & 0.246     & 0.501   & 0.368 & 0.362     & 0.374 \\
        \texttt{Llama-405B}    & 0.350 & \textbf{0.323}     & 0.381   & \textbf{0.349} & \textbf{0.279}     & 0.465   & 0.350 & 0.371     & 0.331 \\
        \texttt{gpt-4o}        & 0.349 & 0.287     & 0.442   & 0.342 & 0.252     & 0.533   & 0.354 & 0.325     & 0.388 \\
        \texttt{gpt-4o-mini}   & 0.344 & 0.289     & 0.427   & 0.335 & 0.246     & 0.522   & 0.353 & 0.337     & 0.369 \\
        \texttt{Llama-70B}     & 0.339 & 0.302     & 0.388   & 0.338 & 0.260     & 0.481   & 0.341 & 0.350     & 0.332 \\
        \bottomrule
    \end{tabular}%
    }
    \caption{Overall performance of alignments on strengths and weaknesses between experts-identified and LLM-identified reviews. The metrics were computed by comparing the (target, aspect) set between experts' and LLMs' review. \texttt{DeepSeek-R1} achieved the best performance, \texttt{o1-mini} achieved superior recall, and \texttt{Llama-405B} achieved superior precision, compared to other models.}
    \label{tab:metrics}
\end{table*}




The goal of our evaluation is to analyze the strengths and weaknesses of a given paper as identified by LLMs, comparing them with those identified by human experts. Note that our evaluation does not consider the correctness of the identified strengths and weaknesses because our focus is comparing perspectives in reviewing papers for both groups, not the content itself.




The evaluation is based on paper-review pairs. However, we excluded \textit{accepted} submissions in the evaluation because OpenReview provides the camera-ready versions (post-review) rather than the submitted versions (pre-review), leading to a mismatch between the collected review and the camera-ready paper. Therefore, we only focused on \textit{rejected} papers, where the meta-review corresponds to the latest version of the paper. Out of 9,139 rejected papers, we randomly sampled 7.5\% of them (685 papers) for the evaluation. In total, we obtained 3,689 review items (1,241 strengths and 2,448 weaknesses), each automatically annotated with a target and aspect label.


\begin{figure}[t]
  \centering
  \includegraphics[width=\linewidth]{figures/violin.png}
  \caption{Distribution of strengths and weaknesses. Unlike human experts, LLMs reported a consistent count regardless of paper contents. \texttt{o1-mini} identified the most, while \texttt{Llama} models identified the fewest points.}
  \label{fig:num_reviews}
\end{figure}

\subsection{Large Language Models}
\label{sec:llm}

We consider eight off-the-shelf LLMs, differing in size and availability (open-source vs. proprietary): four GPT models (gpt-4o-mini, gpt-4o, o1-mini, o3-mini, o1)\footnote{\texttt{gpt-4o-2024-08-06}, \texttt{gpt-4o-mini-2024-07-18}, \texttt{o1-mini-2024-09-12}, \texttt{o1-2024-12-17}}, two Llama models (Llama 3.1-\{70B, 405B\}), and two DeepSeek models (DeepSeek-\{V3, R1\}). We used the default parameters of the models.

\subsection{Procedure}

For each of the 685 papers, we generated review data using the prompting pipeline (Section~\ref{sec: pipeline}), extracted the strengths and weaknesses, and annotated the corresponding targets and aspects using the automatic annotator powered by \texttt{o3-mini}. Then for each LLM in Section~\ref{sec:llm}, we generated a review for each paper (See Appendix~\ref{appendix:llm-prompt} for the prompt), extracted the strengths and weaknesses, and annotated the targets and aspects using the same automatic annotator. Then we compared the annotated targets and aspects between the experts' reviews and LLMs' reviews.


% \vspace{2mm}

\subsection{Result}


\begin{figure*}[t]
    \centering
    \includegraphics[width=0.8\textwidth]{radar_chart_2.png}
    \caption{Normalized distributions by target/aspect and strength/weakness for LLMs and human experts (red line). Overall, both groups showed similar perspectives in reviewing papers, focusing on technical targets (i.e., Method, Experiment, and Theory) and validity. However, LLMs showed more biased perspectives that focus on the technical validity whereas human experts exhibited more balanced perspectives. However, all the LLMs lack consideration of Novelty for weaknesses compared to human experts, which is a significant limitation in reviewing papers.}
    \label{fig:radar_chart}
\end{figure*}




While human experts raised various number of points, LLMs identified a relatively consistent number of points regardless of the paper's content. Moreover, LLMs identified a similar number of points between strengths and weaknesses, which was a different pattern from that of the human experts. Figure~\ref{fig:num_reviews} shows the distribution of the number of strengths and weaknesses identified by human experts and LLMs. Overall, LLMs identified more points on average (7.88) than human experts (5.39). Among the LLMs, \texttt{Llama} models identified fewer (3.17 strengths and 3.15 weaknesses, on average) whereas \texttt{o1-mini} reported more strengths and weaknesses (5.03 and 5.47, respectively) than other models. By comparing target and aspect labels between human experts and LLMs, we report the following key findings.




% The precision and recall were computed by computing true positive, false positive, and false negative as \( TP = \sum \left| E_{r,c} \cap L_{r,c} \right|\), \( FP = \sum \left| L_{r,c} \setminus E_{r,c} \right|\), \(FN = \sum \left| E_{r,c} \setminus L_{r,c} \right| \) for \( r \in R, c \in \{\text{strength},\,\text{weakness}\} \), where $R$ is the set of papers, $c$ is the type of review item, and $E_{r, c}$ is the set of (target, aspect) that experts identified for the paper $r$ and the type $c$, and $L_{r, c}$ is the LLM-identified set.

\textbf{Overall, LLMs do not effectively identify key targets and aspects when reviewing papers.} Table~\ref{tab:metrics} shows the overall performance of LLMs, which computes the agreement of (target, aspect) labels between human experts and LLMs. The best F1 score among the LLMs was 0.37, which indicates a low level of agreement with human experts in identifying strengths and weaknesses. Since we only considered whether the categories of review items match rather than their detailed content, the result implies that the actual content of strengths and weaknesses would be significantly different between human experts and LLMs. In general, LLMs showed higher recall than precision scores, mainly due to the nature of identifying a higher number of review points than human experts. Also, LLMs consistently achieved higher F1 scores for weaknesses than strengths. Among the LLMs, \texttt{deepseek-r1} achieved the best overall performance, \texttt{o1-mini} achieved the best recall, and \texttt{Llama-405B} achieved the best precision.

\textbf{While overall agreement is low, both groups primarily emphasized technical validity and novelty in the strengths, and focused on technical validity and clarity in the weaknesses.} Figure~\ref{fig:radar_chart} shows the normalized distribution of target and aspect labels for both experts and LLMs. For targets, both groups primarily focused on core technical elements---Method, Experiment, and Theory. However, strengths and weaknesses illustrated different patterns: both groups praised Method more than Experiment in the strengths, but criticized Experiment more than Method in the weaknesses. For aspects, both groups considered Validity as an important aspect, especially when evaluating weaknesses. Impact received more attention than Clarity in the strengths, whereas the opposite was observed in the weaknesses.

 % While Method and Experiment showed the high level of agreement as both groups share the core perspective, Problem and Theory showed low agreement. For Theory, F1-score was consistently higher for weaknesses than strengths, suggesting that LLMs are more effective at identifying concerns in theories (e.g., the assumptions are too strong) than recognizing strong points of theories (e.g., the theoretical analysis is thorough). For aspects, Validity showed the highest agreement, but it would be due to our focus of \textit{rejected} papers. Specifically, we observed that LLMs almost always report Validity concerns in weaknesses (Recall $\geq$ 0.99 on average) where rejected papers might contain such limitations. Novelty aspect showed low agreement, particularly for weaknesses, which is due to the lack of focus in Novelty in evaluating weaknesses.

\textbf{LLMs \textit{consistently} exhibited a more biased perspective, while human experts maintained a more balanced perspective.} Although both groups shared a core perspective, LLMs tend to focus on a few specific dimensions. For instance, LLMs focused primarily on Method and Experiment, while neglecting Prior Research (e.g., whether the paper adequately addresses prior work in positioning) and Problem (e.g., whether the task needs community attention), which human experts point out (Problem in the strengths and Prior Research in the weaknesses). For aspects, LLMs mostly focused on Validity in both strengths and weaknesses. In contrast, human experts considered the aspects more evenly among Validity, Novelty, and Clarity. Notably, LLMs exhibited a significant bias for Novelty. LLMs praised Novelty in the strengths, whereas they rarely criticized it in the weaknesses. This is a significant drawback, as a paper review requires a critical examination of novelty, by comparing them against existing work.

\begin{table*}[ht]
  \centering
  \caption{F1 Score for target and aspects between \texttt{DeepSeek-R1}, \texttt{o1-mini}, and \texttt{Llama-405B}. Due to the biased perspective of LLMs, we observed a clear gap between what LLMs mostly focus on (e.g., Method and Experiment targets and Validity aspect) and overlook (e.g., Problem target and Novelty aspect). Full results (F1 score, precision, and recall across models and target/aspect labels) are available in Appendix~\ref{appendix:result}. }
  \label{tab:merged_sw}
  \begin{minipage}[t]{0.48\textwidth}
    \centering
    \textbf{Target F1 score}\\[1ex]
    \footnotesize
    \begin{tabular}{lccc}
      \toprule
      Target       & \texttt{DeepSeek-R1}            & \texttt{o1-mini}              & \texttt{Llama-405B} \\
      \midrule
      Problem            & \swrow{0.30}{0.40}{0.20} & \swrow{0.28}{0.35}{0.20} & \swrow{0.16}{0.16}{0.16} \\
      \grayline
      Method             & \swrow{\textbf{0.73}}{0.75}{0.71} & \swrow{\textbf{0.76}}{0.75}{0.77} & \swrow{\textbf{0.69}}{0.76}{0.63} \\
      \grayline
      Theory             & \swrow{0.47}{0.44}{0.51} & \swrow{0.47}{0.41}{0.53} & \swrow{0.43}{0.46}{0.40} \\
      \grayline
      Experiment         & \swrow{\textbf{0.68}}{0.51}{\textbf{0.85}} & \swrow{\textbf{0.68}}{0.51}{\textbf{0.86}} & \swrow{\textbf{0.66}}{0.52}{\textbf{0.81}} \\
      \bottomrule
    \end{tabular}
  \end{minipage}\hfill
  \begin{minipage}[t]{0.48\textwidth}
    \centering
    \textbf{Aspect F1 score}\\[1ex]
    \footnotesize
    \begin{tabular}{lccc}
      \toprule
      Aspect         & \texttt{DeepSeek-R1}            & \texttt{o1-mini}              & \texttt{Llama-405B} \\
      \midrule
      Novelty               & \swrow{0.39}{0.66}{\textbf{0.12}} & \swrow{0.39}{0.66}{\textbf{0.12}} & \swrow{0.34}{0.66}{\textbf{0.01}} \\
      \grayline
      Impact                & \swrow{0.41}{0.54}{0.29} & \swrow{0.43}{0.56}{0.30} & \swrow{0.32}{0.35}{0.29} \\
      \grayline
      Validity              & \swrow{\textbf{0.77}}{0.60}{0.95} & \swrow{\textbf{0.77}}{0.60}{0.95} & \swrow{\textbf{0.77}}{0.60}{0.95} \\
      \grayline
      Clarity               & \swrow{0.27}{0.17}{0.36} & \swrow{0.40}{0.30}{0.50} & \swrow{0.28}{0.16}{0.40} \\
      \bottomrule
    \end{tabular}
  \end{minipage}
\end{table*}



Due to their biased focus, the level of agreement between LLMs and human experts varied across different labels. Table~\ref{tab:merged_sw} shows F1 scores for specific targets and aspects. For targets and aspects that LLMs focus more on --- Method and Experiment targets and Validity aspect --- LLMs had a much higher level of agreement with human experts compared to other targets and aspects. In the case of Experiment, the F1 score was consistently higher for weaknesses than strengths, suggesting that LLMs are more effective at identifying concerns in experiments (e.g., lack of baselines or scope of evaluation) than recognizing strong points of theories (e.g., experiments are rigorous and thorough). Similarly, for aspects other than Validity, agreement levels were notably lower. In particular, Novelty in the weaknesses, which LLMs largely overlooked, showed a significantly lower F1 score.

% Validity showed the highest agreement, but it would be due to our focus of \textit{rejected} papers. Specifically, we observed that LLMs almost always report Validity concerns in weaknesses (Recall $\geq$ 0.99 on average) where rejected papers might contain such limitations. Novelty aspect showed low agreement, particularly for weaknesses, which is due to the lack of focus in Novelty in evaluating weaknesses.

\textbf{LLMs showed similar patterns in their emphasis in reviewing papers, regardless of their size and reasoning capability.} All LLMs, including both proprietary and open source models, showed similar patterns that focused primarily on technical (Method, Experiment, and Theory) validity than on Novelty for the weaknesses. This consistency indicates that the observed biases could stem from the inherent design and training methods of LLMs, revealing potential room for improvement in the reasoning capability that requires leveraging external information (e.g., identifying comparable related work and analyzing novelty of submissions). 

\textbf{Final acceptance decisions are not accurate.} Table~\ref{tab:rejection_percent} shows the rejection rate reported by each LLM. Overall, the best achieved rejection rate was 24.9\%, which indicates that recall for rejected papers is poor. \texttt{gpt-4o}, \texttt{Llama-405B}, and \texttt{DeepSeek-R1} performed significantly better than other models. \texttt{gpt-4o} and \texttt{Llama-405B} showed relatively high rejection rates while their agreement with human experts on strengths and weaknesses was low (0.348 and 0.349). \texttt{DeepSeek-R1} demonstrated a moderate rejection rate among the models, with a relatively higher agreement score on strengths and weaknesses (0.373). 

\begin{table}[ht]
\centering
\caption{Rejection percentage by model. 100\% is the highest score, as we considered \textit{rejected} papers. All the models were highly positive about the paper acceptance, although the papers were rejected.}
\label{tab:rejection_percent}
\begin{tabular}{lc}
\toprule
Model       & Rejection (\%) \\
\midrule
\texttt{gpt-4o}      & 27.92\% \\
\texttt{Llama-405B}  & 24.30\% \\
\texttt{DeepSeek-r1} & 23.79\% \\
\texttt{gpt-4o-mini} & 9.50\%  \\
\texttt{DeepSeek-v3} & 7.93\%  \\
\texttt{o1-mini}     & 5.45\%  \\
\texttt{o1}          & 3.36\%  \\
\texttt{Llama-70B}   & 0.74\%  \\
\bottomrule
\end{tabular}
\end{table}


% \textbf{ChatGPT o1-pro with Deep Research shows the similar pattern with other LLMs, with poor rejection rate. \notyet{under experiment}}
% \input{sections/6_Implication}

\section{Related Work}

\begin{figure}[bt!]
    \centering
    % First row
    \begin{subfigure}[t]{0.48\linewidth}
        \centering
        \includegraphics[width=\textwidth]{figure/rep11.png}
        \caption{Objects with the prompt: \textit{A white truck that is stationary in the same direction.} \cite{nuprompt}}
    \end{subfigure}
    \hfill
    \begin{subfigure}[t]{0.48\linewidth}
        \centering
        \includegraphics[width=\textwidth]{figure/rep21.png}
        \caption{Frame-based object expression using numerical coordinates \cite{drivelm}.}
    \end{subfigure}
    
    
    % Second row
    \begin{subfigure}[t]{0.48\linewidth}
        \centering
        \includegraphics[width=\textwidth]{figure/TrafficQA-Object_Representation_rep12.jpg}
        \caption{Object referring in \cite{vidstg} with prompt: \textit{What is beneath the adult}.}
    \end{subfigure}
    \hfill
    \begin{subfigure}[t]{0.48\linewidth}
        \centering
        % \includegraphics[width=\textwidth]{figure/rep22.jpg}
        \includegraphics[width=\textwidth]{figure/TrafficQA-Object_Representation_22.jpg}
        \caption{Location of the green bus \textit{[(c1,0.0,0.5,0.4)]} in the video. (Ours)}
        \label{fig:objct_ref4}
    \end{subfigure}
    
    \caption{Different methods for describing objects in images and videos using language expressions. We adopt a tuple-based spatio-temporal object representation for the unique object reference, as shown in (d). }
    \label{fig:object_representation}
\end{figure}


% \begin{table}[htb]
% \centering
% \resizebox{0.5\textwidth}{!}{%
% \begin{tabular}{cccccccc}
% % \hline
% \midrule
% \makecell{\textbf{Dataset}} & \makecell{\textbf{Tasks}} & \makecell{\textbf{QA Gen.}} & \makecell{\textbf{\# Videos}\\\textbf{/Scenes}} & \makecell{\textbf{\# QAs}}  & \makecell{\textbf{\# Grounds.}} & \makecell{\textbf{Domain}} \\

% % \\\textbf{/Capts.}
% % \hline
% \midrule

% HAD \cite{had}         & Video QA & Manual & 5.6k & 45k & - & Driving \\

% DRAMA \cite{malla2023drama}         & Video QA& Manual & 18k  & 102k  & - & Driving \\

% LingoQA \cite{marcu2024lingoqavisualquestionanswering}  & Video QA & Manual & 28k & 419k & - & Driving \\

% NuScenes-QA \cite{qian2024nuscenes}         & Image QA & Template & 850 & 460k &  - & Driving \\

% DriveLM \cite{sima2023drivelm}         & Image QA & Temp. + Man. & 188k  & 4.2M & - & Driving \\

% City-3DQA \cite{sun20243dquestionansweringcity} & Scene QA & Temp + Man. & 193 & 450k & - &  City \\

% \midrule
% HC-STVG \cite{hc-stvg} & Video Grounding & Manual &5.6k & - & 5.6k&General\\

% DVD-ST \cite{dvd-st} & Video Grounding & Manual & 2.7k & - &5.7k & General  \\

% Refer-KITTI \cite{referkitti} & Referred-MOT & Manual & 18 & - & 818 & Driving \\

% NuPrompt \cite{nuprompt}         & Referred-MOT & LLM & 850 & - & 35k  & Driving \\

% \midrule


% \textbf{TUMTraffic-VideoQA (Ours)} & \makecell{Video QA, \\ST Grounding} & Temp. + LLM  &1k & 88k  & 5.7k &  Roadside \\


% % \hline
% \midrule
% \end{tabular}%
% }
% \caption{Summary and comparison of language datasets in the traffic domain for question answering, video grounding, and referred multi-object tracking.}
% \label{tab:related_datasets}
% \end{table}



\begin{table*}[thb!]
\centering
\caption{Summary of visual-language datasets in the traffic domain for question answering, video grounding, and referred multi-object tracking. The table’s upper section presents QA tasks, while the lower section covers grounding and referring tasks. We introduce the first roadside video understanding dataset and unify the tasks in one benchmark. }
\resizebox{\textwidth}{!}{%
\begin{tabular}{c|ccccccccccc}
% \hline
\midrule
\textbf{Dataset} & \textbf{Venue} & \textbf{Tasks} & \textbf{QA Gen.} & \textbf{\# Videos/Scenes} & \textbf{\# QAs/Captions}  & \textbf{\# Grounding} & \textbf{Domain} \\
% \hline
\midrule


% BDD-X \cite{kim2018textual}         & ECCV18 & video-level & Manual & $\sim$7k (v) & $\sim$26k & $\sim$3.7 & $\sim$77h & - & 1 & 4 & Driving \\

% HAD \cite{had}         & CVPR'19 & Video QA & Manual & 5.6k & 45k & - & Driving \\

% SUTD \cite{xu2021sutd}         & CVPR 2021 & video-level & Manual & $\sim$10k (v) & $\sim$63k & $\sim$6.3 & - & 70s & - & - & D + T \\

DRAMA \cite{malla2023drama}         & WACV'23 & Video QA& Manual & 18k  & 102k  & - & Driving \\
LingoQA \cite{marcu2024lingoqavisualquestionanswering}  & ECCV'24 & Video QA & Manual & 28k & 419k & - & Driving & \\

NuScenes-QA \cite{qian2024nuscenes}         & AAAI'24 & Image QA & Template & 850 & 460k &  - & Driving \\

DriveLM \cite{drivelm}         & ECCV'24 & Image QA & Temp. + Man. & 188k  & 4.2M & - & Driving \\

% ELM \cite{zhou2024embodied}         & ECCV'24 & Video-Level & Temp. + LLM & - & $\sim$9M & - &  Driving \\


% SQA-3D \cite{sqa3d}  & ICLR'23 & Scene QA & Manual & 650 & 33.4k & - & Indoor\\

City-3DQA \cite{sun20243dquestionansweringcity} & ACM MM'24& Scene QA & Temp. + Man. & 193 & 450k & - &  City \\

\midrule
HC-STVG \cite{hc-stvg} & ACM MM'22 & Video Grounding & Manual &5.6k & - & 5.6k&General\\

DVD-ST \cite{dvd-st} & -  & Video Grounding & Manual & 2.7k & - &5.7k & General  \\

Refer-KITTI \cite{referkitti} & CVPR'23  & Referred-MOT & Manual & 18 & - & 818 & Driving \\

NuPrompt \cite{nuprompt}         & AAAI'25 & Referred-MOT & LLM & 850 & - & 35k  & Driving \\

% STPR && Video-Level &&5.2k&-&30k &General \\

% VD-STG\cite{vidstg} &&&&\\

\midrule

\textbf{TUMTraffic-VideoQA (Ours)} & - & Video QA, ST Grounding & Temp. + LLM  &1k & 87.3k  & 5.7k &  Roadside \\

% \hline
\midrule
\end{tabular}%
}

\label{tab:related_datasets}
\end{table*}

% Granularity in thousands?

% Can we trust an information about a dataset which was found only in another paper?  

% Modality ?


% \begin{table}[htb]
% \centering
% \resizebox{0.5\textwidth}{!}{%
% \begin{tabular}{c|ccccc}
% % \hline
% \midrule
% \textbf{Dataset} & \textbf{Task} & \textbf{\#Scenes} & \textbf{\#QA}  & \textbf{\#Grounding} & \textbf{Domain} \\
% % \hline
% \midrule

% HAD \cite{had}         & Video QA & $\sim$5.6k & $\sim$45k & - & Driving \\

% DRAMA \cite{malla2023drama}         & Video QA & $\sim$18k  & $\sim$102k  & - & Driving \\

% NuScenes-QA \cite{qian2024nuscenes}         & Image QA & 850 & $\sim$460k &  - & Driving \\

% DriveLM \cite{sima2023drivelm}         & Image QA & $\sim$188k  & $\sim$4.2M & - & Driving \\

% SQA-3D \cite{sqa3d}  & Scene QA & 650 & 33.4k & - & Indoor \\

% City-3DQA \cite{sun20243dquestionansweringcity} & Scene QA & 193 & 450k & - & City \\

% \midrule
% Refer-KITTI\cite{referkitti} & Referred-MOT & 18 & - & 818 & Driving \\

% NuPrompt \cite{nuprompt}         & Referred-MOT & 850 & - & 35k  & Driving \\

% DVD-ST\cite{dvd-st} & Video Grounding & 2.7k & - &5.7k & General \\

% HC-STVG\cite{hc-stvg} & Video Grounding & 5.6k & - & 5.6k & General \\

% \midrule

% \textbf{TUMTraffic-VideoQA(Ours)} & Video QA, Grounding & 1k & 88k  & 5.7k & Traffic \\

% % \hline
% \midrule
% \end{tabular}%
% }
% \caption{Related datasets}
% \label{tab:related_datasets}
% \end{table}


% [x] connect table 1 with introductions. 

\subsection{Vision-Language Datasets in Traffic Scenes}
% DriveLM\cite{drivelm},
% HAD \cite{had} and 
With the rapid advancements in LLMs, significant efforts have been made to integrate language into the development of vision-language foundation models. As summarized in Table \ref{tab:related_datasets}, several pioneering datasets have been introduced for traffic scenarios, particularly focusing on vehicle-centric environments \cite{addatasetseurvey}. NuScenes-QA \cite{qian2024nuscenes} provides a question-answering benchmark tailored for driving scenes. Meanwhile, DRAMA \cite{malla2023drama} is designed for video-level open-ended tasks aimed at evaluating driving instructions and assessing the importance of objects within their environments. Besides, referring to specific traffic participants through natural language—commonly known as referred object grounding and tracking—is a crucial task in traffic scene understanding. Some works \cite{referkitti,nuprompt} extend the KITTI \cite{kitti} and nuScenes \cite{caesar2020nuscenesmultimodaldatasetautonomous} datasets, by associating natural language descriptions with specific vehicles and pedestrians. This facilitates fine-grained identification and tracking of traffic participants, allowing for precise object localization based on language descriptions in complex driving environments. However, most existing efforts primarily focus on driving scenarios and are typically constrained to individual tasks such as question answering, video grounding, or referred multi-object tracking. A significant research gap also remains in the availability of large-scale datasets designed specifically for roadside surveillance scenarios. Our work aims to bridge this gap by providing a comprehensive dataset tailored for multiple tasks in roadside traffic understanding within a unified framework.
% is also an important aspect of traffic scene understanding
% introducing a standardized object representation and 


\subsection{Fine-Grained Video Understanding}

Fine-grained video understanding centers on the precise analysis of intricate video content, targeting tasks that demand nuanced reasoning across spatial and temporal dimensions. Some representative tasks include spatio-temporal grounding \cite{vidstg,hc-stvg}, mapping specific objects or events to precise locations and times within a video based on a given query; video object referring \cite{mevis,referkitti,nuprompt}, which involves tracking objects through space and time given text prompts; video temporal grounding \cite{UniVTG,huang2024vtimellm}, identifying specific moments or intervals in a video that align with a provided textual query. These tasks require high precision, nuanced multimodal alignment, and the ability to capture subtle temporal and spatial dynamics. It is particularly challenging due to the difficulty of properly representing fine-grained video details and the inherent cross-modality misalignment. With the advancement of visual LLMs, recent advancements enhance the capabilities of fine-grained video understanding \cite{videunderstandingsurvey} and facilitate understanding across abstract and detailed levels. 

% , with advanced visual embedding techniques and modality alignment strategies to bridge the gap between textual and visual semantics, significantly





\subsection{Language-Based Object Referring}


Referring objects in visual data, such as images and videos, is typically achieved by associating them with predefined definitions or language descriptions. Figure \ref{fig:object_representation} illustrates four commonly used methods for representing objects through language expressions. The inherent ambiguity of natural language, coupled with the modality gap between visual and linguistic representations, presents significant challenges. Object representation in tasks such as object referring often necessitates careful dataset curation to ensure that linguistic expressions uniquely or collectively correspond to specific objects in videos. For example, some datasets include only scenarios with uniquely identifiable objects \cite{hc-stvg}, while others contain expressions that jointly refer to multiple objects \cite{dvd-st}. However, in complex real-world applications such as autonomous driving, textual descriptions alone are often insufficient to uniquely specify an object. To address this challenge, DriveLM \cite{drivelm} introduces a structured tuple representation, $\textless c, CAM, x, y \textgreater$, where  c  denotes the object identifier,  CAM  specifies the camera, and $\textless x, y \textgreater$ represents the 2D center coordinates within the camera’s coordinate system. Alternatively, ELM \cite{zhou2024embodied} simplifies the problem by converting temporal video tasks into frame-level questions, using a tuple $\textless c, x, y \textgreater$ to identify objects within individual frames without temporal dependencies. Despite the advancements, formulating a unified, precise, and unique language representation for objects in video remains open challenges. 




In this work, we design a spatio-temporal object representation in videos with a four-element tuple format $(c, f_n, x, y)$, where c denotes a unique object identifier, $f_n$ indicates the normalized frame timestamp, and $(x, y)$ corresponds to the object’s normalized spatial coordinates within the frame.  The same object is consistently assigned the identifier  c  throughout the video, while its spatial position changes over time. This formulation enables precise tracking and referencing of objects across both spatial and temporal dimensions, facilitating robust language-based interaction in dynamic environments. Besides, it provides a standardized interface for fine-grained video understanding, enabling more detailed and structured analysis.

 
\section{Conclusion}

We presented \sys, a sparsity-adaptive attention mechanism for efficient long-context LLM inference. Unlike fixed token budget methods, \sys dynamically selects tokens based on cumulative attention scores, adapting to variations in attention sparsity. By leveraging clustering-based sorting and distribution fitting, \sys accurately estimates token importance with low overhead. Our results showed that \sys outperforms existing sparse attention methods, achieving higher accuracy and significant inference speedups, making it a practical solution for long-context LLMs.



% \section{Introduction}

% These instructions are for authors submitting papers to *ACL conferences using \LaTeX. They are not self-contained. All authors must follow the general instructions for *ACL proceedings,\footnote{\url{http://acl-org.github.io/ACLPUB/formatting.html}} and this document contains additional instructions for the \LaTeX{} style files.

% The templates include the \LaTeX{} source of this document (\texttt{acl\_latex.tex}),
% the \LaTeX{} style file used to format it (\texttt{acl.sty}),
% an ACL bibliography style (\texttt{acl\_natbib.bst}),
% an example bibliography (\texttt{custom.bib}),
% and the bibliography for the ACL Anthology (\texttt{anthology.bib}).

% \section{Engines}

% To produce a PDF file, pdf\LaTeX{} is strongly recommended (over original \LaTeX{} plus dvips+ps2pdf or dvipdf). Xe\LaTeX{} also produces PDF files, and is especially suitable for text in non-Latin scripts.

% \section{Preamble}

% The first line of the file must be
% \begin{quote}
% \begin{verbatim}
% \documentclass[11pt]{article}
% \end{verbatim}
% \end{quote}

% To load the style file in the review version:
% \begin{quote}
% \begin{verbatim}
% \usepackage[review]{acl}
% \end{verbatim}
% \end{quote}
% For the final version, omit the \verb|review| option:
% \begin{quote}
% \begin{verbatim}
% \usepackage{acl}
% \end{verbatim}
% \end{quote}

% To use Times Roman, put the following in the preamble:
% \begin{quote}
% \begin{verbatim}
% \usepackage{times}
% \end{verbatim}
% \end{quote}
% (Alternatives like txfonts or newtx are also acceptable.)

% Please see the \LaTeX{} source of this document for comments on other packages that may be useful.

% Set the title and author using \verb|\title| and \verb|\author|. Within the author list, format multiple authors using \verb|\and| and \verb|\And| and \verb|\AND|; please see the \LaTeX{} source for examples.

% By default, the box containing the title and author names is set to the minimum of 5 cm. If you need more space, include the following in the preamble:
% \begin{quote}
% \begin{verbatim}
% \setlength\titlebox{3cm}
% \end{verbatim}
% \end{quote}
% where \verb|<dim>| is replaced with a length. Do not set this length smaller than 5 cm.

% \section{Document Body}

% \subsection{Footnotes}

% Footnotes are inserted with the \verb|\footnote| command.\footnote{This is a footnote.}

% \subsection{Tables and figures}

% See Table~\ref{tab:accents} for an example of a table and its caption.
% \textbf{Do not override the default caption sizes.}

% \begin{table}
%   \centering
%   \begin{tabular}{lc}
%     \hline
%     \textbf{Command} & \textbf{Output} \\
%     \hline
%     \verb|{\"a}|     & {\"a}           \\
%     \verb|{\^e}|     & {\^e}           \\
%     \verb|{\`i}|     & {\`i}           \\
%     \verb|{\.I}|     & {\.I}           \\
%     \verb|{\o}|      & {\o}            \\
%     \verb|{\'u}|     & {\'u}           \\
%     \verb|{\aa}|     & {\aa}           \\\hline
%   \end{tabular}
%   \begin{tabular}{lc}
%     \hline
%     \textbf{Command} & \textbf{Output} \\
%     \hline
%     \verb|{\c c}|    & {\c c}          \\
%     \verb|{\u g}|    & {\u g}          \\
%     \verb|{\l}|      & {\l}            \\
%     \verb|{\~n}|     & {\~n}           \\
%     \verb|{\H o}|    & {\H o}          \\
%     \verb|{\v r}|    & {\v r}          \\
%     \verb|{\ss}|     & {\ss}           \\
%     \hline
%   \end{tabular}
%   \caption{Example commands for accented characters, to be used in, \emph{e.g.}, Bib\TeX{} entries.}
%   \label{tab:accents}
% \end{table}

% As much as possible, fonts in figures should conform
% to the document fonts. See Figure~\ref{fig:experiments} for an example of a figure and its caption.

% Using the \verb|graphicx| package graphics files can be included within figure
% environment at an appropriate point within the text.
% The \verb|graphicx| package supports various optional arguments to control the
% appearance of the figure.
% You must include it explicitly in the \LaTeX{} preamble (after the
% \verb|\documentclass| declaration and before \verb|\begin{document}|) using
% \verb|\usepackage{graphicx}|.

% \begin{figure}[t]
%   \includegraphics[width=\columnwidth]{example-image-golden}
%   \caption{A figure with a caption that runs for more than one line.
%     Example image is usually available through the \texttt{mwe} package
%     without even mentioning it in the preamble.}
%   \label{fig:experiments}
% \end{figure}

% \begin{figure*}[t]
%   \includegraphics[width=0.48\linewidth]{example-image-a} \hfill
%   \includegraphics[width=0.48\linewidth]{example-image-b}
%   \caption {A minimal working example to demonstrate how to place
%     two images side-by-side.}
% \end{figure*}

% \subsection{Hyperlinks}

% Users of older versions of \LaTeX{} may encounter the following error during compilation:
% \begin{quote}
% \verb|\pdfendlink| ended up in different nesting level than \verb|\pdfstartlink|.
% \end{quote}
% This happens when pdf\LaTeX{} is used and a citation splits across a page boundary. The best way to fix this is to upgrade \LaTeX{} to 2018-12-01 or later.

% \subsection{Citations}

% \begin{table*}
%   \centering
%   \begin{tabular}{lll}
%     \hline
%     \textbf{Output}           & \textbf{natbib command} & \textbf{ACL only command} \\
%     \hline
%     \citep{Gusfield:97}       & \verb|\citep|           &                           \\
%     \citealp{Gusfield:97}     & \verb|\citealp|         &                           \\
%     \citet{Gusfield:97}       & \verb|\citet|           &                           \\
%     \citeyearpar{Gusfield:97} & \verb|\citeyearpar|     &                           \\
%     \citeposs{Gusfield:97}    &                         & \verb|\citeposs|          \\
%     \hline
%   \end{tabular}
%   \caption{\label{citation-guide}
%     Citation commands supported by the style file.
%     The style is based on the natbib package and supports all natbib citation commands.
%     It also supports commands defined in previous ACL style files for compatibility.
%   }
% \end{table*}

% Table~\ref{citation-guide} shows the syntax supported by the style files.
% We encourage you to use the natbib styles.
% You can use the command \verb|\citet| (cite in text) to get ``author (year)'' citations, like this citation to a paper by \citet{Gusfield:97}.
% You can use the command \verb|\citep| (cite in parentheses) to get ``(author, year)'' citations \citep{Gusfield:97}.
% You can use the command \verb|\citealp| (alternative cite without parentheses) to get ``author, year'' citations, which is useful for using citations within parentheses (e.g. \citealp{Gusfield:97}).

% A possessive citation can be made with the command \verb|\citeposs|.
% This is not a standard natbib command, so it is generally not compatible
% with other style files.

% \subsection{References}

% \nocite{Ando2005,andrew2007scalable,rasooli-tetrault-2015}

% The \LaTeX{} and Bib\TeX{} style files provided roughly follow the American Psychological Association format.
% If your own bib file is named \texttt{custom.bib}, then placing the following before any appendices in your \LaTeX{} file will generate the references section for you:
% \begin{quote}
% \begin{verbatim}
% \bibliography{custom}
% \end{verbatim}
% \end{quote}

% You can obtain the complete ACL Anthology as a Bib\TeX{} file from \url{https://aclweb.org/anthology/anthology.bib.gz}.
% To include both the Anthology and your own .bib file, use the following instead of the above.
% \begin{quote}
% \begin{verbatim}
% \bibliography{anthology,custom}
% \end{verbatim}
% \end{quote}

% Please see Section~\ref{sec:bibtex} for information on preparing Bib\TeX{} files.

% \subsection{Equations}

% An example equation is shown below:
% \begin{equation}
%   \label{eq:example}
%   A = \pi r^2
% \end{equation}

% Labels for equation numbers, sections, subsections, figures and tables
% are all defined with the \verb|\label{label}| command and cross references
% to them are made with the \verb|\ref{label}| command.

% This an example cross-reference to Equation~\ref{eq:example}.

% \subsection{Appendices}

% Use \verb|\appendix| before any appendix section to switch the section numbering over to letters. See Appendix~\ref{sec:appendix} for an example.

% \section{Bib\TeX{} Files}
% \label{sec:bibtex}

% Unicode cannot be used in Bib\TeX{} entries, and some ways of typing special characters can disrupt Bib\TeX's alphabetization. The recommended way of typing special characters is shown in Table~\ref{tab:accents}.

% Please ensure that Bib\TeX{} records contain DOIs or URLs when possible, and for all the ACL materials that you reference.
% Use the \verb|doi| field for DOIs and the \verb|url| field for URLs.
% If a Bib\TeX{} entry has a URL or DOI field, the paper title in the references section will appear as a hyperlink to the paper, using the hyperref \LaTeX{} package.

% \section*{Acknowledgments}

% This document has been adapted
% by Steven Bethard, Ryan Cotterell and Rui Yan
% from the instructions for earlier ACL and NAACL proceedings, including those for
% ACL 2019 by Douwe Kiela and Ivan Vuli\'{c},
% NAACL 2019 by Stephanie Lukin and Alla Roskovskaya,
% ACL 2018 by Shay Cohen, Kevin Gimpel, and Wei Lu,
% NAACL 2018 by Margaret Mitchell and Stephanie Lukin,
% Bib\TeX{} suggestions for (NA)ACL 2017/2018 from Jason Eisner,
% ACL 2017 by Dan Gildea and Min-Yen Kan,
% NAACL 2017 by Margaret Mitchell,
% ACL 2012 by Maggie Li and Michael White,
% ACL 2010 by Jing-Shin Chang and Philipp Koehn,
% ACL 2008 by Johanna D. Moore, Simone Teufel, James Allan, and Sadaoki Furui,
% ACL 2005 by Hwee Tou Ng and Kemal Oflazer,
% ACL 2002 by Eugene Charniak and Dekang Lin,
% and earlier ACL and EACL formats written by several people, including
% John Chen, Henry S. Thompson and Donald Walker.
% Additional elements were taken from the formatting instructions of the \emph{International Joint Conference on Artificial Intelligence} and the \emph{Conference on Computer Vision and Pattern Recognition}.

% Bibliography entries for the entire Anthology, followed by custom entries
%\bibliography{anthology,custom}
% Custom bibliography entries only
\bibliography{main}


\clearpage
\appendix
\begin{onecolumn}
\section{Appendix}

\label{sec:appendix}



\subsection{Review Generation}
\subsubsection{Prompts for Expert Review Generation}
\label{appendix:gt-prompt}

In this section, we provide prompts for identifying key strength and weakness from review data. Figure~\ref{fig:metareview-summarization} shows the prompt for extracting weakness and strength from meta-review. Figure~\ref{fig:augmentedReview} shows the prompt for using detailed comments from reviews to augment the extracted elements. Figure~\ref{fig:paraphrasing} shows the prompt for removing some extraneous reference. We used the three prompts in a prompt chain, sequentially running the prompts.

\begin{figure*}[h]
    \centering
    \includegraphics[width=\linewidth]{Prompt-Meta-Review-Summarization.pdf}
    \caption{Prompt for Meta-Review Summarization}
    \label{fig:metareview-summarization}
\end{figure*}

\begin{figure*}[h]
    \centering
    \includegraphics[width=\linewidth]{Prompt-AugmentedReview.pdf}
    \caption{Prompt for Generating  Augmented Review }
    \label{fig:augmentedReview}
\end{figure*}

\begin{figure*}[h]
    \centering
    \includegraphics[width=\linewidth]{Prompt-Paraphrasing-AugmentedReview.pdf}
    \caption{Prompt for Paraphrasing  Augmented Review }
    \label{fig:paraphrasing}
\end{figure*}


\clearpage


% \subsection{Additional Experiments Results}

% In this section, we provide detailed results on the focus of LLM and human reviewers when they are reviewing papers.


% confusion matrix  ggests that the er- rors tend to occur in semantically related categories, indicating that the misclassifications are not arbitrary but rather reflect subtle ambiguities inherent in the data. 


\


\subsubsection{Prompts for LLM Review Generation}

\label{appendix:llm-prompt}
Figure~\ref{fig:llm-review-gen} shows the prompt for using LLM to generate reviews from paper.


\begin{figure*}[h]
    \centering
    \includegraphics[width=1\linewidth]{prompt-generate-review.pdf}
    \caption{Prompt for LLM Review Generation}
    \label{fig:llm-review-gen}
\end{figure*}

\clearpage


\subsection{Details of Building Automatic Annotator }
\subsubsection{AI paper writing guidelines}
\label{appendix:guidelines}
To ensure guidelines are comprehensive, we collected guidelines from 9 sources, comprising a total of 243 items, as shown in Table~\ref{tab:guideline_item_count}. An item refers to a specific requirement mentioned in the guidelines, which serves as a distinct criterion for reviewing or writing a paper.
\begin{table}[h]
\centering
\caption{Guidelines and Item Count Summary}
\label{tab:guideline_item_count}
\begin{tabular}{lr}
\toprule
Guideline & Item Count \\
\midrule
ICML Paper Writing Best Practices\footnotemark[1] & 38 \\
ICML 2023 Paper Guidelines\footnotemark[2] & 30 \\
NIPS 2024 Reviewer Guidelines\footnotemark[3] & 18 \\
ACL Checklist\footnotemark[4] & 49 \\
How to Write a Good Research Paper in the Machine Learning Area\footnotemark[5] & 6 \\
ACL Ethics Review Questions\footnotemark[6] & 21 \\
AAAI Reproducibility Checklist\footnotemark[7] & 29 \\
NeurIPS 2021 Paper Checklist Guidelines\footnotemark[8] & 46 \\
ICLR 2019 Guidelines\footnotemark[9] & 6 \\
\midrule
\textbf{Total Count} & \textbf{243} \\
\bottomrule
\end{tabular}
\end{table}

\footnotetext[1]{\url{https://icml.cc/Conferences/2022/BestPractices}}
\footnotetext[2]{\url{https://icml.cc/Conferences/2023/PaperGuidelines}}
\footnotetext[3]{\url{https://neurips.cc/Conferences/2024/ReviewerGuidelines}}
\footnotetext[4]{\url{https://aclrollingreview.org/responsibleNLPresearch/}}
\footnotetext[5]{\url{https://www.turing.com/kb/how-to-write-research-paper-in-machine-learning-area}}
\footnotetext[6]{\url{https://2023.eacl.org/ethics/review-questions/}}
\footnotetext[7]{\url{https://aaai.org/conference/aaai/aaai-25/aaai-25-reproducibility-checklist/}}
\footnotetext[8]{\url{https://neurips.cc/Conferences/2021/PaperInformation/PaperChecklist}}
\footnotetext[9]{\url{https://iclr.cc/Conferences/2019/Reviewer_Guidelines}}




\subsubsection{Prompts}
\label{appendix:annotate-prompt}

In this section, we provide prompts designed to annotate reviews. We designed 4 prompts where each corresponds to one of the four combinations of target/aspect and strength/weakness. Specifically, we designed Target-Strength (Figure~\ref{fig:target-strength-prompt}), Aspect-Strength, (Figure~\ref{fig:aspect-strength-prompt}), Target-Weakness (Figure~\ref{fig:target-weakness-prompt}) , and Aspect-Weakness (Figure~\ref{fig:aspect-weakness-prompt}) prompts.





% Strengths and weaknesses identification prompt 

\begin{figure*}[t]
    \centering
    \includegraphics[width=\linewidth, height=1\textheight, keepaspectratio]{prompt-target-strength.pdf}
    \caption{Prompt for Automatic Target Annotation for Strength}
    \label{fig:target-strength-prompt}
\end{figure*}


\begin{figure*}[t]
    \centering
    \includegraphics[width=\linewidth, height=1\textheight, keepaspectratio]{prompt-target-weakness.pdf}
    \caption{Prompt for Automatic Target Annotation for Weakness}
    \label{fig:target-weakness-prompt}
\end{figure*}

\begin{figure*}[t]
    \centering
    \includegraphics[width=\linewidth, height=1\textheight, keepaspectratio]{prompt-aspect-strength.pdf}
    \caption{Prompt for Automatic Aspect Annotation for Strength}
    \label{fig:aspect-strength-prompt}
\end{figure*}

\begin{figure*}[t]
    \centering
    \includegraphics[width=\linewidth, height=1\textheight, keepaspectratio]{prompt-aspect-weakness.pdf}
    \caption{Prompt for Automatic Aspect Annotation for Weakness}
    \label{fig:aspect-weakness-prompt}
\end{figure*}

\clearpage

\subsubsection{Annotation Comparison}
\label{appendix:annotate-confusion-matrix}

We present a comparison between LLM and human annotations for both target and aspect. Figures~\ref{fig:error-target-matrix} and Figure~\ref{fig:error-aspect-matrix} illustrate the discrepancies. Areas of alignment between LLM and human annotations are shown in green, while red highlights regions with significant discrepancies.




\begin{figure}[h]
    \centering
    \includegraphics[width=0.5\linewidth]{error-target-matrix.png}
    \caption{LLM vs. human target annotation}
    \label{fig:error-target-matrix}
\end{figure}


\begin{figure}[h]
    \centering
    \includegraphics[width=0.5\linewidth]{aspect-error-matrix.png}
    \caption{LLM vs. human aspect annotation}
    \label{fig:error-aspect-matrix}
\end{figure}


While LLM annotations differ from human annotations in some cases, certain discrepancies remain reasonable. Figure~\ref{fig:target-case} and Figure~\ref{fig:aspect-case}  illustrate examples of such reasonable discrepancies.

\begin{figure}[h]
    \centering
    \includegraphics[width=1\linewidth]{target-case.pdf}
    \caption{Cases of Target Annotation Discrepancy}
    \label{fig:target-case}

\end{figure}

\begin{figure}[h]
    \centering
    \includegraphics[width=1\linewidth]{aspect-case.pdf}
    \caption{Cases of  Aspect Annotation Discrepancy}
    \label{fig:aspect-case}

\end{figure}

\clearpage

\subsubsection{Results}
\label{appendix:result}

The following tables present a comprehensive performance comparison of models across different metrics and evaluation targets, including both strengths and weaknesses (Table~\ref{tab:target-all}), as well as separate analyses focusing on strengths (Table~\ref{tab:target_strength_metrics}) and weaknesses (Table~\ref{tab:target_weakness_metrics}). Additionally, we provide a similar comparison across metrics and broader aspects, including both strengths and weaknesses (Table~\ref{tab:aspect-all}), strengths alone (Table~\ref{fig:aspect-strength}), and weaknesses alone (Table~\ref{fig:aspect-weakness}).


\begin{table}[h]
    \centering
    \caption{Performance Comparison of Models Across Metrics and Targets (Including both Strengths and Weaknesses)}    
    \small
    \begin{tabular}{lccccccc}
        \toprule
        Target & Problem & Prior Research & Method & Theory & Experiment & Conclusion & Paper \\
        \midrule
        F1 (gpt-4o-mini) & 0.268 & 0.076 & 0.737 & 0.427 & 0.680 & 0.103 & 0.227 \\
        F1 (gpt-4o) & 0.292 & 0.052 & 0.741 & 0.448 & 0.673 & 0.089 & 0.247 \\
        F1 (o1-mini) & 0.275 & 0.054 & \textbf{0.764} & 0.472 & \textbf{0.684} & \textbf{0.175} & \textbf{0.253} \\
        F1 (o1) & 0.274 & 0.044 & 0.754 & \textbf{0.489} & 0.673 & 0.133 & 0.091 \\
        F1 (llama-70B) & 0.269 & 0.049 & 0.711 & 0.410 & 0.659 & 0.172 & 0.158 \\
        F1 (llama-405B) & 0.158 & 0.031 & 0.690 & 0.427 & 0.662 & 0.167 & 0.134 \\
        F1 (deepseek-r1) & \textbf{0.297} & \textbf{0.081} & 0.729 & 0.473 & 0.682 & 0.164 & 0.152 \\
        F1 (deepseek-v3) & 0.241 & 0.051 & 0.725 & 0.405 & 0.680 & 0.110 & 0.092 \\
        \midrule
        Prec (gpt-4o-mini) & 0.317 & 0.134 & 0.647 & 0.317 & 0.549 & 0.063 & 0.241 \\
        Prec (gpt-4o) & 0.298 & 0.109 & 0.634 & 0.334 & 0.547 & 0.057 & 0.251 \\
        Prec (o1-mini) & 0.315 & 0.130 & 0.639 & 0.342 & 0.549 & 0.107 & 0.274 \\
        Prec (o1) & 0.279 & 0.064 & \textbf{0.648} & \textbf{0.381} & 0.549 & 0.111 & 0.245 \\
        Prec (llama-70B) & \textbf{0.339} & \textbf{0.143} & 0.653 & 0.295 & 0.548 & 0.105 & 0.289 \\
        Prec (llama-405B) & 0.324 & 0.071 & 0.647 & 0.310 & \textbf{0.558} & 0.115 & 0.233 \\
        Prec (deepseek-r1) & 0.321 & 0.099 & 0.639 & 0.327 & 0.549 & \textbf{0.135} & \textbf{0.301} \\
        Prec (deepseek-v3) & 0.288 & 0.100 & 0.645 & 0.280 & 0.547 & 0.076 & 0.249 \\
        \midrule
        Rec (gpt-4o-mini) & 0.233 & 0.053 & 0.870 & 0.691 & 0.983 & 0.274 & 0.232 \\
        Rec (gpt-4o) & 0.297 & 0.034 & 0.899 & 0.723 & 0.965 & 0.202 & \textbf{0.270} \\
        Rec (o1-mini) & 0.266 & 0.034 & \textbf{0.952} & 0.834 & \textbf{0.994} & \textbf{0.536} & 0.249 \\
        Rec (o1) & \textbf{0.353} & 0.034 & 0.905 & 0.736 & 0.963 & 0.167 & 0.056 \\
        Rec (llama-70B) & 0.246 & 0.030 & 0.803 & 0.720 & 0.919 & 0.476 & 0.146 \\
        Rec (llama-405B) & 0.108 & 0.020 & 0.774 & 0.694 & 0.894 & 0.300 & 0.095 \\
        Rec (deepseek-r1) & 0.299 & \textbf{0.069} & 0.859 & \textbf{0.865} & 0.983 & 0.357 & 0.102 \\
        Rec (deepseek-v3) & 0.210 & 0.035 & 0.844 & 0.755 & 0.981 & 0.238 & 0.058 \\
        \bottomrule
    \end{tabular}

    \label{tab:target-all}
\end{table}


\begin{table}[h]
    \centering
    \small
    \caption{Performance Comparison of Models Across Metrics and Targets (Strengths)}
    
    \begin{tabular}{lccccccc}
        \toprule
        Target & Problem & Prior Research & Method & Theory & Experiment & Conclusion & Paper \\
        \midrule
        F1 (gpt-4o-mini) & 0.283 & 0.000 & \textbf{0.760} & 0.424 & 0.511 & 0.118 & 0.232 \\
        F1 (gpt-4o) & 0.329 & 0.000 & 0.756 & 0.446 & \textbf{0.517} & 0.143 & 0.119 \\
        F1 (o1-mini) & 0.345 & 0.000 & 0.753 & 0.411 & 0.511 & 0.300 & \textbf{0.233} \\
        F1 (o1) & 0.384 & 0.000 & 0.749 & \textbf{0.470} & 0.512 & 0.267 & 0.061 \\
        F1 (llama-70B) & 0.245 & 0.000 & 0.750 & 0.420 & 0.516 & 0.242 & 0.198 \\
        F1 (llama-405B) & 0.160 & 0.000 & 0.755 & 0.455 & 0.516 & \textbf{0.333} & 0.079 \\
        F1 (deepseek-r1) & \textbf{0.396} & 0.000 & 0.749 & 0.436 & 0.513 & 0.174 & 0.135 \\
        F1 (deepseek-v3) & 0.331 & 0.000 & 0.755 & 0.423 & 0.509 & 0.114 & 0.086 \\
        \midrule
        Prec (gpt-4o-mini) & 0.315 & 0.000 & 0.622 & 0.286 & 0.343 & 0.071 & 0.198 \\
        Prec (gpt-4o) & 0.295 & 0.000 & 0.616 & 0.299 & 0.350 & 0.091 & 0.182 \\
        Prec (o1-mini) & 0.314 & 0.000 & 0.611 & 0.264 & 0.343 & 0.176 & 0.203 \\
        Prec (o1) & 0.285 & 0.000 & \textbf{0.624} & \textbf{0.322} & 0.346 & 0.222 & 0.172 \\
        Prec (llama-70B) & 0.404 & 0.000 & 0.620 & 0.275 & 0.352 & 0.148 & 0.178 \\
        Prec (llama-405B) & \textbf{0.419} & 0.000 & 0.620 & 0.319 & \textbf{0.358} & \textbf{0.231} & 0.163 \\
        Prec (deepseek-r1) & 0.355 & 0.000 & 0.617 & 0.289 & 0.347 & 0.103 & \textbf{0.279} \\
        Prec (deepseek-v3) & 0.364 & 0.000 & 0.620 & 0.276 & 0.344 & 0.069 & 0.154 \\
        \midrule
        Rec (gpt-4o-mini) & 0.258 & 0.000 & 0.975 & 0.819 & \textbf{0.996} & 0.333 & \textbf{0.281} \\
        Rec (gpt-4o) & 0.371 & 0.000 & 0.978 & 0.872 & 0.991 & 0.333 & 0.089 \\
        Rec (o1-mini) & 0.382 & 0.000 & \textbf{0.980} & \textbf{0.935} & \textbf{0.996} & \textbf{1.000} & 0.274 \\
        Rec (o1) & \textbf{0.588} & 0.000 & 0.936 & 0.872 & 0.987 & 0.333 & 0.037 \\
        Rec (llama-70B) & 0.176 & 0.000 & 0.948 & 0.894 & 0.969 & 0.667 & 0.224 \\
        Rec (llama-405B) & 0.099 & 0.000 & 0.965 & 0.796 & 0.921 & 0.600 & 0.052 \\
        Rec (deepseek-r1) & 0.447 & 0.000 & 0.953 & 0.883 & 0.983 & 0.571 & 0.089 \\
        Rec (deepseek-v3) & 0.303 & 0.000 & 0.963 & 0.904 & 0.982 & 0.333 & 0.059 \\
        \bottomrule
    \end{tabular}
    \label{tab:target_strength_metrics}
\end{table}



\begin{table}[h]
    \centering
    \small
    \caption{Performance Comparison of Models Across Metrics and Targets (Weaknesses)}
    
    \begin{tabular}{lccccccc}
        \toprule
        Target & Problem & Prior Research & Method & Theory & Experiment & Conclusion & Paper \\
        \midrule
        F1 (gpt-4o-mini) & 0.253 & 0.153 & 0.715 & 0.430 & 0.849 & 0.088 & 0.222 \\
        F1 (gpt-4o) & 0.256 & 0.104 & 0.726 & 0.449 & 0.830 & 0.036 & \textbf{0.375} \\
        F1 (o1-mini) & 0.204 & 0.108 & \textbf{0.774} & 0.534 & \textbf{0.857} & 0.050 & 0.272 \\
        F1 (o1) & 0.164 & 0.089 & 0.760 & 0.508 & 0.835 & 0.000 & 0.120 \\
        F1 (llama-70B) & \textbf{0.294} & 0.098 & 0.672 & 0.400 & 0.802 & 0.103 & 0.118 \\
        F1 (llama-405B) & 0.155 & 0.062 & 0.625 & 0.399 & 0.809 & 0.000 & 0.190 \\
        F1 (deepseek-r1) & 0.198 & \textbf{0.163} & 0.709 & \textbf{0.510} & 0.852 & \textbf{0.154} & 0.169 \\
        F1 (deepseek-v3) & 0.151 & 0.103 & 0.696 & 0.387 & 0.850 & 0.105 & 0.099 \\
        \midrule
        Prec (gpt-4o-mini) & \textbf{0.320} & 0.268 & 0.672 & 0.347 & 0.755 & 0.056 & 0.283 \\
        Prec (gpt-4o) & 0.301 & 0.219 & 0.651 & 0.369 & 0.743 & 0.024 & 0.321 \\
        Prec (o1-mini) & 0.315 & 0.259 & 0.666 & 0.420 & 0.754 & 0.038 & 0.345 \\
        Prec (o1) & 0.273 & 0.127 & 0.672 & \textbf{0.440} & 0.752 & 0.000 & 0.317 \\
        Prec (llama-70B) & 0.274 & \textbf{0.286} & \textbf{0.687} & 0.315 & 0.744 & 0.062 & \textbf{0.400} \\
        Prec (llama-405B) & 0.228 & 0.143 & 0.673 & 0.300 & \textbf{0.758} & 0.000 & 0.304 \\
        Prec (deepseek-r1) & 0.287 & 0.197 & 0.661 & 0.365 & 0.750 & \textbf{0.167} & 0.323 \\
        Prec (deepseek-v3) & 0.212 & 0.200 & 0.669 & 0.284 & 0.750 & 0.083 & 0.345 \\
        \midrule
        Rec (gpt-4o-mini) & 0.209 & 0.107 & 0.764 & 0.563 & 0.970 & \textbf{0.214} & 0.183 \\
        Rec (gpt-4o) & 0.222 & 0.068 & 0.821 & 0.574 & 0.939 & 0.071 & \textbf{0.451} \\
        Rec (o1-mini) & 0.151 & 0.068 & \textbf{0.924} & 0.732 & \textbf{0.992} & 0.071 & 0.224 \\
        Rec (o1) & 0.118 & 0.068 & 0.874 & 0.600 & 0.939 & 0.000 & 0.074 \\
        Rec (llama-70B) & \textbf{0.316} & 0.059 & 0.658 & 0.547 & 0.869 & 0.286 & 0.069 \\
        Rec (llama-405B) & 0.118 & 0.040 & 0.583 & 0.593 & 0.867 & 0.000 & 0.138 \\
        Rec (deepseek-r1) & 0.151 & \textbf{0.139} & 0.764 & \textbf{0.847} & 0.984 & 0.143 & 0.115 \\
        Rec (deepseek-v3) & 0.118 & 0.069 & 0.725 & 0.605 & 0.980 & 0.143 & 0.057 \\
        \bottomrule
    \end{tabular}
    \label{tab:target_weakness_metrics}
\end{table}

\begin{table}[h]
    \centering
    \caption{Performance Comparison of Models Across Metrics and Aspects (Including both Strengths and Weaknesses)}
    \small
    \begin{tabular}{lcccc}
        \toprule
        Aspect & Novelty & Impact & Validity & Clarity \\
        \midrule
        F1 (gpt-4o-mini)   & 0.334 & 0.390 & \textbf{0.775} & 0.396 \\
        F1 (gpt-4o)        & 0.378 & \textbf{0.428} & 0.769 & 0.365 \\
        F1 (o1-mini)       & 0.386 & 0.427 & 0.773 & 0.395 \\
        F1 (o1)            & \textbf{0.404} & 0.399 & 0.772 & \textbf{0.401} \\
        F1 (llama-70B)     & 0.334 & 0.322 & 0.769 & 0.327 \\
        F1 (llama-405B)    & 0.337 & 0.318 & 0.772 & 0.278 \\
        F1 (deepseek-r1)   & 0.387 & 0.414 & \textbf{0.775} & 0.266 \\
        F1 (deepseek-v3)   & 0.346 & 0.422 & 0.768 & 0.187 \\
        \midrule
        Prec (gpt-4o-mini) & 0.367 & 0.291 & \textbf{0.671} & 0.317 \\
        Prec (gpt-4o)      & 0.474 & 0.313 & 0.668 & 0.298 \\
        Prec (o1-mini)     & 0.528 & 0.300 & 0.668 & 0.311 \\
        Prec (o1)          & 0.589 & 0.305 & 0.669 & 0.334 \\
        Prec (llama-70B)   & \textbf{0.665} & \textbf{0.318} & 0.667 & 0.337 \\
        Prec (llama-405B)  & 0.587 & 0.302 & \textbf{0.671} & 0.332 \\
        Prec (deepseek-r1) & 0.535 & 0.308 & 0.670 & \textbf{0.339} \\
        Prec (deepseek-v3) & 0.504 & 0.306 & 0.664 & 0.309 \\
        \midrule
        Rec (gpt-4o-mini)  & 0.460 & 0.600 & \textbf{0.990} & \textbf{0.549} \\
        Rec (gpt-4o)       & 0.506 & 0.689 & 0.975 & 0.485 \\
        Rec (o1-mini)      & \textbf{0.507} & \textbf{0.758} & \textbf{0.990} & 0.548 \\
        Rec (o1)           & 0.435 & 0.579 & 0.981 & 0.511 \\
        Rec (llama-70B)    & 0.450 & 0.371 & 0.981 & 0.346 \\
        Rec (llama-405B)   & 0.478 & 0.352 & 0.978 & 0.241 \\
        Rec (deepseek-r1)  & 0.502 & 0.632 & 0.988 & 0.219 \\
        Rec (deepseek-v3)  & 0.478 & 0.683 & 0.982 & 0.134 \\
        \bottomrule
    \end{tabular}
    \label{tab:aspect-all}
\end{table}



\begin{table}[h]
    \centering
    \caption{Performance Comparison of Models Across Metrics and Aspects (Strengths)}
    \small
    \begin{tabular}{lcccc}
        \toprule
        Aspect & Novelty & Impact & Validity & Clarity \\
        \midrule
        F1 (gpt-4o-mini)   & 0.643 & 0.474 & \textbf{0.599} & 0.309 \\
        F1 (gpt-4o)        & 0.654 & 0.520 & 0.593 & 0.202 \\
        F1 (o1-mini)       & 0.656 & \textbf{0.556} & 0.592 & 0.299 \\
        F1 (o1)            & 0.626 & 0.530 & 0.596 & \textbf{0.342} \\
        F1 (llama-70B)     & 0.636 & 0.411 & 0.593 & 0.292 \\
        F1 (llama-405B)    & \textbf{0.660} & 0.345 & 0.596 & 0.157 \\
        F1 (deepseek-r1)   & 0.655 & 0.536 & 0.598 & 0.170 \\
        F1 (deepseek-v3)   & \textbf{0.660} & 0.547 & 0.585 & 0.122 \\
        \midrule
        Prec (gpt-4o-mini) & 0.498 & 0.368 & \textbf{0.431} & 0.222 \\
        Prec (gpt-4o)      & 0.498 & 0.398 & 0.428 & 0.190 \\
        Prec (o1-mini)     & 0.501 & 0.403 & 0.424 & 0.224 \\
        Prec (o1)          & \textbf{0.530} & 0.412 & 0.430 & \textbf{0.261} \\
        Prec (llama-70B)   & 0.497 & \textbf{0.467} & 0.426 & 0.236 \\
        Prec (llama-405B)  & 0.506 & 0.368 & \textbf{0.431} & 0.215 \\
        Prec (deepseek-r1) & 0.503 & 0.400 & \textbf{0.431} & 0.224 \\
        Prec (deepseek-v3) & 0.509 & 0.403 & 0.419 & 0.207 \\
        \midrule
        Rec (gpt-4o-mini)  & 0.907 & 0.667 & \textbf{0.986} & \textbf{0.511} \\
        Rec (gpt-4o)       & \textbf{0.955} & 0.749 & 0.965 & 0.216 \\
        Rec (o1-mini)      & 0.949 & \textbf{0.897} & 0.979 & 0.449 \\
        Rec (o1)           & 0.763 & 0.744 & 0.969 & 0.496 \\
        Rec (llama-70B)    & 0.883 & 0.366 & 0.976 & 0.384 \\
        Rec (llama-405B)   & 0.949 & 0.324 & 0.969 & 0.123 \\
        Rec (deepseek-r1)  & 0.937 & 0.809 & 0.976 & 0.137 \\
        Rec (deepseek-v3)  & 0.940 & 0.851 & 0.965 & 0.086 \\
        \bottomrule
    \end{tabular}
    \label{fig:aspect-strength}

\end{table}


\begin{table}[h]
    \centering
    \small
    \caption{Performance Comparison of Models Across Metrics and Aspects (Weaknesses)}
    \begin{tabular}{lcccc}
        \toprule
        Aspect & Novelty & Impact & Validity & Clarity \\
        \midrule
        F1 (gpt-4o-mini)   & 0.024 & 0.306 & 0.951 & 0.484 \\
        F1 (gpt-4o)        & 0.103 & \textbf{0.335} & 0.945 & \textbf{0.528} \\
        F1 (o1-mini)       & 0.116 & 0.299 & \textbf{0.954} & 0.492 \\
        F1 (o1)            & \textbf{0.182} & 0.268 & 0.949 & 0.459 \\
        F1 (llama-70B)     & 0.032 & 0.233 & 0.945 & 0.362 \\
        F1 (llama-405B)    & 0.013 & 0.291 & 0.947 & 0.399 \\
        F1 (deepseek-r1)   & 0.120 & 0.292 & 0.952 & 0.362 \\
        F1 (deepseek-v3)   & 0.031 & 0.297 & 0.951 & 0.253 \\
        \midrule
        Prec (gpt-4o-mini) & 0.235 & 0.214 & \textbf{0.912} & 0.411 \\
        Prec (gpt-4o)      & 0.450 & 0.228 & 0.907 & 0.406 \\
        Prec (o1-mini)     & 0.556 & 0.197 & 0.911 & 0.397 \\
        Prec (o1)          & 0.647 & 0.198 & 0.908 & 0.406 \\
        Prec (llama-70B)   & \textbf{0.833} & 0.169 & 0.907 & 0.438 \\
        Prec (llama-405B)  & 0.667 & \textbf{0.236} & 0.911 & 0.450 \\
        Prec (deepseek-r1) & 0.568 & 0.215 & 0.908 & \textbf{0.454} \\
        Prec (deepseek-v3) & 0.500 & 0.209 & 0.908 & 0.410 \\
        \midrule
        Rec (gpt-4o-mini)  & 0.013 & 0.533 & 0.994 & 0.587 \\
        Rec (gpt-4o)       & 0.058 & \textbf{0.630} & 0.985 & \textbf{0.754} \\
        Rec (o1-mini)      & 0.065 & 0.619 & \textbf{1.000} & 0.646 \\
        Rec (o1)           & \textbf{0.106} & 0.415 & 0.994 & 0.527 \\
        Rec (llama-70B)    & 0.016 & 0.376 & 0.987 & 0.308 \\
        Rec (llama-405B)   & 0.006 & 0.381 & 0.987 & 0.359 \\
        Rec (deepseek-r1)  & 0.067 & 0.455 & \textbf{1.000} & 0.302 \\
        Rec (deepseek-v3)  & 0.016 & 0.515 & 0.998 & 0.183 \\
        \bottomrule
    \end{tabular}
    \label{fig:aspect-weakness}
\end{table}


\end{onecolumn}


\end{document}
