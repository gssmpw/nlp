%										بسم الله الرّحمن الرّحیم


\documentclass[12pt]{article}
\usepackage{arvin_1.2}
\renewcommand{\labelenumi}{\alph{enumi})} 		% for enum with abc
\renewcommand{\labelenumii}{ ({\roman{enumii}}) } 	% for roman enumii
\newcommand{\vecRep}{\mathrm{vector}}

%Title (change):
\def\C{\mathcal{C}}
\DeclareMathOperator{\wt}{wt}
\DeclareMathOperator{\dist}{dist}
\DeclareMathOperator{\amin}{argmin}
\DeclareMathOperator{\amax}{argmax}
\def\eps{\epsilon}
\addbibresource{main}
\usepackage{authblk}
\title{Unweighted Code Sparsifiers and Thin Subgraphs}
\author[1]{Shayan Oveis Gharan}
 \affil{\small University of Washington, \textsf{shayan@cs.washington.edu}}
\author[2]{Arvin Sahami}
\affil{\small University of British Columbia, \textsf{arvin52@student.ubc.ca}}
\date{February 4, 2025}
%%%%%%%%%%%%%%%%%%%%%%%%%%%%%%%%%%%%%%%

\AddToHookNext{shipout/background}{
 \begin{tikzpicture}[remember picture, overlay,inner sep=0pt,outer sep=0pt,opacity=1]
 \node[anchor=north east] at ([xshift=2.7cm,yshift=3cm]current page.north east) {
  \includegraphics[scale = 0.0175]{bismillah.jpg}
 };
 \end{tikzpicture}
}


%changing d->k
\newcommand{\cd}{k} %code dimension
\newcommand{\gk}{t} %changing graph connectivity symbol from k to t 

\begin{document}
\maketitle
\begin{abstract}
    We show that for every $\cd$-dimensional linear code $\C \subseteq \bF_2^n$ there exists a set $S\subseteq [n]$ of size at most $n/2+O(\sqrt{n\cd})$ such that the projection of $\C$ onto $S$ has distance at least $\frac12\dist(\C)$.
    As a consequence we show that any connected graph $G$ with $m$ edges and $n$ vertices has at least $2^{m-(n-1)}$ many $1/2$-thin subgraphs.
\end{abstract}

\section{Introduction}
Given a code $\C$ over $\mathbb{F}_2^n$, its distance is defined as
$$ \dist(\C) \coloneq \min_{c\in \C, c\neq 0} \wt(c),$$
where $\wt(c) = \norm{c}_0$ is the number of nonzero coordinates of $c$.
We say $\C$ is a {\em linear code} if for any $c,c'\in \C$, $c+c'\in \C$.
We say $\C$ is $\cd$-dimensional if it has $2^\cd$ codewords.
For convenience, we identify the set $S\subseteq [n]$ with its indicator vector $S \in \bF_2^n$ (thus $2^{[n]}$ is identified with $\bF_2^n$). 
\\

The projection of $\C$ onto $S\subseteq [n]$ is the code $\C_S=\{c_S: c\in \C\}$ obtained by taking the $S$ coordinates of every codeword in $\C$. Recently, Khanna-Putterman-Sudan \cite{KPS24}  initiated the study of code sparsification where they proved that any linear code $\C$ of dimension $\cd$ over $\mathbb{F}_2^n$ has a ``weighted'' code sparsifier which projects $\C$ onto at most $\widetilde{O}(\cd/\eps^2)$-coordinates and such that the weighted hamming weight of every code word is preserved (up to a $1\pm\eps$ multiplicative error). 

In our main theorem, we show that every linear code $\C\subseteq \mathbb{F}_2^n$ has {\em unweighted} sparsifier $\C_S$ such that $\abs{S}\approx n/2$ and $\dist(\C_S)\geq \frac12 \dist(\C)$. In other words, (assuming $\cd\ll n)$, we can increase the rate of the code by a factor of 2 while preserving almost the same distance-rate).
Our main theorem is the following.
\begin{thm}\label{thm:main}
    For any linear code $\C\subseteq \mathbb{F}_2^n$, there is a set $S\subseteq [n]$ of size $\leq n(\frac12 +\sqrt{\frac \cd {2.88n}})$ such that for any $c\in \C$, $\wt(c_S)\geq \wt(c)/2$. In particular we have $\dist(\C_S) \geq \dist(\C)/2$.
\end{thm}

We remark that the proof of the theorem is not algorithmic. 
We also briefly explain applications of this theorem to thin subgraphs. 
\\
Given an unweighted (undirected) graph $G=(V,E)$ with $n$ vertices and $m$ edges,
we say a set $T\subseteq E$ is a $\alpha$-thin w.r.t. $G$ if for any nonempty set $S\subsetneq V$,
$$ \abs{T(S,\overline{S})} \leq \alpha \abs{E(S,\overline{S})},$$
i.e., $T$ has at most $\alpha$-fraction of the edges of every cut. Recall that a graph $G$ is $\gk$-edge-connected if every cut in $G$ has at least $\gk$ edges.
The following thin tree conjecture is proposed by Goddyn two decades ago \cite{God04} and has been a subject of intense study since then \cite{AGMOS10,OS11,HO14,AO15,MP19,Mou19,Alg23, KO23}.
\begin{conj}[Thin Tree Conjecture]
    For any $\alpha<1$, there exists $\gk\geq 1$ such that any $\gk$-edge-connected graph $G$ has a spanning tree $T$ that is $\alpha$-thin. 
\end{conj}
We remark that there has been several "spectral" constructions of thin subsets but it remained an open problem whether one can construct thin subsets combinatorially without appealing to eigenvalue arguments (this is specially motivated to address the thin tree conjecture, since $\gk$-edge-connected graphs do not necessarily have spectrally thin trees, see \cite{AO15}).

\begin{cor}
    For any connected graph $G$ with $n$ vertices and $m$ edges, there exists a $\frac12$-thin set $T\subseteq E$ with $\abs{T}\geq m(\frac12-\sqrt{\frac{n-1}{2.88m}})$.
\end{cor}
\begin{proof}
The main observation is that if we identify every cut $(S,\overline{S})$ with the indicator vector of the set of edges of that cut, we obtain a linear code (over $\mathbb{F}_2^m$) of dimension $n-1$. 
Having said that to prove the statement it is enough to use the set $T\subseteq E$ promised in \cref{thm:main} which satisfies $\abs{T}\geq m(\frac12+\sqrt{\frac{n-1}{2.88m}})$; then $\overline{T}$ is a $\frac12$-thin subgraph of $G$ and has size $\abs{\overline{T}}\leq m(\frac12-\sqrt{\frac{n-1}{2.88m}})$. 
\end{proof}

The following stronger statement also follows from our proof.
\begin{thm}
    Any graph $G$ with $m$ edges and $n$ vertices has at least $2^{m-(n-1)}$ many $1/2$-thin subgraphs.
\end{thm}
The bound is tight, as a spanning tree only has a single $1/2$-thin subgraph, the empty-set.
\\
We remark that although the existence of $1/2$-thin subgraphs follows by spectral arguments such as \cite{BSS14}, we are not aware of any exponential lower-bound on the number of spectrally thin subgraphs.
\section{Main Proof}
\begin{defn}[codeword flip]
Given a code word $c\in \C$, the flip corresponding to $c$ is the map $\bF_2^n\to \bF_2^n$ sending $S\in \mathbb{F}_2^n$ to $S+c$. 

When $\C$ is linear, this defines an equivalence relation on subsets of $[n]$. We say $S \sim S'$ if $S$ can be obtained from $S'$ by a codeword flip, i.e., 
$S + S' \in \C$. The transitivity of this relation follows from the linearity of $\C$. 
\end{defn}

Note that the collection of equivalence classes is the quotient space $\bF_2^n/\C$ (this is a vector space over $\bF_2$). Since $\dim \bF_2^n/\C = n - \cd$ there are precisely $\abs{\bF_2^n/\C}=2^{n-\cd}$ equivalence classes. 

\begin{lemma}
    \label{thm:thinChar}
    Fix an equivalence class $H \in \bF_2^n/\C$. Let $S^*=\underset{S\in H}{\amax} \abs{S}$ 
    be a set with the largest size in $H$. Then for any $c\in \C$
    $$ \wt(c_{S^*})\geq \wt(c)/2.$$
    %\dist(\C_{S^*})\geq \frac12\dist(\C).$$
\end{lemma}
\begin{proof}
    We prove this by contradiction. Suppose that there exists a codeword $c\in \C$ such that 
    $$ \wt(c_{S^*}) < \wt(c)/2.$$
    Let $S =c+S^*$. By definition $S\in H$. But the above equation implies $\abs{S}>\abs{S^*}$ which is a contradiction.   
\end{proof}
For an equivalence class $H$, let $S_H \subset [n]$ be a set of the largest size in $H$.
\cref{thm:thinChar} implies that the collection $\set{S_H \colon H \in \bF_2^n/\C}$ contains $2^{n-\cd}$ distinct subsets $S \subset [n]$ s.t. $\dist(\C_S) \geq \frac12\dist(\C)$.

Now, it follows by the pigeon-hole principle that there must be a set $S\approx n/2$ such that $\C_S$ has distance at least half of the distance of $\C$.
\begin{proof}[Proof of  \cref{thm:main}]
Let 
$$ H^* =\underset{H \in \bF_2^n/\C}{\amin} \abs{S_H}$$
be the equivalence class whose largest set is the smallest. The set $S_{H^*}$ is the smallest among $2^{n-\cd}$ distinct sets. 
Hence by the following  lemma,  we must have $\abs{S_{H^*}}\leq n/2+\sqrt{n\cd/2.88}$. This finishes the proof. 
\end{proof}
\begin{lemma}
    For every $n$ the number of subsets of size $\geq n/2+\sqrt{n\cd/2.88}$ is less than $2^{n-\cd}$.
\end{lemma}
\begin{proof}
Let $X_1,\dots,\dots,X_n$ be independent Bernoulli random variable. Then, by Hoeffding's bound 
$$\P[X_1+\dots+X_n \geq n/2+\sqrt{\eps n}]\leq \exp(-2\eps).$$
Equivalently, the number of subsets of $[n]$ of size $\geq n/2+\sqrt{\eps n}$ is at most $2^n\cdot \exp(-2\eps)$. Letting $\exp(-2\eps)=2^{-\cd}$ we get $\eps=\frac{\ln 2}{2}\cd$.
\end{proof}

\printbibliography
\end{document}