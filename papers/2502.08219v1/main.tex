% \documentclass[letterpaper,twocolumn,10pt]{article}
% \usepackage{usenix,epsfig,endnotes}
% \documentclass[preprint,12pt,times,twocolumn]{elsarticle}
\documentclass[3p,times,twocolumn,number]{elsarticle}

% https://tex.stackexchange.com/a/511500
\makeatletter
\let\c@author\relax
\makeatother

\usepackage[numbers]{natbib}
\usepackage{blindtext}
\usepackage{paralist}
\usepackage{amssymb}
\usepackage{color}
\usepackage{graphicx}
\usepackage[acronym]{glossaries}
\loadglsentries{acronyms}
\usepackage[hidelinks]{hyperref}
%\usepackage{draftwatermark}
\usepackage{array}
\usepackage{booktabs}
\usepackage{cleveref}
\usepackage{pifont}
\usepackage{arydshln}
%\usepackage{endnotes}
\usepackage{orcidlink}
\usepackage{fancyvrb}
\usepackage{caption}
\usepackage{subcaption}
\usepackage{xurl}

% prevent single lines on new page
\clubpenalty = 10000
\widowpenalty = 10000
\displaywidowpenalty = 10000

% Capitalize autoref names and make subsection and subsubsections to Section
\usepackage[english]{babel}
\addto\extrasenglish{%
   \renewcommand{\chapterautorefname}{Chapter}
   \renewcommand{\sectionautorefname}{Section}
   \renewcommand{\subsectionautorefname}{Section}
   \renewcommand{\subsubsectionautorefname}{Section}
   \renewcommand{\tableautorefname}{Table}
   \renewcommand{\figureautorefname}{Figure}%
}

\setlength\dashlinedash{0.2pt}
\setlength\dashlinegap{1.5pt}
\setlength\arrayrulewidth{0.3pt}

\newcommand{\todo}[1]{{\color{red}\textbf{TODO:}{ #1}}}
\newcommand{\RQ}[0]{“How can problematic parts in the \gls{foss} ecosystem be identified?”}
\newcommand{\RQQ}[0]{“What is the current state of the \gls{foss} ecosystem?”}

% Contributions:
% This paper
% * highlights that the FLOSS ecosystem as a whole has issues with important and insufficiently maintained packages.
% * showcases that new legal requirements, such as the Cyber Resilience Act, require additional measures (e.g. generalized dependency tracking or standardized interfaces for large scale analysis) to protect the supply chain as a whole.
% * highlights that the current focus of supply chain analysis is focused on platforms such as Github while not all components use Github and are therefore suitable for analysis


\journal{Future Generation Computer Systems}

\begin{document}
\begin{frontmatter}

	\title{Tracking Down Software Cluster Bombs: A Current State Analysis of the Free/Libre and Open Source Software (FLOSS) Ecosystem}

	%% use optional labels to link authors explicitly to addresses:
	%% \author[label1,label2]{}
	%% \affiliation[label1]{organization={},
	%%             addressline={},
	%%             city={},
	%%             postcode={},
	%%             state={},
	%%             country={}}
	\author[1,2]{\orcidlink{0000-0002-2288-9010} Stefan Tatschner\corref{cor1}}
	\author[1,3,4]{\orcidlink{0000-0002-1094-4828} Michael P. Heinl}
	\author[2]{\orcidlink{0009-0008-0767-8208} Nicole Pappler}
	\author[1]{\orcidlink{0009-0001-7615-7579} Tobias Specht}
	\author[5]{\orcidlink{0000-0002-1658-1140} Sven Plaga}
	\author[2]{\orcidlink{0000-0002-3375-8200} Thomas Newe}

	\cortext[cor1]{Corresponding author}

	% Authors Begründung
	% https://www.dfg.de/resource/blob/173732/4166759430af8dc2256f0fa54e009f03/kodex-gwp-data.pdf
	% Leitlinie 14
	%
	% Stefan:
	% Entwicklung und Konzeption des Forschungsvorhabens
	% Erarbeitung, Erhebung, Beschaffung, Bereitstellung der Daten
	% Analyse/Auswertung oder Interpretation der Daten
	% Verfassen des Manuskripts
	%
	% Michael:
	% Analyse/Auswertung oder Interpretation der Daten
	% Schlussfolgerungen
	% Verfassen des Manuskripts

	% Nicole:
	% Entwicklung und Konzeption des Forschungsvorhabens
	% Verfassen des Manuskripts
	%
	% Tobias:
	% Analyse/Auswertung oder Interpretation der Daten
	%
	% Sven:
	% Supervision
	%
	% Tom:
	% Entwicklung und Konzeption des Forschungsvorhabens, Supervision


	\affiliation[1]{organization={Fraunhofer AISEC},
		city={Garching bei München},
		state={Bavaria},
		country={Germany}}
	\affiliation[2]{organization={University of Limerick},
		city={Limerick},
		addressline={V94 T9PX},
		country={Ireland}}
	\affiliation[3]{organization={Technical University of Munich},
		city={Garching bei München},
		state={Bavaria},
		country={Germany}}
	\affiliation[4]{organization={Munich University of Applied Sciences HM},
		city={Munich},
		state={Bavaria},
		country={Germany}}
	\affiliation[5]{organization={Center for Intelligence and Security Studies (CISS)},
		city={Neubiberg},
		state={Bavaria},
		country={Germany}}


	\begin{abstract}
		% More than once, computer history has shown that critical software vulnerabilities can have a large impact on affected components.
		% In the \gls{foss} ecosystem, most software is distributed via package repositories.
		% Nowadays, keeping track of critical dependencies in a software system becomes crucial for maintaining good security practices.
		% Especially due to new legal requirements such as the European Cyber Resilience Act, there is the need that software projects keep a transparent track record with \gls{sbom} and maintain a good health state.
		% This study summarizes the current state of available \gls{foss} package repositories and addresses the challenge of finding problematic spots in a software ecosystem.
		% These parts are analyzed in more detail, quantifying the health state of the \gls{foss} ecosystem.
		% The results show that there are well maintained projects in the \gls{foss} ecosystem but there are also projects with a high impact that are vulnerable to supply chain attacks.
		% This study proposes a method for a health state analysis and shows missing elements, e.g., interfaces, for future research.

		% BEGIN AI-IMPROVED
		Throughout computer history, it has been repeatedly demonstrated that critical software vulnerabilities can significantly affect the components involved.
		In the \gls{foss} ecosystem, most software is distributed through package repositories.
		Nowadays, monitoring critical dependencies in a software system is essential for maintaining robust security practices.
		This is particularly important due to new legal requirements, such as the European Cyber Resilience Act, which necessitate that software projects maintain a transparent track record with \gls{sbom} and ensure a good overall state.
		This study provides a summary of the current state of available \gls{foss} package repositories and addresses the challenge of identifying problematic areas within a software ecosystem.
		These areas are analyzed in detail, quantifying the current state of the \gls{foss} ecosystem.
		The results indicate that while there are well-maintained projects within the \gls{foss} ecosystem, there are also high-impact projects that are susceptible to supply chain attacks.
		This study proposes a method for analyzing the current state and identifies missing elements, such as interfaces, for future research.
		% END AI-IMPROVED
	\end{abstract}

	% \begin{graphicalabstract}
	% \end{graphicalabstract}

	% Required by this Journal.
	% 3 to 5 bullet points.
	\begin{highlights}
		\item Methodology for finding problematic parts in a software ecosystem
		\item Conducted an analysis of a Linux distribution's current state in terms of software dependencies
		\item Analysis of the practicability of new legal requirements
	\end{highlights}

	\begin{keyword}
		Dependency Graph, Vulnerability Databases, Centrality, Software Supply Chain Defects
	\end{keyword}

\end{frontmatter}

\section{Introduction}
\subsection{Motivation}

% BEGIN ORIGINAL
% In computer history, there are many examples where basic components of the software ecosystem suffered from critical vulnerabilities, such as \gls{rce} or Information Disclosure.
% It is in the nature of things that insecure shared components cause repercussions on every software component that depends on the vulnerable component.
% The (in)famous Heartbleed bug (CVE-2014-0160) was a software vulnerability that affected a huge amount of servers.
% This was due to the fact that the affected component, OpenSSL, forms the basis of most encrypted internet traffic.
% The Heartbleed vulnerability is still remembered to the present day.
% END ORIGINAL

% BEGIN AI-IMPROVED
Throughout computer history, there are numerous instances where fundamental components of the software ecosystem have been affected by critical vulnerabilities, such as \gls{rce} or Information Disclosure.
It is inherent that insecure shared components cause repercussions for every software component that relies on the vulnerable component.
The notorious Heartbleed bug (CVE-2014-0160) was a software vulnerability that impacted a vast number of servers.
This was because the affected component, OpenSSL, underpins most encrypted internet traffic.
The Heartbleed vulnerability is still remembered to this day.
% END AI-IMPROVED

% BEGIN ORIGINAL
% Many software components, such as e-mail, web, or database servers, require an implementation of \gls{tls}~\cite{rfc8446}.
% Nowadays, OpenSSL provides a maintained and up-to-date implementation of \gls{tls} and applications can choose to depend on OpenSSL instead of re-implementing the protocol.
% Consequently, a critical vulnerability in OpenSSL will most likely have an impact on the entire ecosystem.
% For instance, at the time of Heartbleed's disclosure, there were around 300k vulnerable servers online.
% Six years later, there were still 200k vulnerable servers online \cite{Morris2024}.
% Further vulnerabilities of shared basic components that also caused a noticeable media response were Ghost (CVE-2015-0235), Log4Shell (CVE-2021-33228), or the attempt to add a backdoor to the xz compression library (CVE-2024-3094).
% END ORIGINAL

% BEGIN AI-IMPROVED
Many software components, such as email, web, or database servers, require an implementation of \gls{tls}~\cite{rfc8446}.
Today, OpenSSL provides a maintained and current implementation of \gls{tls}, allowing applications to rely on OpenSSL rather than re-implementing the protocol.
As a result, a critical vulnerability in OpenSSL is likely to affect the entire ecosystem.
For example, at the time of Heartbleed's disclosure, approximately 300,000 vulnerable servers were online.
Six years later, about 200,000 vulnerable servers remained online \cite{Morris2024}.
Other vulnerabilities of shared basic components that also drew significant media attention include Ghost (CVE-2015-0235), Log4Shell (CVE-2021-33228), and the attempt to insert a backdoor into the xz compression library (CVE-2024-3094).
% END AI-IMPROVED

% BEGIN ORIGINAL
% Recently, there was an \gls{rce} vulnerability (CVE-2023-4863) in the widely used image decoding library \texttt{libwebp}.
% This library is used for decoding the WebP image format used by Google Chrome and Firefox.
% According to an analysis from Google, every user of \texttt{libwebp} was affected by the vulnerability.
% Consequently, this leads to the question “which software components or products are affected?”
% In other words, which software components depend on the vulnerable module and which software components could have repercussions on a large part of the software ecosystem?
% END ORIGINAL

% BEGIN AI-IMPROVED
Recently, an \gls{rce} vulnerability (CVE-2023-4863) was discovered in the widely used image decoding library \texttt{libwebp}.
This library is employed for decoding the WebP image format in Google Chrome and Firefox.
According to an analysis by Google, every user of \texttt{libwebp} was affected by this vulnerability.
This raises the question: “Which software components or products are affected?“
In other words, which software components depend on the vulnerable module, and which could have repercussions on a significant portion of the software ecosystem?
% END AI-IMPROVED

% BEGIN ORIGINAL
% From a technical point of view, sharing code in the form of libraries makes a lot of sense, especially when those libraries implement security-related features \cite{PLAGA2019596}.
% With this approach it is possible to combine efforts and avoid recurring problems or anti-patterns~\cite{heinl2020} by keeping relevant code paths at a single place.
% In practice, a diverse collection of libraries with different approaches is present.
% For instance, well-known \gls{p2p} applications share and expose their underlying networking techniques in order to be reused by other applications~\cite{s21154969}.
% In contrast to this, new technologies such as the QUIC protocol tend to be implemented multiple times until one implementation proves itself in practice~\cite{10.1145/3600160.3605164}.
% END ORIGINAL

% BEGIN AI-IMPROVED
From a technical standpoint, sharing code in the form of libraries is highly beneficial, particularly when these libraries implement security-related features \cite{PLAGA2019596}.
This approach allows for the combination of efforts and helps avoid recurring problems or anti-patterns~\cite{heinl2020} by centralizing relevant code paths.
In practice, there exists a diverse collection of libraries with different approaches.
For example, well-known \gls{p2p} applications share and expose their underlying networking techniques to be reused by other applications~\cite{s21154969}.
In contrast, new technologies such as the QUIC protocol often undergo multiple implementations until one proves itself effective in practice~\cite{10.1145/3600160.3605164}.
% END AI-IMPROVED

\subsection{Problem Statement}\label{sec:problemStatement}

% BEGIN ORIGINAL
% These different approaches lead to a situation in which a few key components fulfilling basic but important tasks are crucial for a lot of applications which have to be built on these key components, as there are often no alternatives available. 
% Not uncommonly, these key components are maintained by a few or even single developers who do this not even as part of their employment but rather during their free time. 
% Those components can cause severe impact to the overall ecosystem and should be maintained with extra care.
% END ORIGINAL

% BEGIN AI-IMPROVED
These various approaches result in a situation where a few key components, which perform basic yet important tasks, become essential for many applications, as there are often no alternatives available.
Often, these key components are maintained by a few or even a single developer, who may work on them during their free time rather than as part of their employment.
These components can significantly impact the overall ecosystem and should be maintained with particular care.
% END AI-IMPROVED


\subsection{Paradigmatic Vulnerabilities}\label{sec:paradigmaticVulns}

% BEGIN ORIGINAL
% The described situation has repeatedly caused severe vulnerabilities to occur in the past. 
% The following examples show that a critical vulnerability in a re-used library causes repercussions on many other components:

% \begin{itemize}
% 	\item \textbf{Heartbleed (CVE-2014-0160)}: This vulnerability in OpenSSL caused information leaks of sensitive data.
% 	      Before 2014, OpenSSL suffered from bad code quality which was presumably caused by project internal problems, such as missing testing or code reviews, due to insufficient founding.
% 	      Major software projects, such as nginx, postfix, or CPython depend on OpenSSL.
% 	\item \textbf{Shellshock (CVE-2014-6271)}: This vulnerability (family) in Bash caused privilege escalation and \gls{rce}.
% 	      According to its Git repository, Bash mostly seems to be maintained by a single person; its Git repository only contains \emph{two} different authors.
% 	      Due to Bash being a central component of most Linux systems, multiple services were affected by this vulnerability, for instance webservers based on \gls{cgi}, \gls{dhcp} clients, or OpenSSH.
% 	\item \textbf{Ghost (CVE-2015-0235)}: A buffer overflow bug affecting the \texttt{gethostbyname()} and \texttt{gethostbyname2()} function calls in the \texttt{glibc} library.
% 	      This vulnerability allows a remote attacker to execute arbitrary code with the permissions of the user running the application.
% 	      Since \texttt{glibc} is a very basic library which provides programming interfaces to communicate with the underlying operating system, it is literally used by almost all software modules.
% 	      Therefore, a large number of programs was affected by Ghost.
% 	\item \textbf{Log4Shell (CVE-2021-33228)}: This vulnerability had existed unnoticed in the Log4j logging framework since 2013. The vulnerability takes advantage of Log4j allowing requests to arbitrary \gls{ldap} and \gls{jndi} servers, allowing attackers to execute arbitrary Java code. The exploit is estimated to have had the potential to affect hundreds of millions of devices~\cite{log4j}.
% 	\item \textbf{The WebP 0day (CVE-2023-4863)}: With a specially crafted WebP lossless file, \texttt{libwebp} may write data out of bounds to the heap. Attacks against this vulnerability can range from \gls{dos} to possible \gls{rce}.
% 	      WebP is broadly applicable in web applications, hence largely browsers or e-mail clients, such as Google Chrome, Firefox, and Thunderbird were affected.
% 	\item \textbf{The xz Backdoor (CVE-2024-3094)}: The widely used xz compression library suffered from a supply chain attack trying to insert a master key for large scale root access.
% 	      A malicious actor gained trust and maintainer access to the source code of the project over time. Eventually, highly sophisticated backdoor code was added to the repository disguised as test files.
% 	      Via its legacy autotools-based build system, the malicious code was added to the resulting shared library\footnote{\url{https://research.swtch.com/xz-script}}.
% 	      The added code was used to overwrite the authentication routines of the SSH service at runtime by exploiting a GCC feature regarding dynamic linking.
% \end{itemize}
% END ORIGINAL

% BEGIN AI-IMPROVED
The described situation has repeatedly led to severe vulnerabilities in the past.
The following examples demonstrate that a critical vulnerability in a reused library can have repercussions on many other components:

\begin{itemize}
	\item \textbf{Heartbleed (CVE-2014-0160)}: This vulnerability in OpenSSL led to information leaks of sensitive data.
	      Before 2014, OpenSSL suffered from poor code quality, presumably due to internal project issues such as the absence of testing or code reviews, resulting from insufficient funding.
	      Major software projects, such as nginx, postfix, and CPython, depend on OpenSSL.
	\item \textbf{Shellshock (CVE-2014-6271)}: This vulnerability (family) in Bash caused privilege escalation and \gls{rce}.
	      According to its Git repository, Bash appears to be primarily maintained by a single person; its Git repository contains only \emph{two} different authors.
	      Because Bash is a central component of most Linux systems, multiple services were affected by this vulnerability, including web servers based on \gls{cgi}, \gls{dhcp} clients, and OpenSSH.
	\item \textbf{Ghost (CVE-2015-0235)}: A buffer overflow bug affecting the \texttt{gethostbyname()} and \texttt{gethostbyname2()} function calls in the \texttt{glibc} library.
	      This vulnerability allows a remote attacker to execute arbitrary code with the permissions of the user running the application.
	      Since \texttt{glibc} is a very basic library that provides programming interfaces to communicate with the underlying operating system, it is used by almost all software modules.
	      Consequently, a large number of programs were affected by Ghost.
	\item \textbf{Log4Shell (CVE-2021-33228)}: This vulnerability had existed unnoticed in the Log4j logging framework since 2013.
	      The vulnerability exploits Log4j's ability to allow requests to arbitrary \gls{ldap} and \gls{jndi} servers, enabling attackers to execute arbitrary Java code.
	      The exploit is estimated to have had the potential to affect hundreds of millions of devices~\cite{log4j}.
	\item \textbf{The WebP 0day (CVE-2023-4863)}: With a specially crafted WebP lossless file, \texttt{libwebp} may write data out of bounds to the heap.
	      Attacks against this vulnerability can range from \gls{dos} to possible \gls{rce}.
	      WebP is widely used in web applications, thus primarily affecting browsers or email clients such as Google Chrome, Firefox, and Thunderbird.
	\item \textbf{The xz Backdoor (CVE-2024-3094)}: The widely used xz compression library suffered from a supply chain attack attempting to insert a master key for large-scale root access.
	      A malicious actor gained trust and maintainer access to the project's source code over time.
	      Eventually, highly sophisticated backdoor code was added to the repository, disguised as test files.
	      Through its legacy autotools-based build system, the malicious code was incorporated into the resulting shared library\footnote{\url{https://research.swtch.com/xz-script}}.
	      The added code was used to overwrite the authentication routines of the SSH service at runtime by exploiting a GCC feature related to dynamic linking.
\end{itemize}
% END AI-IMPROVED

\subsection{Legal Regulations}

% BEGIN ORIGINAL
% When software vulnerabilities in third-party libraries are discovered, vendors that use these libraries in their products are pressurized to fix them as fast as possible, since they might be disclosed and exploited by attackers.
% Identifying and capturing detailed software composition information, including transitive dependencies, is therefore an essential tool to monitor risks of the software supply chain.
% END ORIGINAL

% BEGIN AI-IMPROVED
When software vulnerabilities in third-party libraries are discovered, vendors using these libraries in their products are pressured to fix them as quickly as possible, since they may be disclosed and exploited by attackers.
Identifying and capturing detailed software composition information, including transitive dependencies, is therefore an essential tool for monitoring risks in the software supply chain.
% END AI-IMPROVED

% BEGIN ORIGINAL
% ISO/IEC 5230/2020 \cite{ISO5230} in combination with ISO/IEC 18974:2023 \cite{ISO18974}, also known as OpenChain and OpenChain Security Assurance Specification, define structures and principles in order to stay in control of inbound and outbound software.
% Two main standards to create \glspl{sbom} have been established by now:
% END ORIGINAL

% BEGIN AI-IMPROVED
ISO/IEC 5230/2020 \cite{ISO5230}, in combination with ISO/IEC 18974:2023 \cite{ISO18974}, known as OpenChain and the OpenChain Security Assurance Specification, define structures and principles to maintain control over inbound and outbound software.
Two primary standards for creating \glspl{sbom} have been established so far:
% END AI-IMPROVED

\begin{itemize}
	% BEGIN ORIGINAL
	% \item \textbf{\gls{spdx}}\footnote{\url{https://spdx.dev/}}, a project initiated by the Linux Foundation that also became ISO/IEC 5962:2021 \cite{ISO5962} and
	% \item \textbf{CycloneDX}\footnote{\url{https://cyclonedx.org/}}, a format developed by the \gls{owasp} community.
	% END ORIGINAL

	% BEGIN AI-IMPROVED
	\item \textbf{\gls{spdx}}\footnote{\url{https://spdx.dev/}}, a project initiated by the Linux Foundation, which also became ISO/IEC 5962:2021 \cite{ISO5962}, and
	\item \textbf{CycloneDX}\footnote{\url{https://cyclonedx.org/}}, a format developed by the \gls{owasp} community.
	      % END AI-IMPROVED
\end{itemize}
%
% BEGIN ORIGINAL
% \glspl{sbom} created according to \gls{spdx} and CycloneDX both provide machine-readable formats allowing to analyze whether a particular software application is affected by a new known vulnerability.
% END ORIGINAL

% BEGIN AI-IMPROVED
\glspl{sbom} created according to \gls{spdx} and CycloneDX both provide machine-readable formats, allowing for the analysis of whether a particular software application is affected by a newly known vulnerability.
% END AI-IMPROVED

% BEGIN ORIGINAL
% In order to track security vulnerabilities, the \gls{cve} standard was published by the MITRE Corporation\footnote{\url{https://cve.mitre.org/}}, together with US government agencies.
% Although not directly required by any current regulation, the \gls{cve} system has been widely adopted by the cyber security community.
% It is widely used by tools or software distributors and became the de facto way to refer to software vulnerabilities.
% With security incidents becoming more frequent and more sophisticated, governments are starting to introduce regulatory requirements addressing supply chain issues.
% END ORIGINAL

% BEGIN AI-IMPROVED
In order to track security vulnerabilities, the \gls{cve} standard was published by the MITRE Corporation\footnote{\url{https://cve.mitre.org/}}, in collaboration with US government agencies.
Although not directly required by any current regulation, the \gls{cve} system has been widely adopted by the cybersecurity community.
It is widely used by tools and software distributors, becoming the de facto method for referring to software vulnerabilities.
As security incidents become more frequent and sophisticated, governments are starting to introduce regulatory requirements addressing supply chain issues.
% END AI-IMPROVED

% BEGIN ORIGINAL
% In the US, the government explicitly promotes \glspl{sbom} by executive order\footnote{\url{https://www.whitehouse.gov/briefing-room/presidential-actions/2021/05/12/executive-order-on-improving-the-nations-cybersecurity}}, the \gls{fda} requires the provision of \glspl{sbom} with all medical devices in its Medical Device Cybersecurity amendment to the \gls{fdc}\footnote{\url{https://www.fda.gov/media/119933/download}}, and the most recent Cybersecurity Security Framework\footnote{\url{https://www.nist.gov/cyberframework}} published by the \gls{nist} requires a Supply Chain Risk Management program including inventories of hardware and software.
% Japan recently published the latest version of its \gls{cip}\footnote{\url{https://www.nisc.go.jp/eng/pdf/cip_policy_2024_eng.pdf}} calling for risk management to address attacks originating from the supply chain.
% The European \gls{cra}\footnote{\url{https://www.europarl.europa.eu/doceo/document/TA-9-2024-0130_EN.html}}, adopted on March 12, 2024 by the European Parliament, is the latest addition to these national and international regulations, explicitly requesting \glspl{sbom}.
% END ORIGINAL

% BEGIN AI-IMPROVED
In the US, the government explicitly promotes \glspl{sbom} through an executive order\footnote{\url{https://www.whitehouse.gov/briefing-room/presidential-actions/2021/05/12/executive-order-on-improving-the-nations-cybersecurity}}.
The \gls{fda} requires the provision of \glspl{sbom} with all medical devices in its Medical Device Cybersecurity amendment to the \gls{fdc}\footnote{\url{https://www.fda.gov/media/119933/download}}, and the most recent Cybersecurity Framework\footnote{\url{https://www.nist.gov/cyberframework}} published by the \gls{nist} requires a Supply Chain Risk Management program including inventories of hardware and software.
Japan recently published the latest version of its \gls{cip}\footnote{\url{https://www.nisc.go.jp/eng/pdf/cip_policy_2024_eng.pdf}}, calling for risk management to address attacks originating from the supply chain.
The European \gls{cra}\footnote{\url{https://www.europarl.europa.eu/doceo/document/TA-9-2024-0130_EN.html}}, adopted on March 12, 2024 by the European Parliament, is the latest addition to these national and international regulations, explicitly requesting \glspl{sbom}.
% END AI-IMPROVED

% BEGIN ORIGINAL
% As the number of dependencies listed in \gls{sbom} documents can be very high, dealing with these dependencies will become a crucial part of establishing applicable measures in a cybersecurity governance process.
% END ORIGINAL

% BEGIN AI-IMPROVED
As the number of dependencies listed in \gls{sbom} documents can be very high, managing these dependencies will become a crucial part of establishing effective measures in a cybersecurity governance process.
% END AI-IMPROVED

\subsection{Research Questions}

% BEGIN ORIGINAL
% Based on the increasing legal requirements, this study aims to provide a high level view on the \gls{foss} ecosystem.
% The basic idea of \glspl{sbom} and the legal demand of tracking software dependencies might be a good idea in general.
% However, \emph{identifying} issues in the software supply chain via \glspl{sbom} does not address the root cause of the inherent problem.
% In order to \emph{avoid} such kind of issues, a first step is getting an idea of the health state of the \gls{foss} ecosystem.
% Such an overview will help to find parts in the supply chain where, e.g., financial support or human resources are required.
% For instance, the recent xz Backdoor could have potentially been avoided with better funding of the maintainer in the first place\footnote{\url{https://research.swtch.com/xz-timeline}}.
% These considerations lead to research question \textbf{RQ1} addressed by this paper: \RQ.
% Thinking this idea even further, a more generalized research question \textbf{RQ2} can be formulated: \RQQ.
% END ORIGINAL

% BEGIN AI-IMPROVED
Based on the increasing legal requirements, this study aims to provide a high-level view of the \gls{foss} ecosystem.
The basic concept of \glspl{sbom} and the legal demand for tracking software dependencies might generally be a good idea.
However, \emph{identifying} issues in the software supply chain via \glspl{sbom} does not address the root cause of the inherent problem.
To \emph{avoid} such issues, the first step is to understand the current state of the \gls{foss} ecosystem.
Such an overview will help identify parts of the supply chain where, for example, financial support or human resources are needed.
For instance, the recent xz Backdoor could potentially have been avoided with better funding for the maintainer from the outset\footnote{\url{https://research.swtch.com/xz-timeline}}.
These considerations lead to research question \textbf{RQ1} addressed by this paper: \RQ.
Extending this idea further, a more generalized research question \textbf{RQ2} can be formulated: \RQQ.
% END AI-IMPROVED

% BEGIN ORIGINAL
% Previous research showed that software ecosystems are similarly structured as social networks \cite{9631870,7490780}, hence relevant methods and possible data sources from this area are examined.
% The NixOS \cite{10.1145/1411203.1411255} ecosystem is of interest as it uses the Nix programming language to describe software packages and the relationship between them.
% Analyzing the NixOS package repository was considered suitable to create the dataset for this paper as it avoids error-prone parsing of package manifests.
% END ORIGINAL

% BEGIN AI-IMPROVED
Previous research has shown that software ecosystems are structured similarly to social networks \cite{9631870,7490780}, thus relevant methods and potential data sources from this area are examined.
The NixOS \cite{10.1145/1411203.1411255} ecosystem is of interest because it uses the Nix programming language to describe software packages and their relationships.
NixOS presents several compelling advantages that render it particularly suitable for scientific research.
Its extensive package repository surpasses that of Debian with over 80,000 packages.
Furthermore, NixOS's dependency management system, characterized by its functional package management, ensures precise and reproducible handling, and definition of dependencies.
Additionally, the monorepo structure of NixOS consolidates all package definitions into a single repository, thereby eliminating the need to crawl and parse multiple sources, which significantly streamlines the process of package analysis and data extraction.
These attributes collectively establish NixOS as an ideal platform for researchers seeking a reliable, comprehensive, and efficient system for conducting software-based scientific investigations.
% END AI-IMPROVED

% BEGIN ORIGINAL
% The presented paper proposes a method for finding problematic parts in the \gls{foss} ecosystem as described in \autoref{sec:problemStatement}.
% This method borrows from established techniques known from graph theory and related areas of studies.
% The developed method is applied to a real-world software package repository where metrics which are known from literature, such as project age or number of lines of code, are collected and evaluated.
% This paper contributes insights and related methods to identify critical projects.
% Based on the results on this paper and new legal requirements for dependency tracking there is potential for future research in this field.
% END ORIGINAL

% BEGIN AI-IMPROVED
The presented paper proposes a method for identifying problematic parts of the \gls{foss} ecosystem, as described in \autoref{sec:problemStatement}.
This method draws on established techniques from graph theory and related areas of study.
The developed method is applied to a real-world software package repository where metrics known from the literature, such as project age or the number of lines of code, are collected and evaluated.
This paper provides insights and related methods to identify critical projects.
Based on the results of this paper and new legal requirements for dependency tracking, there is potential for future research in this field.
% END AI-IMPROVED

\section{Related Work}

% BEGIN ORIGINAL
% The identified related work can be structured into three different categories.
% For each of these categories a literature research was conducted representing the theoretical background of the presented analysis in this paper.
% END ORIGINAL

% BEGIN AI-IMPROVED
The identified related work can be organized into three distinct categories.
For each category, a literature review was conducted to represent the theoretical background of the analysis presented in this paper.
% END AI-IMPROVED

% \begin{itemize}
% \item \textbf{Centrality in Graphs}: Covers the theoretical background of the chosen graph-based methods. The applicability of established evaluation methods known from social networks is researched.
% \item \textbf{Package Dependency Analysis}: Previous studies according to the structure of software ecosystems, such as the npm Registry, RubyGems.org, or the Debian project, are examined.
% \item \textbf{Project Status Analysis}: Techniques for quantifying the maintenance status of software projects are researched.
% \end{itemize}

% BEGIN AI-IMPROVED
\begin{itemize}
	\item \textbf{Centrality in Graphs}: This section covers the theoretical background of the selected graph-based methods. The research focuses on the applicability of established evaluation methods known from social networks.
	\item \textbf{Package Dependency Analysis}: This section examines previous studies related to the structure of software ecosystems like the npm Registry, RubyGems.org, or the Debian project.
	\item \textbf{Project Status Analysis}: Research is conducted on techniques for quantifying the maintenance status of software projects.
\end{itemize}
% END AI-IMPROVED

\subsection{Centrality in Graphs}

% BEGIN ORIGINAL
% In the context of network analysis, centrality measures offer several advantages over naive methods, such as only counting incoming edges.
% In contrast to those naive methods, Centrality provides a nuanced understanding of a node's importance based on its position in the network.
% By using centrality, insights into the overall network structure, such as identifying key influencers, potential bottlenecks, or vulnerable nodes can be gained.
% Overall, centrality provides a more comprehensive view of node importance within a network compared to just counting incoming edges.
% END ORIGINAL

% BEGIN AI-IMPROVED
In the context of network analysis, centrality measures offer several advantages over naive methods, such as merely counting incoming edges.
Unlike these naive approaches, centrality provides a nuanced understanding of a node's importance based on its position within the network.
By employing centrality, one can gain insights into the overall network structure, including identifying key influencers, potential bottlenecks, or vulnerable nodes.
Overall, centrality offers a more comprehensive view of node importance within a network compared to simply counting incoming edges.
% END AI-IMPROVED

% BEGIN ORIGINAL
% Centrality algorithms assign numbers or rankings to nodes within a graph corresponding to their network position.
% Applications include identifying the most influential persons in a social network, key infrastructure nodes in the internet, or analyzing urban networks.
% In general, centrality algorithms answer the question “What characterizes an important node?”.
% Since the word “importance” has a wide number of meanings, the available definitions of centrality are versatile \cite{Freeman1978}.
% END ORIGINAL

% BEGIN AI-IMPROVED
Centrality algorithms assign numbers or rankings to nodes within a graph based on their network position.
Applications include identifying the most influential people in a social network, key infrastructure nodes on the internet, or analyzing urban networks.
In general, centrality algorithms answer the question, “What characterizes an important node?”
The word “importance” can have many meanings, making the available definitions of centrality versatile \cite{Freeman1978}.
% END AI-IMPROVED

% BEGIN ORIGINAL
% Among the various centrality algorithms, the eigenvector centrality emerged as the most promising algorithm for this study, as demonstrated in 2019 by Gómez \cite{Gomez2019}.
% While eigenvector centrality focuses on the importance of connections to influential nodes, betweenness centrality highlights nodes that facilitate communication between others, and closeness centrality emphasizes nodes with efficient access to the entire network.
% Other centrality algorithms were deemed unsuitable for this study
% END ORIGINAL

% BEGIN AI-IMPROVED
Among the various centrality algorithms, eigenvector centrality emerged as the most promising algorithm for this study, as demonstrated in 2019 by Gómez \cite{Gomez2019}.
While eigenvector centrality focuses on the importance of connections to influential nodes, betweenness centrality highlights nodes that facilitate communication between others, and closeness centrality emphasizes nodes with efficient access to the entire network.
Other centrality algorithms were deemed unsuitable for this study.
% END AI-IMPROVED

% BEGIN ORIGINAL
% Historically, eigenvector centrality was introduced by Landau \cite{landau1895relativen} for chess tournaments.
% Half a century later, it was rediscovered by Wei \cite{wei1952algebraic} and popularized by Kendall \cite{760e07d1-fd0d-3ce0-afae-f7ab9cd57766} in the context of sport ranking.
% Claude introduced a general definition for graphs based on social connections \cite{claude1966theorie}.
% Eventually, Bonacich \cite{35397813-90c1-3806-8d5d-a07b3340ac3d} reintroduced eigenvector centrality again and made it popular in link analysis.
% END ORIGINAL

% BEGIN AI-IMPROVED
Historically, eigenvector centrality was introduced by Landau \cite{landau1895relativen} for chess tournaments.
Half a century later, it was rediscovered by Wei \cite{wei1952algebraic} and popularized by Kendall \cite{760e07d1-fd0d-3ce0-afae-f7ab9cd57766} in the context of sports ranking.
Claude introduced a general definition for graphs based on social connections \cite{claude1966theorie}.
Eventually, Bonacich \cite{35397813-90c1-3806-8d5d-a07b3340ac3d} reintroduced eigenvector centrality and made it popular in link analysis.
% END AI-IMPROVED

% BEGIN ORIGINAL
% Eigenvector centrality gives a measure of the influence of the node based on the connections of the nodes to which it is connected.
% Similar to degree centrality, eigenvector centrality favors nodes that have a high number of links.
% In contrast to degree centrality, eigenvector centrality also factors in the centrality of the adjacent node.
% END ORIGINAL

% BEGIN AI-IMPROVED
Eigenvector centrality measures the influence of a node based on the connections of the nodes to which it is connected.
Similar to degree centrality, eigenvector centrality favors nodes with a high number of links.
Unlike degree centrality, eigenvector centrality also considers the centrality of the adjacent nodes.
% END AI-IMPROVED

% BEGIN ORIGINAL
% Due to its mathematical foundation, eigenvector centrality requires strongly connected graphs\footnote{\url{https://ocw.mit.edu/courses/14-15-networks-spring-2022/mit14_15s22_lec3.pdf}}.
% In the context of a software repository's dependency graph for a Linux distribution, strong connectivity cannot always be assumed.
% However, a more general variant exists: the Katz centrality algorithm, introduced by Leo Katz in 1953 \cite{Katz1953}.
% Unlike eigenvector centrality, Katz centrality also applies to graphs that are not strongly connected.
% END ORIGINAL

% BEGIN AI-IMPROVED
Due to its mathematical foundation, eigenvector centrality requires strongly connected graphs\footnote{\url{https://ocw.mit.edu/courses/14-15-networks-spring-2022/mit14_15s22_lec3.pdf}}.
In the context of a software repository's dependency graph for a Linux distribution, strong connectivity cannot always be assumed.
However, a more general variant exists: the Katz centrality algorithm, introduced by Leo Katz in 1953 \cite{Katz1953}.
Unlike eigenvector centrality, Katz centrality also applies to graphs that are not strongly connected.
% END AI-IMPROVED

% BEGIN ORIGINAL
% Katz centrality is capable of assigning scores to nodes outside the largest connected component.
% It incorporates an attenuation factor to account for paths of varying lengths, ensuring non-zero scores for a broader range of nodes.
% By emphasizing immediate neighbors through a constant additive term, Katz centrality considers both direct and indirect connections.
% This robustness benefits nodes with fewer connections that remain influential due to their network positions.
% The attenuation factor provides flexibility, allowing for adjustments based on the network's unique characteristics.
% Overall, Katz centrality offers a versatile approach for evaluating disconnected networks, delivering meaningful scores across the graph.
% END ORIGINAL

% BEGIN AI-IMPROVED
Katz centrality is capable of assigning scores to nodes outside the largest connected component.
It incorporates an attenuation factor to account for paths of varying lengths, ensuring non-zero scores for a broader range of nodes.
By emphasizing immediate neighbors through a constant additive term, Katz centrality considers both direct and indirect connections.
This robustness benefits nodes with fewer connections that remain influential due to their network positions.
The attenuation factor provides flexibility, allowing for adjustments based on the network's unique characteristics.
Overall, Katz centrality offers a versatile approach for evaluating disconnected networks, delivering meaningful scores across the graph.
% END AI-IMPROVED

\subsection{Package Dependency Analysis}

% BEGIN ORIGINAL
% In 2015, Wang et al. published a study using a graph-based method to create a distribution wide dependency analysis for Ubuntu~14.04 \cite{7490780}.
% The authors present the challenges with creating a dependency graph by parsing package metadata from package manager such as Debian's \gls{apt}.
% This work illustrates that a graph-based approach is efficient for understanding the software structure of a whole distribution and it can assist in further more detailed investigations.
% END ORIGINAL

% BEGIN AI-IMPROVED
In 2015, Wang et al. published a study using a graph-based method to create a distribution-wide dependency analysis for Ubuntu~14.04 \cite{7490780}.
The authors present the challenges of creating a dependency graph by parsing package metadata from package managers such as Debian's \gls{apt}.
This work illustrates that a graph-based approach is efficient for understanding the software structure of an entire distribution and can assist in further, more detailed investigations.
% END AI-IMPROVED

% BEGIN ORIGINAL
% In 2017, Decan, Mens, and Claes published a comparison of dependency issues in \gls{foss} packaging ecosystems \cite{7884604}.
% The authors presented an empirical analysis of the evolution of dependency graphs of three large package ecosystems.
% The paper highlights solutions each package ecosystem has put into place for dependency update issues, such as dependency constraints.
% The authors conclude that package dependency updates have a non-negligible maintenance cost and that better packaging and dependency analysis tools are required.
% END ORIGINAL

% BEGIN AI-IMPROVED
In 2017, Decan, Mens, and Claes published a comparison of dependency issues in \gls{foss} packaging ecosystems \cite{7884604}.
The authors presented an empirical analysis of the evolution of dependency graphs of three large package ecosystems.
The paper highlights solutions each package ecosystem has implemented for dependency update issues, such as dependency constraints.
The authors conclude that package dependency updates entail a non-negligible maintenance cost and that better packaging and dependency analysis tools are needed.
% END AI-IMPROVED

% BEGIN ORIGINAL
% In 2018, Decan, Mens, and Grosjoen published a more detailed study about the evolution in software packaging ecosystems \cite{Decan2019}.
% The authors state that the majority of packages depend on other packages, but only a small proportion of packages account for most reverse dependencies.
% According to the study, there is a high proportion of so called “fragile” packages, due to a high and increasing number of transitive dependencies over time.
% The study concludes that these findings are instrumental for assessing the quality of a package dependency networks and that it can be improved through more comprehensive dependency management tools and imposed policies.
% END ORIGINAL

% BEGIN AI-IMPROVED
In 2018, Decan, Mens, and Grosjoen published a more detailed study on the evolution of software packaging ecosystems \cite{Decan2019}.
The authors state that most packages depend on other packages, but only a small proportion of packages account for most reverse dependencies.
According to the study, there is a high proportion of so-called “fragile” packages due to a high and increasing number of transitive dependencies over time.
The study concludes that these findings are instrumental for assessing the quality of package dependency networks and that improvements can be made through more comprehensive dependency management tools and imposed policies.
% END AI-IMPROVED

% BEGIN ORIGINAL
% In 2021, Suhaib Mujahid et al. published a study evaluating a centrality-based approach which could detect packages in the npm Registry that are in decline~\cite{9631870}.
% The authors conclude that it is possible to predict that packages in the npm ecosystem will soon become deprecated.
% The article's key point is an analysis of the popular npm package Moment.js. The authors published a chart that shows a declining centrality value since September 2018.
% Two years later the package was considered deprecated by its developers.
% Not until that point in time, the number of packages that depended on Moment.js started to drop.
% END ORIGINAL

% BEGIN AI-IMPROVED
In 2021, Suhaib Mujahid et al. published a study evaluating a centrality-based approach that could detect packages in the npm Registry that are in decline~\cite{9631870}.
The authors conclude that it is possible to predict when packages in the npm ecosystem will soon become deprecated.
The article's key point is an analysis of the popular npm package Moment.js. The authors published a chart showing a declining centrality value since September 2018.
Two years later, the package was considered deprecated by its developers.
It was not until that point in time that the number of packages depending on Moment.js began to drop.
% END AI-IMPROVED

% BEGIN ORIGINAL
% In 2025, Alhamdan and Staicu published a study analyzing the Deno ecosystem \cite{Alhamdan:Staicu:2025}.
% The authors state that even Deno has a smaller attack surface than Node.js, several attacks are not or only partially addressed.
% The paper also highlights that classical URL-related issues such as expired domains or the reliance on insecure transport protocols are still relevant.
% The authors propose the following to improve the security model of the Deno ecosystem: add import permissions, additional access control at file system level, support for compartmentalization, and a manifest file that persists fine-grained permissions.
% END ORIGINAL

% BEGIN AI-IMPROVED
In 2025, Alhamdan and Staicu published a study analyzing the Deno ecosystem \cite{Alhamdan:Staicu:2025}.
The authors state that although Deno has a smaller attack surface than Node.js, several attacks are not addressed or only partially addressed.
The paper also highlights that classical URL-related issues, such as expired domains or reliance on insecure transport protocols, remain relevant.
The authors propose the following improvements to the security model of the Deno ecosystem: add import permissions, additional access control at the file system level, support for compartmentalization, and a manifest file that persists fine-grained permissions.
% END AI-IMPROVED

\subsection{Project Status Analysis}

% BEGIN ORIGINAL
% Unmaintained projects are still a serious problem as also shown by \cite{236368}.
% In 2020, Jailton Coelho et al. published a study \cite{COELHO2020106274} proposing an approach to identify GitHub projects that are not actively maintained.
% The authors introduced the so-called \gls{lma} value as a metric describing the maintenance status of Github projects.
% The authors trained a machine learning model to identify unmaintained or sparsely maintained projects, based on a set of features, for instance the number of commits, forks, or issues.
% The approach was published as an extension for Google Chrome.
% Ironically, this extension is not actively maintained anymore.
% Further, the published extension does not contain the trained model but instead is intended to communicate with a server provided by the authors.
% This server is currently offline and therefore the developed approach cannot be used for this paper.
% However, the authors conducted several correlation analyses.
% The authors state that, e.g., the number of contributors or lines of code can be used to assess the maintenance status of a software project, since those are correlated with their \gls{lma} value.
% END ORIGINAL

% BEGIN AI-IMPROVED
Unmaintained projects continue to pose a serious problem, as also shown by \cite{236368}.
In 2020, Jailton Coelho et al. published a study \cite{COELHO2020106274} proposing a method to identify GitHub projects that are not actively maintained.
The authors introduced the so-called \gls{lma} value as a metric for describing the maintenance status of GitHub projects.
They trained a machine learning model to identify unmaintained or sparsely maintained projects based on a set of features, such as the number of commits, forks, or issues.
The approach was released as an extension for Google Chrome.
Ironically, this extension is no longer actively maintained.
Furthermore, the published extension does not contain the trained model but instead is intended to communicate with a server provided by the authors.
This server is currently offline, and therefore the developed approach cannot be used for this paper.
However, the authors conducted several correlation analyses.
They state that factors such as the number of contributors or lines of code can be used to assess the maintenance status of a software project, as these are correlated with their \gls{lma} value.
% END AI-IMPROVED

% BEGIN ORIGINAL
% In 2020, Rob Pike published a technical article \cite{pike2020} describing the criticality score which is a technique for quantifying criticality.
% In 2021, this technique was further examined by Pfeiffer \cite{pfeiffer}.
% The goal of this score is to find a single value that represents all signals of criticality for a package in a meaningful way.
% The \gls{ossf} group, operated by the Linux Foundation, maintains multiple software projects on Github\footnote{\url{https://github.com/ossf}} which can be used to, e.g., calculate the criticality score of Github projects or surveys about the state of software ecosystems.
% However, their tools are only applicable for a limited scope.
% For instance, the official criticality score tool\footnote{\url{https://github.com/ossf/criticality_score}} is only applicable to Github projects and a Google Cloud account is required to use the tool.
% END ORIGINAL

% BEGIN AI-IMPROVED
In 2020, Rob Pike published a technical article \cite{pike2020} describing the criticality score, a technique for quantifying criticality.
In 2021, this technique was further examined by Pfeiffer \cite{pfeiffer}.
The goal of this score is to find a single value that meaningfully represents all signals of criticality for a package.
The \gls{ossf} group, operated by the Linux Foundation, maintains multiple software projects on GitHub\footnote{\url{https://github.com/ossf}} that can be used to calculate the criticality score of GitHub projects or conduct surveys about the state of software ecosystems.
However, their tools are only applicable within a limited scope.
For instance, the official criticality score tool\footnote{\url{https://github.com/ossf/criticality_score}} is only applicable to GitHub projects, and a Google Cloud account is required to use the tool.
% END AI-IMPROVED

\section{Methodology}
\label{sec:methodology}

% BEGIN ORIGINAL
% This research consists of multiple, consecutive working steps.
% The results were collected in separate tables including additional metadata, e.g., the location of Git repositories or further references.
% As these tables form the database for the evaluation, they will be published as additional data to this paper.
% From a conceptional point of view, the methodology is structured as follows:
% END ORIGINAL

% BEGIN AI-IMPROVED
This research consists of multiple, consecutive working steps.
The results were collected in separate tables, including additional metadata such as the location of Git repositories or further references.
As these tables form the database for the evaluation, they will be published as supplementary data to this paper.
From a conceptual point of view, the methodology is structured as follows:
% END AI-IMPROVED

% BEGIN ORIGINAL
% \begin{enumerate}
% 	\item \textbf{Identification of relevant packages}: Relevant packages are identified by a graph-based approach. The nixpkgs\footnote{\url{https://github.com/NixOS/nixpkgs}} repository was used to create a dependency graph of all (at the time of writing) 82,011 software packages.
% 	      In contrast to similar repositories, e.g., npm, PyPi, or crates.io, the nixpkgs repository additionally contains information about system dependencies.
% 	      Analyzing alternative software repositories would yield ecosystem specific results and exclude system libraries such as libssl (provided by OpenSSL) or libcurl (provided by the curl project) from the analysis.
% 	      The authors considered the nixpkgs repository suitable for this study, since no technical limitations for creating a full dependency graph including system dependencies exist.
% 
% 	      The nixpkgs repository is built upon the special nix language which allows building a large graph data structure without the need to parse text-based metadata.
% 	      Subsequently, techniques developed in the field of social network analysis, also known as \emph{indicators of centrality}, were used to determine the 200 most important nodes in the dependency graph for a subsequent manual analysis.
% 	      Some packages were not relevant for further examination, as they, e.g., contained only documentation.
% 	      35 packages were filtered out after manual review.
% 	      The graph library NetworkX \cite{SciPyProceedings_11} was used for graph processing.
% 
% 	\item \textbf{Identification of relevant vulnerability databases}: The NixOS project does not maintain a vulnerability tracker where actual \glspl{cve} are mapped to real package names and their state is tracked. For this study, the Debian vulnerability database\footnote{\url{https://security-tracker.debian.org/tracker/}} was used as a data source for CVE information. Other large databases, such as OSV\footnote{\url{https://osv.dev/}} which is maintained by Google \cite{10.1145/3597503.3639582}, were considered unsuitable as there is no mapping of CVEs to actual software packages available.
% 	\item \textbf{Addition of missing data}: The identified packages were reviewed and missing data, such as the location of the Git repository, category, license, or the implementation language, were added manually to the table.
% 	\item \textbf{Collection of metrics}: The Git repository of every identified software package was cloned and examined. Multiple values were extracted such as the number of contributors, the age, or the commit activity. These values were added to the table as well.
% 	\item \textbf{Collection of \glspl{cve}}: The package ids from the nixpkgs repository were manually mapped to the corresponding package ids in the Debian repository. The Debian security database was downloaded and the \gls{cve} stats were mapped to the relevant nixpkgs packages.
% 	\item \textbf{Analysis of gathered data}: The data was evaluated and visualized using well-known methods from data science, such as scatter plots, bar graphs, or pie charts.
% \end{enumerate}
% END ORIGINAL

% BEGIN AI-IMPROVED
\begin{enumerate}
	\item \textbf{Identification of relevant packages}: Relevant packages were identified using a graph-based approach. The nixpkgs\footnote{\url{https://github.com/NixOS/nixpkgs}} repository was used to create a dependency graph of all (at the time of writing) 82,011 software packages.
	      Unlike similar repositories, e.g., npm, PyPi, or crates.io, the nixpkgs repository also contains information about system dependencies.
	      Analyzing alternative software repositories would yield ecosystem-specific results and exclude system libraries such as libssl (provided by OpenSSL) or libcurl (provided by the curl project) from the analysis.
	      The authors considered the nixpkgs repository suitable for this study since no technical limitations exist for creating a full dependency graph including system dependencies.

	      The nixpkgs repository is built on the special nix language, which allows building a large graph data structure without needing to parse text-based metadata.
	      Subsequently, techniques developed in the field of social network analysis, also known as \emph{indicators of centrality}, were used to determine the 200 most important nodes in the dependency graph for subsequent manual analysis.
	      Some packages were not relevant for further examination, as they, e.g., contained only documentation.
	      After manual review, 35 packages were filtered out.
	      The graph library NetworkX \cite{SciPyProceedings_11} was used for graph processing.

	\item \textbf{Identification of relevant vulnerability databases}: The NixOS project does not maintain a vulnerability tracker where actual \glspl{cve} are mapped to real package names and their state is tracked. For this study, the Debian vulnerability database\footnote{\url{https://security-tracker.debian.org/tracker/}} was used as a data source for CVE information. Other large databases, such as OSV\footnote{\url{https://osv.dev/}} maintained by Google \cite{10.1145/3597503.3639582}, were considered unsuitable as there is no mapping of CVEs to actual software packages available.
	\item \textbf{Addition of missing data}: The identified packages were reviewed, and missing data, such as the location of the Git repository, category, license, or implementation language, were added manually to the table.
	\item \textbf{Collection of metrics}: The Git repository of every identified software package was cloned and examined. Multiple values were extracted, such as the number of contributors, age, or commit activity. These values were added to the table as well.
	\item \textbf{Collection of \glspl{cve}}: The package IDs from the nixpkgs repository were manually mapped to the corresponding package IDs in the Debian repository. The Debian security database was downloaded, and the \gls{cve} stats were mapped to the relevant nixpkgs packages.
	\item \textbf{Analysis of gathered data}: The data was evaluated and visualized using well-known methods from data science, such as scatter plots, bar graphs, or pie charts.
\end{enumerate}
% END AI-IMPROVED

\section{Data Gathering}

\subsection{Determination of Relevant Packages}

% BEGIN ORIGINAL
% In graph theory and network analysis, indicators of centrality assign numbers or rankings to nodes within a graph corresponding to their network position.
% Applications include identifying the most influential nodes in social networks, computer networks, or urban networks.
% Centrality is even applicable to finding super-spreaders of diseases.
% A high Katz centrality score indicates a strong influence over other nodes in the network.
% It is useful because it indicates not just direct influence, but also implies influence over nodes more than one hop away.
% END ORIGINAL

% BEGIN AI-IMPROVED
In graph theory and network analysis, indicators of centrality assign numbers or rankings to nodes within a graph based on their network position.
Applications include identifying the most influential nodes in social networks, computer networks, or urban networks.
Centrality is even applicable in identifying super-spreaders of diseases.
A high Katz centrality score indicates strong influence over other nodes in the network.
It is useful because it signifies not just direct influence, but also influence over nodes more than one hop away.
% END AI-IMPROVED

% BEGIN ORIGINAL
% From the literature research, the Katz centrality was found to be applicable to a dependency graph data structure describing software dependencies.
% Practically, the centrality algorithm assigns each node a numerical score value.
% Finally, this score can be used to sort the nodes (= the software packages) according to their importance in the software ecosystem.
% END ORIGINAL

% BEGIN AI-IMPROVED
From the literature research, Katz centrality was found to be applicable to a dependency graph data structure describing software dependencies.
Practically, the centrality algorithm assigns each node a numerical score.
Finally, this score can be used to sort the nodes (i.e., the software packages) according to their importance in the software ecosystem.
% END AI-IMPROVED

% BEGIN ORIGINAL
% \begin{figure}[htb]
% 	\centering
% 	\includegraphics[width=0.8\linewidth]{img/katz-centrality.png}
% 	\caption{A random graph data structure with highlighted Katz centrality values.\protect\footnotemark~The more red a node is, the higher the centrality.}
% 	\label{fig:centrality}
% \end{figure}\footnotetext{\url{https://upload.wikimedia.org/wikipedia/commons/9/9e/Wp-01.png}\label{footnote:wiki}}
% END ORIGINAL

% BEGIN AI-IMPROVED
\begin{figure}[htb]
	\centering
	\includegraphics[width=0.8\linewidth]{img/katz-centrality.png}
	\caption{A random graph data structure with highlighted Katz centrality values.\protect\footnotemark~The more red a node is, the higher the centrality.}
	\label{fig:centrality}
\end{figure}\footnotetext{\url{https://upload.wikimedia.org/wikipedia/commons/9/9e/Wp-01.png}\label{footnote:wiki}}
% END AI-IMPROVED

% BEGIN ORIGINAL
% \autoref{fig:centrality} shows an example of a random graph structure where the Katz centrality was used to assign each node an according score.
% This method has some limitations, for instance, a Linux system most likely has not every available package installed.
% Calculating the dependency graph and the centrality scores for such a system might cause different values.
% In this analysis, only the package repository with all nodes in their default configuration are considered.
% No special settings that, e.g., exclude particular libraries were considered.
% END ORIGINAL

% BEGIN AI-IMPROVED
\autoref{fig:centrality} shows an example of a random graph structure where Katz centrality was used to assign each node an appropriate score.
This method has some limitations; for instance, a Linux system likely does not have every available package installed.
Calculating the dependency graph and centrality scores for such a system might yield different values.
In this analysis, only the package repository with all nodes in their default configuration is considered.
No special settings that, for example, exclude particular libraries, were considered.
% END AI-IMPROVED

% BEGIN ORIGINAL
% In order to create a dependency graph for a whole Linux distribution, the NixOS sofware repository, called nixpkgs, was chosen.
% The advantage of this software repository compared to software repositories like Debian or Fedora is its special design: each package is described as a function in the nix programming language.
% The nix language is a functional language which particularly designed to be used for software packaging.
% From a technical point of view, the whole nixpkgs repository is considered a large program.
% Therefore, there is no need to crawl packages from the relevant repository or parse the embedded software package metadata.
% Using the nix language, this information can be used directly with minimal sources of error.
% END ORIGINAL

% BEGIN AI-IMPROVED
In order to create a dependency graph for an entire Linux distribution, the NixOS software repository, called nixpkgs, was chosen.
The advantage of this software repository compared to others like Debian or Fedora is its special design: each package is described as a function in the nix programming language.
The nix language is a functional language specifically designed for software packaging.
From a technical perspective, the entire nixpkgs repository is considered a large program.
Therefore, there is no need to crawl packages from the relevant repository or parse the embedded software package metadata.
Using the nix language, this information can be accessed directly with minimal sources of error.
% END AI-IMPROVED

% BEGIN ORIGINAL
% To summarize the approach, a program in the nix language was written which traverses all software packages and its dependencies.
% The traversed graph was then exported as a \gls{json} object and imported into a Python program for further analysis with the NetworkX library.
% The NetworkX library was finally used to calculate the Katz centrality and sort the dependency graph.
% The first 200 nodes of the sorted graph were considered for further analysis.
% END ORIGINAL

% BEGIN AI-IMPROVED
To summarize the approach, a program in the nix language was written to traverse all software packages and their dependencies.
The traversed graph was then exported as a \gls{json} object and imported into a Python program for further analysis with the NetworkX library.
The NetworkX library was used to calculate the Katz centrality and sort the dependency graph.
The first 200 nodes of the sorted graph were considered for further analysis.
% END AI-IMPROVED

\begin{figure}[htb]
	\centering
	\includegraphics[width=1\linewidth]{img/gallia.eps}
	\caption{A subgraph of dependencies of the gallia \cite{Tatschner_gallia} package.}
	\label{fig:gallia-deps}
\end{figure}

% BEGIN ORIGINAL
% A subgraph which is extracted from the created graph database is shown in \autoref{fig:gallia-deps}.
% A software package is represented by a graph node which can have edges labeled with \texttt{DEPENDS\_ON} to other nodes.
% Not shown in the figure are properties which can be assigned to nodes, such as the package name, software version, or used licenses.
% END ORIGINAL

% BEGIN AI-IMPROVED
A subgraph extracted from the created graph database is shown in \autoref{fig:gallia-deps}.
A software package is represented by a graph node, which can have edges labeled with \texttt{DEPENDS\_ON} to other nodes.
Not shown in the figure are properties that can be assigned to nodes, such as the package name, software version, or used licenses.
% END AI-IMPROVED

% BEGIN ORIGINAL
% The canonical software repositories only include information for libraries in the language specific to the ecosystem.
% For instance, the \gls{pypi} only provides information for Python dependencies.
% System libraries like OpenSSL are not included in the dependency manifests provided by \gls{pypi}.
% The created dependency graph in this study can even be queried for, e.g., system libraries that are required by Python libraries.
% END ORIGINAL

% BEGIN AI-IMPROVED
The canonical software repositories only include information for libraries in the language specific to the ecosystem.
For instance, the \gls{pypi} only provides information for Python dependencies.
System libraries like OpenSSL are not included in the dependency manifests provided by \gls{pypi}.
The created dependency graph in this study can even be queried for, e.g., system libraries that are required by Python libraries.
% END AI-IMPROVED


\subsection{Determination of Relevant Vulnerability Databases}
\label{sec:vulndbs}

% BEGIN ORIGINAL
% A vulnerability database is a structured collection of information about security vulnerabilities in software and hardware systems, detailing aspects such as vulnerability ID, description, severity, and mitigation strategies. Examples include the \gls{nvd}\footnote{\url{https://nvd.nist.gov/}}, which is a U.S. government repository of standards-based vulnerability data or the \gls{osv}\footnote{\url{https://osv.dev/}}, focusing on vulnerabilities in open-source projects.
% These resources aid organizations in identifying and addressing security weaknesses.
% END ORIGINAL

% BEGIN AI-IMPROVED
A vulnerability database is a structured collection of information about security vulnerabilities in software and hardware systems, detailing aspects such as vulnerability ID, description, severity, and mitigation strategies. Examples include the \gls{nvd}\footnote{\url{https://nvd.nist.gov/}}, which is a U.S. government repository of standards-based vulnerability data, or the \gls{osv}\footnote{\url{https://osv.dev/}}, which focuses on vulnerabilities in open-source projects. These resources help organizations identify and address security weaknesses.
% END AI-IMPROVED

% BEGIN ORIGINAL
% Multiple definitions for computer security vulnerabilities were published, for instance:
%
% \begin{itemize}
% 	\item \textbf{RFC4949} \cite{rfc4949}: “A flaw or weakness in a system's design, implementation, or operation and management that could be exploited to violate the system's security policy.”
% 	\item \textbf{ISO 27005} \cite{iso27005}: “A weakness of an asset or group of assets that can be exploited by one or more threats, where an asset is anything that has value to the organization, its business operations, and their continuity, including information resources that support the organization's mission.”
% 	\item \textbf{Committee on National Security Systems (CNSS) Glossary}\footnote{{\url{https://www.niap-ccevs.org/Ref/CNSSI_4009.pdf}}}: “A known weakness in a system, system security procedures, internal controls, or implementation by which an actor or event may intentionally exploit or accidentally trigger the weakness to access, modify, or disrupt normal operations of a system-resulting in a security incident or a violation of the system's security policy.”
% \end{itemize}
% END ORIGINAL

% BEGIN AI-IMPROVED
Various definitions of computer security vulnerabilities have been published, such as:

\begin{itemize}
	\item \textbf{RFC4949} \cite{rfc4949}: “A flaw or weakness in a system's design, implementation, or operation and management that could be exploited to violate the system's security policy.”
	\item \textbf{ISO 27005} \cite{iso27005}: “A weakness of an asset or group of assets that can be exploited by one or more threats, where an asset is anything that has value to the organization, its business operations, and their continuity, including information resources that support the organization's mission.”
	\item \textbf{Committee on National Security Systems (CNSS) Glossary}\footnote{{\url{https://www.niap-ccevs.org/Ref/CNSSI_4009.pdf}}}: “A known weakness in a system, system security procedures, internal controls, or implementation by which an actor or event may intentionally exploit or accidentally trigger the weakness to access, modify, or disrupt normal operations of a system-resulting in a security incident or a violation of the system's security policy.”
\end{itemize}
% END AI-IMPROVED

% BEGIN ORIGINAL
% In cybersecurity, the canonical vulnerability database is the \gls{nvd}\footnote{\url{https://nvd.nist.gov/}} which is operated by the \gls{nist}.
% Vulnerabilities listed in the \gls{nvd} are called \gls{cve}.
% Each \gls{cve} has a unique identifier, such as CVE-2014-0160 which is used to track the state (e.g., vulnerable or fixed) of this vulnerability in software distributions.
% END ORIGINAL

% BEGIN AI-IMPROVED
In cybersecurity, the primary vulnerability database is the \gls{nvd}\footnote{\url{https://nvd.nist.gov/}}, which is operated by the \gls{nist}.
Vulnerabilities listed in the \gls{nvd} are referred to as \gls{cve}.
Each \gls{cve} has a unique identifier, such as CVE-2014-0160, which is used to track the status (e.g., vulnerable or fixed) of this vulnerability in software distributions.
% END AI-IMPROVED

% BEGIN ORIGINAL
% There are so-called \glspl{cna} which are authorized entities with specific scope and responsibility to regularly assign \gls{cve} IDs and publish corresponding \gls{cve} Records.
% For instance, the Linux Kernel organization\footnote{\url{https://kernel.org}} the Python Software Foundation\footnote{\url{https://www.python.org}}, or the curl project\footnote{\url{https://curl.se}} were recently accepted as \glspl{cna} and are allowed to allocate \gls{cve} ids for their managed projects.
% The current Top-Level Root entities which can allocate \gls{cve} ids are\footnote{\url{https://www.cve.org/PartnerInformation/ListofPartners}} the \gls{cisa} and the MITRE Corporation.
% END ORIGINAL

% BEGIN AI-IMPROVED
There are entities known as \glspl{cna} that are authorized with specific scopes and responsibilities to regularly assign \gls{cve} IDs and publish the corresponding \gls{cve} Records.
For example, the Linux Kernel organization\footnote{\url{https://kernel.org}}, the Python Software Foundation\footnote{\url{https://www.python.org}}, and the curl project\footnote{\url{https://curl.se}} were recently accepted as \glspl{cna} and are permitted to allocate \gls{cve} IDs for their managed projects.
The current Top-Level Root entities that can allocate \gls{cve} IDs are\footnote{\url{https://www.cve.org/PartnerInformation/ListofPartners}} the \gls{cisa} and the MITRE Corporation.
% END AI-IMPROVED

% BEGIN ORIGINAL
% The \gls{nvd} has a software ecosystem-agnostic scope, i.e., vulnerabilities which are identified in arbitrary software projects can be submitted to the relevant \gls{cna}.
% However, those vulnerabilities have to be mapped to the relevant software packages in software distribution to be practically useful.
% There are different attempts which are maintained by various different organizations:
% END ORIGINAL

% BEGIN AI-IMPROVED
The \gls{nvd} has a software ecosystem-agnostic scope, meaning vulnerabilities identified in any software project can be submitted to the relevant \gls{cna}.
However, these vulnerabilities must be mapped to the relevant software packages in software distribution to be practically useful.
Various attempts are maintained by different organizations:
% END AI-IMPROVED

% BEGIN ORIGINAL
% \begin{itemize}
% 	\item \textbf{Software Repositories}: Ecosystem specific software repositories, such as \gls{pypi} (for Python) or pkg.go.dev (for Go) maintain own databases that connect \gls{nvd} database entries with particular software packages.
% 	      Additionally, the states, e.g., vulnerable or patched, of the issues are tracked.
% 	\item \textbf{Linux Distributions}: Similar to ecosystem specific repositories, several Linux distributions, such as Arch Linux, Debian, Fedora, or Gentoo maintain separate security databases.
% 	      These databases also connect \gls{nvd} database entries to particular package names in the context of the Linux distribution.
% 	\item \textbf{Software Projects}: A few software projects maintain a good tracking of security issues themselves.
% 	      Recently, there is a development to more \glspl{cna} where software projects implement security tracking and even allocate own \gls{cve} numbers.
% 	\item \textbf{Aggregated Databases}: Additionally, there are independent projects that collect data from all other security databases.
% 	      Examples are the Github Advisory Database, osv.dev, or the vulnerability database provided by mend.io.
% 	      %https://www.mend.io/vulnerability-database/
% \end{itemize}
% END ORIGINAL

% BEGIN AI-IMPROVED
\begin{itemize}
	\item \textbf{Software Repositories}: Ecosystem-specific software repositories, such as \gls{pypi} (for Python) or pkg.go.dev (for Go), maintain their own databases that connect \gls{nvd} database entries with specific software packages. Additionally, the status of issues, such as vulnerable or patched, is tracked.
	\item \textbf{Linux Distributions}: Similar to ecosystem-specific repositories, several Linux distributions, such as Arch Linux, Debian, Fedora, or Gentoo, maintain separate security databases. These databases also link \gls{nvd} database entries to specific package names within the context of the Linux distribution.
	\item \textbf{Software Projects}: A few software projects maintain effective tracking of security issues themselves. Recently, there has been a move towards more \glspl{cna} where software projects implement security tracking and even assign their own \gls{cve} numbers.
	\item \textbf{Aggregated Databases}: Additionally, there are independent projects that collect data from all other security databases. Examples include the Github Advisory Database, osv.dev, or the vulnerability database provided by mend.io.
	      %https://www.mend.io/vulnerability-database/
\end{itemize}
% END AI-IMPROVED

% BEGIN ORIGINAL
% A problem that emerges with such a variety of different databases publishing different views of the same raw data is the usage of a common data format. There is an attempt led by to Google to unify the data structure of these databases\footnote{\url{https://ossf.github.io/osv-schema/}}.
% END ORIGINAL

% BEGIN AI-IMPROVED
A problem that arises with the variety of different databases publishing different views of the same raw data is the use of a common data format.
There is an effort led by Google to unify the data structure of these databases\footnote{\url{https://ossf.github.io/osv-schema/}}.
% END AI-IMPROVED

% BEGIN ORIGINAL
% Due to the lack of a security tracker in NixOS, the security database of the Debian project was considered for this analysis.
% The Debian project offers a software repository of equal size as NixOS.
% Since Debian is a release-based Linux distribution, it was expected that the structure of the package repository is in a fairly comparable state to NixOS which is release-based, too.
% The Debian security database offers an easy to use \gls{json}-based \gls{api}.
% END ORIGINAL

% BEGIN AI-IMPROVED
Due to the absence of a security tracker in NixOS, the security database of the Debian project was considered for this analysis.
The Debian project provides a software repository of comparable size to NixOS.
Since Debian is a release-based Linux distribution, it was expected that the structure of the package repository would be in a state comparable to that of NixOS, which is also release-based.
The Debian security database offers an easy-to-use \gls{json}-based \gls{api}.
% END AI-IMPROVED

\subsection{Addition of Missing Data and Filtering}

% BEGIN ORIGINAL
% The dependency graph contains a lot of information, for instance the package name, the software license, or the supported platforms.
% For this study, the \gls{url} of the Git repository, the programming language of the relevant software project, a category, and the backers were considered relevant.
% END ORIGINAL

% BEGIN AI-IMPROVED
The dependency graph contains a wealth of information, such as the package name, the software license, and the supported platforms. For this study, the \gls{url} of the Git repository, the programming language of the relevant software project, a category, and the backers were considered relevant.
% END AI-IMPROVED

% BEGIN ORIGINAL
% Important information such as the URL to the Git repository of the source code is not present in the created graph data structure.
% The nixpkgs repository somehow needs access to the relevant source code in order to actually build the software packages.
% For this purpose, the whole build step which includes downloading the source code is defined as a nix function which does not expose the \gls{url} of the Git repository.
% Furthermore, signed tarballs or similar are often used instead of directly cloning the Git repository.
% Therefore, there is no choice but to manually search the source code repositories' \glspl{url} and add it to the dataset.
% END ORIGINAL

% BEGIN AI-IMPROVED
Important information, such as the URL to the Git repository of the source code, is not present in the created graph data structure. The nixpkgs repository needs access to the relevant source code to build the software packages. For this purpose, the entire build step, which includes downloading the source code, is defined as a nix function that does not expose the \gls{url} of the Git repository. Additionally, signed tarballs or similar are often used instead of directly cloning the Git repository. Therefore, it is necessary to manually search for the source code repositories' \glspl{url} and add them to the dataset.
% END AI-IMPROVED

% BEGIN ORIGINAL
% The same manual approach applies for finding out the programming language.
% A simple approach for automating this problem would be identifying the used build system.
% However, it turned out that this approach is error-prone, since most build systems support multiple programming languages in lots of different configurations.
% Another possibility would be to count the filename endings of all source code files.
% It turned out that this approach is error-prone, too, since projects can contain supplementary data (e.g., documentation) which would lead to inaccurate results.
% Since the categories were defined by the authors to gain an overview of the found software projects for informational purposes, those also had to be determined by a manual approach.
% END ORIGINAL

% BEGIN AI-IMPROVED
The same manual approach applies to identifying the programming language. A straightforward method for automating this process would be to identify the build system used. However, this approach is error-prone, as most build systems support multiple programming languages in various configurations. Another possibility would be to count the filename extensions of all source code files. This approach is also error-prone because projects can contain supplementary data (e.g., documentation), which would lead to inaccurate results. Since the categories were defined by the authors to gain an overview of the found software projects for informational purposes, these too had to be determined manually.
% END AI-IMPROVED

% BEGIN ORIGINAL
% After adding missing data to the dataset via a manual approach, it turned out that some entries could not be evaluated further.
% For instance, there were some duplicates due to different package versions (e.g. Python 3.10 and Python 3.11).
% A few projects surprisingly do not yet maintain their code in Git but in legacy \glspl{vcs}, such as the \gls{cvs} with the most recent release from 2008.
% All in all, from 200 database entries 35 were filtered out.
% The following analysis was conducted with 165 projects.
% END ORIGINAL

% BEGIN AI-IMPROVED
After adding missing data to the dataset manually, it became apparent that some entries could not be further evaluated.
For instance, there were duplicates due to different package versions (e.g., Python 3.10 and Python 3.11).
Surprisingly, a few projects still maintain their code in legacy \glspl{vcs}, such as the \gls{cvs}, with the most recent release from 2008. In total, 35 entries were filtered out from the 200 database entries.
The following analysis was conducted with 165 projects.
% END AI-IMPROVED

\subsection{Collection of Metrics}

% BEGIN ORIGINAL
% The simplest approach of getting a meaningful number of a project's overall maintenance state would be using the \gls{lma} value from \cite{COELHO2020106274}.
% There are two limitations that prevent using the \gls{lma} in this study.
% First, the authors did not publish their pre-trained model or code that could be used to create an own setup.
% Second, the authors only considered repositories on Github.
% Since a lot of important software projects are hosted on different Git servers, such as the ones provided by the GNU project, the approach shown in \cite{COELHO2020106274} is not applicable to this study.
% END ORIGINAL

% BEGIN AI-IMPROVED
The simplest approach to obtaining a meaningful number for a project's overall maintenance state would be to use the \gls{lma} value from \cite{COELHO2020106274}.
There are two limitations that prevent the use of the \gls{lma} in this study.
First, the authors did not publish their pre-trained model or code that could be used to create a setup of our own.
Second, the authors only considered repositories on Github.
Since many important software projects are hosted on different Git servers, such as those provided by the GNU project, the approach shown in \cite{COELHO2020106274} is not applicable to this study.
% END AI-IMPROVED

% BEGIN ORIGINAL
% However, there are indications that some parameters are correlated with the \gls{lma} value, such as: number of core contributors (aka. “bus factor”), lines of code, or commit activity.
% Besides the \gls{lma}, there is currently no method available that can transform these parameters altogether into a number which can be used to compare the maintenance status of software projects.
% Therefore, in this paper only the collected data is presented and discussed.
% All values are collected from the Git repositories directly using statistical methods and tools, such as pola.rs\footnote{\url{https://pola.rs/}}.
% END ORIGINAL

% BEGIN AI-IMPROVED
However, there are indications that some parameters are correlated with the \gls{lma} value, such as the number of core contributors (also known as the “bus factor”), lines of code, and commit activity.
Besides the \gls{lma}, there is currently no method available that can transform these parameters collectively into a number that can be used to compare the maintenance status of software projects.
Therefore, in this paper, only the collected data is presented and discussed.
All values are collected directly from the Git repositories using statistical methods and tools, such as pola.rs\footnote{\url{https://pola.rs/}}.
% END AI-IMPROVED

\subsection{Collection of CVEs}

% BEGIN ORIGINAL
% Since the NixOS project does not offer a security tracker which maps actual \glspl{cve} to the NixOS defininion of a package, it was decided to use a hybrid approach.
% The NixOS repository was used to create a complete dependency graph and the Debian security database was used to obtain the actual state of security issues in those packages.
% Obtaining these information was trivial, since the Debian project provides a \gls{json} based \gls{api} for the provided database.
% END ORIGINAL

% BEGIN AI-IMPROVED
Since the NixOS project does not offer a security tracker that maps actual \glspl{cve} to the NixOS definition of a package, a hybrid approach was decided upon.
The NixOS repository was used to create a complete dependency graph, and the Debian security database was used to obtain the actual state of security issues in those packages.
Obtaining this information was straightforward, as the Debian project provides a \gls{json}-based \gls{api} for the database.
% END AI-IMPROVED

% BEGIN ORIGINAL
% The NixOS package names could be mapped to those used in Debian by manually searching the names of the source packages in Debian.
% The search functionality was sufficient to find corresponding packages quickly.
% However, this approach is prone to some errors:
% %
% \begin{itemize}
% 	\item \textbf{Human error}: Since the data is mapped by hand, some careless mistakes could happen.
% 	      The authors double checked the data in order to keep these errors minimal.
% 	\item \textbf{Missing packages}: Some packages available in NixOS were not available in Debian.
% 	      Such packages are skipped in this analysis.
% 	      Nine packages were not available in Debian at the time of writing.
% 	\item \textbf{Different splits}: Linux distributions tend to split packages in order to keep the required disk space minimal.
% 	      For instance, a small program shipping a lot of documentation could be split into two  packages: the program itself and the documentation.
% 	      Splitting packages is a distribution specific choice.
% 	      By searching the Debian \emph{source} packages most package splits are considered.
% \end{itemize}
% END ORIGINAL

% BEGIN AI-IMPROVED
The NixOS package names could be mapped to those used in Debian by manually searching the names of the source packages in Debian.
The search functionality was sufficient to find corresponding packages quickly.
However, this approach is prone to some errors:

\begin{itemize}
	\item \textbf{Human error}: Since the data is mapped by hand, some careless mistakes could occur. The authors double-checked the data to minimize these errors.
	\item \textbf{Missing packages}: Some packages available in NixOS were not available in Debian. Such packages are skipped in this analysis. Nine packages were not available in Debian at the time of writing.
	\item \textbf{Different splits}: Linux distributions tend to split packages to minimize the required disk space. For instance, a small program shipping a lot of documentation could be split into two packages: the program itself and the documentation. Splitting packages is a distribution-specific choice. By searching the Debian \emph{source} packages, most package splits are considered.
\end{itemize}
% END AI-IMPROVED

\section{Evaluation}
\subsection{Classification of Projects}

\begin{figure*}[htb]
	\begin{subfigure}[t]{0.45\textwidth}
		\includegraphics[width=\linewidth]{img/languages.pdf}
		\caption{A pie chart showing the segmentation of used programming languages.}
		\label{fig:languages}
	\end{subfigure}\hfill
	\begin{subfigure}[t]{0.45\textwidth}
		\includegraphics[width=\linewidth]{img/licences.pdf}
		\caption{A pie chart showing the distribution of used licenses. The SPDX identifiers are used for license names. Multi licensed projects are noted with “multi”; projects with an empty license field are indicated with “unset”.}
		\label{fig:licences}
	\end{subfigure}
	\medskip
	\begin{subfigure}[t]{0.45\textwidth}
		\includegraphics[width=\linewidth]{img/categories.pdf}
		\caption{A pie chart showing the categories of the examined software projects. The categories have been defined and assigned in this work.}
		\label{fig:categories}
	\end{subfigure}\hfill
	\begin{subfigure}[t]{0.45\textwidth}
		\includegraphics[width=\linewidth]{img/backers.pdf}
		\caption{A pie chart showing the backers of the examined projects. NPO stands for Nonprofit organization.}
		\label{fig:backers}
	\end{subfigure}

	\caption{Four pie charts showing different analyses according to the dataset's classification.}
	\label{fig:three graphs}
\end{figure*}

% BEGIN ORIGINAL
% A huge dependency graph was created of all packages that are provided by the NixOS Linux distribution.
% The created graph contains 82,011 nodes representing software packages and 273,681 edges representing dependency relations.
% END ORIGINAL

% BEGIN AI-IMPROVED
A large dependency graph was created for all packages provided by the NixOS Linux distribution.
The created graph contains 82,011 nodes representing software packages and 273,681 edges representing dependency relations.
% END AI-IMPROVED

% BEGIN ORIGINAL
% \autoref{fig:languages} shows the segmentation of used programming languages.
% Most projects, more than half of all analyzed projects, are written in C.
% The second most used language is Haskell, followed by Python, C++, and Rust.
% END ORIGINAL

% BEGIN AI-IMPROVED
\autoref{fig:languages} shows the segmentation of used programming languages.
Most projects, more than half of all analyzed projects, are written in C.
The second most used language is Haskell, followed by Python, C++, and Rust.
% END AI-IMPROVED

% BEGIN ORIGINAL
% \autoref{fig:licences} shows the used licenses in the dataset.
% The most used licenses are BSD-3-Clause and MIT.
% Some projects had an empty license field in the dataset which is indicated with “unset”.
% Some projects had multiple licenses.
% Those cases were grouped together and indicated with “multi”.
% END ORIGINAL

% BEGIN AI-IMPROVED
\autoref{fig:licences} shows the used licenses in the dataset.
The most used licenses are BSD-3-Clause and MIT.
Some projects had an empty license field in the dataset, which is indicated with “unset”.
Some projects had multiple licenses.
Those cases were grouped together and indicated with “multi”.
% END AI-IMPROVED

% BEGIN ORIGINAL
% \autoref{fig:categories} shows the categories defined and assigned by the authors.
% Those categories help to better understand the dataset and was used for sanity checking.
% The most used category is “Support” which is used to tag software libraries that only implement basic data structures, such as lists.
% The second most used category is “Display Server” which is used for all kinds of X-Server and Wayland related libraries.
% END ORIGINAL

% BEGIN AI-IMPROVED
\autoref{fig:categories} shows the categories defined and assigned by the authors.
These categories help to better understand the dataset and were used for sanity checking.
The most used category is “Support”, which is used to tag software libraries that only implement basic data structures, such as lists.
The second most used category is “Display Server”, which is used for all kinds of X-Server and Wayland-related libraries.
% END AI-IMPROVED

% BEGIN ORIGINAL
% \autoref{fig:backers} shows the backers of the examined projects.
% Most projects are maintained by a single person where no affiliation to a, e.g., company could be determined.
% The second most projects are backed by a \gls{npo}, such as the GNOME Foundation or GNU, followed by companies.
% END ORIGINAL

% BEGIN AI-IMPROVED
\autoref{fig:backers} shows the backers of the examined projects.
Most projects are maintained by a single person, with no affiliation to, for example, a company that could be determined.
The second most projects are backed by a \gls{npo}, such as the GNOME Foundation or GNU, followed by companies.
% END AI-IMPROVED

\subsection{Parameters of Projects}

To better visualize statistical distributions, the following evaluations use box-and-whisker diagrams.
The configuration for all box-and-whisker diagrams, shown in \autoref{fig:box-config}, is as follows\footnote{\url{https://matplotlib.org/stable/api/_as_gen/matplotlib.pyplot.boxplot.html}}.

\begin{figure}[htb]
    \centering
\begin{Verbatim}[fontsize=\footnotesize,frame=single,samepage=true]
     Q1-1.5IQR   Q1   median  Q3   Q3+1.5IQR
                  |-----:-----|
  o      |--------|     :     |--------|    o  o
                  |-----:-----|
flier             <----------->            fliers
                       IQR
\end{Verbatim}
    \caption{Sketch showing the used configuration for box-and-whisker diagrams in this paper.}
    \label{fig:box-config}
\end{figure}



% BEGIN ORIGINAL
% The box extends from the first quartile~(Q1) to the third quartile~(Q3) of the data, with a line at the median~$\tilde x$.
% The whiskers extend from the box to the farthest data point lying within 1.5 times the \gls{iqr} from the box.
% Flier points are those past the end of the whiskers.
% Fliers are indicated with a circle, \texttt{o}.
% Speaking in terms of standard deviation $\sigma$, the box extends to $\pm 0.6745 \sigma$.
% Fliers are beyond $\pm 2.698 \sigma$.
% END ORIGINAL

% BEGIN AI-IMPROVED
The box extends from the first quartile~(Q1) to the third quartile~(Q3) of the data, with a line at the median~$\tilde x$.
The whiskers extend from the box to the farthest data point lying within 1.5 times the \gls{iqr} from the box.
Flier points are those beyond the ends of the whiskers.
Fliers are indicated with a circle, \texttt{o}.
In terms of standard deviation~$\sigma$, the box extends to $\pm 0.6745 \sigma$.
Fliers are beyond $\pm 2.698 \sigma$.
% END AI-IMPROVED

\begin{figure}[htb]
	\includegraphics[width=\linewidth]{img/ages.pdf}
	\caption{The age distribution of the chosen packages at the time of writing. The timestamp of the first commit in the Git repository is considered the start of the project.}
	\label{fig:age}
\end{figure}
%
\begin{figure}[htb]
	\includegraphics[width=\linewidth]{img/ages-box.pdf}
	\caption{A box-and-whisker diagram of the project ages. The median is $\tilde x = 6$; projects older than 32 years are considered fliers.}
	\label{fig:ages-box}
\end{figure}

% BEGIN ORIGINAL
% The authors defined the age of a project as the time difference between the first commit in the Git repository and the date of the evaluation (March 18, 2024).
% \autoref{fig:age} shows the distribution of project ages across all analyzed projects.
% Most projects considered important for the ecosystem are between 10 and 20 years old.
% \autoref{fig:ages-box} shows a box-and-whisker diagram of the age distribution.
% Projects with an age greater than 32 years are considered fliers in this evaluation.
% These projects are: Python (33 years), Perl (36 years), and Emacs (38 years).
% END ORIGINAL

% BEGIN AI-IMPROVED
The authors defined the age of a project as the time difference between the first commit in the Git repository and the date of the evaluation (March 18, 2024).
\autoref{fig:age} shows the distribution of project ages across all analyzed projects.
Most projects considered important for the ecosystem are between 10 and 20 years old.
\autoref{fig:ages-box} shows a box-and-whisker diagram of the age distribution.
Projects with an age greater than 32 years are considered fliers in this evaluation.
These projects are: Python (33 years), Perl (36 years), and Emacs (38 years).
% END AI-IMPROVED

\begin{figure}[htb]
	\includegraphics[width=\linewidth]{img/locs-box.pdf}
	\caption{A box-and-whisker diagram of the number of lines of code. The median is $\tilde x = 12750$; projects with more than 2.2k lines of code are considered fliers. The fliers deviate by several orders of magnitude from the median.}
	\label{fig:locs-box}
\end{figure}

\autoref{fig:locs-box} shows the distribution of lines of code.
This value is widely spread.
The median is $\tilde x = 12750$ and the maximum observed value is 2,992,528.
The flier values deviate by several orders of magnitude from the median.
Projects which a high number of \gls{loc} are QT~(2,992,528~LoC), Emacs~(1,970,007~LoC), Python~(1,806,924~LoC) or Ruby~(1,731,991~LoC).

\begin{figure}[htb]
	\includegraphics[width=\linewidth]{img/bus-box.pdf}
	\caption{A box-and-whisker diagram of the bus factors. A bus factor of 1 in this analysis is defined as one author is responsible for 80\% of all commits. The median is $\tilde x = 6$. The largest observed value is 212.}
	\label{fig:bus-box}
\end{figure}

\autoref{fig:bus-box} shows the distribution of the bus factors.
In the literature, the “bus factor” is known as the minimum number of team members that have to suddenly disappear from a project before the project stalls due to lack of knowledgeable or competent personnel.
There are multiple different definitions how the bus factor can be calculated.
The authors decided to use the definition of number of core contributors from \cite{COELHO2020106274}, i.e., the number of authors who own more than 80\% of all commits.

The distribution looks similar as the distribution of lines of code in \autoref{fig:locs-box}.
However, the projects identified as fliers are different.
Projects with a remarkable high bus factor are:  gdk-pixbuf~(212), QT~(165), glib~(129), and at-spi2-core~(116).

\begin{figure}[htb]
	\includegraphics[width=\linewidth]{img/deps-box.pdf}
	\caption{A box-and-whisker diagram of the reverse dependencies. The median is $\tilde x = 375$. The fliers deviate by one order of magnitude from the median. The maximum observed number is 7,709.}
	\label{fig:deps-box}
\end{figure}

\autoref{fig:deps-box} shows the distribution of the number of reverse dependencies.
The distribution looks familiar: a low median value with a small box and a few fliers that deviate by a magnitude from the median.
Projects with the highest number of reverse dependencies are Python~(7,709), Texinfo~(6,028), Emacs~(5,951), and Perl~(2,444).

\subsection{Open Issues}

The following charts combine the data from the dependency graph with the Debian security database.
As a first step, the Debian security database was examined.
In the database, there are 3,527 packages listed.
At the time of writing, there are 2,391 open issues where a \gls{cve} is considered not patched.

\begin{figure}[htb]
	\includegraphics[width=\linewidth]{img/cves-box.pdf}
	\caption{A box-and-whisker diagram of the overall number of \gls{cve} related entries in the Debian security database.}
	\label{fig:cve-box}
\end{figure}

% BEGIN ORIGINAL
% \autoref{fig:cve-box} shows a box-and-whisker diagram of the amount of overall security issues per package.
% The package with the most entries in the database is the Linux kernel~(3,025 entries) followed by Chromium~(1,571 entries), Firefox~(1,347 entries), and Gitlab~(981 entries).
% The median is $\tilde x = 2$ entries per package.
% END ORIGINAL

% BEGIN AI-IMPROVED
\autoref{fig:cve-box} shows a box-and-whisker diagram of the number of overall security issues per package.
The package with the most entries in the database is the Linux kernel~(3,025 entries), followed by Chromium~(1,571 entries), Firefox~(1,347 entries), and Gitlab~(981 entries).
The median is $\tilde x = 2$ entries per package.
% END AI-IMPROVED

% BEGIN ORIGINAL
% In addition to this, \autoref{fig:cve-box-unpatched} shows the number of unpatched \gls{cve} issues per package.
% The values are widely spread.
% The median is $\tilde x = 0$ and the box size is $\text{IQR} = 0$.
% For the used box-and-whisker diagram configuration all database entries with the number $n$ of open \gls{cve} issues where $n \neq \tilde x \neq 0$ are considered fliers.
% The package with the most open issues is the Linux kernel~(177 entries) followed by TeX Live~(90 entries), gtkwave~(82 entries), and wpewebkit~(37 entries).
% END ORIGINAL

% BEGIN AI-IMPROVED
In addition, \autoref{fig:cve-box-unpatched} shows the number of unpatched \gls{cve} issues per package.
The values are widely spread.
The median is $\tilde x = 0$ and the box size is $\text{IQR} = 0$.
For the box-and-whisker diagram configuration used, all database entries with the number $n$ of open \gls{cve} issues where $n \neq \tilde x \neq 0$ are considered fliers.
The package with the most open issues is the Linux kernel~(177 entries), followed by TeX Live~(90 entries), gtkwave~(82 entries), and wpewebkit~(37 entries).
% END AI-IMPROVED

\begin{figure}[htb]
	\includegraphics[width=\linewidth]{img/cves-unpatched-box.pdf}
	\caption{A box-and-whisker diagram of the unpatched number of \gls{cve} related entries in the Debian security database.}
	\label{fig:cve-box-unpatched}
\end{figure}

% BEGIN ORIGINAL
% In the chosen package list, there are 16 packages with open \gls{cve} issues, as shown by \autoref{fig:cves}.
% The packages are ordered from highest Katz centrality value to lower values, where zlib has the highest value.
% Both systemd and OpenSSL have the highest number of unaddressed issues~(6 entries), followed by gdbm~(4 entries) and Perl/expat~(3 entries).
% END ORIGINAL

% BEGIN AI-IMPROVED
In the chosen package list, there are 16 packages with open \gls{cve} issues, as shown by \autoref{fig:cves}.
The packages are ordered from the highest Katz centrality value to lower values, with zlib having the highest value.
Both systemd and OpenSSL have the highest number of unaddressed issues~(6 entries), followed by gdbm~(4 entries) and Perl/expat~(3 entries).
% END AI-IMPROVED

\begin{figure}[htb]
	\includegraphics[width=\linewidth]{img/cves.pdf}
	\caption{The number of open issues per package in Debian stable. The packages are ordered by the Katz centrality value from high values to lower values, where zlib has the highest value.}
	\label{fig:cves}
\end{figure}

% BEGIN ORIGINAL
% The scatter plot in \autoref{fig:cve-loc} shows the relation between the overall number of \glspl{cve} in a particular package and the number of lines of code.
% There seems to be a linear correlation between those.
% A further analysis revealed that projects with a high number of \glspl{cve} are unmaintained projects were no developer is available for adressing the open issues.
% END ORIGINAL

% BEGIN AI-IMPROVED
The scatter plot in \autoref{fig:cve-loc} shows the relationship between the overall number of \glspl{cve} in a particular package and the number of lines of code.
There seems to be a linear correlation between these.
Further analysis revealed that projects with a high number of \glspl{cve} are unmaintained projects where no developer is available to address the open issues.
% END AI-IMPROVED

\begin{figure}[htb]
	\includegraphics[width=\linewidth]{img/cve-loc.pdf}
	\caption{The number of \glspl{cve} in a particular package from the chosen dataset plotted against the number of lines of code. The linear regression line shows a possible correlation between the number of \glspl{cve} and \gls{loc}.}
	\label{fig:cve-loc}
\end{figure}

\section{Discussion}

\subsection{Structure of the Ecosystem}

% BEGIN ORIGINAL
% This study showed that the current \gls{foss} ecosystem suffers from multiple problems from a infrastructure point of view.
% However, a software ecosystem consisting of 82,011 packages and 273,681 \texttt{DEPENDS\_ON} relations is already too large for a single human to grasp.
% Since required metadata needed to be added manually (cf. \autoref{sec:methodology}), the presented approach is currently only feasible for a limited number of software packages.
% For a large-scale analysis, the required interfaces, such as a standardized metadata format or a web-based \gls{api}, are not available in the ecosystem.
% Furthermore, there is a lot of redundancy and uncertainty in the publicly available data hindering automated evaluation and monitoring.
% END ORIGINAL

% BEGIN AI-IMPROVED
This study showed that the current \gls{foss} ecosystem suffers from multiple problems from an infrastructure point of view.
However, a software ecosystem consisting of 82,011 packages and 273,681 \texttt{DEPENDS\_ON} relations is already too large for a single human to grasp.
Since required metadata needed to be added manually (cf. \autoref{sec:methodology}), the presented approach is currently only feasible for a limited number of software packages.
For a large-scale analysis, the required interfaces, such as a standardized metadata format or a web-based \gls{api}, are not available in the ecosystem.
Furthermore, there is a lot of redundancy and uncertainty in the publicly available data hindering automated evaluation and monitoring.
% END AI-IMPROVED

% BEGIN ORIGINAL
% For instance, the \gls{nvd} database serves as the de facto standard service that provides information regarding vulnerabilities.
% Unfortunately, the \gls{nvd} does not provide standardized information which particular software package in which particular software ecosystem is described.
% To address this problem, there are multiple additional databases available (cf. \autoref{sec:vulndbs}) that utilize a standardized format developed by Google\footnote{\url{https://ossf.github.io/osv-schema/}} which is not used by the \gls{nvd}, though.
% The scope of these databases is always limited, e.g., to Github, a specific programming language ecosystem, or certain Linux distributions.
% However, there is no vulnerability database available that is suitable for a large-scale analysis of the whole \gls{foss} ecosystem including information about software dependencies.
% In order to improve this situation, the development of an alternative software distribution architecture where the relationships between software modules are specified in a standardized cross-language and cross-ecosystem format is a topic for future research.
% END ORIGINAL

% BEGIN AI-IMPROVED
For instance, the \gls{nvd} database serves as the de facto standard service that provides information regarding vulnerabilities.
Unfortunately, the \gls{nvd} does not provide standardized information about which particular software package in which particular software ecosystem is described.
To address this problem, there are multiple additional databases available (cf. \autoref{sec:vulndbs}) that utilize a standardized format developed by Google\footnote{\url{https://ossf.github.io/osv-schema/}}, which is not used by the \gls{nvd}, though.
The scope of these databases is always limited, e.g., to Github, a specific programming language ecosystem, or certain Linux distributions.
However, there is no vulnerability database available that is suitable for a large-scale analysis of the whole \gls{foss} ecosystem, including information about software dependencies.
In order to improve this situation, the development of an alternative software distribution architecture where the relationships between software modules are specified in a standardized cross-language and cross-ecosystem format is a topic for future research.
% END AI-IMPROVED

% BEGIN ORIGINAL
% According to \autoref{fig:languages}, the Haskell programming language is the second most used language.
% This was not expected, since Haskell is not commonly used for systems programming on Linux.
% Further investigation using the proposed categories (cf. \autoref{fig:categories}) revealed that the Haskell software ecosystem is more separated than, e.g., the Python ecosystem.
% In Haskell there is no large “batteries included” standard library like the Python standard library\footnote{\url{https://docs.python.org/3/tutorial/stdlib.html\#batteries-included}}.
% Instead, basic functionality, like data structure implementations, are maintained in separate software modules.
% \autoref{fig:categories} shows a high occurrence of packages in the “Support” category.
% Most of these packages are Haskell modules providing basic functionality, such as lists or dictionaries.
% Consequently, Haskell modules appeared in the dependency graph with a high Katz centrality value.
% Such separated ecosystems are very flexible, since new functionality can be added quickly.
% However, it is not clear if the added complexity in terms of dependency tracking or developer resources is harmful to the Haskell ecosystem; especially compared to ecosystems as Python.
% END ORIGINAL

% BEGIN AI-IMPROVED
According to \autoref{fig:languages}, the Haskell programming language is the second most used language.
This was not expected, since Haskell is not commonly used for systems programming on Linux.
Further investigation using the proposed categories (cf. \autoref{fig:categories}) revealed that the Haskell software ecosystem is more separated than, e.g., the Python ecosystem.
In Haskell, there is no large “batteries included” standard library like the Python standard library\footnote{\url{https://docs.python.org/3/tutorial/stdlib.html\#batteries-included}}.
Instead, basic functionality, like data structure implementations, is maintained in separate software modules.
\autoref{fig:categories} shows a high occurrence of packages in the “Support” category.
Most of these packages are Haskell modules providing basic functionality, such as lists or dictionaries.
Consequently, Haskell modules appeared in the dependency graph with a high Katz centrality value.
Such separated ecosystems are very flexible, as new functionality can be added quickly.
However, it is not clear if the added complexity in terms of dependency tracking or developer resources is harmful to the Haskell ecosystem, especially compared to ecosystems like Python.
% END AI-IMPROVED

% Paragraph on low rate of company backers
% BEGIN ORIGINAL
% According to \autoref{fig:backers}, only 15.7\% of the examined projects' backers have an explicit company affiliation.
% This is surprising considering the added value which the \gls{foss} ecosystems provides to a wide range of different companies.
% Especially for critical projects, such as the ones identified by this paper, broader support of companies would be important to literally avoid single point of failures, e.g., maintenance by a single person as in the case of the xz Backdoor~(CVE-2024-3094).
% Hence, both the general public but also the companies themselves would benefit from a more resilient \gls{foss} ecosystem.
% END ORIGINAL

% BEGIN AI-IMPROVED
According to \autoref{fig:backers}, only 15.7\% of the examined projects' backers have an explicit company affiliation.
This is surprising considering the added value that the \gls{foss} ecosystems provide to a wide range of different companies.
Especially for critical projects, such as the ones identified by this paper, broader support from companies would be important to literally avoid single points of failure, e.g., maintenance by a single person as in the case of the xz Backdoor~(CVE-2024-3094).
Hence, both the general public and the companies themselves would benefit from a more resilient \gls{foss} ecosystem.
% END AI-IMPROVED

% BEGIN ORIGINAL
% However, there might be the risk of companies actively pushing projects towards stricter or non-free licensing.
% Recently, there was such a case in a popular open-source key-value database, Redis\footnote{\url{https://redis.com/blog/redis-adopts-dual-source-available-licensing/}}.
% Due to an active \gls{cla}, the backing company changed the license of the project to a non-free alternative.
% Consequently, developers left the project and created a fork\footnote{\url{https://github.com/valkey-io/valkey}}.
% A further example is the Intel-backed Hyperscan library which was recently converted into closed-source software\footnote{\url{https://www.phoronix.com/news/Intel-Hyperscan-Now-Proprietary}}.
% The authors assume that it is only a matter of time until forks will start to appear.
% Unfortunately, such situations increase the complexity of the open-source ecosystem rather than decreasing it.
% This is especially worth to be mentioned since a majority of the analyzed projects uses permissive license models such BSD and MIT as indicated by \autoref{fig:licences}.
% END ORIGINAL

% BEGIN AI-IMPROVED
However, there might be the risk of companies actively pushing projects towards stricter or non-free licensing.
Recently, there was such a case in a popular open-source key-value database, Redis\footnote{\url{https://redis.com/blog/redis-adopts-dual-source-available-licensing/}}.
Due to an active \gls{cla}, the backing company changed the license of the project to a non-free alternative.
Consequently, developers left the project and created a fork\footnote{\url{https://github.com/valkey-io/valkey}}.
A further example is the Intel-backed Hyperscan library, which was recently converted into closed-source software\footnote{\url{https://www.phoronix.com/news/Intel-Hyperscan-Now-Proprietary}}.
The authors assume that it is only a matter of time until forks will start to appear.
Unfortunately, such situations increase the complexity of the open-source ecosystem rather than decreasing it.
This is especially worth mentioning since a majority of the analyzed projects use permissive license models such as BSD and MIT, as indicated by \autoref{fig:licences}.
% END AI-IMPROVED

% BEGIN ORIGINAL
% The box-and-whisker diagrams in \autoref{fig:locs-box}, \autoref{fig:bus-box}, and \autoref{fig:deps-box} altogether show a common property of the examined software packages.
% On the one hand here are large and actively maintained projects by a community.
% On the other hand there are projects that are maintained by a small number of developers (= a low bus factor).
% END ORIGINAL

% BEGIN AI-IMPROVED
The box-and-whisker diagrams in \autoref{fig:locs-box}, \autoref{fig:bus-box}, and \autoref{fig:deps-box} altogether show a common property of the examined software packages.
On the one hand, there are large and actively maintained projects by a community.
On the other hand, there are projects that are maintained by a small number of developers (i.e., a low bus factor).
% END AI-IMPROVED

% BEGIN ORIGINAL
% At the time of writing, in the Debian security database were 2,391 open \gls{cve} entries.
% Using the presented Katz centrality-based method it is possible to prioritize and sort those entries (cf. \autoref{fig:cve-box-unpatched}).
% END ORIGINAL

% BEGIN AI-IMPROVED
At the time of writing, there were 2,391 open \gls{cve} entries in the Debian security database.
Using the presented Katz centrality-based method, it is possible to prioritize and sort those entries (cf. \autoref{fig:cve-box-unpatched}).
% END AI-IMPROVED

\subsection{CVE Handling}

% BEGIN ORIGINAL
% From the analyzed projects, the curl project stands out as a good example for well maintained software.
% The curl project provides an implementation of a complete \gls{http} stack and has 663 reverse dependencies in NixOS.
% Daniel Stenberg, the main maintainer, makes every effort for improving the project.
% For instance, the project recently has been accepted\footnote{\url{https://daniel.haxx.se/blog/2024/01/16/curl-is-a-cna/}} as a \gls{cna} in order to issue own \glspl{cve} entries.
% Further, the project provides a dashboard\footnote{\url{https://curl.se/dashboard.html}} where multiple development metrics are tracked over time.
% Interesting metrics are, e.g., the “\gls{cve} age in code until fixed” or “curl vulnerabilities: C vs non-C mistakes”.
% Such metrics are helpful to track the project state over time and identify emerging issues early.
% END ORIGINAL

% BEGIN AI-IMPROVED
From the analyzed projects, the curl project stands out as a good example of well-maintained software.
The curl project provides an implementation of a complete \gls{http} stack and has 663 reverse dependencies in NixOS.
Daniel Stenberg, the main maintainer, makes every effort to improve the project.
For instance, the project has recently been accepted\footnote{\url{https://daniel.haxx.se/blog/2024/01/16/curl-is-a-cna/}} as a \gls{cna} in order to issue its own \glspl{cve} entries.
Further, the project provides a dashboard\footnote{\url{https://curl.se/dashboard.html}} where multiple development metrics are tracked over time.
Interesting metrics include the “\gls{cve} age in code until fixed” and “curl vulnerabilities: C vs non-C mistakes.”
Such metrics are helpful to track the project state over time and identify emerging issues early.
% END AI-IMPROVED

% BEGIN ORIGINAL
% The \gls{cve} infrastructure with the \gls{nvd} has received negative feedback in the past.
% For instance, for curl CVE-2020-19909 was filed and graded as a 9.8 CRITICAL issue\footnote{\url{https://daniel.haxx.se/blog/2023/08/26/cve-2020-19909-is-everything-that-is-wrong-with-cves/}}.
% It turned out that it was indeed a bug but with no security implications.
% However, the alarmism spread and curl was tagged as insecure by Linux distributions.
% MITRE rejected Daniel Stenberg's requests to withdraw the faulty \gls{cve} due to the existence of a valid weakness (integer overflow) which could result in valid security impact\footnote{\url{https://curl.se/docs/CVE-2020-19909.html}}.
% END ORIGINAL

% BEGIN AI-IMPROVED
The \gls{cve} infrastructure with the \gls{nvd} has received negative feedback in the past.
For instance, for curl, CVE-2020-19909 was filed and graded as a 9.8 CRITICAL issue\footnote{\url{https://daniel.haxx.se/blog/2023/08/26/cve-2020-19909-is-everything-that-is-wrong-with-cves/}}.
It turned out that it was indeed a bug but with no security implications.
However, the alarmism spread, and curl was tagged as insecure by Linux distributions.
MITRE rejected Daniel Stenberg's requests to withdraw the faulty \gls{cve} due to the existence of a valid weakness (integer overflow), which could result in a valid security impact\footnote{\url{https://curl.se/docs/CVE-2020-19909.html}}.
% END AI-IMPROVED

% BEGIN ORIGINAL
% After further discussion, the \gls{cve} was eventually re-scored as 3.3 and curl was accepted as a \gls{cna}.
% The main point of criticism in this situation was that any person can file \glspl{cve}.
% There is no verification process in place where, e.g., a maintainer must approve that it is indeed security-relevant.
% END ORIGINAL

% BEGIN AI-IMPROVED
After further discussion, the \gls{cve} was eventually re-scored as 3.3, and curl was accepted as a \gls{cna}.
The main point of criticism in this situation was that any person can file \glspl{cve}.
There is no verification process in place where, for example, a maintainer must approve that it is indeed security-relevant.
% END AI-IMPROVED

% BEGIN ORIGINAL
% Greg Kroah-Hartman, a kernel maintainer who is sponsored by the Linux Foundation, also talked about the \gls{nvd} situation in the past\footnote{\url{https://kernel-recipes.org/en/2019/talks/cves-are-dead-long-live-the-cve/}}.
% He describes a similar situation with CVE-2019-12357 where a faulty \gls{cve} was assigned and caused repercussions.
% It turned out that patching the said \gls{cve} is not necessary and the change was eventually reverted.
% In software companies, addressing security-related issues usually has a higher priority than minor bug fixes, hence it is easier to justify developer resources when an appropriate \gls{cve} is available.
% As a consequence, the Linux kernel community became a \gls{cna} in order to avoid faulty \glspl{cve} in the future.
% END ORIGINAL

% BEGIN AI-IMPROVED
Greg Kroah-Hartman, a kernel maintainer who is sponsored by the Linux Foundation, also talked about the \gls{nvd} situation in the past\footnote{\url{https://kernel-recipes.org/en/2019/talks/cves-are-dead-long-live-the-cve/}}.
He describes a similar situation with CVE-2019-12357, where a faulty \gls{cve} was assigned and caused repercussions.
It turned out that patching the said \gls{cve} was not necessary, and the change was eventually reverted.
In software companies, addressing security-related issues usually has a higher priority than minor bug fixes; hence, it is easier to justify developer resources when an appropriate \gls{cve} is available.
As a consequence, the Linux kernel community became a \gls{cna} in order to avoid faulty \glspl{cve} in the future.
% END AI-IMPROVED

\subsection{Limitations}

The presented approach has several limitations that must be acknowledged to provide context for the study's findings and to guide future research efforts.

\begin{enumerate}
	\item \textbf{Number of analyzed packages}:
	      In the dataset of approximately 80,000 records, an upper limit of 200 records has been set for manual analysis.
	      This decision is based on balancing the need for detailed insights with the practical constraints of manual review, which are resource-intensive and time-consuming.
	      This manageable number allows deeper analysis of each record, providing richer and more nuanced insights.
	      Although 200 records do not suffice for comprehensive analysis, valuable insights can still be yielded if they are chosen to reflect the dataset's diversity and key characteristics.
	      This approach ensures a balance is struck between detail and practicality within the constraints of manual analysis.
	\item \textbf{Unidimensional Analysis}:
	      This paper examines projects not limited to those hosted on GitHub, which presents certain challenges for conducting a multi-dimensional analysis, including aspects like community activity.
	      Many projects, such as the Linux kernel, Git, or GNU projects, do not utilize GitHub for their community interactions.
	      Consequently, the tools and techniques from the \gls{ossf}, which are tailored specifically for GitHub-hosted projects, are not applicable to this study.
	      This limitation poses a problem for academic research, restricting the scope of available data and thus the comprehensiveness of the analysis in this paper.
	\item \textbf{Human Error in Manual Analysis}:
	      The lack of programming interfaces required a manual analysis approach, which presents a notable limitation.
	      Manual analysis is prone to human error, as it relies on individual judgment and execution, potentially leading to inconsistencies and inaccuracies.
	      While this approach allows for nuanced examination, the potential for human error remains significant, underscoring the need for automated solutions in future research to enhance result reliability and validity.
	\item \textbf{Ambiguous Mapping of Packages}:
	      A notable limitation of this study is the challenge in mapping package names from a specific software repository to entries in the corresponding vulnerability database.
	      This difficulty arises due to the lack of normalization in package names, leading to ambiguities and inconsistencies.
	      As a result, the process of accurately linking packages to their associated vulnerabilities is hindered, potentially affecting the reliability of our analysis.
	      Future work could focus on developing a standardized naming convention or implementing advanced algorithms to improve the accuracy of these mappings.
\end{enumerate}

The limitations identified in this study present opportunities for further research and future work to address and explore these challenges in greater depth.

\section{Conclusion}

% BEGIN ORIGINAL
% This paper addressed the current challenges of managing large software repositories from a security point of view.
% The \gls{foss} ecosystem nowadays consists of multiple different and interconnected sub-ecosystems.
% Due to upcoming legal regulations, such as the European \gls{cra}, techniques to analyze and track (with \glspl{sbom}) the state of a software project and software ecosystems in general, become more and more relevant.
% END ORIGINAL

% BEGIN AI-IMPROVED
This paper addressed the current challenges of managing large software repositories from a security point of view.
The \gls{foss} ecosystem nowadays consists of multiple different and interconnected sub-ecosystems.
Due to upcoming legal regulations, such as the European \gls{cra}, techniques to analyze and track the state of a software project and software ecosystems in general, with \glspl{sbom}, become more and more relevant.
% END AI-IMPROVED

% BEGIN ORIGINAL
% In this study, it became apparent that the \gls{foss} ecosystem currently suffers from different problems, such as insufficient funding, companies that monetize projects by changing licenses to non-free alternatives, or faulty \gls{cve} reports for justifying developer resources.
% From a technical point of view, there are also multiple problems.
% Especially the lack of standardized interfaces will become a burden when implementing the new legal requirements at a large scale.
% END ORIGINAL

% BEGIN AI-IMPROVED
In this study, it became apparent that the \gls{foss} ecosystem currently suffers from different problems, such as insufficient funding, companies that monetize projects by changing licenses to non-free alternatives, or faulty \gls{cve} reports for justifying developer resources.
From a technical point of view, there are also multiple problems.
Especially the lack of standardized interfaces will become a burden when implementing the new legal requirements at a large scale.
% END AI-IMPROVED

% BEGIN ORIGINAL
% Considering the formulated research question \textbf{RQ1} \RQ, the authors came to the conclusion that sorting software modules by their impact is possible with a centrality-based approach.
% Consequently, for this purpose the software modules have to be available in a dependency graph data structure.
% The standardization of common interfaces for creating \gls{foss} ecosystem-wide examinations will be in the focus of future research.
% The author postulates that the existance of such interfaces and dedicated automated evaluations help to better detect supply chain attacks, such as the xz backdoor, early.
% END ORIGINAL

% BEGIN AI-IMPROVED
Considering the formulated research question \textbf{RQ1} \RQ, the authors concluded that sorting software modules by their impact is possible with a centrality-based approach.
Consequently, for this purpose the software modules have to be available in a dependency graph data structure.
The standardization of common interfaces for creating \gls{foss} ecosystem-wide examinations will be in the focus of future research.
The author postulates that the existence of such interfaces and dedicated automated evaluations helps to better detect supply chain attacks, such as the xz backdoor, early.
% END AI-IMPROVED

% BEGIN ORIGINAL
% Considering the more general research question \textbf{RQ2} \RQQ, the authors tend to a two-minded answer.
% On the one hand, there are well maintained and well funded projects such as Python which have a high impact.
% On the other hand, there are projects with a high impact but with a low bus factor that are vulnerable to supply chain attacks, as shown by the xz Backdoor.
% This paper has contributed insights and methods to identify such critical projects.
% This allows the community and especially companies to benefit from the \gls{foss} ecosystem to back these projects and hopefully prevent incidents such as the xz Backdoor in the future.
% END ORIGINAL

% BEGIN AI-IMPROVED
Considering the more general research question \textbf{RQ2} \RQQ, the authors tend to a two-minded answer.
On the one hand, there are well-maintained and well-funded projects such as Python, which have a high impact.
On the other hand, there are projects with a high impact but with a low bus factor that are vulnerable to supply chain attacks, as shown by the xz Backdoor.
This paper has contributed insights and methods to identify such critical projects.
This allows the community, and especially companies, to benefit from the \gls{foss} ecosystem by supporting these projects and hopefully preventing incidents such as the xz Backdoor in the future.
% END AI-IMPROVED

\section*{Data Availability}

The raw data of this study is available under the CC BY 4.0 Legal Code license at Zenodo \cite{tatschner_2024_11276931}.

\section*{Declaration of generative AI and AI-assisted technologies in the writing process}

During the preparation of this work the authors used ChatGPT in order to improve the language of the paper.
After using this tool, the authors reviewed and edited the content as needed and take full responsibility for the content of the published article.

\section*{Acknowledgments}

% BEGIN ORIGINAL
% Many thanks to David Emeis and Veronique Ehmes for the profitable technical discussions regarding vulnerability databases.
% The authors would like the thank the Wikipedia user Pholme for publishing the image\textsuperscript{\ref{footnote:wiki}} used as \autoref{fig:centrality} under the CC BY-SA 4.0 Deed license.
% END ORIGINAL

% BEGIN AI-IMPROVED
Many thanks to David Emeis and Veronique Ehmes for the productive technical discussions regarding vulnerability databases.
The authors would like to thank the Wikipedia user Pholme for publishing the image\textsuperscript{\ref{footnote:wiki}} used as \autoref{fig:centrality} under the CC BY-SA 4.0 Deed license.
% END AI-IMPROVED

\section*{Funding}

This work was supported by the German Federal Ministry of Education and Research (BMBF) under Grant No. 16KIS1847, ALPAKA.

\bibliographystyle{elsarticle-num}
\bibliography{bibliography}

\end{document}
