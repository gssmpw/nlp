\section{Introduction}
Recent progress in humanoid robots has shown impressive results in achieving complex tasks, and the huge potential to become a general robot platform~\citep{cheng2024tv, zhang2024wococo, cheng2024expressive, scironot2024humanoid}. It is a fundamental skill to support various humanoid motions, enabling them to navigate environments and perform tasks with agility and adaptability.
However, most current humanoid locomotion systems, although showing impressive results in motion-based control~\citep{he2024hover,2024exbody2,fu2024humanplus,cheng2024expressive} and mobile manipulation~\citep{lu2024pmp}, pay limited attention to producing versatile and controllable gait styles, leading to single, tedious, unextendable, and unconstrained movements.
Consider humans, we have versatile athletic abilities, such as running, jumping, and even hopping. Even when only walking, we can fine-tune our frequencies, strides, and foot heights. Bringing such versatility into humanoid locomotion is challenging, but it is the key to exploring the edge of humanoid robots' abilities.
\begin{figure*}[htbp]
\centering
\includegraphics[width=0.98\linewidth]{imgs/FrameworkV6.png}
\vspace{-10pt}
\caption{\small \textbf{Framework of \textsc{HugHBC}.} Illustration with the Unitree H1 robot. \textbf{a): Visualization of parts of commands}. The side view (left) highlights the linear velocity, foot swing height, and body pitch commands. The top-right view shows the angular velocity and waist yaw commands, and the bottom-right view shows the body height command. \textbf{b): Policy inputs/outputs.} The policy is provided with commands, proprioceptive observations, the intervention indicator, and outputs all joints of the robots. \textbf{c): Illustrations of four gaits on the robot without/with external intervention.} By default, the policy controls both the upper-body and the lower-body joints. \textbf{d): External control support.} Feasible external control signals can be seamlessly integrated into the robot's behavior without hurting locomotion performance.}
\label{fig:framework}
\vspace{-20pt}
\end{figure*}
To resolve the challenge and build a unified and general humanoid whole-body controller, in this work, we propose \our, namely, \textbf{H}umanoid's \textbf{U}nified and \textbf{G}eneral \textbf{W}hole-\textbf{B}ody \textbf{C}ontrol.
\our is designed for generating versatile locomotion with dynamic, customizable control, enabling the robot to perform gaits such as walking, standing, jumping, and hopping. Furthermore, \our provides the flexibility to adjust foot behavior parameters 
% stride length, removed
foot swing height and gait frequency, and allows combining posture parameters such as body height, waist rotation, and body pitch. 
To achieve this, \our includes a general command space designed for humanoid locomotion, along with advanced training techniques to learn versatile gaits within \textit{one single policy} (except the hopping gait) using reinforcement learning in simulation, which can be directly transferred onto real robots.

Positioned as a basic controller for humanoid robots to perform a wider range of tasks in diverse real-world scenarios, \our introduces intervention training and supports real-time external control signals of the upper body, like teleoperation, allowing for highly robust loco-manipulation, while maintaining precise locomotion control. An overview of the framework is illustrated in \fig{fig:framework}.

In experiments, we show \our preserves high tracking accuracy on eight different commands under four different gaits; we also ablate the improvement in stability and robustness of the upper body intervention training. We further provide a detailed analysis of how commands combination works, shedding light on the intricate relationships between these commands and how they can be leveraged to optimize movement performance.
Through this work, we aim to significantly broaden the scope of humanoid locomotion capabilities, pushing the boundaries of what is possible with current robotic systems.

The key contributions are summarized as follows:
\begin{itemize}[leftmargin=10pt]
    \item An extended general command space with advanced training techniques designed for versatile humanoid locomotion.
    \item Accurate tracking for eight different commands under four different gaits, using one policy for 3 of the 4 gaits.
    \item A basic humanoid controller that supports external upper-body intervention and enables a wider range of loco-manipulation tasks.
\end{itemize}