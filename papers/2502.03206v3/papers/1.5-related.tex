\section{Related Work}
\subsection{Model-Based Humanoid Controller}
% Trajectory optimization is a widely adopted instance of optimal control, which has been extensively applied in recent decades for the design of locomotion in humanoid systems and the generation of multi-contact gait patterns.

% Model Predictive Control (MPC) is a widely adopted approach within optimization-based techniques for humanoid robots. By utilizing either whole-body dynamics models~\citep{schultz2009modeling, koenemann2015whole} or simplified dynamics models, such as centroidal dynamics~\citep{orin2013centroidal, wensing2016improved} and the linear inverted pendulum (LIP) model~\citep{kajita2010biped}, researchers compute optimal trajectories by solving trajectory optimization (TO) problems~\citep{betts1998survey}. Some studies integrate contact mechanics, such as the zero-moment point (ZMP)~\citep{vukobratovic2004zero}, into simplified dynamic models~\citep{sugihara2009standing, wang2024online}, enhancing control strategies for stability and adaptability.

Controlling humanoid robots has become one of the most fascinating problems since decades ago, many researchers and engineers have built complicated systems and tried to solve them with model-based methods in a perspective of optimal control (OC)~\citep{2021ralFootstep, 2020rasRecedingHorizonPlanning, 2018troMulticontact, 2020troC-CROC, 2018Tower, 2020icraCrocoddyl, 2021troPatternGeneration}.
These works typically employ trajectory optimization with dynamic models of varying levels of complexity, such as the linear inverted pendulum model~\citep{kajita2010biped}, centroidal dynamics model~\citep{orin2013centroidal, wensing2016improved}, or full-body dynamics model~\citep{schultz2009modeling, 2015rasHRP-2humanoid, Xinjilefu}, to perform online optimization, or generate periodic motion control through the hybrid zero dynamics model~\citep{da2019combining,sreenath2011compliant, Hereid2016ICRA}. However, most of them can only generate motion based on predefined contact sequences. Even some have successfully incorporated online optimization to generate real-time motion sequences and contact schedules based on instant environmental feedback and user commands and run on humanoid robots in the real world~\citep{2019rasFootstep}, the nonlinear dynamics and multi-contact optimization of humanoid systems demand significant computational resources, making it challenging to meet real-time performance requirements.
A promising solution is to decouple the whole-body multi-contact optimization control problem into two subproblems: contact planning and motion optimization~\citep{2023troBiConMP, 2020icraCrocoddyl, AmesAaronDecopule}. The goal of the contact planning stage is to generate the desired multi-contact sequence for rich whole-body motion and gait control, including the order and position of both hand and foot contacts~\citep{2016rasmomentumdynamics, 2015rasHRP-2humanoid}. The motion optimization phase optimizes the robot motion trajectory based on the contact sequence. Although decoupling simplifies the problem, model-based approaches still rely on several assumptions, including perfect state estimation and flawless execution of planned movements. However, most assumptions no longer hold in the real world, and the dynamics model is not perfect to describe real robot systems, which results in poor robustness when applied in real environments.

\subsection{Learning-Based Humanoid Controller}
% Humanoid robot control remains a long-standing challenge due to the complexity arising from high degrees of freedom and low stability. 

Recent advancements in learning-based controllers have demonstrated the locomotion capability to go through rough terrains~\citep{scironot2024humanoid, rss2024denoisingworldmodel}, achieving smooth and efficient motions~\citep{chen2024learning}.
However, controllers relying on proprioceptive sensing must predict surrounding terrain through collision detection and swiftly adapt their motion, presenting significant challenges for inherently unstable humanoid robotic systems.
Some recent approaches incorporated depth maps or elevation maps into the policy observations, enabling impressive parkour tasks~\citep{zhuang2024humanoid,long2024learning}.
Some researchers have utilized chain-contact reward functions to learn jumping gaits for humanoid robots~\citep{zhang2024wococo}. 

Additionally, with the support of teleoperation systems for humanoid robots~\citep{cheng2024tv, fu2024mobile} and large-scale humanoid motion datasets~\citep{mahmood2019amass}, researchers have made progress in motion tracking and learned rich whole-body motion representations for humanoid robots.
Some studies focused on upper body tracking combined with maintaining balance in the lower body~\citep{cheng2024expressive}. Some others explored controlling whole-body joints in one policy, differing primarily in their control interfaces/command spaces: \citeauthor{he2024learning} tracked whole-body motion capture keypoints; \citeauthor{fu2024humanplus,2024exbody2} track retargeted joint position; \citeauthor{he2024omnih2o} tracked VR-based head and hands keypoints; \citeauthor{he2024hover} tracked all of these and propose a universal interface approach. Different from them, \citeauthor{lu2024pmp} decoupled the control interface, and combined an IK-based upper-body controller with a learning-based lower-body controller. The lower-body command space includes the task command and the pose command as used in this work, and they introduced the prior knowledge of upper-body movements to the lower-body policy to help its robustness. However, we show that without such a component, we can still construct a robust loco-manipulation controller.

We made several choices in this work: 1) we extend the command space beyond all of these previous works, by introducing additional behavior commands that control the foot and the gait; 2) we employ a learning-based controller to control whole-body joints (instead of only lower-body as in \citeauthor{lu2024pmp}) while supporting external controller (with IK or joint sequences) to take over upper-body joints, since upper-body and lower-body serves as different requirements. Accurate upper-body control is useful for tasks that require precision, while the robot should be robust to arbitrary upper-body intervention under any behavior.
% However, these approaches offer only limited interface implementations, constraining the flexibility of humanoid robots in performing advanced tasks. 
% Furthermore, current controllers exhibit monotonous motion patterns, lacking the ability to generate diverse humanoid gaits. 
% In contrast to previous works, our controller supports multiple motion modes and versatile whole-body control in response to user commands, while providing versatile interfaces for executing loco-manipulation tasks. OLD VERSION
% In contrast to previous works, \our enables multiple gait modes and offers versatile whole-body control in response to user commands. It also provides direct external control interfaces for upper body motion without any loss in tracking performance, eliminating the need for intermediate tracking policies in previous methods.
