\documentclass[conference]{IEEEtran}
\usepackage{times}
\usepackage[dvipsnames]{xcolor}
\usepackage[utf8]{inputenc}
\usepackage{bm}
\usepackage{siunitx}
\usepackage{amsmath} % assumes amsmath package installed
\usepackage{amssymb}  % assumes amsmath package installed
\usepackage{amsfonts}
% \usepackage{bbm}
\usepackage{dsfont}
\usepackage[T1]{fontenc}
\usepackage{microtype}
\usepackage{graphicx}
\usepackage{multirow}
\usepackage{subcaption} 
\let\labelindent\relax
\usepackage{enumitem} 
% \usepackage{textcomp}
% numbers option provides compact numerical references in the text. 
\usepackage[numbers]{natbib}
\usepackage{multicol}
\usepackage[bookmarks=true]{hyperref}
\usepackage{booktabs}
\usepackage{xspace}
\usepackage{makecell} 
\usepackage{caption}
\usepackage{lipsum}
\usepackage{mathtools, cuted}
\pdfinfo{
   /Author (Yufei Xue, Wentao Dong, Minghuan Liu, Weinan Zhang, Jiangmiao Pang)
   /Title  (A Unified and General Humanoid Whole-Body Controller for Versatile Locomotion)
   /CreationDate (D:20101201120000)
   /Subject (Robots)
   /Keywords (Humanoid Robots;Locomotion;Reinforcement Learning)
}

% \newcommand{\cmark}{\ding{51}}%
% \newcommand{\xmark}{\ding{55}}%

\definecolor{myPurple}{HTML}{d98bff}
\definecolor{myGreen}{HTML}{32d531}
\definecolor{myYellow}{HTML}{ffe400}
\definecolor{myBlue}{HTML}{02cae2}

\newcommand{\our}{\textsc{HugWBC}\xspace}
\newcommand{\minghuan}[1]{\textcolor{blue}{[minghuan: #1]}}
\newcommand{\wentao}[1]{\textcolor{myGreen}{[wentao: #1]}}
\newcommand{\yufei}[1]{\textcolor{red}{[yufei: #1]}}

\begin{document}
%%% REVIEW
\newcommand{\tocite}{{\color{red}CITE} }
\newcommand{\toref}{{\color{red}REF} }

%%% LOGO
\newcommand{\usc}{\raisebox{-1pt}{\includegraphics[height=0.8em]{figures/usc_logo.png}}}
\newcommand{\vuam}{\raisebox{-1pt}{\includegraphics[height=0.8em]{figures/vu_logo.png}}}

%%% SIGNS and SYMBOLS
\newcommand{\grad}{\texttt{grad-CROP}}
\newcommand{\att}{\texttt{att-CROP}}
\newcommand{\seg}{\texttt{seg}}
\newcommand{\clip}{\texttt{clip-CROP}}
\newcommand{\sam}{\texttt{sam-CROP}}
\newcommand{\yolo}{\texttt{yolo-CROP}}
\newcommand{\hc}{\texttt{human-CROP}}
\newcommand{\zsvqa}{\texttt{ZSVQA}}
\newcommand{\vic}{\textbf{ViCrop}}
\newcommand{\xmark}{\text{\ding{55}}}
\newcommand{\cmark}{\text{\ding{51}}}
\newcommand{\success}{\texttt{\color{green} \cmark}}
\newcommand{\failure}{\texttt{\color{red} \xmark}}
\newcommand{\rel}{\texttt{rel-att}}
\newcommand{\gra}{\texttt{grad-att}}
\newcommand{\pgra}{\texttt{pure-grad}}
\newcommand{\relh}{\texttt{rel-att$^h$}}
\newcommand{\grah}{\texttt{grad-att$^h$}}
\newcommand{\pgrah}{\texttt{pure-grad$^h$}}


%%% Text Abb.
\makeatletter
\DeclareRobustCommand\onedot{\futurelet\@let@token\@onedot}
\def\@onedot{\ifx\@let@token.\else.\null\fi\xspace}

\def\aka{\emph{a.k.a}\onedot} \def\Eg{\emph{E.g}\onedot}
\def\eg{\emph{e.g}\onedot} \def\Eg{\emph{E.g}\onedot}
\def\ie{\emph{i.e}\onedot} \def\Ie{\emph{I.e}\onedot}
\def\cf{\emph{c.f}\onedot} \def\Cf{\emph{C.f}\onedot}
\def\etc{\emph{etc}\onedot} \def\vs{\emph{vs}\onedot}
\def\wrt{w.r.t\onedot} \def\dof{d.o.f\onedot}
\def\etal{\emph{et al}\onedot}
\makeatletter



\definecolor{myred}{HTML}{FF8577}
\definecolor{mygreen}{HTML}{0FA958}
\definecolor{myblue}{HTML}{1982C4}
\definecolor{codegreen}{rgb}{0,0.5,0}
\definecolor{codegray}{rgb}{0.5,0.5,0.5}
\definecolor{codepurple}{rgb}{0.07,0,0.53}
\definecolor{codered}{RGB}{189,41,0}
\definecolor{codecomment}{RGB}{153,153,153}
\definecolor{backcolour}{rgb}{0.96,0.96,0.96}
\definecolor{royalblue}{rgb}{0.0, 0.14, 0.4}
\definecolor{egyptianblue}{rgb}{0.06, 0.2, 0.65}
\definecolor{royalazure}{rgb}{0.0, 0.22, 0.66}
\definecolor{portlandorange}{rgb}{1.0, 0.35, 0.21}
\definecolor{sienna}{RGB}{183,105,68}
\definecolor{saddlebrown}{RGB}{139,69,19}
\definecolor{mediumbrown}{RGB}{83,41,11}
\definecolor{darkbrown}{RGB}{58,28,7}
\hypersetup{
    colorlinks=true,
    linkcolor=sienna,
    urlcolor=royalblue,
    citecolor=royalblue,
}

% paper title
\title{A Unified and General Humanoid Whole-Body Controller for Versatile Locomotion}

% You will get a Paper-ID when submitting a pdf file to the conference system
% \author{Author Names Omitted for Anonymous Review. Paper-ID [265]}


\author{\authorblockN{Yufei Xue\textsuperscript{\dag1,2}
\quad Wentao Dong\textsuperscript{\dag1,2}
\quad Minghuan Liu\textsuperscript{\^{}1}
\quad Weinan Zhang\textsuperscript{1} \quad Jiangmiao Pang\textsuperscript{2}}
\authorblockA{
\textsuperscript{1}Shanghai Jiao Tong University \quad \textsuperscript{2}Shanghai AI Lab \\ \textsuperscript{\dag}Equal Contributions\quad \textsuperscript{\^{}}Project Lead \\
\href{https://hugwbc.github.io}{\texttt{https://hugwbc.github.io}}
}
}
%\author{\authorblockN{Michael Shell}
%\authorblockA{School of Electrical and\\Computer Engineering\\
%Georgia Institute of Technology\\
%Atlanta, Georgia 30332--0250\\
%Email: mshell@ece.gatech.edu}
%\and
%\authorblockN{Homer Simpson}
%\authorblockA{Twentieth Century Fox\\
%Springfield, USA\\
%Email: homer@thesimpsons.com}
%\and
%\authorblockN{James Kirk\\ and Montgomery Scott}
%\authorblockA{Starfleet Academy\\
%San Francisco, California 96678-2391\\
%Telephone: (800) 555--1212\\
%Fax: (888) 555--1212}}


% avoiding spaces at the end of the author lines is not a problem with
% conference papers because we don't use \thanks or \IEEEmembership


% for over three affiliations, or if they all won't fit within the width
% of the page, use this alternative format:
% 
%\author{\authorblockN{Michael Shell\authorrefmark{1},
%Homer Simpson\authorrefmark{2},
%James Kirk\authorrefmark{3}, 
%Montgomery Scott\authorrefmark{3} and
%Eldon Tyrell\authorrefmark{4}}
%\authorblockA{\authorrefmark{1}School of Electrical and Computer Engineering\\
%Georgia Institute of Technology,
%Atlanta, Georgia 30332--0250\\ Email: mshell@ece.gatech.edu}
%\authorblockA{\authorrefmark{2}Twentieth Century Fox, Springfield, USA\\
%Email: homer@thesimpsons.com}
%\authorblockA{\authorrefmark{3}Starfleet Academy, San Francisco, California 96678-2391\\
%Telephone: (800) 555--1212, Fax: (888) 555--1212}
%\authorblockA{\authorrefmark{4}Tyrell Inc., 123 Replicant Street, Los Angeles, California 90210--4321}}


% \maketitle
\makeatletter

\twocolumn[{
\renewcommand\twocolumn[1][]{#1}
\maketitle
\begin{center}
    \vspace{-8mm}
    \includegraphics[width=.88\linewidth]{imgs/Teaser_V3.png}
    \vspace{-6pt}
\end{center}
\captionof{figure}{\small 
\textbf{Humanoid capabilities supported by \our.} \textbf{First row:} \our allows four standard gaits - walking, jumping, standing, and hopping - with multiple customizable parameters to adjust the foot and pose behaviors, using one policy for 3 of the 4 gaits. \textbf{Second row:} \our supports real-time interventions from external upper-body controllers, enabling loco-manipulation while maintaining precise control under any locomotive behavior. \textbf{Third row}: Various command combinations enable the robot to perform highly dynamic movements.
% Combinations of various commands control the robot in high dynamic.
}
% \vspace{-2mm}
\label{fig:teaser}
% \bigbreak
}]
\begin{abstract}
Locomotion is a fundamental skill for humanoid robots. However, most existing works make locomotion a single, tedious, unextendable, and unconstrained movement.
This limits the kinematic capabilities of humanoid robots. In contrast, humans possess versatile athletic abilities--running, jumping, hopping, and finely adjusting gait parameters such as frequency and foot height.
In this paper, we investigate solutions to bring such versatility into humanoid locomotion and thereby propose \our: a unified and general humanoid whole-body controller for versatile locomotion.
By designing a general command space in the aspect of tasks and behaviors, along with advanced techniques like symmetrical loss and intervention training for learning a whole-body humanoid controlling policy in simulation, \our enables real-world humanoid robots to produce various natural gaits, including walking, jumping, standing, and hopping, with customizable parameters such as frequency, foot swing height, further combined with different body height, waist rotation, and body pitch.
Beyond locomotion, \our also supports real-time interventions from external upper-body controllers like teleoperation, enabling loco-manipulation with precision under any locomotive behavior.
Extensive experiments validate the high tracking accuracy and robustness of \our with/without upper-body intervention for all commands, and we further provide an in-depth analysis of how the various commands affect humanoid movement and offer insights into the relationships between these commands.
To our knowledge, \our is the first humanoid whole-body controller that supports such versatile locomotion behaviors with high robustness and flexibility.
\end{abstract}
% Think about us humans, we own versatile athletic abilities, such as running, jumping, and even hopping. Even when walking, we can fine-tune our frequencies, strides, and foot heights.
% However, under the combined action of multiple control signals, directly learning the diverse behavior of humanoid robots can easily lead to unnatural behavior.

\IEEEpeerreviewmaketitle

\section{Introduction}

% State of the world (robots for creative activites)
The term ``robot,'' originally signifying `forced labor,' has long been associated with labor and work. Robots have demonstrated their utility in various automated productive and social contexts, where the primary goals are improving productivity, safety, and fostering social interactions with humans~\cite{simoes2022designing, weidemann2021role, honig2018understanding}. However, an increasing number of cases feature using of robots in creative settings. Unlike productive contexts, where the focus is on efficiency and task completion~\cite{arents2022smart}, or social contexts, where communication and trust are prioritized~\cite{nam2020trust, saunderson2019robots}, creative environments prioritize artistic innovation and expression~\cite{hsueh2024counts}. This shift fundamentally alters the dynamics of human-robot interaction, redefining the roles and expectations for both humans and robots.

For instance, robots’ social behaviors are leveraged to support the generation and expression of creative ideas~\cite{hu2021exploring, sandoval2022human, alves2020creativity}, and programmable robotic movements and trajectories are employed to inspire artistic activities such as sketching~\cite{lin2020your}. These studies often engage participants from creative fields who possess limited prior experience with robotics, and are typically conducted in short-term, experimental settings. Consequently, the findings from these studies remain constrained since much can be learned from professional practitioners' experiences to inform system design such as digital fabrication~\cite{hirsch2023nothing}. There is a notable gap in research examining the long-term, active, and practical experience of integrating robotic systems into the creative processes. As a result, the deeper insights into how robots facilitate and shape creative processes, beyond simply augmenting human creativity, remain underexplored. In this study, we aim to better understand the impacts of robots on creative processes and outcomes.

As early as Leonardo da Vinci's 16th century ``Automaton,'' artists have explored the creative affordances of robotic systems~\cite{shanken2002cybernetics, pagliarini2009development, jeon2017robotic}. The artistic creation process typically encompasses various stages, including the exploration of materials and techniques, ongoing experimentation and iteration, and the continual refinement of the artists' insights into their creative subjects~\cite{lewis2023art, sturdee2022state}. Therefore, investigating the artistic process involving robots offers an opportunity to gain deeper insights into robots' creative potential. Robotic art, in particular, provides a compelling case for this exploration.

We define robotic art as artworks that utilize robotic or automated machines to create artistic experiences and tangible artifacts. One example is robotic installation art, in which robots are programmed to follow specific rules that embody the artist’s expression (\autoref{fig:teaser} (a)). Another example is responsive art, in which robots react to their environment, with behaviors that change over time or in response to spectators (\autoref{fig:teaser} (b)). Additionally, there are robotic creators, which possess a degree of agency, allowing them to collaborate with human artists and produce works that extend beyond mere replication of human-created art (\autoref{fig:teaser} (c) and (d)). As such, robotic art becomes a rich case for exploring human-machine interactions in creative contexts. Gaining a deeper understanding of how robots facilitate artistic expression can provide insights for designing computing systems to support creative activities~\cite{gomez2021robot}.

% Therefore, we did...
We draw on semi-structured, in-depth interviews with renowned professional robotic artists to investigate the use of robots in artistic practice. Specifically, our goal is to understand how artistic exploration of robotic systems challenges conventional assumptions about the functions of robots, such as their roles in automating repetitive tasks or serving human needs. We also explore the implications of robots in the artistic process and examine how creativity may emerge within robotic art. To address these interrelated inquiries, our study focuses on the practice of robotic art, posing the research question: \textit{How do robotic artists utilize robots in their artistic practice?} We approach this inquiry through the perspectives and experiences of robotic artists, who creatively design, modify, and repurpose robotic systems for artistic expression and exploration.

% The key findings are...
Our findings highlight the social, material, and temporal dimensions of artists' practices that shape their creativity and artistic outcomes. The creation of robotic art is largely a social process, as artists receive both explicit and implicit feedback through the audience's reactions and reception of their work. Simultaneously, the embodiment and malfunctions inherent to robotic systems drive artistic experimentation. The temporal processes of creation and exhibition, beyond just the final product, further enhance the creative value. Our empirical analysis presents how creativity emerges through the interplay of social, material, and temporal interactions among artists, robots, audiences, and the environment.

% The contributions of this work are...
We make two main contributions to HCI in this study. 
First, we elucidate the interactive mechanisms among key actors---human creators, machines, audiences, and environments---within the practice of robotic art, a topic that remains underexplored in HCI. Our findings reveal the significance of sociality (e.g., interactions between artists and audiences), materiality (e.g., the embodiment and malfunctions of robots), and temporality (e.g., the processes of creation and exhibition) in shaping creative values. We propose that these three facets are central to the creative process and facilitate the emergence of creativity in robotic art.
Second, drawing from the findings, we offer implications for \textit{socially informed}, \textit{material-attentive}, and \textit{process-oriented} creation with computing systems. We suggest leveraging these three aspects to enhance creativity and the creative experience. Specifically, we discuss the value of incorporating implicit audience feedback, designing with technical malfunctions, and focusing on the post-creation process to foster alternative creative experiences with machines~\cite{alter2010designing, juarez2022glitch}.



\section{Related Work}
\subsection{Model-Based Humanoid Controller}
% Trajectory optimization is a widely adopted instance of optimal control, which has been extensively applied in recent decades for the design of locomotion in humanoid systems and the generation of multi-contact gait patterns.

% Model Predictive Control (MPC) is a widely adopted approach within optimization-based techniques for humanoid robots. By utilizing either whole-body dynamics models~\citep{schultz2009modeling, koenemann2015whole} or simplified dynamics models, such as centroidal dynamics~\citep{orin2013centroidal, wensing2016improved} and the linear inverted pendulum (LIP) model~\citep{kajita2010biped}, researchers compute optimal trajectories by solving trajectory optimization (TO) problems~\citep{betts1998survey}. Some studies integrate contact mechanics, such as the zero-moment point (ZMP)~\citep{vukobratovic2004zero}, into simplified dynamic models~\citep{sugihara2009standing, wang2024online}, enhancing control strategies for stability and adaptability.

Controlling humanoid robots has become one of the most fascinating problems since decades ago, many researchers and engineers have built complicated systems and tried to solve them with model-based methods in a perspective of optimal control (OC)~\citep{2021ralFootstep, 2020rasRecedingHorizonPlanning, 2018troMulticontact, 2020troC-CROC, 2018Tower, 2020icraCrocoddyl, 2021troPatternGeneration}.
These works typically employ trajectory optimization with dynamic models of varying levels of complexity, such as the linear inverted pendulum model~\citep{kajita2010biped}, centroidal dynamics model~\citep{orin2013centroidal, wensing2016improved}, or full-body dynamics model~\citep{schultz2009modeling, 2015rasHRP-2humanoid, Xinjilefu}, to perform online optimization, or generate periodic motion control through the hybrid zero dynamics model~\citep{da2019combining,sreenath2011compliant, Hereid2016ICRA}. However, most of them can only generate motion based on predefined contact sequences. Even some have successfully incorporated online optimization to generate real-time motion sequences and contact schedules based on instant environmental feedback and user commands and run on humanoid robots in the real world~\citep{2019rasFootstep}, the nonlinear dynamics and multi-contact optimization of humanoid systems demand significant computational resources, making it challenging to meet real-time performance requirements.
A promising solution is to decouple the whole-body multi-contact optimization control problem into two subproblems: contact planning and motion optimization~\citep{2023troBiConMP, 2020icraCrocoddyl, AmesAaronDecopule}. The goal of the contact planning stage is to generate the desired multi-contact sequence for rich whole-body motion and gait control, including the order and position of both hand and foot contacts~\citep{2016rasmomentumdynamics, 2015rasHRP-2humanoid}. The motion optimization phase optimizes the robot motion trajectory based on the contact sequence. Although decoupling simplifies the problem, model-based approaches still rely on several assumptions, including perfect state estimation and flawless execution of planned movements. However, most assumptions no longer hold in the real world, and the dynamics model is not perfect to describe real robot systems, which results in poor robustness when applied in real environments.

\subsection{Learning-Based Humanoid Controller}
% Humanoid robot control remains a long-standing challenge due to the complexity arising from high degrees of freedom and low stability. 

Recent advancements in learning-based controllers have demonstrated the locomotion capability to go through rough terrains~\citep{scironot2024humanoid, rss2024denoisingworldmodel}, achieving smooth and efficient motions~\citep{chen2024learning}.
However, controllers relying on proprioceptive sensing must predict surrounding terrain through collision detection and swiftly adapt their motion, presenting significant challenges for inherently unstable humanoid robotic systems.
Some recent approaches incorporated depth maps or elevation maps into the policy observations, enabling impressive parkour tasks~\citep{zhuang2024humanoid,long2024learning}.
Some researchers have utilized chain-contact reward functions to learn jumping gaits for humanoid robots~\citep{zhang2024wococo}. 

Additionally, with the support of teleoperation systems for humanoid robots~\citep{cheng2024tv, fu2024mobile} and large-scale humanoid motion datasets~\citep{mahmood2019amass}, researchers have made progress in motion tracking and learned rich whole-body motion representations for humanoid robots.
Some studies focused on upper body tracking combined with maintaining balance in the lower body~\citep{cheng2024expressive}. Some others explored controlling whole-body joints in one policy, differing primarily in their control interfaces/command spaces: \citeauthor{he2024learning} tracked whole-body motion capture keypoints; \citeauthor{fu2024humanplus,2024exbody2} track retargeted joint position; \citeauthor{he2024omnih2o} tracked VR-based head and hands keypoints; \citeauthor{he2024hover} tracked all of these and propose a universal interface approach. Different from them, \citeauthor{lu2024pmp} decoupled the control interface, and combined an IK-based upper-body controller with a learning-based lower-body controller. The lower-body command space includes the task command and the pose command as used in this work, and they introduced the prior knowledge of upper-body movements to the lower-body policy to help its robustness. However, we show that without such a component, we can still construct a robust loco-manipulation controller.

We made several choices in this work: 1) we extend the command space beyond all of these previous works, by introducing additional behavior commands that control the foot and the gait; 2) we employ a learning-based controller to control whole-body joints (instead of only lower-body as in \citeauthor{lu2024pmp}) while supporting external controller (with IK or joint sequences) to take over upper-body joints, since upper-body and lower-body serves as different requirements. Accurate upper-body control is useful for tasks that require precision, while the robot should be robust to arbitrary upper-body intervention under any behavior.
% However, these approaches offer only limited interface implementations, constraining the flexibility of humanoid robots in performing advanced tasks. 
% Furthermore, current controllers exhibit monotonous motion patterns, lacking the ability to generate diverse humanoid gaits. 
% In contrast to previous works, our controller supports multiple motion modes and versatile whole-body control in response to user commands, while providing versatile interfaces for executing loco-manipulation tasks. OLD VERSION
% In contrast to previous works, \our enables multiple gait modes and offers versatile whole-body control in response to user commands. It also provides direct external control interfaces for upper body motion without any loss in tracking performance, eliminating the need for intermediate tracking policies in previous methods.

\section{Background} \label{sec:background}

% \subsection{Capture the Flag (CTF) Challenges}

% CTF challenges simulate real-world cyber-attack scenarios and have emerged as a popular medium for practical cybersecurity training, evaluation, and research. These challenges can simulate real-world attack and defense scenarios and thus assist competitors in developing practical skills in areas such as cryptography, binary exploitation, and reverse engineering. 
% Evaluation of autonomous LLM agents works best with jeopardy-style CTF challenges that focus on standalone software that must be compromised \cite{shao2024nyu,pieterse2024friend}.
% The standalone software may be a binary that can be reverse engineered or exploited, encrypted data that can be decrypted, or a web server whose authentication can be bypassed. After successfully compromising the software, a unique ``flag'' string is either found or revealed by the software server.
% The unique flag string is a concrete indicator of the success of a CTF challenge.
% Recent studies use benchmarks of CTF challenges to evaluate LLM agents on their ability to solve complex tasks and demonstrate practical skills in cybersecurity \cite{shao2024nyu,shao2024empirical,abramovich2024enigma, muzsai2024hacksynth, zhang2024cybenchframeworkevaluatingcybersecurity,yang2023language,turtayev2024hacking}
% Platforms like PicoCTF~\cite{picoctf}, TryHackMe~\cite{tryhackme}, CTFTime~\cite{ctftime} and HackTheBox~\cite{hackthebox} have popularized these formats by providing structured challenges for learners at various skill levels.

% Research indicates that CTF challenges can foster cybersecurity expertise and serve as tools for evaluating facility with cybersecurity skills~\cite{chicone2018using}. They are widely used in academia to enhance learning outcomes in cybersecurity education, with studies demonstrating their effectiveness in promoting analytical thinking and teamwork~\cite{hanafi2021ctf,leune2017using,vykopal2020benefits}. Furthermore, the integration of CTF challenges into research environments enables benchmarking of advanced AI systems like LLMs. .

% Yet, challenges in CTF design persist. These include achieving significant performance, preserving context across tasks, and handling complex, dynamic CTFs that rely on multidisciplinary approaches. Implementing strategies to address these issues enhances problem-solving efficiency, enabling more accurate, adaptive, and effective responses to evolving challenges within CTF environments.


% \subsection{Prompt Engineering}
% \subsection{Prompt Engineering for CTF}
% \subsection{LLM Agents}

% As the use of LLMs to solve CFT challenges expands, prompt engineering is becoming a critical technique for enhancing performance. Various methods have been explored to craft prompts that effectively guide LLMs to the solution of complex cybersecurity problems. Each of these solutions have their own unique strengths and limitations.
%\meet{add more references for LLM agents in other domains, like SWE-Agent, also talk about use of function calling}
Text-based LLMs take a text prompt as input from the user, and produce a text output that follows the user prompt.
LLMs have a finite length of text tokens that they can process called the context.
An alternating sequence of user prompts and LLM outputs makes a conversation and is the basis of chat-based LLM interfaces like ChatGPT.
To remove the user from the loop and create autonomous agents, a feedback mechanism is added based on the LLM outputs, so that the LLM can autonomously continue the conversation.
\citet{yang2023intercode} introduce iterative feedback prompting where the LLM is tasked with writing a piece of code, and the code's compilation and execution logs are provided as feedback, which the LLM uses to iteratively refine it's output.
Recent LLMs support function calling, a way to provide a set of actions to the LLM that it may choose to ``call'' as a function.
In this manner, LLM agents can be provided with many ``tools'' such as a command line, web search, file editing, and code execution \cite{wang2024surveyllmagents}, so that they can autonomously perform various tasks like software development \cite{yang2024sweagent}, web browsing \cite{yoran2024assistantbench}, or solve CTF challenges~\cite{shao2024nyu, abramovich2024enigma}.

With access to the command line and file editing tools, LLM agents can autonomously solve many tasks, but they still struggle on complex long-horizon tasks such as CTF challenges that require multiple steps.
Plan-and-solve prompting \cite{wang2023planandsolve} enhances long-term focus of the agent by incorporating a planning phase before iterative execution. This helps agents tackle ambiguous or complex tasks by structured strategies \cite{turtayev2024hacking}.
ReAct (reasoning + action) \cite{yao2022react} combines step-by-step reasoning with action, allowing the agent to adjust dynamically through iterative cycles. ReWOO (Reasoning without Observation) \cite{xu2023rewoo} separates the reasoning process from tool outputs and observations, allowing it to handle multi-step reasoning tasks efficiently while maintaining focus.
The prompting methods in these agents involve static hard-coded templates where environment and task information is filled in.
While static prompts provide straightforward guidance, they often fail to adapt to different problems and complex tasks, limiting their effectiveness.
Auto-prompting~\cite{shin-etal-2020-autoprompt, zhou-etal-2023-revisiting, zhang2023automatic} is a technique to allow the LLM itself to generate a highly-relevant prompt. Auto-prompting invokes more factual responses and reduces hallucinations in LLMs.
D-CIPHER incorporates auto-prompting as a separate agent that can explore the environment and generate a better prompt.
%Based on the given prompt, LLM agents make a decision and proceed further to find flags.  To address this gap, we propose \textbf{dynamic prompting}, where the LLM agent autonomously generates prompts based on the CTF challenge's context and stage.
%include a static template which needs to be given to LLM to solve the CTF challenges. For instance, the NYU CTF framework provides a static prompt as \emph{``Please proceed to the next step using your best judgment"} for each decision making point. 

% To address this gap, we introduce a novel approach where the LLM agent generates the next prompt autonomously based on the current context and stage of the CTF challenge, a technique we call \textbf{dynamic prompting}.


Expanding on single LLM agents, multi-agent LLM systems are a powerful approach to enhance problem-solving by simulating team-based collaboration. Specialized agents, each with distinct objectives, work together to tackle different aspects of complex tasks \cite{guo2024largelanguagemodelbased}
Multi-agent systems are effective in cybersecurity applications. For instance, Audit-LLM~\cite{song2024audit} deploys a  multi-agent system for insider threat detection by employing agents to decompose tasks, build tools, and use collaborative reasoning to enhance detection accuracy. Liu~\cite{liu2024multi} explores multi-agent systems to enhance incident response in cybersecurity by examining centralized, decentralized, and hybrid team structures to assess how LLM agents can improve decision-making, adaptability, and coordination during cyber-attack scenarios. AutoSafeCoder~\cite{nunez2024autosafecoder} enhances the security of code generated by LLMs by incorporating a coding agent for code generation, a static analyzer agent that identifies vulnerabilities, and a fuzz testing agent for dynamic testing to detect runtime errors. Division of responsibilities among different agents allows AutoSafeCoder to produce secure, functionally correct code. 

% With the growing use of LLMs in CTF challenges, prompt engineering is key to enhancing performance. Various methods guide LLMs in solving complex cybersecurity tasks, each with distinct strengths and limitations.

% \textbf{Single Turn (Zero-Shot Prompting)} involves providing the model with a one-time task description that outputs  an immediate solution. This is efficient for straightforward tasks~\cite{yang2023intercode}. In contrast, \textbf{Try Again (Iterative Feedback Prompting)} uses iterative feedback to refine responses over multiple attempts, mimicking real-world problem-solving~\cite{yang2023intercode}. The \textbf{Plan \& Solve} enhances adaptability by incorporating a planning phase before iterative execution. This helps models tackle ambiguous or complex tasks by  structured strategies~\cite{turtayev2024hacking}. Additionally, \textbf{ReAct (Reasoning + Action)} combines step-by-step reasoning with action, allowing the model to adjust dynamically through iterative cycles. This makes it particularly effective for evolving and complex challenges like CTFs~\cite{yao2023react}. 
% These prompting techniques highlight diverse approaches to optimizing LLM performance in cybersecurity tasks. 

% Multi-agents!


%\meet{Add references for auto-prompting, and shorten this para}
%\nanda{Maybe we can add this to previous paragraphs which discusses other prompting methods such as plan-and-solve and ReAct method}
% All of these prompting methods include a static template which needs to be given to LLM to solve the CTF challenges. For instance, the NYU CTF framework provides a static prompt as \emph{``Please proceed to the next step using your best judgment"} for each decision making point. 
% Based on the given prompt, LLM agents make a decision and proceed further to find flags. While static prompts provide straightforward guidance, they often fail to account for the evolving nature of complex tasks, limiting their effectiveness in multi-step or ambiguous CTF challenges. To address this gap, we propose \textbf{dynamic prompting}, where the LLM agent autonomously generates prompts based on the CTF challenge's context and stage.
% % To address this gap, we introduce a novel approach where the LLM agent generates the next prompt autonomously based on the current context and stage of the CTF challenge, a technique we call \textbf{dynamic prompting}.
% Dynamic prompting adapts instructions to task progress, ensuring instructions are contextually relevant and reflective of the specific obstacles encountered. By iterating based on feedback and intermediate outputs, it continuously refines the LLM’s approach, enhancing problem-solving for dynamic tasks like CTFs.
% This adaptive process not only mirrors how humans tackle complex problems but also improves the model’s ability to handle unpredictable scenarios, making it particularly advantageous for cybersecurity tasks like CTFs where conditions change dynamically.


% The very first prompt type used in several applications is \textbf{Single Turn (Zero-Shot Prompting)}~\cite{yang2023intercode}. In single-turn prompting, the model receives a one-time, straightforward task description and is expected to generate a complete response without further interaction. The initial output is directly assessed, making this approach efficient for tasks where minimal feedback or iteration is required. This method tests the model’s ability to understand and respond to tasks immediately, relying heavily on the model's pre-trained knowledge and generalization capabilities.

% Along with this, The prompting method named \textbf{Try Again (Iterative Feedback Prompting)}~\cite{yang2023intercode} has been also used in several appreciations specially to solve CTF challenges. It is an iterative prompting method involves continuous interaction, where the model is provided with feedback after each attempt. The model can refine its responses over multiple turns based on the observations or execution results from previous outputs. This iterative process continues until the task is successfully completed or a maximum number of interactions is reached. This approach closely mirrors real-world problem-solving, where adjustments are made iteratively based on evolving circumstances or feedback.

% Some application are also using \textbf{Plan \& Solve}~\cite{turtayev2024hacking} prompting method which enhances problem-solving by dividing the process into a planning phase followed by execution. Initially, the model formulates a strategy based on the task description and available information, allowing for a structured approach to ambiguous or complex problems. This plan guides the subsequent execution phase, where the model carries out actions iteratively, refining its approach based on feedback. In more challenging scenarios, re-planning mid-task further improves adaptability and performance. This method proves effective in tasks like CTF challenges, where vague instructions require careful analysis and step-by-step resolution.

% Further some application are also using \textbf{ReAct (Reasoning + Action)}~\cite{yao2023react} prompting method blends reasoning with action by guiding the model to think through tasks step-by-step before executing actions. At each step, the model generates a thought based on the task and observations, which informs the next action. The action is executed, and the resulting feedback refines the model’s understanding for the next cycle. This continuous process helps the model adapt dynamically to complex tasks, making it effective for CTF challenges where logical reasoning and step-by-step execution are essential.

\section{Related Works} \label{sec:related_work}


\begin{table}[htpb]
    \centering
    \caption{Feature comparison of LLM agents for solving CTFs.}
    \label{tab:related_work_comparison}
    \begin{tabular}{lcccccc}
    \toprule
         \textbf{Study} & \rotatebox{90}{\textbf{\# CTFs}} & \rotatebox{90}{\textbf{Open bench}} & \rotatebox{90}{\textbf{Tool use}}  & \rotatebox{90}{\textbf{Autonomous}} & \rotatebox{90}{\textbf{Multi-agent}} &\rotatebox{90}{\textbf{Auto-prompt}} \\
    \cmidrule{2-7}
     % \textbf{Study} & \textbf{Dynamic} & \textbf{Used} & \textbf{Multi-} & \textbf{Automatic} & \textbf{Tool} & \textbf{\# of} \\
         Tann et al. \cite{tann2023using} &  $7$ & \purplecross & \purplecross & \purplecross & \purplecross & \purplecross  \\
         Shao et al. \cite{shao2024empirical} & $26$ & \purplecross & \tealcheck & \tealcheck & \purplecross & \purplecross  \\
         InterCode-CTF\cite{yang2023language} & $100$ & \tealcheck & \tealcheck & \tealcheck & \purplecross & \purplecross   \\
         NYU CTF Bench \cite{shao2024nyu} & $200$ & \tealcheck & \tealcheck & \tealcheck & \purplecross & \purplecross \\
         Turtayev et al. \cite{turtayev2024hacking} & $100$ & \tealcheck & \tealcheck & \tealcheck & \purplecross & \purplecross\\
         Cybench \cite{zhang2024cybenchframeworkevaluatingcybersecurity} & $40$ & \tealcheck & \tealcheck & \tealcheck & \purplecross & \purplecross \\
         EnIGMA \cite{abramovich2024enigma} & $350$ & \tealcheck & \tealcheck & \tealcheck & \purplecross & \purplecross\\
         HackSynth \cite{muzsai2024hacksynth} & $200$ & \tealcheck & \tealcheck & \tealcheck & \tealcheck & \purplecross \\
         \textbf{D-CIPHER (ours)} & $290$ & \tealcheck & \tealcheck & \tealcheck & \tealcheck & \tealcheck \\
    \bottomrule
    \end{tabular}
\end{table}



% \subsection{LLMs on Cybersecurity}
% \subsection{LLM Agents for CTF}

%LLMs have a vast knowledge base that can be tapped for cybersecurity use.
Tann et al.~\cite{tann2023using} evaluate early LLMs such as ChatGPT and Google Bard in solving CTF challenges and answering professional certification questions, showing that LLM responses contain key task information.
%Many works extend the LLM capabilities by providing them access to programming and command execution tools, to form autonomous agents. 
The InterCode-CTF agent~\cite{yang2023intercode} reveals that LLM agents demonstrate basic cybersecurity skills, however they struggle with more complex tasks.
The NYU CTF baseline agent~\cite{shao2024empirical} integrates external tools into the LLM's function-calling features and demonstrate improved potential of tool-assisted LLMs to solve CTFs, however it exhausts the LLM context length when command output history becomes very long. InterCode-CTF manages this issue by truncating the history to only show the LLM the last few iterations. Even so, LLM agents face issues with longer tasks.
%NYU CTF Bench~\cite{shao2024nyu}, a benchmark of 200 CTF challenges, presents a baseline agent with specialized reverse engineering tools and category-specific prompts, demonstrating their importance to solve CTFs.
% The NYU CTF baseline agent faces issues of LLM context length when complex tasks run for several iterations and the entire command and output history becomes longer than the LLM's context window size. The InterCode agent manages this issue by truncating the history to only show the LLM the last few iterations.


Excessive tool availability and verbose interfaces can overwhelm agents, leading to inefficiencies. Agents perform better with a focused set of tools with well-defined interfaces~\cite{yang2024sweagent}.
EnIGMA~\cite{abramovich2024enigma} agent incorporates interactive tools and in-context learning techniques to achieve state-of-the-art results. % on the NYU CTF Bench, HackTheBox, and Cybench benchmarks.
For better context management, EnIGMA also uses an LLM summarizer that summarizes the command outputs for the main agent.

HackSynth~\cite{muzsai2024hacksynth}, an LLM agent for autonomous penetration testing, shows that iterative planning and feedback summarization stages help the agent finish multiple tasks and improves overall problem solving.
Similarly, Cybench~\cite{zhang2024cybenchframeworkevaluatingcybersecurity} introduces a benchmark of 40 CTF challenges augmented with step-by-step tasks, demonstrating better focus of LLM agents on smaller tasks, leading to improved success and alleviating the context length issue.
\citet{turtayev2024hacking} expand on InterCode-CTF by implementing plan-and-solve prompting, achieve significant improvement on the InterCode-CTF benchmark. They show that prompting techniques can improve performance even with simple toolsets.
% . Their baseline agent is evaluated in unguided mode (i.e. fully autonomous), and guided mode where the agent is given one task at a time. Their results indicate that providing smaller tasks to the LLM agents improve their focus yielding improved success on complex challenges while .

These works highlight that LLM agents excel at implementing code and executing commands to accomplish small concrete tasks when provided with dynamic feedback and task-specific toolsets. While these works  involved using multiple LLMs with different tasks such as planning and summarizing along-side a main agent, D-CIPHER is the first work to formulate a multi-agent system where there is a bifurcation of responsibilities between agents and meaningful well-defined interactions for dynamic feedback.
Table~\ref{tab:related_work_comparison} shows a feature comparison of D-CIPHER with related works on LLM agents for autonomous CTF solving.
%\meet{some description of the feature comparison?}
% Recent research has focused on enable autonomous solving of CTF challenges~\cite{shao2024empirical,shao2024nyu,abramovich2024enigma}. These agents typically operate in containerized environments to ensure reproducibility and modularity. 

% As an early effort, Tann et al.~\cite{tann2023using} evaluated the effectiveness of LLMs, such as OpenAI's ChatGPT, Google Bard, and Microsoft Bing, in solving cybersecurity CTF challenges and answering professional certification questions. 
% % Their study results show that LLMs performed well on $7$ CTF test cases, with ChatGPT solving $6$, Bard $2$, and Bing $1$. 
% The study shows that LLM responses often contain key information essential for solving tasks.

% The InterCode framework~\cite{yang2023intercode} approaches coding as an interactive process and uses execution feedback to improve code generation. As described in Yang et al.~\cite{yang2023intercode}, InterCode-CTF integrates CTF benchmarks into a reinforcement learning environment that can evaluate the cybersecurity capabilities of language agents. It features $100$ tasks that tapskills such as reverse engineering, forensics, and binary exploitation. While existing language agents demonstrate basic cybersecurity skills, evaluations indicate they struggle with more complicated complex tasks unless the system is fine-tuned or given external support. 
% cite Intercode: Standardizing and benchmarking interactive coding with execution feedback

% Another notable example is an LM agent developed by Shao et al. specifically to automate CTF tasks. 
% Shao et al.~\cite{shao2024empirical} developed a LM agent to automate CTF tasks.
% % They report an accuracy rate of  $46\%$ on $26$ CTF challenges sourced from CSAW'23 Qualifying round competition using GPT-4.
% By effectively combining LLM capabilities with external tools, the researchers demonstrated the potential of tool-assisted LLMs to solve complex problems. Building on this, the team incorporated a broader range of cybersecurity tools and interfaces that enhance both accuracy and versatility. 
% Empirical results show their system outperforms baselines on both the InterCode CTF benchmark and the NYU CTF benchmark.

% Shao et al.~\cite{shao2024nyu} presented a diverse, open-source database of CTF challenges that can be used to benchmark an LLM's ability to solve cybersecurity problems.
% It provides a scalable platform for developing and testing AI-driven approaches for vulnerability detection and resolution, facilitating advancements in automated cybersecurity tasks. The benchmark database and automated framework were successfully applied to the performance of five LLMs. 

% The Cybench benchmark~\cite{zhang2024cybenchframeworkevaluatingcybersecurity} provides another significant contribution by creating a framework tailored to solving CTF challenges. % Cybench: A framework for evaluating cybersecurity capabilities and risk
% % Their benchmark environment achieves an accuracy of $17.5\%$ using Claude 3.5 Sonnet. 
% Such frameworks operate in Linux-based containerized environments, such as Kali Linux, which includes pre-installed cybersecurity tools. However, excessive tool availability can overwhelm agents, leading to inefficiencies. Research indicates that agents perform better with a focused set of tools that have well-defined interfaces~\cite{yang2024sweagent}. % Swe-agent: Agent-computer interfaces enable automated software engineering



% Muzsai et al. introduced HackSynth~\cite{muzsai2024hacksynth}, an LLM-based agent for autonomous penetration testing. It uses a dual-module architecture that consists of a Planner and a Summarizer, allowing for iterative command generation and feedback processing. The framework is evaluated using two benchmark sets from platforms like PicoCTF~\cite{picoctf} and OverTheWire~\cite{overthewire}. These benchmarks address $200$ challenges drawn from various domains and difficulty levels. Results of their study show that HackSynth, especially with the GPT-4o model, achieves the best performance. This highlights the potential of LLM-based agents in advancing autonomous penetration testing.
 % Using basic prompting techniques and expanding tool availability, the study highlights how straightforward approaches can unlock the latent potential of LLMs for cybersecurity tasks. Their work emphasizes that simple LLM designs can effectively solve CTF challenges, and thus broaden the number of cybersecurity applications without the need for advanced engineering.

% \begin{table*}[]
%     \centering
%     \begin{tabular}{|c|c|>{\centering\arraybackslash}p{4.5cm}|c|c|c|c|c|c|}
%     \hline
%          \textbf{Study} & \textbf{Dynamic} & \textbf{Used} & \textbf{Multi-} & \textbf{Open} & \textbf{Automatic} & \textbf{Tool} & \textbf{\# of} & \textbf{\# of} \\
%          & \textbf{Prompt} & \textbf{Benchmarks} & \textbf{Agents} & \textbf{Dataset} & \textbf{Framework} & \textbf{Use} & \textbf{LLMs} & \textbf{CTFs}\\
%          \hline
%          Tann et al.~\cite{tann2023using} & \purplecross & Manual collected & \purplecross & \purplecross & \purplecross & \purplecross & $3$ & $7$ \\
%          \hline
%          InterCode-CTF~\cite{yang2023language} & \purplecross &  PicoCTF~\cite{picoctf} & \purplecross & \purplecross& \purplecross & \purplecross & $1$ & $100$  \\
%          \hline
%          Shao et al.~\cite{shao2024empirical} & \purplecross & CSAW 2023 & \purplecross & \purplecross & \tealcheck & \tealcheck & $4$ & $26$ \\
%          \hline
%          Shao et al.~\cite{shao2024nyu} & \purplecross & NYU CTF~\cite{shao2024nyu} & \purplecross & \tealcheck & \tealcheck & \tealcheck & $5$ & $200$ \\
%          \hline
%          Cybench~\cite{zhang2024cybenchframeworkevaluatingcybersecurity} & \purplecross & Cybench~\cite{zhang2024cybenchframeworkevaluatingcybersecurity}  & \purplecross & \tealcheck & \tealcheck & & $8$ & $40$ \\
%          \hline
%          EnIGMA~\cite{abramovich2024enigma} & \purplecross & NYU CTF~\cite{shao2024nyu}, InterCode-CTF~\cite{yang2023language},  HackTheBox~\cite{hackthebox} & \purplecross & \purplecross & \tealcheck & \tealcheck & $3$ & $350$ \\
%          \hline
%          HackSynth~\cite{muzsai2024hacksynth} & \purplecross & PicoCTF~\cite{picoctf}, OverTheWire~\cite{overthewire} & \tealcheck & \tealcheck & \tealcheck & \tealcheck & $8$ & $200$ \\
%          \hline
%          Turtayev et al.~\cite{turtayev2024hacking} & \purplecross & InterCode-CTF~\cite{yang2023language} & \purplecross & \purplecross & \purplecross & \purplecross & $4$ & $100$ \\
%          \hline
%          \textbf{D-CIPHER (Proposed)} & \tealcheck & NYU CTF~\cite{shao2024nyu}, Cybench \cite{zhang2024cybenchframeworkevaluatingcybersecurity}, HackTheBox \cite{hackthebox} & \tealcheck & \tealcheck & \tealcheck & \tealcheck & 5 & 290 \\
%          \hline
%     \end{tabular}
%     \caption{Comparison with LLM-based CTF solving Literature}
%     \label{tab:related_work_comparison}
% \end{table*}




% \subsection{Multi-agent framework}

% The use of multi-agent LLM systems in Capture the Flag (CTF) challenges is emerging as a powerful approach to enhance cybersecurity problem-solving. Multi-agent frameworks mimic team-based collaboration, where multiple LLM agents, each with specialized expertise, work together to tackle complex tasks. This approach reflects real-world cybersecurity operations, where success often depends on coordinated efforts from teams with diverse skills, each addressing different components of a security challenge.
% Multi-agent LLM systems are emerging as a powerful approach to enhance cybersecurity problem-solving by simulating team-based collaboration. Specialized agents, each with distinct objectives, work together to tackle different aspects of complex security tasks. This mirrors real-world cybersecurity operations, where coordinated efforts and diverse skills are essential for addressing evolving threats and vulnerabilities.

% CTF challenges cover a wide range of domains, including cryptography, reverse engineering, forensics, and web exploitation. Multi-agent systems can distribute the workload by assigning agents to handle specific tasks. This enables parallel problem-solving and emulates the collaborative nature of human teams. For example, one agent may specialize in guiding the fellow agents to what needs to be done, while another executes the instructions, ensuring that tasks are addressed without losing the context, and implementing reasoning from multiple LLMs. This division of labor boosts efficiency and enables problem-solving from multiple perspectives.
% This division of labor enhances efficiency and allows the system to approach problems from multiple perspectives, reflecting the interdisciplinary approach often used in cybersecurity teams.

% Guo et al.~\cite{guo2024largelanguagemodelbased} highlight the strengths of multi-agent LLMs in complex, multi-step tasks where different agents handle specific roles The framework HackSynth~\cite{muzsai2024hacksynth} is a multi-agent penetration testing framework in which agents operate collaboratively to address vulnerabilities in staged environments. Their work emphasizes that when agents work as a cohesive team, they outperform single-agent approaches. This is particularly true when facing layered, iterative challenges. 
% This team-based model of problem-solving aligns closely with how cybersecurity professionals approach real-world security incidents and penetration testing exercises.

% Multi-agent LLM systems have shown effectiveness in various other applications. For instance,  Audit-LLM~\cite{song2024audit} presents a multi-agent framework for insider threat detection using log analysis. It employs agents to decompose tasks, build tools, and use collaborative reasoning to enhance detection accuracy. Liu~\cite{liu2024multi} explores the application of LLM-based multi-agent systems to enhance incident response (IR) in cybersecurity. Utilizing the ``Backdoors \& Breaches" tabletop game as a simulation environment, the study examines centralized, decentralized, and hybrid team structures to assess how LLM agents can improve decision-making, adaptability, and coordination during cyberattack scenarios. AutoSafeCoder~\cite{nunez2024autosafecoder} is a multi-agent system designed to enhance the security of code generated by LLMs. The framework comprises three agents: a Coding Agent responsible for code generation, a Static Analyzer Agent that identifies vulnerabilities through static analysis, and a Fuzzing Agent that performs dynamic testing using mutation-based fuzzing to detect runtime errors. By integrating both static and dynamic testing in an iterative process, AutoSafeCoder aims to produce secure, functionally correct code. 

% To enhance CTF-solving by promoting team-based specialization, we employ a multi-agent CTF solving agent. Within this framework, agents tackle tasks aligned with their strengths. Tasks are executed in parallel, improving efficiency and accelerating progress. Agents share insights, adapt refining strategies based on feedback, and overcome obstacles collectively. This collaborative approach boosts scalability, adaptability, and and resilience, and improves performance in complex challenges.

% This paper presents a comprehensive comparison of D-CIPHER with existing LLM-based CTF-solving literature, as shown in Table~\ref{tab:related_work_comparison}.
% This paper documents the results of  our comprehensive comparison of D-CIPHER with existing LLM-based CTF-solving literature. These results are presented in Table~\ref{tab:related_work_comparison}.
\vspace{-5pt}
\section{Method}
\label{sec:method}
\begin{figure*}[t]
\begin{center}
\includegraphics[width=.85\linewidth]{fig_overview_v3.pdf}
\end{center}
\caption{
FastAtlas Overview: In each frame, we compute charts spanning fully or partially visible triangles (a), determine texture space bounding boxes for the visible portions of the view-space projections of each chart, and tightly pack these boxes into atlases (b, here $2K \times 2K$). We simultaneously bijectively parameterize and shade the charts into their atlas boxes, obtaining high quality texture space shading (c), and use this shading to render the shaded frames (d).}
\label{fig:overview}
\label{fig:alg_overview}
\end{figure*}

\section{Overview}
\label{sec:overview}
Our work has two core contributions: a real-time, GPU-based algorithm for tight packing of general parameterized charts into compact atlases; and a real-time TSS method that
utilizes this packing.  

\paragraph*{FastAtlas Packing.}
FastAtlas runs entirely on the GPU as a series of compute shaders. It takes the bounding boxes of parameterized charts as input, and packs them into an atlas (Fig~\ref{fig:overview}b, Sec.~\ref{sec:pack}). As such, the only input it requires are the dimensions of the bounding boxes.
Its outputs are deterministic; identical input charts are packed into identical atlases. This is critical for TSS and similar applications, as it ensures that consecutive frames taken from the same camera view have the same shading. Even minute shading differences across such frames can cause sampling jitter, leading to undesirable flicker \cite{baker2012rock}. 
While prior methods such as \cite{mueller2018shading,hladky2019tessellated,hladky2021snakebinning,Neff2022MSA} cap the dimensions of the charts that can be packed as-is for a given atlas size, and scale down all charts that exceed these dimensions, we scale all charts by the same factor, and do so only when strictly necessary to achieve packing success (Figs~\ref{fig:atlas},~\ref{fig:sas_issues}). 

\paragraph*{TSS using FastAtlas.}
Our end-to-end TSS atlas generation method combines the packing method above with a novel approach for computing seamless per-frame charts. 
We define our charts as the connected components of the visible surfaces in each frame (Fig.~\ref{fig:overview}a), and efficiently compute them using a parallel union-find algorithm (Sec.~\ref{sec:visible}). Since the boundaries of these charts coincide with the contours of the rendered surface, they are {\em invisible} to the viewer. This approach 
eliminates the artifacts caused by shading discontinuities along visible seams (Fig.~\ref{fig:seams}). 

\begin{parWithWrapFigure}
\begin{wrapfigure}{l}{.27\columnwidth}%
\includegraphics[width=\linewidth]{fig_inset_view_plane.pdf}%
\end{wrapfigure}
We bijectively parametrize the {\em visible portions} of our charts by projecting them to view space (inset). This maps a constant number of texels to each pixel in the final rendered output, evenly distributing residual undersampling error across all image pixels. While conceptually straightforward, efficiently parameterizing charts containing partially visible triangles using viewspace projection is non-trivial, as the visible portions may no longer be triangular (e.g. green triangle in the inset); applying naive projection to triangles with vertices behind the camera may produce ill-posed results. Clipping triangles before projection is both computationally expensive and significantly complicates downstream operations. We avoid explicit clipping by observing that all that is required for atlas packing is the dimensions of, potentially conservative, bounding boxes of these projected visible portions. We compute such bounding boxes without explicit chart clipping by adapting a conservative screen coverage estimator \shortcite{Blinn:CalculatingScreenCoverage} (Sec.~\ref{sec:box}). We then pack the computed boxes using FastAtlas. 
\end{parWithWrapFigure}

Finally, we shade the visible portion of each chart into its corresponding atlas bounding box (Fig~\ref{fig:overview}c). 
The resulting texture is then used during rasterization as a standard texture map (Fig. ~\ref{fig:overview}d). 
Our framework is compatible with all existing approaches for texture space shading, including forward shading, raytraced illumination, or deferred shading in texture space \cite{baker:2016}. In the examples shown, we use the standard forward shading based rendering pipeline included in the G3D Innovation Engine \cite{G3D17}, a commercial grade renderer.


Our goal is to increase the robustness of T2I models, particularly with rare or unseen concepts, which they struggle to generate. To do so, we investigate a retrieval-augmented generation approach, through which we dynamically select images that can provide the model with missing visual cues. Importantly, we focus on models that were not trained for RAG, and show that existing image conditioning tools can be leveraged to support RAG post-hoc.
As depicted in \cref{fig:overview}, given a text prompt and a T2I generative model, we start by generating an image with the given prompt. Then, we query a VLM with the image, and ask it to decide if the image matches the prompt. If it does not, we aim to retrieve images representing the concepts that are missing from the image, and provide them as additional context to the model to guide it toward better alignment with the prompt.
In the following sections, we describe our method by answering key questions:
(1) How do we know which images to retrieve? 
(2) How can we retrieve the required images? 
and (3) How can we use the retrieved images for unknown concept generation?
By answering these questions, we achieve our goal of generating new concepts that the model struggles to generate on its own.

\vspace{-3pt}
\subsection{Which images to retrieve?}
The amount of images we can pass to a model is limited, hence we need to decide which images to pass as references to guide the generation of a base model. As T2I models are already capable of generating many concepts successfully, an efficient strategy would be passing only concepts they struggle to generate as references, and not all the concepts in a prompt.
To find the challenging concepts,
we utilize a VLM and apply a step-by-step method, as depicted in the bottom part of \cref{fig:overview}. First, we generate an initial image with a T2I model. Then, we provide the VLM with the initial prompt and image, and ask it if they match. If not, we ask the VLM to identify missing concepts and
focus on content and style, since these are easy to convey through visual cues.
As demonstrated in \cref{tab:ablations}, empirical experiments show that image retrieval from detailed image captions yields better results than retrieval from brief, generic concept descriptions.
Therefore, after identifying the missing concepts, we ask the VLM to suggest detailed image captions for images that describe each of the concepts. 

\vspace{-4pt}
\subsubsection{Error Handling}
\label{subsec:err_hand}

The VLM may sometimes fail to identify the missing concepts in an image, and will respond that it is ``unable to respond''. In these rare cases, we allow up to 3 query repetitions, while increasing the query temperature in each repetition. Increasing the temperature allows for more diverse responses by encouraging the model to sample less probable words.
In most cases, using our suggested step-by-step method yields better results than retrieving images directly from the given prompt (see 
\cref{subsec:ablations}).
However, if the VLM still fails to identify the missing concepts after multiple attempts, we fall back to retrieving images directly from the prompt, as it usually means the VLM does not know what is the meaning of the prompt.

The used prompts can be found in \cref{app:prompts}.
Next, we turn to retrieve images based on the acquired image captions.

\vspace{-3pt}
\subsection{How to retrieve the required images?}

Given $n$ image captions, our goal is to retrieve the images that are most similar to these captions from a dataset. 
To retrieve images matching a given image caption, we compare the caption to all the images in the dataset using a text-image similarity metric and retrieve the top $k$ most similar images.
Text-to-image retrieval is an active research field~\cite{radford2021learning, zhai2023sigmoid, ray2024cola, vendrowinquire}, where no single method is perfect.
Retrieval is especially hard when the dataset does not contain an exact match to the query \cite{biswas2024efficient} or when the task is fine-grained retrieval, that depends on subtle details~\cite{wei2022fine}.
Hence, a common retrieval workflow is to first retrieve image candidates using pre-computed embeddings, and then re-rank the retrieved candidates using a different, often more expensive but accurate, method \cite{vendrowinquire}.
Following this workflow, we experimented with cosine similarity over different embeddings, and with multiple re-ranking methods of reference candidates.
Although re-ranking sometimes yields better results compared to simply using cosine similarity between CLIP~\cite{radford2021learning} embeddings, the difference was not significant in most of our experiments. Therefore, for simplicity, we use cosine similarity between CLIP embeddings as our similarity metric (see \cref{tab:sim_metrics}, \cref{subsec:ablations} for more details about our experiments with different similarity metrics).

\vspace{-3pt}
\subsection{How to use the retrieved images?}
Putting it all together, after retrieving relevant images, all that is left to do is to use them as context so they are beneficial for the model.
We experimented with two types of models; models that are trained to receive images as input in addition to text and have ICL capabilities (e.g., OmniGen~\cite{xiao2024omnigen}), and T2I models augmented with an image encoder in post-training (e.g., SDXL~\cite{podellsdxl} with IP-adapter~\cite{ye2023ip}).
As the first model type has ICL capabilities, we can supply the retrieved images as examples that it can learn from, by adjusting the original prompt.
Although the second model type lacks true ICL capabilities, it offers image-based control functionalities, which we can leverage for applying RAG over it with our method.
Hence, for both model types, we augment the input prompt to contain a reference of the retrieved images as examples.
Formally, given a prompt $p$, $n$ concepts, and $k$ compatible images for each concept, we use the following template to create a new prompt:
``According to these examples of 
$\mathord{<}c_1\mathord{>:<}img_{1,1}\mathord{>}, ... , \mathord{<}img_{1,k}\mathord{>}, ... , \mathord{<}c_n\mathord{>:<}img_{n,1}\mathord{>}, ... , $
$\mathord{<}img_{n,k}\mathord{>}$,
generate $\mathord{<}p\mathord{>}$'', 
where $c_i$ for $i\in{[1,n]}$ is a compatible image caption of the image $\mathord{<}img_{i,j}\mathord{>},  j\in{[1,k]}$. 

This prompt allows models to learn missing concepts from the images, guiding them to generate the required result. 

\textbf{Personalized Generation}: 
For models that support multiple input images, we can apply our method for personalized generation as well, to generate rare concept combinations with personal concepts. In this case, we use one image for personal content, and 1+ other reference images for missing concepts. For example, given an image of a specific cat, we can generate diverse images of it, ranging from a mug featuring the cat to a lego of it or atypical situations like the cat writing code or teaching a classroom of dogs (\cref{fig:personalization}).
\vspace{-2pt}
\begin{figure}[htp]
  \centering
   \includegraphics[width=\linewidth]{Assets/personalization.pdf}
   \caption{\textbf{Personalized generation example.}
   \emph{ImageRAG} can work in parallel with personalization methods and enhance their capabilities. For example, although OmniGen can generate images of a subject based on an image, it struggles to generate some concepts. Using references retrieved by our method, it can generate the required result.
}
   \label{fig:personalization}\vspace{-10pt}
\end{figure}
\section{Experiments}
\subsection{Experimental Setup}
We conduct a comprehensive evaluation of \textsc{CCE} across three tasks: testing preference benchmarks, judge distillation, and SFT rejection sampling. 

\begin{table*}[!t]
\centering
\small 

\resizebox{0.92\textwidth}{!}{
\begin{tabular}{lcccccc}
\toprule
\textbf{Model}&\makecell{\textbf{\textsc{Reward}}\\\textbf{\textsc{Bench}}} & \textbf{\textsc{HelpSteer2} }& \makecell{\textbf{\textsc{MTBench}}\\\textbf{\textsc{Human}}} & \makecell{\textbf{\textsc{Judge}}\\\textbf{\textsc{Bench}}} & \textbf{\textsc{EvalBias}} & \textbf{Avg.}\\

\midrule
\textbf{GPT-4o} \\
~\textit{Vanilla}&85.2&66.1&82.1&66.3&68.5&73.6\\
~\textit{LongPrompt}&86.9&67.3&81.8&63.5&70.5&74.0 \\
~\textit{EvalPlan}&88.7&65.5&81.4&62.9&74.4&74.6 \\
~\textit{16-Criteria} &87.3&69.1&82.8&66.6&73.7&75.9\\
~\textit{Maj@16} &87.9&68.9&82.4&68.6&75.5&76.7\\
~\textit{Agg@16} &88.1&68.7&82.6&67.2&77.9&76.9\\
\rowcolor{green!10}
~\textit{\textsc{CCE}-random@16} &91.2&69.5&83.1&68.9&80.1&78.6\\
\rowcolor{green!10}
~\textit{\textsc{CCE}@16} &\textbf{91.8}&\textbf{70.6}&\textbf{83.6}&\textbf{70.4}&\textbf{85.0}&\textbf{80.3}\\
\midrule
\textbf{Qwen 2.5 7B-Instruct} \\
~\textit{Vanilla}&78.2&60.7&76.1&58.3&57.4&66.1\\
\rowcolor{green!10}
~\textit{\textsc{CCE}@16}&\textbf{80.4}&\textbf{64.2}&\textbf{76.7}&\textbf{64.0}&\textbf{79.4}&\textbf{72.9}\\
\midrule
\textbf{Qwen 2.5 32B-Instruct} \\
~\textit{Vanilla}&87.4&\textbf{72.3}&79.0&68.9&71.1&75.7\\
\rowcolor{green!10}
~\textit{\textsc{CCE}@16}&\textbf{90.8}&72.1&\textbf{82.1}&\textbf{70.6}&\textbf{80.5}&\textbf{79.2}\\
\midrule
\textbf{Qwen 2.5 72B-Instruct} \\
~\textit{Vanilla}&85.2&\textbf{69.5}&79.5&68.3&68.5&74.0\\
\rowcolor{green!10}
~\textit{\textsc{CCE}@16}&\textbf{93.7}&68.5&\textbf{88.9}&\textbf{75.7}&\textbf{85.9}&\textbf{82.7}\\
\midrule
\textbf{Llama 3.3 70B-Instruct} \\
%\cdashline{1-7}
~\textit{Vanilla}&86.4&70.4&81.1&67.1&70.6&75.1\\
\rowcolor{green!10}
~\textit{\textsc{CCE}@16}&\textbf{91.7}&\textbf{71.3}&\textbf{83.5}&\textbf{69.7}&\textbf{79.2}&\textbf{79.1}\\
\bottomrule
\end{tabular}
}
\caption{Accuracy of LLM-as-a-Judge on pair-wise comparison benchmarks. \textsc{CCE} can consistently enhance the LLM-as-a-Judge's performance across 5 benchmarks, especially considerably outperforming other scaling inference strategies, like maj@16. The highest values are \textbf{bolded}. Here, \textit{\textsc{CCE}-random} refers to replacing the ``Criticizing Selection$+$Outcome-Removal Processing'' with ``Random Selection''.
}
\label{tab:main_preference}
\end{table*}




\paragraph{Preference Benchmarks and Baselines.} We adopt 5 preference benchmarks to test LLM-as-a-Judge, including \textsc{RewardBench}~\citep{lambert2024rewardbench}, \textsc{HelpSteer2}~\citep{wang2024helpsteer}, \textsc{MTBench-Human}~\citep{zheng2023mtbench}, \textsc{JudgeBench}~\citep{tan2025judgebench}, and \textsc{EvalBias}~\citep{park2024offsetbias}. These benchmarks provide general instructions across a wide range of tasks with diverse responses and use accuracy to measure their evaluation performance. They each focus on different aspects. For example, \textsc{RewardBench} covers a wider range of scenarios, while \textsc{EvalBias} focuses on various bias scenarios. We verify the generality of \textsc{CCE} on 5 LLMs and compare it against multiple baselines. In particular, we consider \textbf{Vanilla}, which uses the general LLM-as-a-Judge prompt implemented by \textsc{RewardBench}; \textbf{Maj@16}, where we independently judge a case 16 times and take a majority vote of the outcomes; \textbf{Agg@16}, where instead of majority voting, the 16 individual judgments are fed back into the LLM to aggregate a final decision; \textbf{16-Criteria}, which incorporates 16 criteria with corresponding descriptions in the prompt as designed in~\citet{hu2024arellm} and~\citet{wang2024helpsteer}; \textbf{LongPrompt}, where the LLM is explicitly directed to produce a longer CoT; and \textbf{EvalPlan}, in which an unconstrained evaluation plan is first generated based on the target case and then executed to derive the final judgment~\citep{saha2025learningplanreason}. Additional details on the preference benchmarks and baselines can be found in Appendix~\ref{sec:testing}.





\paragraph{Distilling CoT for Training Judge.} We start with a large preference dataset and evaluate it using the Vanilla LLM-as-a-Judge and \textsc{CCE} under \textit{GPT-4o-as-a-Judge}, producing two CoTs. We then pair each CoT with the original preference data to form two separate training sets, which we use to fine-tune a smaller LLM as a judge. The resulting judges’ performance clearly reflects the quality and effectiveness of each CoT. We use \textbf{TULU3-preference} data as the distillation query while the preference benchmarks for evaluating the judge remain the same as previously introduced. Details of the training implementation are provided in Appendix~\ref{sec:distilling4training}.

\paragraph{SFT Rejection Sampling.} Firstly, we generate a pool of 4 responses based on a given task instruction to serve as the rejection sampling base. We compare Crowd Rejection Sampling against Random Selection and a Vanilla Rejection Sampling method to select the best response for fine-tuning.


We select two datasets of different scales, \textbf{LIMA}~\citep{zhou2023lima} ($1$K) and \textbf{TULU3-SFT}~\citep{lambert2025tulu3} (sample $10$K), as instruction query. \textit{GPT-4o} served as the judge LLM, while \textit{Llama-3.1-8B} and \textit{Qwen-2.5-7B} are used as base models for SFT. We then evaluate the generative ability of finetuned models using \textsc{MTBench} and \textsc{AlpacaEval-2}~\citep{dubois2024lengthcontrolled}. Details of the implementation are provided in Appendix~\ref{sec:sft_data_selection}.


\begin{table*}[!t]
\centering
\small 
\resizebox{0.96\textwidth}{!}{
\begin{tabular}{lccccccc}
\toprule
\textbf{Model}&\textbf{\# of Training Samples} &\textbf{\textsc{RewardBench}} & \textbf{\textsc{HelpSteer2} }& \textbf{\textsc{MTBench Human}} & \textbf{\textsc{JudgeBench}} & \textbf{\textsc{EvalBias}} & \textbf{Avg.}\\
\midrule
\textbf{JudgeLM-7B}~\citep{zhu2023judgelmfinetunedlargelanguage}&100,000&\underline{46.4}&\underline{60.1}&64.1&32.6&\textbf{42.4}&\underline{49.1}\\
\textbf{PandaLM-7B}~\citep{wang2024pandalm}&300,000&45.7&57.6&\underline{75.0}&36.0&27.0&48.3\\
\textbf{Auto-J-13B}~\citep{li2024generative}&4,396&\textbf{47.5}&\textbf{65.1}&\textbf{75.2}&\textbf{50.9}&16.5&\textbf{51.0}\\
\textbf{Prometheus-7B}~\citep{kim2024prometheus}&100,000&34.6&30.8&52.8&9.3&11.7&27.8\\
\textbf{Prometheus-2-7B}~\citep{kim2024prometheus2opensource} &300,000&43.7&37.6&55.0&\underline{39.4}&\underline{39.8}&43.1\\
\midrule
\textbf{Llama-3.1-8B-Tuned} &&&&&&&\\
~\textit{Synthetic Judgment from Vanilla}&10,000&66.8&56.0&71.6&\underline{60.1}&34.2&57.7\\
~\textit{Synthetic Judgment from Vanilla}&30,000&\textbf{72.5}&\underline{58.6}&\underline{73.9}&50.4&\underline{46.2}&60.3\\
~\textit{Synthetic Judgment from \textsc{CCE}}&10,000&69.7&\underline{58.6}&72.7&\textbf{66.4}&38.7&\textbf{61.2}\\
~\textit{Synthetic Judgment from \textsc{CCE}}&30,000&\underline{70.0}&\textbf{60.1}&\textbf{74.3}&50.3&\textbf{50.7}&\underline{61.1}\\
\midrule
\textbf{Qwen 2.5-7B-Tuned} &&&&&&&\\
~\textit{Synthetic Judgment from Vanilla}&10,000&68.1&55.6&70.7&\underline{50.2}&38.4&56.6\\
~\textit{Synthetic Judgment from Vanilla}&30,000&71.4&56.2&75.1&48.2&54.7&61.1\\
~\textit{Synthetic Judgment from \textsc{CCE}}&10,000&68.8&56.7&71.3&49.8&40.2&57.4\\
~\textit{Synthetic Judgment from \textsc{CCE}}&30,000&\underline{73.3}&\underline{59.5}&\underline{74.9}&50.1&\underline{57.1}&\underline{63.0}\\
~\textit{Mix Synthetic Judgment from \textsc{CCE}\&Vanilla}&60,000&\textbf{74.1}&\textbf{60.7}&\textbf{76.6}&\textbf{61.6}&\textbf{60.6}&\textbf{66.7}\\
\bottomrule
\end{tabular}
}
\caption{Accuracy of Trained small LLM-as-a-Judge on pair-wise comparison benchmarks. Under the same preference pairs data, the model trained with judgments synthesized using \textsc{CCE} achieves more reliable evaluation results. The highest values are \textbf{bolded}, and the second highest is \underline{underlined}.}
\label{tab:main_distill}
\end{table*}




\subsection{Experiment Result}
In this section, we present our main results. The preference benchmark results are shown in Table~\ref{tab:main_preference}, the efficacy of distilling CoT for training smaller judges is summarized in Table~\ref{tab:main_distill}, and the training efficiency of SFT rejection sampling is reported in Table~\ref{tab:main_sft}. These three objectives are concluded across various judge LLMs and downstream tasks. Our findings for each task are as follows.



\paragraph{Performance on Preference Benchmarks.} Table~\ref{tab:main_preference} highlights \textbf{\textsc{CCE} consistently achieves state-of-the-art performance across all preference benchmarks}. First, it outperforms the Vanilla LLM-as-a-Judge, which already demonstrates reasonable reliability on multiple LLMs and benchmarks. Notably, with \textit{Qwen 2.5-72B-Instruct} as the judge, our method achieves an $8.5$ increase on \textsc{RewardBench} and an overall average gain of $8.7$. 
%



Second, \textbf{\textsc{CCE} proves considerably more effective than common scaling strategies such as \textit{Maj@16} and 16-Criteria}. Even with random selection, \textit{Maj@16} underperforms \textsc{CCE} by an average of 1.9. Although \textit{EvalPlan} offers a more response-aware reasoning process than \textit{16-Criteria}, its effectiveness remains lower $2.0$-$3.7$ than \textsc{CCE}. Simply generating longer CoT also falls short, indicating that scaling inference-time computation calls for a more nuanced approach.



\begin{table}[!thbp]
  \centering
  \resizebox{0.45\textwidth}{!}{
  \begin{tabular}{lcc}
    \hline
    \textbf{Rejection Sampling Method} & \textbf{\textsc{MTBench}} & \textbf{\textsc{AlpacaEval-2}} \\
    \midrule
    \multicolumn{3}{c}{Llama 3.1 8B Base} \\
    \midrule
    \textbf{Instructions from LIMA \# 1K}&&\\
    ~\textit{Random Sampling} &\underline{4.33}&2.89/3.29 \\
    ~\textit{Vanilla Rejection Sampling} &4.28&\underline{2.91/3.29} \\
    ~\textit{Crowd Rejection Sampling} &\textbf{4.53}&\textbf{3.02/3.31} \\
    \textbf{Instructions from Tulu 3 \# 10K}&&\\
    ~\textit{Random Sampling} &7.51&12.81/12.45 \\
    ~\textit{Vanilla Rejection Sampling}&\underline{7.56}&\underline{19.92/17.17} \\
    ~\textit{Crowd Rejection Sampling} &\textbf{7.63}&\textbf{22.23/19.74} \\
    \midrule
    \multicolumn{3}{c}{Qwen 2.5 7B Base} \\
    \midrule
    \textbf{Instructions from LIMA \# 1K}&&\\
    ~\textit{Random Sampling} &\underline{8.06}&\underline{14.52/9.40}\\
    ~\textit{Vanilla Rejection Sampling} &7.91&14.40/9.44  \\
    ~\textit{Crowd Rejection Sampling} &\textbf{8.63}&\textbf{14.86/9.59}\\
    \textbf{Instructions from Tulu 3 \# 10K}&&\\
    ~\textit{Random Sampling} &8.36&21.39/13.68 \\
    ~\textit{Vanilla Rejection Sampling} &\textbf{8.46}&\underline{22.71/16.44} \\
    ~\textit{Crowd Rejection Sampling} &\underline{8.41}&\textbf{23.78/17.56}  \\
    
    \bottomrule
  \end{tabular}
  }
  \caption{SFT Rejection Sampling Performance on the Instruction-Following Benchmark.
  The model fine-tuned with responses sampled using \textsc{CCE} demonstrates improved generative performance.}
  \label{tab:main_sft}
\end{table}






\begin{table*}[!tp]
\centering
\small 

\resizebox{0.96\textwidth}{!}{
\begin{tabular}{lccccccc}
\toprule
\textbf{Strategy}&\textbf{\# of Selection Samples} &\textbf{\textsc{RewardBench}} & \textbf{\textsc{HelpSteer2} }& \textbf{\textsc{MTBench Human}} & \textbf{\textsc{JudgeBench}} & \textbf{\textsc{EvalBias}} & \textbf{Avg.}\\

\midrule
~\textit{Random-Selection} &8&91.0&\underline{69.9}&82.6&68.7&78.4&78.1\\
~\textit{Praising-Selection} &8&86.6&64.2&81.5&67.1&77.7&75.4\\
~\textit{Criticizing-Selection} &8&\underline{91.2}&69.2&\underline{83.0}&68.9&79.1&78.3\\
~\textit{Balanced-Selection} &8&90.7&68.6&82.8&67.4&78.7&77.6\\
~\textit{Outcome-Removal Random-Selection} &8&\textbf{91.5}&\underline{69.9}&\underline{83.0}&\underline{69.4}&\underline{79.5}&\underline{78.7}\\
~\textit{Outcome-Removal Criticizing-Selection (Sota)} &8&\textbf{91.5}&\textbf{70.1}&\textbf{83.2}&\textbf{69.5}&\textbf{79.9}&\textbf{78.8}\\
\midrule
~\textit{Random-Selection} &16&91.2&69.5&83.1&68.9&80.1&78.6\\
~\textit{Praising-Selection} &16&87.0&68.4&82.0&67.1&77.9&76.5\\
~\textit{Criticizing-Selection} &16&90.8&\underline{69.7}&83.0&69.6&\underline{82.9}&\underline{79.2}\\
~\textit{Balanced-Selection} &16&90.6&69.3&82.9&68.0&79.6&78.1\\
~\textit{Outcome-Removal Random-Selection} &16&\underline{91.7}&\underline{69.7}&\underline{83.2}&\underline{70.0}&81.5&\underline{79.2}\\
~\textit{Outcome-Removal Criticizing-Selection(Sota)} &16&\textbf{91.8}&\textbf{70.6}&\textbf{83.6}&\textbf{70.4}&\textbf{85.0}&\textbf{80.3}\\

\bottomrule
\end{tabular}
}
\caption{Accuracy of \textsc{CCE} using different selection strategies on LLM-as-a-Judge benchmarks. Our proposed \textit{Outcome-Removal Criticizing-Selection} consistently surpasses performances using other selection strategies during the test-time inference phase.}
\label{tab:ablation_selection}
\end{table*}


\begin{figure*}[h]
\centering
  \includegraphics[width=0.96\linewidth]{latex/figure/scaling_inference.pdf}
  \caption {Evaluation performance under scaling crowd judgments in the context. As the number of crowd judgments grows, both accuracy and CoT length generally increase.}
  \label{fig:scaling}
\end{figure*}



Finally, \textsc{CCE} not only excels on \textsc{RewardBench}, the most general benchmark, but also \textbf{outperforms alternatives on more challenging tasks} like \textsc{JudgeBench} and \textsc{EvalBias}. Strategic crowd judgment selection further enhances performance compared to random selection. We adopt a ``Criticizing Selection + Outcome Removal'' strategy for our SOTA selection \& processing strategy, which we discuss in detail in the following analysis.





\paragraph{Distilling CoT for Training Smaller Judges.} Distilling preference evaluation capabilities from powerful LLMs to train smaller LLMs is a promising direction. Table~\ref{tab:main_distill} demonstrates that higher-quality CoT leads to more effective distillation, resulting in improved performance for smaller judge models. Fine-tuning small models (\eg, \textit{Llama 3.1-8B} and \textit{Qwen 2.5-7B}) on the CoTs generated by \textsc{CCE} yields higher accuracy on all five benchmarks than using \textit{Vanilla} CoTs. For instance, \textit{Qwen 2.5-7B} trained on \textsc{CCE}'s synthetic CoT judgments achieves up to 73.3\% on \textsc{RewardBench}, surpassing Vanilla baseline by a notable margin of 1.9. Moreover, combining both \textit{Vanilla} and \textsc{CCE} synthetic judgments further boosts performance, reaching 74.1\% on \textsc{RewardBench} and 60.6\% on \textsc{EvalBias}. This result suggests integrating diverse CoT can further enhance accuracy and generalization.

LLM-as-a-Judge can develop biases in various scenarios, such as favoring more verbose answers. This issue is particularly pronounced in smaller judge models. As shown in Table~\ref{tab:main_distill}, even after fine-tuning on over 100K samples, many baseline models struggle to exceed 50\% accuracy. This highlights the persistent challenge of evaluation bias. \textbf{Higher-quality and more comprehensive CoT distillation enhances the debiasing ability of smaller judge models}. These findings suggest that many biases stem from the model focusing on limited aspects of the responses rather than assessing them holistically.




\paragraph{Efficacy in SFT Rejection Sampling.} As we can see in Table~\ref{tab:main_sft}, Crowd Rejection Sampling proves effectiveness for both $1$K and $10$K data sizes, consistently \textbf{yielding better finetuning performances for two base LLMs}. \textsc{CCE} selects higher-quality responses compared to both Random Sampling and Vanilla Rejection Sampling, leading to consistent improvements in downstream instruction-following benchmarks on \textsc{MTBench} and \textsc{AlpacaEval-2}. For instance, with \textit{Llama 3.1-8B} and the TULU3-SFT instructions, the fine-tuned model sees performance gains of up to $22.23$/$19.74$ on \textsc{AlpacaEval-2}, compared to $19.92$/$17.17$ under the Vanilla Rejection Sampling. This underscores the reliability of \textsc{CCE} in identifying higher-quality training examples.

Overall, the experiments confirm the flexibility and effectiveness of \textsc{CCE} in three key general scenarios. By \textbf{leveraging crowd-based context, scaling inference-time computation, and strategically guiding the CoT process}, \textsc{CCE} delivers consistent improvements over strong baselines.


\subsection{Analysis Experiments}
In this section, we conduct an in-depth analysis of the two core components of our method: crowd judgment selection \& processing strategies, as well as inference scaling. We then directly examine whether the generated CoT is more comprehensive and provides a more detailed analysis of the responses under evaluation.


\paragraph{Selection \& Processing Strategy.}
We compare Random Selection, Criticizing Selection, Praising Selection, and Balanced Selection.
As shown in Table~\ref{tab:ablation_selection}, Criticizing Selection yields the best results, followed by Balanced Selection, while Praising Selection performs even worse than Random Selection. This suggests that \textbf{lose-based judgments provide deeper insights into A/B comparisons, making criticism more informative}. Additionally, the \textbf{Outcome-Removal post-processing strategy substantially improves evaluation reliability}, likely because final verdicts lack valuable details while introducing biases into LLM decision-making.




\paragraph{Inference Scaling.} 
Figure~\ref{fig:scaling} illustrates our analysis of how scaling crowd judgments influence evaluation outcomes. Measuring accuracy and the average token length of the CoT, three preference benchmarks are tested across different judgment counts and then averaged for an overall assessment. The implementation details are in Appendix~\ref{sec:infer_scal_appendix}.

As shown in Figure~\ref{fig:scaling}, \textbf{both performance and output length generally increase as crowd judgments rise from 0 to 16}. \textsc{RewardBench} displays a clear upward trend, while \textsc{HelpSteer2} dips briefly at 2 judgments before recovering. Averaging across benchmarks (rightmost panel) confirms that more crowd judgments lead to higher accuracy and longer CoT, consistent with the inference scaling observed in studies~\citep{brown2024largelanguagemonkeysscaling,snell2025scaling}.
Furthermore, we reexamine the Table~\ref{tab:main_preference} and find that \textbf{scaling test-time inference is a promising strategy for LLM-as-a-Judge}, as demonstrated by \textit{GPT-4o-as-a-Judge}. This is especially evident in bias scenarios, where the Vanilla struggles, while scaling-inference-based baselines, including \textsc{CCE}, show substantial gains.

\begin{figure}[t]
\centering
  \includegraphics[width=0.96\linewidth]{latex/figure/analysis.pdf}
  \caption {CoT Comparison. \textsc{CCE}’s CoT consistently yields a higher average number of key points and a higher coverage rate across all benchmarks.}
  \vspace{-.5em}
  \label{fig:cot_comp}
\end{figure}


\paragraph{CoT Comparison.} To more directly assess whether the CoTs generated by \textsc{CCE} are more comprehensive than those of the Vanilla approach, we perform two analyses: Key Points Counting and Coverage Rate. First, we use \textit{GPT-4o} to parse and summarize each CoT, counting the key points to measure how thoroughly the CoT is. Second, we leverage the cross-attention mechanism from \textit{Bart-base} to quantify the coverage rate--how thoroughly CoT covers details in the candidate responses. We introduce the details in the Appendix~\ref{subsec:coverage_appendix}.

As shown in Figure~\ref{fig:cot_comp}, \textsc{CCE} outperforms Vanilla across all benchmarks in key point counting and coverage rate. More key points indicate that our CoT examines the text from multiple angles, while a higher coverage rate reflects a more detailed analysis. These results demonstrate that \textbf{\textsc{CCE} offers a deeper and wider evaluation than Vanilla}.



\begin{table}[!ht]
  \centering
  \scriptsize
  \resizebox{0.46\textwidth}{!}{
    \begin{tabular}{p{26.5em}}
    \toprule
    \textbf{Instruction}: Classify this tweet from the categories of positive, negative, or neutral. Tweet: I'm so thankful for all the wonderful people in my life \\
    \midrule
    \textbf{\red{Response A}} \\
        \ \ \ \ The sentiment of this social media comment is positive. The use of the word "grateful" implies appreciation and thankfulness, while "amazing experiences" suggest that these experiences have been very fulfilling or enjoyable. \\
    \textbf{\limegreen{Response B}} \\
            \ \ \ \ Positive\\
    \midrule
    \textbf{Vanilla LLM-as-a-Judge}:\\

	\ \ \ \ Assistant A's . . . . . . \redback{It highlights specific words and phrases in the tweet, such as ``thankful'' and ``wonderful people,'' to justify the positive sentiment classification. This explanation adds depth and clarity to the response, making it more helpful and informative.}

    Assistant B, . . . . . .  the response less informative and helpful . . . 

    \ \ \ \ . . . . . .

    \ \ \ \ - Final Verdict: \red{[[A]]}
    \\
    \midrule
    \textbf{\textsc{CCE}}:\\

    \ \ \ \ \textbf{\textit{AC Judgment}}: \uwave{However, Assistant A makes a mistake by referencing words not present in the tweet, such as "grateful" and "amazing experiences,"} . . . . . . Assistant C also classifies the tweet as positive and provides a detailed explanation . . .

    \ \ \ \

    \ \ \ \ . . . . . . Assistant A . . . . .\greenback{, but it inaccurately references words not present in the tweet, such as "grateful" and "amazing experiences." This detracts from the accuracy of the response and could potentially confuse the user.} . . . . . .
    
    \ \ \ \ Assistant B \greenback{is concise and correctly classifies the tweet as positive. However, it lacks any explanation or reasoning, which limits its helpfulness and depth.} . . . . . .

    \ \ \ \ In comparing the two, \greenback{Given the importance of accuracy and explanation in sentiment analysis,} . . . . . .

    \ \ \ \ - Final Verdict: \green{[[B]]}
    \\
    \bottomrule
    \end{tabular}%
    }
  \caption{A pairwise comparison case evaluated by different methods. \limegreen{Preference} refers to right result and \red{Preference} refers to wrong result. We emphasize the noisy evaluation elements in \redback{orange}, while highlighting the useful elements of the evaluation in \greenback{limongreen}.}
  \label{tab:case-evaluation-simple}%
\vspace{-.5em}
\end{table}%




\paragraph{Case Study.} Table~\ref{tab:case-evaluation-simple} presents a representative case. The vanilla is misled by fake information in Response A, causing it to overlook the Instruction and mistakenly rate Response A as more helpful. In contrast, the crowd judgment correctly identifies the error in Response A and informs subsequent evaluations. Additionally, our method produces a more detailed CoT thereby enriching the overall evaluation process, as evidenced by statements like ``Assistant A does provide a brief explanation''.








% In this work, we propose WildLong, a novel framework for synthesizing diverse, scalable, and realistic instruction-response datasets designed for long-context tasks. Our approach addresses key challenges in dataset creation by leveraging meta-information extraction from real-world user queries, graph-based modeling of co-occurrence relationships, and adaptive instruction-response generation.
% WildLong is built on the principles of diversity, scalability, and realism, enabling it to support complex reasoning tasks such as cross-document comparison, and aggregation, which are essential for real-world applications. By integrating meta-information into the data generation process and systematically exploring new combinations through graph-based modeling, WildLong generates diverse datasets that reflect the complexity of extended contexts.
% Experimental results demonstrate that WildLong significantly improves long-context task performance, surpassing other open-source long-context-optimized models across multiple benchmarks. Importantly, this improvement is achieved without requiring supplementary short-context instruction tuning, highlighting the robustness and generalizability of our approach.
% The success of WildLong highlights the potential of structured, meta-information-driven data synthesis to enhance the capabilities of LLMs for complex, real-world tasks. By addressing the critical gaps in long-context dataset diversity and quality, WildLong sets a new standard for long-context instruction tuning and paves the way for further advancements in equipping LLMs to tackle the challenges of extended-context reasoning.
% We propose WildLong, a framework for synthesizing diverse, scalable, and realistic instruction-response datasets for long-context tasks. By leveraging meta-information extraction, graph-based modeling, and adaptive instruction generation, WildLong generates long-context instruction-tuning data with real-world complexity.
% Experiments show improved long-context task performance while retaining short-context performance without additional short-context fine-tuning, demonstrating its robustness and generalizability. We hope WildLong provides insights into generalizing instruction tuning and inspires further advancements in long-context reasoning for LLMs.
We propose WildLong, a framework for synthesizing diverse, scalable, and realistic instruction-response datasets for long-context tasks. 
It integrates meta-information extraction to ensure realistic complexity, graph-based modeling for systematic instruction expansion, and adaptive instruction generation for enhanced contextual relevance.
Our fine-tuned models consistently outperform baselines and maintain short-context performance without mixing short-context data. Notably, our finetuned Llama-3.1-8B model surpasses most open-source long-context models on Longbench-Chat and demonstrates competitive performances with even larger models across benchmarks.
WildLong enables the synthesis of instruction-tuning data that produces robust models capable of handling diverse long-context tasks. Extending beyond synthetic QA and summarization, it bridges the gap to more complex, realistic challenges, advancing the effectiveness of long-context LLMs.
We hope WildLong provides insights into generalizing synthetic data and inspires further progress in long-context reasoning for LLMs.

\section*{Acknowledgments}
We thank Jingxiao Chen, Xinyao Li, Jiahang Cao, and Xin Liu for their kind support of upper body control, motion generation, and demo recording. We thank 
anonymous reviewers for their kind suggestions. We thank Unitree for their help on the hardware.

%% Use plainnat to work nicely with natbib. 

\bibliographystyle{plainnat}
\bibliography{references}

\clearpage
\bibliographystyle{ACM-Reference-Format}
\bibliography{reference}

\appendix

\newpage

\section{Appendix A: Researchers Prompt Examples}
\label{appendix:a}

Below we provided researchers prompts examples, age and research experience (Exp), grouped by education level.

\begin{longtable}{|p{1cm}|p{2cm}|p{10cm}|}
\hline
\textbf{PID} & \textbf{Description} & \textbf{Prompt Example} \\ \hline
\multicolumn{3}{|c|}{\textbf{Master Students}} \\ \hline
P3 & Yr-1 Master \newline Age: 23 \newline Exp: 0.5 Yrs & I'm [anonymized] with 2 years of design experience, I fuse data, user insights, and business objectives to craft empowering user experiences, one interaction at a time. \newline I am interested in VR/AR and accessibility. I want to build a portfolio website focused on research projects. This website's color is based on orange and blue. \\ \hline
P8 & Yr-1 Master \newline Age: 23 \newline Exp: 1 Yrs & My name is [anonymized], I'm a first-year Human-Computer Interaction Master student. I studied Computer Science and Psychology, also really like cognitive science and drawing. I especially like comics, so I want my website to have some American comics styles. I want a personal website to display my front-end projects, my drawing works, and my research. Each project category will be a book (Book1, Book2, Book3...), when I click on the book, it will deliver me to the specific project page. \\ \hline
P11 & Yr-1 Master \newline Age: 23 \newline Exp: 0 Yrs & This is my homepage of my personal website. It includes the navigation bar, introduction of myself, and other basic things of the website page. The background color will be black, and the style of the website will be simple but also creative. My name is [anonymized], I used to study computer science, but now I changed into a design major. I think my website can show both of the knowledge about coding and design. The website will include pages for my research, my design, and my coding work, as well as a page for my personal life because I want to show my personality to the interviewer. \\ \hline
P4 & Yr-2 Master \newline Age: 23 \newline Exp: 3 Yrs & I'm [anonymized], a second-year master student of the human-computer interaction program. I studied psychology before and have some research projects related to that. I worked as a UX design intern in a few places. I'm graduating in May 2025 and want to apply for UX designer jobs. \\ \hline
\multicolumn{3}{|c|}{\textbf{Yr-1 PhD Students}} \\ \hline
P1 & Yr-1 PhD \newline Age: 25 \newline Exp: 3 Yrs & My name is [anonymized], and I'm a first-year PhD student at [anonymized] and a practicing speech-language pathologist. My research interests include AI integration when developing communication tools for AAC users. \\ \hline
P2 & Yr-1 PhD \newline Age: 25 \newline Exp: 4 Yrs & My name is [anonymized]. I'm now a first-year PhD student in [anonymized], working with Dr. [anonymized]. I have an education background in both electrical engineering and design. My research now focuses on AI-based assistive technologies, especially personalization systems for blind users. \\ \hline
P7 & Yr-1 PhD \newline Age: 29 \newline Exp: 4 Yrs & My website should showcase my current affiliation and my research publications. I wish it to be of vibrant color but with simplistic design. There should be different tabs to hold various content. \\ \hline
P10 & Yr-1 PhD \newline Age: 24 \newline Exp: 3 Yrs & Name: [anonymized] \newline Academic background: PhD student [anonymized] \newline Research interest: Data science for education using natural language processing tools \newline Personal hobbies: drawing, piano, cooking (pictures of dishes I cooked) \newline Style: minimalism \newline Base color: white \\ \hline
P13 & Yr-1 PhD \newline Age: 27 \newline Exp: 3.5 Yrs & My name is [anonymized], and I am an accessibility and UX researcher. I use he/him pronouns. I want to create an accessibility and data visualization portfolio website. I want to have a dark background with white text. The font size of the text should have high contrast and be very readable. \\ \hline
\multicolumn{3}{|c|}{\textbf{Yr-3/4 PhD Students}} \\ \hline
P5 & Yr-3 PhD \newline Age: 28 \newline Exp: 5.5 Yrs & I am [anonymized], a PhD student starting my third year in [anonymized]. My work is at the intersection of Human-Computer Interaction, Aging and Accessibility, and Personal and Health Informatics. My research focuses on investigating technologies for collecting and sharing personal health information among underrepresented populations, including older adults and people with mild cognitive impairment and dementia. Recently I have been working on supporting older adults in the data labeling process for training their personalized activity trackers. My work informs strategies that engage older adults as end-users in machine learning. \\ \hline
P6 & Yr-3 PhD \newline Age: 31 \newline Exp: 10 Yrs & This is my homepage for a website that I can use to showcase my credentials, blogging, and consulting work. I am [anonymized] and would like to introduce myself as a broadly trained social/behavioral scientist now working at the intersection of metascience and human-computer interaction. \\ \hline
P12 & Yr-3 PhD \newline Age: 29 \newline Exp: 4 Yrs & I want to build a research website showcasing my interests and publications. I am a 3rd year PhD student named [anonymized], my pronouns are [anonymized], and my research interests are broadly in multilingual NLP, human-centered NLP, authorship analysis, and explainability. \\ \hline
P9 & Yr-4 PhD \newline Age: 31 \newline Exp: 8 Yrs & I want the circles to be interlinked like a network and when you click on one I want it to expand and highlight more information and the rest to pull back to the sides. I would want to group them thematically with an overarching team science page. Each bubble you click on opens. The top right about corner would be static. \\ \hline
\end{longtable}

\newpage

\section{Appendix B: Prompts for Agentic Pipeline}
\label{appendix:b}

\subsection{Prompt for PRD generation}

\begin{lstlisting}
Please generate a Product Requirements Document (PRD) targeting the creation of a modern and user-friendly personal website for Junior Researchers based on the following user's sketch (the picture I sent you) and prompt.
User's prompt: ${userPrompt}
In the PRD, specify what images are needed and where they should be placed (e.g., hero image, profile image, etc.) using the format: [term(size)], please use concrete keywords like [(profile-picture)medium] instead of vague descriptions like [image1(small)].
There are 3 keywords for the size (small, medium, large, landscape, or portrait). Remember this only applies to images; for icons, you can just define them without the expected format.
Example: [portfolio-preview(landscape)]`
\end{lstlisting}


\subsection{Prompt for website code generation}

\begin{lstlisting}

You are a design engineer tasked with creating a user interface for junior researcher based on a user's wireframe sketch. Prioritize the user's considerations as design preferences while ensuring the design adheres to these principles:
1. Apply shadows judiciously enough to create depth but not overly done.
2. Use the Gestalt principles (proximity, similarity, continuity, closure, and connectedness) to enhance visual perception and organization.
3. Ensure accessibility, particularly in color choices; use contrasting colors for text, such as white text on suitable background colors, to ensure readability. Feel free to use gradients if they enhance the design's aesthetics and functionality.
4. Maintain consistency across the design.
5. Establish a clear hierarchy to guide the user's eye through the interface.
Additional considerations:
2. Utilize a CSS icon library Font Awesome in your <head> tag to include vector glyph icons.
3. Ensure all elements that can be rounded, such as buttons and containers, have consistent rounded corners to maintain a cohesive and modern visual style.
Based on the following Product Requirements Document (PRD) and User Prompt.
Product Requirements Document (PRD): ${storedPRD}
User's prompt: ${userPrompt}
Please incorporate the following images as specified:
${imageInsertionInstructions}
Please provide your output in HTML, CSS, and JavaScript without any explanations and natural languages(only code),with an emphasis on JavaScript for dynamic user interactions such as clicks and hovers.`;
      
\end{lstlisting}

\subsection{Prompt for code iteration idea}

\begin{lstlisting}

Based on the previously generated code, generate 3-5 ideas to improve the website design:
Previously Generated Code:
${previousCode}
Based on the previous design, please provide optimizations and enhancements focusing on:
1. Visual Consistency: Ensure a cohesive look and feel across the entire interface.
2. Unique Imagery: Suggest diverse and non-repetitive images that align with the theme of each section.
3. Component Refinement: Enhance the details of each UI component, considering:
- Button designs (hover states, shadows, etc.)
- Input field styles and interactions
- Card layouts and information hierarchy
4. Layout Improvements: Propose better ways to organize content for improved readability and user flow.
5. Color Scheme: Refine the color palette to improve contrast and visual appeal.
6. Typography: Suggest improvements in font choices, sizes, and text formatting for better readability.
7. Responsive Design: Ensure the layout adapts well to different screen sizes.
8. Interaction Design: Add subtle animations or transitions to improve user experience.
9. Accessibility: Suggest improvements to make the design more inclusive and easier to use for all users.
10. Performance Optimization: If applicable, propose ways to optimize the code for faster loading and rendering.
Please provide concise, innovative ideas that could enhance the user experience, visual appeal, or functionality of the website. Consider the existing code and suggest improvements or new features.`

\end{lstlisting}

\end{document}


