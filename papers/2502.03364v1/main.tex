% This class has a lot of options, so please check deepmind.cls for more details.
% This is a minimal set for most needs.
\documentclass[11pt, a4paper, logo, twocolumn, copyright]{googledeepmind}

% Omit dates for reproducibility.
\pdfinfoomitdate 1
\pdftrailerid{redacted}

% This avoids duplicate hyperref bookmark entries when using \bibentry (e.g. via \citeas).
\makeatletter
\renewcommand\bibentry[1]{\nocite{#1}{\frenchspacing\@nameuse{BR@r@#1\@extra@b@citeb}}}
\makeatother

\usepackage{kantlipsum, lipsum}
\usepackage{dsfont}
\usepackage{gdm-colors}

\usepackage[]{mdframed}
\usepackage[capitalize,noabbrev]{cleveref}

% Sometimes you will get errors about pdflink ending up in diffrent position. Try this and
% comment it out again when you are done with your document.
%\hypersetup{draft}

% Set the bibliography options here.
\usepackage[authoryear, sort&compress, round]{natbib}

% Images will be looked for in this path, removes need for explicit path when including images.
\graphicspath{{figures/}}

% Important Information about your paper.
\title{Scaling laws in wearable human activity recognition}

% Can leave this option out if you do not wish to add a corresponding author.
\correspondingauthor{thoddes@google.com}

% Remove these if they are not needed
% \keywords{\LaTeX, Publications process, tools}
% \paperurl{arxiv.org/abs/123}

% Assign your own date to the report.
% Can comment out if not needed or leave blank if n/a.
\renewcommand{\today}{2025-01-31}

\author[1]{Tom Hoddes}
\author[*, 1]{Alex Bijamov}
\author[*, 1]{Saket Joshi}
\author[*,2,3]{Daniel Roggen}
\author[4,5]{Ali Etemad}
\author[2,6]{Robert Harle}
\author[1]{David Racz}

\affil[*]{Equal contributions}
\affil[1]{Google DeepMind}
\affil[2]{Google LLC}
\affil[3]{School of Engineering \& Informatics, University of Sussex, UK}
\affil[4]{Work done while at Google Research}
\affil[5]{Queen's University, Canada}
\affil[6]{Computer Laboratory, University of Cambridge, UK}

\begin{abstract}
Many  deep architectures and self-supervised pre-training techniques have been proposed for human activity recognition (HAR) from wearable multimodal sensors. 
Scaling laws have the potential to help move towards more principled design by linking model capacity with pre-training data volume.
Yet, scaling laws have not been established for HAR to the same extent as in language and vision. 
By conducting an exhaustive grid search on both amount of pre-training data and Transformer architectures, we establish the first known scaling laws for HAR. 
We show that pre-training loss scales with a power law relationship to amount of data and parameter count and that increasing the number of users in a dataset results in a steeper improvement in performance than increasing data per user, indicating that diversity of pre-training data is important, which contrasts to some previously reported findings in self-supervised HAR.
We show that these scaling laws translate to downstream performance improvements on three HAR benchmark datasets of postures, modes of locomotion and activities of daily living: UCI HAR and WISDM Phone and WISDM Watch. 
Finally, we suggest some previously published works should be revisited in light of these scaling laws with more adequate model capacities.
\end{abstract}

\begin{document}

\maketitle



% Incude paper content from external files
\section{Introduction}
\label{sec:introduction}
\section{Introduction}
\label{sec:introduction}
The business processes of organizations are experiencing ever-increasing complexity due to the large amount of data, high number of users, and high-tech devices involved \cite{martin2021pmopportunitieschallenges, beerepoot2023biggestbpmproblems}. This complexity may cause business processes to deviate from normal control flow due to unforeseen and disruptive anomalies \cite{adams2023proceddsriftdetection}. These control-flow anomalies manifest as unknown, skipped, and wrongly-ordered activities in the traces of event logs monitored from the execution of business processes \cite{ko2023adsystematicreview}. For the sake of clarity, let us consider an illustrative example of such anomalies. Figure \ref{FP_ANOMALIES} shows a so-called event log footprint, which captures the control flow relations of four activities of a hypothetical event log. In particular, this footprint captures the control-flow relations between activities \texttt{a}, \texttt{b}, \texttt{c} and \texttt{d}. These are the causal ($\rightarrow$) relation, concurrent ($\parallel$) relation, and other ($\#$) relations such as exclusivity or non-local dependency \cite{aalst2022pmhandbook}. In addition, on the right are six traces, of which five exhibit skipped, wrongly-ordered and unknown control-flow anomalies. For example, $\langle$\texttt{a b d}$\rangle$ has a skipped activity, which is \texttt{c}. Because of this skipped activity, the control-flow relation \texttt{b}$\,\#\,$\texttt{d} is violated, since \texttt{d} directly follows \texttt{b} in the anomalous trace.
\begin{figure}[!t]
\centering
\includegraphics[width=0.9\columnwidth]{images/FP_ANOMALIES.png}
\caption{An example event log footprint with six traces, of which five exhibit control-flow anomalies.}
\label{FP_ANOMALIES}
\end{figure}

\subsection{Control-flow anomaly detection}
Control-flow anomaly detection techniques aim to characterize the normal control flow from event logs and verify whether these deviations occur in new event logs \cite{ko2023adsystematicreview}. To develop control-flow anomaly detection techniques, \revision{process mining} has seen widespread adoption owing to process discovery and \revision{conformance checking}. On the one hand, process discovery is a set of algorithms that encode control-flow relations as a set of model elements and constraints according to a given modeling formalism \cite{aalst2022pmhandbook}; hereafter, we refer to the Petri net, a widespread modeling formalism. On the other hand, \revision{conformance checking} is an explainable set of algorithms that allows linking any deviations with the reference Petri net and providing the fitness measure, namely a measure of how much the Petri net fits the new event log \cite{aalst2022pmhandbook}. Many control-flow anomaly detection techniques based on \revision{conformance checking} (hereafter, \revision{conformance checking}-based techniques) use the fitness measure to determine whether an event log is anomalous \cite{bezerra2009pmad, bezerra2013adlogspais, myers2018icsadpm, pecchia2020applicationfailuresanalysispm}. 

The scientific literature also includes many \revision{conformance checking}-independent techniques for control-flow anomaly detection that combine specific types of trace encodings with machine/deep learning \cite{ko2023adsystematicreview, tavares2023pmtraceencoding}. Whereas these techniques are very effective, their explainability is challenging due to both the type of trace encoding employed and the machine/deep learning model used \cite{rawal2022trustworthyaiadvances,li2023explainablead}. Hence, in the following, we focus on the shortcomings of \revision{conformance checking}-based techniques to investigate whether it is possible to support the development of competitive control-flow anomaly detection techniques while maintaining the explainable nature of \revision{conformance checking}.
\begin{figure}[!t]
\centering
\includegraphics[width=\columnwidth]{images/HIGH_LEVEL_VIEW.png}
\caption{A high-level view of the proposed framework for combining \revision{process mining}-based feature extraction with dimensionality reduction for control-flow anomaly detection.}
\label{HIGH_LEVEL_VIEW}
\end{figure}

\subsection{Shortcomings of \revision{conformance checking}-based techniques}
Unfortunately, the detection effectiveness of \revision{conformance checking}-based techniques is affected by noisy data and low-quality Petri nets, which may be due to human errors in the modeling process or representational bias of process discovery algorithms \cite{bezerra2013adlogspais, pecchia2020applicationfailuresanalysispm, aalst2016pm}. Specifically, on the one hand, noisy data may introduce infrequent and deceptive control-flow relations that may result in inconsistent fitness measures, whereas, on the other hand, checking event logs against a low-quality Petri net could lead to an unreliable distribution of fitness measures. Nonetheless, such Petri nets can still be used as references to obtain insightful information for \revision{process mining}-based feature extraction, supporting the development of competitive and explainable \revision{conformance checking}-based techniques for control-flow anomaly detection despite the problems above. For example, a few works outline that token-based \revision{conformance checking} can be used for \revision{process mining}-based feature extraction to build tabular data and develop effective \revision{conformance checking}-based techniques for control-flow anomaly detection \cite{singh2022lapmsh, debenedictis2023dtadiiot}. However, to the best of our knowledge, the scientific literature lacks a structured proposal for \revision{process mining}-based feature extraction using the state-of-the-art \revision{conformance checking} variant, namely alignment-based \revision{conformance checking}.

\subsection{Contributions}
We propose a novel \revision{process mining}-based feature extraction approach with alignment-based \revision{conformance checking}. This variant aligns the deviating control flow with a reference Petri net; the resulting alignment can be inspected to extract additional statistics such as the number of times a given activity caused mismatches \cite{aalst2022pmhandbook}. We integrate this approach into a flexible and explainable framework for developing techniques for control-flow anomaly detection. The framework combines \revision{process mining}-based feature extraction and dimensionality reduction to handle high-dimensional feature sets, achieve detection effectiveness, and support explainability. Notably, in addition to our proposed \revision{process mining}-based feature extraction approach, the framework allows employing other approaches, enabling a fair comparison of multiple \revision{conformance checking}-based and \revision{conformance checking}-independent techniques for control-flow anomaly detection. Figure \ref{HIGH_LEVEL_VIEW} shows a high-level view of the framework. Business processes are monitored, and event logs obtained from the database of information systems. Subsequently, \revision{process mining}-based feature extraction is applied to these event logs and tabular data input to dimensionality reduction to identify control-flow anomalies. We apply several \revision{conformance checking}-based and \revision{conformance checking}-independent framework techniques to publicly available datasets, simulated data of a case study from railways, and real-world data of a case study from healthcare. We show that the framework techniques implementing our approach outperform the baseline \revision{conformance checking}-based techniques while maintaining the explainable nature of \revision{conformance checking}.

In summary, the contributions of this paper are as follows.
\begin{itemize}
    \item{
        A novel \revision{process mining}-based feature extraction approach to support the development of competitive and explainable \revision{conformance checking}-based techniques for control-flow anomaly detection.
    }
    \item{
        A flexible and explainable framework for developing techniques for control-flow anomaly detection using \revision{process mining}-based feature extraction and dimensionality reduction.
    }
    \item{
        Application to synthetic and real-world datasets of several \revision{conformance checking}-based and \revision{conformance checking}-independent framework techniques, evaluating their detection effectiveness and explainability.
    }
\end{itemize}

The rest of the paper is organized as follows.
\begin{itemize}
    \item Section \ref{sec:related_work} reviews the existing techniques for control-flow anomaly detection, categorizing them into \revision{conformance checking}-based and \revision{conformance checking}-independent techniques.
    \item Section \ref{sec:abccfe} provides the preliminaries of \revision{process mining} to establish the notation used throughout the paper, and delves into the details of the proposed \revision{process mining}-based feature extraction approach with alignment-based \revision{conformance checking}.
    \item Section \ref{sec:framework} describes the framework for developing \revision{conformance checking}-based and \revision{conformance checking}-independent techniques for control-flow anomaly detection that combine \revision{process mining}-based feature extraction and dimensionality reduction.
    \item Section \ref{sec:evaluation} presents the experiments conducted with multiple framework and baseline techniques using data from publicly available datasets and case studies.
    \item Section \ref{sec:conclusions} draws the conclusions and presents future work.
\end{itemize}





\section{Related Work}
\label{sec:related}
\putsec{related}{Related Work}

\noindent \textbf{Efficient Radiance Field Rendering.}
%
The introduction of Neural Radiance Fields (NeRF)~\cite{mil:sri20} has
generated significant interest in efficient 3D scene representation and
rendering for radiance fields.
%
Over the past years, there has been a large amount of research aimed at
accelerating NeRFs through algorithmic or software
optimizations~\cite{mul:eva22,fri:yu22,che:fun23,sun:sun22}, and the
development of hardware
accelerators~\cite{lee:cho23,li:li23,son:wen23,mub:kan23,fen:liu24}.
%
The state-of-the-art method, 3D Gaussian splatting~\cite{ker:kop23}, has
further fueled interest in accelerating radiance field
rendering~\cite{rad:ste24,lee:lee24,nie:stu24,lee:rho24,ham:mel24} as it
employs rasterization primitives that can be rendered much faster than NeRFs.
%
However, previous research focused on software graphics rendering on
programmable cores or building dedicated hardware accelerators. In contrast,
\name{} investigates the potential of efficient radiance field rendering while
utilizing fixed-function units in graphics hardware.
%
To our knowledge, this is the first work that assesses the performance
implications of rendering Gaussian-based radiance fields on the hardware
graphics pipeline with software and hardware optimizations.

%%%%%%%%%%%%%%%%%%%%%%%%%%%%%%%%%%%%%%%%%%%%%%%%%%%%%%%%%%%%%%%%%%%%%%%%%%
\myparagraph{Enhancing Graphics Rendering Hardware.}
%
The performance advantage of executing graphics rendering on either
programmable shader cores or fixed-function units varies depending on the
rendering methods and hardware designs.
%
Previous studies have explored the performance implication of graphics hardware
design by developing simulation infrastructures for graphics
workloads~\cite{bar:gon06,gub:aam19,tin:sax23,arn:par13}.
%
Additionally, several studies have aimed to improve the performance of
special-purpose hardware such as ray tracing units in graphics
hardware~\cite{cho:now23,liu:cha21} and proposed hardware accelerators for
graphics applications~\cite{lu:hua17,ram:gri09}.
%
In contrast to these works, which primarily evaluate traditional graphics
workloads, our work focuses on improving the performance of volume rendering
workloads, such as Gaussian splatting, which require blending a huge number of
fragments per pixel.

%%%%%%%%%%%%%%%%%%%%%%%%%%%%%%%%%%%%%%%%%%%%%%%%%%%%%%%%%%%%%%%%%%%%%%%%%%
%
In the context of multi-sample anti-aliasing, prior work proposed reducing the
amount of redundant shading by merging fragments from adjacent triangles in a
mesh at the quad granularity~\cite{fat:bou10}.
%
While both our work and quad-fragment merging (QFM)~\cite{fat:bou10} aim to
reduce operations by merging quads, our proposed technique differs from QFM in
many aspects.
%
Our method aims to blend \emph{overlapping primitives} along the depth
direction and applies to quads from any primitive. In contrast, QFM merges quad
fragments from small (e.g., pixel-sized) triangles that \emph{share} an edge
(i.e., \emph{connected}, \emph{non-overlapping} triangles).
%
As such, QFM is not applicable to the scenes consisting of a number of
unconnected transparent triangles, such as those in 3D Gaussian splatting.
%
In addition, our method computes the \emph{exact} color for each pixel by
offloading blending operations from ROPs to shader units, whereas QFM
\emph{approximates} pixel colors by using the color from one triangle when
multiple triangles are merged into a single quad.



\section{Method}
\label{sec:method}


\subsection{Scaling Laws}
For our scaling laws, we take a different approach from that in language \cite{hoffmann2022chinchilla} \cite{kaplan2020scalinglawsneurallanguage} and vision \cite{zhai2022scalingvits}. 
In these domains, since data was abundant and compute was the primary constraint, they never completed a full epoch, and thus equated number of steps to amount of data. 
For HAR, however, data is the primary constraint, so we repeat data many times (over 100 epochs) until convergence. Since we are able to train even our largest models to convergence, we do not fix the amount of compute. 
Instead, we focus on the capacity of the models (number of parameters). 
This is more directly tied to inference cost than training, which aligns with the priorities of many HAR deployments.


\subsection{Encoder}
For the encoder backbone, we use a ViT \cite{vit} adapted for accelerometer and gyroscope motion sensors as shown in \cref{mae-architecture}. The input to the encoder consists of a time series window of 128 samples at 50Hz, where each sample has 6 channels (x, y, z for accelerometer and gyroscope). We break the window into ``patches" of 4 samples. We choose this patch size to be as small as possible while still fitting the attention matrix in memory. 
Each patch of shape (4, 6) is flattened and transformed linearly to an embedding of dimension size equal to 1/4 the width of the MLP.

We use a standard Transformer block with 8 attention heads. To determine the optimal encoder capacity (i.e. number of parameters) for a given data scale, we conduct a grid search of 3 different widths (512, 1024, 2048 hidden MLP units) and 3 different depths (5, 10, 20 blocks), resulting in 9 different models from 1M to 63M parameters.


\begin{figure}[htb]
% \vskip 0.2in
\begin{center}
\centerline{\includegraphics[width=1\columnwidth]{HAR_Scaling_Paper_MAE_Diagram.png}}
\caption{Masked Autoencoder adapted for accelerometer and gyroscope. 
During pre-training, a random subset of accelerometer and gyroscope patches are masked out. Non-masked patches are passed to the encoder and the mask
tokens are re-introduced after the encoder. The encoded patches and mask tokens are then processed by a small decoder trained to reconstruct the original input sequence.
}
\label{mae-architecture}
\end{center}
% \vskip -0.2in
\end{figure}


\subsection{Pre-training}
Our pre-training approach follows Masked Autoencoder \cite{maskedautoencoder} closely, but adapted for accelerometer and gyroscope motion sensors as shown in \cref{mae-architecture}. This was chosen because it is simple to implement, scales well, and doesn’t require negative examples or domain specific design choices such as augmentations (e.g. to prevent collapse in dual encoders).
We randomly mask whole patches rather than individual samples. 
We only encode unmasked patches and use a high masking ratio of 70\%. This saves considerably on compute costs.
We use a small decoder consisting of 2 Transformer blocks. We apply a linear projection between the encoder and decoder to decrease the width of the decoder by half.

\subsection{Evaluation}
To evaluate pre-trained encoders, we remove the decoder and use global average pooling to attach a linear classification head to the output of the encoder, which we keep frozen. We use linear evaluation as opposed to full fine-tuning to provide a clearer signal of the information extracted from pre-training alone.

\subsection{Datasets}
We use the Extrasensory dataset \cite{extrasensory2017} for pre-training. This dataset contains more than 300k examples of 20 seconds of sensor data from 60 users. Data has been collected while subjects were engaged in regular everyday behavior for several sensors including accelerometer, gyroscope and magnetometer across multiple phone and wearable devices. The dataset is pre-formatted into 5 folds split by user. Each fold is split into a training and a test set. After filtering for missing data we collected 286140 examples, which equates to approximately 1589 hours of data. 
We ignore the activity labels for pre-training. Within each fold, we vary the amount of data by sampling some percentage of examples. 
The USER sampling strategy takes all the data from a randomly selected percentage of the users in the fold.
The RANDOM sampling strategy puts the data of all users in the fold together and draws a random percentage of examples from that.
To control for non-uniform distribution between different sampling strategies and folds, we report results based on the total hours of pre-training data rather than by percent. We only use the phone data, since the watch data does not contain a gyroscope.

For downstream evaluation, we use popular benchmark datasets UCI HAR and WISDM Phone/Watch datasets. We use the full training dataset for all supervised training.
UCI HAR \cite{human_activity_recognition_using_smartphones_240} contains data from 30 volunteers aged 19-48 engaging in 6 modes of locomotion and postures: walking, walking upstairs, walking downstairs, sitting, standing, laying. The accelerometer and gyroscope data is recorded at 50Hz from a smartphone worn on the waist. 
We use the same random partitioning prescribed by the dataset authors (70\% training and 30\% test sets). We also keep the existing raw data preprocessing pipeline, involving noise filtering, 2.56sec sliding windows with 50\% overlap, resulting in 128 samples per window, and use the raw acceleration, not the low-pass filtered version also present in the dataset.

For WISDM we use the 2019 version of the dataset \cite{wisdm_smartphone_and_smartwatch_activity_and_biometrics_dataset__507} comprising 51 subjects performing 18 activities of daily living (postures, locomotion, house chores, nutrition, work-related activities and others) for 3 minutes each. 
We assign the first 2/3 of users to the training set (subjects 1600-1633), and use the remaining 1/3 for evaluation (subjects 1634-1650) similar to previous benchmarks. We split the WISDM dataset into Phone and Watch body positions and evaluate these separately.

We re-sample all datasets to 50Hz and normalize to the same units. None of the datasets contain a null class.

\subsection{Training schedule}
For every model, data combination, we fix the number of steps for pre-training to 500,000 with a batch size of 2048. This equates to over 100 epochs when using 100\% of the data. We use the Adam optimizer with three different learning rates (1e-3, 1e-4, 1e-5) for every model and take the best result. This ensures that each model has sufficient coverage of the parameter search space regardless of size. We apply dropout during pre-training with a rate of 0.1.

\subsection{Compute}
Our exhaustive grid search results in 1620 (3 learning rates * 6 data sizes * 2 sampling strategies * 5 folds * 9 encoder architectures) different hyperparameter combinations for pre-training. Each run takes between 3 and 35 hours to run on 4 TPUv2 chips with larger models running longer. Our total compute used for pre-training is about 62000 TPU-hours.





\begin{table}[htb]
\caption{Best F1 scores of models trained from scratch (FS) vs linear eval (LE) on pre-trained models for each dataset.} 
\label{f1-table}
% \vskip 0.15in
\begin{center}
\begin{small}
\begin{sc}
\begin{tabular}{lcccr}
\toprule
Data set & FS & LE \\
\midrule
UCI HAR    & 95.1 & 97.9 \\
WISDM Phone    & 31.9 & 34.3 \\
WISDM Watch & 62.6 & 63.1 \\
\bottomrule
\end{tabular}
\end{sc}
\end{small}
\end{center}
% \vskip -0.1in
\vspace{-1.5em}
\end{table}



\section{Results}

\subsection{Supervised training from scratch baseline}
For each dataset, we conduct a thorough capacity and hyperparameter search of from-scratch models to establish baselines for comparing with pre-trained models. The best results are listed in \cref{f1-table}. We also look at the effect of model capacity on from scratch training. We find that smaller models work better for UCI HAR, with our second smallest model of about 2M parameters performing best. The effect of capacity on WISDM is less clear. It also appears that deeper models perform better than wide models.

\begin{figure}[t]
% \vskip 0.2in
\begin{center}
\centerline{\includegraphics[width=0.7\columnwidth]{pretrain_eval_loss_data_law.png}}
\caption{Pre-training test loss vs data size (hours). We fit a power law to each fold and sampling strategy. Equations for each power law can be found in \cref{pretrain-data-law-table}.}
\label{pretrain-loss-data}
\end{center}
% \vskip -0.2in
\end{figure}

\begin{table}[t]
\caption{Power laws of pre-training test loss vs data size. Exponent values for the USER sampling strategy are roughly 3 times greater than RANDOM.}
\label{pretrain-data-law-table}
% \vskip 0.15in
\begin{center}
\begin{small}
\begin{sc}
\begin{tabular}{lcccr}
\toprule
Fold & \multicolumn{2}{c}{Sampling Strategy} \\
& USER & RANDOM \\
\midrule
0    & $L = 0.058D^{-0.049}$ & $L = 0.046D^{-0.017}$ \\
1    & $L = 0.057D^{-0.045}$ & $L = 0.047D^{-0.015}$ \\
2    & $L = 0.052D^{-0.046}$ & $L = 0.043D^{-0.020}$ \\
3    & $L = 0.051D^{-0.052}$ & $L = 0.040D^{-0.019}$ \\
4    & $L = 0.043D^{-0.044}$ & $L = 0.035D^{-0.016}$ \\
\bottomrule
\end{tabular}
\end{sc}
\end{small}
\end{center}
% \vskip -0.1in
\end{table}





\begin{figure}[htb]
%\vskip 0.2in
\begin{center}
\centerline{\includegraphics[width=0.7\columnwidth]{pretrain_eval_loss_capacity_law.png}}
\caption{Pre-training test loss vs model capacity (number of parameters) and associated power law fit and equation.}
\label{pretrain-loss-capacity}
\end{center}
%\vskip -0.2in
\vspace{-1em}
\end{figure}




\begin{figure*}[htb]
%\vskip 0.2in
\begin{center}
\centerline{\includegraphics[width=0.8\linewidth]{pretrain_f1_vs_data.png}}
\caption{Best linear F1 scores vs pre-training dataset size (hours). Each point represents the best F1 score corresponding to a pre-training fold and data size. The best score is chosen from 27 runs consisting of the 9 encoder architectures and 3 learning rates in our search space.}
\label{pretrain-f1-data}
\end{center}
%\vskip -0.2in
\vspace{-1em}
\end{figure*}




\begin{figure*}[htb]
% \vskip 0.2in
\begin{center}
\centerline{\includegraphics[width=0.8\textwidth]{pretrain_capacity.png}}
\caption{Best linear {F1} scores vs model capacity (number of parameters). Each point represents the best F1 score corresponding to an encoder architecture (width and depth). The best score is chosen from all data sizes and learning rates. We indicate the width (mlp hidden dim) by color. At 5M or 20M parameters we have two models that are the same size, with one wider and shallower (5 blocks) and the other narrower and deeper (20 blocks).}
\label{pretrain-f1-capacity}
\end{center}
% \vskip -0.2in
\end{figure*}




\subsection{Scaling laws}
\label{sec:result_scaling}


In \cref{pretrain-loss-data} we establish scaling laws of the pre-training test loss vs hours of data. To calculate the loss, we use the full test set from each Extrasensory fold. 
This allows us to compare different training data amounts and distributions within a fold and fit power-law relationships for each fold. 
We observe roughly the same power-law exponent (or slope on the log-log plot) for a given fold and sampling strategy, giving confidence that this relationship was not due to random chance. 
Furthermore, in \cref{pretrain-data-law-table} we see that the exponent is roughly of 3x greater magnitude (or steeper slope) when data is increased by adding more users, as opposed to uniformly or per-user. This emphasizes that diversity of data is extremely important, and dictates the scaling law. Note that the offset is different for each fold, but that is to be expected, since the test sets are different. 
Similarly, in \cref{pretrain-loss-capacity} we fit a power law between pre-training test loss and model capacity in terms of number of parameters, further demonstrating the existence of a scaling law.



\subsection{Downstream performance}
\label{sec:downstream_performance}

We show that the scaling laws for pre-training translate to similar trends in improved downstream linear classification performance. For each pre-training dataset size, we plot the best F1 score from linear evaluation on downstream datasets UCI HAR and WISDM Phone/Watch. This can be seen in \cref{pretrain-f1-data}. Contrary to previous findings \cite{assessingSSLHAR22,howmuchdataSSLHAR23}, we see consistent improvement as we scale the data size. For UCI HAR, we reach 97.9\% F1 score with linear evaluation. 
To our knowledge, this is on par with the best reported result (98.6\% from \cite{uciharsota}) for this dataset. For all datasets, we surpass from scratch baseline results, with significant improvement for phone datasets UCI HAR (+2.8pp) and WISDM Phone (+2.4pp). 
For WISDM Watch, the improvement is smaller (+0.5pp). This is not surprising given that our pre-training dataset consists of only phone data. Still, the consistent increase in watch performance suggests that we are seeing positive transfer between body positions.


\begin{figure}[htb]
%\vskip 0.2in
\begin{center}
\centerline{\includegraphics[width=\columnwidth]{capacity_vs_data.png}}
\caption{Optimal capacity for a given pre-training data size. Each plot shows the parameter count of the model resulting in the best performance for a given metric (pre-train test loss on the left, UCI HAR test F1 on the right).}
\label{capacity-vs-data}
\end{center}
%\vskip -0.2in
\vspace{-1em}
\end{figure}

In \cref{pretrain-f1-capacity} we study the effect the capacity of the encoder has on downstream performance. For each of the 9 encoder architectures (3 widths by 3 depths), we plot the best F1 score out of 180 runs (5 folds * 6 data sizes * 2 sampling strategies * 3 learning rates) from linear evaluation on downstream datasets UCI HAR and WISDM vs the number of parameters. We find that increasing the number of parameters is crucial to realizing performance improvements across all 3 tasks. The optimal capacity is reached at our biggest model which has about 63M parameters. This is in contrast to our from scratch baselines, where performance can peak at smaller models (e.g. about 2M parameters for UCI HAR).

\begin{figure*}[t]
%\vskip 0.2in
\begin{center}
\centerline{\includegraphics[width=0.8\textwidth]{augmentations_capacity.png}}
\caption{Best linear {F1} scores vs model capacity (number of parameters). Each point represents the best F1 score corresponding to an encoder architecture (width and depth) and whether augmentations were on or off. The best score is chosen from all data sizes and learning rates.}
\label{pretrain-augmentations}
\end{center}
%\vskip -0.2in
\vspace{-1em}
\end{figure*}



\subsection{Optimal capacity vs data size}
In \cref{capacity-vs-data} we study the optimal capacity for a given pre-training data size. For pre-training test loss, we find that optimal model size increases monotonically with more data. Downstream F1 performance tells a different story, with our largest models performing best even with minimal data. We hypothesize that this may be due to epoch-wise double descent \cite{double-descent} behaviour, which we have observed in some cases in this work.

\subsection{Augmentations}
\label{sec:augmentations}
We apply augmentations during pre-training, and study the effect on downstream model performance. In \cref{pretrain-augmentations} We separate results by encoder as in \cref{pretrain-f1-capacity}, but take the best score both with and without augmentations. We see that augmentations always improve performance, especially at larger scales. The optimal capacity without augmentations can be smaller, at either the 4th largest model (about 10M parameters) for UCI HAR or the 3rd largest model (20M parameters) for WISDM Phone. This is not surprising, since augmentations can be thought of as a strong regularizer, and/or an artificial expansion of the dataset.

\subsection{Width vs Depth}
Since we conducted a grid search on width and depth, we have results for a variety of width to depth ratios. There are also 2 model sizes (20M and 5M) for which we have a very deep model (20 blocks) and a very wide model (5 blocks) with the same number of parameters. This allows us to control for the total number of parameters. Looking at \cref{pretrain-f1-capacity}, we can see that increasing both width and depth improve performance, but wider models tend to perform better than deeper models for the same number of parameters.

\section{Conclusion}
\label{sec:conclusion}
\section{Conclusion}
In this work, we propose a simple yet effective approach, called SMILE, for graph few-shot learning with fewer tasks. Specifically, we introduce a novel dual-level mixup strategy, including within-task and across-task mixup, for enriching the diversity of nodes within each task and the diversity of tasks. Also, we incorporate the degree-based prior information to learn expressive node embeddings. Theoretically, we prove that SMILE effectively enhances the model's generalization performance. Empirically, we conduct extensive experiments on multiple benchmarks and the results suggest that SMILE significantly outperforms other baselines, including both in-domain and cross-domain few-shot settings.

% Bibliography components
\bibliographystyle{abbrvnat}
\nobibliography*
\bibliography{main}

\end{document}
