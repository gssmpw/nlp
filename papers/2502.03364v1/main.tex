% This class has a lot of options, so please check deepmind.cls for more details.
% This is a minimal set for most needs.
\documentclass[11pt, a4paper, logo, twocolumn, copyright]{googledeepmind}

% Omit dates for reproducibility.
\pdfinfoomitdate 1
\pdftrailerid{redacted}

% This avoids duplicate hyperref bookmark entries when using \bibentry (e.g. via \citeas).
\makeatletter
\renewcommand\bibentry[1]{\nocite{#1}{\frenchspacing\@nameuse{BR@r@#1\@extra@b@citeb}}}
\makeatother

\usepackage{kantlipsum, lipsum}
\usepackage{dsfont}
\usepackage{gdm-colors}

\usepackage[]{mdframed}
\usepackage[capitalize,noabbrev]{cleveref}

% Sometimes you will get errors about pdflink ending up in diffrent position. Try this and
% comment it out again when you are done with your document.
%\hypersetup{draft}

% Set the bibliography options here.
\usepackage[authoryear, sort&compress, round]{natbib}

% Images will be looked for in this path, removes need for explicit path when including images.
\graphicspath{{figures/}}

% Important Information about your paper.
\title{Scaling laws in wearable human activity recognition}

% Can leave this option out if you do not wish to add a corresponding author.
\correspondingauthor{thoddes@google.com}

% Remove these if they are not needed
% \keywords{\LaTeX, Publications process, tools}
% \paperurl{arxiv.org/abs/123}

% Assign your own date to the report.
% Can comment out if not needed or leave blank if n/a.
\renewcommand{\today}{2025-01-31}

\author[1]{Tom Hoddes}
\author[*, 1]{Alex Bijamov}
\author[*, 1]{Saket Joshi}
\author[*,2,3]{Daniel Roggen}
\author[4,5]{Ali Etemad}
\author[2,6]{Robert Harle}
\author[1]{David Racz}

\affil[*]{Equal contributions}
\affil[1]{Google DeepMind}
\affil[2]{Google LLC}
\affil[3]{School of Engineering \& Informatics, University of Sussex, UK}
\affil[4]{Work done while at Google Research}
\affil[5]{Queen's University, Canada}
\affil[6]{Computer Laboratory, University of Cambridge, UK}

\begin{abstract}
Many  deep architectures and self-supervised pre-training techniques have been proposed for human activity recognition (HAR) from wearable multimodal sensors. 
Scaling laws have the potential to help move towards more principled design by linking model capacity with pre-training data volume.
Yet, scaling laws have not been established for HAR to the same extent as in language and vision. 
By conducting an exhaustive grid search on both amount of pre-training data and Transformer architectures, we establish the first known scaling laws for HAR. 
We show that pre-training loss scales with a power law relationship to amount of data and parameter count and that increasing the number of users in a dataset results in a steeper improvement in performance than increasing data per user, indicating that diversity of pre-training data is important, which contrasts to some previously reported findings in self-supervised HAR.
We show that these scaling laws translate to downstream performance improvements on three HAR benchmark datasets of postures, modes of locomotion and activities of daily living: UCI HAR and WISDM Phone and WISDM Watch. 
Finally, we suggest some previously published works should be revisited in light of these scaling laws with more adequate model capacities.
\end{abstract}

\begin{document}

\maketitle



% Incude paper content from external files
\section{Introduction}
\label{sec:introduction}
\documentclass[../main.tex]{subfiles}
\graphicspath{{../images/}}
\makeatletter
\def\input@path{{../images/}}
\makeatother
\begin{document}
\section{Introduction}
\begin{figure}
\centering
\begin{tikzpicture}
\node[inner sep=0pt] (ws) at (0, 0) {
\includegraphics[height=.4\textwidth, trim={10cm 0 10cm 0},clip]{world_space.png}};
\node[inner sep=0pt] (cs) at (6,0) {\includegraphics[height=.4\textwidth, trim={10cm 1cm 10cm 4cm},clip]{conf_space.png}};
\end{tikzpicture}
\vspace{-5pt}
\label{fig:pbrm_intro}
\caption{\textbf{Left}: Shows world space obstacles as grey spheres. Robots start and goal configuration is colored red and green, respectively. Configurations along the computed path are colored transparent blue. \textbf{Right:} Mapped world space scenario to configuration space. Obstacle region is the grey mesh. Red spheres are collision-free regions computed by the neural SCDF. The optimized shortest path in the convex corridor is the blue curve.}
\vspace{-25pt}
\end{figure}
Motion planning is the problem of finding a collision-free trajectory that connects a given start and goal configuration. The planning takes place in the configuration space of the robot. For single body robots, like mobile robots or drones, the configuration space and the world space are usually the same. This simplifies the planning, since explicit obstacle representations are available which enables geometrical tools like separating hyperplanes, smallest distance to obstacles etc., to be used when designing motion planning algorithms. For multi-body robots like manipulators, the situation is completely different. The world space obstacles are usually mapped to non-convex regions, and to make the problem even harder, the mapping is usually not known. Forming explicit representations of the obstacle region in the configuration space is usually too expensive or intractable. Despite all of this, sampling based planners are used with great success, which mainly is due to their use of implicit representations of the obstacle region. The basic idea is to construct a graph in the configuration space that covers and connects the collision-free region. From this graph, a path can be extracted that connects a given start and goal configuration. The approach is computationally expensive, since the graph is constructed with the smallest geometrical building block available, points, which represents a collision-check. Furthermore, the extracted paths from the graph are non-smooth and jagged due to the stochastic nature of the approach. This adds an additional post-processing step to the process, where the paths are shortcutted and smoothened, before the path can be used for tracking. Clearly a lot of time is invested to form this graph and produce smooth paths. Thus, if the obstacles start to move, then all of this work is done in no use, since all points that make up this graph need to be re-verified, which is simply too time consuming to be done in real time.
\\\\
In this work, we want to address the existing drawbacks of the sampling based planners. Our main contribution is an improved motion planner where each vertex in the graph covers a collision-free region in the form of a sphere instead of a point and where the edges are formed with neighboring intersecting spheres. This representation has the advantage of instead of returning piecewise linear paths, returning a sequence of overlapping spheres, i.e. a convex corridor, that connects a given start and goal configuration, illustrated in Figure \ref{fig:pbrm_intro}. This convex corridor allows us to use convex optimization to produce smooth trajectories, instead of computationally expensive post-processing methods. The representation further allows us to estimate the coverage of the collision-free space, which gives us awareness and feedback in the offline roadmap construction phase. Finally, our representation is simple to adapt to moving obstacles, simply requery for the new radii and recheck for intersections. 
\\\\
The spherical collision-free regions are formed using a signed distance function (SDF), which is a function that returns the smallest distance from an arbitrary point to the boundary of an obstacle. As the name implies, the distance is signed, thus if the point is inside the obstacle it is negative otherwise positive. If the distance is positive, a sphere with radius equal to the distance is guaranteed to cover a collision-free region. Using an SDF in motion planning is not new, but what is novel about our approach is that we express the distance in the configuration space instead of the world space and by doing so allows us to form these convex collision-free regions. We refer to the resulting SDF as a signed configuration distance function (SCDF). Computing an SCDF analytically is non-trivial, our approach is therefore to parameterize the SCDF with a deep neural network and learn the mapping by supervised learning. Our resulting neural SCDF can compute distances for different parameter values of obstacle shapes and we also show how multiple distances can be combined, thus making our approach flexible.
\section{Related work}
Motion planning algorithms can roughly be divided into three families, grid-based, sampling based and optimization based methods. Grid-based methods (GBM) discretize the planning space from which a graph is then compiled. A standard search method is A$^\star$ \citep{a_star}, which is classified as an \textit{informed} search method, since it employs a heuristic function to speed up the search. A$^\star$ guarantees to return an optimal path at the level of discretization used. GBMs usually discretize the planning space by a regular lattice and this limits the GBMs to problems with low dimensionality due to the curse of dimensionality. Thus, GBMs are usually limited to single-body robots where the degrees of freedom (DOF) are low. To overcome the inherent scaling problem with the GBMs, stochastic methods are usually used for multi-body robots. These methods are termed as sampling-based methods (SBM) and core members within this family are the rapidly-exploring random trees (RRT) \citep{rrt} and the probabilistic roadmap (PRM) \citep{prm}. RRT grows a tree from the start configuration and explores the collision-free region in a rapid way until it is able to connect to the goal region. RRT is usually improved by bi-directional planning \citep{rrt_connect}, i.e. an additional tree is grown from the goal configuration and the trees are tested for connection after any tree has been expanded. RRT is a single-query method, thus it searches for a path from scratch each time it is queried. Contrary to this, PRM is a multi-query method, which solves for multiple queries without starting from scratch. PRM does this by creating a roadmap (graph) that covers the collision-free space as an offline step. The graph is then used to solve for multiple queries. PRMs are used in cases where the environment does not change since the extra offline step is too computationally costly and needs to be re-done if the environment is changed. In our work, we address this inherent issue by using a different roadmap representation. Our vertices in the graph cover a collision-free region in the form of spheres and we form the edges by checking for intersecting spheres. If something in the environment changes, we recompute the spheres radii and recheck the intersections, without relying on collision detection. We use a trained neural network to compute the sphere radius, therefore querying for the radius can be done fast, hence our representation enables the PRM for dynamic environments.
\\\\
In the recent decades, optimization based methods (OBM) \citep{chomp, schulman, itomp, stomp} have been introduced as an alternative to SBM for multi-body robots. Like the SBM, the OBMs scale well to higher dimensional problems and produce smoother motion. It is common to use a SDF in the optimization since it is a smooth function, thus enabling gradient-based methods. However, the standard way of expressing the SDF is in world space. The distance therefore needs to be mapped to the configuration space by the forward kinematics. This mapping makes the optimization problem a non-linear program (NLP), which is computationally expensive to solve. Recently, a different approach has been proposed. In \cite{mp_gcs} motion planning is formulated as a convex optimization problem by using the graph of convex sets framework \citep{gcs}. The underlying idea is to decompose the collision-free space into intersecting convex sets from which a convex optimization problem is formulated. In cases where an explicit representation of the obstacles in the configuration space exists, like for single-body robots, creating collision-free convex regions can be done fast \citep{iris}. For multi-body robots, this is non-trivial. Existing work does this successfully \citep{iris_nlp, iris_c} by an optimization based approach, but the methods are still too time consuming to be used in the presence of moving obstacles. Our approach is instead to use deep learning to learn an SDF expressed in the configuration space. With this, we can query for shortest distances to the collision boundary, which allows us to expand spherical regions which are collision-free. Our approach is fast and therefore enables our suggested roadmap planner to be used in dynamic environments.
\\\\
Recent research has focused on learning collision detection \citep{fk_kernel_distance, diffco, graphdistnet} by predicting the signed distance between the robot links and the surrounding obstacles in the world space. The learned SDF is used in trajectory optimization but since the distance is expressed in the world space, the problem becomes an NLP and therefore takes a long time to solve. We take a novel approach and suggest to instead express the signed distance in the configuration space. This allows us to improve the PRM at the same time as it enables convex optimization for trajectory optimization, which runs faster and is more reliable than NLP solvers. In \cite{cspf} a learned signed distance function in the configuration space is proposed similar to our approach. However, their approach is restricted to point cloud representations, while we propose to represent the obstacles as parameterized geometric shapes, e.g. spheres. Furthermore, we also show how to use our learned SCDF to improve an existing roadmap planner.
\section{Problem formulation}
A robot is located in the world space, $\W \subset \R^3 $. The unique location of the robot is given by its configuration $\q \in \C$, where $\C$ is the configuration space. The set of points covered by the robots bodies at a certain configuration is expressed as $\B(\q) \subset \W$. The robot is surrounded by $\NrObst$ obstacles $\O = \bigcup_{i=1}^{\NrObst} \O_i$, where  $\O_i \subset \W$. The representation of the obstacle in the configuration space is the set $\C\O_i = \{\q \in \C \: |\: \B(\q) \cap \O_i \neq \emptyset \}$. The obstacle space is formed as $\Co = \bigcup_{i=1}^{\NrObst} \C \O_i$. The complement is referred to as the free space, $\Cf = \C \setminus \Co$. The path planning problem is a tuple, ($\Cf$, $\qStart$, $\qGoal$), where we want to connect a query pair, consisting of a start, $\qStart$, and goal configuration, $\qGoal$, with a geometric path, $\q(s): [0, 1] \mapsto \Cf$, such that $\q(0)=\qStart$ and $\q(1)=\qGoal$, or report correctly when such a path does not exist.
\end{document}






\section{Related Work}
\label{sec:related}

\section{Related Work} \label{sec:related}

% \textbf{Adversarial Attack}
\textbf{Attacks on SLAM.} 
%With the rise of machine learning, 
The robustness of computer vision systems is being actively investigated. With the emergence of adversarial images in the digital domain by adding optimized noise directly to images~\cite{szegedy2013intriguing,carlini2017towards}, researchers find that such attacks also exist physically in the real world \cite{eykholt2018robust,song2018physical,zhao2019seeing}. To fill the gap between attacks in the digital and physical worlds, recent studies have demonstrated that attacks on real-world computer vision systems are practical \cite{eykholt2018robust,li2019adversarial,man2020ghostimage,sharif2016accessorize,zhao2019seeing,zhou2018invisible}. However, attacks on traditional computer vision methods such as SLAM are relatively less explored. \cite{yoshida2022adversarial} proposes an attack against the scan matching algorithm in LiDAR-based SLAM, while most SLAMs in AR/VR devices rely on different sensors like RGB/depth cameras and IMUs. \cite{ikram2022perceptual} and \cite{chen2024adversary} mislead visual SLAM by poisoning the images with special patterns, and \cite{wang2021can} causes the camera to fail using infrared light. In our work, we demonstrate attacks on Visual-Inertial SLAM (VI-SLAM) by perturbing the IMU readings, rather than cameras, and showing its impact on XR user experience. 

\textbf{Acoustic Injection Attacks.} Among various physical attacks, acoustic injection attacks are attractive due to their low cost. Son~\etal~\cite{son2015rocking} were the first to introduce acoustic attacks on MEMS gyroscopes, demonstrating how these attacks could lead to sensor denial-of-service and result in drone crashes. WALNUT~\cite{trippel2017walnut} expanded on this by developing output biasing and control attacks that enable precise manipulation of MEMS accelerometer outputs using modulated sound waves. Wang et al.~\cite{wang2017sonic} demonstrated a sonic gun, showcasing the vulnerability of various smart devices (\eg drones and self-balancing vehicles) to acoustic attacks. Tu et al. \cite{tu2018injected} designed side-swing and switching attacks to alter the outputs of MEMS gyroscopes and accelerometers. Furthermore, Ji et al. \cite{ji2021poltergeist} fool the object detectors by applying acoustic attack to the image stabilizers commonly used in modern cameras. However, none of the existing works study the relationship between the acoustic injections and SLAM outputs on recent XR devices. 

% \zijian{Do we need one session about security in AR/VR?}
% \yicheng{TODO}
%\jiasi{cite the AIVR paper (UMass Amherst?) paper is we have not already. They add IMU perturbation but w/o SLAM, iirc} \yicheng{Cited}

\textbf{XR Security and Privacy.} 
%Security and privacy concerns in XR systems have gained significant attention. 
For single-user XR systems, researchers have demonstrated various side-channel attacks to extract sensitive information (\eg keystrokes) through video feeds~\cite{ling2019know}, head movements~\cite{nair2023unique, slocum2023going}, architectural hints~\cite{zhang2023its,shang2020arspy}, power usage~\cite{li2024dangers}, and EM side-channel leakages~\cite{al2021vr}. In multi-user XR systems, Su et al.~\cite{su2024remote} use avatar motion data to infer keystrokes in shared VR environments. Slocum et al.~\cite{slocum2024doesn} reveal vulnerabilities in the shared state frameworks of multi-user AR. Similarly, Lebeck et al.~\cite{lebeck2017securing} highlight risks like deceptive virtual objects and emphasize access control for managing shared physical and virtual spaces. Ruth et al.~\cite{ruth2019secure} further propose a secure multi-user AR framework focusing on content sharing and permissions.
Chandio et al.~\cite{chandio2024stealthy} %introduced a multi-modal spatiotemporal attack that 
simultaneously manipulated visual and inertial sensors to disrupt XR pose estimation. However, their study evaluated the attack using offline datasets and assumed the attacker's capability to manipulate IMU data streams through acoustic means, without real experiments. Ours is the first to demonstrate acoustic injection attacks on recent XR devices, like the Hololens 2, in the real world.
 



\section{Method}
\label{sec:method}


\subsection{Scaling Laws}
For our scaling laws, we take a different approach from that in language \cite{hoffmann2022chinchilla} \cite{kaplan2020scalinglawsneurallanguage} and vision \cite{zhai2022scalingvits}. 
In these domains, since data was abundant and compute was the primary constraint, they never completed a full epoch, and thus equated number of steps to amount of data. 
For HAR, however, data is the primary constraint, so we repeat data many times (over 100 epochs) until convergence. Since we are able to train even our largest models to convergence, we do not fix the amount of compute. 
Instead, we focus on the capacity of the models (number of parameters). 
This is more directly tied to inference cost than training, which aligns with the priorities of many HAR deployments.


\subsection{Encoder}
For the encoder backbone, we use a ViT \cite{vit} adapted for accelerometer and gyroscope motion sensors as shown in \cref{mae-architecture}. The input to the encoder consists of a time series window of 128 samples at 50Hz, where each sample has 6 channels (x, y, z for accelerometer and gyroscope). We break the window into ``patches" of 4 samples. We choose this patch size to be as small as possible while still fitting the attention matrix in memory. 
Each patch of shape (4, 6) is flattened and transformed linearly to an embedding of dimension size equal to 1/4 the width of the MLP.

We use a standard Transformer block with 8 attention heads. To determine the optimal encoder capacity (i.e. number of parameters) for a given data scale, we conduct a grid search of 3 different widths (512, 1024, 2048 hidden MLP units) and 3 different depths (5, 10, 20 blocks), resulting in 9 different models from 1M to 63M parameters.


\begin{figure}[htb]
% \vskip 0.2in
\begin{center}
\centerline{\includegraphics[width=1\columnwidth]{HAR_Scaling_Paper_MAE_Diagram.png}}
\caption{Masked Autoencoder adapted for accelerometer and gyroscope. 
During pre-training, a random subset of accelerometer and gyroscope patches are masked out. Non-masked patches are passed to the encoder and the mask
tokens are re-introduced after the encoder. The encoded patches and mask tokens are then processed by a small decoder trained to reconstruct the original input sequence.
}
\label{mae-architecture}
\end{center}
% \vskip -0.2in
\end{figure}


\subsection{Pre-training}
Our pre-training approach follows Masked Autoencoder \cite{maskedautoencoder} closely, but adapted for accelerometer and gyroscope motion sensors as shown in \cref{mae-architecture}. This was chosen because it is simple to implement, scales well, and doesn’t require negative examples or domain specific design choices such as augmentations (e.g. to prevent collapse in dual encoders).
We randomly mask whole patches rather than individual samples. 
We only encode unmasked patches and use a high masking ratio of 70\%. This saves considerably on compute costs.
We use a small decoder consisting of 2 Transformer blocks. We apply a linear projection between the encoder and decoder to decrease the width of the decoder by half.

\subsection{Evaluation}
To evaluate pre-trained encoders, we remove the decoder and use global average pooling to attach a linear classification head to the output of the encoder, which we keep frozen. We use linear evaluation as opposed to full fine-tuning to provide a clearer signal of the information extracted from pre-training alone.

\subsection{Datasets}
We use the Extrasensory dataset \cite{extrasensory2017} for pre-training. This dataset contains more than 300k examples of 20 seconds of sensor data from 60 users. Data has been collected while subjects were engaged in regular everyday behavior for several sensors including accelerometer, gyroscope and magnetometer across multiple phone and wearable devices. The dataset is pre-formatted into 5 folds split by user. Each fold is split into a training and a test set. After filtering for missing data we collected 286140 examples, which equates to approximately 1589 hours of data. 
We ignore the activity labels for pre-training. Within each fold, we vary the amount of data by sampling some percentage of examples. 
The USER sampling strategy takes all the data from a randomly selected percentage of the users in the fold.
The RANDOM sampling strategy puts the data of all users in the fold together and draws a random percentage of examples from that.
To control for non-uniform distribution between different sampling strategies and folds, we report results based on the total hours of pre-training data rather than by percent. We only use the phone data, since the watch data does not contain a gyroscope.

For downstream evaluation, we use popular benchmark datasets UCI HAR and WISDM Phone/Watch datasets. We use the full training dataset for all supervised training.
UCI HAR \cite{human_activity_recognition_using_smartphones_240} contains data from 30 volunteers aged 19-48 engaging in 6 modes of locomotion and postures: walking, walking upstairs, walking downstairs, sitting, standing, laying. The accelerometer and gyroscope data is recorded at 50Hz from a smartphone worn on the waist. 
We use the same random partitioning prescribed by the dataset authors (70\% training and 30\% test sets). We also keep the existing raw data preprocessing pipeline, involving noise filtering, 2.56sec sliding windows with 50\% overlap, resulting in 128 samples per window, and use the raw acceleration, not the low-pass filtered version also present in the dataset.

For WISDM we use the 2019 version of the dataset \cite{wisdm_smartphone_and_smartwatch_activity_and_biometrics_dataset__507} comprising 51 subjects performing 18 activities of daily living (postures, locomotion, house chores, nutrition, work-related activities and others) for 3 minutes each. 
We assign the first 2/3 of users to the training set (subjects 1600-1633), and use the remaining 1/3 for evaluation (subjects 1634-1650) similar to previous benchmarks. We split the WISDM dataset into Phone and Watch body positions and evaluate these separately.

We re-sample all datasets to 50Hz and normalize to the same units. None of the datasets contain a null class.

\subsection{Training schedule}
For every model, data combination, we fix the number of steps for pre-training to 500,000 with a batch size of 2048. This equates to over 100 epochs when using 100\% of the data. We use the Adam optimizer with three different learning rates (1e-3, 1e-4, 1e-5) for every model and take the best result. This ensures that each model has sufficient coverage of the parameter search space regardless of size. We apply dropout during pre-training with a rate of 0.1.

\subsection{Compute}
Our exhaustive grid search results in 1620 (3 learning rates * 6 data sizes * 2 sampling strategies * 5 folds * 9 encoder architectures) different hyperparameter combinations for pre-training. Each run takes between 3 and 35 hours to run on 4 TPUv2 chips with larger models running longer. Our total compute used for pre-training is about 62000 TPU-hours.





\begin{table}[htb]
\caption{Best F1 scores of models trained from scratch (FS) vs linear eval (LE) on pre-trained models for each dataset.} 
\label{f1-table}
% \vskip 0.15in
\begin{center}
\begin{small}
\begin{sc}
\begin{tabular}{lcccr}
\toprule
Data set & FS & LE \\
\midrule
UCI HAR    & 95.1 & 97.9 \\
WISDM Phone    & 31.9 & 34.3 \\
WISDM Watch & 62.6 & 63.1 \\
\bottomrule
\end{tabular}
\end{sc}
\end{small}
\end{center}
% \vskip -0.1in
\vspace{-1.5em}
\end{table}



\section{Results}

\subsection{Supervised training from scratch baseline}
For each dataset, we conduct a thorough capacity and hyperparameter search of from-scratch models to establish baselines for comparing with pre-trained models. The best results are listed in \cref{f1-table}. We also look at the effect of model capacity on from scratch training. We find that smaller models work better for UCI HAR, with our second smallest model of about 2M parameters performing best. The effect of capacity on WISDM is less clear. It also appears that deeper models perform better than wide models.

\begin{figure}[t]
% \vskip 0.2in
\begin{center}
\centerline{\includegraphics[width=0.7\columnwidth]{pretrain_eval_loss_data_law.png}}
\caption{Pre-training test loss vs data size (hours). We fit a power law to each fold and sampling strategy. Equations for each power law can be found in \cref{pretrain-data-law-table}.}
\label{pretrain-loss-data}
\end{center}
% \vskip -0.2in
\end{figure}

\begin{table}[t]
\caption{Power laws of pre-training test loss vs data size. Exponent values for the USER sampling strategy are roughly 3 times greater than RANDOM.}
\label{pretrain-data-law-table}
% \vskip 0.15in
\begin{center}
\begin{small}
\begin{sc}
\begin{tabular}{lcccr}
\toprule
Fold & \multicolumn{2}{c}{Sampling Strategy} \\
& USER & RANDOM \\
\midrule
0    & $L = 0.058D^{-0.049}$ & $L = 0.046D^{-0.017}$ \\
1    & $L = 0.057D^{-0.045}$ & $L = 0.047D^{-0.015}$ \\
2    & $L = 0.052D^{-0.046}$ & $L = 0.043D^{-0.020}$ \\
3    & $L = 0.051D^{-0.052}$ & $L = 0.040D^{-0.019}$ \\
4    & $L = 0.043D^{-0.044}$ & $L = 0.035D^{-0.016}$ \\
\bottomrule
\end{tabular}
\end{sc}
\end{small}
\end{center}
% \vskip -0.1in
\end{table}





\begin{figure}[htb]
%\vskip 0.2in
\begin{center}
\centerline{\includegraphics[width=0.7\columnwidth]{pretrain_eval_loss_capacity_law.png}}
\caption{Pre-training test loss vs model capacity (number of parameters) and associated power law fit and equation.}
\label{pretrain-loss-capacity}
\end{center}
%\vskip -0.2in
\vspace{-1em}
\end{figure}




\begin{figure*}[htb]
%\vskip 0.2in
\begin{center}
\centerline{\includegraphics[width=0.8\linewidth]{pretrain_f1_vs_data.png}}
\caption{Best linear F1 scores vs pre-training dataset size (hours). Each point represents the best F1 score corresponding to a pre-training fold and data size. The best score is chosen from 27 runs consisting of the 9 encoder architectures and 3 learning rates in our search space.}
\label{pretrain-f1-data}
\end{center}
%\vskip -0.2in
\vspace{-1em}
\end{figure*}




\begin{figure*}[htb]
% \vskip 0.2in
\begin{center}
\centerline{\includegraphics[width=0.8\textwidth]{pretrain_capacity.png}}
\caption{Best linear {F1} scores vs model capacity (number of parameters). Each point represents the best F1 score corresponding to an encoder architecture (width and depth). The best score is chosen from all data sizes and learning rates. We indicate the width (mlp hidden dim) by color. At 5M or 20M parameters we have two models that are the same size, with one wider and shallower (5 blocks) and the other narrower and deeper (20 blocks).}
\label{pretrain-f1-capacity}
\end{center}
% \vskip -0.2in
\end{figure*}




\subsection{Scaling laws}
\label{sec:result_scaling}


In \cref{pretrain-loss-data} we establish scaling laws of the pre-training test loss vs hours of data. To calculate the loss, we use the full test set from each Extrasensory fold. 
This allows us to compare different training data amounts and distributions within a fold and fit power-law relationships for each fold. 
We observe roughly the same power-law exponent (or slope on the log-log plot) for a given fold and sampling strategy, giving confidence that this relationship was not due to random chance. 
Furthermore, in \cref{pretrain-data-law-table} we see that the exponent is roughly of 3x greater magnitude (or steeper slope) when data is increased by adding more users, as opposed to uniformly or per-user. This emphasizes that diversity of data is extremely important, and dictates the scaling law. Note that the offset is different for each fold, but that is to be expected, since the test sets are different. 
Similarly, in \cref{pretrain-loss-capacity} we fit a power law between pre-training test loss and model capacity in terms of number of parameters, further demonstrating the existence of a scaling law.



\subsection{Downstream performance}
\label{sec:downstream_performance}

We show that the scaling laws for pre-training translate to similar trends in improved downstream linear classification performance. For each pre-training dataset size, we plot the best F1 score from linear evaluation on downstream datasets UCI HAR and WISDM Phone/Watch. This can be seen in \cref{pretrain-f1-data}. Contrary to previous findings \cite{assessingSSLHAR22,howmuchdataSSLHAR23}, we see consistent improvement as we scale the data size. For UCI HAR, we reach 97.9\% F1 score with linear evaluation. 
To our knowledge, this is on par with the best reported result (98.6\% from \cite{uciharsota}) for this dataset. For all datasets, we surpass from scratch baseline results, with significant improvement for phone datasets UCI HAR (+2.8pp) and WISDM Phone (+2.4pp). 
For WISDM Watch, the improvement is smaller (+0.5pp). This is not surprising given that our pre-training dataset consists of only phone data. Still, the consistent increase in watch performance suggests that we are seeing positive transfer between body positions.


\begin{figure}[htb]
%\vskip 0.2in
\begin{center}
\centerline{\includegraphics[width=\columnwidth]{capacity_vs_data.png}}
\caption{Optimal capacity for a given pre-training data size. Each plot shows the parameter count of the model resulting in the best performance for a given metric (pre-train test loss on the left, UCI HAR test F1 on the right).}
\label{capacity-vs-data}
\end{center}
%\vskip -0.2in
\vspace{-1em}
\end{figure}

In \cref{pretrain-f1-capacity} we study the effect the capacity of the encoder has on downstream performance. For each of the 9 encoder architectures (3 widths by 3 depths), we plot the best F1 score out of 180 runs (5 folds * 6 data sizes * 2 sampling strategies * 3 learning rates) from linear evaluation on downstream datasets UCI HAR and WISDM vs the number of parameters. We find that increasing the number of parameters is crucial to realizing performance improvements across all 3 tasks. The optimal capacity is reached at our biggest model which has about 63M parameters. This is in contrast to our from scratch baselines, where performance can peak at smaller models (e.g. about 2M parameters for UCI HAR).

\begin{figure*}[t]
%\vskip 0.2in
\begin{center}
\centerline{\includegraphics[width=0.8\textwidth]{augmentations_capacity.png}}
\caption{Best linear {F1} scores vs model capacity (number of parameters). Each point represents the best F1 score corresponding to an encoder architecture (width and depth) and whether augmentations were on or off. The best score is chosen from all data sizes and learning rates.}
\label{pretrain-augmentations}
\end{center}
%\vskip -0.2in
\vspace{-1em}
\end{figure*}



\subsection{Optimal capacity vs data size}
In \cref{capacity-vs-data} we study the optimal capacity for a given pre-training data size. For pre-training test loss, we find that optimal model size increases monotonically with more data. Downstream F1 performance tells a different story, with our largest models performing best even with minimal data. We hypothesize that this may be due to epoch-wise double descent \cite{double-descent} behaviour, which we have observed in some cases in this work.

\subsection{Augmentations}
\label{sec:augmentations}
We apply augmentations during pre-training, and study the effect on downstream model performance. In \cref{pretrain-augmentations} We separate results by encoder as in \cref{pretrain-f1-capacity}, but take the best score both with and without augmentations. We see that augmentations always improve performance, especially at larger scales. The optimal capacity without augmentations can be smaller, at either the 4th largest model (about 10M parameters) for UCI HAR or the 3rd largest model (20M parameters) for WISDM Phone. This is not surprising, since augmentations can be thought of as a strong regularizer, and/or an artificial expansion of the dataset.

\subsection{Width vs Depth}
Since we conducted a grid search on width and depth, we have results for a variety of width to depth ratios. There are also 2 model sizes (20M and 5M) for which we have a very deep model (20 blocks) and a very wide model (5 blocks) with the same number of parameters. This allows us to control for the total number of parameters. Looking at \cref{pretrain-f1-capacity}, we can see that increasing both width and depth improve performance, but wider models tend to perform better than deeper models for the same number of parameters.

\section{Conclusion}
\label{sec:conclusion}
\section*{Conclusion}
This paper aims to enhance our understanding of the computational complexity of computing various Shapley value variants. We found that for various ML models --- including decision trees, regression tree ensembles, weighted automata, and linear regression --- both local and global interventional and baseline SHAP can be computed in polynomial time under HMM modeled distributions. This extends popular algorithms, such as TreeSHAP, beyond their empirical distributional scope. We also establish strict complexity gaps between the various SHAP variants (baseline, interventional, and conditional) and prove the intractability of computing SHAP for tree ensembles and neural networks in simplified scenarios. Overall, we present SHAP as a versatile framework whose complexity depends on four key factors: \begin{inparaenum}[(i)] \item model type, \item SHAP variant, \item distribution modeling approach, \item and local vs. global explanations\end{inparaenum}. We believe this perspective provides deeper insight into the computational complexity of SHAP, paving the way for future work.




%We believe that our framework provides a more intricate understanding of SHAP computation complexity across different models, distributions, and variants, paving the way for further research.

Our work opens promising directions for future research. First, expanding our computational analysis to other SHAP-related metrics, such as asymmetric SHAP~\citep{frye20} and SAGE~\citep{covert2020understanding}, would be valuable. Additionally, we aim to explore more expressive distribution classes and relaxed assumptions beyond those in Section \ref{sec:tractable} while maintaining tractable SHAP computation. Finally, when exact computation is intractable (Section \ref{sec:intractable}), investigating the approximability of SHAP metrics through approximation and parameterized complexity theory~\citep{downey2012parameterized} is an important direction.

%Our work opens several promising avenues for future research on the computational properties of explainable AI methods, with a particular focus on SHAP. First, it would be interesting to broaden the computational analysis conducted in this work to include other popular SHAP-related metrics in the literature, such as asymmetric SHAP \cite{frye20} and SAGE \cite{covert2020understanding}. Also, in the future, we aim to explore more expressive distribution classes and relaxed distributional assumptions—extending beyond those examined in Section \ref{sec:tractable} —that still yield tractable SHAP computation. Finally, when exact computation proves intractable (Section \ref{sec:intractable}), it is worthwhile to theoretically investigate the question of the approximability of computing the SHAP metrics across various configurations, through the lens of approximation and parametrized complexity theory \cite{arora2009computational}.

%This paper aims to deepen our understanding of the computational complexity involved in obtaining different Shapley value variants. We found that for a variety of ML models, including decision trees, tree ensembles for regression, weighted automata, and linear regression models — computing both local and global interventional and baseline SHAP can be done in polynomial time when distributions are modeled by HMMs. This extends the distributional scope of popular algorithms like TreeSHAP, which is limited to empirical distributions. Additionally, we demonstrate a strict complexity gap between SHAP variants, showing that interventional and baseline SHAP can be strictly easier to compute than conditional SHAP. Despite these positive results, we uncovered intractability for various SHAP variants in neural networks and tree ensembles. Finally, we provided generalized complexity relations across SHAP variants. We believe that our framework offers a deeper understanding of the complexity involved in computing SHAP across various variants, models, distributions, as well as in both local and global computations, laying the groundwork for future research.

% Bibliography components
\bibliographystyle{abbrvnat}
\nobibliography*
\bibliography{main}

\end{document}
