% CVPR 2024 Paper Template; see https://github.com/cvpr-org/author-kit

\documentclass[10pt,twocolumn,letterpaper]{article}

%%%%%%%%% PAPER TYPE  - PLEASE UPDATE FOR FINAL VERSION
\usepackage{cvpr}              % To produce the CAMERA-READY version
% \usepackage[review]{cvpr}      % To produce the REVIEW version
% \usepackage[pagenumbers]{cvpr} % To force page numbers, e.g. for an arXiv version
\usepackage{graphicx}
\usepackage{amsmath}
\usepackage{amssymb}
\usepackage{booktabs}
\usepackage{makecell}  % for \makecell command
\usepackage{multirow}
\usepackage{textcomp}
\usepackage{comment}
% Import additional packages in the preamble file, before hyperref
%
% --- inline annotations
%
\newcommand{\red}[1]{{\color{red}#1}}
\newcommand{\todo}[1]{{\color{red}#1}}
\newcommand{\TODO}[1]{\textbf{\color{red}[TODO: #1]}}
% --- disable by uncommenting  
% \renewcommand{\TODO}[1]{}
% \renewcommand{\todo}[1]{#1}



\newcommand{\VLM}{LVLM\xspace} 
\newcommand{\ours}{PeKit\xspace}
\newcommand{\yollava}{Yo’LLaVA\xspace}

\newcommand{\thisismy}{This-Is-My-Img\xspace}
\newcommand{\myparagraph}[1]{\noindent\textbf{#1}}
\newcommand{\vdoro}[1]{{\color[rgb]{0.4, 0.18, 0.78} {[V] #1}}}
% --- disable by uncommenting  
% \renewcommand{\TODO}[1]{}
% \renewcommand{\todo}[1]{#1}
\usepackage{slashbox}
% Vectors
\newcommand{\bB}{\mathcal{B}}
\newcommand{\bw}{\mathbf{w}}
\newcommand{\bs}{\mathbf{s}}
\newcommand{\bo}{\mathbf{o}}
\newcommand{\bn}{\mathbf{n}}
\newcommand{\bc}{\mathbf{c}}
\newcommand{\bp}{\mathbf{p}}
\newcommand{\bS}{\mathbf{S}}
\newcommand{\bk}{\mathbf{k}}
\newcommand{\bmu}{\boldsymbol{\mu}}
\newcommand{\bx}{\mathbf{x}}
\newcommand{\bg}{\mathbf{g}}
\newcommand{\be}{\mathbf{e}}
\newcommand{\bX}{\mathbf{X}}
\newcommand{\by}{\mathbf{y}}
\newcommand{\bv}{\mathbf{v}}
\newcommand{\bz}{\mathbf{z}}
\newcommand{\bq}{\mathbf{q}}
\newcommand{\bff}{\mathbf{f}}
\newcommand{\bu}{\mathbf{u}}
\newcommand{\bh}{\mathbf{h}}
\newcommand{\bb}{\mathbf{b}}

\newcommand{\rone}{\textcolor{green}{R1}}
\newcommand{\rtwo}{\textcolor{orange}{R2}}
\newcommand{\rthree}{\textcolor{red}{R3}}
\usepackage{amsmath}
%\usepackage{arydshln}
\DeclareMathOperator{\similarity}{sim}
\DeclareMathOperator{\AvgPool}{AvgPool}

\newcommand{\argmax}{\mathop{\mathrm{argmax}}}     



% It is strongly recommended to use hyperref, especially for the review version.
% hyperref with option pagebackref eases the reviewers' job.
% Please disable hyperref *only* if you encounter grave issues, 
% e.g. with the file validation for the camera-ready version.
%
% If you comment hyperref and then uncomment it, you should delete *.aux before re-running LaTeX.
% (Or just hit 'q' on the first LaTeX run, let it finish, and you should be clear).
\definecolor{cvprblue}{rgb}{0.21,0.49,0.74}
\usepackage[pagebackref,breaklinks,colorlinks,citecolor=cvprblue]{hyperref}

%%%%%%%%% PAPER ID  - PLEASE UPDATE
\def\paperID{7631} % *** Enter the Paper ID here
\def\confName{CVPR}
\def\confYear{2024}

%%%%%%%%% TITLE - PLEASE UPDATE
\title{EyeBench: A Call for More Rigorous Evaluation of Retinal Image Enhancement}

%%%%%%%%% AUTHORS - PLEASE UPDATE
\author{Wenhui Zhu$^{1*}$\enspace Xuanzhao Dong $^{1*}$ \enspace  Xin Li$^{1*}$\enspace Yujian Xiong$^{1*}$ \enspace Xiwen Chen $^{2}$ \\
\enspace Peijie Qiu$^{3}$ \enspace Vamsi Krishna Vasa$^{1}$ \enspace Zhangsihao Yang$^{1}$ \enspace Yi Su$^{4}$ \enspace Oana Dumitrascu$^{5}$ \enspace Yalin Wang$^{1}$ \\
$^{1}$ Arizona State University,  
$^{2}$ Clemson University, 
$^{3}$ Washington University in St. Louis, \\
$^{4}$ Banner Alzheimer's Institute, 
$^{5}$ Mayo Clinic
% {\tt\small xiwenc@g.clemson.edu, arazi@clemson.edu }
% For a paper whose authors are all at the same institution,
% omit the following lines up until the closing ``}''.
% Additional authors and addresses can be added with ``\and'',
% just like the second author.
% To save space, use either the email address or home page, not both
}

\begin{document}


\twocolumn[{%
\renewcommand\twocolumn[1][]{#1}%
\maketitle
\begin{center}
    \centering
    \captionsetup{type=figure}
    \includegraphics[width=1.0\textwidth]{Figures/fig1_v4.jpg}
    \captionof{figure}{ 
    \textbf{Overview of EyeBench.} We introduce EyeBench, a systematic and rigorous benchmark for evaluating retinal image enhancement models. Our evaluation pipeline comprehensively assesses fundus image enhancement quality through both No-Reference and Full-Reference aspects, facilitating a multi-dimensional evaluation. For each aspect, we design a \textbf{distribution-aligned} dataset to ensure fair and clinically meaningful comparisons. 
    Additionally, we include \textbf{clinically consistent} downstream tasks to quantify models' ability in denoising generalization and downstream preserving. Our benchmark also incorporates \textbf{medical experts guided annotations}, adhering to expert protocols, and we statistically validate that EyeBench results \textbf{aligned well with clinic preference assessment}. Finally, we highlight current challenges to inform future development. EyeBench can provide multiple insights from multiple perspectives. 
    }
    \label{banner}
\end{center}%
}]

\def\thefootnote{*}\footnotetext{These authors contributed equally to this paper.}

% \begin{figure*}[!t]
%   \centering
%   \includegraphics[width=\textwidth]{Figures/Banner.png}  % Replace with your figure file and remove the example
%   \caption{The figure presents the analysis of the images in the whole dataset.}
%   \label{fig:overall_ana}
% \end{figure*}

\begin{abstract}
Over the past decade, generative models have achieved significant success in enhancement fundus images.
However, the evaluation of these models still presents a considerable challenge. A comprehensive evaluation benchmark for fundus image enhancement is indispensable for three main reasons: 1) The existing denoising metrics (e.g., PSNR, SSIM) are hardly to extend to downstream real-world clinical research (e.g., lesion preserving, Vessel morphology consistency). 2) There is a lack of comprehensive evaluation for both paired and unpaired enhancement methods, along with the need for  expert protocols to accurately assess clinical value.  3) An ideal evaluation system should provide insights to inform future developments of fundus image enhancement. To this end, we propose a novel comprehensive benchmark, \textbf{EyeBench}, to provide insights that align enhancement models with clinical needs, offering a foundation for future work to improve the clinical relevance and applicability of generative models for fundus image enhancement. EyeBench has three appealing properties: \textbf{1)  multi-dimensional clinical alignment downstream evaluation:} In addition to evaluating the enhancement task, we provide several clinically significant downstream tasks for fundus images, including vessel segmentation, DR grading, denoising generalization, and lesion segmentation.  \textbf{2) Medical expert-guided evaluation design:} We introduce a novel dataset that facilitates comprehensive and fair comparisons between paired and unpaired methods and includes a manual evaluation protocol by medical experts (e.g., the ratio of lesion structure changed, background-color changed, and extra structures generated). \textbf{3) Valuable insights:} Our benchmark study provides a comprehensive and rigorous evaluation of existing methods across different downstream tasks, assisting medical experts in making informed choices. Additionally, we offer analysis of the challenges faced by existing methods, which would shine a light for the further design of generative models for fundus image enhancement.The code is available at \url{https://github.com/Retinal-Research/EyeBench}



\end{abstract}
\section{Introduction}
Non-mydriatic retinal color fundus photography (CFP) is widely used in various fundus disease analyses due to the advantage of not requiring pupillary dilation~\cite{deeplearning1,deeplearning2,deeplearning3,deeplearning4,deeplearning5,deeplearning6}. However, it commonly suffers low quality due to artifacts, uneven illumination, deficient ocular media transparency, poor focus, or inappropriate imaging~\cite{shen2020modeling,fu2019evaluation}. Recently, fundus image enhancement has witnessed significant advancements with the rapid development of the generative model. 
Since these models are not constrained by the need for paired data, a growing number of unpaired image enhancement models have been developed, showing performance comparable to paired methods~\cite{wang2022optimal,zhu2023optimal, zhu2023otre,dong2024cunsb,vasa2024context}. At the same time, due to the difficulty of collecting paired noisy and high-quality images, medical experts now show a stronger preference for these unpaired methods. However, the existing evaluation metrics for fundus image enhancement still comply with the supervised denoising task where the low-high quality fundus image pair synthesis by adding known noises (e.g., Gaussian blur, white noise) into real-world high-quality images. This evaluation heavily relies on conventional metrics such as SSIM and PSNR, which fall short of thoroughly assessing the denoising capabilities and similarity between the latent representations of enhanced images and real high-quality images. Moreover, enhancement evaluation alone does not meet clinical requirements, and a rigorous evaluation framework is needed for both unpaired and paired methods to ensure comprehensive assessment. In this paper, we introduce EyeBench, a comprehensive and rigorous benchmark for evaluating fundus image enhancement methods, which includes the multi-dimensional clinical alignment downstream
evaluation, medical expert-guided evaluation design, and valuable insights. 

First, our benchmark introduces a set of downstream tasks to assess enhanced fundus images, breaking down enhancement quality into clinical preferences, specifically focusing on preserving vessels, disease grading, and lesion structures. We train existing enhancement methods within a standardized framework and apply them to improve fundus image quality for each downstream task. These enhanced images are then processed through respective evaluation workflows for further analysis. As shown in Fig.~\ref{banner}, the downstream tasks include enhancement generalization, vessel segmentation, lesion segmentation, representation, and diabetic retinopathy grading (DR grading). These tasks assess the discrepancies between the generated masks, labels, or representations of the enhanced images and the ground truth or high-quality images. This allows us to determine if vessel structures remain intact and lesion areas are preserved. In addition, the proposed evaluation assesses the enhancement performance and improves the credibility of different enhancement methods for clinical applications.

Second, we annotate unusable images and resample the labels for disease severity levels for each sub-set following guidance from medical experts. To facilitate a more rigorous comparison between paired and unpaired methods under real-world (no-reference) and synthetic (full-reference) noise, we propose a new dataset specifically designed to include dedicated training and testing sets for both paired and unpaired methods under full-reference conditions, allowing for the evaluation of denoising and various downstream tasks. Under no-reference conditions, we restructure the training and testing sets to assess the performance of unpaired methods. Furthermore, we introduce an expert manual evaluation protocol, as shown in Fig.~\ref{banner}, to align with clinical preference by assessing enhanced images. We also conducted a statistical analysis of expert annotation evaluation and EyeBench to validate the necessity of multi-dimensional evaluation.


Third, our multi-dimensional evaluation result (see Fig.~\ref{banner}) will assist medical experts in selecting the appropriate enhancement methods to improve the reliability of subsequent diagnoses and analyses. Specifically for clinically valuable unpaired methods, we provide a detailed analysis of denoising generalization. Additionally, we offer comprehensive analyses and insights into the challenges that these existing methods face and insights for future works.
% WARNING: do not forget to delete the supplementary pages from your submission 
% \clearpage
\pagenumbering{gobble}
\maketitlesupplementary

\section{Additional Results on Embodied Tasks}

To evaluate the broader applicability of our EgoAgent's learned representation beyond video-conditioned 3D human motion prediction, we test its ability to improve visual policy learning for embodiments other than the human skeleton.
Following the methodology in~\cite{majumdar2023we}, we conduct experiments on the TriFinger benchmark~\cite{wuthrich2020trifinger}, which involves a three-finger robot performing two tasks: reach cube and move cube. 
We freeze the pretrained representations and use a 3-layer MLP as the policy network, training each task with 100 demonstrations.

\begin{table}[h]
\centering
\caption{Success rate (\%) on the TriFinger benchmark, where each model's pretrained representation is fixed, and additional linear layers are trained as the policy network.}
\label{tab:trifinger}
\resizebox{\linewidth}{!}{%
\begin{tabular}{llcc}
\toprule
Methods       & Training Dataset & Reach Cube & Move Cube \\
\midrule
DINO~\cite{caron2021emerging}         & WT Venice        & 78.03     & 47.42     \\
DoRA~\cite{venkataramanan2023imagenet}          & WT Venice        & 81.62     & 53.76     \\
DoRA~\cite{venkataramanan2023imagenet}          & WT All           & 82.40     & 48.13     \\
\midrule
EgoAgent-300M & WT+Ego-Exo4D      & 82.61    & 54.21      \\
EgoAgent-1B   & WT+Ego-Exo4D      & \textbf{85.72}      & \textbf{57.66}   \\
\bottomrule
\end{tabular}%
}
\end{table}

As shown in Table~\ref{tab:trifinger}, EgoAgent achieves the highest success rates on both tasks, outperforming the best models from DoRA~\cite{venkataramanan2023imagenet} with increases of +3.32\% and +3.9\% respectively.
This result shows that by incorporating human action prediction into the learning process, EgoAgent demonstrates the ability to learn more effective representations that benefit both image classification and embodied manipulation tasks.
This highlights the potential of leveraging human-centric motion data to bridge the gap between visual understanding and actionable policy learning.



\section{Additional Results on Egocentric Future State Prediction}

In this section, we provide additional qualitative results on the egocentric future state prediction task. Additionally, we describe our approach to finetune video diffusion model on the Ego-Exo4D dataset~\cite{grauman2024ego} and generate future video frames conditioned on initial frames as shown in Figure~\ref{fig:opensora_finetune}.

\begin{figure}[b]
    \centering
    \includegraphics[width=\linewidth]{figures/opensora_finetune.pdf}
    \caption{Comparison of OpenSora V1.1 first-frame-conditioned video generation results before and after finetuning on Ego-Exo4D. Fine-tuning enhances temporal consistency, but the predicted pixel-space future states still exhibit errors, such as inaccuracies in the basketball's trajectory.}
    \label{fig:opensora_finetune}
\end{figure}

\subsection{Visualizations and Comparisons}

More visualizations of our method, DoRA, and OpenSora in different scenes (as shown in Figure~\ref{fig:supp pred}). For OpenSora, when predicting the states of $t_k$, we use all the ground truth frames from $t_{0}$ to $t_{k-1}$ as conditions. As OpenSora takes only past observations as input and neglects human motion, it performs well only when the human has relatively small motions (see top cases in Figure~\ref{fig:supp pred}), but can not adjust to large movements of the human body or quick viewpoint changes (see bottom cases in Figure~\ref{fig:supp pred}).

\begin{figure*}
    \centering
    \includegraphics[width=\linewidth]{figures/supp_pred.pdf}
    \caption{Retrieval and generation results for egocentric future state prediction. Correct and wrong retrieval images are marked with green and red boundaries, respectively.}
    \label{fig:supp pred}
\end{figure*}

\begin{figure*}[t]
    \centering
    \includegraphics[width=0.9\linewidth]{figures/motion_prediction.pdf}
    \vspace{-0.5mm}
    \caption{Motion prediction results in scenes with minor changes in observation.}
    \vspace{-1.5mm}
    \label{fig:motion_prediction}
\end{figure*}

\subsection{Finetuning OpenSora on Ego-Exo4D}

OpenSora V1.1~\cite{opensora}, initially trained on internet videos and images, produces severely inconsistent results when directly applied to infer future videos on the Ego-Exo4D dataset, as illustrated in Figure~\ref{fig:opensora_finetune}.
To address the gap between general internet content and egocentric video data, we fine-tune the official checkpoint on the Ego-Exo4D training set for 50 epochs.
OpenSora V1.1 proposed a random mask strategy during training to enable video generation by image and video conditioning. We adopted the default masking rate, which applies: 75\% with no masking, 2.5\% with random masking of 1 frame to 1/4 of the total frames, 2.5\% with masking at either the beginning or the end for 1 frame to 1/4 of the total frames, and 5\% with random masking spanning 1 frame to 1/4 of the total frames at both the beginning and the end.

As shown in Fig.~\ref{fig:opensora_finetune}, despite being trained on a large dataset, OpenSora struggles to generalize to the Ego-Exo4D dataset, producing future video frames with minimal consistency relative to the conditioning frame. While fine-tuning improves temporal consistency, the moving trajectories of objects like the basketball and soccer ball still deviate from realistic physical laws. Compared with our feature space prediction results, this suggests that training world models in a reconstructive latent space is more challenging than training them in a feature space.


\section{Additional Results on 3D Human Motion Prediction}

We present additional qualitative results for the 3D human motion prediction task, highlighting a particularly challenging scenario where egocentric observations exhibit minimal variation. This scenario poses significant difficulties for video-conditioned motion prediction, as the model must effectively capture and interpret subtle changes. As demonstrated in Fig.~\ref{fig:motion_prediction}, EgoAgent successfully generates accurate predictions that closely align with the ground truth motion, showcasing its ability to handle fine-grained temporal dynamics and nuanced contextual cues.

\section{OpenSora for Image Classification}

In this section, we detail the process of extracting features from OpenSora V1.1~\cite{opensora} (without fine-tuning) for an image classification task. Following the approach of~\cite{xiang2023denoising}, we leverage the insight that diffusion models can be interpreted as multi-level denoising autoencoders. These models inherently learn linearly separable representations within their intermediate layers, without relying on auxiliary encoders. The quality of the extracted features depends on both the layer depth and the noise level applied during extraction.


\begin{table}[h]
\centering
\caption{$k$-NN evaluation results of OpenSora V1.1 features from different layer depths and noising scales on ImageNet-100. Top1 and Top5 accuracy (\%) are reported.}
\label{tab:opensora-knn}
\resizebox{0.95\linewidth}{!}{%
\begin{tabular}{lcccccc}
\toprule
\multirow{2}{*}{Timesteps} & \multicolumn{2}{c}{First Layer} & \multicolumn{2}{c}{Middle Layer} & \multicolumn{2}{c}{Last Layer} \\
\cmidrule(r){2-3}   \cmidrule(r){4-5}  \cmidrule(r){6-7}  & Top1           & Top5           & Top1            & Top5           & Top1           & Top5          \\
\midrule
32        &  6.10           & 18.20             & 34.04               & 59.50             & 30.40             & 55.74             \\
64        & 6.12              & 18.48              & 36.04               & 61.84              & 31.80         & 57.06         \\
128       & 5.84             & 18.14             & 38.08               & 64.16              & 33.44       & 58.42 \\
256       & 5.60             & 16.58              & 30.34               & 56.38              &28.14          & 52.32        \\
512       & 3.66              & 11.70            & 6.24              & 17.62              & 7.24              & 19.44  \\ 
\bottomrule
\end{tabular}%
}
\end{table}

As shown in Table~\ref{tab:opensora-knn}, we first evaluate $k$-NN classification performance on the ImageNet-100 dataset using three intermediate layers and five different noise scales. We find that a noise timestep of 128 yields the best results, with the middle and last layers performing significantly better than the first layer.
We then test this optimal configuration on ImageNet-1K and find that the last layer with 128 noising timesteps achieves the best classification accuracy.

\section{Data Preprocess}
For egocentric video sequences, we utilize videos from the Ego-Exo4D~\cite{grauman2024ego} and WT~\cite{venkataramanan2023imagenet} datasets.
The original resolution of Ego-Exo4D videos is 1408×1408, captured at 30 fps. We sample one frame every five frames and use the original resolution to crop local views (224×224) for computing the self-supervised representation loss. For computing the prediction and action loss, the videos are downsampled to 224×224 resolution.
WT primarily consists of 4K videos (3840×2160) recorded at 60 or 30 fps. Similar to Ego-Exo4D, we use the original resolution and downsample the frame rate to 6 fps for representation loss computation.
As Ego-Exo4D employs fisheye cameras, we undistort the images to a pinhole camera model using the official Project Aria Tools to align them with the WT videos.

For motion sequences, the Ego-Exo4D dataset provides synchronized 3D motion annotations and camera extrinsic parameters for various tasks and scenes. While some annotations are manually labeled, others are automatically generated using 3D motion estimation algorithms from multiple exocentric views. To maximize data utility and maintain high-quality annotations, manual labels are prioritized wherever available, and automated annotations are used only when manual labels are absent.
Each pose is converted into the egocentric camera's coordinate system using transformation matrices derived from the camera extrinsics. These matrices also enable the computation of trajectory vectors for each frame in a sequence. Beyond the x, y, z coordinates, a visibility dimension is appended to account for keypoints invisible to all exocentric views. Finally, a sliding window approach segments sequences into fixed-size windows to serve as input for the model. Note that we do not downsample the frame rate of 3D motions.

\section{Training Details}
\subsection{Architecture Configurations}
In Table~\ref{tab:arch}, we provide detailed architecture configurations for EgoAgent following the scaling-up strategy of InternLM~\cite{team2023internlm}. To ensure the generalization, we do not modify the internal modules in InternML, \emph{i.e.}, we adopt the RMSNorm and 1D RoPE. We show that, without specific modules designed for vision tasks, EgoAgent can perform well on vision and action tasks.

\begin{table}[ht]
  \centering
  \caption{Architecture configurations of EgoAgent.}
  \resizebox{0.8\linewidth}{!}{%
    \begin{tabular}{lcc}
    \toprule
          & EgoAgent-300M & EgoAgent-1B \\
          \midrule
    Depth & 22    & 22 \\
    Embedding dim & 1024  & 2048 \\
    Number of heads & 8     & 16 \\
    MLP ratio &    8/3   & 8/3 \\
    $\#$param.  & 284M & 1.13B \\
    \bottomrule
    \end{tabular}%
    }
  \label{tab:arch}%
\end{table}%

Table~\ref{tab:io_structure} presents the detailed configuration of the embedding and prediction modules in EgoAgent, including the image projector ($\text{Proj}_i$), representation head/state prediction head ($\text{MLP}_i$), action projector ($\text{Proj}_a$) and action prediction head ($\text{MLP}_a$).
Note that the representation head and the state prediction head share the same architecture but have distinct weights.

\begin{table}[t]
\centering
\caption{Architecture of the embedding ($\text{Proj}_i$, $\text{Proj}_a$) and prediction ($\text{MLP}_i$, $\text{MLP}_a$) modules in EgoAgent. For details on module connections and functions, please refer to Fig.~2 in the main paper.}
\label{tab:io_structure}
\resizebox{\linewidth}{!}{%
\begin{tabular}{lcl}
\toprule
       & \multicolumn{1}{c}{Norm \& Activation} & \multicolumn{1}{c}{Output Shape}  \\
\midrule
\multicolumn{3}{l}{$\text{Proj}_i$ (\textit{Image projector})} \\
\midrule
Input image  & -          & 3$\times$224$\times$224 \\
Conv 2D (16$\times$16) & -       & Embedding dim$\times$14$\times$14    \\
\midrule
\multicolumn{3}{l}{$\text{MLP}_i$ (\textit{State prediction head} \& \textit{Representation head)}} \\
\midrule
Input embedding  & -          & Embedding dim \\
Linear & GELU       & 2048          \\
Linear & GELU       & 2048          \\
Linear & -          & 256           \\
Linear & -          & 65536     \\
\midrule
\multicolumn{3}{l}{$\text{Proj}_a$ (\textit{Action projector})} \\
\midrule
Input pose sequence  & -          & 4$\times$5$\times$17 \\
Conv 2D (5$\times$17) & LN, GELU   & Embedding dim$\times$1$\times$1    \\
\midrule
\multicolumn{3}{l}{$\text{MLP}_a$ (\textit{Action prediction head})} \\
\midrule
Input embedding  & -          & Embedding dim$\times$1$\times$1 \\
Linear & -          & 4$\times$5$\times$17     \\
\bottomrule
\end{tabular}%
}
\end{table}


\subsection{Training Configurations}
In Table~\ref{tab:training hyper}, we provide the detailed training hyper-parameters for experiments in the main manuscripts.

\begin{table}[ht]
  \centering
  \caption{Hyper-parameters for training EgoAgent.}
  \resizebox{0.86\linewidth}{!}{%
    \begin{tabular}{lc}
    \toprule
    Training Configuration & EgoAgent-300M/1B \\
    \midrule
    Training recipe: &  \\
    optimizer & AdamW~\cite{loshchilov2017decoupled} \\
    optimizer momentum & $\beta_1=0.9, \beta_2=0.999$ \\
    \midrule
    Learning hyper-parameters: &  \\
    base learning rate & 6.0E-04 \\
    learning rate schedule & cosine \\
    base weight decay & 0.04 \\
    end weight decay & 0.4 \\
    batch size & 1920 \\
    training iters & 72,000 \\
    lr warmup iters & 1,800 \\
    warmup schedule & linear \\
    gradient clip & 1.0 \\
    data type & float16 \\
    norm epsilon & 1.0E-06 \\
    \midrule
    EMA hyper-parameters: &  \\
    momentum & 0.996 \\
    \bottomrule
    \end{tabular}%
    }
  \label{tab:training hyper}%
\end{table}%

\clearpage


% \begin{figure*}[h]
%   \centering
%   \includegraphics[width=\textwidth]{Figures/ana_overall.png}  % Replace with your figure file and remove the example
%   \caption{ Overview of EyeQ~\cite{fu2019evaluation} dataset. (A) highlights attribute distributions (i.e., brightness, contrast, sharpness) and diabetic retinopathy (DR) grades across quality categories (i.e., good, usable, and reject). (B) illustrates histograms for the training (i.e., part A and part B), testing, and validation datasets used in \textbf{Full-Reference} evaluations after resampling, with the workflow of degradation algorithms outlined below. (C) shows histograms for real-world \textbf{No-Reference} experiments after resampling. (D) presents reject-quality samples.
%   }
%   \label{fig:overall_ana}
% \end{figure*}


\section{Existing Methods} \label{Sec: methods}
We aim to explore current image-denoising methods, focusing on paired and unpaired training approaches. To facilitate this discussion, we let $\mathbf{X}_i$, and $\mathbf{Y}_i$ represent low-quality and high-quality images, respectively, with corresponding distribution $\mathbb{P}_{\mathbf{X}_i}$ and $\mathbb{P}_{\mathbf{Y}_i}$, where the disjoint set index $i \in \{1,2 \}$. For all paired methods outlined in Sec.~\ref{Sec:supervised}, we focus on data pairs $(\mathbf{x}_1,\mathbf{y}_1)$ such that $\mathbf{x}_1\sim \mathbb{P}_{\mathbf{X}_1}$ and $\mathbf{y}_1 \sim \mathbb{P}_{\mathbf{Y}_1}$. In contrast, for unpaired methods discussed in Sec.~\ref{Sec:unsupervised}, the data is represented as $\mathbf{x}_1\sim \mathbb{P}_{\mathbf{X}_1}, \mathbf{y}_2 \sim \mathbb{P}_{\mathbf{Y}_2}$, ensuring that no paired information is available.

\subsection{Paired Methods}\label{Sec:supervised}
Leveraging pairs of degraded and clean images, denoted as $(\mathbf{x}_1,\mathbf{y}_1)$, paired methods in retinal fundus image enhancement can be uniformly expressed as:
\begin{equation}\label{eq:paired-methods}
    \hat{\mathbf{y}}_1 = f_\theta(\mathbf{x}_1)
\end{equation}
\noindent Here, $f_\theta$ represents the denoising network, which utilizes a degradation model to simulate noise in fundus images and applies various neural network architectures to restore image quality. In methods such as \textit{SCR-Net}~\cite{li2022structure}, \textit{Cofe-Net}~\cite{shen2020modeling}, \textit{PCE-Net}~\cite{10.1007/978-3-031-16434-7_49} and \textit{GFE-Net}~\cite{li2023generic}, $\hat{\mathbf{y}}_1$ is modeled using a Variational autoencoder (VAE), incorporating additional information (e.g., high-frequency details, latent retinal structure, artifacts, and the Laplacian Pyramid Features) to regularize the denoising process. In contrast, \textit{RFormer}~\cite{deng2022rformer}, $f_\theta$ employs a transformer-based generator, where $f_\theta$ focuses on capturing long-range dependencies present in $\mathbf{x}_1$. 
Notably, \textit{I-SECRET}~\cite{i-secret} leverages a semi-supervised approach to optimize $f_\theta$. In the initial two training phases, paired images are utilized to ensure that $f_\theta$ can preserve structural details and maintain pixel-wise alignment. Subsequently, adversarial learning is applied in an unpaired training setting, where $f_\theta$ functions as an optimized generator. For the sake of consistency in our experiments, we categorize this model as a paired method.

\subsection{Unpaied Methods} \label{Sec:unsupervised}
\begin{comment}
Since collecting paired degraded and clean retinal images is both challenging and expensive, unsupervised training techniques have become widely adopted, yielding significant improvements in image quality. 
\end{comment}
Unpaired methods in retinal image denoising can be broadly categorized into two main approaches:  \textbf{GAN-based} and \textbf{SDE-based} methods, employing the generative models (e.g., GAN~\cite{goodfellow2020generative}, Diffusion~\cite{ho2020denoising,song2020denoising}, Gradient flow~\cite{song2019generative}). Since collecting paired clean and noisy images from the real world is challenging, the prevailing approach frames the denoising task as a style transfer problem.

\noindent\textbf{GAN-based model}. The adversarial learning strategy enhances GAN-based models in generating realistic retinal images with detailed structures. The typical adversarial objective is formulated as follows:
\begin{equation}\label{eq:gan-typical}
\begin{split}
    \min_{G_{\mathbf{X}_1}} \max_{D_{\mathbf{Y}_2}} \mathcal{L}  &:= \mathbb{E}_{\mathbf{y}_2}[\log D_{\mathbf{Y}_2}(\mathbf{y}_2)] \\
   & + \mathbb{E}_{\mathbf{x}_1 }[\log(1 - D_{\mathbf{Y}_2}(G_{\mathbf{X}_1}(\mathbf{x}_1)))]
\end{split}
\end{equation}
\noindent Here, the generator $G_{\mathbf{X}_1}$ and discriminator $D_{\mathbf{Y}_2}$ work in opposition, seeking to converge toward a Nash Equilibrium. 
\begin{comment}
Based on LS-GAN~\cite{mao2017least}, \textit{I-SECRET}~\cite{i-secret} leverage a semi-supervised approach to optimize the generator during the initial training phases, using pairs of synthetic degraded and clean images to ensure that $G_{\mathbf{X}_1}$ preserve structural details, particularly in critical regions (e.g., artifacts), by assigning higher importance to those areas. Additionally, the encoder of $G_{\mathbf{X}_1}$ is employed for contrastive learning, ensuring pixel-wise alignment between $\mathbf{x}_1$ and $G_{\mathbf{X}_1}(\mathbf{x}_1)$. The adversarial learning is ultimately applied to authentic images. 
\end{comment}

\noindent\textit{CycleGAN}~\cite{cyclegan} addresses the limitation of requiring paired data by duplicating the GAN structure. It incorporates cycle consistency and identity regularization to support two-way transformations. 
\begin{comment}
Its full objective can be denoted as:
\begin{equation}\label{eq:cycle-gan}
    \begin{split}
        &\mathcal{L}(G_{\mathbf{X}_1}, G_{\mathbf{Y}_2}, D_{\mathbf{X}_1}, D_{\mathbf{Y}_2}) = \mathcal{L}_{\text{G}}(G_{\mathbf{X}_1}, D_{\mathbf{Y}_2}, \mathbf{X}_1, \mathbf{Y}_2) \\ &  
        + \mathcal{L}_{\text{G}}(G_{\mathbf{Y}_2}, D_{\mathbf{X}_1}, \mathbf{Y}_2, \mathbf{X}_1) + \lambda_{\text{cyc}} \mathcal{L}_{\text{cyc}}(G_{\mathbf{X}_1}, G_{\mathbf{Y}_2}) \\
        &+\lambda_{\text{id}} \mathcal{L}_{\text{id}}(G_{\mathbf{X}_1}, G_{\mathbf{Y}_2})
    \end{split}
\end{equation}
\end{comment}
\begin{comment}
denoted as:
\begin{equation}\label{eq:cyclegan-cycleloss}
\begin{split}
    \mathcal{L}_{\text{cyc}}(G_{\mathbf{X}_1}, G_{\mathbf{Y}_2}) &= \mathbb{E}_{\mathbf{x}_1 \sim \mathbb{P}_{\mathbf{X}_1}} [\| G_{\mathbf{Y}_2}(G_{\mathbf{X}_1}(\mathbf{x}_1)) - \mathbf{x}_1 \|_1]  \\
    &+ \mathbb{E}_{\mathbf{y}_2 \sim \mathbb{P}_{\mathbf{Y}_2}} [\| G_{\mathbf{X}_1}(G_{\mathbf{Y}_2}(\mathbf{y}_2)) - \mathbf{y}_2 \|_1]
\end{split}
\end{equation}
\end{comment}
Specifically, the cycle consistency loss enforces bidirectional mapping, improving alignment and coherence in the generated images. However, these additional structures increase computational complexity and may result in suboptimal performance (e.g., mode collapse, artifacts), especially when dealing with images that exhibit multimodal distributions~\cite{salmona2022can}.
\begin{comment}
. However, the introduction of additional frameworks (i.e., $G_{\mathbf{Y}_2}$ and $D_{\mathbf{Y}_2}$) lead to unnecessary computational costs in retinal image denoising tasks. Additionally, the complicated training process may result in suboptimal performance (e.g., mode collapse, artifacts). To mitigate these limitations, Optimal Transport (OT) theory has been utilized to facilitate smoother probabilistic transportation, leading to the development of \textbf{OT-based GANs}.
\end{comment}

\noindent\textit{Wasserstein-GAN} (WGAN)~\cite{arjovsky2017wasserstein,gulrajani2017improved} is a widely recognized model rooted in OT theory. Instead of solving the primal OT problem directly, WGAN leverages the Kantorovich-Rubinstein duality~\cite{villani2009optimal} to approximate the Wasserstein distance in a computationally feasible manner. The objective function is given by:
\begin{equation}\label{eq:wgan-basic}
    \begin{split}
        \min_{G_{\mathbf{X}_1}} \max_{D_{\mathbf{Y}_2}} \mathcal{L} &:= \mathbb{E}_{\mathbf{y}_2 }[D_{\mathbf{Y}_2}(\mathbf{y}_2)] 
        - \mathbb{E}_{\mathbf{x}_1 } [D_{\mathbf{Y}_2}(G_{\mathbf{X}_1}(\mathbf{x}_1))]
    \end{split}
\end{equation}
\noindent Here, the generator $G_{\mathbf{X}_1}$ learns to map $\mathbb{P}_{\mathbf{X}_1}$ to $\mathbb{P}_{\mathbf{Y}_2}$ by minimizing the Wasserstein distance between them. The discriminator $D_{\mathbf{Y}_2}$ maximizes the difference in continuous scores assigned to the real high-quality images and the synthetic images generated by $G_{\mathbf{X}_1}$. By providing feedback on the quality of the generated images, the discriminator guides the generator toward the optimal mapping.

In contrast to WGAN, which approximates the Wasserstein distance indirectly, \textit{OTT-GAN}~\cite{wang2022optimal} directly solves the Monge's Optimal Transport problem using an adversarial training strategy. The objective function is expressed as:
\begin{equation} \label{eq:ott-basic}
    \begin{split}
        \max_{G_{\mathbf{X}_1}} \min_{D_{\mathbf{Y}_2}} \mathcal{L}:= \mathbb{E}_{\mathbf{x}_1} [ C( \mathbf{x}_1, G_\theta (\mathbf{x}_1) )] 
        +\lambda \mathbf{W}_1(\mathbb{P}_{\mathbf{Y}_2}, \mathbb{P}_{G_{\mathbf{X}_1}(\mathbf{x}_1)} )
    \end{split}
\end{equation}
\noindent Here, the cost function $C$ is defined as mean square error (MSE), and the method in Eq.~\ref{eq:wgan-basic} is employed to approximate the Wasserstein distance, denoted as $\mathbf{W}_1$. Building on \textit{OTT-GAN}, \textit{OTE} and \textit{OTRE}~\cite{zhu2023optimal,zhu2023otre} incorporates the Multi-Scale Structural Similarity Index Measure (MS-SSIM)~\cite{wang2003multiscale,brunet2011mathematical} as the cost function, along with identity regularization, which improve the preservation of structural details during image translation. To further enhance contextual preservation during denoising, \textit{Context-aware OT}~\cite{vasa2024context} extends beyond pixel-based costs. It leverages a pretrained VGG~\cite{mechrez2018contextual} network to capture the earth mover's distance in the feature space, thereby improving perceptual similarity between the input and generated images.

\noindent\textbf{SDE-based model}. \textit{CUNSB-RFIE}~\cite{dong2024cunsb} seeks to identify the Schr\"{o}dinger Bridge (SB), denoted as $\mathbb{Q}^{SB}$, between $\mathbb{P}_{\mathbf{X}_1}$ and $\mathbb{P}_{\mathbf{Y}_2}$. This approach enables a smooth and probabilistically consistent transformation between image distributions, but it results in a loss of high-frequency information during the iterative training process. The main objective function for an arbitrary step $t_i$ is expressed as:
\begin{equation}\label{eq:CUNSB-RFIE}
\begin{split}
    &\min_{\phi} \mathbb{L}(\phi,t_i) := \mathbb{L}_{Adv}(\phi,t_i) + \lambda_{SB} \mathbb{L}_{SB}(\phi,t_i) 
\end{split}
\end{equation}
\noindent Here, $\phi$ parameterizes the generator $G_{\mathbf{X}_1}$ at step $t_i$. The term $\mathbb{L}_{Adv}$ modifies the KL-divergence between the synthetic high-quality image distribution and the ground-truth distribution $\mathbb{P}_{\mathbf{Y}_2}$, while $ \mathbb{L}_{SB}$ approximates the solution to the entropy-regularized optimal transport problem. Consequently, the final static solution shares the same marginal distributions as $\mathbb{Q}^{SB}$~\cite{tong2023improving}. 

\begin{figure*}[h]
  \centering
  \includegraphics[width=0.9\textwidth]{Figures/ana_overall.png}  % Replace with your figure file and remove the example
  \caption{ (A) highlights attribute distributions (i.e., brightness, contrast, sharpness) and diabetic retinopathy (DR) grades across quality categories (i.e., good, usable, and reject). (B) illustrates histograms for the training (i.e., part A and part B), testing, and validation datasets used in \textbf{Full-Reference} evaluations after resampling, with the workflow of degradation algorithms outlined below. (C) shows histograms for real-world \textbf{No-Reference} experiments after resampling. (D) presents samples to be overprocessed.
  }
  \label{fig:overall_ana}
\end{figure*}

\section{Clinic Experts Guided Data Annotation}

Our dataset was sourced from the EyePACS initiative~\cite{diabetic-retinopathy-detection}, with quality annotations derived from the EyeQ dataset~\cite{fu2019evaluation}. We collected 28,791 color fundus images, which were categorized into three quality levels: good, usable, and reject. Additionally, each image was annotated with diabetic retinopathy (DR) severity labels across five levels (0, 1, 2, 3, and 4), where higher values indicate greater severity of DR.
Analyses of attribute distributions (i.e., brightness, contrast, and sharpness) and DR grade across quality categories are presented in Fig.~\ref{fig:overall_ana}(A). The results reveal notable attribute discrepancies across quality categories, with the distributions for good and usable quality images closely aligned. In contrast, the reject category shows significant differences (e.g., a higher prevalence of high-acutance images). Additionally, DR labels are imbalanced, with labels 0 and 2 being more frequent and severe DR cases (labels 3 and 4) relatively underrepresented. Finally, as illustrated in Fig.~\ref{fig:overall_ana}(D), some rejected and usable quality examples tend to be overprocessed, compromising their diagnostic quality and clinical practicability. We re-selected good and usable quality images and applied ratio-preserving resampling to maintain lesion information alignment based on medical expert guidance. Given the low prevalence of severe DR cases, achieving a balanced sample across DR grades is challenging. Therefore, we retained the natural distribution of DR labels across subsets to better represent real-world clinical scenarios and align with objective engineering principles.

\noindent \textbf{Full-Reference Evaluation Dataset.} A total of 16,817 good-quality images were used here, with 10,000 images for training, 600 for validation, and 6,217 for testing, as detailed in Fig.~\ref{fig:overall_ana}(B). To support the experiments, all good-quality images were degraded using the algorithms outlined in ~\cite{shen2020modeling}, which simulate the combinations of illumination, spot artifacts, and blurring. Additionally, the training set was split into two disjoint subsets of 5,000 images each, referred to as $A$ and $B$, with synthesized noisy paired subsets $A^{\ast}$ and $B^{\ast}$, which were used to train the paired ($A^{\ast}$ to $A$) and unpaired ($A^{\ast}$ to $B$) methods. 


\noindent \textbf{No-Reference Evaluation Dataset.}
As shown in Fig.~\ref{fig:overall_ana}(C), A total of 6,434 usable-quality images were included in this study, and all usable-quality images were resampled, resulting in 4,000 real-world noisy images (i.e., real noise) for training and 2,434 testing images. Additionally, 4,000 unpaired good-quality images were resampled from the original set of 10,000 good-quality training images based on the DR label, ensuring the experiment followed the unpaired training scheme.


\begin{comment}

We further analyzed the image quality for each label and plotted histograms for brightness, sharpness, and contrast, shown in Fig.~\ref{fig:overall_ana}. The brightness histogram shows that "good" images have values well-distributed between 75 and 100. In contrast, "reject" images are mainly concentrated between 25 and 50, with some exceeding 150, suggesting overexposure. The sharpness histogram reveals that "good" images exhibit better sharpness, capturing more details such as edges and textures. The contrast histogram shows that "good" images tend to have moderate contrast, while "reject" ones cluster at lower values, indicating blur. Some "reject" images also appear in higher contrast regions, suggesting large variations in brightness. Notably, for the “usable” images, most images fall between the "good" and "reject" labels across the three metrics. However, some images show low values, prompting us to re-evaluate these labels and remove those deemed unusable to maintain data consistency.
\end{comment}

\begin{comment}
\noindent
\textbf{Full-Reference Assessment Dataset.} We grouped the 16,818 high-quality images to construct our dataset. The dataset was divided into three parts: training set, validation set, and test set, containing 10,000, 600, and 6,217 images, respectively(Fig.~\ref{fig:overall_ana}(B)). The training set was further split into two equal parts, A and B, ensuring proportional distribution of DR labels across all subsets for paired training. To simulate noise on high-quality retina images, we introduced variations in Illumination, spot noise, and blurring effects, following the process illustrated at the bottom of the(Fig.~\ref{fig:overall_ana}(B)). Then, we trained the model using part A along with its noise-augmented counterpart, and evaluated performance on the validation and test sets. 
\end{comment}

\begin{comment}
\noindent
\textbf{No-Reference Assessment Dataset.}
(1) We designed two datasets for the unpaired image enhancement experiments. In the first approach, we selected two subsets from the high-quality training data mentioned earlier. During the training process, high-quality images from part A were paired with noise-augmented images from part B. Subsequently, the test sets were used to evaluate the model’s performance, ensuring the effectiveness of the enhancement process. (2) To evaluate the model’s performance in real-world scenarios, we constructed training, validation, and test sets from the full dataset, excluding the high-quality images. A total of 4,000 images were used for training, 400 for validation, and 2,434 for testing. To ensure a balanced training process with the low-quality images, we resampled 4,000 images from the high-quality data, matching the distribution of DR labels in the above set(Fig.~\ref{fig:overall_ana}(C)). These 4,000 high-quality images were then combined with the 4,000 selected images above for training. During the selection process, we filtered the images based on both the quality labels and recommendations from medical doctors. The filtered images, shows in the bottom of Fig.~\ref{fig:overall_ana}(C), aligned with our analysis, confirming that images of excessively low quality no longer have value for enhancement and, therefore, do not require further processing.
\end{comment}

% \section{Existing Methods}
% We aim to explore current image-denoising methodologies, focusing on supervised and unsupervised training approaches. To facilitate this discussion, we divide the data into two disjoint sets.  Let $\mathbf{X}_i$, and $\mathbf{Y}_i$ represent low-quality and high-quality images, respectively, with corresponding distribution $\mathbb{P}_{\mathbf{X}_i}$ and $\mathbb{P}_{\mathbf{Y}_i}$, where the set index $i \in \{1,2 \}$. For all supervised methods outlined in Sec.~\ref{Sec:supervised}, we focus on data pairs $(\mathbf{x}_1,\mathbf{y}_1)$ such that $\mathbf{x}_1\sim \mathbb{P}_{\mathbf{X}_1}$ and $\mathbf{y}_1 \sim \mathbb{P}_{\mathbf{Y}_1}$. In contrast, for unsupervised methods discussed in Sec.~\ref{Sec:unsupervised}, the data is represented as $\mathbf{x}_1\sim \mathbb{P}_{\mathbf{X}_1}, \mathbf{y}_2 \sim \mathbb{P}_{\mathbf{Y}_2}$, ensuring that no paired information is available.

% \subsection{Paired Methods}\label{Sec:supervised}
% Leveraging pairs of degraded and clean images, denoted as $(\mathbf{x}_1,\mathbf{y}_1)$, paired methods in retinal fundus image enhancement can be uniformly expressed as:
% \begin{equation}\label{eq:paired-methods}
%     \hat{\mathbf{y}}_1 = f_\theta(\mathbf{x}_1)
% \end{equation}
% \noindent Here, $f_\theta$ represents the enhancement network, which vary across different architectures. In methods such as \textit{SCR-Net}~\cite{li2022structure}, \textit{Cofe-Net}~\cite{shen2020modeling}, \textit{PCE-Net}~\cite{10.1007/978-3-031-16434-7_49} and \textit{GFE-Net}~\cite{li2023generic}, $\hat{\mathbf{y}}_1$ is modeled using a Variational autoencoder (VAE), incorporating additional information (e.g., high frequency details, latent retinal structure, artifacts, and the Laplacian Pyramid Features) to regularize the denoising process. In contrast, \textit{RFormer}~\cite{deng2022rformer}, $f_\theta$ employs a transformer-based generator, where $f_\theta$ focuses on capturing long-range dependencies present in $\mathbf{x}_1$. 

% \subsection{Unpaied Methods} \label{Sec:unsupervised}
% \begin{comment}
% Since collecting paired degraded and clean retinal images is both challenging and expensive, unsupervised training techniques have become widely adopted, yielding significant improvements in image quality. 
% \end{comment}
% Unpaired methods in retinal image denoising can be broadly categorized into two main approaches:  \textbf{GAN-based} and \textbf{SDE-based} methodologies.

% \textbf{GAN-based model}. The adversarial learning strategy enhances GAN-based models in generating realistic retinal images with detailed structures. The typical adversarial objective is formulated as follows:
% \begin{equation}\label{eq:gan-typical}
% \begin{split}
%     \min_{G_{\mathbf{X}_1}} \max_{D_{\mathbf{Y}_2}} \mathcal{L}  &:= \mathbb{E}_{\mathbf{y}_2}[\log D_{\mathbf{Y}_2}(\mathbf{y}_2)] \\
%    & + \mathbb{E}_{\mathbf{x}_1 }[\log(1 - D_{\mathbf{Y}_2}(G_{\mathbf{X}_1}(\mathbf{x}_1)))]
% \end{split}
% \end{equation}
% \noindent Here, the generator $G_{\mathbf{X}_1}$ and discrimator $D_{\mathbf{Y}_2}$ work in opposition, seeking to converge toward a Nash Equilibrium. Based on LS-GAN~\cite{mao2017least}, \textit{I-SECRET}~\cite{i-secret} leverage a semi-supervised approach to optimize the generator during the initial training phases, using pairs of synthetic degraded and clean images to ensure that $G_{\mathbf{X}_1}$ preserve structural details, particularly in critical regions (e.g., artifacts), by assigning higher importance to those areas. Additionally, the encoder of $G_{\mathbf{X}_1}$ is employed for contrastive learning, ensuring pixel-wise alignment between $\mathbf{x}_1$ and $G_{\mathbf{X}_1}(\mathbf{x}_1)$. The adversarial learning is ultimately applied to authentic images. 

% \textit{CycleGAN}~\cite{cyclegan} addresses the limitation of requiring paired data by duplicating GAN structure. It incorporates cycle consistency and identity regularization to support two-way transformations. 
% \begin{comment}
% Its full objective can be denoted as:
% \begin{equation}\label{eq:cycle-gan}
%     \begin{split}
%         &\mathcal{L}(G_{\mathbf{X}_1}, G_{\mathbf{Y}_2}, D_{\mathbf{X}_1}, D_{\mathbf{Y}_2}) = \mathcal{L}_{\text{G}}(G_{\mathbf{X}_1}, D_{\mathbf{Y}_2}, \mathbf{X}_1, \mathbf{Y}_2) \\ &  
%         + \mathcal{L}_{\text{G}}(G_{\mathbf{Y}_2}, D_{\mathbf{X}_1}, \mathbf{Y}_2, \mathbf{X}_1) + \lambda_{\text{cyc}} \mathcal{L}_{\text{cyc}}(G_{\mathbf{X}_1}, G_{\mathbf{Y}_2}) \\
%         &+\lambda_{\text{id}} \mathcal{L}_{\text{id}}(G_{\mathbf{X}_1}, G_{\mathbf{Y}_2})
%     \end{split}
% \end{equation}
% \end{comment}
% \begin{comment}
% denoted as:
% \begin{equation}\label{eq:cyclegan-cycleloss}
% \begin{split}
%     \mathcal{L}_{\text{cyc}}(G_{\mathbf{X}_1}, G_{\mathbf{Y}_2}) &= \mathbb{E}_{\mathbf{x}_1 \sim \mathbb{P}_{\mathbf{X}_1}} [\| G_{\mathbf{Y}_2}(G_{\mathbf{X}_1}(\mathbf{x}_1)) - \mathbf{x}_1 \|_1]  \\
%     &+ \mathbb{E}_{\mathbf{y}_2 \sim \mathbb{P}_{\mathbf{Y}_2}} [\| G_{\mathbf{X}_1}(G_{\mathbf{Y}_2}(\mathbf{y}_2)) - \mathbf{y}_2 \|_1]
% \end{split}
% \end{equation}
% \end{comment}
% Specifically, the cycle consistency loss enforces a bidirectional mapping, improving alignment and coherence in the generated images. However, these additional structures increase computational complexity and may result in suboptimal performance (e.g., mode collapse, artifacts), especially when dealing with images that exhibit multimodal distributions~\cite{salmona2022can}.
% \begin{comment}
% . However, the introduction of additional frameworks (i.e., $G_{\mathbf{Y}_2}$ and $D_{\mathbf{Y}_2}$) lead to unnecessary computational costs in retinal image denoising tasks. Additionally, the complicated training process may result in suboptimal performance (e.g., mode collapse, artifacts). To mitigate these limitations, Optimal Transport (OT) theory has been utilized to facilitate smoother probabilistic transportation, leading to the development of \textbf{OT-based GANs}.
% \end{comment}

% \textit{Wasserstein-GAN} (WGAN)~\cite{arjovsky2017wasserstein,gulrajani2017improved} is a widely recognized model rooted in OT theory. Instead of solving the primal OT problem directly, WGAN leverages the Kantorovich-Rubinstein duality to approximate the Wasserstein distance in a computationally feasible manner. The objective function is given by:
% \begin{equation}\label{eq:wgan-basic}
%     \begin{split}
%         \min_{G_{\mathbf{X}_1}} \max_{D_{\mathbf{Y}_2}} \mathcal{L} &:= \mathbb{E}_{\mathbf{y}_2 }[D_{\mathbf{Y}_2}(\mathbf{y}_2)] 
%         - \mathbb{E}_{\mathbf{x}_1 } [D_{\mathbf{Y}_2}(G_{\mathbf{X}_1}(\mathbf{x}_1))]
%     \end{split}
% \end{equation}
% \noindent Here, the generator $G_{\mathbf{X}_1}$ learns to map $\mathbb{P}_{\mathbf{X}_1}$ to $\mathbb{P}_{\mathbf{Y}_2}$ by minimizing the Wasserstein distance between them. The discriminator $D_{\mathbf{Y}_2}$ maximizes the difference in continuous scores assigned to the real high-quality images and the synthetic images generated by $G_{\mathbf{X}_1}$. By providing feedback on the quality of the generated images, the discriminator guides the generator toward the optimal mapping.

% In contrast to WGAN, which approximates the Wasserstein distance indirectly, \textit{OTT-GAN}~\cite{wang2022optimal} directly solves the Monge's Optimal Transport problem using an adversarial training strategy. The objective function is expressed as:
% \begin{equation} \label{eq:ott-basic}
%     \begin{split}
%         \max_{G_{\mathbf{X}_1}} \min_{D_{\mathbf{Y}_2}} \mathcal{L}:= \mathbb{E}_{\mathbf{x}_1} [ C( \mathbf{x}_1, G_\theta (\mathbf{x}_1) )] 
%         +\lambda \mathbf{W}_1(\mathbb{P}_{\mathbf{Y}_2}, \mathbb{P}_{G_{\mathbf{X}_1}(\mathbf{x}_1)} )
%     \end{split}
% \end{equation}
% \noindent Here, the cost function $C$ is defined as mean square error (MSE), and the method in Eq.~\ref{eq:wgan-basic} is employed to approximate the Wasserstein distance, denoted as $\mathbf{W}_1$. Building on \textit{OTT-GAN}, \textit{OTE} and \textit{OTRE}~\cite{zhu2023optimal,zhu2023otre} incorporates the Multi-Scale Structural Similarity Index Measure (MS-SSIM)~\cite{wang2003multiscale,brunet2011mathematical} as the cost function, along with identity regularization, which improve the preservation of structural details during image translation. To further enhance contextual preservation during denoising, \textit{Context-aware OT}~\cite{vasa2024context} extends beyond pixel-based costs. It leverages a pretrained VGG~\cite{mechrez2018contextual} network to capture the earth mover's distance in the feature space, thereby improving perceptual similarity between the input and generated images.

% \textbf{SDE-based model}. \textit{CUNSB-RFIE}~\cite{dong2024cunsb} seeks to identify the Schr\"{o}dinger Bridge (SB), denoted as $\mathbb{Q}^{SB}$, between $\mathbb{P}_{\mathbf{X}_1}$ and $\mathbb{P}_{\mathbf{Y}_2}$. This approach enables a smooth and probabilistically consistent transformation between distributions, but it may lead to a deficiency in high-frequency information during the iterative training process. The main objective function for an arbitrary step $t_i$ is expressed as:
% \begin{equation}\label{eq:CUNSB-RFIE}
% \begin{split}
%     &\min_{\phi} \mathbb{L}(\phi,t_i) := \mathbb{L}_{Adv}(\phi,t_i) + \lambda_{SB} \mathbb{L}_{SB}(\phi,t_i) 
% \end{split}
% \end{equation}
% \noindent Here, $\phi$ parameterizes the generator $G_{\mathbf{X}_1}$ at step $t_i$. The term $\mathbb{L}_{Adv}$ modifies the KL-divergence between the synthetic high-quality image distribution and the ground-truth distribution $\mathbb{P}_{\mathbf{Y}_2}$, while $ \mathbb{L}_{SB}$ approximates the solution to the entropy-regularized optimal transport problem. Consequently, the final static solution shares the same marginal distributions as $\mathbb{Q}^{SB}$~\cite{tong2023improving}. 


\begin{table*}[!t]
\centering
\caption{Performance comparison of denoising evaluation in Full-Reference quality assessment experiments. The best performance in each column is highlighted in bold, with the second-best underlined. Visualization results refer to the \textcolor{red}{Appendix C}.}
\tiny
\resizebox{0.8\textwidth}{!}{%
\begin{tabular}{cccccccc}
\toprule
\multirow{2}{*}{} & \multirow{2}{*}{\textbf{Method}} & \multicolumn{2}{c}{\textbf{EyeQ}} & \multicolumn{2}{c}{\textbf{IDRID}} & \multicolumn{2}{c}{\textbf{DRIVE}} \\ \cmidrule(l){3-8} 
                                          &                                  & \textbf{SSIM} $\uparrow$   & \textbf{PSNR} $\uparrow$   & \textbf{SSIM} $\uparrow$   & \textbf{PSNR} $\uparrow$   & \textbf{SSIM} $\uparrow$   & \textbf{PSNR} $\uparrow$   \\ \midrule
\multirow{5}{*}{\textit{Paired Methods}} & SCR-Net~\cite{li2022structure}   & \textbf{0.9606} & 29.698 & 0.6425 & 18.920 & \textbf{0.6824} & 23.280 \\      
                                         & Cofe-Net~\cite{shen2020modeling} & 0.9408           & 24.907           & 0.7397            & 20.058            & 0.6671            & 21.774            \\
                                         & PCE-Net~\cite{10.1007/978-3-031-16434-7_49} & 0.9487           & \textbf{29.895}           & \underline{0.7764}            & \underline{23.201}           & 0.6704            & 24.041           \\
                                         & GFE-Net~\cite{li2023generic}     & \underline{0.9554}           & \underline{29.719}           & \textbf{0.7935}            & \textbf{25.012}           & \underline{0.6793}            & \underline{23.786} \\
                                         &RFormer~\cite{deng2022rformer}
                                          & 0.9260   & 27.163   & 0.5963   & 18.433   & 0.6311   & 22.172\\
                                          & I-SECRET~\cite{i-secret}        & 0.9051 & 23.483 & 0.7157 & 20.173 & 0.5727 & 18.803 \\
                                          \midrule
\multirow{7}{*}{\textit{Unpaired Methods}} 
                                         & CycleGAN~\cite{cyclegan}         & \underline{0.9313}           &\textbf{25.076}           & \textbf{0.7668}            & \textbf{22.511}           & \textbf{0.6681}            & \textbf{22.686}  \\
                                         & WGAN~\cite{gulrajani2017improved} & 0.9266          & 24.793           & 0.7316           & 21.325           & 0.6431            & 20.408 \\
                                         & OTTGAN~\cite{wang2022optimal}    & 0.9275           & 24.065           & 0.7509           & 22.131           & 0.6635            & 21.938 \\
                                         & OTEGAN~\cite{zhu2023optimal}     & \textbf{0.9392}           & \underline{24.812}           & 0.7624           & 22.272           & 0.6642            & 22.183 \\
                                         & Context-aware OT~\cite{vasa2024context} & 0.9144           & 24.088           & 0.7338            & 21.790            & 0.6407            & 21.389 \\
                                         & CUNSB-RFIE~\cite{dong2024cunsb}  & 0.9121           & 24.242           & \underline{0.7651}           & \underline{22.448}           & \underline{0.6659}            & \underline{22.510} \\
\bottomrule
\end{tabular}%
}
\label{tb:deg-exp}
\end{table*}

\begin{table*}[!t]
    \centering
    \caption{
    Performance comparison of vessel and lesion (EX and HE) segmentation in Full-Reference quality assessment experiments. The best performance in each column is highlighted in bold, with the second-best underlined. For visualization results, refer to the \textcolor{red}{Appendix C}.}
    \tiny
    \resizebox{0.9\textwidth}{!}{%
    \begin{tabular}{lcccc|ccc|ccc}
        \toprule
         \multirow{2}[3]{*}{Method} & \multicolumn{4}{c}{Vessel Segmentation} & \multicolumn{3}{c}{EX} & \multicolumn{3}{c}{HE} \\ 
         \cmidrule(lr){2-5}  \cmidrule(lr){6-8}  \cmidrule(lr){9-11}
          
         & AUC $\uparrow$ & PR $\uparrow$ & F1 Score $\uparrow$ & SP $\uparrow$ & AUC  & PR & F1 Score  & AUC & PR & F1 Score \\ \midrule

        SCR-Net~\cite{li2022structure}   & \textbf{0.9227} & \textbf{0.7783} & \textbf{0.7000} & 0.9787 & \textbf{0.9683} & \textbf{0.6041} & \textbf{0.5556} & 0.9377 & 0.3213 & 0.3725\\ 
        cofe-Net~\cite{shen2020modeling} & \underline{0.9188} & \underline{0.7698} & 0.6895 & 0.9801 & 0.9623 & 0.5620 & 0.5349 & 0.9302 & 0.3152 & 0.3281\\
        PCE-Net~\cite{10.1007/978-3-031-16434-7_49} & 0.9146 & 0.7616 & 0.6790 & \textbf{0.9814} & \underline{0.9667} & \underline{0.5876} & 0.5066 & \underline{0.9545} & \underline{0.3639} & \underline{0.3736}\\
        GFE-Net~\cite{li2023generic} & 0.9175 & 0.7669 & 0.6832 & \textbf{0.9814} & 0.9560 & 0.5548 & \underline{0.5380} & \textbf{0.9577} & \textbf{0.4113} & \textbf{0.3751}\\
        RFormer~\cite{deng2022rformer} & 0.8990 & 0.7239 & 0.6374 & \underline{0.9806} & 0.9626 & 0.5593 & 0.4692 & 0.9207 & 0.2677 & 0.3136 \\
        I-SECRET~\cite{i-secret} & 0.9181 & 0.7662 & 0.6838 & 0.9802 & 0.9613 & 0.5424 & 0.4825 & 0.9028 & 0.2629 & 0.2642 \\
    \midrule
        CycleGAN~\cite{cyclegan} & 0.9015 & 0.7278 & 0.6462 & \underline{0.9801}  & 0.9447& 0.4843 & 0.4790 & 0.8970 & 0.1624 & 0.2227\\
        WGAN~\cite{gulrajani2017improved} & 0.9081 & 0.7494 & 0.6768 & 0.9764  & 0.9522 & 0.4942 & 0.4859 & \underline{0.8990} & \underline{0.1847} & \underline{0.2476}\\
        OTTGAN~\cite{wang2022optimal} & 0.9034 & 0.7400 & 0.6609 & \textbf{0.9812}  & 0.9492 & 0.4214 & 0.4365 & 0.8179 & 0.1448 & 0.2233 \\
        OTEGAN~\cite{zhu2023optimal} & \underline{0.9156} & \textbf{0.7678} & \textbf{0.6919} & 0.9797 & \underline{0.9562} & \underline{0.5191} & \underline{0.4868} & \textbf{0.9359} & \textbf{0.2800}& \textbf{0.3165} \\
        Context-aware OT~\cite{vasa2024context} & 0.8871 & 0.7077 & 0.6377 & 0.9791 & 0.9305 & 0.3318 & 0.3707 & 0.8091 & 0.0646 & 0.1184 \\
        CUNSB-RFIE~\cite{dong2024cunsb}  & \textbf{0.9163} & \underline{0.7626} & \underline{0.6872} & 0.9784 & \textbf{0.9572}& \textbf{0.5381} & \textbf{0.4883} & 0.8488 & 0.1489 & 0.1893 \\
        \bottomrule
    \end{tabular}}
    \label{tab-seg}
    \vspace{-0.3cm}
\end{table*}




\begin{table*}[!t]
    \centering
    \caption{Performance comparison with unpaired baselines in No-Reference quality assessment task. The best performance in each column is highlighted in bold, and the second-best is underlined. Visualization results refer to \textcolor{red}{Appendix C}.}
    \tiny
    \resizebox{0.9\textwidth}{!}{%
    \begin{tabular}{lcccc|cc|ccc}
        \toprule
         \multirow{2}[3]{*}{Method} & \multicolumn{4}{c}{DR grading} & \multicolumn{2}{c}{Representation Feature} & \multicolumn{3}{c}{Experts Protocol Evaluation} \\ 
         \cmidrule(lr){2-5}  \cmidrule(lr){6-7}  \cmidrule(lr){8-10}
          
         & ACC $\uparrow$ & Kappa $\uparrow$ & F1 Score $\uparrow$ & AUC $\uparrow$ & FID-Retfound~\cite{zhou2023foundation}$\downarrow$  & FID-Clip~\cite{du2024ret} $\downarrow$ & LPR $\uparrow$ & BPR $\uparrow$ & SPR $\uparrow$\\ \midrule
         
        CycleGAN~\cite{cyclegan}                & \textbf{0.7588} & \underline{0.6006} & \underline{0.7180} & \underline{0.9251}  & \textbf{23.778}& \underline{11.530}  & \underline{0.7707} & 0.8153 & \textbf{0.8726}\\
        
        WGAN~\cite{gulrajani2017improved}       & 0.6446 & 0.3123 & 0.6156 & 0.8874  & 74.885 & 33.076  & 0.4076 & 0.4204 & 0.6561\\
        
        OTTGAN~\cite{wang2022optimal}           & 0.7440 & 0.5688 & 0.7037 & 0.9247  & 51.201 & 20.505 & 0.4586 & 0.7580 & 0.5860 \\
        
        OTEGAN~\cite{zhu2023optimal}            & \underline{0.7539} & \textbf{0.6433} & \textbf{0.7228} & \textbf{0.9326} & \underline{28.987} & \textbf{11.114} & \textbf{0.8280} & \textbf{0.8981} & 0.6178 \\
        
        Context-aware OT~\cite{vasa2024context} & 0.7301 & 0.3811 & 0.6662 & 0.9112 & 61.429 & 34.456 & 0.3566 & 0.3121 & 0.5159 \\
        
        CUNSB-RFIE~\cite{dong2024cunsb}         & 0.6565 & 0.3674 & 0.6341 & 0.8927 & 33.047 & 14.827  & \textbf{0.8280} & \underline{0.8535} & \underline{0.6879} \\
        
        \bottomrule
    \end{tabular}}
    \label{tab-noref}
    \vspace{-0.3cm}
\end{table*}

\section{Experiments}

% \begin{figure*}[h]
%   \centering
%   \includegraphics[width=\textwidth]{Figures/paired experiment evaluation design.pdf}  % Replace with your figure file and remove the example
%   \caption{\textbf{(a)} Illustration of the Downstream Tasks proposed to evaluate the Paired Methods. After degrading the original Images, we infer the Enhanced images on the trained weights (\includegraphics[height=10pt]{Figures/ice.png}). These enhanced images are used for fine-tuning (\includegraphics[height=10pt]{Figures/fire.png}) the Downstream Segmentation Tasks. \textbf{(b)} Protocol followed for Visual Inspection based Experts Evaluation metrics. Each case here is categorized as \textcolor{green}{Positive} or \textcolor{red}{Negative}.}
%   \label{fig:exp_des}
% \end{figure*}

\subsection{Full-Reference Quality Assessment Experiments} \label{section-4.1}
\begin{comment}
We propose to evaluate the paired methods with Degradation Experiment, and two Downstream tasks: Vessel segmentation and Lesion Segmentation. The Paired Experiments are conducted by degrading the high-quality images to synthesize the low-quality counterparts and using the paired combination to learn enhancement. We degrade the original retinal images and then enhance these synthetically degraded images using the trained weights from the Degradation Experiment. The Segmentation models to perform Downstream tasks are then trained upon the enhanced images combined with ground truth masks. The work flow of Downstream tasks is illustrated in Fig. \ref{fig:exp_des}. We leveraged UNet \cite{ronneberger2015unet} as a backbone for accomplishing the Downstream Segmentation tasks. We applied the computational resources from Nvidia RTX3090 GPU to conduct the experiments.
\end{comment}
For full-reference assessment, we used the previously synthesized Full-Reference Assessment Dataset. We strictly followed the training configurations for paired and unpaired methods. For the unpaired method, synthetic low-quality images from training set $A$ (i.e., $A^{\ast}$) were used as input images, while high-quality images from training set $B$ served as the clean reference images. For the paired method, we performed supervised training using low-high-quality image pairs from the training set $A$ (i.e., $A^{\ast}$ and $A$). All models were trained with the parameter report in the original paper. For two segmentation tasks, we trained a vanilla U-Net~\cite{ronneberger2015unet} model from scratch.
The following baselines were considered in this evaluation: \textit{Paired algorithms}: SCR-Net~\cite{li2022structure}, Cofe-Net~\cite{shen2020modeling}, PCE-Net~\cite{10.1007/978-3-031-16434-7_49}, GFE-Net~\cite{li2023generic}, RFormer~\cite{deng2022rformer}, \textit{Unpaired algorithms}: I-SECRET~\cite{i-secret}, CycleGAN~\cite{cyclegan}, WGAN~\cite{gulrajani2017improved}, OTTGAN~\cite{wang2022optimal}, OTEGAN~\cite{zhu2023optimal}, Context-aware OT~\cite{vasa2024context}, CUNSB-RFIE~\cite{dong2024cunsb}.
We generated enhanced images for the trained models separately for the downstream evaluations. Refer to the \textcolor{red}{Appendix A} for more details.

\noindent \textbf{Denoising Evaluation.} We input the noisy images from the Full-Reference testing set into the trained models to generate enhanced images. These enhanced low-quality images were then evaluated using the Peak Signal-to-Noise Ratio (PSNR) and Structural Similarity Index Measure (SSIM).

\noindent \textbf{Denoising generalization Evaluation.} To evaluate denoising generalization, we degraded high-quality images following the same degradation algorithm to synthesize the low-quality images for DRIVE~\cite{drive} and IDRID~\cite{idrid}. Similarly, we fed these enhanced images into the trained model to calculate the PSNR and SSIM between the enhanced and original images (treated as high-quality images).

\noindent \textbf{Vessel Segmentation}. The vessel segmentation task is performed using the DRIVE dataset, which includes annotated masks to further evaluate the ability to preserve blood vessel structures during denoising. We follow the official split, resulting in 20 subjects each in the training and testing sets. We use the enhanced images as training and testing images generated from the generalization evaluation. The vessel segmentation task is evaluated using the Area Under the ROC Curve (AUC), the Area under the Precision-Recall Curve (PR), F1 Score, and Specificity (SP).


\noindent \textbf{Lesion Segmentation}. We use the segmentation masks provided with the IDRID dataset. Since the downstream segmentation tasks were trained and tested solely on enhanced images(obtained from the generalization task), without additional preprocessing or enhancements, we focused only on larger, easier-to-train lesion types, including Hard Exudates (EX) and Hemorrhages (HE). The training set includes 54 subjects, while the testing set includes 27 subjects. Performance is measured using AUC, PR, and F1 score.

\subsection{No-Reference Quality Assessment Experiments}
Evaluating enhancement quality without ground-truth clean images presents a particular challenge for paired methods. Therefore, we focused on unpaired methods to assess real-world denoising capabilities. For this evaluation, we trained the unpaired method using the No-Reference Assessment Dataset, processing 2,434 low-quality testing images to generate enhanced images. These enhanced images were then used in various downstream evaluations, including DR grading, representation feature analysis, and expert assessment by medical professionals. A no-reference quality assessment was conducted on the following baselines: CycleGAN~\cite{cyclegan}, WGAN~\cite{gulrajani2017improved}, OTTGAN~\cite{wang2022optimal}, OTEGAN~\cite{zhu2023optimal}, Context-aware OT~\cite{vasa2024context}, and CUNSB-RFIE~\cite{dong2024cunsb}.
All models were trained using the parameters followed in the original papers. Refer to the \textcolor{red}{Appendix B} for more details.

\noindent\textbf{DR grading.} We trained an NN-MobileNet model~\cite{deeplearning1} for the DR grading task using real-world high-quality images. The enhanced test images are used with the trained NN-MobileNet to infer DR grading classification. Enhancement performance is evaluated based on classification accuracy (ACC), kappa score, F1 score, and AUC. This evaluation primarily aims to assess whether the denoising model disrupts lesion distribution, potentially leading to inconsistencies with the original DR grading labels.

\noindent\textbf{Representation Feature Evaluation.} We employed two fundus image-based foundation models (Retfound~\cite{zhou2023foundation} and Ret-clip~\cite{du2024ret}) to calculate the Fréchet inception distance (FID) between enhanced and real-world high-quality image feature representation, referred to as \textit{FID-Retfound} and \textit{FID-Clip}. \textit{FID-Retfound} measures the preservation of disease-related information, while \textit{FID-Clip} assesses the similarity of spatial structures and continuous features. 

\noindent\textbf{Experts Annotation Evaluation.}
\begin{figure}
    \centering
    \includegraphics[width=1.0\linewidth]{Figures/experimental_design.png}
    \caption{An illustrative medical expert clinical preference evaluation between (a) lesion preserving, (b) background preserving, and (c) structure-preserving.}
    \label{fig:expert-protocol}
\end{figure}
To better align with clinical preferences, we evaluated the enhanced images following protocols provided by medical experts. This evaluation includes the Background Preserving Ratio (BPR), Lesion Preserving Ratio (LPR), and Structure Preserving Ratio (SPR), each used to calculate the proportion of changes in the enhanced images. Importantly, we did not use all 2,434 testing images; instead, we selected 159 images with more prominent lesions, specifically those at DR grading levels 2, 3, and 4. These protocols are shown in Fig.~\ref{fig:expert-protocol}, which evaluate whether the denoised images maintain consistency with the original images regarding background, lesion, and structural integrity, helping to evaluate the practical applicability of these unpaired denoising models in real-world medical settings.

\begin{figure*}[t] \centering \includegraphics[width=\textwidth]{Figures/spearman.pdf}  % Replace with your figure file and remove the example 
\caption{Validation of Expert Clinic Preference Alignment via Spearman’s correlation coefficient ($r$), which is used to assess the correlation between the Experts Protocol preference evaluation and other Eyebench evaluations. Single-dimension evaluations (e.g., denoising, segmentation) may show weak alignment with clinic preferences, while Eyebench multi-dimensional evaluations (e.g., Full-Reference, No-Reference) demonstrated stronger correlation.} 
\label{fig:correlation} 
\end{figure*}

\begin{figure*}[t] \centering \includegraphics[width=\textwidth]{Figures/tsne.png}  % Replace with your figure file and remove the example 
\caption{T-SNE visualizations of the latent representation features extracted from the RET-Clip and RETfound models. Closer proximity of the distributions indicates improved denoising performance of the unpaired method. This analysis demonstrates the effectiveness of the retrieval-enhanced frameworks in capturing and preserving meaningful feature representations. The Euclidean distance between the distribution centroids is showcased under each plot.} 
\label{fig:tsne} 
\end{figure*}


%%%
%  \begin{figure*}[t] \centering \includegraphics[width=0.9\textwidth]{Figures/SDE_analyse_3.pdf}  % Replace with your figure file and remove the example 
% \caption{An illustration of high frequency information smoothing shown in SDE-based method. The yellow box highlights the lesion structure deactivation in iterative generation process.} 
% \label{fig:SDE-analyze} 
% \end{figure*}



\subsection{Experiment Results}

\noindent\textbf{Full-Reference Evaluation.} Overall, paired methods outperform unpaired methods. As shown in Tab.~\ref{tb:deg-exp}, paired methods, particularly GFE-Net, effectively leverage frequency information, achieving higher SSIM (0.9554, 0.7935) and PSNR (29.719, 25.012) on EyeQ and IDRID, respectively. Among unpaired methods, CycleGAN and OTEGAN demonstrate competitive performance, especially on IDRID and DRIVE, where CycleGAN leads in SSIM (0.7668, 0.6681) and PSNR (22.511, 22.696), indicating robust noise reduction and generalization on unseen datasets. In segmentation tasks (Tab.~\ref{tab-seg}), SCR-Net achieves the highest AUC (0.9227), PR (0.7783), and F1 scores (0.7) in vessel segmentation among paired methods. Unpaired models CUNSB-RFIE and OTEGAN also perform comparably, with CUNSB-RFIE achieving the highest AUC (0.9163). For lesion segmentation, GFE-Net excels in HE lesions, while unpaired models CUNSB-RFIE and OTEGAN attain high F1 scores for EX lesions, demonstrating their effectiveness in lesion preservation.  

\begin{comment}
\noindent \textbf{Full-Reference Evaluation.} Tab.~\ref{tb:deg-exp} present the results of all methods in full-reference evaluations. Overall, paired methods outperform unpaired methods. Among paired methods, frequency information is effectively utilized to preserve retinal structures, leading to superior capability. Specifically, GFE-Net attained the best SSIM (0.7935) and PSNR (25.012) on IDRID and performed second-best on the other two datasets. For unpaired methods, CycleGAN achieved the highest PSNR (25.076) and second-highest SSIM (0.9313) on EyeQ, indicating that strong cycle consistency aids in structural preservation. Other OT-related methods demonstrated comparable performance, with OTE-GAN obtaining the highest SSIM (0.9392) and second-highest PSNR (24.812) on EyeQ. Additionally, CUNSB-RFIE showed robust generalization, achieving the second-best results on IDRID and DRIVE. 



Tab.~\ref{tab-seg} demonstrate the necessity for multidimensional evaluation. Specifically, CycleGAN while excel in noise reduction, struggled to accurately segment the vessel and lesion structure. CUNSB-RFIE demonstrate strong generalization.
\end{comment}

Since collecting paired noisy and clean images is challenging in real-world settings, unpaired methods are increasingly prioritized by medical experts.
Notably, some methods excel in noise reduction but face challenges in segmentation (e.g., CycleGAN), whereas SDE-based approaches like CUNSB-RFIE show strong generalization and excel in downstream segmentation. This highlights the need for multidimensional evaluation, as high noise reduction performance does not ensure the preservation of small, clinically significant structures.


\noindent\textbf{No-Reference Evaluation.}
Tab.~\ref{tab-noref} compares several methods for No-reference quality assessment in DR grading, Fréchet Inception Distance (FID) metrics, and Expert Protocol Evaluation. Each method's performance is evaluated across DR grading metrics (ACC, Kappa score, F1 score, AUC), FID scores (Retfound and Clip), and expert assessments (LPR, BPR, SPR). The best and second-best scores in each metric highlight the leading methods.
For DR grading, CycleGAN achieves the highest ACC (0.7588) and ranks second in Kappa, F1, and AUC, indicating strong grading capability. However, OTEGAN surpasses CycleGAN in overall quality metrics, with the highest Kappa (0.6433), F1 (0.7228), and AUC (0.9326), suggesting greater consistency and predictive accuracy for critical assessments. In FID metrics, which evaluate image realism and diversity, OTEGAN and CycleGAN excel. OTEGAN has the lowest FID-Clip score (11.114) and second-best FID-Retfound score (28.987), indicating superior image quality. CycleGAN scores best in FID-Retfound (23.778) and second-best in FID-Clip (11.530), showing strong but slightly less consistent image realism. Expert evaluations also favor OTEGAN and CycleGAN. CycleGAN achieves the highest SPR (0.8726), while OTEGAN and CUNSB-PRIE excels in LPR and BPR, with scores of 0.8280 and 0.8981 / 0.8535. these results suggest that the SDE-based method has more stable modality generation and clinic preference.

In summary, OTEGAN leads across multiple metrics, especially in DR grading, FID, and expert protocols, while CycleGAN follows closely, excelling in accuracy and realism. Other models, like Context-aware OT and CUNSB-RFIE, have strengths in specific areas but lack OTEGAN and CycleGAN consistency. This analysis underscores the effectiveness of OTEGAN in no-reference quality assessments, offering distinct advantages in image realism and expert evaluations.

% \begin{table*}[!t]
% \centering
% \caption{Performance of baseline methods on the enhancement task. The best performance in each column is highlighted in bold, while the second-best is underlined. The results are divided into paired and unpaired methods.}
% \tiny
% \resizebox{0.9\textwidth}{!}{%
% \begin{tabular}{cccccccc}
% \toprule
% \multirow{2}{*}{} & \multirow{2}{*}{\textbf{Method}} & \multicolumn{2}{c}{\textbf{EyeQ}} & \multicolumn{2}{c}{\textbf{IDRID}} & \multicolumn{2}{c}{\textbf{DRIVE}} \\ \cmidrule(l){3-8} 
%                                           &                                  & \textbf{SSIM} $\uparrow$   & \textbf{PSNR} $\uparrow$   & \textbf{SSIM} $\uparrow$   & \textbf{PSNR} $\uparrow$   & \textbf{SSIM} $\uparrow$   & \textbf{PSNR} $\uparrow$   \\ \midrule
% \multirow{5}{*}{\textit{Paired Methods}} & SCR-Net~\cite{li2022structure}   & \textbf{0.9606} & 29.698 & 0.6425 & 18.920 & \textbf{0.6824} & 23.280 \\      
%                                          & Cofe-Net~\cite{shen2020modeling} & 0.9408           & 24.907           & 0.7397            & 20.058            & 0.6671            & 21.774            \\
%                                          & PCE-Net~\cite{10.1007/978-3-031-16434-7_49} & 0.9487           & \textbf{29.895}           & \underline{0.7764}            & \underline{23.201}           & 0.6704            & 24.041           \\
%                                          & GFE-Net~\cite{li2023generic}     & \underline{0.9554}           & \underline{29.719}           & \textbf{0.7935}            & \textbf{25.012}           & \underline{0.6793}            & \underline{23.786} \\
%                                          &RFormer~\cite{deng2022rformer}
%                                           & 0.9260   & 27.163   & 0.5963   & 18.433   & 0.6311   & 22.172\\
%                                           & I-SECRET~\cite{i-secret}        & 0.9051 & 23.483 & 0.7157 & 20.173 & 0.5727 & 18.803 \\
%                                           \midrule
% \multirow{7}{*}{\textit{Unpaired Methods}} 
%                                          & CycleGAN~\cite{cyclegan}         & \underline{0.9313}           &\textbf{25.076}           & \textbf{0.7668}            & \textbf{22.511}           & \textbf{0.6681}            & \textbf{22.686}  \\
%                                          & WGAN~\cite{gulrajani2017improved} & 0.9266          & 24.793           & 0.7316           & 21.325           & 0.6431            & 20.408 \\
%                                          & OTTGAN~\cite{wang2022optimal}    & 0.9275           & 24.065           & 0.7509           & 22.131           & 0.6635            & 21.938 \\
%                                          & OTEGAN~\cite{zhu2023optimal}     & \textbf{0.9392}           & \underline{24.812}           & 0.7624           & 22.272           & 0.6642            & 22.183 \\
%                                          & Context-aware OT~\cite{vasa2024context} & 0.9144           & 24.088           & 0.7338            & 21.790            & 0.6407            & 21.389 \\
%                                          & CUNSB-RFIE~\cite{dong2024cunsb}  & 0.9121           & 24.242           & \underline{0.7651}           & \underline{22.448}           & \underline{0.6659}            & \underline{22.510} \\
% \bottomrule
% \end{tabular}%
% }
% \label{tb:deg-exp}
% \end{table*}

% \begin{table*}[!t]
%     \centering
%     \caption{Performance comparison with paired and unpaired baselines on blood vessel and diabetic lesions (EX and HE) segmentation tasks on the DRIVE~\cite{drive} and IDRID~\cite{idrid} dataset, respectively. The best performance in each column is highlighted in bold, while the second-best is underlined. The results are divided into paired and unpaired methods.}
%     \tiny
%     \resizebox{0.9\textwidth}{!}{%
%     \begin{tabular}{lcccc|ccc|ccc}
%         \toprule
%          \multirow{2}[3]{*}{Method} & \multicolumn{4}{c}{Vessel Segmentation} & \multicolumn{3}{c}{EX} & \multicolumn{3}{c}{HE} \\ 
%          \cmidrule(lr){2-5}  \cmidrule(lr){6-8}  \cmidrule(lr){9-11}
          
%          & AUC $\uparrow$ & PR $\uparrow$ & F1 Score $\uparrow$ & SP $\uparrow$ & AUC  & PR & F1 Score  & AUC & PR & F1 Score \\ \midrule

%         SCR-Net~\cite{li2022structure}   & \textbf{0.9227} & \textbf{0.7783} & \textbf{0.7000} & 0.9787 & \textbf{0.9683} & \textbf{0.6041} & \textbf{0.5556} & 0.9377 & 0.3213 & 0.3725\\ 
%         cofe-Net~\cite{shen2020modeling} & \underline{0.9188} & \underline{0.7698} & 0.6895 & 0.9801 & 0.9623 & 0.5620 & 0.5349 & 0.9302 & 0.3152 & 0.3281\\
%         PCE-Net~\cite{10.1007/978-3-031-16434-7_49} & 0.9146 & 0.7616 & 0.6790 & \textbf{0.9814} & \underline{0.9667} & \underline{0.5876} & 0.5066 & \underline{0.9545} & \underline{0.3639} & \underline{0.3736}\\
%         GFE-Net~\cite{li2023generic} & 0.9175 & 0.7669 & 0.6832 & \textbf{0.9814} & 0.9560 & 0.5548 & \underline{0.5380} & \textbf{0.9577} & \textbf{0.4113} & \textbf{0.3751}\\
%         RFormer~\cite{deng2022rformer} & 0.8990 & 0.7239 & 0.6374 & \underline{0.9806} & 0.9626 & 0.5593 & 0.4692 & 0.9207 & 0.2677 & 0.3136 \\
%         I-SECRET~\cite{i-secret} & 0.9181 & 0.7662 & 0.6838 & 0.9802 & 0.9613 & 0.5424 & 0.4825 & 0.9028 & 0.2629 & 0.2642 \\
%     \midrule
%         CycleGAN~\cite{cyclegan} & 0.9015 & 0.7278 & 0.6462 & \underline{0.9801}  & 0.9447& 0.4843 & 0.4790 & 0.8970 & 0.1624 & 0.2227\\
%         WGAN~\cite{gulrajani2017improved} & 0.9081 & 0.7494 & 0.6768 & 0.9764  & 0.9522 & 0.4942 & 0.4859 & \underline{0.8990} & \underline{0.1847} & \underline{0.2476}\\
%         OTTGAN~\cite{wang2022optimal} & 0.9034 & 0.7400 & 0.6609 & \textbf{0.9812}  & 0.9492 & 0.4214 & 0.4365 & 0.8179 & 0.1448 & 0.2233 \\
%         OTEGAN~\cite{zhu2023optimal} & \underline{0.9156} & \textbf{0.7678} & \textbf{0.6919} & 0.9797 & \underline{0.9562} & \underline{0.5191} & \underline{0.4868} & \textbf{0.9359} & \textbf{0.2800}& \textbf{0.3165} \\
%         Context-aware OT~\cite{vasa2024context} & 0.8871 & 0.7077 & 0.6377 & 0.9791 & 0.9305 & 0.3318 & 0.3707 & 0.8091 & 0.0646 & 0.1184 \\
%         CUNSB-RFIE~\cite{dong2024cunsb}  & \textbf{0.9163} & \underline{0.7626} & \underline{0.6872} & 0.9784 & \textbf{0.9572}& \textbf{0.5381} & \textbf{0.4883} & 0.8488 & 0.1489 & 0.1893 \\
%         \bottomrule
%     \end{tabular}}
%     \label{tab-seg}
%     \vspace{-0.3cm}
% \end{table*}




% \begin{table*}[!t]
%     \centering
%     \caption{Performance comparison with unpaired baselines on No-reference quality assessment task. The best performance in each column is highlighted in bold, and the second-best is underlined.}
%     \tiny
%     \resizebox{0.9\textwidth}{!}{%
%     \begin{tabular}{lcccc|cc|ccc}
%         \toprule
%          \multirow{2}[3]{*}{Method} & \multicolumn{4}{c}{DR grading} & \multicolumn{2}{c}{Fréchet Inception Distance} & \multicolumn{3}{c}{Experts Protocol Evaluation} \\ 
%          \cmidrule(lr){2-5}  \cmidrule(lr){6-7}  \cmidrule(lr){8-10}
          
%          & ACC $\uparrow$ & Kappa $\uparrow$ & F1 Score $\uparrow$ & AUC $\uparrow$ & FID-Retfound~\cite{zhou2023foundation}$\downarrow$  & FID-Clip~\cite{du2024ret} $\downarrow$ & LPR $\uparrow$ & BPR $\uparrow$ & SPR $\uparrow$\\ \midrule
         
%         CycleGAN~\cite{cyclegan}                & \textbf{0.7588} & \underline{0.6006} & \underline{0.7180} & \underline{0.9251}  & \textbf{23.778}& \underline{11.530}  & 0.7707 & 0.8153 & \textbf{0.8726}\\
        
%         WGAN~\cite{gulrajani2017improved}       & 0.6446 & 0.3123 & 0.6156 & 0.8874  & 74.885 & 33.076  & 0.4076 & 0.4204 & 0.6561\\
        
%         OTTGAN~\cite{wang2022optimal}           & 0.7440 & 0.5688 & 0.7037 & 0.9247  & 51.201 & 20.505 & 0.4586 & 0.7580 & 0.5860 \\
        
%         OTEGAN~\cite{zhu2023optimal}            & \underline{0.7539} & \textbf{0.6433} & \textbf{0.7228} & \textbf{0.9326} & \underline{28.987} & \textbf{11.114} & \underline{0.8280} & \textbf{0.8981} & 0.6178 \\
        
%         Context-aware OT~\cite{vasa2024context} & 0.7301 & 0.3811 & 0.6662 & 0.9112 & 61.429 & 34.456 & \textbf{0.8408} & 0.4522 & \underline{0.7389} \\
        
%         CUNSB-RFIE~\cite{dong2024cunsb}         & 0.6565 & 0.3674 & 0.6341 & 0.8927 & 33.047 & 14.827  & \underline{0.8280} & \underline{0.8535} & 0.6879 \\
        
%         \bottomrule
%     \end{tabular}}
%     \label{tab-seg}
%     \vspace{-0.3cm}
% \end{table*}
%%%
% \begin{figure*}[h]
%   \centering
%   \includegraphics[width=\textwidth]{Figures/final_result_illustration.pdf}  % Replace with your figure file and remove the example
%   \caption{test-caption}
%   \label{fig:final-ill}
% \end{figure*}

%%%
\begin{figure}
     \centering
     \includegraphics[width=0.9\linewidth]{Figures/SDE_analysis.png}
     \caption{Illustration of denoising quality and skip connection feature patches of CUNSB-RFIE as steps $t_i$ increase. A higher FID score indicates lower quality, with the skip connection feature patches emphasizing lesion structures. This analysis demonstrates that high-frequency lesion details are gradually smoothed out over the denoising process.  }
     \label{fig:SB-noreference}
 \end{figure}
%  \begin{figure}[t] \centering \includegraphics[width=\textwidth]{Figures/SDE_final_version.pdf}  % Replace with your figure file and remove the example 
% \caption{An illustration of high frequency information smoothing shown in SDE-based method.\textbf{(A)} highlights the attention shift to irrelevant region with red box. \textbf{(B)} highlight the lesion structure deactivation with yellow box.} 
% \label{fig:SDE-analyze} 
% \end{figure}


\section{Further Analysis}

\noindent\textbf{The necessity of multi-dimensional evaluation.} 
To further analyze the importance of the multi-dimensional evaluation of Eyebench, we visualized the correlation between medical experts guiding the protocol evaluation and other evaluations, including single-dimension and our multi-dimension evaluation (Full-Reference and No-Reference), as shown in Fig.~\ref{fig:correlation}. The results demonstrate that our multi-dimensional evaluation closely aligns with clinical preferences, whereas single-dimensional evaluations are likely to exhibit low correlation (e.g., denoising, DR grading, and segmentation). This finding also suggests that relying solely on a single task is insufficient to evaluate enhanced image quality. A multi-dimensional benchmark provides a more comprehensive comparison, offering medical experts greater insight and reference.

\noindent\textbf{Denoising Generalization Ability.} The evaluation of denoising methods necessitates a comparative analysis of the similarity between high-quality and enhanced-quality domains. We conducted a T-SNE~\cite{van2008visualizing} analysis to quantify this similarity and calculated the distance between centroids representing the real-world and enhanced domains, as illustrated in Fig.~\ref{fig:tsne}. Both OTEGAN and CycleGAN leverage high-quality prior learning, improving generalization capabilities. This high-quality regularization narrows the search space for GANs, enabling optimal transport between domains. Notably, the SDE-based method also shows strong potential to compete with GAN-based methods due to their stable modal-distribution modeling. In our Eyebench, CycleGAN, OTEGAN, and CUNSB-RFIE also demonstrated superior denoising generalization ability, indicating that high-quality regularization and SDE-based methods can enhance generalization capabilities.

\noindent\textbf{Trade off in GAN-based Methods.} We found that in these OT-based GANs, a regularization term is enforced between the noised and enhanced images to maintain consistency as enhanced images are transferred to a high-quality image domain. Since these two processes are trade-offs, contextual features or SSIM (structural similarity index) are often used to preserve lesions and vascular structures rather than noise. Thus, selecting an appropriate metric is crucial in this process. Depending on different downstream requirements, adjusting the weight of this regularization term is essential: too high a weight may prevent denoising, while too low a weight may cause the model to misclassify lesions and vessels as noise. A promising solution is the introduction of high-quality image priors in OTE-GAN and CycleGAN. CycleGAN employs cycle consistency, while OTE-GAN leverages structural consistency in high-quality images. This enables the model to partially learn structural and noise priors from high-quality images, preventing excessive structural alterations or incomplete noise modeling during denoising.

\noindent\textbf{Limitation in SDE-based method}. Since the SDE-based approach (i.e., CUNSB-RFIE) enforces smooth probabilistic distribution transport, we observed that high-frequency components, such as retinal lesions, were progressively smoothed out during the iterative generation process, resulting in a noticeable performance decline. We adopted the notation from~\cite{dong2024cunsb}, where $\hat{\mathbf{y}}_1^{t_i}$ for $i \in \{ 1, 2, 3, 4,5\}$ represents the progressively refined high-quality image counterparts. As shown in Fig.~\ref{fig:SB-noreference}, with each increment in step $i$, both FID-Retfound and FID-Clip scores increased, indicating a clear quality degradation. To further interpret this phenomenon, we visualized two skip connection features of the U-Net generator~\cite{dong2024cunsb}. As $t_i$ increases, the model progressively deactivates regions containing lesions. Due to the need for the SDE solver to generate $\hat{\mathbf{y}}_1^{t_i}$ to model the bridge $\mathbb{Q}^{SB}$ during training, high-frequency information tend to becomes smoothed out or diminished in the forward process, as Gaussian noise is added.




\section{Conclusion}
 
With the rapid development of generative models, aligning future methods for denoising fundus images with clinical needs has become essential. In this paper, we propose a new benchmark designed to provide more rigorous, clinically relevant evaluations of enhanced images, enabling broader access for medical experts. Furthermore, these multi-dimensional evaluations have demonstrated a strong correlation with manual expert evaluations, helping to bridge the gap in applying generative model-based denoising methods to real-world clinical requirements.

\noindent\textbf{Limitations and Future Work.} Currently, our evaluations are primarily based on deep learning methods. In the future, we plan to expand our work to include more unsupervised traditional algorithms and apply these methods to MRI enhancement tasks.

\noindent\textbf{Ethics Statement.} This retrospective study used open-access human subject data and did not require ethical approval, as confirmed by their license~\cite{fu2019evaluation,diabetic-retinopathy-detection,idrid,drive}.

{
    \small
    \bibliographystyle{ieeenat_fullname}
    \bibliography{main}
}


 \clearpage\section*{\Large Supplementary Materials - DGR-MIL: Exploring Diverse Global Representation in Multiple Instance Learning for Whole Slide Image Classification}

\thispagestyle{empty}
\appendix

%%%%%%%%% BODY TEXT - ENTER YOUR RESPONSE BELOW
\section{Full-Reference Quality Assessment Experiments Details }\label{Sec:full-reference}

\subsection{Datasets.} For full-reference assessment, we used the previously synthesized Full-Reference Evaluation Dataset. We strictly followed the training configurations for paired and unpaired methods. For the unpaired method, synthetic low-quality images from the training set $A$ (i.e., $A^{\ast}$) were used as input images, while high-quality images from the training set $B$ served as the clean reference images. For the paired method, we performed supervised training using low-high-quality image pairs from the training set $A$ (i.e., $A^{\ast}$ and $A$). 


\subsection{SCR-Net~\cite{li2022structure}}
The model was trained for 150 epochs using Adam optimizer, with an initial learning rate of $2 \times 10^{-4}$ and $\beta_1$ value set to $0.5$, followed by 50 epochs with a learning rate linearly decayed to $0$. The training batch size was 32. All images were resized to $ 256 \times 256$ with a random flipping data augmentation technique. For model architectures, the generator and discriminator architectures followed the architectures and configurations described in~\cite{li2022structure}.

%Network initialization was performed using normal methods from their codes~\cite{li2022structure}, with a scaling factor of $0.02$. Dropout was enabled for the generator.

%Input images were resized to 256×256, and the training batch size was 32. The training process comprised an initial phase with a constant learning rate for 150 epochs, followed by a linear decay over 50 epochs. The Adam optimizer was employed with an initial learning rate 0.0002 and a momentum of 0.5. The encoder consisted of 8 layers. 




\subsection{Cofe-Net~\cite{shen2020modeling}}
\begin{comment}
For this model, we followed the original configuration~\cite{shen2020modeling}. The training process included two stages. First, the Retinal Structure Awareness (RSA) module, which is used to perform the vessel segmentation task, and the Low-Quality Adaptation (LQA) module, which addresses degradation features, were trained separately. Subsequently, the entire framework was trained end-to-end, enabling collaboration between the modules for improved performance. The proposed cofeNet is implemented in PyTorch, using stochastic gradient
descent (SGD) for optimization. The learning rate is initialized to $1 \times 10^{-4}$ for the first 150 epochs and then gradually decayed to zero over the next 150 epochs.
\end{comment}


The model was trained for 300 epochs using the SGD optimizer, with an initial learning rate of $1 \times 10^{-4}$, which was gradually reduced to 0 over the final 150 epochs. The training batch size was 16, and all images were resize to $512 \times 512$.

The loss function comprised four components: main scale error loss ($L_m$), multiple-scale pixel loss ($L^s_p$), multiple-scale content loss ($L^s_c$) and RSA module loss ($L_v$), as described in~\cite{shen2020modeling}, where the $s$ denotes the scale index. The weight for $L^s_p$, $L^s_c$ and $L_v$ was set to $\lambda_p=10$, $\lambda_c=1$ and $\lambda_v=0.1$, respectively, during the training process.


\subsection{PCE-Net~\cite{10.1007/978-3-031-16434-7_49}}
\begin{comment}
The model was implemented based on a U-Net architecture~\cite{10.1007/978-3-031-16434-7_49} and trained for 200 epochs using the Adam optimizer. The learning rate was set to 0.001 for the first 150 epochs and gradually decayed to 0 over the subsequent 50 epochs to ensure smooth convergence. A batch size of 4 was used, and instance normalization was applied to stabilize the optimization process. Input images were resized to 256×256, and data augmentation techniques, including random horizontal and vertical flipping, were employed to enhance generalization. The loss function comprised two components: the enhancement loss ($L_E$) and the weighted feature pyramid constraint loss ($L_C$), as described in~\cite{10.1007/978-3-031-16434-7_49}. The weight for $L_C$ was set to $\lambda_C=0.1$ during training.
\end{comment}


The model was trained for 200 epochs using the Adam optimizer, with an initial learning rate of $1 \times 10^{-3}$, which was gradually reduced to 0 over the final 50 epochs. The training batch size was 4, and all input images were resized to $256 \times 256$. Data augmentation strategies, including random horizontal and vertical flips with a probability of 0.5, were applied to enhance generalization.

The loss function comprised two components: enhancement loss ($L_E$) and the weighted feature pyramid constraint loss ($L_C$), as described in~\cite{10.1007/978-3-031-16434-7_49}. The weight for $L_C$ was set to $\lambda_C=0.1$ during the training process. Additionally, we adopted a U-Net architecture proposed in ~\cite{10.1007/978-3-031-16434-7_49}.

\subsection{GFE-Net~\cite{li2023generic}}
\begin{comment}
GFE-Net employs a symmetric U-Net architecture with 8 down-sampling and 8 up-sampling layers, using a kernel size of $4 \times 4$ and a stride of 2~\cite{li2023generic}.
During training, input images were resized to 256×256, and a batch size of 4 was used. The model was trained for 200 epochs using the Adam optimizer, with an initial learning rate of 0.001 for the first 150 epochs, gradually reduced to 0 over the final 50 epochs. Data augmentation techniques, including random horizontal and vertical flipping, were applied.
The loss function consisted of three components: enhancement loss, cycle-consistency loss, and reconstruction loss, each weighted equally. Default weights of $\lambda = 1.0$ were used for all components during training.
\end{comment}

The model was trained for 200 epochs using the Adam optimizer, with an initial learning rate of $1\times 10^{-3}$, which was gradually reduced to 0 over the final 50 epochs. The training batch size was set to 4, and all input images were resized to $256 \times 256$. Data augmentation strategies, including random horizontal and vertical flips with a probability of 0.5, were applied to enhance generalization.

We employed the same weight (e.g., $\lambda_{all}$ = 1) for all loss losses, including enhancement loss, cycle-consistency loss, and reconstruction loss. Furthermore, we adopted the architecture proposed in~\cite{li2023generic}, implementing a symmetric U-Net with 8 down-sampling and 8 up-sampling layers.

\subsection{I-SECRET~\cite{i-secret}}

The model was trained for 200 epochs using Adam optimizer with an initial learning rate of $1 \times 10^{-4}$ and $\beta$ values set to $0.5$ and $0.999$, respectively. The learning rate followed a cosine decay schedule. The training batch size was set to 8. All images were resized to $256 \times 256$ with random cropping and flipping augmentation strategies.

For model architectures, the generator consisted of 2 down-sampling layers, each with 64 filters and 9 residual blocks. Input and output channels were set to 3 for RGB inputs. The discriminator included 64 filters and 3 layers. Instance normalization and reflective padding were used. The training process employed a least-squares GAN loss~\cite{mao2017least}, a ResNet-based generator, and a PatchGAN-based~\cite{isola2017image} discriminator. GAN and reconstruction losses were weighted at $1.0$, while their importance with the contrastive loss (ICC-loss) and importance-guided supervised loss (IS-loss)~\cite{i-secret} were enabled with weights of $1.0$. 


\subsection{ RFormer~\cite{deng2022rformer}.}
The model was trained for 150 epochs using Adam optimizer, with an initial learning rate of $1 \times 10^{-4}$ and $\beta$ values set to 0.9 and 0.999, respectively. The cosine annealing strategy was employed to steadily decrease the learning rate from the initial value to $1 \times 10^{-6}$ during the training procedure. The training batch size was set to 32. All images were resized to $256 \times 256$ without any additional augmentation strategies. The model architecture followed the design proposed in~\cite{deng2022rformer}, which was consistently maintained throughout our experiments.

%The dataset loading strategy was modified for our pre-processed dataset. Our training was performed on a single A100 GPU. The training batch size was set to 32, while the total number of epochs for training was set to 150. The Adam optimizer with $\beta_1 = 0.9$ and $\beta_2 = 0.999$ was adopted. The initial learning rate was set to $1 \times 10^{-4}$. 



\subsection{CycleGAN~\cite{cyclegan}, WGAN~\cite{gulrajani2017improved}, OTTGAN~\cite{wang2022optimal}, OTEGAN~\cite{zhu2023optimal}  }
The models were trained for 200 epochs using the RMSprop optimizer, with initial learning rates for the generator and discriminator set to $0.5 \times 10^{-4}$ and $1 \times 10^{-4}$, respectively. The learning rate followed a linear decay schedule, decreasing by a factor of 10 every 100 epochs. The training batch size was set to 2. All input images were resized to $256 \times 256$, with random horizontal and vertical flips applied as augmentation strategies.
For CycleGAN, the weighting parameters in the final objective were set to $\lambda_{GAN} = 1$, $\lambda_{Cycle} = 10$, and $\lambda_{Idt} = 5$, corresponding to the weights for the GAN loss, cycle consistency loss, and identity loss, respectively. The Mean Squared Error (MSE) loss was used for the GAN loss, while the cycle consistency and identity losses were computed using the L1-norm. For OTTGAN and OTEGAN, the weighting parameter $\lambda_{OT}$ was set to 40, representing the optimal transport (OT) cost. Furthermore, the OT loss was calculated using the MSE loss for OTTGAN and the MS-SSIM loss for OTEGAN. The generator and discriminator architectures were implemented following the baseline designs described in~\cite{zhu2023optimal,zhu2023otre}.

\subsection{Context-aware OT~\cite{vasa2024context}}
The model was trained for 200 epochs using the RMSprop optimizer, with initial learning rates for the generator and discriminator set to $0.5 \times 10^{-4}$ and $1 \times 10^{-4}$, respectively. The learning rate followed a linear decay schedule, decreasing by a factor of 10 every 50 epochs. The training batch size was set to 2. All input images were resized to $256 \times 256$ without additional augmentation strategies.
A warm-up training strategy was employed, wherein the context-OT loss was introduced after the first 50 epochs. The weighting parameter for this loss was set to $5\times10^{-2}$. We utilized a pre-trained VGG~\cite{mechrez2018contextual} network outlined in~\cite{vasa2024context} to compute the OT loss at feature spaces.
The generator and discriminator architectures followed the designs outlined in~\cite{vasa2024context}.

\subsection{CUNSB-RFIE~\cite{dong2024cunsb}}
The model was trained for 130 epochs using the Adam optimizer, with an initial learning rate of $2 \times 10^ {-4}$. The learning rate was linearly decayed to 0 after the first 80 epochs, and the batch size was set to 8. All input images were resized to $256 \times 256$ without applying any additional augmentation strategies.

The weighting parameters in the final objective were set as $\lambda_{SB} = 1$, $\lambda_{SSIM} = 0.8$, and $\lambda_{NCE} = 1$, corresponding to the weights for entropy-regularized OT loss, task-specific regularization with MS-SSIM~\cite{brunet2011mathematical}, and PatchNCE~\cite{park2020contrastive} loss, respectively.

The generator and discriminator architectures followed the designs described in~\cite{dong2024cunsb}. Specifically, the base number of channels for the generator was set to 32, and 9 ResNet blocks were used in the bottleneck. In addition to the output features of all downsampling layers, the bottleneck's input and middle feature maps were utilized to calculate the PatchNCE regularization.

\subsection{Vessel Segmentation}
A vanilla U-Net model~\cite{ronneberger2015unet} was employed for the downstream vessel segmentation task. The network comprised 4 layers with a base channel size 64 and a channel scale expansion ratio of 2. The training was conducted over 10 epochs using the Adam optimizer, with a batch size of 64 and an initial learning rate of $5 \times 10^{-5}$, which followed a cosine annealing learning rate scheduler. 

Before training, the enhanced images and their corresponding ground-truth vessel segmentation masks were preprocessed. The preprocessing pipeline included random cropping to $ 48 \times 48$ patches, followed by data augmentation techniques such as random horizontal flips, random vertical flips (with a probability of 0.5), and random rotation.


\begin{figure}[ht]
    \centering
    \includegraphics[width=1\linewidth]{Figures/experimental_design.png}
    \caption{An illustrative medical expert clinical preference evaluation between (a) lesion preserving, (b) background preserving, and (c) structure-preserving.}
    \label{fig:expert-protocol}
\end{figure}


\begin{figure*}[t]
  \centering
  \includegraphics[width=\textwidth]{Figures/Eye_Q_denoise.pdf}  % Replace with your figure file and remove the example
  \caption{ Illustration of the Denoising Evaluation on the EyeQ dataset. The first and second columns show the high- and low-quality image references, respectively, while the remaining columns display the synthetic high-quality images generated by all baseline models.
  }
  \label{fig:full-reference-eyeq}
\end{figure*}
%%%
\begin{figure*}[t]
  \centering
  \includegraphics[width=\textwidth]{Figures/denosing_generalization.pdf}  % Replace with your figure file and remove the example
  \caption{ Illustration of the Denoising Generalization Evaluation on the DRIVE and IDRID datasets. The first and second columns show the high- and low-quality image references, respectively, while the remaining columns display the synthetic high-quality images generated by all baseline models.
  }
  \label{fig:full-reference-generalization}
\end{figure*}
%%%
\begin{figure*}[t]
  \centering
  \includegraphics[width=\textwidth]{Figures/segmentation_result_1.PDF}  % Replace with your figure file and remove the example
  \caption{ Illustration of Vessel and Lesion (EX and HE) Segmentation Experiments. The first column shows the reference segmentation masks, while the remaining columns display the segmentation results produced by all baseline models.
  }
  \label{fig:full-reference-segmentation}
\end{figure*}
\begin{figure*}[t]
  \centering
  \includegraphics[width=\textwidth]{Figures/no-reference.pdf}  % Replace with your figure file and remove the example
  \caption{ Illustration of the denoising results in the No-Reference Quality Assessment Experiments. The first column shows the input low-quality image, while the remaining columns display the synthetic high-quality images generated by all unpaired baseline models.
  }
  \label{fig:no-reference}
\end{figure*}


\section{No-Reference Quality Assessment Experiments Details}
We utilized the No-Reference Evaluation Dataset, including all unpaired baseline models for the No-Reference Assessment. These experiments evaluated the models' ability to learn and eliminate real-world noise. We maintained the experimental settings (e.g., hyperparameters) as outlined in Sec.~\ref{Sec:full-reference} to ensure a fair comparison.


\subsection{Lesion Segmentation}
Another U-Net model was employed for the downstream lesion segmentation task. The network consisted of 4 layers, with a base channel size of 64. The channel multiplier was set to 1 in the final layer and 2 in the remaining layers. The model was trained for 300 epochs using the Adam optimizer, with a batch size of 8. The initial learning rate was set to $2 \times 10^{-4}$, and a cosine annealing scheduler was applied, gradually reducing the learning rate to a minimum value of $1 \times 10^{-6}$. 

We utilized extensive data augmentation strategies to enhance model robustness. These included random horizontal and vertical flips, each with a probability of 0.5; random rotations with a probability of 0.8; random grid shuffling over $8\times 8$ grids with a probability of 0.5; and CoarseDropout, which masked up to 12 patches of size 
$20 \times 20$ to a value of 0, also with a probability of 0.5.








\subsection{DR grading.} We trained an NN-MobileNet model~\cite{deeplearning1} for the DR grading task using real-world high-quality images. The enhanced test images are used with the trained NN-MobileNet to infer DR grading classification. Enhancement performance is evaluated based on classification accuracy (ACC), kappa score, F1 score, and AUC. This evaluation primarily aims to assess whether the denoising model disrupts lesion distribution, potentially leading to inconsistencies with the original DR grading labels. 
During the training, we conducted 200 epochs with a batch size of 32 and an input size of $256 \times 256$. The AdamP optimizer was utilized with a $1 \times 10^{-3}$ weight decay and an initial learning rate of $1 \times 10^{-3}$. A dropout rate of $0.2$ was applied during training to mitigate over-fitting. Furthermore, the learning rate was dynamically adjusted using the Cosine Learning Rate Scheduler.

\subsection{Representation Feature Evaluation.}We employed two foundation models for fundus images, RetFound~\cite{zhou2023foundation} and Ret-CLIP~\cite{du2024ret}, to calculate the Fréchet Inception Distance (FID) between enhanced and real-world high-quality image feature representations. These metrics are referred to as \textit{FID-RetFound} and \textit{FID-CLIP}, respectively.

\textit{FID-RetFound} evaluates the preservation of disease-related information, while \textit{FID-CLIP} assesses the similarity of spatial structures and continuous features. To compute these metrics, the enhanced and real-world high-quality images were resized to $224 \times 224$ and normalized before being passed into the respective image encoders. The FID scores were then calculated based on the 1024-dimensional and 512-dimensional feature maps produced by RetFound and Ret-CLIP, respectively.

\subsection{Experts Annotation Evaluation.} To evaluate the quality of the enhanced images, we recruited six trained specialists to perform the manual evaluation. The evaluation criteria, as illustrated in Fig.~\ref{fig:expert-protocol}, included lesion preserving, background preserving, and structure-preserving. Each image was individually checked, and the results were carefully recorded. The six well-trained annotators conducted cross-evaluations on test images enhanced by different models to minimize subjective bias. Ophthalmologists further validated the final results to ensure accuracy and reliability.

\section{Result Illustrations}
We provide additional visualizations for all baseline models in the Full-Reference and No-Reference Quality Assessment Experiments. Specifically, Fig.~\ref{fig:full-reference-eyeq} presents the results of the Denoising Evaluation conducted on the EyeQ dataset. In contrast, Fig.~\ref{fig:full-reference-generalization} illustrates the Denoising Generalization Evaluation results on the DRIVE~\cite{drive} and IDRID~\cite{idrid} datasets. Fig.~\ref{fig:full-reference-segmentation} displays the outcomes of Vessel and Lesion (EX and HE) Segmentation. The results of the No-Reference quality assessment Experiments are outlined in Fig.~\ref{fig:no-reference}.
\end{document}
