\section{Related Work}
Map matching problem can be classified to two broad categories~\cite{mmsurvey}: \textit{Offline Map Matching} and \textit{Online Map Matching}. For \textit{Offline Map Matching}, researchers focus on matching complete trajectories in offline scenarios, which aims for optimal matching route with less constraint on processing time~\cite{survey_magzine}. For \textit{Online Map Matching}, trajectory points are continuously sampled and processed in a streaming manner, which imposes high demands on real-time performance.
\subsection{Offline Map Matching}
Early research on the offline map matching problem was primarily based on the concept of similarity comparison, that is, by defining similarity to identify the road segments most similar to the target trajectory. The authors in~\cite{currentmm, icde12, trajSeg} design algorithms with different measure of similarity (e.g., spatial distance, longest common subsequence) to find the most similar road segment as the matching result. 

% In scenarios where the complexity of maps and trajectories is high, the performance of these methods that match based on similarity is not satisfactory. To further improve the performance, the Hidden Markov Model (HMM) and other HMM-based models have been adopted. The Hidden Markov Model (HMM) assumes that the current state is determined by the previous one, and considers the correct road segments as the hidden states of the trajectory points, which are the observations. 
When maps and trajectories are complex, similarity-based methods often perform poorly. To enhance performance, Hidden Markov Models (HMM) and HMM-enhanced models are used. HMMs assume that the current state depends on the previous one and consider the correct road segments as hidden states of the trajectory points, which are the observations.
Newson et al.~\cite{HMM} are the first to use HMM to address the offline map matching problem. 
Yang et al.~\cite{FMM} integrate HMM with precomputation to achieve high processing speed. 
The authors in~\cite{MDPMM} use MDP to address the limitations of HMM, designing dynamic outlier resolution and improved preprocessing functions for map matching.
Moreover, considering both the spatial structures and the temporal constraints of the trajectories, \cite{stmatching, impstmatching, ifmatching} design methods that characteristic relationships between GPS points.
However, although HMM is widely used in commercial software, its performance in low-sampling-rate scenarios is poor, limiting its potential for cost reduction and efficiency improvement~\cite{praGuide}.

% However, these HMM-based methods are rule-based, which prevents them from fully leveraging the inherent spatio-temporal correlations of trajectories and road networks from extensive datasets. Furthermore, they are inherently greedy for each match, overlooking the impact of current matches on future outcomes, which results in poor robustness and insufficient accuracy.
Recently, deep-learning-based map matching methods have attracted the attention of researchers. 
In order to learn general patterns from trajectory data, \cite{deepmmzhao, deepmmfeng} build deep learning based models to utilize all the trajectory data for joint training and knowledge sharing. 
% Furthermore, based on the attention mechanism, Feng et al.~\cite{deepmmfeng} build a trajectory2road model to map the sparse and noisy trajectory into the accurate road network with data augmentation. 
% To enhance the trajectory data and support the urban applications more effectively, 
The authors in~\cite{mtrajrec} propose MTrajRec to recover the fine-grained points in trajectories and map match them on the road network in an end-to-end manner. 
Jiang et al.~\cite{l2mm} propose L2MM to generate high-quality representations of low-quality trajectories through high-frequency trajectory enhancement. 
Liu et al.~\cite{graphmm} present a graph-based approach that incorporates conditional models to leverage various road and trajectory correlations. 
As for cellular trajectory map matching, \cite{LHMM, FL-AMM, DMM2} design different learning methods (e.g., multi-relational graph learning, federated learning, heuristic reinforcement learning optimizer) to learn the mapping function from cellular trajectory points to road segments. As remote sensing imagery is innovatively applied to various spatiotemporal data tasks~\cite{POIjn,urbancmx}, it is conceivable that the map matching task could similarly leverage diverse types of novel data in the future.
% Shen et al.~\cite{DMM2} adopt a recurrent neural network to identify the most-likely trajectory of roads given a sequence of cell towers and employ a reinforcement learning-based model to optimize the matched outputs. 
% However, although the above methods can extract effective information from complete trajectories for end-to-end matching, applying them to online map matching scenarios leads to inefficiency due to computational redundancy caused by multiple matches.


\subsection{Online Map Matching}
For online map matching, research primarily focuses on the Hidden Markov Model (HMM) or enhanced versions of HMM, focusing on improving efficiency.

% Furthermore, \cite{ohmm, onlinelearning, routepre} propose methods which use various mathematical algorithms to improve efficiency, and combine the spatial, temporal and topological information to support the matching process.
Considering noise and sparseness of GPS data, Goh et al.~\cite{ohmm} propose an enhanced HMM with a Variable Sliding Window method to integrate spatial, temporal, and topological information to map the GPS trajectories to the road network in real time. 
To further address complex and changeable city traffic conditions, Liang et al.~\cite{onlinelearning} design OLMM, which uses online learning to improve accuracy without requiring any prior human labeling.
\cite{fu2021online} proposes an online map matching algorithm based on a second-order HMM with an extended Viterbi algorithm and a self-adaptive sliding window mechanism to enhance trajectory data processing accuracy in complex urban road networks.
To cope with high noise, Jagadeesh et al.~\cite{2017TITSonline} design a HMM to generate partial map-matched paths in an online manner, and use a route choice model to reassess each HMM-generated partial path along with a set of feasible alternative paths.
Furthermore, the authors in~\cite{routepre} propose an algorithm that replaces future GPS points with a probabilistic route prediction model.
Recently, Hu et al.~\cite{AMM} design an adaptive online map matching algorithm called AMM, which calibrate GPS observation data for various measurement errors and under complex urban conditions. 
Additionally, in addressing the issue of cellular trajectory map matching, which involves handling trajectories based on cellular positioning data, \cite{AccReal} design an incremental HMM algorithm that combines digital map hints and a number of heuristics to reduce the noise and provide real-time estimations. 

However, aforementioned online methods are limited by their rule-based nature, which prevents them from fully capturing the spatio-temporal correlations~\cite{hmmAna}. Additionally, these methods are predominantly greedy in their matching process, ignoring the impact of current matches on future outcomes, leading to poor robustness and accuracy~\cite{mmsurvey, surveyVMMT}.