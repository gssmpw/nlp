\section{Preliminaries}
\label{sec:2}
\begin{figure}
  \centering  
  \includegraphics[width=0.7\linewidth]{figures/pre.pdf}
  \vspace{-0.15in}
  \caption{An illustration of Grids Partition, Trajectory Transition Graph, and Link Connection Graph.}
  \vspace{-0.25in}
  \label{fig:pre}
\end{figure}

% We first introduce some basic concepts and then formalize the online map matching problem.

We first introduce the basic concepts of the map matching problem, namely trajectories and road networks, followed by a detailed explanation of the specifically designed data structures for this task. Finally, we formalize the online map matching problem.

\subsection{Basic Concepts}
% \noindent \textbf{Trajectory.} A trajectory $\mathcal{T}$ is a sequence of GPS points $\mathcal{T}=\langle p_1, \ldots, p_n \rangle$, where each point $p_i$ is represented by its GPS location latitude, longitude, and timestamp. As shown in the left part of Fig.~\ref{fig:pre}, we utilize the widely adopted grid representation method~\cite{graphmm, mtrajrec, deeptrajrec} to split the city into $N_g = H \times W$ grids with side length $l_g$. This process results in each trajectory point $p$ falling into a specific grid denoted $g_{ij} = Grid(p)$, where $g_{ij}$ is the grid indexed by $(i,j)$ and $Grid(\cdot)$ can be considered as the function that maps trajectory points to grids. Therefore, a trajectory $\mathcal{T}$ can be represented by a sequence of grids with timestamps $\mathcal{T}=\langle g_1, \ldots, g_n \rangle$.

\noindent \textbf{Trajectory.} A trajectory $\mathcal{T}$ is a sequence of GPS points, denoted as $\mathcal{T} = \langle p_1, p_2, \ldots, p_n \rangle$, where each point $p_i$ represents a location record captured at a specific time. Each point $p_i$ consists of three components: the geographic coordinates latitude ($\text{lat}_i$) and longitude ($\text{lon}_i$), and a timestamp ($\text{time}_i$).

\noindent \textbf{Trajectory Transition Graph.} As shown in the left part of Fig.~\ref{fig:pre}, we utilize the widely adopted grid representation method~\cite{graphmm, mtrajrec, deeptrajrec} to split the city into $N_g = H \times W$ grids with side length $l_g$. This process results in each trajectory point $p$ falling into a specific grid denoted $g_{ij} = Grid(p)$, where $g_{ij}$ is the grid indexed by $(i,j)$ and $Grid(\cdot)$ can be considered as the function that maps trajectory points to grids. Therefore, a trajectory $\mathcal{T}$ can be represented by a sequence of grids with timestamps $\mathcal{T}=\langle g_1, \ldots, g_n \rangle$. By doing so, the relationships between different trajectories can be effectively captured from a global perspective. Specifically, we construct a directed trajectory transition graph $G_T=(V_T,E_T)$, where $V_T$ is the node set of grids and $E_T$ is the edge set representing trajectory transition. Specifically, as shown in the upper right part of Fig.~\ref{fig:pre}, for any trajectory $\mathcal{T}$ in trajectory set, if there is a trajectory point transition from $p_i \in \mathcal{T}$ to $p_{i+1} \in \mathcal{T}$ ($p_i$, $p_{i+1}$ are consecutive and $Grid(p_i)$, $Grid(p_{i+1})$ are not identical), then there is an edge form $Grid(p_i)$ to $Grid(p_{i+1})$. Moreover, the weight assigned to each edge reflects the count of such trajectory transitions observed across the trajectory set.

\noindent \textbf{Road Network.} A road network is a graph where the nodes represent locations, such as intersections, and the edges represent road segments connecting these locations.

\noindent \textbf{Link Connection Graph.} As shown in the lower right part of Fig.~\ref{fig:pre}, a directed link connection graph $G_R=(V_R,E_R)$ is used to model the road network, where each $u_i \in V_R$ is a road segment, and $e_{ij} = (u_i,u_j)$ represents road segment $u_i$ and $u_j$ are connected. To maintain consistency with the grid representation of trajectories, we choose the grid coordinates of the start and end points of road segments, along with their latitude and longitude as node features.

\subsection{Problem Formulation}
% \noindent \textbf{Online Map Matching.} Given the link connection graph $G_R$, the number of matching step $k$, a historical prefix trajectory $\mathcal{T}_{\vartriangleleft i} = \langle p_1, \ldots, p_{i-1} \rangle$, and the matched road $U_{\vartriangleleft i} = \langle u_1, \ldots, u_{i-1} \rangle$ of the historical prefix trajectory, the goal is to match $k$ road segments $U_{i:i+k} = \langle u_i, \ldots, u_{i+k-1} \rangle$ from their candidate sets $C_{i:i+k}$ for $k$ new trajectory points $\mathcal{T}_{i:i+k}= \langle p_i, \ldots, p_{i+k-1} \rangle$ at every $k$ time steps, where the candidate sets are pre-computed based on spatial distance. Note that this process is incremental and requires multiple times to complete the matching of the complete trajectory $\mathcal{T}$.

\noindent In the online scenario, we need to match newly arriving trajectory points at regular intervals. For each matching, we have the link connection graph $G_R$ which represents road network, the $k$ new incoming trajectory points $\mathcal{T}_{i:i+k}= \langle p_i, \ldots, p_{i+k-1} \rangle$ to match, the historical prefix trajectory $\mathcal{T}_{\vartriangleleft i} = \langle p_1, \ldots, p_{i-1} \rangle$ that is already available for previous matching, and the matched road $U_{\vartriangleleft i} = \langle u_1, \ldots, u_{i-1} \rangle$ of the historical prefix trajectory. Formally, we define the Online Map Matching problem and Online Markov Decision Process as follows.

\textit{\underline{Definition 1 (Online Map Matching).}} Given $G_R$, $\mathcal{T}_{i:i+k}$, $\mathcal{T}_{\vartriangleleft i}$ and $U_{\vartriangleleft i}$, the goal is to match $k$ road segments $U_{i:i+k} = \langle u_i, \ldots, u_{i+k-1} \rangle$ from their candidate sets $C_{i:i+k}$. The candidate sets are pre-computed based on spatial distance, which is a common practice in related works~\cite{HMM,AMM, mtrajrec}. The matching process is incremental and requires multiple steps to complete the matching of the entire trajectory $\mathcal{T}$.

\textit{\underline{Definition 2 (Online Markov Decision Process).}} The Markov Decision Process (MDP) provides a powerful framework for representing problems as a sequence of states and transitions, offering both effectiveness and efficiency. Inspired by this, we innovatively model the online map matching problem as an online MDP, meticulously tailored to capture the core aspects of the problem. Specifically, it is characterized by \textit{State}, \textit{Action}, \textit{Transition}, and \textit{Reward}.
\begin{itemize}[leftmargin=10.2pt]
\setlength{\itemsep}{0pt}
\setlength{\parsep}{0pt}
\setlength{\parskip}{0pt}
\item \textit{State}: The state $s_i$ of time step $i$ represents the real-time information of online map matching scenario (i.e., \textit{previously matched road segments} $U_i$, \textit{current trajectory points} $\mathcal{T}_i$, and \textit{candidate road segments} $C_i$), and the historical information of previous matches (i.e., \textit{historical information} of road $H_i^r$ and trajectory $H_i^t$). 

\item \textit{Action}: Action $a_i$ represents the selection of a road segment from the candidate road segments $C_i$. 
\item \textit{Transition}: In our modeling, the selected candidate road segments influence the update of historical information, which in turn affects the next state. Meanwhile, the previously matched road segments within the state are determined by the selected candidate road segments. This can be considered as state transitions in the context of map matching.
\item \textit{Reward}: Reward $r = R(s_i, a_i)$ denotes the reward received after taking action $a_i$ under state $s_i$ with reward function $R$, which is associated with the accuracy of the action.
\end{itemize}

By defining these components, the online map matching problem can be effectively modeled as an online MDP. In particular, previously defined $\mathcal{T}_{\vartriangleleft i}$, $U_{\vartriangleleft i}$, $\mathcal{T}_{i:i+k}$ and $C_{i:i+k}$ are effectively incorporate into \textit{State}, the \textit{Action} determines how to select $U_{i:i+k}$, the \textit{Transition} indicates the evolution of information for $\mathcal{T}_{\vartriangleleft i}$ and $U_{\vartriangleleft i}$ within \textit{State}, and the \textit{Reward} is the assessment of the matched results $U_{i:i+k}$. This approach enables the use of reinforcement learning algorithms to optimize decision-making for efficient and robust map matching in dynamic environments.



