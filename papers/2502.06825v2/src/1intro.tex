\section{Introduction}
Since the widespread availability of GPS services in the 1990s, research related to map matching has been a focal point. Map matching is the process of aligning vehicle trajectories onto the actual road network, which is a fundamental problem in location-based services~\cite{survey_magzine, stsurvey}. 
Online map matching is a sequential process that aligns incoming data incrementally, making it particularly well-suited for real-time applications. This process is a crucial element for various location-based online services and is vital for applications requiring real-time responses such as navigation services~\cite{navigation2}, optimal route planning~\cite{Learning2Route, AnySto}, and travel time estimation~\cite{TTE, TTE2, TTE3}.

% \textcolor{blue}{However, modeling map matching with deterministic rules fails to cope with the high dynamism of the problem, leading to poor performance.}
The map matching method can be categorized into two main types: \textit{rule-based}~\cite{HMM,FMM,MDPMM} and \textit{deep-learning-based}~\cite{deepmmzhao, deepmmfeng, mtrajrec, l2mm, graphmm}. The former treats trajectories as observations and utilizes the Hidden Markov Model (HMM) to infer road segments as hidden states~\cite{HMMbase} or uses Markov Decision Process (MDP) with dynamic programming algorithms like value iteration to model offline matching process~\cite{MDPMM}. With advancements in deep learning, attention has shifted towards the latter approach, framing the problem as an end-to-end task. This method leverages large datasets to directly learn patterns for matching trajectories to road segments. Overall, both types of methods perform well in terms of matching accuracy in high-sampling-rate offline scenarios. However, when the problem shifts to the online scenario, they exhibit significant inefficiencies due to their inherent nature~\cite{praGuide, hmmAna, mmsurvey,surveyVMMT}.

Specifically, for online map matching scenarios, existing solutions are essentially straightforward extended forms of offline methods. For example, \cite{HMM} points out that Microsoft treats the map matching problem as a batch processing task, performing matching after all data has been collected. For online scenarios, they simply use a sliding window as an adaptation of the offline method. To offer a clear perspective, we summarize the two existing extension strategies, as shown in Fig.~\ref{fig:intro}. The first strategy involves repeatedly invoking the DNN or MDP-based offline methods~\cite{mtrajrec,l2mm, graphmm,MDPMM} during online matching process to match the current available trajectory at that moment. The second strategy employs HMM, incrementally matching incoming trajectories in a rule-based manner~\cite{onlinelearning, AMM}. However, current service providers still incur substantial costs because these two categories have the following notable deficiencies in addressing the core challenges of online map matching:

\begin{figure}
  \centering  
  \includegraphics[width=0.7\linewidth]{figures/intro.pdf}
  \vspace{-0.1in}
  \caption{Concept comparison of existing online map matching solutions and our \modelName.}
  \vspace{-0.25in}
  \label{fig:intro}
\end{figure}

\iffalse
For online map matching, current feasible solutions can be categorized into two types: \textit{online adaptations of learning-based methods} and \textit{HMM-based online methods} and. 
The Hidden Markov Model (HMM) is a traditional statistical model that models state transition processes by defining hidden states and observations~\cite{HMMbase}. Recent online map matching methods treat trajectories as observations, aiming to infer the corresponding road segments as the hidden states, designing a series of methods based on HMM to enhance its performance~\cite{onlinelearning, 2017TITSonline, genetic, routepre,AMM}. With the development of deep learning, the research focus has shifted toward deep learning methods for map matching. They model the problem as an end-to-end matching task, aiming to learn the general patterns of the problem from large datasets~\cite{deepmmzhao, deepmmfeng, mtrajrec, l2mm, graphmm}. By repeatedly performing matching on incoming trajectory data, their solutions can be adapted to online scenarios. 
However, despite their potential, both types face significant challenges in achieving the robustness and efficiency required for large-scale applications, indicating substantial gaps that remain unaddressed.
\fi
% However, the efficiency issues introduced by this adaptation are unacceptable for online application scenarios. Worse still, HMM-based online map matching solutions approaches cannot extract general patterns from large datasets, resulting in low adaptability and poor robustness. 

% Additionally, the inherent unpredictability of such environments demands a system that can maintain reliability in its matching results despite the constant flux.
% \noindent (1) \textbf{Low efficiency.} Most existing methods  do not take into account the incremental nature of online map matching, thus necessitating multiple executions of matching originally designed for a whole trajectory. This leads to unnecessary redundant computations in online scenarios, resulting in unacceptable inefficiencies. In light of this issue, a novel solution that is both aligned with the essence of online map matching and efficient in large-scale data scenarios is required.

\noindent (1) \textbf{Low efficiency.} Most current methods, particularly \textcolor{black}{DNN-based or MDP-based}, are tailored to align complete trajectories with road networks. However, when applied incrementally, these methods necessitate the input of all preceding trajectory segments to utilize historical data for enhancing effectiveness. This process results in redundant computations and leads to significant inefficiencies. In light of this issue, it is crucial to develop an innovative solution that not only maintains the accuracy of map matching but also enhances efficiency in real-time scenarios, thereby reducing costs for service providers.

\noindent (2) \textbf{Poor robustness.} Online map matching necessitates the continuous processing of data within a streaming pipeline, emphasizing the significant influence of previous matches on subsequent ones. Earlier matches create an informational context that shapes later decisions. However, existing online methods predominantly utilize a greedy approach, neglecting the essential impact of current matches on future results, thus leading to suboptimal matches. Consequently, addressing this oversight is crucial for advancing the development of a robust online map matching framework.

% Additionally, rule-based Hidden Markov Models (HMM) inherently depend on the linear processing of sequential data, and their structural limitations result in low efficiency in large-scale data scenarios. In light of these issues, a novel solution that is both aligned with the essence of online map matching and efficient in large-scale data scenarios is required.

% A desired model should learn from diverse scenarios and adjust on the fly, optimizing decision-making through ongoing updates and refinements, ensuring precision and dynamic evolution throughout the map matching process.
% Moreover, HMM-based online map matching methods require pre-computed parameters for the current scenario due to the nature of HMMs. Meanwhile, they can not fully leverage the inherent spatio-temporal patterns of trajectories and road networks from large datasets. This results in inadequate robustness and makes them unsuitable for deployment in large-scale online applications across multiple diverse scenarios. An industry-satisfactory solution should balance both high efficiency and high robustness, which is essential for providing high-quality services.
% Existing methods overlook this aspect since they lack a clear modeling of historical and real-time information integration.
% \noindent (2) \textbf{Poor robustness.} Online map matching demands continuous data processing and adaptability in a dynamic road network, requiring a robust response to real-time data across various situations. Existing methods often neglect the necessary dynamic adaptability and the integration of historical data for current decision-making. Moreover, while current online methods prioritize immediate optimality, they fail to consider the influence of past matches on future decisions, which is crucial yet unaddressed. 

\iffalse
Online map matching requires the continuous processing and precise alignment of incoming trajectory data with a dynamically evolving road network. This scenario demands a robust response to real-time data across various situations, emphasizing the need for dynamic adaptability. Consequently, the system must make decisions informed by historical data while adjusting to current environmental conditions. However, existing approaches overlook this critical aspect.
Additionally, existing online methods are inherently based on a greedy approach, meaning they always select the currently optimal match. However, the impact of prior matches on subsequent matches is significant, as the results of earlier matches serve as the informational backdrop for later ones. Therefore, considering the impact of current matches on future ones is essential, yet it has not been addressed.
\fi


\iffalse
\noindent (3) \textbf{Insufficient capture of correlations and heterogeneity.} In map matching, complex multi-dimensional correlations exist between trajectories and the road network. These correlations extend beyond the straightforward correlation between individual trajectories and road segments, encompassing intricate correlations among different trajectories and among different road segments. Successfully capturing and representing these correlations is a non-trivial task that demands advanced data representation strategies and effective feature encoding techniques. An ideal solution must effectively capture the multi-dimensional correlations among trajectories, roads, and trajectory-road relationships.

\noindent (3) \textbf{Trajectory-Road Heterogeneity.} Trajectory and road data are inherently heterogeneous. To achieve accurate matching, it is essential to address this heterogeneity. This concept emphasizes the effective integration of trajectory data and road network data while preserving their unique attributes and context. Specifically, we need to bridge the distinct representations of trajectory and road data, fostering meaningful interactions between these different types of information. Such integration demands techniques that can seamlessly meld these diverse data forms, ensuring they can jointly facilitate an effective matching process. 
\fi

% This requires advanced data representation strategies and effective feature encoding techniques that can bridge the distinct representations of trajectory and road data, fostering meaningful interactions and seamlessly melding these forms to enhance the matching process.
\noindent (3) \textbf{Insufficient handling of trajectory-road heterogeneity.} \textcolor{black}{Intuitively, trajectory data is an unstructured sequence with significant noise, whereas road network data is a relatively fixed, structured topological graph.} This means trajectory and road data are inherently heterogeneous, which demands careful address. Correlations exist both among different trajectories and between various road segments, all of which should be considered. However, existing methods lack specialized designs that effectively capture these complex correlations within and between data types, leading to suboptimal data representations and encoding techniques. We argue that an ideal solution should facilitate the integration of trajectory and road data while preserving their unique attributes and contexts. 

To comprehensively address the aforementioned limitations of existing methods, \textcolor{black}{as shown in Fig.~\ref{fig:intro}}, we introduce \textbf{\modelName}, an efficient and robust online map matching framework with high dynamic adaptability, which is a novel paradigm to address the online map matching problem.
\textbf{First, we meticulously design an Online Markov Decision Process (OMDP) specifically for online map matching scenario,} continuously extracting real-time and historical information from real-time environment to construct the information state, closely integrating with the incremental and streaming characteristics of online map matching. 
Compared to the MDP method previously used for offline map matching~\cite{MDPMM}, this novel modeling approach effectively captures the essential data and dynamically updates the historical information throughout the matching process, which fundamentally simplifies the complexity of the problem and prevents redundant calculations, thereby enhancing efficiency. Additionally, the efficient component that captures sequence correlations effectively coordinates with this modeling approach, enhancing the integration efficiency of historical and real-time information.

\textbf{Second, to endow the model with sufficient dynamic adaptability and robustness for various scenarios,} we employ reinforcement learning to execute matching actions based on the states divided by the OMDP and meticulously design the reward evaluation process. Specifically, we employ the deep Q-learning method, which allows the model to consider the impact of each match on future matches. This enables the model to perform matching from a future-oriented perspective, avoiding current optimal matches that are prone to errors, thereby discarding the simplistic greedy approach to achieve ideal overall matching accuracy. Meanwhile, a carefully designed training process enables the model to grasp the dynamic changes of the matching environment and allows for continuous updates and improvements in the decision-making process based on feedback from the rewards.

\textbf{Third, considering the trajectory-road heterogeneity,} we design distinct graph data structures for each type of data to capture their intrinsic connections, namely trajectory transition graph and link connection graph. By representing both types of data in effective graph structures, their heterogeneity can be mitigated, thus promoting better information integration. Additionally, we design mechanism to bridge the interaction between their representation to facilitate better fusion. Moreover, corresponding graph and recurrent neural networks are utilized to generate effective representations for both trajectories and roads. Recognizing the necessity for effective alignment and integration of trajectory data with road data in map matching task, we design a trajectory-road representation alignment module from the perspective of enhancing representational robustness. This module reduces the distance between the representations of the two data types in the latent space, thereby facilitating effective integration.

In summary, we make the following contributions:
\vspace{-\topsep}
\begin{itemize}[leftmargin=10.2pt]
\setlength{\itemsep}{0pt}
\setlength{\parsep}{0pt}
\setlength{\parskip}{0pt}
\item To the best of our knowledge, we are the first to model the online map matching problem as an Online Markov Decision Process, proposing a new solution paradigm. Additionally, we propose \textbf{\modelName}, a novel two-stage system framework based on deep reinforcement learning, which delivers efficient map matching with high dynamic adaptability and robustness. (Sec.~\ref{sec:2}\&\ref{sec:3})
\item We design graph structures meticulously tailored for both trajectory and road to capture the correlations inherent to each type of data and mitigate the heterogeneity. Our feature encoding modules utilize various representation learning and sequence correlation capture techniques to integrate trajectory and road data, facilitating the map matching process. (Sec.~\ref{sec:4})
\item We propose a novel model learning process for our \textbf{\modelName} framework with carefully designed reward evaluation, which can perform globally optimal matching from a future-oriented perspective. We introduce a trajectory-road representation alignment module that utilizes contrastive learning to facilitate the effective integration of trajectories and roads. Combined with the temporal-difference loss of reinforcement, our framework is capable of robust learning from a variety of scenarios. (Sec.~\ref{sec:5})
\item We conduct a comprehensive evaluation on three real-world datasets. The results show that our method significantly outperforms state-of-the-art in terms of accuracy, efficiency and robustness. (Sec.~\ref{sec:6})
\end{itemize}