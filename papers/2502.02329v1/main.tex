%%
%% This is file `sample-sigconf-authordraft.tex',
%% generated with the docstrip utility.
%%
%% The original source files were:
%%
%% samples.dtx  (with options: `all,proceedings,bibtex,authordraft')
%% 
%% IMPORTANT NOTICE:
%% 
%% For the copyright see the source file.
%% 
%% Any modified versions of this file must be renamed
%% with new filenames distinct from sample-sigconf-authordraft.tex.
%% 
%% For distribution of the original source see the terms
%% for copying and modification in the file samples.dtx.
%% 
%% This generated file may be distributed as long as the
%% original source files, as listed above, are part of the
%% same distribution. (The sources need not necessarily be
%% in the same archive or directory.)
%%
%%
%% Commands for TeXCount
%TC:macro \cite [option:text,text]
%TC:macro \citep [option:text,text]
%TC:macro \citet [option:text,text]
%TC:envir table 0 1
%TC:envir table* 0 1
%TC:envir tabular [ignore] word
%TC:envir displaymath 0 word
%TC:envir math 0 word
%TC:envir comment 0 0
%%
%%
%% The first command in your LaTeX source must be the \documentclass
%% command.
%%
%% For submission and review of your manuscript please change the
%% command to \documentclass[manuscript, screen, review]{acmart}.
%%
%% When submitting camera ready or to TAPS, please change the command
%% to \documentclass[sigconf]{acmart} or whichever template is required
%% for your publication.
%%
%%
% \documentclass[manuscript, screen, review]{acmart}
% \documentclass[manuscript,review,anonymous]{acmart}
\documentclass[manuscript, screen]{acmart}

\usepackage{enumitem}
\usepackage{amsmath}
\usepackage{tabularx}
\usepackage{color}
\usepackage{tabu}                      % only used for the table example
\usepackage{booktabs}                  % only used for the table example
\usepackage{lipsum}                    % used to generate placeholder text
\usepackage{mwe}                       % used to generate placeholder figures
\usepackage[normalem]{ulem}

%%
%% \BibTeX command to typeset BibTeX logo in the docs
\AtBeginDocument{%
  \providecommand\BibTeX{{%
    Bib\TeX}}}

%% Rights management information.  This information is sent to you
%% when you complete the rights form.  These commands have SAMPLE
%% values in them; it is your responsibility as an author to replace
%% the commands and values with those provided to you when you
%% complete the rights form.
\setcopyright{acmlicensed}
% \copyrightyear{2018}
% \acmYear{2018}
\acmDOI{XXXXXXX.XXXXXXX}

%% These commands are for a PROCEEDINGS abstract or paper.
% \acmConference[Conference acronym 'XX]{Make sure to enter the correct
%   conference title from your rights confirmation emai}{June 03--05,
%   2018}{Woodstock, NY}
%%
%%  Uncomment \acmBooktitle if the title of the proceedings is different
%%  from ``Proceedings of ...''!
%%
%%\acmBooktitle{Woodstock '18: ACM Symposium on Neural Gaze Detection,
%%  June 03--05, 2018, Woodstock, NY}
\acmISBN{978-1-4503-XXXX-X/18/06}


%%
%% Submission ID.
%% Use this when submitting an article to a sponsored event. You'll
%% receive a unique submission ID from the organizers
%% of the event, and this ID should be used as the parameter to this command.
%%\acmSubmissionID{123-A56-BU3}

%%
%% For managing citations, it is recommended to use bibliography
%% files in BibTeX format.
%%
%% You can then either use BibTeX with the ACM-Reference-Format style,
%% or BibLaTeX with the acmnumeric or acmauthoryear sytles, that include
%% support for advanced citation of software artefact from the
%% biblatex-software package, also separately available on CTAN.
%%
%% Look at the sample-*-biblatex.tex files for templates showcasing
%% the biblatex styles.
%%

%%
%% The majority of ACM publications use numbered citations and
%% references.  The command \citestyle{authoryear} switches to the
%% "author year" style.
%%
%% If you are preparing content for an event
%% sponsored by ACM SIGGRAPH, you must use the "author year" style of
%% citations and references.
%% Uncommenting
%% the next command will enable that style.
%%\citestyle{acmauthoryear}
\newcommand{\system}{ReSpark}

%%
%% end of the preamble, start of the body of the document source.
\begin{document}

%%
%% The "title" command has an optional parameter,
%% allowing the author to define a "short title" to be used in page headers.
% \title{\system{}: An Example-based Method to Automatically Generate Data Reports}
% \title{\system{}: A Retrieve-then-Adapt Method to Generate Data Reports with Large Language Models}
\title{\system{}: Leveraging Previous Data Reports as References to Generate New Reports with LLMs}

%%
%% The "author" command and its associated commands are used to define
%% the authors and their affiliations.
%% Of note is the shared affiliation of the first two authors, and the
%% "authornote" and "authornotemark" commands
%% used to denote shared contribution to the research.
% Yuan Tian, Chuhan Zhang, Xiaotong Wang, Sitong Pan, Weiwei Cui, Haidong Zhang, Dazhen Deng and Yingcai Wu
\author{Yuan Tian}
\email{yuantian@zju.edu.cn}
\affiliation{%
  \institution{Zhejiang University}
  \state{Zhejiang}
  \country{China}
  \postcode{310000}
}

\author{Chuhan Zhang}
\email{chuhanzhang@zju.edu.cn}
\affiliation{%
  \institution{Zhejiang University}
  \state{Zhejiang}
  \country{China}
  \postcode{310000}
}

\author{Xiaotong Wang}
\email{xtitx327@gmail.com}
\affiliation{%
  \institution{Zhejiang University}
  \state{Zhejiang}
  \country{China}
  \postcode{310000}
}


\author{Sitong Pan}
\email{pstong@zju.edu.cn}
\affiliation{%
  \institution{Zhejiang University}
  \state{Zhejiang}
  \country{China}
  \postcode{310000}
}


\author{Weiwei Cui}
\email{weiwei.cui@microsoft.com}
\affiliation{%
  \institution{Microsoft}
  \state{Beijing}
  \country{China}
  \postcode{310000}
}


\author{Haidong Zhang}
\email{haizhang@microsoft.com}
\affiliation{%
  \institution{Microsoft}
  \state{Beijing}
  \country{China}
  \postcode{310000}
}

\author{Dazhen Deng}
\email{dengdazhen@outlook.com}
\affiliation{%
  \institution{Zhejiang University}
  \state{Zhejiang}
  \country{China}
  \postcode{310000}
}

\author{Yingcai Wu}
\email{ycwu@zju.edu.cn}
\affiliation{%
  \institution{Zhejiang University}
  \state{Zhejiang}
  \country{China}
  \postcode{310000}
}




%%
%% By default, the full list of authors will be used in the page
%% headers. Often, this list is too long, and will overlap
%% other information printed in the page headers. This command allows
%% the author to define a more concise list
%% of authors' names for this purpose.
\renewcommand{\shortauthors}{Tian et al.}

%%
%% The abstract is a short summary of the work to be presented in the
%% article.
\begin{abstract}
\begin{abstract}


The choice of representation for geographic location significantly impacts the accuracy of models for a broad range of geospatial tasks, including fine-grained species classification, population density estimation, and biome classification. Recent works like SatCLIP and GeoCLIP learn such representations by contrastively aligning geolocation with co-located images. While these methods work exceptionally well, in this paper, we posit that the current training strategies fail to fully capture the important visual features. We provide an information theoretic perspective on why the resulting embeddings from these methods discard crucial visual information that is important for many downstream tasks. To solve this problem, we propose a novel retrieval-augmented strategy called RANGE. We build our method on the intuition that the visual features of a location can be estimated by combining the visual features from multiple similar-looking locations. We evaluate our method across a wide variety of tasks. Our results show that RANGE outperforms the existing state-of-the-art models with significant margins in most tasks. We show gains of up to 13.1\% on classification tasks and 0.145 $R^2$ on regression tasks. All our code and models will be made available at: \href{https://github.com/mvrl/RANGE}{https://github.com/mvrl/RANGE}.

\end{abstract}


\end{abstract}

%%
%% The code below is generated by the tool at http://dl.acm.org/ccs.cfm.
%% Please copy and paste the code instead of the example below.
%%
\begin{CCSXML}
<ccs2012>
   <concept>
       <concept_id>10003120.10003145.10003151.10011771</concept_id>
       <concept_desc>Human-centered computing~Visualization toolkits</concept_desc>
       <concept_significance>500</concept_significance>
       </concept>
   <concept>
       <concept_id>10003120.10003121.10003129</concept_id>
       <concept_desc>Human-centered computing~Interactive systems and tools</concept_desc>
       <concept_significance>500</concept_significance>
       </concept>
 </ccs2012>
\end{CCSXML}

\ccsdesc[500]{Human-centered computing~Visualization toolkits}
\ccsdesc[500]{Human-centered computing~Interactive systems and tools}

%%
%% Keywords. The author(s) should pick words that accurately describe
%% the work being presented. Separate the keywords with commas.
\keywords{Visual Storytelling, Visualization Generation, Large Language Models}
%% A "teaser" image appears between the author and affiliation
%% information and the body of the document, and typically spans the
%% page.
\begin{teaserfigure}
  \includegraphics[width=\linewidth]{figs/teaser.png}
  \caption{
    ReSpark overview. 
    ReSpark first preprocesses the dataset (b1), segments the reports in the repository (b2) and then retrieves similar thematic reports based on the similarity. 
    To reuse reports for new data, ReSpark extracts the analysis objectives from the reference report (c) and aligns these objectives with the new data (d1). Based on the aligned objectives, it generates code to transform the data and produce the corresponding charts (d2). Finally, ReSpark generates textual descriptions (d3), summarizing data insights and creating a new report (d).
  }
  \label{fig:teaser}
\end{teaserfigure}

\received{20 February 2007}
\received[revised]{12 March 2009}
\received[accepted]{5 June 2009}

%%
%% This command processes the author and affiliation and title
%% information and builds the first part of the formatted document.
\maketitle

\section{Introduction}

Video generation has garnered significant attention owing to its transformative potential across a wide range of applications, such media content creation~\citep{polyak2024movie}, advertising~\citep{zhang2024virbo,bacher2021advert}, video games~\citep{yang2024playable,valevski2024diffusion, oasis2024}, and world model simulators~\citep{ha2018world, videoworldsimulators2024, agarwal2025cosmos}. Benefiting from advanced generative algorithms~\citep{goodfellow2014generative, ho2020denoising, liu2023flow, lipman2023flow}, scalable model architectures~\citep{vaswani2017attention, peebles2023scalable}, vast amounts of internet-sourced data~\citep{chen2024panda, nan2024openvid, ju2024miradata}, and ongoing expansion of computing capabilities~\citep{nvidia2022h100, nvidia2023dgxgh200, nvidia2024h200nvl}, remarkable advancements have been achieved in the field of video generation~\citep{ho2022video, ho2022imagen, singer2023makeavideo, blattmann2023align, videoworldsimulators2024, kuaishou2024klingai, yang2024cogvideox, jin2024pyramidal, polyak2024movie, kong2024hunyuanvideo, ji2024prompt}.


In this work, we present \textbf{\ours}, a family of rectified flow~\citep{lipman2023flow, liu2023flow} transformer models designed for joint image and video generation, establishing a pathway toward industry-grade performance. This report centers on four key components: data curation, model architecture design, flow formulation, and training infrastructure optimization—each rigorously refined to meet the demands of high-quality, large-scale video generation.


\begin{figure}[ht]
    \centering
    \begin{subfigure}[b]{0.82\linewidth}
        \centering
        \includegraphics[width=\linewidth]{figures/t2i_1024.pdf}
        \caption{Text-to-Image Samples}\label{fig:main-demo-t2i}
    \end{subfigure}
    \vfill
    \begin{subfigure}[b]{0.82\linewidth}
        \centering
        \includegraphics[width=\linewidth]{figures/t2v_samples.pdf}
        \caption{Text-to-Video Samples}\label{fig:main-demo-t2v}
    \end{subfigure}
\caption{\textbf{Generated samples from \ours.} Key components are highlighted in \textcolor{red}{\textbf{RED}}.}\label{fig:main-demo}
\end{figure}


First, we present a comprehensive data processing pipeline designed to construct large-scale, high-quality image and video-text datasets. The pipeline integrates multiple advanced techniques, including video and image filtering based on aesthetic scores, OCR-driven content analysis, and subjective evaluations, to ensure exceptional visual and contextual quality. Furthermore, we employ multimodal large language models~(MLLMs)~\citep{yuan2025tarsier2} to generate dense and contextually aligned captions, which are subsequently refined using an additional large language model~(LLM)~\citep{yang2024qwen2} to enhance their accuracy, fluency, and descriptive richness. As a result, we have curated a robust training dataset comprising approximately 36M video-text pairs and 160M image-text pairs, which are proven sufficient for training industry-level generative models.

Secondly, we take a pioneering step by applying rectified flow formulation~\citep{lipman2023flow} for joint image and video generation, implemented through the \ours model family, which comprises Transformer architectures with 2B and 8B parameters. At its core, the \ours framework employs a 3D joint image-video variational autoencoder (VAE) to compress image and video inputs into a shared latent space, facilitating unified representation. This shared latent space is coupled with a full-attention~\citep{vaswani2017attention} mechanism, enabling seamless joint training of image and video. This architecture delivers high-quality, coherent outputs across both images and videos, establishing a unified framework for visual generation tasks.


Furthermore, to support the training of \ours at scale, we have developed a robust infrastructure tailored for large-scale model training. Our approach incorporates advanced parallelism strategies~\citep{jacobs2023deepspeed, pytorch_fsdp} to manage memory efficiently during long-context training. Additionally, we employ ByteCheckpoint~\citep{wan2024bytecheckpoint} for high-performance checkpointing and integrate fault-tolerant mechanisms from MegaScale~\citep{jiang2024megascale} to ensure stability and scalability across large GPU clusters. These optimizations enable \ours to handle the computational and data challenges of generative modeling with exceptional efficiency and reliability.


We evaluate \ours on both text-to-image and text-to-video benchmarks to highlight its competitive advantages. For text-to-image generation, \ours-T2I demonstrates strong performance across multiple benchmarks, including T2I-CompBench~\citep{huang2023t2i-compbench}, GenEval~\citep{ghosh2024geneval}, and DPG-Bench~\citep{hu2024ella_dbgbench}, excelling in both visual quality and text-image alignment. In text-to-video benchmarks, \ours-T2V achieves state-of-the-art performance on the UCF-101~\citep{ucf101} zero-shot generation task. Additionally, \ours-T2V attains an impressive score of \textbf{84.85} on VBench~\citep{huang2024vbench}, securing the top position on the leaderboard (as of 2025-01-25) and surpassing several leading commercial text-to-video models. Qualitative results, illustrated in \Cref{fig:main-demo}, further demonstrate the superior quality of the generated media samples. These findings underscore \ours's effectiveness in multi-modal generation and its potential as a high-performing solution for both research and commercial applications.
\section{Related Work}

\subsection{Large 3D Reconstruction Models}
Recently, generalized feed-forward models for 3D reconstruction from sparse input views have garnered considerable attention due to their applicability in heavily under-constrained scenarios. The Large Reconstruction Model (LRM)~\cite{hong2023lrm} uses a transformer-based encoder-decoder pipeline to infer a NeRF reconstruction from just a single image. Newer iterations have shifted the focus towards generating 3D Gaussian representations from four input images~\cite{tang2025lgm, xu2024grm, zhang2025gslrm, charatan2024pixelsplat, chen2025mvsplat, liu2025mvsgaussian}, showing remarkable novel view synthesis results. The paradigm of transformer-based sparse 3D reconstruction has also successfully been applied to lifting monocular videos to 4D~\cite{ren2024l4gm}. \\
Yet, none of the existing works in the domain have studied the use-case of inferring \textit{animatable} 3D representations from sparse input images, which is the focus of our work. To this end, we build on top of the Large Gaussian Reconstruction Model (GRM)~\cite{xu2024grm}.

\subsection{3D-aware Portrait Animation}
A different line of work focuses on animating portraits in a 3D-aware manner.
MegaPortraits~\cite{drobyshev2022megaportraits} builds a 3D Volume given a source and driving image, and renders the animated source actor via orthographic projection with subsequent 2D neural rendering.
3D morphable models (3DMMs)~\cite{blanz19993dmm} are extensively used to obtain more interpretable control over the portrait animation. For example, StyleRig~\cite{tewari2020stylerig} demonstrates how a 3DMM can be used to control the data generated from a pre-trained StyleGAN~\cite{karras2019stylegan} network. ROME~\cite{khakhulin2022rome} predicts vertex offsets and texture of a FLAME~\cite{li2017flame} mesh from the input image.
A TriPlane representation is inferred and animated via FLAME~\cite{li2017flame} in multiple methods like Portrait4D~\cite{deng2024portrait4d}, Portrait4D-v2~\cite{deng2024portrait4dv2}, and GPAvatar~\cite{chu2024gpavatar}.
Others, such as VOODOO 3D~\cite{tran2024voodoo3d} and VOODOO XP~\cite{tran2024voodooxp}, learn their own expression encoder to drive the source person in a more detailed manner. \\
All of the aforementioned methods require nothing more than a single image of a person to animate it. This allows them to train on large monocular video datasets to infer a very generic motion prior that even translates to paintings or cartoon characters. However, due to their task formulation, these methods mostly focus on image synthesis from a frontal camera, often trading 3D consistency for better image quality by using 2D screen-space neural renderers. In contrast, our work aims to produce a truthful and complete 3D avatar representation from the input images that can be viewed from any angle.  

\subsection{Photo-realistic 3D Face Models}
The increasing availability of large-scale multi-view face datasets~\cite{kirschstein2023nersemble, ava256, pan2024renderme360, yang2020facescape} has enabled building photo-realistic 3D face models that learn a detailed prior over both geometry and appearance of human faces. HeadNeRF~\cite{hong2022headnerf} conditions a Neural Radiance Field (NeRF)~\cite{mildenhall2021nerf} on identity, expression, albedo, and illumination codes. VRMM~\cite{yang2024vrmm} builds a high-quality and relightable 3D face model using volumetric primitives~\cite{lombardi2021mvp}. One2Avatar~\cite{yu2024one2avatar} extends a 3DMM by anchoring a radiance field to its surface. More recently, GPHM~\cite{xu2025gphm} and HeadGAP~\cite{zheng2024headgap} have adopted 3D Gaussians to build a photo-realistic 3D face model. \\
Photo-realistic 3D face models learn a powerful prior over human facial appearance and geometry, which can be fitted to a single or multiple images of a person, effectively inferring a 3D head avatar. However, the fitting procedure itself is non-trivial and often requires expensive test-time optimization, impeding casual use-cases on consumer-grade devices. While this limitation may be circumvented by learning a generalized encoder that maps images into the 3D face model's latent space, another fundamental limitation remains. Even with more multi-view face datasets being published, the number of available training subjects rarely exceeds the thousands, making it hard to truly learn the full distibution of human facial appearance. Instead, our approach avoids generalizing over the identity axis by conditioning on some images of a person, and only generalizes over the expression axis for which plenty of data is available. 

A similar motivation has inspired recent work on codec avatars where a generalized network infers an animatable 3D representation given a registered mesh of a person~\cite{cao2022authentic, li2024uravatar}.
The resulting avatars exhibit excellent quality at the cost of several minutes of video capture per subject and expensive test-time optimization.
For example, URAvatar~\cite{li2024uravatar} finetunes their network on the given video recording for 3 hours on 8 A100 GPUs, making inference on consumer-grade devices impossible. In contrast, our approach directly regresses the final 3D head avatar from just four input images without the need for expensive test-time fine-tuning.




\section{Problem Formulation}

In this section, we introduce the problem formulation of \system{}. 
First, based on the retrieve-then-adapt idea~\cite{qian2020retrieve}, we discuss our approach for generating data reports by breaking it down into retrieving a related report, deducing, and reproducing a sequence of data analysis segments, each comprising analytical objectives, data processing, and insights.
Based on the formulation, we further conduct a preliminary study to understand the detailed design requirements of what can be reused from previous reports and how to rectify the new data. 
Based on the findings, we propose \system{}'s design considerations.

\subsection{Definition of Data Reports}
\label{subsec:problem_formulation}

To create a data report, data scientists need to explore and analyze the data, obtain data insights, and organize them into coherent narratives and charts~\cite{li2023wherearewesofar}. 
A shortcut for this process might be referring to an existing report and attempting to adapt it with new datasets.
To begin, it is required to retrieve a report that shares a similar topic with the current dataset.
In this section, we assume the retrieval is performed with a reliable search engine or by the users and focus on the parsing and reconstruction of the report.

Given a report, the key to its reconstruction is to decompose the report into logically coherent segments and validate whether the new data can fulfill the needs of different segments or provide similar insights supporting the argument. 
If not, how can adequate transformation of the original segments or new data be employed to make the whole process successful? 

Previous research~\cite{bar2020automatically, li2023edassistant, batch2017interactive} has outlined the analysis workflow as the three steps: 
1) Given a dataset to analyze, a data scientist usually begins by viewing the data and setting an analytical objective. 
2) Then, the data scientist would perform various data transformation steps, usually writing and executing code, and potentially encode the transformed data into a chart~\cite{wang2023dataFormulator}. 
3) Finally, the scientist inspects the results of the analysis and obtains insights into the data. 
We define an ``analysis segment''  as a triplet:
$$ segment := (objective, transformation, insight).$$

After completing an analysis segment, data scientists continue the analysis by creating new analytical objectives from previous ones, hustling to gain new insights, and generating a subsequent analysis segment. 
By repeating this process, the data scientists create a report with a complete analysis workflow and data insights. 
In this study, the segment is considered a basic building block of a data report.

Based on the definitions, we formulate the analysis workflow as a sequence of interconnected analysis segments $S = \{s_0, s_1, \cdots, s_N\}$, where $N$ is the number of segments in a data report.
Each segment $s_j$ is defined by three key components $(o_j, t_j, i_j)$, where $o_j$ is the analytical objective that guides the inquiry, $t_j$ is the data transformations that process the data, and $i_j$ is the insights that emerge from this exploration. 
Moreover, these analysis segments have interdependent dependencies $D = \{d_0, d_1, \cdots\}$, as the objective for one segment might stem from either insight of a previous segment or the data.
Each dependency $d\in D$ denotes a directed link of a tuple of segments $(s_i, s_j)$, representing that the segment $s_j$ stems from $s_i$.
In this light, the analysis workflow can be organized into a tree~\autoref{fig:formulation}a1, where each node denotes an analysis segment, and each edge denotes a dependency. 
Specifically, the initial analysis segment depends on the data. 
Based on this formulation, a data report is essentially a structural form of insights $S = \{s_0, s_1, ..., s_N\}$ distilled from these segments of analysis. 

Therefore, our method revolves around deducing and reproducing the sequence of analysis segments $S$ from the reference report, including analytical objectives, data processing, and insights. 
Finally, we organize the newly gleaned insights into a coherent and informative report, thereby reviving the original report with fresh data insights.

\begin{figure}[!htb] 
  \centering
  \includegraphics[width=0.5\linewidth]{figs/formulation.png}
  \caption{
  Producing a data report (c1) involves analyzing the data (a) and summarizing the analyzed insights into a data report (b). 
  Specifically, the data analysis workflow (a) includes a series of interdependent analysis segments (a1), each corresponding to an analytical objective, data transformation steps, and data insights (a2). 
  To reuse an existing report on a new dataset, we first deduce the data analysis workflow and reproduce it on the new data (c2). 
  }
  \label{fig:formulation}
\end{figure}

\subsection{Preliminary Study}
\label{subsec:preliminary_study_settings}

Based on the definitions, we aim to decompose a data report into segments and apply them to new data. 
However, the organization of data reports is flexible, highly depending on user preference and experiences. 
Besides, the aspects of a report that can be inherited and need adjustment remain unclear. 
To bridge this gap, we conducted a preliminary study with two main objectives:
(1) Investigate the relationship between the narrative structure of data reports and the corresponding analysis segments. 
(2) Identify the similarities and differences in analysis segments among data reports on the same topic and analyze how the differences stem from the data.

We collected data reports of different topics from well-known organizations that publish data reports, such as ONS~\cite{ons}, YouGov~\cite{yougov}, Pew Research Center~\cite{pewResearchCenter} and PPIC~\cite{ppic}. 
% \todo{based on the collected report, make a report repository. }
For the first objective, we analyzed the narrative structures of these reports to assess their alignment with our analysis segments. 

For the second objective, we observed that these organizations often publish reports on similar datasets, such as epidemic data collected at different times, typically issuing one report per dataset. 
These reports usually look similar but vary subtly in their analysis and content, which can be evidence to inspect which features are inherited and which require adjustment. 
Therefore, we further constructed 39 pairs of data reports that share the same topics and conducted pair-wise analysis on them. 
We identified the similarities and differences in each report pair and analyzed how the differences were sourced from the data. 
Treating a data report containing a series of analysis segments, including \textbf{analytical objectives}, \textbf{data transformations}, and \textbf{report content}, we summarized the patterns of similarities and differences regarding these elements.

\subsubsection{Narrative Structure of Reports}

To extract analysis segments from a report, we identified how the report's content aligns with these segments, defined as \textbf{analytical objectives}, \textbf{data transformations}, and \textbf{report content}. 
We explored whether the narrative structure could be segmented so that each part corresponds to a distinct analysis segment.

Our analysis of 35 reports showed that in most cases (32/35), the analytical content was presented as distinct segments, each focused on a single objective rather than interspersed with multiple topics, with related text and possibly a chart grouped together.
Additionally, 11 out of 35 reports included non-analytical content, such as background information, which could either supplement a specific analysis or the entire report and appeared flexibly throughout. 
Some reports (23/35) also included a summary of key insights at the beginning or end, which we excluded to concentrate on the main analytical content.

\subsubsection{Similarities and Differences in Pair Reports}

We summarize the patterns of similarities and differences between the data reports on similar topics. 
These findings lay the foundation for designing and implementing a method to reuse data reports with new data. 

\paragraph{Analytical Objectives}
Analytical objectives are guidance for the exploration of insights and findings from the data.
For example, the report of internet users~\footnote{https://www.ons.gov.uk/businessindustryandtrade/itandinternetindustry/\\bulletins/internetusers/2018} holds an analytical objective to explore the Internet use among each different age group. 
Therefore, we identified the analytical objectives in the collected reports by inspecting the aspects of the data findings that were discussed. 
After identification, we compared the analytical objectives between each pair of reports and analyzed their alignment and variations. 

As a result, all of our collected pair reports reflect similar analytical objectives.
Specifically, 34/39 of them involve analytical objectives that are exactly the same, while 35/39 of them involve slight differences.
Most slight differences source from different \textbf{data contexts and scopes}. 
For example, the analytical objectives of two data with different time ranges will also focus on distinct time frames. 
Others stem from the \textbf{dependencies to previous data insights}. 
Analytical objectives may be formed based on previous insights through logical dependencies, e.g., exploring the reason for an increasing trend. 
Therefore, the changes in previous insights may also cause adjustments in the latter objectives. 
Additionally, some pair reports involve completely different analytical objectives (18/39), which mainly source from \textbf{different data fields}. 
For example, newer data may introduce additional data fields, thus triggering new analysis objectives.
Reports may also incorporate insights from external data sources (4/39), spanning different contexts, scopes, and fields. 
These insights typically maintain a logical dependency on previously established insights, such as generalizing from local to national trends, which also introduces varied analytical objectives. 


\paragraph{Transformation Operations}
Data transformation operations are not explicitly outlined within the data reports. 
Moreover, most of the source data provided by our collected reports have been processed (34/39), making it harder to infer the specific data processing conducted. 
Nonetheless, two facets of data processing can still be discerned from the reports. 
Firstly, the reports mirror the output of the data processing, as the charts and narratives presenting data insights directly originate from these outputs.
Secondly, they also reflect detailed data processing choices, particularly regarding charts, which involve decisions on chart type, encoding, binning, etc. 

Considering these factors, we compared the content referring to similar analytical objectives between each pair of reports and analyzed their similarities and differences in analysis outputs and data processing choices. 
Consequently, we observed that similar analytical objectives always yield similar \textbf{analysis output forms}.
For example, the objective of analyzing trends always results in a chart with the temporal field on the x-axis, indicating a transformed dataset measuring variables across time periods.
However, the \textbf{detailed data processing choices} may vary to accommodate the data difference (13/39). 
Varied formats and scopes of data fields could potentially result in different chart types or levels of binning granularity to better align with visualization rules.

\paragraph{Report Content}
The report content, comprising both textual information and charts, is directly derived from the result of data processing. 
Since analysis results from different datasets naturally differ, resulting in varying values in the report content, our primary focus lies in the similarities and differences beyond mere numerical distinctions. 

As a result, regarding the similarity of report content, we observe that each pair of reports shares a similar \textbf{narrative and visual style}, such as the formality degree in tone and infographic design. 
As our main objective is to reuse the analysis workflow, the inheritance of content style is not our primary focus and is therefore not considered in this study.
The differences primarily manifest in the textual descriptions of data insights. 
Different reports may describe varying types of data insights. 
For example, one report might focus on detailing an outlier, while another might only describe overall trends. 
This difference stems essentially from \textbf{distinct analysis results}, which not only result in numerical disparities but also lead to variations in the reflected data insights.
Under the same data processing steps, one dataset may exhibit a highly significant outlier in the results, while another dataset may not.

\subsection{Expert Interview}
The findings of the preliminary study revealed that various aspects of existing data reports can be leveraged to generate new reports, but these elements require adjustments to align with the new data. 
The study also provides theoretical guidance on how to identify incompatible aspects and the directions in which adjustments should be made. 

To further understand the user requirements in reusing existing reports to analyze data, we conducted an expert interview with two experienced data analysts, EA and EB. 
EA has over two years of experience working as an actuary at an insurance company, where they frequently analyze data, organize results, and present them to clients to guide future decision-making. 
EB is a seasoned researcher in data storytelling and an experienced data journalist. 

We conducted 45-minute interviews with each expert, during which we discussed the following questions: (1) What formats do they typically use to present data analysis results, such as data reports, data stories, or others? (2) Do they encounter scenarios in which they reuse or refer to existing data analysis materials for analyzing new data and proposing new presentation materials? (3) If so, what is their workflow?

For the first two questions, EA noted that he use different formats depending on the case, including Excel files, slides, and data reports. 
He noted that for all three formats, he often refers to existing analysis materials. 
For example, to analyze and present data for a new insurance product,  he may refer to past reports or related reports from other products. 
In these cases, both the data and analysis goals are usually quite similar, making existing materials particularly useful.
EB, on the other hand, discussed the distinction between data reports and data stories. 
She pointed out that data reports are highly structured and commonly used in formal settings such as official documents, presentations, and communications. 
These reports often follow stable templates, making it easy to refer to existing materials when generating new reports. 
In contrast, EB sees data stories as a more creative format and tends not to reference previous materials. 
She prefers to avoid replicating others' approaches and focuses on originality in crafting data stories.

For the third question, Although EA acknowledged using existing materials in multiple formats, he emphasized that the aspects reused and the workflow vary depending on the format. 
For Excel files, these existing files are often with previous SQL scripts and formulas, which can be seen as preserved analysis code. 
In these cases, small modifications (e.g., updating SQL query conditions for new data) are typically sufficient. 
For data reports and slides, however, the process is different. 
These formats often lack original analysis files, and while existing materials can inspire analysis goals, chart creation, and textual descriptions, the data analysis still needs to be conducted manually.
EA further elaborated on the differences between slides and data reports. 
While slides typically feature relevant charts, they can lack detailed narrative descriptions, as slides are often presented orally. 
In contrast, data reports are expected to contain more comprehensive written content, including detailed explanations.


EA and EB further elaborated on the workflow of reusing and referring to previous data reports. 
For cases where the data fields are similar, or the analysis goals remain consistent, both EA and EB noted that these cases allow simply replacing charts with new charts and replacing the numerical conclusions to align with the new data. 
However, when the fields differ, EA mentioned that adjustments are necessary, either by modifying the analysis goals or removing irrelevant sections of the report. 
For new fields, EA would develop new analysis goals and rewrite the descriptive analysis accordingly. In cases where the insights change, the analysis may need to be entirely redone to accommodate the new findings.

\subsection{Expert Interview}

The findings of the preliminary study reveal that various aspects of existing data reports can be leveraged to generate new reports, but these elements require adjustments to align with the new data. 
The study also provides guidance on how to identify incompatible aspects and the directions in which adjustments should be made. However, it is unclear how users reuse past analysis reports for new scenarios.

Therefore, we conducted an expert interview targeted at data analysts who frequently explore new data and compose reports to communicate insights to leaders or clients. Moreover, we expected the interviewees to have certain experience in using LLMs in their analysis and prototyping process.
We interviewed two experienced data analysts, EA and EB. 
EA has over two years of experience working as an actuary at an insurance company, where they frequently analyze a variety of data, including claims history, policyholder demographics, risk factors, and market trends.
Findings are presented to a range of clients, including internal stakeholders (such as underwriters, product managers, and senior executives) and external clients (such as brokers, corporate policyholders, or regulatory bodies), to guide strategic decision-making and ensure compliance with industry standards. 
EB is an assistant professor in the Department of Journalism of a top-tier university. She also holds a Ph.D. in data science and frequently writes data journalism for news media.


We conducted 45-minute interviews with each expert, during which we raised the following questions: (1) What formats do they typically use to present data analysis results, such as slides, reports, dashboards, and spreadsheets? (2) Do they encounter scenarios in which they reuse or refer to existing data analysis materials for analyzing new data and proposing new presentation materials? (3) If so, what is their workflow?

For the first two questions, EA mentioned that he uses different formats depending on the specific case, including Excel files, slides, and data reports. He emphasized that for all three formats, he frequently refers to existing analysis materials. For instance, when analyzing and presenting data for a new insurance product, he often consults past reports or related analyses from similar products. In such cases, both the data and the analysis objectives tend to align closely, making existing materials particularly valuable for efficiency and consistency. Additionally, EA noted that reports are his primary format for conveying data insights, particularly in formal settings.
On the other hand, EB explained that her approach to data presentation is more scenario-dependent, and her use of previous materials varies accordingly. She highlighted that data reports, which are highly structured and commonly employed in formal contexts such as official documents, presentations, and communications, often follow standardized templates. This structure makes it straightforward to adapt or reference existing materials when creating new reports.
However, EB also expressed her interest in crafting ``data stories,'' which, while similar to data reports in format, are more creative and exploratory. Unlike reports, she tends to avoid relying on specific pre-existing materials for data stories, as doing so might constrain her thinking. Instead, she prioritizes originality and creativity, focusing on developing unique narratives that reflect her individual insights and perspectives.


For the third question, Although EA acknowledged using existing materials in multiple formats, he emphasized that the aspects reused and the workflow vary depending on the format. 
For Excel files, these existing files are often with previous SQL scripts and formulas, which can be seen as preserved analysis code. 
In these cases, small modifications (e.g., updating SQL query conditions for new data) are typically sufficient. 
For data reports and slides, however, the process is different. 
These formats often lack original analysis files, and while existing materials can inspire analysis goals, chart creation, and textual descriptions, the data analysis still needs to be conducted manually.
EA further elaborated on the differences between slides and data reports. 
While slides typically feature relevant charts, they can lack detailed narrative descriptions, as slides are often presented orally. 
In contrast, data reports are expected to contain more comprehensive written content, including detailed explanations.


EA and EB further elaborated on the workflow of reusing and referring to previous data reports. 
For cases where the data fields are similar, or the analysis goals remain consistent, both EA and EB noted that these cases allow simply replacing charts with new charts and replacing the numerical conclusions to align with the new data. 
However, when the fields differ, EA mentioned that adjustments are necessary, either by modifying the analysis goals or removing irrelevant sections of the report. 
For new fields, EA would develop new analysis goals and rewrite the descriptive analysis accordingly. In cases where the insights change, the analysis may need to be entirely redone to accommodate the new findings.

\subsection{Design Considerations}

Based on the problem formulation and the findings of the preliminary study, we summarize five design considerations (\textbf{C1}-\textbf{C5}) for an automatic method of reusing data reports with new data. 

\begin{enumerate}[label=\textbf{C\arabic*}]
\item \textbf{Support analytical objectives extraction, correction, and addition. } 
The key of the method is to extract and re-execute the analysis workflow of the existing report, which corresponds to a series of interdependent analytical objectives. 
To achieve this, the method should first split the report to identify distinct analysis segments and then extract the analytical objectives and their dependencies. 
Additionally, it should support smart objective correction and addition based on the dependencies and data features.

\item \textbf{Generate appropriate data processing steps automatically. } 
Based on the preliminary study, the data processing steps are not reflected directly in the existing report. 
Therefore, the method necessitates autonomous reasoning about appropriate data processing steps that yield outputs similar to those in the existing report while also making informed data processing choices adaptable to the new dataset. 

\item \textbf{Produce insightful report content derived from analysis. } 
The textual and visual content that presents data insights constitutes the primary component of a data report. 
The method should produce content that effectively presents new data insights derived from analysis. 
Our preliminary study also revealed that reports often include non-analytical content, such as background information. 
As LLMs are pre-trained on extensive background knowledge, we allow them to generate this content. 
However, any generated non-analytical content will be highlighted, as it may not always be reliable.

\item \textbf{Enable real-time output observation and report modification. } 
Given the complexity of the method, which involves analytical objectives, data processing, report content, and structure, uncertainties naturally arise, potentially leading to deviations from user expectations. 
To address this, the method should provide an interactive interface, enabling users to observe generated outputs in real-time and make necessary adjustments. This ensures that the final report aligns closely with user expectations. 

\item \textbf{Facilitate report structure organization and modification.}
Since the report is generated based on the reference report's workflow, it will naturally exhibit a similar narrative structure. 
The method should allow for re-organizing this inherited structure, including adding and generating (sub-)titles, to ensure the report is well-structured and tailored to the new content.
\end{enumerate}

Based on the design considerations, we develop an intelligent method, \system{}, to deduce and reproduce the authoring workflow of existing data reports on the new data. 
The pipeline of \system{} consists of three stages. 
\textbf{In the pre-processing stage,} \system{} recommends the most relevant reports from a built repository, ranked by similarity to the user's dataset. 
It then dissects the existing report into interconnected segments, each corresponding to the data insights of an analysis segment (\textbf{C1}). 
Based on the segmentation, it extracts the analytical objectives of each segment and deduces their dependencies (\textbf{C1}). 
\textbf{In the analysis stage,} \system{} executes each segment by reusing the information from the original report, identifying the inconsistencies, and customizing the analytical objectives, approaches, and report contents based on the new data (\textbf{C1-C3}). 
\textbf{In the organization stage,} \system{} inherits the original report structure and enables title re-generation (\textbf{C5}). 
Moreover, to enhance usability, we integrate an interactive interface for \system{}, allowing users to inspect real-time outputs, add new analytical objectives, and modify report content as needed (\textbf{C2, C5}). 



\section{\system{}}
\label{sec:respark}

This section describes the implementation of \system{}, including pre-processing, analysis, and organization. 
\system{} utilizes the GPTs from Azure to incorporate the LLM-driven functionalities. 
Specifically, we employ the ``gpt-4-vision-preview'' model as our input involves chart images. 
The prompts are provided in the supplementary materials. 

\subsection{Pre-processing Stage}

Before proceeding with analysis and organization, we need to pre-process the user dataset and reports to (1) acquire necessary data features, (2) recommend the most relevant reports to the data, and (3) extract the analytical objective and their dependencies of the selected report for the subsequent processes. 

\subsubsection{Data Pre-processing}
\label{subsubsec:data_pre_processing}

Based on the findings in the preliminary study (~\autoref{subsec:preliminary_study}), most adjustments entail considerations of data features such as context, scope, fields, and formats. 
Presenting the entire dataset to LLMs is currently impractical due to token limitations, and it does not facilitate a comprehensive understanding of these data features. 
Therefore, we utilize a similar data summary method to LIDA~\cite{dibia2023lida}. 
This method first extracts scope, data type, and unique value count information and samples some values for each data field. 
Subsequently, it employs LLMs to provide brief semantic descriptions for the dataset and each data field. 
The description of the dataset can also help recommend relevant reports(~\autoref{subsubsec:report_retrieval}). 
These pieces of information are then integrated to form a comprehensive data summary.

\subsubsection{Report Pre-processing}
\label{subsubsec:report_pre_processing}

\begin{figure*}[!htb] 
  \centering
  \includegraphics[width=\linewidth]{figs/system.png}
  \caption{
  The interface of \system{}. 
  \system{} consists of four views: data view (b-c), dependency view (d-e), content view (f-g), and generation view (h-k). 
  The data view displays the overall description and data field information. 
  The dependency view displays the extracted interdependent report segments. 
  The content view shows the analytical objective and content of the selected segment. 
  The generation view demonstrates the generated results in real-time. 
  }
  \label{fig:interface}
\end{figure*}

Based on the analysis workflow formulation in~\autoref{subsec:problem_formulation}, our goal is to deduce the analysis segments and their dependencies from the original report for subsequent execution. 
Our preliminary study showed that most reports present analytical content in distinct segments, each focused on a single objective, with related text and visuals grouped together. 
Therefore, ideally, we can find a segmentation that aligns each report section with a specific analysis segment. 
In this light, \system{} should segment the report accordingly, extract the analytical objectives for each segment, and deduce their dependencies.

To achieve this goal, an appropriate report segmentation criteria is very important, as it directly determines the entire analysis workflow (consisting of a sequence of analysis segments). 
The accuracy of segmentation also affects the quality of the extracted analytical objectives and dependencies.

We were initially inspired by prior work in automatic storytelling and insight-mining, which formalizes data insights and their relationships~\cite{ma2023insightpilot, wang2019datashot}. 
For example, Calliope~\cite{shi2020calliope} defines a data story as a series of interrelated insights, with each insight describing data patterns in specific data fields and subsets. 
For instance, ``The average worldwide gross for action movies is increasing over time'' describes an increasing trend measuring ``average (worldwide gross)'' over the breakdown ``year'' within the subset ``genre = action''. 
Based on this definition, we can potentially segment the report by identifying the insight type, measure, breakdown, subset, etc., and combine the insights into segments based on these labels. 
However, we found this definition challenging for segmenting practical data reports, as it doesn't accommodate the flexibility of analyses that involve deeper data transformations, such as creating new fields and describing patterns in derived variables.

Therefore, we need to define new segmentation criteria that accommodate the flexible analysis in the data report. 
Instead of formally defining a ``segment'' or an ``analytical objective'' with a strict data model, we provide a loose description of how segments can be divided and use LLMs to perform segmentation. 
Our preliminary study indicated that most report segments consist of continuous text and possibly a chart.
Based on the study, we execute report segmentation, extract analytical objectives, and deduce the dependencies between segments through the following approach: 
\begin{itemize}
    \item \textbf{Match. } 
    First, for each chart, we match the related paragraph text to form a segment. 
    Based on our preliminary study, we assume that (1) each paragraph corresponds to the nearest preceding or following chart, or none at all, and (2) all text associated with a single chart is continuous. 
    Starting with the first paragraph, LLMs determine whether it matches the nearest preceding or following chart (e.g., describing insights from the chart) or if it doesn't relate to any chart.
    
    \item \textbf{Categorize. } 
    For text that doesn't match a chart, LLMs categorize it to determine if it involves data analysis or serves another purpose, such as providing background information. 
    For continuous text segments that involve data analysis, we further assess whether they belong to the same segment (describe insights derived from the same transformed data). 
    
    \item \textbf{Summarize. } 
    After matching and categorizing, LLMs summarize the analytical objective of each segment and deduce its dependencies with previous segments. 
    We use the six logical relations defined in Calliope to outline dependencies among report segments. 
    LLMs determine whether new content is logically connected to an existing segment or originates directly from the data.
\end{itemize}


\subsubsection{Relevant Report Retrieval}
\label{subsubsec:report_retrieval}


With the pre-processed data and reports, users can select a reference report to analyze the target dataset. 
Based on findings from our preliminary study, various aspects of existing data reports, such as analytical objectives and report content, can serve as helpful reference material. 
However, since the reference report's data may differ from the target dataset, adjustments are necessary to align with the new dataset. 
The closer the reference report's data is to the target dataset, the more aspects can be reused without modification, making the report more suitable for use as a reference.


To facilitate the retrieval of suitable reports, we aim to identify the reports with data similar to the target dataset. 
The core idea is to convert both the dataset and report information into vector embeddings, compute their cosine similarities, and rank the reports from highest to lowest score.
The key question is: which specific information from the dataset and reports should be embedded?
We propose two mechanisms for extracting the embedding of data and report information: topic relevance and field similarity.
\begin{itemize}
    \item \textbf{Topic relevance} refers to the alignment between the topic of the dataset and the report. 
    Typically, datasets and reports within the same domain (e.g., health, economy) exhibit higher topic relevance. 
    For example, a sales dataset is highly topic-relevant to a report analyzing market sales trends. 
    To compute topic relevance, we extract the embedding of the dataset’s name and description, along with the headings and pre-processed analytical objectives of the report. 
    We hypothesize that these elements are semantically related to the corpus's overall topic, and the cosine similarity of their embeddings can reflect their topic relevance.
    \item \textbf{Field similarity} pertains to the alignment of the data fields described in the report with those contained in the dataset. 
    For example, a report on voting intentions across different gender and age groups would exhibit higher field similarity with a dataset containing gender and age information. 
    To compute field similarity, we embed the names and descriptions of the dataset's fields. 
    For the reports, we use LLMs to infer the data fields discussed in the report and embed these inferred fields along with their descriptions. 
    We hypothesize that the cosine similarity between these reflects the degree of field similarity.
\end{itemize}
Finally, we sum the scores from both mechanisms for each report, ranking them from highest to lowest, allowing users to select the most appropriate reference reports.

\subsection{Analysis Stage}
\label{subsec:analysis_stage}

Through the pre-processing stage, we obtain the summary of the new dataset and the segments of the existing report. 
Each segment corresponds to an analytical objective, a dependency on the previous segment or the data, and pieces of report content, including text and charts. 
The next stage is to reproduce the analysis workflow by re-executing each segment on the new data, encompassing reusing and reconstructing the analytical objective, analysis operations, and report contents. 

\subsubsection{Analytical objective correction and insertion}

The analysis workflow is driven by a series of posed analysis objectives. 
Through the pre-processing stage, we obtain the analysis workflow of the existing report by dividing it into segments and extracting each segment's analytical objectives and dependencies. 
To adapt the workflow to new data, \system{} is required to (1) correct the extracted analytical objectives and (2) support the insertion of new objectives according to the data features and segment dependencies. 

\textbf{Analytical objective correction. }
The preliminary study indicates that while existing analytical objectives often remain applicable, alterations or removals may be necessary due to data fields, dependencies, or the data context and scope. 

First, the original objective might involve data fields absent in the new data. 
Given the pre-processed data summary, we employ LLMs to evaluate if the new dataset's fields sufficiently fulfill the objective, considering semantic similarities despite word-to-word differences in field names. 
For example, an objective mentioning ``earn money'' can be related to the data field ``gross.'' 
LLMs are tasked with explaining their decisions to enhance their reasoning~\cite{mialon2023augmented}. 
If the available fields suffice, LLMs should describe the required fields and analysis operations. 
Otherwise, LLMs must explain what external fields are needed to satisfy the objective. 
In such cases, we correct the objective by replacing missing fields with available alternatives. 
For instance, if a movie dataset lacks geographic data, the objective of locating the highest-grossing movies might shift focus to their directors.

Second, the original objective may derive from insights in a prior segment. 
Adjustments might stem from two scenarios. 
If the insight's nature changes (e.g., from an increasing to a decreasing trend), a corresponding shift is needed in the latter related objective (e.g., from identifying causes of growth to exploring reasons behind the downturn). 
Therefore, we provide the model with the newly generated results of the dependent segment and require it to identify and adapt to such variations. 
Additionally, the dependency may call for a context or scope that the data cannot satisfy, such as from local to national or from a 5-year trend to a 20-year trend. 
LLMs must infer whether changes in scope or context affect the objective's applicability, which could lead to its potential removal if the new data does not support similar adjustments. 

Third, minor adjustments are often required for data context and scope adjustments. 
For example, an objective focusing on a 5-year trend needs adjustment to fit a dataset covering only the past three years. 
LLMs should make these modifications based on the context and scope of the provided data.

\textbf{Analytical objective insertion. }
The uniqueness of new datasets and user-driven queries may necessitate adding fresh analytical objectives based on previous insights and dependencies. 
\system{} enables users to embed new objectives at chosen positions, rooted in the data or reliant on preceding analysis segments. 
Users can define the focus data fields and dependencies of these new objectives, and LLMs can suggest potential objectives based on user input.

\subsubsection{Analysis Operation Generation}

% \TODO{code structure, generate a table and a chart}. 
Once the analytical objective has been refined, \system{} generates the requisite analysis operations to fulfill this objective. 
Since these operations are not explicitly detailed in the report, we utilize the code-generation capabilities of LLMs for this phase. 

LLMs are prompted to generate analysis code that aligns with the clarified objective, provided with the data summary and original report content as guides. 
The model is instructed to refer to the original report to deduce the necessary data transformations. 
We also remind the model to generate the code that accommodates the new data, as the reference report content is from a different dataset and only serves as a reference for expected output. 
The model is required first to plan step by step~\cite{kojima2022large} and then generate the Python code that results in transformed data and a chart using matplotlib~\cite{Hunter2007matplotlib} or Seaborn~\cite{Waskom2021seaborn}. 

Upon code generation, we execute it to procure the transformed data and the accompanying chart. 
The execution may also raise errors. 
We relay any execution results, including the transformed data, chart, and potential errors, back to the LLMs. 
The model then assesses whether the code execution is successful and whether the results accurately address the analytical objective and are adequate for generating report content. 
Should the model deduce that revisions are necessary, the cycle of code generation and execution is repeated until satisfactory results are obtained, paving the way for report content creation.

\subsubsection{Report Content Production}

With the execution results in hand, we proceed to generate new report content. 
Given that the code already produces the chart, the model's task in this step is to generate the accompanying textual narrative. 
We instruct the model to produce a narrative that imitates the writing style of the reference report yet is tailored to fit the new data context and the insights derived from the executed analysis. 
We also enable user modification to the report content. 

\subsection{Organization Stage}

After reproducing the analysis workflow and obtaining the new data insights, the next step is to structure the new report. 
As we generate the sequence of segments based on the order of dependencies, the implicit logical structure is adopted naturally. 
Additionally, we inherit the explicit structural elements (such as titles and sections) from the original report. 
Newly inserted analytical objectives are incorporated along with their dependent segments. 
The report and its sections' titles are re-crafted based on the original ones, incorporating new data insights to guide the title generation process. 
User interventions are also supported, allowing for the reorganization of segments into new sections, thereby tailoring the report structure to meet user needs or preferences better.



\section{Interface}
\label{sec:usage_scenario}

We incorporate \system{} with an interactive interface. 
The interface consists of four views: data view, dependency view, content view, and generation view. 
We introduce the interface through a usage scenario. 

\textbf{Pre-processing. }
A data analyst is tasked with creating a data report for a dataset detailing crime in Los Angeles from 2020 to 2023~\footnote{https://catalog.data.gov/dataset/crime-data-from-2020-to-present}. 
S/he decides to use \system{} to assist in creating a data report with the Los Angeles data. 
The analyst begins by uploading the Los Angeles crime dataset(~\autoref{fig:interface}a). 
\system{} then pre-process the dataset. 
After the pre-processing, the analyst can view the dataset information(~\autoref{fig:interface}b), including the file name, overall description, and data field information. 

Next, the analyst opens the report repository, which displays the available reference reports ranked by relevance to the dataset (~\autoref{fig:retrieve}). 
The analyst selects the most relevant report, a 2022 Chicago crime report~\footnote{https://www.illinoispolicy.org/chicago-crime-spikes-in-2022-but-first-drop-in-murder-since-pandemic/}. 
The report is divided into six interdependent segments, each corresponding to an extracted analytical objective and report content(~\autoref{fig:interface}d). 
The analysis mode means the system is ready for analysis. 
With the segmentation and extracted objectives, the analyst can proceed with generating a new report.


\textbf{Segment execution. }
The analyst starts with the first segment (~\autoref{fig:interface}e), focusing on crime trends in Chicago from 2018 to 2022. 
S/he clicks ``generate'' to adapt this analytical objective to the Los Angeles data. 
\system{} modifies the objective to ``How has the total number of crimes changed annually from 2020 to 2023 in Los Angeles?'', which is aligned with the context and scope of the Los Angeles data (~\autoref{fig:interface}h). 
Following this, the analyst clicks ``generate'' again to conduct the analysis. 
\system{} generates the code and executes it (~\autoref{fig:interface}i), obtains the transformed data and chart, and generates the narratives to describe the findings. 
The resulting chart shows the number of crimes and changed percentages each year (~\autoref{fig:interface}j). 
The narrative describes the overall increase and the year-by-year changes, especially the peak in 2022 (~\autoref{fig:interface}k). 
The analyst considers such results reasonable and applies them (~\autoref{fig:interface}g). 

\begin{figure}[!htb] 
  \centering
  \includegraphics[width=0.5\linewidth]{figs/retrieve.png}
  \caption{
    A demonstration of retrieving reports from the report repository.
  }
  \label{fig:retrieve}
\end{figure}


\begin{figure}[!htb] 
  \centering
  \includegraphics[width=0.5\linewidth]{figs/segment_2.png}
  \caption{
    The result of generating the second segment. 
  }
  \label{fig:segment_2}
\end{figure}

\begin{figure}[!htb] 
  \centering
  \includegraphics[width=0.5\linewidth]{figs/delete.png}
  \caption{
  A demonstration of an analytical objective that fails to be corrected and needs to be removed. 
  }
  \label{fig:delete}
\end{figure}

\begin{figure}[!htb] 
  \centering
  \includegraphics[width=0.5\linewidth]{figs/add.png}
  \caption{
  A demonstration of inserting an analysis segment and generating a new analytical objective. 
  }
  \label{fig:add}
\end{figure}

\begin{figure}[!htb] 
  \centering
  \includegraphics[width=0.5\linewidth]{figs/title.png}
  \caption{
    A demonstration of generating new titles and adding section structures. 
  }
  \label{fig:title}
\end{figure}

\begin{figure}[!htb] 
  \centering
  \includegraphics[width=0.5\linewidth]{figs/highlight.png}
  \caption{
    A demonstration of highlighting the sentences that serve non-data analysis purposes. 
  }
  \label{fig:highlight}
\end{figure}

The second segment aims to explore the crime types that drove the increasing trend, which depends on the first segment with a cause-effect logic (~\autoref{fig:segment_2}a). 
As the previous segment results in an overall increasing trend as well, \system{} inherits the logic and corrects the objective to match the Los Angeles data context (~\autoref{fig:segment_2}b). 
Different from the reference report which only shows the changes in 2018 and 2022, \system{} calculates the top ten crime types contributing to the uptick and visualizes them through a bar chart for a clear comparison of cumulative numbers. 
Such varied design choices may stem from the narrative description in the Chicago report, which describes both the overall change and last year's change (~\autoref{fig:segment_2}a1). 
As the analysis progresses, the analyst explores the subsequent two segments, including the decreased crime types and their changes from 2022 to 2023. 

\textbf{Objective removal. }
After the segments above, the Chicago report moves to another analytical objective on homicide trends from 2018 to 2022, which stems directly from the data. 
Similarly, the analyst applies the tailored objective and report content generated by \system{}, which analyses the trend of homicide in LA from 2020 to 2023. 
However, the next segment of the Chicago report generalizes its analysis to include homicide trends from 2000 to 2022(~\autoref{fig:delete}). 
To inherit such a generalization logic on the generated report with the LA data, external data sources are needed, as the provided dataset covers only 2020 to 2023. 
Identified such a case, \system{} marks the analytical objective as ``none'' and visually indicates this error through a red node. 
Consequently, the analyst removes this node.

\textbf{Objective insertion. }
After generating these segments, the analyst finds that there are multiple unused data fields. 
Based on the previous analysis of the homicide trends, the analyst inserts a new node focusing on the time by selecting the ``Time Occ'' field and applying a ``similarity'' logic for parallel analysis (~\autoref{fig:add}). 
Therefore, \system{} forms a new analytical objective to analyze the time distribution of arson. 
%The analysis unveils a tendency for victims to be younger adults, peaking in their mid-to-late 20s and declining after the age of 50. 
% Further investigation into the timing of arsons reveals that most occur during the evening.

\textbf{Structure organization. }
Upon concluding the data analysis, the analyst shifts focus to report structuring, crafting titles, and organizing sections for clarity (~\autoref{fig:title}). 
S/he generates the title and obtains a title that describes the rise in thefts patterns. 
The analyst then organizes the content into two coherent sections. 
The complete data report is detailed in the supplementary materials provided.

\textbf{Highlight. }
For report texts, \system{} highlights sentences that serve non-data analysis purposes  (~\autoref{fig:highlight}). Since these sentences may lack data support, highlighting them can alert users to selectively receive this information.
\section{Comparative Study with \system{} Generation Pipeline}
\label{sec:comparative_study}

We conducted a comparative study to evaluate the effectiveness of the \system{} pipeline. 

\subsection{Study Setup}
Our primary focus was to assess two aspects:
(1) whether the reference report facilitated deeper and more logical data analysis, 
(2) whether the pre-processed segmentation and one-by-one adaptation approach proved beneficial for reusing the analytical logic from the existing report.
We chose to generate two baseline reports with GPT-4o because GPT-4o with Code Interpreter can process CSV files, execute code, and summarize the results. 
Specifically, we generated reports based on the same dataset under three different conditions: (1) using \system{}, (2) using GPT-4o without a reference report, and (3) using GPT-4o with a reference report. This setup serves as an ablation study to evaluate the impact of the reference report and our decomposition-and-adaptation pipeline.

We did not recruit additional data analysts to generate the reports. 
Instead, we directly utilized the reports generated by \system{} and GPT-4o without manual modifications to eliminate potential bias from human intervention. 
This approach allowed us to assess whether the \system{} pipeline could enhance the performance of data report generation without human intervention.

To simulate the practical scenarios of reading data reports, only the charts and narratives were included, excluding the specific data transformation code.
Participants were presented with these three sets and asked to evaluate and compare the quality of the reports within each set. 
To minimize potential order effects, the order of the reports within each set and the order of the three sets provided to participants were counterbalanced.

\subsubsection{Participants. }
We recruited 18 participants (P1-P18, 11 males and 7 females) from various backgrounds, including data visualization (12), data science (4), machine learning (3), network security (1), and human-computer interaction (1). 
Most participants were familiar with data analysis, with an average self-reported score of 3.5 on a 5-point Likert Scale. 
All participants had experience reading data reports, usually from data news, such as official statistics bureau reports, industry research reports, publicly disclosed financial reports, university survey reports, etc.
None of them had experience using our system before the experiment. 

\subsubsection{Data and reference reports. }
We selected three pairs of datasets and reference reports for the study. 
These datasets include the crime data described in~\autoref{sec:usage_scenario}, as well as two popular Kaggle datasets: the Titanic dataset and the cardiovascular disease dataset. 
The datasets were chosen for several reasons: 
They encompass different types of data fields (temporal, quantitative, and categorical), and each contains more than ten data fields. 
They are also relatively clean, requiring minimal pre-processing, making them well-suited for direct analysis. 
Additionally, they are easy to understand and suitable for broad audiences.

% Select Report (10): 
For the reference reports, we used the top-ranked report for each dataset. 
For the crime data, we used a report on Chicago crime, which is relevant both in terms of topic and data characteristics. 
For the cardiovascular disease dataset, we selected a report on the rise of Alzheimer's disease, which is highly relevant to the topic and somewhat similar in data. 
For the Titanic dataset, we chose a report on voting patterns in Britain. 
While this report is not highly relevant in terms of topic, it shares data similarities, such as analyzing different groups (e.g., by sex, age) in relation to specific outcomes (e.g., survival on the Titanic, voting behavior).
Detailed information on the datasets and reference reports are provided in the supplementary materials. 

\subsubsection{Report Generation. }
For each dataset-reference report pair, we generated new reports using three approaches: \system{}, GPT-4o without the reference report, and GPT-4o with the reference report. 
To ensure a fair comparison, all results were generated without manual modifications.

For \system{}, we directly generated new analysis objectives, charts, and text for each segment without any manual adjustments or added objectives.
After completing each segment, we used \system{} to generate a final title for the entire report. 
For the two baselines using GPT-4o, we designed specific input and prompts. 
We prompted the model to follow a process similar to \system{}'s pipeline to generate each report segment. 
This process included defining an objective, writing analysis code, and generating the corresponding section of the report. 
The key differences between the baselines and \system{} are outlined below.

\textbf{Baseline without the reference report.} The model was provided solely with the prompt and CSV dataset. 
The baseline is used to assess whether the reference report facilitates better data analysis.

\textbf{Baseline with the reference report. } In this baseline, the model received the CSV dataset, the textual content of the reference report, and the corresponding chart images. 
We chose to provide text and images instead of a PDF file of the reference report due to limitations in GPT-4o's ability to interpret information from PDFs. 
In our preliminary tests, when we uploaded PDFs to the model and made queries about images within the PDFs, the model occasionally failed to retrieve the necessary information. 
The model was further prompted to adapt the analytical objectives and methodologies from the reference report and make necessary adjustments to align with the dataset. 
Different from \system{}, this baseline did not require the reference report to be segmented into distinct parts with extracted analytical objectives and logic before generating the report. 
We set this baseline to evaluate whether the segmentation pre-processing in \system{} improves the reuse of analytical objectives and logic from the reference report. 

To accommodate the model's output length limitations, we allowed the model to process the (objective, code, text) steps iteratively, one cycle at a time. 
After each cycle, the resulting chart and text were collected as a report segment. 
The model was then prompted with ``Continue'' to produce the next segment. 
In the event of errors (e.g., failed code execution or OpenAI generation errors), we re-generated the output until valid results were obtained. 
By repeating this process, we generated complete data reports for comparison, ensuring that both baselines produced the same number of segments as \system{}'s report. 
Finally, we prompted the model to generate a title for the entire report, as done in \system{}.


\subsubsection{Procedure and tasks. }
Participants began by reviewing three sets of data reports, with each set containing three reports for comparison. They were instructed to thoroughly evaluate and compare the quality of the reports within each set. After carefully reviewing all the reports in a given set, participants completed a five-point Likert scale questionnaire to assess their impressions. The process of reviewing and evaluating each set took approximately 30 minutes.
The questionnaire assesses six dimensions: overall quality, insightfulness, logicality, chart effectiveness, text effectiveness, and consistency.
Specifically, logicality refers to the logical flow and coherence between different sections of the report.
Chart effectiveness and text effectiveness measure how effectively the charts and text convey data insights, respectively.
Consistency assesses whether the content of the charts and text aligns accurately with each other.
After completing the questionnaire, participants engaged in a 20-minute interview to discuss their responses.
All participants were paid with \$7 after the experiment. 


\begin{figure*}[!htb] 
  \centering 
  \includegraphics[width=\linewidth]{figs/evaluation.png}
  \caption{Results of the 5-point Likert scale questionnaire in the comparative study. (a) Overall results summarizing all ratings across the three dataset-reference report pairs. (b-d) Individual results for each dataset-reference report pair.
  }
  \label{fig:evaluation} 
\end{figure*}

\subsection{Results and Discussions}

The results of the comparative study are illustrated in~\autoref{fig:evaluation}. 
Overall, participants provided more positive assessments for the report generated by \system{} in all aspects, as is shown in the overall result (\autoref{fig:evaluation}a). 
We further summarized the important feedback as follows. 

\textbf{Logicality.}
Among all rated aspects, participants consistently rated \system{} highly for logicality. 
This was also one of the most frequently mentioned factors in participants' feedback: 13 out of 18 participants across all three pairs noted that the logicality of \system{} impressed them compared to the baseline reports. 
Specifically, 9 participants from all three reports commented that \system{} reports exhibited a ``clear overall-to-specific structure, making the logic of the report more apparent.'' 
8 participants further highlighted that the paragraphs in \system{} reports ``progress step by step, with a logical and cohesive flow.'' 
P8 pointed out how the report on the cardiovascular disease dataset followed an overall-to-specific structure and went step by step: ``It starts with the prevalence of the disease, then moves to find the leading factors, and then analyzes each key factor in detail.'' 
This progressive logicality also contributed to higher ratings for the overall quality and insightfulness of \system{} reports, with 8 participants indicating that deep analysis was an important factor in assessing insightfulness.

In contrast, both baseline reports received feedback indicating that their narrative structure was ``flat'' and ``lacked a stepwise, progressive flow''. 
13 participants noted that the paragraphs ``had few connections'' and felt ``disjointed'' while reading these reports. 
For example, P2 and P14 observed that baseline reports ``tend to analyze more dimensions, but each factor was not examined in depth''. 
This suggests that the baselines tended to analyze different data attributes or aspects separately, resembling a breadth-first search approach. 
For example, 4 out of 6 participants noted that the baseline report on the Titanic dataset covered different aspects in the first half but ended up repeating analyses. 
This could be due to the report's length (8 segments), which may have caused the model to either forget previous analyses or fail to elaborate deeper on earlier sections.

Some participants also pointed out the trade-off between the depth and breadth of analysis. 
P1 noted that while \system{} generated more in-depth reports, the scope of analysis was narrower. 
For instance, the crime report focused on only a limited set of data attributes. 
This reflects the trade-off between depth and breadth: deeper analysis of certain factors in the same report length may lead to a narrower focus. 
It's akin to a tree with the same number of nodes: a deeper tree has a smaller breadth. 
The design of \system{}, which follows the logical structure of existing reports, does limit the breadth of analysis to some extent. 
This highlights the importance of our system's interactive feature that allows users to see what data attributes have not yet been analyzed and add or delete nodes based on their preferences. 

\textbf{Comparison between the Baselines.} 
Among all three pairs, the scores for GPT without reference report varied across aspects, with no clear standout between the two baselines. 
This indicates that providing a reference report to GPT does not seem to impact the quality of the generated reports significantly. 
However, the results from \system{} were overall better than the baselines. 
Upon closer inspection of the baseline reports, we found that those generated by GPT with the reference report did not clearly inherit the reference report's logical structure. 
As noted earlier, participants found both baseline approaches lacking in logical coherence. 
Since the key difference between GPT with a reference report and \system{} lies in the pre-processed segmentation and one-by-one adaptation approach, this design appears to facilitate the preservation of the logical structure from reference reports.

\textbf{Errors in Chart Design and Text-Chart Consistency.} 
Some reports generated by \system{} are rated lower than certain baseline reports in specific aspects, including chart consistency (Cardiovascular Disease dataset) and chart effectiveness (Titanic dataset). 
Based on the user feedback, the primary reason for these issues is errors in chart design and text-chart consistency. 
While \system{} received higher overall ratings, it still did not resolve the potential errors that could arise from GPT-generated content. 
Specifically, in the cardiovascular disease report, the text described one age group as having the highest cholesterol level, but the chart depicted that same group as having the second highest.
The Titanic dataset report, while praised by participants for its in-depth analysis, included some complex charts, such as \autoref{fig:compararive_error_example.png}, which some participants felt could be improved. 
P15 suggested that \autoref{fig:compararive_error_example.png} could be enhanced by using survival rates on the y-axis rather than relying on two colors to differentiate between those who survived and those who did not.
While \system{} provides manual modification functionality to enable users to improve these issues, there are still notable limitations. We further discuss this limitation in the discussion section.

\begin{figure}[!htb] 
  \centering
  \includegraphics[width=0.5\linewidth]{figs/compararive_error_example.png}
  \caption{
    Example of an inappropriate chart generated in the Titanic dataset report, where the distinction between survivors and non-survivors is made using two colors. 
  }
  \label{fig:compararive_error_example.png}
\end{figure}
\section{Usability Study of \system{} System}
\label{sec:user_study}

To assess the usability of the \system{} system, we further conducted a user-centric experiment. 

\subsection{Study Setup}

In this study, participants were asked to use the \system{} system to generate a data report and provide feedback on their experience.
Specifically, participants were required to generate a report based on a specified dataset and independently select a reference report.
Three datasets were available in total, and participants were randomly assigned to one of them, ensuring an equal distribution of participants across datasets.

\subsubsection{Participants. }
We recruited 12 participants (U1-U12, 9 males and 3 females) from various backgrounds, including data science (5), computer science (3), mathematics (2), sports science (1), and biosystems engineering (1). 
Most participants were familiar with data analysis, with an average self-reported score of 3.25 on a 5-point Likert Scale. 
All participants had experience reading data reports, usually from data news, such as official statistics bureau reports, industry research reports, publicly disclosed financial reports, university survey reports, etc.

\subsubsection{Tasks and Data. }

We used the same three datasets as in the comparative study (\autoref{sec:comparative_study}).
For the reference reports, we selected 8 data reports from various domains, including Health, Education, Security, etc.
Rather than providing a large collection of reports, we chose to sample a smaller set to encourage participants to review each report's information and select one as the reference. 
This approach enabled us to evaluate the effectiveness of our report retrieval method and understand the factors that participants considered when selecting a reference report.

Participants were assigned a specific dataset and tasked with using \system{} to (1) select a reference report from the 8 provided reports, (2) generate a complete report based on the chosen reference, including report segments and headings, and (3) make modifications according to their preferences. 
Modifications included regenerating the model responses, manually editing the analytical objectives and report content, and inserting or removing segments as needed. 
If the participant was unsatisfied with the report selected, s/he could change to another one.
Participants were required to generate a report with at least four segments in total.

\subsubsection{Procedure. }
The entire experiment lasted approximately 70 minutes. 
It began with a 3-minute introduction to the concept of generating a new data report based on a reference report. 
Participants were then presented with a 10-minute tutorial on \system{} interactions, using a COVID-19 report as an example. 
Following the tutorial, they had 3 minutes to explore the system on their own. 

After the exploration phase, participants were tasked with using \system{} to generate a data report based on a specified dataset, which took approximately 35 minutes. 
Specifically, they were required to carefully review the information from the 8 provided reports and select one to serve as the reference report for the analysis of their specified dataset. 
All actions and interactions were logged throughout this process. 
Upon completion of the report, participants were asked to complete a post-study questionnaire. 
This included a 5-point Likert scale assessing both the system's overall usability with the System Usability Scale (SUS) and its individual functionalities. 
Finally, participants took part in a 15-minute feedback interview to provide qualitative insights into their experience. 
All participants were compensated with \$10 for their participation.

\begin{figure*}[!htb] 
  \centering
  \includegraphics[width=\linewidth]{figs/user_study_result.png}
  \caption{The result of the questionnaire in the usability study. }
  \label{fig:user_study_result}
\end{figure*}

\begin{figure*}[!htb] 
  \centering
  \includegraphics[width=\linewidth]{figs/user_involvement.png}
  \caption{The interaction sequences in the usability study, including the assigned datasets (d1-d3), the selected reports and their rankings (a), the interaction sequences during the retrieval (b) and generation (c). }
  \label{fig:user_involvement}
\end{figure*}


\subsection{Results}

The results of the post-study questionnaire are illustrated in~\autoref{fig:user_study_result}.
The user interaction sequence is shown in~\autoref{fig:user_involvement}, highlighting the sequence of actions performed by each participant during the user study. 
We further summarized the important feedback as follows. 

\textbf{Ease of use. }\system{} received an overall SUS score of 88.96, reflecting its high usability. 
The SUS scores for the crime, disease, and Titanic datasets were 86.88, 91.88, and 88.13, respectively. 
Although the Titanic dataset was not highly topic-relevant to the alternative reports provided, the SUS score did not show a significant drop compared to the results of other datasets. 
This suggests that \system{} maintains consistent usability across different scenarios. 

Most participants agree that \system{} is easy to use and does not need to learn a lot of prior knowledge before getting going with the system. 
U4-U6 and U8-U10 noted that the system design and interaction is ``intuitive'' and ``has a low learning curve''. 
Additionally, all participants agreed that the time cost associated with using the system was acceptable. 
U2, U3, and U9 specifically noted that the real-time feedback from the system helped make the processing time feel more manageable.
U9 commented, ``I can see the results being generated gradually, which reassures me that \system{} is steadily progressing. This allows me to stay informed about the model's generation process, making the time spent less noticeable.''

\textbf{Report Retrieval. }
All participants agree that the \system{}'s ranking about the alternative reference reports is suitable. 
We also recorded the users' selected reports, their corresponding ranking in \system{}, and their interactions during the selection process (\autoref{fig:user_involvement}).
The results show that most participants selected the highest-ranked report, and all participants chose one of the top three reports, suggesting the effectiveness of our retrieval mechanism.

Regarding interaction, all participants first reviewed the information in the report list before previewing or selecting alternative reports to assess their suitability.
Notably, most participants did not preview or select all reports. 
Instead, they chose several potential reports to check further based on the information available in the list (\autoref{fig:retrieve}). 
This indicates that the information provided in the report list helped participants make more informed decisions.
Specifically, for the disease and Titanic datasets, participants previewed or selected more reports than for the crime dataset, likely because the disease and Titanic datasets were less similar to the reports in the repository, prompting participants to check more alternatives to find the most appropriate report.

We also interviewed the participants to understand the factors they considered when choosing a reference report and how they ranked these factors.
All participants mentioned field similarity and topic relevance as important criteria, but most (9 out of 12) prioritized field similarity over topic relevance.
They also noted that the list of potential data fields provided in the report list (\autoref{fig:retrieve}) was more useful than the topic information.
U10 explained, ``The topic primarily helps me set expectations for the content of the reference report, such as health or governance, but it doesn't necessarily mean that I can use similar analytical methods. To select a reference to analyze the target dataset, I focus more on the overlapping data fields, such as gender or age, because similar fields are likely to use similar analytical methods.''

Participants also listed other factors they considered when selecting reports.
The most frequently mentioned factor (8 out of 12) was the specific analysis approaches used in the reports, with 4 participants ranking this factor as the highest priority.
Specifically, U11 described his decision process: ``I first judge the analysis methods used in the report, such as trend analysis or group comparison analysis. Then, I assess whether the target dataset is suitable for that approach based on the logical relationships between variables. For example, the Titanic dataset is better suited for group comparison analysis to explore the relationship between survival and other grouped factors.''
U11 further emphasized that considering field similarity is essentially evaluating potential analysis methods because ``similar data fields are likely to be analyzed using the same methods.''
However, he also pointed out that focusing solely on data fields is not comprehensive. ``When data fields differ, it is still possible that the analysis methods in the report are applicable. For example, the Titanic dataset contains a variable about passengers' cabins, which is unique to this dataset but still suitable for group comparison analysis.''

\textbf{User Involvement. }
Most participants agreed that the generated objectives and content were correct and effective~\autoref{fig:user_study_result}.
However, as shown in~\autoref{fig:user_involvement}, all participants used interactions to adjust and modify the results generated by \system{}, including regenerating or modifying objectives and content and manually adding or deleting segments.
This frequent use of interactive features prompted us to further investigate why participants made these modifications.

Modifying objectives was a feature most participants frequently used throughout the process.
However, participants indicated that their modifications were not due to errors or unreasonable results but because the GPT-generated content helped them refine more specified objectives.
For example, U7 commented, ``At first, I didn't know exactly what to analyze, but after seeing the generated analysis objectives, it inspired me to form clearer goals, so I modified them to better align with my intent.''
This trend was also reflected in the content of the modifications, mainly changing the data fields to be analyzed or specifying chart types.
U12, the participant who used this feature the most, expressed a lot of appreciation: ``I just need to make small modifications, like changing a data field or adding a requirement for the chart type, to generate content that better matches my preferences.''
U8 even modified the analysis objective to include statistical tests.
This illustrates the system's flexibility, allowing users to specify detailed requirements in the analysis objectives.

Another feature used by all participants was the ability to manually add or delete segments, which was typically employed later in the analysis process.
Eight participants found this feature ``very necessary'' and found it particularly useful when paired with the feature showing how frequently data fields had been analyzed.
U6 noted, ``The statistics on how often a data field is used for comparisons are helpful. Data fields used frequently align with the analysis topics, while less-used data fields help me identify un-analyzed ones, so I can manually add segments.''
U11 further commented, ``The reference report design in \system{} gives me a warm start and provides inspiration when I'm unsure how to begin my analysis. As I continued, my analysis intent became clearer, so I made modifications. Eventually, when I have a very clear intent, I add segments manually to further refine the analysis.''
This process aligns with our observation of the interaction sequence: Most participants did not make frequent modifications at the beginning.
As their analysis intent became more refined, the interactions involving modifications increased, culminating in manual segment additions and deletions.
This progression from low to high customization reflects users' growing clarity about their analytical goals.

Some participants used the regeneration feature more frequently than modification. 
Two main reasons emerged:
U5 mentioned that he was unsure about his desired results, but he was certain about what he did not want, so he chose to regenerate.
U11 explained that he preferred not to input words manually, so He would regenerate until he found an acceptable result. 
U11 also suggested that \system{} should include more options for customization, such as specifying chart types by options.


\section{Report Quality Assessment: Retrieval Factors vs. Quality of Generated Reports}

Retrieval is a critical initial step in the \system{} pipeline. 
The more similar the reference report's data is to the dataset being analyzed, the fewer adjustments are needed. 
Currently, \system{} considers both topic relevance and field similarity to identify reports with potentially similar data.
However, several critical questions arise: 
\textbf{
How ``similar'' does the reference report's data need to be to the target dataset? 
In addition to topic relevance and field similarity, what other factors might influence the quality of the reports generated by \system{}? }


Therefore, to evaluate \system{}'s performance under varying conditions, we designed a two-round experiment. 
\textbf{In the first round,} we focused on assessing topic relevance and field similarity between different report-dataset combinations. 
Participants were asked to rate 24 combinations (3 datasets $\times$ 8 reports from diverse fields) on topic relevance and field similarity. These ratings allowed us to understand how users perceive the similarity between datasets and reference reports. 
\textbf{In the second round,} based on the results of the first round, we selected 4 reports (from the 8 reports) that varied in topic relevance and field similarity to the 3 datasets.
Note that the downsampling aims to provide a practical sample size for the experiment.
Combining these 4 reports with the 3 datasets, we generated 12 new data reports using \system{}. 
Participants were then asked to evaluate the quality of these reports across multiple dimensions. 
This two-round design enabled us to assess how variations in dataset-reference report similarity affect \system{}'s ability to generate data reports and to identify other factors that may influence \system{}'s performance.


\subsection{Round 1. Topic Relevance and Field Similarity. }
We recruited 13 participants (10 males, 3 females, all familiar with data analysis) to evaluate the topic relevance and field similarity between 24 data-reference report combinations (3 datasets $\times$ 8 reports). 
These datasets and reports were identical to those used in the user study (\autoref{sec:user_study}) and spanned a variety of domains.
During the experiment, participants were first introduced to the concepts of topic relevance and field similarity, as described in \autoref{subsubsec:report_retrieval}. 
They were then asked to rate each data-reference report pair on a 1-5 scale for both aspects. 


The results reveal varying levels of topic relevance and field similarity~\autoref{fig:pre_case_study}. 
Some dataset-report pairs show both high relevance and similarity. 
For example, the 2020-2023 Los Angeles crime dataset is closely aligned with the 2022 Chicago crime report (~\autoref{fig:pre_case_study}a). 
Other pairs exhibited either stronger topic relevance or field similarity. 
For example, the cardiovascular disease dataset was somewhat relevant to a report on opioid use but exhibited lower field similarity (~\autoref{fig:pre_case_study}b): the dataset contained personal health and cardiovascular information, whereas the report focused on trends in opioid use and its geographic distribution.
Conversely, the Titanic dataset and a report on voting patterns in Britain (~\autoref{fig:pre_case_study}c) had no topic relevance but shared demographic data (e.g., gender, age), with the Titanic dataset containing passenger demographics, while the voting report focused on voting tendencies within demographic groups. 
Additionally, some dataset-report pairs exhibited lower topic relevance and lower field similarity, such as the Titanic dataset and the Chicago crime report.



Based on these ratings, we selected four reports covering different levels of topic relevance and field similarity: (1) highly topic-relevant and data-similar, (2) topic-relevant but less data-similar, (3) data-similar but less topic-relevant, and (4) neither topic-relevant nor data-similar. 
These reports, combined with the three datasets, were used to generate 12 new reports, which were then evaluated for quality in the second round of the study.


\begin{figure*}[!htb] 
  \centering 
  \includegraphics[width=0.6\linewidth]{figs/pre_case_study.png}
  \caption{The distribution of the topic relevance and field similarity among the 24 data-reference report combinations. The colored points correspond to the selected 12 data-reference report combinations. }
  \label{fig:pre_case_study} 
\end{figure*}

\begin{figure*}[!htb] 
  \centering 
  \includegraphics[width=0.6\linewidth]{figs/case_study.png}
  \caption{The average ratings of the 12 reports, based on 3 different datasets. (a) The distribution of average ratings among different datasets. (b) The distribution of average ratings on field similarity and topic relevance, respectively. 
  }
  \label{fig:case_study} 
\end{figure*}

\subsection{Round 2. Report Quality under Different Conditions. }
We further recruited 9 participants (6 males and 3 females, all familiar with data analysis) to evaluate the quality of the 12 generated reports. 
Specifically, we showed them the four reports of the same dataset together, repeated three times for three datasets, and required them to rate each report on a 1-5 scale. 

Overall, the ratings of the generated reports are stable, with most reports scoring higher than 3 (~\autoref{fig:case_study}a), indicating that \system{} performs consistently across different cases.
In examining the relationship between the quality of generated reports, topic relevance, and field similarity (~\autoref{fig:case_study}b), field similarity demonstrates a stronger correlation with report quality compared to topic relevance. This aligns with the usability study feedback, where participants ranked field similarity as a more important factor than topic relevance. Specifically, reference reports with higher field similarity scores tend to produce consistently higher-quality reports.

On the other hand, low scores in either field similarity or topic relevance do not necessarily result in poor report quality, but they do lead to greater variability in the results. This stems from \system{}'s mechanism, which makes adjustments to align with the data when discrepancies occur. As a result, \system{} does not rely solely on identifying a highly similar reference report to generate effective outputs.
\section{Limitations and Future Work}
This section delves into the limitations of the current \system{} framework and outlines the potential directions for future research and enhancements. 

\textbf{Accommodate multiple source data and existing reports. }
\system{} currently operates by taking a singular CSV data as input and utilizing an additional report for reference. 
However, data analysts also derive insights from multiple source data to construct a data report~\cite{li2023ai, cao2023dataparticles}. 
Referencing various existing analysis materials instead of a single report to inform a new analysis is also commonplace. 
Future work could incorporate functionalities to accommodate multiple data sources and reference reports. 
Instead of reusing a single report and adjusting the details to match a dataset, integrating multiple sources necessities fully re-formulating a coherent logic from these materials. 
To achieve this goal, a detailed processing pipeline is needed to eliminate redundancies and cluster analytical objectives from multiple materials alongside corresponding data fields, subsequently organizing these objectives into a coherent and logical framework. 

\textbf{Support flexible logical structures. }
\system{} is based on the premise that the subsequent content in the data report may logically depend on the previous content.
Such relation allows for the segmentation of the report in sequential order as a foundation for data analysis reconstruction. 
However, real-world data narratives often exhibit greater flexibility. 
For example, some reports~\footnote{https://www.ons.gov.uk/economy/inflationandpriceindices/bulletins/house\\priceindex/december2023} prioritize key findings by presenting them at the beginning, even though these insights are derived from analyses detailed later in the data report. 
Such cases invert the traditional narrative flow, placing summary conclusions that rely on subsequent content at the forefront. 
Future work could conduct in-depth research on report narrative structures and extract the non-linear dependencies to support such flexible narrative structures. 

\textbf{Beyond static narration and charts.}
\system{} is primarily designed to extract and rebuild the analytical process from the data reports, culminating in the creation of static reports that include both text and chart images. 
The next phase of development could enhance \system{} by integrating interactive elements into these static data reports. 
Features might include clickable charts that drill down into more detailed data and hover-over effects to display additional information. 
To implement such interactive features, a clear linkage between textual content and visual elements~\cite{chen2022crossdata, sultanum2023datatales} should be established and highlighted interactively. 
Moreover, existing research has explored interactive data storytelling techniques that animate the data flow based on data interconnections among narrative components, such as animated unit visualizations and scrollytelling~\cite{cao2023dataparticles, morth2022scrollyvis}. 
\system{} can incorporate such features in the future by extracting the data flow in the generated code and compiling it into animations between report segments. 

\textbf{Support for data pre-processing. } 
\system{} is designed to analyze data directly, progressing seamlessly from one analytical objective to another. 
Therefore, the input data needs to be clean and pre-processed, which does not need additional data wrangling and is ready for analysis. 
While GPT models are capable of generating data wrangling code~\cite{huang2023nl2rigel}, the comprehensive detection of data issues and the application of appropriate cleaning and wrangling operations remain a challenging problem. 
Accurately identifying inconsistencies, handling missing values, and normalizing datasets require significant domain expertise and can vary greatly depending on the dataset's structure and context.
Given these complexities, we consider data pre-processing to be outside the current scope of \system{}. 
However, we recognize its importance as an integral step in the analytical workflow. 
In the future, \system{} could incorporate an additional module dedicated to data pre-processing before generating analytical objectives and their corresponding content. 
This module could leverage existing techniques for automated data cleaning and wrangling, improving the system's ability to handle raw or semi-structured datasets and expanding its applicability across a broader range of real-world scenarios.

\textbf{Verify the AI-generated results. }
As the analytical objective, code, and textual content are all generated by LLMs, a key problem is how to ensure the model's result is correct. 
As is mentioned in the comparative study, \system{}'s result also has inappropriate choices or errors in chart design and text-chart consistency. 
To enable users to check the result of the model and fix it, \system{} provides some interface or functionalities to provide transparency and support modification interaction, including displaying the deduction process of analytical objectives and code, marking subjective statements not supported by data, enabling manual modification and segment insertion/deletion. 
However, some participants in the usability study (U2, U9, U12) also commented that it is not easy for users to identify the errors in the text and code, as they need to align between the generated text, charts, and the underlying code and identify the inconsistencies manually. 
They suggested incorporating mechanisms to verify the AI-generated results more effectively. A potential improvement could involve highlighting the correspondence between the generated code, text, and the associated charts, ensuring that key data points mentioned in the text are accurately reflected in the visualizations~\cite{chen2022crossdata}. For example, linking textual references (e.g., ``the highest cholesterol level'') to the specific data points on the chart could reduce inconsistencies and enhance user trust in the results.

\textbf{Smart retrieval mechanism. }
While the ranking mechanism in \system{}, based on topic relevance and field similarity, was effective in helping participants identify suitable reference reports, several limitations were observed during the study.
First, although topic relevance and field similarity were both recognized as important factors, interviews revealed that most participants prioritized field similarity over topic relevance. To address this, it may be beneficial to assign greater weight to field similarity in the ranking algorithm or make these weighting factors transparent, allowing users to customize their importance based on specific needs.
Second, these two factors may not fully determine the suitability of a reference report. As U11 noted in the usability study, even when data fields differ, the analytical methods used in a report can still be applicable. The current retrieval mechanism risks overlooking reports with valuable analytical methodologies that do not align closely with the target dataset in terms of topic relevance or field similarity.
Future improvements might include integrating the relevance of analytical methods as an additional ranking criterion. 


\section{Conclusion}

This paper introduces \system{}~\footnote{https://\system{}2024.github.io/}, an LLM-driven method leveraging previous data reports as references to generate new reports with LLMs. 
We first retrieve the most related report from the repository as the reference report. 
To deduce the implicit analysis workflow from the existing data report, we formulate it as a sequence of interdependent analysis rounds, including posing analytical intents, performing analysis operations, and obtaining data insights. 
Through such formulation, we divide the existing report into a series of segments with extracted analytical intents and dependencies for reproducing. 
Furthermore, we conduct a preliminary study on similar data reports to survey what to reuse and what to rectify in the existing report. 
Based on the findings, we design a pipeline to generate tailored analytical intents, analysis operations, and report contents through reusing the existing report and addressing the inconsistencies with the new data. 
An interactive interface is designed to allow users to observe real-time outputs, insert analytical intents, and modify report content and structure. 
The effectiveness of \system{} is evaluated through comparative and user studies. 

% \section{Introduction}
% ACM's consolidated article template, introduced in 2017, provides a
% consistent \LaTeX\ style for use across ACM publications, and
% incorporates accessibility and metadata-extraction functionality
% necessary for future Digital Library endeavors. Numerous ACM and
% SIG-specific \LaTeX\ templates have been examined, and their unique
% features incorporated into this single new template.

% If you are new to publishing with ACM, this document is a valuable
% guide to the process of preparing your work for publication. If you
% have published with ACM before, this document provides insight and
% instruction into more recent changes to the article template.

% The ``\verb|acmart|'' document class can be used to prepare articles
% for any ACM publication --- conference or journal, and for any stage
% of publication, from review to final ``camera-ready'' copy, to the
% author's own version, with {\itshape very} few changes to the source.

% \section{Template Overview}
% As noted in the introduction, the ``\verb|acmart|'' document class can
% be used to prepare many different kinds of documentation --- a
% double-anonymous initial submission of a full-length technical paper, a
% two-page SIGGRAPH Emerging Technologies abstract, a ``camera-ready''
% journal article, a SIGCHI Extended Abstract, and more --- all by
% selecting the appropriate {\itshape template style} and {\itshape
%   template parameters}.

% This document will explain the major features of the document
% class. For further information, the {\itshape \LaTeX\ User's Guide} is
% available from
% \url{https://www.acm.org/publications/proceedings-template}.

% \subsection{Template Styles}

% The primary parameter given to the ``\verb|acmart|'' document class is
% the {\itshape template style} which corresponds to the kind of publication
% or SIG publishing the work. This parameter is enclosed in square
% brackets and is a part of the {\verb|documentclass|} command:
% \begin{verbatim}
%   \documentclass[STYLE]{acmart}
% \end{verbatim}

% Journals use one of three template styles. All but three ACM journals
% use the {\verb|acmsmall|} template style:
% \begin{itemize}
% \item {\texttt{acmsmall}}: The default journal template style.
% \item {\texttt{acmlarge}}: Used by JOCCH and TAP.
% \item {\texttt{acmtog}}: Used by TOG.
% \end{itemize}

% The majority of conference proceedings documentation will use the {\verb|acmconf|} template style.
% \begin{itemize}
% \item {\texttt{sigconf}}: The default proceedings template style.
% \item{\texttt{sigchi}}: Used for SIGCHI conference articles.
% \item{\texttt{sigplan}}: Used for SIGPLAN conference articles.
% \end{itemize}

% \subsection{Template Parameters}

% In addition to specifying the {\itshape template style} to be used in
% formatting your work, there are a number of {\itshape template parameters}
% which modify some part of the applied template style. A complete list
% of these parameters can be found in the {\itshape \LaTeX\ User's Guide.}

% Frequently-used parameters, or combinations of parameters, include:
% \begin{itemize}
% \item {\texttt{anonymous,review}}: Suitable for a ``double-anonymous''
%   conference submission. Anonymizes the work and includes line
%   numbers. Use with the \texttt{\acmSubmissionID} command to print the
%   submission's unique ID on each page of the work.
% \item{\texttt{authorversion}}: Produces a version of the work suitable
%   for posting by the author.
% \item{\texttt{screen}}: Produces colored hyperlinks.
% \end{itemize}

% This document uses the following string as the first command in the
% source file:
% \begin{verbatim}
% \documentclass[sigconf,authordraft]{acmart}
% \end{verbatim}

% \section{Modifications}

% Modifying the template --- including but not limited to: adjusting
% margins, typeface sizes, line spacing, paragraph and list definitions,
% and the use of the \verb|\vspace| command to manually adjust the
% vertical spacing between elements of your work --- is not allowed.

% {\bfseries Your document will be returned to you for revision if
%   modifications are discovered.}

% \section{Typefaces}

% The ``\verb|acmart|'' document class requires the use of the
% ``Libertine'' typeface family. Your \TeX\ installation should include
% this set of packages. Please do not substitute other typefaces. The
% ``\verb|lmodern|'' and ``\verb|ltimes|'' packages should not be used,
% as they will override the built-in typeface families.

% \section{Title Information}

% The title of your work should use capital letters appropriately -
% \url{https://capitalizemytitle.com/} has useful rules for
% capitalization. Use the {\verb|title|} command to define the title of
% your work. If your work has a subtitle, define it with the
% {\verb|subtitle|} command.  Do not insert line breaks in your title.

% If your title is lengthy, you must define a short version to be used
% in the page headers, to prevent overlapping text. The \verb|title|
% command has a ``short title'' parameter:
% \begin{verbatim}
%   \title[short title]{full title}
% \end{verbatim}

% \section{Authors and Affiliations}

% Each author must be defined separately for accurate metadata
% identification.  As an exception, multiple authors may share one
% affiliation. Authors' names should not be abbreviated; use full first
% names wherever possible. Include authors' e-mail addresses whenever
% possible.

% Grouping authors' names or e-mail addresses, or providing an ``e-mail
% alias,'' as shown below, is not acceptable:
% \begin{verbatim}
%   \author{Brooke Aster, David Mehldau}
%   \email{dave,judy,steve@university.edu}
%   \email{firstname.lastname@phillips.org}
% \end{verbatim}

% The \verb|authornote| and \verb|authornotemark| commands allow a note
% to apply to multiple authors --- for example, if the first two authors
% of an article contributed equally to the work.

% If your author list is lengthy, you must define a shortened version of
% the list of authors to be used in the page headers, to prevent
% overlapping text. The following command should be placed just after
% the last \verb|\author{}| definition:
% \begin{verbatim}
%   \renewcommand{\shortauthors}{McCartney, et al.}
% \end{verbatim}
% Omitting this command will force the use of a concatenated list of all
% of the authors' names, which may result in overlapping text in the
% page headers.

% The article template's documentation, available at
% \url{https://www.acm.org/publications/proceedings-template}, has a
% complete explanation of these commands and tips for their effective
% use.

% Note that authors' addresses are mandatory for journal articles.

% \section{Rights Information}

% Authors of any work published by ACM will need to complete a rights
% form. Depending on the kind of work, and the rights management choice
% made by the author, this may be copyright transfer, permission,
% license, or an OA (open access) agreement.

% Regardless of the rights management choice, the author will receive a
% copy of the completed rights form once it has been submitted. This
% form contains \LaTeX\ commands that must be copied into the source
% document. When the document source is compiled, these commands and
% their parameters add formatted text to several areas of the final
% document:
% \begin{itemize}
% \item the ``ACM Reference Format'' text on the first page.
% \item the ``rights management'' text on the first page.
% \item the conference information in the page header(s).
% \end{itemize}

% Rights information is unique to the work; if you are preparing several
% works for an event, make sure to use the correct set of commands with
% each of the works.

% The ACM Reference Format text is required for all articles over one
% page in length, and is optional for one-page articles (abstracts).

% \section{CCS Concepts and User-Defined Keywords}

% Two elements of the ``acmart'' document class provide powerful
% taxonomic tools for you to help readers find your work in an online
% search.

% The ACM Computing Classification System ---
% \url{https://www.acm.org/publications/class-2012} --- is a set of
% classifiers and concepts that describe the computing
% discipline. Authors can select entries from this classification
% system, via \url{https://dl.acm.org/ccs/ccs.cfm}, and generate the
% commands to be included in the \LaTeX\ source.

% User-defined keywords are a comma-separated list of words and phrases
% of the authors' choosing, providing a more flexible way of describing
% the research being presented.

% CCS concepts and user-defined keywords are required for for all
% articles over two pages in length, and are optional for one- and
% two-page articles (or abstracts).

% \section{Sectioning Commands}

% Your work should use standard \LaTeX\ sectioning commands:
% \verb|section|, \verb|subsection|, \verb|subsubsection|, and
% \verb|paragraph|. They should be numbered; do not remove the numbering
% from the commands.

% Simulating a sectioning command by setting the first word or words of
% a paragraph in boldface or italicized text is {\bfseries not allowed.}

% \section{Tables}

% The ``\verb|acmart|'' document class includes the ``\verb|booktabs|''
% package --- \url{https://ctan.org/pkg/booktabs} --- for preparing
% high-quality tables.

% Table captions are placed {\itshape above} the table.

% Because tables cannot be split across pages, the best placement for
% them is typically the top of the page nearest their initial cite.  To
% ensure this proper ``floating'' placement of tables, use the
% environment \textbf{table} to enclose the table's contents and the
% table caption.  The contents of the table itself must go in the
% \textbf{tabular} environment, to be aligned properly in rows and
% columns, with the desired horizontal and vertical rules.  Again,
% detailed instructions on \textbf{tabular} material are found in the
% \textit{\LaTeX\ User's Guide}.

% Immediately following this sentence is the point at which
% Table~\ref{tab:freq} is included in the input file; compare the
% placement of the table here with the table in the printed output of
% this document.

% \begin{table}
%   \caption{Frequency of Special Characters}
%   \label{tab:freq}
%   \begin{tabular}{ccl}
%     \toprule
%     Non-English or Math&Frequency&Comments\\
%     \midrule
%     \O & 1 in 1,000& For Swedish names\\
%     $\pi$ & 1 in 5& Common in math\\
%     \$ & 4 in 5 & Used in business\\
%     $\Psi^2_1$ & 1 in 40,000& Unexplained usage\\
%   \bottomrule
% \end{tabular}
% \end{table}

% To set a wider table, which takes up the whole width of the page's
% live area, use the environment \textbf{table*} to enclose the table's
% contents and the table caption.  As with a single-column table, this
% wide table will ``float'' to a location deemed more
% desirable. Immediately following this sentence is the point at which
% Table~\ref{tab:commands} is included in the input file; again, it is
% instructive to compare the placement of the table here with the table
% in the printed output of this document.

% \begin{table*}
%   \caption{Some Typical Commands}
%   \label{tab:commands}
%   \begin{tabular}{ccl}
%     \toprule
%     Command &A Number & Comments\\
%     \midrule
%     \texttt{{\char'134}author} & 100& Author \\
%     \texttt{{\char'134}table}& 300 & For tables\\
%     \texttt{{\char'134}table*}& 400& For wider tables\\
%     \bottomrule
%   \end{tabular}
% \end{table*}

% Always use midrule to separate table header rows from data rows, and
% use it only for this purpose. This enables assistive technologies to
% recognise table headers and support their users in navigating tables
% more easily.

% \section{Math Equations}
% You may want to display math equations in three distinct styles:
% inline, numbered or non-numbered display.  Each of the three are
% discussed in the next sections.

% \subsection{Inline (In-text) Equations}
% A formula that appears in the running text is called an inline or
% in-text formula.  It is produced by the \textbf{math} environment,
% which can be invoked with the usual
% \texttt{{\char'134}begin\,\ldots{\char'134}end} construction or with
% the short form \texttt{\$\,\ldots\$}. You can use any of the symbols
% and structures, from $\alpha$ to $\omega$, available in
% \LaTeX~\cite{Lamport:LaTeX}; this section will simply show a few
% examples of in-text equations in context. Notice how this equation:
% \begin{math}
%   \lim_{n\rightarrow \infty}x=0
% \end{math},
% set here in in-line math style, looks slightly different when
% set in display style.  (See next section).

% \subsection{Display Equations}
% A numbered display equation---one set off by vertical space from the
% text and centered horizontally---is produced by the \textbf{equation}
% environment. An unnumbered display equation is produced by the
% \textbf{displaymath} environment.

% Again, in either environment, you can use any of the symbols and
% structures available in \LaTeX\@; this section will just give a couple
% of examples of display equations in context.  First, consider the
% equation, shown as an inline equation above:
% \begin{equation}
%   \lim_{n\rightarrow \infty}x=0
% \end{equation}
% Notice how it is formatted somewhat differently in
% the \textbf{displaymath}
% environment.  Now, we'll enter an unnumbered equation:
% \begin{displaymath}
%   \sum_{i=0}^{\infty} x + 1
% \end{displaymath}
% and follow it with another numbered equation:
% \begin{equation}
%   \sum_{i=0}^{\infty}x_i=\int_{0}^{\pi+2} f
% \end{equation}
% just to demonstrate \LaTeX's able handling of numbering.

% \section{Figures}

% The ``\verb|figure|'' environment should be used for figures. One or
% more images can be placed within a figure. If your figure contains
% third-party material, you must clearly identify it as such, as shown
% in the example below.
% \begin{figure}[h]
%   \centering
%   \includegraphics[width=\linewidth]{sample-franklin}
%   \caption{1907 Franklin Model D roadster. Photograph by Harris \&
%     Ewing, Inc. [Public domain], via Wikimedia
%     Commons. (\url{https://goo.gl/VLCRBB}).}
%   \Description{A woman and a girl in white dresses sit in an open car.}
% \end{figure}

% Your figures should contain a caption which describes the figure to
% the reader.

% Figure captions are placed {\itshape below} the figure.

% Every figure should also have a figure description unless it is purely
% decorative. These descriptions convey what’s in the image to someone
% who cannot see it. They are also used by search engine crawlers for
% indexing images, and when images cannot be loaded.

% A figure description must be unformatted plain text less than 2000
% characters long (including spaces).  {\bfseries Figure descriptions
%   should not repeat the figure caption – their purpose is to capture
%   important information that is not already provided in the caption or
%   the main text of the paper.} For figures that convey important and
% complex new information, a short text description may not be
% adequate. More complex alternative descriptions can be placed in an
% appendix and referenced in a short figure description. For example,
% provide a data table capturing the information in a bar chart, or a
% structured list representing a graph.  For additional information
% regarding how best to write figure descriptions and why doing this is
% so important, please see
% \url{https://www.acm.org/publications/taps/describing-figures/}.

% \subsection{The ``Teaser Figure''}

% A ``teaser figure'' is an image, or set of images in one figure, that
% are placed after all author and affiliation information, and before
% the body of the article, spanning the page. If you wish to have such a
% figure in your article, place the command immediately before the
% \verb|\maketitle| command:
% \begin{verbatim}
%   \begin{teaserfigure}
%     \includegraphics[width=\textwidth]{sampleteaser}
%     \caption{figure caption}
%     \Description{figure description}
%   \end{teaserfigure}
% \end{verbatim}

% \section{Citations and Bibliographies}

% The use of \BibTeX\ for the preparation and formatting of one's
% references is strongly recommended. Authors' names should be complete
% --- use full first names (``Donald E. Knuth'') not initials
% (``D. E. Knuth'') --- and the salient identifying features of a
% reference should be included: title, year, volume, number, pages,
% article DOI, etc.

% The bibliography is included in your source document with these two
% commands, placed just before the \verb|\end{document}| command:
% \begin{verbatim}
%   \bibliographystyle{ACM-Reference-Format}
%   \bibliography{bibfile}
% \end{verbatim}
% where ``\verb|bibfile|'' is the name, without the ``\verb|.bib|''
% suffix, of the \BibTeX\ file.

% Citations and references are numbered by default. A small number of
% ACM publications have citations and references formatted in the
% ``author year'' style; for these exceptions, please include this
% command in the {\bfseries preamble} (before the command
% ``\verb|\begin{document}|'') of your \LaTeX\ source:
% \begin{verbatim}
%   \citestyle{acmauthoryear}
% \end{verbatim}


%   Some examples.  A paginated journal article \cite{Abril07}, an
%   enumerated journal article \cite{Cohen07}, a reference to an entire
%   issue \cite{JCohen96}, a monograph (whole book) \cite{Kosiur01}, a
%   monograph/whole book in a series (see 2a in spec. document)
%   \cite{Harel79}, a divisible-book such as an anthology or compilation
%   \cite{Editor00} followed by the same example, however we only output
%   the series if the volume number is given \cite{Editor00a} (so
%   Editor00a's series should NOT be present since it has no vol. no.),
%   a chapter in a divisible book \cite{Spector90}, a chapter in a
%   divisible book in a series \cite{Douglass98}, a multi-volume work as
%   book \cite{Knuth97}, a couple of articles in a proceedings (of a
%   conference, symposium, workshop for example) (paginated proceedings
%   article) \cite{Andler79, Hagerup1993}, a proceedings article with
%   all possible elements \cite{Smith10}, an example of an enumerated
%   proceedings article \cite{VanGundy07}, an informally published work
%   \cite{Harel78}, a couple of preprints \cite{Bornmann2019,
%     AnzarootPBM14}, a doctoral dissertation \cite{Clarkson85}, a
%   master's thesis: \cite{anisi03}, an online document / world wide web
%   resource \cite{Thornburg01, Ablamowicz07, Poker06}, a video game
%   (Case 1) \cite{Obama08} and (Case 2) \cite{Novak03} and \cite{Lee05}
%   and (Case 3) a patent \cite{JoeScientist001}, work accepted for
%   publication \cite{rous08}, 'YYYYb'-test for prolific author
%   \cite{SaeediMEJ10} and \cite{SaeediJETC10}. Other cites might
%   contain 'duplicate' DOI and URLs (some SIAM articles)
%   \cite{Kirschmer:2010:AEI:1958016.1958018}. Boris / Barbara Beeton:
%   multi-volume works as books \cite{MR781536} and \cite{MR781537}. A
%   couple of citations with DOIs:
%   \cite{2004:ITE:1009386.1010128,Kirschmer:2010:AEI:1958016.1958018}. Online
%   citations: \cite{TUGInstmem, Thornburg01, CTANacmart}.
%   Artifacts: \cite{R} and \cite{UMassCitations}.

% \section{Acknowledgments}

% Identification of funding sources and other support, and thanks to
% individuals and groups that assisted in the research and the
% preparation of the work should be included in an acknowledgment
% section, which is placed just before the reference section in your
% document.

% This section has a special environment:
% \begin{verbatim}
%   \begin{acks}
%   ...
%   \end{acks}
% \end{verbatim}
% so that the information contained therein can be more easily collected
% during the article metadata extraction phase, and to ensure
% consistency in the spelling of the section heading.

% Authors should not prepare this section as a numbered or unnumbered {\verb|\section|}; please use the ``{\verb|acks|}'' environment.

% \section{Appendices}

% If your work needs an appendix, add it before the
% ``\verb|\end{document}|'' command at the conclusion of your source
% document.

% Start the appendix with the ``\verb|appendix|'' command:
% \begin{verbatim}
%   \appendix
% \end{verbatim}
% and note that in the appendix, sections are lettered, not
% numbered. This document has two appendices, demonstrating the section
% and subsection identification method.

% \section{Multi-language papers}

% Papers may be written in languages other than English or include
% titles, subtitles, keywords and abstracts in different languages (as a
% rule, a paper in a language other than English should include an
% English title and an English abstract).  Use \verb|language=...| for
% every language used in the paper.  The last language indicated is the
% main language of the paper.  For example, a French paper with
% additional titles and abstracts in English and German may start with
% the following command
% \begin{verbatim}
% \documentclass[sigconf, language=english, language=german,
%                language=french]{acmart}
% \end{verbatim}

% The title, subtitle, keywords and abstract will be typeset in the main
% language of the paper.  The commands \verb|\translatedXXX|, \verb|XXX|
% begin title, subtitle and keywords, can be used to set these elements
% in the other languages.  The environment \verb|translatedabstract| is
% used to set the translation of the abstract.  These commands and
% environment have a mandatory first argument: the language of the
% second argument.  See \verb|sample-sigconf-i13n.tex| file for examples
% of their usage.

% \section{SIGCHI Extended Abstracts}

% The ``\verb|sigchi-a|'' template style (available only in \LaTeX\ and
% not in Word) produces a landscape-orientation formatted article, with
% a wide left margin. Three environments are available for use with the
% ``\verb|sigchi-a|'' template style, and produce formatted output in
% the margin:
% \begin{description}
% \item[\texttt{sidebar}:]  Place formatted text in the margin.
% \item[\texttt{marginfigure}:] Place a figure in the margin.
% \item[\texttt{margintable}:] Place a table in the margin.
% \end{description}

%%
%% The acknowledgments section is defined using the "acks" environment
%% (and NOT an unnumbered section). This ensures the proper
%% identification of the section in the article metadata, and the
%% consistent spelling of the heading.
% \begin{acks}
% To Robert, for the bagels and explaining CMYK and color spaces.
% \end{acks}

%%
%% The next two lines define the bibliography style to be used, and
%% the bibliography file.
\bibliographystyle{ACM-Reference-Format}
\bibliography{ref}


%%
%% If your work has an appendix, this is the place to put it.
% \appendix

% \section{Research Methods}

% \subsection{Part One}

% Lorem ipsum dolor sit amet, consectetur adipiscing elit. Morbi
% malesuada, quam in pulvinar varius, metus nunc fermentum urna, id
% sollicitudin purus odio sit amet enim. Aliquam ullamcorper eu ipsum
% vel mollis. Curabitur quis dictum nisl. Phasellus vel semper risus, et
% lacinia dolor. Integer ultricies commodo sem nec semper.

% \subsection{Part Two}

% Etiam commodo feugiat nisl pulvinar pellentesque. Etiam auctor sodales
% ligula, non varius nibh pulvinar semper. Suspendisse nec lectus non
% ipsum convallis congue hendrerit vitae sapien. Donec at laoreet
% eros. Vivamus non purus placerat, scelerisque diam eu, cursus
% ante. Etiam aliquam tortor auctor efficitur mattis.

% \section{Online Resources}

% Nam id fermentum dui. Suspendisse sagittis tortor a nulla mollis, in
% pulvinar ex pretium. Sed interdum orci quis metus euismod, et sagittis
% enim maximus. Vestibulum gravida massa ut felis suscipit
% congue. Quisque mattis elit a risus ultrices commodo venenatis eget
% dui. Etiam sagittis eleifend elementum.

% Nam interdum magna at lectus dignissim, ac dignissim lorem
% rhoncus. Maecenas eu arcu ac neque placerat aliquam. Nunc pulvinar
% massa et mattis lacinia.

\end{document}
\endinput
%%
%% End of file `sample-sigconf-authordraft.tex'.
