\section{Limitations and Future Work}
This section delves into the limitations of the current \system{} framework and outlines the potential directions for future research and enhancements. 

\textbf{Accommodate multiple source data and existing reports. }
\system{} currently operates by taking a singular CSV data as input and utilizing an additional report for reference. 
However, data analysts also derive insights from multiple source data to construct a data report~\cite{li2023ai, cao2023dataparticles}. 
Referencing various existing analysis materials instead of a single report to inform a new analysis is also commonplace. 
Future work could incorporate functionalities to accommodate multiple data sources and reference reports. 
Instead of reusing a single report and adjusting the details to match a dataset, integrating multiple sources necessities fully re-formulating a coherent logic from these materials. 
To achieve this goal, a detailed processing pipeline is needed to eliminate redundancies and cluster analytical objectives from multiple materials alongside corresponding data fields, subsequently organizing these objectives into a coherent and logical framework. 

\textbf{Support flexible logical structures. }
\system{} is based on the premise that the subsequent content in the data report may logically depend on the previous content.
Such relation allows for the segmentation of the report in sequential order as a foundation for data analysis reconstruction. 
However, real-world data narratives often exhibit greater flexibility. 
For example, some reports~\footnote{https://www.ons.gov.uk/economy/inflationandpriceindices/bulletins/house\\priceindex/december2023} prioritize key findings by presenting them at the beginning, even though these insights are derived from analyses detailed later in the data report. 
Such cases invert the traditional narrative flow, placing summary conclusions that rely on subsequent content at the forefront. 
Future work could conduct in-depth research on report narrative structures and extract the non-linear dependencies to support such flexible narrative structures. 

\textbf{Beyond static narration and charts.}
\system{} is primarily designed to extract and rebuild the analytical process from the data reports, culminating in the creation of static reports that include both text and chart images. 
The next phase of development could enhance \system{} by integrating interactive elements into these static data reports. 
Features might include clickable charts that drill down into more detailed data and hover-over effects to display additional information. 
To implement such interactive features, a clear linkage between textual content and visual elements~\cite{chen2022crossdata, sultanum2023datatales} should be established and highlighted interactively. 
Moreover, existing research has explored interactive data storytelling techniques that animate the data flow based on data interconnections among narrative components, such as animated unit visualizations and scrollytelling~\cite{cao2023dataparticles, morth2022scrollyvis}. 
\system{} can incorporate such features in the future by extracting the data flow in the generated code and compiling it into animations between report segments. 

\textbf{Support for data pre-processing. } 
\system{} is designed to analyze data directly, progressing seamlessly from one analytical objective to another. 
Therefore, the input data needs to be clean and pre-processed, which does not need additional data wrangling and is ready for analysis. 
While GPT models are capable of generating data wrangling code~\cite{huang2023nl2rigel}, the comprehensive detection of data issues and the application of appropriate cleaning and wrangling operations remain a challenging problem. 
Accurately identifying inconsistencies, handling missing values, and normalizing datasets require significant domain expertise and can vary greatly depending on the dataset's structure and context.
Given these complexities, we consider data pre-processing to be outside the current scope of \system{}. 
However, we recognize its importance as an integral step in the analytical workflow. 
In the future, \system{} could incorporate an additional module dedicated to data pre-processing before generating analytical objectives and their corresponding content. 
This module could leverage existing techniques for automated data cleaning and wrangling, improving the system's ability to handle raw or semi-structured datasets and expanding its applicability across a broader range of real-world scenarios.

\textbf{Verify the AI-generated results. }
As the analytical objective, code, and textual content are all generated by LLMs, a key problem is how to ensure the model's result is correct. 
As is mentioned in the comparative study, \system{}'s result also has inappropriate choices or errors in chart design and text-chart consistency. 
To enable users to check the result of the model and fix it, \system{} provides some interface or functionalities to provide transparency and support modification interaction, including displaying the deduction process of analytical objectives and code, marking subjective statements not supported by data, enabling manual modification and segment insertion/deletion. 
However, some participants in the usability study (U2, U9, U12) also commented that it is not easy for users to identify the errors in the text and code, as they need to align between the generated text, charts, and the underlying code and identify the inconsistencies manually. 
They suggested incorporating mechanisms to verify the AI-generated results more effectively. A potential improvement could involve highlighting the correspondence between the generated code, text, and the associated charts, ensuring that key data points mentioned in the text are accurately reflected in the visualizations~\cite{chen2022crossdata}. For example, linking textual references (e.g., ``the highest cholesterol level'') to the specific data points on the chart could reduce inconsistencies and enhance user trust in the results.

\textbf{Smart retrieval mechanism. }
While the ranking mechanism in \system{}, based on topic relevance and field similarity, was effective in helping participants identify suitable reference reports, several limitations were observed during the study.
First, although topic relevance and field similarity were both recognized as important factors, interviews revealed that most participants prioritized field similarity over topic relevance. To address this, it may be beneficial to assign greater weight to field similarity in the ranking algorithm or make these weighting factors transparent, allowing users to customize their importance based on specific needs.
Second, these two factors may not fully determine the suitability of a reference report. As U11 noted in the usability study, even when data fields differ, the analytical methods used in a report can still be applicable. The current retrieval mechanism risks overlooking reports with valuable analytical methodologies that do not align closely with the target dataset in terms of topic relevance or field similarity.
Future improvements might include integrating the relevance of analytical methods as an additional ranking criterion. 


\section{Conclusion}

This paper introduces \system{}~\footnote{https://\system{}2024.github.io/}, an LLM-driven method leveraging previous data reports as references to generate new reports with LLMs. 
We first retrieve the most related report from the repository as the reference report. 
To deduce the implicit analysis workflow from the existing data report, we formulate it as a sequence of interdependent analysis rounds, including posing analytical intents, performing analysis operations, and obtaining data insights. 
Through such formulation, we divide the existing report into a series of segments with extracted analytical intents and dependencies for reproducing. 
Furthermore, we conduct a preliminary study on similar data reports to survey what to reuse and what to rectify in the existing report. 
Based on the findings, we design a pipeline to generate tailored analytical intents, analysis operations, and report contents through reusing the existing report and addressing the inconsistencies with the new data. 
An interactive interface is designed to allow users to observe real-time outputs, insert analytical intents, and modify report content and structure. 
The effectiveness of \system{} is evaluated through comparative and user studies. 