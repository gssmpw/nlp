\section{Interface}
\label{sec:usage_scenario}

We incorporate \system{} with an interactive interface. 
The interface consists of four views: data view, dependency view, content view, and generation view. 
We introduce the interface through a usage scenario. 

\textbf{Pre-processing. }
A data analyst is tasked with creating a data report for a dataset detailing crime in Los Angeles from 2020 to 2023~\footnote{https://catalog.data.gov/dataset/crime-data-from-2020-to-present}. 
S/he decides to use \system{} to assist in creating a data report with the Los Angeles data. 
The analyst begins by uploading the Los Angeles crime dataset(~\autoref{fig:interface}a). 
\system{} then pre-process the dataset. 
After the pre-processing, the analyst can view the dataset information(~\autoref{fig:interface}b), including the file name, overall description, and data field information. 

Next, the analyst opens the report repository, which displays the available reference reports ranked by relevance to the dataset (~\autoref{fig:retrieve}). 
The analyst selects the most relevant report, a 2022 Chicago crime report~\footnote{https://www.illinoispolicy.org/chicago-crime-spikes-in-2022-but-first-drop-in-murder-since-pandemic/}. 
The report is divided into six interdependent segments, each corresponding to an extracted analytical objective and report content(~\autoref{fig:interface}d). 
The analysis mode means the system is ready for analysis. 
With the segmentation and extracted objectives, the analyst can proceed with generating a new report.


\textbf{Segment execution. }
The analyst starts with the first segment (~\autoref{fig:interface}e), focusing on crime trends in Chicago from 2018 to 2022. 
S/he clicks ``generate'' to adapt this analytical objective to the Los Angeles data. 
\system{} modifies the objective to ``How has the total number of crimes changed annually from 2020 to 2023 in Los Angeles?'', which is aligned with the context and scope of the Los Angeles data (~\autoref{fig:interface}h). 
Following this, the analyst clicks ``generate'' again to conduct the analysis. 
\system{} generates the code and executes it (~\autoref{fig:interface}i), obtains the transformed data and chart, and generates the narratives to describe the findings. 
The resulting chart shows the number of crimes and changed percentages each year (~\autoref{fig:interface}j). 
The narrative describes the overall increase and the year-by-year changes, especially the peak in 2022 (~\autoref{fig:interface}k). 
The analyst considers such results reasonable and applies them (~\autoref{fig:interface}g). 

\begin{figure}[!htb] 
  \centering
  \includegraphics[width=0.5\linewidth]{figs/retrieve.png}
  \caption{
    A demonstration of retrieving reports from the report repository.
  }
  \label{fig:retrieve}
\end{figure}


\begin{figure}[!htb] 
  \centering
  \includegraphics[width=0.5\linewidth]{figs/segment_2.png}
  \caption{
    The result of generating the second segment. 
  }
  \label{fig:segment_2}
\end{figure}

\begin{figure}[!htb] 
  \centering
  \includegraphics[width=0.5\linewidth]{figs/delete.png}
  \caption{
  A demonstration of an analytical objective that fails to be corrected and needs to be removed. 
  }
  \label{fig:delete}
\end{figure}

\begin{figure}[!htb] 
  \centering
  \includegraphics[width=0.5\linewidth]{figs/add.png}
  \caption{
  A demonstration of inserting an analysis segment and generating a new analytical objective. 
  }
  \label{fig:add}
\end{figure}

\begin{figure}[!htb] 
  \centering
  \includegraphics[width=0.5\linewidth]{figs/title.png}
  \caption{
    A demonstration of generating new titles and adding section structures. 
  }
  \label{fig:title}
\end{figure}

\begin{figure}[!htb] 
  \centering
  \includegraphics[width=0.5\linewidth]{figs/highlight.png}
  \caption{
    A demonstration of highlighting the sentences that serve non-data analysis purposes. 
  }
  \label{fig:highlight}
\end{figure}

The second segment aims to explore the crime types that drove the increasing trend, which depends on the first segment with a cause-effect logic (~\autoref{fig:segment_2}a). 
As the previous segment results in an overall increasing trend as well, \system{} inherits the logic and corrects the objective to match the Los Angeles data context (~\autoref{fig:segment_2}b). 
Different from the reference report which only shows the changes in 2018 and 2022, \system{} calculates the top ten crime types contributing to the uptick and visualizes them through a bar chart for a clear comparison of cumulative numbers. 
Such varied design choices may stem from the narrative description in the Chicago report, which describes both the overall change and last year's change (~\autoref{fig:segment_2}a1). 
As the analysis progresses, the analyst explores the subsequent two segments, including the decreased crime types and their changes from 2022 to 2023. 

\textbf{Objective removal. }
After the segments above, the Chicago report moves to another analytical objective on homicide trends from 2018 to 2022, which stems directly from the data. 
Similarly, the analyst applies the tailored objective and report content generated by \system{}, which analyses the trend of homicide in LA from 2020 to 2023. 
However, the next segment of the Chicago report generalizes its analysis to include homicide trends from 2000 to 2022(~\autoref{fig:delete}). 
To inherit such a generalization logic on the generated report with the LA data, external data sources are needed, as the provided dataset covers only 2020 to 2023. 
Identified such a case, \system{} marks the analytical objective as ``none'' and visually indicates this error through a red node. 
Consequently, the analyst removes this node.

\textbf{Objective insertion. }
After generating these segments, the analyst finds that there are multiple unused data fields. 
Based on the previous analysis of the homicide trends, the analyst inserts a new node focusing on the time by selecting the ``Time Occ'' field and applying a ``similarity'' logic for parallel analysis (~\autoref{fig:add}). 
Therefore, \system{} forms a new analytical objective to analyze the time distribution of arson. 
%The analysis unveils a tendency for victims to be younger adults, peaking in their mid-to-late 20s and declining after the age of 50. 
% Further investigation into the timing of arsons reveals that most occur during the evening.

\textbf{Structure organization. }
Upon concluding the data analysis, the analyst shifts focus to report structuring, crafting titles, and organizing sections for clarity (~\autoref{fig:title}). 
S/he generates the title and obtains a title that describes the rise in thefts patterns. 
The analyst then organizes the content into two coherent sections. 
The complete data report is detailed in the supplementary materials provided.

\textbf{Highlight. }
For report texts, \system{} highlights sentences that serve non-data analysis purposes  (~\autoref{fig:highlight}). Since these sentences may lack data support, highlighting them can alert users to selectively receive this information.