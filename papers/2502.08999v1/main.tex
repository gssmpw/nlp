
%\hyphenation{op-tical net-works semi-conduc-tor IEEE-Xplore}

\documentclass[journal,a4paper]{IEEEtran}
\usepackage{color}
\usepackage{amsmath,amsfonts}
\usepackage{algorithmic}
%\usepackage{algorithm}
\usepackage{array}
%\usepackage[caption=false,font=normalsize,labelfont=sf,textfont=sf]{subfig}
\usepackage{textcomp}
\usepackage{stfloats}
\usepackage{url}
\usepackage{verbatim}
\usepackage{graphicx}
\usepackage{cite}
\usepackage{amssymb}
\usepackage{graphicx}
\usepackage[ruled]{algorithm2e}
\usepackage{makecell}
\usepackage{multicol}
\usepackage{subfigure}
\usepackage{color}
\usepackage{graphicx}
\makeatletter
\newcommand{\removelatexerror}{\let\@latex@error\@gobble}
\makeatother
\usepackage[justification=centering]{caption}
\pagestyle{empty}

\begin{document}

\title{Semantic Communication Meets Heterogeneous Network: Emerging Trends, Opportunities, and Challenges}

\author{Guhan Zheng, Qiang Ni, Aryan Kaushik, and Lixia Yang



\thanks{G. Zheng and Q. Ni are with the School of Computing and Communications, Lancaster University, LA1 4WA, UK (Email: \{g.zheng2, q.ni\}@lancaster.ac.uk).}
\thanks{Aryan Kaushik is with the Department of Computing and Mathematics, Manchester Metropolitan University, M1 5GD, U.K (e-mail: a.kaushik@mmu.ac.uk).}
\thanks{L. Yang is with the School of Electronics and Information, Anhui University, 230601, China. (Email: lixiayang@yeah.net).}
}

\maketitle
\thispagestyle{empty}

\begin{abstract}
Recent developments in machine learning (ML) techniques enable users to extract, transmit, and reproduce information semantics via ML-based semantic communication (SemCom). This significantly increases network spectral efficiency and transmission robustness. In the network, the semantic encoders and decoders among various users, based on ML, however, require collaborative updating according to new transmission tasks. The various heterogeneous characteristics of most networks in turn introduce emerging but unique challenges for semantic codec updating that are different from other general ML model updating. In this article, we first overview the key components of the SemCom system. We then discuss the unique challenges associated with semantic codec updates in heterogeneous networks. Accordingly, we point out a potential framework and discuss the pros and cons thereof. Finally, several future research directions are also discussed.  
\end{abstract}



\section{Introduction}
\IEEEPARstart{T}{he} next-generation communication is expected to present revolutionary breakthroughs in global interconnectivity, intelligent services, and the ultimate communications experience. Leveraging the advancement of key technologies such as artificial intelligence (AI), reconfigurable intelligence surfaces (RIS), and integrated sensing and communication (ISAC), future communication is expected to deliver higher data rates, lower latency, higher energy efficiency, and massive connectivity capabilities \cite{aryan}. These improvements enable the network to support cutting-edge applications including immersive XR experiences, holographic communications, and the artificial intelligence of things (AIoT), driving the evolution of the information society into a smart society.

Nevertheless, future communications still face transmission challenges in terms of spectrum efficiency and robustness. On the one hand, with the exponential growth of communication requirements, spectrum resources are becoming increasingly constrained. The conventional construction of communication systems has mainly been rooted in Shannon's information theory, with the study of communications centered on enhancing the accuracy and efficiency of symbol transmission between the transmitter and receiver \cite{CognitiveSemanticCommunication}. This focus has yielded significant advancements, but as communication technologies evolve, the theoretical Shannon limit is increasingly being approached \cite{shannon}. This makes it difficult to further improve transmission efficiency. On the other hand, in the presence of sophisticated dynamic environments and diversified application requirements, the reliability and adaptability of conventional communication systems remain to be upgraded. In this context, it is difficult to meet the requirements of future networks by relying only on the conventional communication paradigm. To address these limitations, semantic communication (SemCom) has emerged as a promising paradigm.

In typical Semantic Communication (SemCom) systems, conventional codecs are replaced by semantic codecs that leverage machine learning (ML) techniques. The semantic encoder at the transmitter extracts the underlying meaning behind the transmitted information, rather than simply encoding raw symbols. It then transmits only this extracted meaning. On the receiver, the semantic decoder reconstructs the intended message based on the received semantic representation \cite{qin}. By prioritizing the transmission of meaning rather than raw data, SemCom significantly reduces the volume of transmitted information, thereby improving spectral efficiency. Furthermore, SemCom enhances transmission robustness by reducing dependency on precise symbol-level accuracy, making the communication system more resilient to channel impairments and noise.


Predictably, the integration of Semantic Communication (SemCom) into future wireless networks has the potential to significantly enhance the quality of network services. However, a key challenge in SemCom lies in the task-oriented nature of semantic codecs. Specifically, these codecs must be updated whenever they encounter new types of tasks to ensure efficient and accurate transmission \cite{qin}. In point-to-point communication, such updates require the active participation of both the transmitter and the receiver to collaboratively synchronize their semantic encoder and decoder. In the network, this challenge becomes even more complex. Unlike a simple two-party scenario, network-wide updates involve multiple users employing SemCom, each relying on a shared but evolving semantic understanding. Coordinating these updates across numerous parties significantly increases the complexity of maintaining a consistent and up-to-date semantic encoder-decoder model throughout the network. Managing these updates efficiently is crucial for ensuring seamless communication and maintaining the robustness of SemCom-enabled networks.

Several studies have focused on the challenge of collaborative semantic codec update among multi-parties and identified it as one of the primary challenges for SemCom applications \cite{commag}. Distributed learning approaches look most promising for user collaboration on codec updates as the privacy of the data required for user updates can be preserved. The classical federated learning (FL) approach for semantic codec update is first proposed for general networks to prevent data privacy \cite{5,6}. According to the different network scenarios' limitations, various schemes based on FL improvement are also presented \cite{uav,tgcn,iotj}.

\begin{figure*} 
\centering
\includegraphics[width=0.69\textwidth]{system.png} 
\caption{The traditional SemCom framework.} 
\end{figure*}

In addition, growing attention is being given to the heterogeneous characteristics of network users during the semantic codec updating process \cite{commag,heto1,heto2}. This consideration is particularly crucial in diverse network environments, such as non-terrestrial networks and vehicular networks, where significant heterogeneity exists in terms of codec models, infrastructure, and computational capabilities. These variations introduce additional complexities in achieving efficient and synchronized codec updates across different network scenarios. Existing methods designed for general machine learning (ML) models, however, are not always directly applicable to semantic codec updating in heterogeneous networks. Unlike conventional ML-based models, semantic codecs must ensure synchronization and compatibility between two communicating parties, i.e., the transmitter and the receiver, to maintain effective task-oriented communication. This dual-party dependency introduces unique challenges, including the need for real-time coordination, adaptive learning mechanisms, and robustness against heterogeneous network conditions. Addressing these concerns is essential for ensuring seamless and reliable semantic communication across diverse and evolving network environments.

Despite these efforts, the study and development of semantic codec updates in heterogeneous networks is still at an early stage. Many technical challenges still need to be addressed to enable efficient and scalable deployment. Therefore, it is crucial to investigate these challenges associated with semantic codec updates in heterogeneous networks to help pave the way for more robust, efficient, and scalable solutions and facilitate the widespread adoption of SemCom across diverse communication scenarios.


In this article, we first review the key components in classical SemCom systems. We then discuss the main challenges of semantic codec updating in the heterogeneous network. Next, we propose a heterogeneous semantic codec updating framework as a potential solution. The open issues are also discussed following that.




\section{Basic Networking Architecture}
As shown in Fig. 1, we review two classical ML-based SemCom components in heterogeneous networks.



\subsection{Semantic Knowledge Base}

The semantic knowledge base stores background knowledge and contextual information relevant to the SemCom task. It is utilized to help semantic encoders and decoders understand and correctly process semantic information. 

The immediacy and variety of tasks dictate that semantic knowledge bases between users are often heterogeneous and need to be synchronized to adapt to new transmission tasks. Further, the codec also needs to be updated to accommodate the new semantic information. Improper synchronization leads to semantic decoding errors, which may affect the quality of the communication in terms of the completion efficiency. 

\subsection{Semantic Encoder and Decoder}
The semantic encoder is on the transmitter. Its main function is to extract semantic information from the input data. It transforms the raw data into a highly compressed semantic representation via ML. This process focuses on the content and meaning of the information rather than the bit-level symbolic representation of traditional communications.

The semantic decoder is on the receiver. Its main function is to reconstruct the semantic content from the received semantic information. This process relies on the semantic decoder's synchronization of the semantic encoder. If there are differences between the semantic codec, this may lead to decoding errors. The encoder and decoder from two parties hence need to be updated collaboratively in real-time and dynamically.


\subsection{Semantic Channel}
The semantic channel is a medium for information transmission, similar to the physical channel in traditional communication. The concern of the semantic channel, however, is not only the transmission of signals but also semantic integrity and fidelity. In practical communications, this may involve noise in the physical channel as well as noise at the semantic level (e.g., misinterpretation of semantics at the receiver). The robustness of the transmission is usually better than in conventional communications due to the small quantity of data and the fault tolerance of the semantic information.




\section{Key Challenges}
The key challenges in updating the semantic codec model in heterogeneous networks include system heterogeneity, codec heterogeneity, personalized one-to-many model, and data heterogeneity.

\subsection{System Heterogeneity}

In a heterogeneous network, there are many kinds of users transmitting their information through SemCom e.g., autonomous-driving vehicles, portable carrying devices, and edge clouds. They have different computational capabilities (CPU cycles) and communication transmission capabilities (transmission power, bandwidth, etc.). Furthermore, their communication capabilities are constantly changing due to noise during transmission. These constitute heterogeneity of systems that are difficult to be mutually understood, considering privacy and variability.


\subsection{Codec Heterogeneity}
SemCom extracts the meaning of the transmitted information via ML-based models and performs transmission. Therefore, semantic codec models need to be stored and cost a substantial amount of computational resources to use. Confronted with a new transmission task that requires updating the semantic model, the transmitter and the receiver need to collaborate on the update. A key challenge in this process arises from the inherent heterogeneity of users’ initial codec models. For instance, some users may utilize a large, pre-trained encoder optimized for high transmission accuracy, while others may rely on a smaller, lightweight model designed for faster computation with lower resource consumption. This disparity makes it difficult to establish a generally applicable global semantic codec model for new transmission tasks, as users have varying constraints related to storage capacity, computational power, pre-training cost, and accuracy requirements. Users tend to use their personal initial model. Hence, in updating the codecs, considering the codec heterogeneity in the training process is essential.


\subsection{Data Heterogeneity}
Since the semantic codec model is based on the ML model, the users are required to have privacy training data during the update process. This poses a common challenge in ML co-training, i.e., the size, distribution, and other characteristics of these datasets vary. In other words, users have non-identical and independently distributed (Non-IID) datasets. The presence of data heterogeneity affects the transmission accuracy and convergence of semantic codecs updated by users.


\subsection{Personalized One-to-Many Model}

In case users collaborate on updating semantic codecs, it is impractical for one user to tailor the appropriate, personalized encoders or decoders for all other users. We take the downlink as an example. The base station utilizes the semantic encoder to transmit the information to personal devices employed with semantic decoders. From a cost-efficiency perspective, maintaining and updating separate semantic encoders for a large number of personal devices imposes a substantial computational burden on the base station. Moreover, storing multiple distinct encoders significantly increases storage overhead, making such an approach infeasible in large-scale networks. Base stations thus aim to deploy a single, efficient semantic encoder capable of extracting and transmitting information that can be interpreted by various personal devices. Each personal device, in turn, utilizes a personalized heterogeneous semantic decoder to reconstruct the transmitted message based on its specific contextual understanding. This paradigm introduces a personalized one-to-many challenge, where a single (homogeneous) encoder/decoder needs to effectively serve a diverse (heterogeneous) set of decoders/encoders, each with unique semantic interpretations and requirements.


\begin{figure}
\centering
\includegraphics[width=0.45\textwidth]{progress.png} 
\caption{Procedures of the proposed framework for heterogeneous SemCom codec updating.} 
\end{figure}
\section{A Heterogeneous Semantic Codec Updating Framework}
In this section, we propose a novel framework, a heterogeneous semantic codec updating framework, for addressing the aforementioned challenges. Fig. 2 illustrates an epoch of the updating process and the specific process is as follows:

\begin{itemize}

\item Users are constantly assessing each other's trust in order to judge the reliability of their data and computing capabilities. This assessment can be based on historical interaction records, behavioral patterns, or blockchain technology to ensure that the entire system operates in a trustworthy environment and prevents interference from malicious users.

\item In case an update of the semantic codec is required, the user community negotiates and elects a central node (e.g., edge cloud and roadside unit) according to trust value, communication cost, and computing cost. This node is responsible for the same global encoder or decoder model aggregation (based on the one-to-many), ensuring efficient information interaction while reducing communication overhead.

\item The users provide an equal quantity of random data to the selected central node to construct the global dataset while ensuring that the quantity of provided data satisfies the constraint of the minimum allowable quantity of leaked data. 

\item The central node aggregates random data received and assigns global data to individual users. Users perform local training using local computing resources to evaluate the training convergence speed and accuracy related to decoder/encoder model heterogeneity on different devices. This understanding step helps to subsequently optimize the model aggregation strategy and improve the overall training efficiency.

\item Each user performs personalized training based on local data and global data for the entire codec. In each training epoch, they upload the updated encoder/decoder model to the central node and keep the personalized decoder/encoder model.

\item The central node performs weighted aggregation of uploaded encoder/decoder models based on the understanding of the heterogeneity of user models and the heterogeneity in data distribution rather than information detail. Adaptive re-weight aggregation strategies are used to improve the model generalization capability. Eventually, the central node distributes the optimized global encoder/decoder models to individual users.

\end{itemize}



In the aforementioned process, we tested based on existing SemCom codec models from \cite{tccn} (Model 1) and \cite{jsac} (Model 2), as shown in Fig. 3. We assume that users use different SemCom encoder models for semantic extraction of a small but different number of transmitted images from CIFAR 10 \cite{cifar} dataset. One user uses a decoder model based on Model 2 for information restoration, and the transmission accuracy is the mean square error (MSE) \cite{jsac}. 

In this heterogeneous network, It can be seen from Fig. 3 (a) that if the same part of the model (i.e., decoder) is aggregated generally, Model 1 has lower convergence and transmission accuracy than Model 2. Mutual understanding via our proposed framework, reassigning aggregation weights without knowing the users' specifics, Models 1 and 2 both get the convergence speed and training accuracy of the codec updating improved, as illustrated in Figs. 3 (b) and (c), respectively.



\section{Open Research Topics}
Although SemCom is expected to be used in future heterogeneous networks due to its high spectral efficiency and robustness, the following emerging and unique issues hinder its widespread application.


\subsection{Discrimination of Heterogeneous Model and System}
Due to the heterogeneity of the semantic codec models, the traditional model aggregation approaches are not applicable to the collaborative updating of semantic codecs. Leaving heterogeneous models for local training causes incomplete information between users. Each user does not know each other's data information, model information, and system information at the time of training. In the case of one user being slow to update due to model or system reasons, other users may exclude that user from joining the update due to considerations such as accuracy or update efficiency. This is even though that user may have more data information and potentially a higher transmission accuracy after completed updates because the other users are not informed. This prevents that user from deploying semantic communication properly due to discrimination during updating, thus necessitating a high communication load via employing conventional communication paradigam.

\subsection{Fairness of Heterogeneous Data}
Datasets of users are generally considered as data information is Non-IID. Yet, inconsistencies in the data ‘class’ not only influence updates to transmission accuracy and convergence, but also lead to an unfair SemCom after updating, since the user only gets its personalized encoder or decoder model and cannot determine the others’ decoder or encoder model. For instance, in the United Kingdom, a minority ethnic user performs collaborative semantic codec updates with other users in a community of Britons. His training information is related to his ethnicity, e.g., pictures and language. This information, however, is a minority in this community, making the updated semantic codec less accurate for him than for the local Briton community users. This creates an unfair update for the minority ethnic user, who is also involved in the update.


\subsection{Privacy of Personalized One-to-Many Model}

In the network, there is a one-to-many characteristic of semantic models, the semantic encoders and decoders that should form an integral ML model are distributed across different users. During the updating of the SemCom model, if a malicious user can get hold of the semantic encoder/decoder model of one normal user, it can not only affect the coding/decoding of that user but also cause the decoding/coding methods of other users to be deduced. This leads to leakage of transmitted information after SemCom model updating is complete, even though each user's semantic model is individual and heterogeneous. Hence, the semantic codec model is related to the accuracy of signal transmission, users need to consider, not only transmission data privacy and system information privacy but also crucial to consider semantic codec model privacy while employing SemCom. 


\subsection{Security of Active Defense Paradigm Changes}

SemCom achieves efficient information transmission by shifting part of the communication workload to the computation workload. However, this new paradigm enables data, model, and transmission security to interact in a mutually affecting way. The utility function of attackers attacking a SemCom network changes accordingly. Furthermore, the defender's operation and maintenance cost carries on increasing. This is because the defender has to consider not only the computational security but also the communication security. The incomplete information and complexity caused by heterogeneous scenarios exacerbate the difficulty of active defense for the defender. This coupling problem hence changes the traditional active defense analysis posture.



\subsection{Dynamics of Heterogeneous Network}

Several studies have discussed the integration of semantic communication into practical networks, which are often dynamic, e.g., vehicular networks\cite{iotj} and satellite networks\cite{jsac}. In such networks for semantic codec updating, there exist new users joining and veteran users leaving. The locations of users are also changed. It thus is urgently needed to address the above issues such as privacy in heterogeneous networks, while taking into account the dynamics of the user during semantic encoder updates.

\subsection{Enthusiasm of semantic codec Updating}
Facing a brand new task, users in the network associate to perform semantic codec updates to transmit spectral efficiency as well as robustness enhancements for that task. Furthermore, due to heterogeneity, each user is not equally enthusiastic in the same sense as to whether or not to transmit for this task using SemCom, i.e., whether or not to participate in the update. Nevertheless, if some users participate in the update and some refuse, it does not simply affect a single user. Because communication is mutual, users of two different communication paradigms cannot communicate. Moreover, network spectrum resources are limited, users who refuse to update take up more spectrum resources, affecting all users in the network.



\begin{figure}[!htbp]
\centering
\subfigure[{General aggregation}]{
\begin{minipage}[b]{0.38\textwidth}
    \includegraphics[width=2.8in]{1.png}
\end{minipage}}\\
\subfigure[{Model 1}]{
\begin{minipage}[b]{0.38\textwidth}
    \includegraphics[width=2.8in]{M1.png}
\end{minipage}}\\
\subfigure[{Model 2}]{
\begin{minipage}[b]{0.38\textwidth}
    \includegraphics[width=2.8in]{M2.png}
\end{minipage}}
\caption{{Updating convergence speed and transmission accuracy before/after applying our framework.}}
\end{figure}



\section{Conslusion}
This article surveyed the deployment of SemCom in heterogeneous networks, highlighting the challenges during updating semantic codecs to facilitate the widespread adoption of SemCom. Moreover, a heterogeneous semantic codec updating framework was proposed and may constitute a promising candidate for supporting SemCom in heterogeneous networks. Lastly, we also discussed some open issues, such as discrimination, fairness, privacy, network dynamics, and updating enthusiasm. We hope that the challenges and opportunities in this article will pave the way to advance SemCom for large-scale applications on the network in the future.

\section{Acknowledgments}
This work was supported in part by the Open Networks Ecosystem Competition Western Open Radio Access Network (O-RAN) Deployment (ONE WORD) Project.

\ifCLASSOPTIONcaptionsoff
  \newpage
\fi


\begin{thebibliography}{1}
\bibitem{aryan}
R. Singh, A. Kaushik, W. Shin, M. Di Renzo, V. Sciancalepore, D. Lee, H. Sasaki, A. Shojaeifard and O. A. Dobre, "Towards 6G Evolution: Three Enhancements, Three Innovations, and Three Major Challenges,” arXiv2402:10781, Feb. 2024.


\bibitem{CognitiveSemanticCommunication} Y. Li, F. Zhou, L. Yuan, Q. Wu and N. Al-Dhahir, "Cognitive Semantic Communication: A New Communication Paradigm for 6G," \emph{IEEE Communications Magazine}.

\bibitem{shannon} Y. Xiao, G. Shi, Y. Li, W. Saad and H. V. Poor, "Toward Self-Learning Edge Intelligence in 6G," \emph{IEEE Communications Magazine}, vol. 58, no. 12, pp. 34-40, December 2020.

\bibitem{qin} Z. Qin, X. Tao, J. Lu, and G. Y. Li, “Semantic communications: Principles and challenges,” Jun. 2022, [online] Available: http:// arXiv:2201.01389.

\bibitem{commag} G. Zheng, Q. Ni, K. Navaie, H. Pervaiz and C. Zarakovitis, "A Distributed Learning Architecture for Semantic Communication in Autonomous Driving Networks for Task Offloading," \emph{IEEE Communications Magazine}, vol. 61, no. 11, pp. 64-68, November 2023.


\bibitem{5}
G. Shi, Y. Xiao, Y. Li and X. Xie, "From Semantic Communication to Semantic-Aware Networking: Model, Architecture, and Open Problems," \emph{IEEE Communications Magazine}, vol. 59, no. 8, pp. 44-50, August 2021.

\bibitem{6}
Z. Qin, G. Y. Li and H. Ye, "Federated Learning and Wireless Communications," \emph{IEEE Wireless Communications}, vol. 28, no. 5, pp. 134-140, October 2021.


\bibitem{uav}
J. Xu, H. Yao, R. Zhang, T. Mai, S. Huang and S. Guo, "Federated Learning Powered Semantic Communication for UAV Swarm Cooperation," \emph{IEEE Wireless Communications}, vol. 31, no. 4, pp. 140-146, August 2024

\bibitem{tgcn}
G. Zheng, Q. Ni, K. Navaie, H. Pervaiz, A. Kaushik and C. Zarakovitis, "Energy-Efficient Semantic Communication for Aerial-Aided Edge Networks," \emph{IEEE Transactions on Green Communications and Networking}, vol. 8, no. 4, pp. 1742-1751, Dec. 2024

\bibitem{iotj}
G. Zheng et al., "Mobility-Aware Split-Federated With Transfer Learning for Vehicular Semantic Communication Networks," \emph{IEEE Internet of Things Journal}, vol. 11, no. 10, pp. 17237-17248, 15 May, 2024.


\bibitem{heto1}
Y. Peng, F. Jiang, L. Dong, K. Wang, and K. Yang, “Personalized Federated Learning for Generative AI-Assisted Semantic Communications,” Oct. 2024, [online] Available: https://arxiv.org/abs/2410.02450.

\bibitem{heto2}
Y. Wang, W. Ni, W. Yi, X. Xu, P. Zhang, and A. Nallanathan, “Federated Contrastive Learning for Personalized Semantic Communication,” Jun. 2024, [online] Available: https://arxiv.org/abs/2406.09182.



\bibitem{tccn}
E. Bourtsoulatze, D. Burth Kurka and D. Gündüz, "Deep Joint Source-Channel Coding for Wireless Image Transmission," \emph{IEEE Transactions on Cognitive Communications and Networking}, vol. 5, no. 3, pp. 567-579, Sept. 2019.


\bibitem{jsac}
G. Zheng, Q. Ni, K. Navaie and H. Pervaiz, "Semantic Communication in Satellite-Borne Edge Cloud Network for Computation Offloading," \emph{IEEE Journal on Selected Areas in Communications}, vol. 42, no. 5, pp. 1145-1158, May 2024.

\bibitem{cifar}
A. Krizhevsky, ‘‘Learning multiple layers of features from tiny images,’’ Univ. Toronto, Toronto, ON, Canada, Tech. Rep. TR-2009, 2009

\end{thebibliography}

\textbf{Guhan Zheng} is a senior research associate at the School of Computing and Communications, Lancaster University, Lancaster, U.K. His research interests include non-terrestrial networks, semantic communication, and game theories.


\textbf{Qiang Ni} is a professor at the School of Computing and Communications, Lancaster University, Lancaster, U.K. His research areas include future generation communications and networking, including green communications/networking, millimeter-wave wireless, cognitive radio systems, 5G/6G, SDN, cloud networks, edge computing, dispersed computing, Internet of Things, cyber physical systems, artificial intelligence/machine learning, and vehicular networks.

\textbf{Aryan Kaushik} is an associate professor at Manchester Met, UK, since 2024. Previously he has been with University of Sussex, University College London, University of Edinburgh, HKUST, and held visiting appointments at Imperial College London, University of Bologna, University of Luxembourg, Athena RC, and Beihang University. He has been External PhD Examiner internationally such as at UC3M, Spain (2023). He has been an Invited Panel Member at the UK EPSRC ICT Prioritisation Panel in 2023, Chair of IEEE ComSoc ETI on ESIT, Co-Chair of IEEE SIG on AITNTN, Editor of five books by Elsevier and Wiley, and several journals such as IEEE Transactions on Communications, IEEE Transactions on Mobile Computing, IEEE COMST, IEEE OJCOMS (Best Editor Award 2024 and 2023), IEEE Communications Letters (Exemplary Editor 2024 and 2023), IEEE IoT Magazine, IEEE CTN, and several special issues. He has been invited/keynote and tutorial speaker for over 90 academic and industry events, and conferences globally such as at IEEE ICC 2024-25, IEEE GLOBECOM 2023-24, etc., and chairing in Organizing and Technical Program Committees of over 10 flagship IEEE conferences such as IEEE ICC 2024-26, etc., and has been General Chair of over 25 workshops such as at IEEE ICC 2024-25, etc.


\textbf{Lixia Yang} is a professor at the School of Electronic and Information, Anhui University, Hefei, China. His current research interests include signal analysis and design, electromagnetic wave propagation and scattering in complex media and objects, computational electromagnetics, and inverse scattering.



\end{document}


