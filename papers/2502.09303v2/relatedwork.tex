\section{Related Work}
Numerous studies investigated balancing the model training performance (e.g., model accuracy) and cost-effectiveness for FL. For example, \cite{ref15,ref16,ref17} study model compression, gradient quantization, and coding techniques to reduce the model aggregation overhead. Also, \cite{ref18,ref19,ref20} conduct communication/computation resource allocation for FL. Further, \cite{ref21,ref22} aim to reduce the required global aggregation rounds. These works study FL with direct client-to-cloud communications, which can cause increased latency, energy consumption, and congestion over the backhaul links.

To address the above discussed issues, HFL was proposed by \textit{Liu et al.} \cite{ref8}, introducing an intermediate aggregation layer (e.g., formed by ESs) to reduce the frequency of direct communications between clients and the remote cloud. \textit {Feng et al.} \cite{ref24} further optimized subcarrier assignment, transmit power of clients, and computation resources in HFL. \textit {Lin et al.} \cite{ref25} proposed control algorithms for delay-aware HFL, obtaining policies to mitigate communication latency in HFL. 

Considering the constraints of limited resources in wireless networks, developing the joint client selection and the C2E association strategy for HFL emerges as a viable approach to improve model training efficiency, which has attracted a great deal of research attention. For example, \textit {Luo et al.} \cite{ref5} conducted resource allocation and C2E association for HFL, enabling great potentials in low latency and energy-efficient FL. \textit{Deng et al.} \cite{ref10} minimized the total communication cost required for model learning with a target accuracy, by making decisions on edge aggregator selection and node-edge associations under the HFL framework. \textit{Su et al.} \cite{ref26} developed an online learning-based client selection algorithm for ESs, based on empirical learning results, reducing the cumulative delay on computation and communication. \textit{Qu et al.} \cite{ref27} observed the context of local computing and transmission of client-ES pairs, making client selection decisions to maximize the network operator’s utility, under a given budget. 

Besides HFL, several novel architectures have been proposed to address the high communication overhead caused by frequent interactions with the remote server. For example, \textit{Guo et al.} \cite{ref53} proposed a hybrid local stochastic gradient descent (HL-SGD) algorithm, which leverages the availability of fast D2D links to accelerate convergence and reduce training time under non-IID data distributions. However, similar to traditional FL and HFL, the central entity in HL-SGD may still become a bottleneck and suffer from a single point of failure, raising concerns about fault tolerance and scalability. To tackle this issue, a cooperative federated edge learning (CFEL) \cite{ref54,ref55} framework was introduced, which exploits multiple aggregators and eliminates single points of failure, making it more scalable than prior FL frameworks. \textit{Castiglia et al.} \cite{ref54} proposed multi-level local SGD (MLL-SGD) in a two-tier communication network with heterogeneous workers with IID data distributions. \textit{Zhang et al.} \cite{ref55} designed a novel federated optimization method called cooperative edge-based federated averaging (CE-FedAvg), which efficiently learns a shared global model over the collective dataset with non-IID data of all edge devices, under the orchestration of a distributed network of cooperative edge servers.

The above discussed studies have made valuable contributions in the realm of model training by enhancing client participation and optimizing network orchestration, bringing a strong foundation for improving the efficiency and effectiveness of FL systems. However, many of these works either assume that clients are consistently available for model training or do not fully account for the decision-making overhead associated with client recruitment and network orchestration. Specifically, the time required to select clients and establish their association with ESs are often overlooked. Addressing these factors is crucial for the practical implementation of HFL, particularly in dynamic and resource-constrained environments. This work is thus interested in an unexplored research angle and contributes to the on-going research by proposing stagewise decision-making for HFL under intermittent client availability. Interestingly, it employs two complementary plans for client selection and C2E association executed at different timescales to ensure a seamless and efficient model training process.
%\vspace{-3.5mm}