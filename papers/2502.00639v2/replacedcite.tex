\section{Related works}
\paragraph{Diffusion Probabilistic Models.} Denoising Diffusion Model ____ is one of the strongest models for generation tasks, especially for visual generation ____. Extensive research has been conducted from theoretical and empirical perspectives ____. It has achieved phenomenal success in muti-modality generation, including image, video, audio, and 3D shapes. The DM is trained on enormous images and videos from the internet ____. Empowered by modern architecture ____, it has powerful learning capability for Pixel Space Distribution. 

\paragraph{Alignment and Post-training.} After pre-training to learn the distribution of the targeted modality ____, post-training is conducted to align the model toward specific preferences or tune the model to optimize a particular objective. RL has been utilized to align the foundation models toward various objectives ____. Supervised learning can also be applied to the post-training phase ____, either optimizing an equivalent objective ____ or directly differentiating the reward model ____. For DM, most methods use a neural reward model to align the pre-trained model, and there has been a continual effort to design better reward models ____.

\paragraph{Forward Learning Methods.} 
\textcolor{black}{After extensive exploration of model training using forward inferences only ____, forward-learning methods ____ based on stochastic gradient estimation have recently emerged as a promising alternative to classical BP for large-scale machine learning problems ____. Subsequent research ____ has further optimized computational and storage overhead from various perspectives, achieving greater efficiency.}


\begin{figure*}[!th]
    \centering
    \includegraphics[width=\textwidth]{figures/RLR.pdf}
    \vspace{-0.5cm}
    \caption{The computation paradigm of the RLR optimizer.}
    \label{fig:algdiagram}
    \vspace{-0.3cm}
\end{figure*}