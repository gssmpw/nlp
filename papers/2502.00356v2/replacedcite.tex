\section{Related Work}
%Applications
%\textcolor{red}{Zipei's comment: Isn't this paragraph redundant?}

%\textcolor{red}{The modified Bessel function of the second kind, \( K_\nu(x) \) or \textsc{BesselK}, is a fundamental tool in various fields of science and engineering due to its unique mathematical properties____. These properties make \textsc{BesselK} well-suited for solving problems in differential equations____, and physical systems____. A notable characteristic of \textsc{BesselK} is its \textit{rapid decay at infinity}, meaning the function diminishes exponentially as \( x \to \infty \). This property makes it invaluable for modeling boundary conditions or phenomena where the solution must vanish far from the origin. Key applications include heat conduction____, which models temperature distributions in cylindrical or spherical systems; wave propagation____, where it describes damped wave behavior in cylindrical waveguides; and stochastic processes____, which represent probability densities in systems where long-distance contributions are negligible.}

The modified Bessel function of the second kind, \( K_\nu(x) \) or \textsc{BesselK}, arises in the separation of variables solutions of partial differential equations____. A key feature of \textsc{BesselK} is its \textit{rapid decay at infinity}. The function diminishes exponentially as \( x \to \infty \), providing a basis for solutions that vanish at large distances. Applications include heat conduction____, wave propagation____, and stochastic processes____.


\textsc{BesselK} also arises in the widely used Mat\'{e}rn kernel, a covariance function for Gaussian processes. The Mat\'{e}rn kernel finds extensive applications across various fields. In spatial statistics, it is used to model spatial correlations____. In machine learning, it is particularly valuable for Gaussian process regression and Bayesian optimization____. In signal processing, it helps capture correlations in time series and spectral analysis____. 
%These applications underscore the versatility and scientific significance of \textsc{BesselK} in both theoretical and applied domains.
 
At least six different methods for \textsc{BesselK} can be identified in the literature; however, not all of these methods are adopted in existing tools and libraries. (1) \textit{Series Expansions}____: \( K_\nu(x) \) can be expressed as an infinite series in powers of \( x \), which is truncated to a finite number of terms when \( x \) is small. Most existing libraries utilize series expansions for small \( x \), including MATLAB, Mathematica, GSL, SciPy, and Maple. (2) \textit{Continued Fractions}____: \( K_\nu(x) \) can be represented as an infinite fraction with a recursive structure, particularly suitable when \( x \) is moderate or large. This method is also used in most existing libraries for moderate $x$ values. (3) \textit{Asymptotic Expansions}____: This method focuses on the dominant behavior of \( x \) as it becomes large, ignoring lower-order terms, which reduces the accuracy for small \( x \). This method is also used in most existing libraries for large \( x \) values.
%\textcolor{red}{(4) is not a single method, it is a relation and it is used in many methods including (1) and (3).}
%\textcolor{red}{(4) \textit{Recurrence Relations}____: Mathematical formulas that relate the values of a function at different orders $\nu$, enabling efficient computation by leveraging previously calculated values, allowing efficient computation for different orders $\nu$ once the function is known for a particular order. Examples of libraries that employ this approach include MATLAB, SciPy, and Boost C++; }
(4) \textit{Integral Representations}____: \( K_\nu(x) \) is expressed as integrals. This method is flexible and can work with small or large values of $x$. Some libraries that rely on this method include Mathematica, SciPy, and GSL. (5) \textit{Polynomial Fitting}____: Polynomial fitting approximates \( K_\nu(x) \), by fitting polynomials to precomputed values of the function over restricted ranges of \( x \) and \( \nu \). (6) \textit{Differential Equation Solvers}____: Numerical methods for solving differential equations, such as finite difference schemes and Runge-Kutta methods____, can be employed to compute \( K_\nu(x) \). These solvers approximate the solution of \( K_\nu(x) \) by discretizing the domain and applying iterative algorithms to solve the governing differential equation.

% \textcolor{red}{As for the GPU \textsc{BesselK} implementation, Plesner et al.____ developed a GPU library for computing logarithms of modified Bessel functions of the second kind, addressing critical precision limitations and numerical stability issues present in existing libraries. Their approach provided more robust results compared to existing libraries such as GSL, however, their implementation showed large limitations for modified Bessel functions of the second kind with small values ($(x, \nu) \in [0, 150] \times (0, 150]$) in accuracy and runtime, which is a strong shortcoming since most real applications cannot reach that kind of smoothness level, the typical range of $\nu$ is in between $0$ and $3.5$~\cite}


%%%%%%%%%%%%%%%%%%%%%%%%%%%%%%
% 1- Give brief about \emph{ExaGeoStat}and the missing of GPU support of matern kernel
% 2- Related work to solve the BesselK function (GNU, takekawa, and ICS24 reference, ...etc)


%TO-DO: reduce the contexts of \emph{ExaGeoStat}but the necesseity of accurate and fast evaluation of modified bessel function of the second kind, with a comprehensive review of existing method, comparing advantages and disadvantages. Also, mentioning the bottleneck of evaluation of derivative of besselk which is not suitable for gradient-based optimization.

%\emph{ExaGeoStat}, a high-performance software package designed for climate and environmental geostatistics on large-scale computing systems. The software implements exact maximum likelihood estimation (MLE) for large spatial datasets using the Mat\'{e}rn covariance function, enabling both parameter estimation and prediction of missing measurements across geographical locations. \emph{ExaGeoStat} leverages state-of-the-art dense linear algebra libraries (Chameleon) and runtime systems (StarPU) to achieve high performance across different hardware architectures, including multicore CPUs, GPUs, and distributed systems. The software can handle unprecedented large-scale exact computations, demonstrated through synthetic and real-world soil moisture datasets. The authors also provide an R interface (ExaGeoStatR) to make the software accessible to statisticians. The software serves as a benchmark for validating various approximation techniques and provides a complete machine learning pipeline for geostatistical applications. Using the state-of-the-art runtime system, we can further parallelize and reduce the computational complexity of GPU.

%The previous work____ presents new GPU-accelerated methods for generating dense covariance matrices in spatial statistics applications, specifically enhancing the \emph{ExaGeoStat} software framework. The authors propose two implementation schemes: a pure GPU approach for kernels without special functions like the modified Bessel function and a hybrid CPU-GPU approach for kernels that require special functions (like the Mat\'{e}rn kernel) that cannot be computed directly on GPUs. Performance evaluations show that the pure GPU implementation achieves up to 6X speedup compared to CPU-only implementations for the power exponential kernel, while the hybrid approach achieves up to 1.5X speedup for the Mat\'{e}rn kernel. The implementations were tested on different hardware configurations, including Intel processors (Skylake and Icelake) and NVIDIA GPUs (V100 and A100), demonstrating significant performance improvements, especially as the number of spatial locations increases.

%To prevent the computational bottleneck of CPU-only implementation on Mat\'{e}rn kernel function evaluation, it is necessary to implant the code from CPU-only to GPU-only, which prevents generating the matrix and evaluation using CPU. Recent work introduces new algorithms for computing the logarithm of modified Bessel functions of the first and second kind on GPUs, proposing a pure GPU implementation scheme. The modified Bessel function of the second kind is defined as the solution of

%\[x^2 \frac{d^2y}{dx^2} + x\frac{dy}{dx} - (x^2 + \nu^2)y = 0,\] denoted by $K_{\nu}(x)$, where $\nu$ represents for smoothness. The algorithms demonstrate significant improvements in numerical stability and performance, with no underflow or overflow issues, achieving precision equal to or better than current C++ libraries and runtimes that are typically one to two orders of magnitude faster. The authors validate their approach with both numerical experiments and a real-world case study. Another work presents a novel method for efficiently computing the modified Bessel function of the second kind and its derivatives using a parallel numerical integration approach, addressing the computational challenges in existing methods that use series expansion, continued fraction, or asymptotic expansion. The proposed method pre-computes the integration range and uses a fixed number of intervals for numerical integration, making it more suitable for parallel processing on CPUs and GPUs and multi-core processors while maintaining accuracy comparable to existing methods with the usage of PyTorch. Although it gains certain accuracy using the numerical integration, it does not consider the problem region when $x < 0.1$. In summary, the problem region defined in those papers is either far too large in speaking of the smoothness parameter or limited in speaking of the support of the function, which will be unsuitable for environmental studies. To improve computational performance, we propose a new GPU-accelerated algorithm for evaluating the modified Bessel function of the second kind, which can be integrated with spatial statistics frameworks like \emph{ExaGeoStat}and optimized by focusing on specific study regions.