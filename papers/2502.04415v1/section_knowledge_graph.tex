\section{Knowledge Graph}
\label{sec:kg}

To provide \EngineName{} with a powerful geospatial knowledge base, we compiled information from various sources and combined them under a common KG. Our aim was to create a polymorphic database of spatial resources, by integrating natural features and administrative geo-entities under a compact, non-complex ontology, that better facilitates the task of geospatial question answering.  The aforementioned facts were collected from the following sources:
\begin{itemize}
    \item GADM: We collected geospatial features from the Global Administrative Areas (\url{https://gadm.org/}) dataset, focusing on features located in Europe. Utilized features from this dataset include countries, cities, regional units, and national administrative divisions.
    % along with various non-spatial characteristics describing these features, such as total population.
    \item Rivers, Points of Interest, and Ports: This dataset is provided by our partners in the DA4DTE project e-GEOS (\url{https://www.e-geos.it/}) and includes spatial characteristics for various features within these categories. In addition to spatial data, the dataset also contains comprehensive metadata for each feature.
    % This metadata covers topics such as tourism information, links to Wikipedia entries, and other relevant details that improve the understanding and utility of spatial features.
    \item Sentinel-1 Images: We incorporated Sentinel-1 satellite image data, including metadata about the images and the satellite's location at the time each image was captured. Links to the images are stored in the knowledge graph, ensuring that they are easily accessible by the Question-Answering engine and external sources. Sentinel-1 images were collected for the years 2020 and 2021.
    \item Sentinel-2 Images: Similarly to the Sentinel-1 satellite images, we included image links and metadata from Sentinel-2 image collections to enhance the information available in our data model. Sentinel-2 images were collected for the years 2020, 2021 and 2022.
    \item Sea sectors: A collection of sea sectors covering global water spaces and oceans A total of 101 sea sectors along with their polygons were integrated into the knowledge graph of the Marine Regions~\cite{marine-regions} data source.
\end{itemize}

\textbf{Ontology.} We built our ontology on top of well-known and standardized ontologies, namely the YAGO2geo~\cite{DBLP:conf/semweb/KaralisMK19}
ontology and the GeoSPARQL~\cite{perry2012ogc} ontology. The main class of the knowledge graph is named Feature. The Feature class is extended by various subclasses that represent the knowledge provided by the various datasets that we examined in the previously, namely, rivers,
ports, pois (points of interest), Sentinel-1 and Sentinel-2 images and GADM geoentities. 

% Each of these subclasses have their own
% properties, extracted by the metadata provided in the original JSON files for each given object. To incorporate geospatial knowledge, Feature is also extended by GeoSPARQL's spatial Feature class.

\textbf{Translation of named location labels to English.} While integrating our data, we noticed that many
geospatial features were named only in their original languages, with no English equivalents provided. For
instance, the city of Rome was listed only as "Roma" in Italian within the metadata. To provide a consistent framework and to simplify the task of recognizing labels for our Question-Answering engine, we implemented a comprehensive translation pipeline. Using the Mistral-7b LLM~\cite{mistral}, we were able to correctly identify and translate a total of 56657 labels from various European languages.