\documentclass[conference,a4paper]{IEEEtran}
\IEEEoverridecommandlockouts

\usepackage[hidelinks]{hyperref}
\usepackage[cmex10]{amsmath}%American Math Society(AMS) math formatting
\usepackage{amssymb,amsfonts}%AMS extra symbols and fonts
\interdisplaylinepenalty=2500%allow line breaks in multi-line formulas
\usepackage{dblfloatfix}%fix double column figure ordering and placement

\usepackage[ruled,vlined]{algorithm2e}
\usepackage{graphicx}
\graphicspath{{Figures/PDF/}{Figures/PNG/}}

\usepackage{booktabs}
\usepackage{siunitx}
\usepackage[numbers,compress]{natbib}
\usepackage{texnames}
\usepackage{bm,bbm}
\usepackage{orcidlink}
\usepackage{igarss}
\usepackage{listings}

\begin{document}

\title{

\uppercase{TerraQ: Spatiotemporal Question-Answering on Satellite Image Archives}

\thanks{	This work was supported by the ESA project DA4DTE (subcontract 202320239), the Horizon 2020 projects AI4Copernicus (GA No. 101016798) and STELAR (Grant No. 101070122), the Horizon Europe project FAIR2Adapt (GA no. 101188256), and the "NIK. D. XRYSOVERGI" scholarship of the Greek State Scholarship Foundation.}

}

\author{
    \IEEEauthorblockN{Sergios-Anestis Kefalidis\orcidlink{0009-0007-7648-9737}\textsuperscript{1}, 
    Konstantinos Plas\orcidlink{0009-0003-7613-4072}\textsuperscript{1,2}, 
    Manolis Koubarakis\orcidlink{0000-0002-1954-8338}\textsuperscript{1,2}}
    \IEEEauthorblockA{\textsuperscript{1}\textit{Dept. of Informatics and Telecommunications}, 
    \textit{National and Kapodistrian University of Athens}, Athens, Greece\\
    skefalidis@di.uoa.gr, kplas@di.uoa.gr, koubarak@di.uoa.gr\\
    \textsuperscript{2}\textit{Archimedes/Athena RC}, Marousi, Greece
    }
}


\maketitle
\begin{abstract}
	\EngineName{} is a spatiotemporal question-answering engine for satellite image archives. It is a natural language processing system that is built to process requests for satellite images satisfying certain criteria. The requests can refer to image metadata and entities from a specialized knowledge base (e.g., the Emilia-Romagna region). With it, users can make requests like “Give me a hundred images of rivers near ports in France, with less than 20\% snow coverage and more than 10\% cloud coverage”, thus making Earth Observation data more easily accessible, in-line with the current landscape of digital assistants.
\end{abstract}

\begin{IEEEkeywords}
Question-Answering, Knowledge Graph, Geospatial, Temporal, Earth Observation, SPARQL
\end{IEEEkeywords}

\section{Introduction}

The field of Natural Language Processing is undergoing major advancements caused by the rapid development of Language Models~\cite{openai2024gpt4technicalreport, llama2, mistral}. An outcome of this development is the proliferation of digital assistants and natural language interfaces for all manners of systems and knowledge repositories (e.g., Alexa from Amazon, Siri from Apple, ChatGPT from OpenAI, and Claude from Anthropic). The resulting increase in accessibility enables non-technical users to intuitively interact with computer systems and access high-quality information, while also improving efficiency for expert users.  In this climate, the task of Knowledge-Graph Question-Answering (QA), also known as Text-to-SPARQL, is as relevant as ever~\cite{qa-survey1, qa-survey2}. 
% This area is concerned with the creation of natural language interfaces for semi-structured, high-quality knowledge repositories, known as Knowledge Graphs (KGs).

QA systems take as input queries in natural language and generate semantically equivalent SPARQL\footnote{\url{https://www.w3.org/TR/sparql11-query/}} queries over a specific knowledge graph (KG). These SPARQL queries are subsequently executed on an RDF store, which in turn returns the answer. The answer can either be directly presented to the user, or integrated into a larger system and used as part of a Retrieval-Augmented Generation~\cite{rag} pipeline, as is the case with digital assistants that rely on knowledge grounding.

In our work, we are developing \EngineName{} a spatiotemporal QA engine for satellite image archives. 
% The system is designed to handle general spatiotemporal queries as well as queries regarding satellite images satisfying certain criteria. In response, it returns links to relevant satellite images from the Sentinel-1 and Sentinel-2 missions. 
User requests can refer to image metadata and geographic entities (e.g., the Loch Ness Lake or the city of Munich) both of which are included in the target knowledge graph. For example, ``Show me images of Athens with VV polarization.\%''. The goal of our research is to make Earth Observation data archives accessible via natural language, to the benefit of both novice and expert users.

\EngineName{} belongs in the same family of engines as GeoQA2~\cite{geoqa2} and EarthQA~\cite{earthqa}, two template-based question-answering engines previously developed by our group. GeoQA2 is a geospatial QA engine, and EarthQA is a satellite-image archive QA engine that reuses some components of GeoQA2 while also adding some additional specialized components. In comparison to these two systems, \EngineName{} has a number of advantages. First, it does away with template-based query generation. 
% Queries are constructed dynamically via a combination of built-in expert knowledge, heuristics, and language models. 
As a result, it is able to answer a wider array of questions and has improved accuracy. Second, \EngineName{} targets a purpose-built KG with high-quality geospatial information, allowing for more fine-grained results, which was one of the limitations of the original EarthQA paper. Third, unlike EarthQA, the engine does not use specialized components. All thematic information can be integrated into the core engine architecture, making the engine easier to adapt to different domains.

In this paper, we make the following original contributions:
\begin{enumerate}
    \item We present a specialized knowledge graph that interlinks geospatial information about natural features and administrative divisions with satellite image metadata. Knowledge graph resources are integrated into a hierarchical structure, making it easier to expand with additional geospatial information or thematic knowledge.
    \item We develop \EngineName{}, a spatiotemporal QA engine for image archives. The engine is able to answer simple and complex queries both reliably and quickly in dynamic fashion, while also avoiding the use of query-templates or computationally demanding neural models. 
\end{enumerate}

The version of \EngineName{} presented in this paper has been developed in the context of the European Space Agency project \textit{DA4DTE: Demonstrator Precursor Digital Assistant Interface for Digital Twin Earth\footnote{\url{http://da4dte.e-geos.earth/}}} and a demo is available publicly at {\url{http://terraq.di.uoa.gr/}}.

% The rest of the paper is organized as follows. In Section~\ref{sec:kg} we present the target Knowledge Graph. Sections~\ref{sec:engine} and \ref{sec:eval} present the architecture of \EngineName{} and its evaluation. Section~\ref{sec:conclusion} concludes the paper.

\section{Knowledge Graph}
\label{sec:kg}

To provide \EngineName{} with a powerful geospatial knowledge base, we compiled information from various sources and combined them under a common KG. Our aim was to create a polymorphic database of spatial resources, by integrating natural features and administrative geo-entities under a compact, non-complex ontology, that better facilitates the task of geospatial question answering.  The aforementioned facts were collected from the following sources:
\begin{itemize}
    \item GADM: We collected geospatial features from the Global Administrative Areas (\url{https://gadm.org/}) dataset, focusing on features located in Europe. Utilized features from this dataset include countries, cities, regional units, and national administrative divisions.
    % along with various non-spatial characteristics describing these features, such as total population.
    \item Rivers, Points of Interest, and Ports: This dataset is provided by our partners in the DA4DTE project e-GEOS (\url{https://www.e-geos.it/}) and includes spatial characteristics for various features within these categories. In addition to spatial data, the dataset also contains comprehensive metadata for each feature.
    % This metadata covers topics such as tourism information, links to Wikipedia entries, and other relevant details that improve the understanding and utility of spatial features.
    \item Sentinel-1 Images: We incorporated Sentinel-1 satellite image data, including metadata about the images and the satellite's location at the time each image was captured. Links to the images are stored in the knowledge graph, ensuring that they are easily accessible by the Question-Answering engine and external sources. Sentinel-1 images were collected for the years 2020 and 2021.
    \item Sentinel-2 Images: Similarly to the Sentinel-1 satellite images, we included image links and metadata from Sentinel-2 image collections to enhance the information available in our data model. Sentinel-2 images were collected for the years 2020, 2021 and 2022.
    \item Sea sectors: A collection of sea sectors covering global water spaces and oceans A total of 101 sea sectors along with their polygons were integrated into the knowledge graph of the Marine Regions~\cite{marine-regions} data source.
\end{itemize}

\textbf{Ontology.} We built our ontology on top of well-known and standardized ontologies, namely the YAGO2geo~\cite{DBLP:conf/semweb/KaralisMK19}
ontology and the GeoSPARQL~\cite{perry2012ogc} ontology. The main class of the knowledge graph is named Feature. The Feature class is extended by various subclasses that represent the knowledge provided by the various datasets that we examined in the previously, namely, rivers,
ports, pois (points of interest), Sentinel-1 and Sentinel-2 images and GADM geoentities. 

% Each of these subclasses have their own
% properties, extracted by the metadata provided in the original JSON files for each given object. To incorporate geospatial knowledge, Feature is also extended by GeoSPARQL's spatial Feature class.

\textbf{Translation of named location labels to English.} While integrating our data, we noticed that many
geospatial features were named only in their original languages, with no English equivalents provided. For
instance, the city of Rome was listed only as "Roma" in Italian within the metadata. To provide a consistent framework and to simplify the task of recognizing labels for our Question-Answering engine, we implemented a comprehensive translation pipeline. Using the Mistral-7b LLM~\cite{mistral}, we were able to correctly identify and translate a total of 56657 labels from various European languages.
\section{The \EngineName{} Engine}
\label{sec:engine}

\EngineName{} %employs a pipeline architecture, meaning that it 
consists of a number of components, each of which performs a specific task. Information is propagated from one component to the next. This pipeline is split in four distinct conceptual steps.

% \begin{itemize}
%     \item WHERE clause generation
%     \item SELECT/ASK clause generation
%     \item Query generation
%     \item Query rewriting
% \end{itemize}

First, the WHERE clause is generated by combining basic SPARQL/GeoSPARQL building blocks. This is subsequently passed as additional input to the components responsible for the generation of the SELECT/ASK clause. When both clauses have been constructed, the query generator merges them and makes any necessary additions to construct a complete SPARQL query. In the last step, the query is rewritten to make use of materialized geospatial relations.

The complete architecture of \EngineName{} is shown in Figure~\ref{fig:system_arch}. Below, we present the functionality of the system in detail. As our running example we use the request ``Show me all images taken in January 2021 with rivers less than 2km away from towns and forests in the Emilia Romagna region, having cloud coverage less than 10\%''.

\begin{figure*}[h]
\begin{center}
\includegraphics[width=15cm]{earthqa_paper.drawio.png}
\end{center}
\caption{The conceptual architecture of the \EngineName{} system}
\label{fig:system_arch}
\end{figure*} 

\textbf{Dependency Parse Tree Generator.} This module generates a dependency parse tree of the input question using StanfordCoreNLP~\cite{corenlp}. The dependency parse tree is used to identify and store information.

\textbf{Instance Identifier.} This module does named-entity recognition and disambiguation.
%identifies and maps named entities present in the input question to the appropriate resources of the KG. 
In the example question, it identifies the entity “Emilia Romagna” and maps it to the resource \textit{yago:Emilia\_(region\_of\_Italy)} in the KG. The mapping to the KG resource happens in two steps. First, WAT~\cite{WAT} links the named entity to a Wikipedia page. Subsequently, the component searches the Knowledge Graph for the resource that best matches the entity returned by WAT.
In addition to identifying the instance, this component is responsible for creating the block that will be used in the WHERE clause for the identified instance. The generated block is the following:

\begin{lstlisting}[language=SPARQL]
<URI> geo:hasGeometry/geo:asWKT ?iWKTID .
\end{lstlisting}

% We use TagMeDisambiguate based on an evaluation of named-entity recognition and disambiguation models on the geospatial question-answering dataset GeoQuestions201 [PIS+20]. Since then, a few promising systems have been developed but no system has proven to be significantly better for the task of geospatial question-answering, which is most relevant to this project.

\textbf{Concept Identifier.} This module identifies and maps concepts present in the input question to the appropriate resource of the KG ontology. For instance, from the example question, it will identify and map the concepts \textit{River, Town, Forest}. The mapping is done using a class label dictionary and string similarity based on n-grams.
Additionally, this component is responsible for creating the block that will be used in the WHERE clause for the identified concepts:

\begin{lstlisting}[language=SPARQL]
?cID a <URI> ; 
    geo:hasGeometry/geo:asWKT ?cWKTID .
\end{lstlisting}

At the end of the concept identification stage, and after all Instances have been identified, we employ a heuristic of consolidation between concepts and instances. Concepts and Instances that are not separated by any token are consolidated to reduce the complexity of the generated WHERE-clause and help the query generator produce a correct query. For example, in the question “Where is the Tagus river located?” only the Instance of Tagus is kept and the river concept is consolidated into it.

\textbf{Property Identifier.} The property identifier identifies attributes of features or types of features specified by the user in input questions and maps them to the corresponding properties in the knowledge graph. In the example question, the property ``cloud coverage'' of the type of feature image will be identified and mapped to the corresponding property in the KG.
% In the previous engines of \EngineName{}’s family, only one property per question was supported. For \EngineName{}, we wanted to overcome this limitation to be able to answer a wider array of questions.
For each identified concept, we try to match, using string similarity on n-grams, its properties to the words in the sentence. Matched properties are identified as candidate properties for this concept. Multiple concepts might have the same candidate property. To resolve this conflict, we introduced a heuristic that selects the closest concepts as the targets to the properties. This process is similar for instances inside the question.

Again, this component is also responsible for generating the block that will be used in the WHERE clause for the identified properties:

\begin{lstlisting}[language=SPARQL]
INSTANCE/CONCEPT_VARIABLE <URI> ?pID.
\end{lstlisting}

In addition, this component uses the dependency parse tree and Part-of-Speech tags to identify words that denote the use of comparatives and superlatives. These are subsequently matched to the appropriate Concept or Property, using a node-distance heuristic on the dependency parse tree.

\textbf{Spatial relation Identifier.} This module identifies spatial relations present in the input question and maps them to appropriate stSPARQL/GeoSPARQL functions. For instance, in the example question, it will identify the spatial relations ``in'' and ``away from'' and map them to \textit{geof:sfWithin} and \textit{geof:distance} respectively. Then these relations are mapped to the appropriate previously identified Instances and Concepts by using the following heuristic: 

\[
\text{distance} = \text{dependency\_parse\_tree\_distance} + \left(\frac{\text{word\_distance}}{100}\right)
\]

Again, this component is also responsible for generating the block that will be used in the WHERE clause:

\begin{lstlisting}[language=SPARQL]
FILTER (<URI> (FIRST_FEATURE, 
    SECOND_FEATURE))
\end{lstlisting}
and
\begin{lstlisting}[language=SPARQL]
FILTER (geof:distance (FIRST_FEATURE, 
    SECOND_FEATURE, uom:metre) 
    {<, >, <=, >=, =, ~} DISTANCE)
\end{lstlisting}

\textbf{Numeric Solver.} This module is responsible for identifying numbers, understanding their use in the input question and enhancing the previously identified elements with additional information. For this purpose we utilize Part-of-Speech tags and the previously described distance heuristic.
% For identifying numbers, the CD (cardinal number) Part-of-Speech tag produced by CoreNLP is used. Subsequently we use the NER components of CoreNLP to gather additional information about the number (whether it refers to an equality, an inequality or an approximation). We take special care to not include in our actions numbers that are related dates and/or durations.

% After the number and its context are identified we once again employ our distance heuristic to locate the most relevant to the number Concept/Property/Spatial-Relation, which is afterwards enhanced with information about the number and its context.
In our working example, ``less than 2km'' is matched to the spatial function of distance and ``less than 10\%'' is matched to the cloud coverage property.

\textbf{Conjunction Solver.} The Conjunction Solver is responsible for handling conjunctions, as those are identified by the dependency parse tree. To that end, it selects all edges of the parse tree tagged as ``conj:and''. The vertices connected by each of those edges are checked for meaningful conjunctions. Number-to-Property, Number-to-Number and Geospatial-to-Geospatial conjunctions are supported.
For Geospatial-to-Geospatial conjunctions additional spatial relations are generated and stored, as if they were created by the Geospatial Relation Identifier, according to the information provided by the vertices. In our example, this is the case with ``towns and forests''. Number and Property conjunctions faction similarly.

\textbf{Temporal Identifier.} This module uses HeidelTime~\cite{heideltime} to identify temporal keywords in the input question and annotates them with the appropriate date and/or duration. For instance, in the example input question it will identify ``January 2021'' and map it to 2021-01. 

% To that end, the temporal tagger HeidelTime~\cite{heideltime} is used. In addition, durations are identified and can potentially be combined with dates. For example, in the question “Give me images in a span of 3 days in 2023”, the Temporal identifier will combine the timespan of 3 days with the timestamp of 2023 to produce a date-duration object which will be used for query generation.

\textbf{Return Type Identifier.} This module is responsible for identifying the expected form/type of the answer to the question. The supported types are \textit{ Name, Coordinates, Number-Property, Number-Count, Image }. For our example, \textit{Image} is the most appropriate return type.
For identifying the expected return types, this component leverages the sophisticated language understanding of Llama 2~\cite{llama2}. We fine-tune our model to output correctly formatted answers. A fallback mechanism that uses heuristics is provided to enable using \EngineName{} without hardware acceleration (GPU).
% It's worth noting that we refrained from using Llama 2 to determine whether the generated query should be a SELECT or an ASK query. This decision was made because this task can be readily identified through lexical analysis, which offers both speed and effectiveness comparable to the results achieved by Llama 2.

\textbf{Query Form Identifier.} This component is responsible for generating the final ASK/SELECT clause, which will be used by the query generator. It takes as input the return types generated by the return type identifier. For each expected return type, we follow an iterative approach as follows: If the type is \textit{Name}, we search for the next concept. If the type is \textit{Coordinates}, we seek the next concept or instance. When the return type is \textit{Number-Property}, we look for the next property, and if it's \textit{Number-Count}, we search for the next concept. Additionally, we enhance the query by introducing a COUNT aggregation and the necessary GROUP BY clauses. In the case of \textit{Image}, we insert the appropriate code in the query. To determine the 'next' object, we traverse the dependency parse tree.
% To determine the 'next' object, we traverse the dependency parse tree. The traversal begins from the Wh-word of the sentence. Wh-words are those that are tagged as { WP, WFT, WRB } by the CoreNLP part of speech tagger. After locating Wh-words, we identify the objects that are closest based on the dependency parse tree distance. 
% Again, we provide a fallback implementation, for running the engine without hardware acceleration.

\textbf{Query Generator.} The query generator is responsible for generating the final query. Within this stage of the pipeline, it assimilates all the information provided by the preceding components and combines them into a suitable, executable SPARQL or GeoSPARQL query. Information about superlatives, limits and other structures is taken into account in the generation process.

\textbf{Query Enhancer.} The query enhancer is an optional component responsible for modifying the query produced by the Query Generator to fix any mistakes and/or oversights. It is implemented using the Mistral-7b LLM fine-tuned on the dataset GeoQuestions1089~\cite{geoquestions1089}. It serves as a performance-enhancement module that increases the capacity of \EngineName{} to answer complex questions following the Execution Refinement paradigm~\cite{neural-interfaces}.

\textbf{GoST.} The GoST transpiler~\cite{geoquestions1089} takes the query generated by the Query Generator and rewrites it to use materialized geospatial relations if that is possible. Because geospatial relations like \textit{geof:sfWithin} are computationally expensive we do offline materialization using the tool JedAI-Spatial~\cite{jedai-spatial}.
\section{Evaluation}
\label{sec:eval}

To the best of our knowledge, there is no publically available dataset that is suitable for evaluating systems on the task of Text-to-SPARQL for Earth Observation archives. For this reason, we decided to deploy our engine as a pure geospatial QA engine and run an evaluation on the geospatial QA dataset GeoQuestions1089~\cite{geoquestions1089}. Although this did require some tinkering, since GeoQuestions1089 targets the YAGO2geo ontology, the process was straightforward and painless, and we believe that the resulting evaluation is useful for measuring the performance of most dimensions of our engine. Unfortunately, the questions in GeoQuestions1089 do not include temporal information.

To accept an answer as correct, it must match the gold result (included in GeoQuestions1089) exactly. We do not consider partially correct answers as correct. Likewise, for supersets of the answers in the gold set.

The results of our evaluation can be seen in Table~\ref{tab:geoq1089}. We benchmark \EngineName{} with the Query Enhancer disabled, since the model was fine-tuned on the same dataset, which would skew the results. We also compare our engine to GeoQA2 and the engine of Hamzei et al~\cite{hamzei}.

% \begin{table}[hbt]
% 	\centering
% 	\caption{Evaluation on GeoQuestions1089}\label{tab:geoq1089}
% 	\begin{tabular}{c c c c }
% 		\toprule
% 		\textbf{Category} & \textbf{GeoQA2 Accuracy} & \textbf{Hamzei Accuracy} & \textbf{\EngineName{} Accuracy} \\ \hline
%         A & 53.52\% & 28.16\% & 60.56\% \\ \hline
%         B & 62.68\% & 55.22\% & 73.13\% \\ \hline
%         C & 48.36\% & 30.06\% & 47.71\% \\ \hline
%         D & 9.09\% & 4.54\% & 22.73\% \\ \hline
%         E & 24.81\% & 6.56\% & 23.36\% \\ \hline
%         F & 28.57\% & 14.28\% & 38.10\% \\ \hline
%         G & 36.30\% & 12.32\% & 28.77\% \\ \hline
%         H & 23.07\% & 33.33\% & 40.17\% \\ \hline
%         I & 21.73\% &8.69\% & 26.00\% \\ \hline
%         ALL & 40.33\% & 25.92\% & 44.36\% \\
%         \bottomrule
% 	\end{tabular}
% \end{table}

\begin{table}[hbt]
    \centering
    \caption{Evaluation on GeoQuestions1089$_c$ v1.1}\label{tab:geoq1089}
    \resizebox{\columnwidth}{!}{%
        \begin{tabular}{c c c c }
            \toprule
            \textbf{Category} & \textbf{GeoQA2 Accuracy} & \textbf{Hamzei Accuracy} & \textbf{\EngineName{} Accuracy} \\ \hline
            A & 53.52\% & 28.16\% & 60.56\% \\ \hline
            B & 62.68\% & 55.22\% & 73.13\% \\ \hline
            C & 48.36\% & 30.06\% & 47.71\% \\ \hline
            D & 9.09\% & 4.54\% & 22.73\% \\ \hline
            E & 24.81\% & 6.56\% & 23.36\% \\ \hline
            F & 28.57\% & 14.28\% & 38.10\% \\ \hline
            G & 36.30\% & 12.32\% & 28.77\% \\ \hline
            H & 23.07\% & 33.33\% & 40.17\% \\ \hline
            I & 21.73\% &8.69\% & 26.00\% \\ \hline
            ALL & 40.33\% & 25.92\% & 44.36\% \\
            \bottomrule
        \end{tabular}%
    }
\end{table}

We can see that \EngineName{} outperforms the previous state of the art in most question categories without utilizing templates. All in all, there is 4\% uplift in performance over the entire dataset, which is translated to a 10\% improvement relative to GeoQA2. The categories with performance regressions are caused by \EngineName{}’s more dynamic nature. \EngineName{} does not use predefined query templates and employs heuristics for a number of processes as previously described, these heuristics are not performing as well for those particular question categories. 
\section{Conclusion and Future Work}
\label{sec:conclusion}

The success of modern digital assistants has clearly shown that natural language interfaces for computer systems and knowledge repositories can be a great boon for productivity and accessibility. Users nowadays expect to be able to interact with their computers through natural language, reducing the barrier of technical knowledge required for utilizing modern computing capabilities.

This paper presents \EngineName{}, a spatiotemporal QA system for satellite image archives. Our engine targets a high-quality, purpose-built knowledge graph that contains Sentinel-1 and Sentinel-2 image metadata, as well as geospatial information for administrative divisions and natural features. Requests made in natural language are translated to SPARQL queries, which are subsequently executed by an RDF store. This enables users to request, in natural language, satellite images satisfying a number of complex spatial, temporal and thematic criteria. 

Our engine is easily deployable and responsive on commodity hardware, since it does not rely on exceedingly large LLMs. Instead, it utilizes a combination of small-scale LLMs, heuristics and expert knowledge. This development is another step towards our vision of making Earth Observation archives more accessible by both novice and expert users, no matter the available computing capacity.

In the future, we are planning on expanding the capabilities of our system by integrating Visual Question-Answering systems into our Knowledge Graph creation pipeline. This will enable users to express even more specific criteria for image selection, while also maintaining performance.

% \section{Document Structure}

% Conceptually, a scientific document should have four elements:
% \begin{itemize}
% 	\item Introduction: your scientific question, the review of the literature, an outline of the steps you took to answer, a summary of what you observed, and a brief statement of how those observations changed the state-of-the-art (that was outlined in the review of the literature).
% 	\item Methodology: describe what you did so other researchers can replicate your study.
% 	\item Results: the core observations you collected, usually summarized in tables, figures, and images.
% 	\item Discussion: this is the essential part of your work. Do not summarize your results. Tell the reader how they answer (or not) your scientific question and how they change the state-of-the-art.
% \end{itemize}
% This structure is known as ``IMRaD,'' and was consolidated by over a century of scientific practice.
% Fig.~\ref{fig:IMRAD} shows a quote from the book by \citet{Day1998} about this structure, and this piece of code shows how to include a figure in your article.
% The file is in PNG format; reserve this format for matrix-like graphical elements.
% Use PDF for plots, as in Fig.~\ref{fig:Densities}.

% \begin{figure}[hbt]
% 	\centering
% 	\includegraphics[width=.9\linewidth]{Day-IMRaD}
% 	\caption{A screenshot from \citet{Day1998} about the structure of a scientific article.}\label{fig:IMRAD}
% \end{figure}

% The main textual components are controlled by the commands \verb|\section|, \verb|\subsection|, and \verb|\subsubsection|.

% \subsection{This is a subsection}

% Text components are automatically numbered.

% \subsubsection{This is a subsection inside a subsection} This is the innermost text component you should use in your document. 
% It appears as a paragraph with a title.

% \subsubsection{Nice tables} Citing \citet{simplexCT}:
% \begin{quote}
% 	Tables are best suited for looking up precise values, comparing individual values or presenting values involving multiple units of measure. Graphs, on the other hand, are better for detecting trends, anomalies or relations. In other words, graphs show the forest while tables show the trees.
% \end{quote}

% Once you have designed your table, type it using commands from \verb|booktabs|, as in Table~\ref{tab:MyTable}.
% Notice that the second column type was stipulated with commands from the \verb|siunitx| package.
% It switches to mathematical mode and aligns the figures by their decimal point, promoting an immediate visual comparison.

% \begin{table}[hbt]
% 	\centering
% 	\caption{True and estimated classes, with the sample mean.}\label{tab:MyTable}
% 	\begin{tabular}{r S[table-format=3.2] l}
% 		\toprule
% 		\textbf{True LULC} & \textbf{Mean} & \textbf{Estimated LULC} \\ \cmidrule(lr){1-1} \cmidrule(lr){2-2}\cmidrule(lr){3-3}
% 		Sand & 1.32 & Open sea\\
% 		Forest & 7.93 & Forest\\
% 		Open sea & -5 & Open sea\\
% 		Bare soil & 100.41 & Bare soil\\ \bottomrule
% 	\end{tabular}
% \end{table}

% Tables and algorithms captions are usually at the top,
% while figures captions are at the bottom.

% \subsection{Another subsection}

% Your document should have at least two components for each level, i.e., do not use a single \verb|\subsubsection|, but at least two.

% One of \LaTeX's strongest abilities is producing high-quality equations:
% \begin{equation}
% 	i\hbar \frac{\partial}{\partial t} \Psi(\bm{r}, t) = \widehat{H} \Psi(\bm{r}, t),
% \end{equation}
% where $i=\sqrt{-1}$ is the imaginary unit,
% $\hbar$ is the reduced Plank's constant,
% ${\partial}/{\partial t}$ is the partial derivative operator with respect to time,
% $\Psi(\bm{r}, t)$ is the wave equation that depends on the position $\bm r$ and time $t$, and
% $\widehat{H}$ is the Hamiltonian operator.
% Always detail every component of each equation.

% Use \verb|amsmath|'s mathematical environments:
% \begin{equation}
% 	f_{\text{tr} A}(x) =
% 	\Big| \frac{x\beta}{2} \Sigma^{-1} \Big|^a
% 	\frac{e^{-z}}{x\Gamma(ma)} {}_1F_x^{(2a)} 
% 	\Big(a;ma;z I_m - \frac{x\beta}{2}\Sigma^{-1}\Big),
% \end{equation}
% in which we have followed the notation defined by \citet{TheDensitiesandDistributionsoftheLargestEigenvalueandtheTraceofaBetaWishartMatrix}.
% The package offers several options to align equations.
% Often, an equation is wider than a column; in that case, use the \verb|\begin{multline} ... \end{multline}| environment.
% \begin{multline}
% 	\Pr\big(\text{tr} A \leq (x)\big) =
% 	\Big| \frac{x\beta}{2} \Sigma^{-1} \Big|^a
% 	\frac{1}{\Gamma(ma+1)} \\ 
% 	{}_1F_x^{(2a)} \Big(a;ma+1;- \frac{x\beta}{2}\Sigma^{-1}\Big).
% \end{multline}

% \subsection{Algorithms}

% Algorithms summarize, in readable form, the steps that comprise a computational procedure.
% Algorithm~\ref{algo:TwoColumns} shows an example using the \verb|algorithm2e| package.
% This pseudocode spans two columns.

% \begin{algorithm*}[hbt]
% 	\DontPrintSemicolon
% 	\SetKwInput{KwCompute}{Compute}
% 	\SetKwInput{KwDefine}{Define}
% 	\caption{Quasi $U$-Statistics}\label{algo:TwoColumns}
% 	\KwData{The list of folders $\mathcal L$}
% 	\KwData{The number of groups $G$.}
% 	\KwData{The sample size per group $n_1,n_2,\dots,n_G$.}
% 	\KwData{The data per group $\bm X_{1},\ldots,\bm X_{n_1},\bm X_{n_1+1},\ldots,\bm X_{n-n_G+1},\ldots,\bm X_{n_G}$.}
% 	\KwData{The number of bootstrap replications $B$ (we used $200$).}
% 	\KwCompute{Choose the kernel $\phi$ of degree two from 
% 		\begin{align*}
% 			\phi_1(x,y)&=|x-y|^p,\\
% 			\phi_2(x,y)&=\mathbbm{1}\left\{|x-y|>\eta\right\}.
% 	\end{align*}}
% 	\KwCompute{Select $p$ for $\phi_1$, or $\eta$ for $\phi_2$, e.g., with a data-drive optimization method.}
% 	\KwDefine{Define $T$ as
% 		\[
% 		T=\sum\eta_{nij}\phi(\bm X_i,\bm X_j),
% 		\]
% 		where the weights are given by:
% 		\[
% 		\eta_{nij}=\begin{cases}
% 			1 & \text{if } i \text{ and } j \text{ belong to different groups,}\\
% 			-\frac{(n-n_g)}{(n_g-1)} & \text{if } i \text{ and } j \text{ belong to group } g, \text{ for each } 1\leq g\leq G.
% 		\end{cases}
% 		\]}
% 	\While{$\mathcal L$ has elements}{
% 		Read the following element $\ell$ in the list $\mathcal L$\;
% 		\If{$\ell$ is empty}{Skip to the next element\;}
% 		\Else{
% 			\KwCompute{Compute $T$ on the sample $\bm X_{1},\ldots,\bm X_{n_1},\bm X_{n_1+1},\ldots,\bm X_{n-n_G+1},\ldots,\bm X_{n_G}$ and store it in $T_{\text{obs}}$}
% 			\For{$1\leq b \leq B$}{
% 				Build the bootstrap sample $\bm X^{(b)}=\big(\bm X_{1}^{(b)},\ldots,\bm X_{n_1}^{(b)},\ldots,\bm X_{n-n_G+1}^{(b)},\ldots,\bm X_{n_G}^{(b)}\big)$\;
% 				Compute $T^{(b)}$ with $\bm X^{(b)}$
% 			}
% 			Compute $q_\alpha$, the upper $\alpha$ quantile of $T^{(1)},T^{(2)},\dots, T^{(B)}$\;
% 			\KwOut{Decision: reject $H_0$ if $T_{\text{obs}}>q_\alpha$.}
% 			Remove the current set from the list $\mathcal L$\;
% 		}
% 	}
% \end{algorithm*}

% \subsection{Units}

% It is mandatory to state the measurement units.
% The \verb|siunitx| provides commands that implement some of the best ways to do it.
% For instance, you type \verb|\SI{10}{\mega\hertz}| and produce \SI{10}{\mega\hertz}.
% You type \verb|\qtyproduct{2x5}{\meter}| and produce \qtyproduct{2x5}{\meter}, e.g., for stating the spatial resolution of an image.

% \section{Figures}

% Fig.~\ref{fig:Densities} shows a PDF figure.
% Notice that the font size is readable in the final version.

% \begin{figure}[hbt]
% 	\centering
% 	\includegraphics[width=\linewidth]{SinglePlot}
% 	\caption{Two time series mapped onto their Shannon Entropy ($H$) and Statistical Complexity ($C$) points along with \SI{95}{\percent} confidence intervals for the entropy using $D=4$ embedding dimension; cf.\ \citet{AsymptoticDistributionofEntropiesandFisherInformationMeasureofOrdinalPatternswithApplicationsa}.}\label{fig:Densities}
% \end{figure}

% Making reproducible plots and storing the code that produced them is more than advisable.
% The plot displayed in Fig.~\ref{fig:Densities} was created with R~\cite{AmazingR}, and the code is in the \verb|Code| folder accompanying this template.
% It requires a few packages: \verb|ggplot2|~\citep{ggplot2}, \verb|ggthemes|~\citep{ggthemes}, and
% \verb|StatOrdPattHxC|~\citep{AsymptoticDistributionofEntropiesandFisherInformationMeasureofOrdinalPatternswithApplicationsa}.

% Notice that I used \verb|theme_tufte()| to obtain a minimalistic yet informative plot.
% I recommend the books by E.\ Tufte~\citep{Tufte01,VisualExplanationsImagesandQuantitiesEvidenceandNarrative,EnvisioningInformation,BeautifulEvidence} are my preferred reference for communicating with graphics.

% \section{References using \BibTeX}

% This template shows how to use \BibTeX.
% The \texttt{natbib} package allows several types of citations.
% Notice the convenience of using \verb|\citet|, as in the book by \citet{BrockwellDavis91}.
% There is also the \verb|\citep| option, where the reference is inserted between parenthesis \citep{OverviewofthePolSARproV4.0SoftwaretheOpenSourceToolboxforPolarimetricandInterferometricPolarimetricSARDataProcessing}.
% The \verb|references.bib| file provides examples of a journal article~\citep{ABadgingSystemforReproducibilityandReplicabilityinRemoteSensingResearch},
% a technical report~\citep{SystemCalibrationStrategiesforSpaceborneSyntheticApertureRadarforOceanography}, a conference article~\cite{OverviewofthePolSARproV4.0SoftwaretheOpenSourceToolboxforPolarimetricandInterferometricPolarimetricSARDataProcessing}, a book~\cite{BrockwellDavis91}, a manual~\cite{ERMANIIUsersManual}, and a website~\cite{simplexCT}.

\small
\bibliographystyle{IEEEtranN}
\bibliography{references}

\end{document}