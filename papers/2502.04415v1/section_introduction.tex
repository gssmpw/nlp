\section{Introduction}

The field of Natural Language Processing is undergoing major advancements caused by the rapid development of Language Models~\cite{openai2024gpt4technicalreport, llama2, mistral}. An outcome of this development is the proliferation of digital assistants and natural language interfaces for all manners of systems and knowledge repositories (e.g., Alexa from Amazon, Siri from Apple, ChatGPT from OpenAI, and Claude from Anthropic). The resulting increase in accessibility enables non-technical users to intuitively interact with computer systems and access high-quality information, while also improving efficiency for expert users.  In this climate, the task of Knowledge-Graph Question-Answering (QA), also known as Text-to-SPARQL, is as relevant as ever~\cite{qa-survey1, qa-survey2}. 
% This area is concerned with the creation of natural language interfaces for semi-structured, high-quality knowledge repositories, known as Knowledge Graphs (KGs).

QA systems take as input queries in natural language and generate semantically equivalent SPARQL\footnote{\url{https://www.w3.org/TR/sparql11-query/}} queries over a specific knowledge graph (KG). These SPARQL queries are subsequently executed on an RDF store, which in turn returns the answer. The answer can either be directly presented to the user, or integrated into a larger system and used as part of a Retrieval-Augmented Generation~\cite{rag} pipeline, as is the case with digital assistants that rely on knowledge grounding.

In our work, we are developing \EngineName{} a spatiotemporal QA engine for satellite image archives. 
% The system is designed to handle general spatiotemporal queries as well as queries regarding satellite images satisfying certain criteria. In response, it returns links to relevant satellite images from the Sentinel-1 and Sentinel-2 missions. 
User requests can refer to image metadata and geographic entities (e.g., the Loch Ness Lake or the city of Munich) both of which are included in the target knowledge graph. For example, ``Show me images of Athens with VV polarization.\%''. The goal of our research is to make Earth Observation data archives accessible via natural language, to the benefit of both novice and expert users.

\EngineName{} belongs in the same family of engines as GeoQA2~\cite{geoqa2} and EarthQA~\cite{earthqa}, two template-based question-answering engines previously developed by our group. GeoQA2 is a geospatial QA engine, and EarthQA is a satellite-image archive QA engine that reuses some components of GeoQA2 while also adding some additional specialized components. In comparison to these two systems, \EngineName{} has a number of advantages. First, it does away with template-based query generation. 
% Queries are constructed dynamically via a combination of built-in expert knowledge, heuristics, and language models. 
As a result, it is able to answer a wider array of questions and has improved accuracy. Second, \EngineName{} targets a purpose-built KG with high-quality geospatial information, allowing for more fine-grained results, which was one of the limitations of the original EarthQA paper. Third, unlike EarthQA, the engine does not use specialized components. All thematic information can be integrated into the core engine architecture, making the engine easier to adapt to different domains.

In this paper, we make the following original contributions:
\begin{enumerate}
    \item We present a specialized knowledge graph that interlinks geospatial information about natural features and administrative divisions with satellite image metadata. Knowledge graph resources are integrated into a hierarchical structure, making it easier to expand with additional geospatial information or thematic knowledge.
    \item We develop \EngineName{}, a spatiotemporal QA engine for image archives. The engine is able to answer simple and complex queries both reliably and quickly in dynamic fashion, while also avoiding the use of query-templates or computationally demanding neural models. 
\end{enumerate}

The version of \EngineName{} presented in this paper has been developed in the context of the European Space Agency project \textit{DA4DTE: Demonstrator Precursor Digital Assistant Interface for Digital Twin Earth\footnote{\url{http://da4dte.e-geos.earth/}}} and a demo is available publicly at {\url{http://terraq.di.uoa.gr/}}.

% The rest of the paper is organized as follows. In Section~\ref{sec:kg} we present the target Knowledge Graph. Sections~\ref{sec:engine} and \ref{sec:eval} present the architecture of \EngineName{} and its evaluation. Section~\ref{sec:conclusion} concludes the paper.
