\section{Conclusion and Future Work}
\label{sec:conclusion}

The success of modern digital assistants has clearly shown that natural language interfaces for computer systems and knowledge repositories can be a great boon for productivity and accessibility. Users nowadays expect to be able to interact with their computers through natural language, reducing the barrier of technical knowledge required for utilizing modern computing capabilities.

This paper presents \EngineName{}, a spatiotemporal QA system for satellite image archives. Our engine targets a high-quality, purpose-built knowledge graph that contains Sentinel-1 and Sentinel-2 image metadata, as well as geospatial information for administrative divisions and natural features. Requests made in natural language are translated to SPARQL queries, which are subsequently executed by an RDF store. This enables users to request, in natural language, satellite images satisfying a number of complex spatial, temporal and thematic criteria. 

Our engine is easily deployable and responsive on commodity hardware, since it does not rely on exceedingly large LLMs. Instead, it utilizes a combination of small-scale LLMs, heuristics and expert knowledge. This development is another step towards our vision of making Earth Observation archives more accessible by both novice and expert users, no matter the available computing capacity.

In the future, we are planning on expanding the capabilities of our system by integrating Visual Question-Answering systems into our Knowledge Graph creation pipeline. This will enable users to express even more specific criteria for image selection, while also maintaining performance.