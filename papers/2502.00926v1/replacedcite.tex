\section{Related Work}
\label{sec:related}

Pneumatic chambers have been incorporated in robotic gripper surfaces to achieve both macro-scale and micro-scale contact modification.
%
For instance, soft chambers have long been used to envelop or wrap around objects.
%
This macro-scale behaviour drastically increases contact area, sometimes even to the point of establishing form closure ____.
%
Although pneumatic chambers often make up entire grippers or gripper fingers (e.g., PneuNets ____), the load and torque limitations faced by soft grippers have motivated the development of hybrid rigid-soft grippers as a compliant yet strong solution ____.

Some recent rigid-soft grippers augment rigid fingers with soft pneumatic fingerpads, which offer both passive compliance and active pressure-distributing and enveloping capabilities.
%
Most such designs use only a single chamber on each finger ____ or phalange ____, in which the entire fingerpad surface forms a single active region (rather than creating multiple active regions as in ____). 
%
Although these large active regions do increase friction at the macro-scale by enveloping objects, their friction-tuning effects are coupled to the global morphology of the fingerpads. 
%
When the pads inflate, they form bubble shapes that can significantly laterally displace small objects near the sides of the pads.
%
Additionally, most designs apply only positive pneumatic pressure. 
%
The few that apply negative pressure (e.g., ____) typically tune compliance or adhesion rather than friction. 
%
In contrast, our structured fingerpads use a single pneumatic input to actuate an array of small active regions. 
%
Actuation changes the local morphology of the active regions, tuning micro-scale friction with only minor changes to the global pad morphology and thus facilitating stable grasping of small objects. 
%
In certain cases, more macro-scale contact modification is possible: we can apply negative pressure to recess the active regions (enabling interlocking with convex features) or positive pressure to protrude the active regions (enabling partial enveloping of convex objects or interlocking with concave features) as depicted in \cref{fig:macro}. 

\begin{figure}[bt]
    \centering
    \begin{tikzpicture}
% INTERLOCKING
    \coordinate (cs) at (0,0);
    \coordinate (cw) at (0.7,0);
    \coordinate (cl) at (0,-1);
    \coordinate (cew) at (0.1,0);
    \coordinate (cel) at (0,-0.1);
    \coordinate (csx) at ($2*(cew)$);
    \coordinate (csy) at (cl);
    \coordinate (cbx) at ($3*(cew)$);
    \filldraw[fill=green!50!black] (cs) -- ($(cs)+(cel)$) -- ($(cs)+(cel)+(cew)$) -- ($(cs)+(csx)+(csy)$) -- ($(cs)+(cbx)+(cl)$) -- ($(cs)+(cw)-(cbx)+(cl)$) -- ($(cs)+(cw)-(csx)+(cl)$) -- ($(cs)+(cw)-(cew)+(cel)$) -- ($(cs)+(cw)+(cel)$) -- ($(cs)+(cw)$) -- cycle;

    \coordinate (fsl) at (-0.65,0.35);
    \coordinate (fsr) at (1.35,0.35);
    \coordinate (fwl) at (0.75,0);
    \coordinate (fwr) at (-0.75,0);
    \coordinate (fl) at (0,-1.7);
    \coordinate (arcstart) at (0,-0.2);
    \coordinate (arcsep) at (0,-0.5);
    \filldraw[fill=gray!5!white,draw=black] (fsl) -- ($(fsl)+(fwl)$) -- ($(fsl)+(fwl)+(arcstart)$) arc (90:270:0.15) -- ($(fsl)+(fwl)+(arcstart)+(arcsep)$)  arc (90:270:0.15) -- ($(fsl)+(fwl)+(arcstart)+2*(arcsep)$)  arc (90:270:0.15) -- ($(fsl)+(fwl)+(fl)$) -- ($(fsl)+(fl)$) -- cycle;
    \filldraw[fill=gray!5!white,draw=black] (fsr) -- ($(fsr)+(fwr)$) -- ($(fsr)+(fwr)+(arcstart)$) arc (90:-90:0.15) -- ($(fsr)+(fwr)+(arcstart)+(arcsep)$)  arc (90:-90:0.15) -- ($(fsr)+(fwr)+(arcstart)+2*(arcsep)$)  arc (90:-90:0.15) -- ($(fsr)+(fwr)+(fl)$) -- ($(fsr)+(fl)$) -- cycle;

    \coordinate (pic_offset) at (2.75,0);
    \coordinate (fsl2) at ($(fsl) + (pic_offset)$);
    \coordinate (fsr2) at ($(fsr) + (pic_offset)$);
    \filldraw[fill=gray!5!white,draw=black] (fsl2) -- ($(fsl2)+(fwl)$) -- ($(fsl2)+(fwl)+(arcstart)$) arc (90:-90:0.15) -- ($(fsl2)+(fwl)+(arcstart)+(arcsep)$)  arc (90:-90:0.15) -- ($(fsl2)+(fwl)+(arcstart)+2*(arcsep)$)  arc (90:-90:0.15) -- ($(fsl2)+(fwl)+(fl)$) -- ($(fsl2)+(fl)$) -- cycle;
    \filldraw[fill=gray!5!white,draw=black] (fsr2) -- ($(fsr2)+(fwr)$) -- ($(fsr2)+(fwr)+(arcstart)$) arc (90:270:0.15) -- ($(fsr2)+(fwr)+(arcstart)+(arcsep)$)  arc (90:270:0.15) -- ($(fsr2)+(fwr)+(arcstart)+2*(arcsep)$)  arc (90:270:0.15) -- ($(fsr2)+(fwr)+(fl)$) -- ($(fsr2)+(fl)$) -- cycle;
    
    \coordinate (c_shift) at (0.35,-0.25);
    \coordinate (cs2) at ($(cs) + (pic_offset) + (c_shift)$);
    \filldraw[fill=green!50!black] (cs2) circle (0.2);

    \coordinate (c_shiftx) at (0.11,0);
    \coordinate (fsl3) at ($(fsl) + 2*(pic_offset)$);
    \coordinate (fsr3) at ($(fsr) + 2*(pic_offset) + 2*(c_shiftx)$);
    \filldraw[fill=gray!5!white,draw=black] (fsl3) -- ($(fsl3)+(fwl)$) -- ($(fsl3)+(fwl)+(arcstart)$) arc (90:-90:0.15) -- ($(fsl3)+(fwl)+(arcstart)+(arcsep)$)  arc (90:-90:0.15) -- ($(fsl3)+(fwl)+(arcstart)+2*(arcsep)$)  arc (90:-90:0.15) -- ($(fsl3)+(fwl)+(fl)$) -- ($(fsl3)+(fl)$) -- cycle;
    \filldraw[fill=gray!5!white,draw=black] (fsr3) -- ($(fsr3)+(fwr)$) -- ($(fsr3)+(fwr)+(arcstart)$) arc (90:270:0.15) -- ($(fsr3)+(fwr)+(arcstart)+(arcsep)$)  arc (90:270:0.15) -- ($(fsr3)+(fwr)+(arcstart)+2*(arcsep)$)  arc (90:270:0.15) -- ($(fsr3)+(fwr)+(fl)$) -- ($(fsr3)+(fl)$) -- cycle;
    
    \coordinate (c_shifty) at (0,0.3);
    \coordinate (cs3) at ($(cs) + 2*(pic_offset) + (c_shiftx) + (c_shifty)$);
    \coordinate (cl3) at (0,-0.6);
    \filldraw[fill=green!50!black] (cs3) .. controls ($(cs3)+{1/2}*(cw)+{1/4}*(cl3)$) .. ($(cs3)+(cw)$) .. controls ($(cs3)+{0.7}*(cw)+{1/2}*(cl3)$) .. ($(cs3)+(cw)+(cl3)$)  .. controls ($(cs3)+{1/2}*(cw)+{3/4}*(cl3)$) .. ($(cs3)+(cl3)$)  .. controls ($(cs3)+{0.3}*(cw)+{1/2}*(cl3)$) .. cycle;

    \path ($(cs)+{1/2}*(cw)+(0,0.5)$) node {\footnotesize interlocking (convex)}
    ($(cs)+(pic_offset)+{1/2}*(cw)+(0,0.5)$) node {\footnotesize partial enveloping}
    ($(cs)+2*(pic_offset) +{1/2}*(cw)+(0,0.5)$) node {\footnotesize interlocking (concave)};        
\end{tikzpicture}

    \vspace{0.1mm}
    \caption{Different types of macro-scale contact modification.}
    \label{fig:macro}
    \vspace{-3mm}
\end{figure}

Drawing on investigation of tunable surfaces in materials science ____, some recent grip surface designs use active pneumatic control to tune micro-scale contact properties.
%
For instance, Becker et al.\ use local morphology changes to tune friction in their soft gripper by inflating a high-friction elastomeric bladder until it protrudes through holes in a low-friction restraining layer ____.
%
This dual-layer design achieves an impressive increase in shear forces by a factor of five during object sliding. 
%
However, unlike our design, it is implemented in a fully soft gripper, cannot generally envelop or interlock with objects, and does not apply negative pressure. 
%
Tian et al.\ combine soft dry adhesives with a rigid base and use pneumatic control to deflect the adhesive surface in structured active regions ____. 
% 
This rigid-soft gripper increases adhesion force by a factor of 20; however, its dry adhesive is challenging to fabricate and requires specialized equipment. 
%
Moreover, the surface is designed to grip with adhesion alone and does not directly tune friction.
%
Trinh et al.\ design a theoretical rigid-soft fingerpad that, like our design, deflects a flexible surface in an array of small active regions ____. 
% 
However, their design is not fabricated or tested experimentally and does not apply negative pressure. 
% For recent reviews of surfaces with adaptive contact properties in robotics, see ____, and especially ____.
%
In contrast to most other tunable surfaces, our fingerpad design does not rely on moulded components. 
%
Rather, we leverage digitally fabricated and off-the-shelf components that minimize the hands-on fabrication workload and permit straightforward repair and customization. 
%
We also attach our fingerpads to a commercially available robotic gripper rather than using a custom gripper, demonstrating easy adaptation of our design to existing manipulators.