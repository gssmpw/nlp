\section{Discussion}
\label{discussion}
In this section, we discuss some additional meta-observations from the interviews and surveys that did not fit in our taxonomy structure. 

\subsection{Hybrid Quantum-Classical faults vs. Faults in Deep Learning Systems}
Hybrid Quantum-Classical systems bear some similarities to deep learning systems \cite{rieser_tensor_2023}\cite{beer_training_2020}; thus, it  is  natural to compare our taxonomy with similar work in machine learning  \cite{humbatova_taxonomy_2019}. While several categories may seem similar at first, noticeably Optimisation, API, and GPU, the parallel is superficial: the root causes in those categories remain different in Hybrid Quantum-Classical architectures. Other categories are unique to the Hybrid Quantum-Classical architectures.  More specifically, for quantum machine learning (QML), although the optimisation process of the parametric gates is similar to training in deep learning, the encoding of the dataset into qubits is specific to QML and thus, the faults introduced in this architecture cannot be directly related to faults in deep learning.

\subsection{Computer Science vs.  Physics Expertise}
Quantum computing is a multi-disciplinary field. While computer scientists and physicists use the same architectures, they approach them differently  and experience  different problems. This is noticeable in the interviews. Computer scientists typically struggle with conceptualisation, designing the Ansatz and  mapping a problem to a Hamiltonian. Physicists  struggle with the optimisation phase. 
%Several interviewees mentioned problems choosing the right optimiser or dealing with parametrisation issues. 

\subsection{Manual Debugging Methods}
%When discussing approaches to resolving faults leading to wrong behaviours, interviewees mentioned manual debugging strategies that could constitute a good foundation for automated debugging tools. 
Two approaches were mentioned to investigate and resolve the identified faults: one focusing on the optimisation process, and another on the problem conceptualisation. In the first method, the developer detects non-convergence, and first focuses on debugging the implementation and behaviour of the gradients. They would manually debug each gradient at every iteration to check if the results are correctly passed and to detect where the error may originate from. The second approach is more problem-specific and was mentioned by several experts with a Physics background. In case of a wrong behaviour of the system, they would first focus on their problem definition and manually ensure that the Hamiltonian properly represents the problem. The next step would be to manually ensure that the Hamiltonian is correctly mapped into qubits. Mapping a Hamiltonian into qubits is not trivial, and finding an efficient mapping is an open question  \cite{tilly_variational_2022}. They would then focus on the structure of the Ansatz, and ensure no electronic symmetry is broken (see  Section \ref{conceptualisation}). Only then, if no bugs are found in the previous steps, they would proceed to similar steps as the first approach. In \href{https://anonymous.4open.science/r/A-Taxonomy-of-Real-Faults-in-Hybrid-Quantum-Classical-Architectures-EE25/Interviews/Pollished/Interview5_Pollished.docx}{this interview}, the expert gives a detailed explanation of such a procedure. Many of these steps could be automated in the future.


\subsection{Common Problems}
The semi-structured interviews led us to discuss experts' problems beyond our taxonomy. A topic mentioned frequently was the lack of accurate and up-to-date documentation. Quantum platforms tend to mostly document their basic functionalities, leaving advanced features undocumented. Moreover, experts miss a flexible Hybrid Quantum-Classical architecture to experiment on. 
%Most common platforms offer implementations of main algorithms such as VQE. However, it is still challenging to efficiently design a custom hybrid framework.
Tensorflow Quantum \cite{broughton_tensorflow_2021} was introduced with this purpose, but stopped being maintained. Several interviewees mentioned having to implement their connections between machine learning frameworks and quantum simulators, which often introduce compatibility errors and performance issues. With the growing interest in hybrid architecture, researchers and developers would benefit from a dedicated platform that would strongly integrate all components of such architectures and ensure robust and reliable implementations, particularly where quantum and classical parts interact. Although common platforms such as Qiskit and Pennylane allow some support for common NISQ algorithms, such as VQA, experts lack more flexible architecture and templates for various components - typically Ansatz and Optimisation, with clear guidelines for the potential applications and customisations.
