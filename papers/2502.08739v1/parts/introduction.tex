\section{Introduction}

Quantum Computing (QC) offers great promises for boosting current
computational capabilities and enabling solutions to complex and challenging domain-specific problems~\cite{gheorghe-pop_quantum_2020}. In the last four decades, QC has evolved from theoretical concepts to real publicly available hardware and services. However, current computers are limited in their scalability to solve real-life calculations. While the main actors in realising the potential of QC have hitherto been physicists,
computer scientists, and in particular, software engineers, are playing an increasing role in the following
steps~\cite{miranskyy_testing_2019}. Current quantum hardware raises the
need for specific software infrastructures and tools to mitigate its scalability issues~\cite{greiwe_effects_2023}. 

For the foreseeable future, quantum systems are going to be interfaced and complemented with classical computers to achieve large-scale tasks. This has led to a wide-scale adoption of a new type of software architecture called \emph{Hybrid Quantum-Classical architectures}.  
In particular, the current stage of quantum computing  is termed the Noisy
Intermediate-Scale Quantum (NISQ) era, where noisy quantum computation is wrapped around an architecture involving classical optimisation. Several algorithms have been developed
to exploit the potential of current computers and are referred to as
NISQ algorithms~\cite{lau_nisq_2022}, which are but instances of Hybrid Quantum-Classical architectures. %Although scaled-up QC capable of performing fully controllable operation, also termed fault-tolerant quantum computing, will probably not be achieved in the coming few years \cite{lau_nisq_2022}, 
Until we have access to full-scale fault-tolerant quantum computing, we expect a quantum advantage for domain-specific NISQ
algorithms~\cite{herrmann_quantum_2023}. However, even in the presence of fault-tolerant quantum computing, complex hybrid-quantum classical architectures are likely to remain prominent because of the complementary nature of these two computing paradigms.  With such complex architectures comes the need for tailored quality assurance methods and tools. 
%NISQ are a prominent example of Hybrid Quantum--Classical Systems. We expect our taxonomy to be mostly relevant to future Hybrid Systems, potentially including a fault-tolerant quantum part, since we focus on the Hybrid nature of NISQ.

To respond to this need, in this study, we propose a fisrt taxonomy of real faults in Hybrid Quantum-Classical architectures. Although it is focusing on NISQ algorithms, we expect our taxonomy to apply to future Hybrid Quantum-Classical architectures, potentially including fault-tolerant quantum computing. We consider NISQ algorithms, which represent an early version of Hybrid Quantum-Classical architectures, to be an interesting case study to understand the behaviour of Quantum Systems when they are embedded in Classical Computations.
To develop this taxonomy, we considered over 5000 Github issues, preselected 529 of them using rigorously defined inclusion criteria, and analysed them between two authors. We cross-validated our results and selected 133 real faults in Hybrid Quantum-Classical architectures.
 We then structured the results into a taxonomy and further labelled the faults in each classification with more specific descriptions of the nature of the fault. The initial version of the taxonomy was validated and continuously updated through interviews with eleven domain experts. The outcome of the interviews was validated once more through surveys involving a separate group of experts. 


\noindent \textbf{Research questions.}
 To our knowledge, no in-depth analysis of Hybrid Quantum-Classical architecture faults has been conducted. Hence, quantum testing research would benefit from a structured taxonomy of real faults. The following two research questions drive our study: 

\begin{itemize}\itemsep0em
\item  \textit{\textbf{RQ1:} What are typical failure causes in Hybrid Quantum-Classical architectures?} 

The integration of quantum circuits into a loop of classical calculations forms the structure of Hybrid Quantum-Classical architectures. It is natural to expect unique fault categories associated with it, pertaining to the quantum or classical components and their interfaces. We see similarities between such architectures and machine learning architectures, which motivated us to consider similar efforts in that domain~\cite{humbatova_taxonomy_2019} and analyse the similarities in the outcome of our research.  

\item  \textit{\textbf{RQ2: } How can we develop a taxonomy of real faults that captures the specific structure of Hybrid Quantum-Classical architecture?} 

We would like to investigate how real open-source faults (mined from repositories) can be complemented by real faults elicited through interviews with domain experts. We would like to see if the resulting taxonomy can be validated by independent experts and if it can lead to further insights. 

\end{itemize}

Those questions motivated our analysis of real Hybrid Quantum-Classical system faults. 
The contributions of this paper are three-folded.

\begin{itemize}\itemsep0em

\item  \textbf{We benchmarked real Hybrid Quantum-Classical faults.} We empirically analysed public repositories containing Hybrid Quantum-Classical architectures. We searched 5000 fixed issues and selected 133 real faults. 
\item  \textbf{We proposed a first taxonomy of real Hybrid Quantum-Classical architecture faults.} By analysing and categorising those real faults, we developed a taxonomy of real Hybrid Quantum-Classical architecture faults. 
\item  \textbf{We validated this taxonomy against other available datasets and through interviews with experienced developers working in this domain and a survey involving an independent group of experts.} To ensure the validity of our results, we compared them to previous works and presented them to experienced Hybrid Quantum-Classical architecture developers and  researchers. Their feedback was incorporated into our results to create the final taxonomy, validate the outcome, and provide reflective insights on the final results.
\end{itemize}

\noindent \textbf{Structure of the paper.} Section \ref{background} introduces quantum computing and key concepts to understand Hybrid Quantum-Classical architectures. In Section \ref{relatedwork}, we review the related work, both in classical and quantum software engineering. Section \ref{methodology} reviews the methodology of this study before presenting our results in Section \ref{results} and discussing them in Section \ref{discussion}. We state threads to the validity of our study in Section \ref{threats to validity} before concluding in Section \ref{conclusion} and suggest some avenues for future work.