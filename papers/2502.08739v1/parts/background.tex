\section{Background}
\label{background}

This section briefly introduces Quantum Computing and Hybrid Quantum-Classical architectures.

\subsection{Quantum Computing}
Quantum computing (QC) was conceptualised to simulate quantum phenomena based on the postulates of quantum mechanics~\cite{feynman_simulating_1982}. It was later found to have potential applications that could offer significant speed-up over its classical counterpart. The resulting potential speed-up is often referred to as "quantum supremacy"~\cite{arute_quantum_2019}. However, demonstrating quantum supremacy on real hardware remains a long-standing challenge, especially at a scale where it would solve real-life calculations. Most agree this stage of QC will likely last for the next few years if not decades, and refer to it as the NISQ era~\cite{preskill_quantum_2018}. Regardless of the trajectory to large-scale fault-tolerant computing, quantum computers will have a different type of strength than classical computers and are going to be complemented by classical computers in hybrid architectures. 
 NISQ algorithms are a prominent example that hybrid architecture combining small quantum circuits with classical computations could present some computational advantages~\cite{lau_nisq_2022}. Variational Quantum Algorithms (VQA) are the most common example of an efficient combination of a reduced quantum circuit inside a classical optimisation loop~\cite{tilly_variational_2022}. We present VQA as an example of Hybrid Quantum-Classical architecture in more detail at the end of this section.\\

\subsection{Quantum Circuits}

The core of a Hybrid Quantum-Classical architecture is a Quantum circuit. A quantum circuit comprises quantum gates operating on qubits, the quantum analogue to classical bits.  Two distinctive features of qubits are that they may feature a superposition of states,  represented by a linear combination of states, and entanglement, where several qubits form a composite system such that measurement on one determines the state of the other. 

\paragraph{Composition of a circuit}
A quantum circuit used in a Hybrid Quantum-Classical architecture typically begins with the initialisation of qubits and ends with their measurement, where the information is translated into classical bits. The purpose of a quantum circuit is to harness the principles of quantum mechanics, such as superposition and entanglement, to perform computational tasks that would be infeasible or significantly slower for ``quantum parallelism'', i.e., ``evaluating a function on many different values simultaneously''~\cite{nielsenChuang10}.

Fig.~\ref{ghz} represents a simple quantum circuit, creating a three-qubit Greenberger-Horne-Zeilinger (GHZ) state, which is a specific entangled quantum state of three qubits. The GHZ state signifies that the three qubits are entangled in such a way that if the state of one qubit is measured, the states of the other two qubits are also determined. Input variables q[0], q[1], and q[3] represent the 3 input qubits. The horizontal line represents time, the blue elements are quantum logical gates, and the black elements are measurements where the qubit is observed and its state extracted into a classical bit. The maximum number of gates along all horizontal lines in a circuit is called depth. Fig.~\ref{ghz} represents a circuit of depth 3. \\

\begin{figure}[hbt!]
%\includegraphics[width=\linewidth, scale=0.2]{Figures/ghz.png}
\includegraphics[width=\linewidth]{Figures/ghz.png}
\caption{Simple Quantum circuit creating a GHZ state.}
\label{ghz}
\end{figure}

\paragraph{Quantum Noise}
There are different sources of noise in quantum computing, including unwanted entanglement among particles and imperfection in gates. 
Noise accumulates throughout the quantum circuit~\cite{clerk_introduction_2010}. Any quantum system suffers decoherence, induced by the interactions with its environment, causing it to lose its quantum behaviour over time~\cite{lau_nisq_2022}. Hence, current Hybrid Quantum-Classical architectures, e.g., NISQ algorithms, aim to reduce the number of qubits and the gate depth of the circuit to keep the noise level manageable. \\

\subsection{Hybrid Quantum-Classical Architectures}
In this section, we take a closer look at NISQ algorithms as prominent examples of Hybrid Quantum-Classical architectures currently in use, particularly, in open-source repositories for quantum computing. 

\paragraph{Hybrid Structure}
Hybrid Quantum-Classical Architectures typically have four phases. % as shown in Fig.~\ref{NISQStruct}.

\begin{enumerate}\itemsep0em

\item  \textit{Classical pre-processing.} The problem is translated into a cost function, and qubits are initialised.
\item  \textit{Quantum Circuit.} This is the core of a Hybrid Architecture. It typically has a low gate depth and few qubits to manage the noise level. Current Hybrid Architectures utilise around 50 qubits, and 1000 gates at maximum, with an average gate depth of 20~\cite{lau_nisq_2022}.
\item  \textit{Loop of classical computation.} The Quantum Circuit results are processed and a new quantum circuit is initialised. This process is repeated in a loop.
\item  \textit{Classical post-processing.} Once the desired results are reached, it is extracted and processed to answer the original problem.

\end{enumerate}

% \begin{figure}[hbt!]
% \includegraphics[width=\linewidth]{Figures/NISQ_Structure.png}
% \caption{A simple overview of a NISQ System's Structure}
% \label{NISQStruct}
% \end{figure}


\paragraph{Variational Quantum Algorithms (VQAs)}
\label{vqa seq}
VQAs are among the most promising examples of NISQ algorithms \cite{tilly_variational_2022}. 
The main goal of a VQA is to find the optimal parameters for a parameterized quantum circuit, leading to a solution for a given computational problem. We provide an overview of the structure and functioning of a VQA. This structure is typical of Hybrid Architecture. Future Fault-Tolerant Quantum Computing could replace current noisy hardware.

\begin{enumerate}\itemsep0em

\item[1.]  \textit{Problem Definition.} The first step involves defining a computational problem that can benefit from quantum computing and translating it into an objective or cost function. In most applications, it consists of a Hamiltonian representation (a matrix representing system state evolution), connecting the problem domain to the energy level of a quantum system.

\item[2.]  \textit{Optimisation Process.} A parameterized quantum circuit, known as the variational ansatz, is designed. This circuit contains gates with adjustable parameters, denoted as $\theta$. It is initialised with an input $\theta_0$. A quantum state $\ket{\psi(\theta)}$ is measured, and the outcomes are used to compute the expectation value of the objective function $O(\theta, \{\braket{H}_{U(\theta)}\})$.

\item[3--4.]  \textit{Convergence Check and Output.} The optimization process continues until a convergence criterion is met, indicating that further iterations are unlikely to significantly improve the solution. This convergence check ensures that the algorithm has reached a stable and potentially optimal solution. The final set of optimized parameters $\theta_{opt}$ represents the solution to the quantum problem. This solution can be used for further analysis or as the output of the VQA.

\end{enumerate}

An in-depth explanation and a figure detailing figure can be found in the work of Bharti et al. \cite{bharti_noisy_2022}. Fig.~\ref{VQE} \cite{tilly_variational_2022} is a diagrammatic representation of a VQA in which each of the described steps can be found.

\begin{figure}[hbt!]
\includegraphics[width=\linewidth, scale=0.2]{Figures/VQE.pdf}
\caption{Detailed structure of a VQA \cite{tilly_variational_2022}}
\label{VQE}
\end{figure}