\section{Related work}
\label{relatedwork}

\subsection{Classical Computing}
\paragraph{Bug Benchmarks} 
Building datasets of reproducible real faults is a common and valuable practice in classical software engineering. These datasets offer a more realistic and eclectic base for software testing research on artificially generated faults. Noticeably, Defects4J gathers reproducible real faults in Java \cite{just_defects4j_2014}. It was used in several studies, for instance, to inform the design of testing frameworks \cite{martinez_automatic_2017} and to experiment with software engineering techniques \cite{aleti_defects4j_2020}. We expect our study could similarly benefit research in Hybrid Quantum-Classical architectures and serve as an experimentation base for future studies. \\

\paragraph{Real Faults Taxonomy}
Taxonomies are a common way to propose a structured overview of a given problem in software engineering \cite{UsmanEtAl017}. A study that inspired our work proposes a taxonomy of real faults in deep learning systems \cite{humbatova_taxonomy_2019}. They analysed GitHub commits and issues for popular DL frameworks and related Stack Overflow posts to propose a taxonomy of real DL faults. They validated their work by interviewing 20 developers and integrating feedback received to their results. Our work follows a similar methodology applied to Hybrid Quantum-Classical architecture. Building a fault dataset and a taxonomy is a process that can be transferred to a different branch of software engineering since the data available online is similar. Also, the developers' interview structure can be easily adapted to another speciality. 

\subsection{Quantum Computing}
\paragraph{Real Faults in Quantum and Hybrid architectures}
Quantum software testing is challenging. Several studies point to a growing need for developing testing and debugging tools specific to quantum programs \cite{miranskyy_testing_2019}\cite{paltenghi_bugs_2022}\cite{metwalli_tool_2022}\cite{gill_quantum_2022}\cite{pontolillo_multi-lingual_2023}. Some benchmarks of quantum bugs and bug fixes are available in the literature \cite{zhao_bugs4q_2021}\cite{campos_qbugs_2021}\cite{luo_comprehensive_2022}, as well as a low-level Quantum Benchmark Suite, QASMBench, focusing on
NISQ Evaluation and Simulation \cite{li_qasmbench_2023}. Zhao et al. \cite{zhao_identifying_2021} identified some bug patterns in Quantum Programs after an in-depth analysis of Qiskit programs. In our study, we used one of them, namely, the Bugs4Q dataset \cite{zhao_bugs4q_2021} to validate our search process. Although Bugs4Q has a broad scope and does not focus on Hybrid Quantum-Classical architectures, it contains one NISQ bug. In their systematic literature review, Gill et al. \cite{gill_quantum_2022} briefly mention Hybrid Quantum-Classical architectures as a priority research area, without mentioning testing techniques for them. To our knowledge, real faults in Hybrid Quantum-Classical architectures, such as NISQ algorithms,  have not been investigated yet. The closest study to ours concerns faults in Quantum Machine Learning, which is a special case of a NISQ algorithm \cite{zhao_empirical_2023}. We used both datasets to validate our search query and process. We notice a growing interest in recent literature in Quantum testing and quality assurance, and very recently some studies have started to emphasise their investigation around NISQ and Hybrid Quantum-Classical architectures. 

\paragraph{Other taxonomy}
Parallel to our work, Zappin et al. \cite{zappin_when_2024} recently released a comprehensive study characterizing
Hybrid Quantum-Classical issues discussed in
Developer Forums, which represents the closest work to ours to date. However, they solely investigated discussions available on Xanadu Discussion Forums and on QC Stack Exchange. The issues they used do not originate from GitHub and do not follow the strict inclusion criteria we used as described in Section \ref{methodology}, hence there is no overlap between our two works. Their study gave an overview of current causes of issues in a software engineering perspective, gathering them around the following categories: Software Faults, Hardware/Simulator issues, Configuration issues, Developper Errors, Library and Plateform Issues. Our study characterises faults in Hybrid Quantum-Classical architectures around their structuring components, and identify their typical weaknesses, aiming to further understand their nature and stay relevant to future versions of such architectures. These two taxonomies are therefore complementary. They were conducted in parallel, without consulting one another, which highlights once more the relevance of such a work.