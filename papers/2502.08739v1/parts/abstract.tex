With the popularity of Hybrid Quantum-Classical architectures, particularly noisy intermediate-scale quantum (NISQ) architectures,  comes the need for quality assurance methods tailored to their specific faults. In this study, we propose a taxonomy of faults in Hybrid Quantum-Classical architectures accompanied by a dataset of real faults in the identified categories. To achieve this, we empirically analysed open-source repositories for fixed faults. We analysed over 5000 closed issues on GitHub and pre-selected 529 of them based on rigorously defined inclusion criteria. We selected 133 faults that we labelled around symptoms and the origin of the faults. We cross-validated the classification and labels assigned to every fault between two of the authors. As a result, we introduced a taxonomy of real faults in Hybrid Quantum-Classical architectures. Subsequently, we validated the taxonomy through interviews conducted with eleven developers.  The taxonomy was dynamically updated throughout the cross-validation and interview processes.  The final version was validated and discussed through surveys conducted with an independent group of domain experts to ensure its relevance and to gain further insights.