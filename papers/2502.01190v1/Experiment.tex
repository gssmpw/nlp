% 1-1. Dataset
% 1-2. Implementation details
% 2. Comparison
% 3. Ablation Studies

\paragraph{Datasets} We use public music-dance paired dataset for music-driven dance generation, FineDance \cite{li2023finedance}, which includes 22 dance genres. FineDance claims to contain 14.6 hours of music-dance paired data with detailed hand motions and accurate postures. However, it actually has 7.7 hours, totaling 831,600 frames at a frame rate of 30fps, and includes 16 different dance genres. Our entire model is trained and tested on this dataset. We create dance sequences, each comprising 1024 frames long dance sequences (equivalent to 34.13 seconds), using music from the test set of the FineDance dataset as input.
% FineDance implementation
\paragraph{Implementation details} 
We follow the experimental settings of Lodge for the global music feature, the local music feature and the generation of characteristic dance primitives. The global music feature extracted from the music and audio analysis library Librosa \cite{mcfee2015librosa} has a length of 1024 (34.13 seconds), and the local music feature input for the Local Diffusion has a length of 256 (8.53 seconds). As output of the global diffusion, we obtain 13 characteristic dance primitives which consisting of 5 coarse dance motions and 8 fine dance motions. These 8 fine dance motions are then used as input for choreography augmentation augment to 16 fine dance motions by mirroring and aligned with the input music's beat.

\subsection{Comparison on the FineDance dataset}
In this section, we compare our method with the several past works including Lodge. FACT \cite{li2021ai} and MNET \cite{kim2022brand}are auto-regressive dance generation methods. Bailando \cite{siyao2022bailando} is a follow-up approach that employs VQ-VAE to transform dance movements into tokens and achieve outstanding qualitative performance. EDGE represents a significant advancement in the field of dance generation by utilizing diffusion models to achieve substantial performance improvements. Specifically, EDGE introduces a transformer-based diffusion model paired with Jukebox, conferring powerful editing capabilities and setting a new vision in generating realistic and physically plausible dance motions. Lodge is a network designed to generate extremely long dance sequences conditioned on music. It employs a two-stage coarse-to-fine diffusion architecture with characteristic dance primitives as intermediate representations between two diffusion models. It significantly outperforms existing models in generating coherent, high-quality, and expressive dance sequences.

\paragraph{Motion Quality \& Diversity}
Motion quality is assessed using two primary measures. The Frechet Inception Distance (FID) measures the distance between the generated dance motion features and the ground truth dance sequences, indicating the quality of the motion. Separate FID scores are reported for kinematic features ($\mathrm{FID}_k$) and geometric features ($\mathrm{FID}_g$). Additionally, the Foot Skating Ratio (FSR) calculates the proportion of frames where the feet slide on the ground instead of making solid contact, with lower values indicating better physical realism. Motion diversity is evaluated through the diversity in kinematic features (Div$_k$), which assesses the variety in the generated dance motions based on kinematic properties such as speed and acceleration. The diversity in geometric features (Div$_g$) evaluates the diversity of generated dance movements by examining geometric properties and predefined movement templates.



\paragraph{Beat Aligment Score (BAS)}
The Beat Alignment Score (BAS) measures how well the generated dance sequences align with the beats of the accompanying music, reflecting the synchronization between music and dance.

\paragraph{Production Efficiency}
Production efficiency is measured by the average run time required to generate a specific length of dance sequence, such as 1024 frames, to evaluate the computational efficiency of the model. All experiments were conducted on the same computer equipped with 8 Nvidia A100 GPUs.



\begin{table}[ht]
    \centering
    \caption{Comparison with state-of-the-art methods on the FineDance dataset. Wins represent the ratio of victories Lodge (DDPM) achieved in the user study.}
    \resizebox{\textwidth}{!}{
    \begin{tabular}{@{\hskip 3pt}l@{\hskip 3pt}c@{\hskip 3pt}c@{\hskip 3pt}c@{\hskip 3pt}c@{\hskip 3pt}c@{\hskip 3pt}c@{\hskip 3pt}c@{\hskip 3pt}c}
        \toprule
        Method & \text{FID$_k$} $\downarrow$ & \text{FID$_g$} $\downarrow$ & \text{FSR} $\downarrow$ & \text{Div$_k$} $\uparrow$ & \text{Div$_g$} $\uparrow$ & \text{BAS} $\uparrow$ & \text{Run Time} $\downarrow$\\
        \midrule
        Ground Truth & / & / & 6.22\% & 9.73 & 7.44 & 0.2120 &\\
        \midrule
        FACT \cite{li2021ai} & 113.38 & 97.05 & 28.44\% & 3.36 & 6.37 & 0.1831 & 35.88s \\
        MNET \cite{kim2022brand} & 104.71 & 90.31 & 39.36\% & 3.12 & 6.14 & 0.1864 & 38.91s \\
        Bailando \cite{siyao2022bailando} & 82.81 & 28.17 & 18.76\% & 7.74 & 6.25 & 0.2029 & 5.46s  \\
        EDGE \cite{tseng2023edge} & 94.34 & 50.38 & 20.04\% & 8.13 & 6.45 & 0.2116 & 8.59s\\
        Lodge (DDIM) \cite{song2020denoising} & 50.00 & 35.52 & 2.76\% & 5.67 & 4.96 & 0.2269 & 4.57s \\
        Lodge (DDPM) \cite{ho2020denoising} & 45.56 & 34.29 & 5.01\% & 6.75 & 5.64 & 0.2397 & 30.93s \\
        \midrule
        (Lodge + Ours) (DDPM) & \textbf{42.76} & \textbf{32.17} & 4.97\% & 5.79 & 5.24 & \textbf{0.2501} & 31.17s \\
        \bottomrule
    \end{tabular}}
    \label{tab:comparison}
\end{table}


As seen in Table 1, Our proposed model showed an approximately 4\% improvement in BAS scores compared to the baseline model Lodge, and the FID scores increased by an average of about 6\%, achieving state-of-the-art (SOTA) performance. Of course, to fully validate our proposal, experiments on various benchmarks are necessary. However, it is significant that we improved performance on the widely-used FineDataset without significantly increasing computer resources or runtime.