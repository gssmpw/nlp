% 1. Proposed Method
\paragraph{Conclusion} In this work, we propose R-Lodge enhanced version of Lodge which incorporates Dance Recalibration Using Recurrency Block. Proposed method address the burst changes between adjacent coarse dance motions from original Lodge which interfere the smoothness in generated dance motion. Through the evaluation of our generated samples, R-Lodge demonstrates smoother dance motions than Lodge while maintaining a comparable level of overall dance coherence. However, our experiment is limited to the FineDance dataset\cite{li2023finedance}, which hinders generalization to other datasets. Furthermore, we have not yet proposed a solution to improve the weak motion diversity, which is a limitation of the original Lodge approach.
% 2. Future Work
% 2.1. Experiment on different Dataset
% 2.2. Proposing new music feature incorporated with music genre feature
\paragraph{Future Work} For future work, as we mentioned earlier, experimentation with other datasets, e.g. AIST++\cite{li2021ai}, is necessary to demonstrate that our approach is not limited to FineDance\cite{li2023finedance} dataset. Following this, our next objective is to increase the diversity of dance motions generated from input music via music frame genre embedding. Original Lodge reflects the common music genre embedding for whole frames before Local Diffusion. We aim to extract the genre of each individual music frame and integrate it with the corresponding frame of music feature. This integration will enable the global diffusion process to be conditioned not only on the music frame feature but also on its genre embedding. We anticipate it will not compromise the coherence of overall dance motions but rather increase diversity by embodying the subtle variations in genre features from each frame of the music.
