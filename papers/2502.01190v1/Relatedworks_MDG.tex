% EDGE하고 LODGE보면 M2D 관련 related Works 있습니다.
Dance and music are inseparable that music provides the foundation for the movement and emotion of dance while we express the characteristic of music by dance. Accordingly, abundant studies make a goal to generate quality dance conditioned on music.

% Introduce various methods
% 각 방법론을 통한 참고 문헌들, 각 방법들이 마주하는 문제들?
The generation of dance sequences driven by music has been an active area of research, focusing on synchronizing dance movements with musical inputs. Traditional motion-graph methods approached this as a similarity-based retrieval problem, limiting the diversity and creativity of the generated dance sequences. However, recent advancements in deep learning have led to more aesthetically appealing results.

Sequence-based methods using LSTM and Transformer networks predict subsequent dance frames in an autoregressive manner. For example, FACT \cite{li2021ai} inputs music and seed motions into a Transformer network to generate new dance frames frame by frame. However, challenges such as error accumulation and motion freezing persist. VQ-VAE is another well-known approach, as seen in methods like Bailando \cite{siyao2022bailando}, which incorporates reinforcement learning to optimize rhythm and maintain high motion quality. However, the pre-trained codebook in VQ-VAE can limit diversity and hinder generalization.

GAN-based methods like MNET \cite{kim2022brand} employ adversarial training to produce realistic dance clips and achieve genre control, though they often face issues like mode collapse and training instability. Diffusion-based methods have shown significant progress in generating high-quality dance clips. For instance, EDGE \cite{tseng2023edge} uses diffusion inpainting to generate consistent dance segments, while FineDance \cite{li2023finedance} introduces diffusion models to produce diverse and high-quality dance sequences.

Lodge \cite{li2024lodge} represents a notable advancement in this domain, employing a coarse-to-fine diffusion framework to generate extremely long dance sequences. By introducing characteristic dance primitives, Lodge ensures both global choreographic patterns and local motion quality. Additionally, the foot refine block in Lodge addresses artifacts such as foot-skating, enhancing the physical realism of the generated dances. However, Lodge has its limitations. It currently cannot generate dance movements with hand gestures or facial expressions, which are crucial elements in dance performances. Moreover, while Lodge aims to produce diverse dance sequences, its reliance on characteristic dance primitives may limit the overall diversity of the generated dances. This limitation opens avenues for future research to further enhance both the realism and the variety of the dance sequences.