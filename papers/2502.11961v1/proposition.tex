\section{Preliminary results}
\label{sec:polynomial}
In the remainder of this paper we present a number of algorithmic results; here we start with several initial results first showing that we can efficiently check if a given sequence is reconfigurable, and then presenting a condition for a reconfiguration problem to belong to the same approximation class as its static version.

\subsection{Checking if a sequence is reconfigurable}

We show that testing if a sequence $(T_1,\dots,T_\tau)$ is a reconfigurable sequence of a temporal graph $\mathcal{G}=(G,\lambda)$ can be done in polynomial-time. We begin with the smaller problem of checking whether one state can be reconfigured into another. This can be done by computing a perfect matching in a bipartite graph that captures what set changes are possible between time-steps.  Details are deferred to the appendix. 

\begin{lemmarep}
  \label{lemma:check reconfiguration two sets}
Let $G$ be a static graph and let $T_1$ and $T_2$ be two subsets of vertices. We can determine in $\mathcal{O}((|V(G)| + |E(G)|) \cdot \sqrt{|V(G)|})$ time if $T_1$ is reconfigurable into $T_2$.
\end{lemmarep}  
\begin{proof}
  We first contruct a bipartite graph $H$ with vertex set $V(H) = \{v_i \mid x_i \in T_1\} \cup \{u_i \mid x_i \in T_2\}$ and edge set $E(H) = \{v_iu_i \mid x_i \in T_1\cap T_2\} \cup \{v_iu_j \mid x_i \in T_1, x_j\in T_2, x_ix_j \in E(G)\}$.
  We show that there is a reconfiguration bijection between $T_1$ and $T_2$ if and only if there is there is a perfect matching in $H$.
  Let $f$ be a reconfiguration bijection between $T_1$ and $T_2$. Consider the set of edges $M = \{ v_iu_j \mid x_i \in T_1, f(x_i) = x_j\} \cup \{ v_iu_i \mid x_i \in T_1, f(x_i) = x_i\}$. If $f(x_i)=x_i$, then $x_i \in T_1 \cap T_2$ and by construction of $H$, the edge $v_iu_i$ belongs to $H$. If $f(x_i)=x_j$, then $v_i \in T_1$, $v_2 \in T_2$ and $x_ix_j \in E(G)$ and again by construction of $H$, $v_iv_j$ belongs to $H$. Hence, $M \subseteq E(H)$ and since $f$ is a bijection between $T_1$ and $T_2$, $M$ is a perfect matching in $H$.
  %
  Now let $M$ be a perfect matching in $H$. For each edge $v_iu_j \in M$ (possibly $i=j$), we set $f(x_i)=x_j$. Clearly, each vertex of $T_1$ has eaxctly one image and each vertex of $T_2$ has exactly one inverse image, thus $f$ is a bijection. For each $f(x_i)=x_j$ (with $i\neq j$), we have $v_iu_j \in M$ and by construction of $H$, there is an edge between $x_i$ and $x_j$ in $G$. Hence, $f$ is a reconfiguration bijection.
  %
  Finally, the graph $H$ can be constructed in $\mathcal{O}(|E(G)|+|V(G)|)$ and the perfect matching can be computed in $\mathcal{O}((|E(H)|+|V(H)|) \cdot \sqrt{|V(H)|})$~\cite{Hopcroft1971ANA}. We obtain an overall complexity of $\mathcal{O}((|V(G)| + |E(G)|) \cdot \sqrt{|V(G)|})$.
\end{proof}

We call a sequence $T=(T_1,\dots,T_\tau)$ in temporal graph $\mathcal{G}$ \emph{reconfigurable} if for every $1 \leq i  < \tau$, the set $T_i$ is reconfigurable in $\mathcal{G}$ to $T_{i+1}$.  We now extend the previous result to show that testing if a sequence is reconfigurable can also be done in polynomial-time.  


\begin{corollaryrep}
   \label{cor:check reconfiguration sequence}
Let $\mathcal{G}=(G,\lambda)$ be a temporal graph. Let $T=(T_1,\dots,T_\tau)$ be a sequence such that for all $i \in [\tau]$, $T_i \subseteq V(G)$. We can determine whether $T$ is a reconfigurable sequence in $\mathcal{O}(\tau \cdot (|V(G)|+|E(G)|\cdot\sqrt{|V(G)|}))$ time.
\end{corollaryrep}

\begin{proof}
A sequence $(T_1,\dots,T_\tau)$ is reconfigurable if and only if for each $i \in [\tau-1]$, $T_i$ is reconfigurable into $T_{i+1}$ in $G_i$. Hence, by \Cref{lemma:check reconfiguration two sets}, we can test if $(T_1,\dots,T_\tau)$ is reconfigurable in $\mathcal{O}(\tau \cdot (|V(G)|+|E(G)|\cdot\sqrt{|V(G)|}))$.
\end{proof}
 

\subsection{Approximation}

We now turn our attention to the approximation of reconfiguration problems. An optimisation problem belongs to the approximation class $f(n)$-APX if it is possible to approximate this problem in polynomial-time with a $\mathcal{O}(f(n))$ approximation factor. Let $\Pi$ be a minimisation static problem. A $f$-approximation algorithm for $\Pi$ is a polynomial-time algorithm that, given a graph $G$, returns an approximate solution $X_{app}$ such that, we have $|X_{app}|\leq f(G) \cdot |X_{opt}|$, where $X_{opt}$ is an optimal solution for $\Pi$ in $G$. 
 \begin{theoremrep}
   \label{th:approx}
Let $\Pi_S$ be a minimisation static graph problem such that for any two solutions $S_1$ and $S_2$, $S_1 \cup S_2$ is also a solution. Let $\Pi_T$ be the corresponding reconfiguration version of $\Pi_S$. If $\Pi_S$ is $f$-approximable, then $\Pi_T$ is $\tau\cdot f$-approximable.
\end{theoremrep}

\begin{proof}
  Let $app_S$ be a polynomial-time approximation algorithm for $\Pi_S$.
  %
  Let $\mathcal{G}=(G,\lambda)$ be a temporal graph. Let $X_{app} = \bigcup\limits_{t\in [1,\tau]} app_S(G_t)$ be the union of every approximate solution of $\Pi_S$ in all snapshots of $\mathcal{G}$.
  We show that the algorithm $app_T$ that returns the reconfigurable sequence $T_{app} = (X_{app},\dots,X_{app})$ (i.e. $T_{app}$ always contains $X_{app}$ as set of selected vertices) is a $\tau \cdot f(G)$ approximation algorithm for $\Pi_T$.  
  %
 $T_{app}$ is obviously a reconfigurable sequence for $\Pi_T$, thus we only need to show that it achieves the desired ratio.
  For each time-step $t \in [1,\tau]$, let $X^t_{opt}$ denote the optimal solution for $\Pi_S$ in $G_t$. Let $T_{opt}=(T_1,\dots,T_\tau)$ be an optimal reconfiguration sequence for $\Pi_T$ in $\mathcal{G}$. Notice that for each time-step $t$, we have $|X^t_{opt}| \leq |T_t|$.  For all $t$, we have $|app_S(G_t)| \leq f(|G_t|) \cdot |Opt_t|$ and thus, $|app_S(G_t)| \leq f(|G_t|) \cdot |T_t|$. It follows that $|X_{app}| = |T_{app}| \leq \tau \cdot f(G) \cdot |T_{opt}|$. We can conclude that $app_T$ is a $\tau\cdot f(G)$ approximation algorithm.
\end{proof}


{\sc Dominating Set} is known to be Log-APX-complete~\cite{EscoffierP06}, \textit{i.e.} there is a polynomial $\mathcal{O}(log)$-approximation algorithm and there is no polynomial-time approximation algorithm with a constant ratio. Thus, we obtain the following result.

\begin{corollary}
{\sc Temporal Dominating Set Reconfiguration} is Log-APX-complete.
\end{corollary}



\section{Fixed-parameter tractability with respect to the enumeration time of the static version}
\label{sec:enum}

We show in this section that if the number of solutions for a static problem $\Pi_S$ can be bounded by some function $f(G)$, then the reconfiguration version is in FPT with respect to $f(G)$.

\begin{lemmarep}
  \label{lemma:enumeration}
  Let $\Pi_S$ be a static problem and let $\Pi_T$ be the reconfiguration version of $\Pi_S$. Let $\mathcal{G}=(G,\lambda)$ be a temporal graph such that for all $t \in [\tau]$, all solutions in $G_t$ for $\Pi_S$ can be enumerated in $\mathcal{O}(f(G))$. We can solve $\Pi_T$ in $\mathcal{O}(f(G) \cdot \tau \cdot (|V(G)|+|E(G)|)\cdot\sqrt{|V(G)|})$.
\end{lemmarep}


\begin{proof}
  For all $t \in [\tau]$, let $X_t$ denote the set of solutions of $\Pi_S$ in $G_t$. For each $t \in [\tau]$, denote by $Y_t \subseteq X_t$ such that for all $T_t \in Y_t$, there exists a reconfigurable sequence $(T_1,\dots,T_t)$  in the temporal graph containing the $t$ first snapshots of $\mathcal{G}$.
  Notice that $Y_\tau$ contains all set $T_\tau$ such that there is a reconfigurable sequence ending with $T_\tau$ that is a solution for $\Pi_T$. Hence, to obtain the optimal size for $\Pi_T$ in $\mathcal{G}$, it suffices to take the size of the best (minimised or maximised as required by $\Pi_S$) set  $T_\tau \in Y_\tau$ for $\Pi_S$ in $G_\tau$.
  %
  
  We now show by induction how to construct $Y_t$ for each $t \in [\tau]$.
  For the base case, we set $Y_1 = X_1$. Clearly $Y_1$ respects the induction hypothesis.
  Now, for each $t \in [2,\dots,\tau]$, we set $Y_t = \{ T_t \in X_t \mid \exists T_{t-1} \in Y_{t-1} \texttt{ st $T_{t-1}$ is reconfigurable into $T_t$ in $G_{t-1}$}\}$. Let $T_t \in Y_t$. First, since $T_t \in X_t$, $T_t$ is a solution for $\Pi_S$ in $G_t$. Then, for each $T_t \in Y_t$, there is $T_{t-1} \in Y_{t-1}$ such that $T_{t-1}$ is reconfigurable into $T_t$ in $G_{t-1}$. By induction hypothesis, there is a reconfigurable sequence $(T_1,\dots,T_{t-1})$ for $\Pi_T$ in the temporal graph containing the $t-1$ first snapshots of $\mathcal{G}$. Hence, we can conclude thay the sequence $(T_1,\dots,T_{t-1},T_t)$ is a reconfigurable sequence for $\Pi_T$ in the temporal graph containing the $t$ first snapshots of $\mathcal{G}$. Hence, all sets in $Y_t$ respect the induction hypothesis.
  Now let $(T_1,\dots,T_t)$ be a reconfigurable sequence for $\Pi_T$ in the temporal graph containing the $t$ first snapshots of $\mathcal{G}$. That is, $T_{t-1}$ is reconfigurable into $T_t$ in $G_t$ and $(T_1,\dots,T_{t-1})$ be a reconfigurable sequence for $\Pi_T$ in the temporal graph containing the $t-1$ first snapshots of $\mathcal{G}$. By the inductive hypothesis, $T_{t-1} \in Y_{t-1}$ and so, by construction of $Y_t$, $T_t$ belongs to $Y_t$. We can conclude that $Y_t$ respects the recurrence hypothesis.
  
  Since $|X_t| \leq \mathcal{O}(f(G))$, by \Cref{lemma:check reconfiguration two sets} each $Y_t$ can be computed in $\mathcal{O}(f(G)\cdot (|V(G)| + |E(G)|)\cdot \sqrt{|V(G)|})$. Hence, computing $Y_\tau$ can be done in $\mathcal{O}(f(G)\cdot \tau \cdot (|V(G)| + |E(G))|\cdot \sqrt{|V(G)|})$. Notice that it is possible to associate each set $T_t$ with a sequence $(T_1,\dots,T_t)$ to make a constructive algorithm.
\end{proof}


% \subsection{FPT with respect to maximum number of edges in any snapshot}
% \label{sec:FPT with max edges}

Let $|E|_{\max}$ denote $\max_{t\in |\tau|}|E_t|$, the maximum number of edges in any snapshot of the temporal graph $\mathcal{G}$.
We use \Cref{lemma:enumeration} to show inclusion of \textsc{Temporal Dominating Set Reconfiguration} in FPT with respect to $|E|_{\max}$. First, we show that the number of dominating sets is bounded by a function of $|E|_{\max}$ in each snapshot.

\begin{lemmarep}
  Let $\mathcal{G}=(G,\lambda)$ be a temporal graph. For each $t \in [\tau]$, all dominating sets of $G_t$ can be enumerated in $\mathcal{O}(2^{|E|_{max}} \cdot |E|_{max})$.
\end{lemmarep}

\begin{proof}
  Let $G_t$ be a snapshot of $\mathcal{G}$, $X$ be the set of vertices of degree zero in $G_t$ and $Y = V(G) \setminus X$. 
  Notice that in any dominating set $D$ of $G_t$, $X \subseteq D$. Hence, to enumerate all dominating sets in $G_t$, it suffices to list every dominating set in the subgraph induced by $Y$.  For each enumerated dominating set $D$ for $G_t[Y]$, a dominating set for $G_t$ can be obtained by taking the union of $X$ and $D$.
  To list all dominating sets $D$ for $G_t$, we enumerate all subsets of $Y$ and check for each one whether it forms a dominating set. The enumeration of all subsets of $Y$ can be done in $\mathcal{O}(2^{|Y|})$ and checking whether a subset of $Y$ is a dominating set can be done in $\mathcal{O}(|Y| + |E|_{max})$.
  Since we have $|Y| \leq |E|_{max} +1$, we can conclude that all dominating sets of $G_t$ can be enumerated in $\mathcal{O}(2^{|E|_{max}} \cdot |E|_{max})$.
\end{proof}

\noindent Now, we can conclude that \textsc{Temporal Dominating Set Reconfiguration} is in FPT with respect to $|E|_{\max}$.

\begin{corollary}
{\sc Temporal Dominating Set Reconfiguration} can be solved in $\mathcal{O}(2^{|E|_{max}} \cdot \tau \cdot (|V(G)|+|E(G)|\cdot\sqrt{|V(G)|}))$.
\end{corollary}

% \subsection{FPT with respect to minimum-size clique cover}

% Let $G$ be a static graph. A \emph{clique cover} $P = \{X_1,\dots,X_k\}$ of $G$ is a partition of its vertices into cliques. We 

% \begin{lemma}
% Let $\mathcal{G}=(G,\lambda)$ be a temporal graph such that there are two constants $k_1$ and $k_2$ such that for each $t \in [\tau]$, the vertices of $G_t$ can be partitionned into at most $k_1$ vertex-disjoint cliques of size at most $k_2$. For each $t \in [\tau]$, all independent sets of $G_t$ can be enumerated in $\mathcal{O}(k_2^{k_1} \cdot |E(G)|)$.
% \end{lemma}


% \begin{corollary}
% {\sc Reconfiguration Independent Set} can be solved in $\mathcal{O}(k_1^{k_2} \cdot \tau \cdot (|V(G)|+|E(G)|\cdot\sqrt{|V(G)|}))$.
% \end{corollary}

%%% Local Variables:
%%% mode: LaTeX
%%% TeX-master: "main"
%%% End:
