\section{Fixed parameter tractability by lifetime and temporal neighbourhood diversity}
\label{sec:ndiversity}
\newcommand{\tndg}[1][\mathcal{G}]{\ensuremath{TND_{#1}}\xspace}
\newcommand{\ndg}[1][\mathcal{G}]{\ensuremath{ND_{#1}}\xspace}
In this section we present a fixed-parameter algorithm, parametrised by lifetime and the temporal neighbourhood diversity of the temporal graph, to solve a class of reconfiguration problems that we call \emph{temporal neighbourhood diversity locally decidable}. 

We need a number of algorithmic tools to build toward this overall result.  First, in \Cref{subsec:definitions}, we give definitions and notation necessary for this section, including defining temporal neighbourhood diversity, as well as the class of temporal neighbourhood diversity locally decidable problems. These build on analogous definitions in static graphs. 

%In \Cref{subsec:reconfiguration}, we introduce a key subroutine of the parameterized algorithm, which we then use in Subsection \Cref{subsec:algo}
Then, in \Cref{subsec:algo} we describe an overall algorithm to solve our restricted class of problems that is in FPT with respect to temporal neighbourhood diversity and lifetime. This algorithm uses a critical subroutine that constitutes the majority of the technical detail, and is presented in \Cref{subsec:reconfiguration}.  The subroutine uses a reduction to the efficiently-solvable circulation problem to give the key result (in \Cref{lemma:compute reconfigurable sequence}) The result allows us to efficiently generate an optimal reconfigurable sequence that is compatible with that candidate sequence given a candidate reconfiguration sequence of a type specific to temporal neighbourhood diversity locally decidable problems. 


% Then, we present the intuition of the algorithm in \Cref{subsec:algo}. Finally, we describe and show the correctness of the main subroutine of the algorithm in \Cref{subsec:reconfiguration}.


\subsection{Definitions}
\label{subsec:definitions}

% \subsubsection{In static graphs}

Neighbourhood diversity is a static graph parameter
% We first state the formal definitions of neighbourhood diversity and neighbourhood diversity locally decidable problem. This parameter has been
introduced by Lampis~\cite{Lampis12}:


\begin{definition}[Neighbourhood Diversity~\cite{Lampis12}]
  The \emph{neighbourhood diversity} of a static graph $G$ is the minimum value $k$ such that the vertices of $G$ can be partitioned into $k$ classes $V_1,\dots,V_k$ such that 
%
  for each pair of vertices $u$ and $v$ in a same class $V_i$, we have $N(u) \setminus \{v\} = N(v) \setminus \{u\}$.  We call $V_1,\dots,V_k$ a \emph{neigbourhood diversity partition} of $\mathcal{G}$.
\end{definition}

Notice that each set $V_i$ of $P$ forms either an independent set or a clique. Moreover, for any pair of sets $V_i$ and $V_j$ either no vertex of $V_i$ is adjacent to any vertex of $V_j$ or every vertex of $V_i$ is adjacent to every vertex of $V_j$. We distinguish two types of classes: $V_i$ is a \emph{clique-class} if $G[V_i]$ is a clique and $V_i$ is an \emph{independent-class} otherwise.
%
% Observe that the definition of temporal neigbourhood diversity is a generalisation of the neighbourhood diversity in static graphs since the temporal neighbourhood diversity partition of a temporal graph with lifetime one and the neighbourhood diversity partition of its footprint are the same. 

%


 \begin{definition}[Neighbourhood diversity graph]
 Let $G$ be a static graph with neighbourhood diversity partition $V_1,\dots,V_k$. The \emph{neighbourhood diversity graph} of $G$, denoted $\ndg[G]$ is the graph obtained by merging each class $V_i$ into a single vertex. Formally, we have $V(\ndg[G]) = \{V_1,\dots,V_k\}$ and $E(\ndg[G]) = \{V_iV_j \mid \forall v_i \in V_i, \forall v_j\in V_j, v_iv_j \in E(G)\}$.
\end{definition}


%
For clarity, we use the term class when referring to a vertex of \ndg[G], in order to distinguish between the vertices of $G$ and the vertices of \ndg[G].
%
% The definition of temporal neighbourhood diversity in temporal graphs is a generalisation of the neighbourhood diversity which is a parameter defined in static graphs. The neighbourhood diversity of a static graph $G$ is equivalent to the temporal neighbourhood diversity of the temporal graph of lifetime one and containing $G$ as unique snapshot. Hence, in the following, given a static graph $G$, we also use the notation $\tndg[G]$ to refer to the temporal neighbourhood diversity graph of the temporal graph containing $G$ as unique snapshot.

%
Let $X$ be a set of vertices and $Y$ be a subset of classes. We say that $X$ and $Y$ are \emph{compatible} if for all classes $V_i$, $V_i$ belongs to $Y$ if and only if $V_i$ intersects $X$, that is, $\forall V_i, (V_i \cap X \neq \emptyset \Leftrightarrow V_i \in Y)$.
% we have $X \cap V_i \neq \emptyset \Leftrightarrow V_i \in Y$
Notice that there is exactly one subset of classes that is compatible with a set of vertices $X$ whereas several subsets of vertices of $G$ can be compatible with a set of classes $Y$.
%

We now introduce the concept of a neighbourhood diversity locally decidable problem -- we do this first in the static setting in order to build into the temporal setting. 
Intuitively, these are problems for which, given a set of classes $Y$, we can determine the minimum and maximum number of vertices to select in each class that are realised by at least one solution compatible with $Y$, if such a solution exists. The formal definition is as follows:

\begin{definition}[Neighbourhood diversity locally decidable]
  \label{def:ndld}
  A static graph problem $\Pi$ is $f(n)$-\emph{neighbourhood diversity locally decidable} if
  for any static graph $G$ with $n$ vertices and every subset of classes $Y$ of $\ndg[G]$, the following two conditions hold:
  \begin{enumerate}[(a)]
    \item there is a computable function $check_\Pi(Y)$ with time complexity $\mathcal{O}(f(n))$ that determines if there is a solution for $\Pi$ in $G$ that is compatible with $Y$,
    \item if there is such a solution, then
      there exist two computable functions \\$low_\pi : \mathcal{P}(V(\ndg[G])) \times V(\ndg[G]) \to \mathbb{N}$ and $up_\pi : \mathcal{P}(V(\ndg[G])) \times  V(\ndg[G]) \to \mathbb{N}$ with time complexity $\mathcal{O}(f(n))$ such that, for all subsets $X \subseteq V(G)$, we have
\begin{gather*}
  \forall V_i \in V(\ndg[G]), low_\pi(Y,V_i) \leq |V_i \cap X| \leq up_\pi(Y,V_i)\\
  \Leftrightarrow \\
  \text{$X$ is solution for $\Pi$ and is compatible with $Y$}. 
\end{gather*}
  \end{enumerate}
\end{definition}

  In other words, $low_\pi$ and $up_\pi$ are necessary and sufficient lower and upper bounds for the number of selected vertices in each class in a solution to our problem in the graph of low neighbourhood diversity.

Notice that it is easy to compute a solution of minimum (respectively maximum) size that is compatible with a subset of classes $Y$ (if such a solution exists), by arbitrarily selecting exactly $low_\pi(V_i)$ (resp. $up_\pi(V_i)$) vertices inside each class.
%
Hence, $\Pi$ is solvable in $\mathcal{O}(2^k \cdot f(n))$, where $k$ is the neighbourhood diversity of the graph. Indeed, it suffices to enumerate every subset of classes and keep the best solution. It follows that if $check_\pi, low_\pi$ and $up_\pi$ are polynomial-time functions, $\Pi$ is in FPT when parameterised by neighbourhood diversity.

%\subsubsection{Example problems that are neighbourhood diversity locally decidable}

We show that {\sc Dominating Set} and {\sc Independent Set} are $f(n)$-neighbourhood diversity locally decidable.

\begin{lemmarep}
  \label{lemma:ds is ndld} {\sc Dominating set} is $f(n)$-neighbourhood diversity locally decidable where $f(n) \in O(n)$.
\end{lemmarep}

\begin{proof}
  Let $G$ be a static graph and let $Y \subseteq V(\ndg[G])$ be a subset of classes. We show that the following two conditions hold.
  \begin{enumerate}[(a)]
    \item there is a dominating set compatible with $Y$ if and only if $Y$ is a dominating set in $\ndg[G]$, and
    \item if $Y$ is a dominating set in $\ndg[G]$, then for each $V_i \in \ndg[G]$ the lower and upper bounds functions $low : V(\ndg[G])  \to \mathbb{N}$ and $up : V(\ndg[G]) \to \mathbb{N}$ are given by
    \begin{enumerate}[(i)]
      \item $low(V_i) = up(V_i) =0$, if $V_i\not\in Y$,
      \item $low(V_i) = 1$ and $up(V_i) = |V_i|$, if $V_i\in Y$ and $V_i$ is a clique-class or has a neighbour in $Y$ and,
      \item $low(V_i) = up(V_i) =|V_i|$, otherwise.
    \end{enumerate}
  \end{enumerate}
  
  First, we show that there is a dominating set in $G$ compatible with $Y$ if and only if $Y$ is a dominating set in $\ndg[G]$.
  Let $Y$ be a dominating set in $\ndg[G]$, we show that the subset of vertices $X = \bigcup\limits_{V_i \in Y} V_i$ is a dominating set of $G$. For any vertex $v \not \in X$ of $G$, the class $V_i \in N(\ndg[G])$ to which $v$ belongs has a neighbour $V_j$ in $\ndg[G]$ that belongs to $Y$ since otherwise $Y$ would not be a dominating set of $\ndg[G]$. Then, any vertex of $V_j$ belongs to $X$ and is a neighbour of $v$ in $G$ and so, $v$ is dominated. Hence, $X$ is a dominating set of $G$.
  %
  
  Now, let $X$ be a dominating set of $G$, we show that the subset of classes $Y$ compatible with $X$ is a dominating set of $\ndg[G]$. For any class $V_i \not \in Y$ and any vertex $v \in V_i$, $v$ is dominated by a vertex $u$ that belongs to a class $V_j \neq V_i$. Then, $V_j \in Y$ and $V_j$ is adjacent to $V_i$ in $\ndg[G]$ and so, $V_i$ is dominated. Hence, any class that does not belong to $Y$ has a neighbour in $Y$ and so, $Y$ is a dominating set in $\ndg[G]$.

  
Further, suppose that $Y$ is a dominating set in $\ndg[G]$, % we show that the depicted bounds are necessary and sufficient.
and let $X$ be a dominating set of $G$ that is compatible with $Y$, we show that for all classes $V_i$ of $G$, we have $low_\pi(V_i) \leq |X \cap V_i| \leq up_\pi(V_i)$.
  Let $V_i$ be a class of $G$.
  %
  If $V_i\not\in Y$, then we have $|X \cap V_i| = 0$ since otherwise $X$ would not be compatible with $Y$. 
  %
  If $V_i\in Y$, since $X$ is compatible with $Y$, $X$ should contain at least one vertex in $V_i$. Moreover, since $X$ cannot contain more that $|V_i|$ vertices in $V_i$, we have $1 \leq |V_i \cap X| \leq |V_i|$.
  %
  If $V_i$ is not a clique-class or has no neighbour in $Y$, then any vertex $v \in V_i$ is isolated in the subgraph induced by $X \cap \bigcup_{V_j \in Y} V_j$. Hence, $X$ contains necessarily every vertex of $V_i$ and so, $|X \cap V_i| = |V_i|$.

  
  Finally, we show that if $Y$ is a dominating set in $\ndg[G]$, then any set of vertices $X$ such that for all classes $V_i$, we have $low_\pi(V_i) \leq |X \cap V_i| \leq up_\pi(V_i)$,  is a dominating set in $G$ compatible with $Y$. First, since for any class $V_i$ we have $|V_i \cap X| = 0$ if $V_i \not\in Y$ and $1 \leq |V_i \cap X|$ otherwise, $X$ is compatible with $Y$. It remains to show that $X$ is also a dominating set.
  Let $v \not\in X$ be a vertex of $G$ that belongs to a class $V_i$. If $V_i \not\in Y$, then since $Y$ is a dominating set of $\ndg[G]$,  there is a class $V_j \in Y$ that is adjacent to $V_i$ in $\ndg[G]$. Thus, $V_j$ contains a vertex $u \in X$ and $u$ is adjacent to $v$ in $G$ and dominates $v$.
  If $V_i \in Y$ and is a clique-class, then there is a vertex $u \in V_i \cap X$ and $u$ is adjacent to $v$ in $G$. If $V_i$ is an independent-class, then since $v$ does not belong to $X$, $v$, there is necessarily another class $V_j \in Y$ that is adjacent to $V_i$ in $\ndg[G]$, since otherwise, every vertex of $V_i$ would belong to $X$. Thus, $V_j$ contains a vertex $u \in X$ and $u$ is adjacent to $v$ in $G$ and dominates $v$. Hence, in any case, $v$ is dominated by another vertex and thus, $X$ is a dominating set of $G$ that is compatible with $Y$.
\end{proof}

\begin{lemmarep}
  \label{lemma:is is ndld} {\sc Independent Set} is $f(n)$-neighbourhood diversity locally decidable where $f(n) \in O(n)$. 
\end{lemmarep}

\begin{proof}
    Let $G$ be a static graph and let $Y \subseteq V(\ndg[G])$ be a subset of classes. We show the following two conditions.
  \begin{enumerate}[(a)]
    \item there is an independent set compatible with $Y$ if and only if $Y$ is an independent set in $\ndg[G]$, and
    \item if $Y$ is a independent set in $\ndg[G]$, then the lower and upper bounds functions $low : V(\ndg[G])  \to \mathbb{N}$ and $up : V(\ndg[G]) \to \mathbb{N}$ are given by
    \begin{enumerate}[(i)]
      \item $low(V_i) = up(V_i) =0$, if $V_i\not\in Y$,
      \item $low(V_i) = up(V_i) = 1$, if $V_i\in Y$ and $V_i$ is a clique-class and,
      \item $low(V_i)=1$ and $up(V_i) =|V_i|$, otherwise.
    \end{enumerate}

  \end{enumerate}
  
  First, we show that there is an independent set in $G$ compatible with a subset of classes $Y$ if and only if $Y$ is an independent set in $\ndg[G]$.
  Let $Y$ be an independent set in $\ndg[G]$, let $X \subseteq V(G)$ be any subset of vertices constructed by arbitrarily choosing one vertex in each $V_i \in Y$.
  We show that $X$ is an independent set of $G$. Let $u$ be a vertex of $X$ that belongs to some class $V_i$. Per construction of $X$, we have $V_i \in Y$. Let $v$ be a neighbour of $u$. If $v \in V_i$, then $v$ does not belong to $X$ since $X$ contains at most one vertex per class in $Y$. If $v$ belongs to another class $V_j$, then $V_i$ and $V_j$ are adjacent in \ndg[G] and since $Y$ is an independent set, $V_j \not \in Y$ and so, by construction of $X$, $v \not\in X$. So, $X$ is an independent set.
  %
  Now, let $X$ be an independent set of $G$, we show that the subset of classes $Y$ compatible with $X$ is an independent set of $\ndg[G]$. Suppose there exist two adjacent classes $V_i$ and $V_j$ that belong to $Y$. Then, $V_i$ contains a vertex $u \in X$ and $V_j$ contains a vertex $v \in V_j$. Since $u$ and $v$ are adjacent, it contradicts $X$ being an independent set. Hence, $Y$ does not contain two adjacent classes and so, $Y$ is an independent set in \ndg[G].
  %
  
Further, suppose that $Y$ is an independent set in $\ndg[G]$, % we show that the depicted bounds are necessary and sufficient.
and let $X$ be an independent set of $G$ that is compatible with $Y$, we show that for all classes $V_i$ of $G$, we have $low_\pi(V_i) \leq |X \cap V_i| \leq up_\pi(V_i)$.
  Let $V_i$ be a class of $G$.
  %
  If $V_i\not\in Y$, then we have $|X \cap V_i| = 0$ since otherwise $X$ would not be compatible with $Y$. 
  %
  If $V_i\in Y$, since $X$ is compatible with $Y$, $X$ should contain at least one vertex in $V_i$. Moreover, since $X$ cannot contain more that $|V_i|$ vertices in $V_i$, we have $1 \leq |V_i \cap X| \leq |V_i|$.
  %
  If $V_i$ is a clique-class, then $X$ cannot contain more two vertices in $V_i$ since they would be adjacent.
  Hence, $|V_i \cap X|=1$ .
  %
  
  Finally, we show that if $Y$ is an independent set in $\ndg[G]$, then any set of vertices $X$ such that for all classes $V_i$, we have $low_\pi(V_i) \leq |X \cap V_i| \leq up_\pi(V_i)$, is an independent set in $G$ compatible with $Y$. First, since for any class $V_i$ we have $|V_i \cap X| = 0$ if $V_i \not\in Y$ and $1 \leq |V_i \cap X|$ otherwise, $X$ is compatible with $Y$. It remains to show that $X$ is also an independent set.
  Suppose there exist two adjacent vertices $u$ and $v$ that both belong to $X$. The two vertices cannot belong to different classes $V_i$ and $V_j$ since otherwise both $V_i$ and $V_j$ would belong to $Y$, contradicting $Y$ being an independent set in \ndg[G]. Thus, $u$ and $v$ belong to the same class $V_i$. Since, $u$ and $v$ are adjacent, $V_i$ is not an independent class. But then, we have $|X \cap V_i| > up(V_i)$ which is a contradiction. Hence, $X$ does not contain two adjacent vertices and therefore $X$ is an independent set. 
\end{proof}


\subsubsection{In temporal graphs}
We now extend the concept of neighbourhood diversity locally decidable problems to a temporal setting. We use temporal neighbourhood diversity introduced by Enright et al.~\cite{abs-2404-19453}.


\begin{definition}[Temporal Neighbourhood Diversity~\cite{abs-2404-19453}]
The \emph{temporal neighbourhood diversity} of a temporal graph $\mathcal{G}=(G,\lambda)$ is the minimum value $k$ such that the vertices of $G$ can be partitioned into $k$ classes $V_1,\dots,V_k$ such that 
%
  for each pair of vertices $u$ and $v$ in a same class $V_i$, we have $N_t(u) \setminus \{v\} = N_t(v) \setminus \{u\}$ at each time-step $t \in [1,\tau]$.  We call $V_1,\dots,V_k$ a \emph{temporal neigbourhood diversity partition} of $\mathcal{G}$.
\end{definition}

\noindent In the same way as for the static parameter, we can use a graph to represent the partition.

\begin{definition}[Temporal neighbourhood diversity graph]
 Let $\mathcal{G}=(G,\lambda)$ be a temporal graph with temporal neighbourhood partition $V_1,\dots,V_k$. The \emph{temporal neighbourhood diversity graph} of $\mathcal{G}$, denoted $\tndg$ is the temporal graph obtained by merging every class $V_i$ into a single vertex. Formally, we have $V(\tndg) = \{V_1,\dots,V_k\}$, $E(\tndg) = \{V_iV_j \mid \forall v_i \in V_i, \forall v_j\in V_j, v_iv_j \in E(G)\}$ and $\lambda(V_iV_j) = \lambda(u_iu_j)$ for any vertices $u_i \in V_i$ and $u_j \in V_j$.
\end{definition}

\begin{figure}
  \centering
  \scalebox{0.4}{
    \begin{tikzpicture}
      \newcounter{vertex}
      \setcounter{vertex}{1}

      \newcounter{tmp}
      \setcounter{tmp}{1}
      
    \foreach[count=\l from 1] \x/\T in {3/{C,I,C},4/{I,I,C},6/{I,I,I},6/{C,I,I},5/{C,C,I},2/{I,I,C}}{
      
      \setcounter{tmp}{\thevertex}

      \foreach[count=\X from 1] \i in \T {
        \setcounter{vertex}{\thetmp}
        \node[draw,circle,minimum width=2.1cm,label=\l*60+30:{\LARGE $V_\l$}] (\l\X) at ($(\X*10,0)+(\l*60:3)$) {};
        \pgfmathsetmacro{\a}{360/\x}% 
        \foreach \v in {1,...,\x}{
          \node[smallvertex,label=\v*\a:{\letter{\thevertex}}] (\v) at ($(\X*10,0)+(\l*60:3)+(\v*\a:0.5)$) {};
          \stepcounter{vertex}
        }
        \ifthenelse{\equal{\i}{C}}{
          \foreach \a in {1,...,\x}{
            \foreach \b in {1,...,\a} {
              \draw (\a)--(\b);
            }
          }
        }{}
      }
    }
    \foreach \a/\b in {21/31,41/51,41/11,41/61,51/61,11/61,11/51,22/12,62/12,22/62,62/52,22/42,42/52,23/33,53/63,53/13,53/43}
    \draw (\a)--(\b);
    \foreach \L in {1,...,3}
    \node at (10*\L,-5) {\LARGE $G_\L$};
  \end{tikzpicture}
  }
  \caption{\label{fig:tmp neighbourhood diversity} Example of temporal neighbourhood diversity graph \tndg of a temporal graph $\mathcal{G}=(G,\lambda)$. An edge between two sets $V_i$ and $V_j$ indicates that each vertex of $V_i$ is adjacent to all vertices of $V_j$.} 
\end{figure}
 
An example of a temporal neighbourhood diversity graph is depicted in \Cref{fig:tmp neighbourhood diversity}.
%
Let $\Pi$ be a reconfiguration problem. We call $\Pi$ \emph{$f(n)$-temporal neighbourhood diversity locally decidable} if the static version of $\Pi$ is $f(n)$-neighbourhood diversity locally decidable. In both of the problems we consider $f(n)=n$, and we simply write ``temporal neighbourhood diversity locally decidable'' for convenience. For a temporal graph $\mathcal{G}=(G,\lambda)$, we denote by $low^t_\pi$ and $up^t_\pi$ the functions giving the lower and upper bounds on the number of selected vertices in a class $V_i$ needed to obtain solution for the static version of $\Pi$ in $G_t$, as defined in \Cref{def:ndld}.


\subsection{FPT algorithm with respect to temporal neighbourhood diversity and lifetime}
\label{subsec:algo}

We now formulate an FPT algorithm using temporal neighbourhood diversity and lifetime to solve a neighbourhood diversity locally decidable reconfiguration problem $\Pi$ in a temporal graph $\mathcal{G}=(G,\lambda)$. 
A \emph{candidate sequence} is a sequence $Y = (Y_1,\dots,Y_\tau)$ such that for each $t \in [1,\tau]$, $Y_t$ is a subset of classes of $\tndg$. 
We say that $Y$ is \emph{valid} if $check_\Pi(Y_t)= \texttt{true}$ for all $t \in [\tau]$. Let $T=(T_1,\dots,T_\tau)$ be a reconfigurable sequence for $\mathcal{G}$. For the sake of simplicity, we say that the two sequences $Y$ and $T$ are compatible if for each $t \in [\tau], T_t$ and $Y_t$ are compatible. 

The principle of the algorithm is to iterate through each candidate sequence $Y$ and, if $Y$ is valid, compute an optimal reconfigurable sequence $T$ compatible with $Y$. Then, we return the optimal solution among all solutions associated to valid candidate sequences.
Notice that there are at most $2^{tnd \cdot\tau}$ candidate sequences to consider. Hence, we can expect the algorithm to be efficient if the temporal graph has small temporal neighbourhood diversity and short lifetime.
% This algorithm is depicted in \Cref{alg:fpt nd}.
If $\Pi$ is $f(n)$-neighbourhood diversity locally decidable, we show in \Cref{subsec:reconfiguration} how to compute an optimal reconfigurable sequence compatible with a candidate sequence, if one exists, by solving an instance of the circulation problem. Combining the observation that there are at most $2^{tnd \cdot\tau}$ candidate sequences to consider and the result from Subsection~\ref{subsec:reconfiguration} gives us the following result, which we restate at the end of that subsection.

% \begin{algorithm}
%   \SetAlgoLined
%   \KwData{A temporal graph $\mathcal{G}$ and a reconfiguration problem $\Pi$}
%   \KwResult{The score of the optimal solution for $\Pi$ in $\mathcal{G}$}
%   \ForEach{candidate sequence $Y = (Y_1,\dots,T_\tau)$}{
%     \If{$check(Y)$}{
%       \tcc{compute an optimal sequence compatible with $Y$}
%       $T \gets compute\_sequence(\mathcal{G},Y)$\;
%       \lIf{$\Pi$ is a minimisation problem}{$score \gets \min(score,|T|)$}
%       \lElse{$score \gets \max(score,|T|)$}
%     }
%   }
%     \Return score\;
%   \caption{}
%   \label{alg:fpt nd}
% \end{algorithm}

\begin{lemma*}
    Let $\mathcal{G}=(G,\lambda)$ be a temporal graph and let $\Pi$ be an $f(n)$-temporal neighbourhood diversity locally decidable reconfiguration problem. 
 $\Pi$ is solvable in $\mathcal{O}(2^{(tnd\cdot \tau)} \cdot (\tau\cdot tnd \cdot f(n) + \tau^{\nicefrac{11}{2}} \cdot tnd^8))$ where $tnd$ is the temporal neighbourhood diversity of $\mathcal{G}=(G,\lambda)$.
\end{lemma*}








%%% Local Variables:
%%% mode: LaTeX
%%% TeX-master: "main"
%%% End:


\subsection{Computing an optimal reconfigurable sequence compatible with a candidate sequence}
\label{subsec:reconfiguration}

In this subsection, we present an efficient method for computing an optimal reconfigurable sequence $T$ compatible with a particular valid sequence $Y$. 
 Note that,  we cannot compute $T$ by exhaustively generating all possible sequences in $\mathcal{G}$ and selecting the best one compatible with $Y$. This approach would have time complexity of $\mathcal{O}(2^{\tau\cdot|V(G)|})$ for each candidate sequence making it impractical.
%
 % Given a valid candidate sequence $Y$, we describe how to compute an optimal reconfigurable sequence $T = (T_1,\dots,T_\tau)$ for $\Pi$ compatible with $Y$ in $\mathcal{O}(\tau\cdot tnd \cdot f(n) + \tau^{\nicefrac{11}{2}} \cdot tnd^8)$ time. % It corresponds to the function compute\_sequence of \Cref{alg:fpt nd}.

%

As a tool to build our algorithm we construct and solve an instance of the circulation problem: this is a variation of the network flow problem for which there are upper and lower bounds on the capacities of the arcs. Formally, a \emph{circulation graph} $\overrightarrow{F}=(F,c,l,u)$ consists of a directed graph $F$, along with a cost function $c : A(F) \to \mathbb{N}$ on the arcs and two functions on the arcs $l : A(F) \to \mathbb{N}$ and $u : A(F) \to \mathbb{N}$, representing the lower and upper bounds of the flow capacities. The digraph $F$ also has two designated vertices $sou$ and $tar$ representing the source and the target vertices of the flow, respectively.

In the circulation problem, we want to find a \emph{circulation flow} between $sou$ and $tar$, which is a function $g : A(F) \to \mathbb{N}$ that respects two constraints:
\begin{enumerate}[(a)]
  \item \emph{Flow preservation}: the flow entering in any vertex $v \in V(F) \setminus \{sou,tar\}$ is equal to the flow leaving it, that is, $\sum\limits_{u \in N^+(v)} g(u,v) = \sum\limits_{u \in N^-(v)} g(v,u)$.
  \item \emph{Capacity:} for all arcs $(u,v) \in A(F)$, the amount of flow over $(u,v)$ must respect the capacity of $(u,v)$, that is, $l(u,v) \leq g(u,v) \leq u(u,v)$.
\end{enumerate}
%
Finally, the circulation problem is defined as follows.
\prob{Circulation problem}{
A circulation graph $\overrightarrow{F}$, a source vertex $sou$ and a target vertex $tar$
}{
  Find a circulation flow $g$ that minimises  $\sum\limits_{a \in A(F)}g(a)\cdot c(a)$.
}
The circulation problem is known to be solvable in $\mathcal{O}(|A(F)|^{\nicefrac{5}{2}}\cdot |V(F)|^3)$~\cite{Tardos85}. Notice that the circulation problem can easily be transformed into a maximisation problem by modifying all positive costs into negative costs.

We now show that the problem of finding an optimal reconfigurable sequence compatible with $Y$ is reducible to the circulation problem.

% \begin{construction}
%   \TD{constructive version\\}
%   \label{const:flow}
%   Let $\mathcal{G}=(G,\lambda)$ be a temporal graph, let $Y=(Y_1,\dots,Y_\tau)$ be a valid candidate sequence.
%   We construct the following circulation graph $\overrightarrow{F}=(F,c,l,u)$.
%   \begin{itemize}
%     \item For each $Y_t$ and each set $V_i \in Y_t$, introduce two vertices $x^t_i, y^t_i$ along with the arc $(x^t_i,y^t_i)$ and set $c(x^t_i,y^t_i)=0$, $l(x^t_i,y^t_i)=low_\pi(Y_t,V_i)$ and $u(x^t_i,y^t_i)=up_\pi(Y_t,V_i)$.
%     \item Introduce a source vertex $sou$ and for each set $V_i \in Y_1$, introduce the arc $(sou,x^1_i)$ and set $c(sou,x^1_i)=1, l(sou,x^1_i)=0$ and $u(sou,x^1_i)=+\infty$.
%     \item Introduce a target vertex $tar$ and for each set $V_i \in Y_\tau$, introduce the arc $(x^\tau_i,tar)$ and set $c(x^\tau_i,tar)=l(x^\tau_i,tar)=0$ and $u(x^\tau_i,tar)=+\infty$.
%     \item For each $t\in[1,\tau-1]$, and each pair $V_i \in Y_t, V_j \in Y_{t+1}$ such that either $V_i=V_j$ or $V_i$ and $V_j$ are adjacent in $\tndg$ at time step $t$, introduce an arc $(y^t_i,x^{t+1}_j)$ and set $c(y^t_i,x^{t+1}_j)=0, l(y^t_i,x^{t+1}_j)=0$ and $u(y^t_i,x^{t+1}_j)=+\infty$.  
%   \end{itemize}
% \end{construction}


\begin{construction}
  \label{const:flow}
  Let $\mathcal{G}=(G,\lambda)$ be a temporal graph, let $Y=(Y_1,\dots,Y_\tau)$ be a valid candidate sequence.  
  The reconfiguring circulation graph $\overrightarrow{RF}(\mathcal{G},Y)=(F,c,l,u)$ is the circulation graph with vertex set $V(F) = \{sou,tar\} \cup \{x_i^t \mid V_i \in V(\tndg), t \in [1,\tau]\}$ and arc set $E(F) = S \cup \{C_t \mid t \in [1,\tau]\} \cup \{R_t \mid t \in [1,\tau-1]\}  \cup T$ where the arc sets $S,C_t ,R_t$ and $T$ are defined as follows: 
  \begin{itemize}
    \item $S= \{ (sou,x_i^1) \mid V_i \in Y_1\}$ with $c(a)=1$ and $l(a)=0$ and $u(a)=+\infty$ for each a $\in S$,
    \item $C_t= \{(x^t_i,y^t_i) \mid V_i \in Y_t\}$ with $c(x^t_i,y^t_i) = 0$, $l(x^t_i,y^t_i) = low^t_\pi(V_i)$ and  $u(x^t_i,y^t_i) = up^t_\pi(V_i)$ for each $(x^t_i,y^t_i)\in C_t$,
    \item $R_t= \{(y^t_i,v^{t+1}_i) \mid V_i \in Y_t \cap Y_{t+1}\} \cup \{(y^t_i,v^{t+1}_j) \mid V_i \in Y_t,V_j \in Y_{t+1}, V_i$ is adjacent to $V_j$ in $\tndg$ at time $t\}$ with $c(a)=l(a)= 0$ and $u(a)=+\infty$ for each $a\in R_t$,
    \item $T = \{(y_i^\tau,tar) \mid V_i \in V(\tndg)\}$ with $c(a)=l(a)=0$ and $u(a)=+\infty$ for each $a\in T$.
  \end{itemize}
  \end{construction}


An example construction can be found in Figure~\ref{fig:flow}. The idea of the reduction is to represent each token by a unit value of the flow.
  At each time-step $t$, we can decompose the reconfiguring circulation graph into two parts: the checking part $C_t$ and the reconfiguring part $R_t$ (which only exists if $t \neq \tau$). In the checking part, each class $V_i$ is represented by an arc $(x^t_i,y^t_i) \in C_t$ and the value of the flow passing by this arc corresponds to the number of tokens inside $V_i$ at time-step $t$. Since the lower and upper bounds of the arc are the same as those described in \Cref{def:ndld}, we ensure that at each step, the set of selected vertices forms a solution. In the 
reconfiguring part, an arc $(y^t_i,x^t_j) \in R_t$ indicates that the tokens contained in the class $V_i$ can move to the class $V_j$ during time-step $t$. An arc $(y^t_i,x^t_i) \in R_t$ indicates that the tokens in the class $V_i$ can stay in the same vertex during time-step $t$.

% \begin{definition}
%   \label{def:flow}
%   Let $\mathcal{G}=(G,\lambda)$ be a temporal graph and let $\mathcal{Y}=(Y_1,\dots,Y_\tau)$ be a candidate sequence in $\mathcal{G}$ for some reconfiguration problem $\Pi$. The \emph{reconfiguration flow graph} of $\mathcal{G}$ and $\mathcal{Y}$, denoted $\overrightarrow{F}_\Pi(\mathcal{G},\mathcal{Y})=(F,c,l,u)$ is a circulation graph with vertex set $V(F)= \{sou,tar\} \cup \{x^t_i,y^t_i \mid i \in [1,\tau], V^t_i \in Y_t\} \cup \{w^t_i,z^t_i \mid i \in [1,\tau], \}$ 
%   We construct the following circulation graph $\overrightarrow{F}=(F,c,l,u)$. In the following, for each arc $e$, if $c(e),l(e)$ or $u(e)$ are not specified, we set $c(e)=0$, $l(e)=0$ and $u(e) +\infty$.
%   %
%   \begin{itemize}
%     \item For each time-step $t \in [1,\tau]$ each class $V^t_i \in Y^t$, introduce two vertices $x^t_i$ and $y^t_i$ along with the arc $(x^t_i,y^t_i)$ and set $l(x^t_i,y^t_i) = low_{\Pi}(V^t_i)$, $u(x^t_i,y^t_i) = up_{\Pi}(V^t_i)$. 
%     \item For each time-step $t \in [1,\tau-1]$ and each class $V^t_i$ of $G_t$ such that there exists $V^{t+1}_j \in Y^{t+1}$ with $V^t_i \cap V^{t+1}_j \neq \emptyset$, introduce two vertices $w^t_i$ and $z^t_i$ along with the arc $(w^t_i,z^t_i)$ and set $up(w^t_i,z^t_i)=|V^t_i|$. Then introduce the following arcs.
%     \begin{itemize}
%       \item For each $V^t_j \in Y^t$ such that $V^t_j=V^t_i$ or $V^t_j$ is adjacent to $V^t_i$ in $NDG(G_t)$, introduce the arc $(y^t_i,w^t_i)$.
%       \item For each $V^{t+1}_j \in Y^{t+1}$ such that $V^t_i \cap V^{t+1}_j \neq \emptyset$, introduce the arc $(z^t_i,x^{t+1}_j)$ and set $u(z^t_i,x^{t+1}_j) = |V^t_i \cap V^{t+1}_j|$.
%     \end{itemize}
%     \item   Finally, introduce one source vertex $sou$ and one target vertex $tar$. For each class $V^1_i \in Y^t$, introduce the arc $(sou,x^1_i)$ and set $c(sou,x^1_i) = 1$. For each class $V^\tau_i \in Y^\tau$, introduce the arc $(z^\tau_i,tar)$. 
%   \end{itemize}
% \end{definition}



% \begin{construction}
%   \label{const:flow}
%   Let $\mathcal{G}=(G,\lambda)$ be a temporal graph, let $\mathcal{H}=(H,\lambda)$ be a temporal neigbourhood diversity graph of $\mathcal{G}$ and let $S=(S_1,\dots,S_\tau)$ be a sliding dominating set of $\mathcal{H}$.
%   We construct the following circulation graph $\overrightarrow{F}=(F,c,l,u)$.
%   \begin{itemize}
%     \item For each $S_t$ and each set $V_i \in S_t$, introduce two vertices $x^t_i, y^t_i$ along with the arc $(x^t_i,y^t_i)$ and set $c(x^t_i,y^t_i)=0$, $l(x^t_i,y^t_i)=min\_tok(V_i,H_t,S_t)$ and $u(x^t_i,y^t_i)=|V_i|$.
%     \item Introduce a source vertex $sou$ and for each set $V_i \in S_1$, introduce the arc $(s,x^1_i)$ and set $c(sou,x^1_i)=1, l(sou,x^1_i)=0$ and $u(sou,x^1_i)=+\infty$.
%     \item Introduce a target vertex $tar$ and for each set $V_i \in S_\tau$, introduce the arc $(x^\tau_i,tar)$ and set $c(x^\tau_i,tar)=l(x^\tau_i,tar)=0$ and $u(x^\tau_i,tar)=+\infty$.
%     \item For each $t<\tau$, and each pair $V_i \in S_t, V_j \in S_{t+1}$ such that either $V_i=V_j$ or $V_i$ and $V_j$ are adjacent in $H_t$, introduce an arc $(y^t_i,x^{t+1}_j)$ and set $c(y^t_i,x^{t+1}_j)=0, l(y^t_i,x^{t+1}_j)=0$ and $u(y^t_i,x^{t+1}_j)=+\infty$.  
%   \end{itemize}
% \end{construction}





% \begin{figure}
%   \centering
%   \scalebox{0.55}{
%   \begin{tikzpicture}
%     \foreach[count=\l from 1] \x/\T in {3/{C,I,C},4/{I,I,C},6/{I,I,I},6/{C,I,I},5/{C,C,I},3/{I,I,C}}{
%       \foreach[count=\X from 1] \i in \T {
%         \node[draw,circle,minimum width=2cm,label=\l*60+30:{\LARGE $V_\l$}] (\l\X) at ($(\X*10,0)+(\l*60:3)$) {};
%         \pgfmathsetmacro{\a}{360/\x}% 
%         \foreach \v in {1,...,\x}{
%           \node[smallvertex] (\v) at ($(\X*10,0)+(\l*60:3)+(\v*\a:0.5)$) {};
%         }
%         \ifthenelse{\equal{\i}{C}}{
%           \foreach \a in {1,...,\x}{
%             \foreach \b in {1,...,\a} {
%               \draw (\a)--(\b);
%             }
%           }
%         }{}
%       }
%     }
%     \foreach \a/\b in {21/31,41/51,41/11,41/61,51/61,11/61,11/51,22/12,62/12,22/62,62/52,22/42,42/52,23/33,53/63,53/13,53/43}
%     \draw (\a)--(\b);
%     \foreach \L in {1,...,3}
%     \node at (10*\L,-5) {\LARGE $G_\L$};
%   \end{tikzpicture}
%   }
%   \caption{\label{fig:neighbourhood diversity} Example of temporal neigbourhood diversity graph. An edge between two sets $V_i$ and $V_j$ indicates that each vertex of $V_i$ is adjacent to all vertices of $V_j$. } 
% \end{figure}
 

\begin{figure}[ht]
  \centering
  \scalebox{0.75}{
    \begin{tikzpicture}[
tap/.style args = {#1}{decoration={raise=5,
                                      text along path,
                                      text align={align=center},
                                      text={#1}
                                      },
              postaction={decorate},
              },
                   ]
   
     \foreach \x/\y/\l/\c/\u/\t in {0/0/1/1/3/1,0/-2/2/1/4/1,0/-4/3/1/6/1,
       4/1/6/1/3/2,4/-2/2/1/4/2,4/-4/3/6/6/2,
       8/-2/2/1/4/3,8/-4/3/1/6/3,8/1/6/1/3/3,8/-1/5/1/5/3}
     {
       \draw[->,>=stealth] (\x,\y) node[smallvertex,label=90:{$x_\l^\t$}] (x\l\t) {} ->
       node[midway,above] {0/\c/\u}
       ++(1.9,0) node[smallvertex,label=90:{$y_\l^\t$}] (y\l\t) at ++(0.1,0) {}
       ;
     }
     \foreach \a/\b in {21/22,22/23,31/32,32/33,21/32,31/22,11/62,22/63,22/53,62/53}
     \draw[->] (y\a)->(x\b);

     \node[smallvertex,label=90:{$sou$}] (s) at (-3,-2) {};
     \foreach \a in {11,21,31}
     \draw[->,tap={1/0/${+\infty}${}}] (s) -- node[midway,above] {} (x\a);

     \node[smallvertex,label=90:{$tar$}] (t) at (13,-2) {};
     \foreach \a in {63,53,23,33}
     \draw[->] (y\a)--(t);
   \end{tikzpicture}
   }
   \caption{\label{fig:flow} Example of flow graph produced by \Cref{const:flow}. The input temporal graph is the one depicted in \Cref{fig:tmp neighbourhood diversity}. We want to compute a minimum dominating set compatible with the candidate sequence $Y = (Y_1 = \{V_1,V_2,V_3\}, Y_2= \{V_2,V_3,V_6\},Y_3 = \{V_2,V_3,V_5,V_6\})$.}
 \end{figure}


 
\begin{lemmarep}
  \label{lemma:flow reconfiguration} Let $\mathcal{G}$ be a temporal graph with temporal neighbourhood graph \tndg and let $Y=(Y_1,\dots,Y_\tau)$ be a valid candidate sequence for some $f(n)$-neighbourhood diversity locally decidable reconfiguration problem $\Pi$. Let $\overrightarrow{RF}(\mathcal{G},Y)=(F,c,l,u)$ be the reconfiguring circulation graph as described in \Cref{const:flow}.
  There is a reconfigurable sequence $T$ compatible with $Y$ of size $\ell$ if and only if there is a circulation flow $g$ in $\overrightarrow{RF}(\mathcal{G},Y)$ with cost $\ell$.
  % $\mathcal{Y}$ is valid if there is a feasible flow $f$ in $\overrightarrow{F}$ and in that case, the minimum size of a reconfiguration sequence associated to $\mathcal{Y}$ is given by the minimum cost value for 

  % There is a sliding dominating set $(S,M,tok)$ in $\mathcal{G}$ with $\ell$ tokens if and only if there is a circulation flow $f$ with cost $\ell$ in $\overrightarrow{F}$.
\end{lemmarep}


\begin{proof}
  For any vertex $v$ of $\overrightarrow{RF}(\mathcal{G},Y)$, we denote $g^+(v)=\sum\limits_{u \in N^+(v)} g(u,v)$ and $g^-(v)= \sum\limits_{u \in N^-(v)} g(v,u)$, the flow entering in $v$ and leaving $v$, respectively.

  Let $T = (T_1,\dots,T_\tau)$ be a reconfigurable sequence of size $\ell$. For each time-step $t \in [1,\tau-1]$, let $b_t$ be a reconfiguration bijection between $T_t$ and $T_{t+1}$.  We construct a circulation flow $g$ between $sou$ and $tar$ as follows.
  %
  Let \tndg$=(\{V_1,\ldots,V_k\}, \{V_iV_j | \forall v_i\in V_i, \forall v_j\in V_j, v_iv_j\in E(G)\})$. First, we set $g(sou,x^1_i):= |T_1 \cap V_i|$.
  %
  Then, for each time-step $t \in [1,\tau]$, we set $g(x^t_i,y^t_i) := |V_i \cap T_t|$ (i.e.\ the number of tokens contained in the class $V_i$ at time-step $t$).
  %
  For each time-step $t \in [1,\tau-1]$, we set
      % 
    $g(y^t_i,x^{t+1}_j) := | \{ (u,v) \in b_t \mid u \in T_t \cap V_i, v \in T_{t+1} \cap V_j\} |$ (i.e.\ the number of tokens moving from the class $V_i$ to another class $V_j$ during time-step $t$ if $i \neq j$ or the number of tokens moving inside the class $V_i$ or staying in the same vertex in $V_i$ during time-step $t$ if $i = j$).
    %
    % $f(y^t_i,w^t_i) := | \{ uv \in M_t \mid u \in T_t \cap V_i, v \in T_{t+1} \cap V_i\} \cup T_t \cap T_{t+1} \cap V_i|$ (\textit{i.e.} the number of tokens moving inside the class $V_i$ or staying in the same vertex in $V_i$ during time-step $t$)
    %
    Finally, we set $g(y^\tau_i,tar):= |V_i \cap T_\tau|$. 
      % 
    We now show that $g$ is a circulation flow of cost $\ell$. First, notice that the only non-zero cost arcs are those leaving $sou$. Hence, the cost of $g$ is $\sum\limits_{V_i\in Y_1} g(sou,x^1_i) = \sum\limits_{V_i\in Y_1} |V_i \cap T_1| = \ell$.
    Further, we show that the capacity constraints are respected. For each arc $(x^t_i,y^t_i)$, since $\Pi$ is $f(n)$-neighbourhood diversity locally decidable, we have $low^t_\pi(Y_t,V_i) = l(x^t_i,y^t_i) \leq g(x^t_i,y^t_i) \leq up^t_\pi(Y_t,V_i) = u(x^t_i,y^t_i)$. For any other arc $a$, we have $l(a)=0$ and $u(a)=+\infty$ and so, the capacity constraint is necessarily respected. Hence, the capacity constraints are respected for all arcs.
    It remains to show that $g$ respects the flow conservation constraint. 
    %
    For any vertex $x^t_i$, we have $g^-(x^t_i) = g(x^t_i,y^t_i) = |T_t \cap V_i|$. If $t=1$, then $g^+(x^1_i)=g(sou,x^1_i)= |T_1 \cap V^1_i|= g^+(x^t_i)$. Otherwise, by construction of $g$, $g^+(x^t_i)$ is equal to the sum of the number of tokens moving from another class $V_j$ to the class $V_i$ plus the number tokens staying inside the class $V_i$ during the time-step $t$, that is, the number of tokens inside $V_i$ at time-step $t$. So, $g^-(x^t_i) = |T_t \cap V_i| = g^+(x^t_i)$. 
    % 
    For any vertex $y^t_i$, we have $g^+(y^t_i) = g(x^t_i,y^t_i) = |T_t \cap V^t_i|$. If $t = \tau$, then $g^-(y^t_i) = g(y^t_i,tar) = |T_\tau \cap V^\tau_i| = g^+(y^t_i)$. Otherwise, by construction of $g$, $g^-(y^t_i)$ is equal to the number of tokens moving from the class $V_i$ to another class $V_j$ plus the number of tokens staying inside the class $V_i$ during the time-step $t$, that is, the number of tokens inside $V_i$ at time-step $t$. So, $g^+(y^t_i) = |T_t \cap V_i| = g^-(y^t_i)$.
    %

    We now show the reverse.
    Let $g$ be a circulation flow with cost $\ell$ in $\overrightarrow{RF}(\mathcal{G},Y)$.
    We construct a reconfigurable sequence $T = (T_1,\dots,T_\tau)$ compatible with $Y$ of size $\ell$ as follows. First, for each class $V_i \in Y_1$, we construct $T_1$, by selecting $g(x^1_i,y^1_i)$ vertices in the class $V_i$. Since $g$ respects the flow conservation constraint, we have $|T_1 \cap V_i| = g(x^1_i,y^1_i)$ for each class $V_i$. Moreover, since $low^1_\pi(Y_1,V_i) = l(x^1_i,y^1_i) \leq g(x^1_i,y^1_i) < u(x^1_i,y^1_i) = up^1_\pi(Y_1,V_i)$ and since $\Pi$ is $f(n)$-neighbourhood diversity locally decidable, $T_1$ is a solution for the static version of $\Pi$ in $G_1$.
    %
    Now suppose that at time-step $t<\tau-1$, we also have $|T_t \cap V_i| = g(x^t_i,y^t_i)$ for each class $V_i$. We construct $T_{t+1}$ by moving $g(y^t_i,x^{t+1}_j)$ tokens from the class $V_i$ to the class $V_j$ for each pair $i,j$ such that $i\neq j$ and $(y^t_i,x^{t+1}_j)$ is an arc in $\overrightarrow{RF}(\mathcal{G},Y)$. Observe that inside $V_i$, exactly $g(y^t_i,x^{t+1}_i)$ tokens remain on the same vertex. The number of tokens leaving $V_i$ or staying inside $V_i$ is equal to $g^-(y^t_i)$ and since $g$ respects the flow conservation constraint, we have $g^+(y^t_i)= |T_t \cap V_i| = g^-(y^t_i)$. Thus, we have exactly the correct number of tokens in $V_i$ to make the moves. Moreover, we have  $f^+(x^{t+1}_j)= f(x^{t+1}_j,y^{t+1}_j)$ and so, we have $|T_{t+1} \cap V_j| = f(x^{t+1}_j,y^{t+1}_j)$. Since $low^{t+1}_\pi(Y_{t+1},V_j) = l(x^{t+1}_j,y^{t+1}_j) \leq f(x^{t+1}_j,y^{t+1}_j) < u(x^{t+1}_j,y^{t+1}_j) = up^{t+1}_\pi(Y_{t+1},V_i)$ and since $\Pi$ is $f(n)$-neighbourhood diversity locally decidable, $T_{t+1}$ is a solution for the static version of $\Pi$ in $G_{t+1}$. Hence, inductively, we have constructed a reconfigurable sequence $T = (T_1,\dots,T_\tau)$ compatible with $Y$.
\end{proof}


We can now conclude that it is possible to compute an optimal reconfigurable sequence compatible with a candidate sequence in polynomial time.
 
\begin{lemmarep}
  \label{lemma:compute reconfigurable sequence} Let $\Pi$ be a $f(n)$-neighbourhood diversity locally decidable problem, let $\mathcal{G}=(G,\lambda)$ be a temporal graph with lifetime $\tau$ and temporal neighbourhood diversity $tnd$ and let $Y$ be a valid candidate sequence. It is possible to compute an optimal reconfigurable sequence $T$ compatible with $Y$ in
$\mathcal{O}(\tau\cdot tnd \cdot f(n) + \tau^{\nicefrac{11}{2}} \cdot tnd^8)$.
\end{lemmarep}

\begin{proof}
  We first construct the reconfiguring circulation graph $\overrightarrow{RF}(\mathcal{G},Y)=(F,c,l,u)$, as depicted in \Cref{const:flow}. We have $|V(\overrightarrow{RF})|=\mathcal{O}(\tau\cdot \sum\limits_{Y_i \in Y}|Y_i|) = \mathcal{O}(\tau \cdot tnd)$ and $|A(\overrightarrow{RF})|= \mathcal{O}(\tau \cdot \sum\limits_{Y_i \in Y}|Y_i| + \sum\limits_{Y_i \in Y \setminus Y_\tau}(|Y_i|\cdot |Y_{i+1}|)) = \mathcal{O}(\tau \cdot tnd^2)$. Moreover, we call the two functions $low^t_\pi$ and $up^t_\pi$ exactly once per arc $(x^t_i,y^t_i)$ \textit{i.e.}, $\mathcal{O}(\tau\cdot tnd)$ calls in total. Hence,  the construction of $\overrightarrow{RF}$ can be done in $\mathcal{O}(\tau\cdot tnd \cdot f(n) + |A(\overrightarrow{RF})| + |V(\overrightarrow{RF})|)$
  Then, we compute an optimal circulation flow $f$ in $\overrightarrow{RF}$ which can be done in $\mathcal{O}(|A(\overrightarrow{RF})|^{\nicefrac{5}{2}} \cdot |V(\overrightarrow{RF})|^3) = \mathcal{O}((\tau\cdot tnd^2)^{\nicefrac{5}{2}}\cdot (\tau\cdot tnd)^3) = \mathcal{O}(\tau^{\nicefrac{11}{2}}\cdot tnd^8)$~\cite{Tardos85}. We can then construct an optimal reconfigurable sequence $T$ compatible with $Y$, as described in the proof of \Cref{lemma:flow reconfiguration}. We obtain an overall complexity of $\mathcal{O}(\tau\cdot tnd \cdot f(n) + \tau^{\nicefrac{11}{2}} \cdot tnd^8)$.
\end{proof}

We now have the necessary tools for the overall algorithmic result:

\begin{lemmarep}   \label{lemma:fpt nd}
  Let $\mathcal{G}=(G,\lambda)$ be a temporal graph and let $\Pi$ be an $f(n)$-temporal neighbourhood diversity locally decidable reconfiguration problem. 
 $\Pi$ is solvable in $\mathcal{O}(2^{(tnd\cdot \tau)} \cdot (\tau\cdot tnd \cdot f(n) + \tau^{\nicefrac{11}{2}} \cdot tnd^8))$ where $tnd$ is the temporal neighbourhood diversity of $\mathcal{G}=(G,\lambda)$.
\end{lemmarep}

\begin{proof}
  
  For each candidate sequence $Y = (Y_1,\dots,Y_\tau)$, we call the function $check_\Pi$ and we compute an optimal reconfigurable sequence compatible with it. Thus, each iteration can be done in $\mathcal{O}(\tau\cdot tnd \cdot f(n) + \tau^{\nicefrac{11}{2}} \cdot tnd^8)$. For each time-step $t$, there exists $2^{|V(\tndg|)} = 2^{(tnd)}$ possibilities to choose $Y_t$. Hence, there are $2^{tnd\cdot \tau}$ candidate sequences. We obtain an overall complexity of $\mathcal{O}(2^{tnd\cdot\tau} \cdot (f(n)))$.
\end{proof}

\begin{corollaryrep}
  Let $\mathcal{G}=(G,\lambda)$ be a temporal graph with temporal neighbourhood diversity $tnd$ and lifetime $\tau$.
  {\sc Temporal Dominating Set Reconfiguration} and {\sc Temporal Independent Set Reconfiguration} are solvable in $\mathcal{O}(2^{(tnd\cdot\tau)}\cdot(\tau^{\nicefrac{11}{2}} \cdot tnd^8))$ in $\mathcal{G}$.
  
\end{corollaryrep}

\begin{proof}
By \Cref{lemma:ds is ndld,lemma:is is ndld}, the complexity of the functions $check$ for $low_\pi$ and $up_\pi$ {\sc Dominating Set} and {\sc Independent Set} is $\mathcal{O}(|V(\tndg)|+|E(\tndg)|) = \mathcal{O}(tnd^2)$. Hence, by \Cref{lemma:compute reconfigurable sequence}, we can compute and check an optimal sequence in time $\mathcal{O}(\tau^{\nicefrac{11}{2}} \cdot tnd^8)$. Hence, by \Cref{lemma:fpt nd}, {\sc Temporal Dominating Set Reconfiguration} and {\sc Temporal Independent Set} are solvable in $\mathcal{O}(2^{(tnd\cdot\tau)}\cdot(\tau^{\nicefrac{11}{2}} \cdot tnd^8))$.
\end{proof}